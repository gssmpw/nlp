\documentclass[journal,compsoc]{IEEEtran}
% \documentclass[lettersize,journal]{IEEEtran}
\usepackage{epsfig}
\usepackage{graphicx}
\usepackage{amsmath}
\usepackage{amssymb}
\usepackage{algorithm, algorithmic}

\usepackage{diagbox}
\usepackage{float}
\usepackage{afterpage}
\usepackage{bm}
\usepackage{subfig}

\usepackage{multirow}
\usepackage{color}
\usepackage{tablefootnote}
\usepackage{adjustbox}
\usepackage{wrapfig}

\usepackage{hyperref}       % hyperlinks
\usepackage{url}            % simple URL typesetting
\usepackage{booktabs}       % professional-quality tables
\usepackage{amsfonts}       % blackboard math symbols
\usepackage{nicefrac}       % compact symbols for 1/2, etc.
\usepackage{microtype}      % microtypography
\usepackage{times}
\usepackage{epsfig}

\usepackage{bbding}
\usepackage{etoolbox}
\usepackage{paralist}
\usepackage{ulem}
\usepackage{tikz}
\usepackage{color}

\usepackage{makecell}

\usepackage{xcolor,colortbl}



\newcolumntype{Y}{p{0.5cm}<{\centering}}
\newcommand{\mc}[2]{\multicolumn{#1}{c}{#2}}
\definecolor{Gray}{gray}{0.5}
\definecolor{LightCyan}{rgb}{0.88,1,1}

\newcolumntype{a}{>{\columncolor{Gray}}c}
\newcolumntype{b}{>{\columncolor{white}}c}



\DeclareMathOperator*{\cat}{Cat}


\def\etal{\textit{et al}.}
\def\ie{\textit{i.e.}}
\def\eg{\textit{e.g.}}
\def\etc{\textit{etc}}
\def\wrt{\textit{w.r.t. }}

\def\bz{\textcolor{red}}
\def\xc{\textcolor{blue}}

\newcommand{\tb}[1]{\textbf{#1}}
\newcommand{\bc}[1]{\textcolor[RGB]{192,0,0}{\text{#1}}}
\newcommand{\rc}[1]{\textcolor{blue}{\text{#1}}}
\newcommand{\bb}[1]{\textcolor[RGB]{192,0,0}{\textbf{#1}}}
\newcommand{\rb}[1]{\textcolor{blue}{\textbf{#1}}}
\newcommand{\todo}[1]{{\color{blue}{[TODO: #1]}}}

\renewcommand{\thefootnote}{\fnsymbol{footnote}}

%For paragraph heading that is NOT immediately after a (sub)section heading
% \newcommand{\midparaheading}[1]{\vspace*{-0.8em}\paragraph{#1}}

\normalem

\begin{document}

\title{Rotation-Adaptive Point Cloud Domain Generalization via Intricate Orientation Learning}

\author{{Bangzhen~Liu,~Chenxi~Zheng,~Xuemiao~Xu,~Cheng Xu,~Huaidong~Zhang, \\ and~Shengfeng~He,~\IEEEmembership{Senior Member,~IEEE}}

\thanks{This work is supported by the China National Key R\&D Program (No. 2023YFE0202700, 2024YFB4709200), the Key-Area Research and Development Program of Guangzhou City (No. 2023B01J0022), the Guangdong Provincial Natural Science Foundation for Outstanding Youth Team Project (No. 2024B1515040010), the NSFC Key Project (No. U23A20391), the Guangdong Natural Science Funds for Distinguished Young Scholars (No. 2023B1515020097), the AI Singapore Programme under the National Research Foundation Singapore (No. AISG3-GV-2023-011), and the Lee Kong Chian Fellowships. (Bangzhen~Liu and Chenxi~Zheng contributed equally to this work.) (Corresponding authors: Xuemiao~Xu; Cheng~Xu.)
}
\thanks{Bangzhen Liu,~Chenxi~Zheng, Xuemiao Xu, Cheng Xu, and Huaidong Zhang are with the South China University of Technology, Guangzhou, China.~E-mail: liubz.scut@gmail.com,~cszcx@mail.scut.edu.cn, xuemx@scut.edu.cn, cschengxu@gmail.com, and huaidongz@scut.edu.cn. Xuemiao Xu is also with the Guangdong Engineering Center for Large Model and GenAI Technology, the State Key Laboratory of Subtropical Building and Urban Science, the Ministry of Education Key Laboratory of Big Data and Intelligent Robot and the Guangdong Provincial Key Lab of Computational Intelligence and Cyberspace Information.}
\thanks{ Shengfeng He is with the Singapore Management University, Singapore. E-mail: shengfenghe@smu.edu.sg.} 
% the School of Computing and Information Systems, 
}

\markboth{IEEE Transactions on Pattern Analysis and Machine Intelligence}%
{Shell \MakeLowercase{\textit{Liu et al.}}: Rotation-Adaptive Point Cloud Domain Generalization via Intricate Orientation Learning}


\IEEEtitleabstractindextext{
\begin{abstract}
  The vulnerability of 3D point cloud analysis to unpredictable rotations poses an open yet challenging problem: \emph{orientation-aware 3D domain generalization}. Cross-domain robustness and adaptability of 3D representations are crucial but not easily achieved through rotation augmentation. Motivated by the inherent advantages of intricate orientations in enhancing generalizability, we propose an innovative rotation-adaptive domain generalization framework for 3D point cloud analysis. Our approach aims to alleviate orientational shifts by leveraging intricate samples in an iterative learning process. Specifically, we identify the most challenging rotation for each point cloud and construct an intricate orientation set by optimizing intricate orientations. Subsequently, we employ an orientation-aware contrastive learning framework that incorporates an orientation consistency loss and a margin separation loss, enabling effective learning of categorically discriminative and generalizable features with rotation consistency. Extensive experiments and ablations conducted on 3D cross-domain benchmarks firmly establish the state-of-the-art performance of our proposed approach in the context of orientation-aware 3D domain generalization. 
\end{abstract}

\begin{IEEEkeywords}
  Point cloud domain generalization, contrastive learning, rotation robustness, intricate orientation mining
\end{IEEEkeywords}
}

\maketitle

\IEEEdisplaynontitleabstractindextext

\IEEEpeerreviewmaketitle

\section{Introduction}
% Large Language Models~(LLMs) represent a transformative advancement in the field of language processing, demonstrating an unparalleled capacity for text generation and comprehension, which can be further applied in a wide variety of applications.  
% %Large language models (LLMs) have risen to prominence in various fields, offering endless possibilities for artificial intelligence applications. 
% Despite their significant prevalence in recent years, LLMs are frequently challenged with security and privacy issues, such as poor explainability~\cite{}, poor robustness~\cite{}, data dependency~\cite{}, etc. Among them, a specific and notable concern that has garnered increasing attention is the phenomenon of `hallucination', where models generate plausible but factually inaccurate or irrelevant content when employed for specific tasks such as problem-solving.  
% %In particular, the hallucination issue is when these large models are employed for problem-solving, users frequently voice concerns regarding being misled or deceived by the models' nonsensical and erratic outputs. 
% The tendency of these models to produce inaccurate outputs and fabricate facts has severely undermined the safety and usability of LLM applications, which calls for immediate attention in LLM research. 
% %Hallucination in large language models (LLMs) is a critical issue that needs immediate attention in LLM research. The tendency of these models to produce inaccurate outputs and fabricate facts has severely undermined the safety and usability of LLM applications. 
%exceptional 
%including limited explainability, compromised robustness, and a heavy reliance on data, each 
%However, d
Large Language Models (LLMs) have revolutionized language processing, demonstrating impressive text generation and comprehension capabilities with diverse applications. However, despite their growing use, LLMs face significant security and privacy challenges~\cite{siddiq2023generate, hou2023large, kaddour2023challenges, li2024model, 10.1145/3691620.3695510}, which affect their overall effectiveness and reliability. A critical issue is the phenomenon of \emph{hallucination}, where LLMs generate outputs that are coherent but factually incorrect or irrelevant. This tendency to produce misleading information compromises the safety and usability of LLM-based systems. This paper focuses on \emph{fact-conflicting hallucina}tion (FCH), the most prominent form of hallucination in LLMs. FCH occurs when LLMs generate content that directly contradicts established facts. For instance, as illustrated in \figref{fig:example1}, an LLM incorrectly asserts that ``\emph{Haruki Murakami won the Nobel Prize in Literature in 2016}'', whereas the fact is that ``\emph{Haruki Murakami has not won the Nobel Prize, though he has received numerous other literary awards}''. 
Such inaccuracies can significantly lead to user confusion and undermine the trust and reliability that are crucial for LLM applications.

% Large Language Models~(LLMs) have brought transformative advancements to language processing and beyond, showcasing text generation and comprehension abilities with wide-ranging applications. 
% Despite the increasing prevalence, LLMs face critical challenges in security and privacy aspects~\cite{siddiq2023generate, hou2023large, kaddour2023challenges}, heavily impacting their effectiveness and reliability. 
% One notable issue is the phenomenon of \emph{hallucination}, where LLMs produce coherent but factually inaccurate or irrelevant outputs during problem-solving. 
% Such a tendency to generate misleading information jeopardizes the safety and usability of LLM-based applications. 
% This paper concerns the \emph{fact-conflicting hallucination}~(FCH), which is the primary form of hallucinations in LLMs. 
% FCH occurs when LLMs generate content that directly contradicts the well-established facts, as exemplified in \figref{fig:example1}, where an LLM incorrectly believes 
% ``\emph{Haruki Murakami won the Nobel Prize in Literature in 2016}'', deviating from the fact that ``\emph{Haruki Murakami has not won the Nobel Prize but other numerous awards for his work in Literature}''. Such misinformation can cause significant user confusion and undermine the trust and reliability that are essential in various LLM applications. 

%correct answer of 

%is manifested by
%Such misinformation dissemination leads to significant user confusion, eroding the trust and reliability that are crucial in various LLM applications. 

%Large Language Models~(LLMs) represent a transformative advancement in the field of language processing, demonstrating an unparalleled capacity for text generation and comprehension, which can be further applied in a wide variety of applications. Despite their growing prevalence, LLMs encounter critical challenges, particularly in aspects of security and privacy. These include concerns such as limited explainability~\cite{}, compromised robustness~\cite{}, and heavy reliance on data~\cite{}, each posing distinct challenges to their efficacy and reliability. Among these, the phenomenon of ``hallucination'' stands out as a notable concern. This occurs when LLMs, while employed in tasks like problem-solving, generate outputs that are coherent yet factually inaccurate or irrelevant. Such a tendency to produce misleading information not only compromises the safety of LLM applications but also raises urgent questions regarding their usability. 

% Hallucinations in LLMs manifest in several distinct forms, each contributing differently to the challenges identified in their growing applications. 
% %The first, known as ``Input-conflicting hallucination'', arises when there is a discrepancy between the model's output and the user's initial input, reflecting a potential misunderstanding of the task at hand. On the other hand, ``Context-conflicting hallucination'' represents the second type, occurring when LLMs produce inconsistent responses in prolonged or multi-turn interactions, indicative of their limitations in maintaining coherent context. 
% Among the three types categorized in the literature~\cite{huang2023survey,zhang2023hallucination}, ``Fact-conflicting hallucination~(FCH)'' poses a particularly serious concern which is the primary focus of this paper. This phenomenon generates content in direct opposition to established factual knowledge. As illustrated in the example shown in Figure~\ref{fig:example1}, when an LLM was asked about the first person to walk on the moon, it incorrectly answered ``Charles Lindbergh in 1951'', a clear deviation from the factual answer of Neil Armstrong in 1969. This type of hallucination can lead to the dissemination of incorrect information and cause significant confusion among users, undermining the trust and reliability critical in various LLM applications. %Addressing fact-conflicting hallucinations is therefore essential for the advancement of LLMs, ensuring they not only function effectively but also responsibly in their diverse roles.


% According to \cite{huang2023survey} and \cite{zhang2023hallucination}, hallucinations in large language models can be categorized into types such as factual hallucinations and contextual hallucinations. Current benchmark assessments tend to focus on evaluating the propensity of LLMs to generate erroneous facts. The origin of these issues can be traced back to multiple deficiencies, including flaws in training data, training algorithms, and the inference process.

% \begin{figure}[t]
%     \centering
%     \includegraphics[width=0.95\linewidth]{fig/example1-cropped.pdf}\\
%     \caption{A Hallucination Output Example.}
%     %\vspace{-0.5cm}
%     \label{fig:example1}
% \end{figure}

\begin{figure}[t]
\centering
\vspace{3mm}
\hspace{-3mm}
\includegraphics[width=\linewidth]{fig/drowzee-example.pdf}
\\[0.5em]
\caption{A Hallucination Output Example}
\label{fig:example1}
\vspace{-4mm}
\end{figure}
%\lnk{Factual Hallucination and LLM inference current status}

Recent studies have introduced various methods to detect LLM hallucinations. A common approach involves developing specialized benchmarks, such as TruthfulQA~\cite{lin-etal-2022-truthfulqa}, HaluEval~\cite{HaluEval}, and KoLA~\cite{yu2023kola}, to assess hallucinations in tasks like question-answering, summarization, and knowledge graphs. 
While manually labeled datasets provide valuable insights, current methods often rely on simplistic or semi-automated techniques such as string matching, manual validation, or verification through another language model. These approaches reveal significant gaps in automatically and effectively detecting fact-conflicting hallucinations (FCH). 
The primary challenges in FCH detection arise from the lack of dedicated ground truth datasets, the absence of comprehensive test cases designed to trigger FCH, and the lack of a robust testing framework.  
Unlike other types of hallucinations, such as input-conflicting or context-conflicting hallucinations~\cite{ji-etal-2023-rho, shi2023large}, which can often be identified through semantic consistency checks, detecting FCH requires the verification of factual accuracy against external knowledge sources/databases. This process is particularly challenging and resource-intensive, especially for tasks that involve complex logical relationships~\cite{zhang2024fusion}. We identify three primary challenges in addressing this research gap:


% Recent studies have introduced a range of methods for detecting 
% hallucinations. One common approach involves creating comprehensive benchmarks tailored for this purpose. 
% Datasets such as TruthfulQA~\cite{lin-etal-2022-truthfulqa}, HaluEval~\cite{HaluEval}, and KoLA~\cite{yu2023kola} have been designed to evaluate hallucinations across different contexts, including question-answering, summarization, and knowledge graphs. 
% Despite the value of these manually labeled datasets, the current techniques heavily rely on naive and semi-automatic methods, such as string matching, manual validation, or utilizing another LLM for confirmation. 
% Therefore, there is a gap 
% in automatically and effectively testing FCHs, and the primary obstacle in testing FCH is the absence of dedicated ground truth datasets and an extensive testing framework.  
% Unlike other types of hallucinations, e.g., input-conflicting or context-conflicting 
% \cite{ji-etal-2023-rho, shi2023large}, 
% which can be identified through checks for semantic consistency, 
% detecting FCH
% requires the verification of the content's factual accuracy against external sources of knowledge or databases. This makes the process particularly arduous and resource-intensive, especially for tasks processing content with complex logical connections. 
% Here, we highlight three concrete challenges in filling up the identified research gap: 




%The main obstacle in testing for FCH is the absence of dedicated ground truth datasets and specific testing frameworks. Unlike other types of hallucinations~(e.g., input-conflicting and context-conflicting hallucinations, to be detailed in Section~\ref{subsec:cat}) which can be identified through checks for semantic consistency, FCH demands the verification of the content's factual accuracy against external sources of knowledge or databases. This requirement makes the process particularly challenging and resource-intensive, especially for tasks processing contents with inherent logical connections.

% \shil{(I feel the transition is not smooth, we first introducing datasets, and not explaining how they use these datasets to test llm. after these, we can state these methods are not automatic.)}


% To tackle FCH, recent works have developed various techniques for testing and detecting hallucination~\citep{yu2023kola,HaluEval}. The typical and intuitive solution is to develop comprehensive benchmarks for detection. This is done through a process of sampling, filtering, and enhancing ground-truth answers to identify the best and correct answers from given candidates. For example, a well-known hallucination evaluation benchmark HaluEval~\cite{HaluEval} constructs scenarios where LLMs are tested on their ability to select the most factually accurate answers from a set of provided options, with a focus on filtering out hallucinated responses. %\yi{ also talk about the construction of benchmark?}
% Additionally, human annotation plays a critical role in identifying hallucinations in LLM outputs~\cite{Alpaca}. This involves humans determining whether responses contain hallucinated information and considering aspects such as unverifiability, non-factuality, and irrelevance. 



% \lnk{Key challenge: lack of hallucination testing when faced with logic reasoning related problems}
%Bridging the identified research gap in the literature necessitates exploring the inherent challenges faced in detecting FCHs, which are crucial for advancing and enhancing the reliability of LLMs. 

\begin{enumerate}[itemsep=1mm, wide,  labelindent=9pt]
%[itemsep=0ex,leftmargin=0.35cm]
%Challenge\#1: 
%While these benchmarks effectively detect certain hallucinations, they 
\item {\textbf{Automatically constructing and updating benchmark datasets.}} Existing methodologies mainly rely on manually curated benchmarks for detecting specific hallucinations, which fail to encompass the broad and dynamic spectrum of fact-conflicting scenarios in LLMs. 
Meanwhile, due to the ever-evolving nature of knowledge, the need for frequent updates to benchmark data imposes a substantial and continuous maintenance effort.
The reliance on benchmark datasets thus restricts the FCH detection techniques' adaptability, scalability, and  %more importantly, 
detection capability;  
%Challenge\#2:
% in existing test cases. 
\item {\textbf{Efficiently generating FCH test cases.}}
LLMs often answer correctly to simple, straightforward questions due to their extensive training on vast datasets. However, to effectively assess their reasoning capabilities, it is important to generate more complex questions, such as those involving intricate temporal characteristics, that require reasoning rather than just recalling facts. However, constructing such test cases is non-trivial. The challenge lies in designing questions that use familiar knowledge but involve reasoning patterns the LLM may not have been explicitly trained on. Creating such test cases efficiently while ensuring they probe reasoning skills in ways the model has not previously encountered is essential to uncovering latent hallucinations;
% queries that involve temporal concepts, such as ``\emph{Does the human population finally reach six billion by the year 2000?}'' may often be used in applications. However, the correctness of the LLM outputs cannot be guaranteed, potentially leading to misleading information. Currently, there are no satisfactory approaches to automatically verify LLM outputs in such test cases; 
%errors even before the occurrence of large model hallucinations; 
%However, it is known that 
%Another critical issue lies in the verification of temporal logic in existing test cases. 
%It is well known that test cases involving temporal-related questions often face difficulties in automatically verifying the soundness and completeness of these issues. Consequently, the correctness of these test cases cannot be guaranteed, potentially introducing errors even before the occurrence of large model hallucinations;
%Challenge\#3: 
\item {\textbf{Validating the reasoning steps from LLM outputs.}} Even when LLMs finally produce correct answers, the outputs may not indicate an accurate reasoning process, potentially masking false understanding -- a source of FCH. Additionally, the quality of manual validation can differ based on human expertise. As a result, automatically validating reasoning processes, particularly those involving complex logical relationships, is inherently challenging. 
\vspace{1mm}
\end{enumerate}







% \lnk{Key challenge: factual knowledge exploring and new facts generation}
%\yi{we should focus on testing, addressing is a little bit vague.}
% The current research landscape in LLM presents a critical gap in automatically testing FCHs. Predominantly, existing methodologies are anchored to manual benchmarks. %\yi{this sentence is quite chinglish.}
% While these benchmarks are effective in detecting certain types of hallucinations, such as those in Figure~\ref{fig:example1}, they fall short in encompassing the broad and dynamic spectrum of fact-conflicting scenarios inherent to LLMs. %\yi{again, this sentence is not very clear}
% Meanwhile, the need for frequent updates to benchmark data, due to the ever-evolving nature of knowledge, imposes a significant and continuous maintenance effort.
% The reliance on benchmark datasets thus restricts the detection techniques’ adaptability, scalability, and worse, detection capability. 
% From a second perspective, the consistency in the quality of benchmark questions can vary, reflecting the differing levels of experience and skill among the human experts responsible for creating them. This is particularly reflected in the stages such as data labeling and results validation. Additionally, it is important to acknowledge that humans are prone to errors.
% %the scalability and the deof these existing methods are also significantly challenged by their dependency on extensive human intervention, particularly in stages such as data labeling and results validation. %This heavy reliance on manual efforts not only limits the scalability of such approaches but also questions their feasibility in efficiently handling the extensive and intricate datasets characteristic of LLMs.
% Thus, the development of more autonomous, agile, and scalable testing techniques is imperative to effectively identify and tackle FCHs in LLMs.%\yi{in this paper, we focus on testing, but until this paragraph, no terms about ``testing'' explicitly occur.}

% \lnk{Solution to Challenge1: comprehensive logic reasoning based testing framework}

% \lnk{Solution to Challenge2: wiki factual knowledge extraction and prolog rules inference for scalability.}
% \lnk{Key challenge: }

%\textbf{Our Work.}
%To address limitations in the existing techniques, 
%we are the first, to the best of our knowledge, to introduce 
To address the problems outlined above, this paper presents a novel automatic end-to-end metamorphic testing technique based on temporal logic for detecting FCH. To the best of our knowledge, we are the first to create a comprehensive FCH testing framework that utilizes factual knowledge reasoning and metamorphic testing, all seamlessly integrated into the fully automated tool, \tool. 

%\shil{(which four methods?)}
\tool begins by establishing a comprehensive factual knowledge base sourced through crawling information from accessible knowledge bases such as Wikipedia. Each piece of this knowledge acts as a ``seed'' for subsequent transformations. Leveraging logical operators to automatically generate temporal reasoning rules, we transform and augment these seeds and expand factual knowledge into a well-established set of question-answer pairs.
%\yi{into xx}. 
Using the questions and answers in the knowledge set as test cases and ground truth, respectively, we construct a reliable and robust FCH testing benchmark. 


The experiment uses a series of carefully designed template-based prompts to test for FCHs in LLMs. To thoroughly evaluate the reasoning behind the responses, we instruct the LLMs not only to generate answers to the test cases but also to provide detailed justifications for their answers. To reliably identify FCH, we introduce two semantic-aware, similarity-based metamorphic oracles. These oracles extract the key semantic elements from each sentence and map out the logical relationships between them. By comparing the logical and semantic structures of the LLM's responses with the ground truth, the oracles can detect FCH by identifying significant deviations in the LLM's answers from the correct information.




%well-crafted prompts\yi{how prompts generated?} to engage LLMs, testing the alignment of their generated content with our enhanced ground truth. Disparities between LLM outputs and the ground truth signal potential hallucinations. 
%Additionally, in our commitment to fostering collaborative research, we have released our constructed dataset as a benchmark~\cite{drowzee}.

%Our approach directly addresses the need for a comprehensive and flexible testing method by transforming structural factual data into a diverse range of scenarios that LLMs may encounter. This method not only improves the reliability of detection but also enhances its adaptability to various factual contexts.
%Furthermore, we address the scalability challenge by automating the transformation and enlargement of our knowledge base, significantly reducing the dependency on human effort. The well-designed prompts used to test LLMs further streamline the process, making it more efficient in identifying potential hallucinations by comparing LLM outputs with our extended ground truth.

%\textbf{Results and Findings.}
%In evaluating our proposed FCH testing framework and \tool, 
%we undertake 
%to evaluate their effectiveness 
We demonstrate the effectiveness of our approach through comprehensive experiments in multiple contexts. First, our evaluation involves deploying \tool across a wide range of topics drawn from a diverse selection of Wikipedia articles. Second, we test our framework on various open-source and commercial LLMs, thoroughly assessing its applicability and performance across different model architectures. 
Our key findings indicate that \tool succeeds in automatically generating practical test cases and identifying hallucination issues of nine LLMs across nine domains. 
Using these test sets, our experiments show that the rate of hallucination responses produced by various LLMs ranges from 24.7\% to 59.8\% for cases unrelated to temporal reasoning and 16.7\% to 39.2\% for cases requiring temporal reasoning. 
%\shil{shall we differentiate the number for non-temporal and temporal one?}.  
We then categorize these hallucination responses into \emph{erroneous knowledge hallucination} and \emph{erroneous inference hallucination}. 
%\syh{four types?}. 
Through an in-depth analysis, we unveil that the lack of logical reasoning capabilities contributes the most to the FCH issues in LLMs. 
Additionally, we observe that LLMs are particularly prone to generating hallucinations in test cases involving temporal concepts and out-of-distribution knowledge. 
Such an evaluation demonstrates that the 
%Furthermore, we confirm that 
test cases generated using %our 
logical reasoning rules can effectively trigger and detect LLM hallucinations.  %issues in . 


This paper builds upon the earlier version~\cite{DBLP:journals/pacmpl/LiL0SW024} by incorporating hallucination detection through temporal-logic-guided test generation. It includes additional motivational examples (\secref{sec:motivating}), a comprehensive set of reasoning rules for encoding \emph{Metric Temporal Logic} (MTL)~\cite{DBLP:conf/lics/OuaknineW05} formulae (\secref{sec:encoding_MTL}) and automatically generating temporal-logic-related question-answer pairs (\secref{prompt}), and further experimental studies (the {RQ4} at \secref{sec:eval}) that detect hallucinations due to insufficient temporal reasoning capabilities. The main contributions of this work are summarized as follows: 
%We summarize the main contributions of this paper below:
\begin{itemize}[itemsep=1mm,leftmargin=0.35cm]
\item 
%Development of 
\textbf{A novel FCH testing framework.} 
To the best of our knowledge, 
we are the first to develop a novel testing framework based on logic programming and metamorphic testing to automatically detect FCH issues in LLMs. %\yi{hanging sentence}This framework represents a significant advancement over current methodologies, providing a more systematic, comprehensive approach to detection.
%Construction and Release of
\item \textbf{An extensive benchmark based on factual knowledge.} 
To facilitate collaborative efforts and future advances in identifying FCH, 
the source code of \tool and constructed benchmark dataset are publicly available  \cite{drowzee}. 
\item \textbf{Test generation via temporal reasoning.} 
Our tool automatically generates test cases that provide a more comprehensive evaluation of LLMs in handling reasoning tasks and identifying factual inconsistencies. By applying temporal logic-based reasoning rules, we expand the initial seed data from our knowledge base, enhancing the diversity and complexity of the test scenarios. 

\item \textbf{Semantic-aware oracles for LLM answer validation.} We propose and implement two automated verification mechanisms, i.e., the oracles, that analyze the semantic structure similarity between sentences. These oracles are designed to validate the reasoning logic behind the answers generated by LLMs, hereby reliably detecting the occurrence of FCHs. 

\end{itemize}



\section{Related Work} \label{related}




% \subsection{Benchmarks in Coding Scenarios}
% \begin{enumerate}
%     \item Code Generation
%     \item Bug Fixing
% \end{enumerate}

% \subsection{Large Language Model Agents}

% At the heart of the LLM Agent is an Agent Core, which coordinates the core \textit{logic} and \textit{behavioral} characteristics of the agent. In addition, the Agent includes the following key components:

% \begin{itemize}
%     \item Memory Module: It consists of both short-term and long-term memory components that record the agent's internal logs and interactions with the user.
%     \item Tools: These are the tools that the agent can use to perform tasks, usually specific third-party APIs.
%     \item Planning Module: This is used for solving complex problems, such as decomposing tasks and problems, reflexivity or critique.
% \end{itemize}

% \subsection{Multi Agent Collaboration Framework}

% MetaGPT \url{https://arxiv.org/abs/2308.00352}


\parabf{Coding \llm{s}.}
Large Language Models (\llm{s}) have become the go-to solution for a wide array of coding tasks due to their exceptional performance in both code generation and comprehension~\cite{codex}. These models have been successfully applied to various software engineering activities, including program synthesis~\cite{patton2024programming, codex, li2022competition, iyer2018mapping}, code translation~\cite{pan2024lost, roziere2020unsupervised, roziere2021leveraging}, program repair~\cite{xia2023repairstudy, chatrepair, monperrus2018living, bouzenia2024repairagent}, and test generation~\cite{titanfuzz, fuzz4all, deng2023fuzzgpt, lemieux2023codamosa, kang2023testing}. Beyond general-purpose \llm{s}, specialized models have been developed by further training on extensive datasets of open-source code snippets. Notable examples of these code-specific \llm{s} include \codex~\cite{codex}, \codellama~\cite{codellama}, StarCoder~\cite{starcoder,starcodertwo}, and \deepseek~\cite{deepseek}. Additionally, instruction-following code models have emerged, refined through instruction-tuning techniques. These include models such as \codellamainstruct~\cite{codellama}, \deepseekinstruct~\cite{deepseek}, \wizardcoder~\cite{wizardcoder}, \magicoder~\cite{magicoder}, and OpenCodeInterpreter~\cite{zheng2024opencodeinterpreter}.

\parabf{Benchmarking \llm-based coding tasks.}
To assess the capabilities of \llm{s} in coding, a variety of benchmarks have been proposed. Among the most widely utilized are \humaneval~\cite{codex} and \mbpp~\cite{austin2021program}, which are handcrafted benchmarks for code generation that include test cases to validate the correctness of \llm outputs. Other benchmarks have been developed to offer more rigorous tests~\cite{evalplus}, cover additional programming languages~\cite{zheng2023codegeex,cassano2023multipl}, and address different programming domains~\cite{livecodebench, hendrycksapps2021, codecontest, ds1000, arcade}.

More recently, research has shifted towards evaluating \llm{s} on real-world software engineering challenges by operating on entire code repositories rather than isolated coding problems~\cite{swebench, zhang2023repocoder, liu2023repobench}. A notable benchmark in this area is \swebench~\cite{swebench}, which includes tasks requiring repository modifications to resolve actual GitHub issues. The authors of \swebench have also released a more focused subset, \swebenchlite~\cite{swebenchlite}, which contains 300 problems centered on bug fixing that only involves single-file modifications in the ground truth patches. ML-Bench \cite{liu2023mlbench} is a benchmark for evaluating large language models and agents for Machine Learning tasks on reporitory-level code. It involves 18 repositories and focuses on code generation and interactions with Jupyter Notebooks.

\parabf{Repository-level coding.}
The rise of agent-based frameworks~\cite{xi2023rise} has spurred the development of agent-based approaches to software engineering tasks. Devin~\cite{devinwebpage} (and its open-source counterpart OpenDevin~\cite{opendevin}) is among the first comprehensive \llm agent-based frameworks. Devin employs agents to first perform task planning based on user requirements, then allows them to use tools like file editors, terminals, and web search engines to iteratively execute the tasks. \sweagent~\cite{sweagent} introduces a custom agent-computer interface (ACI), enabling the \llm agent to interact with the repository environment through actions like reading and editing files or running bash commands. Another agent-based approach, \autocoderover~\cite{autocoderover}, equips the \llm agent with specific APIs (e.g., searching for methods within certain classes) to effectively identify the necessary modifications for issue resolution. Beside these examples, a variety of other agent-based approaches have been developed in both open-source~\cite{aidar} and commercial products~\cite{bouzenia2024repairagent, coder, repounderstander, lingma, factorydroid, ibmagent, opencsgstarship, marscode, amazonqdeveloper}.

% Unlike these agent-based methods, \tech offers a straightforward and cost-efficient solution for addressing real-world software engineering challenges. Our work is the first to demonstrate that an \emph{agentless} approach can achieve comparable performance without the need for complex tools or modeling intricate environment behavior and feedback.

Unlike existing benchmarks and agent-based frameworks, which focus on the code generation/completion tasks, our proposed \model and \agent focus on the code deployment task, which is under-studied in the field.

\section{Orientational Shift Analysis} \label{sec:analysis}


\textbf{Problem Definition.}
In practical scenarios, objects often originate from various domains, and concurrently, present diverse orientations. To achieve an accurate understanding of these targets, our goal is to grant 3D recognition systems with cross-domain generalizability and robustness to rotational transformations. To this end, we introduce a novel task: orientation-aware 3D domain generalization, and explore the applicability of existing 3D recognition systems under orientational shifts.
Let $\mathbb{X}$ and $\mathbb{Y}$ be the input and label spaces, respectively. We consider a labeled source domain $D_s = {\{(P_i^s, y_i^s)}\}^{n_s}_{i=1}$ with $n_s$ samples and an unlabeled target domain $D_t = \{{P_i^t}\}^{n_t}_{i=1}$. Here, $y^s\in \mathbb{Y}$ is the source labels, and $P^s, P^t \in \mathbb{X}$ represent point clouds with a specified orientation relative to the world coordinate system, where each $P_i \in \mathbb{R}^{N_c \times 3}$ consists of $N_c$ points. The orientation is defined by a $3\times3$ rotation matrix $M_i \in \mathcal{O} \subseteq$ SO(3), where $\mathcal{O}$ is the set of orientations in the given domain and SO(3) represents all rotations in $\mathbb{R}^3$. The objective of orientation-aware 3D domain generalization is to learn a projection function $g: \mathbb{X} \to \mathbb{Y}$ that can be applied to any given target domain $D_t$ with arbitrary orientations in $\mathcal{O}$, by solely training on the labeled source domain $D_s$.


\begin{figure}[t]
    % \flushleft
    \centering
    \subfloat[Random rotation augmentation]{%[b]{0.45\textwidth}
        \label{fig:statement_random}
        % \centering
        \includegraphics[width=0.23\textwidth]{./resources/teaser/analysis/random.png}
    }
    % \hspace{5mm}
    \subfloat[Intricate orientation training]{%[b]{0.45\textwidth}
        \label{fig:statement_ours}
        % \centering
        \includegraphics[width=0.23\textwidth]{./resources/teaser/analysis/full.png}
    }
    \vspace{-2mm}
    \caption{t-SNE visualization of the feature spaces trained with (a) random rotation augmentation and (b) our proposed intricate orientation training. For the sake of simplicity, we only visualize samples from three categories (table, chair, and sofa), which share part of distinguishing features (\eg, slim legs, plane seats, \etc). Samples from ModelNet (the source domain) and ShapeNet (the target domain) are denoted by star (`$\star$') and plus (`$+$'), respectively. The gray dash line denotes the approximated decision boundaries.}
    \label{fig:feature_statement}
    \vspace{-5mm}
\end{figure}


\noindent\textbf{Generalization Analysis of Orientations.}
To accurately understand 3D shapes, information from multiple perspectives is often necessary. Some of the perspectives are easy-to-understand by deep models but less informative, incurring a phenomenon known as ``taking the whole from a part''. Specifically, in the context of rotation-robust point cloud analysis, these learning perspectives refer to those rotated variants with small gradients during training. Random rotation attempts to capture comprehensive information through a Monte Carlo approach, but the vast number of possible rotated angles introduces difficulties and leads to imbalanced learning.
This biased learning problem could give rise to inaccurate selection of point features, making false decisions. As shown in Fig.~\ref{fig:feature_statement}(a), the learned model cannot well distinguish the rotated sample from ``table" and ``chair", as they may contain similar features such as slim legs, which are learned from easy-to-understand samples. 
On the other hand, intricate samples lying beyond current cognitive boundaries can offer complementary knowledge for learning a more accurate discriminative boundary.

We empirically validate these insights by employing Maximum Mean Discrepancy (MMD)~\cite{gretton2012kernel} as a measure of distributional shifts. The MMD calculates the distance between source and target distributions in the Reproducing Kernel Hilbert Space (RKHS). We train a DGCNN~\cite{wang2019dynamic} on the ModelNet (source domain) and test it on ShapeNet (target domain), augmenting the data with random rotations and intricate orientations, respectively. The intricate orientations of the training dataset are obtained by optimizing the rotational angles of each point cloud to maximize the cross-entropy loss with a baseline model trained on aligned data. Fig.~\ref{fig:analysis}(a) depicts the per-class MMD values, demonstrating that training with intricate orientations reduces orientational shifts more effectively. To evaluate discriminability and consistency under varying rotations, we further train a linear SVM on the source domain and test it on the target domain. We augment each point cloud 64 times as illustrated in Section~\ref{sec:experiment} and measure consistency by calculating the mean KL-divergence between their output probabilities. As shown in Fig.~\ref{fig:analysis}(a), the discriminability of the model augmented with intricate orientations matches that of the randomly rotated version, while better preserving consistency across different rotations.
Based on these findings, we aim to leverage intricate samples to reduce orientational shift, enhance output consistency, improve model discriminability to arbitrary rotations, and finally obtain a better classifier that is generalizable towards various domains (Fig.~\ref{fig:analysis}(b)).



% \vspace{-3mm}
\section{Intricate Orientation Learning} \label{sec:method}

\begin{figure}[t]
    \captionsetup[subfloat]{justification=centering,singlelinecheck=false}
    % \flushleft
    \centering
    \subfloat[Per-class maximum mean discrepancy]{%[b]{0.45\textwidth}
        \label{fig:analysis_a}
        % \centering
        \includegraphics[width=0.32\textwidth]{./resources/teaser/analysis/divergence.pdf}
    }
    % \hspace{.4in}
    \subfloat[Consistency \\\& discrimination]{%[b]{0.45\textwidth}
        \label{fig:analysis_b}
        % \centering
        \includegraphics[width=0.16\textwidth]{./resources/teaser/analysis/consistency.pdf}
    }
    \vspace{-3mm}
    \caption{(a) Maximum Mean Discrepancy~\cite{borgwardt2006integrating} between the features of the ModelNet and ShapeNet subdomains in PointDA~\cite{qin2019pointdan}, learned with different orientation augmentations. (b) The mean consistency rate was computed using KL-divergence between multiple augmented variants and the mean accuracy on ShapeNet.}
    \vspace{-5mm}
    \label{fig:analysis}
\end{figure}


The pipeline of our framework is depicted in Fig.~\ref{fig:pipeline}. It consists of an iterative optimization process that alternates between intricate orientation mining and orientation-aware contrastive training. We first generate an alternative set of intricate rotation angles through intricate orientation mining. Then we augment the training point clouds by applying the selected angles from the alternative set, creating a series of intricate sample pairs. To enhance the generalizability of the learned representations, we employ orientation-aware contrastive training using these intricate pairs. Our framework incorporates an orientation consistency loss, which encourages the learning of more representative features while promoting their consistency \wrt random rotations. Additionally, we introduce a margin separation loss to further improve the categorical discriminability of the learned representation space.
In the following sections, we elaborate on each component of our framework in detail.



\begin{figure*}[!b]%th
  \includegraphics[width=\textwidth]{figures/translators_veriplan.pdf}
   \vspace{-12pt}
  \caption{\textit{\ours{} Rule Translator ---} Pipeline of the rule translator described in Section \ref{sec:ruleTranslator}. The translator extracts a set of constraints from the user's initial natural language input that must be adhered to for a correct plan. These constraints are mapped to appropriate LTL properties within the temporal constraint template (Table \ref{fig:template}) for model checking. Using this template, the constraints are converted into LTL and PRISM language for model checking, and then presented back in natural language for user verification.
}
  \label{fig:pipeline}
%   \vspace{-6pt}
\end{figure*}





% \vspace{-3mm}
\subsection{Intricate Orientation Mining}
Let us first consider a standard classification problem over the given data distribution $D = {\{(P_i, y_i)}\}^{n}_{i=1}$. Suppose we have a point cloud classification model, the goal is to search the model parameters $\omega^*$ that optimize the empirical risk $E_{(P, y)\sim D}\left[L(w, P, y)\right]$, where $L$ is a proper loss function. To improve the robustness, the most adopted technique is data augmentation, such that the model can resist possible perturbations on the input data, resulting in the following objective:
\begin{equation}
    \omega^*=\mathop{\arg\min_{\omega}} \mathbb{E}_{(P, y)\sim D} \left[L(\omega, f(P), y)\right],
\end{equation}
where $f$ is the perturbation function. Considering the orientational shift only, the perturbation function of a given point cloud $P_i$ can be regarded as augmenting the aligned pose with a rotation matrix $M_i$. According to Euler's rotation theorem, the rotation matrix is defined in $\mathbb{R}^3$ as $M_i = R_{\theta_{x_i}}\cdot R_{\theta_{y_i}}\cdot R_{\theta_{z_i}}$, 
where $R_{\theta_j}$ is the rotation matrix about the axis-j by angle $\theta_{j} \in \left[-\pi, \pi\right)$ over the cartesian coordinates. In this case, the rotation matrix is parameterized by $\Theta_i=\left[\theta_{x_i}, \theta_{y_i}, \theta_{z_i}\right]$. Since the trigonometric function is differential in the scope $\left[-\pi, \pi\right)$, we can optimize $\Theta_i$ through gradient descent.



The essence of intricate orientation mining lies in the deliberate search for intricate orientations, instead of directly optimizing the model with random rotations. This endeavor aims to identify a rotation matrix $\hat{\Theta}_i$ that introduces perturbations to the point cloud, such that the 3D model is confounded:
\begin{equation}
    \hat{\Theta}_i = \mathop{\arg\max_{\Theta_i}}L(\omega^*, f(\Theta_i, P_i), y_i), \label{formula:iom}
\end{equation}
where $f(\Theta_i, P_i)=M_i\cdot P_i$, and $L$ is a task-specific optimization function~(\eg, cross-entropy loss for classification task, hereafter referred to as $L_{cls}$). By iteratively maximizing this objective, we can calculate the most intricate orientations for all the point clouds within $D$. The details of gradient calculation are provided in the supplement. We adopt Projected Gradient Descent~\cite{madrytowards} to ensure that the optimized $\Theta_i$ can generate a pure rotation matrix. By applying intricate orientation mining to the whole training dataset, we obtain the intricate orientation set $I=\{\hat{\Theta}_{i}\}^{n}_{i=1}$, which is essential for the construction of intricate sample pairs. Note that each orientation is specified for the according sample. To increase the diversity of $I$, we further initialize $\Theta$ with random values and repeat the optimization multiple times. Finally, the intricate set is defined as $I=\{\{\hat{\Theta}^j_{i}\}^{AT}_{j=1}\}^{n}_{i=1}$, where $AT$ is the number of augmentation times. During the training stage, we periodically update the intricate set $I$ for every $T$ epochs. 




% \vspace{-3mm}
\subsection{Orientation-aware Contrastive Training}

Based on the intricate orientation set $I$, we construct the intricate sample pairs and build a contrastive learning framework upon them to obtain categorically discriminative and generalizable features with rotational consistency. As Fig.~\ref{fig:pipeline} shows, our model consists of an optimizable student network $F_s\!=\!C_s\circ E_s$ and a frozen teacher network $F_t\!=\!C_t\circ E_t$, where $E_{s/t}$ is a feature extractor which encodes point cloud to the representation space and derives the final classification result through a linear classifier $C_{s/t}$. The teacher network is optimized by Exponential Moving Average (EMA)~\cite{tarvainen2017mean}.

\noindent\textbf{Inner-sample Rotational Consistency.} The main idea is to maximize the agreement between the original shape and its intricate augmented variants via self-contrastive learning. Each time we sample a mini-batch of $N$ point clouds $\left\{P_{m}\right\}^N_{m=1}$ and randomly select one intricate rotation $\hat{\Theta}_m$ from $I$ for each point cloud. Then we construct the intricate sample pairs $B_{it} = \{(P_{m}, \hat{P}_m )\}^N_{m=1}$, where $\hat{P}_m = f(\hat{\Theta}_{m}, P_m)$ is the augmented version of the original point cloud with the intricate rotation. $B_{it}$ serve as the input of (${F_t}$, $F_s$) and the output logits of $(P_{m}, \hat{P}_m )$ are denoted as $p_m$ and $\hat{p}_m$, respectively. To ensure the consistency of the intricate pairs, we perform knowledge distillation~\cite{hinton2015distilling} on the output probabilities, the orientation consistency loss is therefore formulated as:
\begin{equation}\label{eq5-0}
    \sigma(p/\tau) = \frac{exp(p/\tau)}{\sum_{j=1}^{K}{exp(p^{(j)}/\tau)}},
\end{equation}
\begin{equation}
    L_{oc} = -\frac{1}{N}\sum_{m=1}^{N} \sigma(\hat{p}_m/\tau_t)\log(\sigma(p_m /\tau_s)), \label{eq5}
\end{equation}
where $K$ is the number of categories and $\sigma(\cdot)$ is the softmax function. $\tau \in \mathbb{R^+}$ is the temperature parameter controlling the magnitude of the output logits, where the subscripts in $\left[\tau_s, \tau_t\right]$ denote the source and target domains, respectively. By iteratively refining the distance between the aligned object and its intricate variant, the model can gradually adapt to the harder rotational variations and learn common knowledge concerning different rotations.


\noindent\textbf{Intra-category Discriminability.} To further enhance discriminability, we aim for the learned representation space to maintain rotational consistency while improving the classification ability. To achieve this, we introduce a margin separation loss that yields a classifier with a concise and accurate boundary, capable of handling the unforeseen rotational perturbations. Specifically, we first construct the intra-category intricate pairs by considering the relationships between different categories in the same mini-batch. The intricate pairs can be divided into two groups, including the positive pairs and the negative pairs.
Simply, the positive pairs can be found by traversing the training batch and sorting out samples that belong to the same class. But the embedding space under multiple orientations may not be well simulated by the originally aligned point clouds. Instead of learning from only one intricate orientation, we further augment each point cloud $P_{i}$ with $V$ number of intricate angles sampled from $\{\hat{\Theta}^l_{i}\}^{AT}_{l=1}$ to better simulate the feature space. The loss is designed to minimize the cosine distance between the positive samples as follows:
\begin{equation}
    L_{pos}\!=\!- \frac{1}{K}\sum_{k=1}^{K} \!\frac{1}{V{N^k_p}^2}\!\!\!\sum_{i=1,j=1 \atop j\neq i}^{N^k_{p}, N^k_{p}}\!\sum_{v=1}^{V} \cos(\frac{E_s(\hat{P}_{i,v})}{\tau'},\! \frac{E_s(\hat{P}_j)}{\tau'}),
    \label{eq6}
\end{equation}
where $\cos(\cdot, \cdot)$ denotes the cosine similarity between the two inputs, $N^k_{p}$ is the number of samples of class $k$ within the mini-batch and $\tau'$ is the temperature parameter.

The negative pairs are formed by samples with different labels.
To enlarge the discrepancy of different categories in the representation space, we plan to maximize the distance between each negative pair with:
\begin{equation}
    L_{neg}\!=\!\frac{1}{K}\sum_{k=1}^{K}  \frac{1}{{N^k_p}{N^k_n}}\sum_{i=1}^{N^k_p} \sum_{j=1}^{N^k_{n}} \cos( \frac{E_s(\hat{P}_i)}{\tau'}, \frac{E_s(\hat{P}_j)}{\tau'}),
    \label{eq7}
\end{equation}
where $N^k_{n}$ is the number of negative samples.
The overall margin separation loss is summarized as follows:
\begin{equation}
    L_{ms} = L_{pos} + L_{neg}.
\end{equation}Through contrastive tasks, the point representations under varying rotations are further clustered within the same category, while the other categories are pushed away. A more compact and consistent representation space is formed, thereby enhancing the discriminability for rotated shape classification.

\begin{table*}[t] % table for OSDA setting on Office31
    \caption{Comparison of the macro-average precision score \textit{Avg.}~($\%$) under the orientation-aware 3D domain generalization setting. The value after $\pm$ denotes the standard deviation on the 64 evaluated series. RE, RI, DA and DG represent Rotation-Equivalent, Rotation-Invariant, 3D Domain Adaptation, and 3D Domain Generalization. The top 2 records are marked in \bc{red} and \rc{blue}, respectively.}
    \label{tab:pointda10_avg} 
    \vspace{-4mm}
    \small
    \begin{center}
    \setlength{\tabcolsep}{0.4cm}{
    \resizebox{1\textwidth}{!}{
    \begin{tabular}{l|c|l|l|l|l|l|l|c}
    \hline
    \multicolumn{9}{c}{DGCNN~\cite{wang2019dynamic}}\\
    \hline
    \makecell[c]{Methods}
    &Type & \makecell[c]{M$\to$S} & \makecell[c]{M$\to$S*} & \makecell[c]{S$\to$M} & \makecell[c]{S$\to$S*} & \makecell[c]{S*$\to$M} & \makecell[c]{S*$\to$S} & {AVG} \\


    \hline
    Supervised                                &\multirow{2}{*}{-}                  &{74.4 $\pm$ 5.5}   &{56.7 $\pm$ 3.9}   &{87.3 $\pm$ 14.5}   &{56.7 $\pm$ 3.9}  &{87.3 $\pm$ 14.5}  &{74.4 $\pm$ 5.5}    &{72.8}\\
    w/o Adapt                                 &                  &{54.7 $\pm$ 9.2}   &{25.3 $\pm$ 2.3}   &{57.6 $\pm$ 8.6}   &{27.9 $\pm$ 2.8}  &{45.3 $\pm$ 4.4}  &{37.1 $\pm$ 2.8}    &{41.3}    \\
    \hline
    VN~\cite{Deng_2021_ICCV}                  &\multirow{3}{*}{RE}                  &{61.2 $\pm$ 0.0}    &{27.3 $\pm$ 0.0} &{68.9 $\pm$ 0.0} &{23.5 $\pm$ 0.0} &{35.1 $\pm$ 0.0} &{31.5 $\pm$ 0.0} &{41.3}\\
    SVN~\cite{su2022svnet}                  &                  &{58.4 $\pm$ 0.7}    &{26.3 $\pm$ 0.6} &{64.9 $\pm$ 0.8}  &{23.6 $\pm$ 0.4} &{33.1 $\pm$ 0.8}    &{31.1 $\pm$ 0.8} &{39.6}\\
    EOMP~\cite{luo2022equivariant}            &                  &{56.7 $\pm$ 0.6}    &{27.0 $\pm$ 0.5} &{64.2 $\pm$ 0.9}  &{26.6 $\pm$ 0.5} &{25.6 $\pm$ 0.6}    &{26.9 $\pm$ 0.4} &{37.8}\\
    \hline
    SPRIN~\cite{you2021prin}              &\multirow{5}{*}{RI}             &{58.6 $\pm$ 0.7} &{27.3 $\pm$ 0.7}   &{73.9 $\pm$ 0.6} &{20.7 $\pm$ 0.9}    &{45.7 $\pm$ 0.7} &{37.3 $\pm$ 0.8}  &{43.9}\\
    RIPCA~\cite{li2021closer}              &            &{62.5 $\pm$ 1.2} &\rc{30.5 $\pm$ 1.2}   &{72.9 $\pm$ 0.8} &{27.2 $\pm$ 1.1}    &{47.5 $\pm$ 1.8} &{39.0 $\pm$ 1.2}  &\rc{46.6}\\
    RIConv++~\cite{zhang2022riconv}        &             &{38.9 $\pm$ 0.4} &{19.0 $\pm$ 0.6}   &{57.2 $\pm$ 0.8} &\rc{30.2 $\pm$ 1.0}    &{32.5 $\pm$ 0.7} &{38.4 $\pm$ 0.7}  &{36.0}\\ % special design for Pointnet++
    PaRI~\cite{chen2022devil}              &             &{40.9 $\pm$ 0.0} &{25.0 $\pm$ 0.5}   &{55.8 $\pm$ 0.7} &\bc{32.5 $\pm$ 0.7}    &{44.0 $\pm$ 1.0} &{39.4 $\pm$ 0.1}  &{39.6}\\
    LocoTrans~\cite{chen2024local}              &                  &\rc{65.2 $\pm$ 0.2}   &{30.4 $\pm$ 0.5}    &\rc{76.5 $\pm$ 0.4}  &{25.7 $\pm$ 0.4}  &{37.7 $\pm$ 0.7} &{26.9 $\pm$ 0.0}    &{43.7}\\ 
    \hline
    PointDAN~\cite{qin2019pointdan}           &\multirow{6}{*}{DA}                   &{50.5 $\pm$ 9.4}   &{27.5 $\pm$ 2.5}   &{58.9 $\pm$ 7.0}   &{26.1 $\pm$ 2.4}   &{36.6 $\pm$ 5.1}   &{34.7 $\pm$ 3.1}    &{39.1}    \\
    DefRec~\cite{achituve2021self}            &                   &{53.3 $\pm$ 9.2}   &{24.0 $\pm$ 3.6}   &{58.2 $\pm$ 8.0}   &{21.5 $\pm$ 2.9}   &{37.4 $\pm$ 6.3}   &{30.0 $\pm$ 3.0}    &{37.4}     \\
    GAST~\cite{zou2021geometry}               &                   &{37.9 $\pm$ 6.3}   &{20.7 $\pm$ 1.9}   &{47.4 $\pm$ 2.5}   &{14.7 $\pm$ 2.0}   &{28.9 $\pm$ 1.1}   &{26.0 $\pm$ 1.0}    &{29.3}     \\
    MLSP~\cite{liang2022point}                &                   &{60.9 $\pm$ 9.2}   &{30.0 $\pm$ 3.9}   &{62.6 $\pm$ 5.3}   &{23.4 $\pm$ 3.7}   &{44.9 $\pm$ 4.7}   &{37.3 $\pm$ 3.1}    &{43.2}     \\
    SDDA~\cite{cardace2023self}               &                   &{58.4 $\pm$ 9.8}   &\rc{30.5 $\pm$ 3.8}   &{64.4 $\pm$ 5.3}   &{25.9 $\pm$ 3.8}   &{40.1 $\pm$ 6.1}   &{39.5 $\pm$ 3.4}    &{43.1}    \\
    PCFEA~\cite{wang2024progressive}     &                   &{62.2 $\pm$ 10.3}   &{10.5 $\pm$ 0.2}   &{59.2 $\pm$ 8.7}   &{21.6 $\pm$ 2.4}   &\rc{48.0 $\pm$ 3.9}   &\rc{39.7 $\pm$ 3.9}    &{40.2}    \\
    \hline
    {Metasets~\cite{huang2021metasets}}       &\multirow{3}{*}{DG}                   &{47.4 $\pm$ 9.3}    &{14.0 $\pm$ 0.5} &{38.4 $\pm$ 12.2}  &{10.1 $\pm$ 0.8} &{21.8 $\pm$ 3.9}    &{19.6 $\pm$ 3.8} &{25.2}\\
    {PDG~\cite{wei2022learning}}              &                  &{31.9 $\pm$ 22.4}   &{24.4 $\pm$ 14.3}    &{43.8 $\pm$ 17.2}   &{14.0 $\pm$ 2.7}  &{30.6 $\pm$ 5.2}  &{30.5 $\pm$ 6.8}    &{29.2}\\
    {Ours}                                    &                  &\bb{66.2 $\pm$ 1.5}   &\bb{30.9 $\pm$ 2.2}   &\bb{81.7 $\pm$ 0.7}   &{30.1 $\pm$ 1.3}   &\bb{48.6 $\pm$ 2.3}  &\bb{39.8 $\pm$ 2.4}    &\bb{49.6}    \\
    \hline
    \hline
    \multicolumn{9}{c}{PointNet~\cite{qi2017pointnet}}\\
    \hline
    Supervised                    &\multirow{2}{*}{-}     &{70.1 $\pm$ 7.9}   &{40.9 $\pm$ 4.0}   &{86.4 $\pm$ 18.8}   &{40.9 $\pm$ 4.0}   &{86.4 $\pm$ 18.8}   &{70.1 $\pm$ 7.9}    &{65.8} \\
    w/o Adapt                     &                  &{51.8 $\pm$ 11.7}   &{21.9 $\pm$ 3.4}   &{67.5 $\pm$ 6.6}   &{24.5 $\pm$ 2.2}   &{33.8 $\pm$ 4.2}   &{32.9 $\pm$ 2.6}       &{38.7}   \\
    \hline
    VN~\cite{Deng_2021_ICCV}      &\multirow{2}{*}{RE}    &{57.3 $\pm$ 0.0}   &{22.0 $\pm$ 0.0}   &\rc{68.9 $\pm$ 0.0}   &{20.1 $\pm$ 0.0}   &\rc{34.5 $\pm$ 0.0}   &{26.3 $\pm$ 0.0}  &\rc{38.2} \\
    SVN~\cite{su2022svnet}      &    &{52.7 $\pm$ 0.2}   &{22.3 $\pm$ 0.3}   &{56.2 $\pm$ 0.2}   &{18.1 $\pm$ 0.0}   &{17.0 $\pm$ 0.3}   &{16.7 $\pm$ 0.2}  &{30.5} \\
    \hline
    
    % \hline
    PointDAN~\cite{qin2019pointdan} &\multirow{6}{*}{DA}   &{40.4 $\pm$ 10.4}   &{17.5 $\pm$ 2.5}   &{37.6 $\pm$ 7.4}   &{14.8 $\pm$ 3.5}   &{10.8 $\pm$ 0.5}   &{11.8 $\pm$ 0.8}    &{22.2}    \\
    DefRec~\cite{achituve2021self}  &                   &{34.8 $\pm$ 8.6}   &{15.0 $\pm$ 3.1}   &{35.3 $\pm$ 9.9}   &{14.3 $\pm$ 5.4}   &{13.9 $\pm$ 2.8}   &{11.8 $\pm$ 2.2}    &{20.9}     \\
    GAST~\cite{zou2021geometry}     &                   &{45.3 $\pm$ 4.1}   &\rc{24.6 $\pm$ 1.5}   &{53.6 $\pm$ 3.0}   &{20.3 $\pm$ 1.9}   &{21.4 $\pm$ 2.8}   &{24.4 $\pm$ 3.1}    &{31.6}     \\
    MLSP~\cite{liang2022point}      &                   &{44.2 $\pm$ 10.5}   &{21.8 $\pm$ 4.4}   &{46.6 $\pm$ 8.4}   &{18.3 $\pm$ 3.5}   &{20.2 $\pm$ 5.7}   &{19.0 $\pm$ 3.8}    &{28.4}     \\
    SDDA~\cite{cardace2023self}     &                   &\rc{58.4 $\pm$ 12.3}   &{19.0 $\pm$ 2.5}   &{55.0 $\pm$ 7.1}   &\rc{21.9 $\pm$ 3.7}   &{32.3 $\pm$ 6.9}   &{28.2 $\pm$ 4.5}    &{35.8}    \\
    PCFEA~\cite{wang2024progressive}     &                   &{50.2 $\pm$ 11.3}   &{10.4 $\pm$ 0.2}   &{44.8 $\pm$ 13.2}   &{10.4 $\pm$ 0.5}   &{17.1 $\pm$ 2.3}   &\rc{33.1 $\pm$ 3,2}    &{27.7}    \\
    \hline
    {Metasets~\cite{huang2021metasets}}&\multirow{3}{*}{DG}                   &{31.0 $\pm$ 7.3}    &{11.9 $\pm$ 0.9} &{32.5 $\pm$ 12.7}  &{10.0 $\pm$ 1.0} &{14.3 $\pm$ 4.5}    &{12.3 $\pm$ 2.4} &{18.7}\\
    {PDG~\cite{wei2022learning}}   &                  &{34.0 $\pm$ 11.5}   &{21.1 $\pm$ 3.0}    &{37.3 $\pm$ 17.8}   &{20.0 $\pm$ 4.2}  &{31.0 $\pm$ 8.9}  &{27.0 $\pm$ 5.3}    &{28.4}\\
    {Ours}                         &                    &\bb{63.3 $\pm$ 1.4}   &\bb{27.3 $\pm$ 1.3}   &\bb{81.4 $\pm$ 1.9}   &\bb{29.0 $\pm$ 1.2}   &\bb{42.7 $\pm$ 2.6}   &\bb{34.7 $\pm$ 1.5}       &\bb{46.4} \\
    \hline

    \end{tabular}

    }
    }
    \end{center}
    \vspace{-7mm}
\end{table*}



The entire framework is optimized in an alternative scheme between the intricate orientation mining and the orientation-aware contrastive training.
For the intricate orientation mining, the final objective is the same as Eq.~\ref{formula:iom}, where $L$ is the standard cross-entropy function over the originally aligned point clouds.
The intricate set is periodically updated after $T$ epochs of training to avoid overfitting to the current intricate rotations.
For orientation-aware contrastive training, the final loss function is
\begin{equation}
    L_{final} = L_{cls} + \lambda_{oc} L_{oc} + \lambda_{ms} L_{ms},
    \label{eq9}
\end{equation}
where $L_{cls}$ is the cross-entropy loss on the intricate point clouds. 

\section{Experiments}\label{sec:experiments}
We now evaluate SAPS using both qualitative and quantitative analyses. We first compare its zero-shot performance to R3L on benchmark tasks, then 
%perform delve into ablation studies and
an analysis of how our alignment approach behaves under different conditions.

\paragraph{Environments}
Our agents act by receiving pixel images as input observation, consisting of four consecutive $84 \times 84$ RGB images, stacked along the channel dimension to capture dynamic information such as velocity and acceleration.
We consider environments where we can freely change visual features (background color, camera perspective) or task (rewards, dynamics), therefore we use CarRacing \citep{klimov2016carracing} and LunarLander as both implemented in R3L.
CarRacing requires the agent to drive in a track using pixel observations, whose variations can be in the background color or the target speed, while LunarLander requires the agent to land on a platform, with variations comprising background color and different gravities.
% \AR{appendice per dettagli approfonditi su variazioni}.
No context is provided, hence the agents do not receive any information about the task.
% In the appendix we have other tests with atari env: \Cref{appendix:atari} \AR{riscrivi frase}

\paragraph{Baselines}
We mainly compare SAPS to (R3L), another zero-shot stitching method using relative representations whose approach is similar to ours.
For an additional baseline we also compare to naive zero-shot stitching, where we stitch encoders and controllers with no additional processing, to showcase the progress reached by the methods performing latent alignment techniques.
\section{Conclusion}
In this paper, we introduced Atom of Thoughts (\our), a novel framework that transforms complex reasoning processes into a Markov process of atomic questions. By implementing a two-phase transition mechanism of decomposition and contraction, \our eliminates the need to maintain historical dependencies during reasoning, allowing models to focus computational resources on the current question state. Our extensive evaluation across diverse benchmarks demonstrates that \our serves effectively both as a standalone framework and as a plug-in enhancement for existing test-time scaling methods. These results validate \our's ability to enhance LLMs' reasoning capabilities while optimizing computational efficiency through its Markov-style approach to question decomposition and atomic state transitions.



{\small
\bibliographystyle{ieee_fullname}
\bibliography{egbib}
}

\end{document} 