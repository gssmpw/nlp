\begin{abstract}

In this paper, we focus on 3D scene inpainting, where parts of an input image set, captured from different viewpoints, are masked out. The main challenge lies in generating plausible image completions that are geometrically consistent across views. Most recent work addresses this challenge by combining generative models with a 3D radiance field to fuse information across a relatively dense set of viewpoints. However, a major drawback of these methods is that they often produce blurry images due to the fusion of inconsistent cross-view images.  To avoid blurry inpaintings, we eschew the use of an explicit or implicit radiance field altogether and instead fuse cross-view information in a learned space.  In particular, we introduce a geometry-aware conditional generative model, capable of multi-view consistent inpainting using reference-based geometric and appearance cues. A key advantage of our approach over existing methods is its unique ability to inpaint masked scenes with a limited number of views (i.e., few-view inpainting), whereas previous methods require relatively large image sets for their 3D model fitting step.
Empirically, we evaluate and compare our scene-centric inpainting method
on two datasets, SPIn-NeRF and NeRFiller, which contain images captured at narrow and wide baselines, respectively, and achieve state-of-the-art 3D inpainting performance on both. Additionally, we demonstrate the efficacy of our approach in the few-view setting compared to prior methods.
Our project page is available at \href{https://geomvi.github.io/}{https://geomvi.github.io}.\end{abstract}
