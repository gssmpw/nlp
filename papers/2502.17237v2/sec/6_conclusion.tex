
\section{Conclusion and limitations}
\textit{So, is image retrieval for localization solved?}
Well, almost. While some datasets still show some room for improvement, we note that this is often due to either arguably unsolvable failure cases, wrong labels, and very few cases that can be solved by better models.
We emphasize however that this has been the case for some time, as previous DINO-v2-based models, like SALAD and CliqueMining, show very high results on classic VPR datasets.
What is still missing from literature is models like MegaLoc that achieve good results in a variety of diverse tasks and domains.

\textit{Should you always use MegaLoc?}
Well, almost, except for at least 3 use-cases.
MegaLoc has shown great results on a variety of related tasks, and, unlike other VPR models, achieves good results on landmark retrieval, which make it a great option also for retrieval for 3D reconstruction tasks, besides standard VPR and visual localization tasks.
However, experiments show that MegaLoc is outperformed by CliqueMining in MSLS, which is a dataset made of (almost entirely) forward facing images (\ie photos where the camera is facing the same direction of the street, instead of facing sideways towards the side of the street).
Another use case where MegaLoc is likely to be suboptimal is in very unusual natural environments, like forests or caves, where instead AnyLoc has been shown to work well \cite{Keetha_2023_AnyLoc}.
A third and final use case where other models might be preferred to MegaLoc is for embedded systems, where one might opt for more lightweight models, like the ResNet-18 \cite{He_2016_resnet} versions of CosPlace \cite{Berton_2022_cosPlace}, which has 11M parameters instead of MegaLoc's 228M.
