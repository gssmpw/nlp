The sequential generalized sliding-tile puzzle (SGSTP) is a generalization of the classic 15-Tile Sliding Tile Puzzle (Appendix \ref{sec:geoms}: Figure \ref{fig:STP_and_geom_vis}). In the SGSTP, a set of \( n < m_1 \times m_2 \) tiles, each uniquely labeled \( 1, \dots, n \), are placed on a rectangular grid of size \( m_1 \times m_2 \), denoted by \( G = (V, E) \). The grid has \( m_1 \times m_2 - n \) empty positions that allow tile movement.  

A configuration of tiles is represented as an injective mapping from the set \( \{1, \dots, n\} \) to positions \( V = \{(v_x, v_y) : 1 \leq v_x \leq m_2, 1 \leq v_y \leq m_1 \} \). Each tile must be repositioned from an arbitrary initial configuration \( S = \{s_1, \dots, s_n\} \) to a specified goal configuration \( G = \{g_1, \dots, g_n\} \), such as an ordered row-major layout.  

Let the movement path of tile \( i \), where \( 1 \leq i \leq n \), be expressed as \( p_i : \mathbb{N}_0 \to V \). The puzzle seeks a set of feasible paths \( P = \{p_1, \dots, p_n\} \) that satisfy the following conditions for all \( 1 \leq i, j \leq n \) with \( i \neq j \), and for all time steps \( t \geq 0 \):  

\textbf{Incremental Movement:} \( p_i(t+1) = p_i(t) \text{ or } (p_i(t+1), p_i(t)) \in E \).  
    Tiles move to adjacent, unoccupied positions or stay still.  \\
\textbf{Goal Achievement:} \( p_i(0) = s_i \text{ and } p_i(T) = g_i \text{ for some } T \geq 0 \). 
    Each tile must start at \( s_i \) and reach \( g_i \). \\ 
\textbf{Exclusive Occupancy:} \( p_i(t) \neq p_j(t) \text{ for all } i \neq j \). 
    Two tiles cannot occupy the same position at the same time.  

In this sequential version, tiles move one at a time. Therefore, the head-on collision and corner-following constraints found in the generalized sliding-tile puzzle are omitted, as simultaneous tile movements are not permitted.