\textbf{Complexity scalability}\hspace{.3cm} GSTP is a well-known NP-hard problem due to the need for multi-step planning across a constrained grid \cite{gozon2024computing}. SGP inherits this complexity but introduces greater flexibility in scaling difficulty without altering the game’s core mechanics. This flexibility provides more degrees of freedom, making the task more tractable for VLM agents. Key scaling factors include board size, number of objects, object variability, length of the shortest path solution, and the geom interference factor (see Figure \ref{fig:SGP_scalibility}). The shortest path solution for all episode configurations is calculated using the A* algorithm \cite{hart1968formal}, as detailed in Appendix \ref{sec:optimal_agent}. The interference factor denotes the extent to which objects obstruct one another’s optimal paths, increasing the global solution length beyond the cumulative Manhattan distances of individual paths. This interference can create configurations with short optimal paths but increased planning requirements, significantly raising the problem's difficulty. Available geometric shapes include [“cube,” “pyramid,” “sphere,” “cylinder,” “cone,” “prism”], with colors freely selectable by referencing RGB values. Agents must navigate combinatorial complexity by matching shapes and colors, promoting spatial strategies over the sequential patterns seen in numerical tile puzzles. Episode configurations are randomly generated, requiring models to generalize across tasks. Human and algorithmic benchmarks for these experiments are detailed in Section \ref{sec:baselines}.

