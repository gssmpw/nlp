While VLMs generally demonstrate an effective understanding of how to interact with the environment\footnote{Videos of agents' interactions with iVISPAR available at: 
\href{https://youtu.be/Djis_xkgtW8}{https://youtu.be/Djis\_xkgtW8}.}, as evidenced by low rates of illegal commands, the number of moves across categories varies significantly both between VLMs and within VLMs across modalities (Figure \ref{fig:barplot_action_counts}). This variability highlights challenges in optimal path planning and recognizing invalid successor states, such as occupied destination and out-of-bounds moves, which differ across modalities (Appendix \ref{sec:additional_graphs}: Figure \ref{fig:move_validity_double}). Frequent errors in these categories point to limitations in reasoning about neighboring tiles, either due to an inability to accurately detect obstructed spaces or insufficient precision in localizing the geoms to be moved. A high number of effective actions may indicate a strong understanding of efficient episode-solving strategies; however, if accompanied by a high frequency of ineffective moves, it may instead reflect poor understanding of how to progress effectively toward the goal state. Additionally, high rates of ineffective moves in vision 3D suggest that while VLMs can complete some games, they often struggle to predict the resulting states of their actions accurately.