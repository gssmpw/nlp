
% \documentclass[acmlarge, manuscript, anonymous=true, screen, review]{acmart}
% \documentclass[sigconf]{acmart}
\documentclass[acmsmall]{acmart}

\usepackage[utf8]{inputenc}

\PassOptionsToPackage{table,xcdraw}{xcolor} 

\usepackage{makecell}
\usepackage{textgreek}
\usepackage{multirow}
\usepackage{subcaption}
\usepackage{xcolor,colortbl}
\definecolor{gray}{rgb}{0.1,0.1,0.1}
\usepackage[T1]{fontenc}
\usepackage{graphicx}
\usepackage{tabularx}
\usepackage{longtable}
\usepackage{enumitem}
\usepackage{booktabs}
\usepackage{array}
\usepackage{setspace}
% accepted packages: https://authors.acm.org/proceedings/production-information/accepted-latex-packages

\AtBeginDocument{%
  \providecommand\BibTeX{{%
    \normalfont B\kern-0.5em{\scshape i\kern-0.25em b}\kern-0.8em\TeX}}}

\acmConference[Woodstock '18]{Woodstock '18: ACM Symposium on Neural
  Gaze Detection}{June 03--05, 2018}{Woodstock, NY}
\acmBooktitle{Woodstock '18: ACM Symposium on Neural Gaze Detection,
  June 03--05, 2018, Woodstock, NY}
\acmPrice{15.00}
\acmISBN{978-1-4503-XXXX-X/18/06}

\acmSubmissionID{123-A56-BU3}


\usepackage{times}
\usepackage{latexsym}

\usepackage[T1]{fontenc}

\usepackage[utf8]{inputenc}

\usepackage{microtype}

\usepackage{inconsolata}

\usepackage{graphicx}


\usepackage{amsmath}
\usepackage{amssymb}
\usepackage{multirow}
\usepackage{booktabs}
\usepackage{catchfile}

\usepackage[boxed]{algorithm}
\usepackage{varwidth}
\usepackage[noEnd=true,indLines=false]{algpseudocodex}
\usepackage{cleveref}
\makeatletter
\@addtoreset{ALG@line}{algorithm}
\renewcommand{\ALG@beginalgorithmic}{\small}
\algrenewcommand\alglinenumber[1]{\small #1:}
\makeatother

\usepackage[normalem]{ulem}
\usepackage{todonotes}

\usepackage{lipsum}    %
\usepackage{comment}   %
\usepackage{graphicx}  %
\usepackage{pifont}    %

\usepackage[font=small,labelfont=bf]{caption}
\usepackage{float}     %
\usepackage{booktabs}  %
\usepackage{subcaption}  %

\usepackage{listings}

\usepackage{amsthm}  %


\begin{document}
\title[{Social Media Isn't Just Instagram:\\A Youth-Envisioned Platform for Meaningful Social Connections}]{Social Media Isn't Just Instagram: A Youth-Envisioned Platform for Meaningful Social Connections}

\author{JaeWon Kim}
\orcid{0000-0003-4302-3221}
\affiliation{%
  \institution{University of Washington}
  \city{Seattle}
  \state{WA}
  \country{USA}}
\email{jaewonk@uw.edu}

\author{Hyunsung Cho}
\orcid{0000-0002-4521-2766}
\affiliation{%
  \institution{Human-Computer Interaction Institute, Carnegie Mellon University}
  \city{Pittsburgh}
  \country{USA}}
\email{hyunsung@cs.cmu.edu}

\author{Fannie Liu}
\orcid{0000-0002-5656-3406}
\affiliation{%
  \institution{JPMorgan Chase \& Co.}
  \city{New York}
  \state{NY}
  \country{USA}
}
\email{fannie.liu@jpmchase.com}

\author{Alexis Hiniker}
\orcid{0000-0003-1607-0778}
\affiliation{%
  \institution{The Information School, University of Washington}
  \city{Seattle}
  \country{USA}}
\email{alexisr@uw.edu}

\renewcommand{\shortauthors}{JaeWon Kim, et al.}

\begin{abstract}
We conducted co-design workshops with 23 participants (ages 15–24) to explore how youth envision an ideal remote social connection. Using the Fictional Inquiry (FI) method within a Harry Potter-inspired narrative, we found that youth perceive a disconnect between platforms labeled as ``social media'' (like Instagram) and those where they actually experience meaningful connections (like Minecraft or Discord). Participants envisioned an immersive 3D platform that would bridge this gap by prioritizing meaningful social connections, enabling time that feels well spent through presence and immersion, natural individual expression, intuitive social navigation that leverages physical-world norms, and playful, low-stakes opportunities for gradual friendship development. We introduce the design framework of \textit{spatial integrity}, which encompasses four dimensions of spatial affordances that facilitate meaningful social connections online. The FI method proved effective in generating innovative ideas while empowering youth by fostering a sense of hope and agency over the future of social media through their own design contributions.
\end{abstract}

\begin{CCSXML}
<ccs2012>
   <concept>
       <concept_id>10003120.10003130</concept_id>
       <concept_desc>Human-centered computing~Collaborative and social computing</concept_desc>
       <concept_significance>500</concept_significance>
       </concept>
 </ccs2012>
\end{CCSXML}

\ccsdesc[500]{Human-centered computing~Collaborative and social computing}

\keywords{social media; fictional inquiry; youth; design}

\maketitle

\section{Introduction}

Video generation has garnered significant attention owing to its transformative potential across a wide range of applications, such media content creation~\citep{polyak2024movie}, advertising~\citep{zhang2024virbo,bacher2021advert}, video games~\citep{yang2024playable,valevski2024diffusion, oasis2024}, and world model simulators~\citep{ha2018world, videoworldsimulators2024, agarwal2025cosmos}. Benefiting from advanced generative algorithms~\citep{goodfellow2014generative, ho2020denoising, liu2023flow, lipman2023flow}, scalable model architectures~\citep{vaswani2017attention, peebles2023scalable}, vast amounts of internet-sourced data~\citep{chen2024panda, nan2024openvid, ju2024miradata}, and ongoing expansion of computing capabilities~\citep{nvidia2022h100, nvidia2023dgxgh200, nvidia2024h200nvl}, remarkable advancements have been achieved in the field of video generation~\citep{ho2022video, ho2022imagen, singer2023makeavideo, blattmann2023align, videoworldsimulators2024, kuaishou2024klingai, yang2024cogvideox, jin2024pyramidal, polyak2024movie, kong2024hunyuanvideo, ji2024prompt}.


In this work, we present \textbf{\ours}, a family of rectified flow~\citep{lipman2023flow, liu2023flow} transformer models designed for joint image and video generation, establishing a pathway toward industry-grade performance. This report centers on four key components: data curation, model architecture design, flow formulation, and training infrastructure optimization—each rigorously refined to meet the demands of high-quality, large-scale video generation.


\begin{figure}[ht]
    \centering
    \begin{subfigure}[b]{0.82\linewidth}
        \centering
        \includegraphics[width=\linewidth]{figures/t2i_1024.pdf}
        \caption{Text-to-Image Samples}\label{fig:main-demo-t2i}
    \end{subfigure}
    \vfill
    \begin{subfigure}[b]{0.82\linewidth}
        \centering
        \includegraphics[width=\linewidth]{figures/t2v_samples.pdf}
        \caption{Text-to-Video Samples}\label{fig:main-demo-t2v}
    \end{subfigure}
\caption{\textbf{Generated samples from \ours.} Key components are highlighted in \textcolor{red}{\textbf{RED}}.}\label{fig:main-demo}
\end{figure}


First, we present a comprehensive data processing pipeline designed to construct large-scale, high-quality image and video-text datasets. The pipeline integrates multiple advanced techniques, including video and image filtering based on aesthetic scores, OCR-driven content analysis, and subjective evaluations, to ensure exceptional visual and contextual quality. Furthermore, we employ multimodal large language models~(MLLMs)~\citep{yuan2025tarsier2} to generate dense and contextually aligned captions, which are subsequently refined using an additional large language model~(LLM)~\citep{yang2024qwen2} to enhance their accuracy, fluency, and descriptive richness. As a result, we have curated a robust training dataset comprising approximately 36M video-text pairs and 160M image-text pairs, which are proven sufficient for training industry-level generative models.

Secondly, we take a pioneering step by applying rectified flow formulation~\citep{lipman2023flow} for joint image and video generation, implemented through the \ours model family, which comprises Transformer architectures with 2B and 8B parameters. At its core, the \ours framework employs a 3D joint image-video variational autoencoder (VAE) to compress image and video inputs into a shared latent space, facilitating unified representation. This shared latent space is coupled with a full-attention~\citep{vaswani2017attention} mechanism, enabling seamless joint training of image and video. This architecture delivers high-quality, coherent outputs across both images and videos, establishing a unified framework for visual generation tasks.


Furthermore, to support the training of \ours at scale, we have developed a robust infrastructure tailored for large-scale model training. Our approach incorporates advanced parallelism strategies~\citep{jacobs2023deepspeed, pytorch_fsdp} to manage memory efficiently during long-context training. Additionally, we employ ByteCheckpoint~\citep{wan2024bytecheckpoint} for high-performance checkpointing and integrate fault-tolerant mechanisms from MegaScale~\citep{jiang2024megascale} to ensure stability and scalability across large GPU clusters. These optimizations enable \ours to handle the computational and data challenges of generative modeling with exceptional efficiency and reliability.


We evaluate \ours on both text-to-image and text-to-video benchmarks to highlight its competitive advantages. For text-to-image generation, \ours-T2I demonstrates strong performance across multiple benchmarks, including T2I-CompBench~\citep{huang2023t2i-compbench}, GenEval~\citep{ghosh2024geneval}, and DPG-Bench~\citep{hu2024ella_dbgbench}, excelling in both visual quality and text-image alignment. In text-to-video benchmarks, \ours-T2V achieves state-of-the-art performance on the UCF-101~\citep{ucf101} zero-shot generation task. Additionally, \ours-T2V attains an impressive score of \textbf{84.85} on VBench~\citep{huang2024vbench}, securing the top position on the leaderboard (as of 2025-01-25) and surpassing several leading commercial text-to-video models. Qualitative results, illustrated in \Cref{fig:main-demo}, further demonstrate the superior quality of the generated media samples. These findings underscore \ours's effectiveness in multi-modal generation and its potential as a high-performing solution for both research and commercial applications.
\section{Related Work}
\subsection{Defining Social Media and its Role in Young People's Lives}
Social media does not yet have a widely agreed-upon definition. There is a range of interpretations, from broad (e.g., ``any interactive communication medium that enables two-way interaction and feedback''~\cite{kent2010directions}) to specific (e.g., ``a different kind of Internet-based applications which build the ideological and technological foundations of Web 2.0, and allow users to create content and exchange it with other people through the Internet''~\cite{kaplan2010users}). While some platforms, such as Instagram~\cite{instagram} or Facebook~\cite{facebook-official}, are indisputably considered social media, others, such as YouTube~\cite{youtube} and Discord~\cite{discord}, are less archetypal. There are also different types of social media, and even such classifications have multiple interpretations~\cite{kaplan2010users, sharma2018social, scott2015new}. Some major categories include social networking sites (i.e., ``web-based services that allow individuals to (1) construct a public or semi-public profile within a bounded system, (2) articulate a list of other users with whom they share a connection, and (3) view and traverse their list of connections and those made by others within the system''~\cite{Boyd-2007-SocialNetworkScholarship-i}) and blogs (i.e., ``a way to share that love with the world and encourage an active community of readers who comment on the posts of the author''~\cite{scott2015new}).

% \noindent \textbf{What is social media?} Social media does not yet have a definitive definition. There is a range of interpretations, from broad~\footnote{``any interactive communication medium that enables two-way interaction and feedback''~\cite{kent2010directions})} to specific~\footnote{``a different kind of Internet-based applications which build the ideological and technological foundations of Web 2.0, and allow users to create content and exchange it with other people through the Internet''~\cite{kaplan2010users}}. While some platforms, such as Instagram~\cite{instagram} or Facebook~\cite{facebook-official}, are indisputably considered social media, others, such as YouTube~\cite{youtube} and Discord~\cite{discord}, are less archetypal. There are also different types of social media, and even such classifications have multiple interpretations~\cite{kaplan2010users, sharma2018social, scott2015new}. Some major categories include social networking sites~\footnote{``web-based services that allow individuals to (1) construct a public or semi-public profile within a bounded system, (2) articulate a list of other users with whom they share a connection, and (3) view and traverse their list of connections and those made by others within the system''~\cite{Boyd-2007-SocialNetworkScholarship-i}} and blogs~\footnote{``a way to share that love with the world and encourage an active community of readers who comment on the posts of the author''~\cite{scott2015new}}.

The range of definitions and classifications of social media reflects the wide array of user needs that social media addresses. From a uses and gratifications perspective~\cite{GurevitchKatz-1973-UsesGratificationsResearch-y}, social media supports a variety of utilities, ranging from social interactions to information seeking and entertainment~\cite{whiting2013people}. Even within such broad categories of uses and gratifications, there are diverse ways these needs are pursued. For example, within social interactions, there may be a focus on bridging social capital and weaker ties versus bonding social capital and stronger ties~\cite{phua_uses_2017}. Further, even on a single platform---Instagram, for instance---users engage in a range of activities, from photo sharing for self-promotion and disclosure~\cite{Menon-2022-UsesGratificationsInstagram-x} to more passive consumption of media for entertainment. In other words, Instagram may serve as a personal branding tool for some while functioning as a way to pass the time for others.

% \parHeading{Youth experience with social media} Despite the prevalent narrative that social media is bad, research shows the youth experience with social media is more nuanced. While younger people have bad experiences on social media such as social comparison~\cite{Yau2019-ab}, significant privacy concerns, pressures to curate~\cite{BoydDanah2014ICTS}, time spent feeling meaningless~\cite{HinikerLukoff-2018-WhatMakes-o}, and more mild forms of discomfort~\cite{Landesman-2024-IInstagram-j}, they experience meaningful benefits from those platforms as well. For instance, marginalized populations have been known to benefit particularly from these sites~\cite{AcenaLi-2023-WeReality-s, BellBates-2020-"LetMeDevelopment-u} and these platforms are used as medium for identity development and expression~\cite{Davis2012-bq, Lee-2022-AlgorithmicCrystalTikTok-b}. Given such a nuanced relationship between social media and youth, researchers recommend that it is beneficial to help teens learn to be more resilient with social media experiences rather than focusing on restricting or preventing harm~\cite{Wisniewski2012-nu}.

Despite the prevalent narrative that social media is harmful, the experiences of youth with these platforms are more nuanced. They face challenges such as social comparison~\cite{Yau2019-ab}, significant privacy concerns~\cite{Weinstein2022-rh, kim2024privacysocialnormsystematically, Boyd_2014}, pressures to curate their online presence~\cite{Yau2019-ab}, and feelings of meaningless time spent~\cite{HinikerLukoff-2018-WhatMakes-o}, along with more subtle discomforts~\cite{Landesman-2024-IInstagram-j}. However, these platforms also offer meaningful benefits. For instance, marginalized populations often derive substantial support from these spaces~\cite{AcenaLi-2023-WeReality-s, BellBates-2020-"LetMeDevelopment-u}, and social media serves as an important medium for identity development and self-expression~\cite{Davis2012-bq, Lee-2022-AlgorithmicCrystalTikTok-b}. Given this complex relationship between youth and social media, researchers advocate for strategies that focus on building resilience and equipping teens to navigate these experiences effectively rather than solely aiming to restrict or mitigate harm~\cite{Wisniewski2012-nu, Wisniewski-2018-PrivacyParadox-l, WisniewskiAgha-2023-StrikePrevention-r}.

% paper on Time spent on smartphone feels meaningless vs. meaningful (Kai)

% papers that talk about The bad side
% - correlation between youth social media use and mental health (though not causal)
% - social comparison, mental health, FOMO, filtered and curated posts, only positive ones
% - more mild
% - context collapse -> can't authentically share self

% papers that talk about The good side
% - queer, psychiatric hospitalization
% - identity crystallization
% - political activism
% - emotion regulation

% importance of social media --> shouldn't just ban/restrict --> need to teach resilience



\subsection{Social Technology Design Explorations in HCI}
% HCI research has explored different designs that foster meaningful social interactions online. Ranging from social media platforms such as BeReal~\cite{HinikerKim-2024-SharingDesign-x}, Snapchat~\cite{Bayer2016-fa}, Miitomo~\cite{KaufmanKasunic-2017-BeMeApplication-h}, and TikTok~\cite{Barta2021-yh, Schaadhardt2023-jj} to individual features such as those that support effortful communication~\cite{LiuFannie2021SOUt, LiuZhang-2022-AuggieEncouragingExperiences-n} or genuine connection~\cite{Stepanova2022-vh}. Recently more spatial social technologies such as social virtual reality (VR) platforms~\cite{Zamanifard-2023-SurpriseBirthdayDistance-n, Freeman-2024-MyAudiences-u, Freeman-2020-MyBodyReality-l, Zamanifard-2019-TogethernessCraveRelationships-s}, Gather.town~\cite{duarte2023experience,tu2022meetings}, and VRChat~\cite{chen2024d,deighan2023social,rzeszewski2024social} have been explored.

HCI research has extensively explored designs that foster meaningful social interactions online, examining both platform-level innovations and individual features. Social media platforms such as BeReal~\cite{HinikerKim-2024-SharingDesign-x}, Snapchat~\cite{Bayer2016-fa}, Miitomo~\cite{KaufmanKasunic-2017-BeMeApplication-h}, and TikTok~\cite{Barta2021-yh, Schaadhardt2023-jj} have introduced unique affordances for sharing and connecting, with an emphasis on spontaneity, ephemerality, and creative self-expression. This line of research identified different ways to build and maintain relationships through features that prioritize authenticity and playful interaction. However, users have reported increasingly toxic interactions as they gain popularity, with design choices encouraging users to expand their networks and engage in compulsive behaviors, such as tracking Snap scores~\cite{chambers2022s, van2023snapchat}, which detract from their initial goals of fostering meaningful interactions. 

At the feature level, efforts to support meaningful communication include designs that encourage effortful communication~\cite{LiuFannie2021SOUt, LiuZhang-2022-AuggieEncouragingExperiences-n} and those that foster feelings of genuine connections~\cite{Stepanova2022-vh}. While these features are important, social media experiences are often holistic, involving a complex interplay of multiple design aspects and user behaviors. As a result, such features alone cannot fundamentally alter the overarching narratives or perspectives surrounding social media.

Beyond traditional social media, recent work has investigated spatial social technologies, which emphasize the value of immersive, embodied interactions. Social virtual reality (VR) platforms~\cite{Zamanifard-2023-SurpriseBirthdayDistance-n, Freeman-2024-MyAudiences-u, Freeman-2020-MyBodyReality-l, Zamanifard-2019-TogethernessCraveRelationships-s} provide users with virtual spaces and embodiment for co-presence and shared experiences, facilitating interactions that feel more personal and tangible compared to 2D environments. Similarly, platforms like Gather.town~\cite{duarte2023experience,tu2022meetings} and VRChat~\cite{chen2024d,deighan2023social,rzeszewski2024social} blend elements of gaming and social networking, offering users the ability to navigate virtual environments, customize avatars, and participate in dynamic, context-rich group interactions. However, these platforms often are not generally recognized as social media in the traditional sense.

What youth envision as ideal social media, both at the broader landscape level and the specific feature level remains unclear---especially when considering the possibilities offered by emerging technologies such as spatial computing platforms. Understanding these perspectives is crucial for reimagining the future of social media and ensuring it meets the evolving needs of its users.

% snapscore - \cite{chambers2022s}

% \cite{Stepanova2022-vh}
\subsection{Designing Beyond Doom Narratives}
\parHeading{Social media and design fixation} Design fixation~\cite{jansson1991design}---the tendency to remain constrained by conventional thinking or existing solutions---is a well-documented challenge in design. While generating a wide range of divergent ideas is a critical first step in innovative design, we often impose unnecessary restrictions on our thinking. This self-limiting cognitive process often confines the design process to address localized problems rather than exploring broader, more optimal solutions.

Social media design is no exception. Given the ad-based revenue model, platforms are incentivized to increase time spent on their apps. However, instead of achieving this through meaningful utility and engagement that genuinely enhances users' experiences, they often resort to less healthy, compulsive strategies, such as friend recommendations~\cite{HinikerKim-2024-SharingDesign-x} or engagement metrics (e.g., Snap scores~\cite{rozgonjuk2021comparing}). These tactics drive superficial interactions rather than fostering deeper connections, leaving users dissatisfied and reinforcing the perception that all social media designs and user experiences are fundamentally the same~\cite{Sundaram-Other-SocialMediaSame-g, Pardes-2020-SocialMediaSame-p}.

\parHeading{The Fictional Inquiry (FI) method} FI is an exploratory co-design method that enables participants to transcend real-world constraints through immersion in semi-fictional contexts~\cite{IversenDindler-2007-FictionalInquiry--designSpace-m}. At its core, FI creates an environment where people can imagine beyond present limitations by engaging with carefully crafted scenarios. These scenarios typically incorporate immersive elements such as physical artifacts or role-playing activities, with participants taking on generative roles that help them envision futuristic or fictional settings. The method is particularly effective at helping participants overcome design fixation by encouraging them to draw analogies from fictional scenarios and imagine freely outside real-world constraints. While FI shares the exploratory nature of other ideation techniques, it is distinguished by its emphasis on developing futuristic design materials through collaborative dialogue between designers and users~\cite{IversenDindler-2007-FictionalInquiry--designSpace-m}.

According to Dindler and Iversen~\cite{IversenDindler-2007-FictionalInquiry--designSpace-m}, implementing this method involves three key steps: first, clearly defining the inquiry's purpose, whether it's for staging design situations, exploring future ideas, or driving organizational change; second, developing an appropriate narrative that participants are comfortable with while being distinct enough from current practices to encourage new thinking; and third, creating a compelling plot that introduces tension or conflict within the narrative to motivate specific workshop activities. Unlike traditional role-playing approaches, FI encourages participants to maintain their own identities and expertise while operating within fictional contexts, allowing them to explore new possibilities while drawing from their personal motivations and capabilities, all while circumventing typical societal constraints. 
\section{Study Design}
% robot: aliengo 
% We used the Unitree AlienGo quadruped robot. 
% See Appendix 1 in AlienGo Software Guide PDF
% Weight = 25kg, size (L,W,H) = (0.55, 0.35, 06) m when standing, (0.55, 0.35, 0.31) m when walking
% Handle is 0.4 m or 0.5 m. I'll need to check it to see which type it is.
We gathered input from primary stakeholders of the robot dog guide, divided into three subgroups: BVI individuals who have owned a dog guide, BVI individuals who were not dog guide owners, and sighted individuals with generally low degrees of familiarity with dog guides. While the main focus of this study was on the BVI participants, we elected to include survey responses from sighted participants given the importance of social acceptance of the robot by the general public, which could reflect upon the BVI users themselves and affect their interactions with the general population \cite{kayukawa2022perceive}. 

The need-finding processes consisted of two stages. During Stage 1, we conducted in-depth interviews with BVI participants, querying their experiences in using conventional assistive technologies and dog guides. During Stage 2, a large-scale survey was distributed to both BVI and sighted participants. 

This study was approved by the University’s Institutional Review Board (IRB), and all processes were conducted after obtaining the participants' consent.

\subsection{Stage 1: Interviews}
We recruited nine BVI participants (\textbf{Table}~\ref{tab:bvi-info}) for in-depth interviews, which lasted 45-90 minutes for current or former dog guide owners (DO) and 30-60 minutes for participants without dog guides (NDO). Group DO consisted of five participants, while Group NDO consisted of four participants.
% The interview participants were divided into two groups. Group DO (Dog guide Owner) consisted of five participants who were current or former dog guide owners and Group NDO (Non Dog guide Owner) consisted of three participants who were not dog guide owners. 
All participants were familiar with using white canes as a mobility aid. 

We recruited participants in both groups, DO and NDO, to gather data from those with substantial experience with dog guides, offering potentially more practical insights, and from those without prior experience, providing a perspective that may be less constrained and more open to novel approaches. 

We asked about the participants' overall impressions of a robot dog guide, expectations regarding its potential benefits and challenges compared to a conventional dog guide, their desired methods of giving commands and communicating with the robot dog guide, essential functionalities that the robot dog guide should offer, and their preferences for various aspects of the robot dog guide's form factors. 
For Group DO, we also included questions that asked about the participants' experiences with conventional dog guides. 

% We obtained permission to record the conversations for our records while simultaneously taking notes during the interviews. The interviews lasted 30-60 minutes for NDO participants and 45-90 minutes for DO participants. 

\subsection{Stage 2: Large-Scale Surveys} 
After gathering sufficient initial results from the interviews, we created an online survey for distributing to a larger pool of participants. The survey platform used was Qualtrics. 

\subsubsection{Survey Participants}
The survey had 100 participants divided into two primary groups. Group BVI consisted of 42 blind or visually impaired participants, and Group ST consisted of 58 sighted participants. \textbf{Table}~\ref{tab:survey-demographics} shows the demographic information of the survey participants. 

\subsubsection{Question Differentiation} 
Based on their responses to initial qualifying questions, survey participants were sorted into three subgroups: DO, NDO, and ST. Each participant was assigned one of three different versions of the survey. The surveys for BVI participants mirrored the interview categories (overall impressions, communication methods, functionalities, and form factors), but with a more quantitative approach rather than the open-ended questions used in interviews. The DO version included additional questions pertaining to their prior experience with dog guides. The ST version revolved around the participants' prior interactions with and feelings toward dog guides and dogs in general, their thoughts on a robot dog guide, and broad opinions on the aesthetic component of the robot's design. 


\section{Result}

\subsection{RQ1: What are the design principles and initial prototype characteristics of Echo-Teddy?}

\subsubsection{Design principles of Echo-Teddy}


The design principles of Echo-Teddy are structured to comprehensively support social and social emotional development in autistic children, focusing on four key themes: Potential User, Ethical Consideration, Customization, and Usage. The robot is designed to function as both a peer and an assistant, accommodating the neurodiverse characteristics of autistic children, including gaze aversion, preference for structured patterns, heightened perceptual sensitivity, and attention to hierarchical information and fine details. By integrating these elements, Echo-Teddy creates an interactive, supportive, and engaging experience tailored to the needs of its users.

Echo-Teddy’s verbal and behavioral output aligns with best practices in autism support, ensuring that its speech patterns are age-appropriate and incorporate evidence-based teaching strategies. These strategies reinforce positive behaviors while mitigating interfering behaviors. The robot is designed to elicit and encourage target verbal and behavioral responses, offering structured interactions while maintaining the flexibility needed to adapt to individual learning preferences. Unlike humanoid robots, Echo-Teddy features a soft, fur-covered exterior, a design choice aimed at reducing anxiety and increasing comfort for autistic children who may find highly realistic or human-like robotic faces overwhelming.

Ethical considerations are central to Echo-Teddy’s development. The robot consistently uses positive reinforcement strategies to prevent frustration and enhance engagement. Its speech and behavioral responses dynamically adjust to accommodate each child's individual needs and comfort levels. A key objective in its design is to minimize stressors while fostering an environment that encourages meaningful interaction.

Customization plays a pivotal role in Echo-Teddy’s adaptability. Caregivers and educators can tailor its communication style, interaction topics, and behavioral prompts to align with each child’s unique preferences and developmental goals. This customization extends beyond software, as the robot’s physical appearance can be modified to better suit individual sensory and aesthetic preferences.

Practical usability is another core aspect of Echo-Teddy’s design. The robot is built to be durable and resilient, capable of withstanding minor impacts or water exposure, ensuring longevity in varied educational and home environments. Additionally, Echo-Teddy is designed for independent operation, eliminating the need for constant human intervention and allowing children to interact with it autonomously.

By integrating neurodiverse-friendly interaction models, ethical safeguards, extensive customization options, and practical durability, Echo-Teddy is designed to be an effective and accessible tool for enhancing social communication skills in autistic children. These principles ensure that Echo-Teddy is not only tailored to the unique needs of its users but also remains ethically responsible, adaptable, and functional in real-world applications.

\begin{table*}[hbt!]
\scriptsize
\begin{tabular}{p{2cm} p{3cm} p{7cm}}
\hline
\multicolumn{1}{c}{\textbf{Themes}} & \multicolumn{1}{c}{\textbf{Categories}}   & \multicolumn{1}{c}{\textbf{Design Principles}}                                                                                                   \\
\hline
Potential User                      & Purpose of the Robot                      & This robot should mainly improve the social/socio-emotional skills of autistic children by performing social communication and interaction. \\
                                    &                                           & The robot should act like a facilitator, including peers and assistants.                                                                         \\
                                    & Characteristics of autistic students      & The robot should be designed considering the neurodiverse characteristics of autistic children. (gaze aversion, pattern recognition, perceptual processing, and exceptional focus for paticular topic).
                                    \\
                                    & Output of the robot (Verbal)              & Keep the utterance style and length appropriate to a child of the same age as the user.                                                          \\
                                    &                                           & The robot should activate verbal teaching strategies to induce positive behavior and reduce interfering behaviors.                               \\
                                    &                                           & The robots should be able to elicit the target verbal behavior in children with autism.                                                             \\
                                    & Output of the robot (Behavioral)          & The robot should activate behavioral teaching strategies to induce positive behavior and reduce interfering behaviors.                           \\
                                    &                                           & The robots should be able to elicit the target behavior in children with autism.                                                                    \\
                                    & Appearance of the robot                   & The appearance of the robot should be non-humanoid.                                                                                              \\
\hline
Ethical Consideration               &                                           & Avoid frustration by using positive feedback.                                                                                                    \\
                                    &                                           & Reflect the unique needs of the user in your speech and actions.                                                                                 \\
                                    &                                           & Minimize stress sources and make the participants comfortable.                                                                                   \\
\hline
Customization                      & Preference of the robot                   & The preference of the robot should be customized by the caregiver or instructor of the user.                                                     \\
                                    & Output of the robot (Verbal)              & The subject of the communication should be customized.                                                                                           \\
                                    & Output of the robot (Behavioral)          & The set of behaviors to be stimulated in the child should be customized by the caregiver or the instructor.                                       \\
                                    & Appearance of the robot                   & The appearance of the robot should be customizable.                                                                                              \\
\hline
Usage                               &                                           & Consider the context of use to ensure appropriate sturdiness and prevent damage from shocks or water; the user environment must be considered.   \\
                                    &                                           & The robot should not need additional human support during use.\\
\hline
\end{tabular}
\caption{Design principle for Echo-Teddy.}
\end{table*}

\subsubsection{Initial prototype of Echo-Teddy}

% Hardware
The hardware design of Echo-Teddy is built on a Raspberry Pi platform, chosen for its cost-effectiveness, scalability, and ability to support real-time interaction. The system efficiently transmits audio files and action commands between the Raspberry Pi and the server, ensuring low-latency communication for natural conversations. To further reduce production costs, the microphone and speaker were assembled using custom-purchased components and soldering techniques (Figure \ref{fig:echo-teddy-components}). This approach allowed for greater flexibility in hardware integration while maintaining affordability for broader implementation.

\begin{figure*}[hbt!]
    \centering
    \includegraphics[width=1\textwidth]{imgs/echo-teddy1.pdf}
    \caption{To reduce costs, we purchased microphone and speaker components separately and assembled them using soldering techniques.}
    \label{fig:echo-teddy-components}
\end{figure*}

During the production process, it was observed that placing the Raspberry Pi inside the plush doll led to heat buildup, which caused performance degradation. To address this, a backpack-style enclosure was designed to house the Raspberry Pi externally, allowing for better heat dissipation without compromising portability (Figure \ref{fig:echo-teddy-backpack}). This design also improves ease of maintenance and accessibility for future hardware upgrades.

\begin{figure*}[hbt!]
    \centering
    \includegraphics[width=1\textwidth]{imgs/echo-teddy2.pdf}
    \caption{To prevent the heat buildup, we made backpack to contain the Raspberry Pi.}
    \label{fig:echo-teddy-backpack}
\end{figure*}

The initial prototype version integrates attached motors to enable basic movements, such as nodding, providing simple nonverbal communication cues. Currently, the range of motion is limited to head movements and facial expressions, which are displayed using a dot matrix (Figure \ref{fig:echo-teddy-dot-matrix}). These features are intended to enhance emotional expressiveness and engagement in interactions with users. 

For connectivity, the system utilizes the built-in Wi-Fi module of the Raspberry Pi, ensuring stable access to cloud-based services. Additionally, mobile phone tethering is available as an alternative network solution, enabling portability across different settings, including classrooms, therapy environments, and home use. This flexible connectivity setup ensures that Echo-Teddy remains accessible and functional in diverse user environments.

\begin{figure*}[hbt!]
    \centering
    \includegraphics[width=0.5\textwidth]{imgs/echo-teddy3.pdf}
    \caption{We used dot matrix to express the emotions of Echo-Teddy.}
    \label{fig:echo-teddy-dot-matrix}
\end{figure*}

% Server system
The server system of Echo-Teddy integrates advanced cloud-based technologies to ensure efficient, scalable, and personalized interactions for autistic students. At its core, the system relies on OpenAI’s API for dialogue generation and natural language processing, allowing for context-aware and adaptive conversational interactions. To enhance flexibility and control over language model interactions, a prompt management module has been implemented, enabling LLM administrators to easily update and manage prompt texts for fine-tuned responses.

For speech processing, Echo-Teddy employs AWS Transcribe for Speech-to-Text (STT) functionality, ensuring accurate and efficient transcription of user input. Text-to-Speech (TTS) is handled by both AWS Polly and Naver Clova Voice, with Naver Clova Voice specifically incorporated to support Korean-speaking users. This multi-engine approach ensures high-quality, natural speech synthesis across different languages and user preferences.

The server infrastructure is built using FastAPI, following RESTful API principles, and is hosted on AWS EC2 to maintain reliable and scalable operations. To streamline development and deployment, the system integrates a CI/CD pipeline using GitHub Actions, enabling automated testing, integration, and deployment, reducing maintenance overhead and improving system stability. A key feature of the backend is its ability to transmit audio files along with structured action commands in JSON format, allowing for synchronized multimodal interactions. This ensures that Echo-Teddy’s speech output is coordinated with its physical gestures, enhancing engagement and communication effectiveness.

By leveraging this comprehensive cloud-based system architecture, Echo-Teddy provides a stable, adaptive, and scalable interaction platform designed to support the diverse communication needs of autistic students. The integration of modular and configurable components further ensures that the system remains flexible and customizable, meeting the evolving demands of special education applications.

\subsection{RQ2: What improvements can be made to the initial prototype of Echo-Teddy based on developer reflection-on-action and interviews with experts?}

\subsubsection{Reflection-on-action of developers}

This section reflects on the design and implementation decisions made during the development of the initial prototype of Echo-Teddy, highlighting key challenges encountered and the provisional solutions proposed. These insights provide a foundation for future refinements, ensuring that the robot effectively supports social and emotional development in autistic students.

One of the primary areas of focus was enhancing nonverbal communication capabilities to improve the naturalness of interaction. While the initial prototype allowed Echo-Teddy to express emotions based on user input, its expressive range was limited to seven predefined facial expressions. To increase emotional depth and adaptability, an additional feature involving movable eyebrows could be introduced, allowing for more nuanced and dynamic facial expressions. This modification would enable the robot to better reflect emotional subtleties, making interactions more engaging and responsive to diverse social cues.

A second area of improvement involved implementing a monitoring system for caregivers and educators. Since interactions with Echo-Teddy can provide valuable insights into a child's social communication patterns, enabling convenient tracking of interaction history is essential for individualized intervention planning. To facilitate this, experts recommended optimizing the website for mobile accessibility or developing a dedicated mobile application, allowing caregivers and educators to easily review and analyze communication logs. 

Third, ensuring the reliability of Echo-Teddy’s emotional responses was another critical challenge. Accurate and contextually appropriate emotional reactions to user input are essential for fostering both self-awareness and emotional comprehension in autistic students. However, facial expressions generated from new data should undergo rigorous evaluation to verify accuracy and appropriateness, ensuring that Echo-Teddy's emotional feedback is both credible and developmentally supportive. As noted by Rawal et al. \cite{rawal2022facialemotionexpressionshumanrobot}, effective human-robot interactions depend on accurate emotion recognition and expression, reinforcing the need for continuous validation and refinement of Echo-Teddy’s emotional modeling.

Another key consideration was adapting Echo-Teddy’s appearance to strengthen emotional bonds with users. The physical design of social robots plays a significant role in user engagement and comfort. Research by Ricks and Colton \cite{5509327} suggests that while autistic children benefit from both humanoid and non-humanoid robots, they tend to engage more actively with non-humanoid designs. This finding supports the idea that flexibility in physical form—such as allowing for different shapes or textures—may improve user preference and interaction quality. Providing customizable external features, such as different animal forms or textures, could further optimize engagement based on individual sensory preferences.

Reflecting on these early design decisions has yielded valuable insights into both project management and technical execution. Future development efforts will focus on enhancing Echo-Teddy’s real-time emotional expressiveness, refining its monitoring capabilities, validating emotion-driven responses, and expanding customization options. These refinements will ensure that Echo-Teddy remains an adaptive, engaging, and effective tool for supporting social and emotional development in autistic students.

\subsubsection{Results of interview with experts}

The interview included experts with extensive experience in special education who provided diverse perspectives on the early prototype of the LLM-based social robot Echo-Teddy for students with autism spectrum disorder. The content of the interview was primarily analyzed in terms of seven themes: (1) response speed (processing delay), (2) the robot’s physical form and external features, (3) interaction goals and usage scenarios, (4) nonverbal communication and modeling strategies, (5) the range and characteristics of the target students, (6) generalizability and additional considerations, and (7) implementation costs and scalability. Below is a summary of the experts’ key opinions and improvement suggestions for each theme.

First, experts pointed out that Echo-Teddy experienced a delay of approximately 6–10 seconds when communicating with external servers for STT and TTS. They emphasized that immediate feedback is crucial for students with autism, as delays can lead to loss of attention or increased anxiety during communication. They noted that communication skills must be reinforced through repeated, prompt interactions. To address this, they suggested either (1) upgrading hardware to support faster local processing (e.g., Jetson Nano) or (2) splitting audio data into smaller segments and synthesizing speech sequentially for playback.

Second, the interview included discussions about the rationale behind Echo-Teddy being designed as a teddy bear and the appropriateness of this choice. Experts acknowledged that "soft plush robots can be a positive approach for certain students, as safety and sensory preferences are critical for students with autism." However, they also recommended offering modular options, such as different animal shapes (e.g., dinosaurs), to accommodate students who may react negatively to certain textures or appearances. Furthermore, some experts highlighted the importance of clear visual focus, such as the size and positioning of the robot’s eyes, to encourage eye contact. However, others cautioned against prioritizing eye contact as a universal goal, as it may provoke anxiety or discomfort for some students. Ethical considerations and individual differences were emphasized.

Third, while the development team initially focused on scenarios where students would engage in one-on-one conversations with the robot under the supervision of caregivers or teachers, experts emphasized the importance of fostering peer interactions. They suggested that Echo-Teddy could serve as a mediator to encourage participation and extend social skills in group settings, such as inclusive classrooms. They proposed that the robot could facilitate dialogue between autistic students and their peers, promoting social engagement. However, they also noted potential challenges, such as peers perceiving the robot as a novelty or the risk of students overly depending on the robot.

Fourth, experts emphasized the importance of modeling nonverbal communication (e.g., nodding, gestures, gaze direction) for students with autism, as these skills are as critical as verbal language. They suggested that Echo-Teddy could perform human-like gestures, such as turning its head, waving, or nodding, to encourage imitation behaviors. However, they advised against certain actions, like "hugging," which may not align with typical daily interactions or could cause sensory discomfort. Additionally, they discussed whether immediate reinforcement (e.g., the robot saying, "Good job!") should be provided after modeling or if maintaining a natural conversational flow would be more effective. Designing consistent teaching strategies and robot responses was deemed essential.

Fifth, the experts recommended focusing initially on "high-functioning autistic students who can use spoken language" while also exploring the feasibility of extending Echo-Teddy to students using AAC (augmentative and alternative communication) devices. Some students rely on picture icons or switch operations to express themselves instead of direct speech. Therefore, they emphasized that the robot should not be limited to processing verbal speech alone. If the robot could recognize and respond to electronic voices generated by AAC devices, it could benefit a broader spectrum of students.

Sixth, the experts stressed the importance of ensuring that interactions with the robot could generalize to real-life conversations with other humans. They cautioned that learning communication skills with the robot might have limited impact if not transferred to interactions with peers or family members. Skills like eye contact or body orientation were highlighted as "common competencies" that should be modeled by the robot and reinforced by caregivers or teachers. Practical considerations, such as the robot’s durability, hygiene, and connectivity, were also mentioned as essential factors for long-term use.

Finally, experts noted the financial constraints faced by special education programs, where the budget for equipment is often limited. They suggested that Echo-Teddy could be developed as an open-source project, allowing institutions to build DIY versions tailored to their specific needs. This approach could reduce costs and enable schools or organizations to customize the robot’s appearance, voice, and features to meet diverse sensory and preference requirements.


\section{Discussion}

The expert interviews provided key insights into refining Echo-Teddy’s design and functionality to better support social communication in autistic students. Experts highlighted the need for immediate responsiveness, a carefully designed physical and behavioral model, goal-oriented interaction strategies, and practical implementation considerations. Their feedback emphasized the importance of ensuring that Echo-Teddy not only facilitates meaningful interaction but also generalizes its impact to real-world social settings.

First, one major finding was the critical importance of real-time responsiveness in maintaining engagement for autistic students. Experts observed a 6–10 second processing delay due to cloud-based speech recognition and text-to-speech generation, which they noted could increase anxiety and disrupt interaction flow. Given that immediate and predictable feedback is essential for reinforcing communication skills, delays risk causing students to lose focus or disengage from the conversation. Cano et al. \cite{cano2023design} found that latency reductions in robotic responses not only enhance engagement but also mitigate behavioral disruptions, making this a critical area for refinement. Implementing edge computing strategies or optimizing speech processing pipelines may be essential for achieving the low-latency performance necessary to support effective and anxiety-free interactions for autistic students.

Second, the robot’s form and behavioral modules must balance the individual characteristics of autistic students with educational goals, particularly in distinguishing between modeling social behaviors and facilitating natural conversation. In inclusive classroom settings, incorporating subtle nonverbal cues such as head nods, body tilts, and gaze direction can encourage peer interaction by reinforcing appropriate social responses. However, ensuring design flexibility is crucial, as autistic students have diverse sensory preferences and interaction styles. Tailoring the robot’s activities, gestures, and verbal feedback based on user preferences and real-time adaptation is particularly important. Lee and Park  \cite{maroto2024personalizing} demonstrate that personalization enhances both engagement and intervention effectiveness, reinforcing the importance of customizable interaction strategies.Similarly, Cano et al. \cite{cano2023design} emphasize that user-centered design ensures robots remain effective across a spectrum of student needs, cautioning against one-size-fits-all approaches. For instance, while verbal reinforcement strategies (e.g., the robot saying "Good job!") can be effective for some students, others may find explicit praise stressful or disruptive. To prevent additional stress or confusion, reinforcement should be context-aware, ensuring that it aligns naturally with the conversation flow rather than feeling overly scripted or intrusive. This highlights the need for adaptive response mechanisms, allowing the robot to modulate its speech and gestures based on the child's individual comfort levels and engagement patterns.

Third, to broaden the scope of Echo-Teddy, ensuring scalability and generalizability is essential. Skills learned through robot-assisted interactions should seamlessly transfer to real-life settings, enabling autistic students to apply these communication skills with peers, caregivers, and teachers. Without this transferability, the benefits of robot-mediated interactions risk being isolated to controlled environments rather than supporting meaningful social engagement in daily life.  Choi et al \cite{santos2023applications} emphasize that scalable robot designs enable broader applicability across diverse educational and social settings, ensuring that interventions are not limited to small-scale experiments. A structured learning approach—beginning with caregiver-led activities and progressively expanding to peer interactions—has been shown to enhance social skill transfer beyond interactions with the robot itself. Clabaugh et al. \cite{clabaugh2019long} further highlight the value of long-term personalized interventions, demonstrating that sustained engagement reinforces skill retention and generalization. Such findings indicate that for Echo-Teddy to achieve wide-scale adoption, it must integrate Augmentative and Alternative Communication (AAC) devices, ensuring compatibility with students who rely on visual icons, text-based communication, or switch-access systems. Furthermore, the robot must be equipped with robust network capabilities to ensure seamless connectivity in classroom and home environments. Durable hardware construction is also critical, particularly for long-term educational use in varied settings. By addressing these factors, Echo-Teddy can bridge the gap between controlled intervention settings and real-world application, making it a scalable, adaptable, and impactful tool for special education.









\section{Discussion and Conclusion}

% \begin{quote}
% \textit{"We believe it is unethical for social workers not to learn... about technology-mediated social work."} (\citeauthor{singer_ai_2023}, 2023)
% \end{quote}

In this study, we uncovered multiple ways in which GenAI can be used in social service practice. While some concerns did arise, practitioners by and large seemed optimistic about the possibilities of such tools, and that these issues could be overcome. We note that while most participants found the tool useful, it was far from perfect in its outputs. This is not surprising, since it was powered by a generic LLM rather than one fine-tuned for social service case management. However, despite these inadequacies, our participants still found many uses for most of the tool's outputs. Many flaws pointed out by our participants related to highly contextualised, local knowledge. To tune an AI system for this would require large amounts of case files as training data; given the privacy concerns associated with using client data, this seems unlikely to happen in the near future. What our study shows, however, is that GenAI systems need not aim to be perfect to be useful to social service practitioners, and can instead serve as a complement to the critical "human touch" in social service.

We draw both inspiration and comparisons with prior work on AI in other settings. Studies on creative writing tools showed how the "uncertainty" \cite{wan2024felt} and "randomness" \cite{clark2018creative} of AI outputs aid creativity. Given the promise that our tool shows in aiding brainstorming and discussion, future social service studies could consider AI tools explicitly geared towards creativity - for instance, providing side-by-side displays of how a given case would fit into different theoretical frameworks, prompting users to compare, contrast, and adopt the best of each framework; or allowing users to play around with combining different intervention modalities to generate eclectic (i.e. multi-modal) interventions.

At the same time, the concept of supervision creates a different interaction paradigm to other uses of AI in brainstorming. Past work (e.g. \cite{shaer2024ai}) has explored the use of GenAI for ideation during brainstorming sessions, wherein all users present discuss the ideas generated by the system. With supervision in social service practice, however, there is a marked information and role asymmetry: supervisors may not have had the time to fully read up on their supervisee's case beforehand, yet have to provide guidance and help to the latter. We suggest that GenAI can serve a dual purpose of bringing supervisors up to speed quickly by summarising their supervisee's case data, while simultaneously generating a list of discussion and talking points that can improve the quality of supervision. Generalising, this interaction paradigm has promise in many other areas: senior doctors reviewing medical procedures with newer ones \cite{snowdon2017does} could use GenAI to generate questions about critical parts of a procedure to ask the latter, confirming they have been correctly understood or executed; game studio directors could quickly summarise key developmental pipeline concerns to raise at meetings and ensure the team is on track; even in academia, advisors involved in rather too many projects to keep track of could quickly summarise each graduate student's projects and identify potential concerns to address at their next meeting.

In closing, we are optimistic about the potential for GenAI to significantly enhance social service practice and the quality of care to clients. Future studies could focus on 1) longitudinal investigations into the long-term impact of GenAI on practitioner skills, client outcomes, and organisational workflows, and 2) optimising workflows to best integrate GenAI into casework and supervision, understanding where best to harness the speed and creativity of such systems in harmony with the experience and skills of practitioners at all levels.

% GenAI here thus serves as a tool that supervisors can use before rather then using the session, taking just a few minutes of their time to generate a list of discussion points with their supervisees.

% Traditional brainstorming comes up with new things that users discuss. In supervision, supervisors can use AI to more efficiently generate talking points with their supervisees. These are generally not novel ideas, since an experienced worker would be able to come up with these on their own. However, the interesting and novel use of AI here is in its use as a preparation tool, efficiently generating talking and discussion points, saving supervisors' time in preparing for a session, while still serving as a brainstorming tool during the session itself.

% The idea of embracing imperfect AI echoes the findings of \citeauthor{bossen2023batman} (2023) in a clinical decision setting, which examined the successful implementation of an "error-prone but useful AI tool". This study frames human-AI collaboration as "Batman and Robin", where AI is a useful but ultimately less skilled sidekick that plays second fiddle to Batman. This is similar to \citeauthor{yang2019unremarkable}'s (2019) idea of "unremarkable AI", systems designed to be unobtrusive and only visible to the user when they add some value. As compared to \citeauthor{bossen2023batman}, however, we see fewer instances of our AI system producing errors, and more examples of it providing learning and collaborative opportunities and other new use cases. We build on the idea of "complementary performance" \cite{bansal2021does}, which discusses how the unique expertise of AI enhances human decision-making performance beyond what humans can achieve alone. Beyond decision-making, GenAI can now enable "complementary work patterns", where the nature of its outputs enables humans to carry out their work in entirely new ways. Our study suggests that rather being a sidekick - Robin - AI is growing into the role of a "second Batman" or "AI-Batman": an entity with distinct abilities and expertise from humans, and that contributes in its own unique way. There is certainly still a time and place for unremarkable AI, but exploring uses beyond that paradigm uncovers entirely new areas of system design.

% % \cite{gero2022sparks} found AI to be useful for science writers to translate ideas already in their head into words, and to provide new perspectives to spark further inspiration. \textit{But how is ours different from theirs?}

% \subsection{New Avenues of Human-AI Collaboration}
% \label{subsubsec:discussionhaicollaboration}

% Past HCI literature in other areas \cite{nah2023generative} has suggested that GenAI represents a "leap" \cite{singh2023hide} in human-AI collaboration, 
% % Even when an AI system sometimes produces irrelevant outputs, it can still provide users 
% % Such systems have been proposed as ways to 
% helping users discover new viewpoints \cite{singh2023hide}, scour existing literature to suggest new hypotheses 
% \cite{cascella2023evaluating} and answer questions \cite{biswas2023role}, stimulate their cognitive processes \cite{memmert2023towards}, and overcome "writer's block" \cite{singh2023hide, cooper2023examining} (particularly relevant to SSPs and the vast amount of writing required of them). Our study finds promise for AI to help SSPs in all of these areas. By nature of being more verbose and capable of generating large amounts of content, GenAI seems to create a new way in which AI can complement human work and expertise. Our system, as LLMs tend to do, produced a lot of "bullshit" (S6) \cite{frankfurt2005bullshit} - superficially true statements that were often only "tangentially related" and "devoid of meaning" \cite{halloran2023ai}. Yet, many participants cited the page-long analyses and detailed multi-step intervention plans generated by the AI system to be a good starting point for further discussion, both to better conceptualize a particular case and to facilitate general worker growth and development. Almost like throwing mud at a wall to see what sticks, GenAI can quickly produce a long list of ideas or information, before the worker glances through it and quickly identifies the more interesting points to discuss. Playing the proposed role as a "scaffold" for further work \cite{cooper2023examining}, GenAI, literally, generates new opportunities for novel and more effective processes and perspectives that previous systems (e.g., PRMs) could not. This represents an entirely new mode of human-AI collaboration.
% % This represents a new mode of collaboration not possible with the largely quantitative AI models (like PRMs) of the past.

% Our work therefore supports and extends prior research that have postulated the the potential of AI's shifting roles from decision-maker to human-supporter \cite{wang_human-human_2020}. \citeauthor{siemon2022elaborating} (2022) suggests the role of AI as a "creator" or "coordinator", rather than merely providing "process guidance" \cite{memmert2023towards} that does not contribute to brainstorming. Similarly, \citeauthor{memmert2023towards} (2023) propose GenAI as a step forward from providing meta-level process guidance (i.e. facilitating user tasks) to actively contributing content and aiding brainstorming. We suggest that beyond content-support, AI can even create new work processes that were not possible without GenAI. In this sense, AI has come full circle, becoming a "meta-facilitator".

% % --- WIP BELOW ---

% % Our work echoes and extends previous research on HAI collaboration in tasks requiring a human touch. \cite{gero2023social} found AI to be a safe space for creative writers to bounce ideas off of and document their inner thoughts. \cite{dhillon2024shaping} reference the idea of appropriate scaffolding in argumentative writing, where the user is providing with guidance appropriate for their competency level, and also warns of decreased satisfaction and ownership from AI use. 

% Separately, we draw parallels with the field of creative writing, where HAI collaboration has been extensively researched. Writers note the "irreducibly human" aspect of creativity in writing \cite{gero2023social}, similar to the "human touch" core to social service practice (D1); both groups therefore expressed few concerns about AI taking over core aspects of their jobs. Another interesting parallel was how writers often appreciated the "uncertainty" \cite{wan2024felt} and "randomness" \cite{clark2018creative} of AI systems, which served as a source of inspiration. This echoes the idea of "imperfect AI" "expanding [the] perspective[s]" (S4) of our participants when they simply skimmed through what the AI produced. \cite{wan2024felt} cited how the "duality of uncertainty in the creativity process advances the exploration of the imperfection of GenAI models". While social service work is not typically regarded as "creative", practitioners nonetheless go through processes of ideation and iteration while formulating a case. Our study showed hints of how AI can help with various forms of ideation, but, drawing inspiration from creative writing tools, future studies could consider designs more explicitly geared towards creativity - for instance, by attempting to fit a given case into a number of different theories or modalities, and displaying them together for the user to consider. While many of these assessments may be imperfect or even downnright nonsensical, they may contain valuable ideas and new angles on viewing the case that the practitioner can integrate into their own assessment.

% % \cite{foong2024designing}, describing the design of caregiver-facing values elicitation tools, cites the "twin scenarios" that caregivers face - private use, where they might use a tool to discover their patient's values, and collaborative use, where they discuss the resulting values with other parties close to the patient. This closely mirrors how SSPs in our study reference both individual and collaborative uses of our tool. Unlike in \cite{foong2024designing}, however, we do not see a resulting need to design a "staged approach" with distinct interface features for both stages.

% % --- END OF WIP ---

% Having mentioned algorithm aversion previously, we also make a quick point here on the other end of the spectrum - automation bias, or blind trust in an automated system \cite{brown2019toward}. LLMs risk being perceived as an "ultimate epistemic authority" \cite{cooper2023examining} due to their detailed, life-like outputs. While automation bias has been studied in many contexts, including in the social sector or adjacent areas, we suggest that the very nature of GenAI systems fundamentally inhibits automation bias. The tendency of GenAI to produce verbose, lengthy explanations prompts users to read and think through the machine's judgement before accepting it, bringing up opportunities to disagree with the machine's opinion. This guards against blind acceptance of the system's recommendations, particularly in the culture of a social work agency where constant dialogue - including discussing AI-produced work - is the norm.


% % : Perception of AI in Social Service Work ??

% \subsection{Redefining the Boundary}
% \label{subsubsec:discussiontheoretical}

% As \citeauthor{meilvang_working_2023} (2023) describes, the social service profession has sought to distance itself from comprising mostly "administrative work" \cite{abbott2016boundaries}, and workers have long tried to tried to reduce their considerable time \cite{socialraadgiverforening2010notat} spent on such tasks in favour of actual casework with clients \cite{toren1972social}. Our study, however, suggests a blurring of the line between "manual" administrative tasks and "mental" casework that draws on practitioner expertise. Many tasks our participants cited involve elements of both: for instance, documenting a case recording requires selecting only the relevant information to include, and planning an intervention can be an iterative process of drafting a plan and discussing it with colleagues and superiors. This all stems from the fact that GenAI can produce virtually any document required by the user, but this document almost always requires revision under a watchful human eye.

% \citeauthor{meilvang_working_2023} (2023) also describes a more recent shift in the perceived accepted boundary of AI interventions in social service work. From "defending [the] jurisdiction [of social service work] against artificial intelligence" in the early days of PRM and other statistical assessment tools, the community has started to embrace AI as a "decision-support ... element in the assessment process". Our study concurs and frames GenAI as a source of information that can be used to support and qualify the assessments of SSPs \cite{meilvang_working_2023}, but suggests that we can take a step further: AI can be viewed as a \textit{facilitator} rather than just a supporter. GenAI can facilitate a wide range of discussions that promote efficiency, encourage worker learning and growth, and ultimately enhance client outcomes. This entails a much larger scope of AI use, where practitioners use the information provided by AI in a range of new scenarios. 

% Taken together, these suggest a new focus for boundary work and, more broadly, HCI research. GAI can play a role not just in menial documentation or decision-support, but can be deeply ingrained into every facet of the social service workflow to open new opportunities for worker growth, workflow optimisation, and ultimately improved client outcomes. Future research can therefore investigate the deeper, organisational-level effects of these new uses of AI, and their resulting impact on the role of profession discretion in effective social service work.

% % MH: oh i feel this paragraph is quite new to me! Could we elaborate this more, and truncate the first two paragraphs a bit to adjust the word propotion?


% % Our study extensively documents this for the first time in social service practice, and in the process reveals new insights about how AI can play such a role.



% \subsection{Design Implications}

% % Add link from ACE diagram?

% % EJ: it would be interesting to discuss how LLMs could help "hands-on experience" in the discussion section

% Addressing the struggle of integrating AI amidst the tension between machine assessment and expert judgement, we reframe AI as an \textit{facilitator} rather than an algorithm or decision-support tool, alleviating many concerns about trust and explainablity. We now present a high-level framework (Figure \ref{fig:hai-collaboration}) on human-AI collaboration, presenting a new perspective on designing effective AI systems that can be applied to both the social service sector and beyond.

% \begin{figure}
%     \centering
%     \includegraphics[scale=0.15]{images/designframework.png}
%     \caption{Framework for Human-AI Collaboration}
%     \label{fig:hai-collaboration}
%     \Description{An image showing our framework for Human-AI Collaboration. It shows that as stakeholder level increases from junior to senior, the directness of use shifts from co-creation to provision.}
% \end{figure}
% % MH: so this paradigm is proposed by us? I wonder if this could a part of results as well..?

% % \subsubsection{From Creation to Provision}

% In Section \ref{sec:stage2findings}, we uncovered the different ways in which SSPs of varying seniorities use, evaluate, and suggest uses of AI. These are intrinsically tied to the perspectives and levels of expertise that each stakeholder possesses. We therefore position the role of AI along the scale of \textit{creation} to \textit{provision}. 

% With junior workers, we recommend \textbf{designing tools for co-creation}: systems that aid the least experienced workers in creating the required deliverables for their work. Rather than \textit{telling} workers what to do - a difficult task in any case given the complexity of social work solutions - AI systems should instead \textit{co-create} deliverables required of these workers. These encompass the multitude of use cases that junior workers found useful: creating reports, suggesting perspectives from which to formulate a case, and providing a starting template for possible intervention plans. Notably, since AI outputs are not perfect, we emphasise the "co" in "co-creation": AI should only be a part of the workflow that also includes active engagement on the part of the SSPs and proactive discussion with supervisors. 

% For more experienced SSPs, we recommend \textbf{designing tools for provision}. Again, this is not the mere provision of recommendations or courses of action with clients, but rather that of resources which complement the needs of workers with greater responsibilities. This notably includes supplying materials to aid with supervision, a novel use case that to our knowledge has not surfaced in previous literature. In addition, senior workers also benefit greatly from manual tasks such as routine report writing and data processing. Since these workers are more experienced and can better spot inaccuracies in AI output, we suggest that AI can "provide" a more finished product that requires less vetting and corrections, and which can be used more directly as part of required deliverables.

% % MH: can we seperate here? above is about the guidance to paradigm, below is the practical roadmap for implementation
% In terms of concrete design features, given the constant focus on discussing AI outputs between colleagues in our FGDs, we recommend that AI tools, particularly those for junior workers, \textbf{include collaborative features} that facilitate feedback and idea sharing between users. We also suggest that designers work closely with domain experts (i.e. social work practitioners and agencies) to identify areas where the given AI model tends to make more mistakes, and to build in features that \textbf{highlight potential mistakes or inadequacies} in the AI's output to facilitate further discussion and avoid workers adopting suboptimal suggestions. 

% We also point out a fundamental difference between GenAI systems and previous systems: that GenAI can now play an important role in aiding users \textit{regardless of its flaws}. The nature of GenAI means that it promotes discussion and opens up new workflows by nature of its verbose and potentially incomplete outputs. Rather than working towards more accurate or explainable outcomes, which may in any case have minimal improvement on worker outcomes \cite{li2024advanced}, designers can also focus on \textbf{understanding how GenAI outputs can augment existing user flows and create new ones}.

% % for more senior workers...

% % how to differentiate levels of workers?

% % \subsubsection{Provider}

% % The most basic and obvious role of modern AI that we identify leverages the main strength of LLMs. They have the ability to produce high-quality writing from short, point-form, or otherwise messy and disjoint case notes that user often have \textit{[cite participant here]}. 

% Finally, given the limited expertise of many workers at using AI, it is important that systems \textbf{explicitly guide users to the features they need}, rather than simply relying on the ability of GAI to understand complex user instructions. For example, in the case of flexibility in use cases (Section \ref{subsubsec:control}), systems should include user flows that help combine multiple intervention and assessment modalities in order to directly meet the needs of workers.

% \subsection{Limitations and Future Work}

% While we attempt to mimic a contextual inquiry and work environment in our study design, there is no substitute for real data from actual system deployment. The use of an AI system in day-to-day work could reveal a different set of insights. Future studies could in particular study how the longitudinal context of how user attitudes, behaviours, preferences, and work outputs change with extended use of AI. 

% While we tried to include practitioners from different agencies, roles, and seniorities, social service practice may differ culturally or procedurally in other agencies or countries. Future studies could investigate different kinds of social service agencies and in different cultures to see if AI is similarly useful there.

% As the study was conducted in a country with relatively high technology literacy, participants naturally had a higher baseline understanding and acceptance of AI and other computer systems. However, we emphasise that our findings are not contingent on this - rather, we suggest that our proposed lens of viewing AI in the social sector is a means for engaging in relevant stakeholders and ensuring the effective design and implementation of AI in the social sector, regardless of how participants feel about AI to begin with. 



% % \subsection{Notes}

% % 1) safety and risks and 2) privacy - what does the emphasis on this say about a) design recommendations and b) approach to designing/PD of such systems?


% % W9 was presented with "Strengths" and "SFBT" output options. They commented, "solution focus is always building on the person's strengths". W9 therefore requested being able to output strengths and SFBT at the same time. But this would suggest that the SFBT output does not currently emphasise strengths strongly enough. However, W9 did not specifically evaluate that, and only made this comment because they saw the "strengths" option available, and in their head, strengths are key to SFBT.
% % What does this say about system design and UI in relation to user mental models?
\paragraph{Summary}
Our findings provide significant insights into the influence of correctness, explanations, and refinement on evaluation accuracy and user trust in AI-based planners. 
In particular, the findings are three-fold: 
(1) The \textbf{correctness} of the generated plans is the most significant factor that impacts the evaluation accuracy and user trust in the planners. As the PDDL solver is more capable of generating correct plans, it achieves the highest evaluation accuracy and trust. 
(2) The \textbf{explanation} component of the LLM planner improves evaluation accuracy, as LLM+Expl achieves higher accuracy than LLM alone. Despite this improvement, LLM+Expl minimally impacts user trust. However, alternative explanation methods may influence user trust differently from the manually generated explanations used in our approach.
% On the other hand, explanations may help refine the trust of the planner to a more appropriate level by indicating planner shortcomings.
(3) The \textbf{refinement} procedure in the LLM planner does not lead to a significant improvement in evaluation accuracy; however, it exhibits a positive influence on user trust that may indicate an overtrust in some situations.
% This finding is aligned with prior works showing that iterative refinements based on user feedback would increase user trust~\cite{kunkel2019let, sebo2019don}.
Finally, the propensity-to-trust analysis identifies correctness as the primary determinant of user trust, whereas explanations provided limited improvement in scenarios where the planner's accuracy is diminished.

% In conclusion, our results indicate that the planner's correctness is the dominant factor for both evaluation accuracy and user trust. Therefore, selecting high-quality training data and optimizing the training procedure of AI-based planners to improve planning correctness is the top priority. Once the AI planner achieves a similar correctness level to traditional graph-search planners, strengthening its capability to explain and refine plans will further improve user trust compared to traditional planners.

\paragraph{Future Research} Future steps in this research include expanding user studies with larger sample sizes to improve generalizability and including additional planning problems per session for a more comprehensive evaluation. Next, we will explore alternative methods for generating plan explanations beyond manual creation to identify approaches that more effectively enhance user trust. 
Additionally, we will examine user trust by employing multiple LLM-based planners with varying levels of planning accuracy to better understand the interplay between planning correctness and user trust. 
Furthermore, we aim to enable real-time user-planner interaction, allowing users to provide feedback and refine plans collaboratively, thereby fostering a more dynamic and user-centric planning process.


\bibliographystyle{ACM-Reference-Format}
\bibliography{references, references2}

\section{Secure Token Pruning Protocols}
\label{app:a}
We detail the encrypted token pruning protocols $\Pi_{prune}$ in Figure \ref{fig:protocol-prune} and $\Pi_{mask}$ in Figure \ref{fig:protocol-mask} in this section.

%Optionally include supplemental material (complete proofs, additional experiments and plots) in appendix.
%All such materials \textbf{SHOULD be included in the main submission.}
\begin{figure}[h]
%vspace{-0.2in}
\begin{protocolbox}
\noindent
\textbf{Parties:} Server $P_0$, Client $P_1$.

\textbf{Input:} $P_0$ and $P_1$ holds $\{ \left \langle Att \right \rangle_{0}^{h}, \left \langle Att \right \rangle_{1}^{h}\}_{h=0}^{H-1} \in \mathbb{Z}_{2^{\ell}}^{n\times n}$ and $\left \langle x \right \rangle_{0}, \left \langle x \right \rangle_{1} \in \mathbb{Z}_{2^{\ell}}^{n\times D}$ respectively, where H is the number of heads, n is the number of input tokens and D is the embedding dimension of tokens. Additionally, $P_1$ holds a threshold $\theta \in \mathbb{Z}_{2^{\ell}}$.

\textbf{Output:} $P_0$ and $P_1$ get $\left \langle y \right \rangle_{0}, \left \langle y \right \rangle_{1} \in \mathbb{Z}_{2^{\ell}}^{n'\times D}$, respectively, where $y=\mathsf{Prune}(x)$ and $n'$ is the number of remaining tokens.

\noindent\rule{13.2cm}{0.1pt} % This creates the horizontal line
\textbf{Protocol:}
\begin{enumerate}[label=\arabic*:, leftmargin=*]
    \item For $h \in [H]$, $P_0$ and $P_1$ compute locally with input $\left \langle Att \right \rangle^{h}$, and learn the importance score in each head $\left \langle s \right \rangle^{h} \in \mathbb{Z}_{2^{\ell}}^{n} $, where $\left \langle s \right \rangle^{h}[j] = \frac{1}{n} \sum_{i=0}^{n-1} \left \langle Att \right \rangle^{h}[i,j]$.
    \item $P_0$ and $P_1$ compute locally with input $\{ \left \langle s \right \rangle^{i} \in \mathbb{Z}_{2^{\ell}}^{n}  \}_{i=0}^{H-1}$, and learn the final importance score $\left \langle S \right \rangle \in \mathbb{Z}_{2^{\ell}}^{n}$ for each token, where  $\left \langle S \right \rangle[i] = \frac{1}{H} \sum_{h=0}^{H-1} \left \langle s \right \rangle^{h}[i]$.
    \item  For $i \in [n]$, $P_0$ and $P_1$ invoke $\Pi_{CMP}$ with inputs  $\left \langle S \right \rangle$ and $ \theta $, and learn  $\left \langle M \right \rangle \in \mathbb{Z}_{2^{\ell}}^{n}$, such that$\left \langle M \right \rangle[i] = \Pi_{CMP}(\left \langle S \right \rangle[i] - \theta) $, where: \\
    $M[i] = \begin{cases}
        1  &\text{if}\ S[i] > \theta, \\
        0  &\text{otherwise}.
            \end{cases} $
    % \item If the pruning location is insensitive, $P_0$ and $P_1$ learn real mask $M$ instead of shares $\left \langle M \right \rangle$. $P_0$ and $P_1$ compute $\left \langle y \right \rangle$ with input $\left \langle x \right \rangle$ and $M$, where  $\left \langle x \right \rangle[i]$ is pruned if $M[i]$ is $0$.
    \item $P_0$ and $P_1$ invoke $\Pi_{mask}$ with inputs  $\left \langle x \right \rangle$ and pruning mask $\left \langle M \right \rangle$, and set outputs as $\left \langle y \right \rangle$.
\end{enumerate}
\end{protocolbox}
\setlength{\abovecaptionskip}{-1pt} % Reduces space above the caption
\caption{Secure Token Pruning Protocol $\Pi_{prune}$.}
\label{fig:protocol-prune}
\end{figure}




\begin{figure}[h]
\begin{protocolbox}
\noindent
\textbf{Parties:} Server $P_0$, Client $P_1$.

\textbf{Input:} $P_0$ and $P_1$ hold $\left \langle x \right \rangle_{0}, \left \langle x \right \rangle_{1} \in \mathbb{Z}_{2^{\ell}}^{n\times D}$ and  $\left \langle M \right \rangle_{0}, \left \langle M \right \rangle_{1} \in \mathbb{Z}_{2^{\ell}}^{n}$, respectively, where n is the number of input tokens and D is the embedding dimension of tokens.

\textbf{Output:} $P_0$ and $P_1$ get $\left \langle y \right \rangle_{0}, \left \langle y \right \rangle_{1} \in \mathbb{Z}_{2^{\ell}}^{n'\times D}$, respectively, where $y=\mathsf{Prune}(x)$ and $n'$ is the number of remaining tokens.

\noindent\rule{13.2cm}{0.1pt} % This creates the horizontal line
\textbf{Protocol:}
\begin{enumerate}[label=\arabic*:, leftmargin=*]
    \item For $i \in [n]$, $P_0$ and $P_1$ set $\left \langle M \right \rangle$ to the MSB of $\left \langle x \right \rangle$ and learn the masked tokens $\left \langle \Bar{x} \right \rangle \in Z_{2^{\ell}}^{n\times D}$, where
    $\left \langle \Bar{x}[i] \right \rangle = \left \langle x[i] \right \rangle + (\left \langle M[i] \right \rangle << f)$ and $f$ is the fixed-point precision.
    \item $P_0$ and $P_1$ compute the sum of $\{\Pi_{B2A}(\left \langle M \right \rangle[i]) \}_{i=0}^{n-1}$, and learn the number of remaining tokens $n'$ and the number of tokens to be pruned $m$, where $m = n-n'$.
    \item For $k\in[m]$, for $i\in[n-k-1]$, $P_0$ and $P_1$ invoke $\Pi_{msb}$ to learn the highest bit of $\left \langle \Bar{x}[i] \right \rangle$, where $b=\mathsf{MSB}(\Bar{x}[i])$. With the highest bit of $\Bar{x}[i]$, $P_0$ and $P_1$ perform a oblivious swap between $\Bar{x}[i]$ and $\Bar{x}[i+1]$:
    $\begin{cases}
        \Tilde{x}[i] = b\cdot \Bar{x}[i] + (1-b)\cdot \Bar{x}[i+1] \\
        \Tilde{x}[i+1] = b\cdot \Bar{x}[i+1] + (1-b)\cdot \Bar{x}[i]
    \end{cases} $ \\
    $P_0$ and $P_1$ learn the swapped token sequence $\left \langle \Tilde{x} \right \rangle$.
    \item $P_0$ and $P_1$ truncate $\left \langle \Tilde{x} \right \rangle$ locally by keeping the first $n'$ tokens, clear current MSB (all remaining token has $1$ on the MSB), and set outputs as $\left \langle y \right \rangle$.
\end{enumerate}
\end{protocolbox}
\setlength{\abovecaptionskip}{-1pt} % Reduces space above the caption
\caption{Secure Mask Protocol $\Pi_{mask}$.}
\label{fig:protocol-mask}
%\vspace{-0.2in}
\end{figure}

% \begin{wrapfigure}{r}{0.35\textwidth}  % 'r' for right, and the width of the figure area
%   \centering
%   \includegraphics[width=0.35\textwidth]{figures/msb.pdf}
%   \caption{Runtime of $\Pi_{prune}$ and $\Pi_{mask}$ in different layers. We compare different secure pruning strategies based on the BERT Base model.}
%   \label{fig:msb}
%   \vspace{-0.1in}
% \end{wrapfigure}

% \begin{figure}[h]  % 'r' for right, and the width of the figure area
%   \centering
%   \includegraphics[width=0.4\textwidth]{figures/msb.pdf}
%   \caption{Runtime of $\Pi_{prune}$ and $\Pi_{mask}$ in different layers. We compare different secure pruning strategies based on the BERT Base model.}
%   \label{fig:msb}
%   % \vspace{-0.1in}
% \end{figure}

\textbf{Complexity of $\Pi_{mask}$.} The complexity of the proposed $\Pi_{mask}$ mainly depends on the number of oblivious swaps. To prune $m$ tokens out of $n$ input tokens, $O(mn)$ swaps are needed. Since token pruning is performed progressively, only a small number of tokens are pruned at each layer, which makes $\Pi_{mask}$ efficient during runtime. Specifically, for a BERT base model with 128 input tokens, the pruning protocol only takes $\sim0.9$s on average in each layer. An alternative approach is to invoke an oblivious sort algorithm~\citep{bogdanov2014swap2,pang2023bolt} on $\left \langle \Bar{x} \right \rangle$. However, this approach is less efficient because it blindly sort the whole token sequence without considering $m$. That is, even if only $1$ token needs to be pruned, $O(nlog^{2}n)\sim O(n^2)$ oblivious swaps are needed, where as the proposed $\Pi_{mask}$ only need $O(n)$ swaps. More generally, for an $\ell$-layer Transformer with a total of $m$ tokens pruned, the overall time complexity using the sort strategy would be $O(\ell n^2)$ while using the swap strategy remains an overall complexity of $O(mn).$ Specifically, using the sort strategy to prune tokens in one BERT Base model layer can take up to $3.8\sim4.5$ s depending on the sorting algorithm used. In contrast, using the swap strategy only needs $0.5$ s. Moreover, alternative to our MSB strategy, one can also swap the encrypted mask along with the encrypted token sequence. However, we find that this doubles the number of swaps needed, and thus is less efficient the our MSB strategy, as is shown in Figure \ref{fig:msb}.

\section{Existing Protocols}
\label{app:protocol}
\noindent\textbf{Existing Protocols Used in Our Private Inference.}  In our private inference framework, we reuse several existing cryptographic protocols for basic computations. $\Pi_{MatMul}$ \citep{pang2023bolt} processes two ASS matrices and outputs their product in SS form. For non-linear computations, protocols $\Pi_{SoftMax}, \Pi_{GELU}$, and $\Pi_{LayerNorm}$\citep{lu2023bumblebee, pang2023bolt} take a secret shared tensor and return the result of non-linear functions in ASS. Basic protocols from~\citep{rathee2020cryptflow2, rathee2021sirnn} are also utilized. $\Pi_{CMP}$\citep{EzPC}, for example, inputs ASS values and outputs a secret shared comparison result, while $\Pi_{B2A}$\citep{EzPC} converts secret shared Boolean values into their corresponding arithmetic values.

\section{Polynomial Reduction for Non-linear Functions}
\label{app:b}
The $\mathsf{SoftMax}$ and $\mathsf{GELU}$ functions can be approximated with polynomials. High-degree polynomials~\citep{lu2023bumblebee, pang2023bolt} can achieve the same accuracy as the LUT-based methods~\cite{hao2022iron-iron}. While these polynomial approximations are more efficient than look-up tables, they can still incur considerable overheads. Reducing the high-degree polynomials to the low-degree ones for the less important tokens can imporve efficiency without compromising accuracy. The $\mathsf{SoftMax}$ function is applied to each row of an attention map. If a token is to be reduced, the corresponding row will be computed using the low-degree polynomial approximations. Otherwise, the corresponding row will be computed accurately via a high-degree one. That is if $M_{\beta}'[i] = 1$, $P_0$ and $P_1$ uses high-degree polynomials to compute the $\mathsf{SoftMax}$ function on token $x[i]$:
\begin{equation}
\mathsf{SoftMax}_{i}(x) = \frac{e^{x_i}}{\sum_{j\in [d]}e^{x_j}}
\end{equation}
where $x$ is a input vector of length $d$ and the exponential function is computed via a polynomial approximation. For the $\mathsf{SoftMax}$ protocol, we adopt a similar strategy as~\citep{kim2021ibert, hao2022iron-iron}, where we evaluate on the normalized inputs $\mathsf{SoftMax}(x-max_{i\in [d]}x_i)$. Different from~\citep{hao2022iron-iron}, we did not used the binary tree to find max value in the given vector. Instead, we traverse through the vector to find the max value. This is because each attention map is computed independently and the binary tree cannot be re-used. If $M_{\beta}[i] = 0$, $P_0$ and $P_1$ will approximate the $\mathsf{SoftMax}$ function with low-degree polynomial approximations. We detail how $\mathsf{SoftMax}$ can be approximated as follows:
\begin{equation}
\label{eq:app softmax}
\mathsf{ApproxSoftMax}_{i}(x) = \frac{\mathsf{ApproxExp}(x_i)}{\sum_{j\in [d]}\mathsf{ApproxExp}(x_j)}
\end{equation}
\begin{equation}
\mathsf{ApproxExp}(x)=\begin{cases}
    0  &\text{if}\ x \leq T \\
    (1+ \frac{x}{2^n})^{2^n} &\text{if}\ x \in [T,0]\\
\end{cases}
\end{equation}
where the $2^n$-degree Taylor series is used to approximate the exponential function and $T$ is the clipping boundary. The value $n$ and $T$ determines the accuracy of above approximation. With $n=6$ and $T=-13$, the approximation can achieve an average error within $2^{-10}$~\citep{lu2023bumblebee}. For low-degree polynomial approximation, $n=3$ is used in the Taylor series.

Similarly, $P_0$ or $P_1$ can decide whether or not to approximate the $\mathsf{GELU}$ function for each token. If $M_{\beta}[i] = 1$, $P_0$ and $P_1$ use high-degree polynomials~\citep{lu2023bumblebee} to compute the $\mathsf{GELU}$ function on token $x[i]$ with high-degree polynomial:
% \begin{equation}
% \mathsf{GELU}(x) = 0.5x(1+\mathsf{Tanh}(\sqrt{2/\pi}(x+0.044715x^3)))
% \end{equation}
% where the $\mathsf{Tanh}$ and square root function are computed via a OT-based lookup-table.

\begin{equation}
\label{eq:app gelu}
\mathsf{ApproxGELU}(x)=\begin{cases}
    0  &\text{if}\ x \leq -5 \\
    P^3(x), &\text{if}\ -5 < x \leq -1.97 \\
    P^6(x), &\text{if}\ -1.97 < x \leq 3  \\
    x, &\text{if}\ x >3 \\
\end{cases}
\end{equation}
where $P^3(x)$ and $P^6(x)$ are degree-3 and degree-6 polynomials respectively. The detailed coefficient for the polynomial is: 
\begin{equation*}
    P^3(x) = -0.50540312 -  0.42226581x - 0.11807613x^2 - 0.01103413x^3
\end{equation*}
, and
\begin{equation*}
    P^6(x) = 0.00852632 + 0.5x + 0.36032927x^2 - 0.03768820x^4 + 0.00180675x^6
\end{equation*}

For BOLT baseline, we use another high-degree polynomial to compute the $\mathsf{GELU}$ function.

\begin{equation}
\label{eq:app gelu}
\mathsf{ApproxGELU}(x)=\begin{cases}
    0  &\text{if}\ x < -2.7 \\
    P^4(x), &\text{if}\   |x| \leq 2.7 \\
    x, &\text{if}\ x >2.7 \\
\end{cases}
\end{equation}
We use the same coefficients for $P^4(x)$ as BOLT~\citep{pang2023bolt}.

\begin{figure}[h]
 % \vspace{-0.1in}
    \centering
    \includegraphics[width=1\linewidth]{figures/bumble.pdf}
    % \captionsetup{skip=2pt}
    % \vspace{-0.1in}
    \caption{Comparison with prior works on the BERT model. The input has 128 tokens.}
    \label{fig:bumble}
\end{figure}

If $M_{\beta}'[i] = 0$, $P_0$ and $P_1$ will use low-degree 
polynomial approximation to compute the $\mathsf{GELU}$ function instead. Encrypted polynomial reduction leverages low-degree polynomials to compute non-linear functions for less important tokens. For the $\mathsf{GELU}$ function, the following degree-$2$ polynomial~\cite{kim2021ibert} is used:
\begin{equation*}
\mathsf{ApproxGELU}(x)=\begin{cases}
    0  &\text{if}\ x <  -1.7626 \\
    0.5x+0.28367x^2, &\text{if}\ x \leq |1.7626| \\
    x, &\text{if}\ x > 1.7626\\
\end{cases}
\end{equation*}


\section{Comparison with More Related Works.}
\label{app:c}
\textbf{Other 2PC frameworks.} The primary focus of CipherPrune is to accelerate the private Transformer inference in the 2PC setting. As shown in Figure \ref{fig:bumble}, CipherPrune can be easily extended to other 2PC private inference frameworks like BumbleBee~\citep{lu2023bumblebee}. We compare CipherPrune with BumbleBee and IRON on BERT models. We test the performance in the same LAN setting as BumbleBee with 1 Gbps bandwidth and 0.5 ms of ping time. CipherPrune achieves more than $\sim 60 \times$ speed up over BOLT and $4.3\times$ speed up over BumbleBee.

\begin{figure}[t]
 % \vspace{-0.1in}
    \centering
    \includegraphics[width=1\linewidth]{figures/pumab.pdf}
    % \captionsetup{skip=2pt}
    % \vspace{-0.1in}
    \caption{Comparison with MPCFormer and PUMA on the BERT models. The input has 128 tokens.}
    \label{fig:pumab}
\end{figure}

\begin{figure}[h]
 % \vspace{-0.1in}
    \centering
    \includegraphics[width=1\linewidth]{figures/pumag.pdf}
    % \captionsetup{skip=2pt}
    % \vspace{-0.1in}
    \caption{Comparison with MPCFormer and PUMA on the GPT2 models. The input has 128 tokens. The polynomial reduction is not used.}
    \label{fig:pumag}
\end{figure}

\textbf{Extension to 3PC frameworks.} Additionally, we highlight that CipherPrune can be also extended to the 3PC frameworks like MPCFormer~\citep{li2022mpcformer} and PUMA~\citep{dong2023puma}. This is because CipherPrune is built upon basic primitives like comparison and Boolean-to-Arithmetic conversion. We compare CipherPrune with MPCFormer and PUMA on both the BERT and GPT2 models. CipherPrune has a $6.6\sim9.4\times$ speed up over MPCFormer and $2.8\sim4.6\times$ speed up over PUMA on the BERT-Large and GPT2-Large models.


\section{Communication Reduction in SoftMax and GELU.}
\label{app:e}

\begin{figure}[h]
    \centering
    \includegraphics[width=0.9\linewidth]{figures/layerwise.pdf}
    \caption{Toy example of two successive Transformer layers. In layer$_i$, the SoftMax and Prune protocol have $n$ input tokens. The number of input tokens is reduced to $n'$ for the Linear layers, LayerNorm and GELU in layer$_i$ and SoftMax in layer$_{i+1}$.}
    \label{fig:layer}
\end{figure}

\begin{table*}[h]
\captionsetup{skip=2pt}
\centering
\scriptsize
\caption{Communication cost (in MB) of the SoftMax and GELU protocol in each Transformer layer.}
\begin{tblr}{
    colspec = {c |c c c c c c c c c c c c},
    row{1} = {font=\bfseries},
    row{2-Z} = {rowsep=1pt},
    % row{4} = {bg=LightBlue},
    colsep = 2.5pt,
    }
\hline
\textbf{Layer Index} & \textbf{0}  & \textbf{1}  & \textbf{2} & \textbf{3} & \textbf{4} & \textbf{5} & \textbf{6} & \textbf{7} & \textbf{8} & \textbf{9} & \textbf{10} & \textbf{11} \\
\hline
Softmax & 642.19 & 642.19 & 642.19 & 642.19 & 642.19 & 642.19 & 642.19 & 642.19 & 642.19 & 642.19 & 642.19 & 642.19 \\
Pruned Softmax & 642.19 & 129.58 & 127.89 & 119.73 & 97.04 & 71.52 & 43.92 & 21.50 & 10.67 & 6.16 & 4.65 & 4.03 \\
\hline
GELU & 698.84 & 698.84 & 698.84 & 698.84 & 698.84 & 698.84 & 698.84 & 698.84 & 698.84 & 698.84 & 698.84 & 698.84\\
Pruned GELU  & 325.10 & 317.18 & 313.43 & 275.94 & 236.95 & 191.96 & 135.02 & 88.34 & 46.68 & 16.50 & 5.58 & 5.58\\
\hline
\end{tblr}
\label{tab:layer}
\end{table*}

{
In Figure \ref{fig:layer}, we illustrate why CipherPrune can reduce the communication overhead of both  SoftMax and GELU. Suppose there are $n$ tokens in $layer_i$. Then, the SoftMax protocol in the attention module has a complexity of $O(n^2)$. CipherPrune's token pruning protocol is invoked to select $n'$ tokens out of all $n$ tokens, where $m=n-n'$ is the number of tokens that are removed. The overhead of the GELU function in $layer_i$, i.e., the current layer, has only $O(n')$ complexity (which should be $O(n)$ without token pruning). The complexity of the SoftMax function in $layer_{i+1}$, i.e., the following layer, is reduced to $O(n'^2)$ (which should be $O(n^2)$ without token pruning). The SoftMax protocol has quadratic complexity with respect to the token number and the GELU protocol has linear complexity. Therefore, CipherPrune can reduce the overhead of both the GELU protocol and the SoftMax protocols by reducing the number of tokens. In Table \ref{tab:layer}, we provide detailed layer-wise communication cost of the GELU and the SoftMax protocol. Compared to the unpruned baseline, CipherPrune can effectively reduce the overhead of the GELU and the SoftMax protocols layer by layer.
}

\section{Analysis on Layer-wise redundancy.}
\label{app:f}

\begin{figure}[h]
    \centering
    \includegraphics[width=0.9\linewidth]{figures/layertime0.pdf}
    \caption{The number of pruned tokens and pruning protocol runtime in different layers in the BERT Base model. The results are averaged across 128 QNLI samples.}
    \label{fig:layertime}
\end{figure}

{
In Figure \ref{fig:layertime}, we present the number of pruned tokens and the runtime of the pruning protocol for each layer in the BERT Base model. The number of pruned tokens per layer was averaged across 128 QNLI samples, while the pruning protocol runtime was measured over 10 independent runs. The mean token count for the QNLI samples is 48.5. During inference with BERT Base, input sequences with fewer tokens are padded to 128 tokens using padding tokens. Consistent with prior token pruning methods in plaintext~\citep{goyal2020power}, a significant number of padding tokens are removed at layer 0.  At layer 0, the number of pruned tokens is primarily influenced by the number of padding tokens rather than token-level redundancy.
%In Figure \ref{fig:layertime}, we demonstrate the number of pruned tokens and the pruning protocol runtime in each layer in the BERT Base model. We averaged the number of pruned tokens in each layer across 128 QNLI samples and then tested the pruning protocol runtime in 10 independent runs. The mean token number of the QNLI samples is 48.5. During inference with BERT Base, input sequences with small token number are padded to 128 tokens with padding tokens. Similar to prior token pruning methods in the plaintext~\citep{goyal2020power}, a large number of padding tokens can be removed at layer 0. We remark that token-level redundancy builds progressively throughout inference~\citep{goyal2020power, kim2022LTP}. The number of pruned tokens in layer 0 mostly depends on the number of padding tokens instead of token-level redundancy.
}

{
%As shown in Figure \ref{fig:layertime}, more tokens are removed in the intermediate layers, e.g., layer $4$ to layer $7$. This suggests there is more redundant information in these intermediate layers. 
In CipherPrune, tokens are removed progressively, and once removed, they are excluded from computations in subsequent layers. Consequently, token pruning in earlier layers affects computations in later layers, whereas token pruning in later layers does not impact earlier layers. As a result, even if layers 4 and 7 remove the same number of tokens, layer 7 processes fewer tokens overall, as illustrated in Figure \ref{fig:layertime}. Specifically, 8 tokens are removed in both layer $4$ and layer $7$, but the runtime of the pruning protocol in layer $4$ is $\sim2.4\times$ longer than that in  layer $7$.
}

\section{Related Works}
\label{app:g}

{
In response to the success of Transformers and the need to safeguard data privacy, various private Transformer Inferences~\citep{chen2022thex,zheng2023primer,hao2022iron-iron,li2022mpcformer, lu2023bumblebee, luo2024secformer, pang2023bolt}  are proposed. To efficiently run private Transformer inferences, multiple cryptographic primitives are used in a popular hybrid HE/MPC method IRON~\citep{hao2022iron-iron}, i.e., in a Transformer, HE and SS are used for linear layers, and SS and OT are adopted for nonlinear layers. IRON and BumbleBee~\citep{lu2023bumblebee} focus on optimizing linear general matrix multiplications; SecFormer~\cite{luo2024secformer} improves the non-linear operations like the exponential function with polynomial approximation; BOLT~\citep{pang2023bolt} introduces the baby-step giant-step (BSGS) algorithm to reduce the number of HE rotations, proposes a word elimination (W.E.) technique, and uses polynomial approximation for non-linear operations, ultimately achieving state-of-the-art (SOTA) performance.
}

{Other than above hybrid HE/MPC methods, there are also works exploring privacy-preserving Transformer inference using only HE~\citep{zimerman2023converting, zhang2024nonin}. The first HE-based private Transformer inference work~\citep{zimerman2023converting} replaces \mysoftmax function with a scaled-ReLU function. Since the scaled-ReLU function can be approximated with low-degree polynomials more easily, it can be computed more efficiently using only HE operations. A range-loss term is needed during training to reduce the polynomial degree while maintaining high accuracy. A training-free HE-based private Transformer inference was proposed~\citep{zhang2024nonin}, where non-linear operations are approximated by high-degree polynomials. The HE-based methods need frequent bootstrapping, especially when using high-degree polynomials, thus often incurring higher overhead than the hybrid HE/MPC methods in practice.
}


\end{document}
