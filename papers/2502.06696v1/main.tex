
% \documentclass[acmlarge, manuscript, anonymous=true, screen, review]{acmart}
% \documentclass[sigconf]{acmart}
\documentclass[acmsmall]{acmart}

\usepackage[utf8]{inputenc}

\PassOptionsToPackage{table,xcdraw}{xcolor} 

\usepackage{makecell}
\usepackage{textgreek}
\usepackage{multirow}
\usepackage{subcaption}
\usepackage{xcolor,colortbl}
\definecolor{gray}{rgb}{0.1,0.1,0.1}
\usepackage[T1]{fontenc}
\usepackage{graphicx}
\usepackage{tabularx}
\usepackage{longtable}
\usepackage{enumitem}
\usepackage{booktabs}
\usepackage{array}
\usepackage{setspace}
% accepted packages: https://authors.acm.org/proceedings/production-information/accepted-latex-packages

\AtBeginDocument{%
  \providecommand\BibTeX{{%
    \normalfont B\kern-0.5em{\scshape i\kern-0.25em b}\kern-0.8em\TeX}}}

\acmConference[Woodstock '18]{Woodstock '18: ACM Symposium on Neural
  Gaze Detection}{June 03--05, 2018}{Woodstock, NY}
\acmBooktitle{Woodstock '18: ACM Symposium on Neural Gaze Detection,
  June 03--05, 2018, Woodstock, NY}
\acmPrice{15.00}
\acmISBN{978-1-4503-XXXX-X/18/06}

\acmSubmissionID{123-A56-BU3}

\usepackage[utf8]{inputenc}

%% full size page
\usepackage{fullpage}
\usepackage[margin = 2.5cm]{geometry}

%% AMS packages
\usepackage{amsmath, amsthm, amssymb}
\usepackage{thmtools}
\usepackage{thm-restate}
\usepackage{mathtools}  
\usepackage{xfrac} 
%\usepackage[normalem]{ulem}

\usepackage{placeins} 
\usepackage{times}

%%bibliography
% \usepackage{amsthm}
\usepackage{url}
\usepackage{array}


%% algorithm
%\usepackage{algorithm}
%\usepackage{algpseudocode}
\usepackage[ruled,vlined]{algorithm2e}
% Create a new environment named "algorithm2e" that behaves like "algorithm"
\newenvironment{algorithm2e}[1][]{%
    \begin{algorithm}[#1]%
}{%
    \end{algorithm}
}

%% graphics and colors
\usepackage{graphicx}
\usepackage{color}
%\usepackage[dvipsnames]{xcolor}

%%bibliography
\usepackage[round]{natbib}
\usepackage[hyperindex,breaklinks]{hyperref}
\usepackage{url}

%%color box figure
\usepackage{tcolorbox}

%%draw figure
\usepackage{tikz}

%%misc
\usepackage{nicefrac}

%\citestyle{acmauthoryear}

% \declaretheorem[name=Theorem]{theorem}

%% define theorems, lemmas, claims
\newtheorem{theorem}{Theorem}[section]
\newtheorem{claim}[theorem]{Claim}
\newtheorem{corollary}[theorem]{Corollary}
\newtheorem{proposition}[theorem]{Proposition}
\newtheorem{lemma}[theorem]{Lemma}
\newtheorem{definition}[theorem]{Definition}
\newtheorem{observation}[theorem]{Observation}
\newtheorem{question}[theorem]{Question}
\newtheorem{assumption}[theorem]{Assumption}
\newtheorem*{remark*}{Remark}


% \newenvironment{numberedtheorem}[1]{%
% \renewcommand{\thetheorem}{#1}%
% \begin{theorem}}{\end{theorem}\addtocounter{theorem}{-1}}

% \newenvironment{numberedlemma}[1]{%
% \renewcommand{\thetheorem}{#1}%
% \begin{lemma}}{\end{lemma}\addtocounter{theorem}{-1}}

% \newenvironment{oneshot}[1]{\@begintheorem{#1}{\unskip}}{\@endtheorem}

% \makeatletter
% \newtheorem*{rep@theorem}{\rep@title}
% \newcommand{\newreptheorem}[2]{%
% \newenvironment{rep#1}[1]{%
%  \def\rep@title{#2 \ref{##1}}%
%  \begin{rep@theorem}}%
%  {\end{rep@theorem}}}
% \makeatother


%cleveref package loaded at the end
\usepackage{cleveref}
\crefname{theorem}{theorem}{theorems}
\Crefname{theorem}{Theorem}{Theorems}

\crefname{lemma}{lemma}{lemmas}
\Crefname{lemma}{Lemma}{Lemmas}

\crefname{claim}{claim}{claims}
\Crefname{claim}{Claim}{Claims}

\crefname{corollary}{corollary}{corollaries}
\Crefname{corollary}{Corollary}{Corollaries}

\crefname{proposition}{proposition}{propositions}
\Crefname{proposition}{Proposition}{Propositions}

\crefname{definition}{definition}{definitions}  % Explicitly set "Definition"
\Crefname{definition}{Definition}{Definitions}

\crefname{observation}{observation}{observations}
\Crefname{observation}{Observation}{Observations}

\crefname{question}{question}{questions}
\Crefname{question}{Question}{Questions}

\crefname{assumption}{assumption}{assumptions}
\Crefname{assumption}{Assumption}{Assumptions}

\crefname{algorithm}{algorithm}{algorithms}
\Crefname{algorithm}{Algorithm}{Algorithms}

\crefname{AlgoLine}{line}{lines}  % If using `algorithm2e` with line numbers
\Crefname{AlgoLine}{Line}{Lines}


%% probability notation
\DeclareMathOperator{\cov}{cov}
\DeclareMathOperator{\sgn}{\mathbf{sgn}}
\DeclareMathOperator{\E}{\mathbf{E}}
\DeclareMathOperator{\Var}{\mathbf{Var}}
\DeclareMathOperator{\one}{\mathbf{1}}
\newcommand{\given}{\mid}
\DeclareMathOperator{\Ball}{Ball}
\DeclareMathOperator{\tr}{tr}

% rounding up and down
\newcommand {\roundup}   [1] {{\lceil {#1} \rceil}}
\newcommand {\rounddown} [1] {{\lfloor {#1} \rfloor}}

%% black board letters
\newcommand{\bbB}{\mathbb{B}}
\newcommand{\bbC}{\mathbb{C}}
\newcommand{\bbR}{\mathbb{R}}
\newcommand{\bbZ}{\mathbb{Z}}

%% calligraphic letters 
\newcommand{\calA}{\mathcal{A}}
\newcommand{\calB}{\mathcal{B}}
\newcommand{\calC}{\mathcal{C}}
\newcommand{\calD}{\mathcal{D}}
\newcommand{\calE}{\mathcal{E}}
\newcommand{\calF}{\mathcal{F}}
\newcommand{\calG}{\mathcal{G}}
\newcommand{\calH}{\mathcal{H}}
\newcommand{\calI}{\mathcal{I}}
\newcommand{\calJ}{\mathcal{J}}
\newcommand{\calK}{\mathcal{K}}
\newcommand{\calL}{\mathcal{L}}
\newcommand{\calM}{\mathcal{M}}
\newcommand{\calN}{\mathcal{N}}
\newcommand{\calO}{\mathcal{O}}
\newcommand{\calP}{\mathcal{P}}
\newcommand{\calQ}{\mathcal{Q}}
\newcommand{\calR}{\mathcal{R}}
\newcommand{\calS}{\mathcal{S}}
\newcommand{\calT}{\mathcal{T}}
\newcommand{\calU}{\mathcal{U}}
\newcommand{\calV}{\mathcal{V}}
\newcommand{\calW}{\mathcal{W}}
\newcommand{\calX}{\mathcal{X}}
\newcommand{\calY}{\mathcal{Y}}
\newcommand{\calZ}{\mathcal{Z}}

\newcommand{\N}{\mathbb{N}}
\newcommand{\R}{\mathbb{R}}
\DeclareMathOperator*{\argmin}{arg\,min}
\DeclareMathOperator*{\argmax}{arg\,max}




\newcommand{\err}{\mathrm{err}}
\newcommand{\errstar}{\mathrm{err}^*}

\DeclareMathOperator{\REC}{REC}
\DeclareMathOperator{\NRD}{NRD}
\DeclareMathOperator{\FN}{FN}
\DeclareMathOperator{\FP}{FP}

\DeclareMathOperator{\Tr}{Tr}

\newcommand{\conv}{\mathrm{conv}}
\newcommand{\cone}{\mathrm{cone}}
\newcommand{\inner}[2]{\langle #1, #2\rangle}

\begin{document}
\title[{Social Media Isn't Just Instagram:\\A Youth-Envisioned Platform for Meaningful Social Connections}]{Social Media Isn't Just Instagram: A Youth-Envisioned Platform for Meaningful Social Connections}

\author{JaeWon Kim}
\orcid{0000-0003-4302-3221}
\affiliation{%
  \institution{University of Washington}
  \city{Seattle}
  \state{WA}
  \country{USA}}
\email{jaewonk@uw.edu}

\author{Hyunsung Cho}
\orcid{0000-0002-4521-2766}
\affiliation{%
  \institution{Human-Computer Interaction Institute, Carnegie Mellon University}
  \city{Pittsburgh}
  \country{USA}}
\email{hyunsung@cs.cmu.edu}

\author{Fannie Liu}
\orcid{0000-0002-5656-3406}
\affiliation{%
  \institution{JPMorgan Chase \& Co.}
  \city{New York}
  \state{NY}
  \country{USA}
}
\email{fannie.liu@jpmchase.com}

\author{Alexis Hiniker}
\orcid{0000-0003-1607-0778}
\affiliation{%
  \institution{The Information School, University of Washington}
  \city{Seattle}
  \country{USA}}
\email{alexisr@uw.edu}

\renewcommand{\shortauthors}{JaeWon Kim, et al.}

\begin{abstract}
We conducted co-design workshops with 23 participants (ages 15–24) to explore how youth envision an ideal remote social connection. Using the Fictional Inquiry (FI) method within a Harry Potter-inspired narrative, we found that youth perceive a disconnect between platforms labeled as ``social media'' (like Instagram) and those where they actually experience meaningful connections (like Minecraft or Discord). Participants envisioned an immersive 3D platform that would bridge this gap by prioritizing meaningful social connections, enabling time that feels well spent through presence and immersion, natural individual expression, intuitive social navigation that leverages physical-world norms, and playful, low-stakes opportunities for gradual friendship development. We introduce the design framework of \textit{spatial integrity}, which encompasses four dimensions of spatial affordances that facilitate meaningful social connections online. The FI method proved effective in generating innovative ideas while empowering youth by fostering a sense of hope and agency over the future of social media through their own design contributions.
\end{abstract}

\begin{CCSXML}
<ccs2012>
   <concept>
       <concept_id>10003120.10003130</concept_id>
       <concept_desc>Human-centered computing~Collaborative and social computing</concept_desc>
       <concept_significance>500</concept_significance>
       </concept>
 </ccs2012>
\end{CCSXML}

\ccsdesc[500]{Human-centered computing~Collaborative and social computing}

\keywords{social media; fictional inquiry; youth; design}

\maketitle

\section{Introduction}\label{sec:intro}

In computational finance, Monte Carlo simulations are used extensively to estimate the expected value of financial payoffs based on the solution of stochastic differential equations (SDEs) which model the evolution of stock prices, interest rates, exchange rates and other quantities \cite{glasserman04}.  Monte Carlo methods are very general and flexible, but for high accuracy it requires generating a large number of costly SDE path approximations, which has motivated research into a number of variance reduction or, equivalently, cost reduction techniques. One such method is
Multilevel Monte Carlo (MLMC), which was proposed in \cite{GILES2008} and was adapted for various applications that are summarised in \cite{Giles_overview17} and successfully combined with other methods such as quasi-Monte Carlo methods. The main idea of MLMC is to approximate the payoff using different time stepping resolutions when numerically solving the underlying SDE and to generate an optimal number of samples on each level, such that the overall computational cost is minimised subject to the desired bound on the variance. %, such that the total computational cost is minimised. 
The computational savings come from the fact that most samples are computed on the coarser levels and hence are less expensive while only a few samples from the finest levels are required \cite{GILES2008}.


Among the directions in which the computational cost 
of MLMC methods could further be reduced, an important avenue is the use of lower precision calculations, especially for the first Monte Carlo levels where the targeted accuracy is relatively low. 
 An overview of the research on mixed precision for the standard Monte Carlo (MC) framework is provided in \cite{ChowMixedPrecisionStandardMC} but only a few references study the potential of low precision computation in the MLMC framework \cite{Rounding_error_oliver}. To the best of our knowledge, the only MLMC framework with customised precision in the literature is \cite{brugger2014mixed}, but they use a uniform precision for all operations on each Monte Carlo level instead of optimising 
 the precision of each intermediary variable to reduce as much as possible the cost of path generation.
 
An important motivation for an MLMC framework with variable precision would be performing the low precision computations on reconfigurable hardware devices such as Field Programmable Gate Arrays (FPGAs). FPGAs contain customizable logic blocks and connectors that make it easy to adapt the digital circuit architecture for a specific application, leading to a highly parallel and optimised implementation. Therefore they are successfully exploited in applications that require high speed and have high computational workload, such as signal processing \cite{woods2008fpga}, and real time applications like high frequency trading \cite{HFT1,HFT2}. That is why a number of previous works in hardware architecture design implemented the MLMC algorithm to price financial options using FPGAs as accelerators, which resulted in improved speed and power efficiency compared to full CPU architectures \cite{Schryver2013AMM}. The paper \cite{lindsey2016domain} also proposed 
a Domain Specific Language to automate the configuration of FPGAs for this specific application. However, only \cite{brugger2014mixed} proposed a heuristic to reduce the precision in calculations.

In addition, all aforementioned works considered that the random number generation (RNG) is performed in single or double precision. Yet in most cases an important portion of the workload in the overall MLMC simulation comes from the RNG and in \cite{brugger2014mixed} this limited the total computational savings.
To reduce the cost of MLMC simulations in particular those based on the Geometric Brownian Motion (GBM), \cite{approximateICDF_Oliver, NestedOliver} have proposed to use approximate random numbers that are generated by applying an approximation of the inverse CDF to uniform random numbers. In \cite{NestedOliver}, the authors proposed a way to integrate these lower precision random variables into a \textit{nested} MLMC framework and completed a numerical analysis to bound the resulting error at each MC level by a product of the time step and the error in the random number approximation. The same authors show in \cite{approximateICDF_Oliver} that using approximate random variables reduces the cost of path generation by a factor 7.


In this paper we propose a nested MLMC framework that combines the use of approximate random normal variables and lower precision calculations to reduce the computational cost of MLMC even further than \cite{brugger2014mixed,NestedOliver}. We illustrate the efficiency of our framework in Matlab, after making several assumptions on the cost of operations and size of the errors that we carefully justify. We focus on the case of GBM and use the approximate RNG methods presented in \cite{approximateICDF_Oliver} as well as a new slightly modified method that combines CDF inversion and the central limit theorem. To choose the precision of the variables in the low precision path generation, we introduce a novel method to optimise the bit-widths. This optimisation is performed before the main path generation loop is executed and is based on a linear model of the payoff error  
due to rounding when computing in low precision. The error model relies on algorithmic differentiation in a similar manner to \cite{unifying-bwoptim,bitwidth-AD,ADAPT}. The bit-width optimisation procedure can be performed off-line, so this stage can be excluded from the on-line time complexity of our framework. The user specified desired accuracy is then enforced by calculating on-line the number of samples that need to be generated.

In terms of hardware design, we suggest implementing the low precision path generation on FPGAs and the full-precision ones on a CPU or GPU. 
The FPGA offers enough flexibility to define a separate bit-width for every variable in the low precision path generation, and can be reconfigured periodically to update the bit-widths when the market parameters have changed considerably. 


The paper is organized as follows : \Cref{sec:MLMC} introduces MLMC and nested MLMC to make clear the estimator that is implemented in our framework. Then in \Cref{sec:RNG} we detail the methods that could be used to obtain approximate random normally distributed numbers very cheaply for the low precision path generation. In \Cref{sec:error_model} and \Cref{sec:costModel} we propose an error model and a cost model (resp.) that we then use to formulate the optimisation problem that is solved to obtain the optimal bit-widths of fixed point variables in \Cref{sec:optimisation}. Finally we summarise our results and future directions in \Cref{sec:conclusion}.



\section{Related Work}
Dense retrievers~\cite{karpukhin2020dense,xiong2021approximate,izacard2021unsupervised,Yu2021FewShotCD,xiong2021dense,li2021more} have demonstrated superior ranking performance by conducting semantic matching between queries and documents, which helps overcome the problem of vocabulary mismatch~\cite{belkin1982ask}. These models typically leverage Pre-trained Language Models (PLMs) to encode both queries and documents into a shared embedding space, which significantly enhances retrieval effectiveness. To further optimize this embedding space, dense retrievers are often trained contrastively using relevance signals between queries and documents~\cite{karpukhin2020dense,zhan2021optimizing}. Some studies have also developed zero-shot dense retrievers by training on weak supervision data~\cite{xie2023unsupervised} or leveraging Large Language Models (LLMs) for query expansion and reformulation~\cite{gao2022unsupervised}.


Query expansion is a long-standing research direction, originally proposed to rewrite queries and improve the retrieval accuracy of exact matching-based retrievers, such as BM25~\cite{robertson2009probabilistic}. Early query expansion methods primarily aim to bridge the lexical gap between queries and documents~\cite{carpineto2012survey,rocchio1971relevance} by expanding queries with knowledge bases~\cite{bhogal2007review,qiu1993concept,voorhees1994query} or Pseudo-Relevance Feedback (PRF)~\cite{amati2002probabilistic,robertson1990term,rocchio1971relevance}. These PRF-based methods have proven their effectiveness in enhancing reranking techniques~\cite{li2018nprf,ai2018learning,yu2021pgt} and improving dense retrievers by learning better query representations~\cite{yu2021improving}. 


Recent research of query expansion focuses on Generative Relevance Feedback (GRF) methods, which utilize generation models to directly produce query expansion results~\cite{mackie2023generative, claveau2021neural, wang2023generative, jagerman2023query, mackie2023gprf}. These methods often employ LLMs to generate query-related documents~\cite{wang2023query2doc, jagerman2023query, gao2023precise}, leverage Chain-of-Thought (CoT) reasoning results~\cite{wei2022chain, jagerman2023query, trivedi2023interleaving}, or utilize specific keywords~\cite{li2024can, jagerman2023query} to expand queries, thereby enhancing the ranking capabilities of lexical matching based retrieval models~\cite{jagerman2023query, wang2023query2doc}, dense retrieval models~\cite{wang2023query2doc}, and reranking models~\cite{li2024can}.


Although these studies have demonstrated their advantages in improving retrieval models, directly using LLMs for query expansion and reformulation poses potential risks due to LLM hallucinations~\cite{shuster2021retrieval,huang2023survey}. To mitigate this issue, some works use different instructions to reduce inconsistency in query reformulation~\cite{gao2023precise} or leverage rephrased questions as demonstrations to guide LLMs in generating more effective query expansion results~\cite{koo2024optimizing}. RaFe~\cite{mao2024rafe} further enhances the Retrieval-Augmented Generation (RAG) performance by taking reranking scores as training signals to optimize the query rewriting model. In contrast to these approaches, LLM-QE focuses on modeling the ranking preferences of both retrievers and LLMs, rewarding LLMs for generating more effective expansion results, and exploring its potential to build both unsupervised and supervised dense retrieval models.




\section{Methodology}
\paragraph{Preliminaries.}
We primarily focus on the homologous model merging, in which $\boldsymbol{\theta}_i$ all come from the same base model $\boldsymbol{\theta}_{\rm{base}}$. Given $K$ tasks $\{T_1,T_2,\cdots,T_K\}$ and $K$ corresponding fine-tuned models with parameters $\{\boldsymbol{\theta}_1,\boldsymbol{\theta}_2,\cdots,\boldsymbol{\theta}_K\}$, model merging aims to combine $K$ fine-tuned models into one single model simultaneously performing on $\{T_1,T_2,\cdots,T_K\}$ without post-training~\cite{method_p1_1,method_p1_2}.
Task vector~\cite{ilharco2023editing,yang2024adamerging} is a key element in merging method which could enhances the base model‘s ability or enable the model to handle other tasks. Specifically, for task $T_i$, the task vector $\boldsymbol\tau_i\in \mathbb{R}^D$ is defined as the vector obtained by subtracting the SFT weights $\boldsymbol{\theta}_i$ from the base model weight
$\boldsymbol{\theta}_{\rm{base}}$, \emph{i.e.}, $\boldsymbol\tau_i=\boldsymbol{\theta}_i-\boldsymbol{\theta}_{\rm{base}}$. The merged model could be denoted as $\boldsymbol{\theta}_m=\boldsymbol{\theta}_{\rm{base}}+\sum_i \lambda_i\boldsymbol{\tau}_i$, which $\lambda_i$ is the scaling factor measuring the importance of task vector. For clarification, we also denote the neuron set in $\boldsymbol{\theta}_i$ as $\mathcal{N}_i$, the neuron set in $\boldsymbol{\tau}_i$ as $\mathcal{T}_i$.



\begin{algorithm}[!ht]
    \caption{LED-Merging}
    \label{alg1}
    \begin{algorithmic}[1]
        \REQUIRE  base model $\boldsymbol{\theta}_{\rm{base}}$, SFT models $\{\boldsymbol{\theta}_{i}\mid i\in [K]\}$, mask ratios \{$r_{i} \mid i\in [K]\}$, scaling factors $\{\lambda_i\mid i\in[K]\}$, location datasets $\{\mathcal{X}_{i}\mid i\in[K]\}$
        \ENSURE merged parameter $\boldsymbol{\theta}_{m}$
        \STATE $\mathcal{M}\leftarrow\phi$
        \STATE $\boldsymbol{\theta}_{m}\leftarrow \boldsymbol{\theta}_{\rm{base}}$
        \FOR{$i\in [K]$}
        \STATE $I(\boldsymbol{\theta}_i)=\mathbb{E}_{x\sim \mathcal{X}_i}|\boldsymbol{\theta}_{i}\odot \nabla_{\boldsymbol{\theta}_i}\mathcal{L}(x)|$
        \STATE $I(\boldsymbol{\theta}_{\rm{base}})=\mathbb{E}_{x\sim \mathcal{X}_i}|\boldsymbol{\theta}_{\rm{base}}\odot \nabla_{\boldsymbol{\theta}_{\rm{base}}}\mathcal{L}(x)|$
        
        \STATE calculate $\mathcal{T}^{r_i}_{i}$ following Equation \ref{vote}
        \STATE  $\mathcal{M}\leftarrow \mathcal{M}\cup\{\mathcal{T}^{r_i}_i\}$
       
        
   
        
        
        \ENDFOR  
        \FOR{$i\in [K]$}
        
        \STATE calculate $\text{Disjoint}(\mathcal{T}_i^{r_i})$ use Equation~\ref{disjoint_safety}
        \STATE $\boldsymbol{m}_i \leftarrow \boldsymbol{0}$
        \FOR{$d\in \mathcal{T}_i^{r_i}$}
        \STATE $\boldsymbol{m}_{i,d}=1$
        \ENDFOR
        \STATE $\boldsymbol{\theta}_{m}\leftarrow \boldsymbol{\theta}_{m}+\lambda_i \boldsymbol{\tau}_i\odot \boldsymbol{m}_{i}$
        \ENDFOR
    \end{algorithmic}
\end{algorithm}
    %\vspace{-5pt}
\begin{figure*}[h!]
    \centering
    \includegraphics[width=\linewidth]{figs/pipeline_v2.pdf}
    \vspace{-40mm}
    \caption{Overview of our two-stage training pipeline {\ours}.}
    \label{fig:pipeline}
\end{figure*}


\paragraph{LED-Merging: Location, Election, and Disjoint Merging}
To address the neuron misidentification and interference issues in existing model merging methods, we propose LED-Merging (Location, Election, and Disjoint Merging). Specifically, previous studies \cite{modelstock, ilharco2023editing, tiesmerging} fail to accurately identify safety-related neurons in task vectors with a single magnitude score, namely \textit{neuron misidentification}. Meanwhile, there exists an interference between safety-related and utility-related task vector neurons during the merging process, namely \textit{neuron interference}. To address neuron misidentification, we first locate important neurons both in the base and fine-tuned models and then elect neurons from the task vector considering these two scores together. Subsequently, to mitigate the interference, we introduce a disjoint step, isolating these important neurons so that they influence different base neurons. The whole process is illustrated in Figure~\ref{fig:method}. 




In the location and election step, we consider the importance score from base and fine-tuned models simultaneously to locate task-specific neurons. In this way, it is more accurate than relying on the magnitude score alone because task-specific neurons with high importance score in the fine-tuned model may not necessarily score high in the base model, and vice versa.

{\textbf{Location}}.  We first calculate importance scores for each neuron in a base/fine-tuned model. Given a location dataset $\mathcal{X}_i=\{(x,y)_k\}$, where $x$ is the question and $y$ is the answer, we calculate the importance scores for the weight $\boldsymbol{\theta}_i\in\mathbb{R}^D$ in any  layer as follows~\cite{snip,spareseGPT,sun2024a}:
\begin{equation}
    I(\boldsymbol{\theta}_i)=\mathbb{E}_{x\sim \mathcal{X}_i}[\boldsymbol{\theta}_i\odot \nabla _{\boldsymbol{\theta}_i}\mathcal{L}(x)],
    \label{location}
\end{equation}
which $\mathcal{L}(x)=-\log p(y\mid x)$ is the conditional negative log-likelihood loss. We choose the SNIP score~\cite{snip} because it balances computational efficiency and performance~\cite{cq}. Please refer to Sec.~\ref{sec:ablation} for the comparison between different location methods. After computing importance scores, we choose top-$r_i$ neurons as the important neuron subset $\mathcal{N}_{i}^{r_i}$ from $I(\boldsymbol{\theta}_i)$.
 
 % After computing locating scores, we select the neurons scoring both high in base and fine-tuned models as important neurons in task vectors. Then in the disjoint step,  with preventing  polysemantic neurons  from receiving gradient updates towards different directions,
 % we use set difference to isolate the safety   and utility-related neurons  and construct corresponding masks for merging process,

{\textbf{Election}}. A natural question is how to select important neurons in the task vector $\boldsymbol{\tau}_i$ based on $I(\boldsymbol{\theta}_{\rm{base}})$ and $I(\boldsymbol{\theta}_{i})$. The important neurons in the base model may be different from neurons in the fine-tuned model. Therefore, we introduce the following election strategy to select neurons with high scores in both base and fine-tuned models:
\begin{equation}
    \mathcal{T}_i^{r_i}=\mathcal{N}_i^{r_i}\cap \mathcal{N}_{\rm{base}}^{r_i}.
    \label{vote}
\end{equation}
\emph{Remark}. We compare different choosing methods, including scoring low or high in base or fine-tuned model in Section~\ref{sec:ablation} and find that Equation \ref{vote} achieves the best performance.





{\textbf{Disjoint}}. As important neurons from different task vectors may conflict with each other at the same position, we use the set difference to disjoint the neurons from others to prevent interference:
\begin{equation}
    \text{Disjoint}(\mathcal{T}^{r_i}_{i})=\mathcal{T}^{r_i}_{i}-\mathop{\cup}\limits_{{J}\subsetneqq [K],|J|\geq 2}\mathop{\cap}\limits_{j\in {J}}\mathcal{T}^{r_j}_{j}.
    \label{disjoint_safety}
\end{equation}

Next, we construct a mask $\boldsymbol{m}_i\in\mathbb{R}^D$ to implement disjoint in the merging process. Specifically, this mask $\boldsymbol{m}_i$ is used to select neurons from $\mathcal{T}_i$. The mask ratio is $r_i$, where $r\in(0,1]$. The mask $\boldsymbol{m}_i$ can be derived from:
\begin{equation}
    \boldsymbol{m}_{i,d}=\begin{aligned} &\left\{ \begin{array}{ll} 1, & \text{if } d\in \text{Disjoint}(\mathcal{T}_{i}^{r_i}), \\ 0, & \text{otherwise}. \end{array} \right. \end{aligned}
    \label{mask_safety}
\end{equation}


% \subsection{Merging Models with Masks}
{\textbf{Merging}}. The final
merged task vector $\boldsymbol{\tau}_m$ is as follows:
\begin{equation}
    \boldsymbol{\tau}_m= \sum_i \lambda_i\boldsymbol{\tau}_{i}\odot\boldsymbol{m}_i.
    \label{merged_task_vector}
\end{equation}
We summarize the workflow in Algorithm \ref{alg1}.



\section{Result}
For the evaluation of MicroViT, the ImageNet-1K dataset \cite{russakovsky2015imagenet} comprising 1.28 million training images and 50,000 validation images over 1,000 categories was employed. Following the DeiT training method \cite{touvron2021training}, models were trained for 300 epochs at a 224×224 resolution with an initial learning rate of 0.004, utilizing various data augmentations. The AdamW optimizer \cite{loshchilov2017decoupled} was used with a batch size of 512 across three A6000 GPUs.

We assessed model throughput in various computation environments, including a GPU (RTX-3090), a CPU (Intel i5-13500), and specifically the Jetson Orin Nano edge device. For throughput, the GPU and CPU had a batch size of 256, whereas the edge device used a batch size of 64 with ONNX Runtime. To enhance performance during inference, BN layers were fused with adjacent layers when possible. On the Jetson Orin Nano, we also examined power and energy usage during latency tests with 1000 images at a consistent resolution.

We further evaluate MicroViT on object detection on the COCO dataset \cite{lin2014microsoft} utilizing RetinaNet \cite{ross2017focal} and conduct training for 12 epochs (1$\times$ schedule), adhering to the configuration used by \cite{liu2023efficientvit} in mmdetection \cite{mmdetection}. In the object detection experiments, we employ AdamW \cite{loshchilov2017decoupled} with a batch size of 16, a learning rate of $1\times10^{-3}$, and a weight decay rate of 0.025. 


\begin{table}[!ht]
\centering
\caption{Comparison of All MicroViT Variant with SOTA on ImageNet-1K Dataset. Res, Par and FLPs denotes as input resolution, parameters and Floating Operation. GPU and CPU denotes a inference throughput (img/s) in device respectively.}
\begin{tabular}{ m{2.6cm}|c|c|c|c|>{\centering}m{0.5cm}|c }
\hline
Model   & Res  & Par & FLPs &  GPU  & CPU &   Top-1      \\ \hline
% Fasternet-T1\cite{liu2022convnet}              & 224 & 7.6  & 0.85 & \textbf{7037}  & 149.3  &   & 76.2      \\
MobileNetV2-1.0\cite{sandler2018mobilenetv2}& 224 & 3.5 & 0.314 & 4527 & 82 & 72.0   \\
MobileViT-XXS\cite{mehta2021mobilevit}  & 256 & 1.3 & 0.261 & 3218 & 99  & 69.0   \\
MobileViTV2-0.5\cite{mehta2022separable}  & 256 & 1.4 & 0.480 & 3885 & 68  & 70.2   \\
EdgeNeXt-XXS\cite{maaz2022edgenext}  & 256 & 1.3 & 0.261 & 3975 & 245  & 71.2   \\
Fasternet-T0\cite{chen2023run}  & 224 & 3.9 & 0.340 & 11775 & 311 & 71.9   \\
SHViT-S1\cite{yun2024shvit} & 224  & 6.3 & 0.241 & 15280 & 475 & 72.8 \\
\rowcolor{gray!30}
\textbf{MicroViT-S1}                    & 224  & 6.4 & 0.231 & 17466 & 552 & 72.6 \\ \hline 
EFormerV2-S0\cite{li2023rethinking}& 224  & 3.6 & 0.407 & 1191 & 91 & 73.7 \\ 
EdgeNeXt-XS\cite{maaz2022edgenext}   & 256 & 2.3 & 0.536 & 2935 & 139  & 75.0   \\
EfficientViT-M4\cite{liu2023efficientvit} & 224 & 8.8 & 0.303 & 10093 & 379 & 74.3   \\
MobileViT-XS\cite{mehta2021mobilevit}   & 256 & 2.3 & 0.935 & 1740 & 43 & 74.8   \\
MobileNetV3-L\cite{howard2019searching}& 224 & 3.5 & 0.314 & 4527 & 82 & 75.2  \\
SHViT-S2\cite{yun2024shvit}            & 224  & 11.5 & 0.366 & 12007 & 367 & 75.2 \\
\rowcolor{gray!30}
\textbf{MicroViT-S2}                    & 224 & 10.0 & 0.345 & 14154 & 435 & 74.6 \\ \hline
FastViT-T8\cite{vasu2023fastvit}   & 256 & 4.0 & 0.687 & 3719 & 83 & 76.2   \\
Fasternet-T1\cite{chen2023run}  & 224 & 7.6 & 0.851 & 7151 & 130 & 76.2   \\
EfficientViT-M5\cite{liu2023efficientvit}& 224 & 12.5 & 0.526 & 6807 & 233 & 77.1   \\
SHViT-S3\cite{yun2024shvit}            & 224  & 14.1 & 0.601 & 8180 & 224 & 77.8 \\
% \hline 
\rowcolor{gray!30}
\textbf{MicroViT-S3}                   & 224 & 16.7 & 0.580 & 9288 & 232 & 77.1 \\ \hline 
    \end{tabular}
    \label{tab:imgnet-result}
\end{table}
\subsection{ImageNet-1K Classification Result}
Table \ref{tab:imgnet-result} presents a comparison of various MicroViT variants with state-of-the-art (SOTA) models on the ImageNet-1K dataset. The evaluation focuses on models' computational efficiency and accuracy, highlighting the trade-offs between resource consumption and performance. 

MicroViT-S1 demonstrated superior performance compared to traditional CNN models, surpassing MobileNetV2-1.0\cite{sandler2018mobilenetv2} and Fasternet-T0\cite{mehta2022separable}, with a $3.6 \times$ faster in GPU and $6.7 \times$ in CPU throughput, while maintaining an accuracy advantage of 0.8 over MobileNetV2-1.0. Additionally, MicroViT-S2 outperformed mobile transformers like EfficientFormerV2-S0\cite{li2023rethinking} and EfficientViT-M4\cite{liu2023efficientvit}, achieving $0.3\%$ better accuracy with similar efficiency metrics. Across the MicroViT models, CPU throughput is robust, notably with MicroViT-S1 achieving 552 img/s, which is $8 \times$ faster than several EfficientViT variants, illustrating MicroViT's adaptability to both high-end GPU and CPU settings.


Table \ref{tab:edge-result} presents the performance of MicroViT variants against various SOTA models on the Edge device using ONNX. MicroViT-S1's throughput reaches 773 img/s, efficiently managing large image volumes on the Jetson Orin Nano. This surpasses several SOTA models like MobileViT-XS\cite{mehta2021mobilevit} and EfficientFormer-V2-S0\cite{li2023rethinking}, making MicroViT-S1 optimal for rapid image processing applications. Furthermore, it has a 9.1 ms latency, outperforming MobileNetV2-1.0\cite{sandler2018mobilenetv2} and EdgeNeXt-XS\cite{maaz2022edgenext}, supporting real-time use. It consumes 2147 Joules, achieving high energy efficiency with $\eta=3.7$. Likewise, MicroViT-S2 and MicroViT-S3 balance throughput and energy use, maintaining accuracy, thus ideal for resource-constrained edge devices with superior power efficiency over other lightweight vision transformers.
\begin{table}[!ht]
\centering
\caption{Comparison of All MicroViT Variant and SOTA on ImageNet-1K Dataset with NVIDIA Jetson Orin Nano Edge Device using ONNX format.}
\begin{tabular}{ m{2.6cm}|>{\centering}m{0.5cm}|>{\centering}m{0.6cm}|>{\centering}m{0.9cm}|>{\centering}m{0.8cm}|c }
\hline
\multirow{2}{*}{Model} & Thg & Lat & Avg Pow & Energy  & $\eta$ \\ 
         & img/s & (ms) & (W) & (Joule) & \% / J \\ \hline
MobileNetV2-1.0\cite{sandler2018mobilenetv2} & 234 & 6.7 & 3549 & 23.9 & 3.01 \\
MobileViT-XXS\cite{mehta2021mobilevit}  & 184 & 9.6 & 3428 & 32.4 & 2.13   \\
MobileViTV2-0.5\cite{mehta2022separable}  & 208 & 10.9 & 2887 & 31.6 & 2.22 \\
EdgeNeXt-XXS\cite{maaz2022edgenext}     & 257 & 8.1 & 2805 & 22.6 &  3.15  \\
Fasternet-T0\cite{chen2023run}   & 675 & 8.4 & 2419 & 20.4 &  3.52 \\
SHViT-S1 \cite{yun2024shvit}       & 813  & 12.6 & 2069 & 26.0 & 2.80  \\
\rowcolor{gray!30}
\textbf{MicroViT-S1}                & 773  & 9.1 & 2147 & 19.6 & 3.7 \\ \hline 
EFormerV2-S0\cite{li2023rethinking}  & 257  & 10.9 & 2847 & 31.1 & 2.37   \\ 
EdgeNeXt-XS\cite{maaz2022edgenext}      & 168 & 11.1 & 3031 & 33.6 &  2.32   \\
EfficientViT-M4\cite{liu2023efficientvit} & 587 & 18.0 & 2019 & 36.3 &  2.05   \\
MobileViT-XS\cite{mehta2021mobilevit}   & 96.6 & 14.3 & 3730 & 53.3 & 1.40    \\
MobileNetV3-L\cite{howard2019searching} & 310 & 8.3 & 2743 & 22.6 &  3.33 \\
SHViT-S2 \cite{yun2024shvit}            & 598  & 12.7 & 2306 & 29.4 & 2.56  \\
\rowcolor{gray!30}
\textbf{MicroViT-S2}                      & 567 & 9.9 & 2481 & 24.3 & 3.07 \\ \hline
FastViT-T8\cite{vasu2023fastvit}          & 176 & 8.6 & 3622 & 30.8 &  2.47  \\
Fasternet-T1\cite{chen2023run}     & 421 & 8.8 & 2860 & 25.3 &  3.01  \\
EfficientViT-M5\cite{liu2023efficientvit} & 409 & 21.0 & 2180 & 45.9 & 1.68    \\
SHViT-S3\cite{yun2024shvit}               & 425  & 14.6 & 2500 & 36.6 & 2.13  \\
% \hline 
\rowcolor{gray!30}
\textbf{MicroViT-S3}                    & 398 & 10.9 & 2609 & 28.6 & 2.69  \\ \hline 
    \end{tabular}
    \label{tab:edge-result}
\end{table}

\subsection{Object Detection Result}
We compare MicroViT-3 with efficient models \cite{sandler2018mobilenetv2, howard2019searching, liu2023efficientvit} on the COCO \cite{lin2014microsoft} object detection task, and present the results in Table \ref{tab:obj} Specifically, MicroViT-3 surpasses MobileNetV2 \cite{sandler2018mobilenetv2} by
7.7\% AP with comparable Flops. Compared to the EfficientViT-M4, our MicroViT-3 uses 46.8\% fewer Flops while achieving 3.3\% higher AP, demonstrating its capacity and generalization ability in different vision tasks.

\begin{table}[!ht]
\centering
\caption{Object detection on COCO val2017 with RetinaNet. $AP^b$ denote bounding box average precision. The FLOPs (G) are measured at resolution 1280 $\times$ 800.
}
\begin{tabular}{ m{2.4cm}|>{\centering}m{0.8cm}|>{\centering}m{0.8cm}|>{\centering}m{0.8cm}|>{\centering}m{0.8cm}|c}
\hline
Backbone & $AP^{b}$  & $AP^{b}_{50}$ & $AP^{b}_{75}$ & Par & FLPs  \\ \hline
MobileNetV2\cite{sandler2018mobilenetv2} & 28.3& 46.7& 29.3& 3.4& 300 \\
MobileNetV3\cite{howard2019searching} & 29.9& 49.3& 30.8& 5.4 & 217   \\ 
EfficientViT-M4\cite{liu2023efficientvit}&32.7 &52.2 &34.1 & 8.8 & 299  \\
\rowcolor{gray!30}
\textbf{MicroViT-S3} & 36.0 & 56.6 & 38.2 & 26.7 & 159    \\ \hline

\end{tabular}
\label{tab:obj}
\end{table}

\subsection{Ablation Study}

Table \ref{tab:abl-result} assesses the effects of three ablations on the MicroViT-S2 model as the baseline. The first ablation, low resolution SA with SR=2, slightly increases the parameter count from 11.0M to 11.1M, with GPU throughput remaining nearly unchanged from 12009 to 12029 images per second. However, it increases latency from 9.9 ms to 10.7 ms and decreases Top-1 accuracy to 72.5\%, resulting in a minor drop in efficiency ($\eta$) from 3.0 to 2.8. This indicates a small trade-off between accuracy and computational cost due to the architectural modification.

The Next ablation, without group convolutions, increases the model's complexity significantly to 20.5M parameters, with higher GFLOPs. This results in lower throughput and efficiency ($\eta=1.8$) but achieves the highest accuracy. However, this variant consumes more energy, making it ideal for use cases where accuracy is prioritized over efficiency.

The ablation study shows that even though spatial reduction can increase throughput, the latency is increased, resulting in an efficiency drop. Group convolution successfully increases the efficiency in ESHA compared to other attention models, such as vanilla attention in MobileViT, maintaining lower complexity and energy usage.

\begin{table}[!ht]
\centering
\caption{Ablation study of MicroViT  on ImageNet-1K Dataset. The Baseline model is MicroViT-S2.}
\begin{tabular}{ m{1.6cm}|c|c|c }
\hline
\textbf{Ablation} & Baseline & Low Res Attn & W/O Group  \\ \hline
\textbf{Param}      & 11.0  & 11.1  & 20.5   \\  
\textbf{GFLOPs}     & 0.343 & 0.344 & 0.4977 \\ 
\textbf{GPU}        & 14154 & 14179 & 13511  \\
\textbf{Throughput} & 567   & 570   & 503    \\
\textbf{Latency}    & 9.9   & 10.7  & 12.5   \\
\textbf{Avg Pow}    & 2481  & 2512  & 3288   \\
\textbf{Energy}     & 24.29  & 25.8  & 41.2   \\
\textbf{Top-1}      & 72.7  & 72.5  & 73.2   \\
\textbf{Efficiency ($\eta$)}& 3.07  & 2.8 & 1.8 \\
\hline


    \end{tabular}
    \label{tab:abl-result}
\end{table}


\section{Discussion}
\label{section:discussion}


\subsection{Practical Implications for Feedforward Prompting}

Of course, prompting an LLM continuously before the user submits their prompt is significantly most costly over submitting the prompt just once, once the user is ready.

% But user might not be ready, and the cognitive costs is pretty heavy.


\subsection{}


% Does this work well with Chain of Thought actually?
% Maybe this approach will actually incentivize self-prompt-chaining???
% What are the implications of this?


% A benefit of this is certainly more transparency in the LLM
% LLM is so flexible that adding this kind of structure is still okay for the LLM



% What's more costly, entering a prompt, then responding and saying, no i want this, or typing a prompt, and tuning the prompt/expected output to reduce message exchanges?

% Learning to become a better prompter. One is by trial and error experience. Perhaps another is through this feedforward that tells you what you might be able to anticipate.
\section{Concluding Remarks}
In this paper, we proposed a novel approach utilizing multimodal LLMs to generate gesture-aware speech recognition transcripts for patients with language disorders. Our framework integrates verbal speech and iconic gestures, enabling the generation of enriched transcripts that capture the latent meaning conveyed through both modalities. Through extensive experimentation, we demonstrated that the proposed method effectively contextualizes incomplete or disfluent speech by incorporating gesture information, leading to more accurate and meaningful representations of the speaker's intent. These findings highlight the potential of our approach to significantly contribute to the field of speech and language therapy, offering innovative tools that can enhance the quality of life for individuals with language disorders by facilitating better communication and assessment methods.

\subsection{Ethical Statement} 
Our dataset was obtained from AphasiaBank with the approval of the Institutional Review Board (IRB) and adheres to the data sharing guidelines set by TalkBank\footnote{https://talkbank.org/share/ethics.html}. This includes complying with the Ground Rules for all TalkBank databases, which are based on the American Psychological Association Code of Ethics~\cite{american2002ethical}.

\subsection{Limitation \& Future Work} 
%This study represents a preliminary investigation into using multimodal LLMs to generate gesture-aware speech recognition transcripts. 
While the results are promising, we recognize several limitations and outline our plans to extend this work further.

One primary limitation is the absence of a definitive ground truth for quantitative evaluation. Since our model generates transcripts by synthesizing speech and gesture data from scratch, traditional benchmarks, such as comparisons with standard speech recognition outputs, are insufficient. Moreover, existing original transcripts lack gesture annotations, making direct comparisons challenging. In future work, we aim to address this gap by collaborating with certified pathologists to conduct qualitative assessments, such as A-B preference tests, to evaluate the effectiveness of gesture-enriched transcripts in accurately conveying the speaker's intentions.

To support quantitative evaluations, we plan to develop novel metrics that assess transcript quality, including grammar accuracy, semantic consistency, and the integration of multimodal information. Such metrics will provide a more objective basis for assessing our model's performance and facilitate comparisons with other multimodal and unimodal approaches.

Another limitation of this study is its focus on structured gestures from a specific task, the Peanut Butter Sandwich Task. While this task offers a controlled context for testing our approach, it does not encompass the diversity of gestures and communication patterns seen in everyday scenarios. As part of our future work, we plan to expand the scope of our model to include tasks such as the Cinderella Story Recall Task~\cite{bird1996cinderella}, which involves unstructured and complex narrative gestures. This expansion will allow us to evaluate the adaptability and robustness of our model in handling varied linguistic and gestural contexts.

In summary, while this study establishes a strong foundation for gesture-aware speech recognition, we aim to refine and extend our methods through collaborative qualitative evaluations, the development of robust quantitative metrics, and broader task applications. These efforts will ensure that our approach continues to evolve, ultimately contributing to more effective communication tools and interventions for individuals with language disorders.





\bibliographystyle{ACM-Reference-Format}
\bibliography{references, references2}


\section{Metric}
\label{sec:metric}

\textbf{Mean Squared Error (MSE)} Mean Squared Error (MSE) is a common statistical metric used to assess the difference between predicted and actual values. The formula is:
\begin{equation}
    MSE = \frac{1}{n} \sum_{i=1}^{n} (y_i - \hat{y}_i)^2
\end{equation}
where $ n $ is the number of samples, $ y_i $ is the actual value, and $ \hat{y}_i $ is the predicted value.

\textbf{Relative L2 Error} Relative L2 error measures the relative difference between predicted and actual values, commonly used in time series prediction. The formula is:
\begin{equation}
    \text{Relative L2 Error} = \frac{\| Y_{\text{pred}} - Y_{\text{true}} \|_2}{\| Y_{\text{true}} \|_2}
\end{equation}
where $ Y_{\text{pred}} $ is the predicted value and $ Y_{\text{true}} $ is the actual value.

\textbf{Structural Similarity Index Measure (SSIM)} The Structural Similarity Index (SSIM) measures the similarity between two images in terms of luminance, contrast, and structure. The formula is:
\begin{equation}
    SSIM(x, y) = \frac{(2\mu_x \mu_y + C_1)(2\sigma_{xy} + C_2)}{(\mu_x^2 + \mu_y^2 + C_1)(\sigma_x^2 + \sigma_y^2 + C_2)}
\end{equation}
where $ \mu_x $ and $ \mu_y $ are the mean values, $ \sigma_x $ and $ \sigma_y $ are the standard deviations, $ \sigma_{xy} $ is the covariance.

\section{Related Work}
\subsection{Deep Learning based Weather Forecasting}
\textbf{Global Weather Forecasting.} Global weather forecasting has seen significant progress with deep learning models. FourCastNet, based on Fourier neural operators, provides global forecasts comparable to traditional numerical methods like IFS, but at much higher speeds~\cite{pathak2022fourcastnet}. Pangu, utilizing the Swin Transformer, exceeds NWP methods, incorporating earth-specific location embeddings for better performance~\cite{bi2023accurate}. The Spherical Fourier Neural Operator (SFNO) extends Fourier methods using spherical harmonics, offering more stable long-term predictions~\cite{bonev2023spherical}. FuXi focuses on long-term forecasting, achieving a 15-day forecasts comparable to ECMWF~\cite{chen2023fuxi}. GraphCast leverages message-passing networks to improve efficiency and forecasting accuracy~\cite{lam2023learning}, and GenCast builds on this to enhance ensemble forecasting~\cite{price2023gencast}. Further, diffusion models like those in~\cite{li2024generative} generate probabilistic ensembles by sampling, while NeuralGCM~\cite{kochkov2024neural} focuses on atmospheric circulation with a dynamic core, offering climate simulation capabilities but at higher training and inference costs. 

\textbf{Regional Weather Forecasting.} The goal of regional weather forecasting is to enhance local prediction accuracy with high-resolution models. CorrDiff~\cite{mardani2023generative} combines U-Net and diffusion models to improve local forecasts. MetaWeather~\cite{kim2024metaweather} adapts global forecasts to regional contexts using meta-learning. GNNs are also widely applied in regional forecasting, with Graphcast~\cite{lam2023learning} enhancing accuracy by modeling complex spatial dependencies. MetNet-3~\cite{espeholt2022deep} offers high-accuracy forecasts for weather variables, such as precipitation, temperature, and wind speed, at 2-minute intervals and 1–4 km resolution, outperforming traditional models like HRRR. NowcastNet~\cite{zhang2023skilful} and DGMR~\cite{ravuri2021skilful} excel in short-term extreme precipitation forecasts using deep generative models and radar data. In spatiotemporal prediction, NMO~\cite{wu2024neural} models the evolution of physical dynamics, providing new insights for local weather forecasting. Similarly, SimVP~\cite{gao2022simvp} and PastNet~\cite{wu2024pastnet} achieve good results in forecasting local precipitation evolution using spatiotemporal convolution methods.
    
% Despite these advances, none of these methods effectively address the challenge of balancing global and regional high-resolution forecasts or capturing the fine-grained, dynamic interactions important for extreme event prediction.
    
\subsection{Numerical analysis methods}
Multigrid methods~\cite{mccormick1987multigrid,wesseling1995introduction,hackbusch2013multi,bramble2019multigrid,hiptmair1998multigrid,brandt1983multigrid,borzi2009multigrid} and nested grid strategies~\cite{miyakoda1977one,zhang2012nested,sullivan1996grid} are widely used to solve PDEs and handle multi-scale problems~\cite{debreu2008two,xue2000advanced}. Multigrid methods use grids of different resolutions to transfer information and accelerate iterations. They efficiently solve large-scale problems and improve computational accuracy. By eliminating low-frequency errors on coarse grids and high-frequency errors on fine grids, multigrid methods effectively handle error convergence at different scales~\cite{he2019mgnet,he2023mgno,shao2022fast}. Nested grid strategies embed higher-resolution fine grids into regions of interest based on a global coarse grid to capture local complex physical phenomena in detail. In weather forecasting, this method provides large-scale background fields on a global scale while refining the grid for target regions to accurately simulate the evolution of local weather systems and the occurrence of extreme events~\cite{bacon2000dynamically}. 

% Our proposed neural nested grid method helps address challenges like boundary information loss in regional forecasting and multi-scale feature capture.

\section{Additional Results}
%
We present more additional results in Figure \ref{fig_0.25-day}, \ref{fig_0.5-day}, \ref{fig_1.0-day} \ref{fig_1.5-day}, \ref{fig_2.0-day}, \ref{fig_2.5-day}, \ref{fig_3.0-day}, \ref{fig_3.5-day}, \ref{fig_4.0-day}, \ref{fig_4.5-day}, \ref{fig_5.0-day}, \ref{fig_5.5-day}, \ref{fig_6.0-day}, \ref{fig_6.5-day}, \ref{fig_7.0-day}, \ref{fig_7.5-day},
\ref{fig_8.0-day}, \ref{fig_8.5-day}, \ref{fig_9.0-day}, \ref{fig_9.5-day},
\ref{fig_10.0-day}, including 18 variables that are importmant to weather forecasting, each with results ranging from 6 hours to 10 days. These additional results further demonstrate the effectiveness of OneForecast. Same as the Figure \ref{fig:visual_results}
, the initial conditions is 00:00 UTC, 1 January 2020.


\begin{figure*}[h]
\centering
\includegraphics[width=1\linewidth]{figures/fig_0.25-day.jpg}
\vspace{-20pt}
\caption{6-hour forecast results of different models.}
\label{fig_0.25-day}
\end{figure*}

\begin{figure*}[h]
\centering
\includegraphics[width=1\linewidth]{figures/fig_0.5-day.jpg}
\vspace{-20pt}
\caption{0.5-day forecast results of different models.}
\label{fig_0.5-day}
\end{figure*}

\begin{figure*}[h]
\centering
\includegraphics[width=1\linewidth]{figures/fig_1.0-day.jpg}
\vspace{-20pt}
\caption{1-day forecast results of different models.}
\label{fig_1.0-day}
\end{figure*}

\begin{figure*}[h]
\centering
\includegraphics[width=1\linewidth]{figures/fig_1.5-day.jpg}
\vspace{-20pt}
\caption{1.5-day forecast results of different models.}
\label{fig_1.5-day}
\end{figure*}

\begin{figure*}[h]
\centering
\includegraphics[width=1\linewidth]{figures/fig_2.0-day.jpg}
\vspace{-20pt}
\caption{2-day forecast results of different models.}
\label{fig_2.0-day}
\end{figure*}


\begin{figure*}[h]
\centering
\includegraphics[width=1\linewidth]{figures/fig_2.5-day.jpg}
\vspace{-20pt}
\caption{2.5-day forecast results of different models.}
\label{fig_2.5-day}
\end{figure*}

\begin{figure*}[h]
\centering
\includegraphics[width=1\linewidth]{figures/fig_3.0-day.jpg}
\vspace{-20pt}
\caption{3-day forecast results of different models.}
\label{fig_3.0-day}
\end{figure*}

\begin{figure*}[h]
\centering
\includegraphics[width=1\linewidth]{figures/fig_3.5-day.jpg}
\vspace{-20pt}
\caption{3.5-day forecast results of different models.}
\label{fig_3.5-day}
\end{figure*}

\begin{figure*}[h]
\centering
\includegraphics[width=1\linewidth]{figures/fig_4.0-day.jpg}
\vspace{-20pt}
\caption{4-day forecast results of different models.}
\label{fig_4.0-day}
\end{figure*}

\begin{figure*}[h]
\centering
\includegraphics[width=1\linewidth]{figures/fig_4.5-day.jpg}
\vspace{-20pt}
\caption{4.5-day forecast results of different models.}
\label{fig_4.5-day}
\end{figure*}


\begin{figure*}[h]
\centering
\includegraphics[width=1\linewidth]{figures/fig_5.0-day.jpg}
\vspace{-20pt}
\caption{5.0-day forecast results of different models.}
\label{fig_5.0-day}
\end{figure*}

\begin{figure*}[h]
\centering
\includegraphics[width=1\linewidth]{figures/fig_5.5-day.jpg}
\vspace{-20pt}
\caption{5.5-day forecast results of different models.}
\label{fig_5.5-day}
\end{figure*}

\begin{figure*}[h]
\centering
\includegraphics[width=1\linewidth]{figures/fig_6.0-day.jpg}
\vspace{-20pt}
\caption{6.0-day forecast results of different models.}
\label{fig_6.0-day}
\end{figure*}

\begin{figure*}[h]
\centering
\includegraphics[width=1\linewidth]{figures/fig_6.5-day.jpg}
\vspace{-20pt}
\caption{6.5-day forecast results of different models.}
\label{fig_6.5-day}
\end{figure*}

\begin{figure*}[h]
\centering
\includegraphics[width=1\linewidth]{figures/fig_7.0-day.jpg}
\vspace{-20pt}
\caption{7.0-day forecast results of different models.}
\label{fig_7.0-day}
\end{figure*}

\begin{figure*}[h]
\centering
\includegraphics[width=1\linewidth]{figures/fig_7.5-day.jpg}
\vspace{-20pt}
\caption{7.5-day forecast results of different models.}
\label{fig_7.5-day}
\end{figure*}

\begin{figure*}[h]
\centering
\includegraphics[width=1\linewidth]{figures/fig_8.0-day.jpg}
\vspace{-20pt}
\caption{8.0-day forecast results of different models.}
\label{fig_8.0-day}
\end{figure*}

\begin{figure*}[h]
\centering
\includegraphics[width=1\linewidth]{figures/fig_8.5-day.jpg}
\vspace{-20pt}
\caption{8.5-day forecast results of different models.}
\label{fig_8.5-day}
\end{figure*}

\begin{figure*}[h]
\centering
\includegraphics[width=1\linewidth]{figures/fig_9.0-day.jpg}
\vspace{-20pt}
\caption{9.0-day forecast results of different models.}
\label{fig_9.0-day}
\end{figure*}

\begin{figure*}[h]
\centering
\includegraphics[width=1\linewidth]{figures/fig_9.5-day.jpg}
\vspace{-20pt}
\caption{9.5-day forecast results of different models.}
\label{fig_9.5-day}
\end{figure*}

\begin{figure*}[h]
\centering
\includegraphics[width=1\linewidth]{figures/fig_10.0-day.jpg}
\vspace{-20pt}
\caption{10.0-day forecast results of different models.}
\label{fig_10.0-day}
\end{figure*}


\section{Detailed Mathematical Proof}
\label{sec:proof}
\textbf{Proof of Theorem 1}

Now we have N augmented data and we need to select the best from them. We consider both the quality and the diversity of these data and get the sampling strategy from an optimization problem.

We model the sampling strategy as a multinomial distribution supported on all the augmented data $S = \{\mathbf{X}_j\}_{j=1}^N$, which means that the sampling strategy $\pi=(\pi_1,...,\pi_N)^\top$ is the corresponding probabilities of selecting $\mathbf{X}_1,...,\mathbf{X}_N$, then we can model the expectation of the similarity as:
$$\begin{aligned}
 & \mathbb{E}_{Y_x,Y_{x^{\prime}}\in\mathcal{C}}\{g(x,x^{\prime})\mid S\} \\
 & =\quad\int g(\mathbf{x},\mathbf{x}^{\prime})\boldsymbol{\pi}(\mathbf{x})\mathrm{Pr}_{S}(Y_{x}\in\mathcal{C}\mid\boldsymbol{x}=\mathbf{x})\boldsymbol{\pi}(\mathbf{x}^{\prime})\mathrm{Pr}_{S}(Y_{x}\in\mathcal{C}\mid\boldsymbol{x}=\mathbf{x}^{\prime})d\mathbf{x}d\mathbf{x}^{\prime} \\
 & =\quad\sum_{i,j=1}^Ng(\mathbf{X}_i,\mathbf{X}_j)\pi_i\pi_j\mathrm{Pr}_{S}(Y_x\in\mathcal{C}\mid\boldsymbol{x}=\mathbf{X}_i)\mathrm{Pr}_{S}(Y_x\in\mathcal{C}\mid\boldsymbol{x}=\mathbf{X}_j),
\end{aligned}$$
where the set $\mathcal{C}$ denotes the criterion of selection we are using, the function $g$ can be chosen as any similarity metric function and $x$ means a random variable.

The core to solving the above optimization problem is to use predictive inference to approximate the conditional probability of $\{Y_x\in\mathcal{C}\}$ given $x = \mathbf{X}$
Let $\mu ( \mathbf{x} ) : = \mathbb{E} ( Y\mid \mathbf{X} = \mathbf{x} )$ be the oracle associated with $( \mathbf{X} , Y) .$ Denote $\theta_j=\mathbb{I}\{Y_j\in\mathcal{C}\}$. As the augmented data
$\mathbf{X}_1,...,\mathbf{X}_N$ are independently identically distributed, $\theta_1,...,\theta_N$ can be regarded as independent Bernoulli($q)$ variables with $q=\Pr(Y_j\in\mathcal{C}).$ The probability distribution of the predicted result $W_j$ for $j=1,...,N$ is
$$\Pr(W_j\mid\theta_j)=(1-\theta_j)f_0+\theta_jf_1,\quad$$
where $f_0$ and $f_1$ are the conditional distributions of $W_j$ on $Y_j \in \mathcal{C}$ or not.

Denote $T(w) = \frac{(1-q)f_0(W_j)}{f(W_j)}$, we can rewrite the expectation of the similarity as
$$\mathbb{E}_{Y_x,Y_{x^{\prime}}\in\mathcal{C}}\{g(x,x^{\prime})|S\}=\sum_{i,j=1}^Ng(\mathbf{X}_i,\mathbf{X}_j)\pi_i\pi_j(1-T_i)(1-T_j)=\boldsymbol{\pi}^\top A_\mathbb{T}\boldsymbol{\pi},$$

Next, we use the expectation to control the quality of the data.
$$\mathbb{E}\{\mathbb{I}(Y_x\not\in\mathcal{C})\mid S\}=\sum_{i=1}^N\Pr(Y_i\not\in\mathcal{C}\mid\mathbf{X}_i)\pi_i=\sum_{i=1}^N\pi_iT_i\leq\alpha,$$

In all, the optimization problem can be modeled as 
\begin{align}
    & \arg\min_{\boldsymbol{\pi}}\quad h(\boldsymbol{\pi},\mathbb{T}):=\boldsymbol{\pi}^\top A_\mathbb{T}\boldsymbol{\pi}, \\
    & \text{subject to} \quad
        \begin{cases}
            \sum_{i = 1}^N\pi_iT_i\leq\alpha, \\
            \sum_{i = 1}^N\pi_i = 1, \\
            0\leq\pi_i\leq m^{-1}, \quad 1\leq i\leq N.
        \end{cases}
\end{align}

where $m$ is used to control the maximum selection.

The best selection of K is determined by the strategy $\pi$ which serves as the solution to the above optimization problem.

\section{Additional Experiments}
\label{sec:more_experiments}
\subsection{Long-term forecasting experiment expansion}

In the long-term forecasting experiments, we compare the performance of different backbone models on the SWE benchmark, evaluating the relative L2 error for three variables (U, V, and H). Our setup inputs 5 frames and predicts 50 frames. For the SimVP-v2 model, using \method{} reduces the relative L2 error for SWE (u) from 0.0187 to 0.0154, SWE (v) from 0.0387 to 0.0342, and SWE (h) from 0.0443 to 0.0397. We visualize SWE (h) in 3D as shown in Figure~\ref{fig:case} [\textcolor{red}{I}]. For the ConvLSTM model, applying \method{} reduces the relative L2 error for SWE (u) from 0.0487 to 0.0321, SWE (v) from 0.0673 to 0.0351, and SWE (h) from 0.0762 to 0.0432. For the FNO model, using \method{} reduces the relative L2 error for SWE (u) from 0.0571 to 0.0502, SWE (v) from 0.0832 to 0.0653, and SWE (h) from 0.0981 to 0.0911. Overall, \method{} significantly improves the long-term forecasting accuracy of different backbone models.

\begin{figure*}[h]
    \centering
    \includegraphics[width=\textwidth]{image/casestudy.pdf}
    \caption{
    \textcolor{red}{I.} 3D visualization of the SWE(h), showing Ground-truth, SimVP-V2+BeamVQ predictions, and Error at T=1, 10, 20, 30, 40, 50. The first row shows Ground-truth, the second SimVP-V2+BeamVQ predictions, and the third Error. \textcolor{red}{II.} A case study. Building fire simulation with ventilation settings added to Wu's Prometheus~\cite{wu2024prometheus}. (a) Layout and HRR growth. (b) Comparison of physical metrics for different methods. (c) Ground-truth, ResNet+BeamVQ, and ResNet predictions.
    }
    \label{fig:case} 
\end{figure*}


\subsection{Experiment Statistical Significance}
\label{sec:significance}
To measure the statistical significance of our main experiment results, we choose three backbones to train on two datasets to run 5 times. 
Table~\ref{tab:significance} records the average and standard deviation of the test MSE loss.
The results prove that our method is statistically significant to outperform the baselines
because our confidence interval is always upper than the confidence interval of the baselines. 
Due to limited computation resources, we do not cover all ten backbones and five datasets, 
but we believe these results have shown that our method has consistent advantages.


\begin{table}[h]
\label{tab:significance}
\centering
\begin{scriptsize}
    \begin{sc}
    \caption{ The average and standard deviation of MSE in 5 runs}
    \label{tab:significance}
    \centering
        \renewcommand{\multirowsetup}{\centering}
        \setlength{\tabcolsep}{10pt}
        \begin{tabular}{l|cc|cc}
            \toprule
            
            \multirow{4}{*}{Model} & \multicolumn{4}{c}{Benchmarks}  \\
            \cmidrule(lr){2-5}
            & \multicolumn{2}{c}{NSE} &   \multicolumn{2}{c}{SEVIR}   \\
            \cmidrule(lr){2-5}
           & Ori & + BeamVQ & Ori & + BeamVQ  \\
            \midrule
            ConvLSTM &0.4092$\pm$0.0002 &\textbf{0.1277$\pm$0.0001}  & 0.1762 0.0007  & \textbf{0.1279$\pm$0.0009}  \\
            FNO &  0.2227$\pm$0.0003 &\textbf{0.1007 $\pm$0.0002}& 0.0787$\pm$0.0012 & \textbf{ 0.0437$\pm$0.0013} \\
            CNO & 0.2192 $\pm$0.0008 &\textbf{ 0.1492$\pm$0.0011}& 0.0057$\pm$0.0005 & \textbf{ 0.0053$\pm$0.0006} \\
            \bottomrule
        \end{tabular}
    \end{sc}

\end{scriptsize}
\end{table}


\end{document}
