\subsection{Reimagining Social Media's Potential}
\label{lab:4-6}
Given the nature of the FI co-design workshop, SMH embodies the youth's aspirations and values for what social media could be. Therefore, while they acknowledged potential challenges---such as the platform being overly immersive, addictive, or demanding---they largely viewed these downsides as worthwhile trade-offs for its many benefits. Overall, participants saw SMH as an inspiring alternative to existing social media, offering opportunities for self-expression and the ability to connect in ways that felt deeply personal and intentional. They also appreciated the FI workshops for empowering them to reimagine social media and to be hopeful about its possibilities in ways they had previously thought impossible.

\subsubsection{Fostering Hope for the Positive Potentials of Social Media}
\label{lab:4-6-1}
Participants saw SMH as a platform that offers better user experiences, departs from the repetitive structures of existing platforms, and inspires hope for what social media can become. For example, P06 shared how SMH combines several desirable aspects of social media in a cohesive way, creating an environment focused on meaningful interactions: \inlinequote{You get friends who you can talk to, keep up with, and stay connected to. And you get things you can customize. You kind of get the whole package there. I kind of like all of those here.}

Participants also appreciated how SMH offered a break from the monotony of existing platforms. P11 pointed out that most social media platforms share a common \inlinequote{standard} with repetitive features like \inlinequote{feed, messages, settings,\ldots{} profile.} In contrast, SMH introduced new types of interactions that deviated from familiar patterns. P02 described SMH as \inlinequote{a combination of Discord and BeReal} that brings together casual sharing and topic-based discussions. Similarly, P17 envisioned their SMH as \inlinequote{more interactive, kind of like if Discord and Instagram were combined,} where they could enjoy the interactions of Discord along with profiles and the ability to post content like on Instagram. P22 highlighted that on other platforms, there is limited opportunity for customization or decoration, explaining, \inlinequote{I just have a profile picture.} However, on SMH, they could \inlinequote{really be creative} and \inlinequote{make something that [they] haven't made anywhere else,} which made the experience feel \inlinequote{fun} and \inlinequote{novel,} like a \inlinequote{breath of fresh air.}

Beyond features, participants expressed how SMH inspired hope for the future of social media. P10 pointed out that overcoming the design fixation around social media is crucial: \blockquote{The way that we can make social media better is if we expand the realm of what the definition of social media is. Instead of it being where someone just shares stuff, it can also be places where people connect, and people do this and that.}{10}
Several participants noted that the FI workshop played a pivotal role in helping them gain a broader vision of social technologies and envision new possibilities for social media. P15 reflected on how their perspective evolved throughout the process:
\blockquote{When we started, I literally was giving all these suggestions, which was completely the same thing which existed\ldots{} I was repeating their concepts. So I think throughout this entire interview, I evolved my idea\ldots{} \textbf{actually thinking about what is new}\ldots{} So I think this has been a super hit. I like the entire conversation.}{15}

They appreciated that SMH was not merely a \inlinequote{clone of the same thing\ldots{} without actually adding anything else} but rather a concept that was \inlinequote{completely different\ldots{} not even the same.} Similarly, P14 shared how the workshop shifted their initial skepticism: \blockquote{I don't know what else you could do, because it's online\ldots{} so I really don't know what else there could be to make it more meaningful.}{14} After engaging with the workshop, they felt empowered to imagine broader possibilities: \blockquote{I'm not really a tech person. So I don't really think about [how to make social media better] that much. But I feel like this fake scenario\ldots{} [helped] remind myself, `Girl, you have full control, this is no restrictions.' After I got into that \textbf{I\ldots{} have a way bigger vision of what social media can become.}}{14}

 
\subsubsection{Celebrating Youthful Creativity and Play}
\label{lab:4-6-2}
SMH allows a playful and engaging social media experience, allowing users to immerse themselves in actively creative activities and self-expression. P06 appreciated this ability to personalize their experience on SMH, stating, \inlinequote{I think\ldots{} everyone likes sharing parts of themselves.} P21 likened SMH to a \inlinequote{game} due to its creative freedom. They elaborated, \inlinequote{The creative freedom you have over the platform makes it feel more fictional than it is real.}

Participants found the overall experience of SMH to be youthful, fun, and deeply immersive, offering a space that feels both magical and meaningful. For example, P14 initially leaned toward preferring a 2D version of the platform but shifted their preference to the 3D version, reasoning: \blockquote{Yeah, [I would prefer the 3D version] actually 100\%. I feel like,\ldots{} the only reason I said 2D is because I felt like the need to be like a mature adult. But who cares? I feel like something that's more enjoyable would be [the SMH].}{14} P22 described SMH as \inlinequote{the coolest,} \inlinequote{childlike,} \inlinequote{less generic,} and \inlinequote{a lot more fun} compared to other platforms. P06 echoed this, observing that \inlinequote{the physical space} on SMH felt \inlinequote{a lot more fantastical\ldots{} whimsical, fun} in contrast to \inlinequote{traditional ones.} For some, the platform also evokes nostalgia and happiness. P15 remarked, \inlinequote{Having this app would actually make them feel happy\ldots{} with nostalgia.} They anticipated that SMH would be \inlinequote{a super big hit} with teenagers due to its playful and engaging nature.

\subsubsection{Enabling Otherwise Inaccessible Experiences}
\label{lab:4-6-3}
SMH creates opportunities for users to engage in experiences and interactions that may otherwise be out of reach. P09 highlighted how individuals \inlinequote{experiencing stress} or those who may not have access to \inlinequote{really, really beautiful landscape and environments} could walk around and \inlinequote{admir[e]} the \inlinequote{calming} landscape of SMH. P16 envisioned SMH as a space where users could engage in activities like climbing a mountain in Nepal, saying, \inlinequote{Maybe you're busy\ldots{} you can't afford to travel to Nepal to climb Mount Everest\ldots{} but you have a day, so you can just log in for the entire day and just climb a mountain.} 

P02 saw SMH as a platform that would \inlinequote{seem more like real life, but better.} They imagined being able to engage with friends in idyllic settings, such as \inlinequote{on the beach all the time.} Furthermore, they believed SMH could strengthen friendships through magical features. In their words, \inlinequote{Since people can't read others' emotions in real life, SMH would\ldots{} be a way for you to connect with your friends on a deeper level.} P18 also appreciated how SMH's 3D interface facilitated them to interact with \inlinequote{family across the world} in an immersive way. SMH would offer a sense of presence that goes beyond conventional video calls, enabling users to engage with loved ones \inlinequote{not just through a screen.}