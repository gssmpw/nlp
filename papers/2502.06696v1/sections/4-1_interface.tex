\subsection{Immersive Experiences Through Spatial Elements}
\label{lab:4-1}
Participants saw 3D and VR platforms as key to emotionally engaging, socially fulfilling interactions. P16 noted current social media (which we will refer to as ``Muggle Social Media'' or MSM for short)\inlinequote{does a good job at spreading information [but not] emotions,} arguing that\inlinequote{tight interpersonal interactions} require\inlinequote{talk[ing] to you and see[ing] your face,} which 3D and VR enhance. They envisioned SMH as a 3D open world with realistic, highly flexible social interactions, lifelike avatars, distinct public and personal spaces, and holographic interfaces that convey presence.


\subsubsection{Bringing Realism and Presence Through 3D}
\label{lab:4-1-1}
Many found MSM\inlinequote{flat} or limited by\inlinequote[10]{just a screen,} yearning for greater realism and freedom. P19 compared SMH to\inlinequote{social media like Minecraft} but broader in scope, supporting a \inlinequote[19]{physical world through 3D graphics} with \inlinequote[06]{more of a connection} and \inlinequote[06]{an emphasis on communications.} P06 referred to the 3D massively multiplayer online game (MMO) Sky~\cite{sky}, where players can \inlinequote{talk to other people} or \inlinequote{sit down by a candle and chat} but wanted \inlinequote{more of a connection} than \inlinequote{just sitting there, talking to one random person.}

Navigating shared virtual spaces resonated with P03, who noted that actions like \inlinequote{mov[ing] your legs to walk} or \inlinequote{turn[ing] your head} add a \inlinequote{tiny bit of connection,} making it \inlinequote{easier to have connections with people.} P06 similarly highlighted co-presence in 3D: \blockquote{it does feel more real\ldots{} You turn on your voice, chat\ldots{} walking around a physical world\ldots{} It feels like you're actually in the area of these people; versus traditional social media like Instagram, there's a separation of the screen\dots{} you're sharing the chat room. But you're not really sharing the space [like when] I'm playing Roblox. [In SMH] I'm sharing that space physically, even though it's not really physical.}{06}

\subsubsection{Personal Spaces and Community-Centered Landscapes}
\label{lab:4-1-2}
Participants wanted SMH's 3D world to mirror the structure and functionality of real-world settings like homes, neighborhoods, and shared areas. P19 imagined \inlinequote{separate houses} with \inlinequote{a main house,} where users could \inlinequote{recreat[e] and decorat[e] your house.} 
P03 echoed this: \inlinequote[03]{You each get your own house\ldots{} anything you want}. P21 expanded on the idea, seeing personal spaces as \inlinequote{subspaces} within a larger ecosystem.
% space of all users and within the subspace where \inlinequote{You can have everything that's out in the real world}.

Having a realistic landscape was essential for fostering connection and community for some. P06 described it as having \inlinequote{neighbors} or an \inlinequote{island next door} in a \inlinequote{town area.} This idea of community and proximity reflected participants' desire to recreate the social structures of the real world in a digital setting. By mirroring the physical world, the landscape was also presumed to create a sense of place, grounding social interactions in a familiar and relatable context where users feel comfort and ownership. For example, P09 noted traversal in a corporal form would feel \inlinequote{fulfilling,} \inlinequote{calming,} and \inlinequote{reassuring.}


\subsubsection{Holographic Display and Virtual Reality}
\label{lab:4-1-3}
A recurring theme was moving beyond traditional devices like phones and computers to embrace holographic and VR-based interfaces. Many participants felt mobile devices would seem cumbersome in a magical, immersive world. P01 noted, \inlinequote{If you're already carrying around like a wand everywhere with you, why would you also want to carry a phone?} They envisioned wands projecting holographic screens, like \inlinequote{Jarvis in Iron Man,} or using spell-activated \inlinequote{panels} and \inlinequote{big screen[s].} P15 preferred VR glasses, calling them \inlinequote{more natural,} \inlinequote{easier,} and \inlinequote{more fun.} Other concepts included a \inlinequote{makeup mirror} as a portal, a \inlinequote{watch} that immerses users, or a \inlinequote{mirror glass} for video calls or summoning friends via the charm \ie{\textit{accio}}.

The immersive nature of VR was seen as fostering deeper connections. P01 noted MSM limits \inlinequote{how close you can be with someone,} explaining that\inlinequote{if you're just watching things through a screen\ldots{} technically, there's like a wall dividing you.} They concluded 3D VR platforms are \inlinequote[03]{more immersive and more awesome} than MSM, which is inherently limited \inlinequote{because it's 2D.}
