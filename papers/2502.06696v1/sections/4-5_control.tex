\subsection{Refocusing Interaction Priorities for Meaningful Engagement}
\label{lab:4-5}
Participants expressed a strong desire to refocus their social media experiences on meaningful connections and intentional interactions. This section explores four key themes that emerged from their responses: the need to prioritize genuine connections over algorithmic content recommendations, reducing the influence of celebrities and commercial content, using spatial metaphors to represent relationship strengths, and fostering more intentional and meaningful use of time. These themes reflect participants' vision for social media platforms that better align with their desires for authentic social connection and purposeful engagement.

\subsubsection{Intentional Content Consumption}
\label{lab:4-5-1}
Participants emphasized the importance of refocusing social media on genuine, personal connections rather than extraneous content. P19 criticized how platforms like Instagram have shifted away from its initial focus: \blockquote{We need old Instagram back in 2010, when my mom was posting me and my brother just for the family and no one else. We didn't need the explore page. We didn't need Instagram shopping\ldots{} We don't need extra things in an app that is only made for one thing---connecting\ldots{} We need to figure out how to put boundaries in place between connecting with the people we love and connecting with people we don't know yet\ldots{} you don't need to have so many political takers on this one app\ldots{} It's just a very stressful situation.}{19} This sentiment was echoed by P22, who lamented that on Instagram, \inlinequote{half of the timeline is accounts I don't even follow that were just randomly recommended to me.} They described this experience as \inlinequote{less keeping up with friends and more just getting random stuff shoved in my face.} 

On a related note, a major source of frustration was the infinite scrolling design of current platforms. P11 explained, \inlinequote{There's always more content on social media\ldots{} this fuels the mindset that\ldots{} I need to make sure I'm not missing anything.} P14 also criticized the endless content stream on TikTok, which exposes users to \inlinequote{clickbaity} and often harmful content that \inlinequote{doesn't stimulate [users] at all} and \inlinequote{messes with [users'] attention span.} They expressed a desire to \inlinequote{be more proactive} and \inlinequote{genuinely use [their] brain} instead of \inlinequote{just scrolling mindlessly for an hour.} P09 envisioned a system like an elevator: \inlinequote{You press which floor you want to go to, and then, when you ride that elevator, it just goes straight to that floor\ldots{} Could [SMH] be sort of like that, where you think about what you want to see before entering that space, and then it shows what you were thinking about?} This sentiment was echoed by several participants who envisioned brain-controlled interfaces that would \inlinequote[07]{automatically show [users] things [they] would really like} based on their immediate needs or interests.

\subsubsection{Reduced Celebrity and Commercial Influence}
\label{lab:4-5-2}
Several participants advocated minimizing celebrity influence. P02 stated, \inlinequote{I don't want celebrity stuff\ldots{} I don't want any celebrity drama,} while P01 suggested removing celebrities altogether if the platform prioritized personal enjoyment. If included, they proposed isolating them to a \inlinequote{celebrity zone.} Prioritizing posts from close friends was a key preference. P07 explained, \inlinequote{If I'm just gonna open the app for like 5 seconds\ldots{} I'm gonna wanna see my best friend's post, not some girl I talked to once 3 years ago.} They envisioned SMH offering greater control over interactions and content, noting it would be \inlinequote{more designed for people you actually know\ldots{} like your friends, classmates, or Quidditch team,} rather than celebrities or acquaintances. P19, similarly, envisioned that on SMH, they could \inlinequote{filter out for friends and family, and even then condense your circle a little bit more.}

Other participants expressed a need for control over how non-personal content is organized and placed on the platform. P04 suggested being able to \inlinequote{figure out how I want to arrange strangers' spaces in my\ldots{} platform.} They also wanted the ability to \inlinequote{prioritize things and feel more in control of how much information is fed to them,} imagining a system where apps and notifications are ranked by personal relevance. P18 saw ads as inevitable but wished they could appear in a separate \inlinequote{billboard} that users could skip.

\subsubsection{Spatial Representation of Relationships}
\label{lab:4-5-3}
Youth envisioned SMH as a platform that reflects the dynamics of real-world relationships, using physical distances within the SMH landscape to intuitively represent the strength of social ties. Participants appreciated this design as it prioritized interactions with closer friends, creating a sense of community and authenticity. P02 highlighted, \inlinequote{The fact that only my close friends have access to it\ldots{} is the defining factor,} enabling users to focus on \inlinequote{better conversations with other people [that] already know who I am,} without the pressure to \inlinequote{advertise who I am} as they might feel on MSM.

Some participants envisioned SMH landscape to resemble a college campus or a neighborhood, where physical closeness naturally fosters interaction. P03 imagined, \inlinequote{The people around you\ldots{} would be people from the same house as you,} cultivating a shared sense of \inlinequote{togetherness} and \inlinequote{community building} through common spaces and interactions. Similarly, P01 proposed a tiered layout, where \inlinequote{the top 7 people you're closest to would be the 7 houses in your cul-de-sac area,} while acquaintances would be positioned farther away, requiring a \inlinequote{little hike} to reach them. 

This design approach---aligning physical landscapes with emotional closeness---was seen as a way to streamline interactions and focus on meaningful relationships. P07 noted how, in short bursts of social media use, they preferred seeing content from their closest friends over \inlinequote{some girl I talked to once three years ago.} Unlike MSM feeds, which participants felt prioritized engagement over relationship strength, SMH's spatial arrangement allowed users to engage naturally with those closest to them. P09 further elaborated on how this physical representation could enhance navigation: \inlinequote{have your friends' rooms, or like your kind of close friends list, and maybe you're kind of living in a pod with them, so you can move between the rooms a lot quicker.}

\subsubsection{Intentional and Meaningful Use of Time}
\label{lab:4-5-4}
Participants expressed a strong preference for digital experiences that feel meaningful and intentional, contrasting these with the negative feelings often associated with aimless social media use. P10 described how social media can feel \inlinequote{meaningless} when they \inlinequote{didn't learn something,} while P18 reflected on the regret they feel after \inlinequote{mindlessly scrolling.} Similarly, P14 shared that scrolling through Instagram Reels while \inlinequote{literally just laying in [their] bed} feels like \inlinequote{wasting [their] time.}

In contrast, participants found interactions involving direct communication and engagement more fulfilling. P16 shared that \inlinequote{anytime not scrolling} and instead \inlinequote{talking to people} feels worthwhile, \inlinequote{even if it's just light conversation.} P14 echoed this sentiment, noting that social media use feels more meaningful when they are \inlinequote{actually talking to someone.} They explained that such interactions create a sense of \inlinequote{connectivity,} particularly when keeping in touch with people \inlinequote{not in close proximity.}

The distinction between passive and active engagement was another recurring theme. P10 explained that while doomscrolling does not make them feel \inlinequote{fulfilled,} playing games like Minecraft is different because it involves \inlinequote{progress} they can \inlinequote{save and come back to.} Similarly, P14 shared that the lack of connectivity on social media drives their \inlinequote{doomscrolling,} but they believe platforms that encourage interaction could reduce screen time by making time spent online more meaningful. P06 also note that they noted that they \inlinequote{don't feel bad\ldots{} most of the time [they] use Roblox,} as it involves engaging activities such as \inlinequote{playing games with friends} and \inlinequote{customizing spaces.} 

Given such differences in how time spent is perceived, participants expected time spent on SMH to feel more meaningful than that on MSM. P04 noted that while MSM allows for quicker interpersonal interactions, it often leads to \inlinequote{doomscrolling.} In contrast, SMH would, for example, encourage \inlinequote{thoughtfully engaging with someone's memories,} leaving them feeling \inlinequote{much more fulfilled.} P05, similarly, explained that they would not feel bad about spending more time on a platform like SMH because it involves \inlinequote{more effort and more time\ldots{} a more intentional way of interacting.} P07 affirmed this, stating they would be willing to spend more time on a platform like SMH given interactions where they would find interactions \inlinequote{more enjoyable.}

