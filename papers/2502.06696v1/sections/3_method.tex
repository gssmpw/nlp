\section{Method}

We conducted co-design workshops with 23 participants between the ages of 15 and 24 to examine how we might envision an ideal social media platform free from real-world constraints and from design fixation on archetypal social media platforms. All participants had prior experience with 3D gaming and familiarity with \textit{Harry Potter}~\cite{rowling2015harry, harrypotterfanclub}, a popular series of young adult fantasy novels (and movies based on these stories) in which a subset of the population possesses magical powers. During the workshop, we used the FI method to scaffold participants in generating design ideas.

\subsection{Material: The Narrative Framework}
\section{Entry Interview Miro Board Exercise}
\label{ref:miro}

\begin{figure*}[!h]
  \centering
  \begin{subfigure}{\textwidth}
    \centering
    \includegraphics[width=0.8\linewidth]{inserts/miro-exit.pdf}
    \caption{Miro Board Template}
    % \Description{.}
  \end{subfigure}
  
  \vspace{1em}

  \begin{subfigure}{\textwidth}
    \centering
    \includegraphics[width=0.9\linewidth]{inserts/miro-P01.pdf}
    \caption{P01's Miro Board}
    % \Description{.}
  \end{subfigure}

\caption{Screenshots of Miro Boards Used During Co-Design (Exit) Interviews.}
\end{figure*}

For our study, we selected the \textit{Harry Potter} world as the narrative framework for our FI method. The \textit{Harry Potter} series follows the journey of an orphaned boy who discovers he is a wizard and, as a result, is invited to attend ``Hogwarts School of Witchcraft and Wizardry,'' where he forms deep friendships and gradually uncovers his destiny as he clashes with the story's antagonist.

Out of the numerous magical and fantastical worlds available, we identified the \textit{Harry Potter} universe as particularly well-suited for our FI workshops. The series' extensive popularity and deeply embedded cultural presence provide participants with a shared foundation of understanding, facilitating both recruitment and engagement. The \textit{Harry Potter} series has garnered immense popular appeal and commercial success; by 2023, 98\% of youth aged 18 to 27 in the United States identified as either casual or avid fans of the \textit{Harry Potter} movie franchise, with approximately half describing themselves as avid fans~\cite{harrypotterfan}. Furthermore, the \textit{Harry Potter} book series has sold over 600 million copies~\cite{harrypotterscholastic} and translated into over 80 languages by 2020 globally~\cite{TheRowlingLibrary-2020-BigTranslationsPotter-z}.

The series is set in modern-day England, reimagined to include magical elements, like enchanted objects and spells. Thus, this context leaves room for both incorporating or discarding any aspect of contemporary life, including designs and constraints related to social technologies. Its young protagonists and coming-of-age narrative naturally align with our study's focus on youth identity and relationship building. The characters' experiences or life transitions---which parallel real-world moves between schools or locations---allow for exploring how social technologies might support relationship maintenance across changing contexts. The series' emphasis on community and belonging also lends itself to examining youth identity expression and self-presentation in virtual spaces.

\subsection{Procedure}
The first author conducted all the interviews over Zoom, with the second author joining one of the sessions to validate and ground the FI workshop method. The interviews averaged around 90 minutes. We began the interview by asking which 3D games they had played and what they liked about those experiences. We also asked questions about \textit{Harry Potter}, such as their favorite book in the series. This served to break the ice and help them start immersing themselves in the Harry Potter world.

Following this introduction, we presented the FI narrative: participants were asked to envision ways to support social connection within the Hogwarts world as a non-Wizard (i.e., ``Muggle''~\cite{muggle}) trying to stay connected with everyone across Hogwarts and the Muggle world. We then shared a Miro~\cite{miro} board~(Figure~\ref{fig:miro}) where participants could sort their friends and family into categories inspired by the Harry Potter series, such as fictional houses (e.g., Gryffindor, Hufflepuff), teams (e.g., Quiddich Team members), and old friends (e.g., muggles), reflecting fictional houses, teams, and communities from the story. This activity encouraged them to think of their relationships within the \textit{Harry Potter} context and was intended to further enhance immersion in the fictional design space. Next, we asked general questions about their experiences with social media, focusing on what they found meaningful or meaningless and their vision for an ideal social media platform. 

During the design phase, participants were prompted to describe the capabilities their ideal magical social media would have and what aspects of their lives they would choose to share. Throughout this phase, we sought to keep participants immersed in the narrative, referring to existing platforms as ``Muggle social media'' and maintaining consistent engagement with the Hogwarts setting, as outlined by the fictional inquiry guidelines. Finally, we concluded by asking participants to reflect on how the social media they envisioned for Hogwarts compared to Muggle social media. The semi-structured interview protocol is available in Appendix \ref{protocol}.

\subsection{Participants and Recruitment}
We used a combination of purposive and convenience sampling to recruit participants for our study. The first and second authors shared the recruitment materials (that included a link to a screener survey) on their social media accounts, while the first author also distributed flyers across our institution's campus.

Participants were required to be between 15 and 24 years old, in accordance with UN~\cite{UnitedNationsUnitedNations-Other-Youth|Nations-t} and WHO~\cite{who-youth} definitions for youth, and to consent to be video or audio recorded during a 90-minute interview. Video recording was requested following the FI method guidelines, enabling us to analyze gestures if necessary. We included both teenage and young adult participants (ages 15 to 24) to capture perspectives from distinct life stages, as these groups often navigate significant transitions and share a focus on identity exploration, self-presentation, and relationships.

Participants were also required to be sufficiently familiar with the \textit{Harry Potter} narrative such that they agreed with the statement, ``\textit{I remember some details about the plot, characters, and magical aspects.}'' and to have prior experience with 3D games (e.g., Minecraft, Roblox).  This last criterion was based on the assumption that individuals familiar with 3D games would likely have experience with 2D platforms as well, equipping them with a broader set of cognitive tools for creative idea generation. By contrast, extensive experience with 2D platforms alone does not necessarily ensure the ability to conceptualize or engage with spatial technologies effectively. While we did not impose constraints on social media experience, all participants had at least some engagement with social media.


\begin{table}[htbp!]
\resizebox{\columnwidth}{!}{%
\begin{tabular}{@{}l|ccc|c@{}}
 & Liberal & Moderate & Conservative & Total \\ \hline
Female & 223 & 114 & 45 & 382 \\
Male & 102 & 78 & 53 & 233 \\
Prefer not to say & 2 & 0 & 0 & 2 \\ \hline
Total & 327 & 192 & 98 & 617
\end{tabular}%
}
\caption{Annotator Demographics. All annotators are based in the United States. The table shows the number of annotators across ideology and sex categories, as self-reported to Prolific. The mean age is 38.3 (SD=12.7), and 45 annotators are immigrants (7.3\%).}
\label{tab:demographics}
\end{table}


A total of 84 individuals completed the screener survey. We reached out to the 48 respondents who met the recruitment criteria, and 23 participants provided consent and completed the interview procedure. The mean age of the participant group was 18.4 years (min=16, max=23, SD=1.56), with 15 (65.2\%) identifying as women, 3 (13.0\%) as men, 2 (8.7\%) as non-binary, and 3 (13.0\%) that were gender fluid. The full list of participant information is available in Table \ref{tab:demographics-individual}.


\subsection{Data Analysis}
We analyzed the interview transcripts using reflexive thematic analysis~\cite{BraunClarke-2021-ThematicAnalysisGuide-k}, which provides a systematic yet adaptable framework for interpreting qualitative data. The method's independence from pre-existing theoretical frameworks made it particularly suitable for our study, given our study's exploratory focus on youth-envisioned social media design through the FI method. It enabled us to effectively capture and analyze the diverse perspectives shared by our participants.

Our analysis began with the first two authors independently coding the same two transcripts line by line. During this phase, the codes were descriptive and closely adhered to the data. The authors convened to discuss and resolve any discrepancies in their coding. Following this, each author individually coded two different transcripts using Atlas.ti~\cite{atlas}. After four coding iterations, we reached saturation in our codes. Using this final set of codes, the first author proceeded to code all the transcripts. 

\subsection{Ethical Considerations}
Given J.K. Rowling’s controversial statements regarding transgender individuals~\cite{Other-2020-JKIssues-x, Miller-2024-JKTransantagonism-a}, we included a disclaimer~\footnote{The disclaimer read, \textit{We acknowledge that J.K. Rowling has made statements regarding transgender individuals that many find harmful and discriminatory. We do not support or endorse these views. Our study focuses on the magical context of the Harry Potter series, and we will strive to create an inclusive and respectful environment for all participants, regardless of gender identity or expression. Thank you for your understanding and for helping us advance this important research.}} in our recruitment material and screener survey and reiterated our position during the interview procedure.
