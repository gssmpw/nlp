\subsection{Organic and Intentional Social Interactions}
\label{lab:4-2}
Participants described MSM as \inlinequote[22]{a popularity contest} that felt \inlinequote[14]{really fake} because \inlinequote[14]{people only really show the good moments}. This carefully curated image culture creates \inlinequote[23]{a lot of seriousness}, which is reflected in how \inlinequote[23]{actors or celebrities\ldots{} [even] have somebody else run their Instagram account}. P13 shared an anecdote about helping a friend prepare an Instagram post, including selecting the \inlinequote{post order,} highlighting \inlinequote{how seriously we take social media and our image online.}

In contrast, SMH was envisioned as a platform designed for intentional, engaging interactions with close friends, making it naturally less prone to issues like social comparison. P09 described it as \inlinequote{the antonym to pressure,} \inlinequote{stress-free,} \inlinequote{community oriented,} \inlinequote{wholesome,} \inlinequote{accessible,} and \inlinequote{accepting.} Unlike MSM, which often feels performative and stressful, participants emphasized features and structures fostering organic, intentional, and personalized interactions, leading to closer friendships.

\subsubsection{Flexible and Safe Spaces for Meaningful Conversations}
\label{lab:4-2-1}
Many participants valued direct, synchronous conversations as the best way to build closer relationships. P22 wanted, \inlinequote{actual conversations} for \inlinequote{real connections,} preferring \inlinequote{individualized chatting\ldots{}to get to know people\ldots{}rather than just posting on a thread.} They envisioned \inlinequote[02]{chat rooms you can walk into,} each dedicated to specific topics, such as a room for \inlinequote{high school stuff,} decorated with artifacts from shared experiences to spark conversations. These were seen as\inlinequote{protected space\ldots{}where you can talk\ldots{}without feeling unsafe or judged} contrasting the perceived lack of safety on MSM, where users\inlinequote{don't know everything around it.}

P09 imagined a flexible, \inlinequote{Room of Requirement}~\cite{room-of-requirement}-like system for creating customized rooms \inlinequote{on a moment's notice,} reflecting the \inlinequote{purpose} of conversations. P07 also imagined that personalized chatroom aesthetics could make conversations feel more intimate, such as\inlinequote{photos of you together, inside jokes, or fun memories.} Interest-based spaces, like Discord~\cite{discord}, were also valued for providing a\inlinequote[13]{common point} for conversation and \inlinequote[07]{a good talking point for introductions.}


\subsubsection{Community Gathering Places}
\label{lab:4-2-2}
SMH's envisioned community spaces blended intimate, private interactions with dynamic public areas, fostering shared experiences and casual encounters.

\parHeading{Activity Spaces for Shared Experiences}
Participants saw shared activities as central to friendship-building. P06 envisioned \inlinequote{a little card game area} like Roblox, where \inlinequote{shared experiences} of \inlinequote{something fun} create \inlinequote{real memories}  to be reminisced, \inlinequote{do you remember that time when\ldots?} They added, \inlinequote{Part of the fun is not only are we actually in the space, but doing something together\ldots{} we're experiencing something together, almost.} P12 valued collaborative \inlinequote{projects online,} such as \inlinequote{creat[ing] a house we love,} offering \inlinequote{a sense of self-achievement} and \inlinequote{teamwork} \inlinequote{instead of just like doom scrolling.} Sharing that experience together \inlinequote{on the same topic, on the same project} makes it \inlinequote{a better way to connect, instead of just\ldots{} posting.}Even \inlinequote{mindless scrolling} became meaningful when shared. P18 noted that \inlinequote{see[ing] their reaction, hav[ing] a discussion, and\ldots{}commentat[ing] in real-time, instead of just sending it back and forth,} would feel\inlinequote{10 times more fulfilling.} 

\parHeading{(Public) Third Places for Casual Encounters}
Although most of the interactions participants envisioned centered around friends, participants also emphasized public spaces for serendipitous interactions. P03 described chance encounters while shopping: \inlinequote{you strike a conversation because you're both picking up the same shirt\ldots{} little connections like that happen just because you're in a public space.} Similarly, P08 suggested a marketplace space where selling items could spark conversations: \inlinequote{`Hey, look! This is for sale'\ldots{} have a talk about it\ldots{} and meet new people.} P21 imagined \inlinequote{small arenas} for \inlinequote{Quidditch games\ldots{}[where users] can choose to get seated randomly [and] just start talking to the person next to you\ldots{}and meet new people.} Other suggestions included \inlinequote[17]{public study hall[s]} where participants could \inlinequote[17]{interact with everybody,} leading them to experience that, \inlinequote[17]{all of a sudden, you know this person that plays the piano.}

\subsubsection{Ambient Co-Presence}
\label{lab:4-2-3}
Participants valued the idea of low-intensity connections through ambient co-presence. P18 described a \inlinequote{study sesh} where friends could send their avatars to \inlinequote{sit and study together} in a virtual study room with Lo-Fi music playing in the background. They explained that these spaces would mirror the 24/7 activity of Discord servers, offering a comforting presence even when not actively engaging in direct conversations. P06 highlighted the value of ambient interactions, emphasizing how these experiences allow people to\inlinequote{shar[e] that space physically, even though it's not really physical.} Whether it was \inlinequote{experiencing the cafe together} or \inlinequote{read[ing] a book together} at a library, these shared moments of co-presence were described as uniquely meaningful: \inlinequote{We're experiencing that together\ldots{} just experiencing the space.}

Music would facilitate these interactions, with P19 envisioning a \inlinequote{jukebox} where friends could \inlinequote{queue up songs.} They reflected on their experience: \blockquote{At one point, literally, we sat for an hour in silence, listening to a whole playlist\ldots{} we were just sitting there listening, vibing the music. And it was one of the most awesome nights because we all knew in that moment that collectively we all enjoyed what we were hearing and it was just good. I don't know how else to describe it. It was just nice.}{19} They added that music could also serve as a \inlinequote{conversation starter} while also providing a passive, enjoyable way to connect: \inlinequote{really does bring joy to people because you don't have to connect through talking to someone about music. You can just listen to the music.}

\subsubsection{Effortful and Intentional Navigation}
\label{lab:4-2-4}
Physical travel was seen as a key way to make interactions feel more meaningful and intentional. While teleportation options like \inlinequote[06]{spells} or the \inlinequote[21]{Marauder's Map} were seen as the convenient default, several wanted the option of being able to travel around the virtual world, as if they were taking a \inlinequote[14]{nature walk}. As P21 remarked, \inlinequote{Many times\ldots{} the best memories happen during a walk.} 

For some, the value of their redesigned interactions would come from the intentionality they would bring. P02 described engaging in SMH with deliberate purpose rather than habitual use, preferring a slower pace: \inlinequote{Teleporting would be nice, but not in this situation for me\ldots{} I wouldn't really want anything to be super fast decisions. I want everything to be very thought through.} Similarly, P05 envisioned navigation in SMH as requiring focused effort: \blockquote{[Ideally, SMH would] have some way that you specifically have to search for stuff\ldots{} where you have to intentionally go in looking for the account or the post versus it just being handed to you\ldots{} you would have to walk through a town to get there. It's not challenging to navigate, but you have to intentionally go there to see it. It isn't something you can do mindlessly.}{05}
P14 highlighted how such intentional interactions on SMH would feel more \inlinequote{personal} compared to seeing \inlinequote{a random 2D Insta post that they posted 3 months ago.} Participants also noted that SMH's intentional nature would amplify SMH's focus on meaningful connections. P05 speculated, \inlinequote{Influencers probably wouldn't be as keen on something that is more time-consuming\ldots{} People who are genuinely just trying to connect with others would probably benefit\ldots{} because it is more personal.} 

For others, the fulfillment came from the embodied experience of physical actions. P04 valued \inlinequote{meandering through different people's spaces,} finding that deliberate exploration heightened their sense of being \inlinequote{more present in somebody's space.} P09 echoed this, appreciating how moving physically between rooms created a \inlinequote{feeling of time passing.} They explained, \inlinequote{there's a feeling like you're actually officially leaving a space as opposed to clicking into a different one.} 

