\subsection{Granular, Intuitive, and Fun Privacy Mechanisms}
\label{lab:4-4}
Participants felt that current social media platforms offer limited privacy options, often lacking flexibility and nuance. P16 noted, \inlinequote{It's kind of a pain to organize, and\ldots{} right now in Instagram you can put close friends\ldots{} that's about it.} They expressed a need for more granular controls, particularly in a platform like SMH, where users might share more authentic aspects of their lives. As P11 explained, \inlinequote{If people are posting more like their genuine real life and their experiences, [privacy] would be a little bit more complicated,} emphasizing the importance of having \inlinequote{flexibility} to control access.

\subsubsection{Space-Based Privacy}
\label{lab:4-4-1}
Participants envisioned using spatial or visual cues to establish privacy boundaries, drawing on intuitions from physical spaces. P03 described public areas like \inlinequote{the downstairs area,} \inlinequote{guest bedroom,} or \inlinequote{game room} as spaces where people could \inlinequote{just come in} without permission, while private spaces like their bedroom required \inlinequote{extra consent} as it was a \inlinequote{safe space.} P13 suggested intermediate zones like a \inlinequote{waiting room} for sharing \inlinequote{semi-personal things} before granting further access.

P01 contrasted 2D and 3D settings, noting how boundaries over personal space are clearer: \inlinequote{In 2D, you're just observing.\ldots{} But in a 3D setting, even if you don't leave physical footprints, in a way, they're still there.} Therefore, they expected others to recognize and respect spatially cued privacy norms: \blockquote{I don't think people should be able to request access to areas when you're not there.\ldots{} Why would you want to go into my virtual bedroom when I'm not even home? If I wouldn't do it in real life, I probably wouldn't want to do it online either.}{22} 

P13 elaborated on how privacy norms in physical world contexts transfer to the SMH world, stating: \blockquote{A bedroom is a respected personal space\ldots{} If you label it as a bedroom, it just feels more intimate. It's where you spend most of your time, and it shows your personality the most. It's not like you can easily change the stuff you have in your room.\ldots{} If someone were to come over, you can't just rearrange it in 30 minutes---or 30 seconds---the way you can on Instagram. If you wanted to add someone there, all you'd have to do is remove a couple of things. It's easier than physically changing a space.}{13}

While managing visibility across multiple rooms might initially seem more complex, participants believed this approach would feel intuitive and natural. P11 suggested that spatial contexts could simplify privacy controls, stating, \inlinequote{It would make things easier because you'd be able to easily tell who's allowed and not allowed in certain areas of the house.} 

\subsubsection{Playful Privacy}
\label{lab:4-4-2}
Participants envisioned engaging privacy mechanisms imbued with a sense of playfulness and fantasy. P11 proposed a system where houses or specific objects within personal rooms could incorporate elements like a \inlinequote{secret knock} or an object that reveals a \inlinequote{secret passage to\ldots{} secret rooms} when interacted with in a specific way known only to an exclusive list of friends. Others suggested knowledge-gated access, such as a \inlinequote[21]{secret handshake} with their friends. P02 drew inspiration from the Marauder's Map, envisioning a system where users could only see accessible spaces unlocked by summoning secret \inlinequote{spells.} P05 likened playful privacy mechanisms to \inlinequote{a little Easter egg}, stating \inlinequote{There's that little kid part of everyone} that makes the features enjoyable. They noted that these elements would feel \inlinequote{fun} and \inlinequote{novel} as opposed to MSM's privacy settings, which often come across as uninspired or confusing, \inlinequote{like a wall of settings where sometimes the title of what the setting is doesn't even make sense.} 

\subsubsection{Contextual Privacy}
\label{lab:4-4-3}
Participants reflected on the challenges of navigating context collapse on MSM~\cite{BoydMarwick-2011-ITweetAudience-z}, and hoped SMH would make contextual disclosure easier: \blockquote{On Muggle social media, you've got to create a whole new account. But I'm thinking for this app, it just telepathically knows what's in Muggle area and what's in Hogwarts area. So you don't really have to go through the effort of keeping two lives. You can just walk through a portal. And you're like a new person.}{02}

P03 proposed a framework where every user has a \inlinequote{base} identity but could customize additional aspects of their representation. P22 imagined a contextual profile where \inlinequote{any group you're in\ldots{}you could change your avatar for that,} drawing inspirations Discord where \inlinequote{you can make different profiles for different servers.} P14 highlighted the importance of individual control, suggesting that \inlinequote{each of them should have their own setting, because I feel like everybody has different levels of what they want to keep to themselves.}

Participants also emphasized that contextual privacy goes beyond person-based access control. For example, they suggested varying access based on temporal factors. P14 proposed that \inlinequote{people who don't follow you\ldots{} can't see past versions\ldots{} I'm cool with you seeing the 6-year-old version of me on this, but I don't want you to see my 12-year-old version.} The idea of gradual openness was another suggestion: \blockquote{if you're meeting a complete stranger, you wouldn't want to give them like exactly how you look\ldots{} But then, after a couple of conversations, you sort of get the gist of how they are, and whether you want them to see you\ldots{} so then you can turn it on. Or turn it off.}{21}


\subsubsection{Invisibility}
\label{lab:4-4-4}
Participants were critical of existing platforms for their lack of a robust blocking system. P22 noted that on Instagram, even with contact syncing off, \inlinequote{they might still suggest accounts based on the follow relationships,} exposing a Finsta (Fake + Instagram; secondary account~\cite{VitakHuang-2022-FinstaAccounts-t}) account to people they would rather avoid. They also emphasized the need to block someone thoroughly, including their \inlinequote{alternate accounts,} that \inlinequote{I don't wanna be able to tell that the person I blocked is even around me.} P14 added that on Instagram, even private accounts allow \inlinequote{a random person} to see personal content such as Story highlights~\cite{story-highlights}.

Many users expressed a desire for greater control over their visibility on social media, emphasizing features that would allow them to be effectively ``invisible'' unless they chose otherwise. P03 envisioned users having \inlinequote{an invisibility cloak} for situations where \inlinequote{you don't want anyone to see you.} Similarly, P07 proposed a \inlinequote{super private} profile option where others \inlinequote{can't even search\ldots{} like you're not even on the app.} 

The ability to prevent unsolicited contact was another recurring theme. P22 suggested a model inspired by Discord, where \inlinequote{unprompted contact} happens only with people \inlinequote{you've already interacted with,} and others \inlinequote{can't even be discovered unless you have something in common.} This approach resonated with P13, who felt strongly that \inlinequote{people should only be able to add you if you meet in person and exchange a QR code.} They reflected on an experience where they were able to find personal information about a waitress online based just on their name, noting \inlinequote{it's too easy to find people on social media.}

\subsubsection{Age-Appropriate Spaces}
\label{lab:4-4-5}
Participants emphasized the importance of creating age-appropriate spaces for minors. P21, for example, argued \inlinequote{people who are 50} should not be engaging with \inlinequote{people who are 13.} Age verification was widely seen as a critical first step. P16 suggested incorporating \inlinequote{biometrics} into the verification process, while P19 argued that users should \inlinequote{legally prove} their age. Once reliable age verification is in place, participants advocated for creating separate spaces for minors to which older users do not have access. Conversely, P22 proposed that minors should have restrictions, such as being unable to join general spaces like a \inlinequote{town square.} 



