\section{Discussion}

\begin{tikzpicture}

    \node[circle, draw = black, fill = gray!40, inner sep = 0pt, minimum size = 1cm] (z) at (0, 0) {{$Z$}};
    \node[circle, draw = black, inner sep = 0pt, minimum size = 1cm] (t) at (0, 2) {{$T$}};
    \node[circle, draw = black, inner sep = 0pt, minimum size = 1cm] (yt) at (-1.8, 1.2) {{$Y(T)$}};
    \node[circle, draw = black, inner sep = 0pt, minimum size = 1cm] (xt) at (1.8, 1.2) {{$X(T)$}};
    \node[circle, draw = black, inner sep = 0pt, minimum size = 1cm] (xs0) at (-2.5, -2) {{$X^S(0)$}};
    \node[circle, draw = black, inner sep = 0pt, minimum size = 1cm] (xs1) at (-0.8, -2) {{$X^S(1)$}};
    \node[circle, draw = black, inner sep = 0pt, minimum size = 1cm] (ys0) at (0.8, -2) {{$Y^S(0)$}};
    \node[circle, draw = black, inner sep = 0pt, minimum size = 0.55cm] (ys1) at (2.5, -2) {{$Y^S(1)$}};
    
    \foreach \n in {t, yt, ys0, ys1}{\path [draw, ->] (z) edge (\n);}
    \path [draw, ->] (z) edge node[midway, above] {$g_t$} (xt);
    \path [draw, ->] (z) edge node[midway, below] {$g_0^S$} (xs0);
    \path [draw, ->] (z) edge node[pos=0.35, below, xshift=5pt] {$g_1^S$} (xs1);
    \path [draw, ->] (t) edge (yt);
    \path [draw, ->] (t) edge (xt);
     \node[draw, rectangle, right=20pt of z] {$f_t = g_t^{-1}$};
    
    \node[text width=1.5cm] (realdis) at (2, 3.3) {\small{Real DGP}};
    
    \node[text width = 1.2cm] (pz) at (-1.2, 0.) {\small{$Z \sim P_Z$}};
    \node[text width = 1cm] (pyt) at (-2, 2.1) {\small{$\mathcal{N}(\realmu_T(Z), \sigma_y^2)$}};
    \node[text width = 2cm] (pt) at (0.5, 2.8) {\small{$P_T(\bullet|Z)$}};
    \node[text width = 1.9cm] (pxt) at (1.7, 2.2) {\small{$\delta(X - \R_T(Z))$}};
    
    \node[text width = 3cm] (pxs) at (-1.1, -2.9) {\small{$ \delta(X^S - \S_T(Z))$}};
    \node[text width = 2.3cm] (pys) at (2, -2.9) {\small{$\mathcal{N}(\realmu^S_T(Z), \sigma_{y^S}^2)$}};
    
    \plate[] {plate1} {(z) (yt) (t) (xt) (pz) (pyt) (pt) (pxt) (realdis)} {};
    \plate[] {plate1} {(z) (ys0) (ys1) (xs0) (xs1) (pxs) (pys)} {\small{Simulator DGP}};

\end{tikzpicture}
\subsection{``Social Media is 2D'': Youth Perspectives on What Constitutes Social Media}
Public discourse increasingly portrays social media as fundamentally detrimental to mental health, friendship, and authentic social interaction. This narrative often equates social media with specific platforms like Instagram, TikTok, or X and the ways in which they happen to be designed today~\cite{positech}.

Our study participants expressed similar perspectives, frequently describing social media as centered around superficial interactions that fail to foster meaningful connections. As illustrated in Figure \ref{fig:diagram}(a), there is a disconnect between platforms that youth commonly identify as ``social media'' (such as Instagram and TikTok) and the platforms where they actually experience meaningful connections with friends (such as Discord or Minecraft). When envisioning their ideal social platform, participants described something that would bridge this divide---a space combining the deliberate pursuit of social connection they associate with traditional social media with the organic, engaged interactions they naturally experience in gaming and chat platforms.

This misalignment between expectations and experiences around social media is problematic because negative perceptions of social media use can diminish psychological well-being~\cite{lee2024social, kim2024privacysocialnormsystematically}. Youth experience disappointment in seeking deep social connections on platforms where such interactions are scarce. As shown in Figure \ref{fig:diagram}(b), social media serves diverse uses and gratifications---from information seeking and entertainment to status enhancement and network maintenance. Rather than maintaining an entirely negative view of social media confined to certain platforms, it is crucial to understand each platform's distinct values and limitations.

For instance, critiquing Instagram for not facilitating authentic, deep friendships misses the point; while such connections might occasionally form there, it is not the platform's central form of engagement. However, it is important to note that platforms aren't static---their designs evolve, and user engagement patterns shift over time. Instagram, primarily used for personal branding, sometimes facilitates closer communication through private `finsta' accounts. This fluidity in platform use underscores the importance of teaching~\cite{WilsonWisniewski-2012-FightingMyRegulation-m, Wisniewski-2018-PrivacyParadox-l, TheLearningNetwork-2025-WhatTeensMedia-z} youth to be more discerning about the connections they seek and what's possible on each platform rather than enforcing anxiety-inducing narratives about social media's harmful effects.

\subsection{Youth Seek More Meaningful Connections, Not Less Social Media Minutes}
\label{discussion2}
Our findings confirm prior studies that while the youth are largely dissatisfied with social media and how they spend their time on archetypal social media platforms, they do not necessarily want to reduce their social media usage---rather, they seek platforms that facilitate deeper, more meaningful connections. Our participants identified four key elements that could transform social media from a space of passive consumption to one of genuine connection:

\begin{enumerate}
    \item \textbf{Presence and Immersion.} Youth envision platforms that foster genuine emotional connection through real-time presence. While current platforms feel ``flat'' and emotionally disconnected, participants valued synchronous interactions that create a sense of co-existence. Even simple activities like sharing quick updates about eating ice cream or walking outside, or simply existing in a shared virtual space were seen as meaningful when done with immersive co-presence.
    \item \textbf{Supporting Individual Preferences.} Youth desire more natural and diverse ways to express themselves. Participants envisioned platforms where they could show their personality through both active sharing and subtle signals, similar to how personality naturally emerges in physical spaces. From memory-sharing to customizing their virtual spaces and avatars, they want to express themselves spontaneously and casually rather than feeling constrained to carefully curated posts.
    \item \textbf{Intuitive Social Navigation.} Meaningful social media experiences require design that mirrors how relationships naturally unfold. Youth value platforms that enable organic relationship growth through proximity and repeated low-intensity interactions, similar to how sitting in the same classroom gradually builds connections. They also seek designs that replicate the subtle, embodied norms of physical spaces---where spatial arrangements naturally signal degrees of privacy and social contexts, much like how a bedroom implies different privacy expectations than a living room.
    \item \textbf{Playful Engagement.} Youth emphasize that meaningful connection often grows through joy and shared experiences, yet they find current platforms overly serious and high-stakes. They envision platforms that encourage whimsical exploration and casual interaction---where they can discover hidden objects, explore virtual landscapes, or simply hang out in a ``third place.'' This playfulness creates low-stakes opportunities for friendship development beyond the performative nature of mainstream social media.
\end{enumerate}

These findings challenge the prevailing assumption that engaging youth requires addictive features or endless content streams. Instead, youth express a clear desire for platforms that prioritize meaningful connection over superficial engagement. While some of their specific suggestions may not be immediately implementable, their underlying desires reveal important principles for future platform design: spaces where there is a genuine presence of others, relationships can develop organically, privacy feels intuitive rather than bureaucratic, self-expression can be subtle and multifaceted, and users have genuine control over curating their social experience around relationships they truly value. 

Our participants envision environments where relationships deepen naturally through shared experiences and deliberate engagement rather than through metrics-driven interactions. This vision suggests that platforms prioritizing meaningful connections could redefine online social interaction while maintaining user engagement---challenging the assumption that profitable social media must rely on superficial, compulsive engagement strategies.

\subsection{Spatial Integrity for Meaningful Social Connections in Online Environments}
In the design of virtual environments that foster meaningful social connections, spatial cognition---the way we perceive and interpret space---remains critically relevant. Even in virtual spaces, human interactions are shaped by spatial affordances, mirroring many aspects of physical-world interactions. While virtual spaces need not replicate physical environments exactly---indeed, they can transcend physical limitations in some ways while falling short in others---certain spatial principles remain fundamental to how humans build and maintain relationships. This relevance is reflected in our pervasive use of spatial metaphors when describing virtual interactions, such as ``chat rooms,'' ``cyberspace,'' or ``folders.'' While some platforms like Discord and social VR applications successfully implement spatial design elements, mainstream social media platforms often lack these affordances. Through our interviews, we identify four key dimensions of what we term \textbf{\textit{spatial integrity}} to support meaningful social interactions in online spaces.

\begin{enumerate}
    \item \textbf{Spatial Presence: Understanding the Space and Who's There.} Spatial presence refers to the ability to perceive and navigate within a virtual space, encompassing elements such as proximity/propinquity (knowing who is nearby), navigation (moving within the space), and awareness (understanding one's position relative to others). While gaming platforms and social VR environments often excel at creating this sense of presence, mainstream platforms like Instagram lack basic spatial affordances, limiting users' awareness and intentional navigation within the platform.
    \item \textbf{Spatial Composition: The Meaning of the Space.} Spatial composition refers to ``the presence of fixed places\ldots{} that make interaction possible or likely''~\cite{AdlerSmall-2019-RoleSpaceTies-n}. Platforms like Discord successfully create persistent ``third places'' through dedicated servers and channels that serve as anchors for social connection~\cite{Kim-2025-Discord-third-place}, while many archetypal social media typically offers only transient content streams. Our participants emphasized the importance of dedicated areas---like communal lounges or study spaces---where friends could regularly gather, noting how such persistent spaces enable relationships to develop depth through shared memory and history.
    \item \textbf{Spatial Configuration: How the Space is Organized.} Spatial configuration is ``the segmentation of space into subunits with physical boundaries and pathways between them''~\cite{AdlerSmall-2019-RoleSpaceTies-n}. Clear boundaries and pathways help people recognize territories and manage social norms. Current platforms lack such clarity---a user's feed'' is meant to be personal space yet lacks clear boundaries, while profiles can be accessed without users' awareness. This absence of defined pathways for close friend interactions and boundaries for access control makes social media spaces feel disorderly or invasive, undermining meaningful social boundaries.
    \item \textbf{Spatial Depth: How Much Information Can Be Perceived.} Spatial depth determines how much information users can perceive in a space. 3D spaces enable higher perceptual density, allowing users to process information more intuitively~\cite{kumar2004effect, mcintire2014stereoscopic, tavanti20012d}. Current platforms, limited by 2D interfaces, restrict users to explicit actions like posting and commenting rather than supporting more nuanced forms of expression. Incorporating spatial depth could enable richer self-expression and more natural interactions.
\end{enumerate}

These four dimensions of spatial integrity---presence, composition, configuration, and depth---are critical for designing virtual environments that support meaningful social connections. While some platforms have successfully incorporated these elements, mainstream social media often lacks these affordances, contributing to the dissatisfaction youth feel toward platforms like Instagram that we explored in Section \ref{discussion2}. As spatial computing technologies gain prominence, there is an opportunity to integrate spatial integrity into the design of future platforms. (It is important to note that while spatial depth is generally more pronounced in 3D environments, the other dimensions of spatial integrity can be effectively achieved in 2D spaces, as demonstrated by platforms like Discord~\cite{Kim-2025-Discord-third-place}.)

The concept of spatial integrity offers a framework for understanding why many social media platforms feel disconnected and unsatisfying. By thoughtfully incorporating these principles, we can design online environments that better align with human social-spatial cognition, addressing the limitations our participants identified and better supporting meaningful social interactions and the organic development of relationships.

\subsection{Overcoming Cognitive Fixation and Doom Narratives Through Fictional Inquiry}
Our interviews reveal how mainstream social media's design fixations~\cite{jansson1991design} have shaped youth's perceptions of what social platforms can be. Youth have come to equate social media with certain limiting characteristics: endless content streams, superficial interactions, and 2D interfaces focused on broadcasting posts. Interestingly, while they report meaningful social connections and positive experiences in other online social spaces like Minecraft or Discord, they view mainstream social media as inherently tied to emotionally disconnected experiences and superficial interactions. This cognitive fixation not only overlooks how social features can be designed differently---as demonstrated by their own designs---but also reinforces negative perceptions of social media that prior literature suggests can impact psychological well-being and quality of life~\cite{lee2024social, kim2024privacysocialnormsystematically}.

To help youth envision beyond these constraints, we leveraged the FI method, immersing participants in a magical world setting. The Hogwarts-inspired context provided fresh metaphors for exploring social spaces, privacy, and connection. Unlike traditional design workshops where youth typically propose incremental improvements to existing platforms, these sessions enabled participants to imagine entirely new possibilities for meaningful connection. More importantly, the workshops helped participants expand their understanding of what social platforms could be, empowering them to articulate desires they might have previously dismissed as impossible within conventional social media paradigms. This study demonstrates how design methods can offer a generative path toward hope in domains where narratives of doom are prevalent.

\subsection{Limitation and Future Work}
Our study has several key limitations to consider. First, all participants had prior experience with 3D gaming platforms, potentially biasing them toward favorable attitudes toward incorporating spatial elements or online environments that center play. Second, although we were cognizant of how fixed perceptions around ``social media'' might constrain ideation, our continued use of this terminology throughout the interviews may have inadvertently reinforced existing mental models rather than encouraging completely novel conceptualizations. Third, while the Harry Potter narrative framework successfully transported participants into a magical design space, our specific scenario about a Muggle-born student introducing social technology sometimes led participants to focus on wizard-Muggle divisions rather than the intended focus on connection across distance. This suggests opportunities to refine the narrative approach. 

Future research could empirically test the spatial integrity framework to validate whether incorporating its different dimensions indeed fosters more meaningful social connections in online spaces. This could involve prototyping and evaluating platforms that systematically vary presence, composition, configuration, and depth. 

Another area for future work stems from participants' surprising openness toward the idea of Dobby, a house elf, as a personal assistant. They imagined Dobby having limitless magical powers to help maintain their personal spaces and said they would share almost everything with him---including personal details they would not share with friends. Even when interviewers mentioned that Dobby might report back to the house elf society (a metaphor for social media companies), participants did not seem concerned about privacy and remained willing to share. This is particularly striking given that participants were otherwise skeptical of AI. This raises interesting questions about how youth might engage with AI when it is framed within a familiar and relatable narrative. It suggests that embedding AI in natural, empathetic contexts might lead people to trust and interact with it far more than they otherwise would. Future research should delve into the privacy implications of this dynamic and explore how narrative framing could influence users' relationships with AI in ways that are not immediately obvious.
