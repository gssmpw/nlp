\section{Introduction}
Most youth today use social media daily and spend significant time on platforms such as Instagram and TikTok~\cite{AndersonAnderson-2023-TeensSocial2023-s}. However, these same youth often express deep dissatisfaction with how they spend this time---not because they want less social connection, but because they desire more meaningful online interactions~\cite{HinikerKim-2024-SharingDesign-x}. They often feel constrained by the design patterns of archetypal social media platforms, which prioritize uninteresting and attention-grabbing content consumption over genuine connection~\cite{Landesman-2024-IInstagram-j}. This disconnect has contributed to broader societal concerns about social media's impact on youth well-being~\cite{OtherOther-2023-SocialMediaAdvisory-u}.

Rather than addressing these concerns constructively, public discourse has increasingly framed social media as a societal evil~\cite{Wike-2022-2ViewsSociety-y}, particularly in the U.S., where it is often blamed for issues ranging from democratic instability to what some call the \inlinequote{Cause of the Mental Illness Epidemic}~\cite{Haidt-2023-SocialMediaEvidence-c}. This narrative has prompted restrictive policy responses worldwide: 41 U.S. states suing Meta~\cite{meta-sue}, proposals for ``warning labels'' on social media use~\cite{Murthy-2024-SurgeonGeneralPlatforms-q}, restrictive legislation like the ``Stop Addictive Feeds Exploitation (SAFE) For Kids Act''~\cite{Other-Other-NYState2023-S7694A-p}, and outright bans on social media use for those under 16 in countries like Australia~\cite{Kim-2024-AustraliaBarred-y}. The EU's GDPR~\cite{gdpr} similarly limits teen access.

However, such alarmist narratives oversimplify a complex reality: the effectiveness of these narratives and policies (as well as the causal role of social media in youth mental health) remains unclear. Despite recognizing the issues raised in these narratives, the youth report more positive than negative personal experiences~\cite{Vogels-2023-TeensSocialSurveys-n}, even as they acknowledge potential risks to others~\cite{Vogels-2023-TeensSocialSurveys-n}. Rather than perpetuating fear-based narratives or attempting to completely rebuild existing platforms, researchers and teens themselves emphasize the importance of learning to navigate social media effectively~\cite{WilsonWisniewski-2012-FightingMyRegulation-m, Wisniewski-2018-PrivacyParadox-l, TheLearningNetwork-2025-WhatTeensMedia-z}. Furthermore, such alarmist responses and outright restrictions may be misguided---they impede the development of resilience~\cite{Wisniewski2018-rc} and induce extreme and unhelpful anxiety and mistrust without fostering constructive actions~\cite{kim2024privacysocialnormsystematically}, ultimately leading to lower psychological well-being associated with social media use~\cite{lee2024social}. 

This misalignment suggests that the solution lies not in the wholesale rejection of social media or attempts to limit youth access, but in expanding our understanding of what social platforms can be. Rather than trying to ``fix'' existing platforms or promote fear-based narratives, we need to broaden our vision of what social media could be. To explore alternative visions for social technologies that prioritize meaningful connection, we employed the Fictional Inquiry (FI) method~\cite{IversenDindler-2007-FictionalInquiry--designSpace-m} in co-design workshops with 23 participants (ages 15-24) experienced in 3D gaming environments (e.g., Minecraft~\cite{minecraft}, Roblox~\cite{roblox}). Participants were asked to envision a social media platform for students at Hogwarts, the fictional school from \textit{Harry Potter}~\cite{rowling2015harry, harrypotterfanclub}. This magical context freed participants from current platform constraints while maintaining familiar social dynamics.

Our findings reveal two important patterns in how youth view and experience online social platforms. First, they make a clear distinction in their minds: platforms like Instagram are what they consider ``social media''---where they spend significant time but rarely form meaningful connections. In contrast, they do not typically label platforms like Discord or Minecraft as ``social media,'' even though these are where they actually develop deep friendships and find their time meaningful. Second, because youth associate ``social media'' primarily with platforms like Instagram and its limitations, they become skeptical of social media's overall potential to support authentic relationships. This narrow definition of social media, shaped by their experiences with Instagram-like platforms, leads them to dismiss the entire category as inherently flawed.

Participants identified several key elements missing from archetypal social media platforms: (1) emotional connection through presence and immersion, (2) diverse forms of individual expression that feel more natural, (3) intuitive social navigation---ranging from privacy to relationship development---that enables them to leverage norms from the physical world, and (4) playful, low-stakes opportunities to develop friendships. Although participants expressed deep dissatisfaction with the time they currently spend on archetypal social media platforms, they felt that spending the same amount of time on platforms designed to prioritize meaningful connections would feel valuable.

We contribute empirical findings on what youth perceive as the most significant shortcomings of current social media platforms, as well as what they desire to experience on social media. Additionally, we propose the concept of \textit{spatial integrity} for social interactions in online spaces, which encompasses four dimensions of spatial affordances that facilitate meaningful social connections online. These dimensions include spatial presence (the ability to understand the space and identify who is present), spatial composition (the meaning and significance of the space), spatial configuration (the organizational structure of the space), and spatial depth (the extent of information that can be perceived within the space). Furthermore, we demonstrate that the FI method, in the context of social media design, not only prompted creative idea generation and enabled participants to transcend current social media design patterns but also empowered them to envision the positive potential of social media, leaving them feeling more hopeful about its future possibilities.