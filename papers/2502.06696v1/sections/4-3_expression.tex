\subsection{Expressive and Lower-Stakes Sharing}
\label{lab:4-3}
Participants emphasized that SMH should enable rich personal expression, allowing them to authentically and casually reflect their personalities and interests through memory-sharing and customization, from decorating dorm rooms to avatars.

\subsubsection{Expression Through Personal Area and Avatar Customization}
\label{lab:4-3-1}
Participants wanted to create and personalize virtual environments and avatars that reflect their identities, preferences, and aspirations. Unlike MSM that only offers \inlinequote[10]{two modes}---\inlinequote[10]{light and dark mode[s]}---P21 envisioned a highly customizable world where users could create their own \inlinequote{mini world} within a larger virtual space, complete with \inlinequote{houses, teleport points, and rooms like a kitchen, a bathroom, a bedroom, or even a school.} They emphasized how this space could be shared with friends, \inlinequote{showing a little part of yourself.} P03 imagined a \inlinequote{dark academia} Hogwarts-inspired room filled with \inlinequote{so many bookshelves and\ldots{} Victorian architecture,} to reflect their interests. P01 expanded on these ideas, envisioning a house with rooms \inlinequote{dedicated to each of your interests.} They added how that would enable them to showcase the different facets of their identities: \inlinequote{I'm into multiple things---you could have a room for Harry Potter, a room for neuroscience, and a room for another fandom or interest.} 

\inlinequote[11]{3D avatar[s]} and profiles were another key way participants envisioned expressing themselves. P22 highlighted how platforms like Myspace~\cite{myspace} allowed for a level of customization that reflected individuality, describing how they enjoyed coding profiles to change \inlinequote{the font and really make it reflect me.} P01 also shared that there is an inherent \inlinequote{joy} in \inlinequote{just that process of creating it, and finding out who you are through that process.} Even just choosing a username was seen as a way to \inlinequote{convey to the world and your friends your own identity\ldots{} which is a positive feeling usually.} Reflecting on Tumblr, P01 explained how such personalization is \inlinequote{mainly for you,} not just for others to see. P17 felt similarly, stating, \inlinequote{I don't think it necessarily conveys personal aspects of yourself, but I think it will without you meaning it to.} As they put it, \inlinequote{your brain conjured that up, right? So I think that's part of you.}

Participants highlighted how expression through choice feels authentic. For example, P01 felt that such \inlinequote{creative expression} could feel \inlinequote{a lot more personal} because \inlinequote{that's not just like what life has given you\ldots{} a reflection of your outlook on life.} P13 described this kind of indirect sharing as \inlinequote{mak[ing] social media an extension of everyday interaction,} allowing people to \inlinequote{gain the basic foremost information} about someone without requiring direct conversations. P21 suggested this would make it \inlinequote{easier for people to approach you and say, `Hey, I like that, too.'}

P10 explained that being in a 3D, VR environment amplifies this expressive nature, stating that 2D worlds feel more \inlinequote{compressed} and \inlinequote{able to see as much stuff as you can see in 3D.} P16 noted that in VR, others are \inlinequote{basically right in front of you when they're talking,} making their \inlinequote{mannerisms\ldots{} clearly easily observable.} They elaborated that one could learn a lot about someone based on \inlinequote{the way they hold themselves\ldots{} the way they decorate their house\ldots{} how they choose to dress their in-game character.} This experience feels more authentic, as though meeting with \inlinequote{more real people\ldots{} compared to the profile pictures.}

In addition to expressiveness, participants appreciated how these features support subtle and low-stakes sharing. P16, who described themselves as \inlinequote{fairly private} and not particularly \inlinequote{public} about their preferences, valued the ability to \inlinequote{share things more subtly.} For instance, they explained that \inlinequote{decorat[ing] a house} or customizing an avatar allows others to \inlinequote{get to know more about my personality\ldots{} through my style.} They found this form of self-expression to be \inlinequote{more justified} than posting, which felt \inlinequote{embarrassing} and \inlinequote{just doing this for other people.}

\subsubsection{Real-Time Sharing: Capturing States and Changes}
\label{lab:4-3-2}
Another recurring theme was the ability to reflect real-time states through external artifacts in virtual spaces. P04 shared how they might opt into sharing updates on \inlinequote{what I'm doing at the moment. For example, if I'm at the shops, maybe I'd want to share that.} P21 extended this idea to items purchased such as a \inlinequote{new pair of New Balance shoes,} imagining virtual spaces that automatically displayed \inlinequote{recently added or recently stored items.}

Participants envisioned that the blending of physical and virtual selves would allow for a more personalized medium of sharing. P01 imagined their avatar reflecting what outfit they're wearing in the moment or what song they're listening to, adding that this could help friends intuitively know \inlinequote{`Oh. that's what she's listening to right now' or `That's what she's thinking about.'} P10 described how frequent updates to virtual rooms could reveal changes in personal interests, providing friends with \inlinequote{an insight into\ldots{} what I'm thinking.}

Participants also discussed the ability to adapt avatars to reflect real-time emotions or actions. P21 described avatars that \inlinequote{read your emotions\ldots{} showing [them] on your faces,} while P19 envisioned avatars mimicking real-life gestures. Music was frequently mentioned as a subtle and meaningful way to share current moods.


\subsubsection{Sharing Memories: Multisensory and Emotional Experiences}
\label{lab:4-3-3}
A major theme was the ability to share memories directly with others, like the \inlinequote{Pensieve}~\cite{pensieve}. P04 imagined sharing experiences like \inlinequote{jumping into a pool or taking the first bite of an ice cream sundae} directly, creating more of an emotional connection. Sharing memories could also help reduce misunderstandings by enabling more accurate and nuanced expression, unlike MSM, where communication is mostly \inlinequote[03]{just words.} 

P07 analogically described memory-sharing as \inlinequote{screen sharing, but with your eyes.} P08 highlighted the practicality of memory-sharing, explaining how it could capture transient moments---like crossing a state line---without needing to pull out a camera. P03 highlighted how this would enable them to \inlinequote{just give} memories to their mom in real-time, instead of doing so \inlinequote{after the fact,} where they often found themselves missing the right opportunities.

P04 added that memory-sharing would foster more deliberate consumption of content: \blockquote{[SMH] feels more personal and interactive\ldots{} if I could take more time in a less easy-to-consume format with what others choose to share.\ldots{} On Instagram, when you click on someone's page, you see all the posts at once. You don't go one by one, taking time to read through every comment.\ldots{} Here, you'd have to take more time to hone in on a memory\ldots{}\ldots{} It's the satisfaction of consciously consuming content from someone you're close to, being deliberate in viewing what they share, versus just quickly looking and being done.}{04}

Some participants imagined that these memories would not just be sensory but should also include emotions and perspectives. P02 explained how sharing memories could help others \inlinequote{understand why I said what I said} and prevent \inlinequote[02]{misconstru[al]s.} For some, memories weren't just visual; P01 expressed a desire to \inlinequote{transmit smells through social media,} adding another sensory dimension to shared experiences. Given ethical concerns, many felt that only minimal editing, such as \inlinequote[01]{cropping certain parts you don't want to share} or versions \inlinequote[07]{sped-up or slowed-down,} should be allowed. 

