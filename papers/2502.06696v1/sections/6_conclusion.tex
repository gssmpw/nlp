\section{Conclusion}
Through co-design workshops employing the Fictional Inquiry (FI) method and leveraging a Hogwarts-inspired narrative, we explored the youth's vision for an ideal social platform. The interviews revealed a disconnect between mainstream social media and the kinds of meaningful online connections young people seek. While platforms like Instagram have become synonymous with ``social media,'' youth find their most authentic social interactions happening elsewhere---in gaming environments, chat platforms, and other spaces that they don't typically seek as destinations for social connection.

The participants collectively emphasized four key elements missing from current social platforms: emotional connection through real-time presence and immersion, natural forms of self-expression beyond carefully curated posts, intuitive social navigation that mirrors physical-world relationship development, and playful low-stakes opportunities for friendship growth. These findings challenge the widespread assumption that engaging youth requires addictive features or endless content streams. Rather, young people express a clear desire for time that feels meaningfully spent, imagining that social media platforms enabling meaningful connection through shared experiences and deliberate engagement would provide exactly that.

Importantly, this research demonstrates how moving beyond standard design patterns through methods like Fictional Inquiry can help youth articulate possibilities they might have previously dismissed as impossible within conventional social media paradigms. Rather than accepting current platforms' limitations as inevitable, participants were able to envision radically different approaches to online social connection and feel empowered and hopeful about the positive potential of social media that they initially felt pessimistic about.

Further, we introduce spatial integrity as a framework for facilitating meaningful social connections in online environments, helping to understand why many social platforms feel disconnected and unsatisfying while pointing toward design principles that could better support meaningful online relationships. From 2D platforms to spatial technologies, our study highlights an opportunity and hopeful future for creating social experiences that work with, rather than against, our natural needs and intuitions for social connection.