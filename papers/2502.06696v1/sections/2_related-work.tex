\section{Related Work}
\subsection{Defining Social Media and its Role in Young People's Lives}
Social media does not yet have a widely agreed-upon definition. There is a range of interpretations, from broad (e.g., ``any interactive communication medium that enables two-way interaction and feedback''~\cite{kent2010directions}) to specific (e.g., ``a different kind of Internet-based applications which build the ideological and technological foundations of Web 2.0, and allow users to create content and exchange it with other people through the Internet''~\cite{kaplan2010users}). While some platforms, such as Instagram~\cite{instagram} or Facebook~\cite{facebook-official}, are indisputably considered social media, others, such as YouTube~\cite{youtube} and Discord~\cite{discord}, are less archetypal. There are also different types of social media, and even such classifications have multiple interpretations~\cite{kaplan2010users, sharma2018social, scott2015new}. Some major categories include social networking sites (i.e., ``web-based services that allow individuals to (1) construct a public or semi-public profile within a bounded system, (2) articulate a list of other users with whom they share a connection, and (3) view and traverse their list of connections and those made by others within the system''~\cite{Boyd-2007-SocialNetworkScholarship-i}) and blogs (i.e., ``a way to share that love with the world and encourage an active community of readers who comment on the posts of the author''~\cite{scott2015new}).

% \noindent \textbf{What is social media?} Social media does not yet have a definitive definition. There is a range of interpretations, from broad~\footnote{``any interactive communication medium that enables two-way interaction and feedback''~\cite{kent2010directions})} to specific~\footnote{``a different kind of Internet-based applications which build the ideological and technological foundations of Web 2.0, and allow users to create content and exchange it with other people through the Internet''~\cite{kaplan2010users}}. While some platforms, such as Instagram~\cite{instagram} or Facebook~\cite{facebook-official}, are indisputably considered social media, others, such as YouTube~\cite{youtube} and Discord~\cite{discord}, are less archetypal. There are also different types of social media, and even such classifications have multiple interpretations~\cite{kaplan2010users, sharma2018social, scott2015new}. Some major categories include social networking sites~\footnote{``web-based services that allow individuals to (1) construct a public or semi-public profile within a bounded system, (2) articulate a list of other users with whom they share a connection, and (3) view and traverse their list of connections and those made by others within the system''~\cite{Boyd-2007-SocialNetworkScholarship-i}} and blogs~\footnote{``a way to share that love with the world and encourage an active community of readers who comment on the posts of the author''~\cite{scott2015new}}.

The range of definitions and classifications of social media reflects the wide array of user needs that social media addresses. From a uses and gratifications perspective~\cite{GurevitchKatz-1973-UsesGratificationsResearch-y}, social media supports a variety of utilities, ranging from social interactions to information seeking and entertainment~\cite{whiting2013people}. Even within such broad categories of uses and gratifications, there are diverse ways these needs are pursued. For example, within social interactions, there may be a focus on bridging social capital and weaker ties versus bonding social capital and stronger ties~\cite{phua_uses_2017}. Further, even on a single platform---Instagram, for instance---users engage in a range of activities, from photo sharing for self-promotion and disclosure~\cite{Menon-2022-UsesGratificationsInstagram-x} to more passive consumption of media for entertainment. In other words, Instagram may serve as a personal branding tool for some while functioning as a way to pass the time for others.

% \parHeading{Youth experience with social media} Despite the prevalent narrative that social media is bad, research shows the youth experience with social media is more nuanced. While younger people have bad experiences on social media such as social comparison~\cite{Yau2019-ab}, significant privacy concerns, pressures to curate~\cite{BoydDanah2014ICTS}, time spent feeling meaningless~\cite{HinikerLukoff-2018-WhatMakes-o}, and more mild forms of discomfort~\cite{Landesman-2024-IInstagram-j}, they experience meaningful benefits from those platforms as well. For instance, marginalized populations have been known to benefit particularly from these sites~\cite{AcenaLi-2023-WeReality-s, BellBates-2020-"LetMeDevelopment-u} and these platforms are used as medium for identity development and expression~\cite{Davis2012-bq, Lee-2022-AlgorithmicCrystalTikTok-b}. Given such a nuanced relationship between social media and youth, researchers recommend that it is beneficial to help teens learn to be more resilient with social media experiences rather than focusing on restricting or preventing harm~\cite{Wisniewski2012-nu}.

Despite the prevalent narrative that social media is harmful, the experiences of youth with these platforms are more nuanced. They face challenges such as social comparison~\cite{Yau2019-ab}, significant privacy concerns~\cite{Weinstein2022-rh, kim2024privacysocialnormsystematically, Boyd_2014}, pressures to curate their online presence~\cite{Yau2019-ab}, and feelings of meaningless time spent~\cite{HinikerLukoff-2018-WhatMakes-o}, along with more subtle discomforts~\cite{Landesman-2024-IInstagram-j}. However, these platforms also offer meaningful benefits. For instance, marginalized populations often derive substantial support from these spaces~\cite{AcenaLi-2023-WeReality-s, BellBates-2020-"LetMeDevelopment-u}, and social media serves as an important medium for identity development and self-expression~\cite{Davis2012-bq, Lee-2022-AlgorithmicCrystalTikTok-b}. Given this complex relationship between youth and social media, researchers advocate for strategies that focus on building resilience and equipping teens to navigate these experiences effectively rather than solely aiming to restrict or mitigate harm~\cite{Wisniewski2012-nu, Wisniewski-2018-PrivacyParadox-l, WisniewskiAgha-2023-StrikePrevention-r}.

% paper on Time spent on smartphone feels meaningless vs. meaningful (Kai)

% papers that talk about The bad side
% - correlation between youth social media use and mental health (though not causal)
% - social comparison, mental health, FOMO, filtered and curated posts, only positive ones
% - more mild
% - context collapse -> can't authentically share self

% papers that talk about The good side
% - queer, psychiatric hospitalization
% - identity crystallization
% - political activism
% - emotion regulation

% importance of social media --> shouldn't just ban/restrict --> need to teach resilience



\subsection{Social Technology Design Explorations in HCI}
% HCI research has explored different designs that foster meaningful social interactions online. Ranging from social media platforms such as BeReal~\cite{HinikerKim-2024-SharingDesign-x}, Snapchat~\cite{Bayer2016-fa}, Miitomo~\cite{KaufmanKasunic-2017-BeMeApplication-h}, and TikTok~\cite{Barta2021-yh, Schaadhardt2023-jj} to individual features such as those that support effortful communication~\cite{LiuFannie2021SOUt, LiuZhang-2022-AuggieEncouragingExperiences-n} or genuine connection~\cite{Stepanova2022-vh}. Recently more spatial social technologies such as social virtual reality (VR) platforms~\cite{Zamanifard-2023-SurpriseBirthdayDistance-n, Freeman-2024-MyAudiences-u, Freeman-2020-MyBodyReality-l, Zamanifard-2019-TogethernessCraveRelationships-s}, Gather.town~\cite{duarte2023experience,tu2022meetings}, and VRChat~\cite{chen2024d,deighan2023social,rzeszewski2024social} have been explored.

HCI research has extensively explored designs that foster meaningful social interactions online, examining both platform-level innovations and individual features. Social media platforms such as BeReal~\cite{HinikerKim-2024-SharingDesign-x}, Snapchat~\cite{Bayer2016-fa}, Miitomo~\cite{KaufmanKasunic-2017-BeMeApplication-h}, and TikTok~\cite{Barta2021-yh, Schaadhardt2023-jj} have introduced unique affordances for sharing and connecting, with an emphasis on spontaneity, ephemerality, and creative self-expression. This line of research identified different ways to build and maintain relationships through features that prioritize authenticity and playful interaction. However, users have reported increasingly toxic interactions as they gain popularity, with design choices encouraging users to expand their networks and engage in compulsive behaviors, such as tracking Snap scores~\cite{chambers2022s, van2023snapchat}, which detract from their initial goals of fostering meaningful interactions. 

At the feature level, efforts to support meaningful communication include designs that encourage effortful communication~\cite{LiuFannie2021SOUt, LiuZhang-2022-AuggieEncouragingExperiences-n} and those that foster feelings of genuine connections~\cite{Stepanova2022-vh}. While these features are important, social media experiences are often holistic, involving a complex interplay of multiple design aspects and user behaviors. As a result, such features alone cannot fundamentally alter the overarching narratives or perspectives surrounding social media.

Beyond traditional social media, recent work has investigated spatial social technologies, which emphasize the value of immersive, embodied interactions. Social virtual reality (VR) platforms~\cite{Zamanifard-2023-SurpriseBirthdayDistance-n, Freeman-2024-MyAudiences-u, Freeman-2020-MyBodyReality-l, Zamanifard-2019-TogethernessCraveRelationships-s} provide users with virtual spaces and embodiment for co-presence and shared experiences, facilitating interactions that feel more personal and tangible compared to 2D environments. Similarly, platforms like Gather.town~\cite{duarte2023experience,tu2022meetings} and VRChat~\cite{chen2024d,deighan2023social,rzeszewski2024social} blend elements of gaming and social networking, offering users the ability to navigate virtual environments, customize avatars, and participate in dynamic, context-rich group interactions. However, these platforms often are not generally recognized as social media in the traditional sense.

What youth envision as ideal social media, both at the broader landscape level and the specific feature level remains unclear---especially when considering the possibilities offered by emerging technologies such as spatial computing platforms. Understanding these perspectives is crucial for reimagining the future of social media and ensuring it meets the evolving needs of its users.

% snapscore - \cite{chambers2022s}

% \cite{Stepanova2022-vh}
\subsection{Designing Beyond Doom Narratives}
\parHeading{Social media and design fixation} Design fixation~\cite{jansson1991design}---the tendency to remain constrained by conventional thinking or existing solutions---is a well-documented challenge in design. While generating a wide range of divergent ideas is a critical first step in innovative design, we often impose unnecessary restrictions on our thinking. This self-limiting cognitive process often confines the design process to address localized problems rather than exploring broader, more optimal solutions.

Social media design is no exception. Given the ad-based revenue model, platforms are incentivized to increase time spent on their apps. However, instead of achieving this through meaningful utility and engagement that genuinely enhances users' experiences, they often resort to less healthy, compulsive strategies, such as friend recommendations~\cite{HinikerKim-2024-SharingDesign-x} or engagement metrics (e.g., Snap scores~\cite{rozgonjuk2021comparing}). These tactics drive superficial interactions rather than fostering deeper connections, leaving users dissatisfied and reinforcing the perception that all social media designs and user experiences are fundamentally the same~\cite{Sundaram-Other-SocialMediaSame-g, Pardes-2020-SocialMediaSame-p}.

\parHeading{The Fictional Inquiry (FI) method} FI is an exploratory co-design method that enables participants to transcend real-world constraints through immersion in semi-fictional contexts~\cite{IversenDindler-2007-FictionalInquiry--designSpace-m}. At its core, FI creates an environment where people can imagine beyond present limitations by engaging with carefully crafted scenarios. These scenarios typically incorporate immersive elements such as physical artifacts or role-playing activities, with participants taking on generative roles that help them envision futuristic or fictional settings. The method is particularly effective at helping participants overcome design fixation by encouraging them to draw analogies from fictional scenarios and imagine freely outside real-world constraints. While FI shares the exploratory nature of other ideation techniques, it is distinguished by its emphasis on developing futuristic design materials through collaborative dialogue between designers and users~\cite{IversenDindler-2007-FictionalInquiry--designSpace-m}.

According to Dindler and Iversen~\cite{IversenDindler-2007-FictionalInquiry--designSpace-m}, implementing this method involves three key steps: first, clearly defining the inquiry's purpose, whether it's for staging design situations, exploring future ideas, or driving organizational change; second, developing an appropriate narrative that participants are comfortable with while being distinct enough from current practices to encourage new thinking; and third, creating a compelling plot that introduces tension or conflict within the narrative to motivate specific workshop activities. Unlike traditional role-playing approaches, FI encourages participants to maintain their own identities and expertise while operating within fictional contexts, allowing them to explore new possibilities while drawing from their personal motivations and capabilities, all while circumventing typical societal constraints. 