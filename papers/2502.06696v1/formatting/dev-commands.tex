%!TEX root = main.tex

%%
%% Note: This file includes all new commands that are needed for the preliminary 
%% paper-writing (draft / review) stage of the paper. These commands should NOT
%% be included in the camera-ready version of the paper.
%%
%% All content in this file should take at least one argument. Custom commands 
%% for simple aliases should be provided in "dev-aliases.tex".
%%

%% Note: Custom colors for in-paper comments
\definecolor{LIGHTPINK}{RGB}{237,157,202}
\definecolor{LIGHTRED}{RGB}{210,121,121}
\definecolor{LIGHTORANGE}{RGB}{230,170,50}
\definecolor{LIGHTGOLD}{RGB}{210,194,121}
\definecolor{LIGHTGREEN}{RGB}{121,210,121}
\definecolor{LIGHTAQUA}{RGB}{121,206,210}
\definecolor{LIGHTBLUE}{RGB}{121,124,210}
\definecolor{LIGHTPURPLE}{RGB}{153,102,255}
\definecolor{RED}{RGB}{178,34,34}
\definecolor{GRAY}{RGB}{166,166,166}
\definecolor{WHITE}{RGB}{255,255,255}

%% Note: General TODO and cut commands
\newcommand{\todo}[1]{
    \addcontentsline{toc}{subsection}{  %
        \protect\numberline{}           % Align text appropriately
        \textcolor{RED}{[TODO] #1}}     % Add todo to the table of contents
    \textcolor{RED}{[TODO] \emph{#1}}}  % Typeset the todo note in the text
\newcommand{\cut}[1]{\textcolor{RED}{\st{#1}}}
\newcommand{\revise}[1]{\textcolor{blue}{#1}}

%%
%% Note: Labeled in-paper comments for collaborators (with and without underlined
%% text). Each author should specify a new command for their name with a
%% different comment color, using the example new command for \jane below. 
%% Alternatively, authors can use the more complicated \guest command, but must
%% provide their name each time they leave a comment.
%%

%% Usage: \jane[(OPTIONAL) Text to underline in square brackets]{Comment.}
\newcommandx{\jane}[2][1=] 
    {\setulcolor{LIGHTGREEN}{\ul{#1}} \textcolor{LIGHTGREEN}
    {[\textbf{Jane:} #2]}}
%% Usage: \guest[(OPTIONAL) Text to underline]{Name}{Comment text}
\newcommandx{\guest}[3][1=]
    {\setulcolor{LIGHTORANGE}{\ul{#1}} \textcolor{LIGHTORANGE} 
    {[\textbf{#2:} #3]}}

%%
%% Note: Custom status badges can be added to sections and subsections to label 
%% which ones are ready (or not) for feedback to facilitate collaboration amongst
%% authors. Badges should be added directly to the section heading.
%%
\newcommand{\badge}[2]{\colorbox{#1}{\small\textcolor{WHITE}{\texttt{#2}}}}
\newcommand{\headerBadge}[2]{\hspace*{\fill}\badge{#1}{#2}}

\newcommand{\locked}{\headerBadge{LIGHTPURPLE}{locked}}
\newcommand{\complete}{\headerBadge{LIGHTBLUE}{complete}}
\newcommand{\feedbackProvided}{\headerBadge{LIGHTGREEN}{feedback provided}}
\newcommand{\readyForFeedback}{\headerBadge{LIGHTORANGE}{feedback requested}}
\newcommand{\underRevision}{\headerBadge{LIGHTRED}{under revision}}
\newcommand{\incomplete}{\headerBadge{RED}{incomplete}}


