\subsection*{Introduction}
Welcome, and thank you for participating in our design workshop. Before we begin, I want to ensure that you've read the consent form I sent earlier. Have you had a chance to review it?

\textbf{[Response]}

\noindent Great! Just to reiterate, this session will be recorded for research purposes. Do I have your consent to proceed with the recording?

\textbf{[Response]}

\noindent Thank you. If at any point you feel uncomfortable or wish to stop the interview, please let me know.

\subsection*{General}
Let's start with some general questions about your experiences with 3D gaming and Harry Potter.

\begin{itemize}
    \item Which 3D games do you play and how long have you been playing them?
    \item (Which is your favorite?) Can you describe the 3D elements or the spatial elements in that game a little bit, please?
    \item Have you ever played in VR mode if that is available?
    \item Any AR games you have played? Like Pokémon Go?
    \item What Harry Potter books or movies have you watched?
    \item Which is your favorite?
    \item Which Hogwarts house would you put yourself in?
\end{itemize}

\subsection*{Narrative/Plot}
As mentioned in the consent form, our goal today is to explore innovative ways to improve social media. We particularly invited 3D game players with a love for Harry Potter because we believe this will help us think creatively and ``outside the box'' about what an ideal, magical social media could look like, beyond the confines of a small 2D smartphone screen.

Don't worry about being too creative---we're here to guide you through the design process, not to evaluate you. We want to explore different design directions with your help, and we'll be actively involved while also giving you the space to share your ideas.

To give our brainstorming some structure, we'll start with a specific scenario to solve during this session. I'm going to ask you some guiding questions to help you address the scenario:

\begin{quote}
\textit{Imagine you are a student at Hogwarts. You have friends who are Muggle-born, friends from magical families, and even friends at other wizarding schools around the world. You're also getting to know me, as I'm a new student you've just met at Hogwarts. Keeping in touch with everyone is tough because traditional Muggle social media like Instagram doesn't really portray your wizard self very well. You don't have a smartphone in the first place because Hogwarts people don't need electronic devices, so you can't really charge your smartphone.}\\
\textit{One day, a brilliant Muggle-born student had an idea. They decided to use magic to create a magical social media where everyone---Muggles, Hogwarts students, professors, and friends from other magical schools---could all communicate and share their lives in the most ideal way. Imagine you're using this new magical platform---or maybe it's not even a platform. Whatever it is, what do you think this would look like?}
\end{quote}

\subsection*{Character Map}
Before we dive into the designing part, let's first map out your relationships in your new life as a Hogwarts student. Use the Miro board link (Figure \ref{fig:miro}) to write down the names or nicknames of people that you would like to put in each section. For example, you might want to put ``younger sister'' or ``Jane'' under Family, some of your coworkers under Quidditch Team members, yourself under the house that you think you belong to, your older middle school friends as Muggles, etc.

Please note the two boundaries in the four rectangles for each house. You can place your closer friends in the inner box and not-so-close ones in the outer box. And you can put the same name in multiple boxes (or parallelograms) as you wish.

\begin{itemize}
    \item How do you currently communicate with or stay connected with each of these people via Muggle social media?
    \item What are your biggest pet peeves/issues with Muggle social media that you'd like to solve with magical powers?
    \item What moments do you feel bad/guilty about using social media, if at all?
    \item When does time spent on social media feel meaningless/meaningful/fulfilling, if at all?
    \item Pet peeves around privacy settings?
    \item When do you feel most connected to your friends when interacting with them on social media?
    \item What would a better social media be if you were to use adjectives to describe it? What does “better” mean to you?
\end{itemize}

\subsection*{Design}
Now, let's think about what this magical world social media might look like. Remember, there are no right or wrong answers here. We're looking for your imagination and creativity, so feel free to think outside the box and have fun with it!

\begin{itemize}
    \item What key capabilities would this thing have for communicating with [GROUP]*?
    \item If you could design this thing with any magical powers, what form would it take? (Is it a magical mushroom? A secret room? A creature?)
    \item What would you want to share about yourself on this ideal magical ``thing'' with [GROUP]?
    \item If you had [GROUP'] join this ``thing,'' how might its form adapt to include this new group?
\end{itemize}

*: Iterate for 1) Romantic Relationships and Closest Friends, 2) Closer Hogwarts Friends, 3) Muggle Friends, 4) Friends at Other Magical Schools, 5) Quidditch Team Members, 6) House Elf


\subsection*{Connecting Back}
\begin{itemize}
    \item What do you like about this magical “thing” compared to Muggle social media?
    \item What do you dislike about it?
    \item Imagine using the magical “thing” we discussed earlier. How do you think it could resolve some of the issues people might face with Muggle social media?
    \item How might this new magical “thing” better support your needs for self-presentation and identity management compared to traditional social media?
    \item What challenges might arise in using such a “thing” at Hogwarts?
\end{itemize}

Thank you so much for your time and insights. Your contributions are incredibly valuable to our research. Do you have any final thoughts or questions before we conclude?
