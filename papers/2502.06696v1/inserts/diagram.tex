\begin{figure}[t]
    \centering
    \includegraphics[width=0.9\linewidth]{inserts/diagram.png}
    \caption{(a) Current youth perceptions of social media platforms and where they experience meaningful connections, highlighting the disconnect between platforms commonly recognized as ``social media'' versus where youth actually experience meaningful social connection. The SMH---youth-envisioned ideal social media---is represented by the overlap, combining both aspects. (b) A more nuanced viewpoint clarifies ``social media'' as an umbrella term for a range of platforms with different primary purposes---friendship-building, entertainment, status, information-seeking, tie maintenance, and more---rather than a few archetypal platforms.}
    \Description {This diagram is divided into two parts, offering insights into how youth perceive social media and how they categorize its purposes. On the left, there are two overlapping circles. The circle on the left represents platforms that youth identify as ``social media,'' such as Instagram, Facebook, and TikTok. The circle on the right illustrates platforms where youth feel meaningfully connected with friends, including FaceTime, Discord, Minecraft, and Roblox. In the overlapping area falls Social Media at Hogwarts (SMH) youth-envisioned ideal social media that combines the qualities of both categories. On the right, a Venn diagram shows potential categories of social media platforms by their primary purposes, such as ``Entertainment-Centered,'' ``Status-Centered,'' and ``Information-Seeking-Centered.'' ``Social Media at Hogwarts'' is grouped under ``Strong Tie-Centered.''}
    \label{fig:diagram}
\end{figure}
