\section{Related work}
There exists a large literature on denotational models for probabilistic computation. Many of these works differ substantially from ours in two aspects. First, they do not use random variables and, instead, model probabilistic computation using probability measures____, or continuous distributions____; second, these works do not use Scott domains but instead employ a larger class of dcpo's____, probabilistic coherence spaces____, or categories built from measurable spaces____.  

In the existing literature, there are a limited number of constructions of a probabilistic monad based on random variables. A monad construction quite similar to ours, but not incorporating soft-conditioning, can be found in____. In____, random variables are defined as pairs consisting of a domain and a function, but without discussion of monad commutativity. A strong monad of random variables is defined in____, although the construction differs significantly from ours. Domains of random variables with structures similar to our approach are defined in____, but without presenting monad constructions. Random variables are also used to define quasi-Borel spaces by Heunen et al.____.

The treatment of conditional probability in probabilistic programming has received significant attention. Various approaches to soft conditioning can be found in the literature: Staton____ employs metric spaces, Park et al.____ use sampling functions, and Goubault-Larrecq et al.____ develop a domain-theoretic framework. The relationship between scoring and conditioning has been explored in several works____, with different approaches for implementing conditioning in higher-order languages.