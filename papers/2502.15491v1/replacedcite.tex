\section{Related Work}
Vibration analysis has been used extensively in UAV CM and AD. Simsiriwong and Sullivan use sensor data from multiple accelerometers on the wing structure of a UAV to determine the vibrational characteristics and dynamic properties of a composite UAV wing through frequency analysis ____. Radkowski and Szulim identify and eliminate adverse vibrations appearing in sensor data when a UAV performs aerial maneuvers using vibrational analysis and rotor modelling ____. Wolfram, \textit{et al.} develop models of the UAV drive trains and use various sensors yielding differential pressure, current, voltage, sound, thermocouple, and acceleration data ____. \par
Thresholding methods have also been developed to identify anomalous conditions. Bektash and la Cour-Harbo propose the use of periodograms and power spectrum estimations derived from vibration data to infer the mechanical integrity of the UAV and indicate failures ____. Al-Haddad, \textit{et al.} use the Fast Fourier Transform (FFT) to extract frequency-domain features from the sensor data which is then used to identify faults through the isolation and characterization of fault signatures ____. Banerjee, \textit{et al.} perform in-flight detection of vibrational anomalies related to motor defects by performing the FFT on the vibration signal, computing the Power Spectral Density (PSD), binning data into intervals and comparing to a pre-defined threshold value to identify anomalies ____. \par
ML methods have also been employed to perform UAV CM. Pourpanah, \textit{et al.} use a fuzzy adaptive resonance neural network and a genetic algorithm for feature selection to extract 15 harmonic features from the vibration signal to identify motor and propellor faults ____. Zahra, \textit{et al.} leverage LSTMs for abnormal flight detection and health indicator prediction to estimate the remaining useful life using time-domain signal features, such as RMS, Kurtosis, Skewness, and Crest Factor ____. Wang, \textit{et al.}  use LSTMs to perform point AD through uncertainty interval estimation based on 1-dimensional UAV flight sensor data ____.

Previous work by the authors considered UAV CM through rotor defect detection in pre/post flight settings ____. In past work, a comprehensive analysis was conducted to determine the best-performing ML model and the impact of the various feature sets on the model's predictive performance. However, there was no consideration of the impact of network resource utilization on the optimal solution, and accuracy was the only metric used to compare model performance. To address the limitations of past solutions and the gaps in the state-of-the-art, the work presented in this paper focuses on optimizing both the ML model performance (using additional ML model performance metrics) and the network resource utilization efficiency (using a crafted feature throughput metric).
The contributions of this work are as follows:
\begin{itemize}
    \item The development of a network resource-aware UAV CM framework.
    \item A trade-off analysis between network resource utilization and ML model performance.
\end{itemize}