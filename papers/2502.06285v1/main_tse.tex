

\documentclass[journal,twoside]{IEEEtran}
%\usepackage[preprint]{mlspconf}
\usepackage{graphicx} % Required for inserting images
\usepackage{cite}
\usepackage{comment}
\usepackage{balance}
\usepackage{acronym}
\usepackage{amsmath}
\usepackage{tabularx}
\usepackage[dvipsnames]{xcolor}
%\usepackage{fixltx2e}
\usepackage[hang]{subfigure}
\usepackage{booktabs}
\usepackage{tikz}
\usetikzlibrary{positioning}
%\usepackage{caption}
\usetikzlibrary{shapes.multipart}
\usepackage{amsfonts}
\usepackage{enumitem,bbm}
\usepackage{wrapfig,hyperref}
\usepackage{soul}
\section{Problem Studied}\label{sec:def}
We first present Fixed-Radius Near Neighbor (FRNN) queries and then formalize Aggregation Queries over Nearest Neighbors (AQNNs) that build on them. We then state our problem.

\subsection{Nearest Neighbor Queries}\label{subsec:FRNN}
We build on generalized Fixed-Radius Near Neighbor (FRNN) queries \cite{FRNNSurvey}. Given a dataset \( D \), a query object \( q \), a radius \( r \), and a distance function \( dist \), a generalized FRNN query retrieves all nearest neighbors of \( q \) within radius \( r \). More formally:
\[
NN_D(q, r) = \{x \in D \mid dist(x, q) \leq r\},
\]
where \(x\) is any data point in \(D\) and \(dist(x, q)\) denotes the distance between them. We use \(|NN_D(q,r)|\) to denote the neighborhood size of \(q\). As shown in Fig. \ref{fig:framework}, given a radius \(r\) and a target patient \(q\), patients in the dotted circle are nearest neighbors, and the neighborhood size is 6.

\subsection{Aggregation Queries over Nearest Neighbors}\label{subsec:AQNN} 
Given an FRNN query object \(q\) in dataset \(D\), a radius \(r\), and an attribute \(\texttt{attr}\), an Aggregation Query over Nearest Neighbors (AQNN) is defined as:
\[ \text{agg}(NN_D(q,r)[\texttt{attr}]) \]
where agg is an aggregation function, such as $\mathtt{AVG}$, $\mathtt{SUM}$, and $\mathtt{PCT}$, and \(NN_D(q,r)[\texttt{attr}]\) denotes the bag of values of attribute \texttt{attr} of all FRNN results of \(q\) within radius \(r\). 
% \end{definition}

An AQNN expresses aggregation operations to capture key insights about the neighborhood of a query object. For example, \(\mathtt{AVG}\) can be used to reflect the average heart rate or systolic blood pressure of patients in the neighborhood, providing a measure of typical health conditions. \(\mathtt{SUM}\) is useful for assessing cumulative effects, such as the total cost of treatments in the neighborhood that instructs public policy in terms of health. Similarly, $\mathtt{PCT}$ can be used to find the proportion of patients in the neighborhood of a patient of interest, relative to the population in the dataset.
%\laks{Why is finding the total \#meds to NNs or the total treatment cost of everyone in the NN interesting?}

% \texttt{MIN} and \texttt{MAX} are not included in the aggregation functions because they only capture extreme values, which may not represent the typical characteristics of the nearest neighbors and are more sensitive to outliers. 
% \laks{AVG is also sensitive to outliers, but we still allow it. isn't the real reason we don't consider MIN/MAX because they are amenable to estimation via sampling?} We choose \texttt{PCT} instead of \texttt{COUNT} in order to provide a normalized measure that remains comparable across different neighborhood sizes. It allows for more consistent interpretation of relative popularity \cite{moore1989introduction}.


Fig. \ref{fig:framework} illustrates an example of an AQNN: ``\textit{Find the average systolic blood pressure of patients similar to an insomnia patient \(q\)}''. The aggregation function is \(\mathtt{AVG}\) and the target attribute of interest is systolic blood pressure. Exact query evaluation requires consulting physicians (or predicting embeddings by an expensive machine learning model) for all 500 patients in \(D\) and calculate \(q\)'s nearest neighbors wrt \(r\) \cite{DBLP:journals/isci/RodriguesGSBA21}. We refer to such highly accurate but computationally expensive models as \textit{oracle models}, denoted as \(O\), including deep learning models trained on domain-specific data or human expert annotations \cite{DBLP:conf/sigmod/LuCKC18}. Using oracle models is very expensive \cite{sze2017efficient, DujianPQA, DBLP:journals/pvldb/KangGBHZ20}. To address that, we seek an approximate solution by \textit{proxy models}, denoted as \(P\), that are at least one order of magnitude cheaper than oracle models. In the example, if consulting physicians for one patient incurs one cost unit, calling a cheap machine learning model instead incurs at most \(0.1\) cost unit. Once the similar patients are identified, their systolic blood pressure values are averaged and returned as  output. The use of a proxy model may reduce the accuracy of the neighborhood prediction and hence, we should judiciously call oracle and proxy models to minimize the error of aggregate results.

Note that the values of the target attribute \texttt{attr} are \textit{not} predicted but are instead known quantities.

\subsection{Problem Statement}
Given an AQNN, our goal is to return an approximate aggregate result by leveraging both oracle and proxy models while reducing error and cost.


\begin{acronym}
\acro{gan}[GANs]{Generative Adversarial Networks}
\acro{rl}[RL]{Reinforcement Learning}
\acro{pae}[PAE]{Periodic Autoencoder}
\acro{fld}[FLD]{Fourier Latent Dynamics}
\acro{ppo}[PPO]{Proximal Policy Optimization}
\acro{fft}[FFT]{Fast Fourier Transform}
\acro{pca}[PCA]{Principal Component Analysis}
\acro{dfm}[DFM]{Deep Fourier Mimic}
\acro{dof}[DoF]{Degrees of Freedom}
\acro{mlp}[MLPs]{Multi-Layer Perceptrons}
\end{acronym}





\begin{document}

\title{End-to-End Multi-Microphone Speaker Extraction Using  Relative Transfer Functions }
\author{Aviad Eisenberg, Sharon Gannot, Shlomo E. Chazan}

\author{Aviad Eisenberg, Sharon Gannot,~\IEEEmembership{Fellow,~IEEE}, and Shlomo E. Chazan
\thanks{A. Eisenberg (\texttt{aviad.eisenberg@biu.ac.il}) and S. Chazan (\texttt{shlomi.chazan@biu.ac.il}) are with Bar-Ilan University and OriginAI, Israel; S. Gannot (\texttt{sharon.gannot@biu.ac.il}) is with Bar-Ilan University, Israel.} 
\thanks{The work was partially supported by a grant from the Audition Project, Data Science Program, Council of Higher Education, Israel.}
}

\markboth{IEEE Signal Processing Letters,~Vol.~x, No.~y, Dec.~2024}%March~2017}%
{Eisenberg \textit{et al.}: {E2E Deep BeamFormer for Speaker Extraction}}

\maketitle

\begin{abstract}
This paper introduces a multi-microphone method for extracting a desired speaker from a mixture involving multiple speakers and directional noise in a reverberant environment. In this work, we propose leveraging the instantaneous \ac{RTF}, estimated from a reference utterance recorded in the same position as the desired source. 
The effectiveness of the \ac{RTF}-based spatial cue is compared with \ac{DOA}-based spatial cue and the conventional spectral embedding. 
%
Experimental results in challenging acoustic scenarios demonstrate that using spatial cues yields better performance than the spectral-based cue and that the instantaneous \ac{RTF} outperforms the DOA-based spatial cue.
\end{abstract}

\begin{IEEEkeywords}
Speaker Extraction, Spatial and Spectral Cues, Relative Transfer Function
\end{IEEEkeywords}

\section{Introduction}

%%% single channel separation/extractiom %%%
% \IEEEPARstart{O}{ver} the past few years, significant advancements have been made in the field of speaker separation and speaker extraction, particularly in situations involving a single microphone. Several studies, including \cite{chazan2021single,chen2020dual,luo2019conv,luo2020dual,lutati2022sepit,nachmani2020voice,subakan2021attention,tzinis2022compute,wang2021end,zeghidour2021wavesplit}, have demonstrated remarkable outcomes in separating speakers in ideal, echo-free environments. However, the practical application of these findings is limited since real-world conditions are not devoid of echoes. Conversely, \cite{cord2022monaural} highlighted that although considerable progress has been achieved in speaker separation metrics for non-reverberant scenarios, there has been relatively little advancement in handling reverberant data.

%%%  anecohic reference %%%
% \IEEEPARstart{O}{ver} the past few years, significant advancements have been made in the field of speaker separation and speaker extraction, particularly in situations involving a single microphone \cite{li2024spmamba,wang2023tf,zhao2024mossformer2}.
% Unlike single-microphone setups, utilizing a microphone array makes it feasible to enhance a desired speaker by leveraging available spatial information. Specifically, the model can be trained to enhance a speaker located at a relevant position or \ac{DOA}.
% %
% In most of the cases \noteb{in order to avoid the blind estimation of the desired \ac{DOA}, which is prone to errors in most cases, the oracle \ac{DOA} is used as a reference for the model.} \cite{xu2020neural,xu2021generalized,zhang2021adl,tesch2023spatially,elminshawi2023beamformer,gu2020multi}. In \cite{zhang2021adl}, the elements of the \ac{MVDR} beamformer were estimated, which included the \ac{ATF} and the noise covariance matrix. To avoid the need for matrix inversion, the model directly estimated the inverse covariance. 
% In \cite{elminshawi2023beamformer}, a \ac{DSB} oriented towards the \ac{DOA} was computed. The output of the \ac{DSB} was then utilized as a reference signal for a single-channel \ac{SE} model. The primary objective behind this approach was to facilitate a generalization across various types of microphone arrays, achieving improved performance.

% %%%  ecohic reference %%%
% On the other hand, other studies \cite{vzmolikova2017learning,delcroix2020improving,zorilua2021investigation,han2021multi,ge2022spex} propose a different methodology where the reference signal lacks spatial information and instead focuses on vocal features of the speaker. Typically, the reference signal utilized is the unechoic speech signal of the desired speaker, distinct from the signal present in the mixture, disregarding spatial cues. In ,\cite{ge2022spex} a multi-task model is introduced, which incorporates several tasks. In addition to estimating the \ac{MVDR} parameters using complex-mask estimation, the model also predicts the \ac{DOA} based on the mixture signal. Furthermore, a Cross Entropy loss is computed per speaker using the unechoic reference signal.


%%%  SHARON  %%%
The extraction of a desired speaker from multi-microphone mixtures containing multiple simultaneous speakers is essential in numerous modern applications and devices, including virtual assistants, hearing aids, and smartphones.

In recent years, multi-microphone algorithms based on \acp{DNN} have emerged. In \cite{FaSNet, IFaSNet, markovic22_interspeech}, the network directly processes multi-microphone signals to separate speakers in the scene. However, the absence of a beamformer structure complicates the explainability of the spatial properties of these approaches.


In another category of multi-microphone audio processing, known beamforming criteria or the beamformer weights are estimated directly \cite{ADLMVDR, DeepWithMVDR, NICEbeamRLR, CausalUnet, eabnet2022, Walter2022, Walter2023, Cohen2024Explainable, schwartz2024multi}. Studies, such as \cite{Walter2023, Cohen2024Explainable}, demonstrate that the spatial properties of the filter-and-sum operation can be preserved.



When side information about the desired speaker is available, the speaker separation task is referred to as \ac{TSE}. This side information can take various forms, such as a ``voice signature'' obtained during an enrollment stage, spatial information like the \ac{DOA} of the desired speaker, or visual cues, such as lip movements. A comprehensive survey on speech extraction methods is available in \cite{Zmolikova2023Extraction}. 
%
In \cite{delcroix2018single, Xu2019Ext, delcroix2020improving, Eisenberg2022Extraction, eisenberg2023two}, voice signatures are used to extract the desired source from a mixture recorded by a single microphone. This is typically achieved by inferring a speaker embedding from an enrollment stage.
%
In addition to the speaker embedding, speaker extraction can also leverage spatial information, usually inferred from multi-microphone measurements. The spatial information can be incorporated through \ac{DOA} estimation and/or by steering a beamformer \cite{vzmolikova2017learning,zorilua2021investigation,han2021multi,ge2022spex,elminshawi2023beamformer,gu2020multi,xu2020neural,xu2021generalized,tesch2023spatially,tesch2022tasl,teschtasl2024}. 
%
The works in \cite{teschtasl2024,tesch2022tasl} specifically analyze the spatial properties of the multi-microphone extraction method.

%\noteb{Unlike beamforming techniques, which explicitly utilize spatial information, neural networks incorporate spatial cues implicitly as part of their learning process.}

%%%  disadvantages and our profit  %%%

% The main drawbacks of the approaches above are as follows: When using a non-spatial enrollment signal, the model relies on spectral differences between speakers for separation while ignoring spatial information, which is a strong cue for separation. Additionally, when \ac{DOA} information is used, its estimation process can introduce errors, potentially degrading the output quality. 

\ac{TSE} algorithms that employ ``voice signature'' for enrollment do not fully exploit spatial information, albeit processing multichannel data. Integrating the target speaker’s \ac{DOA} as a spatial cue may enhance performance. However, it is well established that \ac{RTF}-based beamformers typically outperform \ac{DOA}-based beamformers in reverberant environments \cite{Gannot2017}.

To address these limitations, we propose utilizing the instantaneous \ac{RTF} features of the desired speaker. Furthermore, we conduct an extensive comparison between the proposed \ac{RTF}-based method and alternative approaches that use spectral and \ac{DOA} enrollment features, along with a comparison to the \ac{MVDR} beamformer. Our results demonstrate that the proposed method consistently outperforms all other approaches.







\section{Problem Formulation}
A scenario comprising $Q$ concurrently active speakers, captured by $J$ microphones in a reverberant and noisy environment, is addressed. The problem is formulated in the \ac{STFT} domain, with $k \in \{0,\ldots, K-1\}$ and $t \in \{0,\ldots, T-1\}$ representing the frequency index and time-frame index, respectively, with $T$ and $K$ the total number of time-frames and frequency bands, respectively. 
%By assuming that the \acp{ATF} are time-varying, 

The observed signal, as received by the microphone array, can be modeled as:
\begin{equation} 
\x = \sum_{q=1}^{Q} \hq \cdot \sq + n(t,k) \cdot \hn + \mathbf{v}(t,k),
\end{equation}
where $\sq$ denotes the clean, anechoic speech signal of the $q$-th speaker, $\hq$ is a $J \times 1$ vector comprising the time-invariant \acp{ATF} relating the $q$-th source and the microphone array, $n(t,k)$ is a directional noise source, $\hn$ is a $J \times 1$ vector comprising the \acp{ATF} relating the noise and the microphone array, and $\mathbf{v}(t,k)$ represents the sensor noise.



We focus on the scenario where only two concurrent speakers are present, namely $Q=2$, referred to as the desired speaker $\sd$ and the interference speaker $\si$.
Define the reverberant speech sources as received by the microphones as $\tilde{\mathbf{s}}_q(t,k) = s_q(t,k)  \mathbf{h}_q(k)$.

Denote $s_d^{\text{ref}}(t,k)$, the enrollment signal for $\sd$ and the respective reverberant signal as received by the microphone array as $\tilde{\mathbf{s}}_d^{\text{ref}}(t,k)$. 
Given the mixed signal $\x$ and the enrollment, we aim to extract a reverberant replica of the desired speaker, $\tilde{\mathbf{s}}_d(t,k)$.
We stress that the enrollment signal is uttered from the exact position of the desired source to provide the required spatial cue. 



\begin{figure*}[htbp]
\centering
\tikzset{every picture/.style={line width=0.75pt}} %set default line width to 0.75pt        

    

\begin{tikzpicture}[x=0.75pt,y=0.75pt,yscale=-1,xscale=1]
%uncomment if require: \path (0,548); %set diagram left start at 0, and has height of 548

%Image [id:dp9069361724339233] 
\draw (44.64,245.51) node  {\includegraphics[width=38.12pt,height=35.67pt]{mix.png}};
%Shape: Rectangle [id:dp7118385203559439] 
\draw   (19.23,221.73) -- (70.06,221.73) -- (70.06,269.29) -- (19.23,269.29) -- cycle ;

%Image [id:dp18908880465589117] 
\draw (47.74,248.59) node  {\includegraphics[width=38.12pt,height=35.67pt]{mix.png}};
%Shape: Rectangle [id:dp9145815536815001] 
\draw   (22.33,224.81) -- (73.15,224.81) -- (73.15,272.37) -- (22.33,272.37) -- cycle ;


%Image [id:dp889198458016955] 
\draw (50.33,251.84) node  {\includegraphics[width=38.12pt,height=35.67pt]{mix.png}};
%Shape: Rectangle [id:dp21873244194200492] 
\draw   (24.92,228.05) -- (75.74,228.05) -- (75.74,275.62) -- (24.92,275.62) -- cycle ;

%Image [id:dp5455673038209854] 
\draw (53.43,254.92) node  {\includegraphics[width=38.12pt,height=35.67pt]{mix.png}};
%Shape: Rectangle [id:dp03955459689324581] 
\draw   (28.01,231.14) -- (78.84,231.14) -- (78.84,278.7) -- (28.01,278.7) -- cycle ;




%Shape: Rectangle [id:dp623708338705701] 
\draw   (156.1,369.84) -- (217.44,369.84) -- (217.44,386.25) -- (156.1,386.25) -- cycle ;
%Shape: Rectangle [id:dp8730998170177671] 
\draw   (156.1,386.25) -- (217.44,386.25) -- (217.44,402.67) -- (156.1,402.67) -- cycle ;
%Shape: Rectangle [id:dp12701003284697765] 
\draw   (156.1,419.08) -- (217.44,419.08) -- (217.44,435.5) -- (156.1,435.5) -- cycle ;
%Shape: Rectangle [id:dp8997571586346655] 
\draw  [fill={rgb, 255:red, 245; green, 166; blue, 35 }  ,fill opacity=1 ] (156.1,402.67) -- (217.44,402.67) -- (217.44,419.08) -- (156.1,419.08) -- cycle ;
%Image [id:dp8005824136332484] 
\draw (210.08,428.82) node  {\includegraphics[width=8.07pt,height=8.67pt]{hot.png}};

%Shape: Rectangle [id:dp8903454718586092] 
% \draw   (43.62,385.11) -- (104.96,385.11) -- (104.96,418.79) -- (43.62,418.79) -- cycle ;
%Straight Lines [id:da6050170519889082] 
\draw    (60.98,402.67) -- (154.67,402.67) ;
\draw [shift={(156.67,402.67)}, rotate = 180] [color={rgb, 255:red, 0; green, 0; blue, 0 }  ][line width=0.75]    (10.93,-3.29) .. controls (6.95,-1.4) and (3.31,-0.3) .. (0,0) .. controls (3.31,0.3) and (6.95,1.4) .. (10.93,3.29)   ;
%Straight Lines [id:da4466146283568919] 
\draw    (95.1,395.29) -- (111.75,409.6) ;

%Straight Lines [id:da8734877283024587] 
\draw    (217.44,404.02) -- (234.26,404.02) -- (329.82,404.98) ;
%Shape: Trapezoid [id:dp9083817837711585] 
\draw   (188.39,227.14) -- (248.75,244.9) -- (248.71,268.47) -- (188.31,286.06) -- cycle ;

%Straight Lines [id:da7452961115068255] 
\draw    (151.9,258.39) -- (186.06,258.39) ;
\draw [shift={(188.06,258.39)}, rotate = 180] [color={rgb, 255:red, 0; green, 0; blue, 0 }  ][line width=0.75]    (10.93,-3.29) .. controls (6.95,-1.4) and (3.31,-0.3) .. (0,0) .. controls (3.31,0.3) and (6.95,1.4) .. (10.93,3.29)   ;
%Shape: Rectangle [id:dp7969946792723472] 
\draw   (90.17,238.1) -- (151.51,238.1) -- (151.51,271.77) -- (90.17,271.77) -- cycle ;
%Image [id:dp653208688320088] 
\draw (46.17,331.03) node  {\includegraphics[width=38.12pt,height=35.67pt]{ref.png}};
%Shape: Rectangle [id:dp6293543724062918] 
\draw   (20.75,307.24) -- (71.58,307.24) -- (71.58,354.81) -- (20.75,354.81) -- cycle ;


%Rounded Rect [id:dp8404373430997238] 
\draw   (83.6,321.36) .. controls (83.6,316.38) and (87.63,312.35) .. (92.61,312.35) -- (157.24,312.35) .. controls (162.22,312.35) and (166.25,316.38) .. (166.25,321.36) -- (166.25,348.38) .. controls (166.25,353.35) and (162.22,357.38) .. (157.24,357.38) -- (92.61,357.38) .. controls (87.63,357.38) and (83.6,353.35) .. (83.6,348.38) -- cycle ;
%Image [id:dp356963447934719] 
\draw (95.48,347.61) node  {\includegraphics[width=8.14pt,height=8.98pt]{freeze.png}};

%Shape: Rectangle [id:dp5100610627802844] 
\draw   (177.7,318.11) -- (182.26,318.11) -- (182.26,346.31) -- (177.7,346.31) -- cycle ;
%Shape: Rectangle [id:dp09011362126686362] 
\draw   (191.39,318.11) -- (195.95,318.11) -- (195.95,346.31) -- (191.39,346.31) -- cycle ;
%Shape: Rectangle [id:dp39864885227757463] 
\draw   (186.82,318.11) -- (191.39,318.11) -- (191.39,346.31) -- (186.82,346.31) -- cycle ;
%Shape: Rectangle [id:dp2527551882246728] 
\draw   (182.26,318.11) -- (186.82,318.11) -- (186.82,346.31) -- (182.26,346.31) -- cycle ;
%Shape: Rectangle [id:dp04283930934410707] 
\draw   (195.95,318.11) -- (200.51,318.11) -- (200.51,346.31) -- (195.95,346.31) -- cycle ;

%Shape: Rectangle [id:dp8669269247694282] 
\draw   (403.89,243.52) -- (411.63,243.52) -- (411.63,283.07) -- (403.89,283.07) -- cycle ;
%Shape: Rectangle [id:dp4967486514075399] 
\draw   (248.21,164.74) -- (299.03,164.74) -- (299.03,212.3) -- (248.21,212.3) -- cycle ;
%Image [id:dp4494172403007415] 
\draw (273.62,188.52) node  {\includegraphics[width=38.12pt,height=35.67pt]{mix.png}};

%Shape: Rectangle [id:dp5105058612393181] 
\draw   (250.79,167.5) -- (301.61,167.5) -- (301.61,215.06) -- (250.79,215.06) -- cycle ;
%Image [id:dp3601443888825764] 
\draw (276.2,191.28) node  {\includegraphics[width=38.12pt,height=35.67pt]{mix.png}};


%Shape: Rectangle [id:dp264086641906389] 
\draw   (254.23,170.74) -- (305.05,170.74) -- (305.05,218.31) -- (254.23,218.31) -- cycle ;
%Image [id:dp665947925703801] 
\draw (279.64,194.53) node  {\includegraphics[width=38.12pt,height=35.67pt]{mix.png}};

%Shape: Rectangle [id:dp45829617666682276] 
\draw   (256.81,173.5) -- (307.64,173.5) -- (307.64,221.06) -- (256.81,221.06) -- cycle ;
%Image [id:dp7942977661620945] 
\draw (282.23,197.28) node  {\includegraphics[width=38.12pt,height=35.67pt]{mix.png}};




%Shape: Trapezoid [id:dp7414492319000967] 
\draw   (318.45,168.85) -- (378.8,186.61) -- (378.77,210.18) -- (318.36,227.78) -- cycle ;

%Shape: Rectangle [id:dp5945330428321489] 
\draw   (394.3,184.29) -- (398.86,184.29) -- (398.86,212.48) -- (394.3,212.48) -- cycle ;
%Shape: Rectangle [id:dp8698671127419058] 
\draw   (407.99,184.29) -- (412.55,184.29) -- (412.55,212.48) -- (407.99,212.48) -- cycle ;
%Shape: Rectangle [id:dp5735877548764554] 
\draw   (403.43,184.29) -- (407.99,184.29) -- (407.99,212.48) -- (403.43,212.48) -- cycle ;
%Shape: Rectangle [id:dp3180067674583993] 
\draw   (398.86,184.29) -- (403.43,184.29) -- (403.43,212.48) -- (398.86,212.48) -- cycle ;
%Shape: Rectangle [id:dp6133400642974811] 
\draw   (412.55,184.29) -- (417.12,184.29) -- (417.12,212.48) -- (412.55,212.48) -- cycle ;

%Straight Lines [id:da6791007277669958] 
\draw    (407.22,242.63) -- (395.09,214.32) ;
\draw [shift={(394.3,212.48)}, rotate = 66.81] [color={rgb, 255:red, 0; green, 0; blue, 0 }  ][line width=0.75]    (10.93,-3.29) .. controls (6.95,-1.4) and (3.31,-0.3) .. (0,0) .. controls (3.31,0.3) and (6.95,1.4) .. (10.93,3.29)   ;
%Straight Lines [id:da0872755594951693] 
\draw    (407.22,242.63) -- (416.49,214.38) ;
\draw [shift={(417.12,212.48)}, rotate = 108.18] [color={rgb, 255:red, 0; green, 0; blue, 0 }  ][line width=0.75]    (10.93,-3.29) .. controls (6.95,-1.4) and (3.31,-0.3) .. (0,0) .. controls (3.31,0.3) and (6.95,1.4) .. (10.93,3.29)   ;
%Straight Lines [id:da6992460529139395] 
\draw    (407.22,242.63) -- (406.01,214.69) ;
\draw [shift={(405.92,212.69)}, rotate = 87.53] [color={rgb, 255:red, 0; green, 0; blue, 0 }  ][line width=0.75]    (10.93,-3.29) .. controls (6.95,-1.4) and (3.31,-0.3) .. (0,0) .. controls (3.31,0.3) and (6.95,1.4) .. (10.93,3.29)   ;

%Shape: Trapezoid [id:dp8355402445105389] 
\draw   (493.25,227.85) -- (432.89,210.11) -- (432.92,186.54) -- (493.31,168.93) -- cycle ;
%Image [id:dp069542684835052] 
\draw (325.92,216.24) node  {\includegraphics[width=8.07pt,height=8.67pt]{hot.png}};
%Image [id:dp2256840457820628] 
\draw (486.2,215.43) node  {\includegraphics[width=8.07pt,height=8.67pt]{hot.png}};
%Shape: Rectangle [id:dp8752551784030962] 
\draw   (501.92,172.69) -- (552.74,172.69) -- (552.74,220.25) -- (501.92,220.25) -- cycle ;
%Image [id:dp06394701083698995] 
\draw (527.33,196.47) node  {\includegraphics[width=38.12pt,height=35.67pt]{output.png}};


%Shape: Rectangle [id:dp9977813446861878] 
\draw   (257.36,241.36) -- (261.92,241.36) -- (261.92,269.55) -- (257.36,269.55) -- cycle ;
%Shape: Rectangle [id:dp5684426909262685] 
\draw   (271.05,241.36) -- (275.61,241.36) -- (275.61,269.55) -- (271.05,269.55) -- cycle ;
%Shape: Rectangle [id:dp2011255461678516] 
\draw   (266.49,241.36) -- (271.05,241.36) -- (271.05,269.55) -- (266.49,269.55) -- cycle ;
%Shape: Rectangle [id:dp28639196402430267] 
\draw   (261.92,241.36) -- (266.49,241.36) -- (266.49,269.55) -- (261.92,269.55) -- cycle ;
%Shape: Rectangle [id:dp6241942522592696] 
\draw   (275.61,241.36) -- (280.17,241.36) -- (280.17,269.55) -- (275.61,269.55) -- cycle ;

%Straight Lines [id:da42113228694138805] 
\draw    (407.82,283.09) -- (407.63,315.12) ;
%Image [id:dp607326012343921] 
\draw (196.02,274.11) node  {\includegraphics[width=8.07pt,height=8.67pt]{hot.png}};
%Rounded Same Side Corner Rect [id:dp5003048436471262] 
\draw   (398.27,302.28) .. controls (400.87,302.27) and (402.99,304.36) .. (403,306.96) -- (403.07,321.08) .. controls (403.08,323.68) and (400.99,325.8) .. (398.39,325.81) -- (350.43,326.04) .. controls (350.43,326.04) and (350.43,326.04) .. (350.43,326.04) -- (350.32,302.5) .. controls (350.32,302.5) and (350.32,302.5) .. (350.32,302.5) -- cycle ;

%Straight Lines [id:da47487605638440966] 
\draw    (403.13,315.12) -- (407.63,315.12) ;
%Straight Lines [id:da261104543267761] 
\draw    (329.82,404.98) -- (329.82,323.3) ;
%Straight Lines [id:da6689212280241392] 
\draw    (329.82,323.3) -- (349.06,323.79) ;
\draw [shift={(351.06,323.84)}, rotate = 181.46] [color={rgb, 255:red, 0; green, 0; blue, 0 }  ][line width=0.75]    (10.93,-3.29) .. controls (6.95,-1.4) and (3.31,-0.3) .. (0,0) .. controls (3.31,0.3) and (6.95,1.4) .. (10.93,3.29)   ;
%Straight Lines [id:da5045718990414447] 
\draw    (329.34,306.67) -- (347.91,307.04) ;
\draw [shift={(349.91,307.08)}, rotate = 181.13] [color={rgb, 255:red, 0; green, 0; blue, 0 }  ][line width=0.75]    (10.93,-3.29) .. controls (6.95,-1.4) and (3.31,-0.3) .. (0,0) .. controls (3.31,0.3) and (6.95,1.4) .. (10.93,3.29)   ;
%Straight Lines [id:da26468817401398836] 
\draw    (328.67,315.19) -- (347.91,315.68) ;
\draw [shift={(349.91,315.73)}, rotate = 181.46] [color={rgb, 255:red, 0; green, 0; blue, 0 }  ][line width=0.75]    (10.93,-3.29) .. controls (6.95,-1.4) and (3.31,-0.3) .. (0,0) .. controls (3.31,0.3) and (6.95,1.4) .. (10.93,3.29)   ;
%Straight Lines [id:da03502375762013976] 
\draw    (269.56,314.65) -- (328.67,315.19) ;
%Straight Lines [id:da34232013215568435] 
\draw    (226.22,334.66) -- (269.65,334.98) ;
%Straight Lines [id:da04876722940087985] 
\draw    (269.56,314.65) -- (269.65,334.98) ;
%Straight Lines [id:da457846470626037] 
\draw    (329.72,257.99) -- (329.34,306.67) ;
%Straight Lines [id:da7463203294281131] 
\draw    (329.72,257.99) -- (307.91,258.17) ;

% Text Node
\draw (188.73,248.54) node [anchor=north west][inner sep=0.75pt]   [align=left] {Encoder};
% Text Node
\draw (286.54,249.93) node [anchor=north west][inner sep=0.75pt]    {$\sum $};
% Text Node
\draw (318.78,190.26) node [anchor=north west][inner sep=0.75pt]   [align=left] {Encoder};
% Text Node
\draw (286.67,221.55) node [anchor=north west][inner sep=0.75pt]   [align=left] {{\tiny [K,T,C]}};
% Text Node
\draw (57.61,279.42) node [anchor=north west][inner sep=0.75pt]   [align=left] {{\tiny [K,T,C]}};
% Text Node
\draw (55.66,355.69) node [anchor=north west][inner sep=0.75pt]   [align=left] {{\tiny [K,T]}};
% Text Node
\draw (536.39,221.14) node [anchor=north west][inner sep=0.75pt]   [align=left] {{\tiny [K,T]}};
% Text Node
\draw (94.71,244.83) node [anchor=north west][inner sep=0.75pt]    {$\widetilde{\boldsymbol{s}_{d}}\rightarrow \widehat{\boldsymbol{r}_{d}}$};
% Text Node
\draw (204.82,326.84) node [anchor=north west][inner sep=0.75pt]    {$\sum $};
% Text Node
\draw (351.64,303.82) node [anchor=north west][inner sep=0.75pt]  [rotate=-359.79] [align=left] {{\footnotesize arbitrator}};
% Text Node
\draw (40.81,395.88) node [anchor=north west][inner sep=0.75pt]    {$\theta $};
% Text Node
% \draw (98.23,385.62) node [anchor=north west][inner sep=0.75pt]  [rotate=-89.74] [align=left] {{\scriptsize [0..010]}};
% Text Node
% \draw (48.61,394.27) node [anchor=north west][inner sep=0.75pt]    {$\theta \rightarrow $};
% Text Node
\draw (85.58,321.29) node [anchor=north west][inner sep=0.75pt]   [align=left] {SC Encoder};
% Text Node
\draw (434.1,190.39) node [anchor=north west][inner sep=0.75pt]   [align=left] {Decoder};


\end{tikzpicture}
\caption{The multichannel \ac{TSE} algorithm is described, comparing three alternative enrollment features: (1) the instantaneous \ac{RTF} extracted from clean speech utterances impinging on the array from the same position as the desired source, (2) a pre-trained single-channel model that generates a spectral embedding, and (3) the true \ac{DOA} of the desired source mapped to a learned representation. A separate model is trained for each enrollment feature.}
\label{fig:block_diagram}
\end{figure*}

\section{Proposed Model}
In this section, we present the proposed \ac{TSE} architecture, utilizing various types of enrollment features.
\subsection{Architecture}
\label{subsec:Architecture}


The proposed framework consists of a multi-channel encoder, a decoder, and an additional encoder for each enrollment feature. The multi-channel encoder processes the mixture signal, while the enrollment encoder extracts feature-specific embeddings. The decoder then utilizes both encoders to separate the target speaker.
Unlike \cite{eisenberg2023two}, Siamese encoders are not implemented in this study, as the input to both encoders has a different nature, as will be elaborated in the sequel.

The mixture encoder design employs multiple convolution layers, a two-dimensional batch normalization layer, and a `ReLU' activation function. Subsequently, the channel and frequency dimensions are merged, and a \ac{FC} layer is used to reduce dimensionality. A single self-attention layer is then applied. Finally, one of the alternative reference embedding vectors, described in the following paragraphs, is multiplied with each vector along the frame dimension of the mixture embedding on a frame-by-frame basis.

The decoder architecture consists of six self-attention layers and a \ac{FC} layer to restore the original dimension. To enable the use of skip connections as needed, transpose-convolution layers are employed to adapt to the convolution layers in the encoder. Finally, a self-attention layer is applied. 

A block diagram of the entire model is shown in Fig.~\ref{fig:block_diagram}.

\subsection{Alternative Enrollment Information}

In this study, we employed the \ac{RI} components of the \ac{STFT} as the features of the mixture signal. As for the reference features, we explored various alternatives. Two primary approaches emerged: utilizing prior spatial information or leveraging the speech attributes of our target speaker. We utilized either the instantaneous \ac{RTF} or the oracle \ac{DOA} in the former option. In the latter approach, we utilized the embedding of the \ac{SC} model, which does not involve spatial information.

 


\subsubsection{{RTF} Features}
While \ac{ATF} estimation is a blind problem, the \ac{RTF} estimation is a non-blind problem. The \ac{RTF}, proposed in \cite{gannot2001signal}, is widely used for beamforming and localization of sound sources.
The \ac{RTF} of the $q$-th source w.r.t.~microphone $m\in\{1,\ldots,J\}$ is defined as:
\begin{equation}
    \mathbf{r}_q(k) = \frac{1}{h_{m q}(k)} \hq. 
\end{equation}
The \ac{RTF} encodes the spatial information and will be utilized as an auxiliary feature for the proposed model.
%
% Since the \ac{RTF} encodes the \ac{IPD}, it is assumed that the spatial information will dominate over the spectral information.} 
%
%The \ac{RTF} of a multi-channel acoustic transfer function indicates the relationship between the microphones. It is calculated by dividing the transfer function with respect to a reference microphone, typically the first one. 
%
In this study, assuming $\tilde{\mathbf{s}}_d^{\text{ref}}(t,k)$ is noiseless, we employ the instantaneous estimate of the \ac{RTF} \cite{chazan2019multi} as the ratio of the \ac{STFT} of the microphone signals:
%
\begin{equation}
\hat{\mathbf{r}}_d(t,k) = \frac{\tilde{\mathbf{s}}_d^{\text{ref}}(t,k)}{\tilde{s}_d^{\text{ref},m}(t,k)},
\end{equation}
%
where $\tilde{s}_d^{\text{ref},m}(t,k)$ represents the enrollment signal captured by the $m$-th microphone, an arbitrarily selected reference microphone. In our simulation study, we use the enrollment of the desired speaker, but in practice, any signal from the desired location can be utilized.
%Assuming the \ac{RTF} is time-invariant, its estimate can improved by averaging \noteb{the numerator and denominator separately} over the frames dimension of $\hat{\mathbf{r}}_d(t,k)$. 
The encoder architecture for the instantaneous \ac{RTF} features parallels that of the mixture encoder described earlier. The encoder’s output, which is an embedding of the input features, is averaged across the frame dimension to yield a single representation vector, guiding the model toward the desired speaker.

\subsubsection{DOA Features}
As an alternative to the instantaneous \ac{RTF}, we provided the \ac{TSE} framework with the oracle \ac{DOA} of the desired speaker. It is important to note that, in real-world scenarios,  the \ac{DOA} must be estimated. In environments with high reverberation levels, accurate estimation becomes more challenging, which could impact the performance of the extraction model. Unlike the \ac{RI} features of the \ac{STFT} and \ac{RTF}, which share the same dimensions, the \ac{DOA} is an integer. Consequently, we modified the encoder architecture to allow it to learn the \ac{DOA} representation.

Using a lookup table, we introduce an embedding vector that is learned for each \ac{DOA}. Each \ac{DOA} is used to select a corresponding row in the table. This vector is then passed through a self-attention layer. Finally, the embedding vector is used in the bottleneck, as described in \ref{subsec:Architecture}. 


\subsubsection{Spectral Features}

In addition to the previously discussed design options, we explored a scenario where spatial information about the desired speaker is unavailable, and the speaker's voice signature is provided instead. To achieve this, we employed the initial stage of the \ac{SC} model in \cite{eisenberg2023two} with the $m$-th microphone signal.
%the first channel of the multi-channel signal for the \ac{iRTF} estimation as the enrollment signal. 
We then use the embedding vector derived from the enrollment signal as it was used with the \ac{DOA} and \ac{RTF} features. This approach shifted the model’s attention from the spatial information of the desired speaker to their speech characteristics.



\subsection{Objective function}

To train the proposed model for the extraction task, we used the time-domain \ac{SI-SDR} loss function, which is known to be effective in \ac{BSS} tasks. %The loss function is defined as follows:
%
% \begin{equation}
% \text{SI-SDR}\left( \mathbf{s},\hat{\mathbf{s}} \right)= 10 \log_{10} \left( \frac{\norm{\frac{\langle {\hat{\mathbf{s}},\mathbf{s}} \rangle}{\langle {\mathbf{s},\mathbf{s}} \rangle} \mathbf{s}}^2}{\norm{\frac{\langle {\hat{\mathbf{s}},\mathbf{s}} \rangle}{\langle {\mathbf{s},\mathbf{s}} \rangle} \mathbf{s}-\hat{\mathbf{s}}}^2} \right)
% \end{equation}
% %
% Here, $\langle {,} \rangle$ represents the inner product, $\mathbf{s}$ denotes the target speaker in the time domain, and $\hat{\mathbf{s}}$ represents the estimated speaker.

To further enhance the training process, we employed a technique where, for each training sample, we used the same mixture and swapped the desired and interference signals. The corresponding enrollment signal was then used for each of the extracted sources. The two losses were subsequently averaged, resulting in the following expression:
%
\begin{equation}
L_{\text{SI-SDR}} =   \frac{1}{Q}\sum_{q=1}^{Q}\text{SI-SDR}\left(\tilde{s}_q,\hat{\tilde{s}}_q \right) 
\label{eq:loss_sisdr}
\end{equation}



\section{Experimental Study}

\subsection{Datasets}
We evaluate the proposed approach on a simulated dataset. It consists of 40000 utterances for the training set, 5000 for
the validation set, and 300 for the test set. Each example is created by randomly choosing two utterances from the LibriSpeech database and convolving
the signals with four channel \ac{RIR}
simulated by the Image Method \cite{habets2006room}. 

The simulations were conducted in a room with dimensions uniformly distributed in $U[3,10]$~m and a reverberation time of $U[0.2,0.8]$~sec. Microphones were spaced 8~cm apart and placed at least 0.7~m away from the walls. Source positions were uniformly distributed in $U[0^\circ,180^\circ]$, with distances from the microphones in the range $U[1,4]$~m.

Furthermore, to enhance the model's spatial resolution, we introduced directional noise into the mixture with SNR drawn in the range $U[-5,20] $~dB. The noise source was extracted from the audiolabs' dataset\footnote{https://github.com/audiolabs/anechoic-noise} and convolved with  4-channels \ac{RIR} originating from the same room as the two primary speakers. The model’s task becomes more challenging when the noise exhibits directional characteristics because relying on spatial references to identify the desired speaker is complicated by the presence of another directive source. To simulate sensor noise, we introduced pink noise at an \ac{SNR} of 20~dB into the mixture, in addition to the directional noise.

For the mixture simulation, we randomly selected two signals from each speaker, designating one as the desired signal and the other as the enrollment. Both signals from each speaker underwent convolution with the same \ac{RIR}. This choice is based on the assumption that there are brief intervals where only the \noteb{noiseless} desired speaker is present, making it suitable for use as an enrollment signal. Both signals must bear the same spatial information to ensure spatial consistency between the desired source and the corresponding enrollment. The utterances of the desired and interference sources are summed together with the reverberant noise and sensor noise to form the mixture signal.
%Table~\ref{table:reverb_parameters}.  provides an overview of the data parameters.



% \begin{table}[htbp]
% \addtolength{\belowcaptionskip}{6pt}
% \caption{Noisy reverberant data specification.}
% \label{table:noisy_data}
% \centering
% \resizebox{0.8\columnwidth}{!}{
% \begin{tabular}{@{}lll@{}}
% \toprule
% SNR[dB]                 & & $U[-5,20]$ \\ \midrule
% Sensors SNR[dB]           &      & $20$ \\ \midrule
%                         &          & $U[3,10]$                                 \\
% Room dim.~[m]                &         & $U[3,10]$                                  \\
%                         &         & $1.5$                                               \\ \midrule
% Reverb. time~[sec]             &          & $U[0.2,0.8]$                            \\ \midrule
% %                        &         & $\frac{H_x}{2}+{U}[-0.5,0.5]$ \\
% %Mic. Pos. [m]           &         & $\frac{H_y}{2}+{U}[-0.5,0.5]$ \\
% %                        &        & 1.5                                         \\ \midrule
% Distance between mics[m]    & & $0.08$ \\ \midrule

% Min distance from walls[m]   & & $0.7$ \\ \midrule

% Sources Pos. [$^\circ$] &  & \emph{U}{[}0,180{]}                                \\ \midrule
% Sources Distance from mics [m]                   &          & ${U}[1,4]$                        %\\ \midrule
% % SNR [dB]                    &        & \emph{U}{[}10, 25{]}       
%             \\ \bottomrule
% \end{tabular}}
% % \vspace{-0.3cm}
% \label{table:reverb_parameters}
% \end{table}


\subsection{Algorithm Settings}
The speech and noise signals were drawn from the database and downsampled to 8~[KHz]. The frame size of the \ac{STFT} is set to 256 samples with a $50\%$ overlap. Only the first 129 frequency bins are processed due to the symmetry of the \ac{DFT}.

For the training procedure, we employed the Adam optimizer~\cite{kingma2015adam} with a learning rate of 0.001 and a training batch size of 14. The weights are initialized randomly, and the signal lengths are varied randomly for each batch.


\subsection{Evaluation Measures}
To assess the effectiveness of the proposed algorithm, we employ two evaluation metrics:  \ac{SI-SDR} and  \ac{STOI} \cite{Taal2011STOI}. The former indicates the efficiency of speaker separation, while the latter indicates audio intelligibility.



\subsection{Compared Methods}
We compared our proposed methods with our prior work \cite{eisenberg2023two}, using only a single-channel input. Additionally, we compared the proposed methods with two variants of the \ac{MVDR} beamformer. The \ac{MVDR} beamformer aims to estimate the desired signal with minimal distortion while simultaneously minimizing noise. Denoting $\mathbf{g}_d(k)$ as the \ac{MVDR} beamformer at frequency $k$, the estimated signal is given by: 
\begin{equation}
    \hat{s}_d(k,t) = \mathbf{g}^{\text{H}}_d(k) \cdot \mathbf{x}(t,k),
\end{equation}
where $^\textrm{H}$ stands for the Hermitian operator. 
The \ac{MVDR} weights are given by:
\begin{equation}
    \label{eq:mvdr}
    \mathbf{g}_d(k) = \frac{ \mathbf{Q}^{-1}(k)\mathbf{r}^{\text{H}}_d(k)}{\mathbf{r}^{\text{H}}_d(k) \mathbf{Q}^{-1}(k) \mathbf{r}_d(k)},
\end{equation}
where $\mathbf{Q}(k)$ is the spatial covariance matrix of the noise and the interference.
The \ac{RTF}-based \ac{MVDR} aims at estimating the reverberant desired signal, as captured by a reference microphone. 
Given a directional noise and an additional (directional) speaker, we expect $\mathbf{Q}(k)$ to exhibit two main eigenvalues. Unlike the \ac{LCMV} beamformer, the \ac{MVDR} design does not include a constraint to place a null toward the interfering source. Instead, interference suppression is handled implicitly as part of the noise minimization.

We implemented two variants of the \ac{MVDR} beamformer. In the first variant, we assume the availability of a two-second interval containing only the desired speaker and noise signals, as well as an additional two-second segment containing only the noise signals (i.e., the interference is inactive). The \ac{RTF} of the desired speaker is then estimated using the covariance-whitening procedure \cite{Markovich-Golan2018Performance}. 
%
To estimate the spatial covariance matrix $\mathbf{Q}(k)$, we assume access to a segment containing interference plus the noises. The estimate is calculated as:
\begin{equation}
    \hat{\mathbf{Q}}(k) = \frac{1}{T}\sum_{t=0}^{T-1} \tilde{\mathbf{a}}(k,t) \cdot  \tilde{\mathbf{a}}^{\text{H}}(k,t)
\end{equation}
where $ \tilde{\mathbf{a}}(k,t) = \tilde{\mathbf{s}}_i(t,k) +  n(t,k) \cdot \hn + \mathbf{v}(t,k)$. This procedure constitutes the estimated \ac{MVDR} beamformer.



In the second variant, we assume that noiseless enrollments are available. Hence,  the spatial covariance matrix of the noise can be substituted with the identity matrix. This procedure constitutes the Oracle \ac{MVDR} beamformer.




\subsection{Results}
The \ac{SI-SDR} and \ac{STOI} results for all methods evaluated on the dataset described above are shown in the middle part of Table~\ref{table:results}. All three proposed variants outperform the single-channel model and the estimated \ac{MVDR}. Among these variants, the algorithm using the instantaneous \ac{RTF} as a feature vector achieves the best performance. A spectrogram example of the proposed \ac{RTF}-based model is presented in Fig.~\ref{fig:sonograms}.

\begin{figure}[htbp]
\center
{\includegraphics[width=0.48\textwidth, height=0.5\textwidth]{sonograms_diff_doa.png}}
\caption{Spectrogram example of the Proposed-\ac{RTF} model compared with the noisy and ground-truth spectrogram.}
\label{fig:sonograms}
\end{figure}

To further evaluate our model, we simulated a scenario where both speakers have the same \ac{DOA} but differ in their distances from the array. Specifically, we generated data where two speakers are aligned with the array’s center, while directional noise originates from a different direction. In this setup, a reference-based \ac{DOA} model would fail to distinguish the speakers due to their shared \ac{DOA}. Conversely, an \ac{RTF}-based approach can differentiate the sources, as it captures spatial information derived from multiple arrivals caused by room reverberation and signal propagation. We therefore assess the performance of both the Proposed-\ac{RTF} variant and the \ac{RTF}-based \ac{MVDR}.

The numerical results are in the bottom of Table~\ref{table:results}. The results show that the oracle \ac{MVDR} outperforms the proposed method in the intelligibility score, but our model excels when considering separation capabilities. The separation results are on par with those obtained for the case of different \acp{DOA}, which is quite remarkable.



\begin{table}[htbp]
\caption{Results for random locations (middle) and same \ac{DOA} (bottom).}
\begin{center}
\begin{tabular}{@{}lcccccc@{}}
\toprule
  Model & SI-SDR~[dB] $\uparrow$ &  STOI $\uparrow$     \\
\midrule
%\hline
Unprocessed   & -2.6 & 0.54 \\ 


\midrule\midrule
Oracle \ac{MVDR}   & 9.7 &  0.85  \\
\midrule
Single channel   & 6.21 & 0.73 \\

Estimated \ac{MVDR} & 6.3 & 0.79 \\

TSE-RTF   & \textbf{9.2} & \textbf{0.81}  \\

TSE-DOA   & 8.4 & 0.8  \\

TSE-Spectral   & 8.18 &  0.8 \\


  \midrule\midrule

%\hline
Oracle \ac{MVDR}   & 8.5 & \textbf{0.82} \\ 
% \hline
TSE-\ac{RTF}   & \textbf{8.8} & 0.79 \\


\bottomrule
\end{tabular}
\end{center}
\label{table:results}
\end{table}





%\begin{table}[htbp]
% \caption{Results same DOA}
% \begin{center}
% \begin{tabular}{@{}lcccccc@{}}
% \toprule
% \toprule
%   Model & SI-SDR$\uparrow$ &  STOI$\uparrow$     \\
%   \midrule
%   \midrule
% %\hline
% oracle \ac{MVDR}   & 8.5 & \textbf{0.82} \\ 
% % \hline
% Proposed \ac{RTF}   & \textbf{8.8} & 0.79 \\
% \bottomrule
% \bottomrule
% \end{tabular}
% \end{center}
% \label{table:results_same_doa}
% \end{table}

\section{Conclusions}
This study introduces a model for \acf{TSE} using a noiseless enrollment signal by utilizing the \ac{RTF} associated with the desired speaker. We evaluate the effectiveness of different features, specifically the known \ac{DOA} and spectral characteristics. Additionally, we present a comparison with the \ac{MVDR} beamformer. Our results demonstrate the benefits of leveraging the \ac{RTF}, even when both speakers originate from the same direction.


 \balance
\bibliographystyle{IEEEtran}
\bibliography{bib_short,main}




\end{document}
