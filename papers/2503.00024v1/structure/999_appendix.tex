
\section{Pre-annotation and classifiers}\label{app:pre}
\paragraph{Pre-annotation}
We start with the \textbf{Second Reading debates of Bills},\footnote{\url{https://www.parliament.uk/about/how/laws/passage-bill/commons/coms-commons-second-reading/}} where the members debate the main principles of a certain Bill. The advantages of using such debates are: (i) the stance of an argument can be easily identified based on whether they support %for 
the Bills; (ii) debates can be paired with brief Bill introductions,\footnote{e.g., the `long title' on page \url{https://bills.parliament.uk/bills/3858}} providing clear argument topics; and (iii) the arguments 
focus on Bill principles, with fewer discussions on specific amendments and clauses, which require less contextual awareness than other Bill debates like the ones for the Committee Stage.\footnote{\url{https://www.parliament.uk/about/how/laws/passage-bill/commons/coms-commons-comittee-stage/}}
We choose five Bills, including topics relevant to animal welfare and parental leave (see Table \ref{tab:bill} for the Bill introductions), 
which may be easier to annotate and more likely to have emotional arguments.

Three annotators label 245 texts from these debates for \textbf{three layers}: (\emph{L1}) 
whether the text evokes emotions, (\emph{L2}) whether the text contains standalone arguments, and (\emph{L3}) the stance of the text toward the Bill. \emph{L1} and \emph{L2} are labeled `0' (for answer `no') or `1' (for `yes'). If \emph{L2} is labeled `1', annotators proceed to label \emph{L3}, which has four options: `0' for support, `1' for opposition, `2' for inability to identify stance without additional context, and `3' for a neutral stance suggesting additional amendments or policies. Besides, 40 texts from the pilot annotation are also annotated for \emph{L1} and \emph{L2}.  
To potentially speed up the annotation process, the 285 texts are selected from those judged as both emotional and argumentative by GPT4o. Here, we prompt GPT4o with simple questions such as \emph{Does this text try to convince readers something?} and \emph{Is this text emotional?'}.

40 of the outputs are jointly labeled by all annotators, achieving average Cohen's Kappa of 0.622 for \emph{L1}, 0.674 for \emph{L2}, and 0.762 for \emph{L3} across annotator pairs. 
As shown in the `Question' column of Table \ref{tab:pre}, GPT4o already achieves a high precision of 0.82 in detecting argumentative texts using simple prompts. However, its precision for emotional text classification is still low (0.53).

We then convert the annotations for \emph{L3} to \emph{L3$^{*}$}, where we pair argument pairs based on their topics and stances. The categories include: `different topic' for pairs with different topics (from different Bills), `different stance' for pairs with the same topic but different stances, and `same' for pairs with the same topic and stance.

The number of texts annotated for each layer and the corresponding label distribution\se{s} are summarized in Table \ref{tab:pre} (left). 

\paragraph{Automatic Pipeline}
We develop three classifiers based on GPT4o 
to automatically identify the argument pairs needed. The pipeline is as follows: 
\begin{enumerate}[]
    \item \textbf{Argumentative text classification}: our goal is to have a \textbf{high precision} classifier since we have sufficient candidate texts. We find that when we ask GPT-4o to provide the major claim, evidence, and reasoning connecting the evidence to the major claim in the text, its precision increases from 0.82 to 0.96, as shown in the `Argumentative' row of Table \ref{tab:pre}. 
    
    We then retain texts judged as argumentative for \hansard{} using this prompt, while for \deuparl{}, we use a German translation of the same prompt. The overall performance of GPT4o on German data is assessed after completing the stance agreement classification task (see below).

    \item \textbf{Stance agreement classification}: 
    To enable the flexible selection of classifiers with specific performance characteristics (e.g., high recall, high precision), we introduce a parameter into the prompt, with its threshold optimized to achieve different specialized performance levels.
    To do so, we ask GPT4o to rate the likelihood that two given arguments address the same topic and share the same stance on a Likert scale from 0 to 100. We randomly sample 600 argument pairs (with a 2:1:1 ratio for the three categories of \emph{L3$^{*}$}) from the dataset, ‘optimize’ the threshold of ratings for the `same’ category 
    using argument pairs from two Bills, and test the performance on the remaining three Bills to prevent data leakage. We evaluate all possible combinations of Bills for the training and test sets.
    We observe that as the threshold increases, precision on the `same’ category ($P_{same}$) consistently improves, while macro F1 begins to decrease beyond certain thresholds. With a threshold of 100, $P_{same}$ reaches 0.92, but F1 is very low at 0.45. Therefore, we select a threshold of 90 as a more balanced trade-off, achieving $P_{same}=0.81$ and $\textit{F1}=0.76$, to obtain more candidates that are still highly likely to be true positives. 
    
    For \hansard{}, we retain the argument pairs labeled as belonging to the `same' category using this threshold. For \deuparl{}, we apply the German translation of the prompt with the same threshold to identify argument pairs. One annotator evaluates 50 candidates from the outputs of steps 1 and 2: no argument is labeled as non-argumentative, while 12 argument pairs are identified as false positives in the stance agreement task, yielding $P_{same}=0.76$. This value is only 4 percent points lower than the result on English data. Consequently, we retain these prompt settings for the German data.
    
    \item  \textbf{Emotional text classification}: we aim for a \textbf{balanced} classifier because we also need non-emotional arguments. Since this is a subjective task, we ask GPT4o to rate how likely it can feel the emotions 
    in the texts on a \se{L}ikert scale of 0-100, and then `optimize' the threshold of the rates for the `emotional' category on 70\% of the data and check how it performs on the remaining 30\%. Overall, with this step, we can improve the macro F1 to 0.74-0.81 (averaged over three rounds of data splitting), depending on the gold from different annotators. The best threshold for two annotators is 75, while that for the other is 85, so we use the threshold 75 to represent the majority, which has a macro F1 of 0.75, averaged across the three annotators. 

    We use this threshold to select the argument pairs for \hansard{}. For \deuparl{}, we further optimize the threshold using a small-scale set of human annotations and adjust it to 85. This setting is then used to label the binary emotions of arguments. 

\end{enumerate}

\begin{table}[!ht]
\resizebox{\linewidth}{!}{%
\begin{tabular}{@{}lcccc@{}}
\toprule
                               & \multicolumn{2}{c}{Pre-Annotation} & \multicolumn{2}{c}{Automatic Pipeline}   
                               \\
                               & \#                 & \%  & Question  & `Optimized' \\ \midrule
\multicolumn{3}{l}{\emph{L1 - emotion} }                                             \\ \midrule
Emotional                      & 151                & 53.0 & 0.53 (P)  & \multirow{2}{*}{0.75 (F1)} \\ 
Non-emotional                  & 134                & 47.0 & -  \\ \bottomrule
\multicolumn{3}{l}{\emph{L2 - argument}}                                         \\ \midrule
Argumentative                  & 234                & 82.1 & 0.82 (P) & 0.96 (P)  \\
Non-argumentative              & 51                 & 17.9 & - & - \\ \bottomrule
\multicolumn{3}{l}{\emph{L3 - stance} }                                           \\ \midrule    
Support                        & 170                & 72.6 & - \\
Opposition                     & 2                  & 0.9 & -  \\
Neutral                        & 29                 & 12.4  & -\\
Irrelevant                     & 16                 & 14.1& - \\ \midrule
\multicolumn{3}{l}{\emph{L3$^{*}$ - pair stance}}                                            \\ \midrule   
Same           & 2,905            & 8.9  & -   & \multirow{3}{*}{\makecell{0.80 ($P_{same}$) \\ 0.75 (F1)}} \\
Different stance & 3,325              & 10.2 & - \\
Different topic                & 26,486             & 81.0 & - \\ \midrule
Total                          & 32,716             & 100  & - \\ \bottomrule
\end{tabular}}
\caption{Number of texts annotated for each layer and category (\#) and the corresponding label distribution (\%). Performance of GPT4o on the binary emotion classification, argument identification, and stance agreement detection tasks used for automatically identifying the target argument pairs.}\label{tab:pre}
\end{table}


\begin{table*}[!ht]
\resizebox{\linewidth}{!}{
\begin{tabular}{@{}l@{}}
\toprule
\emph{Introduction}                                                                                                                                                                         \\ \midrule
A Bill to Prohibit the export of certain livestock from Great Britain for slaughter.                                                                                                 \\ \midrule
\makecell[l]{A Bill to create offences of dog abduction and cat abduction and to confer a power to make corresponding provision  \\ relating to the abduction of other animals commonly kept as pets.} \\ \midrule
A Bill to make provision about leave and pay for employees with responsibility for children receiving neonatal care.                                                                   \\ \midrule
A Bill to prohibit the import and export of shark fins and to make provision relating to the removal of fins from sharks.                                                            \\ \midrule
A Bill to prohibit the sale and advertising of activities abroad which involve low standards of welfare for animals.                                                                 \\ \bottomrule
\end{tabular}}
\caption{The introductions of the five Bills selected in \protect\bill{}.}\label{tab:bill}
\end{table*}


% Please add the following required packages to your document preamble:
% \usepackage{booktabs}
\begin{table*}[]
\resizebox{\linewidth}{!}{
\begin{tabular}{@{}l|l@{}}
\toprule
English                                                                                                                                                                                  & German                                                                                                                                                                                     \\ \midrule
\makecell[l]{iran, integrat, ukraine, russia, asylum,\\ deportation, israel, gaza, expulsion, \\ displacement, migration, migrant, \\immigrant, refugee, palestine,invasion,\\ repatriation, hamas, hisbollah} & \makecell[l]{ukraine, russland, migrant, \\ immigrant, flüchtling, asyl,\\ gaza, iran, palästina, \\israel, krieg, invasion, \\sanktionen, waffenlieferungen, friedensverhandlungen, \\kriegsverbrechen, flüchtlingskrise, nato,\\ energieversorgung, vertreibung, migrationspolitik,\\ asylverfahren, grenzsicherung, integration, \\abschiebung, aufenthaltsgenehmigung, menschenhandel, \\seenotrettung, rückführung, schutzstatus, \\waffenstillstand, raketenangriffe, besatzung, \\zwei-staaten-lösung, friedensprozess, intifada, \\ hamas, hisbollah, menschenrechte, un-resolution
} \\ \bottomrule
\end{tabular}
}
\caption{Keywords used to filter debates for \hansard{} and \deuparl{}.}\label{tab:keywords}
\end{table*}



% Please add the following required packages to your document preamble:
% \usepackage{booktabs}
\begin{table*}[]
\centering
%\resizebox{!}{\linewidth}{
\begin{tabularx}{\linewidth}{X}
\toprule
\emph{Remove Emotion Prompt}  \\ \midrule
====\textbf{System Prompt}=====\\ I will give you an argumentative text that **can** appeal to emotion.    \\ \\ Your task is to generate an argument with the same stance for the same topic **without emotional language**, by rephrasing the text but maintaining a similar style and length. \\ \\ Briefly explain why the rewritten argument no longer evokes emotions.\\ \\ Answer in the following way:\\ Generated argument: \\ Explanation:\\ ====\textbf{User Prompt}=====\\ Text: \{original argument\}  \\ \midrule
\emph{Add Emotion Prompt}  \\ \midrule
====\textbf{System Prompt}=====\\ I will give you an argumentative text that **cannot** appeal to emotion.\\     \\ Your task is to generate an argument with the same stance on the same topic **with emotions**, by rephrasing the text but maintaining a similar style and length. \\ \\ Briefly explain why the rewritten argument can evoke emotions now.\\ \\ Answer in the following way:\\ Generated argument: \\ Explanation:\\ ====\textbf{User Prompt}=====\\ Text: \{original argument\}                \\ \bottomrule
\end{tabularx}
%}
\caption{Prompts used to remove/add emotions for synthetic arguments.}\label{tab:prompt_synthetic}
\end{table*}


\section{Arguments from others}\label{app:other}
\paragraph{\dagstuhl{}} 
\citet{wachsmuth-etal-2017-computational} collected human ratings on a Likert scale of 1–3 for multiple dimensions of argument quality, including argument effectiveness (convincingness)\footnote{“Argumentation is effective if it persuades the target audience of (or corroborates agreement with) the author’s stance on the issue.” — \citet{wachsmuth-etal-2017-computational}} and emotional appeal. These ratings were applied to 304 argumentative texts from \citet{habernal-gurevych-2016-argument}, which were sourced from a textual debate portal in \textbf{English}. We retain only those arguments whose average convincingness rating (across the three annotators) exceeds 1.5. 
Next, we pair arguments that share the same stance on the same topics and calculate the absolute differences in their emotional appeal ratings. From these pairs, we randomly select 10 topics and then retain the 5 argument pairs with the largest absolute differences in emotional appeal for each topic.

\paragraph{\lynn{}}
\citet{greschner2024fearfulfalconsangryllamas} collected
discrete emotion labels from a reader respective (e.g. joy, disgust etc.) for 300 \textbf{German} arguments associated with 30 statements, drawn from \citet{velutharambath_wuehrl_klinger_2024a}. Each argument was annotated by three annotators. We interpret the number of annotations marking the argument as containing specific emotions (rather than `no emotion') as its emotion score. E.g., if three annotators identify specific emotions in the argument, its emotion score would be 3. Using a procedure similar to the one employed for \dagstuhl{}, we pair arguments referencing the same statement, randomly select 25 statements, and then retain the 
two argument pairs per statement that exhibit the greatest differences in emotion scores. 


\begin{table}[]
\centering
\resizebox{\linewidth}{!}{
\begin{tabular}{@{}llllll@{}}
\toprule
     & \dagstuhl{} & \bill{} & \hansard{} & \lynn{} & \deuparl{} \\ \midrule
\multicolumn{6}{l}{\emph{Increase}}                               \\ \midrule
\rz{}   & \textbf{-0.06}   & \textbf{0.15}         & \textbf{0.05}   & \textbf{-0.38}  & \textbf{0.32}   \\
\rth{} & -0.18    & -0.21        & -0.31  & -0.46   & -0.38  \\ \midrule
\multicolumn{6}{l}{\emph{Decrease}    }                        \\ \midrule
\ro{} & -0.12 & -0.21 & -0.03 & 0.08 & -0.19    \\
\rtw{} & \textbf{0.36}    & \textbf{0.27}         & \textbf{0.29}   & \textbf{0.76}   & \textbf{0.25} \\ \bottomrule
\end{tabular}}
\caption{BWS scores for the 4 argument groups: \rz{}, \ro{}, \rtw{} and \rth{}, derived from the majority votes of the annotation for pairwise comparisons of emotional intensity. `Increase'/`Decrease' denotes the direction to increase/decrease the perceived emotional intensity.}\label{tab:bws}
\vspace{-.3cm}
\end{table}

\section{Prompts}\label{app:prompt}
Table \ref{tab:prompt_synthetic} presents the prompts used to introduce/remove emotions. Table \ref{tab:promptc} illustrates the prompts used for evaluating argument convincingness.

% Please add the following required packages to your document preamble:
% \usepackage{booktabs}
\begin{table*}
\footnotesize
\resizebox{\linewidth}{!}{
\begin{tabular}{@{}cl@{}}
\toprule
\multicolumn{2}{l}{Prompt Template}                                                                                                                                                                                                                                                                                                                                                                                                                                                                                                                                                                                                                                                                                     \\ \midrule
Shared & \begin{tabular}[c]{@{}l@{}}Below, you will find one pair of argumentative texts discussing the same topic with the same stance. The topic may be a binary \\choice, a bill from UK parliamentary debates, or a simple statement. Both arguments either support or oppose the topic, or they \\favor one side if the topic involves a binary choice.\\ \\ Your task is to evaluate each pair to determine **which argumentative text you find more convincing**. There are three label options:\\ 0 (Both arguments are equally convincing.)\\ 1 (Argument 1 is more convincing.)\\ 2 (Argument 2 is more convincing.)\\ \\ **Note**: Truncated sentences or grammatical errors should be **ignored**.\end{tabular} \\ \midrule
1      & \begin{tabular}[c]{@{}l@{}}Please answer your label option **without** any explanations.\\ \\ \{text\}\end{tabular}                                                                                                                                                                                                                                                                                                                                                                                                                                                                                                                                                                                            \\ \midrule
2      & \begin{tabular}[c]{@{}l@{}}Please answer your label option and briefly explain why you choose this label.\\ \\ \{text\}\\ \\ Below is an example answer for you; please follow this format in your response.\\ Label: 2\\ Explanation: because Argument 2 provides more statistics supporting the claim, while Argument 1 contains logical fallacies.\end{tabular}                                                                                                                                                                                                                                                                                                                                             \\ \midrule
3      & \begin{tabular}[c]{@{}l@{}}Please answer your label option and briefly explain why you choose this label.\\ \\ \{text\}\\ \\Below is an example answer for you; please follow this format in your response.\\ Label: 1\\ Explanation: Argument 1 is more convincing, because I totally agree with its point and it evokes my empathy.\end{tabular}                                                                                                                                                                                                                                                                                                                                                                                                                                                            \\ \bottomrule
\end{tabular}}
\caption{Prompt templates for comparing the convincingness of an argument pair. The {text} field contains the two arguments and their topic. The complete prompt is formed by combining the text in the `Shared' row with the text in the corresponding indexed row. For example, Prompt 1 consists of the text from both the `Shared' row and row `1'.}\label{tab:promptc}
\end{table*}



\begin{table}[]
\vspace{-.2cm}
\centering
\setlength\tabcolsep{2pt} 
\resizebox{\linewidth}{!}{%
\begin{tabular}{@{}lcc|cccccc@{}}
\toprule
& \multicolumn{2}{c|}{\textbf{\#Annotators}} & \multicolumn{6}{c}{\textbf{Agreements}} \\
              & \textbf{S} &  \textbf{C} & \multicolumn{3}{c}{\textbf{EMO}} & \multicolumn{3}{c}{\textbf{CONV}} \\ %\midrule
              &&& $\alpha$ & Full & Maj. & $\alpha$ & Full & Maj. \\ \midrule
\dagstuhl{}      & 1        & 4         &  0.506  & 6.5\% & 74.5\%   & 0.540  & 14.0\% & 80.0\%   \\
\bill{} & 1        & 4        &  0.449  & 7.0\% & 76.5\%   & 0.463  & 10.5\% & 78.0\%   \\
\hansard{}       & 1        & 4        &  0.361 & 0.5\% & 68.0\%   & 0.371  & 6.0\% & 75.0\%   \\
\lynn{}       & 2        & 3       &  0.729 & 13.5\% & 87.5\%   & 0.607  & 16.0\% & 82.0\%   \\
\deuparl{}       & 3        & 2       & 0.352  & 8.0\% & 80.5\%   & 0.364  & 4.5\% & 74.5\%   \\ \midrule
Avg    & -        & - & 0.479 & 7.1\% & 77.4\% & 0.469 & 10.2\% & 77.9\% \\ 
\bottomrule
\end{tabular}}
\caption{\textbf{Left}: Number of student (S) and crowdsourcing (C) annotators per batch. \textbf{Right}: Krippendorf's $\alpha$ for the most agreeing annotator pairs (\textbf{$\alpha$}), the percentages of annotation instances where all annotators agree on a certain label (\textbf{Full}),  and the percentage of annotation instances where at least three annotators agree on a certain label (\textbf{Maj.}). 
}
\label{tab:annotator}
\vspace{-.6cm}
\end{table}


\section{Annotation Interface}\label{app:anno}
Figure \ref{fig:anno} shows the screenshots of the annotation interface for convincingness (top) and emotion (bottom) comparisons. We collect the annotations via Google Forms\footnote{\url{https://docs.google.com/forms/}} for crowdsourcing annotators.

\begin{figure*}
    \includegraphics[width=\linewidth]{structure/figs/anno/conv_form.pdf}
    \includegraphics[width=\linewidth]{structure/figs/anno/emo_form.pdf}
    \caption{Screenshots of the annotation interface for convincingness (top) and emotion (bottom) comparison.}\label{fig:anno}
\end{figure*}


\section{Examples}\label{app:exa}
Table \ref{tab:pos} and \ref{tab:neg} provide example instances from \hansard{} and \lynn{}, where emotions have a positive and negative impact, respectively. 

\begin{table*}[!ht]
\centering
\begin{tabularx}{\textwidth}{ X | X }
\toprule
\multicolumn{2}{l}{\textbf{Topic}: The public supports the UK's aid for Ukrainian refugees} \\ \midrule
\rz{}  & \ro{}  \\ \midrule
Members across this House are determined that we, as a country, should open our arms to these people, and this determination has been on full display today. The scenes of devastation and human misery inflicted by President Putin’s barbarous assault on what he calls “Russia’s cousins” in Ukraine have unleashed a tidal wave of solidarity and generosity across the country. British people always step forward and step up in these moments, and since the first tanks rolled into Ukraine, they have come forward in droves with offers of help: community centres have been flooded with critical supplies; the Association of Ukrainians in Great Britain has received millions in donations; and charities such as the Red Cross have been overwhelmed with people giving whatever they can. The outpouring of public support has been nothing short of remarkable. & While this Government, and this whole House, have risen to the occasion with our offer of support to Ukrainians fleeing war, our lethal aid and our stranglehold on economic sanctions on Russia have clearly shown that we will keep upping the ante to ensure that Putin fails. As Members have argued today, it has been abundantly clear in recent days that we can and must do more. It is exactly right, therefore, that my right hon. Friend the Secretary of State for Levelling Up, Housing and Communities set out on Monday the new and uncapped sponsorship scheme, Homes for Ukraine. It is a scheme to allow Ukrainians with no family ties to the UK to be sponsored by individuals or organisations that can offer them a home. It is a scheme that draws not only on the exceptional good will and generosity of the British people, but one that gives them the opportunity to help make a difference.                                                                                                                                                        \\ \midrule
\rth{}   & \rtw{} \\ \midrule
Members of this House have expressed a commitment to welcoming individuals from Ukraine. The recent conflict initiated by President Putin has resulted in significant destruction in Ukraine, prompting a substantial response of support across the country. British citizens have actively contributed since the conflict began, with community centers collecting essential supplies, the Association of Ukrainians in Great Britain receiving financial contributions, and charities like the Red Cross witnessing increased donations.  & In these trying times, the Government and this entire House have demonstrated unwavering courage and compassion by extending our support to Ukrainians escaping the horrors of war. Our determined provision of lethal aid and the relentless imposition of economic sanctions on Russia are powerful affirmations that we will stop at nothing to ensure Putin's defeat. As Members have passionately discussed today, the urgency to do even more has never been clearer. That is why it is so heartening that my right hon. Friend the Secretary of State for Levelling Up, Housing and Communities announced on Monday the new and limitless Homes for Ukraine sponsorship scheme. This initiative opens its arms to Ukrainians without family connections in the UK, allowing them to be warmly embraced by individuals or organizations ready to offer them a sanctuary. It is a testament not only to the extraordinary kindness and generosity of the British people but also to their deep desire to make a meaningful impact in the lives of those in desperate need. \\ \bottomrule
\end{tabularx}
\caption{An example instance from \hansard{} where emotions have a \textbf{positive} impact on argument convincingness.
}\label{tab:pos}
\end{table*}

\begin{table*}[!ht]
\centering
\begin{tabularx}{\textwidth}{ X | X }
\toprule
\multicolumn{2}{l}{Topic: Haie können Krebs bekommen.}    \\ \midrule
\rz{}  & \ro{}  \\  \midrule
Haie sind mehrzellige Lebewesen, wie auch der Mensch. Die Beonderheit von mehrzelligen Lebewesen ist, dass die Zellen sich sowohl stark spezialisieren und untereinander vernetz kommunizieren. Damit werden sie anfällig für bestimmte Zelldefekte, die sich über die genannte Struktur fortpflanzen und den Krebs ausmachen. Haie verfügen, wie auch der Mensch und überhaupt alle mehrzelligen Lebewesen, über nur eine sehr eingeschränkte Möglichkeit diese Defekte zu korrigieren und aufzuhalten, damit können beide gleichermaßen Krebs bekommen & Da auch Fische Krebs bekommen können, ist es auch möglich, dass Haie Krebs bekommen können. Dieser wird durch mutierte Zellen ausgelöst, weshalb dies auch bei Fischarten ausgelöst werden kann. Krebs ist eine weit verbreitete und häufige Krankheit, weshalb Krebs durch Wissenschaftler auch bereits bei Haien festgestellt werden konnte.\\ Krebs kann außerdem auch durch verschiedene Umweltfaktoren wie Umweltverschmutzung ausgelöst werden, diesem Risiko sind Haie ja durchaus ausgesetzt. Deshalb ist die Gefahr einer Erkrankung auch nicht gerade gering.  \\ \midrule
\rth{}  & \rtw{}  \\ \midrule
Haie, ebenso wie Menschen, sind mehrzellige Organismen. Eine charakteristische Eigenschaft solcher Organismen ist die Spezialisierung und Vernetzung ihrer Zellen. Diese Struktur macht sie anfällig für Zellfehler, die sich ausbreiten und zu Krebs führen können. Haie und Menschen besitzen nur begrenzte Mechanismen zur Korrektur und Kontrolle dieser Defekte, was bedeutet, dass beide Arten gleichermaßen anfällig für Krebs sind.  & Die Vorstellung, dass Haie - diese majestätischen und oft missverstandenen Kreaturen der Meere - an Krebs erkranken können, ist zutiefst beunruhigend. Diese Krankheit, die durch die heimtückische Mutation von Zellen verursacht wird, hat bereits viele Fischarten heimgesucht. Die Tatsache, dass auch Haie, die Könige der Ozeane, nicht sicher vor dieser grausamen Krankheit sind, ist erschütternd. Angesichts der weit verbreiteten Umweltverschmutzung, die unsere Ozeane verschlingt, sind Haie einem erheblichen Risiko ausgesetzt, an Krebs zu erkranken. Es ist traurig und alarmierend, dass diese beeindruckenden Tiere, die seit Millionen von Jahren die Meere durchstreifen, nun durch menschliche Einflüsse bedroht sind.
\\ \bottomrule
\end{tabularx}
\caption{An example instance from \lynn{} where emotions have a \textbf{negative} impact on argument convincingness.
}\label{tab:neg}
\end{table*}




\section{LLM}\label{app:llm}
Figure \ref{fig:dis_prompt} illustrates the consistency, positivity and negativity rates of LLMs with different prompts, averaged across instances in all datasets. Table \ref{tab:llm_ranking} displays macro F1 scores and model rankings for LLMs in predicting convincingness rankings of argument pairs ('Static') and the resulting categories of emotional effect (`Dynamic') in English and German.

\begin{figure*}[]
    \centering
    \includegraphics[width=\linewidth]{structure/figs/llm/dis_prompt1.pdf}
    \includegraphics[width=\linewidth]{structure/figs/llm/dis_prompt2.pdf}
    \includegraphics[width=\linewidth]{structure/figs/llm/dis_prompt3.pdf}
    \caption{Consistency, positivity and negativity rates of LLMs with different prompts, averaged across instances in all datasets.}\label{fig:dis_prompt}
\end{figure*}


\begin{table*}[]
\resizebox{\linewidth}{!}{
\begin{tabular}{lcccc|cccc}
\toprule
                           & \multicolumn{4}{c|}{\textbf{EN}}               & \multicolumn{4}{c}{\textbf{DE}}               \\ 
\textbf{Model}                      & Static & Ranking & Dynamic & Ranking & Static & Ranking & Dynamic & Ranking \\ \midrule
gpt-4o-2024-08-06          & \textbf{0.486}  & 1       & 0.411   & 2       & \textbf{0.443}  & 1       & \textbf{0.447}   & 1       \\
Llama-3.3-70B-Instruct     & 0.417  & 2       & \textbf{0.415}   & 1       & 0.372  & 2       & 0.392   & 4       \\
gpt-4o-mini                & 0.416  & 3       & 0.392   & 5       & 0.35   & 4       & 0.394   & 3       \\
Qwen2.5-72B-Instruct       & 0.398  & 4       & 0.398   & 4       & 0.357  & 3       & 0.41    & 2       \\
gpt-3.5-turbo              & 0.39   & 5       & 0.382   & 6       & 0.338  & 6       & 0.381   & 6       \\
Mixtral-8x7B-Instruct-v0.1 & 0.368  & 6       & 0.376   & 7       & 0.35   & 5       & 0.387   & 5       \\
Mistral-7B-Instruct-v0.3   & 0.367  & 7       & 0.407   & 3       & 0.288  & 8       & 0.36    & 9       \\
Llama-3.2-3B-Instruct      & 0.322  & 8       & 0.32    & 10      & 0.281  & 10      & 0.367   & 8       \\
Qwen2.5-0.5B-Instruct      & 0.308  & 9       & 0.342   & 9       & 0.284  & 9       & 0.344   & 10      \\
Qwen2.5-7B-Instruct        & 0.304  & 10      & 0.346   & 8       & 0.319  & 7       & 0.373   & 7       \\
Llama-3.2-1B-Instruct      & 0.286  & 11      & 0.309   & 11      & 0.274  & 11      & 0.343   & 11 \\ \bottomrule     
\end{tabular}}
\caption{Macro F1 scores and model rankings for LLMs in predicting convincingness rankings of argument pairs ('Static') and the resulting categories of emotional effect (`Dynamic') in English and German. For each model, we present the best prompt result to highlight its potential. Human and LLM labels are determined by majority votes from different annotators and rounds, respectively.}\label{tab:llm_ranking}
\end{table*}

