\section{Introduction}

\begin{figure}[!t]
    \centering
    \includegraphics[width=.8\linewidth]{structure/figs/example.pdf.png}
    \caption{An example test case. \protect\rz{} is an argument with emotions and \protect\ro{} is an argument without emotions, both addressing the same topic with the same stance. \protect\rth{} is a counterpart of \protect\rz{} with reduced emotion. We compare the convincingness ranking of the pair (\protect\rz{}, \protect\ro{}) to that of the pair (\protect\rth{}, \protect\ro{}) to observe the effect of emotions on argument convincingness in a dynamic way.}
    \label{fig:example}
    \vspace{-.6cm}
\end{figure}


Emotional appeals have long been recognized as a core component of persuasion \citep{konat2024pathos,habernal2017argumentation}.
Aristotle’s triad of logos, ethos, and pathos \citep{kennedy1991theory} 
emphasizes the multifaceted nature of effective rhetoric. While logical reasoning (\emph{logos}) and the speaker's credibility (\emph{ethos}) are essential, the ability to evoke emotions in the audience (\emph{pathos}) 
\se{may also be crucial} 
\se{in order to} 
make the audience more receptive to the arguments \citep{wachsmuth-etal-2017-computational}.

Despite active research on 
argumentation and argument quality 
in the NLP community \citep[e.g.][]{habernal-gurevych-2016-makes,habernal-gurevych-2016-argument,gleize-etal-2019-convinced,wan-etal-2024-evidence,rescala-etal-2024-language,eger-etal-2017-neural,wachsmuth-etal-2017-computational,wachsmuth2024argument},
the 
pathos dimension has received 
\se{undeservedly little}
attention 
 \citep{evgrafova-etal-2024-analysing,greschner2024fearfulfalconsangryllamas}; emotional appeal is often discussed as a logical fallacy in arguments \citep[e.g.,][]{vijayaraghavan-vosoughi-2022-tweetspin,goffredo-etal-2023-argument,li-etal-2024-reason,mouchel2024logicalfallacyinformedframeworkargument}.
Existing NLP studies exploring the interplay between emotions and \emph{\se{argument} convincingness} often lack a specific focus on the emotional dimension and fail to control for confounding factors \citep[e.g.][]{habernal-gurevych-2016-argument,habernal2017argumentation,wachsmuth-etal-2017-computational}. A confounder refers to a variable that influences both the independent variable (the factor being manipulated: emotions) and the dependent variable (the outcome being measured: convincingness), potentially distorting the observed relationship between them.
To address this gap, we propose a \emph{dynamic} approach inspired by psychological manipulation checks \citep{hoewe2017manipulation,ejelov2020rarely}, where emotional intensity serves as the manipulated variable and convincingness as the dependent variable. To achieve this, we leverage LLMs to rephrase an argument to generate a counterpart that %is more/less likely to 
evokes stronger/weaker emotions, and then compare its convincingness to the original argument, thereby minimizing the effect of confounders. The judgments are evaluated relative to an anchor argument, as illustrated in Figure \ref{fig:example}, to obtain more reliable subjective human evaluation \todo{SE: and add the discussed further experiment} \citep{zhang2017improved,gienapp-etal-2020-efficient,jin-etal-2022-logical,habernal-gurevych-2016-argument}.
This framework enables us to examine how variations in perceived emotional intensity influence judgments of convincingness for a given argument in a controlled manner. 

Besides, we move beyond prior studies that focus predominantly on English arguments or single-domain datasets \citep{habernal-gurevych-2016-argument,habernal2017argumentation,wachsmuth-etal-2017-computational,greschner2024fearfulfalconsangryllamas}. We expand the scope to explore both English and German arguments across diverse text domains, including political debates, online portals, and curated human-written arguments. Our multilingual and cross-domain analysis provides a comprehensive view of how perceived emotional intensity 
affects convincingness across different contexts.

Finally, inspired by recent studies exploring cognitive biases in LLMs \citep{10.1093/pnasnexus/pgae233,echterhoff-etal-2024-cognitive,10.1162/tacl_a_00673,macmillan2024ir}, we further investigate whether LLMs behave like humans when judging argument convincingness under the influence of emotional `bias'. Although emotion is not always considered a fallacy or bias in argumentation \citep{walton2005fundamentals,Duckett02042020,evgrafova-etal-2024-analysing}, understanding its impact on argument evaluation is crucial for developing 
models intended for automated argument evaluation \citep[e.g.,][]{wachsmuth2024argument,rescala-etal-2024-language,mirzakhmedova2024large}.
\noindent\textbf{Our contributions are:}\footnote{Code+data: \url{https://github.com/cyr19/argument_emotion_llm_manipulation}}  
\begin{itemize}[topsep=2pt,itemsep=-1pt,leftmargin=*]
    \item We propose a novel framework to analyze how emotions influence perceived convincingness in a controlled manner. Our findings show that in over half of cases, human judgments remain unaffected by emotional intensity, while emotions more often enhance rather than weaken convincingness.
    
    \item We demonstrate that LLMs can effectively modify the emotional impact of arguments while preserving their original meaning, enabling precise comparisons of argument emotions.
    
    \item We conduct a multilingual, cross-domain analysis, showing that (i) when topics and domains align, emotions impact convincingness similarly in German and English, and (ii) emotions are more likely to enhance convincingness in political debates than in other domains. 
    
    \item We investigate whether LLMs exhibit human-like preferences in evaluating argument convincingness, particularly regarding emotions. While they broadly mirror human patterns, they fail to capture nuanced emotional effects in individual judgments.
\end{itemize}
%We will release code + data upon acceptance.




