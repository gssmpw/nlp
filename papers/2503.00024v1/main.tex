% This must be in the first 5 lines to tell arXiv to use pdfLaTeX, which is strongly recommended.
\pdfoutput=1
% In particular, the hyperref package requires pdfLaTeX in order to break URLs across lines.

\documentclass[11pt]{article}

% Change "review" to "final" to generate the final (sometimes called camera-ready) version.
% Change to "preprint" to generate a non-anonymous version with page numbers.
%\usepackage[review]{acl}
\usepackage[]{acl}

% Standard package includes
\usepackage{times}
\usepackage{latexsym}
%\usepackage{todonotes}
\usepackage{colortbl}

\renewcommand{\baselinestretch}{.99}
% For proper rendering and hyphenation of words containing Latin characters (including in bib files)
\usepackage[T1]{fontenc}
% For Vietnamese characters
% \usepackage[T5]{fontenc}
% See https://www.latex-project.org/help/documentation/encguide.pdf for other character sets

% This assumes your files are encoded as UTF8
\usepackage[utf8]{inputenc}

% This is not strictly necessary, and may be commented out,
% but it will improve the layout of the manuscript,
% and will typically save some space.
\usepackage{microtype}

% This is also not strictly necessary, and may be commented out.
% However, it will improve the aesthetics of text in
% the typewriter font.
\usepackage{inconsolata}

%Including images in your LaTeX document requires adding
%additional package(s)
\usepackage{graphicx}

% Additional packages


\usepackage{booktabs}
\usepackage[intlimits]{amsmath}  % place the subscripts and superscripts in the right position
%\usepackage{amsfonts}            % additional fonts like \mathbb, \mathfrak
%\usepackage{amssymb}    
%\usepackage{graphicx}
%\usepackage[colorinlistoftodos,size=tiny]{todonotes}
\usepackage[colorinlistoftodos,size=tiny,disable]{todonotes}
\usepackage{caption}
\usepackage{pifont}
\usepackage{subcaption}
\usepackage{multirow}
%\usepackage{verse}
%\usepackage{parallel}
%\usepackage{arydshln}
%\usepackage{lmodern}
\usepackage{enumitem}
\usepackage{makecell}
\usepackage{tabularx}
\usepackage{soul}
\usepackage{amssymb}

\newcommand{\hlc}[2][yellow]{{%
    \colorlet{foo}{#1}%
    \sethlcolor{foo}\hl{#2}}%
}


\newcommand{\se}[1]{\textcolor{black}{#1}}
\newcommand{\yc}[1]{\textcolor{black}{#1}} % orange
\newcommand{\ycr}[1]{\textcolor{black}{#1}} % orange
\newcommand{\ycrr}[1]{\textcolor{black}{#1}} % orange
\newcommand{\yca}[1]{\textcolor{black}{#1}}
\newcommand{\senew}[1]{\textcolor{black}{#1}} % blue
\newcommand{\senewest}[1]{\textcolor{black}{#1}} %red

\newcommand{\rz}[0]{\textit{E}}
\newcommand{\ro}[0]{\textit{N}}
\newcommand{\rth}[0]{\textit{G}$^{-}$\textit{(E)}}
\newcommand{\rtw}[0]{\textit{G}$^{+}$\textit{(N)}}

\newcommand{\bill}[0]{\texttt{Bill}$_{\texttt{en}}$}
\newcommand{\hansard}[0]{\texttt{Hansard}$_{\texttt{en}}$}
\newcommand{\deuparl}[0]{\texttt{DeuParl}$_{\texttt{de}}$}
\newcommand{\lynn}[0]{\texttt{EmoDefabel}$_{\texttt{de}}$}
\newcommand{\dagstuhl}[0]{\texttt{Dagstuhl}$_{\texttt{en}}$}

\newcommand{\tstt}[0]{\texttt{TST}$_{\texttt{3}}$}
\newcommand{\tstf}[0]{\texttt{TST}$_{\texttt{4}}$}

\newcommand{\data}[0]{\textsc{EmoVince}}
\renewcommand{\baselinestretch}{0.9825}

\linepenalty=1000
% If the title and author information does not fit in the area allocated, uncomment the following


\title{Do Emotions Really Affect Argument Convincingness? \\ A Dynamic Approach with LLM-based Manipulation Checks}

% Author information can be set in various styles:
% For several authors from the same institution:
% \author{Author 1 \and ... \and Author n \\
%         Address line \\ ... \\ Address line}
% if the names do not fit well on one line use
%         Author 1 \\ {\bf Author 2} \\ ... \\ {\bf Author n} \\
% For authors from different institutions:
% \author{Author 1 \\ Address line \\  ... \\ Address line
%         \And  ... \And
%         Author n \\ Address line \\ ... \\ Address line}
% To start a seperate ``row'' of authors use \AND, as in
% \author{Author 1 \\ Address line \\  ... \\ Address line
%         \AND
%         Author 2 \\ Address line \\ ... \\ Address line \And
%         Author 3 \\ Address line \\ ... \\ Address line}

\author{Yanran Chen and Steffen Eger \\
  %Affiliation / Address line 1 \\
  %Affiliation / Address line 2 \\
  %Affiliation / Address line 3 \\
 NLLG, University of Technology Nuremberg (UTN); \\\url{https://nl2g.github.io/}\\
  \texttt{\{yanran.chen,steffen.eger\}@utn.de} \\
  }
  

\begin{document}

%\newcommand{\attrib}[1]{\nopagebreak{\raggedleft\footnotesize #1\par}}

\maketitle

\begin{abstract}
Emotions have been shown to play a role in argument convincingness, yet this aspect is underexplored in the natural language processing (NLP) community. 
Unlike prior studies that use static analyses, focus on a single text domain or language, or treat emotion as just one of many factors, we introduce a dynamic framework inspired by manipulation checks commonly used in psychology and social science; leveraging LLM-based manipulation checks, this framework examines the extent to which perceived emotional intensity influences perceived convincingness.
Through human evaluation of arguments across different languages, text domains, and topics, we find that in over half of cases, judgments of convincingness remain unchanged despite variations in perceived emotional intensity; when emotions do have an impact, they more often enhance rather than weaken convincingness.
We further analyze how 11 LLMs behave in the same scenario, finding that while LLMs generally mirror human patterns,
they struggle to capture nuanced emotional effects in individual judgments.
\end{abstract}



\section{Introduction}\label{sec:intro}

In computational finance, Monte Carlo simulations are used extensively to estimate the expected value of financial payoffs based on the solution of stochastic differential equations (SDEs) which model the evolution of stock prices, interest rates, exchange rates and other quantities \cite{glasserman04}.  Monte Carlo methods are very general and flexible, but for high accuracy it requires generating a large number of costly SDE path approximations, which has motivated research into a number of variance reduction or, equivalently, cost reduction techniques. One such method is
Multilevel Monte Carlo (MLMC), which was proposed in \cite{GILES2008} and was adapted for various applications that are summarised in \cite{Giles_overview17} and successfully combined with other methods such as quasi-Monte Carlo methods. The main idea of MLMC is to approximate the payoff using different time stepping resolutions when numerically solving the underlying SDE and to generate an optimal number of samples on each level, such that the overall computational cost is minimised subject to the desired bound on the variance. %, such that the total computational cost is minimised. 
The computational savings come from the fact that most samples are computed on the coarser levels and hence are less expensive while only a few samples from the finest levels are required \cite{GILES2008}.


Among the directions in which the computational cost 
of MLMC methods could further be reduced, an important avenue is the use of lower precision calculations, especially for the first Monte Carlo levels where the targeted accuracy is relatively low. 
 An overview of the research on mixed precision for the standard Monte Carlo (MC) framework is provided in \cite{ChowMixedPrecisionStandardMC} but only a few references study the potential of low precision computation in the MLMC framework \cite{Rounding_error_oliver}. To the best of our knowledge, the only MLMC framework with customised precision in the literature is \cite{brugger2014mixed}, but they use a uniform precision for all operations on each Monte Carlo level instead of optimising 
 the precision of each intermediary variable to reduce as much as possible the cost of path generation.
 
An important motivation for an MLMC framework with variable precision would be performing the low precision computations on reconfigurable hardware devices such as Field Programmable Gate Arrays (FPGAs). FPGAs contain customizable logic blocks and connectors that make it easy to adapt the digital circuit architecture for a specific application, leading to a highly parallel and optimised implementation. Therefore they are successfully exploited in applications that require high speed and have high computational workload, such as signal processing \cite{woods2008fpga}, and real time applications like high frequency trading \cite{HFT1,HFT2}. That is why a number of previous works in hardware architecture design implemented the MLMC algorithm to price financial options using FPGAs as accelerators, which resulted in improved speed and power efficiency compared to full CPU architectures \cite{Schryver2013AMM}. The paper \cite{lindsey2016domain} also proposed 
a Domain Specific Language to automate the configuration of FPGAs for this specific application. However, only \cite{brugger2014mixed} proposed a heuristic to reduce the precision in calculations.

In addition, all aforementioned works considered that the random number generation (RNG) is performed in single or double precision. Yet in most cases an important portion of the workload in the overall MLMC simulation comes from the RNG and in \cite{brugger2014mixed} this limited the total computational savings.
To reduce the cost of MLMC simulations in particular those based on the Geometric Brownian Motion (GBM), \cite{approximateICDF_Oliver, NestedOliver} have proposed to use approximate random numbers that are generated by applying an approximation of the inverse CDF to uniform random numbers. In \cite{NestedOliver}, the authors proposed a way to integrate these lower precision random variables into a \textit{nested} MLMC framework and completed a numerical analysis to bound the resulting error at each MC level by a product of the time step and the error in the random number approximation. The same authors show in \cite{approximateICDF_Oliver} that using approximate random variables reduces the cost of path generation by a factor 7.


In this paper we propose a nested MLMC framework that combines the use of approximate random normal variables and lower precision calculations to reduce the computational cost of MLMC even further than \cite{brugger2014mixed,NestedOliver}. We illustrate the efficiency of our framework in Matlab, after making several assumptions on the cost of operations and size of the errors that we carefully justify. We focus on the case of GBM and use the approximate RNG methods presented in \cite{approximateICDF_Oliver} as well as a new slightly modified method that combines CDF inversion and the central limit theorem. To choose the precision of the variables in the low precision path generation, we introduce a novel method to optimise the bit-widths. This optimisation is performed before the main path generation loop is executed and is based on a linear model of the payoff error  
due to rounding when computing in low precision. The error model relies on algorithmic differentiation in a similar manner to \cite{unifying-bwoptim,bitwidth-AD,ADAPT}. The bit-width optimisation procedure can be performed off-line, so this stage can be excluded from the on-line time complexity of our framework. The user specified desired accuracy is then enforced by calculating on-line the number of samples that need to be generated.

In terms of hardware design, we suggest implementing the low precision path generation on FPGAs and the full-precision ones on a CPU or GPU. 
The FPGA offers enough flexibility to define a separate bit-width for every variable in the low precision path generation, and can be reconfigured periodically to update the bit-widths when the market parameters have changed considerably. 


The paper is organized as follows : \Cref{sec:MLMC} introduces MLMC and nested MLMC to make clear the estimator that is implemented in our framework. Then in \Cref{sec:RNG} we detail the methods that could be used to obtain approximate random normally distributed numbers very cheaply for the low precision path generation. In \Cref{sec:error_model} and \Cref{sec:costModel} we propose an error model and a cost model (resp.) that we then use to formulate the optimisation problem that is solved to obtain the optimal bit-widths of fixed point variables in \Cref{sec:optimisation}. Finally we summarise our results and future directions in \Cref{sec:conclusion}.




\section{Related Work}

\paragraph{LLMs for Agent tasks.}

Our research is related to deploying large language models (LLMs) as agents for decision-making tasks in interactive environments~\citep{liu2023agentbench,zhou2023webarena,shridhar2020alfred,toyama2021androidenv}. Earlier works, such as~\citep{yao2023webshopscalablerealworldweb}, fine-tuned models like BERT~\citep{devlin2019bertpretrainingdeepbidirectional} for decision-making in simplified environments, such as online shopping or mobile phone manipulation. With the advent of large language models~\citep{brown2020languagemodelsfewshotlearners,openai2024gpt4technicalreport}, it became feasible to perform decision-making tasks through zero-shot or few-shot in-context learning. To better assess the capabilities of LLMs as agents, several models have been developed~\citep{deng2024mind2web,xiong2024watch,hong2023cogagent,yan2023gpt}. Most approaches~\citep{zheng2024seeact,deng2024mind2web} provide the agent with observation and action history, and the language model predicts the next action via in-context learning. Additionally, some methods~\citep{zhang2023building,li2023camel,song2024trial} attempt to distill trajectories from state-of-the-art language models to train more effective policy models. In contrast, our paper introduces a novel framework that automatically learns a reward model from LLM agent navigation, using it to guide the agents in making more effective plans.

\textbf{LLM Planning.} Our paper is also related to planning with large language models. Early researchers~\citep{brown2020languagemodelsfewshotlearners} often prompted large language models to directly perform agent tasks. Later, \citet{yao2022react} proposed ReAct, which combined LLMs for action prediction with chain-of-thought prompting~\citep{wei2022chain}. Several other works~\citep{yao2023treethoughtsdeliberateproblem,hao2023reasoning,zhao2023large,qiao2024agentplanningworldknowledge} have focused on enhancing multi-step reasoning capabilities by integrating LLMs with tree search methods. Our model differs from these previous studies in several significant ways. First, rather than solely focusing on text generation tasks, our pipeline addresses multi-step action planning tasks in interactive environments, where we must consider not only historical input but also multimodal feedback from the environment. Additionally, our pipeline involves automatic learning of the reward model from the environment without relying on human-annotated data, whereas previous works rely on prompting-based frameworks that require large commercial LLMs like GPT-4~\citep{openai2024gpt4technicalreport} to learn action prediction. Furthermore, \Model supports a variety of planning algorithms beyond tree search.

\textbf{Learning from AI Feedback.} In contrast to prior work on LLM planning, our approach also draws on recent advances in learning from AI feedback~\citep{bai2022constitutional,lee2023rlaif,yuan2024self,sharma2024critical,pan2024autonomous,koh2024tree}. These studies initially prompt state-of-the-art large language models to generate text responses that adhere to predefined principles and then potentially fine-tune the LLMs with reinforcement learning. Like previous studies, we also prompt large language models to generate synthetic data. However, unlike them, we focus not on fine-tuning a better generative model but on developing a classification model that evaluates how well action trajectories fulfill the intended instructions. This approach is simpler, requires no reliance on state-of-the-art LLMs, and is more efficient. We also demonstrate that our learned reward model can integrate with various LLMs and planning algorithms, consistently improving their performance.

\textbf{Inference-Time Scaling.} ~\citet{snell2024scaling} validates the efficacy of inference-time scaling for language models. Based on inference-time scaling, various methods have been proposed, such as random sampling~\citep{wang2022self} and tree-search methods~\citep{hao2023reasoning, zhang2024accessing, guan2025rstar}. Concurrently, several works have also leveraged inference-time scaling to improve the performance of agentic tasks. ~\citet{koh2024tree} adopts a training-free approach, employing MCTS to enhance policy model performance during inference and prompting the LLM to return the reward. ~\citet{gu2024your} introduces a novel speculative reasoning approach to bypass irreversible actions by leveraging LLMs or VLMs. It also employs tree search to improve performance and prompts an LLM to output rewards. ~\citet{yu2024exact} proposes Reflective-MCTS to perform tree search and fine-tune the GPT model, leading to improvements in ~\citet{koh2024visualwebarena}. ~\citet{putta2024agent} also utilizes MCTS to enhance performance on web-based tasks such as ~\citet{yao2023webshopscalablerealworldweb} and real-world booking environments. ~\cite{lin2025qlass} utilizes the stepwise reward to give effective intermediate guidance across different agentic tasks. Our work differs from previous efforts in two key aspects: (1) Broader Application Domain. Unlike prior studies that primarily focus on tasks from a single domain, our method demonstrates strong generalizability across web agents, mathematical reasoning, and scientific discovery domains, further proving its effectiveness. (2) Flexible and Effective Reward Modeling. Instead of simply prompting an LLM as a reward model, we finetune a small scale VLM~\citep{lin2023vila} to evaluate input trajectories. %Our reward scores range continuously between 0 and 1, in contrast to existing methods that rely on discrete scoring (e.g., 0 and 1, or 0, 0.5, and 1) through direct LLM prompting.

% Concurrently, several works have also leveraged inference-time scaling to improve the performance of agentic tasks. ~\citet{pan2024autonomous} demonstrates that LLMs and VLMs, such as the GPT series, can function as evaluators or reward models to provide guidance for fine-tuning or reflection, thereby enhancing digital agents. This lays the groundwork for subsequent studies that directly prompt LLMs as reward models. ~\citet{koh2024tree} adopts a training-free approach, employing MCTS to enhance policy model performance during inference. However, it is limited to web environments~\citep{koh2024visualwebarena}. Moreover, its value function relies on prompting an LLM, which is less effective than our proposed method. We validate our approach through ablation studies, demonstrating that our fine-tuned reward model is more effective. ~\citet{gu2024your} introduces a novel speculative reasoning approach to bypass irreversible actions, such as purchasing a product, by leveraging LLMs or VLMs. It also employs tree search to improve performance, but it remains restricted to the web domain~\citep{koh2024visualwebarena, deng2024mind2web}. Additionally, it lacks reward modeling and instead prompts an LLM to output rewards. ~\citet{yu2024exact} proposes Reflective-MCTS to perform tree search and fine-tune the GPT model, leading to improvements in ~\citep{koh2024visualwebarena}. However, this work focuses solely on a single web agent task, and its reward modeling is derived from multi-agent debate, differing from our more effective and efficient reward modeling approach. ~\citet{putta2024agent} also utilizes MCTS to enhance performance, but it is limited to web-based tasks such as ~\citep{yao2023webshopscalablerealworldweb} and real-world booking environments.
% introduce PDDL domains
% why Gripper env as testing context
% motivation: comparing classical vs LLM planners
% - classical: PDDL solver fast-downward
% - LLM: gpt-4o
% explanation and refinement are two distinguishing features of LLM planners
% - how we demonstrate explanation and refinement in the study
We evaluate user trust in two planners over a set of planning problems and study the potential factors influencing user trust in the planners. In particular, we compare a language-model-based planner, denoted as an \emph{LLM Planner}, with a traditional graph-search-based planner, denoted as a \emph{PDDL Solver}. The PDDL Solver uses Fast Downwards \cite{fastdownward} as its underlying model, processing planning problems described in PDDL to generate an optimal solution. In comparison, the LLM Planner employs GPT-4o to interpret the planning problem and extract a solution generated by the language model. Unlike the PDDL Solver, the LLM Planner can reason through the planning problem, explain its proposed solution, and iteratively refine the solution based on external feedback. This study investigates how the correctness of solutions, the quality of explanations, and the refinement process influence user trust.

\subsection{Planning Problem}
% \begin{wrapfigure}{r}{0.4\textwidth}
% % \begin{figure}[t]
%     \centering
%     \includegraphics[width=\linewidth]{figures/problem-example.pdf}
%     \caption{A running example of a planning problem in our study.}
%     \Description{Planning Problem Example}
%     \label{fig: problem-example}
% % \end{figure}
% \end{wrapfigure}

We describe each planning problem in the \emph{Planning Domain Definition Language (PDDL)} and propose two planners to generate plans that solve the problem. We select the \emph{gripper} planning problems from the International Planning Competition \cite{IPC} for plan generation and evaluation. In a gripper planning problem, a robot moves balls between a set of rooms using two grippers. The objective is to create a plan for the robot to move the balls to the target rooms we defined. We present a few running examples of the gripper problem in Figure \ref{fig: correctness}.

A planning problem consists of a \emph{planning domain} and a \emph{problem description}, expressed in PDDL. 

\paragraph{Planning Domain}
A planning domain refers to the universal aspects of a problem that remains consistent across different instances of the problem. In particular, it defines the types of objects, predicates, and actions that exist in the planning problem. We present an example of the gripper problem in Appendix \ref{app: grippers}.

\paragraph{Problem Description} A problem description specifies the particular instance of a planning task within a given domain. It includes the planning domain to which it pertains, a set of objects, the initial state of these objects, and the goal state to be achieved.

\paragraph{Plan}
A plan is a sequence of actions with specific input parameters. Recall that an action corresponds to a state transition. If a plan (a sequence of actions) transits from the initial state to the goal state defined by a problem, then we consider the plan to be \emph{correct}. If a plan does not transit to the goal state or there exists an action violating its precondition, then the plan is \emph{wrong}.

\begin{figure}[t]
    \centering
    \includegraphics[width=0.8\linewidth]{figures/correct.jpeg}
    \caption{Examples where LLM Planner correctly generates a plan for the gripper planning problem.}
    \Description{Planning Problem Correctness}
    \label{fig: correct}
\end{figure}

\subsection{PDDL Solver}
The PDDL Solver takes the planning domain and the problem description as inputs and then generates a plan described in PDDL. 
% It generates a plan in the following format:
% \vspace{4pt}
% \begin{lstlisting}[language=completion]
% (move robot1 room1 room3)
% (pick robot1 ball2 room3 rgripper1)
% (move robot1 room3 room2) ......
% \end{lstlisting}
Next, we convert the generated plan into natural language for user studies following the procedure in \cite{seipp-et-al-zenodo2022} and display it to users. We present an example in Figure \ref{fig: correct}.

The PDDL Solver applies a graph search algorithm to find a path (i.e., a list of transitions) from the initial state to the goal state. It either generates a \emph{correct} plan---defined as the shortest path between the initial and goal states---or returns a signal indicating that no solution exists for the given problem.

\subsection{LLM Planner}

The LLM Planner addresses planning problems by querying a large language model. In particular, it transmits the planning domain and problem description to the language model using a structured prompt format. The planner then retrieves a natural language plan from the language model. We use GPT-4o as the language model for the planner. To ensure the output adheres to the desired format, we include a few in-context examples within the prompts.

A language model solves a planning problem by interpreting the domain and problem descriptions, simulating state transitions, and generating a sequence of actions to achieve the goal. While effective for reasoning and plan generation, language models may struggle with large state spaces. Unlike the PDDL Solver, the LLM Planner may generate \emph{incorrect} plans that violate the problem specifications (e.g., preconditions of actions) or fail to achieve the goal.

\subsection{Explanation and Refinement}
Alongside the generated plans, we offer detailed explanations of all the plans and revisions of any incorrect plans. This study examines how these explanations and refinements influence human trust in the two planners.

\paragraph{LLM Planner with Explanation (LLM+Expl)}
For each generated plan, we manually provide a natural language explanation. This explanation includes an assessment of the plan’s correctness, identification of any violations of action preconditions, and an analysis of inconsistencies between the final state achieved and the intended goal state. We present examples of explanations in Figure \ref{fig: explain} in Appendix.

In particular, if a plan is correct, the explanation is simply ``the plan successfully satisfies the goal conditions.'' 
If a plan is incorrect, we identify the underlying cause as either a violation of action preconditions or a failure to achieve the goal state. In cases involving precondition violations, we specify the action responsible for the issue. For example, consider the action ``robot moves from room 1 to room 2,'' but the robot is initially located in room 3. This scenario constitutes a violation of the precondition for the ``move'' action. In the latter case, we describe the differences between the final state achieved and the intended goal state, e.g., ``fail to move ball 2 to room 2.''

% \begin{wrapfigure}{r}{0.5\textwidth}
%     \centering
%     \includegraphics[width=0.98\linewidth]{figures/refine.jpeg}
%     \includegraphics[width=0.98\linewidth]{figures/refine-correct.jpeg}
%     \includegraphics[width=0.98\linewidth]{figures/refine-wrong.jpeg}
%     \caption{Plan refinement by the LLM Planner. The top row presents two choices of plan refinement (where the refinement starts). The second and third row shows the refinement outcomes of the two choices, where the second row shows a correctly refined plan and the third row shows an incorrect plan.}
%     \Description{Refinement}
%     \label{fig: refine}
% \end{wrapfigure}

\paragraph{LLM Planner with Refinement (LLM+Refine)}
Note that a plan generated by the LLM Planner could be incorrect. Therefore, we offer a prompting mechanism for the LLM Planner to refine the generated plan according to the user feedback. The mechanism works as follows:

1. Request the user to indicate the step number of the first action in the plan that is incorrect, such as the step where an action’s precondition is violated. We present a sample user interface on the left of Figure \ref{fig: refine} in Appendix.

2. Send the planning domain, problem description, and the original plan to the language model. Then, query the model to rewrite the subsequent steps starting from the user-specified step number. We present a sample input prompt in Figure \ref{fig: refine-prompt} in the Appendix.

3. Replace the original plan with the newly refined plan and display it to the user.

This mechanism allows users to interact with the language model to refine the plan. It enables the language model to focus on a subset of steps, facilitating a deeper interpretation of the incorrect component. However, the correctness of the refined plan is not guaranteed. Figure \ref{fig: refine} in the Appendix shows an example of a correctly refined plan and an incorrectly refined plan.

%\input{structure/3_1_human}
\section{Results}
\label{sec:results}
Following \nksr, we evaluate our method using metrics including the standard Chamfer-$L_1$ Distance~(CD-$L_1 \times 10^{-2}$, $\downarrow$) and F-score~($\uparrow$) with a threshold~($\delta{=}0.010$). 
We also report additional metrics proposed in \nksr~including Chamfer-$L_1$ Distance by Completeness (Comp.~$\times 10^{-2}$, $\downarrow$) and Accuracy (Acc.~$\times 10^{-2}$, $\downarrow$) in the \texttt{Supplementary Material}. 
We evaluate our method on multiple datasets, under two settings including in-domain evaluation for accuracy estimation -- training set and test set are from same dataset, and cross-domain evaluation for generalization ability estimation where training set and test set are from different datasets. 
Additionally, for cross-domain evaluation we use the following datasets prepared by the leading voxel-based baseline, \nksr, and one additional dataset from RangeUDF~\cite{wang2022rangeudf}:

\begin{itemize}
    \item \synthetic{}  is a synthetic dataset created from ShapeNet objects~\cite{chang2015shapenet}. Each scene contains 2-3 objects. 
    Following prior works~\cite{wang2022rangeudf,chibane2020ndf}, we re-scale the synthetic rooms to roughly match real-world scale.
    There are 3750 scenes as training set and \ws{995 scenes} as the test set. 
    \item \scannet{} is a real-world indoor scene dataset. We use the setting from previous work~\cite{wang2022rangeudf, tang2021SACon, peng2020convoccnet, boulch2022poco} where we train on 1201 rooms and test on 312 rooms. 
    \item \carla is a large-scale outdoor driving scene prepared by NKSR~\cite{huang2023neural} using the CARLA simulator~\cite{dosovitskiy2017carla}. 
    \ws{Following NSKR~\cite{huang2023neural}, we test on two subsets including the 'Original' subset (10 random drives simulated on 3 towns) and the 'Novel' subset (3 drives from an additional town only for testing).}
    To avoid exploding GPU memory during training, we follow NKSR~\cite{huang2023neural} to divide a large scene into patches. The resultant training set has {3757} patches. 
    \item \scenenn{}  is a real-world indoor dataset prepared by RangeUDF~\cite{wang2022rangeudf} which we used for cross-domain evaluation. We only use its test set which consists of 20 scenes.
\end{itemize}



\begin{table*}
\centering
\resizebox{\linewidth}{!}{
\setlength{\tabcolsep}{3pt}
\begin{tabular}{LccccccccccccC}
\toprule
Methods & & \multicolumn{3}{c}{\ws{{\bf \synthetic}}}  &  \multicolumn{3}{c}{{\bf \scannet}} & \multicolumn{3}{c}{\ws{{\bf \carla(Original)}}} & \multicolumn{3}{c}{\ws{{\bf \carla(Novel)}}} \\ 
 \cmidrule(lr){3-5} \cmidrule(lr){6-8} \cmidrule(lr){9-11} \cmidrule(lr){12-14} 
&Primitive& CD ($10^{-2}$) $\downarrow$ & F-Score  $\uparrow$ & Latency (s) $\downarrow$  & CD ($10^{-2}$) $\downarrow$ & F-Score  $\uparrow$ & Latency (s) $\downarrow$  & CD (cm) $\downarrow$ & F-Score  $\uparrow$ & Latency (s) $\downarrow$ & CD (cm) $\downarrow$ & F-Score  $\uparrow$ & Latency (s) $\downarrow$ \\        
\midrule
SA-CONet~\cite{tang2021SACon} & Voxels & {0.496} & {93.60} & - & - & - & - & - & - & - & - & - & -\\
ConvOcc~\cite{peng2020convoccnet} & Voxels & {0.420} & {96.40} & - & - & - & - & - & - & - & - & - & -\\
NDF~\cite{chibane2020ndf} & Voxels & {0.408} & {95.20} & - & 0.385  & 96.40  & -  & - & - & - & - & - & -\\
RangeUDF~\cite{wang2022rangeudf} & Voxels & {0.348} & {97.80} & {-} & 0.286 & 98.80 & - & - & - & - & - & - & -\\
\ws{TSDF-Fusion~\cite{zeng20163dmatch}} & -  & - & - & - & - & - & - & 8.1 & 80.2 & - & 7.6 & 80.7 & - \\
\ws{POCO~\cite{boulch2022poco}} & - & - & - & - & - & - & - & 7.0 & 90.1 & - & 12.0 & 92.4 & - \\
\ws{SPSR~\cite{kazhdan2013screened}} & - & - & - & - & - & - & - & 13.3 & 86.5 & - & 11.3 & 88.3 & - \\
\nksr & Voxels &  \underline{0.346} &  \underline{97.41} & \underline{0.40} & \underline{0.246} & \underline{99.51} & \underline{1.54} &  \underline{3.9} &  \underline{93.9} &  \underline{2.0} &  \underline{2.9} &  \underline{96.0} &  \underline{1.8} \\
\nksr (more data) & Voxels & - & - & - & - & - & - & {3.6} & {94.0} & {2.0} & {3.0} & {96.0} & {1.8}\\
Ours~(Minkowski)~\cite{choy20194d} \scriptsize{(w/ KNN)} & Voxels & - & \todo{} & \todo{} & 0.254 & 99.41 & 0.46 & 3.4 & 97.2 &1.9 & 2.7 & 98.1 & 2.0 \\
Ours~(Minkowski)~\cite{choy20194d} & Voxels & - & \todo{} & \todo{} & 0.301 & 98.48 & 0.31 & 3.8 & 96.2 & 1.5 & 3.0 & 97.4 & 1.5\\
\rowcolor{1st} Ours \scriptsize{(w/ KNN)} & Points &{0.321} & {98.34} & {0.13} & {0.243} & {99.61} & {0.48} &{3.2} & {97.5} & {3.2} &{2.6} & {98.3} & {3.4}\\
\rowcolor{1st}Ours & Points & {0.360} & {96.32} & 0.14 & 0.257 & 99.33 & 0.49 & {3.3} & {97.4} & 1.7 & {2.7} & {98.2} & 1.7 \\

\bottomrule
\end{tabular}
}
\caption{\textbf{In-domain evaluation} -- We show that our method achieves the best accuracy (CD and F-score) with significantly improved time efficiency~(inference latency).
Note we retrain \nksr (numbers are underlined) for fairer comparison, \ws{as the training data for \nksr is different from ours -- i.e., they reported some models trained on a ``mix'' of datasets, which is impossible to reproduce.
}
}
\label{tab:indomain}
\end{table*}


\paragraph{Evaluation pipeline}
To evaluate our method, we first extract the mesh with Dual Marching Cubes~\cite{schaefer2004dual} on the predicted SDF, and then compute the CD and F-score between 100k points sampled on the mesh, and 100k points sampled from the ground-truth dense point cloud.
We use the same approach as \nksr to prepare the input point clouds for training and evaluation from the ground-truth dense point clouds through downsampling.
Specifically, for indoor datasets (i.e., \synthetic, 
\scannet and \scenenn), we uniformly sample 10K points sampled from the ground truth dense point cloud. 
For outdoor driving scenes~(i.e., \carla), we follow the evaluation pipeline from \nksr.
We sample sparse input point clouds with a sparse 32-beam LiDAR with a ray distance noise of 0-5 cm and pose noise of $0-3^\circ$, and obtain the ground truth from a noise-free dense 256-beam LiDAR.

\begin{figure*}
\centering
\includegraphics[width=\linewidth]{visualizations/test_set_results.pdf}
\caption{
{\textbf{Qualitative results on \carla and \synthetic}} -- our method achieves high quality surface reconstructions which preserve more details than \nksr~which loses information due to quantization for large and non-uniformly sampled datasets like Carla.
}
\label{fig:qual_results_carla_syn}
\end{figure*}
 
\begin{figure*}
\centering
\vspace{-1em}
\includegraphics[width=.95\linewidth]{visualizations/scannet_results_0.pdf}
\caption{
Qualitative results on \scannet: We compare our method with prior SOTA~\cite{huang2023neural} and Ours~(Minkowski)~\cite{choy20194d} that is more comparable as it only differs from ours in the backbone. Our method achieves reconstruction of similar quality to the SOTA. It also \textit{significantly} outperforms Ours~(Minkowski), highlighting the importance of point-based methods. 
% \TODO{callouts too small? almost no zoom? why?}
}
\vspace{-1em}
\label{fig:scannet_results}
\end{figure*}
  

\paragraph{Implementation details}
We base our feature backbone on PointTransformerV3~\cite{wu2024point} with 4-levels.
The PointNet-style network is a 2-layered residual connection MLP, with hidden dimension of $32$ and output feature dimension of $32$.    
The grid size used in neighborhood function is $0.01$ meters.
Following \nksr, we use the similar coefficients for loss terms -- i.e., $\lambda_{\text{SDF}}$ is $300$ and $\lambda_{\text{mask}}$ is $150$.
However, we empirically set $\lambda_{\text{Eikonal}}$ to $10$~(\nksr does not need this regularizer thanks to its specialized surface solver).
We train our model with a batch size of $4$ on either a single \texttt{NVIDIA RTX A6000 ADA} or an \texttt{NVIDIA L40S}, and a learning rate of $10^{-3}$.
We adopt the Adam optimizer with default parameters.
We set the maximum number of epochs to 200 and employ a cosine learning rate decay starting from epoch 120.


\begin{table*}
\centering
\resizebox{\linewidth}{!}{
\setlength{\tabcolsep}{2pt}
\begin{tabular}{LccccccccccC}
\toprule
Methods & & \multicolumn{3}{c}{{\bf \synthetic $\rightarrow$ \scannet}}  &  \multicolumn{3}{c}{{{\bf \scannet $\rightarrow$ \synthetic}}} & \multicolumn{3}{c}{{{\bf \scannet $\rightarrow$ \scenenn}}} \\ 
 \cmidrule(lr){3-5} \cmidrule(lr){6-8} \cmidrule(lr){9-11}
&Primitive& CD ($10^{-2}$) $\downarrow$ & F-Score  $\uparrow$ & {Latency (s) $\downarrow$ } & CD ($10^{-2}$) $\downarrow$ & F-Score  $\uparrow$ & {Latency (s) $\downarrow$ } & CD ($10^{-2}$) $\downarrow$ & F-Score  $\uparrow$ & {Latency (s) }$\downarrow$ \\       
\midrule
SA-CONet~\cite{tang2021SACon} & Voxels & 0.845 & 77.80 & - & - & - & - & - & - & - \\
ConvOcc~\cite{peng2020convoccnet} & Voxels & 0.776 & 83.30  & - & - & - & - & - & - & - \\
NDF~\cite{chibane2020ndf} & Voxels & 0.452 & 96.00 & - & {0.568} & {88.10} & - & 0.425 & 94.80 & - \\
RangeUDF~\cite{wang2022rangeudf} & Voxels & {0.303} & {98.60} & {-} & 0.481& 91.50 & - & 0.324 & 97.80 & - \\
\nksr & Voxels & {0.329} & {97.37} & {2.02} & {0.351} & {97.41} & {0.46} & {0.268} & {99.18} & {1.95} \\
\rowcolor{1st} Ours (w/ KNN) & Points & {0.284} & {98.65} & {0.54} & {0.327} &{98.37} & {0.13} & {0.277} & {99.00} & {0.50} \\
\bottomrule
\end{tabular}
}
\caption{\textbf{Cross-domain evaluation} -- we achieve the best generalization ability in two cases with much better time efficiency. In the other case where we generalize from \scannet to \scenenn, we achieve accuracy on par with the SOTA baseline~\cite{huang2023neural} with less than a half of their latency.  
}
\vspace{-1.4em}
\label{tab:across_domain}
\end{table*}


\paragraph{Reconstruction latency}
For both our models and NKSR, we record the reconstruction latency for all indoor scenes on a single \texttt{NVIDIA RTX 3090}, and for large outdoor scenes on a single \texttt{NVIDIA L40s} given that more GPU memory is required.
We omit data loading time, and only record the average forward pass time. 

\subsection{In-domain evaluation}
We compare against \nksr~(the current state-of-the-art), RangeUDF~\cite{wang2022rangeudf},  SPSR~\cite{kazhdan2013screened}, NDF~\cite{chibane2020ndf}, ConvOcc~\cite{peng2020convoccnet} and SA-CONet~\cite{tang2021SACon}.     
We further include a baseline that replaces our backbone with MinkowskiNet~\cite{choy20194d} (i.e., Ours~(Minkowski)) to show the degraded performance due to the information loss caused by voxelization.

\paragraph{Quantitative results -- \Cref{tab:indomain}}
Across indoor and outdoor datasets, our method outperforms baselines in terms of accuracy and time efficiency. Especially in outdoor datasets, our method achieves the best surface reconstruction with the smallest latency -- nearly \textit{half} of the second best's latency.
In indoor datasets, which have relatively uniform sampling patterns, we achieve accuracy on par with the previous state-of-the-art, but with significantly improved time efficiency.
Note that we achieve this advantage even with KNN because, in smaller indoor point clouds, the highly engineered KNN implementation has similar time efficiency to that of our neighborhood function.
We further detail our analysis on this matter in the \texttt{Supplementary Material}. 
We also note that our approximate neighborhood function is still effective, as it outperforms the directly comparable baseline MinkowskiNet~\cite{choy20194d}, which shares the same structure except for the backbone and neighborhood function.


\paragraph{Qualitative results -- \Cref{fig:qual_results_carla_syn,fig:scannet_results}}
We show that our method tends to reconstruct surfaces of the best quality among the compared methods.
Especially, on the non-uniform large scale \carla, our method tends to preserve more details than the previous state-of-the-art~\cite{huang2023neural}, which voxelizes the point cloud.   

\subsection{Cross-domain evaluation -- \Cref{tab:across_domain}}
We further test the generalization ability of our method with a cross-domain evaluation.
We evaluate models trained with dataset A on other a different dataset B; we denote this as~A $\rightarrow$ B. 
As shown in \Cref{tab:across_domain}, there are three cases in total.
In two cases (i.e., \synthetic $\rightarrow$ \scannet and \scannet $\rightarrow$ \synthetic), our method achieves the best accuracy with the best time efficiency. 
In another case (\scannet $\rightarrow$ \scenenn), we achieve accuracy on par with SOTA~\cite{huang2023neural} with a much better time efficiency, i.e., less than a half of the latency required by the SOTA~\cite{huang2023neural}.

\subsection{Ablation studies}
Our ablations are executed on \scannet, as it is a real-world dataset, and is equipped with precise ground truth surface meshes.

\begin{table}
\centering
\resizebox{.9\columnwidth}{!}{
\begin{tabular}{LccccccC}
\toprule
{\bf Neighbor Num.} & {CD (10\textsuperscript{-2})} $\downarrow$ & {F-score} $\uparrow$ & Latency (s) $\downarrow$ \\ \midrule
 2 & 0.246 & 99.56 & 109 \\
 4 & 0.244 & 99.59 & 127 \\
 \rowcolor{1st} 
8 & {0.243} & 99.61 & 151 \\
16 & 0.256 & 99.28 & 187 \\
\bottomrule
\end{tabular}
}
\caption{{\bf The impact of neighborhood size} -- larger neighborhoods lead to increased computational cost, and we find that 8 neighbors gives the best balance of cost and quality.}
\label{tab:numpts_neighbor}
\vspace{-1em}
\end{table}

\paragraph{Impact of neighborhood size -- \Cref{tab:numpts_neighbor}}
We analyze the impact of neighborhood size on performance. Larger neighborhood size leads to increased computation overhead. 
We show that the 8-nearest neighboring points gives the best trade-off between accuracy and time efficiency.
Considering a large number (e.g., 16) of neighboring points degrades performance as the the aggregation module has limited capacity to predict the precise SDF from a large local point cloud.

\begin{table}
\centering
\resizebox{.95\columnwidth}{!}{
\begin{tabular}{@{}lcccccc@{}}
\toprule
\makecell{\bf Num. of hidden\\\bf layers in $\aggregation$} & CD (10\textsuperscript{-2}) $\downarrow$ & F-score $\uparrow$ & Latency (s) $\downarrow$ \\ \midrule
 2 & 0.257 & 99.33 & 152 \\
 4 & 0.256 & 99.32 & 166 \\
\bottomrule
\end{tabular}
}
\caption{{\bf Impact of capacity of $\aggregation$} -- we find that increasing the number of layers in $\aggregation$ beyond 2 decreases time efficiency without substantially improving the reconstruction quality.}
\label{tab:agg_capacity}
\vspace{-1em}
\end{table}

\paragraph{Impact of capacity of $\aggregation$ -- \Cref{tab:agg_capacity}} 
We report how the capacity of the aggregation module $\aggregation$ (i.e., different number of hidden layers) impacts the performance.
We observe that aggregation modules of higher capacity give better performance but degraded time efficiency. However, as shown in~\Cref{tab:agg_capacity}, a very large capacity (4 layers) for $\aggregation$ does not help.
We show that we we use 2 layers to have a good trade-off between accuracy and time efficiency. 
We supplement~\Cref{tab:agg_capacity} with an analysis across even more levels in the \texttt{Supplementary Material}.

\begin{table}
\centering
\resizebox{.9\columnwidth}{!}{
\begin{tabular}{@{}lcccc@{}}
\toprule
\textbf{Num. of scales} &KNN & Minkowski & Z-order & Hilbert  \\ \midrule
0 & 1.00 & 0.17 & 0.44  & \cellcolor{1st}0.46  \\
1 & 1.00 & 0.29 & 0.48  & \cellcolor{1st}0.50  \\
2 & 1.00 & 0.38 & 0.49  & \cellcolor{1st}0.52  \\
3 & 1.00 & 0.44 & 0.49  & \cellcolor{1st}0.53  \\ %
\bottomrule
\end{tabular}
}
\caption{\textbf{Recall rate of our Hilbert-curve based $\neighbor$} -- we find that the Hilbert curve consistently outperforms both the Z-order curve~\cite{morton1966computer} and the one-ring neighborhood from Minkowski relative to the exact k-nearest neighbors.
}
\vspace{-1em}
\label{tab:locality_neighbor}
\end{table}

\paragraph{Analysis of neighbors retrieved by~$\neighbor$ -- \Cref{tab:locality_neighbor}}
\at{We now investigate the quality of the point neighborhoods retrieved by various possible implementations for $\neighbor$.
In particular, we are interested to experimentally study whether our serialization indeed preserves locality.
To quantify this, we treat the neighborhood retrieved with KNN as the ground-truth.}
We report the recall rate of a local neighborhood by comparing it with this ground truth~(we ignore the precision rate because we remove false positives with a distance threshold).
We also report the recall rate of the one-ring neighborhood retrieved in Minkowski~\cite{choy20194d}.
We show that the recall rate of our Hilbert $\neighbor$ is the best across variants, and across all scales.

\begin{table}[t]
\centering
\resizebox{\columnwidth}{!}{
\begin{tabular}{L rr rR}
\toprule
Methods & \multicolumn{2}{c}{Uniform} & \multicolumn{2}{c}{Non-Uniform}   \\ 
\cmidrule(r){1-1}
\cmidrule(lr){2-3}
\cmidrule(l){4-5}
\nksr & 0.246 & 480s & 0.273 & 668s  \\
Ours~(Minkowski)~\cite{choy20194d}  & 0.301 & 97s & 0.349 & 94s \\
Ours~(Minkowski)~\cite{choy20194d} {(w/ KNN)} & 0.254 & 145s & 0.294 & 155s \\
\rowcolor{1st} Ours~(w/ serialization) & {0.257} & {152s} & {0.296} & {145s} \\
\rowcolor{1st} Ours~(w/ KNN) & \textbf{0.243} & \textbf{151s} & \textbf{0.273} & \textbf{142s}  \\
\bottomrule
\end{tabular}
}
\caption{
\textbf{The impact of sampling} -- we evaluate uniform vs non-uniform sampling on ScanNet. We find that our method achieves the best accuracy (in terms of CD ($10^{-2}$)) and good time efficiency compared to \nksr~for both sampling types.
}
\vspace{-1em}
\label{tab:nonuniform_scannet}
\end{table}

\paragraph{The impact of sampling pattern --~\Cref{tab:nonuniform_scannet}} 
We report the impact of sampling pattern on performance by evaluating models on ScanNet point clouds that are uniformly or non-uniformly sampled. 
{To non-uniformly sample the ScanNet point clouds, we first partitioned the scene into eight blocks and randomly sampled a different number of points from each block. The number of samples followed an arithmetic sequence with a common difference of 200. Finally, we padded the last block to ensure that the total number of points remained 10K.}
 
We show that our method achieves better robustness to non-uniform sampling than the baselines, highlighting the importance of avoiding quantization of the point cloud for high quality surface reconstruction. 


%\section{Discussion}
\label{section:discussion}


\subsection{Practical Implications for Feedforward Prompting}

Of course, prompting an LLM continuously before the user submits their prompt is significantly most costly over submitting the prompt just once, once the user is ready.

% But user might not be ready, and the cognitive costs is pretty heavy.


\subsection{}


% Does this work well with Chain of Thought actually?
% Maybe this approach will actually incentivize self-prompt-chaining???
% What are the implications of this?


% A benefit of this is certainly more transparency in the LLM
% LLM is so flexible that adding this kind of structure is still okay for the LLM



% What's more costly, entering a prompt, then responding and saying, no i want this, or typing a prompt, and tuning the prompt/expected output to reduce message exchanges?

% Learning to become a better prompter. One is by trial and error experience. Perhaps another is through this feedforward that tells you what you might be able to anticipate.
\section{Concluding Remarks}
In this paper, we proposed a novel approach utilizing multimodal LLMs to generate gesture-aware speech recognition transcripts for patients with language disorders. Our framework integrates verbal speech and iconic gestures, enabling the generation of enriched transcripts that capture the latent meaning conveyed through both modalities. Through extensive experimentation, we demonstrated that the proposed method effectively contextualizes incomplete or disfluent speech by incorporating gesture information, leading to more accurate and meaningful representations of the speaker's intent. These findings highlight the potential of our approach to significantly contribute to the field of speech and language therapy, offering innovative tools that can enhance the quality of life for individuals with language disorders by facilitating better communication and assessment methods.

\subsection{Ethical Statement} 
Our dataset was obtained from AphasiaBank with the approval of the Institutional Review Board (IRB) and adheres to the data sharing guidelines set by TalkBank\footnote{https://talkbank.org/share/ethics.html}. This includes complying with the Ground Rules for all TalkBank databases, which are based on the American Psychological Association Code of Ethics~\cite{american2002ethical}.

\subsection{Limitation \& Future Work} 
%This study represents a preliminary investigation into using multimodal LLMs to generate gesture-aware speech recognition transcripts. 
While the results are promising, we recognize several limitations and outline our plans to extend this work further.

One primary limitation is the absence of a definitive ground truth for quantitative evaluation. Since our model generates transcripts by synthesizing speech and gesture data from scratch, traditional benchmarks, such as comparisons with standard speech recognition outputs, are insufficient. Moreover, existing original transcripts lack gesture annotations, making direct comparisons challenging. In future work, we aim to address this gap by collaborating with certified pathologists to conduct qualitative assessments, such as A-B preference tests, to evaluate the effectiveness of gesture-enriched transcripts in accurately conveying the speaker's intentions.

To support quantitative evaluations, we plan to develop novel metrics that assess transcript quality, including grammar accuracy, semantic consistency, and the integration of multimodal information. Such metrics will provide a more objective basis for assessing our model's performance and facilitate comparisons with other multimodal and unimodal approaches.

Another limitation of this study is its focus on structured gestures from a specific task, the Peanut Butter Sandwich Task. While this task offers a controlled context for testing our approach, it does not encompass the diversity of gestures and communication patterns seen in everyday scenarios. As part of our future work, we plan to expand the scope of our model to include tasks such as the Cinderella Story Recall Task~\cite{bird1996cinderella}, which involves unstructured and complex narrative gestures. This expansion will allow us to evaluate the adaptability and robustness of our model in handling varied linguistic and gestural contexts.

In summary, while this study establishes a strong foundation for gesture-aware speech recognition, we aim to refine and extend our methods through collaborative qualitative evaluations, the development of robust quantitative metrics, and broader task applications. These efforts will ensure that our approach continues to evolve, ultimately contributing to more effective communication tools and interventions for individuals with language disorders.




% Entries for the entire Anthology, followed by custom entries


\bibliography{custom}
%\bibliographystyle{acl_natbib}

%\onecolumn

\appendix

\section{Pre-annotation and classifiers}\label{app:pre}
\paragraph{Pre-annotation}
We start with the \textbf{Second Reading debates of Bills},\footnote{\url{https://www.parliament.uk/about/how/laws/passage-bill/commons/coms-commons-second-reading/}} where the members debate the main principles of a certain Bill. The advantages of using such debates are: (i) the stance of an argument can be easily identified based on whether they support %for 
the Bills; (ii) debates can be paired with brief Bill introductions,\footnote{e.g., the `long title' on page \url{https://bills.parliament.uk/bills/3858}} providing clear argument topics; and (iii) the arguments 
focus on Bill principles, with fewer discussions on specific amendments and clauses, which require less contextual awareness than other Bill debates like the ones for the Committee Stage.\footnote{\url{https://www.parliament.uk/about/how/laws/passage-bill/commons/coms-commons-comittee-stage/}}
We choose five Bills, including topics relevant to animal welfare and parental leave (see Table \ref{tab:bill} for the Bill introductions), 
which may be easier to annotate and more likely to have emotional arguments.

Three annotators label 245 texts from these debates for \textbf{three layers}: (\emph{L1}) 
whether the text evokes emotions, (\emph{L2}) whether the text contains standalone arguments, and (\emph{L3}) the stance of the text toward the Bill. \emph{L1} and \emph{L2} are labeled `0' (for answer `no') or `1' (for `yes'). If \emph{L2} is labeled `1', annotators proceed to label \emph{L3}, which has four options: `0' for support, `1' for opposition, `2' for inability to identify stance without additional context, and `3' for a neutral stance suggesting additional amendments or policies. Besides, 40 texts from the pilot annotation are also annotated for \emph{L1} and \emph{L2}.  
To potentially speed up the annotation process, the 285 texts are selected from those judged as both emotional and argumentative by GPT4o. Here, we prompt GPT4o with simple questions such as \emph{Does this text try to convince readers something?} and \emph{Is this text emotional?'}.

40 of the outputs are jointly labeled by all annotators, achieving average Cohen's Kappa of 0.622 for \emph{L1}, 0.674 for \emph{L2}, and 0.762 for \emph{L3} across annotator pairs. 
As shown in the `Question' column of Table \ref{tab:pre}, GPT4o already achieves a high precision of 0.82 in detecting argumentative texts using simple prompts. However, its precision for emotional text classification is still low (0.53).

We then convert the annotations for \emph{L3} to \emph{L3$^{*}$}, where we pair argument pairs based on their topics and stances. The categories include: `different topic' for pairs with different topics (from different Bills), `different stance' for pairs with the same topic but different stances, and `same' for pairs with the same topic and stance.

The number of texts annotated for each layer and the corresponding label distribution\se{s} are summarized in Table \ref{tab:pre} (left). 

\paragraph{Automatic Pipeline}
We develop three classifiers based on GPT4o 
to automatically identify the argument pairs needed. The pipeline is as follows: 
\begin{enumerate}[]
    \item \textbf{Argumentative text classification}: our goal is to have a \textbf{high precision} classifier since we have sufficient candidate texts. We find that when we ask GPT-4o to provide the major claim, evidence, and reasoning connecting the evidence to the major claim in the text, its precision increases from 0.82 to 0.96, as shown in the `Argumentative' row of Table \ref{tab:pre}. 
    
    We then retain texts judged as argumentative for \hansard{} using this prompt, while for \deuparl{}, we use a German translation of the same prompt. The overall performance of GPT4o on German data is assessed after completing the stance agreement classification task (see below).

    \item \textbf{Stance agreement classification}: 
    To enable the flexible selection of classifiers with specific performance characteristics (e.g., high recall, high precision), we introduce a parameter into the prompt, with its threshold optimized to achieve different specialized performance levels.
    To do so, we ask GPT4o to rate the likelihood that two given arguments address the same topic and share the same stance on a Likert scale from 0 to 100. We randomly sample 600 argument pairs (with a 2:1:1 ratio for the three categories of \emph{L3$^{*}$}) from the dataset, ‘optimize’ the threshold of ratings for the `same’ category 
    using argument pairs from two Bills, and test the performance on the remaining three Bills to prevent data leakage. We evaluate all possible combinations of Bills for the training and test sets.
    We observe that as the threshold increases, precision on the `same’ category ($P_{same}$) consistently improves, while macro F1 begins to decrease beyond certain thresholds. With a threshold of 100, $P_{same}$ reaches 0.92, but F1 is very low at 0.45. Therefore, we select a threshold of 90 as a more balanced trade-off, achieving $P_{same}=0.81$ and $\textit{F1}=0.76$, to obtain more candidates that are still highly likely to be true positives. 
    
    For \hansard{}, we retain the argument pairs labeled as belonging to the `same' category using this threshold. For \deuparl{}, we apply the German translation of the prompt with the same threshold to identify argument pairs. One annotator evaluates 50 candidates from the outputs of steps 1 and 2: no argument is labeled as non-argumentative, while 12 argument pairs are identified as false positives in the stance agreement task, yielding $P_{same}=0.76$. This value is only 4 percent points lower than the result on English data. Consequently, we retain these prompt settings for the German data.
    
    \item  \textbf{Emotional text classification}: we aim for a \textbf{balanced} classifier because we also need non-emotional arguments. Since this is a subjective task, we ask GPT4o to rate how likely it can feel the emotions 
    in the texts on a \se{L}ikert scale of 0-100, and then `optimize' the threshold of the rates for the `emotional' category on 70\% of the data and check how it performs on the remaining 30\%. Overall, with this step, we can improve the macro F1 to 0.74-0.81 (averaged over three rounds of data splitting), depending on the gold from different annotators. The best threshold for two annotators is 75, while that for the other is 85, so we use the threshold 75 to represent the majority, which has a macro F1 of 0.75, averaged across the three annotators. 

    We use this threshold to select the argument pairs for \hansard{}. For \deuparl{}, we further optimize the threshold using a small-scale set of human annotations and adjust it to 85. This setting is then used to label the binary emotions of arguments. 

\end{enumerate}

\begin{table}[!ht]
\resizebox{\linewidth}{!}{%
\begin{tabular}{@{}lcccc@{}}
\toprule
                               & \multicolumn{2}{c}{Pre-Annotation} & \multicolumn{2}{c}{Automatic Pipeline}   
                               \\
                               & \#                 & \%  & Question  & `Optimized' \\ \midrule
\multicolumn{3}{l}{\emph{L1 - emotion} }                                             \\ \midrule
Emotional                      & 151                & 53.0 & 0.53 (P)  & \multirow{2}{*}{0.75 (F1)} \\ 
Non-emotional                  & 134                & 47.0 & -  \\ \bottomrule
\multicolumn{3}{l}{\emph{L2 - argument}}                                         \\ \midrule
Argumentative                  & 234                & 82.1 & 0.82 (P) & 0.96 (P)  \\
Non-argumentative              & 51                 & 17.9 & - & - \\ \bottomrule
\multicolumn{3}{l}{\emph{L3 - stance} }                                           \\ \midrule    
Support                        & 170                & 72.6 & - \\
Opposition                     & 2                  & 0.9 & -  \\
Neutral                        & 29                 & 12.4  & -\\
Irrelevant                     & 16                 & 14.1& - \\ \midrule
\multicolumn{3}{l}{\emph{L3$^{*}$ - pair stance}}                                            \\ \midrule   
Same           & 2,905            & 8.9  & -   & \multirow{3}{*}{\makecell{0.80 ($P_{same}$) \\ 0.75 (F1)}} \\
Different stance & 3,325              & 10.2 & - \\
Different topic                & 26,486             & 81.0 & - \\ \midrule
Total                          & 32,716             & 100  & - \\ \bottomrule
\end{tabular}}
\caption{Number of texts annotated for each layer and category (\#) and the corresponding label distribution (\%). Performance of GPT4o on the binary emotion classification, argument identification, and stance agreement detection tasks used for automatically identifying the target argument pairs.}\label{tab:pre}
\end{table}


\begin{table*}[!ht]
\resizebox{\linewidth}{!}{
\begin{tabular}{@{}l@{}}
\toprule
\emph{Introduction}                                                                                                                                                                         \\ \midrule
A Bill to Prohibit the export of certain livestock from Great Britain for slaughter.                                                                                                 \\ \midrule
\makecell[l]{A Bill to create offences of dog abduction and cat abduction and to confer a power to make corresponding provision  \\ relating to the abduction of other animals commonly kept as pets.} \\ \midrule
A Bill to make provision about leave and pay for employees with responsibility for children receiving neonatal care.                                                                   \\ \midrule
A Bill to prohibit the import and export of shark fins and to make provision relating to the removal of fins from sharks.                                                            \\ \midrule
A Bill to prohibit the sale and advertising of activities abroad which involve low standards of welfare for animals.                                                                 \\ \bottomrule
\end{tabular}}
\caption{The introductions of the five Bills selected in \protect\bill{}.}\label{tab:bill}
\end{table*}


% Please add the following required packages to your document preamble:
% \usepackage{booktabs}
\begin{table*}[]
\resizebox{\linewidth}{!}{
\begin{tabular}{@{}l|l@{}}
\toprule
English                                                                                                                                                                                  & German                                                                                                                                                                                     \\ \midrule
\makecell[l]{iran, integrat, ukraine, russia, asylum,\\ deportation, israel, gaza, expulsion, \\ displacement, migration, migrant, \\immigrant, refugee, palestine,invasion,\\ repatriation, hamas, hisbollah} & \makecell[l]{ukraine, russland, migrant, \\ immigrant, flüchtling, asyl,\\ gaza, iran, palästina, \\israel, krieg, invasion, \\sanktionen, waffenlieferungen, friedensverhandlungen, \\kriegsverbrechen, flüchtlingskrise, nato,\\ energieversorgung, vertreibung, migrationspolitik,\\ asylverfahren, grenzsicherung, integration, \\abschiebung, aufenthaltsgenehmigung, menschenhandel, \\seenotrettung, rückführung, schutzstatus, \\waffenstillstand, raketenangriffe, besatzung, \\zwei-staaten-lösung, friedensprozess, intifada, \\ hamas, hisbollah, menschenrechte, un-resolution
} \\ \bottomrule
\end{tabular}
}
\caption{Keywords used to filter debates for \hansard{} and \deuparl{}.}\label{tab:keywords}
\end{table*}



% Please add the following required packages to your document preamble:
% \usepackage{booktabs}
\begin{table*}[]
\centering
%\resizebox{!}{\linewidth}{
\begin{tabularx}{\linewidth}{X}
\toprule
\emph{Remove Emotion Prompt}  \\ \midrule
====\textbf{System Prompt}=====\\ I will give you an argumentative text that **can** appeal to emotion.    \\ \\ Your task is to generate an argument with the same stance for the same topic **without emotional language**, by rephrasing the text but maintaining a similar style and length. \\ \\ Briefly explain why the rewritten argument no longer evokes emotions.\\ \\ Answer in the following way:\\ Generated argument: \\ Explanation:\\ ====\textbf{User Prompt}=====\\ Text: \{original argument\}  \\ \midrule
\emph{Add Emotion Prompt}  \\ \midrule
====\textbf{System Prompt}=====\\ I will give you an argumentative text that **cannot** appeal to emotion.\\     \\ Your task is to generate an argument with the same stance on the same topic **with emotions**, by rephrasing the text but maintaining a similar style and length. \\ \\ Briefly explain why the rewritten argument can evoke emotions now.\\ \\ Answer in the following way:\\ Generated argument: \\ Explanation:\\ ====\textbf{User Prompt}=====\\ Text: \{original argument\}                \\ \bottomrule
\end{tabularx}
%}
\caption{Prompts used to remove/add emotions for synthetic arguments.}\label{tab:prompt_synthetic}
\end{table*}


\section{Arguments from others}\label{app:other}
\paragraph{\dagstuhl{}} 
\citet{wachsmuth-etal-2017-computational} collected human ratings on a Likert scale of 1–3 for multiple dimensions of argument quality, including argument effectiveness (convincingness)\footnote{“Argumentation is effective if it persuades the target audience of (or corroborates agreement with) the author’s stance on the issue.” — \citet{wachsmuth-etal-2017-computational}} and emotional appeal. These ratings were applied to 304 argumentative texts from \citet{habernal-gurevych-2016-argument}, which were sourced from a textual debate portal in \textbf{English}. We retain only those arguments whose average convincingness rating (across the three annotators) exceeds 1.5. 
Next, we pair arguments that share the same stance on the same topics and calculate the absolute differences in their emotional appeal ratings. From these pairs, we randomly select 10 topics and then retain the 5 argument pairs with the largest absolute differences in emotional appeal for each topic.

\paragraph{\lynn{}}
\citet{greschner2024fearfulfalconsangryllamas} collected
discrete emotion labels from a reader respective (e.g. joy, disgust etc.) for 300 \textbf{German} arguments associated with 30 statements, drawn from \citet{velutharambath_wuehrl_klinger_2024a}. Each argument was annotated by three annotators. We interpret the number of annotations marking the argument as containing specific emotions (rather than `no emotion') as its emotion score. E.g., if three annotators identify specific emotions in the argument, its emotion score would be 3. Using a procedure similar to the one employed for \dagstuhl{}, we pair arguments referencing the same statement, randomly select 25 statements, and then retain the 
two argument pairs per statement that exhibit the greatest differences in emotion scores. 


\begin{table}[]
\centering
\resizebox{\linewidth}{!}{
\begin{tabular}{@{}llllll@{}}
\toprule
     & \dagstuhl{} & \bill{} & \hansard{} & \lynn{} & \deuparl{} \\ \midrule
\multicolumn{6}{l}{\emph{Increase}}                               \\ \midrule
\rz{}   & \textbf{-0.06}   & \textbf{0.15}         & \textbf{0.05}   & \textbf{-0.38}  & \textbf{0.32}   \\
\rth{} & -0.18    & -0.21        & -0.31  & -0.46   & -0.38  \\ \midrule
\multicolumn{6}{l}{\emph{Decrease}    }                        \\ \midrule
\ro{} & -0.12 & -0.21 & -0.03 & 0.08 & -0.19    \\
\rtw{} & \textbf{0.36}    & \textbf{0.27}         & \textbf{0.29}   & \textbf{0.76}   & \textbf{0.25} \\ \bottomrule
\end{tabular}}
\caption{BWS scores for the 4 argument groups: \rz{}, \ro{}, \rtw{} and \rth{}, derived from the majority votes of the annotation for pairwise comparisons of emotional intensity. `Increase'/`Decrease' denotes the direction to increase/decrease the perceived emotional intensity.}\label{tab:bws}
\vspace{-.3cm}
\end{table}

\section{Prompts}\label{app:prompt}
Table \ref{tab:prompt_synthetic} presents the prompts used to introduce/remove emotions. Table \ref{tab:promptc} illustrates the prompts used for evaluating argument convincingness.

% Please add the following required packages to your document preamble:
% \usepackage{booktabs}
\begin{table*}
\footnotesize
\resizebox{\linewidth}{!}{
\begin{tabular}{@{}cl@{}}
\toprule
\multicolumn{2}{l}{Prompt Template}                                                                                                                                                                                                                                                                                                                                                                                                                                                                                                                                                                                                                                                                                     \\ \midrule
Shared & \begin{tabular}[c]{@{}l@{}}Below, you will find one pair of argumentative texts discussing the same topic with the same stance. The topic may be a binary \\choice, a bill from UK parliamentary debates, or a simple statement. Both arguments either support or oppose the topic, or they \\favor one side if the topic involves a binary choice.\\ \\ Your task is to evaluate each pair to determine **which argumentative text you find more convincing**. There are three label options:\\ 0 (Both arguments are equally convincing.)\\ 1 (Argument 1 is more convincing.)\\ 2 (Argument 2 is more convincing.)\\ \\ **Note**: Truncated sentences or grammatical errors should be **ignored**.\end{tabular} \\ \midrule
1      & \begin{tabular}[c]{@{}l@{}}Please answer your label option **without** any explanations.\\ \\ \{text\}\end{tabular}                                                                                                                                                                                                                                                                                                                                                                                                                                                                                                                                                                                            \\ \midrule
2      & \begin{tabular}[c]{@{}l@{}}Please answer your label option and briefly explain why you choose this label.\\ \\ \{text\}\\ \\ Below is an example answer for you; please follow this format in your response.\\ Label: 2\\ Explanation: because Argument 2 provides more statistics supporting the claim, while Argument 1 contains logical fallacies.\end{tabular}                                                                                                                                                                                                                                                                                                                                             \\ \midrule
3      & \begin{tabular}[c]{@{}l@{}}Please answer your label option and briefly explain why you choose this label.\\ \\ \{text\}\\ \\Below is an example answer for you; please follow this format in your response.\\ Label: 1\\ Explanation: Argument 1 is more convincing, because I totally agree with its point and it evokes my empathy.\end{tabular}                                                                                                                                                                                                                                                                                                                                                                                                                                                            \\ \bottomrule
\end{tabular}}
\caption{Prompt templates for comparing the convincingness of an argument pair. The {text} field contains the two arguments and their topic. The complete prompt is formed by combining the text in the `Shared' row with the text in the corresponding indexed row. For example, Prompt 1 consists of the text from both the `Shared' row and row `1'.}\label{tab:promptc}
\end{table*}



\begin{table}[]
\vspace{-.2cm}
\centering
\setlength\tabcolsep{2pt} 
\resizebox{\linewidth}{!}{%
\begin{tabular}{@{}lcc|cccccc@{}}
\toprule
& \multicolumn{2}{c|}{\textbf{\#Annotators}} & \multicolumn{6}{c}{\textbf{Agreements}} \\
              & \textbf{S} &  \textbf{C} & \multicolumn{3}{c}{\textbf{EMO}} & \multicolumn{3}{c}{\textbf{CONV}} \\ %\midrule
              &&& $\alpha$ & Full & Maj. & $\alpha$ & Full & Maj. \\ \midrule
\dagstuhl{}      & 1        & 4         &  0.506  & 6.5\% & 74.5\%   & 0.540  & 14.0\% & 80.0\%   \\
\bill{} & 1        & 4        &  0.449  & 7.0\% & 76.5\%   & 0.463  & 10.5\% & 78.0\%   \\
\hansard{}       & 1        & 4        &  0.361 & 0.5\% & 68.0\%   & 0.371  & 6.0\% & 75.0\%   \\
\lynn{}       & 2        & 3       &  0.729 & 13.5\% & 87.5\%   & 0.607  & 16.0\% & 82.0\%   \\
\deuparl{}       & 3        & 2       & 0.352  & 8.0\% & 80.5\%   & 0.364  & 4.5\% & 74.5\%   \\ \midrule
Avg    & -        & - & 0.479 & 7.1\% & 77.4\% & 0.469 & 10.2\% & 77.9\% \\ 
\bottomrule
\end{tabular}}
\caption{\textbf{Left}: Number of student (S) and crowdsourcing (C) annotators per batch. \textbf{Right}: Krippendorf's $\alpha$ for the most agreeing annotator pairs (\textbf{$\alpha$}), the percentages of annotation instances where all annotators agree on a certain label (\textbf{Full}),  and the percentage of annotation instances where at least three annotators agree on a certain label (\textbf{Maj.}). 
}
\label{tab:annotator}
\vspace{-.6cm}
\end{table}


\section{Annotation Interface}\label{app:anno}
Figure \ref{fig:anno} shows the screenshots of the annotation interface for convincingness (top) and emotion (bottom) comparisons. We collect the annotations via Google Forms\footnote{\url{https://docs.google.com/forms/}} for crowdsourcing annotators.

\begin{figure*}
    \includegraphics[width=\linewidth]{structure/figs/anno/conv_form.pdf}
    \includegraphics[width=\linewidth]{structure/figs/anno/emo_form.pdf}
    \caption{Screenshots of the annotation interface for convincingness (top) and emotion (bottom) comparison.}\label{fig:anno}
\end{figure*}


\section{Examples}\label{app:exa}
Table \ref{tab:pos} and \ref{tab:neg} provide example instances from \hansard{} and \lynn{}, where emotions have a positive and negative impact, respectively. 

\begin{table*}[!ht]
\centering
\begin{tabularx}{\textwidth}{ X | X }
\toprule
\multicolumn{2}{l}{\textbf{Topic}: The public supports the UK's aid for Ukrainian refugees} \\ \midrule
\rz{}  & \ro{}  \\ \midrule
Members across this House are determined that we, as a country, should open our arms to these people, and this determination has been on full display today. The scenes of devastation and human misery inflicted by President Putin’s barbarous assault on what he calls “Russia’s cousins” in Ukraine have unleashed a tidal wave of solidarity and generosity across the country. British people always step forward and step up in these moments, and since the first tanks rolled into Ukraine, they have come forward in droves with offers of help: community centres have been flooded with critical supplies; the Association of Ukrainians in Great Britain has received millions in donations; and charities such as the Red Cross have been overwhelmed with people giving whatever they can. The outpouring of public support has been nothing short of remarkable. & While this Government, and this whole House, have risen to the occasion with our offer of support to Ukrainians fleeing war, our lethal aid and our stranglehold on economic sanctions on Russia have clearly shown that we will keep upping the ante to ensure that Putin fails. As Members have argued today, it has been abundantly clear in recent days that we can and must do more. It is exactly right, therefore, that my right hon. Friend the Secretary of State for Levelling Up, Housing and Communities set out on Monday the new and uncapped sponsorship scheme, Homes for Ukraine. It is a scheme to allow Ukrainians with no family ties to the UK to be sponsored by individuals or organisations that can offer them a home. It is a scheme that draws not only on the exceptional good will and generosity of the British people, but one that gives them the opportunity to help make a difference.                                                                                                                                                        \\ \midrule
\rth{}   & \rtw{} \\ \midrule
Members of this House have expressed a commitment to welcoming individuals from Ukraine. The recent conflict initiated by President Putin has resulted in significant destruction in Ukraine, prompting a substantial response of support across the country. British citizens have actively contributed since the conflict began, with community centers collecting essential supplies, the Association of Ukrainians in Great Britain receiving financial contributions, and charities like the Red Cross witnessing increased donations.  & In these trying times, the Government and this entire House have demonstrated unwavering courage and compassion by extending our support to Ukrainians escaping the horrors of war. Our determined provision of lethal aid and the relentless imposition of economic sanctions on Russia are powerful affirmations that we will stop at nothing to ensure Putin's defeat. As Members have passionately discussed today, the urgency to do even more has never been clearer. That is why it is so heartening that my right hon. Friend the Secretary of State for Levelling Up, Housing and Communities announced on Monday the new and limitless Homes for Ukraine sponsorship scheme. This initiative opens its arms to Ukrainians without family connections in the UK, allowing them to be warmly embraced by individuals or organizations ready to offer them a sanctuary. It is a testament not only to the extraordinary kindness and generosity of the British people but also to their deep desire to make a meaningful impact in the lives of those in desperate need. \\ \bottomrule
\end{tabularx}
\caption{An example instance from \hansard{} where emotions have a \textbf{positive} impact on argument convincingness.
}\label{tab:pos}
\end{table*}

\begin{table*}[!ht]
\centering
\begin{tabularx}{\textwidth}{ X | X }
\toprule
\multicolumn{2}{l}{Topic: Haie können Krebs bekommen.}    \\ \midrule
\rz{}  & \ro{}  \\  \midrule
Haie sind mehrzellige Lebewesen, wie auch der Mensch. Die Beonderheit von mehrzelligen Lebewesen ist, dass die Zellen sich sowohl stark spezialisieren und untereinander vernetz kommunizieren. Damit werden sie anfällig für bestimmte Zelldefekte, die sich über die genannte Struktur fortpflanzen und den Krebs ausmachen. Haie verfügen, wie auch der Mensch und überhaupt alle mehrzelligen Lebewesen, über nur eine sehr eingeschränkte Möglichkeit diese Defekte zu korrigieren und aufzuhalten, damit können beide gleichermaßen Krebs bekommen & Da auch Fische Krebs bekommen können, ist es auch möglich, dass Haie Krebs bekommen können. Dieser wird durch mutierte Zellen ausgelöst, weshalb dies auch bei Fischarten ausgelöst werden kann. Krebs ist eine weit verbreitete und häufige Krankheit, weshalb Krebs durch Wissenschaftler auch bereits bei Haien festgestellt werden konnte.\\ Krebs kann außerdem auch durch verschiedene Umweltfaktoren wie Umweltverschmutzung ausgelöst werden, diesem Risiko sind Haie ja durchaus ausgesetzt. Deshalb ist die Gefahr einer Erkrankung auch nicht gerade gering.  \\ \midrule
\rth{}  & \rtw{}  \\ \midrule
Haie, ebenso wie Menschen, sind mehrzellige Organismen. Eine charakteristische Eigenschaft solcher Organismen ist die Spezialisierung und Vernetzung ihrer Zellen. Diese Struktur macht sie anfällig für Zellfehler, die sich ausbreiten und zu Krebs führen können. Haie und Menschen besitzen nur begrenzte Mechanismen zur Korrektur und Kontrolle dieser Defekte, was bedeutet, dass beide Arten gleichermaßen anfällig für Krebs sind.  & Die Vorstellung, dass Haie - diese majestätischen und oft missverstandenen Kreaturen der Meere - an Krebs erkranken können, ist zutiefst beunruhigend. Diese Krankheit, die durch die heimtückische Mutation von Zellen verursacht wird, hat bereits viele Fischarten heimgesucht. Die Tatsache, dass auch Haie, die Könige der Ozeane, nicht sicher vor dieser grausamen Krankheit sind, ist erschütternd. Angesichts der weit verbreiteten Umweltverschmutzung, die unsere Ozeane verschlingt, sind Haie einem erheblichen Risiko ausgesetzt, an Krebs zu erkranken. Es ist traurig und alarmierend, dass diese beeindruckenden Tiere, die seit Millionen von Jahren die Meere durchstreifen, nun durch menschliche Einflüsse bedroht sind.
\\ \bottomrule
\end{tabularx}
\caption{An example instance from \lynn{} where emotions have a \textbf{negative} impact on argument convincingness.
}\label{tab:neg}
\end{table*}




\section{LLM}\label{app:llm}
Figure \ref{fig:dis_prompt} illustrates the consistency, positivity and negativity rates of LLMs with different prompts, averaged across instances in all datasets. Table \ref{tab:llm_ranking} displays macro F1 scores and model rankings for LLMs in predicting convincingness rankings of argument pairs ('Static') and the resulting categories of emotional effect (`Dynamic') in English and German.

\begin{figure*}[]
    \centering
    \includegraphics[width=\linewidth]{structure/figs/llm/dis_prompt1.pdf}
    \includegraphics[width=\linewidth]{structure/figs/llm/dis_prompt2.pdf}
    \includegraphics[width=\linewidth]{structure/figs/llm/dis_prompt3.pdf}
    \caption{Consistency, positivity and negativity rates of LLMs with different prompts, averaged across instances in all datasets.}\label{fig:dis_prompt}
\end{figure*}


\begin{table*}[]
\resizebox{\linewidth}{!}{
\begin{tabular}{lcccc|cccc}
\toprule
                           & \multicolumn{4}{c|}{\textbf{EN}}               & \multicolumn{4}{c}{\textbf{DE}}               \\ 
\textbf{Model}                      & Static & Ranking & Dynamic & Ranking & Static & Ranking & Dynamic & Ranking \\ \midrule
gpt-4o-2024-08-06          & \textbf{0.486}  & 1       & 0.411   & 2       & \textbf{0.443}  & 1       & \textbf{0.447}   & 1       \\
Llama-3.3-70B-Instruct     & 0.417  & 2       & \textbf{0.415}   & 1       & 0.372  & 2       & 0.392   & 4       \\
gpt-4o-mini                & 0.416  & 3       & 0.392   & 5       & 0.35   & 4       & 0.394   & 3       \\
Qwen2.5-72B-Instruct       & 0.398  & 4       & 0.398   & 4       & 0.357  & 3       & 0.41    & 2       \\
gpt-3.5-turbo              & 0.39   & 5       & 0.382   & 6       & 0.338  & 6       & 0.381   & 6       \\
Mixtral-8x7B-Instruct-v0.1 & 0.368  & 6       & 0.376   & 7       & 0.35   & 5       & 0.387   & 5       \\
Mistral-7B-Instruct-v0.3   & 0.367  & 7       & 0.407   & 3       & 0.288  & 8       & 0.36    & 9       \\
Llama-3.2-3B-Instruct      & 0.322  & 8       & 0.32    & 10      & 0.281  & 10      & 0.367   & 8       \\
Qwen2.5-0.5B-Instruct      & 0.308  & 9       & 0.342   & 9       & 0.284  & 9       & 0.344   & 10      \\
Qwen2.5-7B-Instruct        & 0.304  & 10      & 0.346   & 8       & 0.319  & 7       & 0.373   & 7       \\
Llama-3.2-1B-Instruct      & 0.286  & 11      & 0.309   & 11      & 0.274  & 11      & 0.343   & 11 \\ \bottomrule     
\end{tabular}}
\caption{Macro F1 scores and model rankings for LLMs in predicting convincingness rankings of argument pairs ('Static') and the resulting categories of emotional effect (`Dynamic') in English and German. For each model, we present the best prompt result to highlight its potential. Human and LLM labels are determined by majority votes from different annotators and rounds, respectively.}\label{tab:llm_ranking}
\end{table*}



\end{document}
