\section{\ADDAND: A New Tractable Representation}
\label{sec:ADDAND}

In order to compute the Shannon entropy of a circuit formula $\varphi(X, Y)$, we need to use the probability distribution over the outputs.
Algebraic Decision Diagrams (ADDs) are an influential compact probability representation that can be exponentially smaller than the explicit representation.
Macii and Poncino~\cite{macii1996exact} showed that ADD supports efficient exact computation of entropy.
However, we observed in the experiments that the sizes of ADDs often exponentially explode with large circuit formulas.
We draw inspiration from a Boolean representation known as the Ordered Binary Decision Diagram with conjunctive decomposition (\OBDDAND)~\cite{lai2017new}, which reduces its size through recursive component decomposition and divide-and-conquer strategies. 
This approach enables the representation to be exponentially smaller than the original OBDD.
Accordingly, we propose a probabilistic representation called Algebraic Decision Diagrams with conjunctive decomposition (\ADDAND) and show it supports tractable entropy computation.
\ADDAND is a general form of ADD and is defined as follows:

\begin{definition}\label{ADDAND-definition}
An \ADDAND is a rooted DAG, where each node $u$ is labeled with a symbol $sym(u)$.
If $u$ is a terminal node, $sym(u)$ is a non-negative real weight, also denoted by $\omega(u)$; otherwise, $sym(u)$ is a variable (called \emph{decision} node) or operator $\wedge$ (called \emph{decomposition} node). 
The children of a decision node $u$ are referred to as the \emph{low} child $lo(u)$ and the \emph{high} child $hi(u)$, and connected by dashed lines and solid lines, respectively, corresponding to the cases where $\mathit{var}(u)$ is assigned the value of $\mathit{false}$ and  $\mathit{true}$. 
For a decomposition node, its sub-graphs do not share any variables.
An \ADDAND is imposed with a linear ordering $\prec$ of variables such that given a node $u$ and its non-terminal child $v$, $\mathit{var}(u) \prec \mathit{var}(v) $.


%An \ADDAND is a rooted DAG, where each node $u$ is either terminal or non-terminal.
%Each terminal node $u$ is labeled with a real weight $\omega(u)$ and a set of implied literals $L(u)$.
%Each non-terminal node $u$ is associated with a variable $\mathit{var}(u)$, two children, and a set of implied literals $L(u)$, which satisfies that $\mathit{var}(u)$ does not appear in $L(u)$ and no variable within $L(u)$ appears in any descendant nodes of $u$.
%The children of non-terminal node $u$ are referred to as the \textit{low} child $lo(u)$ and the \textit{high} child $hi(u)$, and connected by dashed lines and solid lines, respectively, corresponding to the cases where $\mathit{var}(u)$ is assigned the value of $\mathit{false}$ and  $\mathit{true}$. 
%An ADD-L is imposed with a linear ordering $\prec$ of variables such that given a node $u$ and its non-terminal child $v$, $\mathit{var}(u) \prec \mathit{var}(v) $.
\end{definition}

Hereafter, we denote the set of variables that appear in the graph rooted at $u$ as $\mathit{Vars}(u)$ and the set of child nodes of $u$ as $Ch(u)$.
We now turn to show that how an \ADDAND defines a probability distribution:
 
\begin{definition}\label{def:ADDAND-weight}
	Let $u$ be an \ADDAND node over a set of variables $Y$ and let $\sigma$ be an assignment over $Y$. 
    The weight of $\sigma$ is defined as follows:
	\begin{equation*}
		   \omega(\sigma,u) =  
		   \begin{cases}  
			   \mathit{\omega}(u) & \text{terminal} \\  
				\prod_{v \in Ch(u)}{\omega(\sigma,v)} & \text{decomposition} \\  
	            \omega(\sigma,lo(u)) & \text{decision and $\sigma \models  \lnot \mathit{var}(u)$}  \\
	            \omega(\sigma,hi(u)) & \text{decision and $\sigma \models  \mathit{var}(u)$}  \\
		   \end{cases}
	\end{equation*}
    The weight of an non-terminal \ADDAND rooted at $u$ is denoted by $\omega(u)$ and defined as $\sum_{\sigma \in 2^{\mathit{Vars}(u)}}\omega(\sigma,u)$.
    The probability of $\sigma$ over $u$ is defined as $p(\sigma, u) = \frac{\omega(\sigma,u)}{\omega(u)}$.
\end{definition}


Figure \ref{fig:ADDAND-Example} depicts an \ADDAND representing the probability distribution of $\varphi_n^{sep}$ in Example \ref{circuit-example} over its outputs wrt $y_1 \prec y_2 \prec \cdots \prec y_{2n}$. The reader can verify that each equivalent ADD wrt $\prec$ has an exponential number of nodes.
In the knowledge compilation field \cite{darwiche2002knowledge,fargier2014knowledge}, we often uses the notion of succinctness to describe the space efficiency of a representation. 
Due to the following observations, we can see that \ADDAND is strictly more succinct than ADD: OBDD and \OBDDAND are subsets of ADD and \ADDAND, respectively, \OBDDAND is strictly more succinct than OBDD ~\cite{lai2017new}; and each \OBDDAND cannot be represented into a non-OBDD ADD.
%We will formally prove this property in the appendix.





\begin{figure}[h]
\vspace{1.5cm} % 在图形上方添加垂直空间
\resizebox{\linewidth}{!} {
\begin{forest}
	for tree={
		if n children=0{circle, draw, inner sep=2pt}{}, % 叶子节点的样式
		if level=1{edge={draw, solid}}{% 如果是根节点的子节点,边为实线
			if n=1{edge={draw, dashed}}{edge={draw, solid}} % 对于其他节点,第一个子节点为虚线,第二个为实线
		}
	}
	[$\wedge$, circle, draw, minimum width=1cm,
	label={[blue,above=0.01cm]above:$\overbrace{(-\frac{3}{4} \cdot \log\frac{3}{4} - \frac{1}{4} \cdot \log\frac{1}{4}) + \cdots + (-\frac{3}{4} \cdot \log\frac{3}{4} - \frac{1}{4} \cdot \log\frac{1}{4})}^{n}$}
	[$y_1$, circle, draw, minimum width=1cm,
	label={[blue,above=0.1cm]above:$-\frac{3}{4} \cdot \log\frac{3}{4} - \frac{1}{4} \cdot \log\frac{1}{4}$},
	name = y1
	[$y_{n+1}$, circle, draw, minimum width=1cm,
	label={[blue,right=0.1cm]right:$0$}
	[$3$, circle, draw, minimum width=1cm,
	label={[blue,below=0.1cm]below:$0$}]
	[$0$, circle, draw, minimum width=1cm,
	label={[blue,below=0.1cm]below:$0$}]
	]
	[$y_{n+1}$, circle, draw, minimum width=1cm,
	label={[blue,right=0.1cm]right:$0$}
	[$0$, circle, draw, minimum width=1cm,
	label={[blue,below=0.1cm]below:$0$}]
	[$1$, circle, draw, minimum width=1cm,
	label={[blue,below=0.1cm]below:$0$}]
	]
	]
	% y_n 节点
	[$y_n$, circle, draw, minimum width=1cm,
	label={[blue,above=0.1cm]above:$-\frac{3}{4} \cdot \log\frac{3}{4} - \frac{1}{4} \cdot \log\frac{1}{4}$}, name = yn
	[$y_{2n}$, circle, draw, minimum width=1cm,
	label={[blue,right=0.1cm]right:$0$}
	[3, circle, draw, minimum width=1cm,
	label={[blue,below=0.1cm]below:$0$}]
	[0, circle, draw, minimum width=1cm,
	label={[blue,below=0.1cm]below:$0$}]
	]
	[$y_{2n}$, circle, draw, minimum width=1cm,
	label={[blue,right=0.1cm]right:$0$}
	[0, circle, draw, minimum width=1cm,
	label={[blue,below=0.1cm]below:$0$}]
	[1, circle, draw, minimum width=1cm,
	label={[blue,below=0.1cm]below:$0$}]
	]
	]
	]
	\begin{tikzpicture}[overlay]
		% 从 y_2 节点的底部到 y_n 节点的顶部绘制一条路径,并在中间添加省略号
		\path (y1.south) -- (yn.north) node[pos=0.5, fill=white] {$\cdots$};
	\end{tikzpicture}
\end{forest}
}




\caption{An \ADDAND representing the probability distribution of $\varphi_n^{sep}$ in Example \ref{circuit-example} over its outputs. 
According to Proposition \ref{prop:Entropy-proposition}, the computed entropy for each node is marked in blue font.
}
\label{fig:ADDAND-Example}
\end{figure}


\subsection{Tractable Computation of Weight and Entropy}
	%\left| \mathit{Sol}(\varphi(Y \mapsto \sigma)) \right| & \text{$u$ is terminal node and $\sigma \models L(u)$} \\ 
	
	The computation of Shannon entropy of \ADDAND depends on the computation of its weight. 
    We first show that for an \ADDAND node $u$, we can compute its weight $\omega(u)$ in polynomial time.
	\begin{proposition}\label{prop:omega-proposition}
		Given a non-terminal node $u$ in \ADDAND, its weight $\mathit{\omega}(u)$ can be recursively computed as follows in polynomial time:
		%$$\mathit{\omega}(u) = 2^{n_0} \cdot \mathit{\omega}(lo(u)) + 2^{n_1} \cdot \mathit{\omega}(hi(u))$$
		\begin{equation*}
			\mathit{\omega}(u) =  
			\begin{cases}  
				\prod_{v \in Ch(u)}{\mathit{\omega}(v)}  & \text{decomposition}   \\
				2^{n_0} \cdot \mathit{\omega}(lo(u)) + 2^{n_1} \cdot \mathit{\omega}(hi(u))  & \text{decision}
				
			\end{cases}
	\end{equation*}
		where $n_0 = |\mathit{Vars}(u)| - |\mathit{Vars}(lo(u))| - 1 $ and $n_1 = |\mathit{Vars}(u)| - |\mathit{Vars}(hi(u))| - 1$. %and $ m = |\mathit{Vars}(u)| - 1 - \sum_{v \in Ch(u)}{ |\mathit{Vars}(v)|}$. 
		%The model counting of the formula represented by $u$ satisfies $CT(u) = \mathit{\omega}(u)$.
		
		\begin{proof}
            The time complexity is immediate by using dynamic programming.
            We prove the equation can compute the weight correctly by induction on the number of variables of the \ADDAND rooted at $u$.
            It is obvious that the weight of a terminal node is the real value labeled. 
            For the case of the $\wedge$ node, since the variables of the child nodes are all disjoint, it can be easily seen from Definition \ref{prop:omega-proposition}.
            Next, we will prove the case of the decision node.
            Assume that when $|\mathit{Vars}(u)| \le n$, this proposition holds. 
            For the case where $|\mathit{Vars}(u)| = n + 1$, we use $Y_0$ and $Y_1$ to denote $\mathit{Vars}(lo(u))$ and $\mathit{Vars}(hi(u))$, and we have $|Y_0| \le n$ and $|Y_1| \le n$.
            Thus, $\mathit{\omega}(lo(u))$ and $\mathit{\omega}(hi(u))$ can be computed correctly.
            According to Definition \ref{def:ADDAND-weight}, $w(u) = \sum_{\sigma \in 2^{\mathit{Vars}(u)}}\omega(\sigma,u)$.
            The assignments over $\mathit{Vars}(u)$ can be divided into two categories: 
            \begin{itemize}
              %\item The assignment $\sigma \not\models L(u)$: $\omega(\sigma, u) = 0$.
              \item The assignment $\sigma \models \lnot \mathit{var}(u)$: 
              It is obvious that $\omega(\sigma, u) = \omega(\sigma_{\downarrow Y_0}, lo(u))$. 
              Each assignment over $Y_0$ can be extended to exactly $2^{n_0}$ different assignments over $\mathit{Vars}(u)$ in this category. Thus, we have the following equation:
              $$\sum_{\sigma \in 2^{\mathit{Vars}(u)} \land \sigma \models \lnot \mathit{var}(u)}\omega(\sigma, u) = 2^{n_0} \cdot \mathit{\omega}(lo(u)).$$
              \item The assignment $\sigma \models \mathit{var}(u)$: This case is similar to the above case.
            \end{itemize}
            To sum up, we can obtain that $\mathit{\omega}(u) = 2^{n_0} \cdot \mathit{\omega}(lo(u)) + 2^{n_1} \cdot \mathit{\omega}(hi(u))$.	
		\end{proof}
		
	\end{proposition}
	
	Now we explain how \ADDAND computes its Shannon entropy in polynomial time.
	\begin{proposition}\label{prop:Entropy-proposition}
		Given an \ADDAND rooted at $u$, its entropy $\mathit{H}(u)$ can be recursively computed in polynomial time as follows:
		\begin{equation*}
		\mathit{H}(u) =  
		\begin{cases}  
			0 & \text{terminal}   \\
			\sum_{v \in Ch(u)}{\mathit{H}(v)}  & \text{decomposition}   \\
			\begin{gathered}
				p_{0} \cdot (\mathit{H}(lo(u)) + n_0)\\
				+ p_{1} \cdot (\mathit{H}(hi(u)) + n_1)\\
				 - p_{0} \cdot \log p_{0} - p_{1} \cdot \log p_{1}
			\end{gathered} & \text{decision}
		
		\end{cases}
		\end{equation*}
		where  $n_0 = |\mathit{Vars}(u)| - |\mathit{Vars}(lo(u))| - 1 $, $n_1 = |\mathit{Vars}(u)| - |\mathit{Vars}(hi(u))| - 1$, $p_{0} = \frac{2^{n_0} \cdot \mathit{\omega}(lo(u))}{\omega(u)}$, and $p_{1} =  \frac{ 2^{n_1} \cdot \mathit{\omega}(hi(u))}{\omega(u)}$.  
       
		
		
		\begin{proof}
        According to Proposition \ref{prop:omega-proposition}, $\omega(u)$ can be computed in polynomial time, and therefore the time complexity in this proposition is obvious. The case of terminal node is obviously correct, the case of decomposition is immediately from the property entropy is additive, and next we show the correctness of the case of decision.

        Let $H_0(u)$ be $-\sum_{\sigma \models \lnot sym(u)} p(u,\sigma) \log p(u, \sigma)$ and $H_1(u)$ be $-\sum_{\sigma \models sym(u)} p(u,\sigma) \log p(u,\sigma)$. Then $H(u) = H_0(u) + H_1(u)$. 
        We use $X$ to denote  $\mathit{Vars}(lo(u))$.
        $H_0(u) = -\sum_{\sigma \models \lnot sym(u)} p_0 \cdot 2^{-n_0} \cdot p(lo(u),\sigma \downarrow X) \cdot \left(\log p_0 - n_0 + \log p(lo(u),\sigma \downarrow X) \right) = p_{0} \cdot \left(-\log p_{0} + n_0 + \mathit{H}(lo(u))\right)$. The case of $H_1$ is dual.
		\end{proof}
		
	\end{proposition}
	
We close this section by explaining why we use the orderedness in the design of \ADDAND.
Actually, Propositions \ref{prop:omega-proposition}--\ref{prop:Entropy-proposition} still hold when we only use the more general read-once property: each variable appears at most once in each path from the root of an \ADDAND to a terminal node.
First, our experimental results indicate that the linear ordering determined by the minfill algorithm in our tool PSE outperforms the dynamic orderings employed in the state-of-the-art model counters, where the former imposes the orderedness and the latter imposes the read-once property.
Second, \ADDAND can provide tractable equivalence checking between probability distributions beyond this study.
