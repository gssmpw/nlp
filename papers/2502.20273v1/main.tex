
% This must be in the first 5 lines to tell arXiv to use pdfLaTeX, which is strongly recommended.
\pdfoutput=1
% In particular, the hyperref package requires pdfLaTeX in order to break URLs across lines.

\documentclass[11pt]{article}

% Change "review" to "final" to generate the final (sometimes called camera-ready) version.
% Change to "preprint" to generate a non-anonymous version with page numbers.
\usepackage{acl}

% Standard package includes
\usepackage{times}
\usepackage{latexsym}
\usepackage{makecell}
\usepackage{array}
\usepackage{multirow}
\usepackage{dirtytalk} % quotation marks

% For proper rendering and hyphenation of words containing Latin characters (including in bib files)
\usepackage[T1]{fontenc}

% This assumes your files are encoded as UTF8
\usepackage[utf8]{inputenc}

% This is not strictly necessary, and may be commented out,
% but it will improve the layout of the manuscript,
% and will typically save some space.
\usepackage[final,nopatch=footnote]{microtype}

% This is also not strictly necessary, and may be commented out.
% However, it will improve the aesthetics of text in
% the typewriter font.
\usepackage{inconsolata}

%Including images in your LaTeX document requires adding
%additional package(s)
\usepackage{rotating}
\usepackage{graphicx}
\usepackage{multirow}
\usepackage{multicol}
\usepackage{booktabs}
\usepackage{soul}
\usepackage{url}
\usepackage{float}
\usepackage{comment}
\usepackage{amsmath}
\usepackage{amssymb}
\usepackage{enumerate}
\usepackage{subcaption}
\usepackage{setspace}
\usepackage{listings}
\usepackage{tabularray}
\usepackage{xcolor}
\usepackage{hyperref}
\usepackage{lscape}
\usepackage{xcolor}
\usepackage{colortbl}
\usepackage{tabularx}
\usepackage[capitalise,noabbrev]{cleveref}
\usepackage{subcaption}    % For subfigures


% For column spacing
\newcommand\Tstrut{\rule{0pt}{1.5em}}       % "top" strut
\newcommand\Bstrut{\rule[-1.5em]{0pt}{0pt}} % "bottom" strut
\newcommand{\TBstrut}{\Tstrut\Bstrut} % top&bottom struts
% For cell spacing
\newcommand\TstrutCell{\rule{0pt}{1.0em}}       % "top" strut
\newcommand\BstrutCell{\rule[-0.5em]{0pt}{0pt}} % "bottom" strut
\newcommand{\TBstrutCell}{\TstrutCell\BstrutCell} 


\lstdefinelanguage{PythonRegex}{
    morekeywords={r},
    morestring=[b]',
    morestring=[b]",
    morecomment=[l]{\#}
}

\lstset{
    language=PythonRegex,
    basicstyle=\ttfamily\footnotesize,
    keywordstyle=\color{blue},
    stringstyle=\color{black},
    commentstyle=\color{gray},
    frame=single,
    breaklines=true,
    captionpos=b
}

% If the title and author information does not fit in the area allocated, uncomment the following
% \setlength\titlebox{7cm}
% and set <dim> to something 5cm or larger.

\title{How Much is Enough?\\The Diminishing Returns of Tokenization Training Data}



\author{
\vspace{1.3mm}
Varshini Reddy\textsuperscript{\dag} 
\quad Craig W. Schmidt\textsuperscript{\dag} 
\quad Yuval Pinter\textsuperscript{\S} 
\quad Chris Tanner\textsuperscript{\dag,\P} \\
\begin{tabular}{ccc}
      \textsuperscript{\dag}Kensho Technologies & \textsuperscript{\S}Ben-Gurion University &  \textsuperscript{\P}MIT \\
     Cambridge, MA  & Beer Sheva, Israel &  Cambridge, MA \\
\end{tabular} \\
\texttt{\small\{varshini.bogolu,craig.schmidt,chris.tanner\}@kensho.com},\,
\texttt{\small uvp@cs.bgu.ac.il} \\
}

\newcommand{\todot}[1]{\textcolor{red}{[TODO: #1]}}
\newcommand{\cws}[1]{\textcolor{blue}{[Craig: #1]}}
\newcommand{\uvp}[1]{\textcolor{cyan}{[Yuval: #1]}}
\newcommand{\ct}[1]{\textcolor{green}{[Chris: #1]}}
\newcommand{\vr}[1]{\textcolor{orange}{[Varshini: #1]}}  

\date{}

\begin{document}
\maketitle

\begin{abstract}


Tokenization, a crucial initial step in natural language processing, is often assumed to benefit from larger training datasets. This paper investigates the impact of tokenizer training data sizes ranging from 1GB to 900GB. Our findings reveal diminishing returns as the data size increases, highlighting a practical limit on how much further scaling the training data can improve tokenization quality. We analyze this phenomenon and attribute the saturation effect to the constraints imposed by the pre-tokenization stage of tokenization. These results offer valuable insights for optimizing the tokenization process and highlight potential avenues for future research in tokenization algorithms.
\end{abstract}


\newcommand{\mtr}[2]{\multirow{#1}{*}{\textbf{#2}}}
\newcommand{\mtc}[2]{\multicolumn{#1}{c}{\textbf{#2}}}
\newcommand{\mtcb}[2]{\multicolumn{#1}{|c|}{\textbf{#2}}} % with border
\newcommand{\mrt}[2]{\multirow{#1}{*}{\rotatebox{90}{#2}}}

\definecolor{C_SECOND_SOFT}{HTML}{e2a8d6}
\definecolor{C_BASE_SOFT}{HTML}{66adbf}

\definecolor{C_BASE}{HTML}{a2c4c9}
\definecolor{C_SECOND}{HTML}{B62699}
\definecolor{C_THIRD}{HTML}{00B9E8}
\definecolor{C_FOURTH}{HTML}{0B008F}
\definecolor{C_BACKGROUND}{HTML}{EDF5FC}

\section{Introduction}
\label{sec:intro}

Tokenizers are a foundational component of any NLP pipeline, as they are responsible for converting raw text into useful sequences of indexed tokens.
Practitioners often default to standard tokenization algorithms such as Byte-Pair Encoding~\citep[BPE;][]{sennrich-etal-2016-neural}, UnigramLM~\citep{kudo-2018-subword} or WordPiece~\citep{wordpiece,devlin-etal-2019-bert}, sourced directly from libraries such as Hugging Face.\footnote{\scriptsize\url{https://github.com/huggingface/tokenizers}}
The training process of a tokenizer involves using a corpus of training data and a specific tokenization algorithm to generate a fixed-size vocabulary, usually containing between 32,000 and 128,000 tokens.

Extensive research explores the influence of a \textit{model's} training data on LLM performance~\citep{scalinglawspretrainingagents,zhang2024when,hoffmann2022trainingcomputeoptimallargelanguage,kaplan2020scalinglawsneurallanguage}. However, the impact of a \emph{tokenizer's} training data remains relatively unexplored. Recent work has begun to address the importance of tokenizers' vocabulary sizes and the training data domains~\citep{gettingtokenizerpretrainingdomain}. Prior work has also explored various aspects of tokenization, including the influence of different tokenization algorithms~\citep{schmidt-etal-2024-tokenization,ali-etal-2024-tokenizer,wordscharactersbriefhistory,survey-tok-algo}, vocabulary size optimization~\citep{gowda-may-2020-finding}, and the interplay between data type and tokenization strategy, especially in multi-lingual applications~\citep{limisiewicz-etal-2023-tokenization,rust-etal-2021-good}.
Yet, to the best of our knowledge, we are the first to investigate how much training data is needed for a tokenizer, and how this affects performance.


%tRecent work has begun to address the importance of vocabulary size and training data domain for tokenizers~\citep{gettingtokenizerpretrainingdomain}, but, to the best of our knowledge, we are the first to investigate the 
%the question of scaling the data for tokenizer training remains open.

%However, the specific effect of training data size on tokenizer performance has yet to be thoroughly investigated.

We address this by examining the impact of scaling tokenizer training data with sizes ranging from 1GB to 900GB. We use English BPE, UnigramLM, and WordPiece tokenizers with vocabulary sizes of 40,960, 64,000, 128,000, and 256,000. For each tokenization, we examine the proportion of vocabulary tokens that are shared with the 900GB reference case, and we use intrinsic metrics to measure the quality of tokenization on a fixed evaluation corpus.

Our results demonstrate that increasing the amount of tokenizer training data leads to diminishing returns, indicating a saturation point where further data provides minimal to no improvements in tokenization quality. Finally, we examine the proportion of pre-tokenization chunks that exactly match a single token in the tokenizer vocabulary, and suggest that this very high proportion is a possible explanation for these diminishing returns. 

\section{Effect of Training Corpus Size on Tokenization}
\label{sec:scaling}

% In our experiments, we focus on the effect of the tokenizer training corpus size, over various vocabulary sizes and tokenizers.

Our training corpus combines the de-duplicated PILE~\citep{pile} and RedPajama~\citep{redpajama} datasets, totaling 900GB of text. We train three tokenizers -- BPE,\footnote{We use a custom BPE tokenizer based on \scriptsize \url{https://github.com/karpathy/minbpe} \footnotesize as a starting point.} UnigramLM,\footnote{\scriptsize \url{https://github.com/huggingface/tokenizers/blob/main/bindings/python/py_src/tokenizers/implementations/sentencepiece_unigram.py}} and WordPiece\footnote{\scriptsize{\url{https://github.com/huggingface/tokenizers/blob/main/bindings/python/py_src/tokenizers/implementations/bert_wordpiece.py}}} -- on progressively larger subsets of the randomly shuffled 900GB corpus.

For smaller dataset sizes, we used 1GB, 5GB, and 15GB subsets. For larger scales, we started with a 30GB subset of our 900GB corpus and cumulatively increased the training data in 30GB intervals up to the full 900GB. This resulted in 33 distinct training runs for each of the four vocabulary sizes for each of the three tokenizers, leading to a total of 396 trained tokenizers. See Appendix~\ref{app:scaling-pretokenization} for a complete description of the pre-tokenization process and our scaling methodology. 


\subsection{Vocabulary analysis}
\label{sec:analysis-vocab}


% This systematic variation in training data size enables us to precisely observe the effect of scale on the intrinsic metrics and Jaccard Index. 

\cref{fig:common_vocab_40960} shows the fraction of shared vocabulary between tokenizers trained on varying amounts of data and a 900GB-trained reference tokenizer using the same algorithm, for a vocabulary size of 40,960.
Vocabulary similarity increases with training data, rising from approximately 58\% to 97\% for BPE, from 40\% to 97\% for UnigramLM, and from 4\% to 92\% for WordPiece.
This trend is consistent across the other vocabulary sizes of 64,000, 128,000, and 256,000, as shown in \cref{app:heatmaps-vocab-sizes}.
Thus, larger training datasets consistently yield more similar vocabularies across algorithms, though BPE and UnigramLM exhibit a more rapid convergence than WordPiece. These results suggest that a substantial portion of the vocabulary learned from the full 900GB dataset can be obtained from tokenizers trained on significantly smaller fractions of the data.  For example, a tokenizer trained on roughly 150-180GB of data captures over 90\% of the vocabulary present in the 900GB-trained counterpart for BPE and UnigramLM, and over 80\% for WordPiece. 

\begin{figure}[!t]
    \centering
    \includegraphics[width=\linewidth]{figs/Common_Vocab_Plot/Common_vocab_40960.png}
    \caption{Proportion of common vocabulary for BPE, UnigramLM, and WordPiece tokenizers (vocabulary size: 40,960) trained with cumulatively increasing data, relative to the vocabulary of the corresponding tokenizer trained with 900GB of data.}
    \label{fig:common_vocab_40960}
\end{figure}


\subsection{Intrinsic metrics analysis}
\label{sec:analysis-metrics}

While we have seen that there are significant differences in the vocabularies of tokenizers trained on, for example, 30GB versus 900GB of data, it is important to examine these differences to determine whether they translate into meaningful improvements in tokenization quality.
To evaluate the trained tokenizers without the additional computational overhead of training full LLMs, we use the intrinsic tokenizer metrics collected by \citet{uzan-etal-2024-greed}, summarized below: 

\begin{itemize}
    \item \textbf{Morphological alignment:} %(Average F1):}
    Measures how well a tokenizer's word segmentations match gold-standard morphological segmentations. Higher scores indicate a greater ability to capture word structure.
    \item \textbf{Cognitive score:} Assesses the correlation between a tokenizer's output and human performance in lexical decision tasks, evaluating how well the tokenizer's behavior aligns with human lexical processing.
    \item \textbf{Entropy score:} Evaluates token distribution quality by penalizing excessively high- or low-frequency tokens, aiming for a balanced representation.
\end{itemize}

\begin{figure}[!t]
    \centering
    \includegraphics[width=1.0\linewidth]{figs/Intrinsic_Metrics/BPE.png} 
    \caption{Intrinsic measures of BPE tokenizers trained for each of the four vocabulary sizes with scaled training data. `Average F1' is calculated over morphological benchmarks.}
    \label{fig:intrinsic_bpe}
\end{figure}

\cref{fig:intrinsic_bpe} gives results on intrinsic metrics for BPE for each of the four vocabulary sizes included in our study.
Contrary to our expectations, these intrinsic measures do not reveal substantial performance gains with increasing data size.
In fact, for BPE, performance plateaus in the 150GB to 180GB range. Despite the observed vocabulary shifts, the core properties reflected by these metrics remain relatively stable across the range of training data volumes. 
% This suggests that while the vocabularies of the trained tokenizers evolve with increasing training data, the fundamental characteristics of the tokenizers remain largely consistent. 
Thus, simply increasing the training data size may not inherently lead to substantial improvements in a tokenizer's effectiveness. A discussion of the observed patterns for the remaining two kinds of tokenizer is presented in \cref{app:intrinsic-vocab-sizes}.

\subsection{Evaluation dataset analysis}
\label{subsec:jaccard-analysis}

To reconcile the apparent discrepancy between observed vocabulary shifts and stable intrinsic scores, we analyzed the impact of our trained tokenizers on a diverse, multi-domain evaluation dataset. This dataset spans several domains: biology, code, finance, history, legal, mathematics, and general text. This diverse composition mitigates potential tokenizer biases that arise from the influence of domain-specific vocabulary prevalence. Each domain-specific corpus contains 1.5 million characters. The sources of the evaluation set are discussed in \cref{app:downstream-others}.

We assessed the impact of scaling training data for vocabulary construction on the evaluation set by computing the Jaccard Index between the actual tokens used in the evaluation text using each trained tokenizer and the same text tokenized with the 900GB-trained reference tokenizer:
\begin{equation}
    J(U,V) = \frac{|U \cap V|}{|U \cup V|},
    \label{eq:jaccard}
\end{equation}
where $U$ and $V$ are the vocabularies of the reference tokenizer and the current tokenizer.

We also computed a weighted version of the Jaccard Index using the normalized token counts over the evaluation data, to account for the differences in the training set sizes:
\begin{equation}
    J_w(U, V) = \frac{\sum_{t \in U \cap V} \min(w_U(t), w_V(t))}{\sum_{t \in U \cup V} \max(w_U(t), w_V(t))},
    \label{eq:weighted_jaccard}
\end{equation}
where $w_U(t)$ and $w_V(t)$ are the normalized token frequencies for token $t$ in vocabularies $U$ and $V$.

\begin{figure}[!t]
    \centering
    \includegraphics[width=1.0\linewidth]{figs/Jaccard/Average_Jaccard_Index_Compiled_40960.png} 
    \caption{Jaccard Index (open markers) and Weighted Jaccard Index (WTD; filled markers) for BPE, UnigramLM, and WordPiece tokenizers (vocabulary size \textbf{40,960}) across varying data sizes, averaged over all evaluation domains.}
    \label{fig:avg_downstream_jaccard_40960}
\end{figure}

\cref{fig:avg_downstream_jaccard_40960} presents the standard (open markers) and weighted (filled markers) Jaccard Index for tokenizers of 40,960 vocabulary size. The weighted scores consistently exceed the unweighted scores, highlighting the significant influence of token frequency on vocabulary overlap. This suggests that high-frequency tokens exhibit greater consistency across varying data sizes, and they account for a significant fraction of the training token counts.

Our analysis revealed that over 80\% of our evaluation text is represented by approximately 20\% of the tokenizer vocabulary (see Appendix~\ref{app:downstream-others} for detailed results across individual domains of our evaluation dataset and vocabulary sizes).
This finding suggests that the majority of tokens that are added with increasing training data are low-frequency and thus less consequential. While the specific composition of the vocabulary evolves with an increase in training data, the core set of tokens responsible for representing the majority of the text remains relatively stable. This observation is consistent with Zipf's law~\citep{zipf1949human}, which postulates an inverse relationship between word frequency and rank in natural language corpora. Thus, increasing a tokenizer's training data beyond a certain point primarily adds low-frequency tokens to the vocabulary, which has a limited impact on the overall tokenization characteristics captured by the intrinsic metrics. This is one explanation as to why the intrinsic metrics are largely unaffected by increases in training data.


% the old parked version
% \input{old_pretokenization_section_3}

\section{The Limiting Role of Pre-Tokenization}
\label{sec:pretokenization}

Pre-tokenization is the initial step in tokenization, which uses regular expressions to split a document into chunks, which are then tokenized separately. \citet{velayuthan-sarveswaran-2025-egalitarian} recently noted that pre-tokenization can have a greater effect on the resulting tokenization than the choice of tokenization algorithm. They call the chunks resulting from the pre-tokenization phase \emph{pre-tokens}.

The weighted Jaccard results in the previous section show that tokenizers produce a core set of common tokens across the range of tokenizer training data. We hypothesize that this is due to pre-tokenization. The highest frequency pre-tokens are prevalent enough that all three tokenization algorithms have a strong incentive to find a single token that exactly matches the pre-token. 
To investigate this, we go back to the aggregated pre-tokens with their associated counts.\footnote{See \cref{app:scaling-pretokenization} for more details on the pre-tokenization aggregation process.}
We calculate the proportion of pre-tokens represented as a single token within each tokenizer's vocabulary. \cref{fig:pre-tokenization} displays this proportion for BPE, UnigramLM, and WordPiece tokenizers, for varying vocabulary sizes.


\begin{figure}[!ht]
    \centering
    \includegraphics[width=1.0\linewidth]{figs/Pre-tokenization/pre-tokenization-combined.png} 
    \caption{Proportion of pre-tokens represented as single tokens in BPE, UnigramLM and WordPiece vocabularies of varying sizes (40,960, 64,000, 128,000, and 256,000) with increasing training data.}
    \label{fig:pre-tokenization}
\end{figure}



The prevalence of these common pre-tokens within the tokenizers' vocabularies is remarkably high across all algorithms, increasing with vocabulary size. This proportion remains relatively stable with more training data at smaller vocabulary sizes and is essentially overlapping at vocabulary sizes of 128,000 and 256,000. The curves are flat because they represent frequent pre-tokens, which are easily found even with very little training data. Larger vocabularies naturally have more capacity to include a greater number of pre-tokens as single tokens, accounting for the overlapping curves.

These observations support our hypothesis that the plateauing intrinsic metrics (\cref{fig:intrinsic_bpe}) and high weighted Jaccard (\cref{fig:avg_downstream_jaccard_40960}) observed with varying training data can at least be partially attributed to the constraints imposed by pre-tokenization. The pre-tokenization step prioritizes the inclusion of commonly occurring pre-tokens as single tokens in the training process. The tokenizers are limited to optimizing the smaller remaining fraction of the tokenized corpus. This limits the tokenizers' ability to fully leverage larger training datasets, as the core vocabulary is largely predetermined by the pre-tokenization process. The additional tokens produced from larger datasets are primarily low-frequency items (i.e., rare words), which have a smaller overall impact on the vocabulary composition and tokenization quality as measured by our intrinsic metrics.

\section{Conclusion}
\label{sec:conclusion}

In this work, we have systematically investigated the impact of tokenizer training data size on the characteristics of trained tokenizers.
Our findings reveal diminishing returns as the tokenizer training data increases beyond 150GB to 180GB.
Our analysis indicates that tokenization algorithms incorporating pre-tokenization may be fundamentally limited in their ability to fully leverage extremely large datasets.
Therefore, rather than focusing solely on \emph{more data}, we advocate for a shift towards developing and employing better vocabulary training methods that are less susceptible to the limitations of pre-tokenization.
Methods that incorporate contextual signals, such as SaGe~\cite{yehezkel-pinter-2023-incorporating}, offer a promising direction for future research.

\section*{Limitations}
This study has several limitations.
First, both our training corpus and most of our evaluation data are English-centric.
This monolingual focus restricts the generalizability of our findings, as the observed trends in tokenizer performance with increasing data size may not be extrapolated to other languages with different morphological structures or tokenization needs.
Future work should explore the impact of data scaling on tokenizers trained and evaluated on diverse language corpora.
Second, our reliance on intrinsic tokenizer metrics as a primary means of evaluation, while enabling efficient analysis, may not fully capture the complex interplay between tokenizer characteristics and downstream LLM performance.
While these metrics provide valuable insights into tokenizer quality, they serve as a proxy for true downstream effectiveness.
Future research should investigate the correlation between intrinsic metrics and downstream task performance across a wider range of language models and tasks to establish a more comprehensive evaluation framework.

\section*{Ethics Statement}
We train our tokenizers on the commonly used public datasets The Pile \cite{pile} and RedPajama \cite{redpajama}, which have not undergone a formal ethics review. While our evaluation set was manually anonymized and checked for abusive language, it may still contain personal opinions that reflect cultural, political, or demographical biases.

\section*{Acknowledgments}
We thank Seth Ebner for many notes and discussions.
This research was supported in part by the Israel Science Foundation (grant No. 1166/23).


% \bibliographystyle{acl_natbib}
\bibliography{anthology,references}

\appendix
\subsection{Lloyd-Max Algorithm}
\label{subsec:Lloyd-Max}
For a given quantization bitwidth $B$ and an operand $\bm{X}$, the Lloyd-Max algorithm finds $2^B$ quantization levels $\{\hat{x}_i\}_{i=1}^{2^B}$ such that quantizing $\bm{X}$ by rounding each scalar in $\bm{X}$ to the nearest quantization level minimizes the quantization MSE. 

The algorithm starts with an initial guess of quantization levels and then iteratively computes quantization thresholds $\{\tau_i\}_{i=1}^{2^B-1}$ and updates quantization levels $\{\hat{x}_i\}_{i=1}^{2^B}$. Specifically, at iteration $n$, thresholds are set to the midpoints of the previous iteration's levels:
\begin{align*}
    \tau_i^{(n)}=\frac{\hat{x}_i^{(n-1)}+\hat{x}_{i+1}^{(n-1)}}2 \text{ for } i=1\ldots 2^B-1
\end{align*}
Subsequently, the quantization levels are re-computed as conditional means of the data regions defined by the new thresholds:
\begin{align*}
    \hat{x}_i^{(n)}=\mathbb{E}\left[ \bm{X} \big| \bm{X}\in [\tau_{i-1}^{(n)},\tau_i^{(n)}] \right] \text{ for } i=1\ldots 2^B
\end{align*}
where to satisfy boundary conditions we have $\tau_0=-\infty$ and $\tau_{2^B}=\infty$. The algorithm iterates the above steps until convergence.

Figure \ref{fig:lm_quant} compares the quantization levels of a $7$-bit floating point (E3M3) quantizer (left) to a $7$-bit Lloyd-Max quantizer (right) when quantizing a layer of weights from the GPT3-126M model at a per-tensor granularity. As shown, the Lloyd-Max quantizer achieves substantially lower quantization MSE. Further, Table \ref{tab:FP7_vs_LM7} shows the superior perplexity achieved by Lloyd-Max quantizers for bitwidths of $7$, $6$ and $5$. The difference between the quantizers is clear at 5 bits, where per-tensor FP quantization incurs a drastic and unacceptable increase in perplexity, while Lloyd-Max quantization incurs a much smaller increase. Nevertheless, we note that even the optimal Lloyd-Max quantizer incurs a notable ($\sim 1.5$) increase in perplexity due to the coarse granularity of quantization. 

\begin{figure}[h]
  \centering
  \includegraphics[width=0.7\linewidth]{sections/figures/LM7_FP7.pdf}
  \caption{\small Quantization levels and the corresponding quantization MSE of Floating Point (left) vs Lloyd-Max (right) Quantizers for a layer of weights in the GPT3-126M model.}
  \label{fig:lm_quant}
\end{figure}

\begin{table}[h]\scriptsize
\begin{center}
\caption{\label{tab:FP7_vs_LM7} \small Comparing perplexity (lower is better) achieved by floating point quantizers and Lloyd-Max quantizers on a GPT3-126M model for the Wikitext-103 dataset.}
\begin{tabular}{c|cc|c}
\hline
 \multirow{2}{*}{\textbf{Bitwidth}} & \multicolumn{2}{|c|}{\textbf{Floating-Point Quantizer}} & \textbf{Lloyd-Max Quantizer} \\
 & Best Format & Wikitext-103 Perplexity & Wikitext-103 Perplexity \\
\hline
7 & E3M3 & 18.32 & 18.27 \\
6 & E3M2 & 19.07 & 18.51 \\
5 & E4M0 & 43.89 & 19.71 \\
\hline
\end{tabular}
\end{center}
\end{table}

\subsection{Proof of Local Optimality of LO-BCQ}
\label{subsec:lobcq_opt_proof}
For a given block $\bm{b}_j$, the quantization MSE during LO-BCQ can be empirically evaluated as $\frac{1}{L_b}\lVert \bm{b}_j- \bm{\hat{b}}_j\rVert^2_2$ where $\bm{\hat{b}}_j$ is computed from equation (\ref{eq:clustered_quantization_definition}) as $C_{f(\bm{b}_j)}(\bm{b}_j)$. Further, for a given block cluster $\mathcal{B}_i$, we compute the quantization MSE as $\frac{1}{|\mathcal{B}_{i}|}\sum_{\bm{b} \in \mathcal{B}_{i}} \frac{1}{L_b}\lVert \bm{b}- C_i^{(n)}(\bm{b})\rVert^2_2$. Therefore, at the end of iteration $n$, we evaluate the overall quantization MSE $J^{(n)}$ for a given operand $\bm{X}$ composed of $N_c$ block clusters as:
\begin{align*}
    \label{eq:mse_iter_n}
    J^{(n)} = \frac{1}{N_c} \sum_{i=1}^{N_c} \frac{1}{|\mathcal{B}_{i}^{(n)}|}\sum_{\bm{v} \in \mathcal{B}_{i}^{(n)}} \frac{1}{L_b}\lVert \bm{b}- B_i^{(n)}(\bm{b})\rVert^2_2
\end{align*}

At the end of iteration $n$, the codebooks are updated from $\mathcal{C}^{(n-1)}$ to $\mathcal{C}^{(n)}$. However, the mapping of a given vector $\bm{b}_j$ to quantizers $\mathcal{C}^{(n)}$ remains as  $f^{(n)}(\bm{b}_j)$. At the next iteration, during the vector clustering step, $f^{(n+1)}(\bm{b}_j)$ finds new mapping of $\bm{b}_j$ to updated codebooks $\mathcal{C}^{(n)}$ such that the quantization MSE over the candidate codebooks is minimized. Therefore, we obtain the following result for $\bm{b}_j$:
\begin{align*}
\frac{1}{L_b}\lVert \bm{b}_j - C_{f^{(n+1)}(\bm{b}_j)}^{(n)}(\bm{b}_j)\rVert^2_2 \le \frac{1}{L_b}\lVert \bm{b}_j - C_{f^{(n)}(\bm{b}_j)}^{(n)}(\bm{b}_j)\rVert^2_2
\end{align*}

That is, quantizing $\bm{b}_j$ at the end of the block clustering step of iteration $n+1$ results in lower quantization MSE compared to quantizing at the end of iteration $n$. Since this is true for all $\bm{b} \in \bm{X}$, we assert the following:
\begin{equation}
\begin{split}
\label{eq:mse_ineq_1}
    \tilde{J}^{(n+1)} &= \frac{1}{N_c} \sum_{i=1}^{N_c} \frac{1}{|\mathcal{B}_{i}^{(n+1)}|}\sum_{\bm{b} \in \mathcal{B}_{i}^{(n+1)}} \frac{1}{L_b}\lVert \bm{b} - C_i^{(n)}(b)\rVert^2_2 \le J^{(n)}
\end{split}
\end{equation}
where $\tilde{J}^{(n+1)}$ is the the quantization MSE after the vector clustering step at iteration $n+1$.

Next, during the codebook update step (\ref{eq:quantizers_update}) at iteration $n+1$, the per-cluster codebooks $\mathcal{C}^{(n)}$ are updated to $\mathcal{C}^{(n+1)}$ by invoking the Lloyd-Max algorithm \citep{Lloyd}. We know that for any given value distribution, the Lloyd-Max algorithm minimizes the quantization MSE. Therefore, for a given vector cluster $\mathcal{B}_i$ we obtain the following result:

\begin{equation}
    \frac{1}{|\mathcal{B}_{i}^{(n+1)}|}\sum_{\bm{b} \in \mathcal{B}_{i}^{(n+1)}} \frac{1}{L_b}\lVert \bm{b}- C_i^{(n+1)}(\bm{b})\rVert^2_2 \le \frac{1}{|\mathcal{B}_{i}^{(n+1)}|}\sum_{\bm{b} \in \mathcal{B}_{i}^{(n+1)}} \frac{1}{L_b}\lVert \bm{b}- C_i^{(n)}(\bm{b})\rVert^2_2
\end{equation}

The above equation states that quantizing the given block cluster $\mathcal{B}_i$ after updating the associated codebook from $C_i^{(n)}$ to $C_i^{(n+1)}$ results in lower quantization MSE. Since this is true for all the block clusters, we derive the following result: 
\begin{equation}
\begin{split}
\label{eq:mse_ineq_2}
     J^{(n+1)} &= \frac{1}{N_c} \sum_{i=1}^{N_c} \frac{1}{|\mathcal{B}_{i}^{(n+1)}|}\sum_{\bm{b} \in \mathcal{B}_{i}^{(n+1)}} \frac{1}{L_b}\lVert \bm{b}- C_i^{(n+1)}(\bm{b})\rVert^2_2  \le \tilde{J}^{(n+1)}   
\end{split}
\end{equation}

Following (\ref{eq:mse_ineq_1}) and (\ref{eq:mse_ineq_2}), we find that the quantization MSE is non-increasing for each iteration, that is, $J^{(1)} \ge J^{(2)} \ge J^{(3)} \ge \ldots \ge J^{(M)}$ where $M$ is the maximum number of iterations. 
%Therefore, we can say that if the algorithm converges, then it must be that it has converged to a local minimum. 
\hfill $\blacksquare$


\begin{figure}
    \begin{center}
    \includegraphics[width=0.5\textwidth]{sections//figures/mse_vs_iter.pdf}
    \end{center}
    \caption{\small NMSE vs iterations during LO-BCQ compared to other block quantization proposals}
    \label{fig:nmse_vs_iter}
\end{figure}

Figure \ref{fig:nmse_vs_iter} shows the empirical convergence of LO-BCQ across several block lengths and number of codebooks. Also, the MSE achieved by LO-BCQ is compared to baselines such as MXFP and VSQ. As shown, LO-BCQ converges to a lower MSE than the baselines. Further, we achieve better convergence for larger number of codebooks ($N_c$) and for a smaller block length ($L_b$), both of which increase the bitwidth of BCQ (see Eq \ref{eq:bitwidth_bcq}).


\subsection{Additional Accuracy Results}
%Table \ref{tab:lobcq_config} lists the various LOBCQ configurations and their corresponding bitwidths.
\begin{table}
\setlength{\tabcolsep}{4.75pt}
\begin{center}
\caption{\label{tab:lobcq_config} Various LO-BCQ configurations and their bitwidths.}
\begin{tabular}{|c||c|c|c|c||c|c||c|} 
\hline
 & \multicolumn{4}{|c||}{$L_b=8$} & \multicolumn{2}{|c||}{$L_b=4$} & $L_b=2$ \\
 \hline
 \backslashbox{$L_A$\kern-1em}{\kern-1em$N_c$} & 2 & 4 & 8 & 16 & 2 & 4 & 2 \\
 \hline
 64 & 4.25 & 4.375 & 4.5 & 4.625 & 4.375 & 4.625 & 4.625\\
 \hline
 32 & 4.375 & 4.5 & 4.625& 4.75 & 4.5 & 4.75 & 4.75 \\
 \hline
 16 & 4.625 & 4.75& 4.875 & 5 & 4.75 & 5 & 5 \\
 \hline
\end{tabular}
\end{center}
\end{table}

%\subsection{Perplexity achieved by various LO-BCQ configurations on Wikitext-103 dataset}

\begin{table} \centering
\begin{tabular}{|c||c|c|c|c||c|c||c|} 
\hline
 $L_b \rightarrow$& \multicolumn{4}{c||}{8} & \multicolumn{2}{c||}{4} & 2\\
 \hline
 \backslashbox{$L_A$\kern-1em}{\kern-1em$N_c$} & 2 & 4 & 8 & 16 & 2 & 4 & 2  \\
 %$N_c \rightarrow$ & 2 & 4 & 8 & 16 & 2 & 4 & 2 \\
 \hline
 \hline
 \multicolumn{8}{c}{GPT3-1.3B (FP32 PPL = 9.98)} \\ 
 \hline
 \hline
 64 & 10.40 & 10.23 & 10.17 & 10.15 &  10.28 & 10.18 & 10.19 \\
 \hline
 32 & 10.25 & 10.20 & 10.15 & 10.12 &  10.23 & 10.17 & 10.17 \\
 \hline
 16 & 10.22 & 10.16 & 10.10 & 10.09 &  10.21 & 10.14 & 10.16 \\
 \hline
  \hline
 \multicolumn{8}{c}{GPT3-8B (FP32 PPL = 7.38)} \\ 
 \hline
 \hline
 64 & 7.61 & 7.52 & 7.48 &  7.47 &  7.55 &  7.49 & 7.50 \\
 \hline
 32 & 7.52 & 7.50 & 7.46 &  7.45 &  7.52 &  7.48 & 7.48  \\
 \hline
 16 & 7.51 & 7.48 & 7.44 &  7.44 &  7.51 &  7.49 & 7.47  \\
 \hline
\end{tabular}
\caption{\label{tab:ppl_gpt3_abalation} Wikitext-103 perplexity across GPT3-1.3B and 8B models.}
\end{table}

\begin{table} \centering
\begin{tabular}{|c||c|c|c|c||} 
\hline
 $L_b \rightarrow$& \multicolumn{4}{c||}{8}\\
 \hline
 \backslashbox{$L_A$\kern-1em}{\kern-1em$N_c$} & 2 & 4 & 8 & 16 \\
 %$N_c \rightarrow$ & 2 & 4 & 8 & 16 & 2 & 4 & 2 \\
 \hline
 \hline
 \multicolumn{5}{|c|}{Llama2-7B (FP32 PPL = 5.06)} \\ 
 \hline
 \hline
 64 & 5.31 & 5.26 & 5.19 & 5.18  \\
 \hline
 32 & 5.23 & 5.25 & 5.18 & 5.15  \\
 \hline
 16 & 5.23 & 5.19 & 5.16 & 5.14  \\
 \hline
 \multicolumn{5}{|c|}{Nemotron4-15B (FP32 PPL = 5.87)} \\ 
 \hline
 \hline
 64  & 6.3 & 6.20 & 6.13 & 6.08  \\
 \hline
 32  & 6.24 & 6.12 & 6.07 & 6.03  \\
 \hline
 16  & 6.12 & 6.14 & 6.04 & 6.02  \\
 \hline
 \multicolumn{5}{|c|}{Nemotron4-340B (FP32 PPL = 3.48)} \\ 
 \hline
 \hline
 64 & 3.67 & 3.62 & 3.60 & 3.59 \\
 \hline
 32 & 3.63 & 3.61 & 3.59 & 3.56 \\
 \hline
 16 & 3.61 & 3.58 & 3.57 & 3.55 \\
 \hline
\end{tabular}
\caption{\label{tab:ppl_llama7B_nemo15B} Wikitext-103 perplexity compared to FP32 baseline in Llama2-7B and Nemotron4-15B, 340B models}
\end{table}

%\subsection{Perplexity achieved by various LO-BCQ configurations on MMLU dataset}


\begin{table} \centering
\begin{tabular}{|c||c|c|c|c||c|c|c|c|} 
\hline
 $L_b \rightarrow$& \multicolumn{4}{c||}{8} & \multicolumn{4}{c||}{8}\\
 \hline
 \backslashbox{$L_A$\kern-1em}{\kern-1em$N_c$} & 2 & 4 & 8 & 16 & 2 & 4 & 8 & 16  \\
 %$N_c \rightarrow$ & 2 & 4 & 8 & 16 & 2 & 4 & 2 \\
 \hline
 \hline
 \multicolumn{5}{|c|}{Llama2-7B (FP32 Accuracy = 45.8\%)} & \multicolumn{4}{|c|}{Llama2-70B (FP32 Accuracy = 69.12\%)} \\ 
 \hline
 \hline
 64 & 43.9 & 43.4 & 43.9 & 44.9 & 68.07 & 68.27 & 68.17 & 68.75 \\
 \hline
 32 & 44.5 & 43.8 & 44.9 & 44.5 & 68.37 & 68.51 & 68.35 & 68.27  \\
 \hline
 16 & 43.9 & 42.7 & 44.9 & 45 & 68.12 & 68.77 & 68.31 & 68.59  \\
 \hline
 \hline
 \multicolumn{5}{|c|}{GPT3-22B (FP32 Accuracy = 38.75\%)} & \multicolumn{4}{|c|}{Nemotron4-15B (FP32 Accuracy = 64.3\%)} \\ 
 \hline
 \hline
 64 & 36.71 & 38.85 & 38.13 & 38.92 & 63.17 & 62.36 & 63.72 & 64.09 \\
 \hline
 32 & 37.95 & 38.69 & 39.45 & 38.34 & 64.05 & 62.30 & 63.8 & 64.33  \\
 \hline
 16 & 38.88 & 38.80 & 38.31 & 38.92 & 63.22 & 63.51 & 63.93 & 64.43  \\
 \hline
\end{tabular}
\caption{\label{tab:mmlu_abalation} Accuracy on MMLU dataset across GPT3-22B, Llama2-7B, 70B and Nemotron4-15B models.}
\end{table}


%\subsection{Perplexity achieved by various LO-BCQ configurations on LM evaluation harness}

\begin{table} \centering
\begin{tabular}{|c||c|c|c|c||c|c|c|c|} 
\hline
 $L_b \rightarrow$& \multicolumn{4}{c||}{8} & \multicolumn{4}{c||}{8}\\
 \hline
 \backslashbox{$L_A$\kern-1em}{\kern-1em$N_c$} & 2 & 4 & 8 & 16 & 2 & 4 & 8 & 16  \\
 %$N_c \rightarrow$ & 2 & 4 & 8 & 16 & 2 & 4 & 2 \\
 \hline
 \hline
 \multicolumn{5}{|c|}{Race (FP32 Accuracy = 37.51\%)} & \multicolumn{4}{|c|}{Boolq (FP32 Accuracy = 64.62\%)} \\ 
 \hline
 \hline
 64 & 36.94 & 37.13 & 36.27 & 37.13 & 63.73 & 62.26 & 63.49 & 63.36 \\
 \hline
 32 & 37.03 & 36.36 & 36.08 & 37.03 & 62.54 & 63.51 & 63.49 & 63.55  \\
 \hline
 16 & 37.03 & 37.03 & 36.46 & 37.03 & 61.1 & 63.79 & 63.58 & 63.33  \\
 \hline
 \hline
 \multicolumn{5}{|c|}{Winogrande (FP32 Accuracy = 58.01\%)} & \multicolumn{4}{|c|}{Piqa (FP32 Accuracy = 74.21\%)} \\ 
 \hline
 \hline
 64 & 58.17 & 57.22 & 57.85 & 58.33 & 73.01 & 73.07 & 73.07 & 72.80 \\
 \hline
 32 & 59.12 & 58.09 & 57.85 & 58.41 & 73.01 & 73.94 & 72.74 & 73.18  \\
 \hline
 16 & 57.93 & 58.88 & 57.93 & 58.56 & 73.94 & 72.80 & 73.01 & 73.94  \\
 \hline
\end{tabular}
\caption{\label{tab:mmlu_abalation} Accuracy on LM evaluation harness tasks on GPT3-1.3B model.}
\end{table}

\begin{table} \centering
\begin{tabular}{|c||c|c|c|c||c|c|c|c|} 
\hline
 $L_b \rightarrow$& \multicolumn{4}{c||}{8} & \multicolumn{4}{c||}{8}\\
 \hline
 \backslashbox{$L_A$\kern-1em}{\kern-1em$N_c$} & 2 & 4 & 8 & 16 & 2 & 4 & 8 & 16  \\
 %$N_c \rightarrow$ & 2 & 4 & 8 & 16 & 2 & 4 & 2 \\
 \hline
 \hline
 \multicolumn{5}{|c|}{Race (FP32 Accuracy = 41.34\%)} & \multicolumn{4}{|c|}{Boolq (FP32 Accuracy = 68.32\%)} \\ 
 \hline
 \hline
 64 & 40.48 & 40.10 & 39.43 & 39.90 & 69.20 & 68.41 & 69.45 & 68.56 \\
 \hline
 32 & 39.52 & 39.52 & 40.77 & 39.62 & 68.32 & 67.43 & 68.17 & 69.30  \\
 \hline
 16 & 39.81 & 39.71 & 39.90 & 40.38 & 68.10 & 66.33 & 69.51 & 69.42  \\
 \hline
 \hline
 \multicolumn{5}{|c|}{Winogrande (FP32 Accuracy = 67.88\%)} & \multicolumn{4}{|c|}{Piqa (FP32 Accuracy = 78.78\%)} \\ 
 \hline
 \hline
 64 & 66.85 & 66.61 & 67.72 & 67.88 & 77.31 & 77.42 & 77.75 & 77.64 \\
 \hline
 32 & 67.25 & 67.72 & 67.72 & 67.00 & 77.31 & 77.04 & 77.80 & 77.37  \\
 \hline
 16 & 68.11 & 68.90 & 67.88 & 67.48 & 77.37 & 78.13 & 78.13 & 77.69  \\
 \hline
\end{tabular}
\caption{\label{tab:mmlu_abalation} Accuracy on LM evaluation harness tasks on GPT3-8B model.}
\end{table}

\begin{table} \centering
\begin{tabular}{|c||c|c|c|c||c|c|c|c|} 
\hline
 $L_b \rightarrow$& \multicolumn{4}{c||}{8} & \multicolumn{4}{c||}{8}\\
 \hline
 \backslashbox{$L_A$\kern-1em}{\kern-1em$N_c$} & 2 & 4 & 8 & 16 & 2 & 4 & 8 & 16  \\
 %$N_c \rightarrow$ & 2 & 4 & 8 & 16 & 2 & 4 & 2 \\
 \hline
 \hline
 \multicolumn{5}{|c|}{Race (FP32 Accuracy = 40.67\%)} & \multicolumn{4}{|c|}{Boolq (FP32 Accuracy = 76.54\%)} \\ 
 \hline
 \hline
 64 & 40.48 & 40.10 & 39.43 & 39.90 & 75.41 & 75.11 & 77.09 & 75.66 \\
 \hline
 32 & 39.52 & 39.52 & 40.77 & 39.62 & 76.02 & 76.02 & 75.96 & 75.35  \\
 \hline
 16 & 39.81 & 39.71 & 39.90 & 40.38 & 75.05 & 73.82 & 75.72 & 76.09  \\
 \hline
 \hline
 \multicolumn{5}{|c|}{Winogrande (FP32 Accuracy = 70.64\%)} & \multicolumn{4}{|c|}{Piqa (FP32 Accuracy = 79.16\%)} \\ 
 \hline
 \hline
 64 & 69.14 & 70.17 & 70.17 & 70.56 & 78.24 & 79.00 & 78.62 & 78.73 \\
 \hline
 32 & 70.96 & 69.69 & 71.27 & 69.30 & 78.56 & 79.49 & 79.16 & 78.89  \\
 \hline
 16 & 71.03 & 69.53 & 69.69 & 70.40 & 78.13 & 79.16 & 79.00 & 79.00  \\
 \hline
\end{tabular}
\caption{\label{tab:mmlu_abalation} Accuracy on LM evaluation harness tasks on GPT3-22B model.}
\end{table}

\begin{table} \centering
\begin{tabular}{|c||c|c|c|c||c|c|c|c|} 
\hline
 $L_b \rightarrow$& \multicolumn{4}{c||}{8} & \multicolumn{4}{c||}{8}\\
 \hline
 \backslashbox{$L_A$\kern-1em}{\kern-1em$N_c$} & 2 & 4 & 8 & 16 & 2 & 4 & 8 & 16  \\
 %$N_c \rightarrow$ & 2 & 4 & 8 & 16 & 2 & 4 & 2 \\
 \hline
 \hline
 \multicolumn{5}{|c|}{Race (FP32 Accuracy = 44.4\%)} & \multicolumn{4}{|c|}{Boolq (FP32 Accuracy = 79.29\%)} \\ 
 \hline
 \hline
 64 & 42.49 & 42.51 & 42.58 & 43.45 & 77.58 & 77.37 & 77.43 & 78.1 \\
 \hline
 32 & 43.35 & 42.49 & 43.64 & 43.73 & 77.86 & 75.32 & 77.28 & 77.86  \\
 \hline
 16 & 44.21 & 44.21 & 43.64 & 42.97 & 78.65 & 77 & 76.94 & 77.98  \\
 \hline
 \hline
 \multicolumn{5}{|c|}{Winogrande (FP32 Accuracy = 69.38\%)} & \multicolumn{4}{|c|}{Piqa (FP32 Accuracy = 78.07\%)} \\ 
 \hline
 \hline
 64 & 68.9 & 68.43 & 69.77 & 68.19 & 77.09 & 76.82 & 77.09 & 77.86 \\
 \hline
 32 & 69.38 & 68.51 & 68.82 & 68.90 & 78.07 & 76.71 & 78.07 & 77.86  \\
 \hline
 16 & 69.53 & 67.09 & 69.38 & 68.90 & 77.37 & 77.8 & 77.91 & 77.69  \\
 \hline
\end{tabular}
\caption{\label{tab:mmlu_abalation} Accuracy on LM evaluation harness tasks on Llama2-7B model.}
\end{table}

\begin{table} \centering
\begin{tabular}{|c||c|c|c|c||c|c|c|c|} 
\hline
 $L_b \rightarrow$& \multicolumn{4}{c||}{8} & \multicolumn{4}{c||}{8}\\
 \hline
 \backslashbox{$L_A$\kern-1em}{\kern-1em$N_c$} & 2 & 4 & 8 & 16 & 2 & 4 & 8 & 16  \\
 %$N_c \rightarrow$ & 2 & 4 & 8 & 16 & 2 & 4 & 2 \\
 \hline
 \hline
 \multicolumn{5}{|c|}{Race (FP32 Accuracy = 48.8\%)} & \multicolumn{4}{|c|}{Boolq (FP32 Accuracy = 85.23\%)} \\ 
 \hline
 \hline
 64 & 49.00 & 49.00 & 49.28 & 48.71 & 82.82 & 84.28 & 84.03 & 84.25 \\
 \hline
 32 & 49.57 & 48.52 & 48.33 & 49.28 & 83.85 & 84.46 & 84.31 & 84.93  \\
 \hline
 16 & 49.85 & 49.09 & 49.28 & 48.99 & 85.11 & 84.46 & 84.61 & 83.94  \\
 \hline
 \hline
 \multicolumn{5}{|c|}{Winogrande (FP32 Accuracy = 79.95\%)} & \multicolumn{4}{|c|}{Piqa (FP32 Accuracy = 81.56\%)} \\ 
 \hline
 \hline
 64 & 78.77 & 78.45 & 78.37 & 79.16 & 81.45 & 80.69 & 81.45 & 81.5 \\
 \hline
 32 & 78.45 & 79.01 & 78.69 & 80.66 & 81.56 & 80.58 & 81.18 & 81.34  \\
 \hline
 16 & 79.95 & 79.56 & 79.79 & 79.72 & 81.28 & 81.66 & 81.28 & 80.96  \\
 \hline
\end{tabular}
\caption{\label{tab:mmlu_abalation} Accuracy on LM evaluation harness tasks on Llama2-70B model.}
\end{table}

%\section{MSE Studies}
%\textcolor{red}{TODO}


\subsection{Number Formats and Quantization Method}
\label{subsec:numFormats_quantMethod}
\subsubsection{Integer Format}
An $n$-bit signed integer (INT) is typically represented with a 2s-complement format \citep{yao2022zeroquant,xiao2023smoothquant,dai2021vsq}, where the most significant bit denotes the sign.

\subsubsection{Floating Point Format}
An $n$-bit signed floating point (FP) number $x$ comprises of a 1-bit sign ($x_{\mathrm{sign}}$), $B_m$-bit mantissa ($x_{\mathrm{mant}}$) and $B_e$-bit exponent ($x_{\mathrm{exp}}$) such that $B_m+B_e=n-1$. The associated constant exponent bias ($E_{\mathrm{bias}}$) is computed as $(2^{{B_e}-1}-1)$. We denote this format as $E_{B_e}M_{B_m}$.  

\subsubsection{Quantization Scheme}
\label{subsec:quant_method}
A quantization scheme dictates how a given unquantized tensor is converted to its quantized representation. We consider FP formats for the purpose of illustration. Given an unquantized tensor $\bm{X}$ and an FP format $E_{B_e}M_{B_m}$, we first, we compute the quantization scale factor $s_X$ that maps the maximum absolute value of $\bm{X}$ to the maximum quantization level of the $E_{B_e}M_{B_m}$ format as follows:
\begin{align}
\label{eq:sf}
    s_X = \frac{\mathrm{max}(|\bm{X}|)}{\mathrm{max}(E_{B_e}M_{B_m})}
\end{align}
In the above equation, $|\cdot|$ denotes the absolute value function.

Next, we scale $\bm{X}$ by $s_X$ and quantize it to $\hat{\bm{X}}$ by rounding it to the nearest quantization level of $E_{B_e}M_{B_m}$ as:

\begin{align}
\label{eq:tensor_quant}
    \hat{\bm{X}} = \text{round-to-nearest}\left(\frac{\bm{X}}{s_X}, E_{B_e}M_{B_m}\right)
\end{align}

We perform dynamic max-scaled quantization \citep{wu2020integer}, where the scale factor $s$ for activations is dynamically computed during runtime.

\subsection{Vector Scaled Quantization}
\begin{wrapfigure}{r}{0.35\linewidth}
  \centering
  \includegraphics[width=\linewidth]{sections/figures/vsquant.jpg}
  \caption{\small Vectorwise decomposition for per-vector scaled quantization (VSQ \citep{dai2021vsq}).}
  \label{fig:vsquant}
\end{wrapfigure}
During VSQ \citep{dai2021vsq}, the operand tensors are decomposed into 1D vectors in a hardware friendly manner as shown in Figure \ref{fig:vsquant}. Since the decomposed tensors are used as operands in matrix multiplications during inference, it is beneficial to perform this decomposition along the reduction dimension of the multiplication. The vectorwise quantization is performed similar to tensorwise quantization described in Equations \ref{eq:sf} and \ref{eq:tensor_quant}, where a scale factor $s_v$ is required for each vector $\bm{v}$ that maps the maximum absolute value of that vector to the maximum quantization level. While smaller vector lengths can lead to larger accuracy gains, the associated memory and computational overheads due to the per-vector scale factors increases. To alleviate these overheads, VSQ \citep{dai2021vsq} proposed a second level quantization of the per-vector scale factors to unsigned integers, while MX \citep{rouhani2023shared} quantizes them to integer powers of 2 (denoted as $2^{INT}$).

\subsubsection{MX Format}
The MX format proposed in \citep{rouhani2023microscaling} introduces the concept of sub-block shifting. For every two scalar elements of $b$-bits each, there is a shared exponent bit. The value of this exponent bit is determined through an empirical analysis that targets minimizing quantization MSE. We note that the FP format $E_{1}M_{b}$ is strictly better than MX from an accuracy perspective since it allocates a dedicated exponent bit to each scalar as opposed to sharing it across two scalars. Therefore, we conservatively bound the accuracy of a $b+2$-bit signed MX format with that of a $E_{1}M_{b}$ format in our comparisons. For instance, we use E1M2 format as a proxy for MX4.

\begin{figure}
    \centering
    \includegraphics[width=1\linewidth]{sections//figures/BlockFormats.pdf}
    \caption{\small Comparing LO-BCQ to MX format.}
    \label{fig:block_formats}
\end{figure}

Figure \ref{fig:block_formats} compares our $4$-bit LO-BCQ block format to MX \citep{rouhani2023microscaling}. As shown, both LO-BCQ and MX decompose a given operand tensor into block arrays and each block array into blocks. Similar to MX, we find that per-block quantization ($L_b < L_A$) leads to better accuracy due to increased flexibility. While MX achieves this through per-block $1$-bit micro-scales, we associate a dedicated codebook to each block through a per-block codebook selector. Further, MX quantizes the per-block array scale-factor to E8M0 format without per-tensor scaling. In contrast during LO-BCQ, we find that per-tensor scaling combined with quantization of per-block array scale-factor to E4M3 format results in superior inference accuracy across models. 


\end{document}