\section{Related Work}
\label{related work}

% put this at end of paper - key things to compare to your approach - MDE solutions, adaptive UI solutions etc

Our study is primarily inspired by Yigitbas et al.’s ____ work on an MDE-based approach for self-adaptive UIs, which also proposes two DSLs conceptually similar to ours. Their ContextML DSL models a user's context-of-use parameters, while AdaptML functions as a rules engine for conditionally applying UI changes. A key methodological difference is how UI changes are applied: while we modify the source code directly, Yigitbas et al. ____ adopt an abstract UI model defined with the Interactive Flow Modelling Language (IFML) ____. One limitation of their approach is the depth of modelling in their DSLs; for instance, ContextML lacks comprehensive support for age-related needs, such as UI preferences, and hearing and mobility impairments. Nonetheless, their study remains the most mature in the MDE + Adaptive UI subdomain so far. We also appreciate that it is evaluated with real end-users, though it does not consider developers' feedback -- a crucial factor in understanding the enablers and barriers to real-world adoption by the software development community.

Another important study is by Bendaly Hlaoui et al. ____, who propose an MDE approach for design-time UI adaptations tailored to disabled users. They achieve this transformation through two models: (1) an accessibility ontological instance representing an end user's context-of-use parameters and (2) a UI model reverse-engineered from an existing non-adapted UI. These models serve as inputs to an adaptation process that applies adaptation rules to transform the non-adapted model into an adapted one, which is then converted into the final UI. Their context-of-use modelling depth is well-developed and has informed the level of comprehensiveness needed for our study. However, it is unclear whether they produced an MDE prototype, as the authors mention the final transformation from an adapted abstract UI model to an executable UI as future work. Without this detail and a subsequent proof-of-concept evaluation, assessing the effectiveness of their reverse engineering approach remains challenging, particularly in terms of its compatibility with modern development tools and frameworks such as Flutter or React Native.

Minon et al. ____ propose an approach similar to ours, incorporating both context-of-use modelling and adaptation rules. Their tool, the Adaptation Integration System (AIS), automatically tailors UIs to meet the accessibility requirements of user groups with visual, hearing, and cognitive impairments at both runtime and design time. AIS includes a compilation of UI adaptation rules designed for these user groups. At design time, the MDE UI tool designer manually inputs a UI model at any abstraction level of the CAMELEON framework ____, along with parameters indicating the user’s disability. The system then generates an adapted UI model accordingly. However, the paper lacks detailed information about its metamodels, making it difficult to assess the depth of AIS’s modelling capabilities. The provided evidence, such as prototype applications and model examples, suggests that their proof-of-concept is less detailed than both ours and that of Yigitbas et al. ____. Additionally, unlike the studies mentioned above, Minon et al. ____ do not include any user or developer evaluation, limiting insights into its practical effectiveness.

A fundamentally different paradigm is proposed by Akiki et al. ____ for MDE-based adaptive UI generation. They adopt a Role-Based UI adaptation mechanism that provides end-users with a minimal feature set and an optimal layout based on their context-of-use scenarios. In this approach, UI elements and adaptation rules are treated as accessibility resources and assigned to user accounts as ‘roles.’ At design time, a user’s account is allocated roles linked to context-of-use factors such as disabilities and culture. However, their modelling approach lacks sufficient depth to address the needs of seniors, as the metamodels support only graphical/layout adaptations and do not include multi-modality features such as text-to-speech or speech-to-text. While their evaluation included eight participants over the age of 50 (out of N=23), they did not explore the impact of age-related impairments or user preferences on end-user satisfaction. This omission raises doubts about the applicability of this approach in addressing the diverse and highly personalised accessibility needs of seniors.

In conclusion, almost all existing studies lack the modelling depth necessary to capture the UI accessibility and adaptation needs of seniors at the metamodel layer, with the exception of Bendaly Hlaoui et al. ____. However, even in that case, the comprehensiveness of the models does not necessarily translate to their usability, as evidenced by our experience with developer user studies. For instance, when examining the DSLs and their examples, several key questions arose. To name a few: Are they intuitive for developers? Do they allow developers to embed metadata, such as references to accessibility guidelines and explanations of model functionality? Do they provide flexibility in modelling groups of users via personas rather than individual users? Unfortunately, in most cases, these questions remained unanswered due to the absence of concrete proof-of-concept implementations in the existing studies. Only Yigitbas et al. ____ had such a prototype, but this ties into another major limitation: the lack of evaluation studies capturing insights from software developers on the usability of DSLs and MDE processes. Without such insights, it is difficult to assess how these proposed approaches can be effectively adopted by real-world software developers.