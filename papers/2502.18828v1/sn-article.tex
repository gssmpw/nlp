%Version 3 October 2023
% See section 11 of the User Manual for version history
%
%%%%%%%%%%%%%%%%%%%%%%%%%%%%%%%%%%%%%%%%%%%%%%%%%%%%%%%%%%%%%%%%%%%%%%
%%                                                                 %%
%% Please do not use \input{...} to include other tex files.       %%
%% Submit your LaTeX manuscript as one .tex document.              %%
%%                                                                 %%
%% All additional figures and files should be attached             %%
%% separately and not embedded in the \TeX\ document itself.       %%
%%                                                                 %%
%%%%%%%%%%%%%%%%%%%%%%%%%%%%%%%%%%%%%%%%%%%%%%%%%%%%%%%%%%%%%%%%%%%%%

%%\documentclass[referee,sn-basic]{sn-jnl}% referee option is meant for double line spacing

%%=======================================================%%
%% to print line numbers in the margin use lineno option %%
%%=======================================================%%

%%\documentclass[lineno,sn-basic]{sn-jnl}% Basic Springer Nature Reference Style/Chemistry Reference Style

%%======================================================%%
%% to compile with pdflatex/xelatex use pdflatex option %%
%%======================================================%%

%%\documentclass[pdflatex,sn-basic]{sn-jnl}% Basic Springer Nature Reference Style/Chemistry Reference Style


%%Note: the following reference styles support Namedate and Numbered referencing. By default the style follows the most common style. To switch between the options you can add or remove “Numbered” in the optional parenthesis. 
%%The option is available for: sn-basic.bst, sn-vancouver.bst, sn-chicago.bst%  
 
%%\documentclass[sn-nature]{sn-jnl}% Style for submissions to Nature Portfolio journals
%%\documentclass[sn-basic]{sn-jnl}% Basic Springer Nature Reference Style/Chemistry Reference Style
\documentclass[sn-mathphys-num]{sn-jnl}% Math and Physical Sciences Numbered Reference Style 

\usepackage{lmodern}
%%\documentclass[sn-mathphys-ay]{sn-jnl}% Math and Physical Sciences Author Year Reference Style
%%\documentclass[sn-aps]{sn-jnl}% American Physical Society (APS) Reference Style
%%\documentclass[sn-vancouver,Numbered]{sn-jnl}% Vancouver Reference Style
%%\documentclass[sn-apa]{sn-jnl}% APA Reference Style 
%%\documentclass[sn-chicago]{sn-jnl}% Chicago-based Humanities Reference Style

%%%% Standard Packages
%%<additional latex packages if required can be included here>

\usepackage{graphicx}%
\usepackage{multirow}%
\usepackage{amsmath,amssymb,amsfonts}%
\usepackage{amsthm}%
% \usepackage{mathrsfs}%
\usepackage[title]{appendix}%
\usepackage{xcolor}%
\usepackage{textcomp}%
\usepackage{manyfoot}%
\usepackage{booktabs}%
\usepackage{algorithm}%
\usepackage{algorithmicx}%
\usepackage{algpseudocode}%
\usepackage{listings}%
\usepackage{array}
\usepackage{colortbl}
\usepackage{longtable}
\usepackage{hyperref}
\usepackage{url}
%%%%

\newcommand{\toolname}[0]{}
\renewcommand{\toolname}[0]{{\textit{AdaptForge}}}

%%%%%=============================================================================%%%%
%%%%  Remarks: This template is provided to aid authors with the preparation
%%%%  of original research articles intended for submission to journals published 
%%%%  by Springer Nature. The guidance has been prepared in partnership with 
%%%%  production teams to conform to Springer Nature technical requirements. 
%%%%  Editorial and presentation requirements differ among journal portfolios and 
%%%%  research disciplines. You may find sections in this template are irrelevant 
%%%%  to your work and are empowered to omit any such section if allowed by the 
%%%%  journal you intend to submit to. The submission guidelines and policies 
%%%%  of the journal take precedence. A detailed User Manual is available in the 
%%%%  template package for technical guidance.
%%%%%=============================================================================%%%%

%% as per the requirement new theorem styles can be included as shown below
% \theoremstyle{thmstyleone}%
\newtheorem{theorem}{Theorem}%  meant for continuous numbers
%%\newtheorem{theorem}{Theorem}[section]% meant for sectionwise numbers
%% optional argument [theorem] produces theorem numbering sequence instead of independent numbers for Proposition
\newtheorem{proposition}[theorem]{Proposition}% 
%%\newtheorem{proposition}{Proposition}% to get separate numbers for theorem and proposition etc.

% \theoremstyle{thmstyletwo}%
\newtheorem{example}{Example}%
\newtheorem{remark}{Remark}%

% \theoremstyle{thmstylethree}%
\newtheorem{definition}{Definition}%

\raggedbottom
%%\unnumbered% uncomment this for unnumbered level heads

\begin{document}

\title[Article Title]{Adaptive and Accessible User Interfaces for Seniors Through Model-Driven Engineering}

%%=============================================================%%
%% GivenName	-> \fnm{Joergen W.}
%% Particle	-> \spfx{van der} -> surname prefix
%% FamilyName	-> \sur{Ploeg}
%% Suffix	-> \sfx{IV}
%% \author*[1,2]{\fnm{Joergen W.} \spfx{van der} \sur{Ploeg} 
%%  \sfx{IV}}\email{iauthor@gmail.com}
%%=============================================================%%

\author*[1]{\fnm{Shavindra} \sur{Wickramathilaka}}\email{shavindra.wickramathilaka@monash.edu}

\author[1]{\fnm{John} \sur{Grundy}}\email{john.grundy@monash.edu}

\author[1]{\fnm{Kashumi} \sur{Madampe}}\email{kashumi.Madampe@monash.edu}

\author[1]{\fnm{Omar} \sur{Haggag}}\email{omar.haggag@monash.edu}

\affil*[1]{\orgdiv{Department of Software Systems and Cybersecurity, Faculty of Information Technology}, \orgname{Monash University}, \orgaddress{\street{Wellington Road}, \city{Clayton}, \postcode{3800}, \state{Victoria}, \country{Australia}}}

%%==================================%%
%% Sample for unstructured abstract %%
%%==================================%%

\abstract{The use of diverse apps among senior users is increasing. However, despite their diverse age-related accessibility needs and preferences, these users often encounter apps with significant accessibility barriers. Even in the best-case scenarios, they are provided with one-size-fits-all user interfaces that offer very limited personalisation support. To address this issue, we describe \toolname, a novel model-driven engineering (MDE)-based approach to support sophisticated adaptations of Flutter app user interfaces and behaviour based on the age-related accessibility needs of senior users. We explain how \toolname\ employs Domain-Specific Languages to capture seniors' context-of-use scenarios and how this information is used via adaptation rules to perform design-time modifications to a Flutter app's source code. Additionally, we report on evaluations conducted with real-world Flutter developers to demonstrate the promise and practical applicability of \toolname, as well as with senior end-users using our adapted Flutter app prototypes.}

\keywords{Model-driven engineering, Adaptive user interfaces, Senior end users, Domain-specific languages, Software accessibility}

%%\pacs[JEL Classification]{D8, H51}

%%\pacs[MSC Classification]{35A01, 65L10, 65L12, 65L20, 65L70}

\maketitle

\section{Introduction} \label{introduction}

According to the World Health Organisation (WHO), one in six people will be 60 years or older by 2030 \cite{who}. Many of these \textit{senior} individuals are likely to experience age-related declines in capabilities such as vision, hearing, cognition, coordination, and mobility \cite{bossini2014}. Despite being a significant demographic segment, seniors often encounter accessibility barriers in software applications, particularly in software User Interfaces (UIs) \cite{bossini2014, Paez2019, connor2017}. Our own senior user studies confirm this phenomenon. For example, seniors find smaller UI elements in mobile applications cumbersome, as reflected in their feedback: \textit{“It's always the fact that everything \texttt{<}text in apps\texttt{>} is small”} and \textit{“It would also be useful if the boxes were bigger”}. Another common challenge they face relates to the weak contrast ratio between the foreground and background in mobile apps, as highlighted by one participant: \textit{“Because you know, your eyesight obviously is not what it used to be. So you need to have a white background perhaps, and the contrast”}.

A major responsibility for this issue can be attributed to software developers, as they often do not consider seniors as an active user community \cite{johnson2017designing}. Consequently, different age-related accessibility needs of seniors are frequently overlooked during the software development lifecycle \cite{bossini2014, Paez2019, shamsujjoha2024developer}. However, this does not imply that developers are apathetic towards the limitations faced by seniors in society. For instance, during our interview study with software developers, the majority of participants demonstrated some awareness of this societal challenge. One participant shared their personal experience: \textit{“I'm already getting to the age where I have to take my glasses off to read up close. So, I understand some of it.”} Another participant acknowledged the diversity of user needs, stating: \textit{“I think people have disadvantages because we live in a world where we are all different with different issues.”}

Despite having such awareness, applying this knowledge in real-life software development projects remains a non-trivial challenge. The age-related needs of seniors are highly diverse, and ageing itself is a highly individualised process that results in significant variability in capabilities, limitations, skills, and experiences for each senior user \cite{czaja2007}. This non-homogeneity in senior user accessibility needs requires developers to consider dozens or even hundreds of potential accessibility scenarios that users might encounter. Addressing these scenarios manually using traditional and currently available development tools is both time-consuming and cost-ineffective, as highlighted in studies such as Akiki et al. \cite{akiki2014}.

One of such traditional approaches to address accessibility challenges is the adoption of accessibility guidelines and standards, such as the Web Content Accessibility Guidelines (WCAG) \cite{wcag}. Compliance with these standards enables developers to create universally accessible UI designs intended to minimise accessibility barriers for all users, including seniors. However, this approach does not account for the fact that individuals differ significantly and seniors, in particular, have highly diverse and non-homogeneous accessibility needs \cite{akiki2014}. We consider this a major limitation that compromises the effectiveness of accessibility efforts.

An approach to addressing this issue is the combination of: (1) Model-Driven Engineering (MDE) \cite{brambilla2017}, which considers software models as primary artefacts for generating application code with minimal to no handwritten coding \cite{brambilla2017}; and (2) Domain-Specific Languages (DSLs) \cite{brambilla2017}, which enable developers to model the specific requirements of a given application domain -- in this case, the age-related accessibility needs of seniors. Developers can first leverage DSLs to comprehensively and formally represent these accessibility needs. Subsequently, the resulting models can be integrated into an MDE pipeline to automatically generate application code that aligns with the age-related requirements specified in the DSL models \cite{wickramathilaka2023}.

The effectiveness of this combined approach has been well demonstrated in several studies \cite{bendaly2018, ghaibi2017, akiki2016, yigitbas2020}, although these studies have not specifically addressed the needs of the senior community. As a result, several gaps remain in terms of the comprehensiveness of senior accessibility needs modelling, as well as in the demonstration and evaluation of the approach \cite{wickramathilaka2023}. With these gaps in mind, we devised the following approach to conduct our study.

\begin{enumerate}
    \item We first ran an exploratory focus group study with a community of seniors to understand their accessibility challenges with mobile apps;
    \item We developed a set of DSLs to model these accessibility challenges as a set of adaptations to UI components to address them;
    \item We prototyped an MDE-based Flutter application automatic adaptation tool that takes our DSLs and creates a new version of the Flutter app applying the desired UI accessibility adaptations;
    \item We evaluated our generated adapted Flutter apps with the same senior community to see how well we had addressed their accessibility needs; and
    \item We evaluated the DSLs and the overall MDE prototype with Flutter developers to see how they perceive the practicality and usefulness of such an approach.
\end{enumerate}

The key contributions that we discuss within this paper are: 
\begin{enumerate}
    \item How our novel DSLs significantly enhance the current capabilities and depth in modelling age-related accessibility and adaptation needs of seniors;
    \item How our novel MDE-based approach enhances traditional development tools and practices by enabling developers to deliver more accessible and personalised apps for seniors while seamlessly integrating with their existing workflows; and
    \item How we generated insights into the expectations of both developers and seniors regarding adaptive UI solutions, specifically developer perspectives, which have not been explored in other related works within this subdomain.
\end{enumerate}

In this paper, we first discuss the motivation behind our work in Section \ref{motivation}, followed by a detailed explanation of our proposed approach in Section \ref{methodology}. We then present the two user studies conducted with senior end-users and software developers for evaluation, followed by a discussion of our findings in Section \ref{evaluation}. Section \ref{related work} summarises related work, and we conclude the paper in Section \ref{conclusion}.

\section{Motivation} \label{motivation}

Consider the elderly persona depicted in Figure \ref{persona example}, based on an exploratory focus group study that we conducted with nine senior participants in Australia. In this scenario, the user, Judy, experiences several age-related impairments: she is far-sighted with low colour contrast sensitivity and requires reading glasses. Additionally, she suffers from arthritis, which limits the mobility of her fingers, making them swollen and difficult to move. Due to her vision impairment, Judy struggles to read text in most of her mobile applications. Her mobility limitations further compound this challenge, as she finds it difficult to use the on-screen keyboard and other touch-based UI elements (e.g., buttons, input boxes) because of her trembling and swollen fingers. Although she could not find a solution for the on-screen keyboard, her granddaughter assisted her in increasing the overall text size through the phone's global accessibility settings. However, this adjustment led to unintended consequences, such as misaligned UI elements and increased vertical scroll space, which now requires her to interact more frequently with the touch screen -- placing additional strain on her impaired fingers.

\begin{figure}
\centering
\fbox{\includegraphics[width=0.95\textwidth]{Figures/fig1_personas_master.png}}
\caption[A compressed representation of motivational personas]{A compressed representation of motivational personas\protect\footnotemark. Meet Judy, a retired nurse who faces significant accessibility barriers when using mobile applications due to her low vision and arthritis. She struggles to read app text because of small font sizes and low-contrast foreground-background colour combinations. Additionally, the small buttons on her on-screen keyboard are challenging to press due to her trembling and swollen fingers. To cope, Judy has attempted to use her phone's global settings to increase text size; however, this adjustment often results in misaligned UI elements that consume more vertical space, forcing her to scroll excessively, which is challenging due to her limited finger mobility.}\label{persona example}
\end{figure}
\footnotetext{The image used in the persona is taken from Freepik.com.}

Unfortunately, the scenario depicted in Figure \ref{persona example} is quite common among seniors. The challenges related to vision and mobility impairments were initially identified in our exploratory focus group study with 9 seniors and were later reaffirmed in a subsequent study involving the same senior community. This follow-up study was conducted across three separate focus groups, with a total of 22 participants. A key finding across all three focus groups was a strong preference for larger and bolder text, along with an enhanced foreground-background contrast ratio to improve readability. Additionally, all three groups expressed a preference for alternative input modalities over traditional mobile input methods, such as on-screen keyboards, to better accommodate their mobility limitations. Participants noted that many of their apps do not provide these, or other adaptions, and one-size-fits-all device preference modifications often work poorly for them.

Our motivation is to mitigate such UI accessibility barriers faced by users like Judy, who experiences both hyperopia and arthritis. Addressing Judy’s specific challenges may seem straightforward, as a developer could simply refer to the relevant accessibility guidelines in WCAG to enhance her user experience within a traditional static development approach. However, the complexity of the problem increases significantly when considering other seniors with diverse combinations of limitations. For instance, there could be a user like Carl, who has arthritis and age-related macular degeneration but is highly confident in using technology due to his past occupation as an engineer. Similarly, Kathryn, a retired librarian with partial hearing loss, may have heightened concerns about data privacy due to her exposure to recent data breaches in Australia. A universal application of accessibility guidelines alone cannot adequately address the personalised UI needs of such individuals, as developers would still find it cost- and resource-prohibitive to accommodate the diverse and individualised needs of senior users \cite{akiki2014}.

Our conclusion is that removing the accessibility barriers faced by seniors in software UIs is crucial. However, achieving this objective requires addressing the various challenges developers encounter when designing and delivering software for senior users. As one developer highlighted during our interview study: \textit{"In day-to-day life, we tackle more with business requirements rather than user requirements, sometimes, unfortunately. Because our products change every day, our requirements change every day, and it's hard to always include all these requirements."} This statement underscores the need for a solution that is flexible, practical, and convenient for developers to adopt. To address this challenge, we propose a Model-Driven Engineering (MDE) strategy \cite{brambilla2017}, where Domain-Specific Language (DSL) models are utilised to capture the diverse accessibility scenarios that senior users may encounter. Developers can then seamlessly integrate these models into their existing or in-development applications, facilitating the automatic generation of adapted and personalised UI instances tailored to the unique needs of each senior user. Proper implementation of the code generation process can significantly enhance convenience for developers. However, to ensure the proposed tool reaches its full potential in addressing the accessibility challenges faced by seniors, other critical factors -- such as development flexibility, scalability, code maintainability, and ease of use -- must also be carefully considered.

\section{Our Approach} \label{methodology}

\subsection{Research methods} \label{methods}

\subsubsection{Problem exploration} \label{Problem exploration}

At the outset of our study, our goal was to gain a deeper understanding of the real-world UI challenges faced by senior app users. To achieve this, we initially conducted a pilot study with seven senior participants, followed by a subsequent focus group involving nine senior participants, with a mean age of 75. All participants were recruited from a well-established senior community in Australia. The findings from these studies revealed that seniors face numerous challenges when interacting with modern app interfaces. Their feedback played a pivotal role in the development of several artefacts, including personas, user narratives, and UI adaptation scenarios, which subsequently informed the design and development phases of our project. A simplified example of these personas is illustrated in Figure \ref{persona example}, providing further context for our methodology.

\subsubsection{DSL design and development} \label{DSL development}

Subsequently, we utilised the personas and use cases generated during our exploratory stage as a baseline to evaluate the current state-of-the-art approaches in the Model-Driven Engineering (MDE)-based UI adaptation domain. While these approaches demonstrate potential, the existing literature lacks comprehensive methods that specifically address the accessibility and adaptation needs of older adults \cite{wickramathilaka2023}. As a potential solution, we designed and developed two novel Domain-Specific Languages (DSLs) -- namely, Context DSL and Adapt DSL -- to effectively capture age-related accessibility and adaptation requirements during the application design phase. Our Context DSL is designed to capture the accessibility parameters of senior users, following a semi-graphical tree structure that allows developers to comprehensively model and create personas of senior users, either individually or by segment. In contrast, our Adapt DSL serves as a companion textual modelling language that references the information stored in the Context DSL model. It functions as a rules engine to conditionally generate or modify an application's source code, ensuring that the UI is tailored to the specific accessibility needs of senior users.

\subsubsection{MDE based Flutter prototype development} \label{prototype development}

To realise our novel DSLs and our overall approach, we developed a proof-of-concept tool suite, \toolname, using the Eclipse Modelling Framework \cite{emf} and the Flutter Framework \cite{flutter}. This suite enables Dart-based source code adaptations for any app developed using the Flutter framework. Developers can define a diverse range of age-related needs of their senior end-users through the two DSLs at the app design stage, which then serve as input to tailor the source code according to the specified models. 

\subsubsection{Evaluation} \label{}

In the final stage of our study, we conducted two separate user studies to evaluate our tool suite: (1) a developer interview study, involving 18 developers with professional experience with the Flutter framework; and (2) a focus group study/acceptance test, conducted with 22 participants across three focus groups from the same senior user community we collaborated with during the exploratory stage. The objective of the senior user study was to assess whether the generated adapted UI instances effectively address the accessibility barriers faced by seniors due to their age-related needs. In this paper, we place greater emphasis on the developer study evaluation, as it provides critical insights into the effectiveness of our novel approach and highlights areas for improvement. This focus is particularly important from the perspective of our primary audience -- the software developer community.

\subsection{Overview of ~\toolname} \label{overview}

\begin{figure}
\centering
\includegraphics[width=1\textwidth]{Figures/fig2_overall_approach.jpg}
\caption[The overall architecture of \toolname]{The overall architecture of \toolname\protect\footnotemark}
\label{overall_architecture}
\end{figure}

\footnotetext{The images of seniors used in the figure are taken from Freepik.com.}

In our proposed \toolname~approach (illustrated in Figure \ref{overall_architecture}), there are three architectural layers: (1) the Metamodel Layer, (2) the Interface Layer, and (3) the Transformation Layer. 

\begin{enumerate}
    \item \textbf{Metamodel Layer:}  
    This layer comprises two \textit{Domain-Specific Languages (DSLs)} - \textit{Adapt DSL} and \textit{Context DSL} — which define the grammar and syntax for their respective model instances in the Interface Layer. These DSLs serve as the foundation for capturing accessibility-related requirements and adaptation rules.
    
    \item \textbf{Interface Layer:}  
    The Interface Layer provides developers with an interactive environment to utilise the two DSL model instances: the \textit{Adaptation Rules Model} and the \textit{Context-of-Use Model}. These models enable developers to comprehensively represent various accessibility scenarios for senior users or other user segments. Additionally, developers must ensure the integration of the DSL models into the Flutter application’s source codebase, which facilitates the adaptation of UI elements based on user-specific accessibility needs.
    
    \item \textbf{Transformation Layer:}  
    Once the DSL models are integrated, developers can trigger the UI adaptation logic defined within them. The Transformation Layer is responsible for performing \textit{model-to-text transformations}, automatically generating alternative Flutter app UI versions that align with the user requirements and accessibility scenarios specified in the DSL models.
\end{enumerate}

% The DSL models 

\subsection{\toolname~Domain Specific Languages} \label{novel dsls}

\subsubsection{Context DSL} \label{context dsl}

\begin{figure}
\centering
\fbox{\includegraphics[width=0.96\textwidth]{Figures/fig3_judy_context_model.png}}
\caption{An example Context-of-use model for the senior persona in \toolname~for senior user Judy. Here we show a graphical editor developed using Eclipse's Sirius modelling workbench.}\label{context model example}
\end{figure}

The term \textit{Context-of-use} was initially coined by Stephanidis et al. \cite{stephanidis2000} and Calvary et al. \cite{calvary2002} to explain how modern-day UI requirements are multidimensional and highly context-dependent. In particular, according to Calvary et al. \cite{calvary2002}, Context-of-use is an n-tuple that comprises the following entities: 1) \textbf{user} of the system and their attributes (e.g. accessibility needs, digital literacy, and language preferences), 2) \textbf{platform}: hardware and software components that the users use to interact with the system (e.g. device type, screen dimensions, and I/O components), and 3) \textbf{environment} in which interactions between the user and the system occur (e.g. ambient noise and lighting levels, or the weather). The concept is fairly simplistic but allows a great deal of flexibility when attempting to model an accessibility scenario of an end-user. In addition, most existing studies in the domain already champion the use of Context-of-Use modelling for adaptive UIs \cite{ghaibi2017, akiki2014, yigitbas2020, braham2022, bacha2011, bongratz2012}. 

\begin{figure}
\centering
\includegraphics[width=1\textwidth]{Figures/fig4_context_dsl_metamodel.jpg}
\caption{The Context DSL metamodel}\label{context dsl metamodel}
\end{figure}

To illustrate how an accessibility scenario for a senior individual can be captured, consider a situation that may negatively impact the app experience for the senior user depicted in persona Figure \ref{persona example}. In this scenario, Judy, who has low colour contrast sensitivity, struggles to differentiate between foreground and background elements in the app if the colour combinations are not well-optimised for her vision condition. Judy is using an iPhone 11 and is currently in her home garden on a bright, sunny day. An example of an accessibility challenge in this context is that Judy wants to use her shopping app, but the high ambient light makes it difficult for her to see the screen clearly because the brightness of the phone is not sufficient under such conditions, exacerbating the existing colour contrast-related issues with the app. Her options in this scenario could be to use her hand to shade the screen or to move to a shaded area. A Context-of-use model can effectively and abstractly model this scenario using a simple tree-node structure. Our own take on such representation is illustrated in the figure \ref{context model example}. 

Our \toolname's~\textit{Context DSL} is heavily influenced by the existing studies we analysed in Wickramathilaka and Mueller \cite{wickramathilaka2023}. %The literature has extensively explored the modelling of accessibility-related scenarios. 
By analysing these existing metamodels, we were able to design a modelling language that specifically addresses the accessibility needs of senior users. This focus is unique, as previous approaches did not consider seniors as their primary target beneficiaries. We incorporated relevant metamodel elements from existing studies into our own and made improvements, such as adding more depth to user preference modelling (e.g., privacy, colour, font, and multimedia preferences), assistive features, and I/O device modelling. The resulting DSL metamodel is illustrated in Figure \ref{context dsl metamodel}. 

\subsubsection{Adapt DSL}

While our \textit{Context DSL} can model diverse accessibility scenarios, another DSL is required by \toolname~to specify and apply adaptation operations when a context-of-use scenario is identified. For this purpose, we introduce \textit{Adapt DSL}. Unlike the abstract and graphical approach taken with Context DSL, Adapt DSL is descriptive and concrete, focusing on the definition of adaptation operations and rules. Consequently, a textual language was chosen, leveraging conditional statements commonly found in programming languages to make it intuitive for its intended users: software developers.

\begin{figure}
\centering
\includegraphics[width=0.9\textwidth]{Figures/fig5_rules_model_example.jpg}
\caption{An example adaptation rules model for Judy (a runtime model instance of Adapt DSL). To explain each section in the model: A) WCAG reference field, B) Overall comment definition of the rule, C) Context DSL referencing conditional statement, D) Unique widget(s) reference field, E) Widget type field, F) Adaptation type field, G) Comment definition field for a granular adaptation operation, and H) Example adaptation operations for a TextFormField presentation adaptation.}\label{rules model example}
\end{figure}

To illustrate the capabilities of Adapt DSL, consider the persona depicted in Figure \ref{persona example}. In the previous subsection on Context DSL, we discussed how a runtime model instance of Context DSL can represent accessibility scenarios for senior end-users. These accessibility scenarios can then be referenced and interpreted in Adapt DSL as complex conditional statements. Based on these statements, concrete adaptation logic can be defined and applied to a Flutter application through adaptation operations. An example adaptation rule instance for the given persona is depicted in Figure \ref{rules model example}.

\begin{figure}
\centering
\includegraphics[width=1\textwidth]{Figures/fig6_adapt_dsl_metamodel.jpg}
\caption{The Adapt DSL metamodel}\label{Adapt DSL metamodel}
\end{figure}

Our Adapt DSL metamodel is shown in Figure \ref{Adapt DSL metamodel}, and its overall design is influenced by several existing studies \cite{bendaly2018, yigitbas2020, bongratz2012, minon2015}. However, it also introduces several significant new contributions to the research space. These include the ability to define and reference WCAG guidelines related to the adaptation rules being established, the capability to define and maintain a documentation thread for each major adaptation rule and its more granular adaptation operations, and the design of the DSL around an established professional software development tool such as Flutter to demonstrate the broader applicability of this approach. Additionally, the \toolname's Adapt DSL is more advanced than similar tools in the literature, particularly in aspects such as the flexibility and granularity of UI adaptation operations and its robustness and generalisability when applying adaptations across a wider range of apps.

\subsection{MDE-based Adaptive user interface modification}

Once the developer has modelled the age-related accessibility adaptation needs of their senior user base using \toolname's \textit{Context DSL} and \textit{Adapt DSL}, these models are transformed into structured data formats and then fed into a Model-Driven Engineering (MDE) pipeline. \toolname's MDE pipeline also takes an off-the-shelf Flutter application source code to adapt as its input.

The \toolname~MDE pipeline queries the adaptation rules model for key properties within a rule's adaptation operations. Upon identifying these key properties, it traverses the Abstract Syntax Tree (AST) of the Flutter app's Dart source code to locate the UI widgets with the specified key properties. Once these widgets are identified, the MDE pipeline applies source code modifications as defined by the adaptation rules model. An example of this adaptation pipeline is illustrated in Figure \ref{pipeline example}.

Both our \textit{Adapt DSL} and the MDE pipeline are tightly integrated with the Flutter framework in our \toolname~prototype proof-of-concept tooling. Consequently, porting our current implementation to a different tool stack (e.g., React, Angular) will require significant rework of both components to their different UI languages. However, our proposed approach is designed to be highly flexible for applications developed within the Flutter framework. Currently, most demonstrable app adaptations can be applied with minimal source code changes to the original codebase. For instance, all screenshots generated for this paper were taken from an open-source Flutter app project\footnote{\url{https://github.com/adeeteya/FlutterFurnitureApp}}, in which we had no involvement in development. We simply loaded the app's source code into \toolname~to demonstrate its adaptability through our approach. For example, to adapt the presentation of both the Text and Icon widgets in one of the app’s UIs (Figure \ref{pipeline example}), only two additional lines of code were required -- key property lines enabling the MDE pipeline to query the Abstract Syntax Tree (AST) and identify the relevant widgets. We believe that most apps developed using Flutter can be similarly adapted with minimal code modifications.

\begin{figure}
\centering
\includegraphics[width=0.9\textwidth]{Figures/fig7_source_code_transformation_example.jpg}
\caption{A simple source code transformation example of \textbf{presentation adaptation}. In this example, two Flutter widgets (Text and Icon) within an AppBar are transformed. The original app (D) and its source code (A) were sourced from an open-source GitHub repository. The MDE pipeline applies the specified adaptation rules (C) to modify the property values of each Flutter widget (B). The adapted app instance (E) now reflects the changes defined in (C).}\label{pipeline example}
\end{figure}

\subsection{An Example Usage} \label{examples}

As noted earlier, we have developed a prototype of \toolname~to demonstrate the capabilities of our approach. As such, there are three types of app adaptations supported by the current proof-of-concept: Presentation, Modality, and Navigation adaptations. Each of the three is explained with examples below.

\subsubsection{Presentation adaptation}

In the example persona of Judy given in Figure \ref{persona example}, we describe how her vision impairment creates accessibility barriers in her app UIs due to small text and insufficient background-foreground colour contrast. However, with the presentation adaptation capabilities of \toolname, we demonstrate how our approach can mitigate such issues. In this paper, we illustrate this through presentation adaptations applied to two types of Flutter widgets: Text and Icon (Figure \ref{pipeline example}). Additionally, our current proof-of-concept tool suite supports presentation adaptations for various other widgets, including Container, ElevatedButton, TextField, TextFormField, DropdownField, Row, Column, AppBar, Image, GridView, and SliverGrid.

Developers can add a \textit{key} property with a unique identifier to any of the aforementioned Flutter widgets and then use the Adapt DSL's rules model to specify stylistic property modifications for specific contextual situations, as modelled by the Context DSL. The \toolname~code generator subsequently injects these modified properties into the relevant widgets, updating the source code accordingly.

\subsubsection{Multi-modality adaptation}

\begin{figure}
\centering
\includegraphics[width=0.9\textwidth]{Figures/fig8_app_adapted_instances_example.jpg}
\caption{An example is provided where the \textit{Add Shipping Address} UI of the open-source furniture app shown in (A), has been adapted using the \toolname~to incorporate text-to-speech and speech-to-text functionality as depicted in (B). Following this, the adapted source code was further modified to transform the form into a dynamic stepper form, resulting in a navigation-adapted UI instance (C).}\label{adapted app instances example}
\end{figure}

Adapting the Dart source code to add text-to-speech and speech-to-text functionality is more complex compared to presentation adaptations. It involves generating the required imports (such as Flutter's Text-to-Speech and Speech-to-Text plugins) and the corresponding Dart code to produce an executable UI. However, this task is feasible through the proposed approach, as demonstrated by the source code example in Figure \ref{multimodality adaptation example}.

\begin{figure}
\centering
\includegraphics[width=0.9\textwidth]{Figures/fig9_modality_adaptation_example.jpg}
\caption{An example of multi-modality adaptation with source code transformation. The previous presentation adaptation example is extended to integrate text-to-speech and speech-to-text functionality for the defined TextFormField widgets. In (C), when the speech-to-text and/or text-to-speech properties are set to true, the MDE pipeline identifies the relevant widget (A.1) and adapts the source code to include the necessary functionality, as shown in (B.1) and (B.2). This transformation is reflected in the app’s front end, where the adaptation evolves from instance (A.2) to (B.3).}\label{multimodality adaptation example}
\end{figure}

\subsubsection{Navigation adaptation}

Compared to the previously demonstrated adaptation types, navigation adaptation operations are arguably the most complex. When a senior user requires changes to the navigation or workflow within their apps, the resulting source code modifications can be highly unpredictable. Consequently, proposing a definitive method to address such adaptation needs is challenging. However, we have demonstrated an example of form-based navigation adaptation using our proposed tool suite and approach, which minimizes the need for manual coding by the developer, as illustrated in Figure \ref{navigation adaptation example}. In this example, a form with multiple input fields is transformed into a step-by-step form, allowing a senior user to sequentially navigate through a complex form in a mobile application. Using our Adapt DSL specification, we can dynamically group and order the input fields in the form, requiring the developer to add only a single line of code to the original app source code (a unique key property as a widget identifier).

\begin{figure}
\centering
\includegraphics[width=0.8\textwidth]{Figures/fig10_navigation_adaptation_example.jpg}
\caption{An example of navigation adaptation with source code transformation. The previously adapted UI source code (A.1) is further transformed into a segmented form through automatically modified source code (B.1). (B.2) shows the additional classes and functionalities applied to the original code to enable this operation. To trigger this transformation, the developer defines the grouping and ordering of the relevant TextFormField widgets through the adaptation rules model (C). The final transformation to the UI instance is shown, evolving from (A.2) to (B.3).}\label{navigation adaptation example}
\end{figure}

\subsubsection{Comment generation}

In addition to the adaptation types discussed above, we provide support to developers in the specification and maintenance of comments for each adaptation rule and its associated granular adaptation operations (shown in Figure \ref{rules model example}). This allows developers to document the purpose of each rule and the specific adaptations it addresses, particularly concerning the senior user's age-related needs. Notably, this documentation can include references to the relevant WCAG guidelines that inform the adaptation rule. When the MDE pipeline modifies the source code, it embeds these comments within the modified Dart files. This practice provides developers with a clear documentation trail, helping them track any adaptations made to the original source code.

\subsection{Implementation}

To develop our prototype \toolname~tool suite, we utilized the Eclipse Modelling Framework (EMF) and its Domain-Specific Language (DSL) workbench tools. For the Context DSL, we implemented the DSL using the Sirius workbench \cite{sirius}. Additionally, we employed Acceleo \cite{acceleo} as a model-to-text transformer to convert the Sirius run-time model instance into a cleaner XML format. The development of the Adapt DSL was conducted using Eclipse's Xtext framework \cite{xtend}, with Xtend serving as the model-to-text transformer to generate a JSON structure from the Adapt DSL's textual run-time model instances.
The current prototype tool suite is also tightly integrated with the Flutter framework. Specifically, Flutter's widget documentation was used to define the widget attribute adaptation classes for the Adapt DSL. In its current implementation, the MDE pipeline exclusively generates and adapts Dart code.


\section{Evaluation} \label{evaluation}

We evaluated \toolname\ using two separate but parallel qualitative user studies. Our approach's primary target audience was software developers. Therefore, we aimed to present \toolname\ in action (as a video demonstration) and gather their feedback regarding its usefulness, practicality, ease of use, potential for real-life adoption, and suggestions for improvement. On the other hand, the ultimate beneficiaries of our study are seniors. Thus, our second study -- a focus group study focused on demonstrating the adaptive UI prototypes and obtaining feedback from seniors to assess whether their age-related accessibility needs have been effectively addressed. Both studies were approved by the Monash Human Research Ethics Committee (MUHREC Project ID: 42470).

This paper primarily focuses on the feedback from software developers, as they are the intended users of \toolname. However, developers will not use the tool in isolation, without considering its impact on end-users. Therefore, we also provide a summary of the feedback received from seniors regarding the UI prototypes generated using \toolname.

\subsection{Evaluation with developers}

\subsubsection{Study protocol}

In this study, we opted to do 1-hour interviews with software developers. Due to the complexities and familiarity necessary to install and set the prototype in a participant's own device, we adopted an approach where we simulated an adaptation workflow with \toolname~running on the first author's personal computer and produced three video demonstrations\footnote{\url{https://drive.google.com/drive/folders/11b_P07ybFD7SoDS1r7rrdWd7wks1la6n?usp=sharing}}.

At the beginning of the interview session, we presented a scenario where a significant portion of a Flutter app's user base was aged above 60. Then we explained how these senior users find accessibility barriers in the app due to their diverse and personalised age-related accessibility needs and our proposal to address this.

In the first video demonstration, we used a simplified representation of Figure \ref{persona example}, which explained Judy's accessibility needs. We then used the Context DSL editor to create a context-of-use model representation of Judy. After the demonstration, we asked questions from the participants with regards to how practical and useful they found the modelling tool, its ease of use, suggestions for improvement, and the likelihood of the participant adopting the modelling tool in a real-life software development project.

In the second demonstration, we continued the same example by defining an Adaptation Rules Model for Judy's accessibility scenarios via Adapt DSL's editor. We then input both the context-of-use model and adaptation rules model into the MDE pipeline and generated an adapted app instance based on them. These adaptations were solely focused on presentation aspects (e.g., making text larger and bolder, changing the theme to black and white, enhancing the input box and button borders etc...). Again a similar set of questions to the previous demonstration were asked from the participant.

The final demonstration was focused on applying complex adaptations such as the integration of text-to-speech and speech-to-text features into the UI and transforming a traditional form into a dynamically grouped wizard pattern-based step-by-step form. We wanted to determine the practical implication of such complex source code adaptations where the source code differs drastically compared to much simpler presentation adaptations.

We then concluded the session where we invited the participant to express their opinions on their overall impressions and the \toolname's potential for adoption in a real-life software development project.

\subsubsection{Participant recruitment}

We recruited 18 participants to the study. Our primary recruitment criteria were for the software developers to have at least 6 months of professional experience in the Flutter framework. The recruitment was done through social media advertisements and authors' personal and professional contacts. Participation in this study was voluntary.

\subsubsection{Data collection}

Each interview session was carried out over Zoom except for one that was done in person. Session durations ran from 45 minutes to 85 minutes. The average duration per session was 59.9 minutes. We audio-recorded the sessions via Zoom and transcribed the data with Otter.ai.

\subsubsection{Data analysis}

After generating the interview transcripts, we carefully cleaned the dataset and familiarised ourselves with it. Subsequently, we subjected the data to a thematic analysis process. The initial coding revealed a diverse set of concepts, requiring iterative refinements to eliminate redundancies and streamline the codes into a hierarchical structure. Ultimately, we classified the codes under the following root themes: (1) Positive Reactions from Developers for \toolname, (2) Negative Reactions from Developers for \toolname, and (3) Suggestions for Improving \toolname.

While classifying the codes under these overarching themes was intuitive, identifying sub-themes within them proved challenging due to the variability in developer feedback. For instance, under the Suggestions for Improving \toolname~theme, developers offered a highly diverse range of suggestions. Nevertheless, we successfully organised the codes into a more manageable, theme-based hierarchy by the end of the analysis.

\subsubsection{Participant information}

\begin{table}
    \centering
    \renewcommand{\arraystretch}{1.2} % Adjust row spacing
    \resizebox{1\textwidth}{!}{
    \setlength{\tabcolsep}{5pt} % Adjust column spacing
    \small % Reduce font size if needed
    \begin{tabular}{|l|>{\raggedright\arraybackslash}p{1.5cm}|>{\raggedright\arraybackslash}p{2cm}|>{\raggedright\arraybackslash}p{2cm}|>{\raggedright\arraybackslash}p{2cm}|>{\raggedright\arraybackslash}p{2cm}|>{\raggedright\arraybackslash}p{3cm}|}
        \hline
        \textbf{ID} & \textbf{Age} & \textbf{Software industry exp.} & \textbf{Mobile dev. Exp.} & \textbf{Flutter exp.} & \textbf{Use accessibility standards?} & \textbf{Importance of addressing age-specific UI needs} \\ \hline
        D1  & 21 - 34  & 5 - 10 years  & 2 - 4 years  & 2 - 3 years  & Yes    & Important  \\ \hline
        D2  & 21 - 34  & 5 - 10 years  & \texttt{<} 1 year  & \texttt{<} 1 year  & Yes    & Essential  \\ \hline
        D3  & 21 - 34  & 2 - 4 years   & \texttt{<} 1 year  & \texttt{<} 1 year  & Yes    & Essential  \\ \hline
        D4  & 21 - 34  & 5 - 10 years  & 5 - 10 years  & 2 - 3 years  & Yes    & Essential  \\ \hline
        D5  & 21 - 34  & 2 - 4 years   & \texttt{<} 1 year  & \texttt{<} 1 year  & Yes    & Essential  \\ \hline
        D6  & 21 - 34  & 5 - 10 years  & 5 - 10 years  & \texttt{>} 5 years  & Yes    & Important  \\ \hline
        D7  & 21 - 34  & 5 - 10 years  & 2 - 4 years  & 2 - 3 years  & Maybe  & Important  \\ \hline
        D8  & 21 - 34  & 2 - 4 years   & 2 - 4 years  & 2 - 3 years  & Maybe  & Important  \\ \hline
        D9  & 21 - 34  & \texttt{<} 1 year  & \texttt{<} 1 year  & \texttt{<} 1 year  & Yes    & Essential  \\ \hline
        D10 & 21 - 34  & 5 - 10 years  & 5 - 10 years  & \texttt{>} 5 years  & Maybe  & Important  \\ \hline
        D11 & 21 - 34  & 2 - 4 years   & 2 - 4 years  & 3 - 4 years  & Maybe  & Essential  \\ \hline
        D12 & 35 - 59  & \texttt{>} 10 years    & 2 - 4 years  & 3 - 4 years  & Yes    & Important  \\ \hline
        D13 & 21 - 34  & \texttt{<} 1 year  & \texttt{<} 1 year  & \texttt{<} 1 year  & No     & Somewhat Important  \\ \hline
        D14 & 21 - 34  & 5 - 10 years  & 5 - 10 years  & 3 - 4 years  & No     & Important  \\ \hline
        D15 & 35 - 59  & \texttt{>} 10 years    & \texttt{>} 10 years   & \texttt{>} 5 years   & Yes    & Essential  \\ \hline
        D16 & 35 - 59  & 5 - 10 years  & 5 - 10 years  & 2 - 3 years  & Maybe  & Somewhat Important  \\ \hline
        D17 & 21 - 34  & 5 - 10 years  & 5 - 10 years  & 3 - 4 years  & Yes    & Somewhat Important  \\ \hline
        D18 & 21 - 34  & 5 - 10 years  & \texttt{>} 10 years  & 3 - 4 years  & No & Somewhat Important  \\ \hline
    \end{tabular}
    }
    \caption{Demographic data of software developers}
    \label{tab:developer_data}
\end{table}

\begin{table}
    \centering
    \renewcommand{\arraystretch}{1.3} % Adjust row height for better vertical alignment
    \resizebox{\textwidth}{!}{ % Scale table to fit the page
    \begin{tabular}{|p{6cm}|*{18}{>{\centering\arraybackslash}p{0.6cm}|}c|}
        \hline
        \textbf{Personalisation Approach} & \textbf{D1} & \textbf{D2} & \textbf{D3} & \textbf{D4} & \textbf{D5} & \textbf{D6} & \textbf{D7} & \textbf{D8} & \textbf{D9} & \textbf{D10} & \textbf{D11} & \textbf{D12} & \textbf{D13} & \textbf{D14} & \textbf{D15} & \textbf{D16} & \textbf{D17} & \textbf{D18} & \textbf{Count} \\ \hline
        
        Design a universally accessible interface based on accessibility guidelines & X & X & X & X &  & X & X &  &  &  & X &  & X & X & X &  & X & X & \textbf{12} \\ \hline
        
        Ensure UI integrates with native accessibility settings &  & X & X & X & X & X & X &  &  &  &  & X &  & X & X & X &  & X & \textbf{11} \\ \hline
        
        Capture personalisation needs through user interviews and personas & X & X & X &  & X &  &  &  & X &  &  &  & X &  & X &  &  &  & \textbf{7} \\ \hline
        
        Provide a configuration dashboard for UI adjustments & X &  &  & X & X & X &  &  & X &  &  &  &  & X & X &  &  &  & \textbf{7} \\ \hline
        
        Special case handling: Stick with general practices unless specific requirements exist &  &  &  &  &  &  &  &  &  &  &  &  &  &  &  &  & X &  & \textbf{1} \\ \hline
        
        Rarely take steps to personalise UI due to time/resource constraints &  &  & X &  &  &  &  & X &  & X &  &  &  &  &  & X &  &  & \textbf{4} \\ \hline
    \end{tabular}
    }
    \caption{Current strategies used by developer participants to address age-specific UI needs}
    \label{tab:personalisation_strategies}
\end{table}


The demographic data for our evaluation study with software developers (N=18) is provided in Table \ref{tab:developer_data}. In terms of age distribution, the majority of participants (n=15, 83.3\%) were between 21–34 years, while the remaining n=3 (16.7\%) fell within the 35–59 age category. We then assessed participants' experience levels across three key dimensions: (1) overall professional experience as developers, (2) experience in mobile development, and (3) experience in the Flutter framework.

Regarding overall professional experience in the software industry, most developers (n=10, 55.6\%) reported having 5–10 years of experience. Additionally, two participants (11.1\%) had more than 10 years of experience, while two (11.1\%) were novice developers with less than one year of experience. With respect to experience in mobile app development, the levels varied; however, only a minority (n=5, 27.8\%) reported having less than one year of experience. A similar trend was observed for experience in the Flutter framework, where again, five participants had less than one year of experience. Notably, we were able to recruit developers with 3–4 years (n=5, 27.8\%) and even more than five years of experience in the framework (n=3, 16.7\%) -- an encouraging finding, given that Flutter 1.0 was only released in late 2018 \cite{flutter}.

Next, we aimed to assess the current priorities and strategies our developer participants follow when developing accessible software, particularly for senior users. First, we inquired whether participants incorporate accessibility standards, such as WCAG, into their development practices. Encouragingly, 10 participants (58.8\%) reported that they do. To further explore developers' perspectives on addressing accessibility barriers in UIs for seniors, we asked them to rate the importance of considering senior age-related accessibility needs using a Likert scale. Notably, 7 participants (41.2\%) stated that it is essential, while another 7 participants (41.2\%) considered it important. This indicates that the vast majority of our sample acknowledged the significance of accessibility considerations in UI development.

The final question in our pre-interview questionnaire (Table \ref{tab:personalisation_strategies}) aimed to gain a concrete understanding of the exact strategies software developers employ when designing UIs and apps for senior users. The most prevalent strategy among participants (n=12, 66.7\%) was designing universally accessible UIs with accessibility standards such as WCAG in mind. Another commonly adopted approach (n=11, 61.1\%) was ensuring that UIs comply with native operating system-level accessibility settings and requirements. Encouragingly, a majority of participants (n=11, 61.1\%) reported having strategies to support personalisation for their users. Specifically, seven participants (38.9\%) stated that they interact with end users to create personas, while seven of them (38.9\%) indicated that they provide a UI configuration dashboard for end-user customization. Conversely, a minority (n=4, 22.2\%) admitted that they rarely take steps to personalise their UIs for seniors, citing time and resource constraints as limiting factors.

\subsubsection{Results}

A key theme that emerged from our analysis was how \toolname\ could either enhance or hinder the developer experience in a software development setting. Nearly all participants (n=17) provided positive feedback on the strengths of \toolname, highlighting how it could make the development process more intuitive and less burdensome when designing personalised software UI for seniors and other user groups. However, some participants (n=10) also expressed concerns, offering critical feedback on potential drawbacks. We identified that these limitations stem from the current implementation of the prototype, which can be improved in future iterations based on the improvement suggestions provided by all participants (n=18). In this section, we examine and summarise the key findings of the study, encompassing positives, negatives, and suggestions for improvement as discussed by the majority of participants.

\subsubsection*{Developer feedback on \toolname's strengths}

\textbf{Ease of use in DSLs:}

The design and usability of DSLs received highly positive feedback from developers, with 17 out of 18 interviews mentioning usability. Individually, 15 developers found Context DSL user-friendly, while Adapt DSL was similarly well-received (n=15). For Context DSL, developers particularly appreciated its tree-like hierarchy, as it aligns with their existing mental models. As [D2] noted, \textit{"I think it's very easy to understand because it's very well structured, and it's like a tree, so it's very easy for developers to understand, because, you know, the structure just makes sense itself."} Similarly, Adapt DSL was praised for its user-friendliness, with developers finding its JSON-like structure intuitive and easy to understand. As [D7] mentioned, \textit{"I have nothing against this design, actually. I find this quite easy to understand. As you mentioned, this is something like JSON, of course, and yeah, to me, it's very user-friendly, maybe because I am used to this kind of text."}

\textbf{Enhanced developer convenience:}

Another recurring theme (n=11) in the data regarding developers' experience with \toolname\ was its potential to provide convenience in addressing age-related needs for end-users. Developers were particularly positive about its automated code generation capabilities through DSL models, which they felt could save time and reduce effort compared to traditional development methods. For example, [D5] noted, \textit{"I think it saves a lot of time for a developer, because, in order to deal with all the adaptations happening in the user interface. Especially, when you have a lot of users with different needs."}

\textbf{Usefulness and practicality of the approach:}

17 out of 18 participants agreed that our proposed \toolname\ tool and its subcomponents are useful and beneficial in achieving our goal: finding a better approach to addressing age-related accessibility needs in UIs for seniors. Even [D14], who was the most critical participant in our study, acknowledged that at least one of our subcomponents—Context DSL—could be situationally useful. This is further evidenced by what was said by [D10] regarding the overall usefulness of \toolname~, \textit{"I'm really interested in using this kind of tool if I'm developing this kind of app, and I think it's really useful and it can save a lot of developers' time, and it will make the app more usable to the senior users as well."}

In addition, a majority of the participants (n=16) stated that the approach that we have taken in this project is a practical one. Of course the developers had a wide range of improvement suggestions to ensure that our next tool iterations mitigate the practical considerations raised regarding our current prototype implementation. For example, the following was noted by [D15]: \textit{"Outside my previous suggestion of generated files, I'm pretty impressed. And I think it could be a maintainable solution. I think it's accessible for developers."}

\textbf{Potential for real-life adoption:}

We specifically asked our study participants whether they would be willing to use our proposed tool and its subcomponents in a real-life UI development project, particularly when designing for a user base that includes a significant proportion of senior users. The feedback we received was overwhelmingly positive, with 16 out of 18 participants agreeing that they would be inclined to adopt our approach. For example, [D1] noted, \textit{"I would definitely use it. And like I said, this might be a must-use model or approach in the future."}. Another participant -- [D7] told us, \textit{"I find it very useful, and I'm kind of impressed. And at the same time, I was wondering why we don't have such a tool at the moment. And I'm really happy that people like you are doing such implementations, and I think it will help 1000s and 1000s of developers, plus millions of users, while using and while implementing, both scenarios to have this kind of tools."}

\subsubsection*{Developer feedback on \toolname's weaknesses and potential mitigation strategies}

\textbf{Incurring extra effort:}

A non-majority (n=7) of participants raised a concern about how \toolname~ could require additional effort from developers. For example, [D10] highlighted the extra effort required to learn and integrate the tool into existing development workflows, stating, \textit{"It takes some extra time when we are implementing this, because this is a totally separate thing. ...... Currently, since we are not making it \texttt{<}app\texttt{>} dynamic, it's mostly static. So if you are making it dynamic, it will take some extra time."} A different but solitary concern was from {D14}, where they pointed out the fact that considering age-related needs in an app might fall outside a developer's typical responsibilities: \textit{"It's a UX guy job. It's not a developer's job to do all of this, right?"}

\underline{\textit{Mitigation strategies:}} A significant portion of our participants (n=8) proposed that we need mechanisms in place to provide developers with guidance when it comes to modelling tasks. Concrete examples, personas, and model documentation metadata could help developers in reducing the learning effort. This is evidenced by [D8] who stated, \textit{“I think one thing that could be better is if we have recommended way of doing something right, so without just giving the operations you can just set up, okay, these are the conditions and then allow users to use that as a template to begin with, so they know what sort of like combinations we how to consider to start with”}.

\textbf{Infeasibility in collecting users' context data:}

A minority of the participants (n=5) commented on how impractical it would be to capture the user, platform, and context parameters of the users in order to facilitate our proposed approach in the first place. To illustrate this challenge, [D4], a UX engineer, highlighted concerns about the onboarding process and user retention, stating, \textit{"How long will it take to configure these things when they \texttt{<}users\texttt{>} install the application? Because that's also a concern. Sometimes, when you are creating \texttt{<}an\texttt{>} application, \texttt{<}it\texttt{>} is much more concerning about how speedy the onboarding process is. If it is too complicated for any use, it doesn't matter that like age \texttt{<}related needs\texttt{>} or like if they need personalised usage, but if we make it a little longer or \texttt{<}add\texttt{>} extra steps, it can be a concern of the user. Like, it can drop user from the installing the app, or drop \texttt{<}them\texttt{>} out \texttt{<}from\texttt{>} doing some actions in the application."}

\underline{\textit{Mitigation strategies:}} Four interviewees provided suggestions on this matter and we identified two potential strategies to resolve it. (1) Seniors volunteering the accessibility information during the onboarding process: This is the more traditional approach and was proposed by both [D12] and [D15]. Despite [D4]'s doubts about putting a burden such as this on the seniors leading to a drop in user base, [D15] is of the opinion that it is a viable approach as evidenced by his statement: \textit{"My mom would definitely use it, for sure. She already uses some of those, like text scaling and stuff like that on her phone"}; and (2) A universal accessibility profile: This method was suggested by both [D8] and [D12] and the latter had a great example to showcase its approach: \textit{"Hey, look, you can create yourself a profile of your impairments over here. And then apps can subscribe to that service so that when that person, when Judy, logs into this new app, she needs to be able to do the New Zealand travel declaration, the New Zealand traveller declaration will go over to that service, grab her profile and then apply it all to the UI, yeah? So she doesn't have to do that for every single app"}

\textbf{Code scalability, maintainability and versioning:}

Developing even a small app would take tens of thousands of code lines. For example, [D14] shared with us that an app he's developing for a small startup reached over 80,000 lines of code. Even in such cases, where the user base is relatively small, implementing personalized UI adaptations using our proposed approach to adaptive code generation could significantly increase the codebase size. With the need to store multiple adaptive permutations, the total number of lines of code could grow exponentially, leading to both code readability and maintainability issues, as noted by a group of participants (n=6). For example, [D14] noted, \textit{"the second a human needs to look at it and fix it and change it \texttt{<}codebase\texttt{>}, we have to have some kind of maintainability or structure or readability"}. Consequently, this issue also leads to issues with version control. As explained by [D10], \textit{"So definitely, we are using something like GitHub or GitLab. So you have to put the code there if you are doing a change again, like UI, change functionality, .... you have to add all these new changes to each version."}

\underline{\textit{Mitigation strategies:}} To prevent the codebase from becoming unwieldy or unmaintainable due to the source code adaptations performed by \toolname, participants suggested several strategies. One notable approach was: Slotting in widgets based on API calls. In this method, the application is designed so that its source code includes allocated spaces for any widgets that may need adaptation. When an adaptation is triggered, the necessary operations and context parameters are sent to an adaptation service REST API, which then serves the required adapted widgets. These widgets are dynamically slotted into the source code, creating an adapted app instance without requiring complex code generation. For example, [D10] described this approach as follows: \textit{"In the Flutter you provide the space. So that provided space, you'll be listening to the back end again. And with the adaptation coming, ..... you're replacing the widget with the new widget coming according to the adaptation from the back \texttt{<}end\texttt{>}"}. A key advantage of this approach is that developers would no longer need to maintain multiple code repositories for each app permutation, reducing code duplication and simplifying maintenance.

\textbf{Deployment concerns:}

Another practical issue raised by participants (n=5) was the deployment of adaptations to end-users in a real-world setting. For example, [D17] explained to us that they optimise the app bundle size so that the app is able to work efficiently even on an older mobile device that has limited resources such as storage. This leads to a challenge where it would be disadvantageous for us to store an adaptation service, DSL models, and conditionally generated code locally. Thus, it would require us to maintain a centralised server, allowing over-the-air (OTA) updates to deliver necessary adaptations to users' apps. However, OTA-based adaptations face significant restrictions under current App Store and Play Store policies. As [D15] pointed out, \textit{"The question of it being in an App Store is kind of a problem, right? Because you have to go through a lot of like getting approvals for app changes."} Even minor updates or bug fixes require going through a lengthy and unpredictable approval process, making frequent or dynamic adaptations impractical within the existing mobile app ecosystem.

\underline{\textit{Mitigation strategies:}} To mitigate the limitations imposed by app store policies on providing updates for UI adaptations, we can leverage tools such as Shorebird or similar services to push over-the-air (OTA) updates to the app almost instantly. This approach was explained by [D12] during our interview: \textit{"The only way you could potentially do it and have the code somewhere else is, I don't know if you've seen 'Shorebird', it does over-the-air updates, so you release your app to the store, and then you can do patches, and you don't have to go through the store review process to get the patch. So you could potentially have a modified version of Shorebird that allows you to pull particular code for those data adaptations."} 

\subsection{Evaluation with seniors}

\subsubsection{Study protocol}

We conducted a focus group study to gather feedback from participants. A presentation-based approach was used to present the questions and demonstrate the application using short video clips\footnote{\url{https://drive.google.com/drive/folders/1_Eooe_XxzAHi8cfUHfWHP8WB4SyP55fs?usp=sharing}}. At the beginning of the session, a Flutter application was shown to participants through a short video clip. The application itself is open-source\footnote{\url{https://github.com/adeeteya/FlutterFurnitureApp}} and is a generic retail furniture app. The demonstration was originally performed on an actual mobile device, where we screen-recorded a user logging in, browsing furniture products, and navigating to a form page to add a shipping address.

The focus group session was structured around three types of adaptations: presentation, multi-modality, and navigation, with each section beginning with a video demonstration. First, participants viewed the app in its original, non-adapted form and provided feedback on usability and accessibility challenges. The second demonstration focused on presentation adaptations, applying \toolname\ to enhance text size, input box borders, and contrast. Participants compared this adapted version with the original. The third demonstration introduced multi-modality adaptations, adding text-to-speech and speech-to-text features for the shipping address form. The fourth showcased a navigation adaptation, transforming the multi-field form into a step-by-step wizard. The session concluded with a discussion on app run-time adaptations to inform future work.

\subsubsection{Participant recruitment}

We recruited 22 senior participants from a University of the Third Age (U3A) chapter in Australia. Each participant was gifted a 30 AUD voucher in a local supermarket as a token of appreciation.

\subsubsection{Data collection}

We held three focus group sessions in person and each ran approximately for an hour. Each session was audio recorded and then transcribed using Otter.ai. During data collection, we inquired about any accessibility barriers or enablers that participants encountered while using the prototype UIs.

\subsubsection{Analysis}

After carefully cleaning the qualitative data transcripts collected from the focus groups, we conducted a thematic analysis. During the initial coding process, the data revealed a diverse set of codes. Through an iterative refinement process, we organised these codes into three overarching root themes: (1) Enablers to Accessibility, (2) Barriers to Accessibility, and (3) Factors Influencing Personalised User Experiences.

The first two themes effectively summarised the insights shared by our senior participants regarding their current experiences with app UIs. These themes captured whether participants found the proposed personalisation methods (demonstrated through UI prototypes) useful and how they envisioned further enhancing their app experiences.

In contrast, the third theme focused on how seniors perceive personalising their app UIs through their own user agency. Within this theme, we further classified the codes into enablers and disablers for personalisation, as well as explored seniors’ perceptions of the methods through which they exercise their personalisation agency. However, this paper reports only on the first two themes, as they focus on feedback about the current version of our prototype, while the third theme will inform future work.

\subsubsection{Participant information}

The demographic information of the senior participants in our evaluation study is provided in Table \ref{tab:senior_demographic_data}. The mean age of the participants was 72.1 years and the median age was 72 years. The majority were women (n=17, 77.3\%), and most identified culturally as Australians (n=14, 63.6\%). Participants had a diverse range of prior occupations, reflecting varied skills, experiences, and backgrounds.

Nearly all participants used spectacles as a visual aid (n=21, 95.5\%), indicating common age-related vision impairments. Additionally, all participants owned a mobile device, with the majority also owning a tablet. The number of hours per week spent using mobile apps varied, but half of the participants (n=11. 50\%) reported using them for more than 10 hours per week. Most participants (n=13, 59.1\%) indicated an average confidence level in using mobile apps, while only two participants (9.1\%) reported below-average confidence.

\begin{sidewaystable}
    \centering
    % \renewcommand{\arraystretch}{1.5}
    \footnotesize
    \begin{tabular}{|l|ll>{\raggedright\arraybackslash}p{2cm}>{\raggedright\arraybackslash}p{2cm}|>{\raggedright\arraybackslash}p{2cm}|l|>{\raggedright\arraybackslash}p{2cm}|>{\raggedright\arraybackslash}p{2cm}|}
        \hline
        \textbf{\centering ID} & \textbf{Age} & \textbf{Gender} & \textbf{Culture} & \textbf{Occupation (prior)} & \textbf{Usage of aids} & \textbf{mobile device usage} & \textbf{App usage (hours-per-week)} & \textbf{Mobile app usage confidence level} \\ \hline
        P1  & 74 & F & Australian & Public servant & Eye glasses & Android phone, Tablet & 4-9 & Average \\ \hline
        P2  & 85 & M & Anglo & Engineer & Eye glasses, Hearing aids & - & \texttt{<}1 & Very low  \\ \hline
        P3  & 75 & M & Australian & - & Eye glasses & iPhone, Tablet & 10-20 & high  \\ \hline
        P4  & 72 & F & Australian & Nurse & Eye glasses & Android phone & 4-9 & Average  \\ \hline
        P5  & 76 & F & Indian & Educator, Academic & Eye glasses & iPhone, Tablet, Kindle & 2-3 & Average  \\ \hline
        P6  & 68 & F & English & Academic librarian & Eye glasses & iPhone, Tablet & 4-9 & Very high  \\ \hline
        P7  & 69 & F & Australian & Dental assistant & Eye glasses & Android phone & 20\texttt{<} & Average  \\ \hline
        P8  & 72 & F & Australian & Nurse & Eye glasses & iPhone, Tablet & 10-20 & Average \\ \hline
        P9  & 83 & M & Australian & Business owner & Eye glasses, Hearing aids & Android phone & \texttt{<}1 & Average  \\ \hline
        P10 & 66 & F & Australian & Marketing & Eye glasses & iPhone & 20\texttt{<} & High  \\ \hline
        P11 & 63 & F & Dutch & Community support & Eye glasses & Android phone & 10-20 & Average  \\ \hline
        P12 & 62 & F & Chinese & - & n/a & Android phone, Tablet & 2-3 & High  \\ \hline
        P13 & 75 & F & Australian & Casual librarian & Eye glasses & iPhone, Tablet & 20\texttt{<} & High  \\ \hline
        P14 & 80 & M & German & Industrial sales & Eye glasses, Hearing aids & iPhone, Tablet & \texttt{<}1 & Average \\ \hline
        P15 & 68 & F & Australian & Trainer, Assessor & Eye glasses & iPhone & 10-20 & Average to low \\ \hline
        P16 & 61 & F & Australian & Information management officer & Eye glasses & Android phone, Tablet & 10-20 & Average  \\ \hline
        P17 & -  & F & Indian & Admin officer & Eye glasses & Android phone & 4-9 & Average \\ \hline
        P18 & 70 & F & Australian & Banking & Eye glasses & Android phone, Tablet & 10-20 & Average  \\ \hline
        P19 & 68 & M & Australian, Irish & Public servant & Eye glasses & Android phone & 2-3 & High \\ \hline
        P20 & 79 & F & Australian & Funeral director & Eye glasses & Android phone, Tablet & 20\texttt{<} & Average  \\ \hline
        P21 & 76 & F & Australian & Service assistant & Eye glasses & iPhone, Tablet & 20\texttt{<} & Average \\ \hline
        P22 & 73 & M & Scottish & Senior manager IT & Eye glasses & Android phone, Tablet & 2-3 & High  \\ \hline 
    \end{tabular}
    \caption{Senior participant demographic information. N = 22 seniors participated in the study.}
    \label{tab:senior_demographic_data}
\end{sidewaystable}

\subsubsection{Results}

Across all three focus groups, we received overwhelmingly positive feedback. Participants strongly resonated with the adaptive enhancements we proposed to make the demonstrated app more tailored to the senior population, particularly when compared to the generic, non-adapted app instances used as a baseline. In the following paragraphs, we summarise the findings regarding these enhanced features included in the adapted UI prototypes, categorised by their adaptation types: Presentation, Multi-modality, and Navigation.

\textbf{Presentation adaptations:} 
When we presented app UI instances with enhanced visual elements to our senior participants, we received positive feedback. For example, they expressed satisfaction with improvements to text readability, particularly the use of bolder and larger fonts. Another widely appreciated enhancement across all three focus groups (n=3) was the improved background-foreground contrast ratios. Beyond general text enhancements, we applied UI adaptations such as changing the app’s colour theme to black-and-white and adding enhanced borders around buttons, input fields, and dropdown menus. However, one concern that emerged was the trade-off between text visibility and the amount of information displayed on a mobile screen. Participants noted that this issue was not unique to our adaptations but was something they had encountered previously when adjusting text size either through their mobile device’s global font settings or using zoom-in gestures.

\textbf{Multi-modality adaptations:}
Our demonstrations on adapting app UIs with embedded alternative input and output features (e.g., Figure \ref{adapted app instances example}) were received very positively. Across all three focus groups, participants welcomed the option for seniors to use an alternative input method, specifically speech-to-text, instead of the traditional on-screen keyboard. To a lesser extent (in two out of three focus groups), participants also found text-to-speech features to be a helpful addition. Despite this positive feedback, seniors expressed scepticism about the viability of speech-to-text technology due to its limitations. A common theme across all focus groups was their prior experiences with inaccurate voice input, often attributed to the diversity in accents, dialects, intonations, and pronunciations. Additionally, the situational appropriateness of alternative input/output modalities was a key concern raised by the majority (two out of three focus groups). Participants noted that they would feel uncomfortable using audio-based features in public settings due to both privacy concerns and social etiquette.

\textbf{Navigation adaptations:}
The adaptations we demonstrated, which transformed a traditionally non-sequential, single-page form into a dynamic, wizard-pattern form, were highly appreciated by participants. A recurring theme across all three focus groups was that this approach made the presented information feel less overwhelming for seniors. Participants expressed that succinctly displaying information on-screen, combined with a logical and intuitive UI flow similar to a wizard-pattern form, was a clear improvement over the traditional static form layout typically used for senior end users.

\subsection{Limitations}

\subsubsection{Limitations with \toolname}

Apart from the weaknesses noted by our developer study participants, another limitation of our current prototype is that the Adapt DSL is tightly coupled with Flutter's widget structure. To make the proposed approach more platform-agnostic, the Adapt DSL would need to be more abstract in specifying adaptation operations. Additionally, the code generation pipeline would have to be extended and rigorously tested to ensure that the UI adaptations align with the specific programming language or framework in use, ensuring flexibility across different platforms. 

Furthermore, more complex adaptation operations, such as senior requiring changes to the UI workflow of an app, may necessitate further extensions beyond what we have proposed. Seniors may have age-related needs that require adaptations to app functionalities such as dynamic page routing, customised workflow patterns, contextual help and guidance, multi-modal navigation support, and backward navigation or undo functionality. Implementing such adaptations would require considerable modifications to the app’s source code, making it difficult to automate these operations through DSL inputs in a platform- or framework-agnostic manner.

The issues mentioned above are further exacerbated by the inherent rigidity of Model-Driven Engineering (MDE) approaches compared to traditional software development tools. MDE pipelines are typically interconnected through multiple model-to-model and model-to-text transformations. As a result, additional care must be taken to ensure that these transformation processes are both flexible and robust, in order to avoid excessive "round-trip maintenance." Without careful design, this could hinder the ability to efficiently modify models and regenerate code. Ensuring flexibility in the transformation steps is critical to guarantee that the final output generates executable applications or functional code stubs without frequent manual intervention.

\subsubsection{Limitations with user studies}

Recruiting participants with at least six months of Flutter experience was challenging, likely due to Flutter's relative novelty, with its 1.0 release in late 2018 \cite{flutter}. Despite successfully recruiting 16 Flutter developers with varying experience levels, two participants ([D2] and [D3]) had familiarity with Flutter but not in a professional capacity. However, they were experts in developing software for seniors and well-versed in state-of-the-art accessibility guidelines such as WCAG \cite{wcag} and ATAG \cite{atag}.

\subsection{Future work}

Our evaluation study with developers revealed many suggestions for improvements and feature additions. Our next iteration of \toolname~will especially focus on enhancements to code generation-related tasks where we want to make our generated code more readable, maintainable, and reusable mitigating the issues raised by our developer participants about the current prototype. In addition, many improvements were also proposed regarding our two novel DSLs such as metamodel enhancements and model editor enhancements. 

Currently, we support only design-time UI adaptations. The next step is to extend these capabilities to enable user-requested run-time adaptations. Rather than pursuing automated self-adaptation, as proposed by Yigitbas et al. \cite{yigitbas2020}, we focus on empowering seniors to personalise their app UIs autonomously, ensuring the app does not dictate their needs. Our focus group evaluation study has already explored the barriers and enablers of run-time personalisation among seniors. To achieve this, we consider two approaches: a traditional accessibility settings-based method and a novel Large Language Model (LLM)-based approach, where LLMs transform multimodal natural language inputs into abstract DSL models, allowing us to reuse the MDE process in \toolname~to trigger app adaptations.

In addition, we believe there is a growing need to provide end-users, including seniors, with mechanisms for accessing contextual privacy information. Our proposed approach is well suited for this task, as similar to how source code modifications address age-related needs, we can extend the DSLs and MDE code generation pipeline to embed privacy-related information e.g. highlighting sensitive data entry items seamlessly.

\section{Related Work} \label{related work}

% put this at end of paper - key things to compare to your approach - MDE solutions, adaptive UI solutions etc

Our study is primarily inspired by Yigitbas et al.’s \cite{yigitbas2020} work on an MDE-based approach for self-adaptive UIs, which also proposes two DSLs conceptually similar to ours. Their ContextML DSL models a user's context-of-use parameters, while AdaptML functions as a rules engine for conditionally applying UI changes. A key methodological difference is how UI changes are applied: while we modify the source code directly, Yigitbas et al. \cite{yigitbas2020} adopt an abstract UI model defined with the Interactive Flow Modelling Language (IFML) \cite{brambilla2014}. One limitation of their approach is the depth of modelling in their DSLs; for instance, ContextML lacks comprehensive support for age-related needs, such as UI preferences, and hearing and mobility impairments. Nonetheless, their study remains the most mature in the MDE + Adaptive UI subdomain so far. We also appreciate that it is evaluated with real end-users, though it does not consider developers' feedback -- a crucial factor in understanding the enablers and barriers to real-world adoption by the software development community.

Another important study is by Bendaly Hlaoui et al. \cite{bendaly2018}, who propose an MDE approach for design-time UI adaptations tailored to disabled users. They achieve this transformation through two models: (1) an accessibility ontological instance representing an end user's context-of-use parameters and (2) a UI model reverse-engineered from an existing non-adapted UI. These models serve as inputs to an adaptation process that applies adaptation rules to transform the non-adapted model into an adapted one, which is then converted into the final UI. Their context-of-use modelling depth is well-developed and has informed the level of comprehensiveness needed for our study. However, it is unclear whether they produced an MDE prototype, as the authors mention the final transformation from an adapted abstract UI model to an executable UI as future work. Without this detail and a subsequent proof-of-concept evaluation, assessing the effectiveness of their reverse engineering approach remains challenging, particularly in terms of its compatibility with modern development tools and frameworks such as Flutter or React Native.

Minon et al. \cite{minon2015} propose an approach similar to ours, incorporating both context-of-use modelling and adaptation rules. Their tool, the Adaptation Integration System (AIS), automatically tailors UIs to meet the accessibility requirements of user groups with visual, hearing, and cognitive impairments at both runtime and design time. AIS includes a compilation of UI adaptation rules designed for these user groups. At design time, the MDE UI tool designer manually inputs a UI model at any abstraction level of the CAMELEON framework \cite{calvary2002}, along with parameters indicating the user’s disability. The system then generates an adapted UI model accordingly. However, the paper lacks detailed information about its metamodels, making it difficult to assess the depth of AIS’s modelling capabilities. The provided evidence, such as prototype applications and model examples, suggests that their proof-of-concept is less detailed than both ours and that of Yigitbas et al. \cite{yigitbas2020}. Additionally, unlike the studies mentioned above, Minon et al. \cite{minon2015} do not include any user or developer evaluation, limiting insights into its practical effectiveness.

A fundamentally different paradigm is proposed by Akiki et al. \cite{akiki2016} for MDE-based adaptive UI generation. They adopt a Role-Based UI adaptation mechanism that provides end-users with a minimal feature set and an optimal layout based on their context-of-use scenarios. In this approach, UI elements and adaptation rules are treated as accessibility resources and assigned to user accounts as ‘roles.’ At design time, a user’s account is allocated roles linked to context-of-use factors such as disabilities and culture. However, their modelling approach lacks sufficient depth to address the needs of seniors, as the metamodels support only graphical/layout adaptations and do not include multi-modality features such as text-to-speech or speech-to-text. While their evaluation included eight participants over the age of 50 (out of N=23), they did not explore the impact of age-related impairments or user preferences on end-user satisfaction. This omission raises doubts about the applicability of this approach in addressing the diverse and highly personalised accessibility needs of seniors.

In conclusion, almost all existing studies lack the modelling depth necessary to capture the UI accessibility and adaptation needs of seniors at the metamodel layer, with the exception of Bendaly Hlaoui et al. \cite{bendaly2018}. However, even in that case, the comprehensiveness of the models does not necessarily translate to their usability, as evidenced by our experience with developer user studies. For instance, when examining the DSLs and their examples, several key questions arose. To name a few: Are they intuitive for developers? Do they allow developers to embed metadata, such as references to accessibility guidelines and explanations of model functionality? Do they provide flexibility in modelling groups of users via personas rather than individual users? Unfortunately, in most cases, these questions remained unanswered due to the absence of concrete proof-of-concept implementations in the existing studies. Only Yigitbas et al. \cite{yigitbas2020} had such a prototype, but this ties into another major limitation: the lack of evaluation studies capturing insights from software developers on the usability of DSLs and MDE processes. Without such insights, it is difficult to assess how these proposed approaches can be effectively adopted by real-world software developers. 

\section{Summary} \label{conclusion}

Senior citizens are a crucial part of our society, yet they continue to experience accessibility barriers in apps. The root cause of this issue lies with us, the software practitioners, making it our responsibility to identify solutions that address the age-related accessibility and adaptation needs of seniors. This requires strategies to overcome challenges such as limited time and resources, which are common in real-world software development. In this study, we propose one such strategy: \toolname, a Model-Driven Engineering (MDE) approach that enables developers to automate the generation of adaptive Flutter app instances based on DSL models that define the accessibility-adaptation needs of senior users.

This paper presents the overall architecture, design, and implementation of \toolname. We also describe two user studies conducted with 18 software developers and 22 senior end-users to evaluate the strengths and weaknesses of the current iteration of our tool prototype.

Additionally, we elaborate on key research contributions: (1) Two novel DSLs that, together, express the age-related accessibility-adaptation needs of seniors, designed to be both comprehensive in modelling capabilities and accessible to developers; (2) A novel MDE approach that utilises these DSLs to perform a dynamic range of source code adaptations in any Flutter app while seamlessly integrating with traditional tools/processes and enhancing the developer experience; and (3) A thorough evaluation involving both software developers and senior end-users to gather real-world insights into designing and developing adaptive apps.

\backmatter

% \bmhead{Supplementary information}

% If your article has accompanying supplementary file/s please state so here. 

% Authors reporting data from electrophoretic gels and blots should supply the full unprocessed scans for key as part of their Supplementary information. This may be requested by the editorial team/s if it is missing.

% Please refer to Journal-level guidance for any specific requirements.

\bmhead{Acknowledgements}

Authors are supported by Australian Research Council (ARC) Laureate Fellowship FL190100035. Our sincere gratitude goes to the participants who took part in the user study.

\section*{Author Contribution}

Shavindra Wickramathilaka: Conceptualisation, Software Development, User Study Conduction, Writing - Original Draft, Review \& Editing. John Grundy: Conceptualisation, Supervision, Writing - Review \& Editing. Kashumi Madampe: Conceptualisation, Supervision, User Study Conduction, Writing - Review \& Editing. Omar Haggag: Conceptualisation, Supervision, Writing - Review \& Editing.

% \begin{appendices}

% \section{Section title of first appendix}\label{secA1}

% An appendix contains supplementary information that is not an essential part of the text itself but which may be helpful in providing a more comprehensive understanding of the research problem or it is information that is too cumbersome to be included in the body of the paper.

%%=============================================%%
%% For submissions to Nature Portfolio Journals %%
%% please use the heading ``Extended Data''.   %%
%%=============================================%%

%%=============================================================%%
%% Sample for another appendix section			       %%
%%=============================================================%%

%% \section{Example of another appendix section}\label{secA2}%
%% Appendices may be used for helpful, supporting or essential material that would otherwise 
%% clutter, break up or be distracting to the text. Appendices can consist of sections, figures, 
%% tables and equations etc.

% \end{appendices}

%%===========================================================================================%%
%% If you are submitting to one of the Nature Portfolio journals, using the eJP submission   %%
%% system, please include the references within the manuscript file itself. You may do this  %%
%% by copying the reference list from your .bbl file, paste it into the main manuscript .tex %%
%% file, and delete the associated \verb+\bibliography+ commands.                            %%
%%===========================================================================================%%

\bibliography{bibliography}% common bib file
%% if required, the content of .bbl file can be included here once bbl is generated
%%\input sn-article.bbl


\end{document}
