%%
%DIF LATEXDIFF DIFFERENCE FILE
%DIF DEL old\main.tex   Wed Dec 11 02:30:59 2024
%DIF ADD new\main.tex   Wed Dec 11 03:07:31 2024
%% This is file `sample-sigconf-authordraft.tex',
%% generated with the docstrip utility.
%%
%% The original source files were:
%%
%% samples.dtx  (with options: `all,proceedings,bibtex,authordraft')
%% 
%% IMPORTANT NOTICE:
%% 
%% For the copyright see the source file.
%% 
%% Any modified versions of this file must be renamed
%% with new filenames distinct from sample-sigconf-authordraft.tex.
%% 
%% For distribution of the original source see the terms
%% for copying and modification in the file samples.dtx.
%% 
%% This generated file may be distributed as long as the
%% original source files, as listed above, are part of the
%% same distribution. (The sources need not necessarily be
%% in the same archive or directory.)
%%
%%
%% Commands for TeXCount
%TC:macro \cite [option:text,text]
%TC:macro \citep [option:text,text]
%TC:macro \citet [option:text,text]
%TC:envir table 0 1
%TC:envir table* 0 1
%TC:envir tabular [ignore] word
%TC:envir displaymath 0 word
%TC:envir math 0 word
%TC:envir comment 0 0
%%
%%
%% The first command in your LaTeX source must be the \documentclass
%% command.
%%
%% For submission and review of your manuscript please change the
%% command to \documentclass[manuscript, screen, review]{acmart}.
%%
%% When submitting camera ready or to TAPS, please change the command
%% to \documentclass[sigconf]{acmart} or whichever template is required
%% for your publication.
%%
%%
\documentclass[manuscript,review,anonymous]{acmart}
%%
%% \BibTeX command to typeset BibTeX logo in the docs
\AtBeginDocument{%
  \providecommand\BibTeX{{%
    Bib\TeX}}}

%% Rights management information.  This information is sent to you
%% when you complete the rights form.  These commands have SAMPLE
%% values in them; it is your responsibility as an author to replace
%% the commands and values with those provided to you when you
%% complete the rights form.
\setcopyright{acmlicensed}
\copyrightyear{2018}
\acmYear{2018}
\acmDOI{XXXXXXX.XXXXXXX}

%% These commands are for a PROCEEDINGS abstract or paper.
\acmConference[Conference acronym 'XX]{Make sure to enter the correct
  conference title from your rights confirmation emai}{June 03--05,
  2018}{Woodstock, NY}
%%
%%  Uncomment \acmBooktitle if the title of the proceedings is different
%%  from ``Proceedings of ...''!
%%
%%\acmBooktitle{Woodstock '18: ACM Symposium on Neural Gaze Detection,
%%  June 03--05, 2018, Woodstock, NY}
\acmISBN{978-1-4503-XXXX-X/18/06}


%%
%% Submission ID.
%% Use this when submitting an article to a sponsored event. You'll
%% receive a unique submission ID from the organizers
%% of the event, and this ID should be used as the parameter to this command.
%%\acmSubmissionID{123-A56-BU3}

%%
%% For managing citations, it is recommended to use bibliography
%% files in BibTeX format.
%%
%% You can then either use BibTeX with the ACM-Reference-Format style,
%% or BibLaTeX with the acmnumeric or acmauthoryear sytles, that include
%% support for advanced citation of software artefact from the
%% biblatex-software package, also separately available on CTAN.
%%
%% Look at the sample-*-biblatex.tex files for templates showcasing
%% the biblatex styles.
%%

%%
%% The majority of ACM publications use numbered citations and
%% references.  The command \citestyle{authoryear} switches to the
%% "author year" style.
%%
%% If you are preparing content for an event
%% sponsored by ACM SIGGRAPH, you must use the "author year" style of
%% citations and references.
%% Uncommenting
%% the next command will enable that style.
%%\citestyle{acmauthoryear}
\usepackage{amsmath}
\usepackage{graphicx}
\usepackage{booktabs}
\usepackage{subcaption}
\usepackage[normalem]{ulem}
\useunder{\uline}{\ul}{}
%%
%% end of the preamble, start of the body of the document source.
%DIF PREAMBLE EXTENSION ADDED BY LATEXDIFF
%DIF UNDERLINE PREAMBLE %DIF PREAMBLE
\RequirePackage[normalem]{ulem} %DIF PREAMBLE
\RequirePackage{color}\definecolor{RED}{rgb}{1,0,0}\definecolor{BLUE}{rgb}{0,0,1} %DIF PREAMBLE
\providecommand{\DIFadd}[1]{{\protect\color{blue}\uwave{#1}}} %DIF PREAMBLE
\providecommand{\DIFdel}[1]{{\protect\color{red}\sout{#1}}} %DIF PREAMBLE
%DIF SAFE PREAMBLE %DIF PREAMBLE
\providecommand{\DIFaddbegin}{} %DIF PREAMBLE
\providecommand{\DIFaddend}{} %DIF PREAMBLE
\providecommand{\DIFdelbegin}{} %DIF PREAMBLE
\providecommand{\DIFdelend}{} %DIF PREAMBLE
\providecommand{\DIFmodbegin}{} %DIF PREAMBLE
\providecommand{\DIFmodend}{} %DIF PREAMBLE
%DIF FLOATSAFE PREAMBLE %DIF PREAMBLE
\providecommand{\DIFaddFL}[1]{\DIFadd{#1}} %DIF PREAMBLE
\providecommand{\DIFdelFL}[1]{\DIFdel{#1}} %DIF PREAMBLE
\providecommand{\DIFaddbeginFL}{} %DIF PREAMBLE
\providecommand{\DIFaddendFL}{} %DIF PREAMBLE
\providecommand{\DIFdelbeginFL}{} %DIF PREAMBLE
\providecommand{\DIFdelendFL}{} %DIF PREAMBLE
\newcommand{\DIFscaledelfig}{0.5}
%DIF HIGHLIGHTGRAPHICS PREAMBLE %DIF PREAMBLE
\RequirePackage{settobox} %DIF PREAMBLE
\RequirePackage{letltxmacro} %DIF PREAMBLE
\newsavebox{\DIFdelgraphicsbox} %DIF PREAMBLE
\newlength{\DIFdelgraphicswidth} %DIF PREAMBLE
\newlength{\DIFdelgraphicsheight} %DIF PREAMBLE
% store original definition of \includegraphics %DIF PREAMBLE
\LetLtxMacro{\DIFOincludegraphics}{\includegraphics} %DIF PREAMBLE
\newcommand{\DIFaddincludegraphics}[2][]{{\color{blue}\fbox{\DIFOincludegraphics[#1]{#2}}}} %DIF PREAMBLE
\newcommand{\DIFdelincludegraphics}[2][]{% %DIF PREAMBLE
\sbox{\DIFdelgraphicsbox}{\DIFOincludegraphics[#1]{#2}}% %DIF PREAMBLE
\settoboxwidth{\DIFdelgraphicswidth}{\DIFdelgraphicsbox} %DIF PREAMBLE
\settoboxtotalheight{\DIFdelgraphicsheight}{\DIFdelgraphicsbox} %DIF PREAMBLE
\scalebox{\DIFscaledelfig}{% %DIF PREAMBLE
\parbox[b]{\DIFdelgraphicswidth}{\usebox{\DIFdelgraphicsbox}\\[-\baselineskip] \rule{\DIFdelgraphicswidth}{0em}}\llap{\resizebox{\DIFdelgraphicswidth}{\DIFdelgraphicsheight}{% %DIF PREAMBLE
\setlength{\unitlength}{\DIFdelgraphicswidth}% %DIF PREAMBLE
\begin{picture}(1,1)% %DIF PREAMBLE
\thicklines\linethickness{2pt} %DIF PREAMBLE
{\color[rgb]{1,0,0}\put(0,0){\framebox(1,1){}}}% %DIF PREAMBLE
{\color[rgb]{1,0,0}\put(0,0){\line( 1,1){1}}}% %DIF PREAMBLE
{\color[rgb]{1,0,0}\put(0,1){\line(1,-1){1}}}% %DIF PREAMBLE
\end{picture}% %DIF PREAMBLE
}\hspace*{3pt}}} %DIF PREAMBLE
} %DIF PREAMBLE
\LetLtxMacro{\DIFOaddbegin}{\DIFaddbegin} %DIF PREAMBLE
\LetLtxMacro{\DIFOaddend}{\DIFaddend} %DIF PREAMBLE
\LetLtxMacro{\DIFOdelbegin}{\DIFdelbegin} %DIF PREAMBLE
\LetLtxMacro{\DIFOdelend}{\DIFdelend} %DIF PREAMBLE
\DeclareRobustCommand{\DIFaddbegin}{\DIFOaddbegin \let\includegraphics\DIFaddincludegraphics} %DIF PREAMBLE
\DeclareRobustCommand{\DIFaddend}{\DIFOaddend \let\includegraphics\DIFOincludegraphics} %DIF PREAMBLE
\DeclareRobustCommand{\DIFdelbegin}{\DIFOdelbegin \let\includegraphics\DIFdelincludegraphics} %DIF PREAMBLE
\DeclareRobustCommand{\DIFdelend}{\DIFOaddend \let\includegraphics\DIFOincludegraphics} %DIF PREAMBLE
\LetLtxMacro{\DIFOaddbeginFL}{\DIFaddbeginFL} %DIF PREAMBLE
\LetLtxMacro{\DIFOaddendFL}{\DIFaddendFL} %DIF PREAMBLE
\LetLtxMacro{\DIFOdelbeginFL}{\DIFdelbeginFL} %DIF PREAMBLE
\LetLtxMacro{\DIFOdelendFL}{\DIFdelendFL} %DIF PREAMBLE
\DeclareRobustCommand{\DIFaddbeginFL}{\DIFOaddbeginFL \let\includegraphics\DIFaddincludegraphics} %DIF PREAMBLE
\DeclareRobustCommand{\DIFaddendFL}{\DIFOaddendFL \let\includegraphics\DIFOincludegraphics} %DIF PREAMBLE
\DeclareRobustCommand{\DIFdelbeginFL}{\DIFOdelbeginFL \let\includegraphics\DIFdelincludegraphics} %DIF PREAMBLE
\DeclareRobustCommand{\DIFdelendFL}{\DIFOaddendFL \let\includegraphics\DIFOincludegraphics} %DIF PREAMBLE
%DIF AMSMATHULEM PREAMBLE %DIF PREAMBLE
\makeatletter %DIF PREAMBLE
\let\sout@orig\sout %DIF PREAMBLE
\renewcommand{\sout}[1]{\ifmmode\text{\sout@orig{\ensuremath{#1}}}\else\sout@orig{#1}\fi} %DIF PREAMBLE
\makeatother %DIF PREAMBLE
%DIF COLORLISTINGS PREAMBLE %DIF PREAMBLE
\RequirePackage{listings} %DIF PREAMBLE
\RequirePackage{color} %DIF PREAMBLE
\lstdefinelanguage{DIFcode}{ %DIF PREAMBLE
%DIF DIFCODE_UNDERLINE %DIF PREAMBLE
  moredelim=[il][\color{red}\sout]{\%DIF\ <\ }, %DIF PREAMBLE
  moredelim=[il][\color{blue}\uwave]{\%DIF\ >\ } %DIF PREAMBLE
} %DIF PREAMBLE
\lstdefinestyle{DIFverbatimstyle}{ %DIF PREAMBLE
	language=DIFcode, %DIF PREAMBLE
	basicstyle=\ttfamily, %DIF PREAMBLE
	columns=fullflexible, %DIF PREAMBLE
	keepspaces=true %DIF PREAMBLE
} %DIF PREAMBLE
\lstnewenvironment{DIFverbatim}{\lstset{style=DIFverbatimstyle}}{} %DIF PREAMBLE
\lstnewenvironment{DIFverbatim*}{\lstset{style=DIFverbatimstyle,showspaces=true}}{} %DIF PREAMBLE
\lstset{extendedchars=\true,inputencoding=utf8}

%DIF END PREAMBLE EXTENSION ADDED BY LATEXDIFF

\begin{document}

%%
%% The "title" command has an optional parameter,
%% allowing the author to define a "short title" to be used in page headers.
\DIFdelbegin %DIFDELCMD < \title[Short Title]{%%%
\DIFdelend \DIFaddbegin \title[Breaking the Familiarity Bias]{\DIFaddend Breaking the Familiarity Bias: Employing Virtual Reality Environments to Enhance Team Formation and Inclusion}

%%
%% The "author" command and its associated commands are used to define
%% the authors and their affiliations.
%% Of note is the shared affiliation of the first two authors, and the
%% "authornote" and "authornotemark" commands
%% used to denote shared contribution to the research.
\author{Mariana Fernandez-Espinosa}
\email{mferna23@nd.edu}
\DIFdelbegin %DIFDELCMD < 

%DIFDELCMD < %%%
\DIFdelend \DIFaddbegin \affiliation{%
  \institution{University of Notre Dame}
  \city{Notre Dame}
  \state{Indiana}
  \country{USA}
}
\DIFaddend \author{Kara Clouse}
\authornote{Authors worked on this project as part of their capstone undergraduate program.}
\DIFdelbegin %DIFDELCMD < 

%DIFDELCMD < %%%
\DIFdelend \DIFaddbegin \affiliation{%
  \institution{University of Notre Dame}
  \city{Notre Dame}
  \state{Indiana}
  \country{USA}
}
\DIFaddend \author{Dylan Sellars}
\authornotemark[1]
\DIFdelbegin %DIFDELCMD < 

%DIFDELCMD < %%%
\DIFdelend \DIFaddbegin \affiliation{%
  \institution{University of Notre Dame}
  \city{Notre Dame}
  \state{Indiana}
  \country{USA}
}
\DIFaddend \author{Danny Tong}
\authornotemark[1]
\DIFdelbegin %DIFDELCMD < 

%DIFDELCMD < %%%
\DIFdelend \DIFaddbegin \affiliation{%
  \institution{University of Notre Dame}
  \city{Notre Dame}
  \state{Indiana}
  \country{USA}
}
\DIFaddend \author{Michael Bsales}
\authornotemark[1]
\DIFdelbegin %DIFDELCMD < 

%DIFDELCMD < %%%
\DIFdelend \DIFaddbegin \affiliation{%
  \institution{University of Notre Dame}
  \city{Notre Dame}
  \state{Indiana}
  \country{USA}
}
\DIFaddend \author{Sophonie Alcindor}
\authornotemark[1]
\DIFdelbegin %DIFDELCMD < 

%DIFDELCMD < %%%
\DIFdelend \DIFaddbegin \affiliation{%
  \institution{University of Notre Dame}
  \city{Notre Dame}
  \state{Indiana}
  \country{USA}
}
\DIFaddend \author{Tim Hubbard}
\DIFdelbegin %DIFDELCMD < 

%DIFDELCMD < %%%
\DIFdelend \DIFaddbegin \affiliation{%
  \institution{University of Notre Dame}
  \city{Notre Dame}
  \state{Indiana}
  \country{USA}
}
\DIFaddend \author{Michael Villano}
\DIFdelbegin %DIFDELCMD < 

%DIFDELCMD < %%%
\DIFdelend \DIFaddbegin \affiliation{%
  \institution{University of Notre Dame}
  \city{Notre Dame}
  \state{Indiana}
  \country{USA}
}
\DIFaddend \author{Diego Gomez-Zara}
\DIFdelbegin %DIFDELCMD < \orcid{1234-5678-9012}
%DIFDELCMD < %%%
\DIFdelend \DIFaddbegin \orcid{0000-0002-4609-6293}
\DIFaddend \email{dgomezara@nd.edu}
\affiliation{%
  \institution{University of Notre Dame}
  \city{Notre Dame}
  \state{Indiana}
  \country{USA}
}
\DIFdelbegin %DIFDELCMD < 

%DIFDELCMD < %%%
\DIFdelend \DIFaddbegin \additionalaffiliation{%
  \institution{Pontificia Universidad Cat\'olica de Chile}
  \department{Facultad de Comunicaciones}
  \city{Santiago}
  \country{Chile}
}
\DIFaddend %%
%% By default, the full list of authors will be used in the page
%% headers. Often, this list is too long, and will overlap
%% other information printed in the page headers. This command allows
%% the author to define a more concise list
%% of authors' names for this purpose.
\renewcommand{\shortauthors}{Fernandez-Espinosa et al.}

%%
%% The abstract is a short summary of the work to be presented in the
%% article.
\begin{abstract}
Team \DIFdelbegin \DIFdel{familiarity }\DIFdelend \DIFaddbegin \DIFadd{closeness }\DIFaddend provides the foundations of trust and communication, contributing to teams' success and viability. However, newcomers often struggle to be included in a team since incumbents tend to interact more with other existing members. Previous research suggests that \DIFdelbegin \DIFdel{computer-mediated }\DIFdelend \DIFaddbegin \DIFadd{online }\DIFaddend communication technologies can help team inclusion by mitigating members' perceived differences. In this study, we test how virtual reality (VR) can \DIFdelbegin \DIFdel{mitigate familiarity bias }\DIFdelend \DIFaddbegin \DIFadd{promote team closeness }\DIFaddend when forming teams. We conducted a between-subject experiment with teams working in-person and VR, where two members interacted first\DIFaddbegin \DIFadd{, }\DIFaddend and then a third member was added later to conduct a \DIFdelbegin \DIFdel{task. We }\DIFdelend \DIFaddbegin \DIFadd{hidden-profile task. Participants }\DIFaddend evaluated how close \DIFdelbegin \DIFdel{participants felt with the newcomer }\DIFdelend \DIFaddbegin \DIFadd{they felt with their teammates }\DIFaddend after the task was completed. Our results show that VR newcomers felt closer to the incumbents than in-person newcomers. However, incumbents' closeness to newcomers did not vary across conditions. We discuss the implications of these findings and offer suggestions for how VR \DIFdelbegin \DIFdel{technologies }\DIFdelend can promote inclusion.
\end{abstract}

%%
%% The code below is generated by the tool at http://dl.acm.org/ccs.cfm.
%% Please copy and paste the code instead of the example below.
%%
\begin{CCSXML}
<ccs2012>
   <concept>
       <concept_id>10003120.10003121.10003122.10011749</concept_id>
       <concept_desc>Human-centered computing~Laboratory experiments</concept_desc>
       <concept_significance>300</concept_significance>
       </concept>
   <concept>
       <concept_id>10003120.10003121.10011748</concept_id>
       <concept_desc>Human-centered computing~Empirical studies in HCI</concept_desc>
       <concept_significance>500</concept_significance>
       </concept>
   <concept>
       <concept_id>10003120.10003130.10003131</concept_id>
       <concept_desc>Human-centered computing~Collaborative and social computing theory, concepts and paradigms</concept_desc>
       <concept_significance>500</concept_significance>
       </concept>
   <concept>
       <concept_id>10003120.10003121.10003124.10011751</concept_id>
       <concept_desc>Human-centered computing~Collaborative interaction</concept_desc>
       <concept_significance>300</concept_significance>
       </concept>
 </ccs2012>
\end{CCSXML}

\ccsdesc[300]{Human-centered computing~Laboratory experiments}
\ccsdesc[500]{Human-centered computing~Empirical studies in HCI}
\ccsdesc[500]{Human-centered computing~Collaborative and social computing theory, concepts and paradigms}
\ccsdesc[300]{Human-centered computing~Collaborative interaction}

%%
%% Keywords. The author(s) should pick words that accurately describe
%% the work being presented. Separate the keywords with commas.
\keywords{Familiarity Bias, Newcomers, Team Formation, Team Inclusion, Virtual Reality}
%% A "teaser" image appears between the author and affiliation
%% information and the body of the document, and typically spans the
%% page.

\received{20 February 2007}
\received[revised]{12 March 2009}
\received[accepted]{5 June 2009}

%%
%% This command processes the author and affiliation and title
%% information and builds the first part of the formatted document.
\maketitle

\section{Introduction}
\label{introduction}
%DIF < Bringing new members to a team is one of the most effective ways to solve problems across various fields \cite{cohen1997makes, assbeihat2016impact}. Whether in face-to-face or virtual settings, collaboration serves as a key predictor of successful innovation, creative team outcomes, and effective decision-making \cite{barczak2010antecedents, stavros2015promoting}.
\DIFdelbegin %DIFDELCMD < 

%DIFDELCMD < %%%
%DIF < Numerous studies have examined variables that can influence team performance by analyzing patterns within teams, such as the balance of skills \cite{cornide2019identification}, personality traits \cite{goldberg2013alternative, lim2023kill}, demographic diversity \cite{tasheva2019integrating, gomez2020impact}, and the degree of team familiarity \cite{witmer2022systematic, maruthappu2016impact, pasarakonda2023team}. However, the behavior of these variables differs between digital and face-to-face environments.
%DIFDELCMD < 

%DIFDELCMD < %%%
%DIF < The COVID-19 pandemic and recent technological advancements significantly accelerated the adoption of remote work \cite{almeida2020challenges, mark2022introduction, ziemba2023remote}, demonstrating that many jobs can be effectively performed from home \cite{aloisi2022essential}. Although remote work has declined from its peak during the pandemic, it remains three to four times more common than in 2019 \cite{usnews_remote_work_2024}. This shift has opened up opportunities for new forms of remote collaboration, such as immersion in virtual environments \cite{rickel1999virtual, weiss1998virtual}. These environments foster new team dynamics, making the exploration of team familiarity a critical factor for effective collaboration, particularly in settings with characteristics not presented in other digital environments. 
%DIFDELCMD < 

%DIFDELCMD < %%%
\DIFdelend Bringing new members to a team is one of the most effective ways to achieve higher levels of originality, innovation, learning, and performance \cite{zeng2021fresh,LEWIS2007159}. Newcomers can bring new resources to a group, including knowledge, skills, and social connections \cite{yuan2020making}. Moreover, organizations and groups are not static. Employees will leave due to turnover, promotions, or transfers to another unit. As such, teams are required to recruit and include new members to continue their work and facilitate new ideas \DIFdelbegin \DIFdel{. Scientific teams, organizational teams, and artistic crews are some examples of teams that benefit from bringing newcomers since they recombine teams' expertise and talents in unprecedented ways }\DIFdelend \cite{guimera2005}. \DIFdelbegin %DIFDELCMD < 

%DIFDELCMD < %%%
\DIFdelend However, incorporating new members into an existing group can be a challenging task \cite{kraut2010dealing}. Newcomers have \DIFdelbegin \DIFdel{also not yet developed the commitment or }\DIFdelend \DIFaddbegin \DIFadd{to develop a commitment and }\DIFaddend understanding of the group\DIFaddbegin \DIFadd{'s }\DIFaddend dynamics, requiring more time to learn and to be included in a group. Moreover, existing team members (i.e., incumbents) are likely to continue interacting with other existing members given their established relationships. \DIFdelbegin \textit{\DIFdel{Team familiarity}}%DIFAUXCMD
\DIFdelend \DIFaddbegin \DIFadd{Team familiarity}\DIFaddend , the existing knowledge of other members of the team \cite{pasarakonda2023team, wilson2008perceived}, \DIFdelbegin \DIFdel{has critical implications for the team's performance }%DIFDELCMD < \cite{assudani2011role, maynard2019really}%%%
\DIFdel{. Research has shown that familiarity among geographically dispersed team members can mitigate the negative effects of spatial separation and positively influence team satisfaction }%DIFDELCMD < \cite{stark2009analysis, bierly2009moderating}%%%
\DIFdel{. Yet, team familiarity }\DIFdelend exacerbates the attachment to previous collaborators and similar individuals in a group, excluding new members and making their inclusion harder \cite{Arcsin2021}.

HCI researchers have studied how online communication systems can \DIFdelbegin \DIFdel{promote the inclusion of new members in a group }\DIFdelend \DIFaddbegin \DIFadd{enhance group dynamics by modifying team members' appearances and facilitating their interactions }\DIFaddend \cite{harris2019joining,10.1145/2998181.2998300,Whiting2020}. \DIFdelbegin \DIFdel{Can }\DIFdelend \DIFaddbegin \DIFadd{While much research has focused on features that foster inclusion among team members, there has been little exploration of the unique dynamics between newcomers and incumbents. Furthermore, while previous studies have investigated how }\DIFaddend online communication systems \DIFdelbegin \DIFdel{help newcomers to feel more connected? As well as help }\DIFdelend \DIFaddbegin \DIFadd{influence team members' perceived closeness, this has been mostly explored at the interpersonal level, with limited consideration of how these perceptions act within a team }\cite{Hall2022}\DIFadd{. Would newcomers be perceived differently depending on the online communication system in which they meet their new team? Would }\DIFaddend incumbents be more \DIFdelbegin \DIFdel{connected with newcomers ? }\DIFdelend \DIFaddbegin \DIFadd{welcoming if the differences between them and the newcomers were less perceptible? 
%DIF > Furthermore, research in VR has examined features such as avatar customization \cite{10.1145/3584931.3606992, pejsa2017me} and immersion \cite{hudson2019or, lionetti2024role}, can shape the perception of new members. 
}

\DIFaddend In this study, we examined how employing a Virtual Reality (VR) \DIFdelbegin \DIFdel{environment affects team familiarity. Our goal is to investigate whether it reduces bias against new members, whether individuals can develop a sense of closeness with virtually met team members, and how VR technology influences team dynamics. Certain inherent qualities of VR systems can introduce new obstacles to building team familiarity and , consequently, could affect effective collaboration. For instance, the absence of communication cues such as eye contact, smiles, and other nonverbal behaviors, coupled with the sense of anonymity that arises when team members use avatars, can avoid concentrating the social interactions among a few }%DIFDELCMD < \cite{dodds2011talk, roth2016avatar, palmer1995interpersonal}%%%
\DIFdelend \DIFaddbegin \DIFadd{application affects team members' closeness to each other. We chose VR because immersion and avatars could potentially reduce bias against new members }\cite{christofi2017virtual, higgins2023perspective, mal2023impact}\DIFaddend . By minimizing the impact of physical \DIFdelbegin \DIFdel{appearance}\DIFdelend \DIFaddbegin \DIFadd{appearances}\DIFaddend , cultural cues, and other factors that often lead to preconceived ideas and negative stereotypes, VR \DIFdelbegin \DIFdel{environments }\DIFdelend could allow team members to focus more on the \DIFdelbegin \DIFdel{substance }\DIFdelend \DIFaddbegin \DIFadd{essence }\DIFaddend of collaboration rather than interpersonal biases \DIFdelbegin %DIFDELCMD < \cite{latoschik2017effect, latoschik2017effect}%%%
\DIFdel{. In this way, VR not only mitigates biases related to team formation but also fosters }\DIFdelend \DIFaddbegin \cite{latoschik2017effect}\DIFadd{, fostering }\DIFaddend a more inclusive atmosphere. \DIFdelbegin %DIFDELCMD < 

%DIFDELCMD < %%%
\DIFdel{While previous work has focused on the examination of the social challenges of communication and collaboration in VR across various metrics---such as creativity, fatigue, and familiarity }%DIFDELCMD < \cite{moore2020familiarity, souchet2022measuring, hirota2019comparison}%%%
\DIFdel{---little work has explored how VR technology affects individuals’ formation of proximity biases when collaborating remotely and if the VR setting can be a way to reduce familiarity bias}\DIFdelend \DIFaddbegin \DIFadd{Furthermore, we are interested in examining whether using VR can mitigate incumbents' familiarity bias, which is the tendency to collaborate and interact with familiar members rather than with new members}\DIFaddend . Given our interest in testing whether the relationships between newcomers and incumbents can be changed by employing VR, our research questions are:
\begin{itemize}
    \item \DIFdelbegin \DIFdel{(R1) Does }\DIFdelend \DIFaddbegin \textbf{\DIFadd{R1:}} \DIFadd{How does }\DIFaddend the use of Virtual Reality \DIFdelbegin \DIFdel{influence the perceived closeness of incumbents }\DIFdelend \DIFaddbegin \DIFadd{affect incumbents' perceived closeness }\DIFaddend to newcomers?
    \item \DIFdelbegin \DIFdel{(R2) }\DIFdelend \DIFaddbegin \textbf{\DIFadd{R2:}} \DIFaddend How does the use of Virtual Reality affect newcomers\DIFdelbegin \DIFdel{’ }\DIFdelend \DIFaddbegin \DIFadd{' }\DIFaddend perceived closeness to incumbents?
    \item \DIFdelbegin \DIFdel{(R3) }\DIFdelend \DIFaddbegin \textbf{\DIFadd{R3:}} \DIFaddend Can Virtual Reality reduce familiarity bias among incumbents toward newcomers?
\end{itemize}

To answer these research questions, we conducted a controlled between-subject experiment \DIFaddbegin \DIFadd{with 29 teams of size three randomly assigned }\DIFaddend in two different settings: in-person \DIFdelbegin \DIFdel{and in a VR environment. In total, we ran 22 sessions (14 in person and 15 in VR). Participants were tasked with collaborating to make a collective decision based on different data provided to each team member, and each participant then evaluated their perceived closeness with the others to assess team familiarity. For the VR setting, we used }\DIFdelend \DIFaddbegin \DIFadd{(IP) and in a VR multi-player application (}\DIFaddend Meta Horizon Workrooms\DIFdelbegin %DIFDELCMD < \cite{meta_horizon_workrooms} %%%
\DIFdel{in which participants could work together in a virtual workspace. In both settings}\DIFdelend \DIFaddbegin \DIFadd{). In each experimental session}\DIFaddend , two of the three participants were initially brought together into a room (i.e., the incumbents), \DIFdelbegin \DIFdel{while }\DIFdelend \DIFaddbegin \DIFadd{and }\DIFaddend the third participant \DIFdelbegin \DIFdel{was in a separate room (}\DIFdelend \DIFaddbegin \DIFadd{(i.e., }\DIFaddend the newcomer) \DIFdelbegin \DIFdel{. The two participants in the same room were given a list of ice-breaker questions to facilitate familiarity. Once they completed this activity, the third participant joined them in the same room. At the end of the experiment, we used a }\DIFdelend \DIFaddbegin \DIFadd{joined later. The team completed a hidden-profile task, which required all team members to share information and have a collaborative discussion to succeed. Participants completed a post-treatment }\DIFaddend survey to evaluate their work experience and their relationships with their teammates. \DIFdelbegin \DIFdel{From this data, we operationalize a metric to measure team familiarity between incumbents and newcomers, which we then compared between the two experimental conditions}\DIFdelend \DIFaddbegin \DIFadd{We employed mixed methods to analyze the survey data, which included behavioral scales and open-ended questions}\DIFaddend . 

Our findings \DIFdelbegin \DIFdel{revealed that incumbent team membersdid not perceive a significant level of closeness to newcomers, regardless of the work setting. However, newcomers reported feeling significantly closer to incumbents in the VR environment, whereas in the }\DIFdelend \DIFaddbegin \DIFadd{reveal that the impact of VR on team members' closeness was significant only for newcomers. Newcomers in VR reported feeling closer to their incumbents than the newcomers working }\DIFaddend in-person\DIFdelbegin \DIFdel{setting, this perceived closeness was noticeably lower. This suggests that VR plays an important role in facilitating the integration of newcomers by creating a more inclusive and engaging environment compared to traditional in-person interactions}\DIFdelend \DIFaddbegin \DIFadd{, and this effect was mediated by the high levels of perceived similarity experienced in VR. However, the incumbents in VR were not affected in the same way}\DIFaddend . Moreover, \DIFdelbegin \DIFdel{team familiarity bias is reduced in the VR setting, suggesting that VR can create a more balanced interaction space by diminishing the impact of pre-existing relationships. }\DIFdelend \DIFaddbegin \DIFadd{employing VR did not cause any significant differences in incumbents' familiarity bias toward newcomers. Through thematic analysis, we found that the VR setting provided participants with psychological safety by reducing social pressures and enhancing participants' sense of presence and engagement in the collaborative task through its immersive virtual setting. %DIF > offered alternative kinesthetic cues, like virtual gestures, to foster connections, particularly for newcomers. It also
}\DIFaddend 

\DIFdelbegin \DIFdel{Our contributionsare twofold}\DIFdelend \DIFaddbegin \DIFadd{This paper provides the following three contributions}\DIFaddend . First, \DIFdelbegin \DIFdel{we provide insights into how VR can mitigate some of the effects of familiarity against newcomers . We discuss the implications of these findings and provide suggestions for future interventions and social systems designed in VR. Second, we provide a comprehensive study conducted in a research facility. The }\DIFdelend \DIFaddbegin \DIFadd{it deepens our understanding of the asymmetric effects of VR on team members' closeness and familiarity perceptions, as newcomers can feel closer to their team, while incumbents may not experience any differences. Second, it offers a controlled between-subject experiment comparing team members' closeness in VR and IP settings, including a }\DIFaddend de-identified \DIFdelbegin \DIFdel{datasets are available at }%DIFDELCMD < [%%%
\DIFdel{Anonymized}%DIFDELCMD < ] %%%
\DIFdel{to facilitate further reproducibility analysis. }\DIFdelend \DIFaddbegin \DIFadd{dataset available at OSF.io}\footnote{\href{https://osf.io/g4b7r/?view_only=dfa60ae466aa4bbeb9922aaedb6482f1}{\DIFadd{https://osf.io/g4b7r/?view\_only=dfa60ae466aa4bbeb9922aaedb6482f1}}} \DIFadd{to facilitate reproducibility and further research. Third, it discusses how the quantitative and qualitative results can enhance future VR applications for teams and collaborations, elaborating on design implications that reduce team familiarity bias and promote team inclusion.
}\DIFaddend 


%DIF <  team members felt similarly close to each other in the Virtual Reality environment, regardless of whether they had met before. However, our findings also highlight challenges that VR environments pose to team familiarity, such as the inability to recognize emotions from other team members, the lack of facial expressions, limited non-verbal cues, and the potential for miscommunication due to the artificial nature of avatar interactions. 
\DIFdelbegin %DIFDELCMD < 

%DIFDELCMD < %%%
\DIFdelend \section{Literature Review}
\label{literature_review}
\DIFaddbegin \DIFadd{We situate our work in the context of prior studies of teams, including closeness and familiarity, and research in VR.
}\DIFaddend 

%DIF < Key word: Social Cues
\DIFaddbegin \subsection{\DIFadd{Teams}}
\DIFadd{Teams consist of two or more individuals collaborating dynamically to achieve a shared objective }\cite{salas2000teamwork}\DIFadd{. As the foundation of organizational work, teams coordinate efforts to tackle complex tasks, demonstrating interdependencies in workflows, goals, and outcomes that reflect their collective responsibility }\cite{kozlowski2006enhancing, gomez2020taxonomy, baron2003group}\DIFadd{.
}\DIFaddend 

\DIFdelbegin \subsection{\DIFdel{Team Familiarity}}
%DIFAUXCMD
\addtocounter{subsection}{-1}%DIFAUXCMD
%DIF < Concept
\DIFdel{Team familiarity }\DIFdelend \DIFaddbegin \subsubsection{\DIFadd{Team Closeness}}
\DIFadd{Team closeness measures team members' subjective perception of how connected they feel to other members of the team }\cite{Gachter2015}\DIFadd{. Individuals can feel closer to certain team members based on their interactions, and not necessarily on their physical proximity }\cite{Wiese2011}\DIFadd{. While team cohesion usually refers to how united team members are as a whole }\cite{salas2015measuring}\DIFadd{, closeness refers to relationships between pairs, which has been studied at the interpersonal level }\cite{rosh2012too}\DIFadd{. 
}

\DIFadd{Researchers have identified several factors that contribute to building closeness }\cite{moreland1982exposure}\DIFadd{. Time is a key element, enabling teams to build shared knowledge and establish trust among members }\cite{harrison2003time, gillespie2012factors, cattani2013tackling}\DIFadd{. Previous relationships also play an important role, as positive prior experiences can reduce uncertainty among interactions between team members }\cite{de2017attuning, jones2019essentials, dittmer2020cut}\DIFadd{. Lastly, similarity affects closeness since team members share common attributes such as demographics, values, or experiences, they are more likely to develop stronger connections and a sense of familiarity }\cite{hinds2000choosing, ruef2003structure, winship2011homophily}\DIFadd{.
}

\subsubsection{\DIFadd{Team Familiarity}}
\DIFadd{Team familiarity }\DIFaddend refers to the degree of knowledge \DIFaddbegin \DIFadd{that }\DIFaddend team members have about each other and the amount of experience they have working together \DIFdelbegin %DIFDELCMD < \cite{muskat2022team}%%%
\DIFdelend \DIFaddbegin \cite{muskat2022team,mukherjee2019prior}\DIFaddend . Previous research has shown that team familiarity can be a predictor of improved performance, enhancing creativity, efficiency, and the overall quality of a team's output \cite{dittmer2020cut, huckman2009team, witmer2022systematic}. For instance, Sosa demonstrated through a sociometric study that \DIFdelbegin \DIFdel{strong ties between team members can lead to more creative problem-solving }%DIFDELCMD < \cite{sosa2011creative}%%%
\DIFdel{. Through semi-structured interviews and surveys conducted in a software company, his findings indicated that }\DIFdelend teams with stronger interpersonal connections and shared knowledge bases were significantly more likely to generate innovative solutions to complex problems \DIFdelbegin \DIFdel{. Moreover, these teams exhibited a higher degree of engagement and satisfaction in their collaborative efforts, underscoring the value of cultivating strong bonds within creative work environments. }\DIFdelend \DIFaddbegin \cite{sosa2011creative}\DIFadd{. }\DIFaddend Similarly, Staats \cite{staats2012unpacking} \DIFdelbegin \DIFdel{researched the geographic location and hierarchical roles within teams that affect team familiarity and, consequently, team performance. This work }\DIFdelend found that increased familiarity \DIFdelbegin \DIFdel{is }\DIFdelend \DIFaddbegin \DIFadd{was }\DIFaddend associated with an \DIFdelbegin \DIFdel{absolute }\DIFdelend improvement in team \DIFdelbegin \DIFdel{effort}\DIFdelend \DIFaddbegin \DIFadd{performance}\DIFaddend . Lastly, \DIFdelbegin \DIFdel{Huckman }\DIFdelend \DIFaddbegin \DIFadd{Salehi }\DIFaddend et al. \DIFdelbegin %DIFDELCMD < \cite{huckman2009team} %%%
\DIFdel{conducted a comprehensive analysis of team dynamics by exploring (1) the degree of familiarity among team members, and (2) the variability in roles played by individuals within teams. The study revealed that familiarity within teams correlates positively and significantly with both the quality of output and adherence to project timelines. Specifically, higher levels of team familiarity were found to be a robust predictor of enhanced team performance, particularly in terms of effort deviation. As such, familiarity helps team members understand where the skills, roles, and assignments of other members, as well as communicate with them in efficient and safe ways }%DIFDELCMD < \cite{gomez2022search, mukherjee2019prior}%%%
\DIFdel{. 
}%DIFDELCMD < 

%DIFDELCMD < %%%
\DIFdel{Researchers have identified several aspects that explain team familiarity. Time is often highlighted by other researchers as the primary factor that enables teams to build and accumulate shared knowledge within teams }%DIFDELCMD < \cite{harrison2003time}%%%
\DIFdel{. Specifically, aspects such as the number of years spent in a specific role }%DIFDELCMD < \cite{huckman2009team}%%%
\DIFdel{, the average time team }\DIFdelend \DIFaddbegin \cite{10.1145/2998181.2998300} \DIFadd{found that crowd worker teams performed better when some of their }\DIFaddend members have worked together \DIFdelbegin %DIFDELCMD < \cite{gillespie2012factors, cattani2013tackling}%%%
\DIFdel{, and the presence of friendships both inside and outside of work all contribute to the development of team familiarity }%DIFDELCMD < \cite{de2017attuning}%%%
\DIFdel{. 
Nevertheless, other studies have shown that team familiarity not only encompasses time as the primary factor but also includes the quality of relationships between members, team cohesion }%DIFDELCMD < \cite{dittmer2020cut}%%%
\DIFdel{, composition, recognition of shared characteristics, diversity, and the effectiveness of their communication }%DIFDELCMD < \cite{rosen2023build, gully1995meta, marlow2018does, rico2019building}%%%
\DIFdel{.
}\DIFdelend \DIFaddbegin \DIFadd{in the past. 
}\DIFaddend 

While \DIFdelbegin \DIFdel{the benefits of team familiarity }\DIFdelend \DIFaddbegin \DIFadd{its benefits }\DIFaddend have been well documented\DIFdelbegin \DIFdel{by multiple studies, it }\DIFdelend \DIFaddbegin \DIFadd{, team familiarity }\DIFaddend also produces negative effects\DIFdelbegin \DIFdel{, particularly those related to the integration of newcomers into established teams. Team familiarity can inherently }\DIFdelend \DIFaddbegin \DIFadd{. It can }\DIFaddend lead to a culture where incumbents develop close networks \DIFdelbegin \DIFdel{that while enhancing internal cohesion and efficiency, may }\DIFdelend \DIFaddbegin \DIFadd{and }\DIFaddend pose significant barriers to new members, such as exclusion \cite{choi2004minority, joardar2007experimental}, intimidation \cite{topa2016newcomers}, favoritism \cite{balthazard2006dysfunctional}, or communication barriers \cite{kraut2010dealing}. \DIFaddbegin \DIFadd{Prior work shows that newcomers are often perceived as less capable or influential, limiting the team’s ability to explore diverse perspectives and solutions. Moreover, high levels of team familiarity can discourage individuals from challenging established viewpoints, ultimately hindering knowledge sharing and the generation of innovative ideas }\cite{assudani2011role, xie2020curvilinear}\DIFadd{.
}\DIFaddend 

\DIFaddbegin \subsubsection{\DIFadd{Newcomers}}
\DIFaddend Including newcomers to a team is important for organizations. Newcomers often bring fresh ideas, innovative approaches, and valuable resources that are crucial for the ongoing vitality of the group \cite{zeng2021fresh}. \DIFdelbegin \DIFdel{Yet}\DIFdelend \DIFaddbegin \DIFadd{However}\DIFaddend , integrating them effectively into an established group poses considerable challenges. They may face strong biases, and their mere presence, even when adhering to social norms, can be perceived as disruptive by existing members \cite{kraut2010dealing, spertus2001scaling}. These issues can make the \DIFaddbegin \DIFadd{existing }\DIFaddend group less desirable for those familiar with the existing dynamics. For instance, Joardar et al. \cite{joardar2007experimental} highlighted that introducing newcomers often triggers resistance from incumbents who may struggle to accept them as part of the group, potentially leading to tension and disruption of effective team functioning. Furthermore, cultural similarity between the newcomer and the incumbents plays a crucial role in facilitating positive group acceptance. Newcomers facing a significantly different socio-cultural environment are often less familiar with local expectations and norms, which can impede their ability to join smoothly \cite{furnham1982social}. The  \DIFdelbegin \DIFdel{resulting }\DIFdelend lack of cultural similarity can exacerbate the groups' challenges, complicating the newcomers' acceptance and diminishing their team cohesion \cite{nesdale2000immigrant}. 

\DIFdelbegin \subsection{\DIFdel{Team Familiarity Bias}}
%DIFAUXCMD
\addtocounter{subsection}{-1}%DIFAUXCMD
\DIFdel{We define familiarity bias as the tendency to collaborate and interact with familiar members rather than with new members. Previous research has shown that interacting with familiar members reduces individuals' uncertainty about future events and interactions }%DIFDELCMD < \cite{hinds2000choosing}%%%
\DIFdel{, and provides the foundations for trust in other membersof the team }%DIFDELCMD < \cite{mcpherson2001birds}%%%
\DIFdel{. }\DIFdelend \DIFaddbegin \subsubsection{\DIFadd{Online Group Communication}}
\DIFadd{The new settings of remote, hybrid, and IP workplaces are reshaping how individuals perceive themselves and assess their relationships with co-workers. Prior research highlights that online communication systems (e.g., text messaging, voice and video call) affect individuals' perceived differences and connections within groups }\cite{Hall2022,DESCHENES2024100351}\DIFadd{. Since individuals are not physically together, systems' design factors such as the presence of nonverbal cues, facial expressions, members' representation, and synchronous conversation will influence how aware participants are of their differences and similarities }\cite{giambatista2010diversity}\DIFadd{. Consequently, online communication systems can moderate how much rich information members can have of each other, leading users to perceive less of their demographic and physical differences }\cite{carte2004capabilities,SUH1999295}\DIFadd{. Providing more information about their characteristics and differences has the risk of avoiding teaming up with others who are different or unfamiliar }\cite{gomez2020impact}\DIFadd{. 
}\DIFaddend 

\DIFdelbegin \DIFdel{Familiarity biases vary significantly depending on the collaboration setting, such as in-person or a socio-technical system. The collaboration setting can amplify members ' perceived differences in a group }%DIFDELCMD < \cite{Shemla2016}%%%
\DIFdel{. }\DIFdelend HCI research has \DIFdelbegin \DIFdel{advanced to mitigate individuals' }\DIFdelend \DIFaddbegin \DIFadd{shown that online technologies can mitigate }\DIFaddend biases toward familiar individuals by altering the \DIFdelbegin \DIFdel{levels of users' identity }%DIFDELCMD < \cite{maloney2020anonymity,Whiting2020,Tzlil2018}%%%
\DIFdel{, in which anonymity could be a potential method to cover peoples' differences and bring them together}\DIFdelend \DIFaddbegin \DIFadd{presentation of user identities }\cite{maloney2020anonymity, Whiting2020, Tzlil2018}\DIFadd{. Several online platforms offer mechanisms like pseudonyms, avatars, and anonymity to depersonalize interactions }\cite{Shemla2016}\DIFadd{. These features reduce the emphasis on personal details, enabling more equitable participation in discussions and decision-making processes, which can lead to fairer and more balanced outcomes }\cite{christopherson2007positive, joinson2001self, walther1996computer}\DIFadd{. In particular, anonymity allows users to create online personas distinct from their offline identities, fostering self-expression and experimentation while promoting inclusivity and reducing biases in collaborative and social settings }\cite{bargh2002can, yurchisin2005exploration, kang2013people}\DIFaddend . However, \DIFdelbegin \DIFdel{anonymity can trigger undesired behaviors }\DIFdelend \DIFaddbegin \DIFadd{while anonymity can help mask individual differences, it may also lead to undesired negative behaviors, such as group polarization and harassment, }\DIFaddend due to the \DIFdelbegin \DIFdel{lack of accountability }%DIFDELCMD < \cite{Ma2016,CHRISTOPHERSON20073038}%%%
\DIFdel{.
Research has also shown the potential benefits of meeting new individuals on the Internet, which enables collaborations and relationships that are not attached to individuals' existing social networks and locations }%DIFDELCMD < \cite{baym2015personal}%%%
\DIFdel{. Online dating, meetup applications , and online communities provide a social infrastructure to enable new relationships }%DIFDELCMD < \cite{ridings2004virtual,Blackwell2015}%%%
\DIFdel{. Yet, how socio-technical systems guide individuals to meet new members or interact with already known ones has not been fully explored.  
}\DIFdelend \DIFaddbegin \DIFadd{reduced sense of accountability }\cite{weisband1993overcoming, Ma2016,CHRISTOPHERSON20073038}\DIFadd{.
}\DIFaddend 

%DIF < Collaborative technologies facilitate connections among team members, enabling them to tackle the opportunities and challenges of cross-boundary collaboration \cite{massey2008collaborative}. These technologies are divided into two primary dimensions based on interaction type. Synchronous interactions occur when team members collaborate simultaneously, utilizing tools like audio-based systems, text-based chat, messaging tools, and video conferencing \cite{robey2000information, lowenthal2023synchronous}. Conversely, asynchronous interactions involve collaboration at different times and locations, employing tools such as emails, repositories, or shared documents.
\DIFdelbegin %DIFDELCMD < 

%DIFDELCMD < %%%
\DIFdel{The choice of the communication system significantly impacts how team members perceive their differences. This is exemplified by research from Carte and Chidambaram }%DIFDELCMD < \cite{carte2004capabilities}%%%
\DIFdel{, which investigated how the type of online communication system influences members' perceived differences. In a nutshell, group meetings and online interactions will be constrained by the system's visual cues and the information provided to the users. This is explained by the amount and richness of the information transmitted through these systems }%DIFDELCMD < \cite{SUH1999295}%%%
\DIFdel{. While individuals can observe others' bodies, gestures, and emotions when meeting face to face, using socio-technical systems (e.g., videoconferencing, text messaging) will provide fewer of these cues. 
In another strand of the literature, recent studies have explored how online technologies affect workers' perceived social proximity, which is how close or far other organizational members seem to them }%DIFDELCMD < \cite{DESCHENES2024100351,van2023organizational}%%%
\DIFdel{. The new settings of remote, hybrid, and in-person workplaces will affect how workers identify themselves with the company and assess their relationships with co-workers. While some scholars argue that online technologies can enable new members to feel connected independently of their geographical distances }%DIFDELCMD < \cite{Taskin2024}%%%
\DIFdel{, others posit that new members' sense of belonging will depend on how strongly existing members employ these online technologies to welcome and interact with them }%DIFDELCMD < \cite{DESCHENES2024100351}%%%
\DIFdel{. }%DIFDELCMD < 

%DIFDELCMD < %%%
\DIFdel{In sum, how members perceive themselves---physically }\DIFdelend \DIFaddbegin \subsection{\DIFadd{Virtual Reality}}
\DIFadd{VR integrates advanced technologies that create immersive and interactive 3D environments }\cite{10.1145/3613904.3642471, 10.1145/3613905.3651085}\DIFadd{. By closely emulating the dynamics of IP conversations, VR enables synchronous and embodied interactions, allowing users to engage with the virtual space and each other through natural body movements and vocal communication }\cite{laato2024making, abbas2023virtual, mccloy2001science}\DIFadd{. Over recent years, many VR applications have increasingly supported collaborative interactions among multiple users, enabling them to be in the same virtual space regardless of their physical locations }\cite{10.1145/3411763.3441346,10.1145/3411764.3445426}\DIFadd{. Previous research has explored several features that enhance group interactions in VR, including the use of avatars, the sense of social presence, }\DIFaddend and \DIFdelbegin \DIFdel{socially--is crucial for understanding the subtle ways in which online technologies can either bridge the group or exacerbate the preference for already-known members of a group. %DIF < Morrison-Smith and Ruiz \cite{morrison2020challenges} discussed how complex managing the identities and differences in remote groups can be. Building common ground in a diverse team requires communication, shared work experiences, common vocabulary, and mental models. However, working on a medium \cite{cundill2019large, bjorn2014does}.
}\DIFdelend \DIFaddbegin \DIFadd{the spatial configuration that VR can provide }\cite{10.1145/3584931.3606992, sykownik2021most, fang2023towards}\DIFadd{. 
}\DIFaddend 

\DIFdelbegin \subsection{\DIFdel{Team Familiarity in Virtual Reality}}
%DIFAUXCMD
\addtocounter{subsection}{-1}%DIFAUXCMD
\DIFdel{Social communication between team members is significantly shaped by non-verbal cues, such as gaze, gestures, body language, and spatial distance }%DIFDELCMD < \cite{carmichael2023connecting, knapp1978nonverbal}%%%
\DIFdel{. These cues play an important role in building interpersonal relationships within teams by providing information that influences how team members perceive and interact with each other }%DIFDELCMD < \cite{knapp1978nonverbal}%%%
\DIFdel{. Prior research has shown that non-verbal communication can be facilitated in VR environments through avatar-based }\DIFdelend \DIFaddbegin \DIFadd{Avatars provide users with the flexibility to choose how they represent themselves in VR, serving as the primary identity cue that shapes perceptions and }\DIFaddend interactions \DIFaddbegin \cite{waltemate2018impact}\DIFadd{. This representation influences user behavior through stereotypes, a phenomenon known as the ``Proteus effect'' }\cite{yee2007proteus}\DIFadd{. For instance, research in VR demonstrates that embodying an elderly avatar in immersive environments can reduce stereotypical attitudes toward older individuals }\cite{yee2006walk}\DIFadd{. Similarly, Groom et al. }\cite{groom2009influence} \DIFadd{found that using avatars of a different race in VR led to measurable shifts in racial attitudes, suggesting that virtual embodiment enables users to challenge biases through flexible and diverse self-representation }\cite{banakou2016virtual}\DIFadd{. Avatar-based interactions }\DIFaddend that mimic non-verbal cues \DIFdelbegin %DIFDELCMD < \cite{freeman2016intimate, clark1991grounding}%%%
\DIFdelend \DIFaddbegin \DIFadd{foster these social interactions }\cite{freeman2016intimate, clark1991grounding, oh2016let}\DIFaddend . Maloney et al. \cite{maloney2020talking} found that non-verbal communication in social VR was perceived as positive and effective, as it offered a less intrusive method for initiating interactions with online strangers, promoting more comfortable and natural engagement. Similar research by \DIFdelbegin \DIFdel{Vosinakis and Xenakis }\DIFdelend \DIFaddbegin \DIFadd{Xenakis et. al }\DIFaddend \cite{xenakis2022nonverbal} highlighted how the reduction of body signals in VR, such as facial expressions and gestures, can decrease reliance on traditional physical cues. This shift helps minimize social biases typically present in face-to-face settings, allowing team members to focus more on communication and collaborative tasks rather than being influenced by appearance or other non-verbal indicators.

%DIF < - Can change your identity
\DIFdelbegin \DIFdel{VR systems have provided new opportunities for synchronous collaboration in immersive environments, offering a persistent 3D space where individuals can interact with others, virtual agents, and representations of both the physical and virtual worlds }%DIFDELCMD < \cite{gomez2023promise}%%%
\DIFdel{. Applications such as Meta’s Horizon Workrooms, Spatial, and VIVE Sync exemplify this by creating immersive virtual spaces where teams can collaborate in real time }%DIFDELCMD < \cite{abramczuk2023meet}%%%
\DIFdel{. A key feature of these systems is the use of avatars, which provide users the flexibility to choose how they represent themselves in virtual environments. The avatar serves as the primary identity cue, providing how individuals are perceived by others and influencing interactions within the digital environment. Prior research has explored how this flexibility in self-representation can impact others' social perceptions. For instance, embodying oneself as an elderly person in immersive environments has been shown to reduce stereotypical attitudes toward the elderly }%DIFDELCMD < \cite{yee2006walk}%%%
\DIFdel{. Similarly, Groom et al. }%DIFDELCMD < \cite{groom2009influence} %%%
\DIFdel{found that incorporating avatars of a different race in VR led to measurable shifts in racial attitudes, suggesting that virtual embodiment can alter social perceptions and help reduce bias by offering usersthe experience of flexible oneself representation }%DIFDELCMD < \cite{banakou2016virtual}%%%
\DIFdelend \DIFaddbegin \DIFadd{Prior research has also demonstrated that social presence is a critical factor for effective collaboration in VR }\cite{kimmel2023lets, sterna2021psychology, yassien2020design}\DIFadd{. VR users can experience an enhanced sense of co-presence, which creates the illusion of sharing a virtual space with their colleagues. The immersive quality of VR helps reduce the sense of detachment often associated with remote work, fostering more engaging interactions }\cite{wienrich2018social, smith2018communication} \DIFadd{and driving users' intention to collaborate }\cite{mutterlein2018specifics}\DIFaddend .

%DIF < - Remote workers are now equal (Social presence, spatial)
\DIFdelbegin \DIFdel{The }\DIFdelend \DIFaddbegin \DIFadd{Lastly, the }\DIFaddend spatial configuration of \DIFdelbegin \DIFdel{a work environment }\DIFdelend \DIFaddbegin \DIFadd{VR environments }\DIFaddend impacts how team members recognize and interact with each other. Proxemics, or spatial behavior, refers to the measurable distances between people as they communicate, influencing interpersonal dynamics \cite{hans2015kinesics}. In a virtual setting, virtual proximity replicates this effect, enabling team members to experience a shared presence in a virtual \DIFdelbegin \DIFdel{office}\DIFdelend \DIFaddbegin \DIFadd{room}\DIFaddend , even if they are physically distant \DIFaddbegin \cite{williamson2021proxemics}\DIFaddend . This shared virtual space fosters a sense of connection and engagement\DIFdelbegin \DIFdel{that reproduces in-person collaboration. According to Sousa et al. }%DIFDELCMD < \cite{sousa2016remote}%%%
\DIFdel{, the concept of remote proxemics in virtual environments allows for natural interaction patterns that mimic physical location, helping team members maintain an awareness of each other’s actions and presence. This sense of shared space helps }\DIFdelend \DIFaddbegin \DIFadd{, helping users }\DIFaddend recreate the interpersonal dynamics, such as proximity and orientation, that are important for effective collaboration \DIFdelbegin \DIFdel{. Furthermore, Slater and Sanchez-Vives }%DIFDELCMD < \cite{sanchez2005presence} %%%
\DIFdel{demonstrated that VR enhances social presence, making users feel like they are sharing the same room with their colleagues, thus reducing the detachment often associated with remote work. 
}\DIFdelend \DIFaddbegin \cite{li2021social, williamson2022digital}\DIFadd{. 
}\DIFaddend 

\DIFdelbegin \subsection{\DIFdel{Hypotheses}}
%DIFAUXCMD
\addtocounter{subsection}{-1}%DIFAUXCMD
\DIFdel{In this research project, we aim to explore whether using virtual reality can mitigate the effects of familiarity bias within teams, particularly between incumbents and newcomers. The reviewed studies provide a conceptual lens to understand how members' familiarity could be mitigated through online communication technologies . We posit that VR will help both incumbents and newcomers feel more connected with their counterparts, where social and professional differences are less pronounced. Since face-to-face interactions will likely trigger participants' perceived differences, we expect that the VR's design and capabilities can shift the focus away from physical attributes that might otherwise reinforce preconceived notions or social hierarchies. As such, using VR may neutralize the social mechanisms that encourage incumbents to stay together, allowing all team members to engage on a more level playing field and reducing the emphasis on superficial differences. Furthermore, since traditional online communication tools often constrain interaction compared to in-person dynamics, we expect that VR's design features will provide more opportunities for newcomers to participate and be included in a group. The reduced social attachment often seen in virtual settings may benefit newcomers by allowing them to feel more included and less overshadowed by the established relationships among incumbents.
Therefore, our hypotheses are as follows:
}%DIFDELCMD < \begin{itemize}
\begin{itemize}%DIFAUXCMD
%DIFDELCMD <     \item %%%
\item%DIFAUXCMD
\DIFdel{H1: Using Virtual Reality will have a positive effect on incumbents' closeness to newcomers. 
    }%DIFDELCMD < \item %%%
\item%DIFAUXCMD
\DIFdel{H2: Using Virtual Reality will have a positive effect on newcomers' closeness to incumbents.
    }%DIFDELCMD < \item %%%
\item%DIFAUXCMD
\DIFdel{H3: Using Virtual Reality will have a negative effect on incumbents' familiarity bias against newcomers. 
}
\end{itemize}%DIFAUXCMD
%DIFDELCMD < \end{itemize}
%DIFDELCMD < %%%
\DIFdelend %DIF > \subsection{Hypotheses}
\DIFaddbegin \DIFadd{Building on the HCI literature reviewed in this section, this study examines how VR can foster inclusion between old and new members of a team by mitigating their differences and biases. While prior research offers valuable insights into how online technologies influence team closeness, familiarity, and their effects on team dynamics, our work aims to explore how these mechanisms operate in VR, given its immersive and spatial nature. By addressing this gap, our study aims to deepen the understanding of closeness among team members and highlight the potential and limitations of VR in enhancing collaboration and integration within teams.
}\DIFaddend 

%DIF <  The prior work demonstrates that synchronous and asynchronous tools moderate the familiarity bias in different ways \cite{}. One example of a synchronous interaction tool is the Metaverse, which offers an immersive and persistent 3D environment where individuals can interact synchronously with others, virtual agents, and both virtual and physical world representations \cite{gomez2023promise}. Meta’s Horizon Workrooms exemplify such immersive virtual spaces where teams can collaborate effectively \cite{Meta2024}.
%DIF > We hypothesize that 

%DIF <  Moore et. al \cite{moore2020familiarity}  explores the impact of attitude familiarity on team performance within virtual reality environments. The study demonstrates that when team members are more familiar with each other’s attitudes, their performance on collaborative tasks, particularly in decision-making, improves significantly. This is attributed to enhanced communication and coordination, which are facilitated by a better understanding of each teammate’s perspective and expectations.
%DIF > \begin{itemize}
%DIF >     \item H1: VR will increase incumbents' closeness to newcomers. 
%DIF >     \item H2: VR will reduce incumbents' familiarity bias against newcomers. 
%DIF >     \item H3: VR will increase newcomers' closeness to incumbents.
%DIF >     \item H4: Users' perceived similarity will mediate the effect of using VR on team members' closeness.
%DIF > \end{itemize}



%DIF < Our work 
%DIF <  While previous work has analyzed various factors influencing team dynamics such as communication \cite{} , diversity \cite{} , geographical distance\cite{}, cultural differences \cite{}. Much of this research has focused on these factor under synchronous remote collaboration tools \cite{} such as Zoom, .. But, little work has explored the new ways of collaboration of new alternatives of synchronous work such as the Metaverse. 
\DIFdelbegin %DIFDELCMD < 

%DIFDELCMD < %%%
%DIF <  To mitigate that lack of familiarity Petersen and Pedersen could be resolved by engaging in learning activities \cite{petersen2002coping}. Moreover 
%DIFDELCMD < 

%DIFDELCMD < %%%
%DIF <  Papers to check
%DIFDELCMD < 

%DIFDELCMD < %%%
%DIF <  Lykourentzou, I., Kraut, R. E., & Dow, S. P. (2017, February). Team dating leads to better online ad hoc collaborations. In Proceedings of the 2017 ACM Conference on Computer Supported Cooperative Work and Social Computing (pp. 2330-2343).
%DIF <  Shin, D., Kim, S., Shang, R., Lee, J., & Hsieh, G. (2023, April). IntroBot: Exploring the use of chatbot-assisted familiarization in online collaborative groups. In Proceedings of the 2023 CHI Conference on Human Factors in Computing Systems (pp. 1-13).
\DIFdelend \section{Methodology}
\label{methodology}
\DIFdelbegin \DIFdel{To investigate how Virtual Reality (VR) can influence familiarity bias between team members, we }\DIFdelend \DIFaddbegin \DIFadd{We }\DIFaddend conducted a between-subject experiment with 87 participants \DIFaddbegin \DIFadd{to test our hypotheses}\DIFaddend . We designed this experiment to assess \DIFdelbegin \DIFdel{and measure the perceived familiarity }\DIFdelend \DIFaddbegin \DIFadd{the perceived closeness }\DIFaddend among team members \DIFdelbegin \DIFdel{in both environments}\DIFdelend \DIFaddbegin \DIFadd{based on the experimental conditions}\DIFaddend , as well as to explore other factors related to team behavior. 

\subsection{Participants}
The study was approved and supervised by [Anonymized]’s Institutional Review Board (IRB) under protocol number [Anonymized]. Participants were recruited \DIFaddbegin \DIFadd{from a private university in the U.S. }\DIFaddend through direct outreach by the research team and via the [Anonymized] Psychology Department's SONA system, an online platform used to recruit participants for experiments. The research team also recruited participants through social media channels. 
%#22-05-7258). 

\subsection{Task Description}
All participants were randomly assigned to teams of size three to perform a collaborative task. Since we wanted to test preference toward incumbents versus newcomers, we randomly asked two participants to meet first and complete an ice-breaker exercise (Appendix \ref{appendix:ice-breaker}). These two participants spent ten minutes \DIFdelbegin \DIFdel{knowing }\DIFdelend \DIFaddbegin \DIFadd{getting to know }\DIFaddend each other. After that, the third member was introduced to the other two participants, and the research assistant (RA) explained the task to them.

\DIFdelbegin \DIFdel{Participants conducted a hidden-profile task }\DIFdelend %DIF > The task, which involves participants revealing individually-held information, raises some questions regarding its efficacy. In the in-person discussion the participants were forbidden from sharing their paper copies with others, but can each participant (in both in-person and VR settings) read it out loud to the group? How was it ensured that a proper group discussion was taking place?: on average how many turns were taken, how long did each participant speak, and what percentage of the information got conveyed from paper to group? As such information is not provided, the question remains whether the task instigated group discussions or individual monologues.
\DIFaddbegin 

\DIFadd{We asked the three participants to complete a task adapted from the ``hidden profile'' paradigm }\DIFaddend \cite{stasser1985pooling}\DIFdelbegin \DIFdel{that required them to }\DIFdelend \DIFaddbegin \DIFadd{, which requires all participants to review candidates and select the best one for a managerial position. The information about the candidates is deliberately distributed unequally among team members to create a ``hidden profile,'' where no single individual has all the information needed to make the optimal choice. This task requires all the group members to share their unique pieces of information and integrate them to identify the most suitable candidate. If only a few members shared information or dominated the group decision-making process, the team would not pick the best candidate. As such, this task can reveal biases in group decision-making, such as members dominating the conversation, low participation, and absence of a collective conversation. Hidden profile tasks have been highly employed in laboratory experiments to test information sharing between group members when making decisions }\cite{Goyal2014,mentis2009,mennecke1997using}\DIFadd{. 
}

\DIFadd{In our study, we requested participants to }\DIFaddend collectively decide on a candidate for the presidency of a university's new satellite campus. Each participant received a copy of the candidates' resumes, \DIFdelbegin \DIFdel{but }\DIFdelend \DIFaddbegin \DIFadd{and }\DIFaddend the information about their qualities and incidents in each copy varied. Since the negative and positive aspects of each candidate varied in each resume's version, the participants needed to share and discuss the differing pieces of information provided in their respective copies. One candidate was appropriate to be hired if all participants shared their versions of the candidates' resumes. \DIFdelbegin \DIFdel{Solving the problem successfully required all participants to share the information provided to them. }\DIFdelend The task for this study was carefully designed to \DIFdelbegin \DIFdel{include data and aspects that are similar to the real world. We }\DIFdelend \DIFaddbegin \DIFadd{resemble real-world decision-making scenarios, such as hiring or promotions, and to make them relatable and meaningful to the participants. We instructed participants not to share the provided copies of the resumes with their teammates. The RA also monitored participants' progress by watching them through a video camera in the IP condition and observing the VR meeting room on a computer in the VR condition. 
}

\DIFadd{We }\DIFaddend chose this task \DIFaddbegin \DIFadd{and the candidates' resume versions }\DIFaddend based on three criteria: (a) that the content of the task was currently relevant in organizational contexts, (b) that the task did not require extensive training to be completed, and (c) that the participants were likely to engage with task since they will not know their teammates beforehand. The resumes are available in the Supplementary Materials.

\subsection{Conditions}
The teams were randomly assigned to perform this task \textit{In Person \DIFaddbegin \DIFadd{(IP)}\DIFaddend } or using \textit{Virtual Reality \DIFaddbegin \DIFadd{(VR)}\DIFaddend }. Meeting \DIFdelbegin \DIFdel{in person }\DIFdelend \DIFaddbegin \DIFadd{IP }\DIFaddend is the most natural communication medium and a ``gold standard'' to compare communication technology \cite{harrison2020framing}. We \DIFdelbegin \DIFdel{located }\DIFdelend \DIFaddbegin \DIFadd{situated }\DIFaddend participants in an experimental room with a web camera to record their conversations. Two of the three participants were together in a physical room, while the third participant remained in a separate room. After \DIFdelbegin \DIFdel{completing this }\DIFdelend \DIFaddbegin \DIFadd{the two participants completed the ice-breaker }\DIFaddend activity, the third participant joined them in the same room. \DIFaddbegin \DIFadd{Participants randomly sat in one of the three chairs. }\DIFaddend Each participant had a paper copy of their respective versions of the candidates' resumes, and we asked them not to \DIFdelbegin \DIFdel{share the papers when resolving the task. We recorded the session }\DIFdelend \DIFaddbegin \DIFadd{display the papers to the other participants. All participants sat together at a table, with their session recorded }\DIFaddend using a web camera \DIFdelbegin \DIFdel{installed on one side of the table, facing all the participants, }\DIFdelend and an omnidirectional microphone to \DIFdelbegin \DIFdel{record their dialogues }\DIFdelend \DIFaddbegin \DIFadd{capture dialogue }\DIFaddend (Figure \ref{figure:in-person-session}).


In the \DIFdelbegin \DIFdel{case of Virtual Reality}\DIFdelend \DIFaddbegin \DIFadd{VR condition}\DIFaddend , all team members were located in different physical rooms with a Meta \DIFdelbegin \DIFdel{Ouculus }\DIFdelend \DIFaddbegin \DIFadd{Oculus }\DIFaddend Quest 2 headset. The \DIFaddbegin \DIFadd{devices had activated hand tracking, enabling users to interact with the application directly using their hands. The }\DIFaddend RA helped participants wear the headset and learn to control the interface\DIFdelbegin \DIFdel{by using their hands}\DIFdelend . They had five minutes to create their virtual avatars, and we instructed them to create avatars that looked similar to \DIFdelbegin \DIFdel{them as much as possible}\DIFdelend \DIFaddbegin \DIFadd{themselves}\DIFaddend . After participants created their avatars, the RA started a session on \textit{Meta Horizon Workrooms} and put the first two participants in a group workspace to complete the ice-breaker exercise, while the third participant waited for them in a private virtual room. The RA added the participant to the workspace after 10 minutes. In the virtual \DIFaddbegin \DIFadd{meeting }\DIFaddend room, each participant had a virtual monitor in front of them with their respective versions of the candidates' resumes. Participants could not see others' monitors. \DIFdelbegin \DIFdel{Since it was a conversational task, participants were only required to use their hands if they wanted to communicate}\DIFdelend \DIFaddbegin \DIFadd{They sat together at a virtual table, and the application randomly assigned them to a seat. Participants could use body movements and hands while talking with others}\DIFaddend . We cast the group workspace (Figure \ref{figure:vr-session}) and recorded participants' conversations to analyze their decisions and identify their final choices. 

\DIFaddbegin \DIFadd{We selected Meta Horizon Workrooms for this experiment for three reasons. First, it provides an accessible, high-quality virtual workspace that simulates real-world collaboration scenarios with multiple concurrent users and low latency. Second, it allows connecting computer monitors for each participant using the Remote Desktop feature, which allows us to display the three different versions of the candidates' resumes. Third, this application allows users to create their own avatars quickly using Meta Avatars.   
}

 \DIFaddend \begin{figure*}[!htb]
\centering
    \begin{subfigure}[b]{.48\textwidth}
        \includegraphics[width=\textwidth]{figures/InPerson.png}
        \Description{Participants working on the task In-Person.}
        \caption{Participants working on the task In-Person}
        \label{figure:in-person-session}
    \end{subfigure}\qquad
    \begin{subfigure}[b]{.47\textwidth}
        \includegraphics[width=\textwidth]{figures/VRSession.png}
        \Description{Participants working on the task in Meta Horizon Workrooms.}
        \caption{Participants working on the task in Meta Horizon Workrooms}
        \label{figure:vr-session}
    \end{subfigure}
\caption{Experimental Conditions}
\DIFdelbeginFL %DIFDELCMD < \Description{Experimental Conditions. On the left, participants are working on the task In-Person. On the right, participants are working on the task in Meta Horizon Workrooms.}
%DIFDELCMD < %%%
\DIFdelendFL %DIF >  \Description{Experimental Conditions. On the left, participants are working on the task In-Person. On the right, participants are working on the task in Meta Horizon Workrooms. Participants were randomly assigned to a seat in both conditions.}
\DIFaddbeginFL \par\noindent\textit{\DIFaddFL{On the left, participants are working on the task In-Person. On the right, participants are working on the task in Meta Horizon Workrooms. Participants were randomly assigned to a seat in both conditions.}}

\DIFaddendFL \label{fig:experimental-conditions}
\end{figure*} 


\subsection{Procedure}
Participants attended our experimental sessions \DIFdelbegin \DIFdel{in person }\DIFdelend \DIFaddbegin \DIFadd{IP }\DIFaddend at our research laboratory. An RA was responsible for conducting the experiment and guiding the participants. At the beginning of the session, the RA \DIFdelbegin \DIFdel{briefly }\DIFdelend explained the purpose of the study to the participants \DIFdelbegin \DIFdel{. If participants have any questions or doubts about the study, the RA answered them}\DIFdelend \DIFaddbegin \DIFadd{and answered their questions}\DIFaddend . The RAs \DIFdelbegin \DIFdel{will emphasize }\DIFdelend \DIFaddbegin \DIFadd{emphasized }\DIFaddend that their participation was voluntary and compensated\DIFaddbegin \DIFadd{, }\DIFaddend and that their responses were confidential. This stage \DIFdelbegin \DIFdel{is expected to last }\DIFdelend \DIFaddbegin \DIFadd{lasted }\DIFaddend five minutes. The RAs provided a consent form to the participants and gave them time to read and ask questions. All participants had to consent to release their collected data to participate in this study; otherwise, the session would not have been conducted. We compensated the participants recruited from the experimental pool with extra credit\DIFdelbegin \DIFdel{. Whereas }\DIFdelend \DIFaddbegin \DIFadd{, whereas }\DIFaddend participants recruited through social media were compensated with \$20 electronic gift cards. Each session lasted 30 minutes on average.

The RA guided the participants to check-in stations and assigned participants nicknames so they could be recognized later for the surveys. The incumbents were nicknamed ``Cat'' and ``Dog,'' while the newcomer's nickname was \DIFdelbegin \DIFdel{''}\DIFdelend \DIFaddbegin \DIFadd{``}\DIFaddend Mouse.'' Each station had a label with the nickname. Participants completed a pre-treatment survey on Qualtrics, which included questions about their \DIFdelbegin \DIFdel{nickname, }\DIFdelend demographics (i.e., gender, ethnicity, race, and age), activity information (i.e., highest education level achieved, and current employment status), social self-efficacy (i.e., expectations to collaborate and form relationships while working together on a task), computational proficiency (i.e., how comfortable participants using computers, phones, and \DIFdelbegin \DIFdel{virtual reality}\DIFdelend \DIFaddbegin \DIFadd{VR}\DIFaddend ). We also checked that participants did not know each other before running the experiment. If that was the case, the session was not conducted\DIFdelbegin \DIFdel{and the participants were compensated}\DIFdelend \DIFaddbegin \DIFadd{, and we fully compensated the participants. Participants sat at different desks and were separated by portable divider curtains so they could not see each other's responses. We also checked that participants did not know each other before running the experiment. If that was the case, the session was not conducted, and we fully compensated the participants}\DIFaddend . Participants sat at different desks and were separated by portable divider curtains so they could not see each other's responses.  
%DIF >  Since the study was conducted in the U.S., we asked whether participants identified themselves as Hispanic/Latino.

After completing the initial survey, the RA led participants to the rooms based on the experimental condition and performed the task. We asked participants to introduce themselves again before they started the task. \DIFdelbegin \DIFdel{The sessions were recorded. }\DIFdelend We gave participants ten minutes to complete the hidden-profile task. 

The RA returned to the participants' rooms after the task was completed and relocated them to the check-in stations. Each participant completed a post-treatment survey on Qualtrics to assess their experiences. Participants had to evaluate their experience with the team and teammates. To ensure that participants recognized the correct team member in each question, we framed the questions regarding the incumbent as \textit{``For the partner you met first...''} and the ones regarding the newcomer as \textit{``For the partner you met later...''}\DIFaddbegin \DIFadd{. }\DIFaddend The third member had a different version of the survey and had to answer the questions based on the incumbents' nicknames. \DIFdelbegin %DIFDELCMD < 

%DIFDELCMD < %%%
\DIFdel{Once all participants }\DIFdelend \DIFaddbegin \DIFadd{After participants had }\DIFaddend completed the final survey, \DIFdelbegin \DIFdel{the RAs confirmed that everyone had completed the experiment, provided their compensations, and gave a short debrief }\DIFdelend \DIFaddbegin \DIFadd{RAs confirmed their completion, provided compensation, and conducted a brief debriefing }\DIFaddend of the experiment.


\subsection{Measurements}
\DIFaddbegin \DIFadd{We validated participants' responses to ensure they were reliable by calculating Cronbach’s ($\alpha$) alpha score. This test provided values above 0.75 for all scales, confirming their internal validity and allowing us to move on to the next steps of our data analysis. We provide the citations, reliability scores, and example items in Table \ref{tab:final_survey}.
}

\DIFaddend \subsubsection{Dependent Variables}
\paragraph{Closeness}
We asked participants how closely they perceived each team member \DIFdelbegin \DIFdel{such as interpersonal closeness, perceived similarity, and belonging}\DIFdelend \DIFaddbegin \DIFadd{in their respective environment}\DIFaddend . Participants evaluated their closeness with each team member using a 7-point Likert scale (See Table \DIFdelbegin \DIFdel{\ref{appendix:tab:final_survey}}\DIFdelend \DIFaddbegin \DIFadd{\ref{tab:final_survey}}\DIFaddend ). We \DIFdelbegin \DIFdel{adapted these items from the }\textit{\DIFdel{Social Connectedness Scale}} %DIFAUXCMD
%DIFDELCMD < \cite{lee1995measuring}%%%
\DIFdel{, which measures participants' feelings of connection to others in their social environment.}\DIFdelend \DIFaddbegin \DIFadd{used the items tested by }\cite{Gachter2015}\DIFadd{, which included questions from the `Inclusion of the Other in the Self' (IOS) Scale, the `Social Connectedness Scale', and the `We Scale.' }\DIFaddend After checking that the items showed \DIFdelbegin \DIFdel{high reliability }\DIFdelend \DIFaddbegin \DIFadd{high-reliability }\DIFaddend levels (Cronbach $\alpha=.90$), we averaged these items into a single score per participant. We defined this average as the closeness score ($C_{i \rightarrow j}$) to assess participant $i$'s closeness feelings to participant $j$. This score offered a comprehensive measure of how \DIFdelbegin \DIFdel{familiar }\DIFdelend \DIFaddbegin \DIFadd{close }\DIFaddend the participants felt \DIFdelbegin \DIFdel{with }\DIFdelend \DIFaddbegin \DIFadd{to }\DIFaddend one another after completing the task.

\paragraph{\DIFaddbegin \DIFadd{Incumbents' }\DIFaddend Familiarity Bias}
Using participants' assessments of their relationships, we \DIFdelbegin \DIFdel{focused on the relationship between the incumbents ($i_1$ and $i_2$) and the relationship between the incumbents with the newcomer ($n$). The goal was to compare how the incumbents feel connected with the other incumbents compared to the newcomers}\DIFdelend \DIFaddbegin \DIFadd{measured how different each incumbent felt connected with their incumbent compared to their newcomer}\DIFaddend . We operationalized the familiarity bias of each incumbent $i_1$ as the difference between \DIFdelbegin \DIFdel{her/his }\DIFdelend \DIFaddbegin \DIFadd{their }\DIFaddend perceived closeness to the other incumbent $i_2$ (i.e., $C_{i_1 \rightarrow i_2}$) and \DIFdelbegin \DIFdel{her/his }\DIFdelend \DIFaddbegin \DIFadd{their }\DIFaddend perceived closeness to the newcomer $n$ (i.e., $C_{i_1 \rightarrow n}$). This score ranged from -1 (i.e., feeling connected with the newcomer only) to 1 (i.e., feeling connected with the incumbent only). To normalize these closeness scores between participants and account for individual differences in rating scales, we calculated the relative difference from the absolute change between the closeness to $i_2$ and the closeness to $n$, and dividing by the closeness to $i_2$ (Equation \ref{eq:familiarity-bias}). We calculated this familiarity score for each incumbent, giving us two scores per team (i.e., $i_1 \rightarrow i_2$ and $i_2 \rightarrow i_1$). 

%($FB_{t}$)
\begin{equation}
FB_{i_1} = \frac{\left(C_{i_{1} \rightarrow i_{2}} - C_{i_{1} \rightarrow n}\right)}{C_{i_{1} \rightarrow i_{2}}}
\label{eq:familiarity-bias}
%\caption{Familiarity bias score for the incumbent $i_1$.}
\end{equation}

\subsubsection{Independent Variables}
\DIFdelbegin \DIFdel{Our independent variables aimed to measure the effect of the experimental conditions, participants' perceptions of each other, their communication, and support provided by the environment on the dependent variables. We validated participants' responses to ensure they were not answered randomly by calculating Cronbach’s ($\alpha$) reliability score. This test provided values above 0.75 for all scales, confirming their internal validity and allowing us to move on to the next steps of our data analysis. We provide the citations, reliability scores, and example items in Table \ref{appendix:tab:final_survey}.
}\DIFdelend %DIF > Our independent variables aimed to measure the effect of the experimental conditions, participants' perceptions of each other, their communication, and support provided by the environment on the dependent variables. 

\DIFdelbegin \paragraph{\DIFdel{Experimental Conditions.}} %DIFAUXCMD
\addtocounter{paragraph}{-1}%DIFAUXCMD
\DIFdel{We first assessed whether the experimental condition (}\DIFdelend \DIFaddbegin \paragraph{\DIFadd{Experimental conditions}} \DIFadd{We used a dummy variable to represent the experimental condition, where zero represented the }\DIFaddend In-Person \DIFdelbegin \DIFdel{vs. VR ) had a direct effect on the participants' closeness and familiarity bias scores. 
We used a dummy variable, one was VR and zero was In-Person. 
}\DIFdelend \DIFaddbegin \DIFadd{sessions, and 1 represented the VR sessions. 
}\DIFaddend 

\paragraph{Perceived \DIFdelbegin \DIFdel{Similarity}\DIFdelend \DIFaddbegin \DIFadd{similarity}\DIFaddend } We asked them how similar they perceived themselves to each participant. We used the items from \cite{brucks2022virtual}, including questions about their similarities with each member. 
\DIFdelbegin \DIFdel{We also }\DIFdelend \DIFaddbegin 

\paragraph{\DIFadd{Gender homophily}} \DIFadd{We }\DIFaddend included a variable `same-gender' to verify that participants felt closer to each other because of gender homophily. 

\paragraph{\DIFdelbegin \DIFdel{Communication Metrics.}\DIFdelend \DIFaddbegin \DIFadd{Support provided by the environment}\DIFaddend } \DIFaddbegin \DIFadd{To assess to what extent the environment supported participants to work on the task, we adapted questions from the }\textit{\DIFadd{Work Environment Satisfaction}} \DIFadd{scale }\cite{tenorio2020syncmeet}\DIFadd{. This scale includes questions measuring how participants feel about working in each environment, including ``I could utilize all my skills and abilities to solve this task in this environment.''
}

\paragraph{\DIFadd{Communication metrics}} \DIFaddend Participants assessed how effective the communication with their partner was using the \textit{Communication Effectiveness Index} (CETI) \cite{lomas1989communicative}. This scale evaluates participants' ability to pay attention, communicate, respond, and interact. We also measured participants' communication by using the \textit{Information Exchange} scale \cite{subramaniam2005influence}. This 7-point Likert scale presents three statements asking whether the partner and the participant shared information, if they learned from each other, and if they exchanged ideas. 

\DIFdelbegin \paragraph{\DIFdel{Environment Metrics.}} %DIFAUXCMD
\addtocounter{paragraph}{-1}%DIFAUXCMD
\DIFdel{To assess to what extent the environment supported participants to work on the task, we adapted questions from the }\textit{\DIFdel{Work Environment Satisfaction}} %DIFAUXCMD
\DIFdel{scale }%DIFDELCMD < \cite{tenorio2020syncmeet}%%%
\DIFdel{. This scale includes questions measuring how participants feel about working in each environment, including ``I could utilize all my skills and abilities to solve this task in this environment.''
}%DIFDELCMD < 

%DIFDELCMD < %%%
\DIFdelend %\subsubsection{Control Variables}
%We controlled for other team dynamic factors we expected would influence participants' closeness feelings: trust with the partner, team viability, psychological safety, and social cohesion. We provide the citations, reliability scores, and example items in Table XX.

\DIFdelbegin \subsection{\DIFdel{Usability Questions}}
%DIFAUXCMD
\addtocounter{subsection}{-1}%DIFAUXCMD
\DIFdelend \DIFaddbegin \paragraph{\DIFadd{Usability}}
\DIFaddend For the VR participants, we verified that using the headset and the workplace application was not an impediment to working on the collaborative task. Participants evaluated the system's usability using the SUS scale \cite{brooke1996sus}, \DIFdelbegin \DIFdel{which included questions such as ``I think that I would like to use this system frequently.''}\DIFdelend \DIFaddbegin \DIFadd{consisting of 10 Likert-scale items with responses ranging from 1 (``Strongly disagree'') to 5 (``Strongly Agree''). SUS scores are calculated on a scale from 0 to 100, with higher scores indicating better usability.
}\DIFaddend 

%DIF >  Please add the following required packages to your document preamble:
%DIF >  \usepackage{graphicx}
\DIFaddbegin \begin{table}[!htb]
\centering
\small
\renewcommand{\arraystretch}{1.2}
\resizebox{\columnwidth}{!}{%
\begin{tabular}{p{0.2\textwidth}p{0.2\textwidth}p{0.45\textwidth}p{0.05\textwidth}p{0.1\textwidth}}
\toprule
\textbf{Dimension} &
  \textbf{Scale} &
  \textbf{Item/Question Example} &
  \textbf{Citation} &
  \textbf{$\alpha$} \\ \midrule
Closeness &
  7-Point Likert Scale, 4 items &
  \textit{``Please select the appropriate number below to indicate to what extent you would use the term `WE' to characterize you and this partner''} &
  \cite{Gachter2015} &
  0.90 \\
Communication Effectiveness &
  5-Likert Scale, 10 items &
  \textit{``Please rate your partner's ability at... Indicating that he/she/they understands what is being said to him/her/them''} &
  \cite{lomas1989communicative} &
  0.89 \\
Information Exchange &
  7-Likert Scale, 3 items &
  \textit{``My partner and I learned from one another''} &
  \cite{subramaniam2005influence} &
  0.93 \\
Perceived Similarity &
  5-Likert Scale, 2 items &
  \textit{``How similar are you to your partner?''} &
  \cite{brucks2022virtual} &
  0.76 \\
Work Support by the Environment &
  5-Likert Scale, 2 items &
  \textit{``How efficiently did you communicate with your partners in this environment?''} &
  \cite{brucks2022virtual} &
  0.88 \\
Usability (VR participants) &
  5-Likert Scale, 10 items &
  \textit{``I thought the system was easy to use''} &
  \cite{brooke1996sus} &
  0.88 \\
  \\ \bottomrule
\end{tabular}%
}
\caption{\DIFaddFL{Final survey items}}
\Description{Final survey items}
\label{tab:final_survey}
\end{table}

\DIFaddend \subsubsection{Open-ended Questions}
Lastly, we included questions in the final survey to know more about the participants' impression of their teammates and working with them. We asked them `Do you feel closer to one partner than another?' and `What were the most significant obstacles to relating with your partners?'
\DIFdelbegin \DIFdel{The authors revised }\DIFdelend \DIFaddbegin 

\subsection{\DIFadd{Quantitative Analysis}}
\DIFadd{Using the }\DIFaddend participants' responses\DIFdelbegin \DIFdel{to contextualize the quantitative findings better. }%DIFDELCMD < 

%DIFDELCMD < %%%
\subsection{\DIFdel{Analysis}}
%DIFAUXCMD
\addtocounter{subsection}{-1}%DIFAUXCMD
\DIFdel{For the Closeness score, we conducted }\DIFdelend \DIFaddbegin \DIFadd{, we conducted the following statistical analyses using R 4.4.0 }\cite{R2024}\DIFadd{. To test whether the experimental conditions made an effect on participants' closeness, we first Welch conducted $t$-tests to check significant differences between participants' closeness scores based on the experimental conditions and relationships. We used this test since the samples in the experimental conditions had different sizes and variances. We also conducted a }\DIFaddend two-way ANOVA to test whether the experimental condition and the \DIFdelbegin \DIFdel{kind }\DIFdelend \DIFaddbegin \DIFadd{type }\DIFaddend of relationship (i.e., \DIFdelbegin \DIFdel{'}\DIFdelend \DIFaddbegin \DIFadd{``}\DIFaddend incumbent $\rightarrow$ incumbent,\DIFdelbegin \DIFdel{' '}\DIFdelend \DIFaddbegin \DIFadd{'' ``}\DIFaddend incumbent $\rightarrow$ newcomer,\DIFdelbegin \DIFdel{' '}\DIFdelend \DIFaddbegin \DIFadd{'' ``}\DIFaddend newcomer $\rightarrow$ incumbent\DIFdelbegin \DIFdel{'}\DIFdelend \DIFaddbegin \DIFadd{''}\DIFaddend ) had an effect on their perceived closeness. \DIFdelbegin \DIFdel{We then conducted a mixed-effect }\DIFdelend \DIFaddbegin \DIFadd{Lastly, we conducted a mixed-effects }\DIFaddend linear regression analysis \DIFdelbegin \DIFdel{(OLS) }\DIFdelend to estimate the factors that most explained their closeness. Each observation was the evaluation of a participant on another. We model the sessions as random effects since each session had three participants. We used their reported closeness scores ($C_{i}$) as the variable to estimate and added the independent variables to our model. We included an interaction term to verify whether the \DIFdelbegin \DIFdel{experimental condition }\DIFdelend \DIFaddbegin \DIFadd{type of relationship }\DIFaddend moderated the effect of the \DIFdelbegin \DIFdel{kind of relationship on their }\DIFdelend \DIFaddbegin \DIFadd{experimental condition on }\DIFaddend perceived closeness. \DIFaddbegin \DIFadd{We used the package }\texttt{\DIFadd{lme4}} \cite{Bates2015} \DIFadd{to create these mixed-effects models.
}\DIFaddend 

\DIFdelbegin \DIFdel{Similarly, we analyzed the Familiarity Bias score using a two-way ANOVA to test whether the experimental condition had an effect on this bias }\DIFdelend \DIFaddbegin \DIFadd{To test whether incumbents' familiarity bias differed based on the conditions, we conducted a Welch }\textit{\DIFadd{t}}\DIFadd{-test to check any statistically significant differences}\DIFaddend . We also \DIFdelbegin \DIFdel{created a mixed-effect }\DIFdelend \DIFaddbegin \DIFadd{conducted a mixed-effects }\DIFaddend linear regression model to estimate the \DIFdelbegin \DIFdel{effects of our independent variables on the incumbents' familiarity bias ($F_{i}$)}\DIFdelend \DIFaddbegin \DIFadd{factors that most explained these biases and used sessions as random effects since each session had two incumbents}\DIFaddend . Given that the familiarity bias was operationalized between incumbents and newcomers, we also computed the relative change between the communication and perceived similarities among them. 
\DIFdelbegin \DIFdel{We also modeled the }\DIFdelend \DIFaddbegin 

\DIFadd{Lastly, we conducted mixed-effects mediation models to examine whether participants' perceived similarity influenced the effect of using VR on closeness. We created different models to account for all the observations, as well as the directionality of the relationships (i.e., newcomers to incumbents, and incumbents to newcomers). We also modeled }\DIFaddend sessions as random effects since each \DIFdelbegin \DIFdel{session had two incumbents. For both }\DIFdelend \DIFaddbegin \DIFadd{participant evaluated two teammates. We created these models using the }\texttt{\DIFadd{mediation}} \DIFadd{package }\cite{Tingley2014}\DIFadd{, which allows moderation models with nested data structures.
}

\DIFadd{For all the mixed-effects regression }\DIFaddend models, we verified the normality assumptions of these models and checked that multicollinearity was not an issue by calculating the models' covariates Variance Inflation Factors (VIF).


\DIFaddbegin \subsection{\DIFadd{Qualitative Analysis}}
\DIFadd{To gain deeper insights into participants’ experiences, we conducted a qualitative analysis of open-ended survey responses, capturing their perspectives on challenges faced during the experiment and perceptions of newcomers and incumbents.
}

\DIFadd{Using an iterative inductive approach }\cite{saldana2021coding}\DIFadd{, the first author conducted initial open coding of 10 responses (five incumbents and five newcomers), identifying 40 codes. Additional coding by the second and third authors expanded the code set, which was later consolidated into a 20-code codebook through affinity mapping }\cite{nielsen2024affinity}\DIFadd{. After coding 43 more responses collaboratively, the codebook was refined to 14 codes during team discussions. The finalized codebook was then shared among four researchers, who independently coded overlapping batches of responses (24 by the first researcher and 21 by each of the others). Collaborative discussions reconciled differences and highlighted key insights, with responses addressing multiple non-mutually exclusive codes.
}

\DIFadd{The first and last authors conducted a thematic analysis }\cite{braun2006using}\DIFadd{, organizing codes into overarching themes and compiling relevant data for each. Drawing on the studies covered in Section \ref{literature_review}, they employed a deductive coding approach, iteratively refining themes over three collaborative meetings. This process clarified theme definitions and resulted in three main themes that provided deeper insights into the experiences of both newcomers and incumbents.
}






 




\DIFaddend \section{Results}
\label{results}
\DIFdelbegin \DIFdel{In total, we }\DIFdelend \DIFaddbegin \DIFadd{We }\DIFaddend recruited 87 participants, whose ages ranged from 18 to 43 years old \DIFdelbegin \DIFdel{. Regarding their gender}\DIFdelend \DIFaddbegin \DIFadd{including undergraduate students, graduate students, and staff members of our institution. Of these}\DIFaddend , 33 \DIFdelbegin \DIFdel{participants }\DIFdelend identified as male and 54 as female\DIFdelbegin \DIFdel{. Fifty-eight participants reported being White, whereas ten reported being }\DIFdelend \DIFaddbegin \DIFadd{; 58 identified as White, 10 as }\DIFaddend Asian-American\DIFdelbegin \DIFdel{and }\DIFdelend \DIFaddbegin \DIFadd{, }\DIFaddend 10 \DIFaddbegin \DIFadd{as }\DIFaddend African-American\DIFdelbegin \DIFdel{. Nine of the participants reported being of an ``Other'' race.Fifteen participants reported being of Latin or Hispanicorigin. When we asked participants how comfortable they were using technologies, most reported feeling }\DIFdelend \DIFaddbegin \DIFadd{, and 9 as “Other.” Fifteen participants identified as Latino/a or Hispanic. Most participants reported being }\DIFaddend more comfortable with Zoom (96\% feeling comfortable or very comfortable) than VR (32\% feeling comfortable or very comfortable). 

From this \DIFdelbegin \DIFdel{pool of participants, we ran }\DIFdelend \DIFaddbegin \DIFadd{participant pool, we conducted a total of 29 sessions between April and September 2024: }\DIFaddend 14 \DIFdelbegin \DIFdel{sessions in person }\DIFdelend \DIFaddbegin \DIFadd{IP sessions }\DIFaddend and 15 \DIFdelbegin \DIFdel{in VR , having 29 sessionsin total. We conducted these sessions between April and September 2024. We found that the VR participants rated positively the VR application and found it easy to use. Regarding the task}\DIFdelend \DIFaddbegin \DIFadd{VR sessions. VR participants evaluated the system's usability with a mean SUS score of 69.42 ($SD=13.81$), indicating close to average usability based on the industry standard benchmark of 68 }\cite{Lewis2018}\DIFadd{. Lastly}\DIFaddend , we did not find \DIFdelbegin \DIFdel{any significant differences between both conditions. Table }%DIFDELCMD < \cite{tab:descriptive} %%%
\DIFdel{presents the main results based on the }\DIFdelend \DIFaddbegin \DIFadd{a significant difference in team performance between the In-Person (IP) condition ($M=0.79, SD=0.43$) and the VR condition ($M=0.73, SD=0.46$). Table \ref{tab:descriptive} provides a descriptive summary of the data categorized by experimental }\DIFaddend conditions. 

\begin{table}[!ht]
\centering
\small
\begin{tabular}{@{}l|ccc|ccc|cc|ccc@{}}
\toprule
& \multicolumn{3}{c}{\textbf{In Person}} & \multicolumn{3}{c}{\textbf{VR}} & \multicolumn{2}{c}{\textbf{\textit{t}-test}} & \multicolumn{3}{c}{\textbf{Total}} \\ \midrule
\textbf{Metric} & $N$ & Mean & SD & $N$ & Mean & SD & Statistic & \textit{p}-value & $N$ & Mean & SD \\ 
  \midrule
  Proportion of female members & 42 & 0.60 & 0.50 & 45 & 0.64 & 0.48 & -0.47 & 0.64 & 87 & 0.62 & 0.49 \\ 
  Age & 42 & 20.40 & 3.81 & 45 & 21.69 & 4.69 & -1.41 & 0.16 & 87 & 21.07 & 4.31 \\ 
  Proportion of Latinos/Hispanics & 42 & 0.14 & 0.35 & 45 & 0.20 & 0.40 & -0.70 & 0.48 & 87 & 0.17 & 0.38 \\ \midrule
  Closeness & 42 & 3.11 & 1.27 & 45 & 3.50 & 1.63 & -1.76 & 0.08 & 87 & 3.31 & 1.48 \\ 
  Perceived Similarity & 42 & 3.21 & 0.84 & 45 & 3.50 & 0.89 & -2.26 & 0.02 & 87 & 3.36 & 0.87 \\ 
  Communication Effectiveness & 42 & 3.78 & 0.70 & 45 & 3.80 & 0.84 & -0.13 & 0.90 & 87 & 3.79 & 0.77 \\ 
  Information Exchange & 42 & 5.96 & 1.07 & 45 & 6.03 & 1.22 & -0.44 & 0.66 & 87 & 6.00 & 1.15 \\ 
  Support by the Environment & 42 & 5.47 & 1.00 & 45 & 5.12 & 1.21 & 2.08 & 0.04 & 87 & 5.29 & 1.12 \\ \midrule
  %Usability Score & 42 & 71.68 & 12.40 & 45 & 69.42 & 13.81 & 1.14 & 0.26 & 87 & 70.51 & 13.16 \\ \midrule
  Incumbents' Closeness to Incumbent & 28 & 4.05 & 1.18 & 30 & 3.85 & 1.55 & 0.57 & 0.57 & 58 & 3.95 & 1.37 \\
  Incumbents' Closeness to Newcomer & 28 & 2.88 & 1.22 & 30 & 3.03 & 1.59 & -0.40 & 0.69 & 58 & 2.96 & 1.41 \\
  Incumbents' Familiarity Bias & 28 & 0.29 & 0.20 & 30 & 0.21 & 0.24 & 1.29 & 0.20 & 58 & 0.25 & 0.22 \\
  Newcomers' Closeness to Incumbents & 14 & 2.39 & 0.76 & 15 & 3.62 & 1.70 & -3.57 & 0.00 & 29 & 3.03 & 1.46 \\ \midrule
  Teams with the Correct Answer (prop.) & 14 & 0.79 & 0.43 & 15 & 0.73 & 0.46 & 0.32 & 0.75 & 29 & 0.76 & 0.44 \\ 
\bottomrule
\end{tabular}
\caption{Descriptive Results of the Study. We present the metrics in four categories across in-person and VR conditions: participants' demographics, participants' final survey metrics, incumbent and newcomer metrics, and the proportion of teams with the correct answer. The number of participants/sessions ($N$), means, and standard deviations (SD) are reported for both conditions, along with the results of Welch's \textit{t}-tests comparing the two conditions. We observed significant differences for Perceived Similarity ($p < 0.05$), Support by the Environment ($p < 0.05$), and Newcomers' Closeness to Incumbents ($p < 0.001$).}
\Description{Descriptive Results of the Study. We present the metrics in four categories across in-person and VR conditions: participants' demographics, participants' final survey metrics, incumbent and newcomer metrics, and the proportion of teams with the correct answer. The number of participants/sessions ($N$), means, and standard deviations (SD) are reported for both conditions, along with the results of Welch's \textit{t}-tests comparing the two conditions. We observed significant differences for Perceived Similarity ($p < 0.05$), Support by the Environment ($p < 0.05$), and Newcomers' Closeness to Incumbents ($p < 0.001$).}
\label{tab:descriptive}
\end{table}

\DIFdelbegin \subsection{\DIFdel{Closeness Scores}}
%DIFAUXCMD
\addtocounter{subsection}{-1}%DIFAUXCMD
\DIFdel{We found that the VR condition' s newcomers felt closer to their incumbents than the In-Person (IP) condition' s newcomers. Figure \ref{fig:closeness-score-condition} shows the reported closeness scores between incumbents, incumbents to newcomers, and newcomers to incumbents per condition.}\DIFdelend \DIFaddbegin \subsection{\DIFadd{Quantitative Results}}
\DIFaddend As expected, incumbents\DIFdelbegin \DIFdel{reported the highest levels of closeness }\DIFdelend \DIFaddbegin \DIFadd{' closeness to the other incumbents was the highest score }\DIFaddend across all the relationships. \DIFdelbegin \DIFdel{The }\DIFdelend \DIFaddbegin \DIFadd{While the }\DIFaddend IP condition's incumbents reported \DIFdelbegin \DIFdel{high levels of }\DIFdelend \DIFaddbegin \DIFadd{a mean }\DIFaddend closeness to their first teammates \DIFdelbegin \DIFdel{($M=4.05, SD=1.18$), whereas they reported low closeness levels for the newcomers ($M=2.88, SD=1.21$). Newcomers'closeness scores also differed significantly per condition. 
In }\DIFdelend \DIFaddbegin \DIFadd{of 4.05 ($SD=1.18$), the VR condition's incumbents reported a mean closeness of 3.85 ($SD=1.55$). Using VR did not make any statistically significant difference to how incumbents perceived other incumbents (Welch's $t=0.57, p>0.10$). 
}

\DIFadd{In both conditions, incumbents reported lower closeness to their newcomers than to their incumbents. Participants in }\DIFaddend the IP condition \DIFdelbegin \DIFdel{, newcomers reported the lowest levels of closeness across all the experimental conditions ($M=2.88, SD=1.21$), whereas, }\DIFdelend \DIFaddbegin \DIFadd{reported a mean closeness of 2.88 ($SD=1.21$), while those }\DIFaddend in the VR condition \DIFdelbegin \DIFdel{, newcomers reported higher closeness scores ($M=3.61, SD=1.70$). While we did not observe a significant difference in incumbents' closeness to newcomers in both conditions}\DIFdelend \DIFaddbegin \DIFadd{reported a slightly higher mean of 3.03 ($SD=1.59$). However, this difference was not statistically significant ($t=-0.40, p>0.10$). 
}

\DIFadd{We found that incumbents' familiarity bias was smaller in the VR condition ($M=0.21, SD=0.24$) than in the IP condition ($M=0.29, SD=0.20$). However, this difference was not statistically significant ($t=1.29, p>0.10$), suggesting that the medium in which participants met did not influence incumbents' attachment to the first member.
}

\DIFadd{Unlike the incumbents' reported closeness scores}\DIFaddend , we found that \DIFdelbegin \DIFdel{newcomers in }\DIFdelend the VR condition\DIFdelbegin \DIFdel{felt more connected with their teammates than the newcomers who met in person. }\DIFdelend \DIFaddbegin \DIFadd{'s newcomers felt closer to their incumbents ($M=3.62, SD=1.70$) than the IP condition's newcomers ($M=2.39, SD=0.76$), a difference that was statistically significant ($t=-3.57, p < 0.001$). The effect size, measured using Cohen's $d$, was $d=0.91$, indicating a large effect. This result suggests that using VR significantly enhanced the newcomers' sense of closeness to incumbents compared to the IP condition.
}\DIFaddend 

\begin{figure}[!hbt]
    \centering
    \DIFdelbeginFL %DIFDELCMD < \includegraphics[width=0.6\columnwidth]{figures/closeness_differences.eps}
%DIFDELCMD <     %%%
\DIFdelendFL \DIFaddbeginFL \includegraphics[width=0.7\columnwidth]{figures/Closeness_score_differences.pdf}
    \DIFaddendFL \caption{Participants' closeness \DIFdelbeginFL \DIFdelFL{scores }\DIFdelendFL per condition \DIFaddbeginFL \DIFaddFL{and relationship. Error bars represent standard deviations. Brackets represent statistically significant differences between two conditions using Tukey HSD tests ($p_{adj} < 0.05$). Each bracket shows its respective }\textit{\DIFaddFL{p}}\DIFaddFL{-adjusted value.}\DIFaddendFL }
    \label{fig:closeness-score-condition}
    \DIFdelbeginFL %DIFDELCMD < \Description{Participants' closeness scores per condition}
%DIFDELCMD < %%%
\DIFdelendFL \DIFaddbeginFL \Description{Participants' closeness per condition and relationship. Error bars represent standard deviations. Brackets represent statistically significant differences between two conditions using Tukey HSD tests ($p_{adj} < 0.05$). Each bracket shows its respective \textit{p}-adjusted value.}
\DIFaddendFL \end{figure}

\DIFdelbegin \DIFdel{The }\DIFdelend \DIFaddbegin \DIFadd{Figure \ref{fig:closeness-score-condition} shows the reported closeness between incumbents, incumbents to newcomers, and newcomers to incumbents per condition. A }\DIFaddend two-way ANOVA test confirms the significant differences in closeness between types of relationships \DIFdelbegin \DIFdel{($F(2,168)=9.28, p < .001$}\DIFdelend \DIFaddbegin \DIFadd{and conditions ($F(2,168)=9.28, p<0.001$}\DIFaddend ). Although the experimental condition did not cause a statistically significant effect on the participants' closeness \DIFdelbegin \DIFdel{scores ($F(1,168)=3.46, p <.10$}\DIFdelend \DIFaddbegin \DIFadd{($F(1,168)=3.46, p<0.10$}\DIFaddend ), we found a significant interaction between the effects of the experimental condition and the type of relationship (\DIFdelbegin \DIFdel{$F(3,168)=4.2, p <0.05$). Posthoc }\DIFdelend \DIFaddbegin \DIFadd{$F(3,168)=4.2, p<0.05$). Post-hoc }\DIFaddend analysis confirmed that the incumbents felt strongly closer to the other incumbent rather than \DIFaddbegin \DIFadd{to }\DIFaddend the newcomer in the \DIFdelbegin \DIFdel{In-Person condition ($\Delta=1.16, p_{adj} < 0.05$). This difference was smaller in the VR conditionand marginally significant ($\Delta=1.02, p_{adj} < .10$). In other words, the closeness gap between incumbents and newcomers was high in both conditions but slightly smaller in VR. In contrast}\DIFdelend \DIFaddbegin \DIFadd{IP condition ($\Delta=1.17, p_{adj}<0.05$). Moreover, in the IP condition}\DIFaddend , \DIFaddbegin \DIFadd{the newcomers felt less close to the incumbents compared to how the incumbents felt to the other incumbents ($\Delta = 1.66, p_{adj} < 0.001$). Across conditions, }\DIFaddend we found \DIFdelbegin \DIFdel{the biggest differences from newcomers' perspectives. We found }\DIFdelend that newcomers' closeness \DIFdelbegin \DIFdel{scores to incumbents were statistically }\DIFdelend \DIFaddbegin \DIFadd{to incumbents was significantly }\DIFaddend higher in VR than IP (\DIFdelbegin \DIFdel{$\Delta=1.22, p_{adj} < .01$). While the connectedness scores of incumbents to incumbents were much higher than newcomers to incumbents in the IP condition ($\Delta=1.45, p_{adj} < .001)$), this difference was smaller in VR ($\Delta=.23, p>.10$). }\DIFdelend \DIFaddbegin \DIFadd{$\Delta=1.23, p_{adj}<0.05$). }\DIFaddend In sum, these results demonstrate the significant differences in participants' perceptions of closeness to others, showing that newcomers using VR felt closer to the rest of the group than those meeting in person. 
%DIF > Interestingly, the newcomers in the IP condition reported the lowest closeness scores across all the relationships and conditions. 

The \DIFdelbegin \DIFdel{mixed-effect }\DIFdelend \DIFaddbegin \DIFadd{mixed-effects }\DIFaddend linear regression model, \DIFdelbegin \DIFdel{which controls for similarity}\DIFdelend \DIFaddbegin \DIFadd{accounting for participants' perceived similarity, gender homophily, communication patterns}\DIFaddend , \DIFdelbegin \DIFdel{communication, }\DIFdelend and environmental factors, \DIFdelbegin \DIFdel{also confirms these effects }\DIFdelend \DIFaddbegin \DIFadd{further confirms the influence of relationships and experimental conditions on closeness }\DIFaddend (Table \ref{tab:mixed-effect-closeness}). By using the \DIFdelbegin \DIFdel{closeness scores }\DIFdelend \DIFaddbegin \DIFadd{relationship }\DIFaddend between incumbents as the baseline\DIFdelbegin \DIFdel{for relationships}\DIFdelend , we found that the closeness \DIFdelbegin \DIFdel{scores }\DIFdelend from incumbents to newcomers \DIFdelbegin \DIFdel{were }\DIFdelend \DIFaddbegin \DIFadd{was }\DIFaddend significantly low (\DIFdelbegin \DIFdel{$\beta=-.59,p<.05$}\DIFdelend \DIFaddbegin \DIFadd{$\beta=-0.40,p<0.05$}\DIFaddend ), as well as the closeness \DIFdelbegin \DIFdel{scores }\DIFdelend reported from newcomers to incumbents (\DIFdelbegin \DIFdel{$\beta=-.94,p<.001$}\DIFdelend \DIFaddbegin \DIFadd{$\beta=-0.63,p<0.001$}\DIFaddend ). The interaction term shows that the newcomers' closeness \DIFdelbegin \DIFdel{scores were }\DIFdelend \DIFaddbegin \DIFadd{was }\DIFaddend significantly higher in the VR condition (\DIFdelbegin \DIFdel{$\beta=.73,p<.05$}\DIFdelend \DIFaddbegin \DIFadd{$\beta=0.49,p<0.05$}\DIFaddend ), indicating that these participants were able to connect more with the team than the ones \DIFdelbegin \DIFdel{working in person}\DIFdelend \DIFaddbegin \DIFadd{in the IP condition}\DIFaddend . Regarding the other independent variables, we found that the participants' perceived similarity had a significant effect on their closeness (\DIFdelbegin \DIFdel{$\beta=.64,p<.001$). 
To confirm that the experimental condition did not make a significant difference in their similarity perceptions, we ran another two-way ANOVA with these two variables. Although the experimental condition ($F(1,169)=4.69, p<.05$) and perceived similarity ($F(1,169)=107.172, p<.001$) had a significant effect on closeness, the interaction between these two was not significant ($F(1,169)=.88, p>.10$), suggesting that participants using VR did not feel more similar or different to their teammates than those meeting in person.
}\DIFdelend \DIFaddbegin \DIFadd{$\beta=0.38,p<0.001$). 
}\DIFaddend 

\begin{table}[ht]
\centering
\small
\begin{tabular}{lccccc}
  \hline
 & \textbf{Estimate} & \textbf{Std. Error} & \textbf{d.f.} & \textbf{\textit{t} value} & \textbf{Pr($>$$|$t$|$)} \\ 
  \hline
  \textbf{Fixed Effects} & & & & & \\
  Intercept & 0.26 & 0.18 & 58.68 & 1.43 & 0.16 \\ 
  Condition (1: VR) & -0.08 & 0.25 & 52.96 & -0.31 & 0.76 \\ 
  Relationship: incumbent $\rightarrow$ newcomer & -0.40* & 0.18 & 137.33 & -2.22 & 0.03 \\ 
  Relationship: newcomer $\rightarrow$ incumbent & -0.63*** & 0.18 & 139.10 & -3.51 & 0.00 \\ 
  Gender Homophily & 0.00 & 0.12 & 158.39 & 0.00 & 1.00 \\ 
  Perceived Similarity & 0.38*** & 0.07 & 146.16 & 5.13 & 0.00 \\ 
  Support by the Environment & 0.07 & 0.07 & 161.67 & 0.94 & 0.35 \\ 
  Comm. Effectiveness & 0.07 & 0.08 & 150.50 & 0.91 & 0.36 \\ 
  Information Exchange & 0.02 & 0.07 & 152.04 & 0.22 & 0.83 \\ 
  (H1) Condition $\times$ Relationship: incumbent $\rightarrow$ newcomer & 0.19 & 0.23 & 135.08 & 0.81 & 0.42 \\ 
  (H2) Condition $\times$ Relationship: newcomer $\rightarrow$ incumbent & 0.49* & 0.24 & 137.05 & 2.04 & 0.04 \\ \midrule
  \textbf{Random Effects} & & & & & \\
  Intercept (Variance) & 0.24 & 0.48 & & & \\
  Residual (Variance) & 0.38 & 0.62 & & & \\
   \hline
\end{tabular}
\caption{Mixed-effects linear regression model estimating participants' closeness with other teammates. We used the ``incumbent $\rightarrow$ incumbent'' as the baseline for relationships. Number of Observations: 174 (i.e., 87 participants evaluating two members). Groups (i.e., Sessions): 29. We computed the Pseudo-R-squared for Generalized Mixed-Effect models: Marginal $R^2=0.35$, Conditional $R^2=0.60$. Significance codes: * $p < .05$, ** $p < .01$, *** $p < .001$.}
\Description{Mixed-effects linear regression model estimating participants' closeness with other teammates. We used the ``incumbent $\rightarrow$ incumbent'' as the baseline for relationships. Number of Observations: 174 (i.e., 87 participants evaluating two members). Groups (i.e., Sessions): 29. We computed the Pseudo-R-squared for Generalized Mixed-Effect models: Marginal $R^2=0.35$, Conditional $R^2=0.60$. Significance codes: * $p < .05$, ** $p < .01$, *** $p < .001$.}
\label{tab:mixed-effect-closeness}
\end{table}

\DIFdelbegin \subsection{\DIFdel{Familiarity Bias}}
%DIFAUXCMD
\addtocounter{subsection}{-1}%DIFAUXCMD
\DIFdel{When analyzing the differences in incumbents' closeness scores to the other incumbent and to the newcomer, Figure \ref{fig:familiarity-score-condition} shows that these differences were smaller in the VR condition ($M=.24, SD=.25$) than in the IP condition ($M=.29, SD=.20$). However, a one-way ANOVA shows that this difference is not statistically significant ($F(1,56)=1.63, p > .10)$). }\DIFdelend %DIF > Given the significant impact of the experimental condition on newcomers' reported closeness to their incumbents

\DIFdelbegin \DIFdel{The lack of significance is also confirmed by the mixed-effect linear regression model (Table \ref{tab:mixed-effect-familiarity}). When we control for the relative differences in perceived similarity and communication}\DIFdelend \DIFaddbegin \DIFadd{Lastly, we analyzed whether the effect of using VR on participants' closeness was mediated by their perceived similarity. Given the significant differences across the types of relationships, we first conducted a mixed-effects mediation analysis with the newcomers' closeness to incumbents. We found that their perceived similarities to the incumbents mediated 32\% of the total relationship between using VR and their closeness to the incumbents (Figure \ref{fig:mediation-analysis}). Consistent with the mixed-effects regression model, this mediation analysis indicates that the experimental condition did not have a significant direct effect on newcomers' closeness to incumbents ($c=0.55,p>0.05,$ Marginal Pseudo-$R^2=0.28$). However}\DIFaddend , we found that \DIFdelbegin \DIFdel{only the incumbents' perceived similarity to the other incumbent---compared to the newcomer---explains their familiarity bias ($\beta=.79,p<.01$). In other words, participants' perceived differences exacerbated their preference for the already-known member. This familiarity bias could also have been explained by gender homophily toward the incumbent ($\beta=.48,p<.10$). The experimental condition did not have }\DIFdelend \DIFaddbegin \DIFadd{meeting in VR helped newcomers perceive themselves more similar to the incumbents than meeting in person ($a=0.94,p<0.01,$ Marginal Pseudo-$R^2=0.21$), and newcomers' perceived similarity had }\DIFaddend a significant effect on \DIFdelbegin \DIFdel{familiarity bias ($\beta=-.29, p>.10$), showing that the medium in which participants met did not influence incumbents' attachment to the first member}\DIFdelend \DIFaddbegin \DIFadd{their closeness to incumbents ($b=0.29,p<0.001,$ Marginal Pseudo-$R^2=0.28$). This finding suggests that using VR enabled newcomers to feel more similar to the current team members, compared to meeting in person, which led to higher closeness feelings when working together}\DIFaddend . 

\DIFdelbegin \DIFdel{It could have been the case that VR moderated the effect of the participants' perceived similarities among themselves. We added an interaction term to this model and checked whether their perceived differences were moderated by the experimental condition. We found a slightly negative effect ($\beta=-.53, p=.055$), suggesting that VR could have helped reduce the impact of perceived similarity on closeness. A potential explanation is that VR avatars reduced the participants' perceived differences since avatars were simple and generic representations of them.
}\DIFdelend \DIFaddbegin \DIFadd{In the case of the incumbents' closeness to newcomers, we found no statistically significant relationships. VR incumbents did not feel more similar to their newcomers than IP incumbents, nor did their perceived similarities explain their closeness with newcomers. 
}\DIFaddend 

\begin{figure}[!hbt]
    \centering
    \DIFdelbeginFL %DIFDELCMD < \includegraphics[width=0.6\columnwidth]{figures/Relative_Closeness_score_differences.eps}
%DIFDELCMD <     %%%
%DIFDELCMD < \caption{%
{%DIFAUXCMD
\DIFdelFL{Incumbents' familiarity bias scores per condition}}
    %DIFAUXCMD
%DIFDELCMD < \label{fig:familiarity-score-condition}
%DIFDELCMD < %%%
\DIFdelendFL \DIFaddbeginFL \includegraphics[width=0.6\columnwidth]{figures/mediation_model.pdf}
    \caption{\DIFaddFL{Mediation analysis showing that the relationship between the medium condition and the newcomers' closeness to incumbents is partially mediated by their perceived similarity. While the direct effect is $c'$, the indirect effect is $a \times b$.}}
    \label{fig:mediation-analysis}
    \Description{Mediation analysis showing that the relationship between the medium condition and the newcomers' closeness to incumbents is partially mediated by their perceived similarity. While the direct effect is $c'$, the indirect effect is $a \times b$.}
\DIFaddendFL \end{figure}

\DIFdelbegin %DIFDELCMD < \begin{table}[ht]
%DIFDELCMD < \centering
%DIFDELCMD < \small
%DIFDELCMD < \begin{tabular}{lccccc}
%DIFDELCMD <   \hline
%DIFDELCMD <  & %%%
\DIFdelFL{Estimate }%DIFDELCMD < & %%%
\DIFdelFL{Std. Error }%DIFDELCMD < & %%%
\DIFdelFL{d.f. }%DIFDELCMD < & %%%
\textit{\DIFdelFL{t}} %DIFAUXCMD
\DIFdelFL{value }%DIFDELCMD < & %%%
\DIFdelFL{Pr($>}|$t$|$) }%DIFDELCMD < \\ 
%DIFDELCMD <   \hline
%DIFDELCMD <   %%%
\textbf{\DIFdelFL{Fixed Effects}} %DIFAUXCMD
%DIFDELCMD < & & & & & \\
%DIFDELCMD <   %%%
\DIFdelFL{Intercept }%DIFDELCMD < & %%%
\DIFdelFL{-0.11 }%DIFDELCMD < & %%%
\DIFdelFL{0.20 }%DIFDELCMD < & %%%
\DIFdelFL{24.05 }%DIFDELCMD < & %%%
\DIFdelFL{-0.54 }%DIFDELCMD < & %%%
\DIFdelFL{0.59 }%DIFDELCMD < \\ 
%DIFDELCMD <   %%%
\DIFdelFL{(H3) Condition (1: VR) }%DIFDELCMD < & %%%
\DIFdelFL{-0.29 }%DIFDELCMD < & %%%
\DIFdelFL{0.26 }%DIFDELCMD < & %%%
\DIFdelFL{28.11 }%DIFDELCMD < & %%%
\DIFdelFL{-1.12 }%DIFDELCMD < & %%%
\DIFdelFL{0.27 }%DIFDELCMD < \\ 
%DIFDELCMD <   %%%
\DIFdelFL{Gender Homophily }%DIFDELCMD < & %%%
\DIFdelFL{0.48$\dagger$ }%DIFDELCMD < & %%%
\DIFdelFL{0.27 }%DIFDELCMD < & %%%
\DIFdelFL{29.15 }%DIFDELCMD < & %%%
\DIFdelFL{1.80 }%DIFDELCMD < & %%%
\DIFdelFL{0.08 }%DIFDELCMD < \\ 
%DIFDELCMD <   %%%
\DIFdelFL{Relative Similarity Score }%DIFDELCMD < & %%%
\DIFdelFL{0.80** }%DIFDELCMD < & %%%
\DIFdelFL{0.26 }%DIFDELCMD < & %%%
\DIFdelFL{49.51 }%DIFDELCMD < & %%%
\DIFdelFL{3.03 }%DIFDELCMD < & %%%
\DIFdelFL{0.00 }%DIFDELCMD < \\ 
%DIFDELCMD <   %%%
\DIFdelFL{Support by the Environment }%DIFDELCMD < & %%%
\DIFdelFL{0.01 }%DIFDELCMD < & %%%
\DIFdelFL{0.13 }%DIFDELCMD < & %%%
\DIFdelFL{46.51 }%DIFDELCMD < & %%%
\DIFdelFL{0.11 }%DIFDELCMD < & %%%
\DIFdelFL{0.91 }%DIFDELCMD < \\ 
%DIFDELCMD <   %%%
\DIFdelFL{Comm.
Effectiveness }%DIFDELCMD < & %%%
\DIFdelFL{0.17 }%DIFDELCMD < & %%%
\DIFdelFL{0.20 }%DIFDELCMD < & %%%
\DIFdelFL{48.78 }%DIFDELCMD < & %%%
\DIFdelFL{0.85 }%DIFDELCMD < & %%%
\DIFdelFL{0.40 }%DIFDELCMD < \\ 
%DIFDELCMD <   %%%
\DIFdelFL{Information Exchange }%DIFDELCMD < & %%%
\DIFdelFL{-0.07 }%DIFDELCMD < & %%%
\DIFdelFL{0.17 }%DIFDELCMD < & %%%
\DIFdelFL{48.90 }%DIFDELCMD < & %%%
\DIFdelFL{-0.40 }%DIFDELCMD < & %%%
\DIFdelFL{0.69 }%DIFDELCMD < \\ 
%DIFDELCMD <   %%%
\DIFdelFL{Condition $\times$ Relative Similarity Score }%DIFDELCMD < & %%%
\DIFdelFL{-0.53$\dagger$ }%DIFDELCMD < & %%%
\DIFdelFL{0.27 }%DIFDELCMD < & %%%
\DIFdelFL{49.38 }%DIFDELCMD < & %%%
\DIFdelFL{-1.96 }%DIFDELCMD < & %%%
\DIFdelFL{0.06 }%DIFDELCMD < \\ 
%DIFDELCMD <   %%%
\textbf{\DIFdelFL{Random Effects}} %DIFAUXCMD
%DIFDELCMD < & & & & & \\
%DIFDELCMD <   %%%
\DIFdelFL{Intercept (Variance) }%DIFDELCMD < & %%%
\DIFdelFL{0.07 }%DIFDELCMD < & %%%
\DIFdelFL{0.26 }%DIFDELCMD < & & & \\
%DIFDELCMD <   %%%
\DIFdelFL{Residual (Variance) }%DIFDELCMD < & %%%
\DIFdelFL{0.66 }%DIFDELCMD < & %%%
\DIFdelFL{0.81 }%DIFDELCMD < & & & \\
%DIFDELCMD <    \hline
%DIFDELCMD < \end{tabular}
%DIFDELCMD < %%%
%DIFDELCMD < \caption{%
{%DIFAUXCMD
\DIFdelFL{Mixed-effects linear regression model estimating incumbents' familiarity bias. Number of Observations: 58. Groups (i.e., Sessions): 29. Significance codes: $\dagger p < .10$,  * $p < .05$, ** $p < .01$, *** $p < .001$}}
%DIFAUXCMD
%DIFDELCMD < \label{tab:mixed-effect-familiarity}
%DIFDELCMD < \end{table}
%DIFDELCMD < %%%
\DIFdelend \DIFaddbegin \subsection{\DIFadd{Qualitative Findings}}
%DIF >  
\DIFaddend 

\DIFdelbegin \subsection{\DIFdel{Open-Ended Questions}}
%DIFAUXCMD
\addtocounter{subsection}{-1}%DIFAUXCMD
\DIFdel{To gain qualitative insights into the perceptions of incumbents and newcomers, we asked participants to report the most significant challenges that they faced during their interactions. In the VR condition, participants mentioned having encountered several key obstacles when relating to their partners. 
One of the most frequently mentioned challenges was unfamiliarity with their partners. Many participants expressed difficulty in building rapport due to the fact that they ``}\textit{\DIFdel{didn't know each other}}%DIFAUXCMD
\DIFdel{'' or had `}\textit{\DIFdel{`never met before.}}%DIFAUXCMD
\DIFdel{'' This lack of prior interaction made it harder to start conversations and feel comfortable. Some noted that ``}\textit{\DIFdel{just starting the conversation was the hardest part}}%DIFAUXCMD
\DIFdel{, '' highlighting the initial awkwardness in their interactions. }\DIFdelend \DIFaddbegin \subsubsection{\DIFadd{Virtual Reality Shielding}}
\DIFadd{For newcomers, the IP setting amplified feelings of exclusion due to visible pre-existing connections among incumbents. For example, P33 mentioned how the incumbents were already focused on the task without considering her/him: }\textit{\DIFadd{``I didn’t want to interrupt. They started talking about the task, and it was a challenge to contribute when I didn’t even know their names.''}}\DIFadd{. Similarly, P24 described feelings of shame and the physical manifestation: }\textit{\DIFadd{``I felt a little bit of shame and I turned red, the situation was awkward.''}} 
\DIFaddend 

\DIFdelbegin \DIFdel{In the case of the VR condition, participants also reported several challenges in building relationships with their partners, but most of them emphasized the limitations of the VR workspace. A recurring issue was the inability to read emotional and non-verbal cues. The absence of facial expressionsand body language in VR created significant barriers }\DIFdelend \DIFaddbegin \DIFadd{However, many newcomers in the VR condition reported that the environment fostered a sense of psychological safety, providing a space where they could participate and engage without fear of judgment or social repercussions }\cite{edmondson2014psychological}\DIFadd{. This helped them feel more comfortable and closer to the incumbents. They described how the VR application provided a kind of shield that made them focus on the task activity. For example, P21 described the avatar appearance: }\textit{\DIFadd{``I didn't face any challenge, the avatar looked funny, and my team members started directly to work on the task once I entered the session.''}}\DIFadd{. Another newcomer explained the feelings of nervous about joining the session but how the customization of her avatar relief this feeling: }\textit{\DIFadd{(P6) At the beginning, I was nervous because I was going to enter the session, but I enjoyed customizing my own avatar, which made me feel more excited about the activity.}} \DIFadd{This sentiment suggests that the VR application allowed participants to engage with the activity in a neutral and focused manner.
}

\DIFadd{Moreover, the mediated nature of communication in VR, combined with the abstraction provided by avatars, allowed newcomers to participate without the social pressures often present in face-to-face interactions. For instance, another newcomer (P15) described how the environment helped them feel less self-conscious when speaking: }\textit{\DIFadd{``I felt like I could just speak up without overthinking. I didn't feel nervous at all.''}}\DIFadd{. 
}

\DIFadd{Yet, incumbents in the VR condition did not experience the same sense of ease when newcomers entered the session. Some incumbents expressed discomfort with the abrupt transition to the task after the newcomer’s arrival. For instance, P8 described feeling uncomfortable when the newcomer joined, and the group had }\DIFaddend to \DIFdelbegin \DIFdel{forming deeper connections. One participant expressedfrustration, noting that }\textit{\DIFdel{"not seeing their facial expressions was hard for me as I wasn't able to gauge what they were thinking as well as if it would've been in person}}%DIFAUXCMD
\DIFdel{." This lack of non-verbal communication made conversations feel more mechanical, as participants could not rely on body language to facilitate understanding, with another participant adding, ``}\textit{\DIFdel{I think seeing people's facial expressions and all make interactions easier}}%DIFAUXCMD
\DIFdel{. }\DIFdelend \DIFaddbegin \DIFadd{immediately shift focus: }\textit{\DIFadd{``The third member joined the session, and we had to jump straight into the activity. It felt awkward because we were talking about other things when she joined.''}} \DIFadd{Moreover, incumbents expressed that these features did not significantly impact their sense of closeness to newcomers, as their connection with other incumbents remained stronger. As P33 participant explained: }\textit{\DIFadd{``It was harder to collaborate with the }[\DIFadd{newcomer}] \DIFadd{because he was so quiet, the }[\DIFadd{incumbent}] \DIFadd{was much more active and we already talked about other things.}}\DIFaddend ''

\DIFdelbegin \DIFdel{Additionally, the nature of VR made it difficult for participants to feel as though they were truly engaging with their partners, leading to feelings of isolation and detachment. Many mentioned the ``}%DIFDELCMD < \textit{%%%
\DIFdel{awkward silence }\DIFdelend \DIFaddbegin \DIFadd{These comments illustrate how the VR application provided a layer of psychological safety, enabling newcomers to engage more freely with incumbents. This might have facilitated a more inclusive atmosphere, where the emphasis was on task completion rather than on social hierarchies or personal judgments.
}

\subsubsection{\DIFadd{Overcoming Non-Verbal Barriers Through Kinesthetic Cues}}
\DIFadd{Many responses from VR participants highlighted the challenges }\DIFaddend of \DIFdelbegin \DIFdel{being in VR}%DIFDELCMD < \MBLOCKRIGHTBRACE%%%
\DIFdel{'' and how the }\DIFdelend \DIFaddbegin \DIFadd{the }\DIFaddend absence of \DIFdelbegin \DIFdel{real-life interaction made it harder to build rapport. For some, the inability to see their partners' faces created enough distraction that they could not fully engage in conversations, with one stating, ``}\textit{\DIFdel{I couldn't see their faces so I didn't focus on the conversation as much as I could have.}}%DIFAUXCMD
\DIFdel{'' The unfamiliarity with VR further compounded these issues, as participants were not only adjusting to the platform but also struggling with basic elements of communication, oneparticipant stating they were "}\textit{\DIFdel{trying to figure out how the VR worked}}%DIFAUXCMD
\DIFdel{. "
}\DIFdelend \DIFaddbegin \DIFadd{facial expressions and other traditional non-verbal cues used to interpret emotions and intentions. For example, P19 indicated an extra effort in recognizing their teammates' facial expressions, which distracted them from focusing on the task: }\textit{\DIFadd{``I couldn't see their faces, so I didn't focus on the conversation as much as I could have. I was trying from the beginning to get some insights from the avatar faces.''}} \DIFadd{Other participants expressed frustration in not having facial expressions since they were very important to understand what others were thinking: }\textit{[\DIFadd{P10}] \DIFadd{``Not seeing their facial expressions was hard for me as I wasn’t able to gauge what they were thinking as well as if it would’ve been in person.''}}
\DIFaddend 

\DIFdelbegin \DIFdel{In summary, participants in the VR condition expressed their frustration with the VR environment's constraints and limitations. Whereas the IP participants mostly focused on describing their problems with starting working the task with a stranger and the lack of familiarity.
}\DIFdelend \DIFaddbegin \DIFadd{Despite their challenges, many VR participants overcame these challenges by employing alternative kinesthetic cues, such as gestures or avatar interactions, which played a significant role in fostering a sense of connection. These features helped them overcome initial barriers. As one newcomer highlighted: }\textit{[\DIFadd{P42}] \DIFadd{``We discovered that we could do a high five between team members, and that was so fun. I didn't feel like I was a stranger. I think I didn’t have any challenges collaborating with them.''}} \DIFadd{Similarly, other body gestures contributed to fostering comfort and connection. As one incumbent expressed: }\textit{[\DIFadd{P29}] \DIFadd{``Some of the team members gave me a thumbs up, and that was nice. From that moment, I felt comfortable with the team.''}} \DIFadd{Moreover, some participants emphasized the immersive nature of VR and how seeing their avatars replicate their own movements made them feel more comfortable and helped them quickly adapt to collaborating with the team. One participant shared: }\textit{[\DIFadd{P3}] \DIFadd{``I noticed that I was moving my hands, and my avatar was doing the same. That made me feel immersed, and it’s the reason I didn’t feel weird during the activity.''}} \DIFadd{Another participant mentioned the rotation of the avatars when someone was referring to another else was helpful to stay engaged in the conversation: }\textit{[\DIFadd{P12}] \DIFadd{``The avatars of the other team members were facing me, and I also noticed that my avatar’s hands were mirroring my movements. Because of that, I didn’t even notice the lack of facial expressions.''}}
\DIFaddend 

%DIF < On the other hand, participants in the Virtual setting reported facing obstacles related to the visual fatigue of using the headset and the lack of facial cues. 
%DIF < \begin{quote}
%DIF <     \textit{(MOUSE9) My eyes were hurting.}
%DIF <     \textit{(CAT4) The inability to recognize emotion.}
%DIF <     \textit{(DOG14) Not really getting to see their facial expressions was hard for me}
%DIF < \end{quote}
\DIFaddbegin \DIFadd{Overall, the VR application provided a combination of limitations and opportunities. Although incumbents largely relied on pre-existing familiarity to maintain connections, newcomers found alternative kinesthetic cues and the immersive features of VR effective in bridging the gap caused by the absence of traditional non-verbal cues.
}\DIFaddend 


%DIF < Most of the obstacles related to the lack of recognition of non-verbal cues were reported by incumbent team members, while newcomers primarily mentioned discomfort associated with using the headset.
%DIF >  \subsubsection{Unfamiliar Beginnings}
%DIF >  Building rapport was a challenge for participants in both VR and in-person settings, particularly when team members were unfamiliar with one another. Incumbents often found it awkward to initiate conversations, as highlighted by one participant: \textit{[P8] ``Just starting the conversation was the hardest part. I didn’t know how to approach them, and it felt awkward at first.''}. 

%DIF < Table X provides descriptive statics for the variables used in the analysis. [EXTRACT INFORMATION FROM QUALTRICS]
%DIF < The analysis of the connectedness scores revealed a clear trend in both experimental conditions: participants, on average, felt closer to the member they met first, as opposed to the member they met later. This observation aligns with our initial hypothesis, suggesting that initial interactions have a strong influence on perceived familiarity.
%DIF >  For newcomers, the IP setting amplified feelings of exclusion due to visible pre-existing connections among incumbents. For example, P33 mentioned how the incumbents were already focused on the task without considering her/him: \textit{``I didn’t want to interrupt. They started talking about the task, and it was a challenge to contribute when I didn’t even know their names.''} In contrast, participants in the VR condition reported a sense of psychological safety for newcomers, reducing social pressures through features like avatars and immersive environments. This allowed newcomers to focus on the task rather than social dynamics (\textit{[P21] ``I think I didn’t face any challenge. The avatar looked funny, and my team members started directly to work on the task.''}) Incumbents in the VR setting did not experience the same benefits as newcomers. They found the abrupt transition to task-related activities awkward and struggled to connect with newcomers (\textit{[P8] ``The third member joined the session, and we had to jump straight into the activity. It felt awkward because we were talking about other things when she joined.''})

%DIF < However, a significant difference emerged when comparing the VR and in-person conditions (Figure). In the VR setting, the gap between the connectedness scores for the first and later-met members was notably smaller. This indicates that VR has a mitigating effect on the formation of familiarity bias. Specifically, our computations showed that VR reduced familiarity bias by 31.69\% compared to the in-person condition. 
%DIF >  While the VR condition newcomers experienced a safer and more inclusive environment than the in-person condition, VR could not fully address incumbents’ reliance on pre-existing connections, suggesting the need for intentional interventions like guided onboarding activities to improve team integration.

%DIF < This reduction suggests that while VR environments may lack some of the immediacy and richness of face-to-face interactions \cite{}, they offer unique features that can help level mitigate familiarity bias, promoting a more balanced perception of team members regardless of the sequence in which they were met.
\DIFaddbegin \subsubsection{\DIFadd{Blurring Realities: The Impact of Physical and Virtual Cues on Presence}}
\DIFadd{Lastly, participants experienced a sense of presence as they navigated the coexistence of physical and virtual environments in VR. While physically situated in their own rooms, participants were tele-transported to a shared virtual workspace, creating an immersive and engaging experience. Many participants found the virtual office highly immersive, describing it as feeling }\textit{\DIFadd{``almost real.''}} \DIFadd{Structured elements, such as avatar placements and the design of the virtual environment, contributed to their sense of presence: }\textit{[\DIFadd{P22}] \DIFadd{``I felt completely immersed because I was in an office, and all of us were placed in seats.''}}
\DIFaddend 

%DIF < From the final survey, we assessed users' perceptions of the factors that complicated their performance in the collaborative task. Through open-ended questions, we gathered qualitative data, which we then categorized into the subgroups outlined in the methodology. The most frequently cited challenges were related to Communication Effectiveness (n=X). A particularly notable issue was the difficulty participants faced in interpreting social cues from team members in the VR environment.
\DIFaddbegin \DIFadd{Some participants also appreciated the blend of physical and virtual elements, noting how real-world objects like desks and computers complemented the virtual environment. This interplay reinforced the sense of being in a shared workspace: }\textit{[\DIFadd{P16}] \DIFadd{``I was in the virtual office, but I felt comfortable because it felt like I was really there with the team.''}}
\DIFaddend 

%DIF < \begin{quote}
%DIF <     \textit{Not really getting to see their facial expressions was hard for me as I wasn't able to gauge what they were thinking as well as if it would've been in person.}
%DIF <     \textit{I think seeing people's facial expressions and all make interactions easier.}
%DIF < \end{quote}
\DIFdelbegin \section{\DIFdel{Discussion}}
%DIFAUXCMD
\addtocounter{section}{-1}%DIFAUXCMD
%DIFDELCMD < \label{discussion}
%DIFDELCMD < %%%
\DIFdelend \DIFaddbegin \DIFadd{However, some participants described the odd sensation of being physically alone yet visually and interactively present with others in the virtual space. The presence of real-world objects, such as their computer or desk, served as grounding elements that reinforced their physical reality but also intensified the artificial nature of the virtual one. As one participant shared: }\textit{[\DIFadd{P52}] \DIFadd{``I was in the virtual office, but I felt weird because, at the same time, I was alone with my computer in an isolated office. That made me feel kind of strange.''}} \DIFadd{Another incumbent elaborated on this feeling of detachment caused by the overlap of real and virtual elements: }\textit{[\DIFadd{P26}] \DIFadd{``It was a challenge because I was able to see the text of the activity on my real computer while also being in a virtual office. That made me feel strange, like a sense of non-presence but presence at the same time.''}}
\DIFaddend 

%DIF <  H1: NOT SUPPORTED (THERE IS NO SIGNIFICANT EVIDENCE)
%DIF <  H2: SUPPORTED 
%DIF <  H3: NOT SUPPORTED (THERE IS NO SIGNIFICANT EVIDENCE) 
\DIFaddbegin \DIFadd{This theme examines the duality experienced by participants as they navigated the coexistence of physical and virtual environments in VR. While participants were physically located in their individual offices, the immersive nature of VR transported them to a shared virtual workspace. This juxtaposition of the tangible and the simulated created a blend of presence that was simultaneously engaging and disorienting.
}\DIFaddend 


%DIF <  \item H1: Using Virtual Reality will have a positive effect on incumbents' closeness to the newcomers.  NOT SUPPORTED (THERE IS NO SIGNIFICANT EVIDENCE)
%DIF <  \item H2: Using Virtual Reality will have a positive effect on newcomers' closeness to the incumbents. SUPPORTED
%DIF <  \item H3: Using Virtual Reality will have a negative effect on incumbents' familiarity bias against newcomers.  NOT SUPPORTED (THERE IS NO SIGNIFICANT EVIDENCE) 
\DIFdelbegin %DIFDELCMD < 

%DIFDELCMD < %%%
\DIFdelend \DIFaddbegin \section{\DIFadd{Discussion}}
\label{discussion}
\DIFaddend In this study, we examined the \DIFdelbegin \DIFdel{perceived closeness between }\DIFdelend \DIFaddbegin \DIFadd{effect of using VR on team formation. We studied how }\DIFaddend incumbents and newcomers \DIFdelbegin \DIFdel{, as well as how familiarity biasis influenced by }\DIFdelend \DIFaddbegin \DIFadd{perceived each other by measuring their perceived closeness and the resultant familiarity bias. We conducted a between-subject experiment with teams meeting in one of }\DIFaddend two different environments: Virtual Reality (VR) and In-Person \DIFdelbegin \DIFdel{. Our }\DIFdelend \DIFaddbegin \DIFadd{(IP). We found a positive and statistically significant effect on newcomers’ closeness to incumbents. The participants' open-ended responses also highlight how VR provided a shield to participants, fostering their safety and interactions with their incumbents. However, our }\DIFaddend findings show no significant evidence supporting that the VR setting positively \DIFdelbegin \DIFdel{affects incumbents’ }\DIFdelend \DIFaddbegin \DIFadd{affected incumbents' }\DIFaddend closeness to newcomers\DIFdelbegin \DIFdel{(H1). However, there was a positive effect on newcomers}\DIFdelend \DIFaddbegin \DIFadd{, and there was no significant difference in mitigating their familiarity bias. This section discusses how our findings advance understanding of VR in collaborative settings.
}

\subsection{\DIFadd{Asymmetric Effects of Virtual Reality}}
\DIFadd{Our RQ1 asked whether the use of VR influences the perceived closeness of incumbents to newcomers, and RQ2 focused on newcomers' closeness to incumbents. We found that newcomers in the VR condition perceived more similarities with their incumbents than the newcomers working IP, which mediated the positive effect of the VR condition on newcomers' closeness. While our findings reveal that newcomers' perceptions of closeness to incumbents were influenced by the VR application, incumbents}\DIFaddend ’ closeness to \DIFdelbegin \DIFdel{incumbents (H2). Finally, familiarity bias was notably }\DIFdelend \DIFaddbegin \DIFadd{newcomers remained unchanged across both experimental conditions. Regardless of the setting, incumbents reported feeling neither similar nor particularly close to newcomers. This result extends prior research in HCI, showing that online communication systems can affect team members' perceptions in different ways }\cite{yee2007proteus}\DIFadd{. The study identifies the asymmetric effects of employing VR on participants' perceived closeness. While VR has been shown to facilitate co-presence and improve the overall interaction experience in collaborative settings, our findings indicate that these benefits do not necessarily extend to all users equally.
}

\DIFadd{Moreover, we asked whether using VR could mitigate incumbents' familiarity bias against newcomers (RQ3). Although it was }\DIFaddend reduced in the VR \DIFdelbegin \DIFdel{environment compared to the In-Person setting, but }\DIFdelend \DIFaddbegin \DIFadd{application with respect to the IP setting, }\DIFaddend this difference was not \DIFdelbegin \DIFdel{fully supported (H3). }\DIFdelend \DIFaddbegin \DIFadd{statistically significant. This result could be attributed to incumbents' pre-established team dynamics. When the session began, incumbents had already had time to engage with their teammates and orient themselves to the task. By the time the newcomer arrived, incumbents were already in a task-focused mindset, leaving little room for social interactions that would foster a connection with the newcomer. This aligns with prior findings that highlight the importance of early social interactions in building team rapport }\cite{10.1145/2998181.2998300, sawyer2010social}\DIFadd{. Without deliberate interventions, VR alone does not seem sufficient to disrupt incumbents’ reliance on pre-existing relationships or to encourage active engagement with newcomers.
}\DIFaddend 

\DIFdelbegin \DIFdel{These findings not only highlight the detrimental effects of familiarity bias on team members' closeness but also clarify the factors contributing to its development and how technology, particularly VR, can help mitigate these effects. One main finding was that the reduction of familiarity bias in VR environments challenges the traditional notions that physical proximity and face-to-face interactions are essential for strong team cohesion and performance. This suggests that virtual environments can, under certain conditions, replicate or even enhance aspects of team familiarity, previously thought to be reliant on physical presence.
As such, employing VR can help new teams bridge the gap between incumbent team members and newcomers, especially when the team is forming. %DIF < improve collaboration and reduce potential biases. Our findings highlight the importance of considering the roles of both incumbents and newcomers in fostering more inclusive and effective team integration. 
}%DIFDELCMD < 

%DIFDELCMD < %%%
%DIF <  First, this paper provides empirical evidence demonstrating how the setting of the work environment can either exacerbate or reduce the perception of familiarity among team members. Our results show that biases surrounding the formation of team members are reduced in the VR environment. Furthermore, our findings indicate that participants in the VR environment were more inclined to connect with their colleagues outside of the experimental setting compared to those in the in-person groups. This suggests that VR not only affects in-experiment interactions but also extends its impact to building connections beyond the formal work context. 
%DIFDELCMD < 

%DIFDELCMD < %%%
%DIF < \subsection{Theoretical Implications}
%DIF < Our findings offer significant contributions to the existing body of knowledge on team dynamics, particularly in the context of virtual environments. 
%DIFDELCMD < 

%DIFDELCMD < %%%
\DIFdel{Moreover, this study extends our current knowledge of }\DIFdelend \DIFaddbegin \subsection{\DIFadd{Virtual Reality: A Safe Space for Newcomers}}
\DIFadd{The findings highlight how VR be employed to promote a more inclusive platform for newcomers. This study extends existing research on }\DIFaddend team familiarity \cite{10.1145/2998181.2998300, muskat2022team}\DIFdelbegin \DIFdel{by demonstrating that VR can mitigate the effects of initial biases that typically form when team members integrate newcomers on the team. This has implications for the broader understanding of how technology influences interpersonal dynamics and suggests that VR might be a }\DIFdelend \DIFaddbegin \DIFadd{, which emphasizes the challenges newcomers face in navigating pre-existing relationships and biases within teams. Prior research has shown that social pressure and anxiety often accompany the process of joining an established team }\cite{kraut2010dealing, choi2004minority}\DIFadd{. Our study suggests that VR can alleviate some of these challenges by making team members look more similar to each other. Features such as avatars and immersive spaces can help reduce the intensity of social cues that amplify social anxieties. This ``virtual shielding'' effect enabled newcomers to focus on collaboration and task completion rather than navigating through the interpersonal complexities of team integration. The VR application offered a comfortable platform for newcomers to engage with established team members, making it a }\DIFaddend valuable tool for \DIFdelbegin \DIFdel{creating more equitable and optimal team interactions, especially in diverse or geographically dispersed teams. VR environments can provide newcomers with a more comfortable platform to engage with established team members, potentially by reducing the social pressure or intimidation that often accompanies in-person interactions. }\DIFdelend \DIFaddbegin \DIFadd{fostering equitable and effective team interactions.
}\DIFaddend 

\DIFdelbegin \DIFdel{Still}\DIFdelend \DIFaddbegin \DIFadd{Extensive research highlights the importance of technology's impact in making individuals' differences visible }\cite{gomez2020impact,carte2004capabilities}\DIFadd{. Our study extends these previous studies by showing how VR can reduce the visibility of demographic or cultural differences through customizable, similar avatars and neutral settings. VR applications in team settings should consider how the avatar design can promote team members' inclusivity, supporting diverse team integration and shifting the focus toward shared goals and contributions. Yet}\DIFaddend , the lack of evidence supporting the idea that VR \DIFdelbegin \DIFdel{has a positive effect on incumbents’ closeness to newcomers highlights }\DIFdelend \DIFaddbegin \DIFadd{positively impacts existing team members underscores }\DIFaddend the need for \DIFdelbegin \DIFdel{further exploration of }\DIFdelend alternative strategies to enhance this integration process. Approaches such as redesigned onboarding \DIFdelbegin \DIFdel{processes }\DIFdelend \DIFaddbegin \DIFadd{procedures }\DIFaddend or pre-training programs may have a more \DIFdelbegin \DIFdel{substantial impact }\DIFdelend \DIFaddbegin \DIFadd{significant effect }\DIFaddend on fostering closer relationships between incumbents and newcomers. Previous research in HCI has \DIFdelbegin \DIFdel{shown }\DIFdelend \DIFaddbegin \DIFadd{demonstrated }\DIFaddend that socio-technical systems can \DIFdelbegin \DIFdel{enable short interactions between members to }\DIFdelend facilitate team formation \DIFdelbegin %DIFDELCMD < \cite{umbelino2021prototeams,lykourentzou2017team}%%%
\DIFdelend \DIFaddbegin \DIFadd{through short, structured interactions }\cite{umbelino2021prototeams, lykourentzou2017team}\DIFaddend . By orchestrating and \DIFdelbegin \DIFdel{altering }\DIFdelend \DIFaddbegin \DIFadd{modifying }\DIFaddend social introductions and interactions, VR \DIFdelbegin \DIFdel{can potentially provide new strategies that are not possible to conduct easily face-to-face. }\DIFdelend \DIFaddbegin \DIFadd{could offer new strategies to address this challenge. For example, ``speed dating''-style interactions could rotate team members through brief, focused conversations, allowing them to quickly learn about each other’s skills, interests, and backgrounds. Additionally, team-based games requiring physical collaboration could create shared experiences that help mitigate familiarity bias. }\DIFaddend Even within the VR setting, it remains \DIFdelbegin \DIFdel{crucial to continue exploring new methods that could }\DIFdelend \DIFaddbegin \DIFadd{essential to explore innovative methods to }\DIFaddend improve team integration and strengthen interpersonal connections between incumbents and newcomers.

\DIFdelbegin \DIFdel{Prior research highlights the essential role of team diversity, noting that the use of technology to present diversity information can be instrumental in forming diverse teams }%DIFDELCMD < \cite{gomez2020impact}%%%
\DIFdel{. Our study builds on this foundation by demonstrating how the VR environment offers unique opportunities for users to represent themselves in ways that may reduce the visibility of certain demographic or cultural differences. This capacity for self-representation within VR can help to reduce the impact of perceived differences among team members, fostering a more inclusive environment that supports the formation and effective functioning of diverse teams. }%DIFDELCMD < 

%DIFDELCMD < %%%
\DIFdelend \subsection{Design Implications}
\DIFdelbegin \DIFdel{The results of this paper provide insights into how HCI practitioners should design VR systems that promote the increment of team familiarity. One implication is the importance of incorporating more expressive }\DIFdelend \DIFaddbegin \DIFadd{Our study identifies multiple aspects of user experience that can guide the design of collaborative VR spaces that foster meaningful connections among team members.
}

\paragraph{\DIFadd{Altering physical and social norms in VR}}
\DIFadd{While it is valuable to bring effective aspects of real-world collaboration into VR, designers should leverage the unique capabilities of VR to overcome the limitations of physical environments and strengthen relationships among team members. For example, VR allows for customizable avatars that can minimize physical appearance-based biases by focusing on shared roles or expertise rather than individual traits. Similarly, controlled }\DIFaddend non-verbal \DIFdelbegin \DIFdel{cues, which can play a critical role in enhancing connections between team members. Prior research has demonstrated that non-verbal communication}\DIFdelend \DIFaddbegin \DIFadd{cues, such as highlighting active speakers or enabling customizable gestures, can ensure clearer communication and equitable participation, especially for individuals who might feel overshadowed in traditional settings. VR can also create blended interaction spaces, such as collaborative environments that adapt dynamically to task needs, enabling teams to move seamlessly between brainstorming, task execution, and social bonding without the constraints of a static physical location. VR is a potential tool that offers the opportunity to create such spaces, breaking down traditional barriers and facilitating the seamless integration of newcomers into teams. %DIF > Apostolopoulos et al. \cite{apostolopoulos2012road} emphasized the growing need for more effective ways to communicate with remote individuals. We contend that this challenge goes beyond communication alone it requires spaces that actively enhance teamwork dynamics and foster inclusion. 
 }

\paragraph{\DIFadd{Integrating Kinesthetic and Interactive Features}}
\DIFadd{Our findings highlight the importance of considering alternative approaches to avatar representation that go beyond improving facial expressions. While expressiveness is valuable and much research has focused on enhancing the realism of avatars to better mimic facial expressions and simulate real-world interactions }\cite{waltemate2018impact, oh2016let}\DIFadd{, other aspects of avatar and interaction design can be tailored specifically to address team formation needs}\DIFaddend , such as \DIFdelbegin \DIFdel{facial expressions and gestures, strengthens social bonds and increases familiarity between individuals }%DIFDELCMD < \cite{fang2021social}%%%
\DIFdel{. VR systems should prioritize the development of avatarscapable of delivering non-verbal signals to enhance and facilitate more effective interactions. For example, incorporating proxemic cues with features that allow users to adjust their virtual proximity to others can improve }\DIFdelend \DIFaddbegin \DIFadd{fostering inclusion and reducing familiarity bias. One design direction is enhancing the kinesthetic and interactive features of avatars. For instance, incorporating haptic feedback, such as a vibration when team members ``high five'' or shake hands, can simulate physical contact and foster a sense of camaraderie. Additionally, enabling shared object manipulation, like collaboratively moving a virtual whiteboard or assembling a puzzle, can promote teamwork and a shared sense of purpose. Body tracking for natural gestures such as nodding, waving, or leaning in and spatial audio integration, which adjusts sound based on proximity, can further enhance }\DIFaddend the sense of \DIFdelbegin \DIFdel{shared presence and strengthen team familiarity.
}%DIFDELCMD < 

%DIFDELCMD < %%%
\DIFdel{Additionally, our results suggest that beyond enhancing non-verbal communication, VR platforms should explore innovative activities and collaborative tasks specifically designed to build familiarity among team }\DIFdelend \DIFaddbegin \DIFadd{presence and connection between team }\DIFaddend members \DIFaddbegin \cite{abbas2023virtual, li2021social, williamson2022digital}\DIFaddend .
\DIFdelbegin \DIFdel{The exploration of more diverse and customizable avatar representations in VR environments could significantly enhance team recognition and interpersonal connection. By allowing users to personalize avatars with detailed characteristics }\DIFdelend \DIFaddbegin 

\paragraph{\DIFadd{Team Formation and Team-Oriented Avatar Customization}}
\DIFadd{Lastly, collaborative VR applications could serve as a valuable tool for supporting team formation and integrating new members. Masking individuals' physical appearances could enable teams to focus on building relationships and getting to know one another. Team members can later transition to meeting in person or using other modalities to continue their collaboration. Moreover, avatar customization in VR applications offers an opportunity to align user representation with the specific dynamics of team collaboration. Rather than emphasizing purely individual personalization, these features could be designed to reflect team-oriented aspects, such as shared goals, complementary skills, or collaborative roles. For instance, by enabling users to tailor their avatars with visual markers }\DIFaddend such as professional \DIFdelbegin \DIFdel{profiles, preferences, or visual cues that reflect their expertise or }\DIFdelend \DIFaddbegin \DIFadd{backgrounds, representations of expertise, or elements reflecting shared }\DIFaddend interests, teams can \DIFdelbegin \DIFdel{foster a deeper understanding of }\DIFdelend \DIFaddbegin \DIFadd{better understand and appreciate }\DIFaddend each member’s \DIFdelbegin \DIFdel{identity and role within the group. These enhanced avatars could include visual markers, such as icons or color coding, that help team members quickly recognize individual skills, preferences, or }\DIFdelend \DIFaddbegin \DIFadd{role and }\DIFaddend contributions.

\DIFdelbegin \DIFdel{The design goal should not simply be to replicate in-person social interactions within VR. Instead, VR systems should aim to address and enhance the inherent limitations found in traditional, in-person interactions. By utilizing the capabilities of VR, such as customizable avatars, controlled non-verbal cues, and blended interaction spaces, VR designs should construct a reality that not only facilitates collaboration but also minimizes the biases and barriers present in physical spaces.
}\DIFdelend %DIF >  \paragraph{Task-Oriented VR Environment}
%DIF >  Another design implication is that task-aligned virtual environments can enhance focus and collaboration. VR spaces that resemble familiar, professional settings, such as offices, can help users concentrate on the activity at hand and foster an atmosphere conducive to teamwork. The structured and goal-oriented design of such environments can potentially reduces distractions and helps participants stay focused, enabling them to collaborate more effectively with their team members and remain engaged in the task. By offering customizable, task-specific environments, VR platforms can enhance team integration and collaboration. For instance, an office-like setting can help establish a sense of shared professional context, making it easier for newcomers and incumbents to interact and build connections. Such environments act as cognitive and social anchors, encouraging team members to engage with one another in a way that promotes mutual understanding and collective progress.


\subsection{Limitations and Future Work}
While our study provides valuable insights, it is important to acknowledge its \DIFaddbegin \DIFadd{main }\DIFaddend limitations. One key limitation is the demographic homogeneity of our participants, who were primarily from a similar age group and educational background. This raises questions about the generalizability of our findings to more diverse populations. Future research should aim to include a more varied participant pool to explore how different cultural and demographic factors influence team familiarity in virtual environments. \DIFdelbegin %DIFDELCMD < 

%DIFDELCMD < %%%
\DIFdelend Second, a larger sample size could increase the statistical validity of our study. \DIFdelbegin \DIFdel{A sample with more than 500 teams (i., 1,500 participants) could detect smaller effect sizes that are in the range of the current familiarity bias results. Despite the sample of this study, we provided relevant insights into the differences between current and new members when using an online communication system. The results suggest that newcomers felt closer to their incumbents by employing an online communication system, but this intervention did not have an effect on the relationships between incumbents. }\DIFdelend Future research should replicate and increase the sample size to obtain results with higher significance levels. 

Third, several experimental design choices could have introduced bias to our results. For example, we did not provide an ice-breaker question to the third participant, which could have also explained the high attachment to the first partner. Despite this design experiment choice, the changes in closeness from incumbent to newcomer would have remained similar between the two experimental conditions. Another example is the short time that incumbents have to meet and the short task that they were required to do. \DIFaddbegin \DIFadd{While experiments conducted in laboratories can offer internal validity, these short-term teams have no past and no future. }\DIFaddend Future studies should examine real teams that are deciding to include new members. Additionally, longitudinal studies could provide deeper insights into how \DIFdelbegin \DIFdel{familiarity bias }\DIFdelend \DIFaddbegin \DIFadd{team members' closeness }\DIFaddend evolves over time in teams that regularly collaborate in VR settings.

Another limitation is the specific nature of the task used in our experiment, which involved decision-making based on shared information. Future studies could explore a wider range of tasks, such as creative tasks, to see how \DIFdelbegin \DIFdel{familiarity is being impact with }\DIFdelend \DIFaddbegin \DIFadd{team members' closeness is being impacted by }\DIFaddend a different type of task that could involve more discussion of ideas between members. \DIFaddbegin \DIFadd{Using only one VR application was also one limitation, as other applications may have offered different features and affected individuals differently. Future work should consider other VR applications as well. }\DIFaddend Finally, as VR technology continues to evolve, it will be important to re-evaluate how different the results could be when using more advanced and usable VR devices. Future research could explore how advancements in VR, such as improved sensory feedback and haptic technology, might further enhance or alter the dynamics of team interactions in virtual environments. 



%DIF < Future work -> Exploration and creation a multi-level model adding the other subgroups?
%DIF >  H1: NOT SUPPORTED (THERE IS NO SIGNIFICANT EVIDENCE)
%DIF >  H2: SUPPORTED 
%DIF >  H3: NOT SUPPORTED (THERE IS NO SIGNIFICANT EVIDENCE) 
\DIFaddbegin 

%DIF >  \item H1: Using Virtual Reality will have a positive effect on incumbents' closeness to the newcomers.  NOT SUPPORTED (THERE IS NO SIGNIFICANT EVIDENCE)
%DIF >  \item H2: Using Virtual Reality will have a positive effect on newcomers' closeness to the incumbents. SUPPORTED
%DIF >  \item H3: Using Virtual Reality will have a negative effect on incumbents' familiarity bias against newcomers.  NOT SUPPORTED (THERE IS NO SIGNIFICANT EVIDENCE) 

\DIFaddend \section{Conclusion}
\label{conclusion}
This study \DIFdelbegin \DIFdel{has explored the concept of team familiarity , a key factor known to influence team performance and enhance collaboration. To address the different behaviors of team familiarity across two different collaborative settings. }\DIFdelend \DIFaddbegin \DIFadd{explored how meeting on Virtual Reality can affect closeness and familiarity among team members. }\DIFaddend Through a controlled between-subject experiment conducted in In-Person and \DIFdelbegin \DIFdel{Virtual Reality (VR)}\DIFdelend \DIFaddbegin \DIFadd{VR}\DIFaddend , we examined empirical data about the perception of closeness between incumbents and newcomers. \DIFdelbegin %DIFDELCMD < 

%DIFDELCMD < %%%
\DIFdelend The findings of this study demonstrate how employing VR affects team members' closeness in \DIFdelbegin \DIFdel{disparate ways. Regarding the incumbents, the high levels of perceived closeness among incumbent members were consistent with the literature, which suggests that individuals familiar with each other tend to feel more connected. However, our results showed no significant difference in incumbents’ closeness to each other across different environments, indicating that online communication systems, such as VR, do not have a notable impact on existing members' relationships. In contrast, the newcomers ' perceived closeness to incumbents was significantly higher in the VR setting. This finding suggests that VR can foster a more inclusive atmosphere for newcomers , facilitating smoother integration and promoting stronger interpersonal connections. While VR did not reduce incumbents' familiarity bias toward newcomers, it clearly supports newcomers in building closer relationships with their established teammates, potentially enhancing team dynamics by reducing the social distance between new and existing }\DIFdelend \DIFaddbegin \DIFadd{asymmetric ways. While newcomers in VR felt closer to their teams than newcomers working in person, incumbents did not feel significantly different toward their newcomers. The findings suggest that VR could participants' appearances, resulting in higher closeness toward existing team }\DIFaddend members. 


%DIF > Regarding the incumbents, the high levels of perceived closeness among incumbent members were consistent with the literature, which suggests that individuals familiar with each other tend to feel more connected. However, our results showed no significant difference in incumbents’ closeness to each other across different environments, indicating that online communication systems, such as VR, do not have a notable impact on existing members' relationships. In contrast, the newcomers' perceived closeness to incumbents was significantly higher in the VR setting. This finding suggests that VR can foster a more inclusive atmosphere for newcomers, facilitating smoother integration and promoting stronger interpersonal connections. %While VR did not reduce incumbents' familiarity bias toward newcomers, it supported newcomers in building closer relationships with their established teammates.
\DIFaddbegin 

\DIFaddend As organizations increasingly adopt hybrid and remote collaboration tools, it \DIFdelbegin \DIFdel{is critical }\DIFdelend \DIFaddbegin \DIFadd{will be important }\DIFaddend to explore and leverage technologies like VR to bridge social gaps and promote equal participation. Future research and development should focus on enhancing these virtual environments to further mitigate familiarity biases and improve overall team performance. We hope this work will \DIFdelbegin \DIFdel{encourage }\DIFdelend \DIFaddbegin \DIFadd{inspire }\DIFaddend HCI practitioners and designers to continue exploring innovative ways \DIFdelbegin \DIFdel{to integrate }\DIFdelend \DIFaddbegin \DIFadd{for integrating }\DIFaddend VR into collaborative practices, ultimately creating more cohesive and \DIFdelbegin \DIFdel{high-performing }\DIFdelend \DIFaddbegin \DIFadd{effective }\DIFaddend teams in virtual settings.



\begin{acks}
% Lucy and Sloan
\end{acks}

%%
%% The next two lines define the bibliography style to be used, and
%% the bibliography file.
\bibliographystyle{ACM-Reference-Format}
\bibliography{sample-base}

%TC:ignore 
\appendix

\section{Ice-breaker questions}
\label{appendix:ice-breaker}
Please go through these questions together, you do not need to answer them all.
\begin{itemize}
    \item Share [University Name] Introductions %(Name, Dorm, Grade, Major, Hometown)
    \item What's the last TV show or movie you watched and enjoyed?
    \item Do you have any pets, and if not, what kind of pet would you like to have?
    \item If you could travel anywhere in the world, where would you go and why?
    \item What was your dream job as a kid?
    \item What's your favorite type of food, and why do you love it?
    \item What's a hobby or activity you enjoy doing in your free time?
    \item What kind of music do you like to listen to, and do you have any favorite artists?
    \item What's a skill or talent you wish you had, and why?
    \item What's the best piece of advice you've ever received, and did you follow it?
\end{itemize}

\DIFdelbegin \section{\DIFdel{Final Survey}}
%DIFAUXCMD
\addtocounter{section}{-1}%DIFAUXCMD
%DIF <  Please add the following required packages to your document preamble:
%DIF <  \usepackage{graphicx}
%DIFDELCMD < \begin{table}[!htb]
%DIFDELCMD < \centering
%DIFDELCMD < \small
%DIFDELCMD < \renewcommand{\arraystretch}{2}
%DIFDELCMD < \resizebox{\columnwidth}{!}{%
%DIFDELCMD < \begin{tabular}{p{0.2\textwidth}p{0.25\textwidth}p{0.40\textwidth}p{0.05\textwidth}p{0.1\textwidth}}
%DIFDELCMD < \toprule
%DIFDELCMD < \textbf{Dimension} &
%DIFDELCMD <   \textbf{Scale} &
%DIFDELCMD <   \textbf{Item/Question Example} &
%DIFDELCMD <   \textbf{Citation} &
%DIFDELCMD <   \textbf{$\alpha$} \\ \midrule
%DIFDELCMD < Closeness &
%DIFDELCMD <   7-Point Likert Scale, 4 items &
%DIFDELCMD <   \textit{``Please select the appropriate number below to indicate to what extent you would use the term `WE' to characterize you and this partner''} &
%DIFDELCMD <   N/A &
%DIFDELCMD <   0.90 \\
%DIFDELCMD < Communication Effectiveness &
%DIFDELCMD <   5-Likert Scale, 10 items &
%DIFDELCMD <   \textit{``Please rate your partner's ability at... Indicating that he/she/they understands what is being said to him/her/them''} &
%DIFDELCMD <   \cite{lomas1989communicative} &
%DIFDELCMD <   0.89 \\
%DIFDELCMD < Information Exchange &
%DIFDELCMD <   7-Likert Scale, 3 items &
%DIFDELCMD <   \textit{``My partner and I learned from one another''} &
%DIFDELCMD <   \cite{subramaniam2005influence} &
%DIFDELCMD <   0.93 \\
%DIFDELCMD < Subjective Similarity &
%DIFDELCMD <   5-Likert Scale, 2 items &
%DIFDELCMD <   \textit{``How similar are you to your partner?''} &
%DIFDELCMD <   \cite{brucks2022virtual} &
%DIFDELCMD <   0.76 \\
%DIFDELCMD < Work Support by the Environment &
%DIFDELCMD <   5-Likert Scale, 2 items &
%DIFDELCMD <   \textit{``How efficiently did you communicate with your partners in this environment?''} &
%DIFDELCMD <   \cite{brucks2022virtual} &
%DIFDELCMD <   0.88 \\
%DIFDELCMD < Usability (VR participants) &
%DIFDELCMD <   5-Likert Scale, 10 items &
%DIFDELCMD <   \textit{``I thought the system was easy to use''} &
%DIFDELCMD <   \cite{brooke1996sus} &
%DIFDELCMD <   0.88 \\
%DIFDELCMD <   \\ \bottomrule
%DIFDELCMD < \end{tabular}%
%DIFDELCMD < }
%DIFDELCMD < %%%
%DIFDELCMD < \caption{%
{%DIFAUXCMD
\DIFdelFL{Final survey items}}
%DIFAUXCMD
%DIFDELCMD < \label{appendix:tab:final_survey}
%DIFDELCMD < \end{table}
%DIFDELCMD < %%%
\DIFdelend %DIF > \section{Final Survey}







%TC:endignore 

\end{document}
\endinput
%%
%% End of file `sample-sigconf-authordraft.tex'.
