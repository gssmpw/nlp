\section{Conclusion}
\label{conclusion}
This study explored how meeting in Virtual Reality can affect closeness and familiarity among team members. We examined empirical data about the perception of closeness between incumbents and newcomers through a controlled, between-subjects experiment conducted in both In-Person (IP) and Virtual Reality (VR). Our findings demonstrate how employing VR affects team members' closeness in asymmetric ways. While newcomers in VR felt closer to their teams than newcomers working in person, incumbents did not feel significantly different toward their newcomers across conditions. The findings suggest that VR could abstract participants' appearances, resulting in higher closeness toward existing team members. 

As teams and organizations increasingly adopt hybrid and remote collaboration tools, HCI researchers and practitioners will play an important role in bridging social gaps and promoting equal participation by leveraging online communication systems. Future research and development should focus on enhancing these virtual environments to improve teams' cohesion and performance further. We hope this work will inspire future studies and systems to continue exploring innovative ways to integrate VR into collaborative practices, ultimately creating more cohesive and effective teams in virtual settings.

