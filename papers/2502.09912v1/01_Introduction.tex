\section{Introduction}
\label{introduction}
Bringing new members to a team is one of the most effective ways to foster originality, innovation, learning, and performance \cite{zeng2021fresh,LEWIS2007159}. Newcomers can bring new perspectives and resources to a group, including knowledge, skills, and social connections \cite{yuan2020making}. Moreover, organizations and groups are not static. Employees will leave due to turnover, promotions, or transfers to another unit. Thus, teams must often recruit and include new members to continue their work and facilitate new ideas \cite{guimera2005}. 

However, incorporating new members into an existing group can be a challenging task \cite{kraut2010dealing}. Newcomers need time to understand the group's dynamics, feel included, and commit to the team. Furthermore, team familiarity (i.e., team members' prior experience working with one another \cite{pasarakonda2023team, muskat2022team}) exacerbates the attachment to previous collaborators and similar individuals in a group, making newcomers' inclusion more difficult \cite{Arcsin2021}. As a result, existing team members (i.e., incumbents) tend to rely on their established relationships, trusting and interacting primarily among themselves. 
 
Previous research in Human-Computer Interaction (HCI) has examined how online communication systems can enhance group dynamics by modifying team members' appearances and facilitating their interactions \cite{harris2019joining,10.1145/2998181.2998300,Whiting2020}. While much work has focused on features that promote inclusion among team members, little has been explored on how these technologies affect newcomers and incumbents differently. Additionally, although previous studies have investigated how online communication systems influence perceived closeness among team members, this has been mostly explored at the interpersonal level, with limited consideration of how these perceptions function within an entire team \cite{Hall2022}. This gap raises several questions about how online communication systems can differently affect the inclusion of new members. Would newcomers be perceived differently depending on the online communication system in which they meet their new team? Would incumbents be more welcoming if the differences between them and the newcomers were less perceptible? 

In this study, we examined how employing a collaborative Virtual Reality (VR) application affects team members' closeness to each other. We investigated VR because its high-level immersion and customizable avatars can potentially reduce the influence of physical appearances, cultural cues, and other factors that often lead to preconceived ideas and negative stereotypes among team members \cite{christofi2017virtual, higgins2023perspective, mal2023impact}. Potentially, collaborative VR could foster a more inclusive atmosphere and allow team members to focus more on the essence of collaboration rather than interpersonal biases \cite{latoschik2017effect}. Moreover, we ask whether using VR can mitigate incumbents' familiarity bias---their tendency to collaborate and interact more with familiar members than with new members. Given our interest in exploring how VR can change the relationships between newcomers and incumbents, our research questions are:
\begin{itemize}
    \item[\textbf{RQ1:}] How does using Virtual Reality affect incumbents' perceived closeness to newcomers?
    \item[\textbf{RQ2:}] How does using Virtual Reality affect newcomers' perceived closeness to incumbents?
    \item[\textbf{RQ3:}] Can Virtual Reality reduce familiarity bias among incumbents toward newcomers?
\end{itemize}

To answer these research questions, we conducted a controlled, between-subject experiment with 29 teams of three members each. Teams were randomly assigned to work in one of two settings: in-person or a VR multi-player application (Meta Horizon Workrooms). In each experimental session, two of the three participants were initially brought together into a room to get to know each other (i.e., the incumbents), while the third participant (i.e., the newcomer) joined later. Teams then completed a hidden-profile task, which required all team members to share information and have a collaborative discussion to succeed. Afterward, participants completed a post-treatment survey to evaluate their work experience and their relationships with their teammates. We employed mixed methods to analyze the survey data, which included behavioral scales and open-ended questions. 

Our findings reveal that using VR significantly impacted team members' closeness only for newcomers. Newcomers in VR reported feeling closer to their incumbents than the newcomers working in-person, and this effect was mediated by the high levels of perceived similarity experienced in VR. However, the incumbents in VR were not affected in the same way, and employing VR did not cause any significant differences in incumbents' familiarity bias toward newcomers. Through thematic analysis \cite{braun2006using}, we found that the VR setting provided participants with psychological safety by reducing social pressures, while the immersive virtual environment enhanced participants' sense of presence and engagement in the collaborative task. 

This paper provides the following three contributions. First, it deepens our understanding of the asymmetric effects of VR on team members' closeness and familiarity perceptions, as newcomers can feel closer to their team, while incumbents may not experience any differences. Second, it offers a controlled between-subject experiment comparing team members' closeness in VR and in-person settings, including a de-identified dataset available at OSF.io \cite{Gomez-Zara_2025} to facilitate reproducibility and further research. Third, it discusses how the quantitative and qualitative results can enhance future VR applications for teams and collaborations, elaborating on design implications that promote team members' closeness and inclusion.

