\section{Results}
\label{results}
We recruited 87 participants, whose ages ranged from 18 to 43 years old, and included undergraduate students, graduate students, and staff members of our institution. Of these, 33 identified as male and 54 as female; 58 identified as White, 10 as Asian-American, 10 as African-American, and nine as ``Other.'' Fifteen participants identified as Latino or Hispanic. Most participants reported being more comfortable with Zoom (96\% feeling comfortable or very comfortable) than VR (32\% feeling comfortable or very comfortable). 

From this participant pool, we conducted a total of 29 sessions between April and September 2024: 14 IP sessions and 15 VR sessions. VR participants evaluated the system's usability with a mean SUS score of 69.42 ($SD=13.81$), indicating close to average usability based on the industry standard benchmark of 68 \cite{Lewis2018}. Lastly, we did not find a significant difference in the proportion of teams that answered correctly between the In-Person (IP) condition ($M=0.79, SD=0.43$) and the VR condition ($M=0.73, SD=0.46$). Table \ref{tab:descriptive} provides a descriptive summary of the data categorized by experimental conditions. 

\begin{table*}[!ht]
\centering
\small
\begin{tabular}{@{}l|ccc|ccc|cc|ccc@{}}
\toprule
& \multicolumn{3}{c}{\textbf{In Person}} & \multicolumn{3}{c}{\textbf{VR}} & \multicolumn{2}{c}{\textbf{\textit{t}-test}} & \multicolumn{3}{c}{\textbf{Total}} \\ \midrule
\textbf{Metric} & $N$ & Mean & SD & $N$ & Mean & SD & Statistic & \textit{p}-value & $N$ & Mean & SD \\ 
  \midrule
  Proportion of female members & 42 & 0.60 & 0.50 & 45 & 0.64 & 0.48 & -0.47 & 0.64 & 87 & 0.62 & 0.49 \\ 
  Age & 42 & 20.40 & 3.81 & 45 & 21.69 & 4.69 & -1.41 & 0.16 & 87 & 21.07 & 4.31 \\ 
  Proportion of Latinos/Hispanics & 42 & 0.14 & 0.35 & 45 & 0.20 & 0.40 & -0.70 & 0.48 & 87 & 0.17 & 0.38 \\ \midrule
  Closeness & 42 & 3.11 & 1.27 & 45 & 3.50 & 1.63 & -1.76 & 0.08 & 87 & 3.31 & 1.48 \\ 
  Perceived Similarity & 42 & 3.21 & 0.84 & 45 & 3.50 & 0.89 & -2.26 & 0.02 & 87 & 3.36 & 0.87 \\ 
  Communication Effectiveness & 42 & 3.78 & 0.70 & 45 & 3.80 & 0.84 & -0.13 & 0.90 & 87 & 3.79 & 0.77 \\ 
  Information Exchange & 42 & 5.96 & 1.07 & 45 & 6.03 & 1.22 & -0.44 & 0.66 & 87 & 6.00 & 1.15 \\ 
  Support by the Environment & 42 & 5.47 & 1.00 & 45 & 5.12 & 1.21 & 2.08 & 0.04 & 87 & 5.29 & 1.12 \\ \midrule
  %Usability Score & 42 & 71.68 & 12.40 & 45 & 69.42 & 13.81 & 1.14 & 0.26 & 87 & 70.51 & 13.16 \\ \midrule
  Incumbents' Closeness to Incumbent & 28 & 4.05 & 1.18 & 30 & 3.85 & 1.55 & 0.57 & 0.57 & 58 & 3.95 & 1.37 \\
  Incumbents' Closeness to Newcomer & 28 & 2.88 & 1.22 & 30 & 3.03 & 1.59 & -0.40 & 0.69 & 58 & 2.96 & 1.41 \\
  Incumbents' Familiarity Bias & 28 & 0.29 & 0.20 & 30 & 0.21 & 0.24 & 1.29 & 0.20 & 58 & 0.25 & 0.22 \\
  Newcomers' Closeness to Incumbents & 14 & 2.39 & 0.76 & 15 & 3.62 & 1.70 & -3.57 & 0.00 & 29 & 3.03 & 1.46 \\ \midrule
  Teams with the Correct Answer (prop.) & 14 & 0.79 & 0.43 & 15 & 0.73 & 0.46 & 0.32 & 0.75 & 29 & 0.76 & 0.44 \\ 
\bottomrule
\end{tabular}
\caption{Descriptive Results of the Study. We present the metrics in four categories across in-person and VR conditions: participants' demographics, participants' final survey metrics, incumbent and newcomer metrics, and the proportion of teams with the correct answer. The number of participants/sessions ($N$), means, and standard deviations (SD) are reported for both conditions, along with the results of Welch's \textit{t}-tests comparing the two conditions. We observed significant differences for Perceived Similarity ($p < 0.05$), Support by the Environment ($p < 0.05$), and Newcomers' Closeness to Incumbents ($p < 0.001$).}
\Description{Descriptive Results of the Study. We present the metrics in four categories across in-person and VR conditions: participants' demographics, participants' final survey metrics, incumbent and newcomer metrics, and the proportion of teams with the correct answer. The number of participants/sessions ($N$), means, and standard deviations (SD) are reported for both conditions, along with the results of Welch's \textit{t}-tests comparing the two conditions. We observed significant differences for Perceived Similarity ($p < 0.05$), Support by the Environment ($p < 0.05$), and Newcomers' Closeness to Incumbents ($p < 0.001$).}
\label{tab:descriptive}
\end{table*}

\subsection{Quantitative Results}
As expected, incumbents' closeness to the other incumbents was the highest score across all the relationships. While the IP condition's incumbents reported a mean closeness to their first teammates of 4.05 ($SD=1.18$), the VR condition's incumbents reported a mean closeness of 3.85 ($SD=1.55$). Using VR did not make any statistically significant difference to how incumbents perceived other incumbents (Welch's $t=0.57, p>0.10$). 

In both conditions, incumbents reported lower closeness to their newcomers than to their incumbents. Participants in the IP condition reported a mean closeness of 2.88 ($SD=1.21$), while those in the VR condition reported a slightly higher mean of 3.03 ($SD=1.59$). However, this difference was not statistically significant ($t=-0.40, p>0.10$). 

Unlike the incumbents' reported closeness scores, we found that the VR condition's newcomers felt closer to their incumbents ($M=3.62, SD=1.70$) than the IP condition's newcomers ($M=2.39, SD=0.76$), a difference that was statistically significant ($t=-3.57, p < 0.001$). The effect size, measured using Cohen's $d$, was $d=0.91$, indicating a large effect. This result suggests that using VR significantly enhanced the newcomers' sense of closeness to incumbents compared to the IP condition.

\begin{figure}[!hbt]
    \centering
    \includegraphics[width=\columnwidth]{figures/Closeness_score_differences.pdf}
    \caption{Participants' closeness per condition and relationship. Error bars represent standard deviations. Brackets represent statistically significant differences between two conditions using Tukey HSD tests ($p_{adj} < 0.05$). Each bracket shows its respective \textit{p}-adjusted value.}
    \label{fig:closeness-score-condition}
    \Description{Participants' closeness per condition and relationship. Error bars represent standard deviations. Brackets represent statistically significant differences between two conditions using Tukey HSD tests ($p_{adj} < 0.05$). Each bracket shows its respective \textit{p}-adjusted value.}
\end{figure}

Figure \ref{fig:closeness-score-condition} shows the reported closeness between incumbents, incumbents to newcomers, and newcomers to incumbents per condition. A two-way ANOVA test confirms the significant differences in closeness between types of relationships and conditions ($F(2,168)=9.28, p<0.001$). Although the experimental condition did not cause a statistically significant effect on the participants' closeness ($F(1,168)=3.46, p<0.10$), we found a significant interaction between the effects of the experimental condition and the type of relationship ($F(3,168)=4.2, p<0.05$). Post-hoc analysis confirmed that the incumbents felt strongly closer to the other incumbent rather than to the newcomer in the IP condition ($\Delta=1.17, p_{adj}<0.05$). Moreover, in the IP condition, the newcomers felt less close to the incumbents compared to how the incumbents felt to the other incumbents ($\Delta = 1.66, p_{adj} < 0.001$). Across conditions, we found that newcomers' closeness to incumbents was significantly higher in VR than IP ($\Delta=1.23, p_{adj}<0.05$). In sum, these results demonstrate the significant differences in participants' perceptions of closeness to others, showing that newcomers using VR felt closer to the rest of the group than those meeting in person. 

The mixed-effects linear regression model, accounting for participants' perceived similarity, gender homophily, communication patterns, and environmental factors, further confirms the influence of relationships and experimental conditions on closeness (Table \ref{tab:mixed-effect-closeness}). By using the relationship between incumbents as the baseline, we found that the closeness from incumbents to newcomers was significantly low ($\beta=-0.40,p<0.05$), as well as the closeness reported from newcomers to incumbents ($\beta=-0.63,p<0.001$). Regarding the other independent variables, we found that the participants' perceived similarity had a significant effect on their closeness ($\beta=0.38,p<0.001$). The first interaction term shows that the incumbents' closeness was not statistically higher in the VR condition ($\beta=0.19,p>0.05$). This indicates that VR did not lead to an increased sense of closeness for incumbents toward newcomers, providing no evidence to support H1.

The second interaction term shows that the newcomers' closeness was significantly higher in the VR condition ($\beta=0.49,p<0.05$), indicating that VR facilitated stronger connections between newcomers and their incumbents compared to the IP condition. This result supports H2, demonstrating that VR enhances newcomers' sense of closeness to incumbents.

We found that incumbents' familiarity bias was smaller in the VR condition ($M=0.21, SD=0.24$) than in the IP condition ($M=0.29, SD=0.20$). However, this difference was not statistically significant ($t=1.29, p>0.10$), suggesting that the medium in which participants met did not influence incumbents' attachment to the first member. Consequently, H3 is not supported.

\begin{table*}[ht]
\centering
\small
\begin{tabular}{lccccc}
  \hline
 & \textbf{Estimate} & \textbf{Std. Error} & \textbf{d.f.} & \textbf{\textit{t} value} & \textbf{Pr($>$$|$t$|$)} \\ 
  \hline
  \textbf{Fixed Effects} & & & & & \\
  Intercept & 0.26 & 0.18 & 58.68 & 1.43 & 0.16 \\ 
  Condition (1: VR) & -0.08 & 0.25 & 52.96 & -0.31 & 0.76 \\ 
  Relationship: incumbent $\rightarrow$ newcomer & -0.40* & 0.18 & 137.33 & -2.22 & 0.03 \\ 
  Relationship: newcomer $\rightarrow$ incumbent & -0.63*** & 0.18 & 139.10 & -3.51 & 0.00 \\ 
  Gender Homophily & 0.00 & 0.12 & 158.39 & 0.00 & 1.00 \\ 
  Perceived Similarity & 0.38*** & 0.07 & 146.16 & 5.13 & 0.00 \\ 
  Support by the Environment & 0.07 & 0.07 & 161.67 & 0.94 & 0.35 \\ 
  Comm. Effectiveness & 0.07 & 0.08 & 150.50 & 0.91 & 0.36 \\ 
  Information Exchange & 0.02 & 0.07 & 152.04 & 0.22 & 0.83 \\ 
  (H1) Condition $\times$ Relationship: incumbent $\rightarrow$ newcomer & 0.19 & 0.23 & 135.08 & 0.81 & 0.42 \\ 
  (H2) Condition $\times$ Relationship: newcomer $\rightarrow$ incumbent & 0.49* & 0.24 & 137.05 & 2.04 & 0.04 \\ \midrule
  \textbf{Random Effects} & & & & & \\
  Intercept (Variance) & 0.24 & 0.48 & & & \\
  Residual (Variance) & 0.38 & 0.62 & & & \\
   \hline
\end{tabular}
\caption{Mixed-effects linear regression model estimating participants' closeness with other teammates. We used the ``incumbent $\rightarrow$ incumbent'' as the baseline for relationships. Number of Observations: 174 (i.e., 87 participants evaluating two members). Groups (i.e., Sessions): 29. We computed the Pseudo-R-squared for Generalized Mixed-Effect models: Marginal $R^2=0.35$, Conditional $R^2=0.60$. Significance codes: * $p < .05$, ** $p < .01$, *** $p < .001$.}
\Description{Mixed-effects linear regression model estimating participants' closeness with other teammates. We used the ``incumbent $\rightarrow$ incumbent'' as the baseline for relationships. Number of Observations: 174 (i.e., 87 participants evaluating two members). Groups (i.e., Sessions): 29. We computed the Pseudo-R-squared for Generalized Mixed-Effect models: Marginal $R^2=0.35$, Conditional $R^2=0.60$. Significance codes: * $p < .05$, ** $p < .01$, *** $p < .001$.}
\label{tab:mixed-effect-closeness}
\end{table*}

Lastly, we analyzed whether the effect of using VR on participants' closeness was mediated by their perceived similarity. Given the significant differences across the types of relationships, we first conducted a mixed-effects mediation analysis with the newcomers' closeness to incumbents. We found that their perceived similarities to the incumbents mediated 32\% of the total relationship between using VR and their closeness to the incumbents (Figure \ref{fig:mediation-analysis}). Consistent with the mixed-effects regression model, this mediation analysis indicates that the experimental condition did not have a significant direct effect on newcomers' closeness to incumbents ($c=0.55,p>0.05,$ Marginal Pseudo-$R^2=0.28$). However, we found that meeting in VR helped newcomers perceive themselves more similar to the incumbents than meeting in person ($a=0.94,p<0.01,$ Marginal Pseudo-$R^2=0.21$), and newcomers' perceived similarity had a significant effect on their closeness to incumbents ($b=0.29,p<0.001,$ Marginal Pseudo-$R^2=0.28$). This finding suggests that using VR enabled newcomers to feel more similar to the current team members, compared to meeting in person, which led to higher closeness feelings when working together. In the case of the incumbents' closeness to newcomers, we found no statistically significant relationships. VR incumbents did not feel more similar to their newcomers than IP incumbents, nor did their perceived similarities explain their closeness with newcomers. Therefore, H4 is partially supported. 

\begin{figure}[!hbt]
    \centering
    \includegraphics[width=\columnwidth]{figures/mediation_model.pdf}
    \caption{Mediation analysis showing that the relationship between the medium condition and the newcomers' closeness to incumbents is partially mediated by their perceived similarity. While the direct effect is $c'$, the indirect effect is $a \times b$.}
    \label{fig:mediation-analysis}
    \Description{Mediation analysis showing that the relationship between the medium condition and the newcomers' closeness to incumbents is partially mediated by their perceived similarity. While the direct effect is $c'$, the indirect effect is $a \times b$.}
\end{figure}

\subsection{Qualitative Results}

\subsubsection{Virtual Reality Shielding}
This theme addresses why newcomers in VR felt significantly closer to incumbents than newcomers in IP (supporting H2). Seventy-three percent (73\%) of the newcomers in VR reported a sense of psychological safety, enabling them to engage without fear of judgment or social repercussions \cite{edmondson2014psychological}. Meanwhile, 42\% of the participants in the IP condition reported amplified feelings of exclusion due to visible pre-existing connections among incumbents. For example, P33 mentioned how the incumbents were already focused on the task without considering them: \textit{``I didn't want to interrupt. They started talking about the task, and it was a challenge to contribute when I didn't even know their names.''}. Similarly, P24 described feelings of shame and the physical manifestation: \textit{``I felt a little bit of shame, and I turned red, the situation was awkward.''} 

By contrast, 11 of the 15 newcomers in the VR condition reported that the VR setting helped them feel more comfortable and closer to the incumbents. Participants described how the VR application acted as a `shield,' enabling them to focus on the task rather than potential social anxieties. For example, P21 described how the avatar's cartoonish appearance contributed to a more relaxed social environment: \textit{``I didn't face any challenge, the avatar looked funny, and my team members started directly to work on the task once I entered the session.''} Another newcomer (P6) explained that they had feelings of nervousness about joining the session but how the customization of the avatar relieved this feeling: \textit{``At the beginning, I was nervous because I was going to enter the session, but I enjoyed customizing my own avatar, which made me feel more excited about the activity.''} This sentiment suggests that the VR application allowed participants to engage with the activity in a neutral and focused manner.

Moreover, the mediated nature of communication in VR, combined with the abstraction provided by avatars, allowed newcomers to participate without the social pressures often present in face-to-face interactions. For instance, newcomer P15 described how the environment helped them feel less self-conscious when speaking: \textit{``I felt like I could just speak up without overthinking. I didn't feel nervous at all.''}. These comments illustrate how the VR application provided a layer of psychological safety, enabling newcomers to engage more freely with incumbents. This might have facilitated a more inclusive atmosphere, where the emphasis was on task completion rather than on social hierarchies or personal judgments.

\subsubsection{The Impact of Time Over Medium}
Nevertheless, incumbents in the VR condition did not experience the same ease in connecting with newcomers (rejecting H1) as they did with fellow incumbents (rejecting H3). Among the 30 incumbents, 30\% expressed discomfort with the abrupt transition to the task after the newcomer's arrival. For instance, P8 described feeling uncomfortable when the newcomer joined, and the group had to immediately shift focus: \textit{``The [newcomer] joined the session, and we had to jump straight into the activity. It felt awkward because we were talking about other things when she joined.''} Moreover, 56\% of incumbents expressed that features of the VR environment, such as the avatars' appearance and the application's immersiveness, did not significantly impact their sense of closeness to newcomers, as their connection with other incumbents remained stronger. As participant P33 explained: \textit{``It was harder to collaborate with the [newcomer] because he was so quiet, the [incumbent] was much more active and we already talked about other things.}''

In both experimental conditions, 13 incumbent participants emphasized that time was the primary factor in developing a sense of closeness with other incumbents. Having more time to interact before the newcomer joined allowed incumbents to establish a prior relationship, reinforcing their connection. In contrast, the limited time available to engage with the newcomer hindered the development of a similar bond, making it more challenging to establish the same level of closeness. For example, P4 mentioned: \textit{``I had more time to interact with the [incumbent] and established a connection.}'' This finding suggests that closeness among participants was primarily built through longer interactions rather than initial visual cues. Neither physical appearance in the IP condition nor avatar appearance in VR significantly influenced their sense of closeness. Instead, the duration of interaction played an important role in developing familiarity and strengthening a sense of closeness. Notably, VR did not offer a significant advantage in mitigating familiarity bias, as the medium itself did not compensate for the incumbents' pre-existing relationships formed through prior time spent together.

\subsubsection{Overcoming Non-Verbal Constraints Through Kinesthetic Cues}
The participants' responses also underscore how using VR affected their perceptions of each other, ultimately affecting how similar or different they felt with each other (partially supporting H4). Compared to participants working IP, participants in VR reported that most of their attention went to the coexistence of physical and virtual elements. While physically situated in their own rooms, participants were tele-transported to a shared virtual workspace, creating an immersive and engaging experience. Structured elements, such as avatar placements and the design of the virtual environment, played a crucial role in reinforcing this presence, as specifically noted by four participants. For instance, P22 described: \textit{``I felt completely immersed because I was in an office, and all of us were placed in seats.''} Beyond the virtual environment itself, seven participants (15\%) also highlighted the role of real-world objects in enhancing their sense of presence. They appreciated the blend of physical and virtual elements, noting how real-world objects like desks and computers complemented the virtual setting. This interplay reinforced the feeling of being in a shared workspace, as P16 explained:\textit{``I was in the virtual office, but I felt comfortable because it felt like I was really there with the team.''} 

Moreover, one-third of the participants emphasized the immersive nature of VR. Among them, six newcomers mentioned how seeing their avatars replicate their own movements made them feel more comfortable and helped them quickly adapt to collaborating with the team. One newcomer (P3) shared: \textit{``I noticed that I was moving my hands, and my avatar was doing the same. That made me feel immersed, and it's the reason I did not feel weird during the activity.''} Another participant (P12) mentioned that the rotation of the avatars when someone was referring to themselves was helpful to stay engaged in the conversation: \textit{``The avatars of the other team members were facing me, and I also noticed that my avatar's hands were mirroring my movements. Because of that, I did not even notice the lack of facial expressions.''} In contrast, participants who worked IP paid less attention to their environment and were more conscious of the incumbents who were working on the task. For instance, a newcomer working in the IP condition (P7) said \textit{``they [incumbents] were able to connect beforehand, and I was not able to talk or get to know them. They felt like they knew each other very well, which made me uncomfortable sharing my ideas to a full extent.''}

Another main explanation of this difference between VR and IP was the absence of facial expressions and other traditional non-verbal cues in VR, which was reported by 71\% of the VR participants. These features and cues are crucial for interpreting emotions, intentions, and perceived differences. This limitation likely influenced how incumbents and newcomers perceived each other's similarities and how closeness developed within the team. Two newcomer participants stated that the simplified and cartoonish avatars helped them focus on both the task and their conversations with others. For example, P18 noted that seeing another participant's avatar smile created a sense of excitement, making them feel more comfortable during the activity: \textit{``It was curious how some [incumbent] smiled, and I felt excited.''}

Despite the benefits of this design, 32 of the 45 participants in the VR condition described how the lack of facial expressions made communication more demanding and, at times, distracting. For example, P19 shared that trying to interpret their teammates' emotions through avatars required extra effort, which took their focus away from the task: \textit{``I couldn't see their faces, so I didn't focus on the conversation as much as I could have. I was trying from the beginning to get some insights from the avatar faces.''}. Other participants expressed frustration in not having facial expressions since they were very important to understanding what others were thinking: \textit{[P10] ``Not seeing their facial expressions was hard for me as I wasn't able to gauge what they were thinking as well as if it would've been in person.''}

To overcome these challenges, 19 participants in the VR condition adapted by using alternative kinesthetic cues, such as gestures or avatar interactions, which played a significant role in fostering a sense of connection. For instance, P42 described how a playful interaction helped them feel included: \textit{``We discovered that we could do a high five between team members, and that was so fun. I didn't feel like I was a stranger. I think I didn't have any challenges collaborating with them.''} Similarly, P29 noted how simple gestures, like a thumbs-up, helped build comfort and connection: \textit{``Some of the team members gave me a thumbs up, and that was nice. From that moment, I felt comfortable with the team.''}

\subsubsection{Blurring Realities: The Impact of Physical and Virtual Cues on Presence}
Our last theme provides additional context for understanding the nuanced effects of VR on team dynamics, highlighting how the environment influences behavior. More than half of the participants (53\%) found the virtual office highly immersive, describing it as feeling \textit{``almost real.''} However, thirteen participants (28\%) described the odd sensation of being physically alone yet visually and interactively present with others in the virtual space. The presence of real-world objects, such as their computer or desk, served as grounding elements that reinforced their physical reality but also intensified the artificial nature of the virtual one. As one participant (P52) shared: \textit{``I was in the virtual office, but I felt weird because, at the same time, I was alone with my computer in an isolated office. That made me feel kind of strange.''} Another incumbent (P26) elaborated on this feeling of detachment caused by the overlap of real and virtual elements: \textit{``It was a challenge because I was able to see the text of the activity on my real computer while also being in a virtual office. That made me feel strange, like a sense of non-presence but presence at the same time.''}

This theme characterizes the duality experienced by participants as they navigated the coexistence of physical and virtual environments in VR. While participants were physically located in their individual offices, the immersive nature of VR transported them to a shared virtual workspace. This juxtaposition of the tangible and the simulated created a blend of presence that was both engaging and disorienting.

