\section{Literature Review}
\label{literature_review}
\DIFaddbegin \DIFadd{We situate our work in the context of prior studies of teams, including closeness and familiarity, and research in VR.
}\DIFaddend 

%DIF < Key word: Social Cues
\DIFaddbegin \subsection{\DIFadd{Teams}}
\DIFadd{Teams consist of two or more individuals collaborating dynamically to achieve a shared objective }\cite{salas2000teamwork}\DIFadd{. As the foundation of organizational work, teams coordinate efforts to tackle complex tasks, demonstrating interdependencies in workflows, goals, and outcomes that reflect their collective responsibility }\cite{kozlowski2006enhancing, gomez2020taxonomy, baron2003group}\DIFadd{.
}\DIFaddend 

\DIFdelbegin \subsection{\DIFdel{Team Familiarity}}
%DIFAUXCMD
\addtocounter{subsection}{-1}%DIFAUXCMD
%DIF < Concept
\DIFdel{Team familiarity }\DIFdelend \DIFaddbegin \subsubsection{\DIFadd{Team Closeness}}
\DIFadd{Team closeness measures team members' subjective perception of how connected they feel to other members of the team }\cite{Gachter2015}\DIFadd{. Individuals can feel closer to certain team members based on their interactions, and not necessarily on their physical proximity }\cite{Wiese2011}\DIFadd{. While team cohesion usually refers to how united team members are as a whole }\cite{salas2015measuring}\DIFadd{, closeness refers to relationships between pairs, which has been studied at the interpersonal level }\cite{rosh2012too}\DIFadd{. 
}

\DIFadd{Researchers have identified several factors that contribute to building closeness }\cite{moreland1982exposure}\DIFadd{. Time is a key element, enabling teams to build shared knowledge and establish trust among members }\cite{harrison2003time, gillespie2012factors, cattani2013tackling}\DIFadd{. Previous relationships also play an important role, as positive prior experiences can reduce uncertainty among interactions between team members }\cite{de2017attuning, jones2019essentials, dittmer2020cut}\DIFadd{. Lastly, similarity affects closeness since team members share common attributes such as demographics, values, or experiences, they are more likely to develop stronger connections and a sense of familiarity }\cite{hinds2000choosing, ruef2003structure, winship2011homophily}\DIFadd{.
}

\subsubsection{\DIFadd{Team Familiarity}}
\DIFadd{Team familiarity }\DIFaddend refers to the degree of knowledge \DIFaddbegin \DIFadd{that }\DIFaddend team members have about each other and the amount of experience they have working together \DIFdelbegin %DIFDELCMD < \cite{muskat2022team}%%%
\DIFdelend \DIFaddbegin \cite{muskat2022team,mukherjee2019prior}\DIFaddend . Previous research has shown that team familiarity can be a predictor of improved performance, enhancing creativity, efficiency, and the overall quality of a team's output \cite{dittmer2020cut, huckman2009team, witmer2022systematic}. For instance, Sosa demonstrated through a sociometric study that \DIFdelbegin \DIFdel{strong ties between team members can lead to more creative problem-solving }%DIFDELCMD < \cite{sosa2011creative}%%%
\DIFdel{. Through semi-structured interviews and surveys conducted in a software company, his findings indicated that }\DIFdelend teams with stronger interpersonal connections and shared knowledge bases were significantly more likely to generate innovative solutions to complex problems \DIFdelbegin \DIFdel{. Moreover, these teams exhibited a higher degree of engagement and satisfaction in their collaborative efforts, underscoring the value of cultivating strong bonds within creative work environments. }\DIFdelend \DIFaddbegin \cite{sosa2011creative}\DIFadd{. }\DIFaddend Similarly, Staats \cite{staats2012unpacking} \DIFdelbegin \DIFdel{researched the geographic location and hierarchical roles within teams that affect team familiarity and, consequently, team performance. This work }\DIFdelend found that increased familiarity \DIFdelbegin \DIFdel{is }\DIFdelend \DIFaddbegin \DIFadd{was }\DIFaddend associated with an \DIFdelbegin \DIFdel{absolute }\DIFdelend improvement in team \DIFdelbegin \DIFdel{effort}\DIFdelend \DIFaddbegin \DIFadd{performance}\DIFaddend . Lastly, \DIFdelbegin \DIFdel{Huckman }\DIFdelend \DIFaddbegin \DIFadd{Salehi }\DIFaddend et al. \DIFdelbegin %DIFDELCMD < \cite{huckman2009team} %%%
\DIFdel{conducted a comprehensive analysis of team dynamics by exploring (1) the degree of familiarity among team members, and (2) the variability in roles played by individuals within teams. The study revealed that familiarity within teams correlates positively and significantly with both the quality of output and adherence to project timelines. Specifically, higher levels of team familiarity were found to be a robust predictor of enhanced team performance, particularly in terms of effort deviation. As such, familiarity helps team members understand where the skills, roles, and assignments of other members, as well as communicate with them in efficient and safe ways }%DIFDELCMD < \cite{gomez2022search, mukherjee2019prior}%%%
\DIFdel{. 
}%DIFDELCMD < 

%DIFDELCMD < %%%
\DIFdel{Researchers have identified several aspects that explain team familiarity. Time is often highlighted by other researchers as the primary factor that enables teams to build and accumulate shared knowledge within teams }%DIFDELCMD < \cite{harrison2003time}%%%
\DIFdel{. Specifically, aspects such as the number of years spent in a specific role }%DIFDELCMD < \cite{huckman2009team}%%%
\DIFdel{, the average time team }\DIFdelend \DIFaddbegin \cite{10.1145/2998181.2998300} \DIFadd{found that crowd worker teams performed better when some of their }\DIFaddend members have worked together \DIFdelbegin %DIFDELCMD < \cite{gillespie2012factors, cattani2013tackling}%%%
\DIFdel{, and the presence of friendships both inside and outside of work all contribute to the development of team familiarity }%DIFDELCMD < \cite{de2017attuning}%%%
\DIFdel{. 
Nevertheless, other studies have shown that team familiarity not only encompasses time as the primary factor but also includes the quality of relationships between members, team cohesion }%DIFDELCMD < \cite{dittmer2020cut}%%%
\DIFdel{, composition, recognition of shared characteristics, diversity, and the effectiveness of their communication }%DIFDELCMD < \cite{rosen2023build, gully1995meta, marlow2018does, rico2019building}%%%
\DIFdel{.
}\DIFdelend \DIFaddbegin \DIFadd{in the past. 
}\DIFaddend 

While \DIFdelbegin \DIFdel{the benefits of team familiarity }\DIFdelend \DIFaddbegin \DIFadd{its benefits }\DIFaddend have been well documented\DIFdelbegin \DIFdel{by multiple studies, it }\DIFdelend \DIFaddbegin \DIFadd{, team familiarity }\DIFaddend also produces negative effects\DIFdelbegin \DIFdel{, particularly those related to the integration of newcomers into established teams. Team familiarity can inherently }\DIFdelend \DIFaddbegin \DIFadd{. It can }\DIFaddend lead to a culture where incumbents develop close networks \DIFdelbegin \DIFdel{that while enhancing internal cohesion and efficiency, may }\DIFdelend \DIFaddbegin \DIFadd{and }\DIFaddend pose significant barriers to new members, such as exclusion \cite{choi2004minority, joardar2007experimental}, intimidation \cite{topa2016newcomers}, favoritism \cite{balthazard2006dysfunctional}, or communication barriers \cite{kraut2010dealing}. \DIFaddbegin \DIFadd{Prior work shows that newcomers are often perceived as less capable or influential, limiting the team’s ability to explore diverse perspectives and solutions. Moreover, high levels of team familiarity can discourage individuals from challenging established viewpoints, ultimately hindering knowledge sharing and the generation of innovative ideas }\cite{assudani2011role, xie2020curvilinear}\DIFadd{.
}\DIFaddend 

\DIFaddbegin \subsubsection{\DIFadd{Newcomers}}
\DIFaddend Including newcomers to a team is important for organizations. Newcomers often bring fresh ideas, innovative approaches, and valuable resources that are crucial for the ongoing vitality of the group \cite{zeng2021fresh}. \DIFdelbegin \DIFdel{Yet}\DIFdelend \DIFaddbegin \DIFadd{However}\DIFaddend , integrating them effectively into an established group poses considerable challenges. They may face strong biases, and their mere presence, even when adhering to social norms, can be perceived as disruptive by existing members \cite{kraut2010dealing, spertus2001scaling}. These issues can make the \DIFaddbegin \DIFadd{existing }\DIFaddend group less desirable for those familiar with the existing dynamics. For instance, Joardar et al. \cite{joardar2007experimental} highlighted that introducing newcomers often triggers resistance from incumbents who may struggle to accept them as part of the group, potentially leading to tension and disruption of effective team functioning. Furthermore, cultural similarity between the newcomer and the incumbents plays a crucial role in facilitating positive group acceptance. Newcomers facing a significantly different socio-cultural environment are often less familiar with local expectations and norms, which can impede their ability to join smoothly \cite{furnham1982social}. The  \DIFdelbegin \DIFdel{resulting }\DIFdelend lack of cultural similarity can exacerbate the groups' challenges, complicating the newcomers' acceptance and diminishing their team cohesion \cite{nesdale2000immigrant}. 

\DIFdelbegin \subsection{\DIFdel{Team Familiarity Bias}}
%DIFAUXCMD
\addtocounter{subsection}{-1}%DIFAUXCMD
\DIFdel{We define familiarity bias as the tendency to collaborate and interact with familiar members rather than with new members. Previous research has shown that interacting with familiar members reduces individuals' uncertainty about future events and interactions }%DIFDELCMD < \cite{hinds2000choosing}%%%
\DIFdel{, and provides the foundations for trust in other membersof the team }%DIFDELCMD < \cite{mcpherson2001birds}%%%
\DIFdel{. }\DIFdelend \DIFaddbegin \subsubsection{\DIFadd{Online Group Communication}}
\DIFadd{The new settings of remote, hybrid, and IP workplaces are reshaping how individuals perceive themselves and assess their relationships with co-workers. Prior research highlights that online communication systems (e.g., text messaging, voice and video call) affect individuals' perceived differences and connections within groups }\cite{Hall2022,DESCHENES2024100351}\DIFadd{. Since individuals are not physically together, systems' design factors such as the presence of nonverbal cues, facial expressions, members' representation, and synchronous conversation will influence how aware participants are of their differences and similarities }\cite{giambatista2010diversity}\DIFadd{. Consequently, online communication systems can moderate how much rich information members can have of each other, leading users to perceive less of their demographic and physical differences }\cite{carte2004capabilities,SUH1999295}\DIFadd{. Providing more information about their characteristics and differences has the risk of avoiding teaming up with others who are different or unfamiliar }\cite{gomez2020impact}\DIFadd{. 
}\DIFaddend 

\DIFdelbegin \DIFdel{Familiarity biases vary significantly depending on the collaboration setting, such as in-person or a socio-technical system. The collaboration setting can amplify members ' perceived differences in a group }%DIFDELCMD < \cite{Shemla2016}%%%
\DIFdel{. }\DIFdelend HCI research has \DIFdelbegin \DIFdel{advanced to mitigate individuals' }\DIFdelend \DIFaddbegin \DIFadd{shown that online technologies can mitigate }\DIFaddend biases toward familiar individuals by altering the \DIFdelbegin \DIFdel{levels of users' identity }%DIFDELCMD < \cite{maloney2020anonymity,Whiting2020,Tzlil2018}%%%
\DIFdel{, in which anonymity could be a potential method to cover peoples' differences and bring them together}\DIFdelend \DIFaddbegin \DIFadd{presentation of user identities }\cite{maloney2020anonymity, Whiting2020, Tzlil2018}\DIFadd{. Several online platforms offer mechanisms like pseudonyms, avatars, and anonymity to depersonalize interactions }\cite{Shemla2016}\DIFadd{. These features reduce the emphasis on personal details, enabling more equitable participation in discussions and decision-making processes, which can lead to fairer and more balanced outcomes }\cite{christopherson2007positive, joinson2001self, walther1996computer}\DIFadd{. In particular, anonymity allows users to create online personas distinct from their offline identities, fostering self-expression and experimentation while promoting inclusivity and reducing biases in collaborative and social settings }\cite{bargh2002can, yurchisin2005exploration, kang2013people}\DIFaddend . However, \DIFdelbegin \DIFdel{anonymity can trigger undesired behaviors }\DIFdelend \DIFaddbegin \DIFadd{while anonymity can help mask individual differences, it may also lead to undesired negative behaviors, such as group polarization and harassment, }\DIFaddend due to the \DIFdelbegin \DIFdel{lack of accountability }%DIFDELCMD < \cite{Ma2016,CHRISTOPHERSON20073038}%%%
\DIFdel{.
Research has also shown the potential benefits of meeting new individuals on the Internet, which enables collaborations and relationships that are not attached to individuals' existing social networks and locations }%DIFDELCMD < \cite{baym2015personal}%%%
\DIFdel{. Online dating, meetup applications , and online communities provide a social infrastructure to enable new relationships }%DIFDELCMD < \cite{ridings2004virtual,Blackwell2015}%%%
\DIFdel{. Yet, how socio-technical systems guide individuals to meet new members or interact with already known ones has not been fully explored.  
}\DIFdelend \DIFaddbegin \DIFadd{reduced sense of accountability }\cite{weisband1993overcoming, Ma2016,CHRISTOPHERSON20073038}\DIFadd{.
}\DIFaddend 

%DIF < Collaborative technologies facilitate connections among team members, enabling them to tackle the opportunities and challenges of cross-boundary collaboration \cite{massey2008collaborative}. These technologies are divided into two primary dimensions based on interaction type. Synchronous interactions occur when team members collaborate simultaneously, utilizing tools like audio-based systems, text-based chat, messaging tools, and video conferencing \cite{robey2000information, lowenthal2023synchronous}. Conversely, asynchronous interactions involve collaboration at different times and locations, employing tools such as emails, repositories, or shared documents.
\DIFdelbegin %DIFDELCMD < 

%DIFDELCMD < %%%
\DIFdel{The choice of the communication system significantly impacts how team members perceive their differences. This is exemplified by research from Carte and Chidambaram }%DIFDELCMD < \cite{carte2004capabilities}%%%
\DIFdel{, which investigated how the type of online communication system influences members' perceived differences. In a nutshell, group meetings and online interactions will be constrained by the system's visual cues and the information provided to the users. This is explained by the amount and richness of the information transmitted through these systems }%DIFDELCMD < \cite{SUH1999295}%%%
\DIFdel{. While individuals can observe others' bodies, gestures, and emotions when meeting face to face, using socio-technical systems (e.g., videoconferencing, text messaging) will provide fewer of these cues. 
In another strand of the literature, recent studies have explored how online technologies affect workers' perceived social proximity, which is how close or far other organizational members seem to them }%DIFDELCMD < \cite{DESCHENES2024100351,van2023organizational}%%%
\DIFdel{. The new settings of remote, hybrid, and in-person workplaces will affect how workers identify themselves with the company and assess their relationships with co-workers. While some scholars argue that online technologies can enable new members to feel connected independently of their geographical distances }%DIFDELCMD < \cite{Taskin2024}%%%
\DIFdel{, others posit that new members' sense of belonging will depend on how strongly existing members employ these online technologies to welcome and interact with them }%DIFDELCMD < \cite{DESCHENES2024100351}%%%
\DIFdel{. }%DIFDELCMD < 

%DIFDELCMD < %%%
\DIFdel{In sum, how members perceive themselves---physically }\DIFdelend \DIFaddbegin \subsection{\DIFadd{Virtual Reality}}
\DIFadd{VR integrates advanced technologies that create immersive and interactive 3D environments }\cite{10.1145/3613904.3642471, 10.1145/3613905.3651085}\DIFadd{. By closely emulating the dynamics of IP conversations, VR enables synchronous and embodied interactions, allowing users to engage with the virtual space and each other through natural body movements and vocal communication }\cite{laato2024making, abbas2023virtual, mccloy2001science}\DIFadd{. Over recent years, many VR applications have increasingly supported collaborative interactions among multiple users, enabling them to be in the same virtual space regardless of their physical locations }\cite{10.1145/3411763.3441346,10.1145/3411764.3445426}\DIFadd{. Previous research has explored several features that enhance group interactions in VR, including the use of avatars, the sense of social presence, }\DIFaddend and \DIFdelbegin \DIFdel{socially--is crucial for understanding the subtle ways in which online technologies can either bridge the group or exacerbate the preference for already-known members of a group. %DIF < Morrison-Smith and Ruiz \cite{morrison2020challenges} discussed how complex managing the identities and differences in remote groups can be. Building common ground in a diverse team requires communication, shared work experiences, common vocabulary, and mental models. However, working on a medium \cite{cundill2019large, bjorn2014does}.
}\DIFdelend \DIFaddbegin \DIFadd{the spatial configuration that VR can provide }\cite{10.1145/3584931.3606992, sykownik2021most, fang2023towards}\DIFadd{. 
}\DIFaddend 

\DIFdelbegin \subsection{\DIFdel{Team Familiarity in Virtual Reality}}
%DIFAUXCMD
\addtocounter{subsection}{-1}%DIFAUXCMD
\DIFdel{Social communication between team members is significantly shaped by non-verbal cues, such as gaze, gestures, body language, and spatial distance }%DIFDELCMD < \cite{carmichael2023connecting, knapp1978nonverbal}%%%
\DIFdel{. These cues play an important role in building interpersonal relationships within teams by providing information that influences how team members perceive and interact with each other }%DIFDELCMD < \cite{knapp1978nonverbal}%%%
\DIFdel{. Prior research has shown that non-verbal communication can be facilitated in VR environments through avatar-based }\DIFdelend \DIFaddbegin \DIFadd{Avatars provide users with the flexibility to choose how they represent themselves in VR, serving as the primary identity cue that shapes perceptions and }\DIFaddend interactions \DIFaddbegin \cite{waltemate2018impact}\DIFadd{. This representation influences user behavior through stereotypes, a phenomenon known as the ``Proteus effect'' }\cite{yee2007proteus}\DIFadd{. For instance, research in VR demonstrates that embodying an elderly avatar in immersive environments can reduce stereotypical attitudes toward older individuals }\cite{yee2006walk}\DIFadd{. Similarly, Groom et al. }\cite{groom2009influence} \DIFadd{found that using avatars of a different race in VR led to measurable shifts in racial attitudes, suggesting that virtual embodiment enables users to challenge biases through flexible and diverse self-representation }\cite{banakou2016virtual}\DIFadd{. Avatar-based interactions }\DIFaddend that mimic non-verbal cues \DIFdelbegin %DIFDELCMD < \cite{freeman2016intimate, clark1991grounding}%%%
\DIFdelend \DIFaddbegin \DIFadd{foster these social interactions }\cite{freeman2016intimate, clark1991grounding, oh2016let}\DIFaddend . Maloney et al. \cite{maloney2020talking} found that non-verbal communication in social VR was perceived as positive and effective, as it offered a less intrusive method for initiating interactions with online strangers, promoting more comfortable and natural engagement. Similar research by \DIFdelbegin \DIFdel{Vosinakis and Xenakis }\DIFdelend \DIFaddbegin \DIFadd{Xenakis et. al }\DIFaddend \cite{xenakis2022nonverbal} highlighted how the reduction of body signals in VR, such as facial expressions and gestures, can decrease reliance on traditional physical cues. This shift helps minimize social biases typically present in face-to-face settings, allowing team members to focus more on communication and collaborative tasks rather than being influenced by appearance or other non-verbal indicators.

%DIF < - Can change your identity
\DIFdelbegin \DIFdel{VR systems have provided new opportunities for synchronous collaboration in immersive environments, offering a persistent 3D space where individuals can interact with others, virtual agents, and representations of both the physical and virtual worlds }%DIFDELCMD < \cite{gomez2023promise}%%%
\DIFdel{. Applications such as Meta’s Horizon Workrooms, Spatial, and VIVE Sync exemplify this by creating immersive virtual spaces where teams can collaborate in real time }%DIFDELCMD < \cite{abramczuk2023meet}%%%
\DIFdel{. A key feature of these systems is the use of avatars, which provide users the flexibility to choose how they represent themselves in virtual environments. The avatar serves as the primary identity cue, providing how individuals are perceived by others and influencing interactions within the digital environment. Prior research has explored how this flexibility in self-representation can impact others' social perceptions. For instance, embodying oneself as an elderly person in immersive environments has been shown to reduce stereotypical attitudes toward the elderly }%DIFDELCMD < \cite{yee2006walk}%%%
\DIFdel{. Similarly, Groom et al. }%DIFDELCMD < \cite{groom2009influence} %%%
\DIFdel{found that incorporating avatars of a different race in VR led to measurable shifts in racial attitudes, suggesting that virtual embodiment can alter social perceptions and help reduce bias by offering usersthe experience of flexible oneself representation }%DIFDELCMD < \cite{banakou2016virtual}%%%
\DIFdelend \DIFaddbegin \DIFadd{Prior research has also demonstrated that social presence is a critical factor for effective collaboration in VR }\cite{kimmel2023lets, sterna2021psychology, yassien2020design}\DIFadd{. VR users can experience an enhanced sense of co-presence, which creates the illusion of sharing a virtual space with their colleagues. The immersive quality of VR helps reduce the sense of detachment often associated with remote work, fostering more engaging interactions }\cite{wienrich2018social, smith2018communication} \DIFadd{and driving users' intention to collaborate }\cite{mutterlein2018specifics}\DIFaddend .

%DIF < - Remote workers are now equal (Social presence, spatial)
\DIFdelbegin \DIFdel{The }\DIFdelend \DIFaddbegin \DIFadd{Lastly, the }\DIFaddend spatial configuration of \DIFdelbegin \DIFdel{a work environment }\DIFdelend \DIFaddbegin \DIFadd{VR environments }\DIFaddend impacts how team members recognize and interact with each other. Proxemics, or spatial behavior, refers to the measurable distances between people as they communicate, influencing interpersonal dynamics \cite{hans2015kinesics}. In a virtual setting, virtual proximity replicates this effect, enabling team members to experience a shared presence in a virtual \DIFdelbegin \DIFdel{office}\DIFdelend \DIFaddbegin \DIFadd{room}\DIFaddend , even if they are physically distant \DIFaddbegin \cite{williamson2021proxemics}\DIFaddend . This shared virtual space fosters a sense of connection and engagement\DIFdelbegin \DIFdel{that reproduces in-person collaboration. According to Sousa et al. }%DIFDELCMD < \cite{sousa2016remote}%%%
\DIFdel{, the concept of remote proxemics in virtual environments allows for natural interaction patterns that mimic physical location, helping team members maintain an awareness of each other’s actions and presence. This sense of shared space helps }\DIFdelend \DIFaddbegin \DIFadd{, helping users }\DIFaddend recreate the interpersonal dynamics, such as proximity and orientation, that are important for effective collaboration \DIFdelbegin \DIFdel{. Furthermore, Slater and Sanchez-Vives }%DIFDELCMD < \cite{sanchez2005presence} %%%
\DIFdel{demonstrated that VR enhances social presence, making users feel like they are sharing the same room with their colleagues, thus reducing the detachment often associated with remote work. 
}\DIFdelend \DIFaddbegin \cite{li2021social, williamson2022digital}\DIFadd{. 
}\DIFaddend 

\DIFdelbegin \subsection{\DIFdel{Hypotheses}}
%DIFAUXCMD
\addtocounter{subsection}{-1}%DIFAUXCMD
\DIFdel{In this research project, we aim to explore whether using virtual reality can mitigate the effects of familiarity bias within teams, particularly between incumbents and newcomers. The reviewed studies provide a conceptual lens to understand how members' familiarity could be mitigated through online communication technologies . We posit that VR will help both incumbents and newcomers feel more connected with their counterparts, where social and professional differences are less pronounced. Since face-to-face interactions will likely trigger participants' perceived differences, we expect that the VR's design and capabilities can shift the focus away from physical attributes that might otherwise reinforce preconceived notions or social hierarchies. As such, using VR may neutralize the social mechanisms that encourage incumbents to stay together, allowing all team members to engage on a more level playing field and reducing the emphasis on superficial differences. Furthermore, since traditional online communication tools often constrain interaction compared to in-person dynamics, we expect that VR's design features will provide more opportunities for newcomers to participate and be included in a group. The reduced social attachment often seen in virtual settings may benefit newcomers by allowing them to feel more included and less overshadowed by the established relationships among incumbents.
Therefore, our hypotheses are as follows:
}%DIFDELCMD < \begin{itemize}
\begin{itemize}%DIFAUXCMD
%DIFDELCMD <     \item %%%
\item%DIFAUXCMD
\DIFdel{H1: Using Virtual Reality will have a positive effect on incumbents' closeness to newcomers. 
    }%DIFDELCMD < \item %%%
\item%DIFAUXCMD
\DIFdel{H2: Using Virtual Reality will have a positive effect on newcomers' closeness to incumbents.
    }%DIFDELCMD < \item %%%
\item%DIFAUXCMD
\DIFdel{H3: Using Virtual Reality will have a negative effect on incumbents' familiarity bias against newcomers. 
}
\end{itemize}%DIFAUXCMD
%DIFDELCMD < \end{itemize}
%DIFDELCMD < %%%
\DIFdelend %DIF > \subsection{Hypotheses}
\DIFaddbegin \DIFadd{Building on the HCI literature reviewed in this section, this study examines how VR can foster inclusion between old and new members of a team by mitigating their differences and biases. While prior research offers valuable insights into how online technologies influence team closeness, familiarity, and their effects on team dynamics, our work aims to explore how these mechanisms operate in VR, given its immersive and spatial nature. By addressing this gap, our study aims to deepen the understanding of closeness among team members and highlight the potential and limitations of VR in enhancing collaboration and integration within teams.
}\DIFaddend 

%DIF <  The prior work demonstrates that synchronous and asynchronous tools moderate the familiarity bias in different ways \cite{}. One example of a synchronous interaction tool is the Metaverse, which offers an immersive and persistent 3D environment where individuals can interact synchronously with others, virtual agents, and both virtual and physical world representations \cite{gomez2023promise}. Meta’s Horizon Workrooms exemplify such immersive virtual spaces where teams can collaborate effectively \cite{Meta2024}.
%DIF > We hypothesize that 

%DIF <  Moore et. al \cite{moore2020familiarity}  explores the impact of attitude familiarity on team performance within virtual reality environments. The study demonstrates that when team members are more familiar with each other’s attitudes, their performance on collaborative tasks, particularly in decision-making, improves significantly. This is attributed to enhanced communication and coordination, which are facilitated by a better understanding of each teammate’s perspective and expectations.
%DIF > \begin{itemize}
%DIF >     \item H1: VR will increase incumbents' closeness to newcomers. 
%DIF >     \item H2: VR will reduce incumbents' familiarity bias against newcomers. 
%DIF >     \item H3: VR will increase newcomers' closeness to incumbents.
%DIF >     \item H4: Users' perceived similarity will mediate the effect of using VR on team members' closeness.
%DIF > \end{itemize}



%DIF < Our work 
%DIF <  While previous work has analyzed various factors influencing team dynamics such as communication \cite{} , diversity \cite{} , geographical distance\cite{}, cultural differences \cite{}. Much of this research has focused on these factor under synchronous remote collaboration tools \cite{} such as Zoom, .. But, little work has explored the new ways of collaboration of new alternatives of synchronous work such as the Metaverse. 
\DIFdelbegin %DIFDELCMD < 

%DIFDELCMD < %%%
%DIF <  To mitigate that lack of familiarity Petersen and Pedersen could be resolved by engaging in learning activities \cite{petersen2002coping}. Moreover 
%DIFDELCMD < 

%DIFDELCMD < %%%
%DIF <  Papers to check
%DIFDELCMD < 

%DIFDELCMD < %%%
%DIF <  Lykourentzou, I., Kraut, R. E., & Dow, S. P. (2017, February). Team dating leads to better online ad hoc collaborations. In Proceedings of the 2017 ACM Conference on Computer Supported Cooperative Work and Social Computing (pp. 2330-2343).
%DIF <  Shin, D., Kim, S., Shang, R., Lee, J., & Hsieh, G. (2023, April). IntroBot: Exploring the use of chatbot-assisted familiarization in online collaborative groups. In Proceedings of the 2023 CHI Conference on Human Factors in Computing Systems (pp. 1-13).
\DIFdelend