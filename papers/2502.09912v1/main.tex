%%
%% This is file `sample-sigconf-authordraft.tex',
%% generated with the docstrip utility.
%%
%% The original source files were:
%%
%% samples.dtx  (with options: `all,proceedings,bibtex,authordraft')
%% 
%% IMPORTANT NOTICE:
%% 
%% For the copyright see the source file.
%% 
%% Any modified versions of this file must be renamed
%% with new filenames distinct from sample-sigconf-authordraft.tex.
%% 
%% For distribution of the original source see the terms
%% for copying and modification in the file samples.dtx.
%% 
%% This generated file may be distributed as long as the
%% original source files, as listed above, are part of the
%% same distribution. (The sources need not necessarily be
%% in the same archive or directory.)
%%
%%
%% Commands for TeXCount
%TC:macro \cite [option:text,text]
%TC:macro \citep [option:text,text]
%TC:macro \citet [option:text,text]
%TC:envir table 0 1
%TC:envir table* 0 1
%TC:envir tabular [ignore] word
%TC:envir displaymath 0 word
%TC:envir math 0 word
%TC:envir comment 0 0
%%
%%
%% The first command in your LaTeX source must be the \documentclass
%% command.
%%
%% For submission and review of your manuscript please change the
%% command to \documentclass[manuscript, screen, review]{acmart}.
%%
%% When submitting camera ready or to TAPS, please change the command
%% to \documentclass[sigconf]{acmart} or whichever template is required
%% for your publication.
%%
%%
%\documentclass[manuscript]{acmart}
\documentclass[sigconf]{acmart}
%%
%% \BibTeX command to typeset BibTeX logo in the docs
\AtBeginDocument{%
  \providecommand\BibTeX{{%
    Bib\TeX}}}

%% Rights management information.  This information is sent to you
%% when you complete the rights form.  These commands have SAMPLE
%% values in them; it is your responsibility as an author to replace
%% the commands and values with those provided to you when you
%% complete the rights form.
\copyrightyear{2025}
\acmYear{2025}
\setcopyright{cc}
\setcctype{by-nd}
\acmConference[CHI '25]{CHI Conference on Human Factors in Computing Systems}{April 26-May 1, 2025}{Yokohama, Japan}
\acmBooktitle{CHI Conference on Human Factors in Computing Systems (CHI '25), April 26-May 1, 2025, Yokohama, Japan}\acmDOI{10.1145/3706598.3714127}
\acmISBN{979-8-4007-1394-1/25/04}




%%
%% Submission ID.
%% Use this when submitting an article to a sponsored event. You'll
%% receive a unique submission ID from the organizers
%% of the event, and this ID should be used as the parameter to this command.
%%\acmSubmissionID{123-A56-BU3}

%%
%% For managing citations, it is recommended to use bibliography
%% files in BibTeX format.
%%
%% You can then either use BibTeX with the ACM-Reference-Format style,
%% or BibLaTeX with the acmnumeric or acmauthoryear sytles, that include
%% support for advanced citation of software artefact from the
%% biblatex-software package, also separately available on CTAN.
%%
%% Look at the sample-*-biblatex.tex files for templates showcasing
%% the biblatex styles.
%%

%%
%% The majority of ACM publications use numbered citations and
%% references.  The command \citestyle{authoryear} switches to the
%% "author year" style.
%%
%% If you are preparing content for an event
%% sponsored by ACM SIGGRAPH, you must use the "author year" style of
%% citations and references.
%% Uncommenting
%% the next command will enable that style.
%%\citestyle{acmauthoryear}
\usepackage{amsmath}
\usepackage{graphicx}
\usepackage{booktabs}
\usepackage{subcaption}
\usepackage[normalem]{ulem}
\useunder{\uline}{\ul}{}
%%
%% end of the preamble, start of the body of the document source.
\begin{document}

%%
%% The "title" command has an optional parameter,
%% allowing the author to define a "short title" to be used in page headers.
\title[Breaking the Familiarity Bias]{Breaking the Familiarity Bias: Employing Virtual Reality Environments to Enhance Team Formation and Inclusion}

%%
%% The "author" command and its associated commands are used to define
%% the authors and their affiliations.
%% Of note is the shared affiliation of the first two authors, and the
%% "authornote" and "authornotemark" commands
%% used to denote shared contribution to the research.
\author{Mariana Fernandez-Espinosa}
\orcid{0009-0004-1116-2002}
\affiliation{%
  \institution{University of Notre Dame}
  \city{Notre Dame}
  \state{Indiana}
  \country{USA}
}
\author{Kara Clouse}
\orcid{0009-0009-6466-7043 }
\authornote{Authors worked on this project as part of Notre Dame's Interdisciplinary Traineeship for Socially Responsible and Engaged Data Scientists (iTREDS).}
\affiliation{%
  \institution{University of Notre Dame}
  \city{Notre Dame}
  \state{Indiana}
  \country{USA}
}
\author{Dylan Sellars}
\orcid{}
\authornotemark[1]
\affiliation{%
  \institution{University of Notre Dame}
  \city{Notre Dame}
  \state{Indiana}
  \country{USA}
}
\author{Danny Tong}
\orcid{0009-0002-2449-0991}
\authornotemark[1]
\affiliation{%
  \institution{University of Notre Dame}
  \city{Notre Dame}
  \state{Indiana}
  \country{USA}
}
\author{Michael Bsales}
\orcid{}
\authornotemark[1]
\affiliation{%
  \institution{University of Notre Dame}
  \city{Notre Dame}
  \state{Indiana}
  \country{USA}
}
\author{Sophonie Alcindor}
\orcid{0009-0005-8035-2587}
\authornotemark[1]
\affiliation{%
  \institution{University of Notre Dame}
  \city{Notre Dame}
  \state{Indiana}
  \country{USA}
}
\author{Tim Hubbard}
\orcid{0000-0002-6029-7446}
\affiliation{%
  \institution{University of Notre Dame}
  \city{Notre Dame}
  \state{Indiana}
  \country{USA}
}
\author{Michael Villano}
\orcid{0000-0002-5212-326X}
\affiliation{%
  \institution{University of Notre Dame}
  \city{Notre Dame}
  \state{Indiana}
  \country{USA}
}
\author{Diego Gomez-Zara}
\orcid{0000-0002-4609-6293}
\email{dgomezara@nd.edu}
\affiliation{%
  \institution{University of Notre Dame}
  \city{Notre Dame}
  \state{Indiana}
  \country{USA}
}
\additionalaffiliation{%
  \institution{Pontificia Universidad Cat\'olica de Chile}
  \department{Facultad de Comunicaciones}
  \city{Santiago}
  \country{Chile}
}
%%
%% By default, the full list of authors will be used in the page
%% headers. Often, this list is too long, and will overlap
%% other information printed in the page headers. This command allows
%% the author to define a more concise list
%% of authors' names for this purpose.
\renewcommand{\shortauthors}{Fernandez-Espinosa et al.}

%%
%% The abstract is a short summary of the work to be presented in the
%% article.
\begin{abstract}
Team closeness provides the foundations of trust and communication, contributing to teams' success and viability. However, newcomers often struggle to be included in a team since incumbents tend to interact more with other existing members. Previous research suggests that online communication technologies can help team inclusion by mitigating members' perceived differences. In this study, we test how virtual reality (VR) can promote team closeness when forming teams. We conducted a between-subject experiment with teams working in-person and VR, where two members interacted first, and then a third member was added later to conduct a hidden-profile task. Participants evaluated how close they felt with their teammates after the task was completed. Our results show that VR newcomers felt closer to the incumbents than in-person newcomers. However, incumbents' closeness to newcomers did not vary across conditions. We discuss the implications of these findings and offer suggestions for how VR can promote inclusion.
\end{abstract}

%%
%% The code below is generated by the tool at http://dl.acm.org/ccs.cfm.
%% Please copy and paste the code instead of the example below.
%%
\begin{CCSXML}
<ccs2012>
   <concept>
       <concept_id>10003120.10003121.10003122.10011749</concept_id>
       <concept_desc>Human-centered computing~Laboratory experiments</concept_desc>
       <concept_significance>300</concept_significance>
       </concept>
   <concept>
       <concept_id>10003120.10003121.10011748</concept_id>
       <concept_desc>Human-centered computing~Empirical studies in HCI</concept_desc>
       <concept_significance>500</concept_significance>
       </concept>
   <concept>
       <concept_id>10003120.10003130.10003131</concept_id>
       <concept_desc>Human-centered computing~Collaborative and social computing theory, concepts and paradigms</concept_desc>
       <concept_significance>500</concept_significance>
       </concept>
   <concept>
       <concept_id>10003120.10003121.10003124.10011751</concept_id>
       <concept_desc>Human-centered computing~Collaborative interaction</concept_desc>
       <concept_significance>300</concept_significance>
       </concept>
 </ccs2012>
\end{CCSXML}

\ccsdesc[300]{Human-centered computing~Laboratory experiments}
\ccsdesc[500]{Human-centered computing~Empirical studies in HCI}
\ccsdesc[500]{Human-centered computing~Collaborative and social computing theory, concepts and paradigms}
\ccsdesc[300]{Human-centered computing~Collaborative interaction}

%%
%% Keywords. The author(s) should pick words that accurately describe
%% the work being presented. Separate the keywords with commas.
\keywords{Familiarity Bias, Newcomers, Team Formation, Team Inclusion, Virtual Reality}
%% A "teaser" image appears between the author and affiliation
%% information and the body of the document, and typically spans the
%% page.

%\received{20 February 2007}
%\received[revised]{12 March 2009}
%\received[accepted]{5 June 2009}

%%
%% This command processes the author and affiliation and title
%% information and builds the first part of the formatted document.
\maketitle


\section{Introduction}

\begin{figure*}
    \centering
    \includegraphics[width=\textwidth]{figures/Introduction.pdf}
    \caption{Showing the novel problem statement applied to traffic prediction use case. Multiple unstructured observations from the past are used to reconstruct a hidden traffic state from which a full traffic state is forecast with a set of query locations. }
    \label{fig:intro}
\end{figure*}

% Was sagen denn die anderen warum Traffic Prediction gut ist? 
Forecasting the traffic in the near future is an important task for city management.
Data from the near past is used to predict future traffic states with spatio-temporal Graph Neural Networks \cite{bui22}.
Accurate prediction provides the opportunity to optimize traffic flow, reduce traffic jams and increase air quality \cite{Po19}.

% Wieso ist Sparsity in allen Dimensionen wichtig.
While traffic prediction relies on the availability of data from traffic sensors, there exists a plethora of reasons why sensors may stop working temporarily, such as simple errors, energy saving, or overloaded communication systems.
Considering small- or medium-sized cities, the coverage of sensors may be low because the sensors are too expensive or not available.
Also, the sensors are typically static and do not adapt to changes in the traffic flow (e.g. caused by a construction site), which motivates moving sensors that for example could be mounted on cars. 
However, both missing and moving sensors introduce sparsity, since measurements may not be available for all locations at all times.
This sparsity must be explicitly addressed in traffic prediction for a realistic application scenario, which is illustrated in figure \ref{fig:intro}.
From one hour of data on Sunday morning, only few observations of the traffic state are available at each timestep.
The number of observations may differ throughout the observed time and the observation itself can be distributed arbitrarily in the city. 
We assume a relatively low number of sensors to account for resource saving and sensor failure in our proposed framework SUSTeR.
The task is to predict the dense traffic state one timestep after the observations at all possible sensor locations.
We study this problem on the traffic dataset Metr-LA and PEMS-BAY to test our assumption that only a fraction of the sensor values would be enough for good predictions.
By modifying an existing traffic dataset, we are able to compare our results from very sparse observations to the bottom line with all information available.
A successful study will provide insights in how sensors in new cities can be reduced before installing them and further mobile sensors would save more resources and are able to adapt to new traffic situations.
We argue that in order to be adaptable to other cities and changes in traffic flows, prior information like the road network should be neglected and just the sparse observations considered.
This comes with the added benefit of making our solution applicable in regions where no openly available road network is maintained or pathways change frequently (e.g. flood areas, animal observations). 


The aforementioned problem is novel and more challenging than the commonly considered traffic prediction problem, since there exist very few observations in each input sample.
Current works for the traffic prediction problem do not consider any missing values. \cite{Li2021, Shao22}
A common method among state of the art approaches is the usage of Graph Neural Networks on graphs that model the sensor network \cite{bui22}.
The values of a sensor are applied to the same graph node for each timestep which prohibits any non-stationary sensors . 
With fixed sensor locations, the resulting sensor network is highly correlated with the road network.
Streets connecting two intersections with sensors should be also an interesting point for correlations in the sensor network.
However, variable observations and high temporal sparsity rates can not be modeled adequately in a static network.
We show in our experiments that the road network has only a small influence on the traffic predictions.

Besides the traffic prediction for future timesteps, some works explore the field of traffic speed imputation \cite{Cini22, Cuza22} where missing sensor values are predicted.
But the amount of missing values is assumed to be at most 80\%, which on average are still over 40 given sensors in each timestep in the Metr-LA dataset with a total of 207 sensors.
We consider up to 99.9\% missing values which are on average 2.4 observations in each timestep that are used as input.
Such high sparsity rates drastically decrease the chance that multiple values are present in one input sample from the same sensor location, which makes it challenging to recognize and learn temporal correlations for each location on its own.

High sparsity rates (>95\%) result in few sensor values, but if a reconstruction of the traffic state would be possible, we question if spatio-temporal graphs require nodes for each sensor.
In SUSTeR we utilize only a small amount of graph nodes for the encoding of information and do not relate such nodes to the sensor network.
We call this the hidden graph (see figure \ref{fig:intro}), which is still able to reconstruct the complete traffic state.
Due to the reduced number of nodes SUSTeR achieves faster runtimes, as shown in the experiments.
This hidden graph is not embedded directly in the spatial domain, which is why the assignment of observations, as well as the querying of the future traffic, is done with an encoder and a decoder, implemented as neural networks.
The decoding from the hidden graph to future values depends on a set of query locations.
Figure \ref{fig:intro} shows the query locations as given from outside and in combination with the reconstructed traffic state the future values are predicted.

To construct the hidden graph we encode observations from each timestep into from multiple graphs, one for each timestep. 
The graphs are created in a residual style and information is added to the node embeddings from the previous timesteps.
We choose this method to incorporate all timesteps equally into the hidden state because the redundant information along the past is non-existing for high sparsity rates.
From the sequence of graphs where our framework inserted the observations step by step we apply STGCN \cite{Yu18}, an algorithm for traffic prediction to find and learn the spatio-temporal correlations on our small number of graph nodes.
The first future timestep of the STGCN is our hidden graph in which the traffic state is reconstructed. 

% Recent work has an implicit embedding of the graph nodes into the spatial domain as the assignment from the sensor to graph node is fixed one by one.
% Because the graph has the same structure as the road network spatio-temporal correlations can be learned between those sensors.
% We reduce the number of nodes and use a non-linear assignment learned data-driven from the observations.

We find in the experiments that SUSTeR outperforms the plain STGCN and modern traffic prediction frameworks like D2STGNN for high sparsity rates $(\geq 99\%)$.
This is equivalent to only $0.2$ to $2.4$ observation for each timestep on average.
SUSTeR uses fewer parameters than the baselines and can train faster and with less training data.
Our main contributions can be summarized as follows:
\begin{itemize}
    \item We introduce a sparse and unstructured variant of the traffic prediction problem with sparsity in all dimensions. The sensors report only a fraction of their values and are arbitrarily distributed in the spatial domain.
    \item We propose SUSTeR, a framework around the STGCN architecture, which maps sparse observations onto a dense hidden graph to reconstruct the complete traffic state.
    Our code is available at github.\footnote{https://github.com/ywoelker/SUSTeR}
    \item We conducts experiments that show that SUSTeR outperforms the baselines in very sparse situations ($\geq 95\%$) and has a competitive performance in low sparsity rates.
    % \item SUSTeR trains a third faster than the next competitor.
\end{itemize}

\section{Literature Review}
\label{literature_review}
We situate our work in the context of prior studies of teams---including closeness and familiarity---and VR research.

\subsection{Teams}
Teams consist of two or more individuals collaborating to achieve a shared objective \cite{salas2000teamwork}. As the foundation of organizational work, teams coordinate efforts to tackle complex tasks, demonstrating interdependencies in workflows, goals, and outcomes that reflect their collective responsibility \cite{kozlowski2006enhancing, gomez2020taxonomy, baron2003group}.

\subsubsection{Team Closeness}
Team closeness measures team members' subjective perception of how connected they feel to other members of the team \cite{Gachter2015}. Individuals can feel closer to specific team members based on their interactions and not necessarily on their physical proximity \cite{Wiese2011}. While team cohesion usually refers to how united team members are as a whole \cite{salas2015measuring}, closeness relates to relationships between pairs, which has been studied at the interpersonal level \cite{rosh2012too}. 

Researchers have identified several factors that contribute to building closeness \cite{moreland1982exposure}. Time is a key element, enabling teams to build shared knowledge and establish trust among members \cite{harrison2003time, gillespie2012factors, cattani2013tackling}. Previous relationships also play an important role, as positive prior experiences can reduce uncertainty among interactions between team members \cite{de2017attuning, jones2019essentials, dittmer2020cut}. Lastly, similarity affects closeness since team members who share common attributes such as demographics, values, or experiences are more likely to develop stronger connections and a sense of familiarity \cite{hinds2000choosing, ruef2003structure, winship2011homophily}.

\subsubsection{Team Familiarity}
Team familiarity refers to the amount of experience they have working together \cite{muskat2022team,mukherjee2019prior}. Previous research has shown that team familiarity can be a predictor of improved performance, enhancing creativity, efficiency, and the overall quality of a team's output \cite{dittmer2020cut, huckman2009team, witmer2022systematic}. For instance, Sosa demonstrated through a sociometric study that teams with stronger interpersonal connections and shared knowledge bases were significantly more likely to generate innovative solutions to complex problems \cite{sosa2011creative}. Similarly, Staats \cite{staats2012unpacking} found that increased familiarity was associated with improving team performance. Lastly, Salehi et al. \cite{10.1145/2998181.2998300} found that crowd worker teams performed better when some of their members had previously worked together. 

While its benefits have been well documented, team familiarity also produces adverse effects. It can lead to a culture where incumbents develop close networks and pose significant barriers to new members, such as exclusion \cite{choi2004minority, joardar2007experimental}, intimidation \cite{topa2016newcomers}, favoritism \cite{balthazard2006dysfunctional}, or communication barriers \cite{kraut2010dealing}. Prior work shows newcomers are often perceived as less capable or influential, limiting the team's ability to explore diverse perspectives and solutions. Moreover, high levels of team familiarity can discourage individuals from challenging established viewpoints, ultimately hindering knowledge sharing and the generation of innovative ideas \cite{assudani2011role, xie2020curvilinear}.

\subsubsection{Newcomers}
Including newcomers to a team is important for organizations. Newcomers often bring fresh ideas, innovative approaches, and valuable resources that are crucial for the ongoing vitality of the group \cite{zeng2021fresh}. However, integrating newcomers effectively into an established group poses considerable challenges. They may face strong biases, and their mere presence, even when adhering to social norms, can be perceived as disruptive by existing members \cite{kraut2010dealing, spertus2001scaling}. These issues can make the existing group less desirable for those familiar with the existing dynamics. For example, Joardar et al. \cite{joardar2007experimental} highlighted that the introduction of newcomers often triggers resistance from incumbents who may struggle to accept them as part of the group, potentially leading to tension and disruption of effective team functioning. Furthermore, cultural similarity between the newcomer and the incumbents plays a crucial role in facilitating positive group acceptance. Newcomers facing a significantly different socio-cultural environment are often less familiar with local expectations and norms, which can impede their ability to join smoothly \cite{furnham1982social}. The lack of cultural similarity can exacerbate the groups' challenges, complicating the newcomers' acceptance and diminishing their team cohesion \cite{nesdale2000immigrant}. 

\subsubsection{Online Group Communication}
The new settings of remote, hybrid, and in-person (IP) workplaces are reshaping how individuals perceive themselves and assess their relationships with co-workers. Prior research highlights that online communication systems affect individuals' perceived differences and connections within groups \cite{Hall2022,DESCHENES2024100351}. Since individuals are not physically together, systems' design factors such as nonverbal cues, facial expressions, members' representation, and synchronous conversation will influence how aware participants are of their differences and similarities \cite{giambatista2010diversity}. Consequently, online communication systems can moderate how much rich information members can have of each other, leading users to perceive less of their demographic and physical differences \cite{carte2004capabilities,SUH1999295}. Providing more information about their characteristics and differences has the risk of avoiding teaming up with others who are different or unfamiliar \cite{gomez2020impact}. 

HCI research has shown that online technologies can mitigate biases toward familiar individuals by altering the presentation of user identities \cite{maloney2020anonymity, Whiting2020, Tzlil2018}. Several online platforms offer mechanisms like pseudonyms, avatars, and anonymity to depersonalize interactions \cite{Shemla2016}. These features reduce the emphasis on personal details, enabling more equitable participation in discussions and decision-making processes, which can lead to fairer and more balanced outcomes \cite{christopherson2007positive, joinson2001self, walther1996computer}. In particular, anonymity allows users to create online personas distinct from their offline identities, fostering self-expression and experimentation while promoting inclusivity and reducing biases in collaborative and social settings \cite{bargh2002can, yurchisin2005exploration, kang2013people}. However, it may also lead to undesired negative behaviors, such as group polarization and harassment, due to the reduced sense of accountability \cite{weisband1993overcoming, Ma2016, christopherson2007positive}.

\subsection{Virtual Reality}
VR integrates advanced technologies that create immersive and interactive 3D environments \cite{wohlgenannt2020virtual, 10.1145/3613905.3651085, hubbard2024cross}. By closely emulating the dynamics of IP conversations, VR enables synchronous and embodied interactions, allowing users to engage with the virtual space and each other through natural body movements and vocal communication \cite{laato2024making, abbas2023virtual, mccloy2001science}. In recent years, many VR applications have increasingly supported collaborative interactions among multiple users, enabling them to be in the same virtual space regardless of their physical locations \cite{li2021social,10.1145/3411764.3445426}. Previous research has explored several features that enhance group interactions in VR, including the use of avatars, the sense of social presence, and the spatial configuration that VR can provide \cite{10.1145/3584931.3606992, sykownik2021most, fang2023towards}. 

Avatars provide users with the flexibility to choose how they represent themselves in VR, serving as the primary identity cue that shapes perceptions and interactions \cite{waltemate2018impact}. This representation influences user behavior through stereotypes, a phenomenon known as the ``Proteus effect'' \cite{yee2007proteus}. For instance, research in VR demonstrates that embodying an elderly avatar in immersive environments can reduce stereotypical attitudes toward older individuals \cite{yee2006walk}. Similarly, Groom et al. \cite{groom2009influence} found that using avatars of a different race in VR led to measurable shifts in racial attitudes, suggesting that virtual embodiment enables users to challenge biases through flexible and diverse self-representation \cite{banakou2016virtual}. Avatar-based interactions that mimic non-verbal cues foster these social interactions \cite{freeman2016intimate, clark1991grounding, oh2016let}. Maloney et al. \cite{maloney2020talking} found that non-verbal communication in social VR was perceived as positive and effective, as it offered a less intrusive method for initiating interactions with online strangers. Similar research by Xenakis et. al \cite{xenakis2022nonverbal} highlighted how the reduction of body signals in VR, such as facial expressions and gestures, can decrease reliance on traditional physical cues. 

Prior research has also demonstrated that social presence is a critical factor for effective collaboration in VR \cite{kimmel2023lets, sterna2021psychology, yassien2020design}. VR users can experience an enhanced sense of co-presence, which creates the illusion of sharing a virtual space with their colleagues. The immersive quality of VR helps reduce the sense of detachment often associated with remote work, fostering more engaging interactions \cite{wienrich2018social, smith2018communication} and driving users' intentions to collaborate \cite{mutterlein2018specifics}.

Lastly, the spatial configuration of VR environments impacts how team members recognize and interact with each other. Proxemics, or spatial behavior, refers to the measurable distances between people as they communicate, influencing interpersonal dynamics \cite{hans2015kinesics}. In a virtual setting, virtual proximity replicates this effect, enabling team members to experience a shared presence in a virtual room, even if they are physically distant \cite{williamson2021proxemics}. This shared virtual space fosters a sense of connection and engagement, helping users recreate the interpersonal dynamics, such as proximity and orientation, that are important for effective collaboration \cite{li2021social, williamson2022digital}. 

VR environments influence individuals' perceptions and behaviors during collaboration by abstracting physical features and reducing appearance-based cues that often highlight differences among team members. This shift redirects attention toward collaborative interactions, potentially diminishing social biases and the perceived distinction between incumbents and newcomers. Building on this reasoning, we propose that the mode of interaction affects both how incumbents perceive closeness to newcomers and how newcomers perceive closeness to incumbents. Our first two hypotheses are:

\begin{quote}
    \textbf{H1}: \textit{When onboarding new members, incumbents will perceive greater closeness to newcomers when meeting in VR than when meeting in person.}
\end{quote}
\begin{quote}
    \textbf{H2}: \textit{When onboarding new members, newcomers will perceive greater closeness to incumbents when meeting in VR than when meeting in person.}
\end{quote}

When individuals unfamiliar with one another come together to assemble a team, they initially have limited information to assess each other or establish expectations for their interactions. To alleviate uncertainty, team members often rely on readily available cues, such as physical appearance or demographic traits, to quickly form impressions \cite{allport1954nature,hinds2000choosing}. However, these initial judgments can reinforce social biases, favoring familiar individuals or strengthening preexisting group distinctions. Prior research suggests that VR can reduce such biases through depersonalization and flexible identity representation. By shifting attention away from physical and cultural differences as well as preexisting relationships, VR can foster a more inclusive environment that reduces perceived distinctions between incumbents and newcomers. Therefore, we hypothesize:
\begin{quote}
    \textbf{H3}: \textit{Incumbents' familiarity bias against newcomers will be lower when meeting in VR than when meeting in person.}
\end{quote}

Lastly, one mechanism that could explain how VR fosters closeness is \textit{perceived similarity}---team members who see themselves as more alike tend to experience higher trust and a stronger sense of bonding \cite{zellmer2008and, mannix2005differences}. In VR, avatars and immersive environments can attenuate physical or demographic differences \cite{lopez2019investigating, marini2022can}, potentially making team members feel more similar and accentuating shared goals. This reduction in perceived differences may enhance perceptions of similarity, ultimately fostering closeness. Thus, we hypothesize:
\begin{quote}
    \textbf{H4}: \textit{Perceived similarity will mediate the relationship between VR (versus in-person) and team closeness, such that using VR increases perceived similarity among team members, thereby enhancing closeness.}
\end{quote}

\section{Iterative Data-based V-model}\label{03_Methodology}

Even though the existing development and V\&V frameworks for emerging complex systems, particularly automated vehicles, differ in their terminology, methodological specifics, and different emphasis, there is a general common sense that arises from the inherent nature of the engineering processes. Furthermore, while corresponding safety assurance is crucial for approval, it is highly dependent on the system and the environment, which is also referred to as ODD. Therefore, the overall methodology should be decoupled from the application-specific safety assurance, generalized across existing frameworks, and formalized along the transitions between simulation and the real world, while striving to support and enable safety argumentation along the resulting iterative process reference model in a general form. This requires, similar to the classical V-model, a certain level of abstraction for widespread applicability. Therefore, a detailed, fully-fledged safety case that leads to direct release cannot be provided. Consequently, the iterative data-based V-model is introduced to build on the generality of the classical V-model, bridging existing frameworks while simultaneously addressing the challenges of emerging systems and technologies in a sophisticated manner. Thereby, a particular emphasis is laid on cognitive cyber-physical systems of the real world that are safety critical to adress challenges of arising autonomous technologies.

\subsection{Fundamental Principles}
Adressing emerging autonomous technologies in a general manner highlights the criticality of a generic scneario-based approach \cite{riedmaier2020survey, elster2021fundamental, VVMOverall}, as corresponding database generation could be exhaustive and costly. Consequently, this would limit the suitability of the methodology towards solid technological applications that justify the efforts. Elsewise, a silent testing \cite{Tesla_shadow, templeton2019} approach for corresponding data acquisition, as scenario database equivalent, is also limited by the fact that dedicated systems and applications must already be operating on a large scale in the target environment. Beyond that, direct data collection in the real world might also be inappropriate, as it could lead to hazards and does not provide the necessary scale of datasets. As a consequence, the general methodology of the emerging autonomous systems is based on systematically generated data on a large scale through simulation. This observation is inline with the project series AI Family \cite{KIFamilie} that covers a range of sub-projects on AI assurance \cite{KIAbsicherungSynData}, transfer \& scaling \cite{KIDeltaSynData}, data tooling \cite{KIDataTooling}, and the hybridization of knowledge and data for automated driving applications \cite{KIWissen_D1_D4, KIWissen_D2}. Although the project series dealt with core topics such as the exploitation of simulation and real data, an overarching reference process has not been established. 

\subsection{General Methodology}\label{sec:methodo}
The proposed iterative data-based V-model builds on the VVM \cite{VVM} project by taking up the general V-model structure that is enhanced by a consistent alignment of the ODD along the V-stages. Beyond the application of automated driving, the ODD is interpreted as a generic concept that systematically defines the environment and context of the respective system. In addition, the iterative data-based V-model reflects Waymo's dynamic lifecycle approach \cite{favaro2023building}. In this way, the open world associated challenges as well as systematic gaps are addressed in a natural, iterative and continuous refinement manner. While the approach focuses on the product and provides guidance through the reference process, the chosen level of abstraction also allows for underlying process refinement along the safety argumentation. In addition, the iterative data-based V-model generalizes the scenario-based database approach. Furthermore, the iterative data-based V-model considers AI-specific development aspects and takes up ideas of the data engine \cite{karpathy_cvpr21} to increase efficiency. However, a central part remains a V\&V that is adapted to the challenge of the targeted use of simulation and the real world. Along a systematic consideration of the strengths of innovative frameworks while addressing existing limitations when applied to complex systems, as discussed in Section (\ref{sec:diff}). This results in an iterative data-based V-model, which is outlined in Figure \ref{fig:ours}. The individual aspects are described in more detail below. 

\textbf{Operational Design Domain:} The cycle initially starts with the definition of the ODD. Here, the ODD states an application independent foundation for defining the system's context, e.g. the environment and operating conditions. Besides that, the ODD represents the top-level target designation and belonging requirements specification. Moreover, the ODD provides guidance for the requirement definition of subordinate systems and functions. Thus, taking into account the system ODD independently of the granularity of the functionality or component to be developed ensures top-level aligment.

\textbf{Function-specific ODD:} Here, the specific functionality of the system, subsystem, or component to be developed is specified by means of a dedicated ODD. Thereby, following the top-down decomposition, the relevant top-level targets of the ODD are stringently broken down and transferred to the respective function-specific ODD. Moreover, specifics such as the desired behavior of the functionality that does not result directly from the top-level targets is included. Accordingly, the explicit function-specific ODD compactly defines the requirements specification while simultaneously minimizing specification uncertainties \cite{burton2023closing} through its alignment with the ODD.

\textbf{Data-specific ODD:} In this context, the previously outlined requirements specification is transfered into a data perspective. In this way, the data-specific ODD accounts for the fact that the availability, quality, and nature of the data can have a direct impact on how the system performs the task at hand. Thus, the definition of the data-specific ODD is responsible
for transforming the function-specific ODD requirements into the data-based representation. Making this transformation explicit within the framework should minimize the systematic gaps. In particular, the fact that development and V\&V are data-driven highlights the importance of a precise translation of the formally defined specifications and requirements into a data-specific representation.

\begin{figure*}[]
	\centering	
	\includegraphics[width=0.7\linewidth]{img/LRT_UL_X_3.png}
	\caption{The iterative data-based V-model, which formalizes and merges the various existing methods. The initial loop starts with the definition of the ODD. The explicit formalization of the process from the real world to simulation and back and the data-based characteristic address the challenges of complex systems that embrace AI. The iterative approach, on the other hand, addresses the challenges of open world complexity and offers continuous system and confidence improvement in an intuitive way.}
	\label{fig:ours}
\end{figure*} 

Consequently, the data-specific ODD, which is responsible for the syntactic transition into the data domain while preserving the semantics to meet required demands, constitutes an interface. In more detail, the data-specific ODD represents an handover area between system domain experts and funtion experts (e.g. AI experts), who use different terminologies (linguistic gap) and have different understanding (knowledge gap) due to different backgrounds (specialization gap). This entails the risk that the semantics of the requirements are not fully translated and usually leads to a specification gap. The framework aims to tackle and minimize this risk, which has already been described as specification uncertainties \cite{burton2023closing} that lead to specification insufficiencies \cite{burton2023addressing} and result in a semantic gap. The goal of the data-specific ODD is to enhance the efficiency and effectiveness of developing and ultimately safeguarding emerging complex systems that incorporate AI. This target-oriented addressing of the semantic gap can lead to overarching improved performance, reliability, and robustness.

\textbf{Architectural Design Domain / Data Design Domain:}  The two components of this stage address the transition to development. The architectural design domain defines a framework of possible architectures of the system to be developed on basis of the data-specific ODD. Thus, on the one hand, this step defines requirements for the architecture while on the other hand first assumptions of the design process that constitute from the data-specific ODD are made explicit. Thus, this stage represents a design abstrachtion layer for general descisions during the development and improvement processes.

In parallel, the definition of the data design domain transfers the requirements of the data-specific ODD into the requirements for
generating the data. However, the generation of data is not part of this stage. This deliberate separation of the definition of a design domain from the realization is an integral part of the framework and aims at the explicit separation, formulation, and documentation of requirements and assumptions while at the same time disclosing process related
uncertainties. In addition, this enables a decoupled validation of the design domains and the implementation via the continuous refinement process and indicates the need for action. For example, if deficiencies are identified during monitoring, it is possible to check whether the data design domain was valid and the problem arises from the incompleteness of the dataset in relation to it, or whether a refinement of the previous specifications and assumptions is necessary. 

\textbf{Architecture Definition / Sim. Dataset Generation:} This stage represents a deliberated engineering stage that focus on both the architecture setup as well as the simulation based data generation. The architectural definition selects a certain architectur within the corresponding architectural design domain and determines the specifics like parameters. 

The simulation data generation characterizes the systematic synthesis of data with respect to the data design domain. While the architecture selects a dedicated realization from the set of possible architectures, the creation of the dataset aims at completeness. In line with other frameworks like \cite{VVMOverall, favaro2023building, karpathy_cvpr21} and standards such as SOTIF \cite{iso21448}, the iterative data-based V-model refers to a complete coverage of the whole space by a sufficient decomposition through trigger constraints, yet in a data-based way. Moreover, the strategic generation of synthetic data offers the possibility of uncovering trigger conditions at an early stage and countering the challenge of "unknown unknowns". For example, logical inference based on existing trigger conditions can be used for this purpose.  

\textbf{System Development:}  This level represents the final implementation, which takes up and connects the previously decoupled paths of the architecture and data, and yields to a resulting system. Thereby, the system can consider individual dedicated (AI) models up to a system of systems, depending on the task. In addition, with regard to the overall process reference model, the system development phase completes the left leg of the V, which represents the design phase, and forms the intersection with the V\&V phase, the right leg of the V that is integral to the iterative data-based V-model.

\textbf{System Evaluation:} The first evaluation stage within the V\&V phase is represented by system evaluation. Here, the previously stipulated (functional) requirements are verified on the basis of the test split of the generated simulation data. Accordingly, this first evaluation stage can also be referred to as a virtual system testing. In more detail, this simulative performance evaluation is carried out using specified trigger conditions provided in terms of the dataset and assesses performance through corresponding indicators and acceptance criteria, which have to be derived from the individually customizable safety argumentation. 

\textbf{Real World Dataset Generation:} In this context, real-world data is generated. Data generation takes place in accordance with the data design domain. For safety reasons, the real world dataset represents a subset of the simulation dataset. Moreover, the effort in the real world can be limited for reasons of cost-effectiveness if this is permitted by the safety argument. As a consequence, the dedicated requirement for the scope of the real world dataset depends on the individual functionality to be developed and the corresponding safety argumentation and assurance. Therefore, a general specification cannot be provided by the process reference model.

\textbf{System Transfer:}  In order to counteract the gap between simulation and real world, the system transfer stage is introduced explicitly. The systems functionality can thus be adapted on the basis of the real data from the previous stage.

\textbf{Transfer Evaluation:} This second evaluation stage within the V\&V phase constitutes the transfer evaluation of the system. Here, the previously stipulated requirements are verified on the basis of the test split of the generated real world data. Consequently, this second evaluation stage can also be described as a simulation-based system test, which is carried out by means of real world data. The procedure as well as the performance indicators and acceptance criteria can be adopted from the first evaluation stage, the system evaluation, as the main difference is the changed origin of the data. Consequently, the desired functionality of the system and the associated quality and assessment criteria can be adopted.

\textbf{Open-Loop Evaluation (Silent Testing):}  This is the third evaluation stage, which marks the transition back to the real world. More precisely, the developed system is evaluated in the real world with the real input, whilst the system output is not applied in the real world. This open-loop evaluation thus represents a first step in the gradual return to the real world. In recent years, the open-loop evaluation has received increasing attention and can also be interpreted as silent testing \cite{wang2021online}, shadow mode, and virtual assessment of automation in field operation (VAAFO) \cite{wang2020reduction}.

\textbf{Closed-Loop Evaluation:} In the context of the fourth evaluation stage, the term closed-loop refers to the previous open-loop, which is now closed, meaning that the evaluation formalizes the system-based feedback into the real world. Here, the functionality is initially verified in the real world. Furthermore, this closed-loop evaluation phase is best viewed as a real world system test or, with regard to the application of automated driving, as a vehicle-in-the-loop (VIL). In the case of automated driving, this system evaluation can be carried out on the proving ground, for example. 

\textbf{Field Operation Evaluation:} The fifth and final evaluation phase is carried out in the real world on a larger scale than the previous one and aims to validate the system in the real world. It can be seen that V\&V, which was mostly considered jointly, is separate in the real world. While verification in the real world is possible on the proving ground, for example, validation requires further operation in the field. In principle, this kind of evaluation setting matches the consideration of on-the-road tests in automated driving applications.

\textbf{Deployment - System Operation \& Monitoring:} Once the system has successfully passed all evaluations, the acceptance criteria stipulated by the safety argumentation are fulfilled and the system can be depolyed. Continuous trust building arise in the course of system operation and ongoing monitoring. Furthermore, intelligent data harvesting can be performed during operation. In automated driving, for example, each individual vehicle can collect and select local data such that the cumulative gain can be used for subsequent refinement measures. 

\textbf{Detection of Deficiencies:} By means of the safety argumentation, corresponding acceptance criteria, and system specific performance indicators, deficiencies can be detected. If, e.g., an "unknown unknown" is detected at the overall level, which can have or has had catastrophic consequences, the approval of the systems must be withdrawn and the development and assurance process must cycled again from the beginning, taking into account the adapted requirements. While this is defined here in a structured way, we can see a very similar approach in practice today in the example of Cruise LCC in the USA, according to the incident of hitting a pedestrian \cite{equipmentrecallreport, NHTSARecall23E, NHTSARecallLetter}.

\textbf{Continuous Refinement:} In case of insufficencies, the overall loop is reinitiated by starting with a refinement of the ODD. A continuous transfer of the acquired findings back into the simulation and a step-by-step return to the real world is an integral part of the framework. Even in the long term, the use of synthetically generated data is targeted for safety, and efficiency reasons. In terms of trigger conditions, real world data might be limited to uncovered trigger conditions, while synthetic data allows the consideration of conceivable but unseen trigger conditions. Thus, the use of exclusively real data represents a subspace of imaginable trigger conditions and is therefore only effective in combination with synthetically augmented data. Accordingly, synthetic data can increase safety and efficiency in the development process. In general, moreover, the continuous refinement of the framework leverages the iterative nature of error analysis and specification adaptation, which increases safety and efficiency throughout the entire product lifecycle.

\textbf{Failure Handling:} Only the forward-looking transitions are indicated in Figure \ref{fig:ours}. Nevertheless, a large number of subordinate feedback flows are present, which have been omitted for the sake of clarity. Dealing with failed evaluations is of particular importance, which is why this is described in more general terms. 

On the one hand, an error can be caused by the design and realization of the functionality. On the other hand, it can be caused by gaps in the data used. Analyzing the error case can provide more information about the underlying cause. If an error case is caused by data and the data is within the data design domain, there is an error in the dataset generation. However, if the error case data is part of the data-specific ODD but not the data design domain, there is a gap in the specification of the data design domain. Otherwise, if the error case data is neither part of the data-specific ODD, the data-specific ODD definition itself must be adapted. An adaptation of a specification, e.g., the data-specific ODD or the data design domain, always requires stepping back to this level and revising the subsequent levels.

However, the functionality itself can also be the source of an error. If the selected function is within the architectural design domain, it is conceivable that the choice of architecture, although permissible, was not appropriate. If the realized functionality falls within the data-specific ODD but is not covered by the architectural design domain, the architectural design domain must be updated. If the function causes a behavior that does not correspond to the data-specific ODD but does correspond to the function-specific ODD, the data-specific ODD must be updated. Ultimately, gaps in the function-specific ODD can also lead to a subsequent error, which may require this specification to be updated. Furthermore, it is generally assumed that the assessments are conducted within the ODD. Otherwise, an ODD refinement is deemed to take place.

\textbf{Safety Argumentation Decoupling:} Along the process outlined above, the safety argumentation and safety assessment are customizable. While the VVM project \cite{VVMOverall} and Waymo's safety determination lifecycle \cite{favaro2023building} specify the absence of unreasonable risk in detail with respect to the system, this is omitted by the iterative data-based V-model, analogous to the classical V-model \cite{brohl1993v}. Nevertheless, it is evident that distinct evaluation stages throughout the V\&V process outlined above require the definition of acceptance criteria based on a system-specific safety argumentation. In this way, Waymo's case credibility assessment \cite{favaro2023building} is built-in by design, albeit with greater flexibility. This is due to the fact that the safety argumentation is not predefined. Therefore, to achieve the overall objectives, the framework enforces the creation, evaluation, and refinement of the safety argumentation. In particular, this customizability of the safety argumentation within the proposed framework implicitly leads to reasonableness, confidence, and coverage assessments along the interative refinement and the corresponding approval, similar to Waymo's case credibility assessment. Since the safety argumentation depends on the system as well as its environment and context, in other words specific on the ODD, and is also subject to application-specific standards and regulations, the safety argumentation is required to be decoupled and customizable to achive the desired generality of the methodology.


\subsection{Classification and Delimitation of the Methodology}\label{sec:classi}

The framework of the proposed iterative data-based V-model takes up existing further developments of the V-model \cite{VVMOverall} as well as innovative frameworks \cite{favaro2023building, karpathy_cvpr21} for handling complex systems. It addresses the characteristics of complex systems that integrate AI. This is illustrated by the use of data-based methods and the separate consideration of the architecture. In particular, complex systems are handled by means of dedicated stages and the formalized exploitation of simulation and real data. Central aspects from the AI Family project \cite{KIFamilie}, in particular the two sub-projects AI Delta Learning \cite{KIDeltaSynData} and AI Data Tooling \cite{KIDataTooling}, are thus taken up and formally integrated into an overall perspective. At the same time, the iterative data-based V-model offers the opportunity to take the results of the above-mentioned projects \cite{KIFamilie, KIDeltaSynData, KIDataTooling} into account. For instance, the stringent safety argumentation that was developed for a pedestrian detection AI \cite{KIAbsicherungSynMethoden, KIAbsicherungSynAbsicherung} can be considered. In addition, approaches for generating synthetic data \cite{KIAbsicherungSynData, KIDeltaSynData} can also be incorporated. This illustrates the generic and unifying character of the iterative data-based V-model framework. Moreover, the framework offers the possibility of using the emerging imaginative intelligence \cite{wang2024does, li2024sora} in the development process and is therefore also equipped for future developments.

Furthermore, the transition from a scenario-based approach to a general data-based approach opens up the broad applicability of the methodology. Additionally, the present approach enables scalability by the consideration of different levels of granularity of a system and, thus, also the system complexity. Thereby, the methodology addresses the architecture layer, the behavioral layer, as well as the in-service operational layer of Waymo \cite{karpathy_cvpr21} or the capability, engineering, and real world layer of the VVM project \cite{VVMAPerspectives}. Due to the claim of a generic framework and the necessity of individual performance measures and acceptance criteria, the framework does not claim to be a comprehensive framework for safety assessment and assurance. Nevertheless, Waymo's Case Credibility Assessment \cite{favaro2023building} approach is inherently integrated. This is due to the fact that the individual definition of the safety argumentation and thus the acceptance criteria and performance indicators require a suitability assessment, while the multiple assessments under increasingly realistic conditions require to address the coverage assessment. The iterative approach implies a process refinement within the structure defined by the framework. Hence, a native implementation of credibility \cite{koopman2019credible} can be envisioned. In particular, with increasing time and scale of the system, validation of the process and product takes place, thus addressing trust and credibility. This analysis of the proposed framework in relation to the improved V-model \cite{VVMOverall} of the VVM project, the safety determination lifecycle \cite{favaro2023building} of Waymo, and the data engine \cite{karpathy_cvpr21} of Tesla demonstrates that the iterative data-based V-model specifically combines the various methods and perspectives and formalizes them at the macro process level in order to maintain the generality of the classical V-model \cite{brohl1993v}.

While the framework does not explicitly address safety assessment and assurance, it opens up the possibility of data-driven functional safety assurance along with the potential to incorporate the results of current research from projects such as SUNRISE \cite{SUNRISE} and V4SAFETY \cite{V4SAFETY}. While allowing for statistical assurance of AI systems, the framework can also be applied to more traditional approaches. Ultimately, this leaves space for the use of different concepts. This is particularly important with regard to safety and explainability of AI systems, as research in this area is still raising open questions and different solutions are conceivable \cite{neto2022safety}. The future applicability of the process reference model is therefore addressed by the safety argumentation flexibility.

For an extended analysis of the iterative data-based V-model in relation to the previously analyzed process reference frameworks from Section \ref{02_New}, major characteristics are compared in Table \ref{tab:compare_frameworks}. Thereby, the advantages and disadvantages of the individual frameworks are illustrated in a compact and abstract manner while demonstrating that the iterative data-based V-model is able to unite the various frameworks except for the safety assessment, which is purposefully detached in accordance with the classical V-model. The individual criteria in Table \ref{tab:compare_frameworks} result from the analysis of the innovative development processes from Section \ref{02_New}, in particular from the discussion as well as the fundamental principles from Section \ref{03_Methodology}.

More specifically, the proposed framework extends the improved V-model specifically towards a continuous integration process that allows to respond to changes in the real world and address the open long-tail distribution challenge over time. Along with this standardization of the different frameworks, there is also a simplification. As discussed above, compared to VVM and Waymo, due to the chosen abstraction of the process, only one perspective is required to address different levels of system's granularity and complexity. Likewise, data-based processes, such as the one of Tesla's data engine, can be considered upfront, whilst a variety of system types, such as traditional or and mixed systems, can be considered in parallel. This is supplemented in Table \ref{tab:compare_frameworks}, which contains a more detailed assessment of the proposed framework in relation to previously investigated process reference frameworks. 

\newcommand\RotText[1]{\rotatebox{90}{\parbox{3.9cm}{\raggedright#1}}}

\begin{table}[]
	\centering
	\caption{Comparison of the previously analyzed process reference frameworks with the established iterative data-based V-model.}
	\begin{tabularx}{\linewidth}{l *{5}{>{\raggedright\arraybackslash}X}}
		\toprule
		& \RotText{Classical V-model} & \RotText{Improved V-model \quad \quad \quad \quad \quad  \tiny(VVM Project)} & \RotText{Safety Determination Lifecycle \tiny(Waymo)} & \RotText{Data Engine \quad \quad \quad \quad \quad \quad \quad \quad } & \RotText{Iterative data-based V-model \tiny(proposed)} \\
		\midrule
		Design phase & \cmark & \cmark & (\cmark) & \cmark & \cmark \\
		V\&V phase & \cmark & \cmark & \cmark & \cmark & \cmark \\
		Application independence & \cmark & \xmark & (\cmark) & (\cmark) & \cmark \\
		\midrule
		Generic across system granularities & \cmark & \xmark & \xmark & (\xmark) & \cmark \\
		Use of specific databases & \xmark & \cmark & (\cmark) & \cmark & \cmark \\
		Use of generic databases & \xmark & \xmark & (\xmark) & \xmark & \cmark \\
		Formalized simulation exploitation & \xmark & \xmark & \xmark & \xmark & \cmark \\
		\midrule
		Iterative product refinement & \xmark & \xmark & \cmark & \cmark & \cmark \\
		Iterative process refinement & \xmark & \xmark & \cmark & (\xmark) & (\cmark) \\
		Continuous trust building & \xmark & \xmark & \cmark & (\xmark) & \cmark \\
		\midrule
		System monitoring & \xmark & \xmark & \cmark & \cmark & \cmark \\
		Safety assessment (e.g. w.r.t. residual risk) & \xmark & \cmark & \cmark & \xmark & \xmark \\
		Safety argumentation customizability & \cmark & \xmark & \xmark & (\xmark) & \cmark \\
		\midrule
		Appropriate for traditional systems & \cmark & \cmark & \cmark & \xmark & \cmark \\
		Appropriate for AI systems & \xmark & (\cmark) & (\cmark) & \cmark & \cmark \\
		Appropriate for complex systems incl. AI & \xmark & (\cmark) & \cmark & (\cmark) & \cmark \\
		\midrule
		Overall generality and transferability & \cmark & (\xmark) & (\cmark) & (\cmark) & \cmark \\
		Overall suitability for emerging AI systems & \xmark & (\cmark) & (\cmark) & (\xmark) & \cmark \\
		\bottomrule
	\end{tabularx}%
	\label{tab:compare_frameworks}
\end{table}%

\subsection{Summary and Discussion of the Proposed Framework}

Consequently, it can be summarized that the iterative data-based V-model
\begin{itemize}
	\item represents a systematic update of the classical V-model,
	\item recognises the data-driven AI development and V\&V,
	\item generalizes across innovative development frameworks,
	\item thus formalizes a unifying process reference model.
\end{itemize}

Therefore, like the classical V-model, the iterative data-based V-model
\begin{itemize}
	\item decouples the process from specific safety assurance,
	\item unites multiple perspectives and approaches into a single,
	\item applies across different levels of system granularities,
	\item thus enables the desired generality and transferability.
\end{itemize}

In addition, the various advantages of the different innovative development processes are taken into account by the iterative data-based V-model, that 
\begin{itemize}
	\item considers system environment/context through the ODD,
	\item maps the open world in a managable (trigger) datasets,
	\item accounts for the prospective and retrospective view,
	\item enables iterative refinement throughout the lifecycle,
	\item thus harmonizes advantages of existing methodologies.
\end{itemize}

Furthermore, the data-based iterative V-model also addresses respective disadvantages of existing process reference models, as it
\begin{itemize}
	\item relaxes assumptions w.r.t. databases or hardware,
	\item formalizes the exploitation of simulation and real world,
	\item applies to traditional, AI-based, or mixed systems,
	\item represents an application-agnostic methodology,
	\item and overcomes limitations of existing frameworks.
\end{itemize}

The iterative data-based V-model harmonizes the respective advantages from Table \ref{tab:compare_frameworks_ad_disad} while counteracting the disadvantages from Table \ref{tab:compare_frameworks_ad_disad}. Thereby, the classical V-model is deliberately extended to the needs of complex systems incorporating AI. 

The proposed framework also has some implications. Particularly, while decoupling the framework from safety assessment and argumentation enables broad applicability, it does not inherently ensure the development of safe systems. Therefore, for safety-critical applications, individual safety assessments and arguments are necessary. However, the framework provides the desired flexibility and adaptability. Morover, among other things, the harmonization of different approaches has a specific implication on the formalization of the process, which is illustrated in Figure \ref{fig:ours}. This formalization explicitly addresses various data sources, from synthetic to real data. In particular, the systematic consideration of synthetic data facilitates the early uncovering of trigger conditions and counteracts the challenge of “unknown unknowns”. The framework thus enables greater efficiency and safety through the systematic consideration of synthetic data. Furthermore, the continuous refinement provided by the framework and thus the iterative nature of error analysis and specification adaptation throughout the entire product lifecycle ensures and increases safety.

Beyond that, the formalization also entails the separation in the depth of the design domain and design implementation as well as in the separation in the breadth of the architecture and the parameterizing data, in order to systematically close gaps in development phase. Overall, a unique feature of the framework is its ability to cover different types of systems, levels of granularity and complexity, and application areas from a unified perspective. The property to adapt to varying system levels and complexities is illustrated abstractly in the following.








\section{Results}
We identified key contexts of parental involvement, perceptions of AI-generated content, preferences for AI-assisted content creation, and collaborative patterns in shared interactions with a robot. We present the findings based on our research questions as follows.

%We identified four key areas that inform the design of an AI-assisted educational robot to support parental involvement in young children's learning activities. We present our results based on the study phases as follows: (1) \textit{Phase 1: Parent Contextual Needs and Scenarios}: understanding the real-life contexts and challenges parents face; (2) \textit{Phase 2.1: Parent Perspectives on AI-Generated Learning Content}: capturing parents' attitudes and concerns regarding the use of AI in generating educational content; (3) \textit{Phase 2.2: Parent Use of the LLM-Assisted Content Supervision Mechanism}: examining how parents review, edit, and supervise AI-generated content; and (4) \textit{Parent Use of the Robot Involvement Adjustment Mechanism}: exploring how parents delegate roles between themselves and the robot during learning activities.

\subsection{(RQ1) What contexts do parents encounter when involving in young children's learning activities?}

Each participant provided examples for the eight scenarios representing unique combinations of the three two-dimensional factors using the \texttt{SET-scenario cards}. We present P2's scenarios as an example in Table~\ref{tab:scneario-example}. The full set of scenario examples from all participants is documented as supplementary materials.\footnote{SET Scenarios: \url{https://osf.io/zfksg/?view_only=b59bd41287f543ce82ab85950aaf004f}} Beyond capturing the contexts shared by each parent, we analyzed these examples to identify and summarize key contextual patterns for the three factors of parental involvement: \textit{skills}, \textit{energy}, and \textit{time}.

\subsubsection{\textbf{Skill:} Parents face challenges in pedagogical skills, particularly with advanced or unfamiliar concepts.}
Parents mentioned several \textit{skills} in supporting their child's (1) intellectual, (2) pedagogical, and (3) social-emotional development, highlighting key challenges across these areas. While some parents (7/20) reported low confidence in \textit{intellectual} activities, especially advanced STEM topics (P1, P7, P12, P14, P17–19), many (16/20) felt confident in literacy (\textit{e.g.,} reading, spelling; P1–6, P8, P9, P11, P13, P14, P19, P20) and basic STEM (P6, P8–10, P12–15). Confidence often stemmed from personal expertise or interests, consistent with the Hoover-Dempsey and Sandler (HDS) framework \cite{green2007parents}. For example, P15, a physicist, felt confident teaching physics-related activities. The majority of parents (14/20) struggled with \textit{pedagogical} skills, such as explaining concepts (P7, P8, P13, P17, P18), answering or formulating questions (P3, P4, P6, P7, P9, P13), identifying developmental benchmarks (P4, P6, P10, P11), and allowing their child to learn from mistakes (P2, P12, P19). A smaller group of parents (7/20) expressed confidence in these areas, particularly explaining concepts (P14, P18, P19) and answering questions (P4, P5, P10, P14). \textit{Social-emotional} skills presented additional challenges. Some parents (6/20) struggled with teaching emotion regulation (P2, P17), behavioral management (P5, P15, P20), and interpersonal conflict resolution (P3, P15). Others (5/20) lacked confidence in encouraging participation in learning activities (P5, P11) or maintaining patience during learning support (P4, P10, P18, P19). Conversely, several parents (10/20) felt confident teaching emotion regulation (P1, P3, P10–13, P16, P18–20) and norms of polite communication (P7, P12, P16, P20).

\subsubsection{\textbf{Energy:} Parents' motivation depends on their physical and emotional status as well as the child's willingness to learn.}
%\paragraph{Parents' motivation depends on their physical and emotional status, time, and the child's willingness to learn.}
Parents suggested that their motivation to facilitate learning activities was affected by (1) physical status, (2) emotional status, and (3) time. Commenting on their \textit{physical status}, most parents (16/20) indicated low motivation when they need rest due to feeling ``\textit{hungry},'' ``\textit{sick},'' or ``\textit{tired}'' (P1--11, P14--17, P19), and many parents (9/20) reported being highly motivated when they are ``\textit{well rested}'' or after having ``\textit{a really good meal}'' (P1--4, P6, P7, P9, P11, P16). Regarding \textit{emotional state}, many parents (9/20) lacked motivation when they needed a mental break or ``\textit{me time}'' if they felt emotionally exhausted (P3, P7, P11--14, P17--19) or after spending time with their child (P5, P8, P20). In addition, some parents (7/20) lost motivation if their child appeared to be disinterested (P11, P12, P14) or poorly behaved (P4, P15, P17, P20). In contrast, many parents (16/20) were motivated when their child needed support (P12, P15), expressed interest and invited the parent to participate (P1, P4, P7, P8, P10, P12–14, P16–19), or is well behaved and ready to learn (P2, P5, P13, P18, P19, P20). Some parents (6/20) were highly motivated when they wanted to connect with their child (P3, P14, P15) or when they were personally interested in the activity (P5, P10, P12, P14).


\subsubsection{\textbf{Time:} Parents' availability depended on work, chores, other family members.}
Parents discussed (1) work and commitment, (2) household chores, and (3) family needs as factors that determined whether they had time, \textit{i.e.,} availability and presence, to facilitate learning activities. Most parents (19/20) were not available when they needed to be at \textit{work} (P1--4, P6, P7, P10--14, P16--19) and had other personal or professional engagements (P4, P5, P9, P11, P15, p20). Many parents (17/20) stated that \textit{household chores}, such as laundry, meal preparation, and cleaning, also determined their availability to be with their child (P1--6, P9--17, P20). Although some parents involved their child in chores (P1, P2, P7, P10, P15, P17), not all chores were seen as being appropriate or safe for children. Parents' availability also depended on the ability of other family members to provide support (P2--4, P6--8, P16--20), \textit{e.g.,} when a spouse helped with chores or an older child watches a younger sibling. Parents had less time if other family members needed them (P3, P7, P8, P12, P13, P16--18, P20), \textit{e.g.,} when a younger child is crying or a family member is sick. Finally, parents described their availability using specific time frames, \textit{e.g.,} ``\textit{weekday mornings} (P5, P8, P19),'' ``\textit{weekdays after dinner and before bedtime} (P5, P10, P18),'' ``\textit{anytime on weekends} (P1, P3, P4, P6, P8, P10, P16),'' or ``\textit{unstructured time} (P2, P11, P14, P16, P17, P20).'' They often structured their time and consider themselves available when they are physically present with their child (P1, P3--7, P14--16), such as during grocery shopping, car rides, or trips to the park together.





\subsection{(RQ2) How do parents perceive AI-generated content for young children?}\label{sec-result-2}

Parents showed mixed attitudes toward AI-generated learning content for young children. They discussed their perceived benefits and risks and envisioned ways to mitigate their concerns.

\subsubsection{Mixed Attitude towards AI-generated content}
Parents expressed a range of attitudes towards allowing AI to generate content for young children, ranging from skepticism and concern (P2--6, P10, P13--15) to open-minded caution (P9, P8, P16, P19, P20), acceptance (P6, P7, P11, P17, P18) and, in some cases, neutral (P1, P12). Parents who were \textit{\textbf{skeptical and concerned}} questioned whether AI-generated content met quality and safety standards, \textit{e.g.,} P3 questioned, ``\textit{Who's generating the content? Where is it getting the content from? Is it good? Is it safe?}'' On the other hand, parents who hold an attitude of \textit{\textbf{open-minded caution}} recognize the risks of using AI-generated content but feel open to use it under specific conditions. P16 highlighted model training, stating, ``\textit{I wouldn't be against it if the people training it were proficient in what the AI is teaching.}'' Similarly, P20 emphasized personal oversight, explaining, ``\textit{I can do my own evaluation to determine whether or not I think the content is good regardless of who it came from.}'' Furthermore, parents who have an attitude of \textit{\textbf{acceptance}} assume people who created the system have already ensure the appropriateness for children, \textit{e.g.,} P7 stated, ``\textit{I'm assuming because it's AI, there would be more research behind it.  So I would be okay with it.}'' Finally, parents who hold a \textit{\textbf{neutral}} attitude typically don't have much experience with AI and therefore feel unsure about their attitude for AI-generated content, \textit{e.g.,} P1 had ``\textit{not even thought about it until before this study.}''

\subsubsection{Perceived Benefits and Risks}
Parents identified several benefits of AI-generated content for young children. Some parents (P2, P4, P8, P10, P11, P16) highlighted AI's potential in \textit{\textbf{adaptability}} to adjust learning content to their child's evolving developmental needs, \textit{e.g.,} P11 expected AI to help ``\textit{adjust content as the child grows.}'' In addition, parents (P2, P3, P4, P16, P19) discussed \textit{\textbf{customization}}, illustrating that ``\textit{one of the big benefits would be to create material that are related to his[child's] interests and things that would be motivating to him[child]} (P4).'' Parents (P6, P7, P8, P10, P12, P17) also emphasized \textit{\textbf{efficiency}} of AI, explaining ``\textit{because it[AI] can access a huge amount of information very fast} (P12),'' enabling a ``\textit{quicker way to learn or to see something} (P7).'' Moreoever, a few parents (P1, P11, P18, P20) noted AI's potential to foster \textit{\textbf{affordability}}, suggesting that AI-generated content could enhance the scalability and accessibility of learning resources, making ``\textit{more learning materials available, more variety available} (P1),'' and making things ``\textit{cheaper and more accessible for people} (P20).'' Finally, a few parents (P14, P15) expected easier \textit{\textbf{pedagogical integration}} with AI, enabling parents to ``\textit{teach children things that sometimes parents don't know because not all parents know everything} (P14).''

Meanwhile, parents described their perceived risks of AI-generated content for children. Most parents (P1--3, P5, P11--15, P17, P19) were concerned about \textit{\textbf{age-inappropriateness}} of the content, which could be ``\textit{violent and don’t match family values} (P1),'' ``\textit{physically harmful and sexually inappropriate} (P3),'' and ``\textit{stuff about body image and certain people being better than other people} (P5).'' In addition many parents (P2, P3, P9, P14, P16, P17, P18) expressed concerns about the \textit{\textbf{inaccuracy}} of the information presented through AI-generated content, worrying that AI could provide ``\textit{factually inaccurate}'' learning materials or content that might imply theories that are ``\textit{misframed or misconstructed} (P2).'' Moreover, parents (P2--4, P14, P15) raised concerns about the \textit{\textbf{training data quality}} for AI models. P2 emphasized transparency stating, ``\textit{I'd want to know a lot more about where that training data came from or who supervised that learning process}.'' A few parents (P6, P7) expressed concerns about children's \textit{\textbf{over dependence}} on AI instead of developing their own cognitive abilities. P6 worried that constant use of AI could discourage critical thinking, stating, ``\textit{if they have a question, instead of thinking through the question, they just ask AI, not using their own brain}.'' Finally, two parents shared concerns over \textit{\textbf{message dilution}}, where AI oversimplifies complex ideas and diminishes their original intent. P15 worried that AI might dilute sociopolitical issues, such as racial diversity and gender identity. Similarly, P20 emphasized concern about whether the core message being conveyed to the child aligns with parental values, stating ``\textit{I'm more concerned about the message the book is trying to impart on the child}.

\subsubsection{Envisioned Risk Mitigation Methods}
Parents described what methods they envisioned to address their concerns. First, some parents (P3, P5, P11, P12, P14, P16) stressed the need to enable \textit{\textbf{parental review and verification}}. For example, P5 stated ``\textit{I would read it to make sure that it was actually something I wanted to read with her}.'' In addition, a few parents (P2, P17, P19) expressed that \textit{\textbf{social and public validation}} could also enhance their trust in AI-generated content, \textit{e.g.,} P2 described that ``\textit{if a thousand people used it...and endorsed this model, that would give me more confidence in it}.'' Moreover, some parents (P2, P9, P15) discussed \textit{\textbf{model and data transparency}}, emphasizing the need to understand how AI models are trained. As explained by P9, ``\textit{being able to know exactly what's going on or how it works...would make me feel more secure about what my child is learning}.'' Lastly, a few parents (P1, P15, P18) highlighted the importance of \textbf{expert involvement} in creating AI-generated content. For instance, P1 emphasized the need for oversight by ``\textit{people with a background in human development.}''

\subsection{(RQ3) How would parents prefer to collaborate with LLM on supervising content creation under different contexts?}

\begin{figure*}[b]
\includegraphics[width=\textwidth]{figures/figure-result-03-hho.pdf}
   \vspace{-6pt}
  \caption{Summary of parent's use of LLM-assisted content supervision mechanism: (1) content evaluation criteria, (2) use pattern, (3) perceived value.}
  \label{fig:result-03}
   \vspace{-6pt}
\end{figure*}

We found three main themes for parent-AI collaboration on content creation using the \textit{editor interface}: (1) \textit{Content evaluation and criteria}, referring to what parents pay attention to when reviewing and revising LLM-generated content. (2) \textit{Contextual usage patterns}, describing how parents envision using the LLM-powered interface in various contexts. (3) \textit{Perceived value and benefits}, covering what values parents believe LLM brings.

\subsubsection{Theme 1: Content evaluation and criteria}
Parents focus on balancing \textit{difficulty} and \textit{variety} of concepts as well as ensuring the \textit{quality} of questions when reviewing, regenerating, and revising LLM-generated content for young children.

\textit{\textbf{Parents aim to give their children the right level of challenge while reinforcing skills they can confidently accomplish.}} Many (8/20) avoided overly easy questions to prevent boredom but strategically included them at the start or after difficult questions to build confidence. As P2 explained, ``\textit{I want her to get the answers and then have it get increasingly difficult as she goes so she doesn't get discouraged at the beginning}.'' Meanwhile, most parents (11/20) valued challenges that stretch their child's abilities without overwhelming them. P12 concerned that ``\textit{underestimating her would be damaging for her},'' while P20 expressed interest in seeing how his child would handle harder concepts, saying, ``\textit{I'm actually really interested to see if she can answer.}'' Finally, parents (7/20) were also cautious of content that might be too advanced, \textit{e.g.,}``\textit{she[child] doesn't know uppercase or lowercase yet, so that doesn't mean anything to her} (P20).'' Additionally, \textit{\textbf{parents aim to maintain engagement by introducing diverse concepts and question types throughout the activity.}} Many parents (11/20) expressed concerns over repetitive content and preferred diverse topics to challenge their child differently. For instance, P20 changed the concept of a question to ``addition'', explaining, ``\textit{I just made the last question a `how many,' so this one I want a different concept}.'' Finally, \textit{\textbf{parents evaluate the quality of LLM-generated learning content based on standards} such as question clarity and coherence (9/20), wording precision (6/20), visual clarity (5/20), and cognitive load (P12, P20).} P12 raised issues with wording, stating, ``\textit{I don't think she's going to fully understand front legs versus back legs when it's a front view},'' while P6 expressed concerns about visual clarity: ``\textit{from the pictures, you can't really tell how many bugs with black bodies are flying in the air}.'' P20 also reflected on cognitive load, saying, ``\textit{I think it's just too long, too much information for her to process}.'' 

\subsubsection{Theme 2: Contextual usage patterns}
We discussed parents' preferences and behaviors when collaborating with LLM under two main contexts: (1) when parents have limited time or energy and (2) when they have sufficient time and energy.

\textit{When parents have limited time or energy}, most were still \textit{willing} to invest minimal effort (P4, P6--9, P11, P12, P15, P18), often opting to \textit{\textbf{skim through the LLM-generated content with minor self-editing}}. For example, P9 shared, ``\textit{I might skip quickly, skim through it, make sure there isn't anything that I feel is not appropriate}.'' This approach allows involvement with minimal time commitment. However, some parents (P4, P11, P15) emphasized that the \textit{\textbf{LLM output must be high-quality enough to require minimal editing}}, otherwise they may not use it at all. P11 explained, ``\textit{The more that stuff can be in really good shape before it gets to parents, the more we can minimize how much work we have to do ahead of time}.''

Some parents were \textit{unwilling} to invest effort when time or energy was limited. They preferred to either \textit{\textbf{reuse previously reviewed activities}} (P8, P11, P12) or directly \textit{\textbf{use LLM-generated content without review}} (P5, P6, P9, P10, P18, P20). As P8 explained, ``\textit{if I don't have time, I would have to be using something he's already done before, so I don't have to supervise it},'' while P10 noted, ``\textit{if the AI-generated questions were enough to keep him engaged, then it would be worth it}.'' A few parents (P1, P7) preferred to \textit{\textbf{avoid using the system entirely}}, as they feel uncomfortable leaving their child engaged with the content without supervision. As P1 explained, ``\textit{if I'm either physically or mentally not present. It's just not happening}.''

\textit{When parents have sufficient time and energy}, most of them (P6–P10, P12, P18, P20) choose to \textit{\textbf{review and edit the content in detail, even customizing questions}} to better supervise and personalize learning for their child. P9 shared, ``\textit{If I had more time and motivation, I would take the time to do it myself. I enjoy writing, so I'd probably spend time customizing the content}.'' Similarly, P12 noted, ``\textit{If I had all the time, I would go through and be picky with the wording and content of the questions}.''

In contrast, some parents (P1, P2, P7, P11, P12, P18) still prefer to \textit{\textbf{skim through the content with minor editing}}, as they found the detailed process too effortful even when time allowed, but they cannot fully trust LLM or themselves to come up with good questions. For example, P11 shared, ``\textit{I would probably scroll through and try to do as little editing as possible},'' while P7 expressed doubt, stating, ``\textit{I don't know that I would come up with better questions than this one from AI}.'' A few parents opted to \textit{\textbf{avoid using the system entirely}}, preferring to spend their time on other activities (P15) or relying on their ability to engage their child without the system (P4, P5). For instance, P15 shared, ``\textit{I would rather spend that time playing an imaginative game with her than spending time designing this,}'' P5 similarly expressed confidence, saying, ``\textit{I think I can and do ask her questions about stuff we read}.''

\subsubsection{Theme 3: Perceived value and benefits}
We found that parents perceive the value of the system to include not only \textit{content supervision}, but also \textit{content co-creation with LLM} and \textit{parent empowerment through pedagogical insights}.

First, and unsurprisingly, most parents suggested that the system allows them to \textit{\textbf{supervise the learning content generated by LLM}}. For example, P2, while feeling skeptical about trusting AI, noted, ``\textit{I don't know what AI model was used, still, I can confirm everything myself},'' reflecting the value parents place on maintaining oversight of the content presented to their children. Second, some parents (P6–10, P12, P18, P20) appreciated that the system allows them to \textit{\textbf{co-create personalized learning content with the LLM}} for their child without having to start from scratch. For example, P18 appreciated the ability to adapt the content to their child's needs, saying, ``\textit{tailoring it to her difficulty levels and seeing the ability to modify the content alleviates some concerns}.'' P10 highlighted how the LLM creates a draft to work from, stating, ``\textit{I do appreciate the concepts and the kinds of questions that it [LLM] provides, and how it has that template there}.'' This flexibility allowed parents to easily modify content while leveraging the assistance from LLM. Third, some parents found that the system \textit{\textbf{empowered parents with pedagogical insights}}. As many parents do not possess formal pedagogical knowledge--such as understanding how to effectively teach their child--they often struggle with determining what questions to ask or which concepts are age-appropriate. Since parents brought up the same value after interacting with the robot as well, we discuss this value more in-depth in Section \ref{sec-6.4.2}.

\subsection{(RQ4) How would parents prefer to collaborate with an AI-assisted robot to engage in learning activities with their children under different contexts?}

\begin{figure*}[b!]
\includegraphics[width=\textwidth]{figures/figure-result-04-hho.pdf}
   \vspace{-6pt}
  \caption{Summary of parent's use of robot involvement adjustment mechanisms: (1) usage pattern, (2) parenting education.}
  \label{fig:result-03}
   \vspace{-6pt}
\end{figure*}

We identified two major themes in the use of parent-robot collaboration mechanisms (\textit{i.e.,} \textit{mode-switching} and \textit{role-delegation}) within the \textit{activity interface}: (1) \textit{Contextual mode utilization}, referring to how parents adjust their involvement based on varying time and energy levels, and (2) \textit{Perceived educational impact on parenting}, highlighting how parents value the process for enhancing their skills and knowledge in parenting.

\subsubsection{Theme 1: Contextual mode utilization}

We discussed parents' preferences when collaborating with the AI-assisted robot across four contexts: (1) sufficient energy and time, (2) sufficient time but low energy, (3) sufficient energy but limited time, and (4) low energy and time. The impact of parental skill is discussed in specific cases.

(1) \textit{Parents have sufficient energy and time}: many parents (P2, P5–7, P9, P15, P18) preferred the \textit{\textbf{parent takeover mode}}, where they facilitate activities themselves while using LLM-generated content as a resource. For example, P18 shared, ``\textit{if I'm feeling motivated, I'd probably take over, but still look at some AI-generated questions to prompt me or remind me of things to ask or do with her}.'' Similarly, P15 noted, ``\textit{with full energy and time, I would use the parent-only mode because I want to interact with her and give her all my attention}.'' Parents valued the ability to take full control while using LLM-generated content for supplemental support when they have sufficient energy and time.

In addition, some parents (P2, P8–12, P15, P20) envisioned using \textit{\textbf{collaboration modes}}--where both the parent and the robot share responsibilities (\textit{i.e.,} parent-led or robot-led mode)--with the parents' \textit{skills} in specific areas relative to the robot playing a critical role in determining the pattern of role delegation. Parents often chose to involve the robot when they felt it could enhance their child's engagement especially in high-stakes tasks like quizzes. For example, P15 noted, ``\textit{I would use the robot for quizzes as a playful element to keep her engaged}.'' Similarly, P2 highlighted the objectivity of the robot in quizzing: ``\textit{I like the idea of reading her the book and then a neutral third party gets to test her on it}.'' On the other hand, parents took on specific roles when they believed their involvement would benefit their child more. P11 shared, ``\textit{I would let the robot read and ask questions but step in if he wasn't understanding or needed guidance},'' while P12 emphasized the emotional aspect of teaching: ``\textit{I can explain in a way that she understands, whereas the robot might come across as too harsh}.''

(2) \textit{Parents have sufficient time but lack energy}: some parents (P3, P7, P10--12, P15, P18, P20) opted for \textit{\textbf{collaboration modes}}, with their involvement influenced by their motivation levels and partially by their \textit{skill} relative to the robot. For example, P10 noted, ``\textit{when I'm not motivated, having the robot do the quiz takes some heat off me}.'' Similarly, P9 mentioned, ``\textit{I'd probably read the book, but have the robot do everything else}.'' Additionally, some parents (P1, P2, P4, P6–8, P15) chose to use \textit{\textbf{robot takeover mode}}--where the robot facilitates everything--while they remained nearby to supervise. For instance, P15 noted, ``\textit{I'd be around, but I wouldn't physically do much because I'm not feeling well}.'' Similarly, P2 noted, ``\textit{If I'm not motivated, I could see myself handing it all over to the robot}.''

(3) \textit{Parents have sufficient energy but lack time}: many parents (P2, P3, P5, P7--11, P15) opted for the \textit{\textbf{robot takeover mode}}--where the robot facilitates everything--while adjusting their usage based on \textit{how much they trust LLM}. Parents with higher trust allowed their child to use the content directly without review (P3, P5, P7, P9, P10). For example, P7 mentioned, ``\textit{If I'm trying to take a walk, I might do the robot takeover, then I can physically be gone}.'' In contrast, parents with less trust preferred to supervise while multitasking (P2, P7, P8), review content beforehand using the editor (P2, P8), or use the system only if the LLM model met high-quality standards (P11, P15). For example, P2 shared, ``\textit{If I'm not there, I wouldn't want them to do it, unless I had used the editor to review},'' while P7 described a multitasking scenario: ``\textit{I could be working from home while the robot takes over, and I'm nearby to supervise}.'' Moreover, a few parents (P2, P12) chose to \textit{\textbf{avoid using the system entirely}} due to their lack of trust in using LLM-generated content directly and insufficient time to review it, or because the system design did not support independent use for young children (P4, P12, P20). For instance, P2 noted, ``\textit{If I'm absent, I don't know if I'd want them to do any of this},'' while P12 stated, ``\textit{I know my daughter is sensitive, and if [the questions are too hard and] the robot keeps telling her she's wrong, she might take it personally and give up}.''

(4) \textit{Parents lack both energy and time}: Some parents chose \textit{\textbf{robot takeover mode}}, adjusting their usage based on their trust in LLM-generated content. Others \textit{\textbf{avoided using the system entirely}} due to low trust and insufficient time to review (P2, P12), or because the system design did not support independent use by young children (P4, P12, P20). Refer to the previous case—parents with sufficient energy but lacking time—as the usage patterns and contextual reasons are very similar.

\begin{figure*}[!t]
  \includegraphics[width=\textwidth]{figures/figure-quant-hho.pdf}
  \caption{Parent perception on child's math and literacy ability before and after the reading session. The result suggested that parents adjusted their perception after observing their child doing the activity and they tend to underestimate them, especially for advanced math concepts and phonological awareness concept for literacy. The horizontal lines represents significance from the Wilcoxon Signed-Ranked Test: $p < .01^{**}$, $p < .05^{*}$.}
  \label{fig:quant-result}
   \vspace*{-10pt}
\end{figure*}

\subsubsection{Theme 2: Perceived Educational Impact on Parenting} \label{sec-6.4.2}

Supported by mixed-method data, many parents thought \texttt{PAiREd} has value in parenting education, providing them with pedagogical strategies and giving them opportunities to observe their child's proficiency level systematically through observation. If the system provides a comprehensive framework and ample ideas, parents may not decrease their involvement in an activity just because they don't have the pedagogical skill; in fact, they may even increase their involvement. In addition, parents will be able to observe and adjust their understanding about what their child can do or cannot do, instead of under- or over- estimate their child's ability.

Several parents (P1, P4, P6, P7, P10--12, P18) appreciated that the system \textit{\textbf{offered ideas they might not have considered on their own}}, providing new topics to explore with their child. For instance, P1 emphasized, ``\textit{I hadn't even thought of all the different types of concepts},'' and P10 highlighted that the system ``\textit{gives more of a structure…even the drop down list of concepts is insightful, offering lenses I wouldn't normally consider when reading}.'' Others (P2, P6--9, P11, P12) valued that the LLM \textit{\textbf{generated example questions for each concept}}, allowing them to start with ready-made content without worrying whether their own questions reflected the intended learning goals. For example, P8 mentioned, ``\textit{What's nice about the AI-generated ones is that you can specifically choose a variety of concepts, whereas creating them on your own, you don't always know what the concepts are}.'' Additionally, some parents noted that the system allowed them to systematically select questions they were unsure their child could answer, which \textit{\textbf{provided a structured way to observe and assess their child's proficiency level}}. For example, P7 remarked, ``\textit{it gives her the opportunity to show me things she knows that I otherwise wouldn't have asked},'' while P8 shared, ``\textit{I was curious to see how he does if he doesn't know how to answer this, rather than just setting him up to succeed}.''

Before and after parent-child pairs engage in the activity, we asked parents to rate their perception on their child's math and literacy abilities. Our quantitative results suggest that parents adjusted their understanding of their child's proficiency in certain concepts after reading together. Specifically, \textit{\textbf{parents tended to underestimate what their child can or cannot do, especially with more advanced math concepts}} (Math-L3: $p < .01^{**}$, Math-L4: $p < .01^{**}$) and the phonological awareness concept in literacy ($p < .05^{*}$). Figure \ref{fig:quant-result} summarizes the significance results from the Wilcoxon Signed-Rank Test.

%A One-Way ANOVA revealed significant differences between levels (L1-L4), necessitating separate analyses. Subsequent Repeated Measures ANOVAs for each level showed significant improvement in Math L4 post-intervention. Post-hoc pairwise tests, using both parametric (Paired T-Tests) and non-parametric (Wilcoxon Signed-Rank) methods to ensure robustness, confirmed significance in Literacy L2, Math L3, and Math L4. This multi-layered approach—combining One-Way ANOVA, Repeated Measures ANOVA, and diverse post-hoc tests—ensured level independence, accounted for within-subject variability, and provided a comprehensive, robust understanding of level-specific improvements while minimising misinterpretation risks.






\section{Conclusion}

This paper introduces a variant of the multivariate time series traffic prediction problem with a focus on highly sparse and unstructured observations.
To address this problem we propose SUSTeR, a framework which handles sparse unstructured observations by creating hidden graphs in a residual fashion, which are then used with a conventional spatio-temporal GNN.
SUSTeR achieves better predictions for high sparsity (80\% - 99.9\% missing data) than existing baselines and remains competitive in denser settings or even when using only half the amount of the training data.
In addition, its training is considerably faster than the next-best competitor due to a smaller model size.

% We conduct experiments on a unstructured and sparse version of the traffic dataset Metr-LA and compare the performance of SUSTeR with traffic prediction baselines.
% The consideration of the sparsity within SUSTeR outperforms other approaches at sparsity rates $\geq$99\%.
% Experiments were performed up to a sparsity with only 2.4 observations within a sample where without missing data such a sample contains 12$\times$207 values.
% Further, the ablation studies explore the influence of our design choices and show the robustness of our framework.


\section{Future Work}

We plan to explore the interpretability within SUSTeR to obtain an intuitive understanding of the graph nodes within the hidden graph.
Small design choices are made within SUSTeR to make this possible, from observations that are not relying on each other in the same timestep, variable amounts of observations, a learnable assignment function from the observation to the hidden node, and an explicit learned laplacian matrix. 
The problem of sparse unstructured observations, which should be reconstructed into a hidden state, is present in many other domains.
In particular ocean data is a very promising application field for SUSTeR where sparse ARGO\footnote{https://argo.ucsd.edu} observations would perfectly match the problem definition to predict ocean states. 
There, observations are typically spatially and temporally sparse - comparable to the highest dropout rate in this paper - and observations are non-stationary and change their position freely.
We see SUSTeR as a bridge of the well-studied spatio-temporal mining methods into a new area of domains, in which such methods previously were not applicable.

\section{Conclusion}\label{sec:conclusion}
%This work explores the impact of grid-connected and wireless measurement setups on capacitive human body communication, revealing significant differences in both channel \revise{gain} and frequency behavior. 
While conventional data acquisition setups are effective for quantifying the forward path loss, which depends on the conductive properties of the human body, they substantially alter the return path behavior by artificially modifying the capacitive coupling to earth ground.
Therefore, a wireless, wearable-sized data acquisition system is essential for quantitatively evaluating the full \ac{HBC} communication channel in a realistic environment with minimal measurement interference. 
To address this challenge, this work introduces \textit{BodySense}, an evaluation platform for human body communication that is fully wireless, compact enough for wearable applications, and designed for extendability.
To validate the proposed system, the measured channel gains of a classical, grid-connected setup and a wireless setup have been determined for distances of \qty{10}{\centi\meter}, \qty{30}{\centi\meter}, and \qty{50}{\centi\meter} between transmitter and receiver for a frequency range between \qty{4}{\mega\hertz} and \qty{64}{\mega\hertz}.
A comparison between the two scenarios yields an average overestimation of \qty{18.15}{\db} over all investigated distances for the classical case, highlighting the importance of evaluating capacitive \ac{HBC} in realistic conditions.
When comparing the energy consumption of capacitive \ac{HBC} with \ac{BLE}, we achieved results comparable to state-of-the-art \ac{BLE} frontends. 
This demonstrates its potential as a promising alternative to conventional \ac{RF} links, offering opportunities to further enhance the overall energy efficiency of wearable devices and move closer to the realization of battery-free, body-worn sensor nodes.



%This paper proposes \textit{Bodysense}, a fully wireless, wearable-sized system designed to accurately evaluate capacitive human body communication. Experimental evaluation has revealed significant differences in both channel loss and frequency behavior. This paper demonstrated that while conventional data acquisition setups are effective for quantifying the forward path loss, which depends on the conductive properties of the human body, they substantially alter the return path behavior by artificially modifying the capacitive coupling to earth ground. Thus, the proposed wearable-sized data acquisition system is essential for quantitatively evaluating the full \ac{HBC} communication channel in a realistic environment with minimal measurement interference. 
%To address this issue, this paper presents \textit{Bodysense}, a fully wireless, wearable-sized, and extendable evaluation platform for human body communication.
%To validate the proposed system, the measured channel gains of a classical, grid-connected setup and a wireless setup have been determined for distances of \qty{10}{\centi\meter}, \qty{30}{\centi\meter}, and \qty{50}{\centi\meter} between transmitter and receiver for a frequency range between \qty{4}{\mega\hertz} and \qty{64}{\mega\hertz}.
%A comparison between the two scenarios yields an average overestimation of \qty{18.15}{\db} over all investigated distances for the classical case, highlighting the importance of evaluating capacitive \ac{HBC} with a measurement setup that is similar or ideally identical to the envisaged use case.


\begin{acks}
This work was supported by the Alfred P. Sloan Foundation (G-2024-22427) and the University of Notre Dame's Lucy Family Institute for Data \& Society. We also thank the anonymous reviewers for their feedback and suggestions.
\end{acks}

%%
%% The next two lines define the bibliography style to be used, and
%% the bibliography file.
\bibliographystyle{ACM-Reference-Format}
\bibliography{main}

\appendix
\onecolumn

\part{Appendix} 

\newcommand{\appendixnumberline}[1]{Appendix\space}

\renewcommand{\appendixname}{Appendix}
\renewcommand{\thesection}{\appendixname~\Alph{section}}
\renewcommand{\thesubsection}{\Alph{section}.\arabic{subsection}}

\section{Proofs}
\label{appendix_sec:proofs}
This section contains all omitted proofs in the paper.

\subsection{Proof of Lemma~\ref{lemma:equivalence_between_perspective_relaxation_and_convexification}}

\begin{namedlemma}
    [~\ref{lemma:equivalence_between_perspective_relaxation_and_convexification}]
    The closed convex hull of the set
    \begin{align*}
        \textstyle \left\{ (\tau, \bbeta, \bz) \middle|
        \| \bbeta \|_\infty \leq M, \, \bz \in \{0, 1\}^p, \, \mathbf{1}^\top \bz \leq k, \, \beta_j ( 1 - z_j) = 0 ~~ \forall j \in [p], \, \sum_{j \in [p]} \beta_j^2 \leq \tau \right\}
    \end{align*}
    is given by the set
    \begin{align*}
        \textstyle \left\{ (\tau, \bbeta, \bz)  \;\middle|\; -M z_j\leq \bbeta_j \leq M z_j ~ \forall j \in [p], \, \bz \in [0, 1]^p, \, \mathbf{1}^\top \bz \leq k, \, \sum_{j \in [p]} \beta_j^2 / z_j \leq \tau \right\}.
    \end{align*}
\end{namedlemma}

\begin{proof}
    Let $\mathcal T$ represent the first set mentioned in the statement of the lemma. Using the definition of the perspective function and applying the big-M formulation technique, we have
    \begin{align*}
        \textstyle \mathcal T = \left\{ (\tau, \bbeta, \bz)  \;\middle|\; -M z_j\leq \bbeta_j \leq M z_j ~ \forall j \in [p], \, \bz \in \{0, 1\}^p, \, \mathbf{1}^\top \bz \leq k, \, \sum_{j \in [p]} \beta_j^2 / z_j \leq \tau \right\}.
    \end{align*}
    As the epigraph of a perspective function constitutes a cone \citep[Lemma~1 \& 2]{shafiee2024constrained}, we may write $\mathcal T = \mathrm{Proj}_{(\tau, \bbeta, \bz)}(\overline {\mathcal T})$, where 
    \begin{align*}
        \textstyle \overline {\mathcal T} = \left\{ (\tau, \bbeta, \bt, \bz) \;\middle|\; \bm 1^\top \bt = \tau, \, \bz \in \{0, 1\}^p, \, \mathbf{1}^\top \bz \leq k, \, \bm A_j \begin{bmatrix} t_j \\ \beta_j \end{bmatrix} + \bm B_j z_j \in \mathbb K_j ~ \forall j \in [p] \right\}
    \end{align*}
    admits a mixed-binary conic representation with
    \begin{align*}
        \bm A = \begin{bmatrix} 1 & 0 \\ 0 & 1 \\ 0 & 0 \\ 0 & 1 \\ 0 & -1 \end{bmatrix}, \,
        \bm B = \begin{bmatrix} 0 \\ 0 \\ 0 \\ M \\ M \end{bmatrix}, \,
        \mathbb K_j = \mathbb L_+ \times \R_+ \times \R_+ \qquad \forall j \in [p].
    \end{align*}
    Here, $\mathbb L_+ \in \R^3$ denotes the rotated second order cone, that is, $\mathbb L_+ = \{ (t, \beta, z) \in \R_+ \times \R \times \R_+: \beta^2 \leq t z  \}$.
    Thus, using \citep[Lemma~4]{shafiee2024constrained}, the set $\overline{\mathcal T}$ satisfies all the requirements of \citep[Theorem~1]{shafiee2024constrained}, and therefore, its continuous relaxation gives the closed convex hull of $\overline{\mathcal T}$, that is,
    \begin{align*}
        \textstyle \cl \conv(\overline {\mathcal T}) = \left\{ (\tau, \bbeta, \bt, \bz) \;\middle|\; \bm 1^\top \bt = \tau, \, \bz \in [0, 1]^p, \, \mathbf{1}^\top \bz \leq k, \, \bm A_j \begin{bmatrix} t_j \\ \beta_j \end{bmatrix} + \bm B_j z_j \in \mathbb K_j ~ \forall j \in [p] \right\}.
    \end{align*}
    The prove concludes by applying Fourier-Motzkin elimination method to project out the variable $\bt$.
\end{proof} 

\begin{namedlemma}
    [~\ref{lemma:fenchel_conjugate_of_g_closed_form_expression}]
    The conjugate of $g$ is given by
    \begin{equation*}
        g^*(\balpha) = \TopSum_k({\bf H}_M(\balpha)).
    \end{equation*}
\end{namedlemma}

\begin{proof}
    Using the definition of the implicit function $g$ in~\eqref{eq:function_g_definition}, we have
    \begin{align}
        \label{eq:max:g*}
        g^*(\balpha) = \left\{
        \begin{array}{cl}
            \max & \balpha^\top \bbeta -  \frac{1}{2} \sum_{j \in [p]} {\beta_j^2}/{z_j} \\[1ex]
            \st & \bbeta \in \R^p, \, \bz \in [0, 1]^p, \, \bm 1^\top \bz \leq k, \\[1ex]
            & -M z_j \leq \beta_j \leq M z_j ~ \forall j \in [p]
        \end{array}
        \right.
    \end{align}
    For any fixed feasible $\bz$, the maximization problem over $\bbeta$ is a simple constrained quadratic problem, that can be solved analytically by the vector $\beta^\star$ whose $j$'th element is given by
    $\beta_j^\star = \sgn(\alpha_j) \min(\vert{\alpha_j}, M) z_j.$
    Substituting the optimizer $\beta^\star$, the objective function of the maximization problem in~\eqref{eq:max:g*} simplifies to
    \begin{align*}
        \balpha^\top \bbeta^\star - \frac{1}{2} \sum_{j \in [p]} {\beta_j^\star}^2 / z_j 
        &= \sum_{j \in [p]} \alpha_j \cdot \sgn(\alpha_j) \min(\vert{\alpha_j}, M) z_j - \frac{\left( \sgn\left( \alpha_j \right) \min\left(\vert{\alpha_j}, M \right) z_j \right)^2}{2z_j} \\
        &= \sum_{j \in [p]} ( \vert{\alpha_j} \min(\vert{\alpha_j}, M) - \frac{1}{2} \min(\alpha_j^2, M^2) ) z_j %\\
        % &= \begin{cases} \frac{1}{2} \alpha_j^2 z_j & \text{if } \vert{\alpha_j} \leq M  \\ \left( M \vert{\alpha_j} - \frac{1}{2} M^2 \right) z_j & \text{if } \vert{\alpha_j} > M
        % \end{cases} \\
        = H_M(\alpha_j) z_j,
    \end{align*}
    where the second equality holds as $\bz$ is a binary vector, and the last equality follows from the definition of the Huber loss function. We thus arrive at
    \begin{align*}
        g^*(\balpha) = \max_{\bz \in [0,1]^p} \left\{ \textstyle \sum_{j \in [p]} H_M (\alpha_j) z_j: \bm 1^\top \bz \leq k \right\} = \TopSum_k ({\mathbf{H}}_M(\balpha)).
    \end{align*}
    This completes the proof.
\end{proof}

\subsection{Proof of Lemma~\ref{lemma:equivalence_between_proximal_operator_and_huber_isotonic_regression}}

\begin{namedlemma}
    [~\ref{lemma:equivalence_between_proximal_operator_and_huber_isotonic_regression}]
    For any $\bmu \in \R^p$, we have 
    $$\prox_{\rho g^*}(\bmu) = \sgn(\bmu) \odot \bnu^\star, $$ 
    where $\odot$ denotes the Hadamard (element-wise) product, $\bnu^\star$ is the unique solution of the following optimization problem
    \begin{align}
        \label{A:obj:KyFan_Huber_isotonic_regression}
        \begin{array}{cl}
            \min\limits_{\bnu \in \R^p} & \frac{1}{2} \sum_{j \in [p]} (\nu_j - \vert{\mu_j})^2 + \rho \sum_{j \in \calJ} H_M (\nu_j) \\[2ex]
            \st & \quad \nu_j \geq \nu_l \; \text{ if } \; \vert{\mu_j} \geq \vert{\mu_l} ~~ \forall j, l \in [p],
        \end{array} 
    \end{align}
    and $\calJ$ is the set of indices of the top $k$ largest elements of~$ \vert{\mu_j}, j \in [p]$. 
\end{namedlemma}

\begin{proof}
    For simplicity, let $\balpha^\star = \prox_{\rho g^*}(\bmu)$, that is,
    \begin{align}
        \label{eq:alpha:star}
        \balpha^\star = \argmin_{\bm \alpha \in \R^p} ~ \frac{1}{2} \Vert{\bm \alpha - \bm \mu}_2^2 + \rho g^*(\bm \alpha).
    \end{align}
    We first show that $\sgn(\balpha^\star) = \sgn(\bmu)$ (step 1) and then establish that for every $j, l \in [p]$ with $\vert{\mu_j} \geq \vert{\mu_l}$, we have $\vert{\alpha_j^\star} \geq \vert{\alpha_l^\star}$ (step 2). We then conclude the proof using these observations.

    \begin{itemize}[label=$\diamond$,leftmargin=*]
        \item \textbf{Step 1.} We prove the sign-preserving property through a proof by contradiction. For the sake of contradiction, suppose that there exists some $j \in [p]$ such that $\sgn(\alpha_j^\star) \neq \sgn(\mu_j)$.
        Hence, we can construct a new $\balpha'$ by flipping the sign of $\alpha_j^\star$, i.e., $\alpha_j' = -\alpha_j^\star$, and keeping the rest of the elements the same as $\balpha^\star$.
        Now under the assumption that $\sgn(\alpha_j^\star) \neq \sgn(\mu_j)$, we have $\left\lvert{\alpha_j^\star - \mu_j}\right\rvert > \left\lvert{\lvert{\alpha_j^\star}\rvert - \lvert{\mu_j}\rvert}\right\rvert = \left\lvert{\alpha_j' - \mu_j}\right\rvert$, so the $j$-th term in the first summation of the objective function will decrease while everything else remains the same.
        This leads to a smaller objective value for $\balpha'$ than $\balpha^\star$, which contradicts the optimality of $\balpha^\star$.
        Thus, the claim follows.
        
        \item \textbf{Step 2.} We prove the relative magnitude-preserving property through a proof by contradiction. For the sake of contradiction, suppose that there exists some $j, l \in [p]$ such that $\vert{\mu_j} \geq \vert{\mu_l}$ but $\vert{\alpha_j^\star} < \vert{\alpha_l^\star}$.
        Then, we can construct a new $\balpha'$ by swapping $\alpha_j^\star$ and $\alpha_l^\star$, i.e., $\alpha_j' = \alpha_l^\star$ and $\alpha_l' = \alpha_j^\star$, and keeping the rest of the elements the same as $\balpha^\star$.
        Under the assumption that $\vert{\mu_j} \geq \vert{\mu_l}$ but $\vert{\alpha_j^\star} < \vert{\alpha_l^\star}$, we have $\left\lvert{\alpha_j^\star - \mu_j}\right\rvert + \left\lvert{\alpha_l^\star - \mu_l}\right\rvert > \left\lvert{\alpha_l^\star - \mu_j}\right\rvert + \left\lvert{\alpha_j^\star - \mu_l}\right\rvert =
        \left\lvert{\alpha_j' - \mu_j}\right\rvert + \left\lvert{\alpha_l' - \mu_l}\right\rvert$, so the sum of the $j$-th and $l$-th terms in the first summation of the objective function will decrease while everything else remains the same.
        This leads to a smaller objective value for $\balpha'$ than $\balpha^\star$, which contradicts the optimality of $\balpha^\star$. Thus, the claim~follows.
    \end{itemize}
    Using these two observations, we are ready to prove that $\balpha^\star = \sgn(\bmu) \odot \bnu^\star$.
    We first reparametrize the minimization problem~\eqref{eq:alpha:star} by substituting the decision variable $\balpha$ with a new variable $\bnu \in \R_+^p$ satisfying $\balpha = \sgn(\bmu) \odot \bnu$. By the sign-preserving property in step 1, it is easy to show the equivalence between the optimization problem in~\eqref{eq:alpha:star} and the following optimization problem
    \begin{align*}
        \min_{\bnu \in \R^p_+} ~ \textstyle \frac{1}{2} \sum_{j \in [p]} (\nu_j - \vert{\mu_j})^2 + \rho \TopSum_k \left( \mathbf{H}_M ( \bnu ) \right).
    \end{align*}
    By the relative magnitude-preserving property in step 2, we can further set the equivalence between the minimization problem in~\eqref{eq:alpha:star} and the following optimization problem
    \begin{align*}
        \begin{array}{cl}
            \displaystyle \min_{\bnu \in \R_+^p} & \frac{1}{2} \sum_{j \in [p]} (\nu_j - \vert{\mu_j})^2 + \rho \sum_{j \in \calJ} H_M (\nu_j), \\ 
            \st & \quad \nu_j \geq \nu_l \; \text{ if } \; \vert{\mu_j} \geq \vert{\mu_l}.
        \end{array} 
    \end{align*}
    Lastly, the nonnegative constraint on $\bnu$ can be removed as the second summation term in the objective function implies that $\nu_j \geq 0$. Thus, we have shown that any feasible point $\balpha$ in the minimization problem~\eqref{eq:alpha:star} can be reconstructed by any feasible point $\bnu$ in the minimization problem in the statement of lemma, while maintaining the same objective value. Hence, we may conclude that $\balpha^\star = \sgn(\bmu) \odot \bnu^\star$, as required.
\end{proof}

\subsection{Proof of Lemma~\ref{lemma:PAVA_algorithm_exact_solution}}






%Assuming that the input vector $\bmu$ has already been sorted so that the elements are in nonincreasing order in terms of their absolute values, the algorithm runs in linear time complexity, $O(p)$, where $p$ is the number of elements in the input vector $\bmu$.

\begin{namedlemma}
    [~\ref{lemma:PAVA_algorithm_exact_solution}]
    The vector $\hat \bnu$ in Algorithm~\ref{alg:PAVA_algorithm} solves~\eqref{obj:KyFan_Huber_isotonic_regression} exactly.
\end{namedlemma}

\begin{proof}
    The minimization problem~\eqref{obj:KyFan_Huber_isotonic_regression} is an instance of a generalized isotonic regression problem taking the form
    \begin{align}
        \label{obj:KyFan_Huber_isotonic_regression_rewritten_as_generalized_isotonic_regression}
        \min_{\bnu} \sum_{j=1}^{p} h_j(\nu_j) \quad \st \quad \nu_1 \geq \nu_2 \geq \cdots \geq \nu_J,
    \end{align}
    where $h_j(\nu) = \frac{1}{2} (\nu - \mu_j)^2 + \rho_j H_M(\nu)$, $\rho_j = \rho$ if $j \in \calJ$ and $\rho_j = 0$ otherwise, and the set $\calJ$ is the set of indices of top k largest elements of $\vert{\mu_j}$, as defined in the statement of Lemma~\ref{lemma:equivalence_between_proximal_operator_and_huber_isotonic_regression}.
    Thanks to~\cite{best2000minimizing,ahuja2001fast}, the optimizer of~\eqref{obj:KyFan_Huber_isotonic_regression_rewritten_as_generalized_isotonic_regression} satisfies two key properties: 
    \begin{itemize}[label=$\diamond$,leftmargin=*]
        \item \textbf{Property 1: Optimal solution for a merged block is single-valued.} 
        Suppose we have two adjacent blocks $[a_1, a_2]$ and $[a_2+1, a_3]$ such that the optimal solution of each block is single-valued, that is, the minimization problems
        \begin{align*}
            \left\{
            \begin{array}{cl}
                \min\limits_{\bnu_{a_1:a_2}} & \sum_{j=a_1}^{a_2} h_j(\nu_j) \\
                \st & \nu_{a_1} \geq \cdots \geq \nu_{a_2}
            \end{array}
            \right. \quad \text{and} \quad
            \left\{
            \begin{array}{cl}
                \min\limits_{\bnu_{a_2+1:a_3}} & \sum_{j=a_2+1}^{a_3} h_j(\nu_j) \\
                \st & \nu_{a_2+1} \geq \cdots \geq \nu_{a_3} \\
            \end{array}
            \right.
        \end{align*}
        are solved by $\bnu_{a_1:a_2}^\star$ and $\bnu_{a_2+1:a_3}^\star$ with $\nu_{a_1}^\star = \cdots = \nu_{a_2}^\star$ and $\nu_{a_2+1}^\star = \cdots = \nu_{a_3}^\star$, respectively.
        If $\nu_{a_1}^\star \leq \nu_{a_2+1}^\star$, then the optimal solution for the merged block $[a_1, a_3]$ is single-valued, that is, the minimization problem
        \begin{align*}
            \left\{
            \begin{array}{cl}
                \min\limits_{\bnu_{a_1:a_3}} & \sum_{j=a_1}^{a_3} h_j(\nu_j) \\
                \st & \nu_{a_1} \geq \cdots \geq \nu_{a_3}
            \end{array}
            \right.
        \end{align*}
        is solved by $\bnu_{a_1:a_3}^\star$ with $\nu_{a_1}^\star = \cdots = \nu_{a_3}^\star$.

        \item \textbf{Property 2: No isotonic constraint violation between single-valued blocks implies the solution is optimal.} Suppose that we have $s$ blocks $[a_1, a_2], [a_2+1, a_3], \ldots, [a_{s}+1, a_{s+1}]$ (with $a_1=1$ and $a_{s+1}=p$) such that the optimal solution for each block is single-valued, that is, $\nu^\star_{a_l+1} = \dots = \nu^\star_{a_{l+1}}$ for all $l \in [s]$. Then, if $\hat{\nu}_{a_1} \geq \hat{\nu}_{a_2+1} \geq \ldots \hat{\nu}_{a_{s}}$, then $\hat{\bnu}$ is the optimal solution to~\eqref{obj:KyFan_Huber_isotonic_regression_rewritten_as_generalized_isotonic_regression}.
    \end{itemize}
    
    Using these two properties, it is now easy to see why Algorithm~\ref{alg:PAVA_algorithm} returns the optimal solution. 
    We start by constructing blocks which have length 1.
    The initial value restricted to each block is optimal.
    Then, we iteratively merge adjacent blocks and update the values of $\nu_j$'s whenever there is a violation of the isotonic constraint.
    By the first property, the optimal solution for the merged block is single-valued.
    Therefore, we can compute the optimal solution for the merged block by solving a univariate optimization problem.
    We keep merging blocks until there is no isotonic constraint violation.
    When this happens, by construction, the solution for each block is single-valued and optimal.
    By the second property, the final vector $\hat{\bnu}$ is the optimal solution to~\eqref{obj:KyFan_Huber_isotonic_regression_rewritten_as_generalized_isotonic_regression}, as required.
\end{proof}

\subsection{Proof of Lemma~\ref{lemma:PAVA_merging_linear_time_complexity}}

\begin{namedlemma}
    [~\ref{lemma:PAVA_merging_linear_time_complexity}]
    The merging step (lines 11-14) in Algorithm~\ref{alg:PAVA_algorithm} can be performed in linear time complexity $\mathcal O(p)$.
\end{namedlemma}

\begin{proof}
A detailed implementation of line 11-14 (Step 3) of the PAVA Algorithm~\ref{alg:PAVA_algorithm} that achieves a linear time complexity is presented in Algorithm~\ref{alg:up_and_down_block_algorithm_for_merging_in_PAVA}. In the following, we first show that Algorithm~\ref{alg:up_and_down_block_algorithm_for_merging_in_PAVA} accomplishes the objective in lines 11-14 of Algorithm~\ref{alg:PAVA_algorithm}. We then establish that Algorithm~\ref{alg:up_and_down_block_algorithm_for_merging_in_PAVA} runs in linear time complexity.

\begin{algorithm}[hb]
    \caption{Up and Down Block Algorithm for Merging in PAVA}
    \label{alg:up_and_down_block_algorithm_for_merging_in_PAVA}
    \begin{flushleft}
    \textbf{Input:} vector $\bmu \in \mathbb{R}^p$, nonnegative weights $\brho \in \mathbb{R}_{+}^p$ ($\rho_{[1:k]}=\rho, \rho_{k+1:p}=0$), vector $\hat{\bnu}$ ($\hat{\nu}_j = \text{prox}_{\rho_j H_M}(\vert{\mu_j})$), integer $k \in \mathbb{N}$ (first $k$ elements subject to Huber penalty), and threshold $M > 0$ for the Huber loss function. %\\
    %\textbf{Output:} vector $\hat{\bnu} \in \mathbb{R}^p$, which is the optimal solution to Problem~\eqref{obj:KyFan_Huber_isotonic_regression}.\\
    \end{flushleft}
    \begin{algorithmic}[1]
        \STATE \COMMENT{Initialization for the first block}
        \STATE Initialize $b=1$, $P_1 = \rho_1$, $S_1 = \vert{\mu_1}$, $N_b=1$, $\nu_1$, $r_1 = 1$.
        \STATE $\nu_{\text{prev}} = \hat{\nu}_1$, $j=2$
        \WHILE{$j \leq n$}
            \STATE $b = b + 1$
            \STATE $P_b = \rho_j$, $S_b = \vert{\mu_j}$, $N_b=1$, $\nu = \hat{\nu}_j$
            \STATE \COMMENT{If the value for the current singleton block is greater that of the previous block (isotonic violation), merge the current block with the previous block}
            \IF{$\nu > v_{\text{prev}}$}
                \STATE $b = b - 1$
                \STATE $P_b = P_b + \rho_j$, \, $S_b = S_b + \vert{\mu_j}$, \, $N_b = N_b + 1$, \, $\nu = \text{prox}_{\frac{P_b}{N_b} H_{M}}(\frac{S_b}{N_b})$
                \STATE \COMMENT{Look forward: keep merging the current block with the next block if the isotonic violation persists}
                \WHILE{$j < n$ \AND $\nu \leq \hat{\nu}_j$}
                    \STATE $j = j + 1$
                    \STATE $P_b = P_b + \rho_j$, \, $S_b = S_b + \vert{\mu_j}$, \, $N_b = N_b + 1$, \, $\nu = \text{prox}_{\frac{P_b}{N_b} H_{M}}(\frac{S_b}{N_b})$
                \ENDWHILE
                \STATE \COMMENT{Look backward: keep merging the current block with the previous block if the isotonic violation persists}
                \WHILE{$b > 1$ \AND $\nu_{b-1} < \nu$}
                    \STATE $b = b - 1$
                    \STATE $P_b = P_b + P_{b+1}$, \, $S_b = S_b + S_{b+1}$, \, $N_b = N_b + N_{b+1}$, \, $\nu = \text{prox}_{\frac{P_b}{N_b} H_{M}}(\frac{S_b}{N_b})$
                \ENDWHILE
            \ENDIF
            \STATE \COMMENT{Save the current block's value and the index of the last element in the block}
            \STATE $\nu_b = \nu$, $r_b = j$
            \STATE \COMMENT{Start fresh on the next element}
            \STATE $\nu_{\text{prev}} = \nu$, $j = j + 1$
        \ENDWHILE
        \STATE \COMMENT{Modify the output vector to have the same new value for all elements in each block}
        \FOR{$l = 1, ..., b$}
            \STATE $\hat{\nu}_{[r_{l-1}+1:r_l]} = \nu_l$
        \ENDFOR
        \STATE \textbf{return} $\hat{\bnu}$
    \end{algorithmic}
\end{algorithm}

% \begin{algorithm}[H]
%     \caption{Modified PAVA with Huber Penalty for Nonincreasing Isotonic Regression}
%     \label{alg:up_and_down_block_algorithm_for_merging_in_PAVA}
%     \begin{flushleft}
%     \textbf{Input:} vector $\bmu \in \mathbb{R}^n$ (observations), nonnegative weights $w \in \mathbb{R}_{\ge 0}^n$, integer $k \in \mathbb{N}$ (first $k$ elements subject to Huber penalty), scalar $\rho > 0$ (Huber penalty coefficient), and threshold $M > 0$.\\
%     \textbf{Output:} vector $x \in \mathbb{R}^n$ (monotone, nonincreasing sequence approximating $y$).\\
%     \end{flushleft}
%     \begin{algorithmic}[1]
%         \STATE $M, P$
%         \STATE $y_1 \gets y,\; y_2 \gets y,\; w_1 \gets w,\; w_2 \gets w$
%         \FOR{$j = 0$ to $k-1$}
%             \STATE $w_1[j] \gets w[j] + \frac{\rho}{2}$
%             \STATE $y_1[j] \gets y[j] \cdot \frac{w[j]}{w_1[j]}$
%             \STATE $y_2[j] \gets y[j] - \frac{\rho M}{2w_2[j]}$
%         \ENDFOR
%         \STATE Initialize boolean array $\text{use\_y1}$ of length $n$:
%         \FOR{$j = 0$ to $n-1$}
%             \STATE $\text{use\_y1}[j] \gets (y_1[j] \leq M)$
%         \ENDFOR
    
%         \STATE Allocate arrays $x_1^{\text{block}}, w_1^{\text{block}}, x_2^{\text{block}}, w_2^{\text{block}}$, and $\text{use\_x1\_block}$ of length $n$
%         \STATE Allocate array $r$ of length $n+1$
%         \STATE $r[0] \gets -1,\; r[1] \gets 0$
%         \STATE $b \gets 1$ \COMMENT{Number of blocks}
    
%         \STATE $x_1^{\text{block}}[0] \gets y_1[0],\; w_1^{\text{block}}[0] \gets w_1[0]$
%         \STATE $x_2^{\text{block}}[0] \gets y_2[0],\; w_2^{\text{block}}[0] \gets w_2[0]$
%         \STATE $\text{use\_x1\_block}[0] \gets \text{use\_y1}[0]$
    
%         \STATE $j \gets 1$
%         \WHILE{$j < n$}
%             \STATE $b \gets b + 1$
    
%             \STATE \textbf{Compute current values:}
%             \IF{$\text{use\_y1}[j] = \text{True}$}
%                 \STATE $x_{\text{curr}} \gets y_1[j],\; w_{\text{curr}} \gets w_1[j]$
%             \ELSE
%                 \STATE $x_{\text{curr}} \gets \max(y_2[j], M),\; w_{\text{curr}} \gets w_2[j]$
%             \ENDIF
    
%             \STATE \textbf{Compute previous block values:}
%             \STATE $\ell \gets b - 2$ \COMMENT{Index of previous block}
%             \IF{$\text{use\_x1\_block}[\ell] = \text{True}$}
%                 \STATE $x_{\text{prev}} \gets x_1^{\text{block}}[\ell],\; w_{\text{prev}} \gets w_1^{\text{block}}[\ell]$
%             \ELSE
%                 \STATE $x_{\text{prev}} \gets \max(x_2^{\text{block}}[\ell], M),\; w_{\text{prev}} \gets w_2^{\text{block}}[\ell]$
%             \ENDIF
    
%             \STATE \textbf{Check for nonincreasing violation:} 
%             \IF{$x_{\text{prev}} < x_{\text{curr}}$}
%                 \STATE $b \gets b - 1$
%                 \STATE Merge current element with previous block:
    
%                 \STATE $S_1 \gets (w_1^{\text{block}}[b-1] \cdot x_1^{\text{block}}[b-1]) + (w_1[j] \cdot y_1[j])$
%                 \STATE $W_1 \gets w_1^{\text{block}}[b-1] + w_1[j]$
%                 \STATE $x_{1,\text{merged}} \gets S_1 / W_1$
    
%                 \STATE $S_2 \gets (w_2^{\text{block}}[b-1] \cdot x_2^{\text{block}}[b-1]) + (w_2[j] \cdot y_2[j])$
%                 \STATE $W_2 \gets w_2^{\text{block}}[b-1] + w_2[j]$
%                 \STATE $x_{2,\text{merged}} \gets S_2 / W_2$
    
%                 \STATE $\text{use\_x1\_merged} \gets (x_{1,\text{merged}} \leq M)$
    
%                 \COMMENT{k-up step: merge forward if violation persists}
%                 \WHILE{$j < n-1$ \AND $\bigl(x_{1,\text{merged}} \cdot \text{use\_x1\_merged} + (1-\text{use\_x1\_merged}) \cdot \max(x_{2,\text{merged}}, M)\bigr) \leq \bigl(y_1[j+1] \cdot \text{use\_y1}[j+1] + (1-\text{use\_y1}[j+1]) \cdot \max(y_2[j+1], M)\bigr)$}
%                     \STATE $j \gets j + 1$
%                     \STATE $S_1 \gets S_1 + w_1[j] \cdot y_1[j],\; W_1 \gets W_1 + w_1[j],\; x_{1,\text{merged}} \gets S_1 / W_1$
%                     \STATE $S_2 \gets S_2 + w_2[j] \cdot y_2[j],\; W_2 \gets W_2 + w_2[j],\; x_{2,\text{merged}} \gets S_2 / W_2$
%                     \STATE $\text{use\_x1\_merged} \gets (x_{1,\text{merged}} \leq M)$
%                 \ENDWHILE
    
%                 \COMMENT{k-down step: merge backward if violation persists}
%                 \WHILE{$b > 1$ \AND $\bigl(x_1^{\text{block}}[b-2] \cdot \text{use\_x1\_block}[b-2] + (1-\text{use\_x1\_block}[b-2]) \cdot \max(x_2^{\text{block}}[b-2], M)\bigr) < \bigl(x_{1,\text{merged}} \cdot \text{use\_x1\_merged} + (1-\text{use\_x1\_merged}) \cdot \max(x_{2,\text{merged}}, M)\bigr)$}
%                     \STATE $b \gets b - 1$
%                     \STATE $S_1 \gets S_1 + w_1^{\text{block}}[b-1] \cdot x_1^{\text{block}}[b-1],\; W_1 \gets W_1 + w_1^{\text{block}}[b-1],\; x_{1,\text{merged}} \gets S_1 / W_1$
%                     \STATE $S_2 \gets S_2 + w_2^{\text{block}}[b-1] \cdot x_2^{\text{block}}[b-1],\; W_2 \gets W_2 + w_2^{\text{block}}[b-1],\; x_{2,\text{merged}} \gets S_2 / W_2$
%                     \STATE $\text{use\_x1\_merged} \gets (x_{1,\text{merged}} \leq M)$
%                 \ENDWHILE
    
%                 \STATE $x_1^{\text{block}}[b-1] \gets x_{1,\text{merged}},\; w_1^{\text{block}}[b-1] \gets W_1$
%                 \STATE $x_2^{\text{block}}[b-1] \gets x_{2,\text{merged}},\; w_2^{\text{block}}[b-1] \gets W_2$
%                 \STATE $\text{use\_x1\_block}[b-1] \gets \text{use\_x1\_merged}$
%                 \STATE \textit{No violation}
%             \ELSE
%                 \STATE \COMMENT{No violation}
%                 \STATE $x_1^{\text{block}}[b-1] \gets y_1[j],\; w_1^{\text{block}}[b-1] \gets w_1[j]$
%                 \STATE $x_2^{\text{block}}[b-1] \gets y_2[j],\; w_2^{\text{block}}[b-1] \gets w_2[j]$
%                 \STATE $\text{use\_x1\_block}[b-1] \gets \text{use\_y1}[j]$
%             \ENDIF
    
%             \STATE $r[b] \gets j$
%             \STATE $j \gets j + 1$
%         \ENDWHILE
    
%         \COMMENT{Expand blocks to form final $x$}
%         \STATE $x \gets$ empty array of length $n$
%         \STATE $f \gets n-1$
    
%         \FOR{$\ell = b$ down to $1$}
%             \STATE $start\_idx \gets r[\ell-1] + 1$
%             \STATE $end\_idx \gets r[\ell]$
%             \IF{$\text{use\_x1\_block}[\ell-1] = \text{True}$}
%                 \STATE $block\_value \gets x_1^{\text{block}}[\ell-1]$
%             \ELSE
%                 \STATE $block\_value \gets \max(x_2^{\text{block}}[\ell-1], M)$
%             \ENDIF
%             \FOR{$idx = end\_idx$ down to $start\_idx$}
%                 \STATE $x[idx] \gets block\_value$
%             \ENDFOR
%             \STATE $f \gets start\_idx - 1$
%         \ENDFOR
    
%         \STATE \textbf{return} $x$
%     \end{algorithmic}
% \end{algorithm}


To prove the first claim, we show that the parameters $P_b, S_b,$ and $\nu_b$ amount to
\begin{align*}
    \textstyle
    P_b = \sum_{j \in \calB(b)} \rho_j, ~ 
    S_b = \sum_{j \in \calB(b)} \vert{\mu_j}, ~ 
    \nu_b = \prox_{\sum_{j \in \calB(b)} \rho_j H_M}(|\mu_j|)
\end{align*}
for each block index $b$, where $\calB(b)$ denoting the set of indices in the $b$'th block. It is easy to verify that Algorithm~\ref{alg:up_and_down_block_algorithm_for_merging_in_PAVA} recursively computes $P_b$ and $S_b$. Thus, we will focus on $\nu_b$.
Note that the computation of the proximal operator in $\nu_b$ is reduced to solving a univariate optimization problem for each $b$ and satisfies
\begin{align*}
    \nu_b =& \argmin_{v \in \R} \sum_{j \in \calB(b)} \left( \frac{1}{2} (v - \vert{\mu_j})^2 + \rho_j H_M(v) \right) \\
    %= & \argmin_{v} \sum_{j \in \calB(b)} \left( \frac{1}{2} (v^2 - 2v\vert{\mu_j} + \mu_j^2) + \rho_j H_M(v) \right) \\
    = & \argmin_{v} \sum_{j \in \calB(b)} \left( \frac{1}{2} v^2 - v\vert{\mu_j} + \rho_j H_M(v) \right) \\
    %= & \argmin_{v} \left( \sum_{j \in \calB(b)} \frac{1}{2} v^2 - \sum_{j \in \calB(b)} v\vert{\mu_j} + \sum_{j \in \calB(b)} \rho_j H_M(v) \right) \\
    %= & \argmin_{v} \left( N_b \frac{1}{2} v^2 - S_b \vert{\mu_j} + P_b H_M(v) \right) \\
    = & \argmin_{v} \left( \frac{1}{2} v^2 - \frac{S_b}{N_b} \vert{\mu_j} + \frac{P_b}{N_b} H_M(v) \right) 
    = \argmin_{v} \left( \frac{1}{2} \left( v - \frac{S_b}{N_b} \right)^2 + \frac{P_b}{N_b} H_M(v) \right) 
    = \prox_{\frac{P_b}{N_b} H_{M}}(\frac{S_b}{N_b}).
\end{align*}

Thus, Algorithm~\ref{alg:up_and_down_block_algorithm_for_merging_in_PAVA} merges two adjacent blocks if the isotonic violation persists, and the output of the proximal operator is the minimizer of the univariate function in the merged block.
This is exactly the same as the objective in lines 11-14 of Algorithm~\ref{alg:PAVA_algorithm}. Hence, the first claim follows.

To show that the algorithm runs in linear time, notice that in the while loop $j \leq p$ in Algorithm~\ref{alg:up_and_down_block_algorithm_for_merging_in_PAVA}, the variable $j$ is incremented by $1$ in each iteration, and the loop terminates when $j = p$.
Although there are two while loops inside the main while loop, the total number of iterations in the two inner while loops is at most $p$.
This is because we start with $p$ blocks, and each iteration of the inner while loops either merges two blocks forward or merges two blocks backward.
The total number of merging operations is at most $p-1$.
Thus, the total number of iterations in the while loop $j \leq p$ is at most $p$.
Lastly, since we can evaluate the proximal operator of the Huber loss function, $\mathbf{H}_M$, in constant time complexity, the total time complexity of Algorithm~\ref{alg:up_and_down_block_algorithm_for_merging_in_PAVA} is $O(p)$.
\end{proof}

\subsection{Proof of Theorem~\ref{theorem:pava_algorithm_linear_time_complexity_and_exact_solution}}

\begin{namedtheorem}
    [~\ref{theorem:pava_algorithm_linear_time_complexity_and_exact_solution}]
    For any $\bmu \in \R^p$, Algorithm~\ref{alg:PAVA_algorithm} returns the \textit{exact} evaluation of $\prox_{\rho g^*}(\bmu)$ in $\tilde {\mathcal O}(p)$.
\end{namedtheorem}

\begin{proof}
    By Lemmas~\ref{lemma:equivalence_between_proximal_operator_and_huber_isotonic_regression} and~\ref{lemma:PAVA_algorithm_exact_solution}, the output of Algorithm~\ref{alg:PAVA_algorithm} computes $\prox_{\rho g^*}$ exactly. 
    The linear time complexity statement also holds thanks to Lemma~\ref{lemma:PAVA_merging_linear_time_complexity}.
\end{proof}

\subsection{Proof of Theorem~\ref{theorem:compute_g_value_algorithm_correctness}}

\begin{namedtheorem}
    [~\ref{theorem:compute_g_value_algorithm_correctness}]
        For any $\bbeta \in \R^p$, Algorithm~\ref{alg:compute_g_value_algorithm} computes the exact value of $g(\bbeta)$, defined in~\eqref{eq:function_g_definition}, in $\mathcal O(p + p \log k)$.
\end{namedtheorem}

\begin{proof}
% Add proof content here

We first show that the algorithm correctly computes the value of $g(\bbeta)$ and then analyze its computational complexity. Define the mixed-binary set
\begin{align*}
    \calS_0 = \left\{ (t, \bbeta) \;\middle|\; \textstyle \frac{1}{2} \sum_{j \in [p]} \beta_j^2 \leq t, \, \|\bbeta \|_\infty \leq M, \, \|\bbeta \|_0 \leq k \right\}.
\end{align*}
Using the perspective and big-M reformulation techniques, the set $\calS_0$ admits the equivalent representation
\begin{align*}
    \calS_0 = \left\{ (t, \bbeta) \;\middle|\; \exists \bz \in \{0,1\}^p ~ \st ~ \textstyle \frac{1}{2} \sum_{j \in [p]} \beta_j^2 / z_j \leq t, \, \bm 1^\top \bz \leq k, \, -M z_j \leq \beta_j \leq M z_j ~~ \forall j \in [p] \right\}.
\end{align*}
Following the proof of Lemma~\ref{lemma:equivalence_between_perspective_relaxation_and_convexification}, one can show that the closed convex hull of $\calS_0$ is given by 
\begin{align*}
    \cl \conv(\calS_0) = \left\{ (t, \bbeta) \;\middle|\; \exists \bz \in [0,1]^p ~ \st ~ \textstyle \frac{1}{2} \sum_{j \in [p]} \beta_j^2 / z_j \leq t, \, \bm 1^\top \bz \leq k, \, -M z_j \leq \beta_j \leq M z_j ~~ \forall j \in [p] \right\}.
\end{align*}
Therefore, the implicit function $g$ can be written as the evaluation of the support function of $\cl\conv(\calS_0)$ at $(1, \bm 0)$, that is,
\begin{align}
    \label{eq:g:S0}
    g(\bbeta) = \min  \{ t : (t, \bbeta) \in \cl\conv(\calS_0) \}.
\end{align}
Notice that the set $\calS_0$ is sign- and permutation-invariants. Hence, by ~\citep[Theorem~4]{kim2022convexification}, its closed convex hull admits the following (different) lifted represenation
\begin{align}
    \label{eq:diff:conv}
    \cl \conv(\calS_0) = \left\{ (t, \bbeta) \;\middle|\; \exists \bphi \in \R^p ~ \st ~
    \begin{array}{l}
        \frac{1}{2} \sum_{j \in [p]} \phi_j^2 \leq t, \, \vert{\bbeta} \preceq_m \bphi, \\
        0 \leq \phi_k \leq \ldots \leq \phi_1 \leq M, \\
        \phi_{k+1} = \phi_{k+2} = \ldots = \phi_n = 0 
    \end{array}
    \right\},
\end{align}
where the absolute value operator $\vert{\cdot}$ is applied to a vector in an element-wise fashion, and the constraint $\vert{\bbeta} \preceq_m \bphi$ denotes that $\bphi$ majorizes $\vert{\bbeta}$, that is,
\begin{align*}
    \textstyle \vert{\bbeta} \preceq_m \bphi  \quad \iff \quad  \sum_{j \in [l]} \vert{\beta_j} \leq \sum_{j \in [l]} \phi_j \quad \forall l \in [p-1] \quad \text{and} \quad \sum_{j \in [p]} \phi_j = \sum_{j \in [p]} \vert{\beta_j}.
\end{align*}
Using this alternative convex hull description of $\calS_0$ in~\eqref{eq:diff:conv} and the implicit formulation~\eqref{eq:g:S0}, we may conclude that
\begin{align}
    g(\bbeta) = \min\limits_{\bphi \in \R^p}
    \textstyle \left\{ \frac{1}{2} \sum_{j \in [p]} \phi_j^2 :  \vert{\bbeta} \preceq_m \bphi, \, 0 \leq \phi_k \leq \ldots \leq \phi_1 \leq M, \, \phi_{k+1} = \phi_{k+2} = \ldots = \phi_n = 0
    \right\}. \label{appendix_obj:compute_g_value_majorization_formulation}
\end{align}
In the following we show that Algorithm~\ref{alg:compute_g_value_algorithm} can efficiently solve the minimization problem in~\eqref{appendix_obj:compute_g_value_majorization_formulation}. At the first iteration $j=1$ of the algorithm, we have
\begin{align*}
    \textstyle k \phi_1 \geq \sum_{j \in [k]} \phi_j = \sum_{j \in [p]} \phi_j \geq \sum_{j \in [p]} \vert{\beta_j} \quad \Rightarrow \quad \phi_1 \geq \frac{1}{k} \sum_{j \in [p]} \vert{\beta_j}.
\end{align*}
At the same time, we also need to satisfy $\vert{\beta_1} \leq \phi_1$ from the first majorization constraint. We now discuss two cases
\begin{itemize}[label=$\diamond$,leftmargin=*]
    \item \textbf{Case 1:} If $\frac{1}{k} \sum_{j \in [p]} \vert{\beta_j} \geq \vert{\beta_1}$, in order to solve the minimization problem in~\eqref{appendix_obj:compute_g_value_majorization_formulation}, we set $\phi_1 = \frac{1}{k} \sum_{j=1}^n \vert{\beta_j}$. Notice that $\phi_1 \leq M$ is automatically satisfied because $\phi_1 = \frac{1}{k} \sum_{j \in [p]} \vert{\beta_j} = \frac{1}{k} \sum_{j \in [p]} M z_j \leq M$. This leads to $\phi_2 = \ldots = \phi_k = \frac{1}{k} \sum_{j \in [p]} \vert{\beta_j}$.
    To see this, for the sake of contradition, assume that $\exists j \in \{2, \ldots, k\}$ such that $\phi_j < \frac{1}{k} \sum_{j \in [p]} \vert{\beta_j}$. 
    Since $\phi_j \leq \phi_1 = \frac{1}{k} \sum_{j \in [p]} \vert{\beta_j}$, we have $\sum_{j \in [k]} \phi_j < \sum_{j \in [k]} \frac{1}{k} \sum_{j \in [p]} \vert{\beta_j} = \sum_{j \in [p]} \vert{\beta_j}$, which contradicts the majorization constraint.

    \item \textbf{Case 2:} If $\frac{1}{k} \sum_{j \in [n]} \vert{\beta_j} < \vert{\beta_1}$, we can set $\phi_1 = \vert{\beta_1}$. Notice that $\phi_1 \leq M$ is automatically satisfied because $\vert{\beta_1} \leq M z_1 \leq M$.
    Then we are left with $k-1$ coefficients to set, and we can follow the same argument as we did for $j=1$ with slight difference that the majorization constraints are changed to
    \begin{align*}
        \textstyle
        \sum_{j=2}^l \phi_j \geq \sum_{j=2}^l \vert{\beta_j} \quad \forall l \in \{2, \ldots, p-1\} \quad \text{and} \quad \sum_{j=2}^p \phi_j = \sum_{j=2}^p \vert{\beta_j}.
    \end{align*}
\end{itemize}
We repeat this process until we set all $k$ coefficients $\phi_1, \ldots, \phi_k$, as implemented by Algorithm~\ref{alg:compute_g_value_algorithm}.
The output of the algorithm coincides with the optimal value of the minimization problem in~\eqref{appendix_obj:compute_g_value_majorization_formulation}. Hence, the first claim follows.

As for the complexity claim, it is easy to see that Algorithm~\ref{alg:compute_g_value_algorithm}.
only requires partial sorting step on Line 2, which has a complexity of $\mathcal O(p \log k)$. The summation step on Line 3 has a complexity of $\mathcal O(p)$. The for-loop step on Line 4-8 has a complexity of $\mathcal O(k)$, so does the final summation step on Line 9. Therefore, the overall computational complexity of Algorithm~\ref{alg:compute_g_value_algorithm} is $\mathcal O(p + p \log k)$. This concludes the proof.
\end{proof}

\newpage
\section{Experimental Setup Details}
\label{appendix:experimental_setup}

\subsection{Setup for Evaluating Proximal Operators}
\label{appendix:setup_for_evaluating_proximal_operators}
The synthetic data generation process is as follows.
We sample the input vector $\bgamma \in \bbR^p$ from the standard multivariate Gaussian distribution, $\bgamma \sim \calN(\mathbf{0}, \bI_p)$, where $\bI_p$ denotes the identity matrix with dimension $p$.
We vary the dimension $p \in \{2^0, 2^1, ..., 2^{10}\} \times 10^2$ and set the cardinality $k$ to be $10$, the box constraint $M$ to be $1.0$, and the weight parameter $\rho$  to be $1.0$.
We report the running time for evaluating these proximal operators.
To obtain the mean and standard deviation of the running time, we repeat each setting 5 times, each with a different random seed.

\subsection{Setup for Solving the Perspective Relaxation}
\label{appendix:setup_for_solving_the_perspective_relaxation}

We generate our synthetic datasets in the following procedure.
First, we sample each feature vector $\bx_i \in \bbR^p $ from a Gaussian distribution, $\bx_i \sim \calN(\mathbf{0}, \bSigma)$, where the covariance matrix has entries $\Sigma_{jl} = \sigma^{\vert{j-l}}$.
The variable $\sigma \in (0, 1)$ controls the features correlation: if we increase $\sigma$, feature columns in the design matrix $\bX$ become more correlated.
Throughout the experimental section, we set $\sigma=0.5$.
Next, we create the sparse coefficient vector $\bbeta^*$ with $k$ equally spaced nonzero entries, where $\beta^*_j = 1$ if $j \text{ mod } (p/k) = 0$ and $\beta^*_j = 0$ otherwise.
After these two steps, we build the prediction vector $\by$.
If our loss function is squared error loss (regression task), we set $y_i = \bx_i^T \bbeta^* + \epsilon_i$, where $\epsilon_i$ is a Gaussian random noise with $\epsilon_i \sim \calN(0, \frac{\Vert{\bX \bbeta^*}}{\text{SNR}})$, and $\text{SNR}$ stands for the signal-to-noise ratio.
In all our experiments, we choose $\text{SNR}=5$.
If our loss function is logistic loss (classification task), we set $y_i \sim Bern(\bx_i^T \bbeta^* + \epsilon_i)$, where $Bern(P)$ is a Bernoulli random variable with $\bbP(y_i = 1) = P$ and $\bbP(y_i = -1) = 1 - P$.
For this experiment, we vary the feature dimension $p \in \{1000, 2000, 4000, 8000, 16000\}$.
We control the sample size by using a parameter called $n$-to-$p$ ratio, or sample to feature ratio.
For the results in the main paper, we set $n$-to-$p$ ratio to be $1.0$, the box constraint $M$ to be $2$, the number of nonzero coefficients k (also the cardinality constraint) to be $10$, and $\ell_2$ regularization coefficient $\lambda_2$ to be $1.0$.
Again, we report and compare the running times, with means and standard deviations calculated based on 5 repeated simulations with different random seeds.

\subsection{Setup for Certifying Optimality}
\label{appendix:setup_for_certifying_optimality}

\paragraph{Datasets and Preprocessing}
We run on both synthetic and real-world datasets.
For the synthetic datasets, we run on the largest synthetic instances ($n=16000$ and $p=16000$).
For the real-world datasets, we use the dataset cancer drug response~\cite{liu2020deepcdr} for linear regression and DOROTHEA~\cite{asuncion2007uci} for logistic regression.

The cancer drug response dataset has 822 samples and orginally has 34674 features.
However, many feature only has a single value, so we prune all these features, which result in 2200 features.
The DOROTHEA dataset has 1950 samples and 100000 features.
After pruning redundant features, we have 91598 features.

For both the cancer drug response and DOROTHEA dataset, we center each feature to have mean $0$ and norm equal to $1$.

\paragraph{Choice of Hyperparameters}
For the cardinality constraint $k$, we set $k=10$ for both synthetic datasets.
For the cancer drug response dataset, we set $k=5$.
For DOROTHEA, we set $k=15$.
In practice, this choice can be made more judiciously by doing 5 fold cross validation with a heuristic sparse learning algorithm first.
However, since our emphasis here is simply to compare certification speed, we just pick a variety of $k$'s.

For the $\ell_2$ regularization coefficient, we set $\lambda_2=1$.
For the box constraint, we set $M=2$ for the synthetic datasets and DOROTHEA.
The infinity norm of the final optimal solution less than this value.
For the cancer drug response dataset, we set $M=5$, which is also bigger than the infinity norm of the final optimal solution.

\paragraph{Branch and Bound}
For our method, we write a customized branch-and-bound (BnB) framework.
We use Algorithm~\ref{alg:main_algorithm} to solve the relaxation at each node and use Equation~\eqref{eq:fenchel_duality_theorem_F_y(Ax)+G(x)} to calculate the safe lower bound to prune the search space.
To find feasible solutions, we use an effective approach called beamsearch~\cite{liu2022fasterrisk} from the existing literature.
For branching, we branch on the feature based on the best feasible solution found by the beamsearch algorithm at each node.
For the nonzero coefficients of this solution, we branch on the variable which would lead to the largest loss increase if the coefficient to $0$.
The intuition is that such a variable is important and should be branched early in the BnB framework.

\subsection{Computing Platforms}
When investigating how much GPU can accelerate our computation, we run the experiments with both CPU and GPU implementations on the Nvidia RTXA5000s.
For everything else, we run the experiments with the CPU implementation on AMD Milan with CPU speed 2.45 Ghz and 8 cores.

\section{Additional Numerical Results}
\label{appendix:numerical}

% \subsection{Perturbation Study regarding Proximal Operators}
% \label{appendix:numerical_proximal_operators}

% \subsubsection{Perturbation Study on $k$ Values}

% \subsubsection{Perturbation Study on $M$ Values}

% \subsubsection{Perturbation Study on $\rho$ Values}

\subsection{Perturbation Study regarding Solving the Perspective Relaxation}
\label{appendix:numerical_solve_convex_relaxation}

\subsubsection{Perturbation Study on $M$ Values}

\begin{figure*}[!ht]
    \centering
    \includegraphics[width=0.9\textwidth]{sections/Plots/big_M_perturbation/convex_relaxation_comparison_n_p_ratio_1.0_M_1.2.png}
    \caption{Solve the perspective relaxation in Problem~\eqref{obj:original_sparse_problem_perspective_formulation_convex_relaxation}.
    We set $M=1.2$, $\lambda_2=1.0$, $n$-to-$p$ ratio to be 1.}
    \label{fig:solve_convex_relaxation_M_1.2_lambda2_1.0_n_p_ratio_1.0}
\end{figure*}

\begin{figure*}[!ht]
    \centering
    \includegraphics[width=0.9\textwidth]{sections/Plots/big_M_perturbation/convex_relaxation_comparison_n_p_ratio_1.0_M_1.5.png}
    \caption{Solve the perspective relaxation in Problem~\eqref{obj:original_sparse_problem_perspective_formulation_convex_relaxation}.
    We set $M=1.5$, $\lambda_2=1.0$, $n$-to-$p$ ratio to be 1.}
    \label{fig:solve_convex_relaxation_M_1.5_lambda2_1.0_n_p_ratio_1.0}
\end{figure*}

\begin{figure*}[!ht]
    \centering
    \includegraphics[width=0.9\textwidth]{sections/Plots/big_M_perturbation/convex_relaxation_comparison_n_p_ratio_1.0_M_3.0.png}
    \caption{Solve the perspective relaxation in Problem~\eqref{obj:original_sparse_problem_perspective_formulation_convex_relaxation}.
    We set $M=3.0$, $\lambda_2=1.0$, $n$-to-$p$ ratio to be 1.}
    \label{fig:solve_convex_relaxation_M_3.0_lambda2_1.0_n_p_ratio_1.0}
\end{figure*}

\begin{figure*}[!ht]
    \centering
    \includegraphics[width=0.9\textwidth]{sections/Plots/big_M_perturbation/convex_relaxation_comparison_n_p_ratio_1.0_M_5.0.png}
    \caption{Solve the perspective relaxation in Problem~\eqref{obj:original_sparse_problem_perspective_formulation_convex_relaxation}.
    We set $M=5.0$, $\lambda_2=1.0$, $n$-to-$p$ ratio to be 1.}
    \label{fig:solve_convex_relaxation_M_5.0_lambda2_1.0_n_p_ratio_1.0}
\end{figure*}

\begin{figure*}[!ht]
    \centering
    \includegraphics[width=0.9\textwidth]{sections/Plots/big_M_perturbation/convex_relaxation_comparison_n_p_ratio_1.0_M_10.0.png}
    \caption{Solve the perspective relaxation in Problem~\eqref{obj:original_sparse_problem_perspective_formulation_convex_relaxation}.
    We set $M=10.0$, $\lambda_2=1.0$, $n$-to-$p$ ratio to be 1.}
    \label{fig:solve_convex_relaxation_M_10.0_lambda2_1.0_n_p_ratio_1.0}
\end{figure*}

\newpage

\subsubsection{Perturbation Study on $\lambda_2$ Values}


\begin{figure*}[!ht]
    \centering
    \includegraphics[width=0.9\textwidth]{sections/Plots/lambda2_perturbation/convex_relaxation_comparison_lambda2_0.1.png}
    \caption{Solve the perspective relaxation in Problem~\eqref{obj:original_sparse_problem_perspective_formulation_convex_relaxation}.
    We set $M=2.0$, $\lambda_2=0.1$, $n$-to-$p$ ratio to be 1.}
    \label{fig:solve_convex_relaxation_M_2.0_lambda2_0.1_n_p_ratio_1.0}
\end{figure*}

\begin{figure*}[!ht]
    \centering
    \includegraphics[width=0.9\textwidth]{sections/Plots/lambda2_perturbation/convex_relaxation_comparison_lambda2_10.0.png}
    \caption{Solve the perspective relaxation in Problem~\eqref{obj:original_sparse_problem_perspective_formulation_convex_relaxation}.
    We set $M=2.0$, $\lambda_2=10.0$, $n$-to-$p$ ratio to be 1.}
    \label{fig:solve_convex_relaxation_M_2.0_lambda2_10.0_n_p_ratio_1.0}
\end{figure*}

\newpage

\subsubsection{Perturbation Study on $n$-to-$p$ Ratios}


\begin{figure*}[!ht]
    \centering
    \includegraphics[width=0.9\textwidth]{sections/Plots/n_p_ratio_perturbation/convex_relaxation_comparison_n_p_ratio_0.1_M_2.0.png}
    \caption{Solve the perspective relaxation in Problem~\eqref{obj:original_sparse_problem_perspective_formulation_convex_relaxation}.
    We set $M=2.0$, $\lambda_2=1.0$, $n$-to-$p$ ratio to be 10.0.}
    \label{fig:solve_convex_relaxation_M_2.0_lambda2_1.0_n_p_ratio_10.0}
\end{figure*}


\begin{figure*}[!ht]
    \centering
    \includegraphics[width=0.9\textwidth]{sections/Plots/n_p_ratio_perturbation/convex_relaxation_comparison_n_p_ratio_10.0_M_2.0.png}
    \caption{Solve the perspective relaxation in Problem~\eqref{obj:original_sparse_problem_perspective_formulation_convex_relaxation}.
    We set $M=2.0$, $\lambda_2=1.0$, $n$-to-$p$ ratio to be 0.1.}
    \label{fig:solve_convex_relaxation_M_2.0_lambda2_1.0_n_p_ratio_0.1}
\end{figure*}

\newpage

\section{Additional Discussions}
We first provide common calculus rules for conjugate functions, whose proof can be found in standard optimization textbooks such as~\citep{beck2017first}.

\begin{itemize}[label=$\diamond$,leftmargin=*]
    \item \textbf{Separable Sum Rule:} Let $f(\bx) = \sum_{j \in [p]} f_j(x_j)$, where $f_j: \R \rightarrow \R$ is convex for all $j \in [p]$. Then, the conjugate of $f$ is given by $f^*(\bmu) = \sum_{j \in [p]} f_j^*(\mu_j)$.
    
    \item \textbf{Scalar Multiplication Rule:}  Let $g : \R^p \to \R$ be convex and $\alpha > 0$ be a scalar. Then, the conjugate of $f(\bx) = \alpha g(\bx)$ is given by $f^*(\bmu) = \alpha g^*(\bmu/\alpha)$.
    
    \item \textbf{Addition to Affine Function Rule:} Let $g : \R^p \to \R$ be convex and $\ba, \bb \in \mathbb{R}^p$ be two vectors. Then, the conjugate of $f(\bx) = g(\bx) + \ba^\top\bx + b$ is given by $f^*(\bmu) = g^*(\bmu - \ba) - \bb$.
    
    \item \textbf{Composition with Invertible Linear Mapping Rule:} Let $g : \R^p \to \R$ be convex and $\bA \in \mathbb{R}^{p \times p}$ be an invertible matrix. Then, the convex conjugate of $f(\bx) = g(\bA \bx)$ is given by $f^*(\bmu) = g^*(\bA^{-\top} \bmu)$.
    
    \item \textbf{Infimal Convolution Rule:} Let $g, h : \R^p \to \R$ be convex. Then, the convex conjugate of $f(\bx) = \inf_{by} ~ g(\by) + h(\bx - \by)$ is given by $f^*(\bmu) = g^*(\bmu) + h^*(\bmu)$.
\end{itemize}
These rules are useful for discussions in~\ref{appendix_sec:convex_conjugate_for_GLM_loss_functions} and~\ref{appendix_sec:safe_lower_bound_more_discussions}.


\subsection{Convex Conjugate for GLM Loss Functions}
\label{appendix_sec:convex_conjugate_for_GLM_loss_functions}

The convex conjugates of some of GLM loss functions are summarized bellow.
\begin{itemize}[label=$\diamond$,leftmargin=*]
    \item \textbf{Linear Regression:} 
    $$F(\bX \bbeta) = \Vert{\bX \bbeta - \by}_2^2 \quad \& \quad F^*(-\bzeta) = \frac{1}{4} \Vert{\bzeta}_2^2 - \by^T \bzeta.$$
    \item \textbf{Logistic Regression:} 
    $$F(\bX \bbeta) = \sum_{i \in [n]} \log(1 + \exp(-y_i (\bX \bbeta)_i)) \quad \& \quad F^*(-\bzeta) = \sum_{i \in [n]} \left( 1- \frac{\zeta_i}{y_i} \right) \log \left( 1-\frac{\zeta_i}{y_i} \right) + \frac{\zeta_i}{y_i} \log \left( \frac{\zeta_i}{y_i} \right).$$ 
    \item \textbf{Poisson Regression:} 
    $$F(\bX \bbeta) = \sum_{i \in [n]} \left( \exp(\bX \bbeta)_i - y_i (\bX \bbeta)_i \right) \quad \& \quad F^*(-\bzeta) = \sum_{i \in [n]} h(-\zeta_i + y_i), $$
    where $h(z) = z \log(z) - z$ if $z > 0$ and $h(z)=0$ if $z = 0$.
    \item \textbf{Gamma Regression:}
    $$F(\bX \bbeta) = \sum_{i \in [n]} \left( y_i \exp(-(\bX \bbeta)_i) + (\bX \bbeta)_i\right) \quad \& \quad F^*(-\bzeta) = \sum_{i \in [n]} y_i h(\frac{1-\zeta_i}{y_i}), $$
    where $h(z) = z \log(z) - z$ if $z > 0$ and $h(z)=0$ if $z = 0$.
    \item \textbf{Squared Hinge Loss:}
    For binary classification with labels $y_i \in \{-1, +1\}$,
    $$F(\bX \bbeta) = \sum_{i \in [n]} \max(0, 1-y_i (\bX \bbeta)_i)^2 \quad \& \quad F^*(-\bzeta) = \sum_{i \in [n]}  h(- y_i \zeta_i),$$
    where $h(z) = z + \frac{z^2}{4}$ if $z \leq 0$ and $h(z)=\infty$ if $z > 0$.
    % \item \textbf{Multinomial Logistic Regression:}
    % For multiclass classification with $K$ classes with coefficients $\bbeta \in \mathbb{R}^{p \times K}$, let $y_{ik}$ be a binary indicator such that $y_{ik}=1$ if the $i$-th sample belongs to class $k$, and $y_{ik}=0$ otherwise.
    % $$F(\bX \bbeta) = \sum_{i \in [n]} \left( \log\left( \sum_{j=1}^K \exp((\bX \bbeta)_{ij}) \right) - \sum_{k=1}^K y_{ik} (\bX \bbeta)_{ik} \right) $$
    % $$F^*(-\bzeta) = \sum_{i \in [n]} \begin{cases}
    %     \sum_{k=1}^K (y_{ik} - \zeta_{ik}) \log(y_{ik} - \zeta_{ik}) & \text{if } \sum_{k=1}^K \zeta_{ik} = 0, \zeta_{ik} \le y_{ik} \text{ for all } k \\
    %     +\infty & \text{otherwise} \end{cases} $$
\end{itemize}




\subsection{Safe Lower Bound}
\label{appendix_sec:safe_lower_bound_more_discussions}

The linear regression problem with eigen-perspective relaxation is formulated as
\begin{align*}
    P^\star_{\text{eig-conv}} = \min_{\bbeta \in \R^p} \bbeta^\top \bQ_{\text{eig}} \bbeta - 2\by^\top \bX \bbeta +  2 \lambda_{\text{eig}} g(\bbeta),
\end{align*}
where $\bQ_{\text{eig}} = \bX^\top \bX - \lambda_{\text{min}}(\bX^\top \bX) \bI$, $\lambda_{\text{eig}} = \lambda_2 + \lambda_{\text{min}}(\bX^\top \bX)$, and $\lambda_{\text{min}}(\cdot)$ denotes minimum eigenvalue of the input matrix.
Using the standard version of weak duality theorem, we have
\begin{align*}
    P_{\text{MIP}}^\star \geq P_{\text{eig-conv}}^\star \geq - F^*(-\hat{\bzeta}) - G^*(\hat{\bzeta}),
\end{align*}
where $F(\bbeta)= \bbeta^\top \bQ_{\text{eig}} \bbeta$, $G(\bbeta) = -2\by^\top \bX \bbeta + 2 \lambda_{\text{eig}} g(\bbeta)$, and $\hat{\bzeta} = -\nabla F(\hat{\bbeta}) = -2\bQ_{\text{eig}} \hat{\bbeta}$.
The conjugate functions admit the following closed form expressions
\begin{align*}
    F^*(-\hat{\bzeta}) &= \frac{1}{4} \hat{\bzeta}^\top \bQ_{\text{eig}}^{\dagger} \hat{\bzeta} = \hat{\bbeta} \bQ_{\text{eig}} \hat{\bbeta} \quad \& \quad G^*(\hat{\bzeta}) = 2\lambda_{\text{eig}} \, g^* \left(\frac{-\bQ_{\text{eig}}\hat{\bbeta} +  \bX^\top \by}{\lambda_{\text{eig}}} \right), 
\end{align*}
where we use $(\cdot)^{\dagger}$ to denote the pseudo-inverse of a matrix. We may conclude that
\begin{align*}
    P_{\text{MIP}}^\star & \geq \hat{\bbeta} \bQ_{\text{eig}} \hat{\bbeta} + 2\lambda_{\text{eig}} \, g^* \left(\frac{-\bQ_{\text{eig}}\hat{\bbeta} +  \bX^\top \by}{\lambda_{\text{eig}}}\right).
\end{align*}
The above lower bound can be viewed as a generalization of the safe lower bound formula from~\citep[Theorem~3.1]{liu2024okridge}. Specifically, as $M$ approaches $\infty$, the above lower bound matches the lower bound in in~\citep[Theorem~3.1]{liu2024okridge}. 
Furthermore, Our proof uses a simple weak duality argument and is concise, in contrast to the lengthy two-page algebraic proof of~\citep[Theorem~3.1]{liu2024okridge}.


\end{document}
\endinput
%%
%% End of file `sample-sigconf-authordraft.tex'.
