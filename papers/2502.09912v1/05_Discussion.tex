\section{Discussion}
\label{discussion}
In this study, we examined how VR could promote team members' closeness when new members join an existing team. We conducted a between-subject experiment with teams meeting in one of two different environments---Virtual Reality (VR) and In-Person (IP)---and studied how incumbents and newcomers perceived each other by measuring their perceived closeness and the resultant familiarity bias. We found a positive and statistically significant effect of using a VR application on newcomers' closeness to incumbents. However, we found no significant evidence supporting that the VR setting positively affected incumbents' closeness to newcomers, and there was no significant difference in mitigating their familiarity bias. The participants' open-ended responses highlight how VR provided a shield to newcomers, fostering their safety and interactions with their incumbents. This section discusses how our findings advance understanding of VR in collaborative settings.

\subsection{Asymmetric Effects of Virtual Reality}
Our RQ1 asked whether using VR when forming teams influences the perceived closeness of incumbents to newcomers, and RQ2 focused on newcomers' closeness to incumbents. We found that newcomers in the VR condition perceived themselves more similar to their incumbents than the newcomers working IP, which mediated the positive effect of the VR condition on newcomers' closeness. While our findings reveal that the VR application fostered newcomers' perceptions of closeness to incumbents, incumbents' closeness to newcomers remained unchanged across both experimental conditions. Regardless of the setting, incumbents reported feeling neither similar nor particularly close to newcomers. This result extends prior research in HCI, showing that online communication systems can affect team members' perceptions in different ways \cite{yee2007proteus}. The current study identifies the asymmetric effects of employing VR on participants' perceived closeness. While VR has been shown to facilitate co-presence and improve the overall interaction experience in collaborative settings, our findings indicate that these benefits do not necessarily extend to all users equally.

Moreover, we asked whether VR could mitigate incumbents' familiarity bias against newcomers (RQ3). Although it was reduced in the VR application with respect to the IP setting, this difference was not statistically significant. This result could be attributed to incumbents' pre-established team dynamics. When the session began, incumbents had already had time to engage with their teammates and orient themselves to the task. By the time the newcomer arrived, incumbents were already in a task-focused mindset, leaving little room for social interactions that would foster a connection with the newcomer. This observation aligns with prior findings that highlight the importance of early social interactions in building team rapport \cite{10.1145/2998181.2998300, sawyer2010social}. Without deliberate interventions, VR alone does not seem sufficient to disrupt incumbents' reliance on pre-existing relationships or to encourage active engagement with newcomers.

\subsection{Virtual Reality: A Safe Space for Newcomers}
These findings highlight how VR can be employed to promote a more inclusive platform for newcomers. This study extends existing research on team familiarity \cite{10.1145/2998181.2998300, muskat2022team}, emphasizing the challenges newcomers face in navigating pre-existing relationships and biases within teams. Previous research indicates that newcomers often experience social pressure and anxiety when integrating into an established team \cite{kraut2010dealing, choi2004minority}. Our study suggests that VR can alleviate some of these challenges by making team members look more similar to each other. Features such as avatars and immersive spaces can help reduce the intensity of social cues that amplify social anxieties. This ``virtual shielding'' effect enabled newcomers to focus on collaboration and task completion rather than navigating the interpersonal complexities of team integration. The VR application offered a comfortable platform for newcomers to engage with established team members, making it a valuable tool for fostering equitable and effective team interactions.

Extensive research highlights the importance of technology's impact in making individuals' differences visible \cite{gomez2020impact,carte2004capabilities}. Our study extends these previous studies by showing how VR can reduce the visibility of demographic or cultural differences through customizable, similar avatars and neutral settings. VR applications in team settings should consider how the avatar design can promote team members' inclusion, supporting diverse team integration and shifting the focus toward shared goals and contributions. Yet, the lack of evidence supporting the idea that VR positively impacts existing team members underscores the need for alternative strategies to enhance this integration process. Approaches such as redesigned onboarding procedures or pre-training programs may have a more significant effect on fostering closer relationships between incumbents and newcomers. Previous research in HCI has demonstrated that socio-technical systems can facilitate team formation through short, structured interactions \cite{umbelino2021prototeams, lykourentzou2017team}. By orchestrating social introductions and interactions, VR could offer new strategies to address this challenge. For example, ``speed dating'' interactions could rotate team members through brief conversations, allowing them to learn about each other quickly. Additionally, team-based games requiring physical collaboration could create shared experiences that help mitigate familiarity bias. Even within the VR setting, exploring innovative methods to improve team integration and strengthen interpersonal connections between incumbents and newcomers will continue to be essential for VR adoption.

\subsection{Design Implications}

\paragraph{Altering Physical and Social Norms in VR}
While it is valuable to bring effective aspects of real-world collaboration into VR, designers should leverage the unique capabilities of VR to overcome the limitations of physical environments and strengthen relationships among team members. For example, VR allows for customizable avatars that can minimize physical appearance-based biases by focusing on shared roles or expertise rather than individual traits \cite{GatherTown, FrameVR}. Similarly, controlled non-verbal cues, such as highlighting active speakers or enabling customizable gestures, can ensure clearer communication and equitable participation, especially for individuals who might feel overshadowed in traditional settings. VR can also create blended interaction spaces, such as collaborative environments that adapt dynamically to task needs, enabling teams to move seamlessly between brainstorming, task execution, and social bonding without the constraints of a static physical location. Therefore, VR can be a potential tool that offers the opportunity to create such spaces, breaking down traditional barriers and facilitating the seamless integration of newcomers into teams. 
 
\paragraph{Integrating Kinesthetic and Interactive Features}
Our findings highlight the importance of considering richer approaches to avatar representation beyond improving facial expressions. While expressiveness is valuable and much research has focused on enhancing the realism of avatars \cite{waltemate2018impact, oh2016let}, other aspects of avatar design and interactions can be tailored specifically to address team formation needs, such as fostering inclusion and reducing familiarity bias. One design direction is enhancing the kinesthetic and interactive features of avatars. For example, incorporating haptic feedback, such as a vibration when team members ``high five'' or shake hands, can simulate physical contact and foster a sense of camaraderie. Additionally, enabling shared object manipulation, like collaboratively moving a virtual whiteboard or assembling a puzzle, can promote teamwork and a shared sense of purpose. Body tracking for natural gestures such as nodding, waving, or leaning in and spatial audio integration can further enhance the sense of presence and connection between team members \cite{abbas2023virtual, li2021social, williamson2022digital}.

\paragraph{Team Formation and Team-Oriented Avatar Customization}
Lastly, collaborative VR applications could serve as a valuable tool for supporting team formation. Masking individuals' physical appearances could enable teams to focus on building relationships and getting to know one another \cite{Whiting2020}. Team members can later transition to meeting in person or using other modalities to continue their collaboration. Moreover, avatar customization in VR applications offers an opportunity to align user representation with the specific dynamics of team collaboration. Rather than emphasizing purely individual personalization, these features could be designed to reflect team-oriented aspects, such as shared goals, complementary skills, or collaborative roles. For instance, teams can better identify each member's role and contributions by enabling users to tailor their avatars with visual markers such as professional backgrounds, representations of expertise, or elements reflecting shared interests.

\subsection{Limitations and Future Work}
While our study provides valuable insights, it is important to acknowledge its main limitations. One key limitation is the demographic homogeneity of our participants, who were primarily from a similar age group and educational background. This limitation raises questions about the generalizability of our findings to more diverse populations. Future research should aim to include a more varied participant pool to explore how different cultural and demographic factors influence team familiarity in virtual environments. Second, a larger sample size could increase the statistical validity of our study. Future research should replicate and increase the sample size to obtain results with higher significance levels. 

Third, several experimental design choices could have introduced bias to our results. For example, we did not provide an ice-breaker question to the third participant, which could have also explained the high attachment to the first partner. Despite this choice in the design of the experiment, the changes in closeness from incumbent to newcomer would have remained similar between the two experimental conditions. Another example is the short time that incumbents had to meet during this brief experiment and the short task that the team was required to do. While experiments conducted in laboratories can offer internal validity, these short-term teams have no past and no future. Future studies should examine real teams that are deciding to include new members. Additionally, longitudinal studies could provide deeper insights into how team members' closeness evolves in teams that regularly collaborate in VR settings.

Another limitation is the specific nature of the task used in our experiment, which involved decision-making based on shared information. Future studies could explore a broader range of tasks, such as creative tasks, to see how team members' closeness is impacted by a different type of task that could involve more discussion of ideas between members. Using only one VR application was also a limitation, as other applications may have offered different features and affected individuals differently. Future work should consider other VR applications as well. Finally, as VR technology continues to evolve, it will be necessary to re-evaluate how different the results could be when using more advanced and usable VR devices. Future research could explore how advancements in VR, such as improved sensory feedback and haptic technology, might further enhance or alter the dynamics of team interactions in virtual environments. 




