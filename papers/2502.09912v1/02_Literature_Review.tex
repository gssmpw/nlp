\section{Literature Review}
\label{literature_review}
We situate our work in the context of prior studies of teams---including closeness and familiarity---and VR research.

\subsection{Teams}
Teams consist of two or more individuals collaborating to achieve a shared objective \cite{salas2000teamwork}. As the foundation of organizational work, teams coordinate efforts to tackle complex tasks, demonstrating interdependencies in workflows, goals, and outcomes that reflect their collective responsibility \cite{kozlowski2006enhancing, gomez2020taxonomy, baron2003group}.

\subsubsection{Team Closeness}
Team closeness measures team members' subjective perception of how connected they feel to other members of the team \cite{Gachter2015}. Individuals can feel closer to specific team members based on their interactions and not necessarily on their physical proximity \cite{Wiese2011}. While team cohesion usually refers to how united team members are as a whole \cite{salas2015measuring}, closeness relates to relationships between pairs, which has been studied at the interpersonal level \cite{rosh2012too}. 

Researchers have identified several factors that contribute to building closeness \cite{moreland1982exposure}. Time is a key element, enabling teams to build shared knowledge and establish trust among members \cite{harrison2003time, gillespie2012factors, cattani2013tackling}. Previous relationships also play an important role, as positive prior experiences can reduce uncertainty among interactions between team members \cite{de2017attuning, jones2019essentials, dittmer2020cut}. Lastly, similarity affects closeness since team members who share common attributes such as demographics, values, or experiences are more likely to develop stronger connections and a sense of familiarity \cite{hinds2000choosing, ruef2003structure, winship2011homophily}.

\subsubsection{Team Familiarity}
Team familiarity refers to the amount of experience they have working together \cite{muskat2022team,mukherjee2019prior}. Previous research has shown that team familiarity can be a predictor of improved performance, enhancing creativity, efficiency, and the overall quality of a team's output \cite{dittmer2020cut, huckman2009team, witmer2022systematic}. For instance, Sosa demonstrated through a sociometric study that teams with stronger interpersonal connections and shared knowledge bases were significantly more likely to generate innovative solutions to complex problems \cite{sosa2011creative}. Similarly, Staats \cite{staats2012unpacking} found that increased familiarity was associated with improving team performance. Lastly, Salehi et al. \cite{10.1145/2998181.2998300} found that crowd worker teams performed better when some of their members had previously worked together. 

While its benefits have been well documented, team familiarity also produces adverse effects. It can lead to a culture where incumbents develop close networks and pose significant barriers to new members, such as exclusion \cite{choi2004minority, joardar2007experimental}, intimidation \cite{topa2016newcomers}, favoritism \cite{balthazard2006dysfunctional}, or communication barriers \cite{kraut2010dealing}. Prior work shows newcomers are often perceived as less capable or influential, limiting the team's ability to explore diverse perspectives and solutions. Moreover, high levels of team familiarity can discourage individuals from challenging established viewpoints, ultimately hindering knowledge sharing and the generation of innovative ideas \cite{assudani2011role, xie2020curvilinear}.

\subsubsection{Newcomers}
Including newcomers to a team is important for organizations. Newcomers often bring fresh ideas, innovative approaches, and valuable resources that are crucial for the ongoing vitality of the group \cite{zeng2021fresh}. However, integrating newcomers effectively into an established group poses considerable challenges. They may face strong biases, and their mere presence, even when adhering to social norms, can be perceived as disruptive by existing members \cite{kraut2010dealing, spertus2001scaling}. These issues can make the existing group less desirable for those familiar with the existing dynamics. For example, Joardar et al. \cite{joardar2007experimental} highlighted that the introduction of newcomers often triggers resistance from incumbents who may struggle to accept them as part of the group, potentially leading to tension and disruption of effective team functioning. Furthermore, cultural similarity between the newcomer and the incumbents plays a crucial role in facilitating positive group acceptance. Newcomers facing a significantly different socio-cultural environment are often less familiar with local expectations and norms, which can impede their ability to join smoothly \cite{furnham1982social}. The lack of cultural similarity can exacerbate the groups' challenges, complicating the newcomers' acceptance and diminishing their team cohesion \cite{nesdale2000immigrant}. 

\subsubsection{Online Group Communication}
The new settings of remote, hybrid, and in-person (IP) workplaces are reshaping how individuals perceive themselves and assess their relationships with co-workers. Prior research highlights that online communication systems affect individuals' perceived differences and connections within groups \cite{Hall2022,DESCHENES2024100351}. Since individuals are not physically together, systems' design factors such as nonverbal cues, facial expressions, members' representation, and synchronous conversation will influence how aware participants are of their differences and similarities \cite{giambatista2010diversity}. Consequently, online communication systems can moderate how much rich information members can have of each other, leading users to perceive less of their demographic and physical differences \cite{carte2004capabilities,SUH1999295}. Providing more information about their characteristics and differences has the risk of avoiding teaming up with others who are different or unfamiliar \cite{gomez2020impact}. 

HCI research has shown that online technologies can mitigate biases toward familiar individuals by altering the presentation of user identities \cite{maloney2020anonymity, Whiting2020, Tzlil2018}. Several online platforms offer mechanisms like pseudonyms, avatars, and anonymity to depersonalize interactions \cite{Shemla2016}. These features reduce the emphasis on personal details, enabling more equitable participation in discussions and decision-making processes, which can lead to fairer and more balanced outcomes \cite{christopherson2007positive, joinson2001self, walther1996computer}. In particular, anonymity allows users to create online personas distinct from their offline identities, fostering self-expression and experimentation while promoting inclusivity and reducing biases in collaborative and social settings \cite{bargh2002can, yurchisin2005exploration, kang2013people}. However, it may also lead to undesired negative behaviors, such as group polarization and harassment, due to the reduced sense of accountability \cite{weisband1993overcoming, Ma2016, christopherson2007positive}.

\subsection{Virtual Reality}
VR integrates advanced technologies that create immersive and interactive 3D environments \cite{wohlgenannt2020virtual, 10.1145/3613905.3651085, hubbard2024cross}. By closely emulating the dynamics of IP conversations, VR enables synchronous and embodied interactions, allowing users to engage with the virtual space and each other through natural body movements and vocal communication \cite{laato2024making, abbas2023virtual, mccloy2001science}. In recent years, many VR applications have increasingly supported collaborative interactions among multiple users, enabling them to be in the same virtual space regardless of their physical locations \cite{li2021social,10.1145/3411764.3445426}. Previous research has explored several features that enhance group interactions in VR, including the use of avatars, the sense of social presence, and the spatial configuration that VR can provide \cite{10.1145/3584931.3606992, sykownik2021most, fang2023towards}. 

Avatars provide users with the flexibility to choose how they represent themselves in VR, serving as the primary identity cue that shapes perceptions and interactions \cite{waltemate2018impact}. This representation influences user behavior through stereotypes, a phenomenon known as the ``Proteus effect'' \cite{yee2007proteus}. For instance, research in VR demonstrates that embodying an elderly avatar in immersive environments can reduce stereotypical attitudes toward older individuals \cite{yee2006walk}. Similarly, Groom et al. \cite{groom2009influence} found that using avatars of a different race in VR led to measurable shifts in racial attitudes, suggesting that virtual embodiment enables users to challenge biases through flexible and diverse self-representation \cite{banakou2016virtual}. Avatar-based interactions that mimic non-verbal cues foster these social interactions \cite{freeman2016intimate, clark1991grounding, oh2016let}. Maloney et al. \cite{maloney2020talking} found that non-verbal communication in social VR was perceived as positive and effective, as it offered a less intrusive method for initiating interactions with online strangers. Similar research by Xenakis et. al \cite{xenakis2022nonverbal} highlighted how the reduction of body signals in VR, such as facial expressions and gestures, can decrease reliance on traditional physical cues. 

Prior research has also demonstrated that social presence is a critical factor for effective collaboration in VR \cite{kimmel2023lets, sterna2021psychology, yassien2020design}. VR users can experience an enhanced sense of co-presence, which creates the illusion of sharing a virtual space with their colleagues. The immersive quality of VR helps reduce the sense of detachment often associated with remote work, fostering more engaging interactions \cite{wienrich2018social, smith2018communication} and driving users' intentions to collaborate \cite{mutterlein2018specifics}.

Lastly, the spatial configuration of VR environments impacts how team members recognize and interact with each other. Proxemics, or spatial behavior, refers to the measurable distances between people as they communicate, influencing interpersonal dynamics \cite{hans2015kinesics}. In a virtual setting, virtual proximity replicates this effect, enabling team members to experience a shared presence in a virtual room, even if they are physically distant \cite{williamson2021proxemics}. This shared virtual space fosters a sense of connection and engagement, helping users recreate the interpersonal dynamics, such as proximity and orientation, that are important for effective collaboration \cite{li2021social, williamson2022digital}. 

VR environments influence individuals' perceptions and behaviors during collaboration by abstracting physical features and reducing appearance-based cues that often highlight differences among team members. This shift redirects attention toward collaborative interactions, potentially diminishing social biases and the perceived distinction between incumbents and newcomers. Building on this reasoning, we propose that the mode of interaction affects both how incumbents perceive closeness to newcomers and how newcomers perceive closeness to incumbents. Our first two hypotheses are:

\begin{quote}
    \textbf{H1}: \textit{When onboarding new members, incumbents will perceive greater closeness to newcomers when meeting in VR than when meeting in person.}
\end{quote}
\begin{quote}
    \textbf{H2}: \textit{When onboarding new members, newcomers will perceive greater closeness to incumbents when meeting in VR than when meeting in person.}
\end{quote}

When individuals unfamiliar with one another come together to assemble a team, they initially have limited information to assess each other or establish expectations for their interactions. To alleviate uncertainty, team members often rely on readily available cues, such as physical appearance or demographic traits, to quickly form impressions \cite{allport1954nature,hinds2000choosing}. However, these initial judgments can reinforce social biases, favoring familiar individuals or strengthening preexisting group distinctions. Prior research suggests that VR can reduce such biases through depersonalization and flexible identity representation. By shifting attention away from physical and cultural differences as well as preexisting relationships, VR can foster a more inclusive environment that reduces perceived distinctions between incumbents and newcomers. Therefore, we hypothesize:
\begin{quote}
    \textbf{H3}: \textit{Incumbents' familiarity bias against newcomers will be lower when meeting in VR than when meeting in person.}
\end{quote}

Lastly, one mechanism that could explain how VR fosters closeness is \textit{perceived similarity}---team members who see themselves as more alike tend to experience higher trust and a stronger sense of bonding \cite{zellmer2008and, mannix2005differences}. In VR, avatars and immersive environments can attenuate physical or demographic differences \cite{lopez2019investigating, marini2022can}, potentially making team members feel more similar and accentuating shared goals. This reduction in perceived differences may enhance perceptions of similarity, ultimately fostering closeness. Thus, we hypothesize:
\begin{quote}
    \textbf{H4}: \textit{Perceived similarity will mediate the relationship between VR (versus in-person) and team closeness, such that using VR increases perceived similarity among team members, thereby enhancing closeness.}
\end{quote}
