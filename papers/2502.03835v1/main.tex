% CVPR 2025 Paper Template; see https://github.com/cvpr-org/author-kit

\documentclass[10pt,twocolumn,letterpaper]{article}

%%%%%%%%% PAPER TYPE  - PLEASE UPDATE FOR FINAL VERSION
% \usepackage{cvpr}              % To produce the CAMERA-READY version
% \usepackage[review]{cvpr}      % To produce the REVIEW version
\usepackage[pagenumbers]{cvpr} % To force page numbers, e.g. for an arXiv version
\usepackage{multirow} % 包含 multirow 宏包
% Import additional packages in the preamble file, before hyperref
\newcommand{\CG}{\mathcal{G}\xspace}
\newcommand{\CV}{\mathcal{V}\xspace}
\newcommand{\CE}{\mathcal{E}\xspace}
\newcommand{\CA}{\mathcal{A}\xspace}
\newcommand{\CF}{\mathcal{F}\xspace}
\newcommand{\CR}{\mathcal{R}\xspace}
\newcommand{\CB}{\mathcal{B}\xspace}
\newcommand{\CX}{\mathcal{X}\xspace}
\newcommand{\CK}{\mathcal{K}\xspace}
\newcommand{\CM}{\mathcal{M}\xspace}
\newcommand{\CC}{\mathcal{C}\xspace}
\newcommand{\CL}{\mathcal{L}\xspace}
\newcommand{\CI}{\mathcal{I}\xspace}
\newcommand{\CQ}{\mathcal{Q}\xspace}
\newcommand{\CO}{\mathcal{O}\xspace}
\newcommand{\CP}{\mathcal{P}\xspace}
\newcommand{\CS}{\mathcal{S}\xspace}
\newcommand{\CT}{\mathcal{T}\xspace}
\newcommand{\CJ}{\mathcal{J}\xspace}
\usepackage[para]{footmisc}
\usepackage{subfig}
% \usepackage{subcaption}
% \usepackage{array}
% \usepackage{colortbl}



% It is strongly recommended to use hyperref, especially for the review version.
% hyperref with option pagebackref eases the reviewers' job.
% Please disable hyperref *only* if you encounter grave issues, 
% e.g. with the file validation for the camera-ready version.
%
% If you comment hyperref and then uncomment it, you should delete *.aux before re-running LaTeX.
% (Or just hit 'q' on the first LaTeX run, let it finish, and you should be clear).
\definecolor{cvprblue}{rgb}{0.21,0.49,0.74}
\usepackage[pagebackref,breaklinks,colorlinks,allcolors=cvprblue]{hyperref}

%%%%%%%%% PAPER ID  - PLEASE UPDATE
\def\paperID{9969} % *** Enter the Paper ID here
\def\confName{CVPR}
\def\confYear{2025}

%%%%%%%%% TITLE - PLEASE UPDATE
\title{Single-Domain Generalized Object Detection by Balancing Domain Diversity and Invariance}

%%%%%%%%% AUTHORS - PLEASE UPDATE
\author{Zhenwei He, Hongsu Ni\\
Chongqing University of Technology\\
{\tt\small hzw@cqut.edu.cn, nhs2@stu.cqut.edu.cn }}
% For a paper whose authors are all at the same institution,
% omit the following lines up until the closing ``}''.
% Additional authors and addresses can be added with ``\and'',
% just like the second author.
% To save space, use either the email address or home page, not both
% \and
% Second Author\\
% Institution2\\
% First line of institution2 address\\
% {\tt\small secondauthor@i2.org}
% }

\begin{document}
\maketitle
% \begin{abstract}

% Recent works to jointly reconstruct 3D human and object from a single RGB image, are mostly model-based, that fail to capture the fine details of the clothed human body and object surface. In this paper, we introduce ReCHOR, a novel, model-free, first-method to produce realistic clothed human-object reconstructions from a monocular view. This is extremely challenging due to human-object occlusions, diverse interactions and depth ambiguity, as it needs to infer both 3D spatial awareness and high resolution details. Our core idea is based on estimating neural implicit representations for human and object respectively by an attention-based neural implicit model that attends to pixel-aligned features from both the global human-object image for spatial awareness and  the local separate view of human and object images for high quality details. Additionally, the network is conditioned on semantic features from an initial estimated human-object pose prior and a generative diffusion model that inpaints occluded regions, thus enabling the retrieval of details from them.
% We also propose a synthetic dataset with rendered scenes of diverse, inter-occluded 3D human and object scans, to train our network. We evaluate our method on the synthetic and real world BEHAVE dataset. Our experiments show that our method outperforms the SOTA in achieving realistic clothed human-object reconstructions.
Recent approaches to jointly reconstruct 3D humans and objects from a single RGB image represent 3D shapes with template-based or coarse models, which fail to capture details of loose clothing on human bodies. In this paper, we introduce a novel implicit approach for jointly reconstructing realistic 3D clothed humans and objects from a monocular view. For the first time, we model both the human and the object with an implicit representation, allowing to capture more realistic details such as clothing. This task is extremely challenging due to human-object occlusions and the lack of 3D information in 2D images, often leading to poor detail reconstruction and depth ambiguity. To address these problems, we propose a novel attention-based neural implicit model that leverages image pixel alignment from both the input human-object image for a global understanding of the human-object scene and from local separate views of the human and object images to improve realism with, for example, clothing details. Additionally, the network is conditioned on semantic features derived from an estimated human-object pose prior, which provides 3D spatial information about the shared space of humans and objects. To handle human occlusion caused by objects, we use a generative diffusion model that inpaints the occluded regions, recovering otherwise lost details. For training and evaluation, we introduce a synthetic dataset featuring rendered scenes of inter-occluded 3D human scans and diverse objects. Extensive evaluation on both synthetic and real-world datasets demonstrates the superior quality of the proposed human-object reconstructions over competitive methods.
\end{abstract}    
% \section{Introduction}
\label{sec:intro}
% Image editing methods in diffusion models depend on user-defined control directions - users can unlock their creativity using these methods by specifying the desired manipulation through prompts~\cite{gandikota2023concept}, reference images~\cite{ruiz2022dreambooth, kumari2022customdiffusion, gal2022image, chen2024trainingfreeregionalpromptingdiffusion}, or attribute vectors~\cite{parmar2023zero,hertz2022prompt}. In this work, we ask a fundamentally different question: \emph{Can we automatically discover the underlying visual structure of a concept within diffusion model's knowledge?} %Rather than requiring user-specified controls, we aim to decompose the model's internal knowledge into meaningful directions.

% This question touches on a fundamental limitation in how we interact with diffusion models. Current control methods ~\cite{zhang2023addingconditionalcontroltexttoimage, gandikota2023concept, ye2023ipadaptertextcompatibleimage,ye2023ipadaptertextcompatibleimage, hertz2024stylealignedimagegeneration, li2023photomaker, shi2024instantbooth, chen2024trainingfreeregionalpromptingdiffusion} require users to specify their desired manipulations in advance, limiting interactive creativity. This contrasts with natural human artistic workflows, where creators dynamically explore creative ideas while jointly refining them toward meaningful artistic outcomes~\cite{hoffmann2016modeling}. This synergy between specification and exploration is not new to generative models. Early GAN architectures naturally developed disentangled latent spaces that enabled continuous\cite{harkonen2020ganspace,radford2015unsupervised, wu2021stylespace, shen2020interfacegan}, compositional control over generated images. Users could explore these spaces to discover interesting variations that would be difficult to describe in words~\cite{wu2021stylespace}, then combine them to achieve their creative goals~\cite{grabe2022towards}. 


% While diffusion models have largely superseded GANs in conditional image synthesis~\cite{dhariwal2021diffusion},  their underlying structure remains less understood. Diffusion models achieve remarkable diversity through high-dimensional latents, unlike GANs' compact latent spaces.  With a single prompt, diffusion models can generate radically different variations through different random initializations of input noise. We ask - Is it possible to discover interpretable structure within this vast space of variations?

Text-to-image diffusion models are capable of generating remarkable visual variations from a single prompt through different random initializations. However, this vast creative potential remains largely opaque to users---while we can generate diverse images, we lack understanding of the underlying structure of these variations. This presents a fundamental challenge: how can we discover and expose the latent visual capabilities encoded within these models?

\let\thefootnote\relax \footnote{$^{*}$Correspondence to \texttt{gandikota.ro@northeastern.edu}}

The challenge touches on a key limitation in how we interact with diffusion models today. Current control methods require users to explicitly specify their desired edits in advance through prompts~\cite{gandikota2023concept}, reference images~\cite{zhang2023addingconditionalcontroltexttoimage, chen2024trainingfreeregionalpromptingdiffusion, ruiz2022dreambooth,kumari2022customdiffusion, Ryu_lora, hu2021lora}, or attribute vectors~\cite{ye2023ipadaptertextcompatibleimage, hertz2024stylealignedimagegeneration, li2023photomaker, shi2024instantbooth,parmar2023zero,hertz2022prompt}. That contrasts sharply with natural human creative workflows, where artists dynamically explore creative ideas and jointly refine them toward meaningful artistic outcomes~\cite{hoffmann2016modeling}. The need for pre-specified controls creates a barrier between users and the full creative potential of these models.

Interestingly, earlier generative models like GANs~\cite{gans,karras2019style,brock2018large} naturally developed more interpretable internal structures. Their compact latent spaces often exhibited emergent disentanglement~\cite{harkonen2020ganspace,radford2015unsupervised, wu2021stylespace, shen2020interfacegan}, enabling continuous and compositional control over generated images. Users could explore these spaces to discover interesting variations that would be difficult to describe in words~\cite{wu2021stylespace}, then combine them to achieve their creative goals~\cite{grabe2022towards}.

Diffusion models have largely superseded GANs in conditional image synthesis~\cite{dhariwal2021diffusion}, achieving greater diversity through much higher-dimensional latents. And yet an understanding of the underlying structure of these larger latent spaces has remained elusive. In this work, we ask a fundamental question: \emph{Can we automatically discover the visual structure within a diffusion model's knowledge of a concept?} Rather than requiring user-specified controls, we aim to decompose the model's internal representations into expressive directions that users can explore and combine.

To address these needs, we present \textbf{SliderSpace}, a framework that brings systematic explorability to diffusion models. Given just a text prompt, SliderSpace discovers a canonical set of meaningful, diverse, and controllable directions within the model's knowledge of that concept. Each direction is implemented as a low-rank adapter~\cite{hu2021lora} that can be scaled and composed with others, allowing users to explore and smoothly combine different aspects of variation, as shown in Figure~\ref{fig:intro}.

We ground SliderSpace discovery in three key requirements for meaningful decomposition of a diffusion model's visual manifold: 
\begin{enumerate}
    \item \textbf{Unsupervised Discovery:} The decomposition process should emerge from the intrinsic structure of the model's learned representation, rather than being guided by predefined attributes. This ensures we capture the true topology of the model's knowledge space rather than projecting our assumptions onto it.
    
    \item \textbf{Semantic Orthogonality:} Each discovered control must represent a distinct semantic direction. This is enforced in a semantic feature space, like CLIP, where every slider has an orthogonal effect in embeddings. This prevents discovering multiple controls that create similar semantic effects, making the system more efficient and easier.
    
    \item \textbf{Distribution Consistency:} Directions must induce consistent transformations across both random seeds and prompt variations. 
\end{enumerate}

These requirements naturally lead to our proposed framework, which we formalize in Section~\ref{sec:method}. As we show in our experiments, SliderSpace is architecture-agnostic, working with both conventional U-Net based models like Stable Diffusion~\cite{rombach2022high, rombach2022sd20, podell2023sdxl, turbo, dmd} and recent transformer-based architectures like Flux~\cite{flux}.

We demonstrate the expressiveness of SliderSpace through three applications: First, we show how SliderSpace can decompose high-level concepts into diverse and expressive components, revealing the natural axes of variation in the model's understanding. Second, we explore artistic style variation, where SliderSpace discovers directions that match or exceed the diversity of manually curated artist lists while being judged more useful by human evaluators. Finally, we show how SliderSpace can help reverse the mode collapse commonly observed in distilled diffusion models, restoring diversity while maintaining generation speed.

Beyond providing practical creative control, SliderSpace opens new avenues for understanding and utilizing the latent capabilities of diffusion models. By mapping these models' visual potential into intuitive, composable directions, we take a step toward making their creative possibilities more accessible and interpretable to users.

% Image editing methods in diffusion models unlock the creativity of users. In this work we ask an alternate question: \emph{Can we organize and expose what of the diffusion model is already capable of?}.
% Existing methods for controlling image generation typically require users to manually specify edit directions for desired changes. This process is time-consuming, requires technical expertise, and limits the spontaneity of the creative process. For instance, if a user wants to adjust the smile of a generated person, they must explicitly request this edit, often through imprecise prompt engineering or model fine-tuning. This approach of predefined controls or manual specifications restricts users from fully exploring the latent capabilities of the model. There may be interesting stylistic variations or attributes that the model can generate, but users have no easy way to discover or utilize these.

% Natural visual disentanglement was an emergent property in the latent space of Generative Adversarial Models (GANs) \cite{harkonen2020ganspace,radford2015unsupervised, wu2021stylespace, shen2020interfacegan}. In particular, it has been observed that StyleGAN~\cite{karras2019style} stylespace neurons offer detailed control over many meaningful aspects of images that would be difficult to describe in words~\cite{wu2021stylespace}. However, diffusion models do not share such a compact latent space~\cite{park2023unsupervised}; and efforts to uncover such a space in the semantic embeddings of the text conditioning have met with limited success \nik{Nick - is there a specific citation you were thinking about?}.

% In this work we introduce \textbf{SliderSpace}, which takes a step towards uncovering an analogous low dimensional representation of diffusion models' visual breadth; in essence treating the diffusion model as many generators sharing parameters, where a particular generator is defined by a specific prompt. For a given prompt we sample many random seeds (and optionally prompt expansions using an LLM), generate the corresponding images, and apply an off the shelf feature extractor (in this work CLIP, but our method can be applied to any differentiable feature extractor). We use PCA to analyze these features, and for each of the leading $k$ principal components we train a LoRA \cite{} which causes the diffusion model to produces images which increase the feature magnitude along that component when passed back through the same feature extractor. This leads to a 'Slider' for each principal component, because each LoRA can be scaled and applied to the original diffusion model, continuously varying those visual features in the generated results (as measured, in our case, by CLIP).

% There are many other works that enhance the controllability of diffusion models. One common approach is enabling users to add spatial constraints to a generation either manually, or via a reference image \cite{zhang2023addingconditionalcontroltexttoimage, chen2024trainingfreeregionalpromptingdiffusion}, a second is leveraging more abstract embeddings (e.g. identity, style) extracted from a reference image \cite{ye2023ipadaptertextcompatibleimage, hertz2024stylealignedimagegeneration, li2023photomaker, shi2024instantbooth}, a third is finetuning a foundation model to better generate a concept important to the user \cite{ruiz2022dreambooth, kumari2022customdiffusion, Ryu_lora, hu2021lora}, and a fourth (most relevant to this work) is finding low-rank adaptors of the model based on a prompt or small training set which can be scaled to provide continous control over one aspect of generated image (e.g. night vs day, basic vs luxury, etc.) \cite{gandikota2023concept}. SliderSpace is complementary to all of these methods and offers something distinct. All of the other methods we are aware require the user (and / or model designer) to know in advance what type of control they want. In contrast SliderSpace assists users in discovering and controlling hidden capabilities present in the diffusion model's distribution of possible generations.

%We propose that truly intuitive creative control in a text-to-image model should meet three key criteria: \emph{discoverability}, \emph{intuitiveness}, and \emph{specificity}. The model should reveal controllable attributes that may not be immediately obvious, offer controls that are easy to understand and manipulate, and ensure each control affects a distinct attribute of the generated image.

% We demonstrate the utility and power of SliderSpace using three applications built on top of SDXL-DMD \cite{dmd}, because its fast generation speed lends itself well to the continuous control offered by SliderSpace.

% First, we study concept decomposition (Section \ref{sec:concept_exp}), where we learn sliders for a specific concept (e.g. 'monster', 'waterfall', 'car'). Through quantitative metrics of diversity and text alignment we demonstrate that the learned sliders dramatically boost the diversity of generations when randomly applied without harming text alignment; we also ask humans to qualitatively judge these results in a user study where they find the SliderSpace results to be more 'Diverse', 'Useful', and 'Creative' than our baselines.

% Second, we attempt to compare the automatic discoveries of SliderSpace to a large scale manual study of artistic styles (Section \ref{sec:art_exp}), open-sourced by ParrotZone \cite{parrotzone}. In this study SDXL was prompted with over 4300 artist names,  and based on visual inspection the cases of successful stylistic mimicry recorded. Quantitatively SliderSpace more closely matches the distribution of artistic variation discovered by ParrotZone than other baselines, and in our user studies was judged to be significantly more 'Diverse' and 'Useful' than the baselines. To our surprise humans even judged SliderSpace results to be slightly more 'Diverse' than the results generated by the manually discovered artist names of \cite{parrotzone}.

% Third, we attempt to use SliderSpace to reverse the mode collapse commonly observed in distilled few-step diffusion models relative to the original teacher model (Section \ref{sec:diverse_exp}). We quantitatively demonstrate that applying SliderSpace to SDXL-DMD leads to more closely matching the distribution of images by the original teacher, SDXL.

%Through extensive experiments on various state-of-the-art text-to-image models, we demonstrate that SliderSpace significantly enhances user control and creative expression in AI-assisted image generation tasks. Our method enables a range of applications, including concept decomposition and control, diversity improvement in generated images, customization dissection and edits, and the exploration of artistic styles inherent in the model.

% SliderSpace goes beyond providing a practical tool for enhanced creative control. By mapping the visual potential of diffusion models it can open new avenues for generative creativity and deepens our understanding of each model's hidden potential.
% \section{Related work}
\label{sec:formatting}

\subsection{Text-to-Video Generation}

T2V generation has made notable progress, evolving from early GAN-based models \cite{saito2017temporal,tulyakov2018mocogan,fu2023tell,li2018video,wu2022nuwa,yu2022generating} to newer transformer \cite{yan2021videogpt,arnab2021vivit,esser2021taming,ramesh2021zero,yu2022scaling} and diffusion models \cite{kirkpatrick2017overcoming,sohl2015deep,song2020denoising,zhang2022gddim}. Early efforts like MoCoGAN~\cite{tulyakov2018mocogan} focused on short video clips but faced issues with stability and coherence. The introduction of transformers improved sequential data handling, enhancing video generation, while diffusion models further improved video quality by progressively denoising the input. 
Despite these advances, T2V models still struggle to reflect human preferences, with the generated videos generally lacking aesthetic quality. Additionally, the scarcity of paired video preference data hinders effective model training and may lead to insufficient flexibility and poor quality in the generated videos.


\subsection{RLHF}

\iffalse
Aligning LLMs \cite{dai1901transformer,radford2019language,zhang2023opt} typically involves two steps: supervised fine-tuning followed by Reinforcement Learning with Human Feedback (RLHF) \cite{gao2023scaling,stiennon2020learning,rafailov2024direct}. Although effective, RLHF is computationally expensive and can lead to issues like reward hacking. Methods like DPO have streamlined alignment by leveraging feedback data directly, improving efficiency.

In contrast, diffusion model alignment is still evolving, focusing mainly on enhancing visual quality through curated datasets. Techniques like DOODL \cite{wallace2023end} and AlignProp \cite{prabhudesai2023aligning} target aesthetic improvements but face challenges with complex tasks such as text-image alignment. Reinforcement learning methods like DPOK \cite{fan2024reinforcement} and DDPO \cite{black2023training} improve reward optimization but struggle with scalability. DPO-SDXL integrates DPO into T2I generation, boosting both alignment and aesthetics.

However, aligning video generation remains a largely unaddressed challenge, especially when dealing with motion consistency and semantic coherence across frames.
\fi

RLHF \cite{gao2023scaling,stiennon2020learning,rafailov2024direct} is a method that utilizes human feedback to guide machine learning models. Early RLHF algorithms, such as DDPG~\cite{lillicrap2015continuous} and PPO~\cite{schulman2017proximal}, typically relied on complex reward models to quantify human feedback. These reward models require a large amount of annotated data and face challenges during tuning. As research has progressed, more efficient preference learning methods have emerged, among which DPO has become a new framework. DPO does not depend on a separate reward model; instead, it obtains human preferences through pairwise comparisons and directly optimizes these preferences. This shift not only simplifies the application of RLHF but also enhances the alignment of models with human values. Furthermore, DPO has been successfully introduced into T2I tasks~\cite{wallace2024diffusion,yang2024using}, providing new insights for generative models in addressing the alignment of human preferences and showcasing DPO's potential in the field of AIGC~\cite{shi2024instantbooth,
qing2024hierarchical,menapace2024snap,koley2024s}. However, there remains a gap in current research regarding the application of DPO strategies to T2V tasks. Effectively integrating DPO into T2V tasks presents a challenging endeavor.


% \section{Preliminary}
\label{sec:preliminary}
In this section, we first introduce the mathematical formulation of flow-based text-to-image generative models~\cite{Xingchao_2022,Lipman_2022}, which forms the foundation of modern T2I systems~\cite{sd3,sdxl,imagen3,imagen}. We then describe classifier-free guidance~\cite{ho2022classifier}, a key technique to control the generation process through text conditioning.

\subsection{Flow-based text-to-image generative models}
In state-of-the-art T2I models~\cite{sd3}, the image generation process is modeled by learning, through a neural network, a flow $\psi$ that generates a probability path $(p_t)_{0\le t\le 1}$ bridging the source distribution $p_0$ and the target distribution $p_1$ ~\cite{Xingchao_2022,Lipman_2022}. This framework encompasses diffusion models~\cite{sohl2015deep,ddpm} as a special case. In particular, a commonly used formulation sets a Gaussian distribution as the source: $p_0 = \mathcal{N}(\mathbf{0}, \mathbf{I})$ and a delta distribution centered on a sample $\mathbf{x}_1$ from the data distribution $q$ as the target: $p_1 = \delta_{\mathbf{x}_1}$.
Then, it incorporates an affine conditional flow $\psi_t(\mathbf{x} | \mathbf{x}_1) = a_t \mathbf{x}_1 + b_t \mathbf{x}$ with the boundary condition $(a_0, b_0) = (0, 1),\ (a_1, b_1) = (1, 0)$ to bridge them. The neural network typically approximates quantities such as velocity fields, $x_0$ prediction or $x_1$ prediction. In this modeling, these quantities can be viewed as affine transformations of the marginal probability path score $\nabla_{\mathbf{x}} \log p_t(\mathbf{x})$.

\subsection{Classifier-free guidance in flow-based models}
Classifier-free guidance~\cite{ho2022classifier} is a method for sampling from a model conditioned by a text input $\mathbf{y}$ by guiding an unconditional image generation model modeled using the score $\nabla_{\mathbf{x}} \log p_t(\mathbf{x})$. This enables the sampling from
\[
q_w(\mathbf{x}, \mathbf{y}) \propto q(\mathbf{x})q(\mathbf{y}|\mathbf{x})^w \propto q(\mathbf{x})^{1-w}q(\mathbf{x}|\mathbf{y})^w
\]
where $w \in \mathbb{R}$ is the guidance scale typically used with $w > 1$. The score satisfies
\[
\nabla_{\mathbf{x}} \log q_w(\mathbf{x}, \mathbf{y}) = (1-w)\nabla_{\mathbf{x}} \log q(\mathbf{x}) + w\nabla_{\mathbf{x}} \log q(\mathbf{x}|\mathbf{y})
\]
so by training the network to learn both the unconditional score $\nabla_{\mathbf{x}} \log q(\mathbf{x})$ and conditional score $\nabla_{\mathbf{x}} \log q(\mathbf{x}|\mathbf{y})$, flexible sampling from the conditional distribution can be achieved through a weighted sum of the network outputs.

\begin{abstract}
Single-domain generalization for object detection (S-DGOD) aims to transfer knowledge from a single source domain to unseen target domains. In recent years, many models have focused primarily on achieving feature invariance to enhance robustness. However, due to the inherent diversity across domains, an excessive emphasis on invariance can cause the model to overlook the actual differences between images. This overemphasis may complicate the training process and lead to a loss of valuable information. To address this issue, we propose the Diversity Invariance Detection Model (DIDM), which focuses on the balance between the diversity of domain-specific and invariance cross domains. Recognizing that domain diversity introduces variations in domain-specific features, we introduce a Diversity Learning Module (DLM). The DLM is designed to preserve the diversity of domain-specific information with proposed feature diversity loss while limiting the category semantics in the features. In addition, to maintain domain invariance, we incorporate a Weighted Aligning Module (WAM), which aligns features without compromising feature diversity. We conducted our model on five distinct datasets, which have illustrated the superior performance and effectiveness of the proposed model.
\end{abstract}

\section{Introduction}
\label{sec:intro}

In recent years, deep learning has led to significant improvements in the performance of detection models. However, traditional detectors often rely on the assumption of Independent and Identically Distributed (i.i.d.) data. This assumption becomes a limitation when the model encounters a domain shift, leading to a significant drop in performance and restricting the applicability of these models in real-world scenarios.
\begin{figure}
    \centering
    \includegraphics[width=1\linewidth]{72.png}
    \caption{A brief overview of DIDM is presented. The DLM is designed for the domain-specific feature to preserve their diversity and suppress the semantic information. The WAM is employed to align features without compromising the feature diversity with loss weight.}
    \label{fig:enter-label1}
\end{figure}

To address the domain shift problem, domain generalization (DG) has been developed to transfer knowledge from the source domains to the unseen target domain \cite{dou2019domain, liu2024unbiased, zhou2022domain,bi2024learning}. These methods enhance model robustness, making it more robust to more application scenarios. Recently, single-domain generalization \cite{liu2020towards,wu2022single,wu2024g,lee2024object,vidit2023clip} has gained significant attention as it reduces data requirements by training the model on a single domain, which is a much more challenging task compared to the traditional DG.

Most existing single-domain generalization models aim to reduce domain bias by learning domain-invariant features, which often restrict the model and prevent the capture of domain-specific information. This approach requires the model to extract similar, or even identical, features from images across different domains. Considering significant domain differences, such a constraint is counterintuitive and can limit effective feature extraction, potentially complicating the training process. To address this, we proposed that a domain generalization (DG) model should not only consider the learning of domain invariance but also the diversity of the domain-specific information. The learning of domain invariance ensures cross-domain robustness, while the diversity of domain-specific features enhances feature learning and model training.

Based on the above discussion, specifically, in this paper, we introduce the Diversity Invariance Domain Generalization Detection Model (DIDM), designed to not only learn the domain invariance but also keep the diversity domain-specific. To manage these two feature types, we incorporate the Diversity Learning Module (DLM) to capture domain-specific features and the Weighted Aligning Module (WAM) to focus on domain-invariant features, as illustrated in \cref{fig:enter-label1}. 

The Diversity Learning Module (DLM) is designed to preserve the domain diversity and limit the semantic limitation. Firstly, to eliminate the semantic information among the domain-specific features, we apply an Entropy Maximization loss to reduce category-level semantics. However, focusing solely on constraining domain-specific information may limit the model's ability to learn diverse features. To address this, we introduce a feature diversity loss (FD) to reinforce the diversity of domain-specific information. Finally, the DLM offers three key advantages: first, by reducing category semantics in domain-specific features, it supports domain-invariant training; second, it preserves feature diversity across domains, enhancing the training phase of the model. Additionally, by encouraging feature diversity with FD loss, the module compels the model to capture richer semantic information from images, resulting in more semantically nuanced features.

In addition to domain-specific feature learning with DLM, domain invariance learning may also influence the feature diversity in the Domain Generalization (DG) model. Thus, we introduce the Weighted Aligning Module (WAM) to alleviate diversity loss with feature alignment. The proposed WAM is designed with two main objectives: First, by feature alignment across domains, it enables the model to learn domain-invariant features for the detection task; Second, by incorporating loss weights, the model prevents the overemphasizes the feature alignment. When the feature differences do not influence the detection, the model down-weight the alignment to preserve the domain diversity. By combining both the DLM and WAM, our model achieves effective domain generalization on the detection task. In conclusion, the contribution of the proposed DIDM is presented as follows:

\begin{itemize}
\item We introduce the Diversity Learning Module (DLM) to preserve the feature diversity while limiting the semantic information within domain-specific features. The feature diversity loss is proposed to enhance the feature diversity.

\item The Weighted Aligning Module (WAM) is designed to align features without influencing the feature diversity. The loss weight is proposed to prevent an overemphasis on domain-specific features.

\item Experiments were conducted under five distinct weather conditions, effectively showcasing the robustness of our methodology.
\end{itemize}

%-------------------------------------------------------------------------
%------------------------------------------------------------------------
\begin{figure*}
    \centering
    \includegraphics[width=1\linewidth]{67.png}
    \caption{The general architecture of DIDM. The input image is processed through the backbone to extract both original and enhanced features, as well as to obtain the objective domain-specific feature. Subsequently, this feature is processed through the Diversity Learning Module (DLM) and the Weighted Aligning Module (WAM). DLM is implemented for domain-specific features with maximum entropy loss and feature diversity loss, where the diversity is enhanced and semantic information is suppressed. Weighted Aligning Module (WAM) imposes a constraint on feature alignment, which prevents the model's overemphasis on the alignment and compromises diversity learning. Together, the DLM and WAM work synergistically to ensure that the model focuses on robust features.}
    \label{fig:enter-label2}
\end{figure*}


\section{Related Work}
\label{sec:Work}

\textbf{Domain Generalization (DG).} Researchers have been working extensively in single domain generalization and have developed numerous methods to enhance it \cite{dou2019domain, liu2024unbiased, li2024prompt, wu2024g, zhou2022domain}. These methods can generally be grouped into three main categories: First, the data argumentation approach \cite{kang2022style,lee2022wildnet,li2023deep,somavarapu2020frustratingly} increases the diversity of source domain data by applying various data manipulation techniques. For instance, Somavarapu et al. \cite{somavarapu2020frustratingly} introduce a straightforward image stylization transform to generate diverse samples, exploring variability across sources. Similarly, Wang et al. \cite{wang2021learning} employ adversarial training to create varied input images, further enhancing generalization. Second, some kinds of DG methods focus on the presentation learning of domain invariance. Shao et al. \cite{shao2019multi} use multi-adversarial discriminative training to extract both shared and distinctive feature representations across multiple source domains. Last, Various learning strategies \cite{chen2023meta,du2020learning,peng2022semantic,seo2020learning,wang2023improving,zhao2021learning} are employed to improve model generalization. For example, Zhao et al. \cite{zhao2021learning} apply a meta-learning approach that simulates the model’s adaptation to new, unseen domains during training, thereby boosting its adaptability in unfamiliar environments. These approaches illustrate significant advancements in domain generalization, highlighting different techniques designed to improve models' performance on previously unseen domains.

\textbf{Object Detection.} 
To mitigate the impact of domain bias on model performance, researchers have conducted numerous experiments and proposed various methods in object detection. One prominent approach is Domain Adaptive Object Detection (DAOD) \cite{chen2021i3net,deng2021unbiased,krishna2023mila,li2022cross,oza2023unsupervised,zhao2020review}, which enhances the model's robustness in new domains by simultaneously training on both source and target domains. For instance, a common strategy involves minimizing the distance between global and local features, as outlined in \cite{cao2023contrastive,chen2020harmonizing,deng2021unbiased,saito2019strong}. However, these methods require access to the target domain during model training, which imposes limitations on practical applications. Therefore, researchers have proposed Domain Generalized Object Detection (DGOD) \cite{chang2024unified,qin2024towards}. For instance, Chang et al. \cite{chang2024unified} enhance the model's generalization ability by decoupling depth estimation from dynamic perspective enhancement. However, the effectiveness of DGOD largely depends on the number of accessible source domains, and collecting multiple source domains can be costly. Consequently, researchers have introduced the more challenging Single Domain Generalized Object Detection (S-DGOD) \cite{vidit2023clip,danish2024improving,lee2024object,pan2018two,choi2021robustnet}, which is generally categorized into feature normalization and invariant-based methods. Firstly, feature normalization methods such as IBN-Net, proposed by Pan et al. \cite{pan2018two}, enable the network to adjust its normalization strategy according to different tasks and datasets by combining Instance Normalization (IN) and Batch Normalization (BN). IterNorm, introduced by Huang et al. \cite{huang2019iterative}, avoids feature decomposition through a Newton iteration method, thereby improving the efficiency of the normalization process. Secondly, invariant-based approaches, such as UFR proposed by Liu et al. \cite{liu2024unbiased}, enhance the model's generalization performance by eliminating prototype bias and attentional bias. Wu et al. \cite{wu2022single} propose a cyclic disentanglement self-distillation approach specifically for single-domain generalization in object detection, which enhances feature disentanglement. These approaches demonstrate significant progress in the field of target detection and contribute to the rapid advancement of the discipline.

%-------------------------------------------------------------------------
\section{Methodology}
\label{sec:method}
In this section, we introduce the proposed Diversity Invariance Detection Model (DIDM), which has two main components: the Diversity Learning Module (DLM) and the Weighted Aligning Module (WAM). These components work together to balance domain-invariant feature learning and feature diversity. Specifically, the DLM is designed to preserve feature diversity and restricting semantic information for domain-specific features, while the WAM aligns features across domains to enhance domain invariance.

\subsection{Preliminaries}

\textbf{Problem Description.} 
For single-domain generalization detection tasks, the model is trained using one source domain $D_s=\{{(x^i_s,y^i_s,b^i_s)}\}^{N_s}_{i=1}$, where $N_s$ stand for the number of samples,  $x^i_s$ presents the input image, $b^i_s$ and $y^i_s$ are the ground truth bounding boxes and corresponding labels, respectively. The model is trained for the detection of the unseen target domain $D_t=\{{(x^i_t)}\}^{N_t}_{i=1}$. Note that the source and target domains share the same label space.

\textbf{Data Augmented.} 
For the single-domain generalization task, training a model on single source dataset may lead to overfitting to domain-specific styles and features. Besides, obtaining diverse datasets across multiple domains is often costly. To address this, we apply data augmentation techniques \cite{bachman2014learning,hsu2020every,laine2016temporal} to increase the diversity of the source domain images, thereby enhancing the model's adaptability and robustness to variations across different domains. These augmentations include techniques such as noise addition, blurring, and random cropping. Each source image $x_s$ is transformed through these augmentation functions $A(.)$ to produce the enhanced image $x_a=A(x_s)$.

\subsection{Overview}

The net structure of the proposed model is shown in \cref{fig:enter-label2}, where Faster R-CNN \cite{ren2016faster} is used as the base detector. Given a source image $x_s$, we can generate an argument image $x_a=A(x_s)$. During the training phase, both $x_s$ and $x_a$ are fed into the model to produce their corresponding feature maps $F_s$ and $F_a$, respectively. Only the $F_s$ is fed into RPN to get a series of proposals $P \in R^{K \times 4}$. Then, the $F_s$, $F_a$ and $P$ are fed into the RoI-Pooling layer to get the features of proposals for the original and argument image: $z_{s}=RP(F_{s},P)$, $z_{a}=RP(F_{a},P)$.

After that, the pooled feature $z_{s}$ and $z_{a}$ are fed into the further network for detection results, supervised by the loss function of Faster-RCNN. To support domain alignment and feature diversity learning, $z_{s}$ and $z_{a}$ are sent into the Weighted Aligning Module (WAM) and Diversity Learning Module (DLM). The goal of this process is to achieve domain alignment while preserving domain-specific diversity. In the DLM, feature diversity loss is applied to enhance the feature diversity, while entropy maximization loss eliminates the semantic information. In the WAM, the loss weight is used to prevent overemphasizing the domain alignment, which may influence the feature diversity. By integrating the supervision from both the DLM and WAM, the proposed Diversity Invariance Detection Model (DIDM) can achieve improved performance on the domain generalization task.

\subsubsection{Diversity Learning Module (DLM)}
\label{sec:methodDLM}
Traditional domain alignment methods primarily focus on learning domain-invariant features, overlooking the importance of domain-specific information. This limitation restricts the model's ability to achieve feature diversity, impacting feature extraction. To address this, we introduce the Diversity Learning Module (DLM), which restricts the domain-specific feature to achieve feature diversity. The DLM operates as follows: first, to obtain domain-specific features that capture domain information, we draw inspiration from OCR \cite{jing2023order}, assuming the relationship between the original and augmented features as $z_{a}= \lambda_1 z_{s}+(1- \lambda_1) z_d$. Here, $\lambda_1$ is a gradually increasing hyperparameter, reflecting the model's improving capacity to learn domain-invariant features over the process of training. As this capacity grows, the domain-specific information in $z_{s}$ and $z_{a}$ correspondingly diminishes. Ultimately, this approach enables us to obtain the domain-specific feature $z_d$ during training.

Second, considering that the domain-specific feature $z_d$ should not contain semantic information, we introduce the entropy maximization loss for $z_d$, where a classifier is trained with the cross-entropy loss:
\begin{equation}
\begin{aligned}
\mathcal{L}_\mathcal{C} = -\sum\limits_{i=1}^C y_ilog([Softmax(f(z_d))])
\end{aligned}
\end{equation}
where $C$ represents the number of categories, $f(z_d) \in R^{K \times C}$ is prediction function for $z_d$. Inspired by \cite{jing2023order}, to estimate the category semantic information in $z_d$, we implement to maximization of the entropy of the prediction results from $z_d$.
\begin{equation}
\begin{aligned}
\mathcal{L}_\mathcal{H} = -\mathcal{H}(y|z_d) = -\mathcal{H}[Softmax(f(z_d))]
\label{eq:pythagoras}
\end{aligned}
\end{equation}
where $\mathcal{H}$ represents entropy computation. By minimizing \cref{eq:pythagoras}, we can maximize the conditional entropy based on the classifier $f(.)$. When the output of the classifier for all categories is uniform across all categories, the conditional entropy reaches its maximum, thereby reducing the semantic information contained in $f(z_d)$.
%------------------------------------------------------------------------

\begin{table*}
    \centering
    \setlength{\tabcolsep}{8.6pt}
    \fontsize{9.9}{15}\selectfont\begin{tabular}{c|c|cccc|c}
    \bottomrule
         \textbf{Methods}&  \textbf{daytime-clear} &  \textbf{night-sunny}&  \textbf{dusk-rainy}&  \textbf{night-rainy}&  \textbf{daytime-foggy}&  \textbf{Average} \\
         \hline
          Faster R-CNN \cite{ren2016faster}&  54.7&  34.0&  30.5&  14.0&  32.2& 27.7\\
          SW \cite{pan2019switchable}&  50.6&  33.4&  26.3&  13.7&  30.8& 26.1\\
         IBN-Net \cite{pan2018two}&  49.7&  32.1&  26.1&  14.3&  29.6& 25.5\\
         IterNorm \cite{huang2019iterative}&  43.9&  29.6&  22.8&  12.6&  28.4& 23.4\\
         ISW \cite{choi2021robustnet}&  51.3&  33.2&  25.9&  14.1&  31.8& 26.3\\
         S-DGOD \cite{wu2022single}&56.1 &36.6 &28.2 &16.6 &33.5 &28.8\\
         CLIP-Gap \cite{vidit2023clip}&51.3 &36.9 &32.3 &18.7 &38.5 &31.6 \\
         Prompt-Driven \cite{li2024prompt}&53.6 &38.5 &33.7 &19.2 &39.1 &32.6 \\
         OA-DG \cite{lee2024object}&55.8 &38.0 &33.9 &16.8 &38.3 &31.8 \\
         \hline
         \hline
         Ours &\textbf{56.1} & \textbf{42.0} & \textbf{35.4} & \textbf{19.2} & \textbf{39.3} &\textbf{34.0} \\
         \toprule
    \end{tabular}
    \caption{Single-Domain Generalization Object Detection Results. We trained the model using daytime-clear as the source and subsequently tested it within that domain, while also assessing its generalization across the other four domains. The results presented in the table indicate that our Diversity Invariant Detection Model (DIDM) significantly enhances the model's generalization performance, with bold text highlighting optimal performance. This underscores the adaptability and effectiveness of DIDM across various environments.}
    \label{tab:my_label1}
\end{table*}
%------------------------------------------------------------------------
Lastly, in addition to estimating semantic information, we introduce a \textbf{feature diversity (FD) loss} to promote diverse characteristics among features. Specifically, let the domain-specific feature for each proposal be $z_d^i$. We calculate the cosine similarity between each pair $z_d^i$ and $z_d^j$ for all proposals. Given that each proposal may capture different information due to variations in location or domain, our goal is to increase the distinctions between $z_d^i$ and $z_d^j$, where $i,j \in [1, K]$ and $K$ is the number of proposals. Consequently, the feature diversity loss can be expressed as:
\begin{equation}
\begin{aligned}
\mathcal{L}_\mathcal{FD} = -\sum\limits_{i=1}^K log(\dfrac{e^{sim(f'(z_d^i),f'(z_d^i))}}{\sum\limits_{j=1}^K e^{sim(f'(z_d^i),f'(z_d^j))}} )
\end{aligned}
\end{equation}
Here $sim(.)$ denotes the cosine similarity between feature mappings. $\hat{y}_i$ is set as 1 when $i=j$ and to 0 otherwise. $f'(.)$ is the function of the fully connected layer. The diversity loss is designed to encourage a wide range of predicted categories, maximizing the dissimilarity among features.

The total loss for the Diversity Learning Module is calculated as follows:
\begin{equation}
\begin{aligned}
\mathcal{L}_\mathcal{DLM} = \mathcal{L}_\mathcal{C}+\mathcal{L}_\mathcal{H}+\lambda_2 \mathcal{L}_\mathcal{FD}
\end{aligned}
\end{equation}
where $\lambda_2$ is the hyperparameter for balancing the loss, with the supervision of the DLM, our model can both limit the semantic information of the domain-specific feature, while keeping the feature diversity. 

\subsubsection{Weighted Aligning Module (WAM)}
\label{sec:methodWAM}

DLM focuses on learning domain-specific features, while WAM is introduced to promote domain invariance. Considering that overemphasis on domain invariance may influence domain diversity, we implement the loss weight within the feature alignment in WAM. Specifically, during the training phase, we align the original feature $z_{s}$ and the argument features $z_{a}$. Simultaneously, we minimizing the difference between $z_{s}$ and $z_{a}$, while maximizing the distance between $z_{s}$, $z_{a}$ and $z_d$ for the alignment:
\begin{align}
\mathcal{L}_{fa}&=1- \mathcal{L}(z_{s},z_{a})\\
\mathcal{L}_{fs}=&\dfrac {1}{2}(\mathcal{L}(z_{s},z_d) + \mathcal{L}(z_{a},z_d))\\
\mathcal{L}(z_{s},z_{a}) =& \dfrac{sim(z_{s}^i,z_{a}^i)}{sim(z_{s}^i,z_{a}^i) + \sum\limits_{j \neq i, j=1}^K sim(z_{s}^i,z_{a}^j)}
\end{align}
By aligning features across different domains, we encourage the model to learn domain-invariant features. However, solely constraining domain disparity can reduce feature diversity. To prevent the model from overly restricting domain-specific information, we introduce a weighted parameter for feature alignment. When detection results based on $z_{s}$ and $z_{a}$ are similar, the differences between $z_{s}$ and $z_{a}$ have minimal impact, suggesting that the alignment loss should be down-weighted to preserve feature diversity. Specifically, the weight for the alignment loss is defined as:
\begin{align}
\beta &= 2-e^{-\dfrac {1}{2}(\mathcal{L}_{c}+\mathcal{L}_{b})}
\label{eq:pythagoras8}
\end{align}
where $\mathcal{L}_{c} = \sum\limits_{i=1}^K KL(p_{s}^i,p_{a}^i)$, and
$\mathcal{L}_{b} = ||b_{s}^i-b_{a}^i||_2^2$ from \cite{danish2024improving}. $p_{s}^i, p_{a}^i$ are the classification results of $z_{s}$ and $z_{a}$, respectively, while $b_{s}^i-b_{a}^i$ are the bounding box regression results. Finally, the loss function of the weighted alignment module (WAM) is defined as follows:
\begin{align}
\mathcal{L}_\mathcal{WAM}=\beta \mathcal{L}_{fa}+ \mathcal{L}_{fs}
\end{align}
By training with the loss weight $\beta$, the model down weight the alignment loss when the differences between the original and argument features do not influence the detection results. This operation prevents the overemphasis on the feature alignment. Thus, the WAM can align the feature without compromising the feature diversity. Thus, the feature learning of the model is further enhanced.

\subsubsection{Overall Optimization Objective}
The overall optimization objective of the model is the following:
\begin{align}
\mathcal{L}_{total}=\mathcal{L}_{det}+\alpha \mathcal{L}_\mathcal{DLM}+ \gamma \mathcal{L}_\mathcal{WAM}
\label{eq:pythagoras10}
\end{align}
Here $\mathcal{L}_{det}=\mathcal{L}_{reg}+\mathcal{L}_{cls}$ is the detection loss, $\alpha$ and $\gamma$ is the hyperparameters for balancing the loss. With the co-training of the detection loss, DLM and WMA, our model achieve effective domain generalization on the task.
% %------------------------------------------------------------------------
\begin{figure*}
    \centering
    \includegraphics[width=1\linewidth]{25.pdf}
    \caption{Qualitative evaluation results for night-sunny and night-rainy weather conditions are presented. The \textbf{top row} showcases the detection visualizations generated by the \textbf{Faster R-CNN} \cite{ren2016faster} model, while the \textbf{bottom row} displays the detection visualizations produced by the \textbf{DIDM}. The images on the \textbf{left} depict \textbf{night-sunny} conditions, whereas those on the \textbf{right} illustrate \textbf{night-rainy} conditions.}
    \label{fig:enter-label3}
\end{figure*}
%------------------------------------------------------------------------
%------------------------------------------------------------------------
\section{Experiments}
\label{sec:experiment}
\subsection{Experimental Setup}

\textbf{Datasets.} 
To verify the validity of the method, we utilize the datasets created by \cite{wu2022single}. Five distinct weather conditions are included: daytime-clear, night-sunny, dusk-rainy, night-rainy, and daytime-foggy. These datasets are primarily derived from 27,708 images captured during daytime-clear conditions and 26,158 images taken at night-sunny conditions, collected from BDD-100K \cite{yu2020bdd100k}. Additionally, 3,775 images with foggy conditions were sourced from Adverse Weather \cite{hassaballah2020vehicle} and Cityscape datasets \cite{cordts2016cityscapes}. The datasets with rain rendered using \cite{wu2021vector}, including 3,501 images from dusk-rainy days and 2,494 images from rainy nights. In this paper, we employ 19,395 daytime images of sunny conditions as the training set to train the model, while another 8,313 images are used to evaluate the model's performance. The remaining four datasets are treated as unseen target domains to assess the model's generalization capabilities. In these datasets, we focus on seven common categories: car, bike, bus, rider, person, motor, and truck.
%------------------------------------------------------------------------
\begin{table}
    \centering
    \setlength{\tabcolsep}{5.8pt}
     \fontsize{7.89}{12}\selectfont\begin{tabular}{@{  }c|p{0.8cm}@{}p{0.8cm}@{}p{0.75cm}@{}c@{  }c@{  }c@{  }c@{  }|c@{  }}
    \bottomrule
     Methods & Bus & Bike & Car &Motor & Person & Rider &Truck & mAP \\
        \hline
           FR \cite{ren2016faster} & 33.7 & 27.4 & 61.1 & 10.6 & 40.2 & 25.7 & 39.6 & 34.0 \\
           SW \cite{pan2019switchable} &38.7 &29.2 &49.8 &16.6 &31.5 &28.0 &40.2 &33.4 \\
           IBN-Net \cite{pan2018two} &37.8 &27.3 &49.6 &15.1 &29.2 &27.1 &38.9 &32.1 \\
           IterNorm \cite{huang2019iterative} &38.5 &23.5 &38.9 &15.8 &26.6 &25.9 &38.1 &29.6 \\
           ISW \cite{choi2021robustnet} &38.5 &28.5 &49.6 &15.4 &31.9 &27.5 &41.3 &33.2 \\
         S-DGOD \cite{wu2022single} &40.6 &35.1 &50.7 &19.7 &34.7 &32.1 &43.4 &36.6\\
         CLIP-Gap \cite{vidit2023clip} &37.7 &34.3 &58.0 &19.2 &37.6 &28.5 &42.9 &36.9 \\
         Prompt-D \cite{li2024prompt}&\textbf{49.9} &35.0 &59.0 &21.3 &40.4 &29.9 &42.9 &38.5 \\
        \hline
        \hline
         Ours&43.5  &\textbf{40.1}  &\textbf{65.1}  &\textbf{22.4}  &\textbf{45.2}  &\textbf{32.5}  &\textbf{45.3}  &\textbf{42.0} \\
         \toprule
    \end{tabular}
    \caption{The quantitative results (\%) on the night-sunny.}
    \label{tab:my_label2}
\end{table}
%------------------------------------------------------------------------

\textbf{Implementation Details.}
We utilize Faster R-CNN \cite{ren2016faster} as a two-stage detector, with ResNet101 \cite{deng2021unbiased} serving as the backbone network. The base network employs weights that have been pre-trained on the ImageNet dataset \cite{deng2009imagenet}. To train the model, we apply the stochastic gradient descent (SGD) algorithm with a momentum parameter of 0.9. The initial learning rate is set to 0.02 and decays every three epochs, while the batch-size is 4. All reported mean average precision (mAP) values are based on an Intersection over Union (IoU) threshold of 0.5.

\textbf{Data Augmentation Setting.}
To enhance the diversity of the source domain, we employ techniques such as random clipping, the addition of Gaussian noise, grayscale enhancement, and color transformation to expand the dataset.

\subsection{Performance Analysis}
The DIDM we studied is compared here with four feature-based normalization methods, they are SW \cite{pan2019switchable}, IBN-Net \cite{pan2018two}, IterNorm \cite{huang2019iterative} and ISW \cite{choi2021robustnet}. Comparisons were also made with the latest S-DGOD \cite{wu2022single}, CLIP-Gap \cite{vidit2023clip}, Prompt-Driven \cite{li2024prompt}, and OA-DG \cite{lee2024object}methods.
%------------------------------------------------------------------------
%------------------------------------------------------------------------
\begin{figure*}
    \centering
    \includegraphics[width=1\linewidth]{26.pdf}
    \caption{Qualitative evaluation results for night-sunny and night-rainy weather conditions are presented. The \textbf{top row} showcases the detection visualizations generated by the \textbf{Faster R-CNN} \cite{ren2016faster} model, while the \textbf{bottom row} displays the detection visualizations produced by the \textbf{DIDM}. The images on the \textbf{left} depict \textbf{daytime-foggy} conditions, whereas those on the \textbf{right} illustrate \textbf{dusk-rainy} conditions.}
    \label{fig:enter-label4}
\end{figure*}
%------------------------------------------------------------------------
%------------------------------------------------------------------------

\begin{table}
    \centering
    \setlength{\tabcolsep}{5.8pt}
     \fontsize{7.89}{12}\selectfont\begin{tabular}{@{  }c|p{0.8cm}@{}p{0.8cm}@{}p{0.75cm}@{}c@{  }c@{  }c@{  }c@{  }|c@{  }}
    \bottomrule
     Methods & Bus & Bike & Car &Motor & Person & Rider &Truck & mAP \\
        \hline
           FR \cite{ren2016faster} & 25.1 & 6.1 & 35.7 & 0.2 & 9.1 & 3.9 & 17.8 & 14.0 \\
           SW \cite{pan2019switchable} &22.3 &7.8 &27.6 &0.2 &10.3 &10.0 &17.7 &13.7 \\
           IBN-Net \cite{pan2018two} &24.6 &10.0 &28.4 &0.9 &8.3 &9.8 &18.1 &14.3 \\
           IterNorm \cite{huang2019iterative} &21.4 &6.7 &22.0 &0.9 &9.1 &10.6 &17.6 &12.6 \\
           ISW \cite{choi2021robustnet} &22.5 &11.4 &26.9 &0.4 &9.9 &9.8 &17.5 &14.1 \\
           S-DGOD \cite{wu2022single} &24.4 &11.6 &29.5 &9.8 &10.5 &11.4 &19.2 &16.6\\
          CLIP-Gap \cite{vidit2023clip} &28.6 &12.1 &36.1 &9.2 &12.3 &9.6 &22.9 &18.7 \\
          Prompt-D \cite{li2024prompt}&25.6 &12.1 &35.8 &\textbf{10.1} &\textbf{14.2} &\textbf{12.9} &22.9 &19.2 \\
        \hline
        \hline
         Ours&\textbf{31.6}  &\textbf{12.1}  &\textbf{38.3}  &3.8  &12.8  &10.6 &\textbf{25.0}  &\textbf{19.2} \\
         \toprule
    \end{tabular}
    \caption{The quantitative results (\%) on the night-rainy.}
    \label{tab:my_label4}
\end{table}
%------------------------------------------------------------------------
\textbf{Results on all datasets.}
\cref{tab:my_label1} presents the testing and generalization results of DIDM on the DWD datasets \cite{wu2022single}. In this study, we utilized the daytime-clear dataset to train the model, tested it on the same dataset, and evaluated its generalization capabilities on four additional datasets: night-sunny, dusk-rainy, night-rainy, and daytime-foggy. As shown in \cref{tab:my_label1}, our approach achieves the highest average generalization performance of 34.0\%. When compared to the baseline network, Faster R-CNN \cite{ren2016faster}, our method demonstrates improvements of 7.1\% and 4.9\% on the daytime-foggy and dusk-rainy datasets, respectively. Additionally, performance on the night-sunny dataset was significantly enhanced by 8.0\%, while the night-rainy dataset, characterized by a complex environment, saw an improvement of 5.2\%.
%------------------------------------------------------------------------

\begin{table}
    \centering
    \setlength{\tabcolsep}{5.8pt}
     \fontsize{7.89}{12}\selectfont\begin{tabular}{@{  }c|p{0.8cm}@{}p{0.8cm}@{}p{0.75cm}@{}c@{  }c@{  }c@{  }c@{  }|c@{  }}
    \bottomrule
     Methods & Bus & Bike & Car &Motor & Person & Rider &Truck & mAP \\
        \hline
           FR \cite{ren2016faster} & 22.9 & 27.6 & 55.8 & 29.6 & 33.1 & 34.9 & 21.3 & 32.2 \\
           SW \cite{pan2019switchable} &30.6 &36.2 &44.6 &25.1 &30.7 &34.6 &23.6 &30.8 \\
           IBN-Net \cite{pan2018two} &29.9 &26.1 &44.5 &24.4 &26.2 &33.5 &22.4 &29.6 \\
           IterNorm \cite{huang2019iterative} &29.7 &21.8 &42.4 &24.4 &26.0 &33.3 &21.6 &28.4 \\
           ISW \cite{choi2021robustnet} &29.5 &26.4 &49.2 &27.9 &30.7 &34.8 &24.0 &31.8 \\
           S-DGOD \cite{wu2022single} &32.9 &28.0 &48.8 &29.8 &32.5 &38.2 &24.1 &33.5\\
          CLIP-Gap \cite{vidit2023clip} &36.1 &34.3 &58.0 &33.1 &39.0 &43.9 &25.1 &38.5 \\
          Prompt-D \cite{li2024prompt}&36.1 &\textbf{34.5} &58.4 &33.3 &\textbf{40.5} &\textbf{44.2} &26.2 &39.1 \\
        \hline
        \hline
         Ours&\textbf{38.5}  &31.6  &\textbf{62.1}  &\textbf{35.8}  &36.8  &42.7  &\textbf{27.3}  &\textbf{39.3} \\
         \toprule
    \end{tabular}
    \caption{The quantitative results (\%) on the daytime-foggy.}
    \label{tab:my_label5}
\end{table}
%------------------------------------------------------------------------
\textbf{Results on night-sunny conditions.}
As shown in the results presented in \cref{tab:my_label2}, our method exhibits outstanding performance across several object detection categories, particularly in the Bike, Car, and Person categories. It achieves significant performance improvements compared to all methods in the table. In these categories, our method attains detection accuracies of 40.1\%, 65.1\%, and 45.2\%, respectively. This demonstrates the advantages of our approach in feature normalization and target detection. In addition, we achieve an overall performance improvement of 5.4\%, 5.1\%, and 3.5\% compared to the latest S-DGOD \cite{wu2022single}, CLIP-Gap \cite{vidit2023clip} and prompt-D \cite{li2024prompt}, respectively. This further demonstrates the superior adaptability and robustness of our method in handling data across different domains. \cref{fig:enter-label3} further confirms that the detection accuracy of our method surpasses that of the baseline Faster R-CNN \cite{ren2016faster}.

\textbf{Results on night-rainy conditions.}
The night-rainy presents a particularly challenging situation. Low visibility and complex lighting significantly impacted the color-dependent model, while the rain further diminished visual clarity. 
\cref{tab:my_label4} illustrates that our method significantly outperforms other approaches in automobile category detection, particularly in bus detection, with improvements of 7.2\%, 3\%, and 6\% compared to S-DGOD \cite{wu2022single}, CLIP-Gap \cite{vidit2023clip} and Prompt-D \cite{li2024prompt}, respectively. 
\cref{fig:enter-label3} illustrates that our method can recognize objects more accurately in harsh environments, highlighting its advantages in enhancing the robustness of object detection.
%------------------------------------------------------------------------

\begin{table}
    \centering
    \setlength{\tabcolsep}{5.8pt}
     \fontsize{7.89}{12}\selectfont\begin{tabular}{@{  }c|p{0.8cm}@{}p{0.8cm}@{}p{0.75cm}@{}c@{  }c@{  }c@{  }c@{  }|c@{  }}
    \bottomrule
     Methods & Bus & Bike & Car &Motor & Person & Rider &Truck & mAP \\
        \hline
           FR \cite{ren2016faster} & 36.9 & 22.8 & 65.5 & 10.3 & 22.3 & 15.6 & 39.8 & 30.5 \\
           SW \cite{pan2019switchable} &35.2 &16.7 &50.1 &10.4 &20.1 &13.0 &38.8 &26.3 \\
           IBN-Net \cite{pan2018two} &37.0 &14.8 &50.3 &11.4 &17.3 &13.3 &38.4 &26.1 \\
           IterNorm \cite{huang2019iterative} &32.9 &14.1 &38.9 &11.0 &15.5 &11.6 &35.7 &22.8 \\
           ISW \cite{choi2021robustnet} &34.7 &16.0 &50.0 &11.1 &17.8 &12.6 &38.8 &25.9 \\
         S-DGOD \cite{wu2022single} &37.1 &19.6 &50.9 &13.4 &19.7 &16.3 &40.7 &28.2\\
         CLIP-Gap \cite{vidit2023clip} &37.8 &22.8 &60.7 &16.8 &26.8 &18.7 &42.4 &32.3 \\
         Prompt-D \cite{li2024prompt}&39.4 &25.2 &60.9 &\textbf{20.4} &29.9 &16.5 &43.9 &33.7 \\
        \hline
        \hline
         Ours&\textbf{41.6}  &\textbf{26.3}  &\textbf{66.6}  &16.6  &\textbf{30.9}  &\textbf{21.9}  &\textbf{44.1}  &\textbf{35.4} \\
         \toprule
    \end{tabular}
    \caption{The quantitative results (\%) on the dusk-rainy.}
    \label{tab:my_label3}
\end{table}
%------------------------------------------------------------------------
\textbf{Results on daytime-foggy conditions.}
As shown in \cref{tab:my_label5}, our method demonstrates superior generalization performance under daytime-foggy conditions. Additionally, \cref{fig:enter-label4} illustrates that our method achieves higher accuracy in detecting small figure targets in foggy environments compared to Faster R-CNN \cite{ren2016faster}.

\textbf{Results on dusk-rainy conditions.}
As shown in \cref{tab:my_label3}, our method demonstrates exceptional performance across all categories, particularly in the Bus, Car, and Truck categories. Furthermore, the overall performance improves by 3.1\% and 1.7\% compared to the CLIP-Gap \cite{vidit2023clip} and Prompt-D \cite{li2024prompt}, the most recent methods, respectively. The robustness of our method in handling challenging environments is also evident. The visual analysis presented in \cref{fig:enter-label4} illustrates the diversity and invariance of the equilibrium domain, which contributes to effective model training.
%------------------------------------------------------------------------
\begin{table}
    \centering
    \fontsize{9}{15}\selectfont\begin{tabular}{c@{  }|ccc@{}|ccc}
    \bottomrule
         Methods& $\mathcal L_\mathcal C+\mathcal L_\mathcal H$&  $\mathcal L_\mathcal{FD}$&  $\mathcal L_\mathcal{WAM}$&  DC&  DF& NS\\
         \hline
         \multirow{6}{*}{FR \cite{ren2016faster}}%%合并六行
         &  &  &  &54.7  &32.2  &34.0 \\
         &  $\checkmark$&  &  &55.8  &37.1  &39.3 \\
         &  &$\checkmark$  &  &54.1  &37.6  &39.4 \\
         &  $\checkmark$&  $\checkmark$&  &55.9  &38.0  &39.7 \\
         &  &  &  $\checkmark$&  \textbf{56.5}&  37.7& 40.3\\
         &  $\checkmark$&  $\checkmark$&  $\checkmark$&  56.1&  \textbf{39.3}& \textbf{42.0}\\
    \toprule
    \end{tabular}
    \caption{Ablation study. Here, DC, DF, and NS stand for \textbf{daytime-clear}, \textbf{daytime-foggy} and \textbf{night-sunny}, respectively.}
    \label{tab:my_label6}
\end{table}
%------------------------------------------------------------------------
\subsection{Ablation Study}
To assess the influence of each component on the performance of DIDM, we performed an ablation study. The model was trained utilizing the daytime-clear dataset and subsequently evaluated on both the daytime-foggy and night-sunny. Through a comprehensive analysis of the individual contributions of DLM and WAM, we confirmed the efficacy of both modules in the context of object detection. Furthermore, we illustrated the synergistic effect achieved by integrating these two modules, which significantly improves the model's generalization capabilities and detection accuracy. The conclusive results are detailed in \cref{tab:my_label6}.

The first component is the DLM, as illustrated in \cref{tab:my_label6}. The model's performance can be significantly improved by utilizing $\mathcal L_\mathcal H$ and $\mathcal L_\mathcal{FD}$, respectively. This suggests that either constraining the semantic information of domain-specific features or preserving the diversity of these features can enhance the model's detection capabilities. Furthermore, the model's performance is further augmented by employing both $\mathcal L_\mathcal H$ and $\mathcal L_\mathcal{FD}$. This demonstrates that the DLM component effectively enhances the model's ability to extract semantic information from images. In \cref{tab:my_label6}, we analyze the role of the WAM and observe that the model's generalization performance improves by 5.5\% and 6.3\% for daytime-foggy and night-sunny conditions, respectively, compared to the baseline Faster R-CNN \cite{ren2016faster}. Additionally, the model's test map achieves a score of 56.5\%. This indicates that the WAM enhances the model's ability to learn domain-invariant features while maintaining a degree of diversity in domain-specific features by incorporating loss weights.

Ultimately, we integrated the DLM and WAM components and found that the model's generalization performance was significantly improved while maintaining its testing performance to a certain extent. This indicates that the synergy between DLM and WAM effectively balance between the diversity of domain-specific and invariance cross domains. This balance enhances the model's learning and feature extraction capabilities while ensuring stable detection performance across various environments.
%------------------------------------------------------------------------
\begin{figure}
  \centering
  \begin{subfigure}{0.49\linewidth}
  \includegraphics[width=1\linewidth]{65.png}
    \caption{Analysis of $\alpha$ when $\gamma$ is 0.45}
    \label{fig:short-a}
  \end{subfigure}
  \hfill
  \begin{subfigure}{0.49\linewidth}
  \includegraphics[width=1\linewidth]{64.png}
    \caption{Analysis of $\gamma$ when $\alpha$ is 0.45}
    \label{fig:short-b}
  \end{subfigure}
  \caption{Hyperparameter Analysis.}
  \label{fig:short}
\end{figure}
%------------------------------------------------------------------------
\subsection{Hyperparameter Analysis}

Appropriate hyperparameter settings are essential for optimizing model performance. For the parameters $\alpha$ and $\gamma$ in \cref{eq:pythagoras10}, we experimented with various hyperparameter configurations to identify the optimal settings. As illustrated in \cref{fig:short-a,fig:short-b}, the model demonstrates the best performance when $\alpha$ =0.45 and $\gamma$ = 0.45.

\section{Conclusion}
\label{sec:conclusion}
Existing domain-generalized object detection models (S-DGOD) primarily focus on learning domain-invariant features to mitigate the negative impact of domain bias on the model's generalization ability. However, this approach often overlooks the inherent differences between various domains and may complicate the training process and lead to a loss of valuable information. To address this issue, we propose a Diversity Invariant Detection Model (DIDM) in this paper. This approach aims to balance domain invariance and domain diversity, thereby enhancing the model's feature extraction capabilities. In DIDM, we propose to introduce the Diversity Learning Module (DLM) and Weighted Aligning Module (WAM) for the domain-specific feature and domain invariance, respectively. For DLM, the feature diversity (FD) loss is implemented with entropy maximization loss to eliminate the semantic information while keeping the feature diversity. Additionally, a Weighted Aligning Module (WAM) is incorporated to prevent the overemphasis on the feature alignment with loss weight. With the combination of the DLM and WAM, the DIDM can efficiently manage the domain-specific and domain-invariance when preserving the feature diversity, resulting in improved detection performance across multiple domains. This strategy not only enhances the model's adaptability to domain variations but also ensures stable detection performance in diverse environments. Both experimental data and comprehensive analysis validate the effectiveness of DIDM.

%-------------------------------------------------------------------------
%-------------------------------------------------------------------------

{
    \small
    \bibliographystyle{ieeenat_fullname}
    \bibliography{main}
}

% WARNING: do not forget to delete the supplementary pages from your submission 
% 
\clearpage
% \setcounter{page}{1}
% \maketitlesupplementary
\begin{center}
Supplementary Material
\end{center}

% {
%     \onecolumn
%     \centering
%     \Large
%     \textbf{\thetitle}\\
%     \vspace{0.5em}Supplementary Material \\
%     \vspace{1.0em}
% }

\section{Proof of \cref{theorem:dr}}
We require some additional regularity assumptions:
\begin{assumption} 1) The number of classes $C$ is bounded w.r.t the number of samples $N$, 2) the missingness mechanism $P(A=1|Y,\theta)$, as well as its estimated counterpart $P(A=1|Y,\theta)$, are bounded below by some constant $\epsilon > 0$, 3) the quantities $P(Y|X,\theta)$ and $P(A|Y,\theta)$ are estimated using auxiliary samples independent of samples used for the sample averaging.
\label{assumption:extra}
\end{assumption}
Assumptions 1 and 2 are natural. For the missingness mechanism, the ground truth being bounded means that there is a non-vanishing proportion of samples for every class. The boundedness of the estimate can be enforced by clipping the estimate. Assumption 3 is called sample splitting in \cite{kennedy-dr}.

For convenience we use operator $\E_N$ to denote the average of $N$ samples i.e. $\frac{1}{N}\sum_{i=1}^N$. Note that this is by itself a random variable, in contrast to $\E$ which is a fixed number.

\begin{proof}[Proof of \cref{theorem:dr}] Because $C$ is bounded (assumption \ref{assumption:extra}), we can fix a class $c$ and prove the theorem.
Let us define the influence function $\phi$, parameterized by $\theta$, as
\begin{equation}
\phi(O | \theta)(c) = P(Y=c|X,\theta) + \frac{\one(A=1)}{P(A=1|Y,\theta)} (\one(Y=c) - P(Y=c|X,\theta)) - P(Y=c)
\end{equation}
As we have done in the main text, we use $\phi(O)$ to denote the same function but all estimated quantities are replaced with their truths. In other words, we use $\phi(O)$ for $\phi(O|\theta_0)$ where $\theta_0$ is the truth, given that our model contains $\theta_0$ e.g. when the model is consistent.

Recall that:
\begin{equation}
\begin{aligned}
\Psi_{dr}(\theta)(c) &= \frac{1}{N}\sum_{i=1}^N \left\{P(Y=c|X,\theta) + \frac{\one(A=1)}{P(A=1|Y,\theta)} (\one(Y=c) - P(Y=c|X,\theta))\right\}\\
&= \E_N [\phi(O|\theta)(c)] + P(Y=c)
\end{aligned}
\end{equation}

We will show that:
\begin{equation}
\Psi_{dr}(\theta)(c) - P(Y=c) = (\E_N - \E)[\phi(O)(c)] + o_P(N^{-1/2})
\label{eq:proof-linearity}
\end{equation}
To do that, we use the following decomposition
\begin{equation}
\begin{aligned}
\Psi_{dr}(\theta)(c) - P(Y=c) &= \E_N [\phi(O|\theta)(c)] \\
&= (\E_N - \E)[\phi(O)(c)] + (\E_N - \E)[\phi(O|\theta)(c) - \phi(O)(c)] + \E[\phi(O|\theta)(c)]
% &+ (\E_n - \E)[\phi(O;\theta) - \phi(O)]\\
% &+ \E[P(Y=c|X,\theta)] - \E[P(Y=c|X)] + \E[\phi(O,\theta)]
\end{aligned}
\end{equation}
and analyze the second and third term. The third term is:
\begin{equation}
\begin{aligned}
\E[\phi(O|\theta)(c)] &= \E[P(Y=c|X,\theta)] + \E\left[\frac{\one(A=1)}{P(A=1|Y,\theta)}(\one(Y=c) - P(Y=c|X,\theta))\right]- P(Y=c) \\
&= \E\left[P(Y=c|X,\theta) + \frac{P(A=1|Y)}{P(A=1|Y,\theta)}(P(Y=c|X) - P(Y=c|X,\theta))\right] - \E[P(Y=c|X)]\\
&= \E\left[(P(Y=c|X,\theta) - P(Y=c|X)) (P(A=1|Y,\theta) -P(A=1|Y)) \frac{1}{P(A=1|Y,\theta)}\right]\\
\end{aligned}
\end{equation}
by Cauchy-Schwarz inequality:
\begin{equation}
\begin{aligned}
\E[\phi(O|\theta)(c)] &\le \frac{1}{\epsilon} \|P(A=1|Y,\theta) - P(A=1|Y)\|_2 \|P(Y=c|X,\theta) - P(Y=c|X)\|_{L_2(P)}\\
&= \frac{1}{\epsilon} o_P(N^{-1/4} N^{-1/4}) = o_P(N^{-1/2})
\end{aligned}
\end{equation}
by assumption \ref{assumption:4th-root-n} and that $P(A=1|Y,\theta) > \epsilon$ (assumption \ref{assumption:extra}). The second term can be bounded by Chebyshev inequality
% \begin{equation}
% \begin{aligned}
% \E[\E_N[\phi(O|\theta)(c) - \phi(O)(c)]] &= \E[\phi(O|\theta)(c) - \phi(O)(c)]\\
% \var[\E_N[\phi(O|\theta)(c) - \phi(O)(c)]] &= \frac{1}{N}\var[\phi(O|\theta)(c) - \phi(O)(c)] \le 
% \end{aligned}
% \end{equation}
\begin{equation}
P(|(\E_N - \E)[\phi(O|\theta)(c) - \phi(O)(c)]| \ge t) \le \frac{\var[\E_N[\phi(O|\theta)(c) - \phi(O)(c)]]}{t^2} = \frac{\var[\phi(O|\theta)(c) - \phi(O)(c)]}{Nt^2}
\end{equation}
note here that $\theta$ is independent of the samples used for $\E_N$ by assumption \ref{assumption:extra}. For any $\varepsilon > 0$, by picking $t = \frac{1}{\sqrt{N\varepsilon}}$ we get
\begin{equation}
P\left(\left|\frac{(\E_N - \E)[\phi(O|\theta)(c) - \phi(O)(c)]}{N^{-1/2}}\right| \ge \frac{1}{\sqrt{\varepsilon}}\right) \le \varepsilon \var[\phi(O|\theta)(c) - \phi(O)(c)]
\end{equation}
by the definition of $O_P$, we then get
\begin{equation}
(\E_N - \E)[\phi(O|\theta)(c) - \phi(O)(c)] = O_P(N^{-1/2}\var[\phi(O|\theta)(c) - \phi(O)(c)])
\end{equation}
Because $\phi$ is a continuous function of $P(Y|X,\theta)$ and $P(A|Y,\theta)$ (given $P(A|Y,\theta) > \epsilon$, assumption \ref{assumption:extra}), by the continuous mapping theorem and the fact that $P(Y|X,\theta)$ and $P(A|Y,\theta)$ are convergent in probability (assumption \ref{assumption:4th-root-n}), we get $\var[\phi(O|\theta)(c) - \phi(O)(c)] = o_P(1)$. This gives
\begin{equation}
(\E_N - \E)[\phi(O|\theta)(c) - \phi(O)(c)] = o_P(N^{-1/2})
\end{equation}
Therefore, we have shown that the second and third term are both $o_P(N^{-1/2})$, proving \cref{eq:proof-linearity}. As the final step, multiply both sides of this equation by $\sqrt{N}$ we get:
\begin{equation}
\sqrt{N}(\Psi_{dr}(\theta)(c) - P(Y=c)) = \sqrt{N} (\E_N - \E)[\phi(O)(c)] + o_P(1) \rightsquigarrow \mathcal{N}(0, \var[\phi(O)(c)])
\end{equation}
by the central limit theorem, and $\var[\phi(O)(c)] = \E[\phi(O)(c)^2]$ because $\E[\phi(O)(c)] = 0$.
\end{proof}

While we started with the definition of $\phi$, \cref{eq:proof-linearity} shows that $\phi$ is indeed an influence function. Now we show that $\phi$ is also the efficient influence function, by using the characterization of the model's tangent space \cite{tsiatis-missingdata}. Note that the joint probability factorizes as $P(X,A,Y) = P(X)P(Y|X)P(A|Y)$, therefore the tangent space $\mathcal{T}$ factorizes as $\mathcal{T} = \mathcal{T}_{X} \oplus \mathcal{T}_{Y|X} \oplus \mathcal{T}_{A|Y}$ where $\mathcal{T}_X = \{h(X): \E[h] = 0\}$, $\mathcal{T}_{Y|X} = \{h(X,Y): \E[h|X] = 0\}$, $\mathcal{T}_{A|Y} = \{h(A,Y): \E[h|Y] = 0\}$, and the 3 subspaces are pairwise orthogonal. All influence functions are orthogonal to the tangent space, but the influence function that is also in the tangent space has the smallest variance and is called the efficient influence function. As $\phi$ is already an influence function, we need only show that $\phi$ is in $\mathcal{T}$. We write $\phi$ as
\begin{equation}
\phi(O)(c) = (P(Y=c|X) - P(Y=c)) + \left[\frac{\one(A=1)}{P(A=1|Y)} - 1\right](\one(Y=c) - P(Y=c|X)) + (\one(Y=c) - P(Y=c|X))
\end{equation}
and note that the first, second and third term are in $\mathcal{T}_X$, $\mathcal{T}_{A|Y}$ and $\mathcal{T}_{Y|X}$ respectively. Therefore, $\phi$ is indeed in $\mathcal{T}$. The efficient influence function has the smallest variance of all influence function, and therefore our estimator being asymptotically linear in $\phi$ (\cref{eq:proof-linearity}) has the smallest mean squared error in a local asymptotic minimax sense \cite{kennedy-dr, asymptoticstatistics}

\section{Further background and related work}
\paragraph{Discussion on semi-supervised EM.}
It appears that semi-supervised EM was first used for parameter estimation when the missingness mechanism is non-ignorable in \cite{ibrahim1996parameter}, but has not been used for label shift estimation.
Perhaps this is because the semi-supervised situation where additional unlabeled data is available during training is rarer than the test-time adaptation case. EM is well suited to take advantage of the extra unlabeled data to improve the classifier under very scarce and long-tailed labeled data. While the connection between pseudo-labeling and EM has been explored before \cite{entropyminimization}, the situation with label shift has not until recently \cite{simpro}. Here the application of EM is much more interesting, because other than simply giving pseudo-labeling a rigorous formulation, EM also estimates the missingness mechanism (equivalently the label distribution shift), which is important for shift correction and thus high-quality pseudo-labels \cite{acr}. The application of confidence thresholding can be seen as a sparse variant of EM \cite{neal1998view}.

\paragraph{The doubly-robust risk.} 
\label{subsec:dr-risk}
A technique that also derives from the theory of semi-parametric efficiency is orthogonal statistical learning \citep{foster2023orthogonal}. The idea is to minimize the doubly-robust risk:
\label{subsec:method-dr-risk}
\begin{equation}
\label{eq:dr-risk}
\mathcal{R}(\theta_2) = \frac{1}{N} \sum_{i=1}^N \Bigg[ l(x_i, \hat y_i|\theta_2) + \frac{\one(a_i=1)}{P(A=a_i|Y=y_i, \theta_1)} (l(x_i, y_i | \theta_2) - l(x_i, \hat y_i | \theta_2))\Bigg]
\end{equation}
where $l(x,y|\theta) = -\sum_{c=1}^C [y]_c \log P(Y=c|X=x,\theta)$ is the negative cross-entropy. 
The notation $[y]_c$ means that we are using the $c$-entry in a C-dimension probability vector $y$. 
Thus, $y_i$ denotes the one-hot label of observation $i$, while $\hat y_i$ denotes the pseudo-label, which can be one-hot or all-zero. 
Finally, we use $\theta_1$ to denote that $P(a|y,\theta_1)$ is an estimation from a previous stage, but it can be estimated with $\theta_2$ as well. 
The risk $\mathcal{R}(\theta_2)$ can be used as a training loss in a straightforward fashion. 
Similar to the doubly robust estimation of $P(Y)$, the doubly robust risk provides approximately unbiased estimation of the risk. 
This property has been used in \citep{arelabelsinformative, onnonrandommissinglabels, drst} also in the semi-supervised learning setting.
More broadly, it is at the heart of one of the core techniques in heterogenous treatment effect estimation in causal estimation \cite{kennedy2023towards, foster2023orthogonal, wager2018estimation}. 
The focus here is not the estimation of $\mathcal{R}(\theta_2)$ per se, but the quality of the learned model \cite{foster2023orthogonal}.
By using the doubly-robust risk, we can achieve an optimality result similar in spirit to our theorem \cref{theorem:dr}, but for the generalization error.
While this is appealing, in practice there are 2 problems with this approach. First, the inverse probability weight $P(A=a_i|Y=y_i,\theta_1)$ can be very large if the class ratio is highly unlabeled, making training unstable \cite{kallus2020deepmatch, pham2023stable}. 
This problem exists for our estimation as well. However, it is much easier to control for estimation than for training because of the iterative nature of model update. Secondly, we can further write $\mathcal{R}$ as:
\begin{equation}
\mathcal{R}(\theta_2) = \frac{1}{N}\sum_{i=1}^N l\left(x_i, \hat y_i + \frac{\one(a_i=1)}{P(A=a_i|Y=y_i,\theta_1)} (y_i - \hat y_i)\Bigg\vert\theta_2\right)
\end{equation}
which is a cross-entropy loss with new meta-pseudo-labels. However, these labels are not meant to be learned exactly, and furthermore they can be negative. Thus, theoretical works have to put stringent assumptions on the models. In \cref{subsec:ablation-1}, we show that experimentally that the instability problem makes doubly-robust risk performance worse than our 2-stage approach.

\section{Training and hyperparameter settings.}
\label{subsec:training-setting}
For neural network training, we follow the implementation and hyperparameter settings of \cite{simpro}. In particular, we adapt the core code of SimPro for Supervised, MLE and EM. For MLE, we update $P(A|Y)$ using the Adam optimizer with learning rate 1e-3, while for EM we use a momentum update similar to SimPro's update of $P(Y|A)$ because it has a a closed-form solution at each mini-batch. We use Wide ResNet-28-2 on all methods and all datasets in this section, including Imagenet-127, because we are motivated by the fact that stage-1's goal is not classification accuracy but the estimation of a finite-dimensional parameter. When using Wide ResNet-28-2 for Imagenet-127, we use the hyperparameters of CIFAR-100, except we lower the batch size of unlabeled data to 2 times that of labeled data instead of 8 for memory reason. We do not perform additional hyperparameter tuning. All experiments can be performed on 1 A6000 RTX GPU, and are run 3 times. We report the total variation distance between the estimated and the ground truth unlabeled class distribution, similar to its usage in Theorem 3.1 of \cite{lsc}, and the top-1 classification accuracy.

In the second stage of our algorithm, we freeze our estimation and plug it in SimPro and BOAT.
We keep exactly the same hyperparameter settings that SimPro and BOAT use. In particular, for Imagenet-127, we now use ResNet-50 and run each experiment once.
In SimPro, we set the unlabeled class distribution $P(Y|A=0)$ at the E-step;  however, we still keep a running estimate of the class distribution $P(Y)$ in the logit adjustment loss \cref{eq:simpro-la-loss}. While it is possible to use the first stage estimate in the logit adjustment loss, we observe that doing so results in lower accuracy than using the the running average. This is conceptually consistent with the role of the running average - serving not as an accurate estimate of $P(Y)$ but to make the classifier's class distribution uniform through the logit adjustment loss, which is good for the test set. Similarly, in BOAT, we only replace $\Delta_c = \log P(Y|A=1) - \log P(Y|A=0)$ in equation (4) of \cite{boat}, which is adjusting a classifier's predictions from the labeled to the unlabeled class distribution, with our SimPro + DR estimate instead of their on-the-fly estimate. 


% \section{Additional experiments}
% % \begin{table*}[t]
\centering
\caption{Total Variation Distance on CIFAR-10-LT ($N_l = 500$, $M_l = 4000$) with different class imbalance ratios $\gamma_l$ and $\gamma_u$ under five different unlabeled class distributions.}
\label{tab:cifar10-tv}
\resizebox{\textwidth}{!}{
\begin{tabular}{lccccccccccc}
\toprule
& & \multicolumn{2}{c}{consistent} & \multicolumn{2}{c}{uniform} & \multicolumn{2}{c}{reversed} & \multicolumn{2}{c}{middle} & \multicolumn{2}{c}{head-tail} \\
\cmidrule(lr){3-4} \cmidrule(lr){5-6} \cmidrule(lr){7-8} \cmidrule(lr){9-10} \cmidrule(lr){11-12}
& & $\gamma_l = 150$ & $\gamma_l = 100$ & $\gamma_l = 150$ & $\gamma_l = 100$ & $\gamma_l = 150$ & $\gamma_l = 100$ & $\gamma_l = 150$ & $\gamma_l = 100$ & $\gamma_l = 150$ & $\gamma_l = 100$ \\
Model & Estimator & $\gamma_u = 150$ & $\gamma_u = 100$ & $\gamma_u = 1$ & $\gamma_u = 1$ & $\gamma_u = 1/150$ & $\gamma_u = 1/100$ & $\gamma_u = 150$ & $\gamma_u = 100$ & $\gamma_u = 150$ & $\gamma_u = 100$ \\
\midrule
Supervised & MLLS & 0.269 ± 0.252 & 0.038 ± 0.006 & 0.251 ± 0.046 & 0.255 ± 0.060 & 0.429 ± 0.028 & 0.493 ± 0.050 & 0.333 ± 0.042 & 0.320 ± 0.009 & 0.457 ± 0.034 & 0.444 ± 0.043 \\
Supervised & RLLS & 0.043 ± 0.001 & 0.044 ± 0.010 & 0.348 ± 0.034 & 0.305 ± 0.068 & 0.769 ± 0.016 & 0.678 ± 0.028 & 0.430 ± 0.008 & 0.368 ± 0.013 & 0.539 ± 0.018 & 0.503 ± 0.020 \\
\midrule
MLE & IPW & 0.027 ± 0.001 & 0.027 ± 0.000 & 0.319 ± 0.072 & 0.243 ± 0.010 & 0.674 ± 0.020 & 0.646 ± 0.041 & 0.438 ± 0.020 & 0.454 ± 0.026 & 0.547 ± 0.049 & 0.491 ± 0.059 \\
MLE & OR & 0.045 ± 0.004 & 0.042 ± 0.000 & 0.215 ± 0.026 & 0.203 ± 0.032 & 0.433 ± 0.017 & 0.395 ± 0.033 & 0.193 ± 0.006 & 0.209 ± 0.037 & 0.307 ± 0.147 & 0.249 ± 0.130 \\
MLE & DR & 0.090 ± 0.002 & 0.079 ± 0.000 & 0.407 ± 0.027 & 0.360 ± 0.007 & 0.425 ± 0.007 & 0.421 ± 0.029 & 0.256 ± 0.001 & 0.286 ± 0.031 & 0.435 ± 0.136 & 0.362 ± 0.122 \\
\midrule
EM & IPW & 0.035 ± 0.002 & 0.040 ± 0.001 & 0.021 ± 0.001 & 0.029 ± 0.015 & 0.303 ± 0.187 & 0.091 ± 0.010 & 0.119 ± 0.011 & 0.105 ± 0.022 & 0.104 ± 0.026 & 0.104 ± 0.051 \\
EM & OR & 0.037 ± 0.003 & 0.042 ± 0.002 & 0.016 ± 0.001 & 0.024 ± 0.012 & 0.269 ± 0.183 & 0.090 ± 0.008 & 0.122 ± 0.012 & 0.103 ± 0.022 & 0.072 ± 0.012 & 0.073 ± 0.024 \\
EM & DR & 0.034 ± 0.004 & 0.037 ± 0.001 & 0.014 ± 0.001 & 0.027 ± 0.020 & 0.264 ± 0.191 & 0.092 ± 0.005 & 0.111 ± 0.019 & 0.097 ± 0.026 & 0.077 ± 0.016 & 0.073 ± 0.028 \\
\midrule
SimPro & IPW & 0.070 ± 0.011 & 0.058 ± 0.000 & 0.046 ± 0.001 & 0.049 ± 0.005 & 0.254 ± 0.074 & 0.223 ± 0.098 & 0.097 ± 0.025 & 0.067 ± 0.002 & 0.105 ± 0.066 & 0.110 ± 0.079 \\
SimPro & OR & 0.071 ± 0.012 & 0.058 ± 0.000 & 0.045 ± 0.001 & 0.049 ± 0.006 & 0.040 ± 0.003 & 0.059 ± 0.017 & 0.074 ± 0.006 & 0.075 ± 0.002 & 0.033 ± 0.003 & 0.033 ± 0.003 \\
SimPro & DR & 0.017 ± 0.004 & 0.026 ± 0.001 & 0.019 ± 0.002 & 0.018 ± 0.003 & 0.039 ± 0.003 & 0.058 ± 0.025 & 0.091 ± 0.007 & 0.031 ± 0.001 & 0.015 ± 0.003 & 0.019 ± 0.007 \\
\bottomrule
\end{tabular}
}
\end{table*}
% 

\begin{table*}[t]
\centering
\caption{Total Variation Distance on CIFAR-100-LT ($N_l = 50$, $M_l = 400$) with different class imbalance ratios $\gamma_l$ and $\gamma_u$ under five different unlabeled class distributions.}
\label{tab:cifar100-tv}
\resizebox{\textwidth}{!}{
\begin{tabular}{lccccccccccc}
\toprule
& & \multicolumn{2}{c}{consistent} & \multicolumn{2}{c}{uniform} & \multicolumn{2}{c}{reversed} & \multicolumn{2}{c}{middle} & \multicolumn{2}{c}{head-tail} \\
\cmidrule(lr){3-4} \cmidrule(lr){5-6} \cmidrule(lr){7-8} \cmidrule(lr){9-10} \cmidrule(lr){11-12}
& & $\gamma_l = 20$ & $\gamma_l = 10$ & $\gamma_l = 20$ & $\gamma_l = 10$ & $\gamma_l = 20$ & $\gamma_l = 10$ & $\gamma_l = 20$ & $\gamma_l = 10$ & $\gamma_l = 20$ & $\gamma_l = 10$ \\
Model & Estimator & $\gamma_u = 20$ & $\gamma_u = 10$ & $\gamma_u = 1$ & $\gamma_u = 1$ & $\gamma_u = 1/20$ & $\gamma_u = 1/10$ & $\gamma_u = 20$ & $\gamma_u = 10$ & $\gamma_u = 20$ & $\gamma_u = 10$ \\
\midrule
Supervised & MLLS & 0.707 ± 0.016 & 0.313 ± 0.100 & 0.445 ± 0.172 & 0.309 ± 0.119 & 0.383 ± 0.075 & 0.397 ± 0.006 & 0.570 ± 0.001 & 0.373 ± 0.107 & 0.543 ± 0.009 & 0.231 ± 0.057 \\
Supervised & RLLS & 0.520 ± 0.007 & 0.133 ± 0.003 & 0.337 ± 0.125 & 0.253 ± 0.082 & 0.424 ± 0.060 & 0.463 ± 0.003 & 0.454 ± 0.021 & 0.306 ± 0.074 & 0.460 ± 0.028 & 0.241 ± 0.040 \\
\midrule
MLE & IPW & 0.075 ± 0.000 & 0.071 ± 0.001 & 0.229 ± 0.001 & 0.167 ± 0.002 & 0.565 ± 0.005 & 0.443 ± 0.007 & 0.415 ± 0.000 & 0.311 ± 0.005 & 0.343 ± 0.000 & 0.280 ± 0.001 \\
MLE & OR & 0.065 ± 0.002 & 0.061 ± 0.001 & 0.200 ± 0.007 & 0.143 ± 0.001 & 0.526 ± 0.011 & 0.399 ± 0.023 & 0.360 ± 0.003 & 0.256 ± 0.012 & 0.328 ± 0.003 & 0.266 ± 0.005 \\
MLE & DR & 0.149 ± 0.019 & 0.145 ± 0.010 & 0.243 ± 0.004 & 0.214 ± 0.019 & 0.568 ± 0.005 & 0.464 ± 0.014 & 0.403 ± 0.014 & 0.309 ± 0.012 & 0.365 ± 0.007 & 0.320 ± 0.004 \\
\midrule
EM & IPW & 0.097 ± 0.008 & 0.092 ± 0.004 & 0.239 ± 0.007 & 0.179 ± 0.003 & 0.478 ± 0.012 & 0.329 ± 0.020 & 0.262 ± 0.016 & 0.202 ± 0.003 & 0.312 ± 0.002 & 0.227 ± 0.001 \\
EM & OR & 0.121 ± 0.007 & 0.108 ± 0.005 & 0.261 ± 0.007 & 0.189 ± 0.004 & 0.489 ± 0.013 & 0.335 ± 0.020 & 0.274 ± 0.016 & 0.211 ± 0.004 & 0.336 ± 0.003 & 0.235 ± 0.001 \\
EM & DR & 0.125 ± 0.005 & 0.111 ± 0.004 & 0.269 ± 0.007 & 0.194 ± 0.005 & 0.497 ± 0.010 & 0.336 ± 0.024 & 0.281 ± 0.019 & 0.219 ± 0.008 & 0.336 ± 0.007 & 0.233 ± 0.004 \\
\midrule
SimPro & IPW & 0.125 ± 0.001 & 0.100 ± 0.005 & 0.166 ± 0.007 & 0.141 ± 0.009 & 0.353 ± 0.023 & 0.261 ± 0.008 & 0.202 ± 0.003 & 0.158 ± 0.005 & 0.277 ± 0.009 & 0.197 ± 0.003 \\
SimPro & OR & 0.133 ± 0.005 & 0.100 ± 0.004 & 0.160 ± 0.007 & 0.138 ± 0.010 & 0.322 ± 0.014 & 0.253 ± 0.008 & 0.202 ± 0.003 & 0.156 ± 0.005 & 0.269 ± 0.006 & 0.191 ± 0.004 \\
SimPro & DR & 0.122 ± 0.003 & 0.106 ± 0.006 & 0.188 ± 0.009 & 0.149 ± 0.006 & 0.343 ± 0.023 & 0.257 ± 0.007 & 0.219 ± 0.010 & 0.172 ± 0.002 & 0.279 ± 0.007 & 0.198 ± 0.004 \\
\bottomrule
\end{tabular}
}
\end{table*}

\end{document}
