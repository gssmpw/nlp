In this section, we will discuss the implications of our results. 

\subsection{Slightly Improved Rigidity Lower Bounds Would Yield Razborov Rigidity}
In \cref{sec:lower_bound}, we proved nearly tight rigidity lower bound for Kronecker powers (\cref{thm:kronecker_lb}) and the distance matrices (\cref{thm:hamming_lb}) in the small-rank regime. It is natural to ask whether we can improve the rigidity lower bound such that it works for better rank or sparsity parameters. However, the following theorems show that such an improvement may be very challenging: even improving our rigidity lower bounds slightly would imply a Razborov rigidity lower bound, a famous hard problem in the fields of matrix rigidity and communication complexity.


\StrongerKro*

\StrongerMaj*

\begin{proofof}{\cref{thm:stronger_lb_kro_imply_Razborov}}
    Assume to the contrary that $\BK{A_n}_{n \in \mathbb{N}}$ is not Razborov rigid. Thus, there is a constant $c > 0$, such that for any $n$, we have $\boolRigidity{\kro{A}}{2^{\log^c n}} \le \frac{1}{3} \cdot q^{2n}$. Let $k < n$ be a parameter to be determined. We can use \cref{thm:amplification_kro} on the matrix $\kro[k]{A}$ and its $(n/k)$-th Kronecker power $\kro{A}$: as $\boolRigidity{\kro[k]{A}}{2^{\log^c k}} \le \frac{1}{3} \cdot q^{2k}$, we get
    \begin{align*}
        \boolRigidity{\kro{A}}{2^{1 + \log^c k} \cdot n} \le q^{2n} \bk*{\frac{1}{2} - \frac{1}{2} \cdot \bk*{\frac{1}{6} - \alpha}^{n/k}},
    \end{align*}
    where $\alpha$ is the difference between the fraction of $1$'s and $-1$'s in $A^{\otimes k}$. In particular, since $A$ is not the all-$1$'s or all-$(-1)$'s matrix (since its rank is not $1$), it follows that for sufficiently large $k$, we have $\alpha < 1/12$.\footnote{One can show $\alpha < 1/12$ by directly counting, although it also follows from our rigidity lower bound above. We need to show that there are roughly the same number of $1$ and $-1$ entries in $\kro[k]{A}$. To argue this, one can use \cref{thm:kronecker_lb} on $\kro[k]{A}$ to get $\boolRigidity{\kro[k]{A}}{1} \ge \boolRigidity{\kro[k]{A}}{c_1 k} \ge q^{2k} \bk*{\frac{1}{2} - c_2^k}$, which means both the all-$1$'s and all-$(-1)$'s matrices are bad approximations of $\kro[k]{A}$, so there are roughly the same number of $1$ and $-1$ entries in $\kro[k]{A}$.} Picking $k \defeq 2^{\bk{\eps\log n / 2}^{1/c}}$, so that $2^{\log^c k} = n^{\eps/2}$, we get
    \begin{align*}
        \boolRigidity{\kro{A}}{n^{1 + \eps}} 
        \le \boolRigidity{\kro{A}}{2^{1 + \log^c k} \cdot n} 
        \le q^{2n} \bk*{\frac{1}{2} - \frac{1}{2} \cdot {\frac{1}{12^{n/ 2^{\bk{\eps\log n / 2}^{1/c}}}}}}
        < q^{2n} \bk*{\frac{1}{2} - \frac{1}{2^{n/2^{(\log n)^{o(1)}}}}},
    \end{align*}
    contradicting \eqref{ineq:kro_rigidity_lb_for_larger_rank}.
\end{proofof}

\begin{proofof}{\cref{thm:stronger_lb_maj_imply_Razborov}}
    Simialr to the proof of \cref{thm:stronger_lb_kro_imply_Razborov}, we assume to the contrary that $\BK{M_n}_{n \in \mathbb{N}}$ is not Razborov rigid with $\boolRigidity{M_n}{2^{\log^c n}} \le \frac{1}{3} \cdot 4^{n}$ for some constant $c > 0$, and apply \cref{thm:amplification_maj} on the matrices $M_k$ and $M_n$ to get
    \begin{align*}
        \boolRigidity{M_n}{2^{\log^c k}} \le 4^{n} \bk*{\frac{1}{2} - \Omega\bk*{\frac{\sqrt{k}}{\sqrt{n}}}}.
    \end{align*}
    Taking $k = 2^{\bk{\log\log n + \log \beta}^{1/c}}$, we have $2^{\log^c k} = \beta \log n$, and 
    \begin{align*}
        \boolRigidity{M_n}{\beta\log n}
        \le 4^{n}\bk*{\frac{1}{2} - \Omega\bk*{\frac{2^{\frac{1}{2}\bk{\log\log n + \log \beta}^{1/c}}}{n^{1/2}}}}
        < 4^n \bk*{\frac{1}{2} - \frac{2^{\bk*{\log \log n}^{o(1)}}}{n^{1/2}}},
    \end{align*}
    contradicting \eqref{ineq:maj_rigidity_lb_for_larger_rank}.
\end{proofof}

\subsection{Obstructions to Constructing Depth-2 Circuits via Rigidity} \label{sec:obstructiondepth2}

One important application of matrix rigidity in the small-rank regime is that rigidity upper bounds for matrix $A$ can be used to construct small depth-$d$ linear circuits to compute the Kronecker power $\kro{A}$ of $A$. Specifically, \cite{alman2021kronecker} proved that
\begin{lemma}[\cite{alman2021kronecker}, Lemma 1.4]
    \label{lm:depth-2_circuit}
    Let $A \in \BK{-1,1}^{q\times q}$ be a matrix over some field $\F_p$. For any rank parameter $r \le q$, we define
    \begin{align*}
        c \defeq \log_q \bk{(r + 1) \cdot (r + \R_{A}(r) / q)}.
    \end{align*}
    Then, for any positive integer $n$, the Kronecker power $\kro{A}$ has a depth-$d$ synchronous circuit of size $O(d \cdot q^{(1 + c/d)n})$.
\end{lemma}

By applying this lemma, \cite{alman2021kronecker} obtained a depth-2 synchronous circuit of size $O(2^{1.47582n})$ for Walsh-Hadamard matrix $H_n$ from a rank-1 rigidity upper bound $\R_{H_4}(1) \le 96$, improving on the folklore bound of $O(2^{1.5n})$.

Notably, although most prior work on matrix rigidity had given \emph{asymptotic} upper and lower bounds, it is unclear how to use asymptotic rigidity upper bounds for $H_n$, such as those from~\cite{alman2017probabilistic}, to construct improved circuits in this way (see \cite[Page 2]{alman2021kronecker}). Alman instead used the constant-sized bound $\R_{H_4}(1)$. It was later shown that rank-1 rigidity upper bounds cannot give any further improvements~\cite{alman2023smaller}.

We are able to use our new rigidity lower bound, \cref{thm:kronecker_lb}, to prove that only constant-sized rigidity upper bounds may be effective in conjunction with Lemma~\ref{lm:depth-2_circuit}. We prove that, starting from the matrix $A = H_k$ with a large $k$ and picking any rank $1 < r \le O(k)$, the depth-2 circuit for $H_n$ constructed by \cref{lm:depth-2_circuit} will have size at least $\Omega(2^{3n/2})$.
\begin{lemma}
    Let $c_1, c_2$ be the constants in \cref{thm:kronecker_lb}, and $k$ is a large parameter. For any rank $1 < r \le c_1 k$, we have 
    \begin{align*}
        \log_{2^k} \bk{(r + 1) \cdot (r + \R_{H_k}(r) / 2^k)} \ge 1.
    \end{align*}
\end{lemma}
\begin{proof}
    By \cref{thm:kronecker_lb}, we have $\R_{H_k}(r) \ge \R_{H_k}(c_1 n) \ge 4^k \bk{1/2 - c_2^k} \ge 4^k /3$ for large $k$. Hence, 
    \begin{align*}
        &(r + 1) \cdot (r + \R_{H_k}(r) / 2^k) 
        > 3 \R_{H_k}(r) / 2^k \ge 2^k
    \end{align*}
    as $r \ge 2$, which proves the desried inequality.
\end{proof}

