\newcommand{\UinC}{\tilde{U}}
\newcommand{\VinC}{\tilde{V}}
\newcommand{\rinC}{\tilde{r}}
\newcommand{\LinC}{\tilde{L}}
\newcommand{\vecUinCsub}{\tilde{\vec{u}_i}^{(\text{sub})}}
\newcommand{\vecVinCsub}{\tilde{\vec{v}_j}^{(\text{sub})}}
\newcommand{\vecUinC}{\tilde{\vec{u}_i}}
\newcommand{\vecVinC}{\tilde{\vec{v}_j}}
\newcommand{\vecUinCrow}{\tilde{\vec{u}}_{(k)}}
\newcommand{\vecVinCrow}{\tilde{\vec{v}}_{(k)}}

In this section, we will prove \cref{thm:rigidity_lb_from_singular_value} which establishes a rigidity lower bound for any binary matrix in a finite field in terms of its largest singular value. After that, we will apply \cref{thm:rigidity_lb_from_singular_value} to get nearly tight rigidity lower bounds for two types of matrices in small ranks.

\subsection{Rigidity Lower Bounds by Largest Singular Value}
In this subsection, we prove \cref{thm:rigidity_lb_from_singular_value}. 

\RigidityLbFromSingularValue*

The key idea behind the proof is to consider the matrix $A$ over the number field $\mathbb{C}$, even though we are arguing about its (Boolean) rigidity over $\F_p$. Specifically, the proof consists of two parts:
\begin{itemize}
    \item Showing that if a binary matrix can be approximated well by some low-rank matrix in the \emph{finite field $\F_p$}, then it can also be approximated well by some binary ``low-rank'' matrix in the \emph{number field $\mathbb{C}$} (for a more restricted notion than low-rank which we will define shortly, where the low-rank decomposition must use small entries).
    \item Bounding the ``rigidity'' (using this restricted rank notion) of binary matrices in $\mathbb{C}$ in terms of their largest singular values.
\end{itemize}

Before getting to the full proof, we give the key technical lemma behind the first step:

\begin{lemma}
\label{lm:transformation_of_low-rank_matrix_to_C}
    Suppose $L\in \F_p^{N \times N}$ is a matrix of rank $r$ over $\F_p$. Then, the matrix $\LinC\in \BK{-1,1}^{N \times N}$, defined as the entry-wise Booleanization of $L$, i.e., 
    \[\LinC[i,j] = \bool(L[i,j]), \quad \forall (i,j)\in [N]^2,\]
    satisfies the following properties:
    \begin{itemize}
        \item Viewed as a matrix over $\C$, the rank of $\LinC$ over $\C$ is at most $\rinC \defeq (p^3 + 1)^r$.
        \item Moreover, $\LinC$ admits a low-rank decomposition $\LinC = \UinC^\top \VinC$ over $\C$, where $\UinC, \VinC \in \mathbb{C}^{\rinC \times N}$ and each entry of $\UinC$ and $\VinC$ has magnitude at most $C^r$ for some constant $C$ which depends only on $p$.
    \end{itemize}
\end{lemma}

\begin{proof}
    Since $L$ is a rank-$r$ matrix over $\F_p$, it admits a low-rank decomposition $L \overset{\F_p}{=} U^{\top} V$, where $U, V \in \F_p^{r \times N}$. We can interpret $U$, $V$ as integer matrices from $\Z^{r \times N}$ whose entries are integers between $0$ and $p-1$, and express the decomposition as $L \equiv U^{\top} V \pmod{p}$.

    Let $\omega \defeq e^{2\pi i/p}$ be a $p$-th root of unity. By polynomial interpolation, there is a polynomial $f \in \mathbb{C}[x]$ with degree at most $p$, such that $f(\omega^{k}) = \bool \bk{k}$ for all $k \in [p]$. Then, the entries of $\LinC$ can be written as
        \begin{align*}
        \LinC[i,j] = \bool(L[i,j]) = f\bk*{\omega^{L[i,j]}} = f\bk*{\omega^{(U^{\top} V)[i,j]}}, \quad \forall (i,j) \in [N]^2.  \numberthis \label{ineq:lb_LinC_entries_def}
    \end{align*}
    We will use \eqref{ineq:lb_LinC_entries_def} to construct the desired low-rank decomposition of $\LinC$ below.

    Let $\vec{u}_i$ be the $i$-th column vector of $U$ for any $i \in [N]$, and $\vec{v}_j$ similarly be the $j$-th column vector of $V$. Write $\vec{u}_i = \bk{u_{i1}, \ldots, u_{ir}}^{\top}$ and $\vec{v}_j = \bk{v_{j1}, \ldots, v_{jr}}^{\top}$. We will expand $\LinC[i,j]$ as a sparse polynomial in  $u_{i1}v_{j1}, \ldots, u_{ir}v_{jr}$, namely
    \begin{align*}
        \LinC[i,j] = F\bk*{u_{i1}v_{j1}, \ldots, u_{ir}v_{jr}},  \numberthis \label{eq:L=F(uv) in sec_rigidity_lb}
    \end{align*}
    where $F$ is a $r$-variable polynomial which we construct next, which has degree at most $p^{3}$ in each variable.

    By polynomial interpolation, there is a polynomial $g \in \mathbb{C}[x]$ with degree at most $p^2$, such that $g(k) = \omega^{k}$, for all $k \in \Bk[big]{p^2}$.
    Expand $g$ into a sum of monomials as $g(x) = c_0 + c_1 x + \cdots + c_{p^2} x^{p^2}$. Then,
    \begin{align*}
        \omega^{(U^{\top} V)[i,j]} = {\prod_{k=1}^{r} \omega^{u_{ik} v_{jk}}}
        = {\prod_{k=1}^r g(u_{ik} v_{ik})}
        = {\prod_{k=1}^r \bk*{c_0 + c_1 \bk{u_{ik} v_{ik}} + \cdots + c_{p^2} \bk{u_{ik} v_{ik}}^{p^2}}},
    \end{align*}
    meaning that there is a $r$-variable polynomial $G$ of degree at most $p^{2}$ in each variable, such that 
    $\omega^{(U^{\top} V)[i,j]} = G\bk*{u_{i1}v_{j1}, \ldots, u_{ir}v_{jr}}$. Then, as $f$ is a polynomial of degree at most $p$, 
    \begin{align*}
        \LinC[i,j] = f\bk*{\omega^{(U^{\top} V)[i,j]}} = f\bk[BBig]{G\bk*{u_{i1}v_{j1}, \ldots, u_{ir}v_{jr}}}
    \end{align*}
    can be further expressed as $F(u_{i1}v_{j1}, \ldots, u_{ir}v_{jr})$, where $F \defeq f \circ G$ is a polynomial of degree at most $p^{3}$ in each variable. Moreover, expanding $F$ as
    \begin{align*}
        F(u_{i1}v_{j1}, \ldots, u_{ir}v_{jr}) = \sum_{\vec{\alpha} \in \Bk{p^3 + 1}^{r}} C_{\vec{\alpha}} \vec{u}_i^{\vec{\alpha}} \vec{v}_i^{\vec{\alpha}},
    \end{align*}
    we have further that each coefficient $C_{\vec{\alpha}}$ is bounded in magnitude by $C^{r}$ for some constant $C$ (depending on $p$). This is because both $f$ and $g$ are polynomials that only depend on $p$, with their $\l_1$-norm (sum of magnitudes of all the coefficients) being bounded by constants $C_f, C_g$, respectively, that only depend on $p$. Then, the $\l_1$-norm of $G$ is bounded by $\bk{C_g}^r$ and the $\l_1$-norm of $F$ is bounded by $C_f \cdot \bk{C_g}^{rp}$, which means that each coefficient $C_{\vec{\alpha}}$ of $F$ is bounded by $C_f \cdot \bk{C_g}^{rp} \eqdef C^r$ for some constant $C$ that only depends on $p$.

    With the low-degree polynomial expression \eqref{eq:L=F(uv) in sec_rigidity_lb} of $\LinC$, we 
    can construct matrices $\UinC, \VinC \in \mathbb{C}^{\rinC \times N}$ such that $\LinC = \UinC^{\top} \VinC$, as follows:
    \begin{itemize}
        \item Let $\vecUinC$ be the $i$-th column vector of $\UinC$ for any $i \in [N]$ and $\vecVinC$ be the $j$-th column vector of $\VinC$. Both $\vecUinC$ and $\vecVinC$ are $\bk{p^3 + 1}^r$-dimensional vectors, with their coordinates indexed by all possible $\vec{\alpha} \in \Bk{p^3 + 1}^r$.
        \item In $\vecUinC$, the entry indexed by $\vec{\alpha}$ is $\vecUinC[\vec{\alpha}] \defeq C_{\vec{\alpha}} \vec{u}_i^{\vec{\alpha}}$.
        \item In $\vecVinC$, the entry indexed by $\vec{\alpha}$ is $\vecVinC[\vec{\alpha}] \defeq \vec{v}_j^{\vec{\alpha}}$.
    \end{itemize}
    Then, we can check $\LinC = \UinC^{\top} \VinC$ by 
    \begin{align*}
        \UinC^{\top} \VinC[i,j] = \angbk{\vecUinC, \vecVinC} = \sum_{\vec{\alpha} \in \Bk{p^3 + 1}^{r}} C_{\vec{\alpha}} \vec{u}_i^{\vec{\alpha}} \vec{v}_i^{\vec{\alpha}} = \LinC[i,j], \quad \forall (i,j) \in [N]^2.
    \end{align*}
    Hence, the matrix $\LinC$ has rank at most $\rinC = (p^3+1)^r$ over $\mathbb{C}$, with a low-rank decomposition $\LinC = \UinC^{\top} \VinC$ such that each entry of $\UinC$ and $\VinC$ is bounded in magnitude by $C^{r}$ for some constant $C$ which depends only on $p$.
    \end{proof}

We now prove \cref{thm:rigidity_lb_from_singular_value} using \cref{lm:transformation_of_low-rank_matrix_to_C}.
\begin{proofof}{\cref{thm:rigidity_lb_from_singular_value}}
    Suppose that $\boolRigidity{A}{r} = s$, then by the definition of matrix rigidity, 
    there is a rank-$r$ matrix $L$ in $\F_p$, such that 
    \begin{align*}
        \bool\bk{A[i,j]} = \bool\bk{L[i,j]}, \quad \forall (i,j) \in I, \numberthis \label{eq:A=UV mod p in sec_rigidity_lb}
    \end{align*}
    for some set $I \subset [N]^2$ of indices with $\abs{I} \ge N^2 - s.$ 
    By applying \cref{lm:transformation_of_low-rank_matrix_to_C} on $L$, the matrix $\LinC$ defined by applying Booleanization entry-wise on $L$ admits a low-rank decomposition $\LinC = \UinC^\top \VinC$ over $\C$, where $\UinC, \VinC \in \C^{\rinC \times N}$ with each entry being bounded in magnitude by $C^r$. Hence, \eqref{eq:A=UV mod p in sec_rigidity_lb} can be written as 
    \begin{align*}
        A[i,j] = \bool\bk{A[i,j]} = \bool\bk{L[i,j]} = \LinC[i,j] = \bk*{\UinC^\top \VinC}[i,j], \quad \forall (i,j)\in I.\numberthis \label{eq:lb_wirte_rigidity_over_C}
    \end{align*}
    
    Now we use \eqref{eq:lb_wirte_rigidity_over_C} to give a lower bound on the sparsity $s = N^2 - \abs{I}$ in terms of the rank $r$ and the largest singular value $\sigma_1$ of $A$. As $A$ and $\LinC$ are both binary matrices taking values from $\BK{-1, 1}$, we have
    \begin{align*}
        s = \nnz(A - \LinC) = \frac{1}{4}\sum_{(i,j) \in [N]^2} \bk*{A[i,j] - \LinC[i,j]}^2
        = \frac{1}{4}\sum_{(i,j) \in [N]^2} \bk*{2 - A[i,j] \LinC[i,j]}. \numberthis \label{eq:s < sum(2 - AL) in sec_rigidity_lb}
    \end{align*}
    As $\LinC$ is a rank-$\rinC$ matrix with $\LinC = \UinC^{\top} \VinC$, we can write $\LinC = \sum_{k=1}^{\rinC} {\vecUinCrow}^{\top} \vecVinCrow$, where $\vecUinCrow$ and $\vecVinCrow$ are the $k$-th row vector of $\UinC$ and $\VinC$, respectively. Moreover, for any $k \le \rinC$, 
    \begin{align*}
        &\abs[BBig]{\sum_{(i,j) \in [N^2]} A[i,j] \bk*{{\vecUinCrow}^{\top} \vecVinCrow}[i,j]}
        =\abs[BBig]{\sum_{(i,j) \in [N^2]} A[i,j] {\vecUinCrow}[i] \vecVinCrow[j]}\\
        {}={}& \abs*{\vecUinCrow A \vecVinCrow^{\top}}
        {}\le{} \norm[Big]{\vecUinCrow}_2 \norm[Big]{A \vecVinCrow^{\top}}_2
        {}\le{} \norm[Big]{\vecUinCrow}_2 \cdot \bk*{\sigma_1 \norm[Big]{\vecVinCrow^{\top}}_2}
        {}\le{} \sigma_1 C^{2r} N,
    \end{align*}
    where the first inequality is the Cauchy-Schwartz inequality, the second inequality uses the definition of the largest singular value $\sigma_1$ and the third inequality is because each entry of $\vecUinCrow$ and $\vecVinCrow$ is bounded by $C^r$, hence $\norm[Big]{\vecUinCrow}$ and $\norm[Big]{\vecVinCrow}$ are both bounded by $C^r \sqrt{N}$. Summing over all $k$'s, we get 
    \begin{align*}
        \abs[BBig]{\sum_{(i,j) \in [N^2]} A[i,j] \LinC[i,j]}
        \le \sum_{k=1}^{\rinC} \abs[BBig]{\sum_{(i,j) \in [N^2]} A[i,j] \bk*{{\vecUinCrow}^{\top} \vecVinCrow}[i,j]}
        \le \sigma_1 C^{2r} \rinC N. \numberthis \label{eq: sum(AL) < C^r in sec_rigidity_lb}
    \end{align*}
    Plugging \eqref{eq: sum(AL) < C^r in sec_rigidity_lb} to \eqref{eq:s < sum(2 - AL) in sec_rigidity_lb}, we get 
    \begin{align*}
        s \ge \frac{N^2}{2} - \frac{1}{4} \abs[BBig]{\sum_{(i,j) \in [N^2]} A[i,j] \LinC[i,j]}
        \ge N^2 \bk*{\frac{1}{2} - \frac{\sigma_1 C^{2r} \rinC}{4N} }. \numberthis \label{eq:s < N^2(1/2-tiny) in sec_rigidity_lb}
    \end{align*}
    As $\rinC \defeq \bk{p^3 + 1}^r$ is also exponential in $r$, there is a constant $c$ (depending on $p$) such that $C^{2r} \rinC \le c^r$ for any $r \in \N$. Hence, we can conclude the desired result from \eqref{eq:s < N^2(1/2-tiny) in sec_rigidity_lb} that
    \begin{align*}
        &\boolRigidity{A}{r} \ge N^2 \bk*{\frac{1}{2} - \frac{c^r \sigma_1}{N}}. \qedhere
    \end{align*}
\end{proofof}

\subsection{Rigidity Lower Bound for Kronecker Matrices}
In this subsection, we apply \cref{thm:rigidity_lb_from_singular_value} to Kronecker matrices to get nearly tight rigidity lower bounds for small ranks.

\KroneckerLb*

    According to \cref{thm:rigidity_lb_from_singular_value}, in order to prove Theorem~\ref{thm:kronecker_lb}, it suffices to bound the largest singular value of $\kro{A}$. The following lemma will help us to do this.
    \begin{lemma}
        \label{lm:singular_value_of_A}
        For any matrix $A \in \BK{-1,1}^{q \times q}$ with $\rank(A) > 1$, the largest singular value $\sigma_1 \defeq \sigma_1(A)$ of $A$ is strictly smaller than $q$.
    \end{lemma}
    \begin{proof}
        By the definition of singular value, $\sigma_1^2$ is one of the eigenvalues of the matrix $B = A^{\top} A$, hence there is an eigenvector $\vec{v} \in \mathbb{C}^{q}$ such that 
        $ B\vec{v} = A^{\top} A \vec{v} = \sigma_1^2 \vec{v}.$ 
        Suppose $\vec{v} = \bk{v_1, \ldots, v_q}$ and assume $j \in [q]$ is the index maximizing $\abs{v_j}$. Then,
        \begin{align*}
            \sigma_1^2 \abs{v_j} = \abs[BBig]{\bk*{B \vec{v}}[j]}
            = \abs*{\sum_{i=1}^{q} B[i,j] v_i}
            \le \sum_{i=1}^q \abs[BBig]{B[i,j]} \abs[Big]{v_i}
            \le \sum_{i=1}^q \abs[BBig]{B[i,j]} \abs[Big]{v_j}.
        \end{align*}
        Hence, $\sigma_1^2 \le \sum_{i=1}^q \abs{B[i,j]}$. Moreover, as $\abs*{B[i,j]} = \abs*{\sum_{\l = 1}^q A[\l, i] A[\l, j]} \le q$ for any $i \in [q]$, we get $\sigma_1^2 \le q^2$, i.e., $\sigma_1 \le q$.

        We further check the equality $\sigma_1 = q$ cannot hold. Otherwise, we have $\abs*{\sum_{\l = 1}^q A[\l, i] A[\l, j]} = q$, which means $A[\l, i] = A[\l, j]$ for all $i, \l \in [q]$, i.e., all the columns of $A$ are same as the $j$-th column. This means that $A$ has rank 1, a contradiction. Hence, we have $\sigma_1 < q$, as desired.
    \end{proof}

\begin{proofof}{\cref{thm:kronecker_lb}}
    Using \cref{lm:singular_value_of_A}, the largest singular value $\sigma_1(\kro{A})$ of $\kro{A}$ can be represented as $\sigma_1^n$, where $\sigma_1 < q$ is the largest singular value of $A$. Let $c_1 > 0$ be a parameter to be determined. By \cref{thm:rigidity_lb_from_singular_value}, the rigidity of $\kro{A}$ can be bounded as 
    \begin{align*}
        \boolRigidity{\kro{A}}{c_1 n} 
        \ge q^n \bk*{\frac{1}{2} - \frac{c^{c_1 n} \cdot \sigma_1^n}{q^n}}
        = q^n \bk*{\frac{1}{2} - \bk*{\frac{c^{c_1}\sigma_1}{q}}^n}
    \end{align*}
    for some constant $c>0$ which is defined in \cref{thm:rigidity_lb_from_singular_value}. As $\sigma_1 < q$, there is a sufficiently small constant $c_1 > 0$ such that $c_2 \defeq {c^{c_1}\sigma_1}/{q} < 1$, which implies the desired result.
\end{proofof}

\subsection{Rigidity Lower Bound for the Distance Matrix}
For any $n \in \N$, we define the $2^n \times 2^n$ matrix $M_n$ called the \defn{distance matrix} as $M_n \defeq \maj{A}$, where $A = \bk*{\begin{matrix}
    1 & -1\\
    -1 & 1
\end{matrix}}$. In this subsection, we prove the following rigidity lower bound for $M_n$:
\HammingLb*

    Similar to the previous subsection, we need to bound the largest singular value of $M_k$. Here we explicitly compute the singular values as follows:
    \begin{claim}
    \label{clm:list_eigenvalues}
        All the eigenvalues of $M_n$ are listed below:
        \begin{align*}
            \lambda_y = \sum_{z \in \BK{0,1}^n} \bk*{2\indicator{|z| \le n/2}-1} (-1)^{\angbk{y,z}},\quad \forall y \in \BK{0,1}^n.
        \end{align*}
    \end{claim}
    \begin{proof}
        By the definition of the distance matrix $M_n = \maj{A}$, we can represent each entry of $M_n$ as 
        \begin{align*}
            M_n[x,y] = \Maj\bk{A[x_1,y_1], \ldots, A[x_n, y_n]} = 2\indicator{|{x-y}| \le n/2}-1
        \end{align*}
        for each $x, y \in \BK{0,1}^n$. Below, we check that $\lambda_y$ is an eigenvalue of $M_n$, with associated eigenvector $v_y \in \BK{-1,1}^{2^n}$ defined as $v_y[x] = (-1)^{\angbk{y,x}}$ for any $x \in \BK{0, 1}^n$: 
        \begin{align*}
            M_n v_y [x] 
            {}={}& \sum_{z \in \BK{0,1}^n} M_n[x,z] v_y[z]
            = \sum_{z \in \BK{0,1}^n} \bk*{2\indicator{|{x-z}| \le n/2}-1} (-1)^{\angbk{y,z}}\\
            {}={}& (-1)^{\angbk{y,x}}\sum_{z \in \BK{0,1}^n} \bk*{2\indicator{|{x\oplus z}| \le n/2}-1} (-1)^{\angbk{y,x\oplus z}}\\
            {}={}& v_y[x]\sum_{z \in \BK{0,1}^n} \bk*{2\indicator{|{z}| \le n/2}-1} (-1)^{\angbk{y,z}}
            {}={} \lambda_y v_y[x].
        \end{align*}
        Here, we use $x \oplus z$ to denote the bit-wise XOR of $x$ and $z$: $(x_1 \oplus z_1, \ldots, x_n \oplus z_n)$. As the vectors $v_y$ form an orthogonal basis, we have enumerated all the eigenvalues of $M_n$ as  $\BK{\lambda_y : y \in \BK{0,1}^n}$.
    \end{proof}

Now that we have listed all the singular values of $M_n$, we can bound the largest one.

    \begin{claim}
    \label{clm:singular_bound_hamming}
        The largest singular value $\sigma_1(M_n)$ of $M_n$ is bounded above by $O\bk{2^n/\sqrt{n}}$.
    \end{claim}
    \begin{proof}
        As $M_n$ is a symmetric matrix, we only need to show that for any eigenvalue $\lambda_y$ defined in \cref{clm:list_eigenvalues}, we have $\abs{\lambda_y} = O\bk{2^n/\sqrt{n}}$.

        If $y$ is the zero vector, i.e., $y = (0, \ldots, 0)$, then 
            $\lambda_y = \sum_{z \in \BK{0,1}^n} \bk*{2\indicator{|{z}| \le n/2}-1}.$
        For each $z \in \BK{0,1}^n$ with $|{z}| < n/2 $, we pair it with $\overline{z} \defeq (1-z_1, \ldots, 1-z_n)$ in the sum above, so that the paired summands will cancel out since  
        $\bk*{2\indicator{|{z}| \le n/2}-1} + \bk*{2\indicator{|{\overline{z}}| \le n/2}-1} = 0.$
        Hence, 
        \begin{align*}
            |\lambda_y| 
            = \abs*{\sum_{\substack{z \in \BK{0,1}^n, |z| = n/2}} \bk*{2\indicator{|{z}| \le n/2}-1}}
            \le {\sum_{\substack{z \in \BK{0,1}^n\\ |z| = n/2} } 1} 
            \le \binom{n}{n/2} = O\bk*{2^n/\sqrt{n}}.
        \end{align*}

        Otherwise, $y$ is non-zero, say, $y_1 \neq 0$. Then, for each $z\in \BK{0,1}^n$ with $|z| \notin \Bk*{\frac{n}{2}-1, \frac{n}{2}+1}$, we pair it with $z^{\oplus 1} \defeq (1 - z_1, z_2, \ldots, z_n)$. Thus, $\indicator{|z| \le n/2} = \indicator{|z^{\oplus 1}| \le n/2}$ and $(-1)^{\angbk{y,z}} + (-1)^{\angbk{y, z^{\oplus 1}}} = 0$, hence their corresponding terms in the sum defining $\lambda_y$ again cancel out. Thus,
        \begin{align*}
            |\lambda_y| 
            =& \abs*{\sum_{\substack{z \in \BK{0,1}^n, |z| \in \Bk*{\frac{n}{2} - 1, \frac{n}{2} + 1}}} \bk*{2\indicator{|{z}| \le n/2}-1}}
            \le {\sum_{\substack{z \in \BK{0,1}^n\\ |z| \in \Bk*{\frac{n}{2} - 1, \frac{n}{2} + 1} } }1} 
            = O\bk*{2^n/\sqrt{n}}. \qedhere
        \end{align*}
    \end{proof} 
\begin{proofof}{\cref{thm:hamming_lb}}
    Plugging the singular value bound in \cref{clm:singular_bound_hamming} into \cref{thm:rigidity_lb_from_singular_value}, we show that there is a constant $c > 1$, such that for any rank $r$,
    \begin{align*}
        &\boolRigidity{M_n}{r}
        \ge 4^n \bk*{\frac{1}{2} - \frac{c^r \cdot 2^n /\sqrt{n}}{2^n}}
        = 4^n \bk*{\frac{1}{2} - \frac{c^r }{\sqrt{n}}}. 
    \end{align*}
    For any $\eps > 0$, we can take a sufficiently small constant $\beta > 0$ such that $\beta \log c < \eps$, then 
    \begin{align*}
        &\boolRigidity{M_n}{\beta \log n}
        \ge 4^n \bk*{\frac{1}{2} - \frac{c^{\beta \log n} }{\sqrt{n}}}
        > 4^n \bk*{\frac{1}{2} - \frac{1 }{n^{1/2 - \eps}}}. \qedhere
    \end{align*}
\end{proofof}
