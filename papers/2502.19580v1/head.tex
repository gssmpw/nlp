\usepackage[T1]{fontenc}
\usepackage{amsfonts,amsmath,amsthm,amssymb,mathtools}  %

\mathtoolsset{centercolon}
\usepackage{xfrac,nicefrac}
\usepackage{xcolor}
\usepackage{mathdots}
\usepackage{mleftright}  %
\let\left\mleft
\let\right\mright

\usepackage{xspace}
\xspaceaddexceptions{]\}}  %
\usepackage{regexpatch}

\usepackage{bm,bbm,dsfont}  %
\usepackage{caption}
\usepackage[normalem]{ulem}
\usepackage{enumitem}

\usepackage{graphicx}
\usepackage{float}
\usepackage{subcaption}  %
\usepackage{tcolorbox}
\usepackage{tikz}
\usetikzlibrary{decorations.pathreplacing}
\usetikzlibrary{calc}
\usetikzlibrary{positioning}
\usetikzlibrary{arrows.meta}

\usepackage[linesnumbered,boxed,ruled,vlined]{algorithm2e}
\usepackage{algpseudocode}

\usepackage{thmtools,thm-restate}
\theoremstyle{plain}
\newtheorem{prob}{Problem}  %
\newtheorem{open}[prob]{Open Question}
\newtheorem{theorem}{Theorem}[section]  %
\newtheorem{lemma}[theorem]{Lemma}
\newtheorem{fact}[theorem]{Fact}
\newtheorem{observation}[theorem]{Observation}
\newtheorem{prop}[theorem]{Proposition}
\newtheorem{cor}[theorem]{Corollary}
\newtheorem{claim}[theorem]{Claim}
\theoremstyle{definition}  %
\newtheorem{definition}[theorem]{Definition}
\newtheorem{remark}[theorem]{Remark}
\newtheorem{property}[theorem]{Property}
\newenvironment{proofof}[1]{\begin{proof}[Proof of #1]}{\end{proof}}
\newenvironment{proofsketch}{\begin{proof}[Proof Sketch]}{\end{proof}}
\newenvironment{proofsketchof}[1]{\begin{proof}[Proof Sketch of #1]}{\end{proof}}

\usepackage[colorlinks,citecolor=blue,linkcolor=blue,urlcolor=red]{hyperref}
\usepackage[capitalise]{cleveref}
\crefname{algocf}{Algorithm}{Algorithms}
\Crefname{algocf}{Algorithm}{Algorithms}
\crefname{prob}{Problem}{Problems}
\crefname{claim}{Claim}{Claims}
\crefname{cor}{Corollary}{Corollaries}
\crefname{fact}{Fact}{Facts}

\DeclarePairedDelimiter{\ceil}{\lceil}{\rceil}
\DeclarePairedDelimiter{\floor}{\lfloor}{\rfloor}
\DeclarePairedDelimiter{\norm}{\lVert}{\rVert}
\DeclarePairedDelimiter{\bk}{(}{)}
\DeclarePairedDelimiter{\Bk}{[}{]}
\DeclarePairedDelimiter{\BK}{\{}{\}}
\DeclarePairedDelimiter{\rb}{(}{)}
\DeclarePairedDelimiter{\sqb}{[}{]}
\DeclarePairedDelimiter{\cb}{\{}{\}}
\DeclarePairedDelimiter{\angbk}{\langle}{\rangle}
\DeclarePairedDelimiter{\abs}{\lvert}{\rvert}

\DeclareMathOperator*{\E}{\mathbb{E}}
\DeclareMathOperator*{\PrAux}{Pr}
\let\Pr\PrAux
\DeclareMathOperator{\poly}{poly}
\DeclareMathOperator{\polylog}{polylog}
\DeclareMathOperator*{\argmax}{arg\,max}
\DeclareMathOperator*{\argmin}{arg\,min}

\newcommand{\F}{\mathbb{F}}
\renewcommand{\d}{\mathrm{d}}
\renewcommand{\tilde}{\widetilde}
\newcommand{\eqdef}{\eqqcolon}
\newcommand{\defeq}{\coloneqq}
\newcommand{\eps}{\varepsilon}
\newcommand{\T}{\mathcal{T}}
\newcommand{\N}{\mathbb{N}}
\newcommand{\R}{\mathbb{R}}
\newcommand{\Z}{\mathbb{Z}}
\renewcommand{\l}{\ell}
\renewcommand{\emptyset}{\varnothing}
\renewcommand{\epsilon}{\eps}
\newcommand{\Patrascu}{\textup{P{\v{a}}tra{\c{s}}cu}\xspace}

\newcommand{\numberthis}{\addtocounter{equation}{1}\tag{\theequation}}
\newcommand{\smallsub}{\scriptscriptstyle}
\newcommand{\tallsub}{{\textstyle\mathstrut}}
\newcommand{\tall}{\vphantom\sum}
\newcommand{\matwrap}[1]{{\begin{matrix}#1\end{matrix}}}
\newcommand{\matwrapdisplay}[1]{{\begin{matrix}\displaystyle #1\end{matrix}}}


\usepackage{regexpatch}
\makeatletter
\xpatchcmd\thmt@restatable{%
\csname #2\@xa\endcsname\ifx\@nx#1\@nx\else[{#1}]\fi
}{%
\ifthmt@thisistheone
\csname #2\@xa\endcsname\ifx\@nx#1\@nx\else[{#1}]\fi
\else
\csname #2\@xa\endcsname[{Restated}]
\fi}{}{}
\makeatother

\newcommand{\creflastconjunction}{, and\nobreakspace}


\newcommand{\jingxunworking}{\begin{center}\color{blue} ------------------------------------WORKING------------------------------------ \end{center}}
\newcommand{\jingxun}[1]{{\color{blue}[Jingxun: #1]}}
\newcommand{\josh}[1]{{\color{red}[Josh: #1]}}
\newcommand{\defn}[1]{\textbf{\emph{#1}}}

\newcommand{\nnz}{\textup{nnz}}
\newcommand{\rank}[1][\F_p]{\textup{rank}^{#1}}
\newcommand{\sparsity}{\textup{sparsity}}
\renewcommand{\R}{\mathcal{R}}
\newcommand{\signR}{\text{sign-}\mathcal{R}}
\newcommand{\ip}{\text{IP}}
\newcommand{\Pcc}{\text{P}^{\text{cc}}}
\newcommand{\BPP}{\text{BPP}^{\text{cc}}}
\newcommand{\NP}{\text{NP}^{\text{cc}}}
\newcommand{\PH}{\text{PH}^{\text{cc}}}
\newcommand{\AM}{\text{AM}^{\text{cc}}}
\newcommand{\LTFoLTF}{\text{LTF} \circ \text{LTF}}
\newcommand{\bool}{\textup{bool}}
\newcommand{\sign}{\textup{sign}}

\renewcommand{\vec}[1]{\bm{\mathrm{#1}}}

\newcommand{\circledM}{\tikz[baseline=(char.base)]{\node[shape=circle,draw,inner sep=0.4pt,minimum size=0.4em] (char) {\fontsize{3}{4}\selectfont M};}}
\newcommand{\maj}[2][n]{#2^{\circledM #1}}
\newcommand{\kro}[2][n]{#2^{\otimes #1}}
\newcommand{\Maj}{\textup{Maj}}

\newcommand{\indicator}[1]{\mathbbm{1}{\Bk*{#1}}}
\newcommand{\boolRigidity}[2]{\R_{#1}^{\F_p^{\bool}}\bk*{#2}}

\newcommand{\pre}{\text{pre}}

\newcommand{\C}{\mathbb{C}}
