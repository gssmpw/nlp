\section{Experiments}
\begin{table}[!ht]
\centering

\caption{\textbf{Performance on on DrivAerNet}: we evaluate drag coefficient $c_d$ Mean Squared Error (MSE), Mean Absolute Error (MAE), Max Absolute Error (Max AE), coefficient of determination ($R^2$) of drag coefficient ($c_d$) prediction and inference time on the official test set. We evaluated the inference time on A100 single GPU. $^{\dagger}$ numbers from the authors.}
\label{table:drivaernet_main}
\resizebox{1.03\linewidth}{!}{

\begin{tabular}{lccccc}
\toprule
\textbf{Model} & $c_d$ \textbf{Mean SE} ($\downarrow$) & $c_d$ \textbf{Mean AE} ($\downarrow$)& $c_d$ \textbf{Max AE} ($\downarrow$)& $c_d$ \boldmath$R^2$ ($\uparrow$) & \textbf{Time (sec)} ($\downarrow$)\\
\midrule
PointNet++~\citep{qi2017pointnet++} &  7.813E-5 & 6.755E-3 & 3.463E-2 & 0.896 & 0.200 \\ %
DeepGCN~\citep{li2019deepgcns} & 6.297E-5 & 6.091E-3 & 3.070E-2 & 0.916 & 0.151 \\
MeshGraphNet~\citep{pfaff2020meshgraphnet} & 6.0E-5 & 6.08E-3 & 2.965E-2 & 0.917 & 0.25 \\
AssaNet~\citep{qian2021assanet} & 5.433E-5  & 5.81E-3 & 2.39E-2 & 0.927 & 0.11 \\
PointNeXt~\citep{qian2022pointnext} & 4.577E-5 & 5.2E-3 & 2.41E-2 & 0.939 & 0.239 \\ %
PointBERT~\citep{yu2022point} & 6.334E-5 & 6.204E-3 & 2.767E-2 & 0.915 & 0.163 \\
DrivAerNet DGCNN~\citep{elrefaie2024drivaernet}$^{\,\dagger}$ & 8.0E-5 & 6.91E-3 & \textbf{8.80E-3} & 0.901 & 0.52 \\
\midrule
FIGConvNet (Ours) & \textbf{3.225E-5} & \textbf{4.423E-3} & 2.134E-2 & \textbf{0.957} & \textbf{0.051} \\ \bottomrule
\end{tabular}

} %

\end{table}

\begin{table}[!ht]
\centering

\caption{\textbf{Comparing Convolution Kernel Size (local and global) on DrivAerNet Normalized Pressure ($\bar{P}$) Prediction}: we evaluate Mean Squared Error (MSE), Mean Absolute Error (MAE), Max Absolute Error (Max AE), of normalized pressure and the coefficient of determination ($R^2$) of drag coefficient and inference time on the official test set. The local convolution suffers from long inference time. $(r_x, r_y, r_z) = (4, 4, 4)$ and kernel size $K \ge 2r - $ is global. (Sec.~\ref{sec:reparameterization})}
\label{tab:drivaernet_local_vs_global}
\resizebox{1.03\linewidth}{!}{
\begin{tabular}{lccccc}
\toprule
\textbf{Kernel size} & $\bar{P}$ \textbf{Mean SE} ($\downarrow$) & $\bar{P}$ \textbf{Mean AE} ($\downarrow$)& $\bar{P}$ \textbf{Max AE} ($\downarrow$)& $c_d$ \boldmath$R^2$ ($\uparrow$) & \textbf{Time (sec)} ($\downarrow$) \\
\midrule
 $3 \times 3 \times 3$  & 0.046845  & 0.11895 & 5.95431 & 0.93 & 0.054 \\
  $5 \times 5 \times 5$ & 0.046364 & 0.11489 & 5.7173 & 0.943 & 0.061 \\
\midrule
$7 \times 7 \times 7$ (Global) & 0.044959 & \textbf{0.1124} & \textbf{5.6795} & 0.955 & 0.079 \\
$7 \times 7 \times 7$ (2D Reparameterization) & \textbf{0.043818} & 0.11285 & 5.73351 & \textbf{0.957} & \textbf{0.051} \\
\bottomrule
\end{tabular}
} %

\end{table}



We evaluated our approach using two automotive computational fluid dynamics datasets, comparing it with strong baselines and state-of-the-art methods:
\textbf{DrivAerNet}~\citep{elrefaie2024drivaernet}: Contains $4k$ meshes with CFD simulation results, including drag coefficients and mesh surface pressures. We adhere to the official evaluation metrics and the data split.
\textbf{Ahmed body}: Comprise surface meshes with approximately $100k$ vertices, parameterized by height, width, length, ground clearance, slant angle, and fillet radius. Following~\citep{li2023geometry}, we use about $10\%$ of the data points for testing. The wind tunnel inlet velocity ranges from $10m/s$ to $70m/s$, which we include as an additional input to the network.


\subsection{Experiment Setting}

The car models in both data sets consist of triangular or quadrilateral meshes with faces and pressure values defined on vertices for the DrivAerNet and faces on the Ahmed body data set. As the network cannot directly process a triangular or quadrilateral face, we convert a face to a centroid point and predict the pressure on these centroid vertices for the Ahmed body dataset.

To gauge the performance of our proposed network, we considered a large number of state-of-the-art dense prediction network architectures (e.g., semantic segmentation) for comparison including Dynamic Graph CNN~(DGCNN)~\citep{wang2019dynamic}, PointTransformers~\citep{zhao2021point}, PointCNN~\citep{qi2017pointnet,li2018pointcnn}, and geometry-informed neural operator(GINO)\citep{li2023geometry}.
For the DrivAerNet dataset, we follow the DrivAerNet~\citep{elrefaie2024drivaernet} and sample $N$ number of points from the point cloud and evaluate the MSE, MAE, Max Error, and the coefficient of determination $R^2$ of drag prediction. For the Ahmed body dataset, we follow the same setting as~\citep{li2023geometry} and evaluate the pressure prediction.



\subsection{Results on DrivAerNet}

Table~\ref{table:drivaernet_main} presents the performance comparison of various methods in the DrivAerNet dataset. Our FIGConvNet outperforms all state-of-the-art methods in drag coefficient prediction while maintaining fast inference times. PointNet variants (e.g., PointNet++, PointNeXt) perform well compared to transformer-based networks like PointBERT, likely due to the dataset's small size. For all baselines except DrivAerNet DGCNN, we incorporate both pressure prediction and drag coefficient prediction losses.

We analyze the impact of the size of the convolution kernel on the prediction of pressure (Table~\ref{tab:drivaernet_local_vs_global}). Larger kernels approach global convolution, but lead to performance saturation and slower inference. Our reparameterized 3D convolution achieves comparable performance with improved speed.

Figure~\ref{fig:cd_vs_gt} visualizes the ground truth versus predicted drag coefficients, demonstrating the network's ability to capture the distribution accurately. Figure~\ref{fig:drivaer_num_points} shows the effect of the sample point count on the precision of the prediction, revealing robustness over a wide range but potential overfitting with very high point counts. Qualitative pressure predictions are shown in Figure~\ref{fig:drivaernet_main}.

To assess the impact of factorized grid dimensions, we varied grid sizes (Table~\ref{table:drivaernet_hyperparam}). Larger grids improved the accuracy of pressure prediction, but degraded the coefficient of determination ($\mathbf{R}^2$) of the drag coefficient and increased the inference time.

Lastly, we remove the feature fusion between factorized grids proposed in Sect.~\ref{sec:fusion}. We observe that having no fusion in FIG convolution degrades performance but the gap is smaller when the grid $(r_x, r_y, r_z)$ is larger. This suggests that while fusion remains important, its significance decreases with increasing grid size.


\begin{table}[!ht]
\centering

\caption{\textbf{Choosing the rank: impact of the choice of on DrivAerNet on performance}: we evaluate normalized pressure $\bar{P}$ Mean Squared Error (MSE), Mean Absolute Error (MAE), Max Absolute Error (Max AE), coefficient of determination ($R^2$) of drag coefficient $c_d$ and inference time on the official test set. We trained for only 50 epochs for this experiment. Note that the car is facing +x axis and is the longest while -z is the gravity axis and is the shortest. See Sec.~\ref{sec:factorized_grids} for $(r_x, r_y, r_z)$ definition.
}
\label{table:drivaernet_hyperparam}
\resizebox{0.90\linewidth}{!}{

\begin{tabular}{lccccc}
\toprule
\textbf{$(r_x,r_y,r_z)$} & $\bar{P}$ \textbf{Mean SE} ($\downarrow$) & $\bar{P}$ \textbf{Mean AE} ($\downarrow$)& $\bar{P}$ \textbf{Max AE} ($\downarrow$)& $c_d$ \boldmath$R^2$ ($\uparrow$) & \textbf{Time (sec)} ($\downarrow$)\\
\midrule
(1, 1, 1) & 0.05278 & 0.1275 & 5.8266 & \textbf{0.9328 }& \textbf{0.0305} \\ 
(3, 2, 2) & 0.05199 & 0.1250 & 5.8284 & 0.9249 & 0.0396 \\
(5, 3, 2) & 0.05131 & 0.1223 & 6.2350 & 0.8735 & 0.0399 \\
(10, 6, 4) & 0.05079 & \textbf{0.1221}& 5.6506 & 0.9243 & 0.0493 \\ 
(10, 10, 10) & \textbf{0.04999} & 0.1254 & \textbf{5.5343} & 0.8926 & 0.0610 \\ 
\bottomrule
\end{tabular}

} %

\end{table}

\begin{table}[!ht]
\centering

\caption{Impact of the factorized grid fusion (Sec.~\ref{sec:fusion}) on DrivAerNet: we evaluate normalized pressure $\bar{P}$ Mean Squared Error (MSE), Mean Absolute Error (MAE), Max Absolute Error (Max AE), coefficient of determination ($R^2$) of drag coefficient $c_d$, and inference time on the official test set. We trained for 50 epochs for this experiment. For no communication rows, we set the fusion layer in Sec.~\ref{sec:fusion} to be identity and kept all the rest of the network the same.
}
\label{table:drivaernet_communication}
\resizebox{0.95\linewidth}{!}{

\begin{tabular}{lccccc}
\toprule
\textbf{$(r_x,r_y,r_z)$} & $\bar{P}$ \textbf{Mean SE} ($\downarrow$) & $\bar{P}$ \textbf{Mean AE} ($\downarrow$)& $\bar{P}$ \textbf{Max AE} ($\downarrow$)& $c_d$ \boldmath$R^2$ ($\uparrow$) & \textbf{Time (sec)} ($\downarrow$) \\
\midrule
(3, 2, 2) & \textbf{0.05199} & \textbf{0.1250} & \textbf{5.8284} & \textbf{0.9249} & 0.0396 \\
(3, 2, 2) No Fusion & 0.053455 & 0.12683 & 6.28512 & 0.90413 & \textbf{0.0361} \\
\midrule
(5, 3, 2) & \textbf{0.05131} & \textbf{0.1223} & 6.2350 & 0.8735 & \textbf{0.0399} \\
(5, 3, 2) No Fusion & 0.052921 & 0.12287 & \textbf{6.02101} & \textbf{0.88638} & 0.0451 \\
\bottomrule
\end{tabular}

} %

\end{table}




\begin{figure}[htbp]
    \centering
    
    \begin{subfigure}[t]{0.45\textwidth}
        \centering
        \includegraphics[width=0.99\linewidth]{Images/drag_num_points.pdf}
        \caption{Number of Sample Points on Drag Prediction: The networks are robust to the number of sample points used for drag prediction.} 
        \label{fig:drivaer_num_points}
    \end{subfigure}
    \hfill
    \begin{subfigure}[t]{0.45\textwidth}
        \centering
        \includegraphics[width=0.99\linewidth]{Images/C_d_GT_vs_Predicted.png}
        \caption{Drag prediction vs. Ground truth drag on DrivAerNet. The drag prediction closely matches the drag ground truth with $R^2$ of 0.95.}
        \label{fig:cd_vs_gt}
    \end{subfigure}
\end{figure}


\begin{figure*}[htbp]
    \centering
    \vspace{1em}
    \resizebox{0.99\linewidth}{!}{
    \begin{tabular}{cccc}

    \includegraphics[width=0.25\linewidth]{Images/DrivAerNet/0001_input_mesh.pdf} &
    \includegraphics[width=0.25\linewidth]{Images/DrivAerNet/0001_gt_p_mesh.pdf} &
    \includegraphics[width=0.25\linewidth]{Images/DrivAerNet/0001_pred_p_pc.pdf} &
    \includegraphics[width=0.25\linewidth]{Images/DrivAerNet/0001_p_ae_pc.pdf} \\


    \includegraphics[width=0.25\linewidth]{Images/DrivAerNet/3296_input_mesh.pdf} &
    \includegraphics[width=0.25\linewidth]{Images/DrivAerNet/3296_gt_p_mesh.pdf} &
    \includegraphics[width=0.25\linewidth]{Images/DrivAerNet/3296_pred_p_pc.pdf} &
    \includegraphics[width=0.25\linewidth]{Images/DrivAerNet/3296_p_ae_pc.pdf} \\


    Input Mesh & Ground Truth Pressure & Pressure Prediction & Pressure Absolute Error
    \end{tabular}
    }    
    \caption{\textbf{Normalized Pressure Prediction and Error Visualization on DrivAerNet.} Our network predicts both drag coefficients and per vertex pressure. We visualize the ground truth pressure and prediction along with the absolute error of the pressure. Note that the pressures are normalized to highlight the errors clearly.}
    \label{fig:drivaernet_main}
\end{figure*}

\subsection{Results on Ahmed body}
\begin{table*}[ht]
\centering
\caption{\textbf{Ahmed Body Per Vertex Pressure Prediction Error} measured the normalized L2 pressure error per vertex on the test set. The top three rows are from~\citet{li2023geometry}. 
}
\label{tab:main_ahmed}
\resizebox{.60\linewidth}{!}{
\begin{tabular}{lccc}
\toprule
\textbf{Model} & \textbf{Pressure Error} & \textbf{Model Size (MB)} \\
\midrule %
UNet (interp) &11.16\% & 0.13 \\ %
FNO (interp)~\cite{li2020fourier} & 12.59\% & 924.34 \\ %
GINO~\cite{li2023geometry} & 9.01\% & 923.63 \\
\midrule
FIGConvNet (Ours) & 0.89\% & 68.29 \\ %
\bottomrule
\end{tabular}

}
\end{table*}


Table~\ref{tab:main_ahmed} compares the performance of our method in the Ahmed body data set with state-of-the-art approaches~\cite{li2023geometry}, reporting normalized pressure MSE and model size. Although GINO outperforms UNet and FNO, it achieves only $9\%$ pressure error. In contrast, our method attains a significantly lower normalized pressure error of $0.89\%$ with a smaller model footprint.

We further analyze the impact of grid resolution on network performance (Table~\ref{tab:ahmed_resolution}). Our approach demonstrates robust pressure prediction across a wide range of grid resolutions, even with small grids. However, we observe that very high grid resolutions lead to overfitting on training data, resulting in decreased test performance.

