\section{Introduction}

The automotive industry stands at the forefront of technological advancement and is heavily relying on computational fluid dynamics (CFD) to optimize vehicle designs for enhanced aerodynamics and fuel efficiency. Accurate simulation of complex fluid dynamics around automotive geometries is crucial to achieving optimal performance. However, traditional numerical solvers, including finite difference and finite element methods, often prove computationally intensive and time-consuming, particularly when dealing with large-scale simulations, as encountered in CFD applications. 
The demand for efficient solutions in the automotive sector requires the exploration of innovative approaches to accelerate fluid dynamics simulations and overcome the limitations of current solvers.


In recent years, deep learning methodologies have emerged as promising tools in scientific computing, advancing traditional simulation techniques, in biochemistry~\citep{jumper2021highly}, seismology~\citep{yang2021seismic}, climate change mitigation~\citep{wen2023real}, and weather~\citep{pathak2022fourcastnet,lam2022graphcast} to name a few. In fluid dynamics, recent attempts have been made to develop domain-specific deep learning methods to emulate fluid flow evolution in 2D and 3D proof-of-concept settings~\citep{jacob2021deep,li2020neural,pfaff2020learning,kossaifi2023multigrid}.
Although most of these works focused on solving problems using relatively low-resolution grids, industrial automotive CFD requires working with detailed meshes that contain millions of points.

To address the time-consuming and computationally intensive nature of conventional CFD solvers on detailed meshes, recent studies~\citep{jacob2021deep,li2023geometry} have explored replacing CFD simulations with deep learning-based models to accelerate the process.
In particular,~\citet{jacob2021deep} studies the DrivAer dataset~\citep{heft2012introduction}, utilize Unet~\citep{ronneberger2015u} architectures, and aim to predict the car surface drag coefficient -- the integration of surface pressure and friction -- directly by bypassing integration. Furthermore, the architecture is applied on 3D voxel grids that requires $O(N^3)$ complexity, forcing the method to scale only to low-resolution 3D grids.
\citet{li2023geometry} propose a neural operator method for Ahmed body~\citep{ahmed1984some} car dataset and aims to predict the pressure function on the car surface. This approach utilizes graph embedding to a uniform grid and performs 3D global convolution through fast Fourier transform (FFT). Although, in principle, this method handles different girding, the FFT in the operator imposes a complexity of $O(N^3 \log N^3)$, which becomes computationally prohibitive as the size of the grid increases.
Both methods face scalability challenges due to their cubic complexity, which severely limits their representational power for high-resolution simulations.
Consequently, there is a pressing need for a specialized domain-inspired method capable of handling 3D fine-grained car geometries with meshes comprising tens of millions of vertices~\citep{jacob2021deep}. Such massive datasets require a novel approach in both design and implementation.

\textbf{In this work}, we propose a new quadratic complexity neural CFD approach $O(N^2)$, significantly improving scalability over existing 3D neural CFD models that require cubic complexity $O(N^3)$. Our method outperforms the state-of-the-art by reducing the absolute mean squared error by 70\%.

The key innovations of our approach include Factorized Implicit Grids and Factorized Implicit Convolution. With Factorized Implicit Grids, we approximate high-resolution domains using a set of implicit grids, each with one lower-resolution axis. For example, a domain $1k \times 1k \times 1k$ containing $10^9$ elements can be represented by three implicit grids with dimensions $5 \times 1k \times 1k$, $1k \times 4 \times 1k$, and $1k \times 1k \times 3$. This reduces the total number of elements to only $5M + 4M + 3M = 12M$, a significant reduction from the original $10^9$. Our Factorized Implicit Convolution method approximates 3D convolutions using these implicit grids, employing reparameterization techniques to accelerate computations.

We validate our approach on two large-scale CFD datasets. DrivAerNet \citep{heft2012introduction,elrefaie2024drivaernet} and Ahmed body dataset \citep{ahmed1984some}. Our experiments focus on the prediction of surface pressure and drag coefficients. The results demonstrate that our network is an order of magnitude faster than existing methods while achieving state-of-the-art performance in both drag coefficient prediction and per-face pressure prediction.
