


\section{Conclusion and Limitations}
In this work, we proposed a deep learning method for automotive drag coefficient prediction using a network with factorized implicit global convolutions. This approach efficiently captures the global context of the geometry, outperforming state-of-the-art methods on two automotive CFD datasets. On the DrivAerNet dataset, our method achieved an $R^2$ value of 0.95 for drag coefficient prediction, while on the Ahmed body dataset, it attained a normalized pressure error of $0.89\%$.

However, our approach has some limitations. The FIG ConvNet directly regresses the drag coefficient without incorporating physics-based constraints, which could lead to overfitting and poor generalization to unseen data. Additionally, our method is currently limited to the automotive domain with a restricted model design, potentially limiting its applicability to other fields.
Looking ahead, we plan to address these limitations and further improve our model. Future work will focus on incorporating physics-based constraints such as Reynolds number and wall shear stress to enhance generalization.
