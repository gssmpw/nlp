\section{Appendix}


\section{Datasets}

The foundation of CFD in the automotive industry provides insight into design and engineering. The comprehensive previous texts provide a solid overview of computational methods in fluid dynamics and dedicate a comprehensive overview of traditional CFD techniques~\citep{ferziger2019computational} along with specification in automotive aerodynamics, also instrumental in understanding the principles~\citep{katz2016automotive}. Solvers, such as OpenFOAM, a GPU-accelerated open-source solver, along with commercialized licensed solvers are widely used for solving CFD equations in automotive simulations~\citep{jasak2007openfoam}. 

Such simulations consist of two main components, i) car designs, complex geometry often developed special software, and ii) running large scale computation to solve multivariate coupled equations. 
Significant advancements have been achieved by the Ahmed body shape~\citep{ahmed1984some}, a generic car model simple enough to enable high-fidelity industry standard simulations while retaining the main features characterizing the flow of modern cars.
Since then, attempts have been made to improve the realism of the shapes. Shape-net~\citep{chang2015shapenet} in particular has provided a valuable resource for simple CFD simulations of cars~\citep{umetani2018learning}. 
Extending Ahmed's body setting, the DrivAer data set introduces more complex and realistic car geometries~\citep{heft2012introduction}, with subsequent efforts, producing large-scale aerodynamics simulations on such geometries\citep{varney2020experimental}. 
On such dataset, prior work attempts to predict car surface drag coefficients directly by bypassing the surface pressure prediction, pioneered by ~\cite{jacob2021deep}. However, this approach deploys an architecture applied to 3D voxel grids, forcing the method to scale only to low-resolution 3D grid version of the data. The lack of resolution obscures the fine details of geometry, causing the network to predict the same results for cars with different information. This is in contrast to our work that predicts pressure fields on large scales and detailed meshes.



\textbf{Ahmed body} consists of generic automotive geometries~\citep{ahmed1984some}, simple enough to enable high-fidelity industry standard simulations but retaining the main features characterizing the flow of modern cars. It was generated and used in previous studies~\citep{li2023geometry} and contains simulations with various wind-tunnel inlet velocities.

The Ahmed body data set, generated using vehicle aerodynamic simulation in Ahmed body shapes~\citep{ahmed1984some}, consists of steady-state simulation of OpenFOAM solver in 3D meshes each with $10M$ vertices parameterized by height, width, length, ground clearance, slant angle, and fillet radius. The data set is generated and used in previous studies~\citep{li2023geometry} and contains GPU-accelerated simulations with surface mesh sizes of $100k$ on more than $500$ car geometries, each taking 7-19 hours. We follow the same setting as this study using the $10\%$ shape testing. The dataset is proprietary from NVIDIA Corp. Following this work, both of the deployed datasets are in the process of being made publicly available for further research. 


\begin{table}[ht]
\centering

\caption{\textbf{Ahmed body Controlled Experiment} We vary the grid resolution and kernel size for analysis. $(r_x, r_y, r_z)$ is $(6, 2, 2)$ (Sec.~\ref{sec:factorized_grids}) e.g. The three grid resolutions we used for the first three rows are $6\PLH 280\PLH 180,560\PLH 2\PLH 180,560\PLH 208\PLH 2$.}
\label{tab:ahmed_resolution}
\vspace{2mm}

\resizebox{0.6\linewidth}{!}{

\begin{tabular}{lcccc}
\toprule
\textbf{Max Resolution} & \textbf{Kernel Size} & \textbf{Pressure Error} & \textbf{Model Size (MB)}\\
\midrule
\multirow{3}{*}{$560\PLH 208\PLH 180$} & 3 & 3.40\% & 105.0 \\ %
& 7 & 3.31\% & 417.1 \\ %
& 11 & 2.56\% & 979.7 \\ %

\midrule
\multirow{3}{*}{$280\PLH 104\PLH 90$} & 3 & 2.89\% & 105.0 \\ %
& 7 & 3.05\% & 417.1 \\ %
& 11 & 2.93\% & 979.7 \\ %

\midrule
\multirow{2}{*}{$140\PLH 42\PLH 45$} & 9 & 1.65\% & 667.29 \\ %
& 11 & 2.59\% & 979.7 \\ %

\bottomrule
\end{tabular}
}
\end{table}




\textbf{DrivAerNet} datasets is the parametric extension of DrivAer datasets. DriveAer car geometries are more complex real-world automotive designs used by the automotive industry and solver development~\citep{heft2012introduction}. Solving the aerodynamic equation for such geometries is a challenging task, and GPU-accelerated solvers are used to provide fast and accurate solvers, generating training data for deep learning purposes~\citep{varney2020experimental,jacob2021deep}. To train our model on the DrivAer dataset, and to demonstrate the applicability of our approach to real-world applications, we use industry simulations from \cite{jacob2021deep}.
DrivAerNet with $50$ parameters in the design space. The dataset consists of $4000$ data points generated using
Reynolds-average Navier-Stokes (RANS) formulation on OpenFoam solver on $0.5$M mesh faces. 










\subsection{Baseline Network Configurations}

We list the network configurations used in the experiment in the appendix. We use OpenPoint~\cite{qian2022pointnext} for the baseline implementation and, with configuration, you can specify the network architecture.

\subsection{Baseline Implementations}

We use the OpenPoint, an open-soruce 3D point cloud library~\citep{qian2022pointnext} to implement PointNet++~\citep{qi2017pointnet++}, DeepGCN~\citep{li2019deepgcns}, AssaNet~\citep{qian2021assanet}, PointNeXt~\citep{qian2022pointnext}, and PointBERT~\citep{yu2022point}. In this section, we share the network architecture configuration used in the experiment.

\begin{minipage}{\linewidth}
\begin{lstlisting}[language=yaml, caption=\textbf{PointNet++ Configuration}, label=lst:pointnetpp_config]
  NAME: BaseSeg
  encoder_args:
    NAME: PointNet2Encoder
    in_channels: 3
    width: null
    strides: [2, 4, 1]
    mlps: [[[64, 64, 128]],
          [[128, 128, 256]],
          [[256, 512, 512]]]
    layers: 3
    use_res: False
    radius: 0.05
    num_samples: 32
    sampler: fps
    aggr_args:
      NAME: 'convpool'
      feature_type: 'dp_fj'
      anisotropic: False
      reduction: 'max'
    group_args:
      NAME: 'ballquery'
    conv_args: 
      order: conv-norm-act
    act_args:
      act: 'relu'
    norm_args:
      norm: 'bn'
  decoder_args:
    NAME: PointNet2Decoder
    fp_mlps: [[128, 128], [256, 128], [512, 128]]

\end{lstlisting}
\end{minipage}



\begin{minipage}{\linewidth}
\begin{lstlisting}[language=yaml, caption=\textbf{DeepGCN Configuration}, label=lst:deepgcn_config]
  NAME: BaseSeg
  encoder_args:
    NAME: DeepGCN
    in_channels: 3
    channels: 64
    n_classes: 256 
    emb_dims: 256
    n_blocks: 14
    conv: 'edge'
    block: 'res'
    k: 9
    epsilon: 0.0
    use_stochastic: False
    use_dilation: True 
    dropout: 0
    norm_args: {'norm': 'in'}
    act_args: {'act': 'relu'}
\end{lstlisting}
\end{minipage}



\begin{minipage}{\linewidth}
\begin{lstlisting}[language=yaml, caption=\textbf{AssaNet Configuration}, label=lst:assanet_config]
  NAME: BaseSeg
  encoder_args:
    NAME: PointNet2Encoder
    in_channels: 3
    strides: [4, 4, 4, 4]
    blocks: [3, 3, 3, 3]
    width: 128
    width_scaling: 3
    double_last_channel: False
    layers: 3
    use_res: True 
    query_as_support: True
    mlps: null 
    stem_conv: True
    stem_aggr: True
    radius: [[0.1, 0.2], [0.2, 0.4], [0.4, 0.8], [0.8, 1.6]]
    num_samples: [[16, 32], [16, 32], [16, 32], [16, 32]]
    sampler: fps
    aggr_args:
      NAME: 'ASSA'
      feature_type: 'assa'
      anisotropic: True 
      reduction: 'mean'
    group_args:
      NAME: 'ballquery'
      use_xyz: True
      normalize_dp: True
    conv_args:
      order: conv-norm-act
    act_args:
      act: 'relu'
    norm_args:
      norm: 'bn'
  decoder_args:
    NAME: PointNet2Decoder
    fp_mlps: [[64, 64], [128, 128], [256, 256], [512, 512]]
\end{lstlisting}
\end{minipage}




\begin{minipage}{\linewidth}
\begin{lstlisting}[language=yaml, caption=\textbf{PointNeXt Configuration}, label=lst:pointnext_config]

  NAME: BaseSeg
  encoder_args:
    NAME: PointNextEncoder
    blocks: [1, 2, 3, 2, 2]
    strides: [1, 4, 4, 4, 4]
    width: 64
    in_channels: 3
    sa_layers: 1
    sa_use_res: True
    radius: 0.1
    radius_scaling: 2.5
    nsample: 32
    expansion: 4
    aggr_args:
      feature_type: 'dp_fj'
    reduction: 'max'
    group_args:
      NAME: 'ballquery'
      normalize_dp: True
    conv_args:
      order: conv-norm-act
    act_args:
      act: 'relu' # leakrelu makes training unstable.
    norm_args:
      norm: 'bn'  # ln makes training unstable
  decoder_args:
    NAME: PointNextDecoder
\end{lstlisting}
\end{minipage}



\begin{minipage}{\linewidth}
\begin{lstlisting}[language=yaml, caption=\textbf{PointBERT Configuration}, label=lst:pointbert_config]
  NAME: BaseSeg
  encoder_args:
    NAME: PointViT
    in_channels: 3
    embed_dim: 512 
    depth: 8
    num_heads: 8
    mlp_ratio: 4.
    drop_rate: 0.
    attn_drop_rate: 0.0
    drop_path_rate: 0.1
    add_pos_each_block: True
    qkv_bias: True
    act_args:
      act: 'gelu'
    norm_args:
      norm: 'ln'
      eps: 1.0e-6
    embed_args:
      NAME: P3Embed
      feature_type: 'dp_df'
      reduction: 'max'
      sample_ratio: 0.0625
      normalize_dp: False 
      group_size: 32
      subsample: 'fps' # random, FPS
      group: 'knn'
      conv_args:
        order: conv-norm-act
      layers: 4
      norm_args: 
        norm: 'ln2d'
  decoder_args:
    NAME: PointViTDecoder
    channel_scaling: 1
    global_feat: cls,max
    progressive_input: True
\end{lstlisting}
\end{minipage}



\subsection{FIGConvNet Configuration}

We share the network configuration used in FIGConvNet experiments in the appendix. The code will be released upon acceptance, and the network configuration below uniquely defines the architecture.

\subsection{FIG ConvNet Architecture Details}

In this section, we provide architecture details used in our network using the configuration files used in our experiments.


\begin{minipage}{\linewidth}
\begin{lstlisting}[language=yaml, caption=\textbf{FIGConvNet Configuration}, label=lst:figconvnet_config]
num_levels: 2
kernel_size: 5
hidden_channels:
  - 16
  - 32
  - 48
num_down_blocks: [1, 1]  # defines the number of FIGConv blocks per hierarchy in encoder
num_up_blocks: [1, 1]  # defines the number of FIGConv blocks per hierarchy in decoder
resolution_memory_format_pairs:  # defines the grid resolutions
  - [  5, 150, 100]
  - [250,   3, 100]
  - [250, 150,   2]
\end{lstlisting}
\end{minipage}




\subsection{Warp-based Radius Search}

\floatname{algorithm}{Procedure}
\renewcommand{\algorithmicrequire}{\textbf{Input:}}
\renewcommand{\algorithmicensure}{\textbf{Output:}}

The algorithm~\ref{alg:warp-radius} describes how we efficiently find the input points within the radius of a query point in parallel. It follows the common three-step computational pattern in GPU computing when encountering dynamic number of results: Count, Exclusive Sum, Allocate, and Fill. We achieve excellent performance by leveraging NVIDIA's Warp Python framework, which compiles to native CUDA and provides spatially efficient point queries with its hash-grid primitive.

\begin{algorithm}
    \caption{GPU-accelerated points in a radius search}
    \begin{algorithmic}
        \Require{input points $p$, query points $q$, radius $r$}
        \Ensure{Results Array, Result Offset}
    \Procedure{CountRadiusResults}{query points, input points, radius $r$}
    \State Step 1: Count number of results

        \ForAll{query points $q$}
        \While{candidate $p\gets$ hash-grid query($q$,$r$)}
            \If{$\|q-p\|<$ radius}
                count[q]++
            \EndIf
        \EndWhile
        \EndFor
        \EndProcedure
        \Procedure{ComputeOffset}{count}
        \State offset $\gets$ exclusive-sum(count)
        \State total number results $\gets$ offset[last]
        \State results-array $\gets$ alloc(total number results)
        \EndProcedure
        \Procedure{FillRadiusResults}{query points, input points, radius $r$, offset}
        \ForAll{query points $q$}
        \State q-count $\gets 0$
        \While{candidate $p\gets$ hash-grid query($q$,$r$)}
        \If{$\|q-p\|<$ radius}
            \State results-array[offset[q-count]] $\gets p$
            \State q-count++
        \EndIf
        \EndWhile
        \EndFor
        \EndProcedure

        \Procedure{PointsInRadius}{input points, query points, radius}
        \State count $\gets$ \Call{CountRadiusResults}{query points, input points, radius}
        \State offset, allocated results array $\gets$ \Call{ComputeOffset}{count}
        \State results array $\gets$ \Call{FillRadiusResults}{query points, input points, radius, offset, results array}
        \EndProcedure
    \end{algorithmic}
    \label{alg:warp-radius}
\end{algorithm}
