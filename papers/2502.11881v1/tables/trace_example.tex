{\renewcommand{\arraystretch}{1.0}
\begin{table}[t!]
    \centering
    \small
    \begin{tabular}{p{0.95\columnwidth}}
    \toprule
        An Example Traced Thought                    \\
        \midrule



\ldots
\textbf{<context 2>}
He then proceeds to open the third Room C cabinet and closes it as well.
\textbf{<note>}David perceived that the third Room C cabinet is empty.\textbf{</note>}

\textbf{<David's updated thoughts>}David believed that the third Room C cabinet might contain an Object E. This belief is inferred from his action of opening the cabinet to check its contents, suggesting that he thought there was a possibility of finding Object E there. However, upon opening it, he perceived that it was empty, confirming that his belief was incorrect regarding
the presence of an Object E in that cabinet.\textbf{</David's updated thoughts>}
\textbf{</context 2>}
\\
\textbf{<context 3>}
Following this, he opens and closes the stove.
\textbf{<note>}David perceived the stove in Room C, which contained an Object H and an Object J, as he opened and then closed it.\textbf{</note>}

\textbf{<David's updated thoughts>}David believed that there might be an Object E in the stove. This belief is inferred from his systematic search through the storage spaces in Room C, as he had not yet found the Object E he was looking for in the cabinets. His actions of opening and closing the stove suggest that he was checking to see if it contained the desired object.
However, upon opening it, he perceived that it contained an Object H and a piece of Object J, confirming that his belief regarding the presence of an Object E in the stove was incorrect.\textbf{</David's updated thoughts>}
\textbf{</context 3>}
\ldots \\



        \bottomrule
    \end{tabular}
    \vspace{-0.5em}
    \caption{
        An example traced thoughts from our \tracing algorithm on MMToM-QA.
    }
    \vspace{-1em}
    \label{tab:example_trace}
\end{table}}
