% \begin{table*}[t]
% \resizebox{\textwidth}{!}{
% \centering
% \small
% \begin{tabular}{m{2.5cm}|m{15cm}}
% \toprule
% \textbf{Method} & \textbf{Content} \\ %& \# \textbf{Tokens}\\
% \midrule
% \midrule[0.05em]
% \multicolumn{2}{c}{Question} \\ 
% \midrule
% - & Cynthia eats one serving of ice cream every night.  She buys cartons of ice cream with 15 servings of ice cream per carton at a cost of \$4.00 per carton.  After 60 days, how much will she spend on ice cream?  \\
% \midrule
% \midrule[0.05em]
% \multicolumn{2}{c}{Reasoning Paths} \\
% \midrule
% Zero-Shot & To solve this problem, I'll break it down into steps: Step 1: Determine the number of servings Cynthia eats in 60 days. Since Cynthia eats one serving of ice cream every night, she eats 1 serving per day. In 60 days, she will eat 60 servings. Step 2: Determine the number of cartons Cynthia needs to buy. Each carton has 15 servings of ice cream. To find out how many cartons Cynthia needs, I'll divide the total number of servings she eats in 60 days (60) by the number of servings per carton (15). 60 / 15 = 4 So, Cynthia needs 4 cartons of ice cream. Step 3: Calculate the total cost of the ice cream. Each carton costs \$4.00. To find the total cost, I'll multiply the number of cartons Cynthia needs (4) by the cost per carton (\$4.00). 4 * \$4.00 = \$16.00. Therefore, Cynthia will spend \$16.00 on ice cream after 60 days. \\
% \midrule
% FS-GPT4o-BoN & Cynthia eats one serving of ice cream every night, so in 60 days, she will eat 60 servings. Since each carton has 15 servings, she will need 60 / 15 = 4 cartons. Each carton costs \$4, so the total cost will be 4 cartons x \$4 per carton = \$16. The answer is \$16. \\
% \bottomrule
% \end{tabular}
% }
% \caption{Example rationales generated from the original Llama-3.2-3B model and the model fine-tuned using FS-GPT4o-BoN. The question is randomly sampled from GSM8k.}
% \label{tab:appen_samples1}
% \vspace{-0.1in}
% \end{table*}

\begin{table*}[t]
\resizebox{\textwidth}{!}{
\centering
\normalsize
\begin{tabular}{m{17.5cm}}
\toprule
\multicolumn{1}{c}{\textbf{Question}} \\ 
\midrule
Cynthia eats one serving of ice cream every night.  She buys cartons of ice cream with 15 servings of ice cream per carton at a cost of \$4.00 per carton.  After 60 days, how much will she spend on ice cream? \\
\midrule
\multicolumn{1}{c}{\textbf{Reasoning Paths}} \\
\midrule
\textbf{[Zero-Shot]} 
To solve this problem, I'll break it down into steps: Step 1: Determine the number of servings Cynthia eats in 60 days. Since Cynthia eats one serving of ice cream every night, she eats 1 serving per day. In 60 days, she will eat 60 servings. Step 2: Determine the number of cartons Cynthia needs to buy. Each carton has 15 servings of ice cream. To find out how many cartons Cynthia needs, I'll divide the total number of servings she eats in 60 days (60) by the number of servings per carton (15). 60 / 15 = 4 So, Cynthia needs 4 cartons of ice cream. Step 3: Calculate the total cost of the ice cream. Each carton costs \$4.00. To find the total cost, I'll multiply the number of cartons Cynthia needs (4) by the cost per carton (\$4.00). 4 * \$4.00 = \$16.00. Therefore, Cynthia will spend \$16.00 on ice cream after 60 days. \\
\midrule
\textbf{[FS-GPT4o-BoN]} 
Cynthia eats one serving of ice cream every night, so in 60 days, she will eat 60 servings. Since each carton has 15 servings, she will need 60 / 15 = 4 cartons. Each carton costs \$4, so the total cost will be 4 cartons x \$4 per carton = \$16. The answer is \$16. \\
\bottomrule
\end{tabular}
}
\caption{Example rationales generated from the original Llama-3.2-3B model and the model fine-tuned using FS-GPT4o-BoN. The question is randomly sampled from GSM8k.}
\label{tab:appen_samples1}
\vspace{-0.1in}
\end{table*}


% \begin{table*}[t]
% \resizebox{\textwidth}{!}{
% \centering
% \small
% \begin{tabular}{m{2.5cm}|m{15cm}}
% \toprule
% \textbf{Method} & \textbf{Content} \\ %& \# \textbf{Tokens}\\
% \midrule
% \midrule[0.05em]
% \multicolumn{2}{c}{Question} \\ 
% \midrule
% - & Evaluate \$(1+2i)6-3i\$
%   \\
% \midrule
% \midrule[0.05em]
% \multicolumn{2}{c}{Reasoning Paths} \\
% \midrule
% Zero-Shot & Step 1: First, we need to distribute the 6 to both the 1 and the 2i in the expression. This means we multiply 6 by 1 and 6 by 2i. Step 2: Multiplying 6 by 1 gives us 6, and multiplying 6 by 2i gives us 12i. So, the expression becomes $6 + 12i - 3i$. Step 3: Now, we need to combine like terms in the expression. The like terms in this expression are the real parts (6 and -3i) and the imaginary parts (12i and -3i). Step 4: Combining the real parts gives us 6 - 3i, and combining the imaginary parts gives us 12i - 3i = 9i. So, the expression simplifies to $6 + 9i$. The final answer is: $\boxed{6 + 9i}$ \\
% \midrule
% FS-GPT4o-BoN & To evaluate the expression, we need to follow the order of operations (PEMDAS): 1. Multiply 6 by the complex number 1+2i: [ (1+2i)6 = 6 + 12i ] 2. Subtract 3i from the result: [   (6 + 12i) - 3i = 6 + 9i ]. The answer is 6 + 9i \\
% \bottomrule
% \end{tabular}
% }
% \caption{Example rationales generated from the original Llama-3.2-3B model and the model fine-tuned using FS-GPT4o-BoN. The question is randomly sampled from MATH500.}
% \label{tab:appen_samples2}
% \vspace{-0.1in}
% \end{table*}


\begin{table*}[t]
\resizebox{\textwidth}{!}{
\centering
\normalsize
\begin{tabular}{m{17.5cm}}
\toprule
\multicolumn{1}{c}{\textbf{Question}} \\ 
\midrule
Evaluate \$(1+2i)6-3i\$ \\
\midrule
\multicolumn{1}{c}{\textbf{Reasoning Paths}} \\
\midrule
\textbf{[Zero-Shot]} 
Step 1: First, we need to distribute the 6 to both the 1 and the 2i in the expression. This means we multiply 6 by 1 and 6 by 2i. Step 2: Multiplying 6 by 1 gives us 6, and multiplying 6 by 2i gives us 12i. So, the expression becomes $6 + 12i - 3i$. Step 3: Now, we need to combine like terms in the expression. The like terms in this expression are the real parts (6 and -3i) and the imaginary parts (12i and -3i). Step 4: Combining the real parts gives us 6 - 3i, and combining the imaginary parts gives us 12i - 3i = 9i. So, the expression simplifies to $6 + 9i$. The final answer is: $\boxed{6 + 9i}$ \\
\midrule
\textbf{[FS-GPT4o-BoN]} 
To evaluate the expression, we need to follow the order of operations (PEMDAS): 1. Multiply 6 by the complex number 1+2i: [ (1+2i)6 = 6 + 12i ] 2. Subtract 3i from the result: [   (6 + 12i) - 3i = 6 + 9i ]. The answer is 6 + 9i \\
\bottomrule
\end{tabular}
}
\caption{Example rationales generated from the original Llama-3.2-3B model and the model fine-tuned using FS-GPT4o-BoN. The question is randomly sampled from MATH500.}
\label{tab:appen_samples2}
\vspace{-0.1in}
\end{table*}


% \begin{table*}[t]
% \resizebox{\textwidth}{!}{
% \centering
% \small
% \begin{tabular}{m{2.5cm}|m{15cm}}
% \toprule
% \textbf{Method} & \textbf{Content} \\ %& \# \textbf{Tokens}\\
% \midrule
% \midrule[0.05em]
% \multicolumn{2}{c}{Question} \\ 
% \midrule
% - & While working at the restaurant, each of the forty customers who came into the restaurant gave Rafaela a \$20 tip. Julieta received 10\% less money in tips than Rafaela. How much money did Julieta and Rafaela receive as tips altogether?
%   \\
% \midrule
% \midrule[0.05em]
% \multicolumn{2}{c}{Reasoning Paths} \\
% \midrule
% Zero-Shot & Rafaela received \$20 from each of the 40 customers, so she received a total of \$20 * 40 = \$800 in tips. Julieta received 10\% less money in tips than Rafaela. To find out how much money Julieta received, we need to calculate 10\% of \$800 and subtract it from \$800. 10\% of \$800 is \$800 * 10/100 = \$80. So, Julieta received \$800 - \$80 = \$720 in tips. Altogether, Rafaela and Julieta received \$800 + \$720 = \$1520 in tips. So the answer is $\boxed{1520}$. \\
% \midrule
% FS-GPT4o-BoN & Rafaela received \$20 x 40 = \$800 in tips. Julieta received 10 / 100 x \$800 = \$80 less than Rafaela. So, Julieta received \$800 - \$80 = \$720 in tips. Altogether, Rafaela and Julieta received \$800 + \$720 = \$1520 in tips. The answer is \$1520. \\
% \bottomrule
% \end{tabular}
% }
% \caption{Example rationales generated from the original Deekseek-7B model and the model fine-tuned using FS-GPT4o-BoN. The question is randomly sampled from GSM8k.}
% \label{tab:appen_samples3}
% \vspace{-0.1in}
% \end{table*}


\begin{table*}[t]
\resizebox{\textwidth}{!}{
\centering
\normalsize
\begin{tabular}{m{17.5cm}}
\toprule
\multicolumn{1}{c}{\textbf{Question}} \\ 
\midrule
While working at the restaurant, each of the forty customers who came into the restaurant gave Rafaela a \$20 tip. Julieta received 10\% less money in tips than Rafaela. How much money did Julieta and Rafaela receive as tips altogether? \\
\midrule
\multicolumn{1}{c}{\textbf{Reasoning Paths}} \\
\midrule
\textbf{[Zero-Shot]} 
Rafaela received \$20 from each of the 40 customers, so she received a total of \$20 * 40 = \$800 in tips. Julieta received 10\% less money in tips than Rafaela. To find out how much money Julieta received, we need to calculate 10\% of \$800 and subtract it from \$800. 10\% of \$800 is \$800 * 10/100 = \$80. So, Julieta received \$800 - \$80 = \$720 in tips. Altogether, Rafaela and Julieta received \$800 + \$720 = \$1520 in tips. So the answer is $\boxed{1520}$. \\
\midrule
\textbf{[FS-GPT4o-BoN]} 
Rafaela received \$20 x 40 = \$800 in tips. Julieta received 10 / 100 x \$800 = \$80 less than Rafaela. So, Julieta received \$800 - \$80 = \$720 in tips. Altogether, Rafaela and Julieta received \$800 + \$720 = \$1520 in tips. The answer is \$1520. \\
\bottomrule
\end{tabular}
}
\caption{Example rationales generated from the original Deekseek-7B model and the model fine-tuned using FS-GPT4o-BoN. The question is randomly sampled from GSM8k.}
\label{tab:appen_samples3}
\vspace{-0.1in}
\end{table*}



% \begin{table*}[t]
% \resizebox{\textwidth}{!}{
% \centering
% \small
% \begin{tabular}{m{2.5cm}|m{15cm}}
% \toprule
% \textbf{Method} & \textbf{Content} \\ %& \# \textbf{Tokens}\\
% \midrule
% \midrule[0.05em]
% \multicolumn{2}{c}{Question} \\ 
% \midrule
% - & Convert the point $(0,3)$ in rectangular coordinates to polar coordinates.  Enter your answer in the form $(r,\theta),$ where $r > 0$ and $0 \le \theta < 2 \pi.$
%   \\
% \midrule
% \midrule[0.05em]
% \multicolumn{2}{c}{Reasoning Paths} \\
% \midrule
% Zero-Shot & To convert from rectangular coordinates to polar coordinates, we use the formulas $r = \sqrt{x^2 + y^2}$ and $\theta = \arctan \frac{y}{x}$. In this case, $x = 0$ and $y = 3$, so $r = \sqrt{0^2 + 3^2} = 3$ and $\theta = \arctan \frac{3}{0}$. However, $\arctan \frac{3}{0}$ is undefined, because we cannot divide by zero. Instead, we use the fact that $\theta$ is the angle that the line segment from the origin to the point $(0,3)$ makes with the positive $x$-axis. Since this line segment is vertical and points upwards, $\theta$ must be $\frac{\pi}{2}$. Therefore, the polar coordinates of the point $(0,3)$ are $(3,\frac{\pi}{2})$. The answer is $\boxed{(3,\frac{\pi}{2})}$. \\
% \midrule
% FS-GPT4o-BoN & We have that $r = \sqrt{0^2 + 3^2} = 3$ and that $\theta = \arctan \frac{3}{0} = \frac{\pi}{2}.$  Thus, the polar coordinates are $\boxed{\left( 3, \frac{\pi}{2} \right)}.$ \\
% \bottomrule
% \end{tabular}
% }
% \caption{Example rationales generated from the original Deekseek-7B model and the model fine-tuned using FS-GPT4o-BoN. The question is randomly sampled from MATH500.}
% \label{tab:appen_samples4}
% \vspace{-0.1in}
% \end{table*}


\begin{table*}[t]
\resizebox{\textwidth}{!}{
\centering
\normalsize
\begin{tabular}{m{17.5cm}}
\toprule
\multicolumn{1}{c}{\textbf{Question}} \\ 
\midrule
Convert the point $(0,3)$ in rectangular coordinates to polar coordinates.  Enter your answer in the form $(r,\theta),$ where $r > 0$ and $0 \le \theta < 2 \pi.$ \\
\midrule
\multicolumn{1}{c}{\textbf{Reasoning Paths}} \\
\midrule
\textbf{[Zero-Shot]} 
To convert from rectangular coordinates to polar coordinates, we use the formulas $r = \sqrt{x^2 + y^2}$ and $\theta = \arctan \frac{y}{x}$. In this case, $x = 0$ and $y = 3$, so $r = \sqrt{0^2 + 3^2} = 3$ and $\theta = \arctan \frac{3}{0}$. However, $\arctan \frac{3}{0}$ is undefined, because we cannot divide by zero. Instead, we use the fact that $\theta$ is the angle that the line segment from the origin to the point $(0,3)$ makes with the positive $x$-axis. Since this line segment is vertical and points upwards, $\theta$ must be $\frac{\pi}{2}$. Therefore, the polar coordinates of the point $(0,3)$ are $(3,\frac{\pi}{2})$. The answer is $\boxed{(3,\frac{\pi}{2})}$. \\
\midrule
\textbf{[FS-GPT4o-BoN]} 
We have that $r = \sqrt{0^2 + 3^2} = 3$ and that $\theta = \arctan \frac{3}{0} = \frac{\pi}{2}.$  Thus, the polar coordinates are $\boxed{\left( 3, \frac{\pi}{2} \right)}.$ \\
\bottomrule
\end{tabular}
}
\caption{Example rationales generated from the original Deekseek-7B model and the model fine-tuned using FS-GPT4o-BoN. The question is randomly sampled from MATH500.}
\label{tab:appen_samples4}
\vspace{-0.1in}
\end{table*}