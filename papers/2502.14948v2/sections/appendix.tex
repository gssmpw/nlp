\section{Prompt Template}
\label{app:prompt-template}
\begin{tcolorbox}[
    colback=gray!10,           % Light grey background
    colframe=gray!75,          % Dark grey border
    title=Prompt for Generating Problem Description,      % Title of the box
    fonttitle=\bfseries,       % Make the title bold
    width=\textwidth,          % Set the box width to the page width
    sharp corners,             % Optional: sharp corners (remove for rounded)
    boxrule=0.75mm,            % Border thickness
    coltitle=black             % Title text color
]
[Code Snippet]

{\tt for (first\_p, second\_p) in zip\_longest(diag1, diag2):}

{\tt\ \ \ \ assert first\_p[0] == pytest.approx(second\_p[0])}

{\tt\ \ \ \ assert first\_p[1] == pytest.approx(second\_p[1])}

\ 

[Template]

Polycarp is reading a book consisting of $n$ pages numbered from $1$ to $n$. Every time he finishes the page with the number divisible by $m$, he writes down the last digit of this page number. For example, if $n=15$ and $m=5$, pages divisible by $m$ are $5, 10, 15$. Their last digits are $5, 0, 5$ correspondingly, their sum is $10$.

Your task is to calculate the sum of all digits Polycarp has written down.

You have to answer $q$ independent queries.

\ 

[Instruction]

Please gain inspiration from the previous random code snippet and template to create a high-quality python programming problem.

Rules:

- Never mention the ``code snippet''.

- Don't write the solution.

- Do not specify constraints nor example inputs/outputs.

- The inspiration is just an inspiration. You can deviate from it.

- The solution of the problem should be only one function, not an entire program.

- The problem should be self-contained.
\end{tcolorbox}

\begin{tcolorbox}[
    colback=gray!10,           % Light grey background
    colframe=gray!75,          % Dark grey border
    title=Prompt for Generating Function Signature,      % Title of the box
    fonttitle=\bfseries,       % Make the title bold
    width=\textwidth,          % Set the box width to the page width
    sharp corners,             % Optional: sharp corners (remove for rounded)
    boxrule=0.75mm,            % Border thickness
    coltitle=black             % Title text color
]
\textless Problem1\textgreater

Write a function to find the similar elements from the given two tuple lists.

\textless /Problem1\textgreater

\textless Signature1\textgreater

similar\_elements(test\_tup1: list, test\_tup2: list) -\textgreater list

\textless/Signature1\textgreater

\textless Problem2\textgreater

Write a python function to identify non-prime numbers.

\textless/Problem2\textgreater

\textless Signature2\textgreater
\end{tcolorbox}

\begin{tcolorbox}[
    colback=gray!10,           % Light grey background
    colframe=gray!75,          % Dark grey border
    title=Prompt for Generating Test Input,      % Title of the box
    fonttitle=\bfseries,       % Make the title bold
    width=\textwidth,          % Set the box width to the page width
    sharp corners,             % Optional: sharp corners (remove for rounded)
    boxrule=0.75mm,            % Border thickness
    coltitle=black             % Title text color
]
\textless Q1\textgreater

Write a function to find the longest string in a list of strings. If the strings are not comparable (due to being of different lengths), the function should return None.
function signature: longest\_string(strings: list[str])

\textless/Q1\textgreater

\textless ANALYSIS1\textgreater

- Case 1: `strings` is a list of strings.

- Case 2: `strings` is empty.

\textless/ANALYSIS1\textgreater

\textless INPUTS1\textgreater

{\tt longest\_string(['dog', 'cat', 'elephant']) \# Consider case 1.}


{\tt longest\_string([]) \# Consider case 2.}

\textless/INPUTS1\textgreater

\textless Q2\textgreater

Write a function to check if the given string represents a sequence of ASCII characters.
The function should be able to handle different types of sequences, such as lists, tuples, and NumPy arrays.
The function should return True if the sequence contains only ASCII characters, and False otherwise.

function signature: is\_ascii(seq: list) -\textgreater bool

\textless/Q2\textgreater

\textless ANALYSIS2\textgreater

\end{tcolorbox}

\begin{tcolorbox}[
    colback=gray!10,           % Light grey background
    colframe=gray!75,          % Dark grey border
    title=Prompt for Generating Test Output,      % Title of the box
    fonttitle=\bfseries,       % Make the title bold
    width=\textwidth,          % Set the box width to the page width
    sharp corners,             % Optional: sharp corners (remove for rounded)
    boxrule=0.75mm,            % Border thickness
    coltitle=black             % Title text color
]
\textless Q1 \textgreater

Write a function to find the similar elements from the given two tuple lists.

function signature: {\tt similar\_elements(test\_tup1: Tuple, test\_tup2: Tuple)}

\textless/Q1 \textgreater

\textless INPUT1 \textgreater

{\tt similar\_elements((3, 4, 5, 6),(5, 7, 4, 10))}

\textless /INPUT1 \textgreater

\textless ANALYSIS1 \textgreater

- In the first tuple (3, 4, 5, 6), the elements 4 and 5 are present.

- In the second tuple (5, 7, 4, 10), the elements 4 and 5 are also present.

- Since 4 and 5 are present in both tuples, they should be included in the output.

- The other elements in the tuples (3, 6, 7, and 10) are not present in both tuples, so they should not be included in the output.

- So the expected output is (4, 5).

\textless /ANALYSIS1 \textgreater

\textless OUTPUT1 \textgreater

{\tt assert similar\_elements((3, 4, 5, 6),(5, 7, 4, 10)) == (4, 5)}

\textless /OUTPUT1 \textgreater

\textless Q2 \textgreater

Write a python function to identify non-prime numbers.

function signature: {\tt is\_not\_prime(n: int)}

\textless /Q2 \textgreater

\textless INPUT2 \textgreater

{\tt is\_not\_prime(2)}

\textless /INPUT2 \textgreater

\textless ANALYSIS2 \textgreater

- One of the fundamental properties of prime numbers is that they can only be divided evenly by 1 and themselves.

- 2 is considered a prime number because it can only be divided evenly by 1 and itself.

- So 2 is a prime, and the expected output is False.

\textless /ANALYSIS2 \textgreater

\textless OUTPUT2 \textgreater

{\tt assert is\_not\_prime(2) == False}

\textless /OUTPUT2 \textgreater

\textless Q3 \textgreater

Write a function to find all words which are at least 4 characters long in a string by using regex.

function signature: {\tt find\_char\_long(text: str)}

\textless /Q3 \textgreater

\textless INPUT3 \textgreater

{\tt find\_char\_long('Jing Eco and Tech')}

\textless /INPUT3 \textgreater

\textless ANALYSIS3 \textgreater

- For the first word 'Jing', it's 4 characters long, and 4 >= 4, so it should be included.

- For the second word 'Eco', it's 3 characters long, and 3 < 4, so it should NOT be 
included.

- For the third word 'and', it's 3 characters long, and 3 < 4, so it should NOT be included.

- For the fourth word 'Tech', it's 4 characters long, and 4 >= 4, so it should be included.

- To sum up, the output is ['Jing', 'Tech'].

\textless /ANALYSIS3 \textgreater

\textless OUTPUT3 \textgreater

{\tt assert find\_char\_long('Jing Eco and Tech') == ['Jing', 'Tech']}

\textless /OUTPUT3 \textgreater

\textless Q4 \textgreater

{}

function signature: {}

\textless /Q4 \textgreater

\textless INPUT4 \textgreater

{}

\textless /INPUT4 \textgreater

\textless ANALYSIS4 \textgreater
\end{tcolorbox}

\section{Performance Change Across Iterations}
\label{app:performance-change}
\begin{figure*}[ht]
    \centering
    \begin{subfigure}[b]{0.4\textwidth}
        \centering
        \includegraphics[width=\textwidth]{figs/code-gen-perf.pdf}
    \end{subfigure}
    % \begin{subfigure}[b]{0.33\textwidth}
    %     \centering
    %     \includegraphics[width=\textwidth]{figs/test-gen-perf.pdf}
    % \end{subfigure}
    \begin{subfigure}[b]{0.4\textwidth}
        \centering
        \includegraphics[width=\textwidth]{figs/test-gen-fp-perf.pdf}
    \end{subfigure}
    \caption{Performance of Solver-Verifier Framework.}
    \label{fig:solver-verifier-perf}
\end{figure*}