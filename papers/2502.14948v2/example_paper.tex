%%%%%%%% ICML 2025 EXAMPLE LATEX SUBMISSION FILE %%%%%%%%%%%%%%%%%

\documentclass{article}
\usepackage{subcaption}
% Recommended, but optional, packages for figures and better typesetting:
\usepackage{microtype}
\usepackage{graphicx}
% \usepackage{subfigure}
\usepackage{booktabs} % for professional tables
\usepackage{multirow}

% hyperref makes hyperlinks in the resulting PDF.
% If your build breaks (sometimes temporarily if a hyperlink spans a page)
% please comment out the following usepackage line and replace
% \usepackage{icml2025} with \usepackage[nohyperref]{icml2025} above.
\usepackage{hyperref}


% Attempt to make hyperref and algorithmic work together better:
\newcommand{\theHalgorithm}{\arabic{algorithm}}

% Use the following line for the initial blind version submitted for review:
% \usepackage{icml2025}

% If accepted, instead use the following line for the camera-ready submission:
\usepackage[accepted]{icml2025}

% For theorems and such
\usepackage{amsmath}
\usepackage{amssymb}
\usepackage{mathtools}
\usepackage{amsthm}
\usepackage{dsfont}
\usepackage{tcolorbox}
\usepackage{enumitem}


% if you use cleveref..
\usepackage[capitalize,noabbrev]{cleveref}

%%%%%%%%%%%%%%%%%%%%%%%%%%%%%%%%
% THEOREMS
%%%%%%%%%%%%%%%%%%%%%%%%%%%%%%%%
\theoremstyle{plain}
\newtheorem{theorem}{Theorem}[section]
\newtheorem{proposition}[theorem]{Proposition}
\newtheorem{lemma}[theorem]{Lemma}
\newtheorem{corollary}[theorem]{Corollary}
\theoremstyle{definition}
\newtheorem{definition}[theorem]{Definition}
\newtheorem{assumption}[theorem]{Assumption}
\theoremstyle{remark}
\newtheorem{remark}[theorem]{Remark}

% Todonotes is useful during development; simply uncomment the next line
%    and comment out the line below the next line to turn off comments
%\usepackage[disable,textsize=tiny]{todonotes}
\usepackage[textsize=tiny]{todonotes}


% The \icmltitle you define below is probably too long as a header.
% Therefore, a short form for the running title is supplied here:
\icmltitlerunning{Learning to Solve and Verify}

\begin{document}

\twocolumn[
\icmltitle{Learning to Solve and Verify:\\
           A Self-Play Framework for Code and Test Generation}
% Test-Diven coding improvement?
% Focus: discriminator of the verifier; False positive rate is very important?

% It is OKAY to include author information, even for blind
% submissions: the style file will automatically remove it for you
% unless you've provided the [accepted] option to the icml2025
% package.

% List of affiliations: The first argument should be a (short)
% identifier you will use later to specify author affiliations
% Academic affiliations should list Department, University, City, Region, Country
% Industry affiliations should list Company, City, Region, Country

% You can specify symbols, otherwise they are numbered in order.
% Ideally, you should not use this facility. Affiliations will be numbered
% in order of appearance and this is the preferred way.
\icmlsetsymbol{equal}{*}

\begin{icmlauthorlist}
\icmlauthor{Zi Lin}{equal,yyy}
\icmlauthor{Sheng Shen}{comp}
\icmlauthor{Jingbo Shang}{yyy}
\icmlauthor{Jason Weston}{comp}
\icmlauthor{Yixin Nie}{comp}
% \icmlauthor{Firstname6 Lastname6}{sch,yyy,comp}
% \icmlauthor{Firstname7 Lastname7}{comp}
% \icmlauthor{}{sch}
% \icmlauthor{Firstname8 Lastname8}{sch}
% \icmlauthor{Firstname8 Lastname8}{yyy,comp}
% \icmlauthor{}{sch}
% \icmlauthor{}{sch}
\end{icmlauthorlist}

\icmlaffiliation{yyy}{University of California, San Diego}
\icmlaffiliation{comp}{Meta}
% \icmlaffiliation{sch}{School of ZZZ, Institute of WWW, Location, Country}

\icmlcorrespondingauthor{Zi Lin}{lzi@ucsd.edu}
\icmlcorrespondingauthor{Yixin Nie}{ynie@meta.com}

% You may provide any keywords that you
% find helpful for describing your paper; these are used to populate
% the "keywords" metadata in the PDF but will not be shown in the document
\icmlkeywords{Machine Learning, ICML}

\vskip 0.3in
]

% this must go after the closing bracket ] following \twocolumn[ ...

% This command actually creates the footnote in the first column
% listing the affiliations and the copyright notice.
% The command takes one argument, which is text to display at the start of the footnote.
% The \icmlEqualContribution command is standard text for equal contribution.
% Remove it (just {}) if you do not need this facility.

%\printAffiliationsAndNotice{}  % leave blank if no need to mention equal contribution
\printAffiliationsAndNotice{*Work done during internship at Meta GenAI.\ } % otherwise use the standard text.
% \footnote{1. Footnote}

\newcommand{\jingbo}[1]{\textcolor{blue}{\textbf{Jingbo:} #1}}

\begin{abstract}
% \jingbo{The first sentence seems saying that verifying the correctness of a generated code is more difficult than generating the code itself. However, the later part of the abstract suggests that we are generating unit tests to improve the code genereation. This falls back to the issue Taylor mentioned in your thesis proposal. 
% I think we need to make a few points in some order: (0) Code generation has a long way to go because its correctness is still low. (1) Generating unit tests for a particular problem is simpler than verifying the soundness of the code --- that is, if the generated code passes all the generated unit tests, it can still be wrong. (2) A set of simple unit tests can effectively improve the code generation. (3) Given the generated code, one can better design the unit tests. (4) Putting all these points together will help us motivate the self-play framework.}
Recent advances in large language models (LLMs) have improved their performance on coding benchmarks. 
However, improvement is plateauing due to the exhaustion of readily available high-quality data.
%However, the improvement plateaued due to the exhaustion of readily available high-quality data.
% However, their ability remains unknown when applied to diverse coding scenarios. 
% This challenge is primarily due to the lack of high-quality instruction-following data.
% However, the improvement 
% Recent advances in large language models have improved their ability to generate code, but verifying its correctness remains a challenge, as effective test generation remains a bottleneck. 
% While LLMs can autonomously produce code and unit tests, their test generation capability falls behind due to limited fine-tuning data.
%Prior work involves using synthetic \textit{self-instruct} data from larger models. However, training on a model’s own outputs can lead to error accumulation.
Prior work has shown the potential of synthetic \textit{self-instruct} data, but naively training on a model’s own outputs can cause error accumulation, especially in coding tasks, where generalization may collapse due to overly simple or erroneous training data, highlighting the need for rigorous quality checks on synthetic data.
% As a result, there is a critical need for strongly performing methods to select high-quality synthetic data. 
In this work, we explore an effective approach whereby the model itself verifies the correctness of 
its own data. %the generated data. 
We thus propose {\sc Sol-Ver}, a self-play solver-verifier framework that jointly improves a single model’s code and test generation capacity. By iteratively refining code (LLM-as-a-solver) and tests (LLM-as-a-verifier) together, we boost both capabilities without relying on human annotations or larger teacher models.
%
%One effective approach is to have the model itself verify the correctness of the generated data. 
%In this work, we propose {\sc Sol-Ver}, a self-play solver-verifier framework that jointly improves a single model’s code and test generation capacity. By iteratively refining code (LLM-as-a-solver) and tests (LLM-as-a-verifier) together, we boost both capabilities without relying on human annotations or larger teacher models.
Experiments with the Llama 3.1 8B model demonstrate substantial performance enhancements, achieving average relative improvements of 19.63\% in code generation and 17.49\% in test generation on MBPP and LiveCodeBench.
\end{abstract}

\section{Introduction}
% \textsc{Sol-Ver}


% This seems to be a better flow: synthetic data by the same size model is not good -> data has a lot of fales positive (fales positive is high cite paper: alphaCode) ->  data need to be filtered/checked -> we build unit test verifier for filtering. -> iteratively improve both

Large language models (LLMs) have demonstrated impressive ability in code generation, significantly enhancing the programming efficiency and productivity of human developers~\cite{li2022competition,roziere2023code,codealpaca}.
The ability to code is largely driven by high-quality online resources where coding problems, human-written solutions, and corresponding unit tests are freely available. However, as these free resources being depleted, the momentum of LLMs' improvement diminishes.
% Building on these successes, researchers are exploring additional methods to optimize LLM performance across more diverse coding scenarios.

To address the scarcity of supervised data for code generation, recent studies have adopted synthetic data generation techniques such as {\sc Self-Instruct}~\cite{wang2023self} to augmenting LLM training sets. 
% leveraging the robust code generation abilities of teacher models, enabling the distillation of synthetic data that further enhances LLMs.
% the capabilities of LLMs in diverse programming contexts.
Specifically, previous work collects and designs code instructions and generates corresponding responses using a high-capacity teacher LLM. This generated data is then employed to fine-tune a student LLM, thereby enhancing its code generation abilities. 
Although synthetic code data produced in this manner has demonstrated success, it relies on the availability of a strong teacher model, presumably with a larger parameter size and higher computation costs.
Additionally, existing work has shown that training a model on data generated by itself is ineffective because errors introduced during generation tend to accumulate over iterations~\cite{dubey2024llama}. As a result, there is a critical need for effective methods to verify the generated data.

Despite this pressing need, evaluating the correctness of the generated code is not trivial and often demands substantial programming expertise, even for human annotators.
Recently, some research has explored the use of LLM-as-a-judge~\cite{mcaleese2024llm,alshahwan2024automated,dong2024self}, which can automatically provide feedback on the code it has generated, for example by running the code against unit tests it produces. However, as we will show in this work, the current capability of LLMs to generate unit tests, herein referred to as LLM-as-a-verifier, is substantially worse than their ability to produce code solutions, herein referred to as LLM-as-a-solver (Section~\ref{sec:base-performance}). This is because of the lack of high-quality data specifically for unit test generation during the post-training phase, i.e., most fine-tuning data focuses on code generation, while only a small portion targets unit test generation.
%(both data quantity and quality)

% To address this limitation, one effective approach is to utilize synthetic data generated by LLMs, known as \textit{self-instruct} data~\cite{wang2023self}.
% This method has shown promising results when the model is trained on data generated by larger, more competent models. However, previous research has revealed that training models on their own generated data may not be beneficial (Llama 3.1 report). This may be due to the model accumulating its own mistakes during fine-tuning. Therefore, it is crucial to check the correctness of the generated solutions. 

In this work, we propose \textsc{Sol-Ver}, a self-play solver-verifier framework to iteratively train a model for both code and test generation. The main idea is to let the LLM-as-a-solver and LLM-as-a-verifier help each other. Specifically, we ask the model to generate code solutions and unit tests for the same set of coding problems. By  executing the generated test against the generated code, we obtain feedback for training, involving two steps: (1) SFT training: we take the passed examples for fine-tuning the model, and (2) DPO training: we take both passed and failed examples as preference pairs to further train the model aligning with the preference. These training steps are for both code generation and unit test generation, and they can be repeated in an iterative manner. 

The experimental results on Llama 3.1 8B model show that we can successfully improve the model’s performance on both code and test generation without relying on human-annotated data or larger models. Specifically, on MBPP and LiveCodeBench, we achieve an average of 19.63\% and 17.49\% relative improvement for code and test generation respectively.

\begin{figure*}[h]
    \centering
    \includegraphics[width=0.9\textwidth]{figs/icml-overview.pdf}
    \caption{An overview of the {\sc Sol-Ver} framework.
We train an LLM to both generate coding solutions (solver) and unit tests (verifier) in an iterative self-play framework, whereby synthetic preference pairs are constructed at each iteration depending on whether the code passes the generated tests or not.
We show that this approach enables the model to self-improve in both capabilities (see \autoref{tab:full-performance}).
    }
\if 0
insimultan
    We thus propose SOL-VER, a
self-play solver-verifier framework that jointly im-
proves a single model’s code and test generation
capacity. By iteratively refining code (LLM-as-
a-solver) and tests (LLM-as-a-verifier) together,
we boost both capabilities without relying on hu-
man annotations or larger teacher models
\fi 
    \label{fig:enter-label}
\end{figure*}

In summary, our work makes the following contributions:
\begin{itemize}[leftmargin=*]
    \item \textbf{Identification of a {\em critical gap}:} We analyze and highlight the significant gap in LLMs' abilities between code generation and unit test generation.
    \item \textbf{Novel \textit{Self-Play} Framework:} We propose a novel iterative framework where the model simultaneously functions as a code solver and a verifier. This methodology effectively self-aligns the model's outputs with desired performance criteria without relying on external annotations or teacher models.
    \item \textbf{High-Quality Synthetic Data Generation:} We contribute a generalizable method for creating high-quality synthetic data for both code and unit test generation. This data augmentation approach can be extended to various model training scenarios in the coding domain.
\end{itemize}



\section{Related Work}
% \paragraph{Synthetic Data Generation for Coding}
Various empirical studies applying scaling laws to the training of foundation models have highlighted the critical role of the data size~\cite{kaplan2020scaling,hoffmann2022training}. To address the need for larger datasets, synthetic data generation has become a popular and cost-effective solution, which leverages advanced LLMs to produce high-quality data.
One notable method is {\sc Self-Instruct}~\cite{wang2023self}, which employs a pre-trained LLM to generate instruction-output pairs from a small seed dataset.

In the realm of code, previous work commonly devises synthetic coding instructions using a stronger teacher model (e.g., ChatGPT or GPT-4) and then finetunes a weaker student model (e.g., {\sc CodeAlpaca}~\cite{codealpaca} and {\sc CodeLlama}~\cite{roziere2023code}) with the generated data to distill knowledge from the teacher. For example, code alpaca consists of 20K automatically generated code instructions by applying {\sc Self-Instruct} on ChatGPT using 21 seed tasks. To further enhance the code abilities of LLMs, \citet{luo2023wizardcoder} proposes \textit{Code Eval-Instruct} that employs various heuristics to increase the complexity of seed code instructions. Magicoder~\cite{wei2024magicoder} proposes to generate new coding problems by drawing inspiration from random snippets collected from open source code.


While previous work has shown significant improvements for models trained on data generated by larger, more competent models, training an LLM on its own generated data is not helpful and can even degrade performance~\cite{zhou2024lima}. Therefore, to prevent the model from learning errors present in its own generated data, some post-processing steps are essential. For example, Llama 3.1~\cite{dubey2024llama} utilizes error feedback from test execution and adopts an iterative self-correction procedure to revise potential errors.
% Add CodeDPO here (02/03):
CodeDPO~\cite{zhang2024codedpo} replaces teacher models with a self-generation-and-validation process that uses a PageRank-like algorithm to rank code snippets by correctness and efficiency, yielding diverse preference optimization data without external resources.

% I feel our work is very similar to AutoIF (https://arxiv.org/pdf/2406.13542), but they do not focus on code genration tasks.

In this work, we propose leveraging both positive and negative examples generated by the model, treating pairs of passing and failing responses as chosen-rejected pairs for Direct Preference Optimization (DPO)~\cite{rafailov2024direct}. Note that our method is complementary to self-correction, rather than orthogonal. By improving the quality of unit tests, our framework enhances the accuracy of unit test execution feedback, and thereby can benefit self-correction scenarios as well. While CodeDPO also employs a self-verification method, it is fundamentally constrained by the quality of its generated tests. In contrast, our approach simultaneously improves the generation of both code and tests in a self-play method, mitigating this bottleneck and enabling more robust training.
% no experiment has show this.
\section{A Self-play Solver-verifier Framework}
\subsection{Problem Formulation}
\label{sec:problem-formulation}
\paragraph{Setup} We consider that an LLM can play two roles:
\vspace{-2mm}
\begin{itemize}[leftmargin=*]
    \item \textbf{Solver ($S$):} Given a coding problem description $P$, it produces a candidate solution $C$ (e.g., a piece of code).
    \item \textbf{Verifier ($V$):} Given a proposed solution $C$ and the original problem $P$, the verifier tries to produce test cases $\mathbf{T}$\footnote{We use bold Italic to represent a set.} (e.g., a set of inputs and expected outputs) and can catch errors in $C$ if it is incorrect. Essentially, it produces and selects challenging unit tests to determine if the code is correct or not.%\footnote{In practice, solver and verifier can generate multiple candidates.} 
    %through multi-time inference.}
\end{itemize}

The objective of the solver is to produce a correct solution $C$ that will pass any tests the verifier can come up with. The objective of the verifier is to produce a set of tests $\mathbf{T}$ that will fail any incorrect solutions and thus distinguish correct solutions from incorrect ones.

Let $p(\mathbf{P})$ be the distribution over problem statements. We can think of having a training set of problems or a domain from which we can sample problems. The sampling strategies we consider are detailed  in Section~\ref{sec:practical-settings}.

The solver $S_{\theta}$ is a model parameterized by $\theta$ that, given a problem $P$, generates a candidate solution $C$: $C \sim S_{\theta}(\cdot | P)$. The verifier $V_{\phi}$ is a model parameterized by $\phi$, given the problem $P$ and a candidate solution $C$, generates a test suite $\mathbf{T}$: $\mathbf{T} \sim V_{\phi}(\cdot | P, C)$.

In practice, the solver and verifier can be the same LLM.

\textbf{Scoring Function}\ \ \ We define a function that executes $C$ on the tests $\mathbf{T}$ as $\text{Score}(C, \mathbf{T}) \in [0, 1]$, which is the fraction of tests passed by solution $C$. A score of 1 means $C$ passes every test $T$; a score of 0 means it failed all tests. Formally,
\begin{align}
    \text{Score}(C, \mathbf{T}) = \mathbb{E}_{T\sim \mathbf{T}}[\mathbb{I}(C(T)=\text{expected\_output}(T))]
\end{align}
where $\mathbb{I}$ is the indicator function, and $C(T)$ means running one single test on code solution $C$.

We sample a set of problems $P \sim p(\mathbf{P})$, generate some candidate solutions $C \sim S_{\theta}(\cdot | P)$, and generate candidate tests $\mathbf{T} \sim V_{\phi}(\cdot | P, C)$. We now have tuple $(P, C, \mathbf{T})$. We consider:

\begin{align}
y = \begin{cases}
1 & \text{if } \text{Score}(C,\mathbf{T}) = 1 \text{ (i.e., passes all tests)} \\
0 & \text{otherwise}.
\end{cases}
\end{align}

We employ two stages of  training to make use of both chosen ($y=1$) and rejected ($y=0$) examples for training the solver and verifier, described as follows:

\textbf{Stage 1: SFT Training}\ \ \ For pairs where $y = 1$, we have a correct solution-test pair. These are high-quality examples that reflect desired behavior, i.e., the solution $C$ solves the problem $P$, and the test suite $\mathbf{T}$ properly validates that the solution is correct. We use $(P, C, \mathbf{T}, y = 1)$ tuples to fine-tune the model directly. The training signal here encourages the model (1) as a solver, to generate similar correct solutions for similar problems, and (2) as a verifier, to produce meaningful tests that confirm correctness. We call this the supervised fine-tuning (SFT) stage, where we optimize for both solver and verifier:

\begin{align}
    \mathcal{L}_{\text{SFT}_\text{solver}}(\theta) = -\mathbb{E}_{(P, C, \mathbf{T}):y=1}[\log S_{\theta}(C|P)] \\
    \mathcal{L}_{\text{SFT}_\text{verifier}}(\phi) = -\mathbb{E}_{(P, C, \mathbf{T}):y=1}[\log V_{\phi}(\mathbf{T}|P, C)]
\end{align}

In practice, both objectives can be trained using a mixture of data consisting of chosen examples for solver and verifier.

\textbf{Stage 2: DPO Training}\ \ 
\label{sec:dpo-training}
We now aim to form pairwise comparisons (preferences) to train both solver and verifier roles more effectively. In practice, we adopted the Direct Preference Optimization (DPO) method, but any preference tuning methods can be used at this stage.

For the solver perspective, for each problem $P$, and each chosen tuple $(P, C^+, \mathbf{T}, y=1)$, we find a rejected tuple $(P, C^-, \mathbf{T}, y=0)$. Following standard DPO training~\cite{rafailov2024direct}, we can formulate our policy objective as:
\begin{align}
    &\mathcal{L}_{\text{DPO}_\text{solver}}(S^*_{\theta}; S_{\theta}) = \nonumber \\ &-\mathbb{E}[\log\sigma(\beta\log \frac{S^*_{\theta}(C^+|P)}{S_{\theta}(C^+|P)} - \beta\log \frac{S^*_{\theta}(C^-|P)}{S_{\theta}(C^-|P)}]
\end{align}

where $\beta$ is the hyperparameter to regulate the strength of weight updates; $S_{\theta}(C|P)$ is the probability that our model (with parameter $\theta$) assigns to generating code solution $C$ given problem $P$.

For the verifier perspective, similarly, for each problem $P$, and each chosen tuple $(P, C, \mathbf{T}^+): y=1$, find a rejected tuple $(P, C, \mathbf{T}^-):y=0$\footnote{This can be achieved by selecting any expected output that is not equal to the chosen one in the sampling space for $T$.}. The verifier-related DPO loss is then:
\begin{align}
    &\mathcal{L}_{\text{DPO}_\text{verifier}}(V^*_{\phi}; V_{\phi}) = \nonumber \\ &-\mathbb{E}[\log\sigma(\beta\log \frac{V^*_{\phi}(\mathbf{T}^+|P, C)}{V_{\phi}(\mathbf{T}^+|P, C)} - \beta\log \frac{V^*_{\phi}(\mathbf{T}^-|P, C)}{V_{\phi}(\mathbf{T}^-|P, C)}]
\end{align}

\subsection{Synthetic Data Generation}
\label{sec:practical-settings}
In this section, we describe our approach to generating the synthetic data, including problem description generation, code generation, test generation and preference data generation. All related prompts can be found in Appendix~\ref{app:prompt-template}.

\textbf{Problem Description Generation}\ \ \ Following Magicoder~\cite{wei2024magicoder}, we generate a large collection of programming problem descriptions that span a diverse range of topics, including those in the long tail distribution. To achieve this diversity, we sample random code snippets from various sources and prompt the model to generate programming problems inspired by these examples. This allows us to tap into a wide range of topics and create a comprehensive set of problem descriptions (as demonstrated in Figure~\ref{fig:embedding-plot}).

% \begin{figure*}[h]
%     \centering
%     \includegraphics[width=\textwidth]{figs/test-gen-overview.pdf}
%     \caption{An overview of the test generation pipeline.}
%     \label{fig:test-generation-overview}
% \end{figure*}

It is noted that code generation tasks can follow different problem description formats despite having the same content. For example, here is the same problem but stated in different ways:
\begin{itemize}[nosep,leftmargin=*]
    \item Write a python function to remove the kth element from a given list.
    \item In the ancient Library of Alexandria, scrolls are stored in a mystical list. The librarian needs to remove a specific scroll whenever a visitor requests it.
\end{itemize}

To make our prompt sets accommodate these diverse problem description formats, we adopt some templates from the training set of different coding benchmarks (e.g., MBPP, APPS) into the original prompt to generate the problem description.

To create a self-contained coding problem description, we need to ensure it not only includes a clear problem statement but also a well-defined starter code. The starter code should contain all necessary built-in libraries and a detailed function signature description to show what should be the input and output. Therefore, after obtaining the original problem description generated by the model, we then ask the model to generate function signatures. After deduplication, we get 103,280 problem descriptions in total.

\textbf{Code Generation}\ \ \ We prompt the LLM to solve each problem given the generated function signature. Following the Llama 3.1~\cite{dubey2024llama}, we also require the model to explain its thought process in comments, which improves code generation in both accuracy and readability.

\textbf{Test Generation}\ \ \ 
\label{sec:test-generation}
We use a pipelined approach to generate unit test sets. Specifically, we first ask the model to generate a set of valid inputs and then ask it to generate expected outputs based on the inputs.

For input generation, we ask the model to generate different types of function inputs to cover different cases including general, corner or difficult cases. For example, the following problem description should contain two different cases:

\fbox{%
  \parbox{\columnwidth}{%
\textbf{Problem Description:}

Write a function to find the longest string in a list of strings. If the strings are not comparable due to being of different lengths, the function should return None.

{\tt longest\_string(strings: list[str])}
% \newline

\textbf{Case 1:} {\tt strings} is a list of strings.

\textbf{Case 2:} {\tt strings} is empty.

  }%
}

For output generation, we ask the model to generate the expected output based on the problem description and input. We also find that two strategies can boost the performance of unit test generation (as demonstrated in Section~\ref{sec:base-performance}):
\begin{itemize}[nosep, leftmargin=*]
    \item \textbf{Majority Voting}: The majority voting mechanism in the self-consistency approach~\cite{wang2022self} asks the model to generate multiple candidate outputs for a query, and aggregates them using a majority voting procedure to select the most commonly occurring answer.
    \item \textbf{Chain-of-Thought reasoning:} Following \citet{wei2022chain}, before outputting the expected values, we asked the model to first generate its reasoning steps.
\end{itemize}

Additionally, we employ the following strategies to ensure the quality and robustness of the generated unit tests:

\begin{itemize}[nosep, leftmargin=*]
    \item \textbf{Test Coverage Optimization:} We sample multiple candidate test cases and strategically select a subset that maximizes branch coverage of the solution code (a maximum coverage problem), ensuring comprehensive testing of different execution paths.
    \item \textbf{Output Diversification:} We notice that if the outputs of the test cases are not diverse enough, the solution code can cheat by exploiting patterns in test cases. For example, if all test cases return the same value (e.g., {\tt True}), the model could trivially pass by implementing a function that always returns that value. To address this issue, we explicitly select test samples with diverse output values.
\end{itemize}

\textbf{Synthetic Preference Data Generation}\ \ \ 
The key goal of constructing synthetic data for preference tuning, which is DPO training in our case, is to construct pairs of ``chosen'' and ``rejected'' responses to the given prompts. As illustrated in Section~\ref{sec:dpo-training}, the ``chosen'' examples are those examples where the solver  agrees with the verifier, i.e., the generated codes can pass the generated tests. However, identifying ``rejected'' examples is considerably more complex, as it is not always clear which side is at fault when they disagree.

In previous work, \citet{dong2024self} adopts an automated quality cross verification process by selecting both test case and functions with an accuracy rate grater than 0.5.
We utilize a similar cross validation strategy, but with a focus on both code and test generation. Specifically, if at least one generated solution passes all the generated tests, we will assume the solution and the test are ``correct''.
For training the solver, any other sampled code solutions that fail to pass these ``correct'' tests are treated as ``rejected'' examples.
Similarly, for training the verifier, when we find a ``correct'' test $f(x) == y$, we revisit the original sampling space for generating expected outputs, and treat any expected outputs $y' \neq y$ as ``rejected'' tests $f(x) == y'$. In this way, we can reuse all the chain-of-thought explanations generated by the model  during the majority voting process.

\section{Experiments}
This section presents experiments on several reward model benchmarks, including experimental setup (\cref{sec:exp_setup}), results (\cref{sec:exp_result}), and analyses (\cref{sec:exp_analysis}).


\begin{table*}
    \centering
    \small
    % \resizebox{\linewidth}{!}{
    \begin{tabular}{lccccccc}
    \toprule
    \multirow{2}{*}{Model} & \multicolumn{2}{c}{RM-Bench} & \multirow{2}{*}{JudgeBench} & \multicolumn{3}{c}{IFBench} & \multirow{2}{*}{Overall} \\
    \cmidrule{2-3} \cmidrule{5-7}
    & Normal & Hard & & Simple & Normal & Hard & \\
    \midrule
ArmoRM-Llama3-8B-v0.1 &$76.7$&$34.6$&$51.9$&$72.3$&$66.2$&$59.5$&$56.5$\\
INF-ORM-Llama3.1-70B &$77.5$&$25.1$&$59.1$&$78.7$&$69.2$&$53.8$&$55.7$\\
Skywork-Reward-Llama-3.1-8B-v0.2 &$78.0$&$31.8$&$57.8$&$78.7$&$69.2$&$59.8$&$58.1$\\
Skywork-Reward-Gemma-2-27B &$82.7$&$35.1$&$55.8$&$\boldsymbol{87.2}$&$68.4$&$56.1$&$59.2$\\
internlm2-7b-reward &$72.6$&$19.9$&$56.2$&$74.5$&$61.7$&$55.7$&$52.0$\\
internlm2-20b-reward &$74.4$&$26.1$&$61.7$&$74.5$&$68.4$&$58.7$&$56.4$\\
\midrule
GPT-4o &$71.4$&$27.9$&$64.6$&$\underline{85.1}$&$66.2$&$54.4$&$56.3$\\
GPT-4o mini &$60.5$&$15.0$&$51.9$&$70.2$&$59.4$&$51.9$&$45.9$\\
o3-mini &$76.0$&$38.6$&$66.6$&$81.9$&$\underline{76.3}$&$64.6$&$62.8$\\
Llama3-8B Instruct &$\phantom{0}9.3$&$20.2$&$\phantom{0}2.6$&$12.8$&$12.8$&$13.6$&$11.3$\\
DeepSeek-R1 &$83.7$&$50.1$&$\boldsymbol{74.4}$&$72.3$&$74.4$&$64.0$&$69.1$\\
DeepSeek-R1-Distill-Llama-8B &$42.1$&$56.8$&$47.7$&$53.2$&$55.6$&$54.2$&$50.3$\\
\midrule
\ourmethodllama &$79.3$&$53.5$&$52.9$&$70.2$&$63.9$&$67.8$&$63.2$\\
\quad w/ search engine &$76.0$&$49.9$&$55.2$&$74.5$&$69.2$&$67.8$&$62.5$\\
\ourmethodmini &$\boldsymbol{86.0}$&$\boldsymbol{60.2}$&$\underline{68.2}$&$78.7$&$69.2$&$\boldsymbol{78.0}$&$\boldsymbol{72.5}$\\
\quad w/ search engine &$\underline{84.2}$&$\underline{59.7}$&$60.7$&$68.1$&$\boldsymbol{80.5}$&$\underline{76.1}$&$\underline{70.3}$\\
    \bottomrule
    \end{tabular}
    % }
    \caption{Experimental results (\%) of all investigated baselines and \ourmethod. The overall score is the average of RM-Bench, JudgeBench, and the micro-averaged score of three subsets of IFBench. By default, \ourmethod relies on its parametric knowledge, and ``w/ search engine'' denotes using Google API as an external source.}
    \label{tab:main_exp}
\end{table*}


\subsection{Experimental Setup}
\label{sec:exp_setup}

\paragraph{\ourmethod Implementation}
We adopt the advanced and lightweight ArmoRM~\citep{wang2024interpretable} as the base reward model to compute human preference scores. As \ourmethod is agnostic to reward models, one can also adopt other advanced reward models. 
We use GPT-4o mini~\citep{OpenAI2024} as the LLM backbone for implementing all modules and developing \ourmethodmini. We also employ the open-source LLM Llama3-8B Instruct~\citep{dubey2024llama} as the backbone and develop \ourmethodllama, except for the instruction-following verification agent, which requires strong coding capabilities and is instead powered by Qwen2.5-Coder 7B~\citep{hui2024qwen2}.
We adopt two knowledge sources for the factuality verification agent: an external search engine using Google API and the LLM’s parametric parameters. More details are placed in appendix~\ref{sec:app_method}.



\paragraph{Evaluation Benchmarks}
Reward model benchmarks typically involve an instruction and a response pair and require selecting the better response as the chosen one.
We use RM-Bench~\citep{liu2024rm}, JudgeBench~\citep{tan2024judgebench}, and a new benchmark \ourdataset as evaluation benchmarks, as both RM-Bench and JudgeBench include response pairs involving factual correctness. We select the chat subset of RM-Bench as the evaluation set, using both the normal and hard settings. For JudgeBench, we use the knowledge subset as the evaluation set.
We further construct a new benchmark \ourdataset to evaluate reward models on selecting responses that better follow constraints in instructions as there is no existing relevant benchmark.
Specifically, we first construct instructions with several implicit constraints, integrating the constraint information with the primary task objective through paraphrasing. The constraints include both hard constraints, such as length, format, and keywords, as well as soft constraints, such as content and style. We then use GPT-4o to generate $8$ responses for each instruction with a sampling temperature of $1.0$. For each instruction, we create a response pair, selecting the one that satisfies all constraints as the chosen response and otherwise rejected. Based on the number of unsatisfied constraints (UC) in the rejected response, we split \ourdataset instances into three subsets: simple (\#UC$\geq$3), normal (\#UC$=$2), and hard (\#UC$=$1),  containing $47$, $133$, and $264$ instances respectively. 
We report the micro-averaged accuracy across the three subsets as the final metric for \ourdataset. More evaluation details on these benchmarks are provided in appendix~\ref{sec:app_exp}.



\paragraph{Baselines}
\looseness = -1
We mainly investigate two categories of baselines: (1) typical reward models, which are specifically trained for reward modeling and typically implemented as regression models to score each response and select the one with the highest reward score as the chosen response. We investigate several advanced and representative reward models, including ArmoRM~\citep{wang2024interpretable}, INF-ORM-Llama3.1-70B~\citep{infly2024inf}, Skywork-Reward~\citep{liu2024skywork}, internlm2 reward~\citep{cai2024internlm2}. (2) LLMs as generative reward models, where large language models serve as generative reward models to score responses or perform pairwise comparisons to select the best response~\citep{lambert2024rewardbench}. We evaluate proprietary models, including GPT-4o~\citep{OpenAI20244o}, GPT-4o mini~\citep{OpenAI2024}, o3-mini~\citep{openai2025o3mini}, and open-source LLMs, including Llama3-8B Instruct~\citep{dubey2024llama}, 
% {\color{red}Qwen2.5-Coder 7B~\citep{hui2024qwen2}}, 
DeepSeek-R1, and R1 distilled Llama3-8B model~\citep{guo2025deepseek}. 
We evaluate all the baselines using the code repository provided by \citet{lambert2024rewardbench}.


\subsection{Experimental Results}
\label{sec:exp_result}



Table~\ref{tab:main_exp} presents the experimental results, and we can observe that:
(1) Existing reward models fall short in selecting more factual responses or better adhering to hard constraints in instructions, which may limit their reliability in real-world applications.
(2) \ourmethod significantly outperforms the base reward model AromRM and the corresponding LLM backbone GPT-4o mini and Llama3-8B Instruct. It demonstrates that designing an appropriate reward agentic workflow can effectively enhance reward model performance.
(3) Even when using Llama3-8B Instruct as the LLM backbone, \ourmethodllama outperforms reward models with much more parameters and more advanced proprietary LLMs such as GPT-4o, which suggests that \ourmethod is more cost-efficient without requiring additional reward modeling training data or more parameters to achieve advanced performance.
(4) Using a search engine as an external knowledge source for factuality slightly reduces performance in RM-Bench and JudgeBench. One possible reason is that the retrieved information may contain noise or irrelevant information~\citep{chen2024benchmarking}. We leave the detailed analysis and design of retrieval-augmented agents for future work.
(5) \ourmethod achieves significant improvements on IFBench, particularly in the hard subset. It suggests that while not perfectly solved, existing LLMs can effectively analyze hard constraints and generate verification code, which can help the training of advanced LLMs~\citep{lambert2024t}.

In conclusion, incorporating additional verification agents for specific scenarios~\cite{mu2024rule, lambert2024t}, particularly those with verifiable correctness, can develop more reliable and advanced reward systems, presenting a promising direction for future reward model development.



\subsection{Analysis}
\label{sec:exp_analysis}


\begin{table}
    \centering
    \small
    \resizebox{\linewidth}{!}{
    \setlength{\tabcolsep}{3pt}
    \begin{tabular}{lccc}
    \toprule
    Model & RM-Bench & JudgeBench & IFBench \\
    \midrule
    \ourmethodmini &$73.1$&$68.2$&$75.5$\\
    \hspace{2mm}\textit{-- factuality verifier} &$54.0$&$52.9$&$73.6$\\
    \hspace{2mm}\textit{-- if verifier} &$74.7$&$66.2$&$60.4$\\
    \hspace{2mm}\textit{-- both} &$55.4$&$58.8$&$58.8$\\
    \midrule[0.1pt]
    Oracle setting &$76.7$&$70.1$&$77.5$\\
    \midrule
    \ourmethodllama &$66.4$&$52.9$&$66.9$\\
    \hspace{2mm}\textit{-- factuality verifier} &$51.9$&$51.6$&$65.8$\\
    \hspace{2mm}\textit{-- if verifier} &$58.0$&$57.5$&$57.2$\\
    \hspace{2mm}\textit{-- both} &$44.8$&$55.5$&$57.2$\\
    \midrule[0.1pt]
    Oracle setting &$79.5$&$73.1$&$68.5$\\
    \bottomrule
    \end{tabular}
    }
    \caption{Experimental results (\%) of ablation study and the oracle setting. \textit{-- factuality verifier} and \textit{-- if verifier} refer to the reduction of the corresponding verification agent into a single LLM scorer.
    The results are the micro-averaged scores of all the corresponding subsets.}
    \label{tab:analysis}
\end{table}

We first conduct an ablation study on the verification agents in \ourmethod. Specifically, we investigate three settings: \textit{-- factuality verifier}, \textit{-- if verifier}, and \textit{-- both}, where the corresponding verification agents are reduced to \textbf{a single step}: using an additional LLM backbone to directly score the response, which is equivalent to the simple ensemble of the reward model ArmoRM with the corresponding LLM as a generative reward model~\citep{costereward2024}.
The ablation results are shown in Table~\ref{tab:analysis}. We can observe that removing the well-designed verification agent leads to a significant performance decrease. It demonstrates the importance of well-designed verification agents, and we encourage the community to develop more advanced verification agents for a more reliable \ourmethod.



We also observe the oracle setting of \ourmethod that invokes the most appropriate verification agents, that is, invoking the factuality agent on RM-Bench and JudgeBench, and the instruction-following verification agent on IFBench. The experimental results are shown in Table~\ref{tab:analysis}, and we observe that both \ourmethodmini and \ourmethodllama perform significantly better in the oracle setting. This further demonstrates the effectiveness of the verification agents and suggests that the planner in \ourmethod still has a large room for improvement and we leave developing a more advanced planner for future work. This also suggests that in some specific and well-defined scenarios, one can adopt the corresponding verification agent alone to achieve better results.

\begin{figure*}
    \centering
    \includegraphics[width=0.98\linewidth]{figures/best_of_n.pdf}
    \caption{Best-of-n results (\%) on TriviaQA, IFEval, and CELLO using the base reward model ArmoRM and \ourmethod to search. ``+Oracle'' denotes using the oracle setting of \ourmethod as mentioned in \cref{sec:exp_analysis}.}
    \label{fig:enter-label}
\end{figure*}


\begin{table*}
    \centering
    \small
    \begin{tabular}{lccccccc}
    \toprule
    DPO Training Data & MMLU & MMLU-Pro & TriviaQA & TruthfulQA & IFEval & CELLO & MT-Bench\\
    \midrule
    -- & $58.9$ & $28.8$ & $54.8$ & $39.5$ & $43.3$ & $51.5$ & $5.2$ \\
    \midrule
    Original UF & $58.7$ & $29.3$ & $54.0$ & $42.0$ & $56.8$ & $62.0$  & $6.0$ \\
    ArmoRM-UF & $58.1$ & $29.9$ & $52.5$ & $45.0$ & $58.6$ & $60.8$ & $6.0$ \\
    \ourmethodllama-UF & $59.1$ & $30.5$ & $55.1$ & $44.1$ & $\mathbf{59.4}$ & $60.1$ & $5.8$ \\
    \midrule
    ArmoRM-OP & $58.4$ & $30.4$ & $51.6$ & $44.4$ & $52.7$ & $58.1$ & $6.0$ \\
    \ourmethodllama-OP & $\mathbf{59.5}$ & $\mathbf{31.3}$ & $\mathbf{55.3}$ & $\mathbf{48.5}$ & $58.2$ & $\mathbf{65.7}$ & $\mathbf{6.1}$ \\
    \bottomrule
    \end{tabular}
    \caption{Experimental results (\%) of LLMs trained with DPO on different training data. ``ArmoRM-UF'' denotes using ArmoRM to construct preference pairs from UltraFeedback. ``UF'' and ``OP'' are short for UltraFeedback and on-policy data, respectively. ``Original UF'' refers to using the original GPT-4 annotated preference pairs from UltraFeedback to train the LLM. ``--'' denotes the original LLM zephyr-7b-sft-full without further DPO training.}
    \label{tab:dpo_results}
\end{table*}



















\section{Ablation Study}
\subsection{Prompt and Test Analysis}
\textbf{Prompt Coverage Analysis}\ \ \ To analyze the domain coverage of our generated coding problem set, in Figure~\ref{fig:embedding-plot}, we visualize the embedding distributions of our synthetic problem descriptions with those from established coding benchmarks, including MBPP, APPS, and LiveCodeBench.
Specifically, we use Gecko, a compact and versatile text embedding model distilled from LLMs~\cite{lee2024gecko} for obtaining the sentence embeddings.
% By mapping these embeddings into a common feature space, we can effectively compare the semantic and structural characteristics of our generated problems with those from existing datasets.
The embedding distribution plot reveals that our synthetic problem set exhibits a broad and diverse coverage, encompassing a wide range of topics, difficulty levels, and programming paradigms present in the compared benchmarks.

\begin{figure}[t]
    \centering
    \includegraphics[width=0.9\columnwidth]{figs/embedding-plot.pdf}
    \caption{Prompt distribution comparison with other standard coding benchmarks. We use Principal Component Analysis (PCA) for embedding dimension reduction.}
    \label{fig:embedding-plot}
\end{figure}

\textbf{Progress Analysis per Iteration}\ \ \ To evaluate the iterative advancements of our model, we present a case study on test generation across iterations in Table~\ref{tab:case-study}. The results illustrate that \textsc{Sol-Ver} progressively refines its test generation for the same set of coding problems, thereby enhancing the quality of the synthetic data. These enhancements include the generation of more accurate expected values and better adherence to required format specifications.
Additionally, we monitor the execution results at each iteration and display the distribution of pass and error rates in Figure~\ref{fig:error-dist}. As shown, the pass rate increases with each iteration, primarily due to a reduction in assertion errors, indicating an improvement in the accuracy of the predicted expected outputs.

\begin{figure}
    \centering
    \includegraphics[width=\columnwidth]{figs/error-dist.pdf}
    \caption{Pass or error distribution of synthetic data generated at each iteration.}
    \label{fig:error-dist}
\end{figure}


\begin{table}[h]
\caption{Code generation performance for different settings of the scoring function for selecting DPO pairs. Results indicate better results are obtained with less, but higher quality, data ($\epsilon>0$).}
\begin{center}
\resizebox{\columnwidth}{!}{
\begin{tabular}{l|cccc}
\toprule
$\epsilon$    & \multicolumn{1}{c}{$> 0$} & \multicolumn{1}{c}{$> 0.5$} & \multicolumn{1}{c}{$ > 0.75$} & \multicolumn{1}{c}{\textsc{Sol-Ver}}\\
\midrule
Data Size &25,525 & 20,457 & 13,158 & 12,525\\
MBPP          & \multicolumn{1}{c}{36.00} & \multicolumn{1}{c}{37.00}   & \multicolumn{1}{c}{\textbf{40.80}}  &   \multicolumn{1}{c}{\textbf{40.80}} \\
LiveCodeBench & \multicolumn{1}{c}{22.41} & \multicolumn{1}{c}{25.75}   & \multicolumn{1}{c}{\textbf{26.18}}   & \multicolumn{1}{c}{25.97}  \\
\hline \hline
$\text{Score}(C^-, T)$ & Random & Lowest & Median & \textsc{Sol-Ver}\\
\midrule
MBPP                   & \textbf{40.80}  & 39.00  & 38.60  & \textbf{40.80}\\
LiveCodeBench          & \textbf{26.18}  & \textbf{26.18}  & 25.98  & 25.97\\
\bottomrule
\end{tabular}}
\end{center}
\label{tab:scoring-func}
\end{table}

\subsection{Discussion on the Scoring Function}
In Section~\ref{sec:problem-formulation}, we define the scoring function for selecting the chosen / rejected solution-test pair as a binary function.
In our experiments for Iter 1, we find that only 45\% examples can get agreed solution-test pairs, where there is at least one code generated by the model will pass all the generated test. Consequently, the total number of preference tuning pairs is limited by the number of these selected examples.

To explore whether we can utilize the rest of the data where the pass rate is not necessarily 100\%, but is still high enough to rely on, we conduct a series of experiments to discuss the potential of using a \textit{soft pass rate} for selecting preference pairs. Specifically, we change the chosen / rejected pair as $(C^-, C^+)$, where for the same test suite $\mathbf{T}$, $\text{Score}(C^+, \mathbf{T}) > \text{Score}(C^-, \mathbf{T})$, and $\text{Score}(C^+, \mathbf{T}) \geq \epsilon$, where $\epsilon$ is a threshold to determine above which pass rate the test set is relatively reliable. For simplicity, we discuss three cases for $\epsilon$: (1) $\epsilon > 0$ (can be any number); (2) $\epsilon > 0.5$; (3) $\epsilon > 0.75$. For $\text{Score}(C^-, \mathbf{T})$, we also consider three cases: (1) $\text{Score}(C^-, \mathbf{T})$ is a random score; (2) $\text{Score}(C^-, \mathbf{T})$ is the lowest score among all sampling candidate; (3) $\text{Score}(C^-, \mathbf{T})$ is the median score from the lowest to $\text{Score}(C^+, \mathbf{T})$.

In Table~\ref{tab:scoring-func}, we first present various $\epsilon$ settings for assigning $\text{Score}(C^-, \mathbf{T})$ as a random score. After identifying the optimal setting from our results ($\epsilon > 0.75$), we examine the impact on $\text{Score}(C^-, \mathbf{T})$. The findings reveal that randomly selecting the threshold for a high-quality test set significantly degrades performance, despite an increase in data size. This underscores that the quality of synthetic data is more critical than its quantity. Regarding $\text{Score}(C^-, \mathbf{T})$, we found that its impact is less sensitive compared to $\epsilon$. Considering the overall performance, we have chosen to retain the original settings for \textsc{Sol-Ver} to maintain simplicity. We also encourage future research to explore the potential of scoring functions design in more diverse ways.

\section{Conclusion}

In this work, we introduced {\sc Sol-Ver}, a novel self-play solver-verifier framework that iteratively enhances both code and unit test generation capabilities of large language models (LLMs). By enabling the model to act simultaneously as a solver and a verifier, our approach effectively leverages synthetic data to bridge the critical gap between code generation and unit test generation. Experimental results on the Llama 3.1 8B model demonstrate significant improvements in generating both high-quality code solutions and corresponding unit tests without the necessity of human-annotated data or reliance on larger teacher models. {\sc Sol-Ver} not only advances the state-of-the-art in automated code and test generation but also offers a scalable solution that can be adapted to various programming tasks.

There are several avenues for further enhancement of {\sc Sol-Ver}. Future work could explore more controlled input generation techniques to optimize the relevance and quality of synthetic data. Additionally, extending the framework to accommodate more complex coding scenarios beyond function-level generation can broaden its applicability.
Further optimization of the iterative training process could also reduce computational overhead, making the approach more efficient.
Finally, evaluating the framework on larger and more diverse models will help to determine its generalizability and effectiveness across different architectures.

% \newpage
% \section*{Impact Statement}
% This paper presents work aimed at advancing the field of machine learning by enhancing large language models’ capabilities in code and test generation through the {\sc Sol-Ver} framework. By improving the verification ability of LLMs, our approach enables the development of more accurate and reliable software solutions, thereby increasing productivity and reducing the likelihood of errors across diverse coding scenarios. The iterative self-play process, which leverages the model itself to verify and enhance its outputs, eliminates the need for extensive human annotations or larger teacher models, making the methodology both efficient and scalable. This advancement has the potential to democratize access to sophisticated coding tools, empowering a broader range of developers and industries to innovate and create robust applications. Additionally, enhanced automated coding capabilities can accelerate software development cycles, foster educational initiatives in programming, and support the creation of more complex and reliable software systems. 


%As a result, the {\sc Sol-Ver} framework can further , enabling more effective and intelligent software engineering practices that can drive future advancements in various sectors.
%{\sc Sol-Ver} are built based on open-source models and datasets (all available online). We do not anticipate any major ethical concerns.

% Authors are required to include a statement of the potential broader impact of their work, including its ethical aspects and future societal consequences. This statement should be in a separate section at the end of the paper (co-located with Acknowledgements, before References), and does not count toward the paper page limit. In many cases, where the ethical impacts and expected societal implications are those that are well established when advancing the field of Machine Learning, substantial discussion is not required, and a simple statement such as: 

% “This paper presents work whose goal is to advance the field of Machine Learning. There are many potential societal consequences of our work, none which we feel must be specifically highlighted here.”

% The above statement can be used verbatim in such cases, but we encourage authors to think about whether there is content which does warrant further discussion, as this statement will be apparent if the paper is later flagged for ethics review.

% In particular, Reviewers and ACs may flag submissions for ethics review. Flagged submissions will be sent to an ethics review committee for comments. Ethics reviewers do not have the authority to reject papers, but in extreme cases papers may be rejected by the program chairs on ethical grounds, regardless of scientific quality or contribution.

% Note: Impact Statements are optional for papers submitted to the position paper track this year.

% \section{Electronic Submission}
% \label{submission}

% Submission to ICML 2025 will be entirely electronic, via a web site
% (not email). Information about the submission process and \LaTeX\ templates
% are available on the conference web site at:
% \begin{center}
% \textbf{\texttt{http://icml.cc/}}
% \end{center}

% The guidelines below will be enforced for initial submissions and
% camera-ready copies. Here is a brief summary:
% \begin{itemize}
% \item Submissions must be in PDF\@. 
% \item If your paper has appendices, submit the appendix together with the main body and the references \textbf{as a single file}. Reviewers will not look for appendices as a separate PDF file. So if you submit such an extra file, reviewers will very likely miss it.
% \item Page limit: The main body of the paper has to be fitted to 8 pages, excluding references and appendices; the space for the latter two is not limited in pages, but the total file size may not exceed 10MB. For the final version of the paper, authors can add one extra page to the main body.
% \item \textbf{Do not include author information or acknowledgements} in your
%     initial submission.
% \item Your paper should be in \textbf{10 point Times font}.
% \item Make sure your PDF file only uses Type-1 fonts.
% \item Place figure captions \emph{under} the figure (and omit titles from inside
%     the graphic file itself). Place table captions \emph{over} the table.
% \item References must include page numbers whenever possible and be as complete
%     as possible. Place multiple citations in chronological order.
% \item Do not alter the style template; in particular, do not compress the paper
%     format by reducing the vertical spaces.
% \item Keep your abstract brief and self-contained, one paragraph and roughly
%     4--6 sentences. Gross violations will require correction at the
%     camera-ready phase. The title should have content words capitalized.
% \end{itemize}

% \subsection{Submitting Papers}

% \textbf{Anonymous Submission:} ICML uses double-blind review: no identifying
% author information may appear on the title page or in the paper
% itself. \cref{author info} gives further details.

% \medskip

% Authors must provide their manuscripts in \textbf{PDF} format.
% Furthermore, please make sure that files contain only embedded Type-1 fonts
% (e.g.,~using the program \texttt{pdffonts} in linux or using
% File/DocumentProperties/Fonts in Acrobat). Other fonts (like Type-3)
% might come from graphics files imported into the document.

% Authors using \textbf{Word} must convert their document to PDF\@. Most
% of the latest versions of Word have the facility to do this
% automatically. Submissions will not be accepted in Word format or any
% format other than PDF\@. Really. We're not joking. Don't send Word.

% Those who use \textbf{\LaTeX} should avoid including Type-3 fonts.
% Those using \texttt{latex} and \texttt{dvips} may need the following
% two commands:

% {\footnotesize
% \begin{verbatim}
% dvips -Ppdf -tletter -G0 -o paper.ps paper.dvi
% ps2pdf paper.ps
% \end{verbatim}}
% It is a zero following the ``-G'', which tells dvips to use
% the config.pdf file. Newer \TeX\ distributions don't always need this
% option.

% Using \texttt{pdflatex} rather than \texttt{latex}, often gives better
% results. This program avoids the Type-3 font problem, and supports more
% advanced features in the \texttt{microtype} package.

% \textbf{Graphics files} should be a reasonable size, and included from
% an appropriate format. Use vector formats (.eps/.pdf) for plots,
% lossless bitmap formats (.png) for raster graphics with sharp lines, and
% jpeg for photo-like images.

% The style file uses the \texttt{hyperref} package to make clickable
% links in documents. If this causes problems for you, add
% \texttt{nohyperref} as one of the options to the \texttt{icml2025}
% usepackage statement.


% \subsection{Submitting Final Camera-Ready Copy}

% The final versions of papers accepted for publication should follow the
% same format and naming convention as initial submissions, except that
% author information (names and affiliations) should be given. See
% \cref{final author} for formatting instructions.

% The footnote, ``Preliminary work. Under review by the International
% Conference on Machine Learning (ICML). Do not distribute.'' must be
% modified to ``\textit{Proceedings of the
% $\mathit{42}^{nd}$ International Conference on Machine Learning},
% Vancouver, Canada, PMLR 267, 2025.
% Copyright 2025 by the author(s).''

% For those using the \textbf{\LaTeX} style file, this change (and others) is
% handled automatically by simply changing
% $\mathtt{\backslash usepackage\{icml2025\}}$ to
% $$\mathtt{\backslash usepackage[accepted]\{icml2025\}}$$
% Authors using \textbf{Word} must edit the
% footnote on the first page of the document themselves.

% Camera-ready copies should have the title of the paper as running head
% on each page except the first one. The running title consists of a
% single line centered above a horizontal rule which is $1$~point thick.
% The running head should be centered, bold and in $9$~point type. The
% rule should be $10$~points above the main text. For those using the
% \textbf{\LaTeX} style file, the original title is automatically set as running
% head using the \texttt{fancyhdr} package which is included in the ICML
% 2025 style file package. In case that the original title exceeds the
% size restrictions, a shorter form can be supplied by using

% \verb|\icmltitlerunning{...}|

% just before $\mathtt{\backslash begin\{document\}}$.
% Authors using \textbf{Word} must edit the header of the document themselves.

% \section{Format of the Paper}

% All submissions must follow the specified format.

% \subsection{Dimensions}




% The text of the paper should be formatted in two columns, with an
% overall width of 6.75~inches, height of 9.0~inches, and 0.25~inches
% between the columns. The left margin should be 0.75~inches and the top
% margin 1.0~inch (2.54~cm). The right and bottom margins will depend on
% whether you print on US letter or A4 paper, but all final versions
% must be produced for US letter size.
% Do not write anything on the margins.

% The paper body should be set in 10~point type with a vertical spacing
% of 11~points. Please use Times typeface throughout the text.

% \subsection{Title}

% The paper title should be set in 14~point bold type and centered
% between two horizontal rules that are 1~point thick, with 1.0~inch
% between the top rule and the top edge of the page. Capitalize the
% first letter of content words and put the rest of the title in lower
% case.

% \subsection{Author Information for Submission}
% \label{author info}

% ICML uses double-blind review, so author information must not appear. If
% you are using \LaTeX\/ and the \texttt{icml2025.sty} file, use
% \verb+\icmlauthor{...}+ to specify authors and \verb+\icmlaffiliation{...}+ to specify affiliations. (Read the TeX code used to produce this document for an example usage.) The author information
% will not be printed unless \texttt{accepted} is passed as an argument to the
% style file.
% Submissions that include the author information will not
% be reviewed.

% \subsubsection{Self-Citations}

% If you are citing published papers for which you are an author, refer
% to yourself in the third person. In particular, do not use phrases
% that reveal your identity (e.g., ``in previous work \cite{langley00}, we
% have shown \ldots'').

% Do not anonymize citations in the reference section. The only exception are manuscripts that are
% not yet published (e.g., under submission). If you choose to refer to
% such unpublished manuscripts \cite{anonymous}, anonymized copies have
% to be submitted
% as Supplementary Material via OpenReview\@. However, keep in mind that an ICML
% paper should be self contained and should contain sufficient detail
% for the reviewers to evaluate the work. In particular, reviewers are
% not required to look at the Supplementary Material when writing their
% review (they are not required to look at more than the first $8$ pages of the submitted document).

% \subsubsection{Camera-Ready Author Information}
% \label{final author}

% If a paper is accepted, a final camera-ready copy must be prepared.
% %
% For camera-ready papers, author information should start 0.3~inches below the
% bottom rule surrounding the title. The authors' names should appear in 10~point
% bold type, in a row, separated by white space, and centered. Author names should
% not be broken across lines. Unbolded superscripted numbers, starting 1, should
% be used to refer to affiliations.

% Affiliations should be numbered in the order of appearance. A single footnote
% block of text should be used to list all the affiliations. (Academic
% affiliations should list Department, University, City, State/Region, Country.
% Similarly for industrial affiliations.)

% Each distinct affiliations should be listed once. If an author has multiple
% affiliations, multiple superscripts should be placed after the name, separated
% by thin spaces. If the authors would like to highlight equal contribution by
% multiple first authors, those authors should have an asterisk placed after their
% name in superscript, and the term ``\textsuperscript{*}Equal contribution"
% should be placed in the footnote block ahead of the list of affiliations. A
% list of corresponding authors and their emails (in the format Full Name
% \textless{}email@domain.com\textgreater{}) can follow the list of affiliations.
% Ideally only one or two names should be listed.

% A sample file with author names is included in the ICML2025 style file
% package. Turn on the \texttt{[accepted]} option to the stylefile to
% see the names rendered. All of the guidelines above are implemented
% by the \LaTeX\ style file.

% \subsection{Abstract}

% The paper abstract should begin in the left column, 0.4~inches below the final
% address. The heading `Abstract' should be centered, bold, and in 11~point type.
% The abstract body should use 10~point type, with a vertical spacing of
% 11~points, and should be indented 0.25~inches more than normal on left-hand and
% right-hand margins. Insert 0.4~inches of blank space after the body. Keep your
% abstract brief and self-contained, limiting it to one paragraph and roughly 4--6
% sentences. Gross violations will require correction at the camera-ready phase.

% \subsection{Partitioning the Text}

% You should organize your paper into sections and paragraphs to help
% readers place a structure on the material and understand its
% contributions.

% \subsubsection{Sections and Subsections}

% Section headings should be numbered, flush left, and set in 11~pt bold
% type with the content words capitalized. Leave 0.25~inches of space
% before the heading and 0.15~inches after the heading.

% Similarly, subsection headings should be numbered, flush left, and set
% in 10~pt bold type with the content words capitalized. Leave
% 0.2~inches of space before the heading and 0.13~inches afterward.

% Finally, subsubsection headings should be numbered, flush left, and
% set in 10~pt small caps with the content words capitalized. Leave
% 0.18~inches of space before the heading and 0.1~inches after the
% heading.

% Please use no more than three levels of headings.

% \subsubsection{Paragraphs and Footnotes}

% Within each section or subsection, you should further partition the
% paper into paragraphs. Do not indent the first line of a given
% paragraph, but insert a blank line between succeeding ones.

% You can use footnotes\footnote{Footnotes
% should be complete sentences.} to provide readers with additional
% information about a topic without interrupting the flow of the paper.
% Indicate footnotes with a number in the text where the point is most
% relevant. Place the footnote in 9~point type at the bottom of the
% column in which it appears. Precede the first footnote in a column
% with a horizontal rule of 0.8~inches.\footnote{Multiple footnotes can
% appear in each column, in the same order as they appear in the text,
% but spread them across columns and pages if possible.}

% \begin{figure}[ht]
% \vskip 0.2in
% \begin{center}
% \centerline{\includegraphics[width=\columnwidth]{icml_numpapers}}
% \caption{Historical locations and number of accepted papers for International
% Machine Learning Conferences (ICML 1993 -- ICML 2008) and International
% Workshops on Machine Learning (ML 1988 -- ML 1992). At the time this figure was
% produced, the number of accepted papers for ICML 2008 was unknown and instead
% estimated.}
% \label{icml-historical}
% \end{center}
% \vskip -0.2in
% \end{figure}

% \subsection{Figures}

% You may want to include figures in the paper to illustrate
% your approach and results. Such artwork should be centered,
% legible, and separated from the text. Lines should be dark and at
% least 0.5~points thick for purposes of reproduction, and text should
% not appear on a gray background.

% Label all distinct components of each figure. If the figure takes the
% form of a graph, then give a name for each axis and include a legend
% that briefly describes each curve. Do not include a title inside the
% figure; instead, the caption should serve this function.

% Number figures sequentially, placing the figure number and caption
% \emph{after} the graphics, with at least 0.1~inches of space before
% the caption and 0.1~inches after it, as in
% \cref{icml-historical}. The figure caption should be set in
% 9~point type and centered unless it runs two or more lines, in which
% case it should be flush left. You may float figures to the top or
% bottom of a column, and you may set wide figures across both columns
% (use the environment \texttt{figure*} in \LaTeX). Always place
% two-column figures at the top or bottom of the page.

% \subsection{Algorithms}

% If you are using \LaTeX, please use the ``algorithm'' and ``algorithmic''
% environments to format pseudocode. These require
% the corresponding stylefiles, algorithm.sty and
% algorithmic.sty, which are supplied with this package.
% \cref{alg:example} shows an example.

% \begin{algorithm}[tb]
%    \caption{Bubble Sort}
%    \label{alg:example}
% \begin{algorithmic}
%    \STATE {\bfseries Input:} data $x_i$, size $m$
%    \REPEAT
%    \STATE Initialize $noChange = true$.
%    \FOR{$i=1$ {\bfseries to} $m-1$}
%    \IF{$x_i > x_{i+1}$}
%    \STATE Swap $x_i$ and $x_{i+1}$
%    \STATE $noChange = false$
%    \ENDIF
%    \ENDFOR
%    \UNTIL{$noChange$ is $true$}
% \end{algorithmic}
% \end{algorithm}

% \subsection{Tables}

% You may also want to include tables that summarize material. Like
% figures, these should be centered, legible, and numbered consecutively.
% However, place the title \emph{above} the table with at least
% 0.1~inches of space before the title and the same after it, as in
% \cref{sample-table}. The table title should be set in 9~point
% type and centered unless it runs two or more lines, in which case it
% should be flush left.

% % Note use of \abovespace and \belowspace to get reasonable spacing
% % above and below tabular lines.

% \begin{table}[t]
% \caption{Classification accuracies for naive Bayes and flexible
% Bayes on various data sets.}
% \label{sample-table}
% \vskip 0.15in
% \begin{center}
% \begin{small}
% \begin{sc}
% \begin{tabular}{lcccr}
% \toprule
% Data set & Naive & Flexible & Better? \\
% \midrule
% Breast    & 95.9$\pm$ 0.2& 96.7$\pm$ 0.2& $\surd$ \\
% Cleveland & 83.3$\pm$ 0.6& 80.0$\pm$ 0.6& $\times$\\
% Glass2    & 61.9$\pm$ 1.4& 83.8$\pm$ 0.7& $\surd$ \\
% Credit    & 74.8$\pm$ 0.5& 78.3$\pm$ 0.6&         \\
% Horse     & 73.3$\pm$ 0.9& 69.7$\pm$ 1.0& $\times$\\
% Meta      & 67.1$\pm$ 0.6& 76.5$\pm$ 0.5& $\surd$ \\
% Pima      & 75.1$\pm$ 0.6& 73.9$\pm$ 0.5&         \\
% Vehicle   & 44.9$\pm$ 0.6& 61.5$\pm$ 0.4& $\surd$ \\
% \bottomrule
% \end{tabular}
% \end{sc}
% \end{small}
% \end{center}
% \vskip -0.1in
% \end{table}

% Tables contain textual material, whereas figures contain graphical material.
% Specify the contents of each row and column in the table's topmost
% row. Again, you may float tables to a column's top or bottom, and set
% wide tables across both columns. Place two-column tables at the
% top or bottom of the page.

% \subsection{Theorems and such}
% The preferred way is to number definitions, propositions, lemmas, etc. consecutively, within sections, as shown below.
% \begin{definition}
% \label{def:inj}
% A function $f:X \to Y$ is injective if for any $x,y\in X$ different, $f(x)\ne f(y)$.
% \end{definition}
% Using \cref{def:inj} we immediate get the following result:
% \begin{proposition}
% If $f$ is injective mapping a set $X$ to another set $Y$, 
% the cardinality of $Y$ is at least as large as that of $X$
% \end{proposition}
% \begin{proof} 
% Left as an exercise to the reader. 
% \end{proof}
% \cref{lem:usefullemma} stated next will prove to be useful.
% \begin{lemma}
% \label{lem:usefullemma}
% For any $f:X \to Y$ and $g:Y\to Z$ injective functions, $f \circ g$ is injective.
% \end{lemma}
% \begin{theorem}
% \label{thm:bigtheorem}
% If $f:X\to Y$ is bijective, the cardinality of $X$ and $Y$ are the same.
% \end{theorem}
% An easy corollary of \cref{thm:bigtheorem} is the following:
% \begin{corollary}
% If $f:X\to Y$ is bijective, 
% the cardinality of $X$ is at least as large as that of $Y$.
% \end{corollary}
% \begin{assumption}
% The set $X$ is finite.
% \label{ass:xfinite}
% \end{assumption}
% \begin{remark}
% According to some, it is only the finite case (cf. \cref{ass:xfinite}) that is interesting.
% \end{remark}
% %restatable

% \subsection{Citations and References}

% Please use APA reference format regardless of your formatter
% or word processor. If you rely on the \LaTeX\/ bibliographic
% facility, use \texttt{natbib.sty} and \texttt{icml2025.bst}
% included in the style-file package to obtain this format.

% Citations within the text should include the authors' last names and
% year. If the authors' names are included in the sentence, place only
% the year in parentheses, for example when referencing Arthur Samuel's
% pioneering work \yrcite{Samuel59}. Otherwise place the entire
% reference in parentheses with the authors and year separated by a
% comma \cite{Samuel59}. List multiple references separated by
% semicolons \cite{kearns89,Samuel59,mitchell80}. Use the `et~al.'
% construct only for citations with three or more authors or after
% listing all authors to a publication in an earlier reference \cite{MachineLearningI}.

% Authors should cite their own work in the third person
% in the initial version of their paper submitted for blind review.
% Please refer to \cref{author info} for detailed instructions on how to
% cite your own papers.

% Use an unnumbered first-level section heading for the references, and use a
% hanging indent style, with the first line of the reference flush against the
% left margin and subsequent lines indented by 10 points. The references at the
% end of this document give examples for journal articles \cite{Samuel59},
% conference publications \cite{langley00}, book chapters \cite{Newell81}, books
% \cite{DudaHart2nd}, edited volumes \cite{MachineLearningI}, technical reports
% \cite{mitchell80}, and dissertations \cite{kearns89}.

% Alphabetize references by the surnames of the first authors, with
% single author entries preceding multiple author entries. Order
% references for the same authors by year of publication, with the
% earliest first. Make sure that each reference includes all relevant
% information (e.g., page numbers).

% Please put some effort into making references complete, presentable, and
% consistent, e.g. use the actual current name of authors.
% If using bibtex, please protect capital letters of names and
% abbreviations in titles, for example, use \{B\}ayesian or \{L\}ipschitz
% in your .bib file.

% \section*{Accessibility}
% Authors are kindly asked to make their submissions as accessible as possible for everyone including people with disabilities and sensory or neurological differences.
% Tips of how to achieve this and what to pay attention to will be provided on the conference website \url{http://icml.cc/}.

% \section*{Software and Data}

% If a paper is accepted, we strongly encourage the publication of software and data with the
% camera-ready version of the paper whenever appropriate. This can be
% done by including a URL in the camera-ready copy. However, \textbf{do not}
% include URLs that reveal your institution or identity in your
% submission for review. Instead, provide an anonymous URL or upload
% the material as ``Supplementary Material'' into the OpenReview reviewing
% system. Note that reviewers are not required to look at this material
% when writing their review.

% % Acknowledgements should only appear in the accepted version.
% \section*{Acknowledgements}

% \textbf{Do not} include acknowledgements in the initial version of
% the paper submitted for blind review.

% If a paper is accepted, the final camera-ready version can (and
% usually should) include acknowledgements.  Such acknowledgements
% should be placed at the end of the section, in an unnumbered section
% that does not count towards the paper page limit. Typically, this will 
% include thanks to reviewers who gave useful comments, to colleagues 
% who contributed to the ideas, and to funding agencies and corporate 
% sponsors that provided financial support.

% \section*{Impact Statement}

% Authors are \textbf{required} to include a statement of the potential 
% broader impact of their work, including its ethical aspects and future 
% societal consequences. This statement should be in an unnumbered 
% section at the end of the paper (co-located with Acknowledgements -- 
% the two may appear in either order, but both must be before References), 
% and does not count toward the paper page limit. In many cases, where 
% the ethical impacts and expected societal implications are those that 
% are well established when advancing the field of Machine Learning, 
% substantial discussion is not required, and a simple statement such 
% as the following will suffice:

% ``This paper presents work whose goal is to advance the field of 
% Machine Learning. There are many potential societal consequences 
% of our work, none which we feel must be specifically highlighted here.''

% The above statement can be used verbatim in such cases, but we 
% encourage authors to think about whether there is content which does 
% warrant further discussion, as this statement will be apparent if the 
% paper is later flagged for ethics review.


% % In the unusual situation where you want a paper to appear in the
% % references without citing it in the main text, use \nocite
% \nocite{langley00}

\bibliography{example_paper}
\bibliographystyle{icml2025}


%%%%%%%%%%%%%%%%%%%%%%%%%%%%%%%%%%%%%%%%%%%%%%%%%%%%%%%%%%%%%%%%%%%%%%%%%%%%%%%
%%%%%%%%%%%%%%%%%%%%%%%%%%%%%%%%%%%%%%%%%%%%%%%%%%%%%%%%%%%%%%%%%%%%%%%%%%%%%%%
% APPENDIX
%%%%%%%%%%%%%%%%%%%%%%%%%%%%%%%%%%%%%%%%%%%%%%%%%%%%%%%%%%%%%%%%%%%%%%%%%%%%%%%
%%%%%%%%%%%%%%%%%%%%%%%%%%%%%%%%%%%%%%%%%%%%%%%%%%%%%%%%%%%%%%%%%%%%%%%%%%%%%%%
\newpage
\appendix
\onecolumn
\subsection{Lloyd-Max Algorithm}
\label{subsec:Lloyd-Max}
For a given quantization bitwidth $B$ and an operand $\bm{X}$, the Lloyd-Max algorithm finds $2^B$ quantization levels $\{\hat{x}_i\}_{i=1}^{2^B}$ such that quantizing $\bm{X}$ by rounding each scalar in $\bm{X}$ to the nearest quantization level minimizes the quantization MSE. 

The algorithm starts with an initial guess of quantization levels and then iteratively computes quantization thresholds $\{\tau_i\}_{i=1}^{2^B-1}$ and updates quantization levels $\{\hat{x}_i\}_{i=1}^{2^B}$. Specifically, at iteration $n$, thresholds are set to the midpoints of the previous iteration's levels:
\begin{align*}
    \tau_i^{(n)}=\frac{\hat{x}_i^{(n-1)}+\hat{x}_{i+1}^{(n-1)}}2 \text{ for } i=1\ldots 2^B-1
\end{align*}
Subsequently, the quantization levels are re-computed as conditional means of the data regions defined by the new thresholds:
\begin{align*}
    \hat{x}_i^{(n)}=\mathbb{E}\left[ \bm{X} \big| \bm{X}\in [\tau_{i-1}^{(n)},\tau_i^{(n)}] \right] \text{ for } i=1\ldots 2^B
\end{align*}
where to satisfy boundary conditions we have $\tau_0=-\infty$ and $\tau_{2^B}=\infty$. The algorithm iterates the above steps until convergence.

Figure \ref{fig:lm_quant} compares the quantization levels of a $7$-bit floating point (E3M3) quantizer (left) to a $7$-bit Lloyd-Max quantizer (right) when quantizing a layer of weights from the GPT3-126M model at a per-tensor granularity. As shown, the Lloyd-Max quantizer achieves substantially lower quantization MSE. Further, Table \ref{tab:FP7_vs_LM7} shows the superior perplexity achieved by Lloyd-Max quantizers for bitwidths of $7$, $6$ and $5$. The difference between the quantizers is clear at 5 bits, where per-tensor FP quantization incurs a drastic and unacceptable increase in perplexity, while Lloyd-Max quantization incurs a much smaller increase. Nevertheless, we note that even the optimal Lloyd-Max quantizer incurs a notable ($\sim 1.5$) increase in perplexity due to the coarse granularity of quantization. 

\begin{figure}[h]
  \centering
  \includegraphics[width=0.7\linewidth]{sections/figures/LM7_FP7.pdf}
  \caption{\small Quantization levels and the corresponding quantization MSE of Floating Point (left) vs Lloyd-Max (right) Quantizers for a layer of weights in the GPT3-126M model.}
  \label{fig:lm_quant}
\end{figure}

\begin{table}[h]\scriptsize
\begin{center}
\caption{\label{tab:FP7_vs_LM7} \small Comparing perplexity (lower is better) achieved by floating point quantizers and Lloyd-Max quantizers on a GPT3-126M model for the Wikitext-103 dataset.}
\begin{tabular}{c|cc|c}
\hline
 \multirow{2}{*}{\textbf{Bitwidth}} & \multicolumn{2}{|c|}{\textbf{Floating-Point Quantizer}} & \textbf{Lloyd-Max Quantizer} \\
 & Best Format & Wikitext-103 Perplexity & Wikitext-103 Perplexity \\
\hline
7 & E3M3 & 18.32 & 18.27 \\
6 & E3M2 & 19.07 & 18.51 \\
5 & E4M0 & 43.89 & 19.71 \\
\hline
\end{tabular}
\end{center}
\end{table}

\subsection{Proof of Local Optimality of LO-BCQ}
\label{subsec:lobcq_opt_proof}
For a given block $\bm{b}_j$, the quantization MSE during LO-BCQ can be empirically evaluated as $\frac{1}{L_b}\lVert \bm{b}_j- \bm{\hat{b}}_j\rVert^2_2$ where $\bm{\hat{b}}_j$ is computed from equation (\ref{eq:clustered_quantization_definition}) as $C_{f(\bm{b}_j)}(\bm{b}_j)$. Further, for a given block cluster $\mathcal{B}_i$, we compute the quantization MSE as $\frac{1}{|\mathcal{B}_{i}|}\sum_{\bm{b} \in \mathcal{B}_{i}} \frac{1}{L_b}\lVert \bm{b}- C_i^{(n)}(\bm{b})\rVert^2_2$. Therefore, at the end of iteration $n$, we evaluate the overall quantization MSE $J^{(n)}$ for a given operand $\bm{X}$ composed of $N_c$ block clusters as:
\begin{align*}
    \label{eq:mse_iter_n}
    J^{(n)} = \frac{1}{N_c} \sum_{i=1}^{N_c} \frac{1}{|\mathcal{B}_{i}^{(n)}|}\sum_{\bm{v} \in \mathcal{B}_{i}^{(n)}} \frac{1}{L_b}\lVert \bm{b}- B_i^{(n)}(\bm{b})\rVert^2_2
\end{align*}

At the end of iteration $n$, the codebooks are updated from $\mathcal{C}^{(n-1)}$ to $\mathcal{C}^{(n)}$. However, the mapping of a given vector $\bm{b}_j$ to quantizers $\mathcal{C}^{(n)}$ remains as  $f^{(n)}(\bm{b}_j)$. At the next iteration, during the vector clustering step, $f^{(n+1)}(\bm{b}_j)$ finds new mapping of $\bm{b}_j$ to updated codebooks $\mathcal{C}^{(n)}$ such that the quantization MSE over the candidate codebooks is minimized. Therefore, we obtain the following result for $\bm{b}_j$:
\begin{align*}
\frac{1}{L_b}\lVert \bm{b}_j - C_{f^{(n+1)}(\bm{b}_j)}^{(n)}(\bm{b}_j)\rVert^2_2 \le \frac{1}{L_b}\lVert \bm{b}_j - C_{f^{(n)}(\bm{b}_j)}^{(n)}(\bm{b}_j)\rVert^2_2
\end{align*}

That is, quantizing $\bm{b}_j$ at the end of the block clustering step of iteration $n+1$ results in lower quantization MSE compared to quantizing at the end of iteration $n$. Since this is true for all $\bm{b} \in \bm{X}$, we assert the following:
\begin{equation}
\begin{split}
\label{eq:mse_ineq_1}
    \tilde{J}^{(n+1)} &= \frac{1}{N_c} \sum_{i=1}^{N_c} \frac{1}{|\mathcal{B}_{i}^{(n+1)}|}\sum_{\bm{b} \in \mathcal{B}_{i}^{(n+1)}} \frac{1}{L_b}\lVert \bm{b} - C_i^{(n)}(b)\rVert^2_2 \le J^{(n)}
\end{split}
\end{equation}
where $\tilde{J}^{(n+1)}$ is the the quantization MSE after the vector clustering step at iteration $n+1$.

Next, during the codebook update step (\ref{eq:quantizers_update}) at iteration $n+1$, the per-cluster codebooks $\mathcal{C}^{(n)}$ are updated to $\mathcal{C}^{(n+1)}$ by invoking the Lloyd-Max algorithm \citep{Lloyd}. We know that for any given value distribution, the Lloyd-Max algorithm minimizes the quantization MSE. Therefore, for a given vector cluster $\mathcal{B}_i$ we obtain the following result:

\begin{equation}
    \frac{1}{|\mathcal{B}_{i}^{(n+1)}|}\sum_{\bm{b} \in \mathcal{B}_{i}^{(n+1)}} \frac{1}{L_b}\lVert \bm{b}- C_i^{(n+1)}(\bm{b})\rVert^2_2 \le \frac{1}{|\mathcal{B}_{i}^{(n+1)}|}\sum_{\bm{b} \in \mathcal{B}_{i}^{(n+1)}} \frac{1}{L_b}\lVert \bm{b}- C_i^{(n)}(\bm{b})\rVert^2_2
\end{equation}

The above equation states that quantizing the given block cluster $\mathcal{B}_i$ after updating the associated codebook from $C_i^{(n)}$ to $C_i^{(n+1)}$ results in lower quantization MSE. Since this is true for all the block clusters, we derive the following result: 
\begin{equation}
\begin{split}
\label{eq:mse_ineq_2}
     J^{(n+1)} &= \frac{1}{N_c} \sum_{i=1}^{N_c} \frac{1}{|\mathcal{B}_{i}^{(n+1)}|}\sum_{\bm{b} \in \mathcal{B}_{i}^{(n+1)}} \frac{1}{L_b}\lVert \bm{b}- C_i^{(n+1)}(\bm{b})\rVert^2_2  \le \tilde{J}^{(n+1)}   
\end{split}
\end{equation}

Following (\ref{eq:mse_ineq_1}) and (\ref{eq:mse_ineq_2}), we find that the quantization MSE is non-increasing for each iteration, that is, $J^{(1)} \ge J^{(2)} \ge J^{(3)} \ge \ldots \ge J^{(M)}$ where $M$ is the maximum number of iterations. 
%Therefore, we can say that if the algorithm converges, then it must be that it has converged to a local minimum. 
\hfill $\blacksquare$


\begin{figure}
    \begin{center}
    \includegraphics[width=0.5\textwidth]{sections//figures/mse_vs_iter.pdf}
    \end{center}
    \caption{\small NMSE vs iterations during LO-BCQ compared to other block quantization proposals}
    \label{fig:nmse_vs_iter}
\end{figure}

Figure \ref{fig:nmse_vs_iter} shows the empirical convergence of LO-BCQ across several block lengths and number of codebooks. Also, the MSE achieved by LO-BCQ is compared to baselines such as MXFP and VSQ. As shown, LO-BCQ converges to a lower MSE than the baselines. Further, we achieve better convergence for larger number of codebooks ($N_c$) and for a smaller block length ($L_b$), both of which increase the bitwidth of BCQ (see Eq \ref{eq:bitwidth_bcq}).


\subsection{Additional Accuracy Results}
%Table \ref{tab:lobcq_config} lists the various LOBCQ configurations and their corresponding bitwidths.
\begin{table}
\setlength{\tabcolsep}{4.75pt}
\begin{center}
\caption{\label{tab:lobcq_config} Various LO-BCQ configurations and their bitwidths.}
\begin{tabular}{|c||c|c|c|c||c|c||c|} 
\hline
 & \multicolumn{4}{|c||}{$L_b=8$} & \multicolumn{2}{|c||}{$L_b=4$} & $L_b=2$ \\
 \hline
 \backslashbox{$L_A$\kern-1em}{\kern-1em$N_c$} & 2 & 4 & 8 & 16 & 2 & 4 & 2 \\
 \hline
 64 & 4.25 & 4.375 & 4.5 & 4.625 & 4.375 & 4.625 & 4.625\\
 \hline
 32 & 4.375 & 4.5 & 4.625& 4.75 & 4.5 & 4.75 & 4.75 \\
 \hline
 16 & 4.625 & 4.75& 4.875 & 5 & 4.75 & 5 & 5 \\
 \hline
\end{tabular}
\end{center}
\end{table}

%\subsection{Perplexity achieved by various LO-BCQ configurations on Wikitext-103 dataset}

\begin{table} \centering
\begin{tabular}{|c||c|c|c|c||c|c||c|} 
\hline
 $L_b \rightarrow$& \multicolumn{4}{c||}{8} & \multicolumn{2}{c||}{4} & 2\\
 \hline
 \backslashbox{$L_A$\kern-1em}{\kern-1em$N_c$} & 2 & 4 & 8 & 16 & 2 & 4 & 2  \\
 %$N_c \rightarrow$ & 2 & 4 & 8 & 16 & 2 & 4 & 2 \\
 \hline
 \hline
 \multicolumn{8}{c}{GPT3-1.3B (FP32 PPL = 9.98)} \\ 
 \hline
 \hline
 64 & 10.40 & 10.23 & 10.17 & 10.15 &  10.28 & 10.18 & 10.19 \\
 \hline
 32 & 10.25 & 10.20 & 10.15 & 10.12 &  10.23 & 10.17 & 10.17 \\
 \hline
 16 & 10.22 & 10.16 & 10.10 & 10.09 &  10.21 & 10.14 & 10.16 \\
 \hline
  \hline
 \multicolumn{8}{c}{GPT3-8B (FP32 PPL = 7.38)} \\ 
 \hline
 \hline
 64 & 7.61 & 7.52 & 7.48 &  7.47 &  7.55 &  7.49 & 7.50 \\
 \hline
 32 & 7.52 & 7.50 & 7.46 &  7.45 &  7.52 &  7.48 & 7.48  \\
 \hline
 16 & 7.51 & 7.48 & 7.44 &  7.44 &  7.51 &  7.49 & 7.47  \\
 \hline
\end{tabular}
\caption{\label{tab:ppl_gpt3_abalation} Wikitext-103 perplexity across GPT3-1.3B and 8B models.}
\end{table}

\begin{table} \centering
\begin{tabular}{|c||c|c|c|c||} 
\hline
 $L_b \rightarrow$& \multicolumn{4}{c||}{8}\\
 \hline
 \backslashbox{$L_A$\kern-1em}{\kern-1em$N_c$} & 2 & 4 & 8 & 16 \\
 %$N_c \rightarrow$ & 2 & 4 & 8 & 16 & 2 & 4 & 2 \\
 \hline
 \hline
 \multicolumn{5}{|c|}{Llama2-7B (FP32 PPL = 5.06)} \\ 
 \hline
 \hline
 64 & 5.31 & 5.26 & 5.19 & 5.18  \\
 \hline
 32 & 5.23 & 5.25 & 5.18 & 5.15  \\
 \hline
 16 & 5.23 & 5.19 & 5.16 & 5.14  \\
 \hline
 \multicolumn{5}{|c|}{Nemotron4-15B (FP32 PPL = 5.87)} \\ 
 \hline
 \hline
 64  & 6.3 & 6.20 & 6.13 & 6.08  \\
 \hline
 32  & 6.24 & 6.12 & 6.07 & 6.03  \\
 \hline
 16  & 6.12 & 6.14 & 6.04 & 6.02  \\
 \hline
 \multicolumn{5}{|c|}{Nemotron4-340B (FP32 PPL = 3.48)} \\ 
 \hline
 \hline
 64 & 3.67 & 3.62 & 3.60 & 3.59 \\
 \hline
 32 & 3.63 & 3.61 & 3.59 & 3.56 \\
 \hline
 16 & 3.61 & 3.58 & 3.57 & 3.55 \\
 \hline
\end{tabular}
\caption{\label{tab:ppl_llama7B_nemo15B} Wikitext-103 perplexity compared to FP32 baseline in Llama2-7B and Nemotron4-15B, 340B models}
\end{table}

%\subsection{Perplexity achieved by various LO-BCQ configurations on MMLU dataset}


\begin{table} \centering
\begin{tabular}{|c||c|c|c|c||c|c|c|c|} 
\hline
 $L_b \rightarrow$& \multicolumn{4}{c||}{8} & \multicolumn{4}{c||}{8}\\
 \hline
 \backslashbox{$L_A$\kern-1em}{\kern-1em$N_c$} & 2 & 4 & 8 & 16 & 2 & 4 & 8 & 16  \\
 %$N_c \rightarrow$ & 2 & 4 & 8 & 16 & 2 & 4 & 2 \\
 \hline
 \hline
 \multicolumn{5}{|c|}{Llama2-7B (FP32 Accuracy = 45.8\%)} & \multicolumn{4}{|c|}{Llama2-70B (FP32 Accuracy = 69.12\%)} \\ 
 \hline
 \hline
 64 & 43.9 & 43.4 & 43.9 & 44.9 & 68.07 & 68.27 & 68.17 & 68.75 \\
 \hline
 32 & 44.5 & 43.8 & 44.9 & 44.5 & 68.37 & 68.51 & 68.35 & 68.27  \\
 \hline
 16 & 43.9 & 42.7 & 44.9 & 45 & 68.12 & 68.77 & 68.31 & 68.59  \\
 \hline
 \hline
 \multicolumn{5}{|c|}{GPT3-22B (FP32 Accuracy = 38.75\%)} & \multicolumn{4}{|c|}{Nemotron4-15B (FP32 Accuracy = 64.3\%)} \\ 
 \hline
 \hline
 64 & 36.71 & 38.85 & 38.13 & 38.92 & 63.17 & 62.36 & 63.72 & 64.09 \\
 \hline
 32 & 37.95 & 38.69 & 39.45 & 38.34 & 64.05 & 62.30 & 63.8 & 64.33  \\
 \hline
 16 & 38.88 & 38.80 & 38.31 & 38.92 & 63.22 & 63.51 & 63.93 & 64.43  \\
 \hline
\end{tabular}
\caption{\label{tab:mmlu_abalation} Accuracy on MMLU dataset across GPT3-22B, Llama2-7B, 70B and Nemotron4-15B models.}
\end{table}


%\subsection{Perplexity achieved by various LO-BCQ configurations on LM evaluation harness}

\begin{table} \centering
\begin{tabular}{|c||c|c|c|c||c|c|c|c|} 
\hline
 $L_b \rightarrow$& \multicolumn{4}{c||}{8} & \multicolumn{4}{c||}{8}\\
 \hline
 \backslashbox{$L_A$\kern-1em}{\kern-1em$N_c$} & 2 & 4 & 8 & 16 & 2 & 4 & 8 & 16  \\
 %$N_c \rightarrow$ & 2 & 4 & 8 & 16 & 2 & 4 & 2 \\
 \hline
 \hline
 \multicolumn{5}{|c|}{Race (FP32 Accuracy = 37.51\%)} & \multicolumn{4}{|c|}{Boolq (FP32 Accuracy = 64.62\%)} \\ 
 \hline
 \hline
 64 & 36.94 & 37.13 & 36.27 & 37.13 & 63.73 & 62.26 & 63.49 & 63.36 \\
 \hline
 32 & 37.03 & 36.36 & 36.08 & 37.03 & 62.54 & 63.51 & 63.49 & 63.55  \\
 \hline
 16 & 37.03 & 37.03 & 36.46 & 37.03 & 61.1 & 63.79 & 63.58 & 63.33  \\
 \hline
 \hline
 \multicolumn{5}{|c|}{Winogrande (FP32 Accuracy = 58.01\%)} & \multicolumn{4}{|c|}{Piqa (FP32 Accuracy = 74.21\%)} \\ 
 \hline
 \hline
 64 & 58.17 & 57.22 & 57.85 & 58.33 & 73.01 & 73.07 & 73.07 & 72.80 \\
 \hline
 32 & 59.12 & 58.09 & 57.85 & 58.41 & 73.01 & 73.94 & 72.74 & 73.18  \\
 \hline
 16 & 57.93 & 58.88 & 57.93 & 58.56 & 73.94 & 72.80 & 73.01 & 73.94  \\
 \hline
\end{tabular}
\caption{\label{tab:mmlu_abalation} Accuracy on LM evaluation harness tasks on GPT3-1.3B model.}
\end{table}

\begin{table} \centering
\begin{tabular}{|c||c|c|c|c||c|c|c|c|} 
\hline
 $L_b \rightarrow$& \multicolumn{4}{c||}{8} & \multicolumn{4}{c||}{8}\\
 \hline
 \backslashbox{$L_A$\kern-1em}{\kern-1em$N_c$} & 2 & 4 & 8 & 16 & 2 & 4 & 8 & 16  \\
 %$N_c \rightarrow$ & 2 & 4 & 8 & 16 & 2 & 4 & 2 \\
 \hline
 \hline
 \multicolumn{5}{|c|}{Race (FP32 Accuracy = 41.34\%)} & \multicolumn{4}{|c|}{Boolq (FP32 Accuracy = 68.32\%)} \\ 
 \hline
 \hline
 64 & 40.48 & 40.10 & 39.43 & 39.90 & 69.20 & 68.41 & 69.45 & 68.56 \\
 \hline
 32 & 39.52 & 39.52 & 40.77 & 39.62 & 68.32 & 67.43 & 68.17 & 69.30  \\
 \hline
 16 & 39.81 & 39.71 & 39.90 & 40.38 & 68.10 & 66.33 & 69.51 & 69.42  \\
 \hline
 \hline
 \multicolumn{5}{|c|}{Winogrande (FP32 Accuracy = 67.88\%)} & \multicolumn{4}{|c|}{Piqa (FP32 Accuracy = 78.78\%)} \\ 
 \hline
 \hline
 64 & 66.85 & 66.61 & 67.72 & 67.88 & 77.31 & 77.42 & 77.75 & 77.64 \\
 \hline
 32 & 67.25 & 67.72 & 67.72 & 67.00 & 77.31 & 77.04 & 77.80 & 77.37  \\
 \hline
 16 & 68.11 & 68.90 & 67.88 & 67.48 & 77.37 & 78.13 & 78.13 & 77.69  \\
 \hline
\end{tabular}
\caption{\label{tab:mmlu_abalation} Accuracy on LM evaluation harness tasks on GPT3-8B model.}
\end{table}

\begin{table} \centering
\begin{tabular}{|c||c|c|c|c||c|c|c|c|} 
\hline
 $L_b \rightarrow$& \multicolumn{4}{c||}{8} & \multicolumn{4}{c||}{8}\\
 \hline
 \backslashbox{$L_A$\kern-1em}{\kern-1em$N_c$} & 2 & 4 & 8 & 16 & 2 & 4 & 8 & 16  \\
 %$N_c \rightarrow$ & 2 & 4 & 8 & 16 & 2 & 4 & 2 \\
 \hline
 \hline
 \multicolumn{5}{|c|}{Race (FP32 Accuracy = 40.67\%)} & \multicolumn{4}{|c|}{Boolq (FP32 Accuracy = 76.54\%)} \\ 
 \hline
 \hline
 64 & 40.48 & 40.10 & 39.43 & 39.90 & 75.41 & 75.11 & 77.09 & 75.66 \\
 \hline
 32 & 39.52 & 39.52 & 40.77 & 39.62 & 76.02 & 76.02 & 75.96 & 75.35  \\
 \hline
 16 & 39.81 & 39.71 & 39.90 & 40.38 & 75.05 & 73.82 & 75.72 & 76.09  \\
 \hline
 \hline
 \multicolumn{5}{|c|}{Winogrande (FP32 Accuracy = 70.64\%)} & \multicolumn{4}{|c|}{Piqa (FP32 Accuracy = 79.16\%)} \\ 
 \hline
 \hline
 64 & 69.14 & 70.17 & 70.17 & 70.56 & 78.24 & 79.00 & 78.62 & 78.73 \\
 \hline
 32 & 70.96 & 69.69 & 71.27 & 69.30 & 78.56 & 79.49 & 79.16 & 78.89  \\
 \hline
 16 & 71.03 & 69.53 & 69.69 & 70.40 & 78.13 & 79.16 & 79.00 & 79.00  \\
 \hline
\end{tabular}
\caption{\label{tab:mmlu_abalation} Accuracy on LM evaluation harness tasks on GPT3-22B model.}
\end{table}

\begin{table} \centering
\begin{tabular}{|c||c|c|c|c||c|c|c|c|} 
\hline
 $L_b \rightarrow$& \multicolumn{4}{c||}{8} & \multicolumn{4}{c||}{8}\\
 \hline
 \backslashbox{$L_A$\kern-1em}{\kern-1em$N_c$} & 2 & 4 & 8 & 16 & 2 & 4 & 8 & 16  \\
 %$N_c \rightarrow$ & 2 & 4 & 8 & 16 & 2 & 4 & 2 \\
 \hline
 \hline
 \multicolumn{5}{|c|}{Race (FP32 Accuracy = 44.4\%)} & \multicolumn{4}{|c|}{Boolq (FP32 Accuracy = 79.29\%)} \\ 
 \hline
 \hline
 64 & 42.49 & 42.51 & 42.58 & 43.45 & 77.58 & 77.37 & 77.43 & 78.1 \\
 \hline
 32 & 43.35 & 42.49 & 43.64 & 43.73 & 77.86 & 75.32 & 77.28 & 77.86  \\
 \hline
 16 & 44.21 & 44.21 & 43.64 & 42.97 & 78.65 & 77 & 76.94 & 77.98  \\
 \hline
 \hline
 \multicolumn{5}{|c|}{Winogrande (FP32 Accuracy = 69.38\%)} & \multicolumn{4}{|c|}{Piqa (FP32 Accuracy = 78.07\%)} \\ 
 \hline
 \hline
 64 & 68.9 & 68.43 & 69.77 & 68.19 & 77.09 & 76.82 & 77.09 & 77.86 \\
 \hline
 32 & 69.38 & 68.51 & 68.82 & 68.90 & 78.07 & 76.71 & 78.07 & 77.86  \\
 \hline
 16 & 69.53 & 67.09 & 69.38 & 68.90 & 77.37 & 77.8 & 77.91 & 77.69  \\
 \hline
\end{tabular}
\caption{\label{tab:mmlu_abalation} Accuracy on LM evaluation harness tasks on Llama2-7B model.}
\end{table}

\begin{table} \centering
\begin{tabular}{|c||c|c|c|c||c|c|c|c|} 
\hline
 $L_b \rightarrow$& \multicolumn{4}{c||}{8} & \multicolumn{4}{c||}{8}\\
 \hline
 \backslashbox{$L_A$\kern-1em}{\kern-1em$N_c$} & 2 & 4 & 8 & 16 & 2 & 4 & 8 & 16  \\
 %$N_c \rightarrow$ & 2 & 4 & 8 & 16 & 2 & 4 & 2 \\
 \hline
 \hline
 \multicolumn{5}{|c|}{Race (FP32 Accuracy = 48.8\%)} & \multicolumn{4}{|c|}{Boolq (FP32 Accuracy = 85.23\%)} \\ 
 \hline
 \hline
 64 & 49.00 & 49.00 & 49.28 & 48.71 & 82.82 & 84.28 & 84.03 & 84.25 \\
 \hline
 32 & 49.57 & 48.52 & 48.33 & 49.28 & 83.85 & 84.46 & 84.31 & 84.93  \\
 \hline
 16 & 49.85 & 49.09 & 49.28 & 48.99 & 85.11 & 84.46 & 84.61 & 83.94  \\
 \hline
 \hline
 \multicolumn{5}{|c|}{Winogrande (FP32 Accuracy = 79.95\%)} & \multicolumn{4}{|c|}{Piqa (FP32 Accuracy = 81.56\%)} \\ 
 \hline
 \hline
 64 & 78.77 & 78.45 & 78.37 & 79.16 & 81.45 & 80.69 & 81.45 & 81.5 \\
 \hline
 32 & 78.45 & 79.01 & 78.69 & 80.66 & 81.56 & 80.58 & 81.18 & 81.34  \\
 \hline
 16 & 79.95 & 79.56 & 79.79 & 79.72 & 81.28 & 81.66 & 81.28 & 80.96  \\
 \hline
\end{tabular}
\caption{\label{tab:mmlu_abalation} Accuracy on LM evaluation harness tasks on Llama2-70B model.}
\end{table}

%\section{MSE Studies}
%\textcolor{red}{TODO}


\subsection{Number Formats and Quantization Method}
\label{subsec:numFormats_quantMethod}
\subsubsection{Integer Format}
An $n$-bit signed integer (INT) is typically represented with a 2s-complement format \citep{yao2022zeroquant,xiao2023smoothquant,dai2021vsq}, where the most significant bit denotes the sign.

\subsubsection{Floating Point Format}
An $n$-bit signed floating point (FP) number $x$ comprises of a 1-bit sign ($x_{\mathrm{sign}}$), $B_m$-bit mantissa ($x_{\mathrm{mant}}$) and $B_e$-bit exponent ($x_{\mathrm{exp}}$) such that $B_m+B_e=n-1$. The associated constant exponent bias ($E_{\mathrm{bias}}$) is computed as $(2^{{B_e}-1}-1)$. We denote this format as $E_{B_e}M_{B_m}$.  

\subsubsection{Quantization Scheme}
\label{subsec:quant_method}
A quantization scheme dictates how a given unquantized tensor is converted to its quantized representation. We consider FP formats for the purpose of illustration. Given an unquantized tensor $\bm{X}$ and an FP format $E_{B_e}M_{B_m}$, we first, we compute the quantization scale factor $s_X$ that maps the maximum absolute value of $\bm{X}$ to the maximum quantization level of the $E_{B_e}M_{B_m}$ format as follows:
\begin{align}
\label{eq:sf}
    s_X = \frac{\mathrm{max}(|\bm{X}|)}{\mathrm{max}(E_{B_e}M_{B_m})}
\end{align}
In the above equation, $|\cdot|$ denotes the absolute value function.

Next, we scale $\bm{X}$ by $s_X$ and quantize it to $\hat{\bm{X}}$ by rounding it to the nearest quantization level of $E_{B_e}M_{B_m}$ as:

\begin{align}
\label{eq:tensor_quant}
    \hat{\bm{X}} = \text{round-to-nearest}\left(\frac{\bm{X}}{s_X}, E_{B_e}M_{B_m}\right)
\end{align}

We perform dynamic max-scaled quantization \citep{wu2020integer}, where the scale factor $s$ for activations is dynamically computed during runtime.

\subsection{Vector Scaled Quantization}
\begin{wrapfigure}{r}{0.35\linewidth}
  \centering
  \includegraphics[width=\linewidth]{sections/figures/vsquant.jpg}
  \caption{\small Vectorwise decomposition for per-vector scaled quantization (VSQ \citep{dai2021vsq}).}
  \label{fig:vsquant}
\end{wrapfigure}
During VSQ \citep{dai2021vsq}, the operand tensors are decomposed into 1D vectors in a hardware friendly manner as shown in Figure \ref{fig:vsquant}. Since the decomposed tensors are used as operands in matrix multiplications during inference, it is beneficial to perform this decomposition along the reduction dimension of the multiplication. The vectorwise quantization is performed similar to tensorwise quantization described in Equations \ref{eq:sf} and \ref{eq:tensor_quant}, where a scale factor $s_v$ is required for each vector $\bm{v}$ that maps the maximum absolute value of that vector to the maximum quantization level. While smaller vector lengths can lead to larger accuracy gains, the associated memory and computational overheads due to the per-vector scale factors increases. To alleviate these overheads, VSQ \citep{dai2021vsq} proposed a second level quantization of the per-vector scale factors to unsigned integers, while MX \citep{rouhani2023shared} quantizes them to integer powers of 2 (denoted as $2^{INT}$).

\subsubsection{MX Format}
The MX format proposed in \citep{rouhani2023microscaling} introduces the concept of sub-block shifting. For every two scalar elements of $b$-bits each, there is a shared exponent bit. The value of this exponent bit is determined through an empirical analysis that targets minimizing quantization MSE. We note that the FP format $E_{1}M_{b}$ is strictly better than MX from an accuracy perspective since it allocates a dedicated exponent bit to each scalar as opposed to sharing it across two scalars. Therefore, we conservatively bound the accuracy of a $b+2$-bit signed MX format with that of a $E_{1}M_{b}$ format in our comparisons. For instance, we use E1M2 format as a proxy for MX4.

\begin{figure}
    \centering
    \includegraphics[width=1\linewidth]{sections//figures/BlockFormats.pdf}
    \caption{\small Comparing LO-BCQ to MX format.}
    \label{fig:block_formats}
\end{figure}

Figure \ref{fig:block_formats} compares our $4$-bit LO-BCQ block format to MX \citep{rouhani2023microscaling}. As shown, both LO-BCQ and MX decompose a given operand tensor into block arrays and each block array into blocks. Similar to MX, we find that per-block quantization ($L_b < L_A$) leads to better accuracy due to increased flexibility. While MX achieves this through per-block $1$-bit micro-scales, we associate a dedicated codebook to each block through a per-block codebook selector. Further, MX quantizes the per-block array scale-factor to E8M0 format without per-tensor scaling. In contrast during LO-BCQ, we find that per-tensor scaling combined with quantization of per-block array scale-factor to E4M3 format results in superior inference accuracy across models. 


% You can have as much text here as you want. The main body must be at most $8$ pages long.
% For the final version, one more page can be added.
% If you want, you can use an appendix like this one.  

% The $\mathtt{\backslash onecolumn}$ command above can be kept in place if you prefer a one-column appendix, or can be removed if you prefer a two-column appendix.  Apart from this possible change, the style (font size, spacing, margins, page numbering, etc.) should be kept the same as the main body.
%%%%%%%%%%%%%%%%%%%%%%%%%%%%%%%%%%%%%%%%%%%%%%%%%%%%%%%%%%%%%%%%%%%%%%%%%%%%%%%
%%%%%%%%%%%%%%%%%%%%%%%%%%%%%%%%%%%%%%%%%%%%%%%%%%%%%%%%%%%%%%%%%%%%%%%%%%%%%%%


\end{document}


% This document was modified from the file originally made available by
% Pat Langley and Andrea Danyluk for ICML-2K. This version was created
% by Iain Murray in 2018, and modified by Alexandre Bouchard in
% 2019 and 2021 and by Csaba Szepesvari, Gang Niu and Sivan Sabato in 2022.
% Modified again in 2023 and 2024 by Sivan Sabato and Jonathan Scarlett.
% Previous contributors include Dan Roy, Lise Getoor and Tobias
% Scheffer, which was slightly modified from the 2010 version by
% Thorsten Joachims & Johannes Fuernkranz, slightly modified from the
% 2009 version by Kiri Wagstaff and Sam Roweis's 2008 version, which is
% slightly modified from Prasad Tadepalli's 2007 version which is a
% lightly changed version of the previous year's version by Andrew
% Moore, which was in turn edited from those of Kristian Kersting and
% Codrina Lauth. Alex Smola contributed to the algorithmic style files.
