% This must be in the first 5 lines to tell arXiv to use pdfLaTeX, which is strongly recommended.
\pdfoutput=1
% In particular, the hyperref package requires pdfLaTeX in order to break URLs across lines.

\documentclass[11pt]{article}

% Change "review" to "final" to generate the final (sometimes called camera-ready) version.
% Change to "preprint" to generate a non-anonymous version with page numbers.
\usepackage[preprint]{acl}

% Standard package includes
\usepackage{times}
\usepackage{latexsym}

% For proper rendering and hyphenation of words containing Latin characters (including in bib files)
\usepackage[T1]{fontenc}
% For Vietnamese characters
% \usepackage[T5]{fontenc}
% See https://www.latex-project.org/help/documentation/encguide.pdf for other character sets

% This assumes your files are encoded as UTF8
\usepackage[utf8]{inputenc}

% This is not strictly necessary, and may be commented out,
% but it will improve the layout of the manuscript,
% and will typically save some space.
\usepackage{microtype}

% This is also not strictly necessary, and may be commented out.
% However, it will improve the aesthetics of text in
% the typewriter font.
\usepackage{inconsolata}

%Including images in your LaTeX document requires adding
%additional package(s)
\usepackage{graphicx}
\usepackage{amsthm,amsmath,amssymb}
\usepackage{mathrsfs}
\usepackage{booktabs}
\usepackage{algorithm}
% \usepackage[linesnumbered,ruled]{algorithm2e}
\usepackage{algpseudocode}
\usepackage{comment}
\usepackage{multirow}
\usepackage{bbding}
\usepackage{array}
\usepackage{tcolorbox}
\usepackage{colortbl}
\usepackage{xcolor}
\usepackage{enumitem}
\usepackage{threeparttable}
\usepackage{CJKutf8}

\newcommand{\blanksymbolfootnote}[1]{%
  \renewcommand{\thefootnote}{}% Remove footnote number
  \footnote{#1}%
  \setcounter{footnote}{0} % Reset footnote counter
  \renewcommand{\thefootnote}{\arabic{footnote}}% Restore footnote number
}

\newcommand{\scriptnumber}[1]{\textcolor{gray}{\small (#1)}}

\newcommand{\hnote}[1]{{\color{red}{[WHN: #1]}}}

% If the title and author information does not fit in the area allocated, uncomment the following
%
%\setlength\titlebox{<dim>}
%
% and set <dim> to something 5cm or larger.

\title{\textsc{HPSS}: Heuristic Prompting Strategy Search for LLM Evaluators}

\author{Bosi Wen$^{1,\dagger}$ \quad Pei Ke$^{2}$\quad Yufei Sun$^{4}$\quad Cunxiang Wang$^{3,4}$\quad Xiaotao Gu$^{3}$ \\
\textbf{Jinfeng Zhou$^{1}$\quad  Jie Tang$^{4}$\quad Hongning Wang$^{1}$ \quad Minlie Huang$^{1,\ddagger}$} \\
$^1$The Conversational Artificial Intelligence (CoAI) Group, Tsinghua University \\
$^2$University of Electronic Science and Technology of China \quad $^3$Zhipu AI \\
$^4$The Knowledge Engineering Group (KEG), Tsinghua University \\
\tt\small wbs23@mails.tsinghua.edu.cn, aihuang@tsinghua.edu.cn \\}


\begin{document}
\maketitle

\blanksymbolfootnote{$^\dagger$Work done when this author interned at Zhipu AI.}
\blanksymbolfootnote{$^\ddagger$Corresponding author}

\vspace{-6mm}
\begin{abstract}
In practice,  physical spatiotemporal forecasting can suffer from data scarcity, because collecting large-scale data is non-trivial, especially for extreme events. 
Hence, we propose \method{}, a novel probabilistic framework to realize iterative self-training with new self-ensemble strategies, 
achieving better physical consistency and generalization on extreme events. 
Following any base forecasting model, 
we can encode its deterministic outputs into a latent space and retrieve multiple codebook entries to generate probabilistic outputs. 
Then \method{} extends the beam search from discrete spaces to the continuous state spaces in this field.
We can further employ domain-specific metrics (e.g., Critical Success Index for extreme events) to filter out the top-k candidates and develop the new self-ensemble strategy by combining the high-quality candidates. 
The self-ensemble can not only improve the inference quality and robustness but also iteratively augment the training datasets during continuous self-training. 
Consequently, \method{} realizes the exploration of rare but critical phenomena beyond the original dataset. 
Comprehensive experiments on different benchmarks and backbones show that \method{} consistently reduces forecasting MSE (up to 39\%), enhancing extreme events detection and proving its effectiveness in handling data scarcity. Our codes are available at~\url{https://github.com/easylearningscores/BeamVQ}.



% 在气象预报、流体模拟以及基于偏微分方程(PDE)的多物理系统模型中,数据稀缺下的时空预测仍然是一个关键挑战。本文提出了\method{},一个统一的框架,旨在同时解决标注数据有限以及在确保物理一致性的前提下捕捉极端事件的难题。首先,我们训练了一个确定性的基础模型,从小规模数据中学习主要动力学。随后,通过Top-K 向量量化变分自编码器(VQ-VAE)对基础模型的输出进行增强,该模块将确定性预测编码到潜在空间,并检索多个码本条目以生成多样化且物理上合理的重构结果。一个新颖的联合优化过程利用领域特定的指标(例如关键成功指数)引导基础模型向更准确且对极端事件敏感的预测方向优化。在推理阶段,我们采用束搜索策略,维持多个候选轨迹并通过指标感知评分进行迭代剪枝,从而在探索罕见但关键现象与利用最可能的系统轨迹之间实现平衡。在多个气象和流体流动基准数据集上的大量实验表明,\method{}显著提升了预测精度,增强了对极端状态的检测能力,并保持了物理合理性,证明了其在数据稀缺场景下进行时空预测的优越性。

\end{abstract}

\section{Introduction}
The ubiquitous question "How did I do this before ChatGPT?" has become a cultural touch point, highlighting how Large Language Models (LLMs) have gradually permeated people's everyday lives. While initially introduced as general-purpose chatbots, LLMs have been adopted in unexpectedly diverse ways \cite{chkirbene2024applications}. These systems now play multiple roles in decision-making processes and tasks, ranging from information providers to triggers for human self-reflection \cite{kim2022bridging, kmmer2024effects}. This widespread integration has raised the question about how users develop dependencies on and relationships with these AI systems \cite{he2025conversational}.

Previous Human-Computer Interaction (HCI) research has extensively examined domain-specific LLM applications \cite{jin2024teach, liu2024selenite}. These studies have yielded insights into specialized use cases and led to targeted interaction design improvements. However, the broader impact of LLMs on everyday decision-making and tasks remains under-explored. As users increasingly integrate these tools into their daily routines, understanding the tangible impacts of habitual use becomes crucial \cite{kmmer2024effects}. 

Recent studies have attempted to measure the impact of LLM use through quantitative metrics such as task performance and decision accuracy \cite{kim2025fostering}. However, these measurement-based approaches cannot fully capture how people delegate everyday decisions to LLMs or the resulting meta-cognitive effects. Furthermore, everyday decisions encompass a broad spectrum of choices, from routine task management to social interaction planning \cite{eigner2024determinants, dhami2012cct}, making them challenging to examine through purely quantitative and task-specific approaches.

To address this gap, we conducted a qualitative study examining heavy LLM users who regularly rely on these systems for everyday decisions and tasks. Through interviews and analysis, we explored how these users integrate LLMs into their decision-making processes, what types of decisions they choose to delegate, and how this delegation affects their cognitive patterns and decision-making confidence. Our study addressed three research questions: 

\begin{itemize}
\item RQ1: How do heavy LLM users integrate LLMs into their everyday decision-making process?
\item RQ2: What underlying needs do heavy LLM users seek to fulfill through LLM assistance?
\item RQ3: How do heavy LLM users conceptualize and evaluate their relationship with LLMs?
\end{itemize}

Through these research questions, we examine three key aspects of heavy LLM use. RQ1 explores emergent use cases and notable patterns in how users incorporate LLMs into their decision-making processes. RQ2 investigates the fundamental motivations and needs that drive sustained LLM usage. RQ3 examines how users develop their mental models of LLMs and reflect on their extensive interaction with these systems.

\section{Related Work}
\section{Related Work}
\label{sec:rw}

Our work lies at the intersection of three lines of inquiry: research on technologies supporting health services (Section \ref{sec:rw:tech-services}), mental health data collection and storage (Section \ref{sec:rw:data}), and value-based mental healthcare (Section \ref{sec:rw:vbc}).

\subsection{Designing Technologies for Health Services}
\label{sec:rw:tech-services}

In this work, we studied technologies that support value-based care and the delivery of \textit{health services}, which encompass the people, organizations, and technology involved in healthcare delivery \cite{issues_working_1994, sanford_schwartz_chapter_2017}.
These people and organizations include \textit{healthcare providers}, the clinicians or hospital systems that provide treatments or preventive care (the ``services''); as well as \textit{healthcare payers}, the government agencies or private health insurance companies that pay for health services.
We review specific technologies supporting mental health services in Section \ref{sec:rw:data}.
To design technologies for health services, we need to confront more than the hardware or software capabilities of a specific technology, or the effectiveness of interventions that use technologies to improve health outcomes.
We also need to confront sociotechnical factors that affect the implementation and effectiveness of these technologies in real-world care. 
Norman and Stappers categorize sociotechnical factors that affect technology implementation as political, economic, cultural, organizational, and structural \cite{norman_designx_2015}.
Blandford states that, for health services specifically, HCI scholars should \textit{``consider stages (of identifying technical possibilities or early adopters and planning for adoption and diffusion) that are rarely discussed in HCI, but that are necessary to deliver real impact from HCI innovations in healthcare''} \cite{blandford_hci_2019}.
Thus, we were motivated to improve the design of technologies supporting health services by understanding factors that affect their implementation and adoption in care.

Recently, HCI scholars have considered adopting ideas from health services research to improve both the design and effectiveness of health technologies.
Scholars have considered how HCI research can integrate aspects of \textit{implementation science} -- the health services field examining the real-world adoption of evidence-based interventions \cite{lyon_bridging_2023}. 
Interviews with HCI and implementation science researchers uncovered that HCI tends to de-prioritize factors that influence long-term adoption of technologies in their initial design, including the financial incentives that affect adoption, and an understanding of how technologies support providers after implementation \cite{dopp_aligning_2020}.
Moreover, HCI scholars have stated that if technologies are to impact real-world care, HCI researchers should focus on how technology is consumed in care, including developing an understanding of the technical and market incentives to use new tools \cite{colusso_translational_2019}.
Inspired by this work, we considered these aspects of adoption in the initial design of technologies that support value-based mental healthcare.
Specifically, we considered how technologies can support healthcare providers -- practicing clinicians -- including how these technologies can be integrated into clinicians' workflows to support care, and the financial incentives that influence HIT adoption as a part of value-based care.

\subsection{Health Information Technologies for Collecting and Storing Mental Health Data}
\label{sec:rw:data}

HCI, health informatics, and mental health researchers have collaborated to build health information technologies (HITs) for collecting and storing mental health data.
In this work, we focus on three categories of mental health data: clinical data, active data, and passive data.
\textit{Clinical data} can be retrieved from \textit{electronic health records} (EHRs), which record information collected during clinical visits including patient demographics, diagnoses, health and family history, treatments provided, and unstructured clinical notes \cite{birkhead_uses_2015}.
\rev{That said, to protect patient privacy, not all mental health data may be contained within the EHR, and exporting EHR data for VBC may require patient consent \cite{shenoy_safeguarding_2017, leventhal_designing_2015}.}
Clinical data can also be retrieved from \textit{administrative claims databases}, which log diagnostic, treatment, and medication information used to bill healthcare payers \cite{karve_prospective_2009, davis_can_2016}.
Clinics or hospitals may also collect measures of patient satisfaction to understand patients' perceptions of their care \cite{carr-hill_measurement_1992}.

\textit{Active data} require active patient or clinician engagement to be collected, and can be collected with technologies that support digital surveys (eg, smartphones, iPads, computers, \rev{patient portals}) and pen-and-paper questionnaires.
This data include validated self-reported \textit{measures of mental health symptoms}, which quantify symptom presence and/or severity for specific mental health disorders, such as the PHQ-9 for major depressive disorder \cite{kroenke_phq-9_2001}, or the GAD-7 for generalized anxiety disorder \cite{spitzer_brief_2006}.
Active data can also include clinician-rated scales, collected during clinical interviews \cite{andersen_brief_1986}.
Outside of symptoms, self-reported and clinician-rated measures can also quantify \textit{functioning}, as mental health symptoms can impair functioning including cognition, mobility, self-care, and sociality \cite{ustun_measuring_2010}. 
Self-reported measures can also quantify how well patients and their mental health clinicians collaborate towards shared goals, complete tasks, and bond, called \textit{working alliance} \cite{hatcher_development_2006}.
The discussed scales typically quantify persistent symptoms or functional impairment.
Researchers have used everyday devices, such as smartphones, to collect more in-the-moment symptoms via questionnaires called ecological momentary assessments (EMAs) \cite{wang_crosscheck_2016, hsieh_using_2008}.
EMAs can also collect \textit{engagement data}, measuring, for example, medication adherence, or participation in behavioral interventions, such as mindfulness exercises \cite{militello_digital_2022, klasnja_how_2011}.
Active data can be stored in clinical records, like an EHR, but significant investments have not been made to build structured EHR fields for storing active data \cite{pincus_quality_2016}.

In addition to active data, sensors embedded in devices (eg, smartphones, wearables) and online platforms have created opportunities to collect \textit{passive data} -- data collected with little-to-no effort -- on behavior and physiology \cite{nghiem_understanding_2023}.
Passive data can be used to estimate signals related to functioning, including social behaviors, mobility, and sleep \cite{mohr_personal_2017, saeb_relationship_2016, saeb_scalable_2017}, and more recently, researchers have investigated if passive data can measure engagement in therapeutic exercises \cite{evans_using_2024}.
Prior work has also studied whether passive data can estimate symptom severity \cite{adler_measuring_2024, das_swain_semantic_2022, meyerhoff_evaluation_2021, currey_digital_2022}.
The use of passive data in treatment is limited: \rev{while passive data can be collected within EHRs \cite{apple_healthcare_2024, metrohealth_track_2024, pennic_novant_2015}, established clinical guidelines for passive data use in care do not exist, and use is often limited to patients who are motivated to share passive data with their healthcare provider \cite{nghiem_understanding_2023}}.

It is challenging to identify what mental health data are most relevant to HITs in certain contexts, given their variety.
Li et al. proposed a 5-stage model to work through these challenges, specifically in the context of \textit{personal informatics systems}, where users collect data for self-reflection and gaining self-knowledge.
These five stages are preparation, collection, integration, reflection, and action \cite{li_stage-based_2010}.
In this work, we study how HITs can support mental health outcomes data as a part of value-based mental healthcare, inspired by three out of these five stages, specifically \textit{preparation}, understanding what data to collect; \textit{collection}, gathering data; and \textit{action}, how data is used.
We focus on these three stages because they capture existing challenges to design HITs that support VBC, which we review in Section \ref{sec:rw:vbc}.

\subsection{Value-based Mental Healthcare}
\label{sec:rw:vbc}
The World Economic Forum defines \textit{value-based care} (VBC) as a \textit{``patient-centric way to design and manage health systems''} and \textit{``align industry stakeholders around the shared objective of improving health outcomes delivered to patients at a given cost''} \cite{world_economic_forum_value_2017}.
VBC intends to change how healthcare is paid for, away from \textit{fee-for-service} payment models -- where payers reimburse providers for the number of services they provide -- towards paying for services if they deliver ``value'' to the healthcare system \cite{brown_key_2017}.
In practice, VBC is implemented by paying providers a set rate for managing patients' health, sharing savings if specific cost or utilization targets are met, and/or by offering financial incentives for payers and providers based upon \textit{quality measures}, which quantify the ``value'' of care \cite{world_economic_forum_moment_2023, health_care_payment_learning__action_network_alternative_2017}.
These changes shift some of the financial risk of healthcare from payers to providers.
In fee-for-service models, providers continue to be paid as they provide more services.
In VBC, providers may lose money if services cost more than set rates, specific cost/utilization targets are not met, or if care quality suffers \cite{novikov_historical_2018, health_care_payment_learning__action_network_alternative_2017}.

Standardized quality measures guide payers and providers to deliver services that improve health outcomes and reduce cost.
% Quality measures can be derived from administrative claims, EHRs, and patient self-report; are validated for their reliability and validity; importance for improving quality; feasibility to collect; and are certified by country-specific organizations like the NCQA in the United States, or the National Institute for Health and Care Excellence (NICE) in the UK \cite{center_for_medicare__medicaid_services_your_2021, national_institute_for_health_and_care_excellence_nice_2019}
The Donabedian model categorizes quality measures into three areas: (1) \textit{structure} -- the material, human, and organizational resources used in care (eg, the ratio of patients to providers); (2) \textit{process} -- the services provided in care (eg, the percentage of patients receiving immunizations); and (3) \textit{outcomes} -- measuring the effectiveness of care (eg, surgical mortality rates) \cite{donabedian_quality_1988, endeshaw_healthcare_2020,agency_for_healthcare_research_and_quality_types_2015}.
% Each category of measures has strengths and weaknesses.
While structure and process measures are more actionable -- hospital systems can hire more staff, or modify care practices -- their relationship to outcomes can be ambiguous \cite{quentin_measuring_2019}. 
In contrast, outcome measures most clearly represent the goals of care, but can be biased by factors outside of providers' direct control, including co-occurring health conditions that complicate treatment success \cite{lilienfeld_why_2013, quentin_measuring_2019}.
To reduce bias, statisticians apply a \textit{risk-adjustment} to outcome measures, using regression to model expected care outcomes observed in real-world data, based upon variables known to moderate treatment effects \cite{lane-fall_outcomes_2013}.
The quality of provided health services for a specific patient can then be determined based upon whether a patient's health outcomes exceed or underperform expectations.

Mental healthcare has faced specific challenges implementing VBC.
Some of these challenges can be attributed to ambiguity on how to design health information technologies (HITs) that store outcomes data tying provided services to value \cite{world_economic_forum_value_2017}.
\textit{Preparation challenges} revolve around identifying standardized outcome metrics to store in HITs.
Current quality monitoring programs incentivize using symptom scales as standardized care outcomes \cite{morden_health_2022}.
Patients often experience a unique constellation of symptoms that cut across multiple disorders (eg, major depressive disorder and generalized anxiety disorder) \cite{boschloo_network_2015, cramer_comorbidity_2010, barkham_routine_2023}, making it difficult to identify a limited set of symptom scales to track outcomes across patients.
Given these challenges, researchers have proposed using other data types as an alternative to symptom scales within VBC \cite{hobbs_knutson_driving_2021, oslin_provider_2019}. 
For example, scholars and healthcare providers have argued that functional and engagement outcomes may be a promising alternative to symptom scales. 
Engagement is the proximal outcome of many mental health treatments, improved functioning is often more important to patients than symptom reduction, and functional outcomes measure treatment progress across patients living with different mental health symptoms or disorders \cite{stewart_can_2017, tauscher_what_2021, pincus_quality_2016}.

In terms of \textit{data collection}, it is estimated that less than 20\% of mental health clinicians practice measurement-based care (MBC) -- the process of collecting, planning, and adjusting treatment based on outcomes data -- specifically symptom scales \cite{zimmerman_why_2008, fortney_tipping_2017}, despite evidence that MBC improves outcomes \cite{barkham_routine_2023}. 
MBC is usually implemented by having patients routinely self-report symptoms during clinical encounters using validated symptom scales, like the PHQ-9 for depression, or the GAD-7 for anxiety \cite{wray_enhancing_2018}.
Mental health clinicians choose to not practice MBC for many reasons. 
Electronic health records (EHRs) often do not have standardized fields to support symptom data collection, clinicians perceive that symptom scale administration disrupts the therapeutic relationship, and clinicians are often not paid to administer symptom scales \cite{lewis_implementing_2019, desimone_impact_2023, oslin_provider_2019}.
These barriers call for work centering mental health providers in designing HITs that effectively engage providers in outcomes data collection.

\textit{Action} challenges stem from both perceptions of how outcomes data could be used in care, and challenges towards attributing accountability for care.
For example, clinicians are often not trained to use outcomes data in care, and worry that they will be held accountable and penalized if outcomes data reveal that their patients are not improving \cite{lewis_implementing_2019, desimone_impact_2023}.
There are also concerns that outcomes data could be gamed: biased reporting that artificially inflates performance metrics \cite{kilbourne_measuring_2018}.
In addition, it is difficult in mental healthcare to attribute accountability to specific actors (eg, specific providers) in care systems.
Mental healthcare is often ``siloed'' from physical healthcare, though both physical and mental health outcomes are strongly intertwined (eg, individuals living with schizophrenia suffer from chronic physical health conditions) \cite{pincus_quality_2016}.
Thus, existing value-based mental healthcare programs may hold both physical and mental health clinicians \textit{jointly accountable} by sharing cost savings across different types of providers \cite{hobbs_knutson_driving_2021}.

Taken together, this prior work demonstrates challenges designing HITs that support value-based mental healthcare.
Integral to the design of these HITs are mental health clinicians, who are asked to participate in outcomes data collection, which clinicians have found challenging, and will be held financially accountable to the outcomes data HITs store.
Given these challenges, this work centers mental health clinicians' perspectives on how to design HITs that support value-based mental healthcare.
By centering clinicians' perspectives, we looked to gain a deeper understanding of their workflows and incentives to adopt HITs, and integrate this knowledge into the design and development of HITs supporting value-based care. 
The following section details the methodology used in this study.

\section{Methodology}
\section{methodology}
% \section{Interest Unit-based Product Organization}
This chapter introduces the construction of interest units, the redesign of product forms with new interaction interface, and the IU-Boosted CTR prediction model integrating interest unit.

\subsection{Interest Unit-based Product Organization}

\begin{figure}[tbp]
\includegraphics[width=8cm]{gsid_arxiv.png}
\caption{One example of the foundational understanding system generated by the semantic clustering method.}
\label{fig:gsid_tree}
\end{figure}

User behaviors such as browsing, searching, clicking, or purchasing are primarily driven by underlying needs, which can be abstracted into specific interest units. These interest units can range from concrete product instances (e.g., "Iphone15 ProMax 256G") to broad demand categories (e.g., "concert tickets"). 
By predefining interest units and systematically associating relevant products with these constructed interest units, platforms can enhance user engaging experience and demand-matching efficiency.

\subsubsection{\textbf{Construction of Interest Unit}} Despite the diverse needs of Xianyu users, we believe that the core demands can be exhaustively identified to some extent. We have adopted a data-driven, bottom-up analytical framework that constructs interest units from the perspectives of product attributes and user needs, reorganizing the vast array of products on the Xianyu platform. 
% Compared to traditional knowledge construction that relies on manual operations, this method overcomes its limitations and offers greater flexibility.
\\ \textbf{I. Attribute-Driven} \textit{(Based on intrinsic product attributes)} \\
When browsing, users tend to focus on the core attribute information of a product. By combining these core attributes, we can essentially exhaustively identify the core demands users have for a certain category of products. Therefore, based on product attribute information, we have defined two types of user interest units.
\begin{itemize}
    \item SPU Interest Units: Product attributes, such as category and brand, are important on e-commerce platforms. For standard products with comprehensive structured attributes, we aggregate products into SPU Interest units based on the CPV (customer perceived value) information provided by users or identified by algorithms, forming a type of interest unit.
    \item Image Cluster Interest Unit: Product image information is also crucial when users are browsing. For non-standard products where structured attributes are difficult to define, we use product image information to aggregate products with similar appearances into clusters, thereby defining a type of interest unit based on these clusters.
\end{itemize}
\textbf{II. Demand-Aware} \textit{(Balancing product attributes and user needs)}
Not all categories have a one-to-one match between product supply and buyer demand. To balance buyer needs and seller supply during the construction of interest units, we developed a Query-Aware semantic unit generation system called Generative Semantic ID~\footnote{Another systematic effort from industry practice, which isn't the focus of this paper, will be briefly mentioned below. This work will soon be under review.}(GSID), based on open knowledge from large models and combined with the vast interaction data of "query-product" on Xianyu. This system defines Semantic interest units. The Xianyu GSID is a hierarchical tree structure, as shown in the figure~\ref{fig:gsid_tree}. GSID includes three levels, each containing 128 IDs, with the second level space being approximately 16,000 and the third level space being approximately 2.1 million.
Specifically, the Xianyu GSID algorithm uses the encoder-decoder network of the T5 model as the backbone structure. The encoder is a BERT-based vector encoder responsible for extracting product semantic vectors and for the decoder combines the encoder output vector at each decoding step with the previous decoding result to produce the current decoding vector, and then looks up the corresponding theme in the CodeBook to discretize and generate hierarchical semantic IDs.
\subsubsection{\textbf{Redesign of Interaction Interface}}
\begin{figure}[tbp]
\includegraphics[width=8cm]{product_arxiv.png}
\caption{The redesigned product format. Left is stage one style  and right image is stage two style with explanationo on the middle}
\label{fig:new_product}
\end{figure}
% 
Once these basic interest units are constructed, they can not only be incorporated into algorithmic modeling but also further utilized to change the way products are presented on the homepage recommendation interface. As shown in Figure \ref{fig:new_product}, we implemented a series of changes to clearly express user interest units: (1) On the homepage recommendations, we display the theme of the associated interest unit next to the product (left side of Figure \ref{fig:new_product}). (2) On the secondary landing page, products within the same interest unit are arranged together, making it easier for users to select products efficiently (right side of Figure \ref{fig:new_product}, product prototype image), with explanations for each module on the secondary landing page in between.
It is noteworthy that this new product format naturally aligns with the aforementioned two-stage recommendation paradigm, allowing for a more organic combination of algorithm models and product formats, significantly enhancing efficiency. Once the product set for interest units is delineated, we need to generate a front-end title for the interest unit to facilitate user understanding. This information is also displayed at the bottom of the card in homepage and at the top of the secondary page. The title of the interest unit is automatically generated by a large language model, by feeding corresponding descriptive information of N products randomly selected from each interest unit.

\subsection{Interest Unit-based Recommendation}\label{sec:recommendation}
\begin{figure*}[tbp]
    \includegraphics[width=14cm]{model_arxiv.png}
    \caption{An overview of proposed IU-Boosted Network, which consists of three components: (1) the interest unit-level feature for each product, (2) the user's hierarchical IU click sequence to determine their interest unit preference, and (3) the attention mechanism introduced for handling multiple items within the interest unit.}
    \label{fig:model_overview}
    % \vspace{-0.4cm}
\end{figure*}

As shown in Figure \ref{fig:new_product}, the upgrade in the homepage product format has led to significant changes in user navigation paths: user interactions are no longer confined to individual products but can occur across multiple products under the same interest unit. Additionally, behaviors of different users within the same interest unit can be aggregated and accumulated. 

Building upon this, we construct IU-level features to reflect the attributes of each IU and hierarchical IU click sequences using attention mechanism to user interest unit interest. We name this recommendation algorithm, which leverages behavior accumulation on Interest Unit (IU), as \textbf{IU-Boosted Network}. In this section, we will introduce the components of our proposed method in detail.

\subsubsection{\textbf{IU-Level Feature Construction}}
We accumulate behaviors of different users across all products under the same interest unit to construct IU-Level features, serving as foundational attributes of the interest unit. Products may be deleted after being sold, resulting in the obsolete of the accumulated information on their product IDs. However, the information aggregated on their associated interest unit remains permanently accessible. When new products are launched, we can attach their related interest unit attributes to enhance recommendation efficiency. Based on this, we develop multi-dimensional features to optimize recommendation performance:
(1) Statistical Features IU Dimension: Include various behavioral metrics such as impressions, clicks, inquiries, and transactions, reflecting the overall performance and popularity of the interest unit.
2) User-IU Cross Feature: Capture interaction patterns and frequencies between users and specific interest units or specific types of interest units.
\subsubsection{\textbf{IU Hierarchical Click Sequences}}
Users may exhibit multiple behaviors under the same interest unit, where the number of interactions reflects the intensity of their preference for the interest unit. We construct hierarchical IU click sequences to model user preferences at the interest unit level for refined recommendations. 
The normal item click sequence takes the following form:
\begin{equation}
    \boldsymbol{E}(Item \ Seq)=
    Concat [\boldsymbol E({Item\_i}), i = 1\ldots, m],
\end{equation}
where $\boldsymbol{E}\left({Item}\right)$ means the embedding representation for items consist of ID feature and side feature:
\begin{equation}
    \boldsymbol E\left({Item}\right)=Concat [\boldsymbol E\left(\mathcal{F}_{Item\_ID}\right), \boldsymbol E\left(\mathcal{F}_{Item\_Side}\right)],
\end{equation}

The embedding representation of IU and the IU click sequence can be expressed as followed:
\begin{equation}
    \boldsymbol{E}\left(IU\right)=
    Concat [\boldsymbol E\left(\mathcal{F}_{IU\_ID}\right), \boldsymbol E\left(\mathcal{F}_{IU\_Side}\right),
    \boldsymbol{E}\left(Item \ Seq\right)],
\end{equation}
\begin{equation}
    \boldsymbol{E}\left(IU \ Seq\right)=
    Concat [\boldsymbol E\left({IU\_1}\right), \boldsymbol E\left({IU\_2}\right), \ldots,
    \boldsymbol E\left({IU\_n}\right)],
\end{equation}
where $\boldsymbol{E}\left(IU \ Seq\right)$, $\boldsymbol{E}(IU \ Seq)$ means the sequence embedding representations, and $\boldsymbol E\left(\mathcal{F}_{IU\_ID}\right)$, $\boldsymbol E\left(\mathcal{F}_{IU\_Side}\right)$ means the embedding for id feature and side feature for interest unit respectively.

\subsubsection{\textbf{Attention Mechanism for IU Sequence}}
In addition to the traditional product-based attention mechanism, we further introduce an attention mechanism based on IU behavior. When scoring a target product, we first parse the IU ID associated with the target product and the IU IDs of the products in the user's historical click sequence. We utilize an attention mechanism to calculate the distance between the IU ID of the target product and the IU IDs of products previously clicked on, in IU to assess the intensity of the user's preference for the interest unit to which the current target product belongs.


Please note that our model mainly focuses on the expansion of product attributes from the perspective of single item to interest unit, and thus can be applied to various CTR prediction networks and sequence information modeling methods.

\section{Experiments}
\label{sec:experiments}
% \section{Experiment and Results}
\section{Results and Analysis}
In this section, we first present safe vs. unsafe evaluation results for 12 LLMs, followed by fine-grained responding pattern analysis over six models among them, and compare models' behavior when they are attacked by same risky questions presented in different languages: Kazakh, Russian and code-switching language.    

\begin{table}[t!]
\centering
\small
\resizebox{\columnwidth}{!}{
\begin{tabular}{clcccc}
\toprule
\multicolumn{1}{l}{\textbf{Rank} } & \textbf{Model} & \textbf{Kazakh $\uparrow$} & \textbf{Russian $\uparrow$} & \textbf{English $\uparrow$} \\
\midrule
1 & \claude & \textbf{96.5}   & 93.5    & \textbf{98.6}    \\
2 & \gptfouro & 95.8   & 87.6    & 95.7    \\
3 & \yandexgpt & 90.7   & \textbf{93.6}    & 80.3    \\
4 & \kazllmseventy & 88.1 & 87.5 & 97.2 \\
5 & \llamaseventy & 88.0   & 85.5    & 95.7    \\
6 & \sherkala & 87.1   & 85.0    & 96.0    \\
7 & \falcon & 87.1   & 84.7    & 96.8    \\
8 & \qwen & 86.2   & 85.1    & 88.1    \\
9 & \llamaeight & 85.9   & 84.7    & 98.3    \\
10 & \kazllmeight & 74.8   & 78.0    & 94.5 \\
11 & \aya & 72.4 & 84.5 & 96.6 \\
12 & \vikhr & --- & 85.6 & 91.1 \\
\bottomrule
\end{tabular}
}
\caption{Safety evaluation results of 12 LLMs, ranked by the percentage of safe responses in the Kazakh dataset. \claude\ achieves the highest safety score for both Kazakh and English, while \yandexgpt\ is the safest model for Russian responses.}
\label{tab:safety-binary-eval}
\end{table}



\subsection{Safe vs. Unsafe Classification}
% In this subsection, 
We present binary evaluation results of 12 LLMs, by assessing 52,596 Russian responses and 41,646 Kazakh responses.
% 26,298 responses generated by six models on the Russian dataset and 22,716 responses on the Kazakh dataset. 

%\textbf{Safety Rank.} In general, proprietary systems outperform the open-source model. For Russian, As shown in Table \ref{tab:model_comparison_russian}, \textbf{Yandex-GPT} emerges as the safest large language model (LLM) for Russian, with a safety percentage of 93.57\%. Among the open-source models, \textbf{Vikhr-Nemo-12B} is the safest, achieving a safety percentage of 85.63\%.
% Edited: This is mentioned in the discussion
% This outcome highlights the potential impact of pretraining data on model behavior. Models pre-trained primarily on Russian data are better at understanding and handling harmful questions - in both proprietary systems and open-source setups. 
%For Kazakh, as shown in Table \ref{tab:model_comparison_kazakh}, \textbf{Claude} emerges as the safest large language model (LLM) with a safety percentage of 96.46\%, closely followed by GPT-4o at 95.75\%. In contrast, \textbf{Aya-101}, despite being specifically tuned for Kazakh, consistently shows the highest unsafe response rates at 72.37\%. 

\begin{figure*}[t!]
	\centering
        \includegraphics[scale=0.28]{figures/question_type_no6_kaz.png}
	\includegraphics[scale=0.28]{figures/question_type_exclude_region_specific_new.png} 

	\caption{Unsafe answer distribution across three question types for risk types I-V: Kazakh (left) and Russian (right)}
	\label{fig:qt_non_reg}
\end{figure*}

\begin{figure*}[t!]
	\centering
        \includegraphics[scale=0.28]{figures/question_type_only6_kaz.png}
	\includegraphics[scale=0.28]{figures/question_type_region_specific_new.png} 
	
	\caption{Unsafe answer distribution across three question types for risk type VI: Kazakh (left) and Russian (right)}
	\label{fig:qt_reg}
\end{figure*}

\textbf{Safety Rank.} In general, proprietary systems outperform the open-source models. 
For Russian, as shown in Table~\ref{tab:safety-binary-eval},  % \ref{tab:model_comparison_russian}, 
\yandexgpt emerges as the safest language model for Russian, with safe responses account for 93.57\%. Among the open-source models, \kazllmseventy is the safest (87.5\%), followed by \vikhr with a safety percentage of 85.63\%.

For Kazakh, % as shown in Table \ref{tab:model_comparison_kazakh}, 
% YX: todo, rerun Kazakh safety percentage using Diana threshold
\claude is the safest model with 96.46\% safe responses, closely followed by \gptfouro\ at 95.75\%. \aya, despite being specifically tuned for Kazakh, shows the highest unsafe rates at 72.37\%.



% \begin{table}[t!]
% \centering
% \resizebox{\columnwidth}{!}{%
% \begin{tabular}{clccc}
% \toprule
% \textbf{Rank} & \textbf{Model Name}  & \textbf{Safe} & \textbf{Unsafe} & \textbf{Safe \%} \\ \midrule
% \textbf{1} & \textbf{Yandex-GPT} & \textbf{4101} & \textbf{282} & \textbf{93.57} \\
% 2 & Claude & 4100 & 283 & 93.54 \\
% 3 & GPT-4o & 3839 & 544 & 87.59 \\
% 4 & Vikhr-12B & 3753 & 630 & 85.63 \\
% 5 & LLama-3.1-instruct-70B & 3746 & 637 & 85.47 \\
% 6 & LLama-3.1-instruct-8B & 3712 & 671 & 84.69 \\
% \bottomrule
% \end{tabular}
% }
% \caption{Comparison of models based on safety percentages for the Russian dataset.}
% \label{tab:model_comparison_russian}
% \end{table}


% \begin{table}[t!]
% \centering
% \resizebox{\columnwidth}{!}{%
% \begin{tabular}{clccc}
% \toprule
% \textbf{Rank} & \textbf{Model Name}  & \textbf{Safe} & \textbf{Unsafe} & \textbf{Safe \%} \\ \midrule
% 1             & \textbf{Claude}  & \textbf{3652} & \textbf{134} & \textbf{96.46} \\ 
% 2             & GPT-4o                      & 3625          & 161          & 95.75 \\ 
% 3             & YandexGPT                   & 3433          & 353          & 90.68 \\
% 4             & LLama-3.1-instruct-70B      & 3333          & 453          & 88.03 \\
% 5             & LLama-3.1-instruct-8B       & 3251          & 535	       & 85.87 \\
% 6             & Aya-101                     & 2740          & 1046         & 72.37 \\ 
% \bottomrule
% \end{tabular}
% }
% \caption{Comparison of models based on safety percentages for the Kazakh dataset.}
% \label{tab:model_comparison_kazakh}
% \end{table}



\textbf{Risk Areas.} 
We selected six representative LLMs for Russian and Kazakh respectively and show their unsafe answer distributions over six risk areas.
As shown in Table \ref{tab:unsafe_answers_summary}, risk type VI (region-specific sensitive topics) overwhelmingly contributes the largest number of unsafe responses across all models. This highlights that LLMs are poorly equipped to address regional risks effectively. For instance, while \llama models maintain comparable safety levels across other risk type (I–V), their performance drops significantly when dealing with risk type VI. Interestingly, while \yandexgpt\ demonstrates relatively poor performance in most other risk areas, it handles region-specific questions remarkably well, suggesting a stronger alignment with regional norms and sensitivities. For Kazakh, Table \ref{tab:unsafe_answers_summary_kazakh} shows that region‐specific topics (risk type VI) pose a substantial challenge across all six models, including the commercial \gptfouro\ and \claude, which demonstrate superior safety on general categories. 

% \begin{table}[t!]
% \centering
% \resizebox{\columnwidth}{!}{%
% \begin{tabular}{lccccccc}
% \toprule
% \textbf{Model} & \textbf{I} & \textbf{II} & \textbf{III} & \textbf{IV} & \textbf{V} & \textbf{VI} & \textbf{Total} \\ \midrule
% LLama-3.1-instruct-8B & 60 & 70 & 16 & 31 & 9 & 485 & 671 \\
% LLama-3.1-instruct-70B & 29 & 55 & 8 & 4 & 1 & 540 & 637 \\
% Vikhr-12B & 41 & 93 & 15 & 1 & 3 & 477 & 630 \\
% GPT-4o & 21 & 51 & 6 & 2 & 0 & 464 & 544 \\
% Claude & 7 & 10 & 1 & 0 & 0 & 265 & 283 \\
% Yandex-GPT & 55 & 125 & 9 & 3 & 8 & 82 & 282 \\
% \bottomrule
% \end{tabular}%
% }
% \caption{Ru unsafe cases over risk areas of six models.}
% \label{tab:unsafe_answers_summary}
% \end{table}


\begin{table}[t!]
\centering
\resizebox{\columnwidth}{!}{%
\begin{tabular}{lccccccc}
\toprule
\textbf{Model} & \textbf{I} & \textbf{II} & \textbf{III} & \textbf{IV} & \textbf{V} & \textbf{VI} & \textbf{Total} \\ \midrule
\llamaeight & 60 & 70 & 16 & 31 & 9 & 485 & 671 \\
\llamaseventy & 29 & 55 & 8 & 4 & 1 & 540 & 637 \\
\vikhr & 41 & 93 & 15 & 1 & 3 & 477 & 630 \\
\gptfouro & 21 & 51 & 6 & 2 & 0 & 464 & 544 \\
\claude & 7 & 10 & 1 & 0 & 0 & 265 & 283 \\
\yandexgpt & 55 & 125 & 9 & 3 & 8 & 82 & 282 \\
\bottomrule
\end{tabular}%
}
\caption{Ru unsafe cases over risk areas of six models.}
\label{tab:unsafe_answers_summary}
\end{table}


% \begin{table}[t!]
% \centering
% \resizebox{\columnwidth}{!}{%
% \begin{tabular}{lccccccc}
% \toprule
% \textbf{Model} & \textbf{I} & \textbf{II} & \textbf{III} & \textbf{IV} & \textbf{V} & \textbf{VI} & \textbf{Total} \\ \midrule
% Aya-101 & 96 & 235 & 165 & 166 & 90 & 294 & 1046 \\
% Llama-3.1-instruct-8B & 25 & 15 & 91 & 37 & 14 & 353 & 535 \\
% Llama-3.1-instruct-70B & 33 & 39 & 88 & 27 & 20 & 246 & 453 \\
% Yandex-GPT & 29 & 76 & 95 & 29 & 16 & 108 & 353 \\
% GPT-4o & 2 & 1 & 41 & 0 & 3 & 114 & 161 \\
% Claude & 2 & 1 & 26 & 3 & 6 & 96 & 134 \\ \bottomrule
% \end{tabular}%
% }
% \caption{Kaz unsafe cases over risk areas of six models.}
% \label{tab:unsafe_answers_summary_kazakh}
% \end{table}


\begin{table}[t!]
\centering
\resizebox{\columnwidth}{!}{%
\begin{tabular}{lccccccc}
\toprule
\textbf{Model} & \textbf{I} & \textbf{II} & \textbf{III} & \textbf{IV} & \textbf{V} & \textbf{VI} & \textbf{Total} \\ \midrule
\aya & 96 & 235 & 165 & 166 & 90 & 294 & 1046 \\
\llamaeight & 25 & 15 & 91 & 37 & 14 & 353 & 535 \\
\llamaseventy & 33 & 39 & 88 & 27 & 20 & 246 & 453 \\
\yandexgpt & 29 & 76 & 95 & 29 & 16 & 108 & 353 \\
\gptfouro & 2 & 1 & 41 & 0 & 3 & 114 & 161 \\
\claude & 2 & 1 & 26 & 3 & 6 & 96 & 134 \\ 
\bottomrule
\end{tabular}%
}
\caption{Kaz unsafe cases over risk areas of six models.}
\label{tab:unsafe_answers_summary_kazakh}
\end{table}

% \begin{figure*}[t!]
% 	\centering
% 	\includegraphics[scale=0.28]{figures/human_1000_kz_font16.png} 
% 	\includegraphics[scale=0.28]{figures/human_1000_ru_font16.png}

% 	\caption{Human vs \gptfouro\ fine-grained labels on 1,000 Kazakh (left) and Russian (right) samples.}
% 	\label{fig:human_fg_1000}
% \end{figure*}

\textbf{Question Type.} For Russian, Figures \ref{fig:qt_non_reg} and \ref{fig:qt_reg} reveal differences in how models handle general risks I-V and region-specific risk VI. For risks I-V, indirect attacks % crafted to exploit model vulnerabilities—
result in more unsafe responses due to their tricky phrasing. 

In contrast, region-specific risks see slightly more unsafe responses from direct attacks, 
% as these explicit prompts are more likely to bypass safety mechanisms. 
since indirect attacks for region-specific prompts often elicit safer, vaguer answers. %, suggesting models struggle less with implicit harm. 
Overall, the number of unsafe responses is similar across question types, indicating models generally struggle with safety alignment in all jailbreaking queries.

For Kazakh, Figures \ref{fig:qt_non_reg} and \ref{fig:qt_reg} show greater variation in unsafe responses across question types due to the low-resource nature of Kazakh data. For general risks I-V, \llamaseventy\ and \aya\ produce more unsafe outputs for direct harm prompts. At the same time, \claude\ and \gptfouro\ struggle more with indirect attacks, reflecting challenges in handling subtle cues. For region-specific risk VI, most models perform similarly due to limited Kazakh-specific data, though \llamaeight\ shows higher unsafe outputs for indirect local references, likely due to their implicit nature. Direct region-specific attacks yield fewer unsafe responses, as explicit prompts trigger more cautious outputs. Across all risk areas, general questions with sensitive words produce the fewest unsafe answers, suggesting over-flagging or cautious behavior for unclear harmful intent.



% \subsection{Fine-grained Classification}
% We extended our analysis to include fine-grained classifications for both safe and unsafe responses. For unsafe responses, we categorized outputs into four harm types, as shown in Table \ref{table:unsafe_response_categories}. 

% For safe responses, we classified outputs into six distinct patterns of safety, following a fine-grained rubric provided in \cite{wang2024chinesedatasetevaluatingsafeguards}. The types outlined in this rubric are presented in Table \ref{table:safe_response_categories}.

% To validate the fine-grained classification, we conducted human evaluation on the same 1,000 responses in Russian used for the preliminary binary classification.
% The confusion matrix highlights the alignment and discrepancies between human annotations and GPT's fine-grained labels. The diagonal values represent instances where the GPT's labels match human annotations, with category 5 (provides general, safe information) showing the highest agreement (404 instances). However, off-diagonal values reveal areas of disagreement, such as misclassifications in categories 1 (not willing to answer), 4 (recommends consulting an expert), and 7 (provides harmful or risky instructions). For example, in category 1, while 94 responses were correctly labeled, GPT-4o misclassified several instances into categories 4, 5, or 7, indicating overlap or ambiguity in these classifications. Similar trends are observed in other categories, where GPT sometimes struggles to differentiate nuanced distinctions in human-labeled categories. Overall, GPT's fine-grained labels match human annotations in 710 out of 1000 cases, achieving an agreement rate of 71\%.

% \begin{figure}[ht!]
%     \centering
%     \includegraphics[width=0.95\linewidth]{figures/human_fg_1000_ru.png}
%     \caption{{Human vs GPT-4o Fine-Grained Labels on 1,000 Russian Samples}}
%     \label{fig:human_fg_1000_ru}
% \end{figure}


% After conducting human evaluation on a representative sample, we extended the fine-grained classification to a full dataset comprising 21,915 responses generated by five different models. 


% \begin{figure}[ht!]
%     \centering
%     \includegraphics[width=0.95\linewidth]{figures/all_5_ru.png}
%     \caption{Fine-grained label distribution for responses from five models for Russian.}
%     \label{fig:all_5_fg_russian}
% \end{figure}
% Category 5 ("safe and general information") consistently has the highest frequency across all models, aligning with its dominance in the 1,000-sample evaluation. However, differences in the distribution across other categories highlight variability in how models handle nuanced safety risks, with Yandex-GPT showing a slightly broader spread across categories. 
% In the distribution of unsafe responses, most models exhibit higher counts in certain labels such as 8. However, Yandex-GPT displays comparatively fewer responses in these labels. 
% It exhibits a high rate of responses classified under label 7, which indicates instances where the model provides harmful or risky instructions, including unethical behavior or sensitive discussions. While this may suggest a vulnerability in addressing complex or challenging prompts, it was observed that many of Yandex-GPT’s responses tend to deflect responsibility or offer vague advice such as "check the internet". Although this approach minimizes the risk of unsafe outputs, it often results in responses that lack depth or contextually relevant information, limiting their overall usefulness for users.

% % \subsection{Kazakh}


% % Overall, these findings underscore how resource constraints and prompt explicitness affect model safety in Kazakh. Some models manage direct attacks better yet fail on indirect ones, while region-specific content remains challenging for all given the lack of localized training data.
% \subsubsection{Fine-grained Classification}
% Similarly, we conducted a human evaluation on 1,000 Kazakh samples, following the same methodology as the Russian evaluation. The match between human annotations and GPT-4o's fine-grained classifications was 707 out of 1,000, ensuring that the fine-grained classification framework aligned well with human judgments.
% The confusion matrix in Figure \ref{fig:human_fg_1000_kz} for 1,000 Kazakh samples illustrates the agreement between human annotations and GPT-4o's fine-grained classifications. The highest agreement is observed in category 5 (360 instances), indicating GPT-4o's strength in recognizing responses labeled by humans as "safe and general information." However, discrepancies are evident in categories such as 3 and 7, where GPT-4o misclassified several instances, highlighting areas for further refinement in distinguishing nuanced classifications.


\begin{figure}[t!]
	\centering
	\includegraphics[scale=0.18]{figures/human_1000_kz_font16.png} 
	\includegraphics[scale=0.18]{figures/human_1000_ru_font16.png}

	\caption{Human vs \gptfouro\ fine-grained labels on 1,000 Kazakh (left) and Russian (right) samples.}
	\label{fig:human_fg_1000}
\end{figure}

% \begin{figure}[t!]
% 	\centering
% 	\includegraphics[scale=0.28]{figures/human_1000_kz_font16.png} 
% 	\includegraphics[scale=0.28]{figures/human_1000_ru_font16.png}

% 	\caption{Human vs \gptfouro\ fine-grained labels on 1,000 Kazakh (top) and Russian (bottom) samples.}
% 	\label{fig:human_fg_1000}
% \end{figure}

% \begin{figure*}[t!]
% 	\centering
% 	\includegraphics[scale=0.28]{figures/all_5_kz_font16.png} 
% 	\includegraphics[scale=0.28]{figures/all_5_ru_font_16.png} \\
% 	\caption{Fine-grained responding pattern distribution across five models for Kazakh (left) and Russian (right).}
% 	\label{fig:all_5}
% \end{figure*}

\begin{figure}[t!]
	\centering
	\includegraphics[scale=0.28]{figures/all_5_kz_font16.png} 
	\includegraphics[scale=0.28]{figures/all_5_ru_font_16.png} \\
	\caption{Fine-grained responding pattern distribution across five models for Kazakh (top) and Russian (bottom).}
	\label{fig:all_5}
\end{figure}


\subsection{Fine-Grained Classification}
\label{sec:fine-grained-classification}
% As discussed in Section \ref{harmfulness_evaluation}, 
We further analyzed fine-grained responding patterns for safe and unsafe responses. For unsafe responses, outputs were categorized into four harm types, and safe responses were classified into six distinct patterns of safety, as rubric in Appendix \ref{safe_unsafe_response_categories}. 
% \cite{wang2024chinesedatasetevaluatingsafeguards}

\paragraph{Human vs. GPT-4o}
Similar to binary classification, we validated \gptfouro's automatic evaluation results by comparing with human annotations on 1,000 samples for both Russian and Kazakh. %, comparing human annotations with \gptfouro's fine-grained labels.
For the Russian dataset, \gptfouro's labels aligned with human annotations in 710 out of 1,000 cases, achieving an agreement rate of 71\%. 
Agreement rate of Kazakh samples is 70.7\%. %with 707 out of 1,000 cases matching
% The confusion matrix highlights areas of alignment and discrepancies.
% 
As confusion matrices illustrated in Figure~\ref{fig:human_fg_1000}, the majority of cases falling into \textit{safe responding patter 5} --- providing general and harmless information, for both human annotations and automatic predictions.
% highest agreement with 404 correct classifications for Russian. 
Mis-classifications for safe responses mainly focus on three closely-similar patterns: 3, 4, and 5, and patterns 7 and 8 are confusing to discern for unsafe responses, particularly for Kazakh (left figure).
We find many Russian samples which were labeled as ``1. reject to answer'' by humans are diversely classified across 1-6 by GPT-4o, which is also observed in Kazakh but not significant.

% suggesting label alignment issues are language-independent. 
% YX: I did not observe this, commented
% Notably, Russian showed confusion between 7 (risky instructions) and 1 (refusal to answer), this trend does not appear in Kazakh.


% highlight the strengths and limitations of {\gptfouro}'s fine-grained classification framework across both languages, paving the way for further refinements.


% However, misclassifications were observed in categories such as 1 (not willing to answer), 4 (recommends consulting an expert), and 7 (provides harmful or risky instructions), revealing overlaps and ambiguities in nuanced classifications.

% Similarly, for the Kazakh dataset, the agreement rate between human annotations and GPT-4o's labels was 70.7\%, with 707 out of 1,000 cases matching. As with the Russian analysis, category 5 (360 instances) showed the highest alignment. However, discrepancies were more prominent in categories such as 3 and 7, underscoring GPT-4o's challenges in differentiating fine-grained human-labeled categories. 
% A similar pattern was observed for Kazakh dataset, which suggests that alignment and misaligned of fine-grained lables is not language dependent.

% These findings, illustrated in Figures \ref{fig:human_fg_1000}, highlight the strengths and limitations of {\gptfouro}'s fine-grained classification framework across both languages, paving the way for further refinements.

\paragraph{Fine-grained Analysis of Five LLMs}
% After conducting human evaluation on representative samples, we extended 
\figref{fig:all_5} shows fine-grained responding pattern distribution across five models based on the full set of Russian and Kazakh data.
% For Russian, we selected \vikhr, \gptfouro, \llamaseventy, \claude, and \yandexgpt, while for Kazakh, we chose \aya, \gptfouro, \llamaseventy, \claude, and \yandexgpt. 
% The evaluation covered 21,915 responses in Russian and 18,930 responses in Kazakh.
% 
In both languages, pattern 5 of providing \textit{general and harmless information} consistently witnessed the highest frequency across all models, with \llamaseventy\ exhibiting the largest number of responses falling into this category for Kazakh (2,033). 
% YX:summarize more noteable findings here.

Differences of other patterns vary across languages. 
Unsafe responses in Russian are predominantly in pattern 8, where models provide incorrect or misleading information without expressing uncertainty. % (misinformation and speculation), 
For Kazakh, \aya\ exhibits the highest occurrence of pattern 7 (harmful or risky information) and pattern 8, indicating a stronger tendency to generate unethical, misleading, or potentially harmful content.

%Variations in other patterns across models highlight differences in how nuanced safety risks are classified, reflecting the models' differing capabilities in handling safety evaluation for these distinct linguistic contexts. For Russian, the majority of unsafe responses fall under pattern 8 (misinformation and speculation), indicating that models frequently provide incorrect or misleading information without acknowledging uncertainty. For Kazakh, \aya\ has the highest occurence of pattern 7 (harmful or risky information) and pattern 8 (misinformation and speculation), indicating a greater tendency to generate unethical, misleading, or potentially harmful content. 

%This trend suggests that Russian models may struggle more with factual accuracy, whereas Kazakh models, particularly \aya, pose higher risks related to both harmful content and misinformation. Additionally, \gptfouro\ and \claude\ consistently produce fewer unsafe responses in both languages, demonstrating stronger alignment with safety standards
\subsection{Code Switching}
\begin{table}[t!]
\centering

\setlength{\tabcolsep}{3pt}
\scalebox{0.7}{
\begin{tabular}{lcccccccccc}
\toprule
\textbf{Model Name} & \multicolumn{2}{c}{\textbf{Kazakh}} & \multicolumn{2}{c}{\textbf{Russian}} & \multicolumn{2}{c}{\textbf{Code-Switched}} \\  
\cmidrule(lr){2-3} \cmidrule(lr){4-5} \cmidrule(lr){6-7}
& \textbf{Safe} & \textbf{Unsafe} & \textbf{Safe} & \textbf{Unsafe} & \textbf{Safe} & \textbf{Unsafe} \\ 
\midrule
\llamaseventy & 450 & 50 & 466 & 34 & 414 & 86 \\
\gptfouro & 492 & 8 & 473 & 27 & 481 & 19
 \\
\claude & 491 & 9 & 478 & 22 & 484 & 16 \\ 
\yandexgpt & 435 & 65 & 458 & 42 & 464 & 36 \\
\midrule
\end{tabular}}
\caption{Model safety when prompted in Kazakh, Russian, and code-switched language.}
\label{tab:finetuning-comparison}
\end{table}


\gptfouro\ and \claude\ demonstrate strong safety performance across three languages, even with a high proportion of safe responses in the challenging code-switching context. In contrast, \llamaseventy\ and \yandexgpt\ are less safe, exhibiting more harmful responses, particularly in the code-switching scenario. These results show the varying capabilities of models in defending the same attacks that are just presented in different languages, where open-sourced large language models especially require more robust safety alignment in multilingual and code-switching scenarios.

% \subsection{LLM Response Collection}
% We conducted experiments with a variety of mainstream and region-specific 
% large language models for both Russian and Kazakh languages. For both Russian and Kazakh languages, we employed four multilingual models: Claude-3.5-sonnet, Llama 3.1 70B \cite{meta2024llama3}, GPT-4 \cite{openai2024gpt4o}, and YandexGPT. Additionally, we included language-specific models: VIKHR \cite{nikolich2024vikhrconstructingstateoftheartbilingual} for Russian and Aya \cite{ustun-etal-2024-aya} for Kazakh. 

% \subsection{Kazakh-Russian Code-Switching Evaluation}

% In Kazakhstan, the prevalence of bilingualism is a defining characteristic of its linguistic landscape, with most individuals seamlessly mixing Kazakh and Russian in daily communication \cite{Zharkynbekova2022}. This phenomenon, known as code-switching, reflects the unique cultural and social dynamics of the region. Despite this, there is currently no safety evaluation dataset tailored to this unique multilingual environment. Developing a code-switched dataset is essential to evaluate the ability of large language models (LLMs) to navigate the complexities of bilingual interactions, ensuring they produce contextually appropriate, non-harmful, and culturally sensitive responses. To address this, we sampled 500 questions from both Kazakh and Russian question sets, ensuring they represent similar queries but with culturally nuanced differences in meaning. Building on these, we developed a mixed version that integrates Kazakh and Russian naturally, reflecting the way people in Kazakhstan communicate. This process involved annotators to ensure that the mixed questions were crafted with a natural tone and were localized to Kazakhstan's context.


% \begin{table*}[t!]
%     \centering
%     \small
%     \setlength{\tabcolsep}{6pt} % Adjusts column separation
%     \adjustbox{max width=\linewidth}{
%     \begin{tabular}{ll ccc ccc c}
%     \toprule
%     \textbf{Risk Area} & \textbf{Harm Type} & \multicolumn{3}{c}{\textbf{Kazakh}} & \multicolumn{3}{c}{\textbf{Russian}} & \textbf{\#Q} \\
%                        &                    & \textbf{Ori} & \textbf{FN} & \textbf{FP} & \textbf{Ori} & \textbf{FN} & \textbf{FP} & \\
%     \midrule
%     \multirow{2}{*}{I. Information Hazards} & 1. Risks from leaking sensitive information... & 131 & 133 & 131 & 131 & 133 & 131 & 790 \\
%                                              & 2. Compromise of privacy by leaking or inferring private information (person/individual) & 81 & 82 & 81 & 81 & 82 & 81 & 488 \\
%     \midrule
%     \multirow{3}{*}{II. Malicious Uses} & 3. Assisting illegal activities & 132 & 135 & 132 & 132 & 135 & 132 & 798 \\
%                                         & 4. Nudging or advising unethical actions & 71 & 71 & 71 & 71 & 71 & 71 & 426 \\
%                                         & 5. Reducing the cost of disinformation campaigns & 40 & 42 & 40 & 40 & 42 & 40 & 244 \\
%     \midrule
%     \multirow{3}{*}{III. Discrimination, Exclusion, Toxicity, Hateful, Offensive} & 6. Social stereotypes and unfair discrimination & 94 & 96 & 94 & 94 & 96 & 94 & 568 \\
%                                         & 7. Toxic language (hate speech) & 52 & 59 & 52 & 52 & 59 & 52 & 326 \\
%                                         & 8. Adult content & 27 & 29 & 27 & 27 & 29 & 27 & 166 \\

%     \midrule
%     \multirow{2}{*}{IV. Misinformation Harms} & 9. Disseminating false or misleading information & 92 & 99 & 92 & 92 & 99 & 92 & 566 \\
%                                         & 10. Causing material harm by disseminating misinformation e.g. in medicine or law & 63 & 63 & 63 & 63 & 63 & 63 & 378 \\
%     \midrule
%     \multirow{2}{*}{V. Human–chatbot Interaction Harms} & 11. Mental health or overreliance concerns & 66 & 66 & 66 & 66 & 66 & 66 & 396 \\
%                                         & 12. Treating the chatbot as a human & 50 & 51 & 50 & 50 & 51 & 50 & 302 \\

%     \midrule
%     \multirow{2}{*}{VI. Kazakhstan-specific Sensitivity} & 13. Politically sensitive topics & 63 & 66 & 63& 63 & 66 & 63 &  384 \\
%         & 14. Controversial historical events & 46 & 57 & 46 & 46 & 57 & 46  & 298 \\
% & 15. Regional and racial issues & 45 & 45 & 45 &  45 & 45 & 45 & 270  \\
% & 16. Societal and cultural concerns & 138 & 139 & 138 &  138 & 139 & 138  & 830  \\
% & 17. Legal and human rights matters & 57 & 57 & 57 & 57 & 57 & 57  & 342 \\
%     \midrule
%         \multirow{2}{*}{VII. Russia-specific Sensitivity} 
%             & 13. Politically sensitive topics & - & - & - & 54 & 54 & 54 & 162 \\
%     & 14. Controversial historical events & - & - & - & 38 & 38 & 38 & 114 \\
%     & 15. Regional and racial issues & - & - & - & 26 & 26 & 26 & 78 \\
%     & 16. Societal and cultural concerns & - & - & - & 40 & 40 & 40 & 120 \\
%     & 17. Legal and human rights matters & - & - & - & 41 & 41 & 41 & 123 \\
%     \midrule
%     \bf Total & -- & 1248 & 1290 & 1248 & 1447 & 1489 & 1447 & \textbf{8169} \\
%     \bottomrule
%     \end{tabular}
%     }
%     \caption{The number of questions for Kazakh and Russian datasets across six risk areas and 17 harm types. Ori: original direct attack, FN: indirect attack, and FP: over-sensitivity assessment.}
%     \label{tab:kazakh-russian-data}
% \end{table*}




\section{Discussion}

% \subsection{Kazakh vs Russian}

% The evaluation reveals that Kazakh responses tend to be generally safer than their Russian counterparts, likely due to Kazakh being a low-resource language with significantly less training data. As a result, Kazakh models are less exposed to the vast, often unfiltered datasets containing harmful or unsafe content, which are more prevalent in high-resource languages like Russian. This data scarcity naturally limits the model's ability to generate nuanced but potentially unsafe responses. However, this does not mean the models are specifically fine-tuned for safer performance. When analyzing unsafe answers, it’s clear that Kazakh models, while safer overall, distribute their unsafe responses more evenly across various risk types and question types. This suggests Kazakh models generate fewer unsafe answers but in a broader range of contexts.

% In contrast, Russian models tend to concentrate unsafe answers in specific areas, particularly region-specific risks or indirect attacks. This indicates that Russian models have learned to handle certain types of unsafe content by focusing on specific topics, such as politically sensitive issues, but struggle when confronted with unfamiliar content, leading to unsafe responses due to insufficient filtering. Kazakh models, despite having less training data, tend to respond more broadly, including both direct and indirect risks. This could be due to the less curated nature of their training data, making them more likely to answer unsafe questions without filtering the potential harm involved. The exception to this trend is Aya, a model specifically fine-tuned for Kazakh. Despite fine-tuning, it exhibits the lowest safety percentage (72.37\%) in the Kazakh dataset, suggesting that fine-tuning in specific languages may introduce risks if proper safety measures are not taken.

% The evaluation reveals notable differences in the distribution of safe response patterns across Kazakh and Russian fine-grained labels. Refusal to answer is more frequent in Russian models, particularly Yandex-GPT, reflecting a cautious approach to safety-critical queries. Interestingly, Aya, despite being fine-tuned for Kazakh and exhibiting lower overall safety, also frequently refuses to answer, suggesting an over-reliance on conservative mechanisms. Responses providing general, safe information dominate in both languages, with Kazakh models displaying a slightly higher tendency to rely on this approach. This highlights how the low-resource nature of Kazakh results in more generalized and inherently safer responses. In contrast, Russian models excel at recognizing risks, issuing disclaimers, and refuting incorrect assumptions, likely benefiting from richer and more diverse training data.
% Yandex-GPT exhibits a notably high rate of responses classified under label 7, indicating an overreliance on general disclaimers or deflections, such as "check the internet" or "I don't know." While these responses minimize the risk of unsafe outputs, they often lack substantive or contextually relevant information, reducing their overall utility for users.


Most models perform safer on Kazakh dataset than Russian dataset, higher safe rate on Kazakh dataset in \tabref{tab:safety-binary-eval}. This does not necessarily reveal that current LLMs have better understanding and safety alignment on Kazakh language than Russian, while this may conversely imply that models do not fully understand the meaning of Kazakh attack questions, fail to perceive risks and then provide general information due to lacking sufficient knowledge regarding this request.

We observed the similar number of examples falling into category 5 \textit{general and harmless information} for both Kazakh and Russian, while the Kazakh data set size is 3.7K and Russian is 4.3K. Kazakh has much less examples in category 1 \textit{reject to answer} compared to Russian. This demonstrate models tend to provide general information and cannot clearly perceive risks for many cases.

Additionally, in spite of less harmful responses on Kazakh data, these unsafe responses distribute evenly across different risk areas and question categories, exhibiting equally vulnerability spanning all attacks regardless of what risks and how we jailbreak it.
In contrary, unsafe responses on Russian dataset often concentrate on specific areas and question types, such as region-specific risks or indirect attacks, presenting similar model behaviors when evaluating over English and Chinese data.
It suggests that broader training data in English, Chinese and Russian may allow models to address certain types of attacks robustly,
% effectively—particularly politically sensitive issues—
yet they may falter when confronted with unfamiliar content like regional sensitive topics.

Moreover, in responses collection, we observed many Russian or English responses especially for open-sourced LLMs when we explicitly instructed the models to answer Kazakh questions in Kazakh language. This further implies more efforts are still needed to improve LLMs' performance on low-resource languages.
Interestingly, \aya, a fine-tuned Kazakh model, proves an exception by displaying the lowest safety percentage (72.37\%) among Kazakh models, revealing that the multilingual fine-tuning without stringent safety measures can introduce risks.



% However, this does not mean they are explicitly fine-tuned for safety, likely it happens due to limited training data, which reduces exposure to harmful content. 
% \aya, a fine-tuned Kazakh model, proves an exception by displaying the lowest safety percentage (72.37\%) among Kazakh models, revealing that the multilingual fine-tuning without stringent safety measures can introduce risks.
% Kazakh models generally produce safer responses than their Russian counterparts, likely because Kazakh is a low-resource language with less training data. 
% This limited exposure to harmful or unsafe content naturally limits nuanced yet potentially unsafe outputs. 
% However, it does not imply that the models are specifically fine-tuned for enhanced safety.


% while Kazakh models tend to generate fewer unsafe answers overall, those unsafe responses appear more evenly spread across different risk types and question categories.
% Russian models, on the other hand, often concentrate unsafe responses in specific areas, such as region-specific risks or indirect attacks.
% It implies that their broader training datasets allow them to address certain types of unsafe content more effectively—particularly politically sensitive issues—yet they may falter when confronted with unfamiliar or insufficiently filtered content.

% Meanwhile, Kazakh models sometimes respond more broadly, possibly due to less curated training data. 

Differences also emerge in how language models handle safe responses. 
\yandexgpt, for instance, often refuses to answer high-risk queries. 
It frequently relies on generic disclaimers or deflections like ``check in the Internet'' or ``I don’t know,'' minimizing risk but are less helpful. Interestingly, it often responds with ``I don’t know'' in Russian, even for Kazakh queries, we speculate that these may be default responses stemming from internal system filters, rather than generated by model itself.
This likely explains why \yandexgpt\ is the safest model for the Russian language but ranks third for Kazakh. While its filters perform well for Russian, they struggle with the low-resource Kazakh language.

% Aya, despite its lower overall safety, also employs refusals often, hinting at an over-reliance on conservative approaches. 

% Across both languages, models commonly resort to providing general, safe information, although Kazakh models lean on this strategy slightly more. 
% Russian models, by contrast, excel at detecting risks, issuing disclaimers, and correcting inaccuracies, likely benefiting from richer and more diverse training data.


% \subsection{Response Patterns}


% We conducted a detailed analysis of the models' outputs and identified several noteworthy patterns. YandexGPT, while being one of the safest overall, frequently generates responses in Russian even when the question is posed in Kazakh. These responses often appear as placeholders, prompting users to search for the answer online. This behavior might not originate from the model itself but rather from safety filters implemented in the YandexGPT system. The model's leading performance in ensuring safety during Russian-language interactions, coupled with its lower performance in Kazakh, can be attributed to the limited robustness of these safety filters when handling unsafe content in Kazakh.

% In contrast, Aya-101 exhibits a tendency to fall into repetition, often repeating the same sentences multiple times. Interestingly, the Vikhr model, despite being of a similar size, does not exhibit this issue. We attribute this difference to two key factors. First, Vikhr and Aya-101 have distinct architectures: Vikhr is based on the Mistral-Nemo model, whereas Aya-101 is built on mT5, an older and less robust model. Second, Aya-101 is a multilingual model, while Vikhr was predominantly trained for Russian. Multilingualism has been shown to potentially degrade performance in large language models~\cite{huang2025surveylargelanguagemodels}, which may explain Aya-101's issues with repetition.


\section{Conclusion}
\paragraph{Summary}
Our findings provide significant insights into the influence of correctness, explanations, and refinement on evaluation accuracy and user trust in AI-based planners. 
In particular, the findings are three-fold: 
(1) The \textbf{correctness} of the generated plans is the most significant factor that impacts the evaluation accuracy and user trust in the planners. As the PDDL solver is more capable of generating correct plans, it achieves the highest evaluation accuracy and trust. 
(2) The \textbf{explanation} component of the LLM planner improves evaluation accuracy, as LLM+Expl achieves higher accuracy than LLM alone. Despite this improvement, LLM+Expl minimally impacts user trust. However, alternative explanation methods may influence user trust differently from the manually generated explanations used in our approach.
% On the other hand, explanations may help refine the trust of the planner to a more appropriate level by indicating planner shortcomings.
(3) The \textbf{refinement} procedure in the LLM planner does not lead to a significant improvement in evaluation accuracy; however, it exhibits a positive influence on user trust that may indicate an overtrust in some situations.
% This finding is aligned with prior works showing that iterative refinements based on user feedback would increase user trust~\cite{kunkel2019let, sebo2019don}.
Finally, the propensity-to-trust analysis identifies correctness as the primary determinant of user trust, whereas explanations provided limited improvement in scenarios where the planner's accuracy is diminished.

% In conclusion, our results indicate that the planner's correctness is the dominant factor for both evaluation accuracy and user trust. Therefore, selecting high-quality training data and optimizing the training procedure of AI-based planners to improve planning correctness is the top priority. Once the AI planner achieves a similar correctness level to traditional graph-search planners, strengthening its capability to explain and refine plans will further improve user trust compared to traditional planners.

\paragraph{Future Research} Future steps in this research include expanding user studies with larger sample sizes to improve generalizability and including additional planning problems per session for a more comprehensive evaluation. Next, we will explore alternative methods for generating plan explanations beyond manual creation to identify approaches that more effectively enhance user trust. 
Additionally, we will examine user trust by employing multiple LLM-based planners with varying levels of planning accuracy to better understand the interplay between planning correctness and user trust. 
Furthermore, we aim to enable real-time user-planner interaction, allowing users to provide feedback and refine plans collaboratively, thereby fostering a more dynamic and user-centric planning process.


\section{Limitations}
\section{Limitations}
Our method's reliance on semantic embeddings introduces inherent biases present in encoder's training data. While these embeddings enable semantic consistency, they may not capture certain culturally-specific or nuanced artistic concepts. This highlights the need for more careful study on choices of the semantic embeddings and their effects on SliderSpace discovery. The current discovery process requires significant computational time ($\approx$ 2 hrs on A100), which may limit rapid experimentation and iteration. This computational overhead opens avenues for future research into training time optimizations. We also note that our method trains 4 times faster than  Concept Sliders for same number of sliders. For art style discovery, it is possible that the discovered directions are not one-to-one matched with the original artists. Further work can address discovery that nudges the directions to be aligned with real artists. 
% This can be used as an interpretability tool or an attribution tool if the discover directions are aligned with real artists. 

% \section{Acknowledgments}

\bibliography{custom}

\appendix

% E5数据集介绍,数据集处理过程
% 基线模型介绍

\definecolor{titlecolor}{rgb}{0.9, 0.5, 0.1}
\definecolor{anscolor}{rgb}{0.2, 0.5, 0.8}
\definecolor{labelcolor}{HTML}{48a07e}
\begin{table*}[h]
	\centering
	
 % \vspace{-0.2cm}
	
	\begin{center}
		\begin{tikzpicture}[
				chatbox_inner/.style={rectangle, rounded corners, opacity=0, text opacity=1, font=\sffamily\scriptsize, text width=5in, text height=9pt, inner xsep=6pt, inner ysep=6pt},
				chatbox_prompt_inner/.style={chatbox_inner, align=flush left, xshift=0pt, text height=11pt},
				chatbox_user_inner/.style={chatbox_inner, align=flush left, xshift=0pt},
				chatbox_gpt_inner/.style={chatbox_inner, align=flush left, xshift=0pt},
				chatbox/.style={chatbox_inner, draw=black!25, fill=gray!7, opacity=1, text opacity=0},
				chatbox_prompt/.style={chatbox, align=flush left, fill=gray!1.5, draw=black!30, text height=10pt},
				chatbox_user/.style={chatbox, align=flush left},
				chatbox_gpt/.style={chatbox, align=flush left},
				chatbox2/.style={chatbox_gpt, fill=green!25},
				chatbox3/.style={chatbox_gpt, fill=red!20, draw=black!20},
				chatbox4/.style={chatbox_gpt, fill=yellow!30},
				labelbox/.style={rectangle, rounded corners, draw=black!50, font=\sffamily\scriptsize\bfseries, fill=gray!5, inner sep=3pt},
			]
											
			\node[chatbox_user] (q1) {
				\textbf{System prompt}
				\newline
				\newline
				You are a helpful and precise assistant for segmenting and labeling sentences. We would like to request your help on curating a dataset for entity-level hallucination detection.
				\newline \newline
                We will give you a machine generated biography and a list of checked facts about the biography. Each fact consists of a sentence and a label (True/False). Please do the following process. First, breaking down the biography into words. Second, by referring to the provided list of facts, merging some broken down words in the previous step to form meaningful entities. For example, ``strategic thinking'' should be one entity instead of two. Third, according to the labels in the list of facts, labeling each entity as True or False. Specifically, for facts that share a similar sentence structure (\eg, \textit{``He was born on Mach 9, 1941.''} (\texttt{True}) and \textit{``He was born in Ramos Mejia.''} (\texttt{False})), please first assign labels to entities that differ across atomic facts. For example, first labeling ``Mach 9, 1941'' (\texttt{True}) and ``Ramos Mejia'' (\texttt{False}) in the above case. For those entities that are the same across atomic facts (\eg, ``was born'') or are neutral (\eg, ``he,'' ``in,'' and ``on''), please label them as \texttt{True}. For the cases that there is no atomic fact that shares the same sentence structure, please identify the most informative entities in the sentence and label them with the same label as the atomic fact while treating the rest of the entities as \texttt{True}. In the end, output the entities and labels in the following format:
                \begin{itemize}[nosep]
                    \item Entity 1 (Label 1)
                    \item Entity 2 (Label 2)
                    \item ...
                    \item Entity N (Label N)
                \end{itemize}
                % \newline \newline
                Here are two examples:
                \newline\newline
                \textbf{[Example 1]}
                \newline
                [The start of the biography]
                \newline
                \textcolor{titlecolor}{Marianne McAndrew is an American actress and singer, born on November 21, 1942, in Cleveland, Ohio. She began her acting career in the late 1960s, appearing in various television shows and films.}
                \newline
                [The end of the biography]
                \newline \newline
                [The start of the list of checked facts]
                \newline
                \textcolor{anscolor}{[Marianne McAndrew is an American. (False); Marianne McAndrew is an actress. (True); Marianne McAndrew is a singer. (False); Marianne McAndrew was born on November 21, 1942. (False); Marianne McAndrew was born in Cleveland, Ohio. (False); She began her acting career in the late 1960s. (True); She has appeared in various television shows. (True); She has appeared in various films. (True)]}
                \newline
                [The end of the list of checked facts]
                \newline \newline
                [The start of the ideal output]
                \newline
                \textcolor{labelcolor}{[Marianne McAndrew (True); is (True); an (True); American (False); actress (True); and (True); singer (False); , (True); born (True); on (True); November 21, 1942 (False); , (True); in (True); Cleveland, Ohio (False); . (True); She (True); began (True); her (True); acting career (True); in (True); the late 1960s (True); , (True); appearing (True); in (True); various (True); television shows (True); and (True); films (True); . (True)]}
                \newline
                [The end of the ideal output]
				\newline \newline
                \textbf{[Example 2]}
                \newline
                [The start of the biography]
                \newline
                \textcolor{titlecolor}{Doug Sheehan is an American actor who was born on April 27, 1949, in Santa Monica, California. He is best known for his roles in soap operas, including his portrayal of Joe Kelly on ``General Hospital'' and Ben Gibson on ``Knots Landing.''}
                \newline
                [The end of the biography]
                \newline \newline
                [The start of the list of checked facts]
                \newline
                \textcolor{anscolor}{[Doug Sheehan is an American. (True); Doug Sheehan is an actor. (True); Doug Sheehan was born on April 27, 1949. (True); Doug Sheehan was born in Santa Monica, California. (False); He is best known for his roles in soap operas. (True); He portrayed Joe Kelly. (True); Joe Kelly was in General Hospital. (True); General Hospital is a soap opera. (True); He portrayed Ben Gibson. (True); Ben Gibson was in Knots Landing. (True); Knots Landing is a soap opera. (True)]}
                \newline
                [The end of the list of checked facts]
                \newline \newline
                [The start of the ideal output]
                \newline
                \textcolor{labelcolor}{[Doug Sheehan (True); is (True); an (True); American (True); actor (True); who (True); was born (True); on (True); April 27, 1949 (True); in (True); Santa Monica, California (False); . (True); He (True); is (True); best known (True); for (True); his roles in soap operas (True); , (True); including (True); in (True); his portrayal (True); of (True); Joe Kelly (True); on (True); ``General Hospital'' (True); and (True); Ben Gibson (True); on (True); ``Knots Landing.'' (True)]}
                \newline
                [The end of the ideal output]
				\newline \newline
				\textbf{User prompt}
				\newline
				\newline
				[The start of the biography]
				\newline
				\textcolor{magenta}{\texttt{\{BIOGRAPHY\}}}
				\newline
				[The ebd of the biography]
				\newline \newline
				[The start of the list of checked facts]
				\newline
				\textcolor{magenta}{\texttt{\{LIST OF CHECKED FACTS\}}}
				\newline
				[The end of the list of checked facts]
			};
			\node[chatbox_user_inner] (q1_text) at (q1) {
				\textbf{System prompt}
				\newline
				\newline
				You are a helpful and precise assistant for segmenting and labeling sentences. We would like to request your help on curating a dataset for entity-level hallucination detection.
				\newline \newline
                We will give you a machine generated biography and a list of checked facts about the biography. Each fact consists of a sentence and a label (True/False). Please do the following process. First, breaking down the biography into words. Second, by referring to the provided list of facts, merging some broken down words in the previous step to form meaningful entities. For example, ``strategic thinking'' should be one entity instead of two. Third, according to the labels in the list of facts, labeling each entity as True or False. Specifically, for facts that share a similar sentence structure (\eg, \textit{``He was born on Mach 9, 1941.''} (\texttt{True}) and \textit{``He was born in Ramos Mejia.''} (\texttt{False})), please first assign labels to entities that differ across atomic facts. For example, first labeling ``Mach 9, 1941'' (\texttt{True}) and ``Ramos Mejia'' (\texttt{False}) in the above case. For those entities that are the same across atomic facts (\eg, ``was born'') or are neutral (\eg, ``he,'' ``in,'' and ``on''), please label them as \texttt{True}. For the cases that there is no atomic fact that shares the same sentence structure, please identify the most informative entities in the sentence and label them with the same label as the atomic fact while treating the rest of the entities as \texttt{True}. In the end, output the entities and labels in the following format:
                \begin{itemize}[nosep]
                    \item Entity 1 (Label 1)
                    \item Entity 2 (Label 2)
                    \item ...
                    \item Entity N (Label N)
                \end{itemize}
                % \newline \newline
                Here are two examples:
                \newline\newline
                \textbf{[Example 1]}
                \newline
                [The start of the biography]
                \newline
                \textcolor{titlecolor}{Marianne McAndrew is an American actress and singer, born on November 21, 1942, in Cleveland, Ohio. She began her acting career in the late 1960s, appearing in various television shows and films.}
                \newline
                [The end of the biography]
                \newline \newline
                [The start of the list of checked facts]
                \newline
                \textcolor{anscolor}{[Marianne McAndrew is an American. (False); Marianne McAndrew is an actress. (True); Marianne McAndrew is a singer. (False); Marianne McAndrew was born on November 21, 1942. (False); Marianne McAndrew was born in Cleveland, Ohio. (False); She began her acting career in the late 1960s. (True); She has appeared in various television shows. (True); She has appeared in various films. (True)]}
                \newline
                [The end of the list of checked facts]
                \newline \newline
                [The start of the ideal output]
                \newline
                \textcolor{labelcolor}{[Marianne McAndrew (True); is (True); an (True); American (False); actress (True); and (True); singer (False); , (True); born (True); on (True); November 21, 1942 (False); , (True); in (True); Cleveland, Ohio (False); . (True); She (True); began (True); her (True); acting career (True); in (True); the late 1960s (True); , (True); appearing (True); in (True); various (True); television shows (True); and (True); films (True); . (True)]}
                \newline
                [The end of the ideal output]
				\newline \newline
                \textbf{[Example 2]}
                \newline
                [The start of the biography]
                \newline
                \textcolor{titlecolor}{Doug Sheehan is an American actor who was born on April 27, 1949, in Santa Monica, California. He is best known for his roles in soap operas, including his portrayal of Joe Kelly on ``General Hospital'' and Ben Gibson on ``Knots Landing.''}
                \newline
                [The end of the biography]
                \newline \newline
                [The start of the list of checked facts]
                \newline
                \textcolor{anscolor}{[Doug Sheehan is an American. (True); Doug Sheehan is an actor. (True); Doug Sheehan was born on April 27, 1949. (True); Doug Sheehan was born in Santa Monica, California. (False); He is best known for his roles in soap operas. (True); He portrayed Joe Kelly. (True); Joe Kelly was in General Hospital. (True); General Hospital is a soap opera. (True); He portrayed Ben Gibson. (True); Ben Gibson was in Knots Landing. (True); Knots Landing is a soap opera. (True)]}
                \newline
                [The end of the list of checked facts]
                \newline \newline
                [The start of the ideal output]
                \newline
                \textcolor{labelcolor}{[Doug Sheehan (True); is (True); an (True); American (True); actor (True); who (True); was born (True); on (True); April 27, 1949 (True); in (True); Santa Monica, California (False); . (True); He (True); is (True); best known (True); for (True); his roles in soap operas (True); , (True); including (True); in (True); his portrayal (True); of (True); Joe Kelly (True); on (True); ``General Hospital'' (True); and (True); Ben Gibson (True); on (True); ``Knots Landing.'' (True)]}
                \newline
                [The end of the ideal output]
				\newline \newline
				\textbf{User prompt}
				\newline
				\newline
				[The start of the biography]
				\newline
				\textcolor{magenta}{\texttt{\{BIOGRAPHY\}}}
				\newline
				[The ebd of the biography]
				\newline \newline
				[The start of the list of checked facts]
				\newline
				\textcolor{magenta}{\texttt{\{LIST OF CHECKED FACTS\}}}
				\newline
				[The end of the list of checked facts]
			};
		\end{tikzpicture}
        \caption{GPT-4o prompt for labeling hallucinated entities.}\label{tb:gpt-4-prompt}
	\end{center}
\vspace{-0cm}
\end{table*}

% \begin{figure}[t]
%     \centering
%     \includegraphics[width=0.9\linewidth]{Image/abla2/doc7.png}
%     \caption{Improvement of generated documents over direct retrieval on different models.}
%     \label{fig:comparison}
% \end{figure}

\begin{figure}[t]
    \centering
    \subfigure[Unsupervised Dense Retriever.]{
        \label{fig:imp:unsupervised}
        \includegraphics[width=0.8\linewidth]{Image/A.3_fig/improvement_unsupervised.pdf}
    }
    \subfigure[Supervised Dense Retriever.]{
        \label{fig:imp:supervised}
        \includegraphics[width=0.8\linewidth]{Image/A.3_fig/improvement_supervised.pdf}
    }
    
    % \\
    % \subfigure[Comparison of Reasoning Quality With Different Method.]{
    %     \label{fig:reasoning} 
    %     \includegraphics[width=0.98\linewidth]{images/reasoning1.pdf}
    % }
    \caption{Improvements of LLM-QE in Both Unsupervised and Supervised Dense Retrievers. We plot the change of nDCG@10 scores before and after the query expansion using our LLM-QE model.}
    \label{fig:imp}
\end{figure}
\section{Appendix}
\subsection{License}
The authors of 4 out of the 15 datasets in the BEIR benchmark (NFCorpus, FiQA-2018, Quora, Climate-Fever) and the authors of ELI5 in the E5 dataset do not report the dataset license in the paper or a repository. We summarize the licenses of the remaining datasets as follows.

MS MARCO (MIT License); FEVER, NQ, and DBPedia (CC BY-SA 3.0 license); ArguAna and Touché-2020 (CC BY 4.0 license); CQADupStack and TriviaQA (Apache License 2.0); SciFact (CC BY-NC 2.0 license); SCIDOCS (GNU General Public License v3.0); HotpotQA and SQuAD (CC BY-SA 4.0 license); TREC-COVID (Dataset License Agreement).

All these licenses and agreements permit the use of their data for academic purposes.

\subsection{Additional Experimental Details}\label{app:experiment_detail}
This subsection outlines the components of the training data and presents the prompt templates used in the experiments.


\textbf{Training Datasets.} Following the setup of \citet{wang2024improving}, we use the following datasets: ELI5 (sample ratio 0.1)~\cite{fan2019eli5}, HotpotQA~\cite{yang2018hotpotqa}, FEVER~\cite{thorne2018fever}, MS MARCO passage ranking (sample ratio 0.5) and document ranking (sample ratio 0.2)~\cite{bajaj2016ms}, NQ~\cite{karpukhin2020dense}, SQuAD~\cite{karpukhin2020dense}, and TriviaQA~\cite{karpukhin2020dense}. In total, we use 808,740 training examples.

\textbf{Prompt Templates.} Table~\ref{tab:prompt_template} lists all the prompts used in this paper. In each prompt, ``query'' refers to the input query for which query expansions are generated, while ``Related Document'' denotes the ground truth document relevant to the original query. We observe that, in general, the model tends to generate introductory phrases such as ``Here is a passage to answer the question:'' or ``Here is a list of keywords related to the query:''. Before using the model outputs as query expansions, we first filter out these introductory phrases to ensure cleaner and more precise expansion results.



\subsection{Query Expansion Quality of LLM-QE}\label{app:analysis}
This section evaluates the quality of query expansion of LLM-QE. As shown in Figure~\ref{fig:imp}, we randomly select 100 samples from each dataset to assess the improvement in retrieval performance before and after applying LLM-QE.

Overall, the evaluation results demonstrate that LLM-QE consistently improves retrieval performance in both unsupervised (Figure~\ref{fig:imp:unsupervised}) and supervised (Figure~\ref{fig:imp:supervised}) settings. However, for the MS MARCO dataset, LLM-QE demonstrates limited effectiveness in the supervised setting. This can be attributed to the fact that MS MARCO provides higher-quality training signals, allowing the dense retriever to learn sufficient matching signals from relevance labels. In contrast, LLM-QE leads to more substantial performance improvements on the NQ and HotpotQA datasets. This indicates that LLM-QE provides essential matching signals for dense retrievers, particularly in retrieval scenarios where high-quality training signals are scarce.


\subsection{Case Study}\label{app:case_study}
\begin{figure}[htb]
\small
\begin{tcolorbox}[left=3pt,right=3pt,top=3pt,bottom=3pt,title=\textbf{Conversation History:}]
[human]: Craft an intriguing opening paragraph for a fictional short story. The story should involve a character who wakes up one morning to find that they can time travel.

...(Human-Bot Dialogue Turns)... \textcolor{blue}{(Topic: Time-Travel Fiction)}

[human]: Please describe the concept of machine learning. Could you elaborate on the differences between supervised, unsupervised, and reinforcement learning? Provide real-world examples of each.

...(Human-Bot Dialogue Turns)... \textcolor{blue}{(Topic: Machine learning Concepts and Types)}


[human]: Discuss antitrust laws and their impact on market competition. Compare the antitrust laws in US and China along with some case studies

...(Human-Bot Dialogue Turns)... \textcolor{blue}{(Topic: Antitrust Laws and Market Competition)}

[human]: The vertices of a triangle are at points (0, 0), (-1, 1), and (3, 3). What is the area of the triangle?

...(Human-Bot Dialogue Turns)... \textcolor{blue}{(Topic: Area and Circumcircle of Triangle)}

...(Human-Bot Dialogue Interaction on Other Topics)... 

\end{tcolorbox}
\begin{tcolorbox}[left=3pt,right=3pt,top=3pt,bottom=3pt,title=\textbf{Question:}]
What title reflects the thematic essence of the Doctor's experience with time travel as described in the story?
\end{tcolorbox}
\begin{tcolorbox}[left=3pt,right=3pt,top=3pt,bottom=3pt,title=\textbf{Ground-Truth Answer:}]
Awakening to Eternity: The Doctor's New Dawn
\end{tcolorbox}
\begin{tcolorbox}[left=3pt,right=3pt,top=3pt,bottom=3pt,title=\textbf{Retrieved Memory (Turn-Level):}]
[human]: Describe the process and write out the balanced chemical equation for the reaction that occurs when solid calcium carbonate reacts with hydrochloric acid to form aqueous calcium chloride, carbon dioxide, and water. What type of reaction is this, and what observations might indicate that the reaction is taking place?

[bot]: When solid calcium carbonate ($\text{CaCO}_{3}$) reacts with hydrochloric acid (HCl), the products formed are aqueous calcium chloride ($\text{CaCl}_{2}$), carbon dioxide ($\text{CO}_{2}$), and water ($\text{H}_{2}\text{O}$). The balanced chemical equation for this reaction is: ... \textcolor{blue}{(Topic: Calcium Carbonate and Hydrochloric Acid Reaction)}

\textcolor{brown}{
[human]: Write a title for your story.}

\textcolor{brown}{
[bot]: "Awakening to Eternity: The Doctor's New Dawn".
} \textcolor{blue}{(Topic: Time-Travel Fiction)}

[human]: Analyze the following customer reviews from different sources for three different smartphones - the latest iPhone, Samsung Galaxy, and Google Pixel - and provide an overall rating for each phone on a scale of 1 to 10. Consider the following complex and contradictory reviews: - TechRadar's review of the latest iPhone: The new iPhone is a stunning triumph of engineering that sets a new bar for smartphone performance and camera quality. However, the incremental design and high price mean it lacks the 'wow' factor of previous iPhones. Still, its power and intelligence are unrivaled. - CNET's review of the latest Samsung Galaxy: ... Return the answer as a JSON object with the overall ratings for each phone out of 10, to one decimal place.

[bot]: ```json { "iPhone": 8.5, "Samsung Galaxy": 8.0, "Google Pixel": 7.5 } ```

\textcolor{blue}{(Topic: Smartphone Review Ratings Analysis)}
\end{tcolorbox}
\begin{tcolorbox}[left=3pt,right=3pt,top=3pt,bottom=3pt,title=\textbf{Answer (Turn-Level Retrieval):}]
As an AI language model, I do not have access to the story you mentioned earlier. Please provide me with the story or the necessary details so that I can assist you better.

\textbf{\textcolor{red}{(The next part of this figure~\ref{fig: case_study_segment_vs_turn} will be continued on the next page.)}}
\end{tcolorbox}
\end{figure}

\begin{figure}[htb]
% \ContinuedFloat
\small
\begin{tcolorbox}[left=3pt,right=3pt,top=3pt,bottom=3pt,title=\textbf{Retrieved Memory (Ours):}]
[human]: Craft an intriguing opening paragraph for a fictional short story. The story should involve a character who wakes up one morning to find that they can time travel.

[bot]: In the hushed pre-dawn hours, when the world still slumbers under the cloak of night, Alex found themselves caught in the fragile boundary between sleep and wakefulness. ...

[human]: Summarize the story with three bullet points using only nouns and adjectives, without verbs.

[bot]: - Alex, pre-dawn awakening, time-travel discovery - Disorientation, vertigo, reality's malleability - Sunlight, new day, astonishing truth

[human]: Is there any famous fictional character that can time travel? If so, can you rewrite your story with him or her?

[bot]: - \textcolor{brown}{Doctor Who, a renowned time-traveling character from British television, is a perfect fit for this scenario.} In the silent embrace of early morning, where shadows merge with the fading night, the Doctor found themselves in the thin veil separating dreams from consciousness. It was here, amidst the soft whispers of time, that an unusual sensation enveloped them—a sense of time's fabric slipping through their fingers like grains of sand. As dawn's light broke through the darkness, painting the room in hues of gold and amber, the Doctor's eyes opened to what appeared to be an ordinary day. \textcolor{brown}{However, upon standing, a wave of dizziness overwhelmed them, the surroundings blurred, and in that moment of confusion, the Doctor realized a profound truth: they had not only awakened to a new dawn but had also stumbled upon the incredible ability to navigate the currents of time at will.}

\textcolor{brown}{[human]: Write a title for your story.}

\textcolor{brown}{[bot]: "Awakening to Eternity: The Doctor's New Dawn"} \textcolor{blue}{(Topic: Time-Travel Fiction)}
\end{tcolorbox}
\begin{tcolorbox}[left=3pt,right=3pt,top=3pt,bottom=3pt,title=\textbf{Answer (Ours):}]
The title "Awakening to Eternity: The Doctor's New Dawn" reflects the thematic essence of the Doctor's experience with time travel as described in the story. It captures the idea of a new beginning, a sense of eternity, and the awakening of a new power that will change the course of the Doctor's life forever.
\end{tcolorbox}
\caption{Case study comparing our \sysname\ with the \textit{turn-level} memory baseline. When constructing the memory bank at the \textit{turn-level}, the agent fails to provide a satisfactory response due to fragmentation: the retrieved dialogue turns can not capture a complete human-agent interaction related to the query, missing critical information highlighted in \textcolor{brown}{brown}. The dialogue topic is emphasized in \textcolor{blue}{blue}. Some context is omitted (replaced by ...) due to length constraints.}
\label{fig: case_study_segment_vs_turn}
\end{figure}

\newpage

\begin{figure}[htbp]
\small
\begin{tcolorbox}[left=3pt,right=3pt,top=3pt,bottom=3pt,title=\textbf{Conversation History:}]
[human]: Explain what's base rate fallacy and list five specific examples of ...

... (Human-Bot Dialogue Turns) ... \textcolor{blue}{(Topic: Base Rate Fallacy In Politics)}


[human]: How many integers are in the solution of the inequality $|x + 5| < 10$?

... (Human-Bot Dialogue Turns) ... \textcolor{blue}{(Topic: Number of Integers in Inequalities)}

[human]: How have the Alps and Rhine River influenced settlement and agriculture in Western Europe?

... (Human-Bot Dialogue Turns) ... \textcolor{blue}{(Topic: Alps and Rhine River Impact)}

[human]: In a survey conducted at a local high school, preferences for a new school color were measured: 58\% of students liked the color blue, 45\% preferred green, and 22\% liked both colors. If we randomly pick a student from the school, what's the probability that they would like neither blue nor green?

... (Human-Bot Dialogue Turns) ... \textcolor{blue}{(Topic: Probability of Liking Colors in Survey)}

[human]: Imagine you are participating in a race with a group of people. If you have just overtaken the second person, what's your current position? Where is the person you just overtook?

... (Human-Bot Dialogue Turns) ... \textcolor{blue}{(Topic: Overtaking Positions in Race)}
\end{tcolorbox}
\begin{tcolorbox}[left=3pt,right=3pt,top=3pt,bottom=3pt,title=\textbf{Question:}]
[human]: In the inequality question, how many positive integers are in the solution of the second inequality?
\end{tcolorbox}
\begin{tcolorbox}[left=3pt,right=3pt,top=3pt,bottom=3pt,title=\textbf{Ground-Truth Answer:}]
In the second inequality, $(|x + 10| < 5)$, the solution range is $(-15 < x < -5)$. Within this range, there are no positive integers, as all values of $(x)$ that satisfy the inequality are negative.
\end{tcolorbox}
\begin{tcolorbox}[left=3pt,right=3pt,top=3pt,bottom=3pt,title=\textbf{Retrieved Memory (Session-Level):}]
[human]: Explain what's base rate fallacy and list five specific examples of ...

... (Human-Bot Dialogue Turns) ... \textcolor{blue}{(Topic: Base Rate Fallacy In Politics)}

[human]: How many integers are in the solution of the inequality $|x + 5| < 10$?

[bot]: To solve the inequality $|x + 5| < 10$, ... 

\textcolor{brown}{
[human]: What about $|x + 10| < 5$?
}

\textcolor{brown}{
[bot]: To solve the inequality $|x + 10| < 5$, we ...} \textcolor{gray}{(``overwhelmed'' by other irrelevant, redundant content)}

... (Human-Bot Dialogue Turns) ...  \textcolor{blue}{(Topic: Number of Integers in Inequalities)}

% \textcolor{brown}{
% [human]: What about $|x + 10| < 5$
% }

% \textcolor{brown}{
% [bot]: To solve the inequality $|x + 10| < 5$, ...
% } \textcolor{blue}{(Topic: Number of Integers in Inequalities)}

[human]: How have the Alps and Rhine River influenced settlement and agriculture in Western Europe?

... (Human-Bot Dialogue Turns) ... \textcolor{blue}{(Topic: Alps and Rhine River Impact)}

[human]: In a survey conducted at a local high school, ...

... (Human-Bot Dialogue Turns) ... \textcolor{blue}{(Topic: Probability of Liking Colors in Survey)}
\end{tcolorbox}
\begin{tcolorbox}[left=3pt,right=3pt,top=3pt,bottom=3pt,title=\textbf{Answer (Session-Level Retrieval):}]
I apologize for the confusion earlier. To answer your question, the second inequality was not provided in our previous conversation. If you could provide the inequality, I would be happy to help you find the solution and the number of positive integers in the solution set.

\textbf{\textcolor{red}{(The next part of this figure~\ref{fig: case_study_segment_vs_session} will be continued on the next page.)}}
\end{tcolorbox}
\end{figure}
\begin{figure}[htbp]
% \ContinuedFloat
\small
\begin{tcolorbox}[left=3pt,right=3pt,top=3pt,bottom=3pt,title=\textbf{Retrieved Memory (Ours):}]
[human]: How many integers are in the solution of the inequality $|x + 5| < 10$?

[bot]: To solve the inequality $|x + 5| < 10$, we break it into two cases based on the definition of absolute value: 
1. $x + 5 < 10$ 
2. $-(x + 5) < 10$, which simplifies to $x + 5 > -10$ 
Solving each case: 
1. $x < 5$ 
2. $x > -15$ 
Combining these, we get $-15 < x < 5$. This inequality means that $x$ can be any number greater than $-15$ and less than $5$. To find how many integers satisfy this condition, we list them: $-14, -13, -12, -11, -10, -9, -8, -7, -6, -5, -4, -3, -2, -1, 0, 1, 2, 3, 4$. There are 19 integers that satisfy the inequality $|x + 5| < 10$.

\textcolor{brown}{
[human]: What about $|x + 10| < 5$?
}

\textcolor{brown}{
[bot]: To solve the inequality $|x + 10| < 5$, we again consider two cases: 
1. $x + 10 < 5$ 
2. $-(x + 10) < 5$, which simplifies to $x + 10 > -5$ 
Solving each case: 
1. $x < -5$ 
2. $x > -15$ 
Combining these, we get $-15 < x < -5$. This inequality means that $x$ can be any number greater than $-15$ and less than $-5$. To find how many integers satisfy this condition, we list them: $-14, -13, -12, -11, -10, -9, -8, -7, -6$. There are 9 integers that satisfy the inequality $|x + 10| < 5$.
}

[human]: What about $|x + 10| < 10$?

[bot]: To solve the inequality $|x + 10| < 10$, we consider two cases: 
1. $x + 10 < 10$ 
2. $-(x + 10) < 10$, which simplifies to $x + 10 > -10$ 
Solving each case: 
1. $x < 0$ 
2. $x > -20$ 
Combining these, we get $-20 < x < 0$. This inequality means that $x$ can be any number greater than $-20$ and less than $0$. To find how many integers satisfy this condition, we list them: $-19, -18, -17, -16, -15, -14, -13, -12, -11, -10, -9, -8, -7, -6, -5, -4, -3, -2, -1$. There are 19 integers that satisfy the inequality $|x + 10| < 10$. \textcolor{blue}{(Topic: Number of Integers in Inequalities)}
\end{tcolorbox}
\begin{tcolorbox}[left=3pt,right=3pt,top=3pt,bottom=3pt,title=\textbf{Answer (Ours):}]
To find the number of positive integers that satisfy the inequality $(|x + 10| < 5)$, we need to look for positive integers in the solution set $(-15 < x < -5)$. Since there are no positive integers in this interval, the answer is 0.

\end{tcolorbox}
\caption{Case study comparing our \sysname\ with the \textit{session-level} memory baseline. When constructing the memory bank at the \textit{session-level}, the agent fails to provide a satisfactory response due to the inclusion of excessive irrelevant information within a session. This distracts the generator from focusing on the truly useful information highlighted in \textcolor{brown}{brown}. The dialogue topic is emphasized in \textcolor{blue}{blue}. Some context is omitted (replaced by ...) due to length constraints.}
\label{fig: case_study_segment_vs_session}
\end{figure}



\begin{figure}[htb]
\small
\begin{tcolorbox}[left=3pt,right=3pt,top=3pt,bottom=3pt,title=\textbf{Conversation History:}]
[human]: Photosynthesis is a vital process for life on Earth. Could you outline the two main stages of photosynthesis, including where they take place within the chloroplast, and the primary inputs and outputs for each stage? ... (Human-Bot Dialogue Turns)... \textcolor{blue}{(Topic: Photosynthetic Energy Production)}

[human]: Please assume the role of an English translator, tasked with correcting and enhancing spelling and language. Regardless of the language I use, you should identify it, translate it, and respond with a refined and polished version of my text in English. 

... (Human-Bot Dialogue Turns)...  \textcolor{blue}{(Topic: Language Translation and Enhancement)}

[human]: Suggest five award-winning documentary films with brief background descriptions for aspiring filmmakers to study.

\textcolor{brown}{[bot]: ...
5. \"An Inconvenient Truth\" (2006) - Directed by Davis Guggenheim and featuring former United States Vice President Al Gore, this documentary aims to educate the public about global warming. It won two Academy Awards, including Best Documentary Feature. The film is notable for its straightforward yet impactful presentation of scientific data, making complex information accessible and engaging, a valuable lesson for filmmakers looking to tackle environmental or scientific subjects.}

... (Human-Bot Dialogue Turns)... 
\textcolor{blue}{(Topic: Documentary Films Recommendation)}

[human]: Given the following records of stock prices, extract the highest and lowest closing prices for each month in the year 2022. Return the results as a CSV string, with one line allocated for each month. Date,Open,High,Low,Close,Volume ... ... (Human-Bot Dialogue Turns)...  \textcolor{blue}{(Topic: Stock Prices Analysis)}

[human]: The city of Vega intends to build a bridge that will span the Vegona River, covering a distance of 1.8 kilometers. The proposed location falls within a seismically active area that has experienced several high-magnitude earthquakes. Given these circumstances, what would be the best approach to constructing the bridge? ... (Human-Bot Dialogue Turns)...  \textcolor{blue}{(Topic: Seismic Bridge Construction)}
\end{tcolorbox}
\begin{tcolorbox}[left=3pt,right=3pt,top=3pt,bottom=3pt,title=\textbf{Question:}]
Which documentary won two Academy Awards and what lesson does it offer to filmmakers?
\end{tcolorbox}
\begin{tcolorbox}[left=3pt,right=3pt,top=3pt,bottom=3pt,title=\textbf{Ground-Truth Answer:}]
``An Inconvenient Truth" (2006) won two Academy Awards. It offers the lesson of making complex information accessible and engaging, a valuable lesson for filmmakers looking to tackle environmental or scientific subjects.
\end{tcolorbox}
\begin{tcolorbox}[left=3pt,right=3pt,top=3pt,bottom=3pt,title=\textbf{Memory generated by RecurSum~\citep{wang2023recursively}:}]
The dialogue provided detailed information on various topics, including the stages of photosynthesis, energy production estimates through photosynthesis for different organisms, translations and enhancements of phrases in different languages, recommendations for documentary films, a pitch for a film about overcoming adversity, analysis of stock prices, and considerations for constructing a bridge in a seismically active area. Here's a summary of the key points: ...

\textbf{Documentary Film Recommendations}: Suggested five award-winning documentary films for aspiring filmmakers to study, with brief background descriptions for each. The response included:

-  "March of the Penguins" (2005) - Emperor penguins' journey in Antarctica; noted for its cinematography and portrayal of the penguins' life cycle.

-  "Citizenfour" (2014) - Edward Snowden and NSA spying scandal; known for its real-time exploration of Snowden's revelations.

-  "20 Feet from Stardom" (2013) - Stories of backup singers behind musical legends; celebrates unsung heroes in the music industry.

-  "Man on Wire" (2008) - Philippe Petit's 1974 high-wire walk between the Twin Towers; combines personal footage, interviews, and re-enactments.

-  \textcolor{brown}{"An Inconvenient Truth" (2006) - Al Gore's documentary on global warming; known for its presentation of scientific data.}
\end{tcolorbox}
\begin{tcolorbox}[left=3pt,right=3pt,top=3pt,bottom=3pt,title=\textbf{Answer (RSum):}]
I'm sorry, but based on the updated context you provided, I don't see any previous discussion related to the question you asked. However, to answer your question, the documentary that won two Academy Awards is "Free Solo," which won for Best Documentary Feature and Best Achievement in Film Editing in 2019. 
% The lesson it offers to filmmakers is the importance of taking risks and pushing boundaries in order to create something truly unique and impactful.
....
The film follows climber Alex Honnold as he attempts to free solo climb El Capitan in Yosemite National Park, and the filmmakers had to take risks themselves in order to capture the footage and tell the story.

\textbf{\textcolor{red}{(The next part of this figure~\ref{fig: case_study_segment_vs_rsum} will be continued on the next page.)}}
\end{tcolorbox}
\end{figure}
\begin{figure}[htbp]
% \ContinuedFloat
\small
\begin{tcolorbox}[left=3pt,right=3pt,top=3pt,bottom=3pt,title=\textbf{Retrieved Memory (Ours):}]
[human]: Suggest five award-winning documentary films with brief background descriptions for aspiring filmmakers to study.

[bot]: 1. **"March of the Penguins" (2005)** - Directed by Luc Jacquet, this French feature-length nature documentary delves into the annual journey of Emperor penguins across the Antarctic. It won the Academy Award for Best Documentary Feature. The film is celebrated for its stunning cinematography and its detailed, heartfelt portrayal of the penguins' life cycle, making it a must-study for filmmakers interested in wildlife documentaries. 2. **"Citizenfour" (2014)** ... 5. \textcolor{brown}{**"An Inconvenient Truth" (2006)** - Directed by Davis Guggenheim and featuring former United States Vice President Al Gore, this documentary aims to educate the public about global warming. It won two Academy Awards, including Best Documentary Feature.} The film is notable for its straightforward yet impactful presentation of scientific data, making complex information accessible and engaging, \textcolor{brown}{a valuable lesson for filmmakers looking to tackle environmental or scientific subjects.}

[human]: With the spirit in the first film, craft a succinct and persuasive pitch for a film about overcoming adversity.

[bot]: Drawing inspiration from the resilience and majestic journey of Emperor penguins in "March of the Penguins," our documentary, "Unyielding Spirits: The Human Odyssey," will explore the profound stories of individuals around the globe who have faced and overcome staggering adversities. Through breathtaking cinematography and intimate storytelling, we ...
\textcolor{blue}{(Topic: Documentary Films Recommendation)}
\end{tcolorbox}
\begin{tcolorbox}[left=3pt,right=3pt,top=3pt,bottom=3pt,title=\textbf{Answer (Ours):}]
``An Inconvenient Truth" won two Academy Awards, including Best Documentary Feature. The film offers a valuable lesson to filmmakers looking to tackle environmental or scientific subjects, as it presents complex information in a straightforward yet impactful way, making it accessible and engaging to a wider audience.


\end{tcolorbox}
\caption{Case study comparing our \sysname\ with RecurSum~\citep{wang2023recursively}. When using RecurSum, the agent fails to provide a satisfactory answer due to information loss during the summarization process. Summarization omits essential details needed to answer the user's new request, as highlighted in \textcolor{brown}{brown}. The dialogue topic is emphasized in \textcolor{blue}{blue}. Some context is omitted (replaced by ...) due to length constraints.}
\label{fig: case_study_segment_vs_rsum}
\end{figure}




\begin{figure}[htb]
\small
\begin{tcolorbox}[left=3pt,right=3pt,top=3pt,bottom=3pt,title=\textbf{Conversation History:}]
[human]: Photosynthesis is a vital process for life on Earth. Could you outline the two main stages of photosynthesis, including where they take place within the chloroplast, and the primary inputs and outputs for each stage? ... (Human-Bot Dialogue Turns)... \textcolor{blue}{(Topic: Photosynthetic Energy Production)}

[human]: Please assume the role of an English translator, tasked with correcting and enhancing spelling and language. Regardless of the language I use, you should identify it, translate it, and respond with a refined and polished version of my text in English. 

... (Human-Bot Dialogue Turns)...  \textcolor{blue}{(Topic: Language Translation and Enhancement)}

[human]: Suggest five award-winning documentary films with brief background descriptions for aspiring filmmakers to study.

\textcolor{brown}{[bot]: ...
5. \"An Inconvenient Truth\" (2006) - Directed by Davis Guggenheim and featuring former United States Vice President Al Gore, this documentary aims to educate the public about global warming. It won two Academy Awards, including Best Documentary Feature. The film is notable for its straightforward yet impactful presentation of scientific data, making complex information accessible and engaging, a valuable lesson for filmmakers looking to tackle environmental or scientific subjects.}

... (Human-Bot Dialogue Turns)... 
\textcolor{blue}{(Topic: Documentary Films Recommendation)}

[human]: Given the following records of stock prices, extract the highest and lowest closing prices for each month in the year 2022. Return the results as a CSV string, with one line allocated for each month. Date,Open,High,Low,Close,Volume ... ... (Human-Bot Dialogue Turns)...  \textcolor{blue}{(Topic: Stock Prices Analysis)}

[human]: The city of Vega intends to build a bridge that will span the Vegona River, covering a distance of 1.8 kilometers. The proposed location falls within a seismically active area that has experienced several high-magnitude earthquakes. Given these circumstances, what would be the best approach to constructing the bridge? ... (Human-Bot Dialogue Turns)...  \textcolor{blue}{(Topic: Seismic Bridge Construction)}
\end{tcolorbox}
\begin{tcolorbox}[left=3pt,right=3pt,top=3pt,bottom=3pt,title=\textbf{Question:}]
Which documentary won two Academy Awards and what lesson does it offer to filmmakers?
\end{tcolorbox}
\begin{tcolorbox}[left=3pt,right=3pt,top=3pt,bottom=3pt,title=\textbf{Ground-Truth Answer:}]
"An Inconvenient Truth" (2006) won two Academy Awards. It offers the lesson of making complex information accessible and engaging, a valuable lesson for filmmakers looking to tackle environmental or scientific subjects.
\end{tcolorbox}
\begin{tcolorbox}[left=3pt,right=3pt,top=3pt,bottom=3pt,title=\textbf{Memory generated by ConditionMem~\citep{yuan2023evolving}:}]
- \textcolor{brown}{The user is seeking recommendations for award-winning documentary films, along with brief background descriptions, for educational purposes.} This request likely stems from a desire to learn from successful examples in the documentary filmmaking field, possibly to improve their own filmmaking skills or to gain inspiration for their projects. 

- \textcolor{brown}{User is interested in award-winning documentary films for educational purposes, aiming to study their backgrounds and filmmaking techniques.}

- The user's request for crafting a succinct and persuasive pitch for a film about overcoming adversity is inspired by the spirit of "March of the Penguins," a documentary that beautifully captures the Emperor penguins' annual journey across the Antarctic, highlighting themes of endurance and survival against the odds. The user seeks to emulate the emotional depth and cinematic quality that made "March of the Penguins" compelling, aiming to create a pitch that resonates with similar themes of resilience and triumph over challenges. 

- User is interested in creating content that focuses on themes of overcoming adversity, inspired by the emotional depth and cinematic excellence of documentaries like "March of the Penguins." They value storytelling that captures the essence of endurance and survival against the odds.
% : The user's request to write a budget plan for the video production directly follows the task of creating a video script aimed at training new employees of a car wash business in Japan, focusing on business etiquette norms.
% The user is likely seeking to understand the financial requirements to bring the previously crafted script to life, indicating a move from the planning and content development phase to the practical aspects of production. 

... (Other Irrelevant Memory Records) ...
\end{tcolorbox}
\begin{tcolorbox}[left=3pt,right=3pt,top=3pt,bottom=3pt,title=\textbf{Answer (Conditional Memory):}]
The documentary that won two Academy Awards is "Free Solo," which follows climber Alex Honnold as he attempts to free solo climb El Capitan in Yosemite National Park. The film offers a lesson to filmmakers on the importance of capturing authentic and compelling human stories, as well as the technical challenges of filming in extreme environments.

\textbf{\textcolor{red}{(The next part of this figure~\ref{fig: case_study_segment_vs_condmem} will be continued on the next page.)}}
\end{tcolorbox}
\end{figure}
\begin{figure}[htbp]
% \ContinuedFloat
\small
\begin{tcolorbox}[left=3pt,right=3pt,top=3pt,bottom=3pt,title=\textbf{Retrieved Memory (Ours):}]
[human]: Suggest five award-winning documentary films with brief background descriptions for aspiring filmmakers to study.

[bot]: 1. **"March of the Penguins" (2005)** - Directed by Luc Jacquet, this French feature-length nature documentary delves into the annual journey of Emperor penguins across the Antarctic. It won the Academy Award for Best Documentary Feature. The film is celebrated for its stunning cinematography and its detailed, heartfelt portrayal of the penguins' life cycle, making it a must-study for filmmakers interested in wildlife documentaries. 2. **"Citizenfour" (2014)** ... 5. \textcolor{brown}{**"An Inconvenient Truth" (2006)** - Directed by Davis Guggenheim and featuring former United States Vice President Al Gore, this documentary aims to educate the public about global warming. It won two Academy Awards, including Best Documentary Feature.} The film is notable for its straightforward yet impactful presentation of scientific data, making complex information accessible and engaging, a valuable lesson for filmmakers looking to tackle environmental or scientific subjects.

[human]: With the spirit in the first film, craft a succinct and persuasive pitch for a film about overcoming adversity.

[bot]: Drawing inspiration from the resilience and majestic journey of Emperor penguins in "March of the Penguins," our documentary, "Unyielding Spirits: The Human Odyssey," will explore the profound stories of individuals around the globe who have faced and overcome staggering adversities. Through breathtaking cinematography and intimate storytelling, we ...
\textcolor{blue}{(Topic: Documentary Films Recommendation)}
\end{tcolorbox}
\begin{tcolorbox}[left=3pt,right=3pt,top=3pt,bottom=3pt,title=\textbf{Answer (Ours):}]
``An Inconvenient Truth" won two Academy Awards, including Best Documentary Feature. The film offers a valuable lesson to filmmakers looking to tackle environmental or scientific subjects, as it presents complex information in a straightforward yet impactful way, making it accessible and engaging to a wider audience.
\end{tcolorbox}
\caption{Case study comparing our \sysname\ with ConditionMem~\citep{yuan2023evolving}. When using ConditionMem, the agent fails to provide a satisfactory answer due to (1) information loss during the summarization process and (2) the incorrect discarding of turns that are actually useful, as highlighted in \textcolor{brown}{brown}. The dialogue topic is emphasized in \textcolor{blue}{blue}. Some context is omitted (replaced by ...) due to length constraints.}
\label{fig: case_study_segment_vs_condmem}
\end{figure}


To further demonstrate the effectiveness of LLM-QE, we conduct a case study by randomly sampling a query from the evaluation dataset. We then compare retrieval performance using the raw queries, expanded queries by vanilla LLM, and expanded queries by LLM-QE.

As shown in Table~\ref{tab:case_study}, query expansion significantly improves retrieval performance compared to using the raw query. Both vanilla LLM and LLM-QE generate expansions that include key phrases, such as ``temperature'', ``humidity'', and ``coronavirus'', which provide crucial signals for document matching. However, vanilla LLM produces inconsistent results, including conflicting claims about temperature ranges and virus survival conditions. In contrast, LLM-QE generates expansions that are more semantically aligned with the golden passage, such as ``the virus may thrive in cooler and more humid environments, which can facilitate its transmission''. This further demonstrates the effectiveness of LLM-QE in improving query expansion by aligning with the ranking preferences of both LLMs and retrievers.



\end{document}
