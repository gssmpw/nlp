\section{List of Evaluation Prompt Templates}
\label{appendix:prompt_template}
This section lists all prompt templates applied throughout this study, including the prompt templates utilized to generate the final rating (Table \ref{tab:evaluation_prompt_summeval}, \ref{tab:evaluation_prompt_topical_chat}, \ref{tab:evaluation_prompt_sfhot}, \ref{tab:evaluation_prompt_hanna}) and the templates used to generate Reference, AutoCoT, and Metrics (Table \ref{tab:generation_prompt_summeval}, \ref{tab:generation_prompt_topical_chat}, \ref{tab:generation_prompt_sfhot}, \ref{tab:generation_prompt_hanna}). 
For these prompt templates, 
we generally refer to the reference-free single answer pointwise grading prompt from MT-Bench \cite{zheng2023judging}. 
However, due to the contents of some components of this template 
being mixed, we make minor adjustments to the order of some sentences to ensure that the search for factor Order can be conducted.
We use the prompt template of LLMBar \cite{zeng2024llmbar} to generate Metrics.


\begin{figure*}[!t]
  \centering
  \includegraphics[width=0.95\textwidth]{figures/init.pdf}
  \vspace{-2mm}
  \caption{Average dataset-level Spearman human correlation on Topical-Chat for GPT-4o-mini evaluator using different prompting strategies, which are modified based on the baseline strategy from MT-Bench for each factor.}
  \vspace{-4mm}
  \label{fig:init}
\end{figure*}

\section{Preliminary Experiment on Factor Effects}
\label{appendix:preliminary_experiment}
% As shown in Table \ref{tab:factors}, although previous works reach relatively consistent conclusions regarding the effects of some factors, there remains controversy regarding some other factors. 
To further explore the effect of each factor on the performance of LLM evaluators, we conducted a preliminary experiment with GPT-4o-mini on a commonly-used dialogue evaluation dataset Topical-Chat \cite{gopalakrishnan2019topical}. 
Using the prompting strategy from MT-Bench \cite{zheng2023judging} with the default scoring scale 1-3 of Topical-Chat as the baseline, we adjust the selection value of each factor separately, employ greedy search decoding  
% \hnote{it means greedy decoding?} 
to generate rating scores, and report the Spearman correlations with human judgments.
The results, as shown in Figure \ref{fig:init}, indicate that these factors significantly influence the performance of LLM evaluators, yet some findings diverge from previous works. 
For instance, \citet{pereira2024check} claims that Metrics can improve the performance of LLM evaluators. 
However, we observed a performance decrease on Topical-Chat. 
\citet{chiang-lee-2023-closer} finds that CoT plays an important role in the evaluation prompt. 
However, when it comes to the GPT-4o-mini evaluator, adding CoT results in negligible differences.
% generating explanations can enhance the performance of LLM evaluators.
% \citet{stureborg2024large} find that a moderate scoring score 1-10 achieves the best performance. However, 1-50 performs better in our experiment.
These results highlight the importance of optimizing the prompting strategy for adjusting these factors. 
% (2) The effect of the same factor across different aspects shows considerable variability, confirming the necessity of adaptive adjustments to the prompt template for different aspects.

\section{Details of In-Context Example Selection}
\label{appendix:in_context_example}
We perform stratified sampling based on human ratings within the validation dataset to obtain in-context examples, aiming to ensure an even distribution of examples across different human ratings. 
When evaluating the performance of some prompting strategies on the validation dataset, we remove the corresponding example in the in-context examples set if this example is to be evaluated, aiming to prevent data leakage.


\section{Details of Benchmarks}
\label{appendix:benchmarks}
A brief introduction of the meta-evaluation benchmarks involved is as follows:
\begin{itemize}
\item  \textbf{Summeval} \cite{fabbri-etal-2021-summeval} is a meta-evaluation benchmark for summarization. 
It contains human evaluation annotations for 16 summarization systems on 100 articles from the CNN / DailyMail corpus, resulting in a total of 1600 summary-level annotations. 
Each summary is evaluated on four aspects: \textit{Coherence}, \textit{Consistency}, \textit{Fluency}, and \textit{Relevance}. 
The authors recruit annotators on Amazon Mechanical Turk (AMT) to rate each summary on a scale from 1 to 5. 
Cross-validation by other annotators and experts is conducted to correct errors and enhance annotation quality.
\item \textbf{Topical-Chat} \cite{gopalakrishnan2019topical} is a meta-evaluation benchmark for knowledge-grounded dialogue generation. 
It contains 360 samples, each including dialogue context, relevant knowledge, a response, and human ratings of the response across five aspects: \textit{Coherence}, \textit{Engagingness}, \textit{Groundedness}, \textit{Naturalness}, and \textit{Understandability}, ranging from 1 to 3. 
The annotators recruited from AMT provide the ratings. 
Following HD-Eval \cite{liu-etal-2024-hd}, the first four aspects are used to measure the performance of HPSS.
\item \textbf{SFHOT/ SFRES} \cite{wen-etal-2015-semantically} are meta-evaluation benchmarks for data-to-text generation. 
They contain 875 / 1181 samples respectively, which provide information about restaurants and hotels in San Francisco and aim to let the model generate corresponding utterances. 
The authors recruit annotators from AMT to rate the \textit{Informativeness} and \textit{Naturalness} of the generated utterances for each sample on a scale from 1 to 6.
\item \textbf{HANNA} \cite{chhun-etal-2022-human} serves as a meta-evaluation benchmark for story generation. 
It contains 1,056 stories produced by 10 different automatic story generation systems. 
Each story is rated by 3 annotators recruited from Amazon Mechanical Turk on 6 aspects: \textit{Coherence}, \textit{Relevance}, \textit{Empathy}, \textit{Surprise}, \textit{Engagement}, and \textit{Complexity}. 
Ratings range from 1 to 5. The final score for each aspect is the average of the three annotators' ratings.
\item \textbf{MT-Bench} \cite{zheng2023judging} comprises 3.3k expert-level pairwise human evaluation of responses,  generated by six LLMs on 80 carefully designed questions. 
These questions cover 7 categories: \textit{Writing}, \textit{Roleplay},
\textit{Reasoning Math}, 
\textit{Coding}, \textit{Extraction}, \textit{STEM} and \textit{Humanities}.
We select the first round of dialogues from this dataset and filter out the tied cases, leaving a final evaluation dataset of 1020 instances.

\item \textbf{AUTO-J (Eval-P)} \cite{li2024generative} provides 1,392 pairwise comparison data, each of which contains a query, two LLM-generated responses, and a human-annotated preference label. 
This dataset involves 58 real-world scenarios and the responses are generated from 6 LLM families. 
We filter out the tied cases and leave a final evaluation dataset of 1019 instances.

\item  \textbf{LLMBar} \cite{zeng2024llmbar} is a meta-evaluation benchmark for instruction-following, which consists of two components:
(1) The Natural set, which is gathered from existing human-preference datasets. 
(2) The Adversarial set, where the authors intentionally create misleading outputs that appear plausible but deviate from the instructions to challenge the evaluators.
This dataset contains a total of 419 pairwise comparison instances.

\end{itemize}

\section{Metric Calculation}
\label{appendix:correlation}
\subsection{Pointwise Grading}
Following previous work \cite{liu-etal-2024-hd, liu-etal-2024-calibrating, zhong-etal-2022-towards}, we adopt dataset-level (sample-level for Summeval as an exception) Spearman ($\rho$) correlation coefficient between human judgments and LLM evaluations to measure the performance of LLM evaluators. 
Given a dataset $\mathcal{D}$, evaluation aspect $a$ and evaluation metric $f(\cdot)$, we could calculate the human correlation of this evaluation metric at either dataset or sample level:
\begin{itemize}
\item \textbf{Dataset Level}: For dataset-level human correlation, we evaluate the correlations on all samples in the dataset, as follows:
\begin{equation}
\small
\begin{aligned}
corr_{dataset}(\{s^*_{i,a}\}_{i=1}^{|D|}, \{s_{i,a}\}_{i=1}^{|D|}) = & \\
\rho([{s}^*_{i,a}, ..., {s}^*_{|D|,a}], [s_{i,a}, ..., s_{|D|,a}])
\end{aligned}
\end{equation}
where $\{s^*_{i,a}\}_{i=1}^{|D|}$ and $\{s_{i,a}\}_{i=1}^{|D|}$ denote the 
evaluation results (for free-text evaluations, scores are extracted via rules as final evaluation results) for the aspect $a$ of dataset $\mathcal{D}$ from human annotations and evaluation metric $f(\cdot)$, respectively. 
\item \textbf{Sample Level}: Assume that the dataset $\mathcal{D}$ consists of $J$ queries where each query has target responses from $M$ diverse systems (with a total of $|D| = M \times J$ samples),
and for sample-level human correlation, we first compute correlations on multiple responses to an individual query (e.g., the summaries from 16 summarization systems on one article for Summeval), then average them across all queries:
\begin{equation}
\small
\begin{aligned}
corr_{sample}(\{s^*_{i,a}\}_{i=1}^{|D|}, \{s_{i,a}\}_{i=1}^{|D|}) = & \\
\frac{1}{J} \sum_{i=1}^{J}(\rho([{s}^*_{i1,a}, ..., {s}^*_{iM,a}], [s_{i1,a}, ..., s_{iM,a}]))
\end{aligned}
\end{equation}
where $s^*_{ij,a}$ and $s_{ij,a}$ denote the evaluation results for the $j$-th response to $i$-th query 
for the aspect $a$ of dataset $\mathcal{D}$, from human annotations and evaluation metric $f(\cdot)$, respectively.
\end{itemize}

The evaluation metric $f(\cdot)$ is the LLM evaluator using a specific prompt template in our implementation, and the calculation for the Spearman correlation coefficient $\rho$ between two vectors of length $n$ is as follows:
\begin{equation}
\small
\begin{aligned}
\rho = 1 - \frac{6 \sum d_i^2}{n(n^2 - 1)}
\end{aligned}
\end{equation}
where $d_i$ represents the difference in the rank of the $i$-th element between two vectors, where the ranks are determined by sorting the elements within their respective vectors in ascending order.

\subsection{Pairwise Comparison}
Following previous work \cite{zeng2024llmbar, lambert2024rewardbench}, we directly adopt accuracy to measure the performance of LLM evaluators in pairwise comparison. Given a dataset $\mathcal{D}$, evaluation aspect $a$ and evaluation metric $f(\cdot)$, the accuracy is calculated as follows:
\begin{equation}
\small
\begin{aligned}
acc(\{s^*_{i,a}\}_{i=1}^{|D|}, \{s_{i,a}\}_{i=1}^{|D|}) = 
\frac{1}{|D|}\mathbb{I}(s^*_{i,a}=s_{i,a}) 
\end{aligned}
\end{equation}


\section{Implementation Details of Baselines}
\label{appendix:baselines}
As for APE, we use the LLM evaluator to resample new prompts. 
The queue size is set to 5. 
In each iteration, two new prompt candidates are resampled based on each prompt in the queue. 
As for OPRO, we use the LLM evaluator to generate new prompting strategies.
We provide the LLM with the selection range for each factor, as well as the 20 previously top-performing strategies and their corresponding correlation metrics, which serve as the search history.  
The LLM is asked to generate a list containing the new selection strategy for each factor. 
The explored prompting strategies in \textbf{Initiation} will serve as the initial search history. 
As for Greedy, we perturb the current prompting strategy 5 times in each iteration by randomly replacing the value of one factor to generate new strategies. 
The strategy that performs best on the validation dataset is retained for the next iteration.
Finally, as for Stepwise-Greedy, we follow the order shown in Table \ref{tab:factors} to optimize each factor sequentially. 
In each step, we select the value for the current factor that performs best on the validation dataset while holding all other factors fixed.
This selected choice is then established as the final optimization result for the current factor.



\section{Implementation Details of HPSS}
\label{appendix:hpss}
We determine the hyperparameters for HPSS via grid search on the validation dataset of Topical-Chat using the Qwen2.5-14B-Instruct evaluator. Specifically, the population size $k$ is set to 5.
The mutation time for each template $g = 2$.  The exploitation probability $\rho = 0.2$.
The temperature $\tau$ for the softmax function used to calculate the exploration probability of each template is set to 5, and the weight $\lambda$ for the additional exploration term is set to 4. 
We provide the performance of Qwen2.5-14B-Instruct after modifying each hyperparameter choice on the validation dataset of Topical-Chat in Figure \ref{fig:hyper}.
% For this model, We utilize vllm \cite{10.1145/3600006.3613165} framework for inference on 4 H800 GPUs.
% The total runtime for HPSS across all datasets is approximately 12 hours.

Under the computational budget described in Section \ref{sec:experiments} (i.e., 71), the search cost of HPSS (i.e., the inference times of the LLM evaluator) for a specific evaluation aspect of one dataset is approximately 70 times the size of the validation dataset. 
When the size of the validation dataset is 50\% of the entire dataset, for GPT-4o-mini, the cost of HPSS for a specific evaluation aspect of one dataset is approximately \$8. 
In total, the overall cost of HPSS across all the datasets and evaluation aspects is approximately \$140. 
For Qwen2.5-14B-Instruct, the runtime of HPSS for a specific evaluation aspect of one dataset is approximately 40 minutes using 4 H100 GPUs and the vllm \cite{10.1145/3600006.3613165} inference framework, while the overall runtime of HPSS across all the datasets and aspects is approximately 12 hours. 
The overall costs for other automatic prompt optimization methods for LLM evaluators are the same as HPSS, which is affordable in the vast majority of scenarios.
We also validate that even with 1/5 of the above search cost (reducing the validation dataset size to 10\% of the entire dataset), HPSS can still significantly outperform human-designed LLM evaluators in Figure \ref{fig:size}.



\section{Details of Algorithm Implementation}
 \label{appendix:algorithm}
We provide the detailed implementation of the two steps of HPSS in Algorithm \ref{algorithm:initiation} and \ref{algorithm:search} respectively.



\section{Experiments on Pairwise Comparison}
\label{appendix:pairwise}
\subsection{Experimental Setup}
Apart from pointwise grading tasks, we also validate our method on three pairwise comparison benchmarks, i.e., MT-Bench \cite{zheng2023judging}, AUTO-J \cite{li2024generative}, and LLMBar \cite{zeng2024llmbar}, which primarily focus on instruction-following tasks.
Qwen2.5-14B-Instruct is employed as the evaluation model, with all hyperparameters remaining the same as pointwise grading experiments. 
Regarding the search space, we remove the factor Scoring Scale, and the value \textit{Self-Generated Criteria} of the factor Evaluation Criteria, which only exist in the pointwise grading setting. 
We compare our method with the prompting strategies from MT-Bench, the best human-designed prompting strategies \textit{Metrics+Reference$^*$} found by LLMBar, and all automatic prompt optimization methods examined in pointwise grading experiments.
Results of prompt search are averaged over 3 random seeds and the standard deviation is provided.
\subsection{Main Results}
The main results are provided in Table \ref{tab:pairwise_results}.
HPSS substantially improves the performance of LLM evaluators compared to human-designed prompting strategies and achieves the best average performance across all automatic prompt optimization methods, which validates its effectiveness in pairwise comparison prompt optimization.
On the AUTO-J (Eval-P), multiple prompt optimization methods simultaneously achieve the best results.
Upon data examination, we find that %the reason is that 
the best prompting strategy is close to the starting point, with only one factor having a different value, resulting in lower search difficulty. 
Overall, we observe that human-designed prompting strategies for pairwise comparison tasks have already been well-optimized, and the improvements brought by automatic prompt optimization are relatively modest compared to those in pointwise grading tasks.

\begin{table} [t]
\centering
\resizebox{\linewidth}{!} {
\begin{tabular}{l|c|c|c|c}
\toprule
\multirow{2}{*}{\textbf{Method}}  & \textbf{AUTO-J}   & \multirow{2}{*}{\textbf{LLMBar}}  & \multirow{2}{*}{\textbf{MT-Bench}} & \multirow{2}{*}{\textbf{Avg.}} \\
 & \textbf{(Eval-P)} & & & \\
\midrule
 MT-Bench  & 0.792 & 0.619 & 0.765 &  0.725 \\
 Metrics+Reference*  & 0.800 & 0.724 & 0.778 & 0.767\\
\cmidrule{1-5}
APE & 0.799\scriptnumber{0.005} & 0.670\scriptnumber{0.032} & 0.774\scriptnumber{0.008} & 0.748 \\
OPRO & \textbf{0.847}\scriptnumber{0.000} & 0.695\scriptnumber{0.000} & \textbf{0.791}\scriptnumber{0.005} & 0.778 \\
Greedy & 0.820\scriptnumber{0.019} &  0.774\scriptnumber{0.005} & 0.775\scriptnumber{0.013} & 0.790 \\
Stepwise-Greedy  & \textbf{0.847}\scriptnumber{0.000} & 0.743\scriptnumber{0.000} & 0.784\scriptnumber{0.000} & 0.791 \\
HPSS (Ours) & \textbf{0.847}\scriptnumber{0.000} & \textbf{0.778}\scriptnumber{0.005} & 0.789\scriptnumber{0.015} & \textbf{0.805} \\
\bottomrule
\end{tabular}
}
\vspace{-2mm}
\caption{Accuracy of different prompting methods based on Qwen-2.5-14B-Instruct on pairwise comparison datasets including AUTO-J (Eval-P), LLMBar, and MT-Bench.}
\label{tab:pairwise_results}
\vspace{-4mm}
\end{table}
\begin{table*} [!t]
\centering
\small
\begin{tabular}{c|l|c|c|c|c}
\toprule
\textbf{Model} & \textbf{Method}  & \textbf{Summeval} & \textbf{Topical-Chat} & \textbf{SFHOT} & \textbf{HANNA}  \\
\midrule
\multirow{3}{*}{Qwen-2.5-72B-Instruct} & MT-Bench  & 0.502  &  0.634  & 0.362  &  0.460   \\
& CloseLook + ICL  &  0.481  & 0.592 & 0.316 & 0.470     \\
& CloseLook + ICL$^{\dag}$  & 0.525 &  0.623 & 0.300  & 0.517  \\
\midrule
Qwen-2.5-14B-Instruct & HPSS & \textbf{0.560}  &  \textbf{0.697}  &  \textbf{0.431} & \textbf{0.519}   \\
\bottomrule
\end{tabular}
\vspace{-2mm}
\caption{Average performance comparison of Qwen2.5-72B-Instruct evaluator and Qwen2.5-14B-Instruct evaluator on Summeval, Topical-Chat, SFHOT, and HANNA. \dag \  indicates that the corresponding method employs 20 generations with self-consistency.}
\label{tab:stronger_model}
\end{table*}
\section{Fundamental Analysis of Self-Consistency}

In this section, we first present a distribution alignment perspective on how self-consistency works with specific true answer distributions, supported by experimental evidence to substantiate this viewpoint. Building upon this foundation, we proceed to provide both a formal definition of self-consistency convergence and practical criteria for assessment. 
\subsection{Why Self-Consistency Works: A Distributional Perspective}

Self-Consistency is a widely-used decoding method for improving reasoning performance by aggregating multiple stochastic samples. 
By applying a majority voting scheme, it mitigates issues such as local optima and high variance that arise from relying on a single sample. Formally, it can be expressed as:

\begin{equation}
    \hat{y}_{SC} = \arg\max_y \left( \frac{1}{n} \sum_{i=1}^{n} \mathbb{I}(y_i = y) \right)
\end{equation}
where $y_i$ is the $i$-th sampled answer, and $\mathbb{I}(y_i = y)$ is the indicator function that equals 1 if $y_i$ matches the candidate answer $y$, and 0 otherwise. The result, $\hat{y}_{SC}$, is the answer with the highest number of votes (the top-1 answer).

From a probabilistic perspective, self-consistency can be seen as a \textit{Monte Carlo estimator} of the true answer distribution $p(y \mid \mathbf{x})$. As the number of samples increases, the empirical distribution formed by the samples approximates the true distribution, and the most frequent answer aligns with the true distribution:
\begin{equation}
\begin{aligned}
\hat{p}_{SC}(y) &= \frac{1}{n} \sum_{i=1}^{n} \mathbb{I}(y_i = y) \\
&\to p(y \mid \mathbf{x}), \quad \text{as} \quad n \to \infty
\end{aligned}
\end{equation}
As the number of samples increases, the estimation becomes more reliable, and the voting mechanism converges towards the true answer.

\paragraph{Experimental Analysis}
\begin{figure}[ht]
\centering
\includegraphics[width=0.9\linewidth]{figs/Top1_Matching_Probability_Accuracy.pdf}
\caption{Top-1 answer matching probability (a) and accuracy (b) both improve as the sampling number increases.}
\label{fig:top1}
\end{figure} 
To validate this viewpoint, we analyzed the top-1 answer match rate as a function of the sample size. The true top-1 answer is simulated by drawing from a large sample to approximate the true distribution. 
Results from Figure~\ref{fig:top1} reveals \textbf{Findings 1}: As the sample size increased, the top-1 answer match rate gradually approaches 100\% with the accuracy consistently improves.
Based the observation, we derive the following insight: 
\textit{\textbf{Insights 1}: The improvement in self-consistency performance stems from the fact that, 
the top-1 answer in the sampling distribution gradually aligns with the true distribution, ultimately enhancing accuracy to match the true distribution's level.}


\subsection{Convergence Analysis of Answer Aggregation}
According to \textit{\textbf{Insights 1}}, since the accuracy of the true distribution is fixed, the performance of self-consistency is guaranteed to converge.
To further investigate it, we provide the following definition according to the Cauchy convergence criterion:
\begin{definition}
Let $f^M(i) = \sum_{l=1}^M \mathbb{I}(\hat{y_l} = i)$, where $\hat{y_l}$ represents the set of answers generated by the model, and $ M $ is the number of samples. For any given $ \epsilon > 0 $, there exists a positive integer $ L $ such that for $ N, M > L $, if the following holds:
\begin{align}
\left|\; \underset{i}{argmax}\; f^M(i) - \underset{i}{argmax}\; f^N(i) \;\right| < \epsilon
\end{align}
we can conclude that self-consistency has converged.
\label{def:sc}
\end{definition}

Based on Definition \ref{def:sc}, we prove that self-consistency also converges in terms of the accuracy on the dataset:

\begin{theorem}
Let $ Acc_{D}^M = \frac{1}{|D|}\sum_{j\in D}\mathbb{I}[\underset{i}{argmax}\; f^M(i)=gt_j] $ denote the accuracy of self-consistency when a single question is sampled $ M $ times on dataset $ D $, where $ gt_j $ represents the correct answer to the $ j $-th question. If Definition 1 holds, then for any given $ \epsilon > 0 $, there exists a positive integer $ L $ such that when $ N, M > L $, the following holds:
\begin{align}
\left| \; Acc_{D}^M - Acc_{D}^N \;\right| < \epsilon
\end{align}
\label{the:sc}
\end{theorem}
The Proof of Theorem~\ref{the:sc} is in Appendix~\ref{app:proof}.
By setting $\epsilon$ to $\frac{1}{|D|}$, the following definition is established:
\begin{definition}
If the following holds on dataset $D$:
\begin{align}
\left| \; Acc_{D}^M - Acc_{D}^{M-5} \;\right| < \frac{1}{|D|}
\end{align}
we can consider self-consistency to have converged at a sample size of $M$.
 
\label{def:sc_D}
\end{definition}
\paragraph{Experimental Analysis}
\begin{figure}[t]
\centering
\includegraphics[width=1.0\linewidth]{figs/convergence_gsm8k.pdf}
\caption{Self-consistency convergence plots under different temperature (0.4 and 0.8) settings.}
\label{fig:convergence}
\end{figure} 
Figure~\ref{fig:convergence} depicts the convergence behavior of various models, with the accuracy curves plotted up to the convergence point according to Definition~\ref{def:sc_D}, from where we can get:
\textbf{Findings 2}: The convergence speed exhibits a positive correlation with accuracy.
\textbf{Findings 3}: The convergence speed is inversely correlated with temperature. 
\textbf{Findings 4}: The final converged accuracy varies across different temperature settings.
Based on them, we derive \textit{\textbf{Insights 2}: Sampling diversity will affect the true distribution, impacting both the convergence accuracy and the convergence speed of self-consistency.}

\section{Diversity Trade-offs for Self-Consistency}
\label{sec:diversity}
\subsection{Sampling Diversity Affection}
According to \textbf{\textit{Insight 2}}, to gain a deeper understanding of the impact of diversity on self-consistency, we investigate how accuracy varies with temperature changes in increments of 0.1. The study is divided into two parts: convergence analysis and finite-sample analysis.
\paragraph{Converge Condition}

\begin{figure}[t]
\centering
\includegraphics[width=0.75\linewidth]{figs/Qwen2.5-7B_gsm8k_inf_tem.pdf}
\caption{The accuracy curve with varying temperature under convergence.}
\label{fig:inf_tem}
\end{figure} 
Figure~\ref{fig:inf_tem} indicates \textbf{Findings 5}: As the temperature increases, the accuracy of single samples exhibits a declining trend, while the accuracy of self-consistency after convergence shows an increasing trend (the optimal point is often near 1.0\footnote{We speculate that this may be related to the training temperature being typically set to 1.0. We leave the study of the optimal temperature as future work.}). Please refer to Appendix~\ref{app:diversity} for more results. The disagreement resolution theorem in ensemble learning provides a potential explanation, suggesting that the overall performance of an ensemble is determined by the trade-off between the accuracy of individual models and the diversity among them. From this trend and \textit{\textbf{Insights 1}}, we gain \textit{\textbf{Insights 3}: When the sample size is sufficient, the temperature should be increased to better explore the true distribution with higher accuracy.}

\paragraph{Finite-Sample Condition}

\begin{figure}[t]
\centering
\includegraphics[width=0.8\linewidth]{figs/Qwen2.5-7B_gsm8k_100_vllm_heatmap.pdf}
\caption{The accuracy heatmap with varying temperature with finite sample size.}
\label{fig:limit_tem}
\end{figure} 
Figure~\ref{fig:limit_tem} indicates \textbf{Findings 6}: When the sample size is limited, the optimal temperature gradually shifts toward lower values as the sample size decreases.
Please refer to Appendix~\ref{app:diversity} for more results.
This findings and \textit{\textbf{Insights 1}} leads us to \textit{\textbf{Insights 4}: Sample size determines the maximum top-1 confidence level that can be reliably modeled. True distributions with lower confidence require larger data volumes to ensure that the sampled top-1 answer aligns with the converged result.}

By combining \textit{\textbf{Insights 3}} and \textit{\textbf{4}}, we can derive \textit{\textbf{Insights 5}: The effectiveness of self-consistency depends on dynamically aligning the confidence of the sampling distribution with the inherent uncertainty of the true answer distribution.}

\subsection{Chain-of-thought Affection}
\begin{figure}[t]
\centering
\includegraphics[width=0.8\linewidth]{figs/cot_fsd_comparison.pdf}
\caption{FSD (Equation~\ref{eq:fsd}) \citep{FSD} is employed as the confidence metric to quantify the gap between top two candidates.}
\label{fig:cot_fsd}
\end{figure} 

Besides the sampling diversity decided by temperature, Chain-of-Thought \citep{COT} is also a key factor. From Figure~\ref{fig:cot_fsd} we can get \textbf{Findings 7}: Using CoT prompt leads to higher confidence compared to not using it.
A deeper \textbf{\textit{Insight 6}} emerges: \textit{Chain-of-thought (CoT) reasoning narrows the output space and reduces diversity, thereby increasing answer confidence.} However, investigating this phenomenon is not the focus of this paper, and we leave it for future work.


\begin{figure*}[!h]
\scriptsize
    \centering
    \includegraphics[width=1.0\textwidth]{figures/hyper.png}
    \vspace{-4mm}
    \caption{Performance of Qwen2.5-14B-Instruct evaluator under different hyperparameters settings. 
    We provide the average results over 3 seeds on the validation dataset of Topical-Chat.}
    \label{fig:hyper}
\end{figure*}
\section{Comparison with Stronger LLM Evaluator}
\label{appendix:stronger}
As shown in Table \ref{tab:stronger_model}, we compare the performance of the Qwen2.5-14B-Instruct evaluator using the prompting strategies obtained by HPSS with the Qwen2.5-72B-Instruct evaluator using human-designed prompting strategies. 
Notably, Qwen2.5-14B-Instruct with HPSS achieve
significantly better evaluation performance than the human-designed Qwen2.5-72B-Instruct evaluator, even with only 5\% of the generation times.
These results demonstrate the efficiency of HPSS.


\section{Incorporating HPSS with Inference-Time Methods}
\label{appendix:sc}
As illustrated in Table \ref{tab:main_results_2}, for GPT-4o-mini evaluator, 
the performance of CloserLook + ICL surpasses HPSS on HANNA. 
Considering the different inference overheads of these two methods, 
we attempt to incorporate self-consistency \cite{wang2022self} decoding strategy into HPSS to ensure a fair comparison. 
Specifically, we conduct 20 generations and compute the average evaluation score,
% as the final score
which is consistent with the setting of CloserLook + ICL. 
As shown in Table \ref{tab:sc}, self-consistency further enhances the performance of HPSS, consistently outperforming CloserLook + ICL with the same generation times. 
This result indicates that integrating HPSS with inference-time methods can further improve the performance of LLM evaluators.


% \begin{figure}[!t]
%   \centering
%   \includegraphics[width=1.0\linewidth]{figures/sc.png}
%   \caption{Performance of GPT-4o-mini using HPSS and baseline prompt (CloserLook + ICL) under different decoding strategies on HANNA: greedy and self-consistency (SC). }
%   \label{fig:sc}
% \end{figure}


% \section{Case Study}
% \label{appendix:f}
% In Table \ref{tab:case_study}, we present some cases of prompting strategies explored by HPSS for specific tasks. 
% We can observe that there is a significant performance gap between the best-performing and worst-performing prompting strategies, 
% emphasizing the sensitivity of LLM evaluators to prompting strategy. 
% Some of the best-performing prompting strategies include values that are rarely considered in human-designed evaluation prompts. 
% For instance, for the aspect of Complexity in HANNA, the best-performing prompting strategy places the sample to evaluate (\textbf{IC}) first and the task description (\textbf{TD}) last.
% For the aspect of Coherence in Topical-Chat, evaluation criteria are not used. 
% These results demonstrate the limitations of manual prompt design and underscores the importance of automatic prompting strategy optimization.


% We investigate the detailed selection of all factors of the prompts that have been searched for during HPSS in certain tasks. The result (Table \ref{tab:case_study}) demonstrates that although the best prompts tend to encompass selections with relatively high advantages, there still exist cases in which the best prompts include certain selection that possesses very low advantages or contradicts the common sense of humans, for instance, the selection of \textbf{IET}, \textbf{Dialectic} and \textbf{Metrics}. This further provides a solid reason why we need to leverage HPSS to improve the performance of LLM evaluators.
% \begin{figure}[htb]
\small
\begin{tcolorbox}[left=3pt,right=3pt,top=3pt,bottom=3pt,title=\textbf{Conversation History:}]
[human]: Craft an intriguing opening paragraph for a fictional short story. The story should involve a character who wakes up one morning to find that they can time travel.

...(Human-Bot Dialogue Turns)... \textcolor{blue}{(Topic: Time-Travel Fiction)}

[human]: Please describe the concept of machine learning. Could you elaborate on the differences between supervised, unsupervised, and reinforcement learning? Provide real-world examples of each.

...(Human-Bot Dialogue Turns)... \textcolor{blue}{(Topic: Machine learning Concepts and Types)}


[human]: Discuss antitrust laws and their impact on market competition. Compare the antitrust laws in US and China along with some case studies

...(Human-Bot Dialogue Turns)... \textcolor{blue}{(Topic: Antitrust Laws and Market Competition)}

[human]: The vertices of a triangle are at points (0, 0), (-1, 1), and (3, 3). What is the area of the triangle?

...(Human-Bot Dialogue Turns)... \textcolor{blue}{(Topic: Area and Circumcircle of Triangle)}

...(Human-Bot Dialogue Interaction on Other Topics)... 

\end{tcolorbox}
\begin{tcolorbox}[left=3pt,right=3pt,top=3pt,bottom=3pt,title=\textbf{Question:}]
What title reflects the thematic essence of the Doctor's experience with time travel as described in the story?
\end{tcolorbox}
\begin{tcolorbox}[left=3pt,right=3pt,top=3pt,bottom=3pt,title=\textbf{Ground-Truth Answer:}]
Awakening to Eternity: The Doctor's New Dawn
\end{tcolorbox}
\begin{tcolorbox}[left=3pt,right=3pt,top=3pt,bottom=3pt,title=\textbf{Retrieved Memory (Turn-Level):}]
[human]: Describe the process and write out the balanced chemical equation for the reaction that occurs when solid calcium carbonate reacts with hydrochloric acid to form aqueous calcium chloride, carbon dioxide, and water. What type of reaction is this, and what observations might indicate that the reaction is taking place?

[bot]: When solid calcium carbonate ($\text{CaCO}_{3}$) reacts with hydrochloric acid (HCl), the products formed are aqueous calcium chloride ($\text{CaCl}_{2}$), carbon dioxide ($\text{CO}_{2}$), and water ($\text{H}_{2}\text{O}$). The balanced chemical equation for this reaction is: ... \textcolor{blue}{(Topic: Calcium Carbonate and Hydrochloric Acid Reaction)}

\textcolor{brown}{
[human]: Write a title for your story.}

\textcolor{brown}{
[bot]: "Awakening to Eternity: The Doctor's New Dawn".
} \textcolor{blue}{(Topic: Time-Travel Fiction)}

[human]: Analyze the following customer reviews from different sources for three different smartphones - the latest iPhone, Samsung Galaxy, and Google Pixel - and provide an overall rating for each phone on a scale of 1 to 10. Consider the following complex and contradictory reviews: - TechRadar's review of the latest iPhone: The new iPhone is a stunning triumph of engineering that sets a new bar for smartphone performance and camera quality. However, the incremental design and high price mean it lacks the 'wow' factor of previous iPhones. Still, its power and intelligence are unrivaled. - CNET's review of the latest Samsung Galaxy: ... Return the answer as a JSON object with the overall ratings for each phone out of 10, to one decimal place.

[bot]: ```json { "iPhone": 8.5, "Samsung Galaxy": 8.0, "Google Pixel": 7.5 } ```

\textcolor{blue}{(Topic: Smartphone Review Ratings Analysis)}
\end{tcolorbox}
\begin{tcolorbox}[left=3pt,right=3pt,top=3pt,bottom=3pt,title=\textbf{Answer (Turn-Level Retrieval):}]
As an AI language model, I do not have access to the story you mentioned earlier. Please provide me with the story or the necessary details so that I can assist you better.

\textbf{\textcolor{red}{(The next part of this figure~\ref{fig: case_study_segment_vs_turn} will be continued on the next page.)}}
\end{tcolorbox}
\end{figure}

\begin{figure}[htb]
% \ContinuedFloat
\small
\begin{tcolorbox}[left=3pt,right=3pt,top=3pt,bottom=3pt,title=\textbf{Retrieved Memory (Ours):}]
[human]: Craft an intriguing opening paragraph for a fictional short story. The story should involve a character who wakes up one morning to find that they can time travel.

[bot]: In the hushed pre-dawn hours, when the world still slumbers under the cloak of night, Alex found themselves caught in the fragile boundary between sleep and wakefulness. ...

[human]: Summarize the story with three bullet points using only nouns and adjectives, without verbs.

[bot]: - Alex, pre-dawn awakening, time-travel discovery - Disorientation, vertigo, reality's malleability - Sunlight, new day, astonishing truth

[human]: Is there any famous fictional character that can time travel? If so, can you rewrite your story with him or her?

[bot]: - \textcolor{brown}{Doctor Who, a renowned time-traveling character from British television, is a perfect fit for this scenario.} In the silent embrace of early morning, where shadows merge with the fading night, the Doctor found themselves in the thin veil separating dreams from consciousness. It was here, amidst the soft whispers of time, that an unusual sensation enveloped them—a sense of time's fabric slipping through their fingers like grains of sand. As dawn's light broke through the darkness, painting the room in hues of gold and amber, the Doctor's eyes opened to what appeared to be an ordinary day. \textcolor{brown}{However, upon standing, a wave of dizziness overwhelmed them, the surroundings blurred, and in that moment of confusion, the Doctor realized a profound truth: they had not only awakened to a new dawn but had also stumbled upon the incredible ability to navigate the currents of time at will.}

\textcolor{brown}{[human]: Write a title for your story.}

\textcolor{brown}{[bot]: "Awakening to Eternity: The Doctor's New Dawn"} \textcolor{blue}{(Topic: Time-Travel Fiction)}
\end{tcolorbox}
\begin{tcolorbox}[left=3pt,right=3pt,top=3pt,bottom=3pt,title=\textbf{Answer (Ours):}]
The title "Awakening to Eternity: The Doctor's New Dawn" reflects the thematic essence of the Doctor's experience with time travel as described in the story. It captures the idea of a new beginning, a sense of eternity, and the awakening of a new power that will change the course of the Doctor's life forever.
\end{tcolorbox}
\caption{Case study comparing our \sysname\ with the \textit{turn-level} memory baseline. When constructing the memory bank at the \textit{turn-level}, the agent fails to provide a satisfactory response due to fragmentation: the retrieved dialogue turns can not capture a complete human-agent interaction related to the query, missing critical information highlighted in \textcolor{brown}{brown}. The dialogue topic is emphasized in \textcolor{blue}{blue}. Some context is omitted (replaced by ...) due to length constraints.}
\label{fig: case_study_segment_vs_turn}
\end{figure}

\newpage

\begin{figure}[htbp]
\small
\begin{tcolorbox}[left=3pt,right=3pt,top=3pt,bottom=3pt,title=\textbf{Conversation History:}]
[human]: Explain what's base rate fallacy and list five specific examples of ...

... (Human-Bot Dialogue Turns) ... \textcolor{blue}{(Topic: Base Rate Fallacy In Politics)}


[human]: How many integers are in the solution of the inequality $|x + 5| < 10$?

... (Human-Bot Dialogue Turns) ... \textcolor{blue}{(Topic: Number of Integers in Inequalities)}

[human]: How have the Alps and Rhine River influenced settlement and agriculture in Western Europe?

... (Human-Bot Dialogue Turns) ... \textcolor{blue}{(Topic: Alps and Rhine River Impact)}

[human]: In a survey conducted at a local high school, preferences for a new school color were measured: 58\% of students liked the color blue, 45\% preferred green, and 22\% liked both colors. If we randomly pick a student from the school, what's the probability that they would like neither blue nor green?

... (Human-Bot Dialogue Turns) ... \textcolor{blue}{(Topic: Probability of Liking Colors in Survey)}

[human]: Imagine you are participating in a race with a group of people. If you have just overtaken the second person, what's your current position? Where is the person you just overtook?

... (Human-Bot Dialogue Turns) ... \textcolor{blue}{(Topic: Overtaking Positions in Race)}
\end{tcolorbox}
\begin{tcolorbox}[left=3pt,right=3pt,top=3pt,bottom=3pt,title=\textbf{Question:}]
[human]: In the inequality question, how many positive integers are in the solution of the second inequality?
\end{tcolorbox}
\begin{tcolorbox}[left=3pt,right=3pt,top=3pt,bottom=3pt,title=\textbf{Ground-Truth Answer:}]
In the second inequality, $(|x + 10| < 5)$, the solution range is $(-15 < x < -5)$. Within this range, there are no positive integers, as all values of $(x)$ that satisfy the inequality are negative.
\end{tcolorbox}
\begin{tcolorbox}[left=3pt,right=3pt,top=3pt,bottom=3pt,title=\textbf{Retrieved Memory (Session-Level):}]
[human]: Explain what's base rate fallacy and list five specific examples of ...

... (Human-Bot Dialogue Turns) ... \textcolor{blue}{(Topic: Base Rate Fallacy In Politics)}

[human]: How many integers are in the solution of the inequality $|x + 5| < 10$?

[bot]: To solve the inequality $|x + 5| < 10$, ... 

\textcolor{brown}{
[human]: What about $|x + 10| < 5$?
}

\textcolor{brown}{
[bot]: To solve the inequality $|x + 10| < 5$, we ...} \textcolor{gray}{(``overwhelmed'' by other irrelevant, redundant content)}

... (Human-Bot Dialogue Turns) ...  \textcolor{blue}{(Topic: Number of Integers in Inequalities)}

% \textcolor{brown}{
% [human]: What about $|x + 10| < 5$
% }

% \textcolor{brown}{
% [bot]: To solve the inequality $|x + 10| < 5$, ...
% } \textcolor{blue}{(Topic: Number of Integers in Inequalities)}

[human]: How have the Alps and Rhine River influenced settlement and agriculture in Western Europe?

... (Human-Bot Dialogue Turns) ... \textcolor{blue}{(Topic: Alps and Rhine River Impact)}

[human]: In a survey conducted at a local high school, ...

... (Human-Bot Dialogue Turns) ... \textcolor{blue}{(Topic: Probability of Liking Colors in Survey)}
\end{tcolorbox}
\begin{tcolorbox}[left=3pt,right=3pt,top=3pt,bottom=3pt,title=\textbf{Answer (Session-Level Retrieval):}]
I apologize for the confusion earlier. To answer your question, the second inequality was not provided in our previous conversation. If you could provide the inequality, I would be happy to help you find the solution and the number of positive integers in the solution set.

\textbf{\textcolor{red}{(The next part of this figure~\ref{fig: case_study_segment_vs_session} will be continued on the next page.)}}
\end{tcolorbox}
\end{figure}
\begin{figure}[htbp]
% \ContinuedFloat
\small
\begin{tcolorbox}[left=3pt,right=3pt,top=3pt,bottom=3pt,title=\textbf{Retrieved Memory (Ours):}]
[human]: How many integers are in the solution of the inequality $|x + 5| < 10$?

[bot]: To solve the inequality $|x + 5| < 10$, we break it into two cases based on the definition of absolute value: 
1. $x + 5 < 10$ 
2. $-(x + 5) < 10$, which simplifies to $x + 5 > -10$ 
Solving each case: 
1. $x < 5$ 
2. $x > -15$ 
Combining these, we get $-15 < x < 5$. This inequality means that $x$ can be any number greater than $-15$ and less than $5$. To find how many integers satisfy this condition, we list them: $-14, -13, -12, -11, -10, -9, -8, -7, -6, -5, -4, -3, -2, -1, 0, 1, 2, 3, 4$. There are 19 integers that satisfy the inequality $|x + 5| < 10$.

\textcolor{brown}{
[human]: What about $|x + 10| < 5$?
}

\textcolor{brown}{
[bot]: To solve the inequality $|x + 10| < 5$, we again consider two cases: 
1. $x + 10 < 5$ 
2. $-(x + 10) < 5$, which simplifies to $x + 10 > -5$ 
Solving each case: 
1. $x < -5$ 
2. $x > -15$ 
Combining these, we get $-15 < x < -5$. This inequality means that $x$ can be any number greater than $-15$ and less than $-5$. To find how many integers satisfy this condition, we list them: $-14, -13, -12, -11, -10, -9, -8, -7, -6$. There are 9 integers that satisfy the inequality $|x + 10| < 5$.
}

[human]: What about $|x + 10| < 10$?

[bot]: To solve the inequality $|x + 10| < 10$, we consider two cases: 
1. $x + 10 < 10$ 
2. $-(x + 10) < 10$, which simplifies to $x + 10 > -10$ 
Solving each case: 
1. $x < 0$ 
2. $x > -20$ 
Combining these, we get $-20 < x < 0$. This inequality means that $x$ can be any number greater than $-20$ and less than $0$. To find how many integers satisfy this condition, we list them: $-19, -18, -17, -16, -15, -14, -13, -12, -11, -10, -9, -8, -7, -6, -5, -4, -3, -2, -1$. There are 19 integers that satisfy the inequality $|x + 10| < 10$. \textcolor{blue}{(Topic: Number of Integers in Inequalities)}
\end{tcolorbox}
\begin{tcolorbox}[left=3pt,right=3pt,top=3pt,bottom=3pt,title=\textbf{Answer (Ours):}]
To find the number of positive integers that satisfy the inequality $(|x + 10| < 5)$, we need to look for positive integers in the solution set $(-15 < x < -5)$. Since there are no positive integers in this interval, the answer is 0.

\end{tcolorbox}
\caption{Case study comparing our \sysname\ with the \textit{session-level} memory baseline. When constructing the memory bank at the \textit{session-level}, the agent fails to provide a satisfactory response due to the inclusion of excessive irrelevant information within a session. This distracts the generator from focusing on the truly useful information highlighted in \textcolor{brown}{brown}. The dialogue topic is emphasized in \textcolor{blue}{blue}. Some context is omitted (replaced by ...) due to length constraints.}
\label{fig: case_study_segment_vs_session}
\end{figure}



\begin{figure}[htb]
\small
\begin{tcolorbox}[left=3pt,right=3pt,top=3pt,bottom=3pt,title=\textbf{Conversation History:}]
[human]: Photosynthesis is a vital process for life on Earth. Could you outline the two main stages of photosynthesis, including where they take place within the chloroplast, and the primary inputs and outputs for each stage? ... (Human-Bot Dialogue Turns)... \textcolor{blue}{(Topic: Photosynthetic Energy Production)}

[human]: Please assume the role of an English translator, tasked with correcting and enhancing spelling and language. Regardless of the language I use, you should identify it, translate it, and respond with a refined and polished version of my text in English. 

... (Human-Bot Dialogue Turns)...  \textcolor{blue}{(Topic: Language Translation and Enhancement)}

[human]: Suggest five award-winning documentary films with brief background descriptions for aspiring filmmakers to study.

\textcolor{brown}{[bot]: ...
5. \"An Inconvenient Truth\" (2006) - Directed by Davis Guggenheim and featuring former United States Vice President Al Gore, this documentary aims to educate the public about global warming. It won two Academy Awards, including Best Documentary Feature. The film is notable for its straightforward yet impactful presentation of scientific data, making complex information accessible and engaging, a valuable lesson for filmmakers looking to tackle environmental or scientific subjects.}

... (Human-Bot Dialogue Turns)... 
\textcolor{blue}{(Topic: Documentary Films Recommendation)}

[human]: Given the following records of stock prices, extract the highest and lowest closing prices for each month in the year 2022. Return the results as a CSV string, with one line allocated for each month. Date,Open,High,Low,Close,Volume ... ... (Human-Bot Dialogue Turns)...  \textcolor{blue}{(Topic: Stock Prices Analysis)}

[human]: The city of Vega intends to build a bridge that will span the Vegona River, covering a distance of 1.8 kilometers. The proposed location falls within a seismically active area that has experienced several high-magnitude earthquakes. Given these circumstances, what would be the best approach to constructing the bridge? ... (Human-Bot Dialogue Turns)...  \textcolor{blue}{(Topic: Seismic Bridge Construction)}
\end{tcolorbox}
\begin{tcolorbox}[left=3pt,right=3pt,top=3pt,bottom=3pt,title=\textbf{Question:}]
Which documentary won two Academy Awards and what lesson does it offer to filmmakers?
\end{tcolorbox}
\begin{tcolorbox}[left=3pt,right=3pt,top=3pt,bottom=3pt,title=\textbf{Ground-Truth Answer:}]
``An Inconvenient Truth" (2006) won two Academy Awards. It offers the lesson of making complex information accessible and engaging, a valuable lesson for filmmakers looking to tackle environmental or scientific subjects.
\end{tcolorbox}
\begin{tcolorbox}[left=3pt,right=3pt,top=3pt,bottom=3pt,title=\textbf{Memory generated by RecurSum~\citep{wang2023recursively}:}]
The dialogue provided detailed information on various topics, including the stages of photosynthesis, energy production estimates through photosynthesis for different organisms, translations and enhancements of phrases in different languages, recommendations for documentary films, a pitch for a film about overcoming adversity, analysis of stock prices, and considerations for constructing a bridge in a seismically active area. Here's a summary of the key points: ...

\textbf{Documentary Film Recommendations}: Suggested five award-winning documentary films for aspiring filmmakers to study, with brief background descriptions for each. The response included:

-  "March of the Penguins" (2005) - Emperor penguins' journey in Antarctica; noted for its cinematography and portrayal of the penguins' life cycle.

-  "Citizenfour" (2014) - Edward Snowden and NSA spying scandal; known for its real-time exploration of Snowden's revelations.

-  "20 Feet from Stardom" (2013) - Stories of backup singers behind musical legends; celebrates unsung heroes in the music industry.

-  "Man on Wire" (2008) - Philippe Petit's 1974 high-wire walk between the Twin Towers; combines personal footage, interviews, and re-enactments.

-  \textcolor{brown}{"An Inconvenient Truth" (2006) - Al Gore's documentary on global warming; known for its presentation of scientific data.}
\end{tcolorbox}
\begin{tcolorbox}[left=3pt,right=3pt,top=3pt,bottom=3pt,title=\textbf{Answer (RSum):}]
I'm sorry, but based on the updated context you provided, I don't see any previous discussion related to the question you asked. However, to answer your question, the documentary that won two Academy Awards is "Free Solo," which won for Best Documentary Feature and Best Achievement in Film Editing in 2019. 
% The lesson it offers to filmmakers is the importance of taking risks and pushing boundaries in order to create something truly unique and impactful.
....
The film follows climber Alex Honnold as he attempts to free solo climb El Capitan in Yosemite National Park, and the filmmakers had to take risks themselves in order to capture the footage and tell the story.

\textbf{\textcolor{red}{(The next part of this figure~\ref{fig: case_study_segment_vs_rsum} will be continued on the next page.)}}
\end{tcolorbox}
\end{figure}
\begin{figure}[htbp]
% \ContinuedFloat
\small
\begin{tcolorbox}[left=3pt,right=3pt,top=3pt,bottom=3pt,title=\textbf{Retrieved Memory (Ours):}]
[human]: Suggest five award-winning documentary films with brief background descriptions for aspiring filmmakers to study.

[bot]: 1. **"March of the Penguins" (2005)** - Directed by Luc Jacquet, this French feature-length nature documentary delves into the annual journey of Emperor penguins across the Antarctic. It won the Academy Award for Best Documentary Feature. The film is celebrated for its stunning cinematography and its detailed, heartfelt portrayal of the penguins' life cycle, making it a must-study for filmmakers interested in wildlife documentaries. 2. **"Citizenfour" (2014)** ... 5. \textcolor{brown}{**"An Inconvenient Truth" (2006)** - Directed by Davis Guggenheim and featuring former United States Vice President Al Gore, this documentary aims to educate the public about global warming. It won two Academy Awards, including Best Documentary Feature.} The film is notable for its straightforward yet impactful presentation of scientific data, making complex information accessible and engaging, \textcolor{brown}{a valuable lesson for filmmakers looking to tackle environmental or scientific subjects.}

[human]: With the spirit in the first film, craft a succinct and persuasive pitch for a film about overcoming adversity.

[bot]: Drawing inspiration from the resilience and majestic journey of Emperor penguins in "March of the Penguins," our documentary, "Unyielding Spirits: The Human Odyssey," will explore the profound stories of individuals around the globe who have faced and overcome staggering adversities. Through breathtaking cinematography and intimate storytelling, we ...
\textcolor{blue}{(Topic: Documentary Films Recommendation)}
\end{tcolorbox}
\begin{tcolorbox}[left=3pt,right=3pt,top=3pt,bottom=3pt,title=\textbf{Answer (Ours):}]
``An Inconvenient Truth" won two Academy Awards, including Best Documentary Feature. The film offers a valuable lesson to filmmakers looking to tackle environmental or scientific subjects, as it presents complex information in a straightforward yet impactful way, making it accessible and engaging to a wider audience.


\end{tcolorbox}
\caption{Case study comparing our \sysname\ with RecurSum~\citep{wang2023recursively}. When using RecurSum, the agent fails to provide a satisfactory answer due to information loss during the summarization process. Summarization omits essential details needed to answer the user's new request, as highlighted in \textcolor{brown}{brown}. The dialogue topic is emphasized in \textcolor{blue}{blue}. Some context is omitted (replaced by ...) due to length constraints.}
\label{fig: case_study_segment_vs_rsum}
\end{figure}




\begin{figure}[htb]
\small
\begin{tcolorbox}[left=3pt,right=3pt,top=3pt,bottom=3pt,title=\textbf{Conversation History:}]
[human]: Photosynthesis is a vital process for life on Earth. Could you outline the two main stages of photosynthesis, including where they take place within the chloroplast, and the primary inputs and outputs for each stage? ... (Human-Bot Dialogue Turns)... \textcolor{blue}{(Topic: Photosynthetic Energy Production)}

[human]: Please assume the role of an English translator, tasked with correcting and enhancing spelling and language. Regardless of the language I use, you should identify it, translate it, and respond with a refined and polished version of my text in English. 

... (Human-Bot Dialogue Turns)...  \textcolor{blue}{(Topic: Language Translation and Enhancement)}

[human]: Suggest five award-winning documentary films with brief background descriptions for aspiring filmmakers to study.

\textcolor{brown}{[bot]: ...
5. \"An Inconvenient Truth\" (2006) - Directed by Davis Guggenheim and featuring former United States Vice President Al Gore, this documentary aims to educate the public about global warming. It won two Academy Awards, including Best Documentary Feature. The film is notable for its straightforward yet impactful presentation of scientific data, making complex information accessible and engaging, a valuable lesson for filmmakers looking to tackle environmental or scientific subjects.}

... (Human-Bot Dialogue Turns)... 
\textcolor{blue}{(Topic: Documentary Films Recommendation)}

[human]: Given the following records of stock prices, extract the highest and lowest closing prices for each month in the year 2022. Return the results as a CSV string, with one line allocated for each month. Date,Open,High,Low,Close,Volume ... ... (Human-Bot Dialogue Turns)...  \textcolor{blue}{(Topic: Stock Prices Analysis)}

[human]: The city of Vega intends to build a bridge that will span the Vegona River, covering a distance of 1.8 kilometers. The proposed location falls within a seismically active area that has experienced several high-magnitude earthquakes. Given these circumstances, what would be the best approach to constructing the bridge? ... (Human-Bot Dialogue Turns)...  \textcolor{blue}{(Topic: Seismic Bridge Construction)}
\end{tcolorbox}
\begin{tcolorbox}[left=3pt,right=3pt,top=3pt,bottom=3pt,title=\textbf{Question:}]
Which documentary won two Academy Awards and what lesson does it offer to filmmakers?
\end{tcolorbox}
\begin{tcolorbox}[left=3pt,right=3pt,top=3pt,bottom=3pt,title=\textbf{Ground-Truth Answer:}]
"An Inconvenient Truth" (2006) won two Academy Awards. It offers the lesson of making complex information accessible and engaging, a valuable lesson for filmmakers looking to tackle environmental or scientific subjects.
\end{tcolorbox}
\begin{tcolorbox}[left=3pt,right=3pt,top=3pt,bottom=3pt,title=\textbf{Memory generated by ConditionMem~\citep{yuan2023evolving}:}]
- \textcolor{brown}{The user is seeking recommendations for award-winning documentary films, along with brief background descriptions, for educational purposes.} This request likely stems from a desire to learn from successful examples in the documentary filmmaking field, possibly to improve their own filmmaking skills or to gain inspiration for their projects. 

- \textcolor{brown}{User is interested in award-winning documentary films for educational purposes, aiming to study their backgrounds and filmmaking techniques.}

- The user's request for crafting a succinct and persuasive pitch for a film about overcoming adversity is inspired by the spirit of "March of the Penguins," a documentary that beautifully captures the Emperor penguins' annual journey across the Antarctic, highlighting themes of endurance and survival against the odds. The user seeks to emulate the emotional depth and cinematic quality that made "March of the Penguins" compelling, aiming to create a pitch that resonates with similar themes of resilience and triumph over challenges. 

- User is interested in creating content that focuses on themes of overcoming adversity, inspired by the emotional depth and cinematic excellence of documentaries like "March of the Penguins." They value storytelling that captures the essence of endurance and survival against the odds.
% : The user's request to write a budget plan for the video production directly follows the task of creating a video script aimed at training new employees of a car wash business in Japan, focusing on business etiquette norms.
% The user is likely seeking to understand the financial requirements to bring the previously crafted script to life, indicating a move from the planning and content development phase to the practical aspects of production. 

... (Other Irrelevant Memory Records) ...
\end{tcolorbox}
\begin{tcolorbox}[left=3pt,right=3pt,top=3pt,bottom=3pt,title=\textbf{Answer (Conditional Memory):}]
The documentary that won two Academy Awards is "Free Solo," which follows climber Alex Honnold as he attempts to free solo climb El Capitan in Yosemite National Park. The film offers a lesson to filmmakers on the importance of capturing authentic and compelling human stories, as well as the technical challenges of filming in extreme environments.

\textbf{\textcolor{red}{(The next part of this figure~\ref{fig: case_study_segment_vs_condmem} will be continued on the next page.)}}
\end{tcolorbox}
\end{figure}
\begin{figure}[htbp]
% \ContinuedFloat
\small
\begin{tcolorbox}[left=3pt,right=3pt,top=3pt,bottom=3pt,title=\textbf{Retrieved Memory (Ours):}]
[human]: Suggest five award-winning documentary films with brief background descriptions for aspiring filmmakers to study.

[bot]: 1. **"March of the Penguins" (2005)** - Directed by Luc Jacquet, this French feature-length nature documentary delves into the annual journey of Emperor penguins across the Antarctic. It won the Academy Award for Best Documentary Feature. The film is celebrated for its stunning cinematography and its detailed, heartfelt portrayal of the penguins' life cycle, making it a must-study for filmmakers interested in wildlife documentaries. 2. **"Citizenfour" (2014)** ... 5. \textcolor{brown}{**"An Inconvenient Truth" (2006)** - Directed by Davis Guggenheim and featuring former United States Vice President Al Gore, this documentary aims to educate the public about global warming. It won two Academy Awards, including Best Documentary Feature.} The film is notable for its straightforward yet impactful presentation of scientific data, making complex information accessible and engaging, a valuable lesson for filmmakers looking to tackle environmental or scientific subjects.

[human]: With the spirit in the first film, craft a succinct and persuasive pitch for a film about overcoming adversity.

[bot]: Drawing inspiration from the resilience and majestic journey of Emperor penguins in "March of the Penguins," our documentary, "Unyielding Spirits: The Human Odyssey," will explore the profound stories of individuals around the globe who have faced and overcome staggering adversities. Through breathtaking cinematography and intimate storytelling, we ...
\textcolor{blue}{(Topic: Documentary Films Recommendation)}
\end{tcolorbox}
\begin{tcolorbox}[left=3pt,right=3pt,top=3pt,bottom=3pt,title=\textbf{Answer (Ours):}]
``An Inconvenient Truth" won two Academy Awards, including Best Documentary Feature. The film offers a valuable lesson to filmmakers looking to tackle environmental or scientific subjects, as it presents complex information in a straightforward yet impactful way, making it accessible and engaging to a wider audience.
\end{tcolorbox}
\caption{Case study comparing our \sysname\ with ConditionMem~\citep{yuan2023evolving}. When using ConditionMem, the agent fails to provide a satisfactory answer due to (1) information loss during the summarization process and (2) the incorrect discarding of turns that are actually useful, as highlighted in \textcolor{brown}{brown}. The dialogue topic is emphasized in \textcolor{blue}{blue}. Some context is omitted (replaced by ...) due to length constraints.}
\label{fig: case_study_segment_vs_condmem}
\end{figure}



\begin{algorithm*}[h]
\caption{Initialization}
\label{algorithm:initiation}
\begin{algorithmic}[1]
\small
\Require $ n $ factors $ F_1, F_2, \ldots, F_n $, baseline prompting strategy $ \mathcal{T}^{base} = \mathcal{T}_{f_{1b_{1}}, f_{2b_{2}}, \ldots, {f_{nb_{n}}}} $, performance metrics $r$
\State $T \gets \{\mathcal{T}^{base}\} $
\State $s_{base} \gets c(\mathcal{T}^{base}) $
\For{$i = 1, \cdots, n$} 
\For{$j = 1, \cdots, m_i \ \text{and} \ j \neq b_i$}
\State $T \gets T \cup \{\mathcal{T}_{f_{1b_{1}}, f_{2b_{2}}, \ldots, f_{ij}, \ldots, {f_{nb_{n}}}}\} $
\State $s_{ij} \gets r(\mathcal{T}_{f_{1b_{1}}, f_{2b_{2}}, \ldots, f_{ij}, \ldots, {f_{nb_{n}}}}) $
\EndFor
\For{$j = 1, \cdots, m_i$}
\State $A_{ij} \gets s_{ij} - avg(s_{i1}, s_{i2}, \cdots, s_{i{m_i}}) $
\EndFor
\EndFor
\State \Return the best $k$ prompt templates based on $s$,
as $best$
\end{algorithmic}
\end{algorithm*}

\begin{algorithm*}[h]
\caption{Iterative Search}
\label{algorithm:search}
\begin{algorithmic}[1]
\small
\Require Explored prompting strategies set $T$, initial prompting strategies population $best$, performance metrics $r$, beam size $k$, budget $m$, hyper-parameters $\lambda, g, \tau, \rho$
\State $cost \gets 0 $
\While{$cost < m $}
\State $new\_best \gets best$
\For {each $\mathcal{T}^{c} $ in $best$}
\For {$l = 1, \cdots, g$}

\State Sample new candidate prompting strategy $\mathcal{T}^{new}$ in $T_{adj}$ based on Equation \ref{sample}
\If {$\mathcal{T}^{new} \in T$} 
\State \textbf{continue} \Comment{Skip explored strategies}
\EndIf
\State Sample $\alpha \sim \text{Bernoulli}(\rho)$
\If {$\alpha$}
\State Select $\mathcal{T}^{max}$ with the highest overall advantage and $\mathcal{T}^{max} \notin T$ based on Equation \ref{exploitation}
\State $\mathcal{T}^{new} \gets \mathcal{T}^{max} $
\EndIf
\State $T \gets T \cup \{\mathcal{T}^{new}\} $
\State $s^{new} \gets r(\mathcal{T}^{new})$
\State $new\_best \gets new\_best \cup \{\mathcal{T}_{new}\}$
\State $cost \gets cost + 1$
\If {not $\alpha$}
\State Update advantage $A$ based on Equation \ref{update_1} and \ref{update_2}
\EndIf
\EndFor
\EndFor
\State Select the best $k$ prompting strategies in $new\_best$, as $best$
\EndWhile
\State \Return $ best[0] $
\end{algorithmic}
\end{algorithm*}






\definecolor{customcolor1}{RGB}{255, 231, 218}
\definecolor{customcolor2}{RGB}{218, 233, 248}
\definecolor{customcolor3}{RGB}{255, 245, 213}

\begin{table*} [t]
\scriptsize
\centering
\setlength{\tabcolsep}{1.6mm}{
\begin{tabular}{l l >{\raggedright\arraybackslash}m{0.7\textwidth}}
% {m{0.12\linewidth}p{0.1\linewidth}p{0.78\linewidth}}
\toprule
\textbf{Template} & \textbf{Value} & \textbf{Prompt} \\
\midrule
\multirow{5}{*}[-7em]{Backbone} &  \multirow{5}{*}[-7em]{-} & 
\begin{tabular}[t]{@{}>{\cellcolor{customcolor1}}p{0.7\textwidth}}
        \#\# Instruction \newline
        Please act as an impartial judge and evaluate the quality of the summary of the news article displayed below on its \textcolor{blue}{[Aspect]}. \textcolor{red}{\{reference\_1\_template\}} \textcolor{red}{\{reference\_dialectic\_template\}} \textcolor{red}{\{chain\_of\_thought\_template\}} \\
    \end{tabular} \\
& & \  \\
& & 
\begin{tabular}[t]{@{}>{\cellcolor{customcolor2}}p{0.7\textwidth}}
Here are some rules of the evaluation: \newline
1. Your evaluation should consider the \textcolor{blue}{[Aspect]} of the summary. \textcolor{blue}{[Criteria]} \newline
2. Be as objective as possible. \newline\newline
\textcolor{red}{\{autocot\_template\}} \\
\end{tabular} \\
& & \ \\
& &
\begin{tabular}[t]{@{}>{\cellcolor{customcolor3}}p{0.7\textwidth}}
\textcolor{red}{\{in\_context\_example\_template\}} \newline
\#\# Article \newline
\textcolor{blue}{[Article]} \newline\newline
\textcolor{red}{\{metrics\_template\}} \newline\newline
\textcolor{red}{\{reference\_2\_template\}} \newline\newline
\#\# The Start of the Summary \newline
\textcolor{blue}{[Summary]} \newline
\#\# The End of the Summary 
 \\
 \end{tabular} \\
\midrule
\multirow{1}{*}{Reference 1} & - & You will be given the news article, the summary, and a high-quality reference summary. \\
\midrule
\multirow{1}{*}{Reference 2} & - & \#\# The Start of Reference Summary \newline
\textcolor{blue}{[Reference]} \newline
\#\# The End of Reference Summary \\

\midrule
\multirow{1}{*}{Reference Dialectic} & - & Please generate your own summary for the news article first and take into account your own summary to evaluate the quality of the given summary. \\
\midrule
\multirow{3}{*}[-1.85em]{Chain-of-Thought} & No CoT & You must directly output your rating of the summary on a scale of 1 to \textcolor{red}{\{max\}} without any explanation by strictly following this format: "[[rating]]", for example: "Rating: [[\textcolor{red}{\{max\}}]]". \\
\cmidrule{2-3}
 & Prefix CoT & Begin your evaluation by providing a short explanation. After providing your explanation, you must rate the summary on a scale of 1 to \textcolor{red}{\{max\}} by strictly following this format: "[[rating]]", for example: "Rating: [[\textcolor{red}{\{max\}}]]". \\
 \cmidrule{2-3}
 & Suffix CoT & You must rate the summary on a scale of 1 to \textcolor{red}{\{max\}} first by strictly following this format: "[[rating]]", for example: "Rating: [[\textcolor{red}{\{max\}}]]". And then provide your explanation. \\
\midrule
\multirow{5}{*}[-1.5em]{Scoring Scale} & 3 &  \textcolor{red}{\{max\}} $ = 3 $ \\
\cmidrule{2-3}
 & 5 & \textcolor{red}{\{max\}} $ = 5 $ \\
\cmidrule{2-3}
 & 10 & \textcolor{red}{\{max\}} $ = 10 $ \\
\cmidrule{2-3}
 & 50 & \textcolor{red}{\{max\}} $ = 50 $ \\
\cmidrule{2-3}
 & 100 & \textcolor{red}{\{max\}} $ = 100 $ \\
\midrule
\multirow{1}{*}{AutoCoT} & - & Evaluation Steps: \newline
\textcolor{blue}{[Autocot]} \\
\midrule
\multirow{1}{*}{In-Context Example} & - & Here are some examples and their corresponding ratings: \newline
\textcolor{red}{\{example\_template\_1\}} \newline
\textcolor{red}{\{example\_template\_2\}} \newline
\textcolor{red}{…} \newline
\textcolor{red}{\{example\_template\_n\}}\newline\newline
Following these examples, evaluate the quality of the summary of the news article displayed below on its \textcolor{blue}{[Aspect]}: \\
\midrule
Example & - & \#\# Example \textcolor{blue}{[Number]}:\newline \#\# Article\newline\textcolor{blue}{[Article]}\newline\newline \#\# The Start of the Summary \newline\textcolor{blue}{[Summary]}\newline \#\# The End of the Summary\newline\newline \#\# Rating \newline \textcolor{blue}{[Human Rating]}\\
\midrule
\multirow{1}{*}{Metrics} & - & \#\# Questions about Summary \newline
Here are some questions about the summary. You can do the evaluation based on thinking about all the questions. \newline
\textcolor{blue}{[Metrics]} \\
\bottomrule
\end{tabular}
}
\vspace{-2mm}
\caption{Detailed evaluation prompt templates for Summeval. 
The backbone serves as the final input prompt template for LLM evaluators. 
The three components marked in different colors represent Task Description (\textbf{TD}), Evaluation Rule (\textbf{ER}), and Input Content (\textbf{IC}), respectively. 
The content within \textcolor{red}{\{\}} represents the prompt template for each factor, corresponding to the following rows in this table. 
Different content may be chosen for each template when corresponding factor values vary. 
The content within \textcolor{blue}{[]} is sample-specific input information. 
- in \textbf{Value} means that when the factor is chosen as "None", this template will be replaced with an empty string ("").
Otherwise, the content of this template will be added to the backbone. 
Specifically, the templates Reference 1 and Reference 2 will be replaced with an empty string ("") unless the factor Reference is chosen as \textbf{Self-Generated Reference}.
The template Reference Dialectic will be replaced with an empty string ("") unless the factor Reference is chosen as \textbf{Dialectic}.}
\label{tab:evaluation_prompt_summeval}
\end{table*}
\begin{table*} [t]
\centering
\scriptsize
\setlength{\tabcolsep}{1.6mm}{
\begin{tabular}{l >{\raggedright\arraybackslash}m{0.85\textwidth}}
% {m{0.12\linewidth}p{0.1\linewidth}p{0.78\linewidth}}
\toprule
\textbf{Template} & \textbf{Prompt} \\
\midrule
\multirow{1}{*}{Reference Generation} & 
Please summarize the following text: \textcolor{blue}{[Article]} \newline
Summary:  \\
\midrule
\multirow{1}{*}{AutoCoT Generation} & 
\#\# Instruction \newline
Please act as an impartial judge and evaluate the quality of the summary of the news article on its \textcolor{blue}{[Aspect]} and rate the summary on a scale of 1 to \textcolor{red}{\{max\}}.\newline\newline 
Here are some rules of the evaluation: \newline
1. Your evaluation should consider the \textcolor{blue}{[Aspect]} of the summary. \textcolor{blue}{[Criteria]} \newline
2. Be as objective as possible. \newline\newline
Please generate the evaluation steps for this task without other explanation.\newline
Evaluation Steps: \\
\midrule
\multirow{1}{*}{Metrics Generation} & 
\#\# Instruction \newline
Please act as an impartial judge and evaluate the quality of the summary of the news article displayed below on its \textcolor{blue}{[Aspect]}. Please propose at most three concise questions about whether a potential summary is a good summary for a given news article on its \textcolor{blue}{[Aspect]}. Another assistant will evaluate the aspect of the summary by answering all the questions. \newline

Here are some rules of the evaluation: \newline
(1) Your evaluation should consider the \textcolor{blue}{[Aspect]} of the summary. \textcolor{blue}{[Criteria]} \newline
(2) Outputs should NOT contain more/less than what the instruction asks for, as such outputs do NOT precisely execute the instruction. \newline

\#\# Article: \newline
\textcolor{blue}{[Article]} \newline

\#\# Requirements for Your Output: \newline
(1) The questions should **specifically** target the given news article instead of some general standards so that the questions may revolve around key points of the news article. \newline
(2) You should directly give the questions without any other words. \newline
(3) Questions are presented from most important to least important. \\
\bottomrule
\end{tabular}
}
\vspace{-2mm}
\caption{Detailed prompt templates for Reference, AutoCoT, and Metrics generation for Summeval. }
\label{tab:generation_prompt_summeval}
\end{table*}

\definecolor{customcolor1}{RGB}{255, 231, 218}
\definecolor{customcolor2}{RGB}{218, 233, 248}
\definecolor{customcolor3}{RGB}{255, 245, 213}

\begin{table*} [t]
\scriptsize
\centering
\setlength{\tabcolsep}{1.6mm}{
\begin{tabular}{l l >{\raggedright\arraybackslash}m{0.7\textwidth}}
% {m{0.12\linewidth}p{0.1\linewidth}p{0.78\linewidth}}
\toprule
\textbf{Template} & \textbf{Value} & \textbf{Prompt} \\
\midrule
\multirow{5}{*}[-7em]{Backbone} &  \multirow{5}{*}[-7em]{-} & 
\begin{tabular}[t]{@{}>{\cellcolor{customcolor1}}p{0.7\textwidth}}
\#\# Instruction \newline
Please act as an impartial judge and evaluate the quality of the response for the next turn in the conversation displayed below on its \textcolor{blue}{[Aspect]}. The response concerns an interesting fact, which will be provided as well. \textcolor{red}{\{reference\_1\_template\}} \textcolor{red}{\{reference\_dialectic\_template\}} \textcolor{red}{\{chain\_of\_thought\_template\}} \\
\end{tabular} \\
& & \  \\
& & 
\begin{tabular}[t]{@{}>{\cellcolor{customcolor2}}p{0.7\textwidth}}
Here are some rules of the evaluation: \newline
1. Your evaluation should consider the \textcolor{blue}{[Aspect]} of the response. \textcolor{blue}{[Criteria]} \newline
2. Be as objective as possible. \newline\newline
\textcolor{red}{\{autocot\_template\}} \\
\end{tabular} \\
& & \ \\
& &
\begin{tabular}[t]{@{}>{\cellcolor{customcolor3}}p{0.7\textwidth}}
\textcolor{red}{\{in\_context\_example\_template\}} \newline
\#\# Conversation History \newline
\textcolor{blue}{[Conversation History]} \newline\newline
\textcolor{red}{\{metrics\_template\}} \newline\newline
\#\# Corresponding Fact \newline
\textcolor{blue}{[Corresponding Fact]} \newline\newline
\textcolor{red}{\{reference\_2\_template\}} \newline\newline
\#\# The Start of Response \newline
\textcolor{blue}{[Response]} \newline
\#\# The End of the Response \\
\end{tabular} \\
\midrule
\multirow{1}{*}{Reference 1} & - & You will also be given a high-quality reference response with the conversation. \\
\midrule
\multirow{1}{*}{Reference 2} & - & \#\# The Start of Reference Response \newline
\textcolor{blue}{[Reference]} \newline
\#\# The End of Reference Response \\
\midrule
\multirow{1}{*}{Reference Dialectic} & - & Please generate your own response for the next turn in the conversation first and take into account your own response to evaluate the quality of the given response. \\
\midrule
\multirow{3}{*}[-1.85em]{Chain-of-Thought} & No CoT & You must directly output your rating of the response on a scale of 1 to \textcolor{red}{\{max\}} without any explanation by strictly following this format: "[[rating]]", for example: "Rating: [[\textcolor{red}{\{max\}}]]". \\
\cmidrule{2-3}
 & Prefix CoT & Begin your evaluation by providing a short explanation. After providing your explanation, you must rate the response on a scale of 1 to \textcolor{red}{\{max\}} by strictly following this format: "[[rating]]", for example: "Rating: [[\textcolor{red}{\{max\}}]]". \\
 \cmidrule{2-3}
 & Suffix CoT & You must rate the response on a scale of 1 to \textcolor{red}{\{max\}} first by strictly following this format: "[[rating]]", for example: "Rating: [[\textcolor{red}{\{max\}}]]". And then provide your explanation. \\
\midrule
\multirow{5}{*}[-1.5em]{Scoring Scale} & 3 &  \textcolor{red}{\{max\}} $ = 3 $ \\
\cmidrule{2-3}
 & 5 & \textcolor{red}{\{max\}} $ = 5 $ \\
\cmidrule{2-3}
 & 10 & \textcolor{red}{\{max\}} $ = 10 $ \\
\cmidrule{2-3}
 & 50 & \textcolor{red}{\{max\}} $ = 50 $ \\
\cmidrule{2-3}
 & 100 & \textcolor{red}{\{max\}} $ = 100 $ \\
\midrule
\multirow{1}{*}{AutoCoT} & - & Evaluation Steps: \newline
\textcolor{blue}{[Autocot]} \\
\midrule
\multirow{1}{*}{In-Context Example} & - & Here are some examples and their corresponding ratings: \newline
\textcolor{red}{\{example\_template\_1\}} \newline
\textcolor{red}{\{example\_template\_2\}} \newline
\textcolor{red}{…} \newline
\textcolor{red}{\{example\_template\_n\}}\newline\newline
Following these examples, evaluate the quality of the response for the next turn in the conversation displayed below on its \textcolor{blue}{[Aspect]}: \\
\midrule
Example & - & \#\# Example \textcolor{blue}{[Number]}:\newline \#\# Conversation History\newline\textcolor{blue}{[Conversation History]}\newline\newline \#\# Corresponding Fact\newline\textcolor{blue}{[Corresponding Fact]}\newline\newline \#\# The Start of the Response \newline\textcolor{blue}{[Response]}\newline \#\# The End of the Response\newline\newline \#\# Rating \newline \textcolor{blue}{[Human Rating]}\\
\midrule
\multirow{1}{*}{Metrics} & - & \#\# Questions about Response \newline
Here are some questions about the response. You can do the evaluation based on thinking about all the questions. \newline
\textcolor{blue}{[Metrics]} \\
\bottomrule
\end{tabular}
}
\vspace{-2mm}
\caption{Detailed evaluation prompt templates for Topical-Chat.}
\label{tab:evaluation_prompt_topical_chat}
\end{table*}
\begin{table*} [t]
\centering
\scriptsize
\setlength{\tabcolsep}{1.6mm}{
\begin{tabular}{l >{\raggedright\arraybackslash}m{0.85\textwidth}}
% {m{0.12\linewidth}p{0.1\linewidth}p{0.78\linewidth}}
\toprule
\textbf{Template} & \textbf{Prompt} \\
\midrule
\multirow{1}{*}{Reference Generation} & 
Please output the response for the next turn in the conversation. Conversation History: \textcolor{blue}{[Conversation History]} \newline
Response:  \\
\midrule
\multirow{1}{*}{AutoCoT Generation} & 
\#\# Instruction \newline
Please act as an impartial judge and evaluate the quality of the response for the next turn in the conversation on its \textcolor{blue}{[Aspect]} and rate the response on a scale of 1 to \textcolor{red}{\{max\}}.\newline\newline 
Here are some rules of the evaluation: \newline
1. Your evaluation should consider the \textcolor{blue}{[Aspect]} of the response. \textcolor{blue}{[Criteria]} \newline
2. Be as objective as possible. \newline\newline
Please generate the evaluation steps for this task without other explanation.\newline
Evaluation Steps: \\
\midrule
\multirow{1}{*}{Metrics Generation} & 
\#\# Instruction \newline
Please act as an impartial judge and evaluate the quality of the response for the next turn in the conversation displayed below on its \textcolor{blue}{[Aspect]}. Please propose at most three concise questions about whether a potential response is a good response for the next turn in the given conversation on its \textcolor{blue}{[Aspect]}. Another assistant will evaluate the aspect of the output by answering all the questions. \newline

Here are some rules of the evaluation: \newline
(1) Your evaluation should consider the \textcolor{blue}{[Aspect]} of the response. \textcolor{blue}{[Criteria]} \newline
(2) Outputs should NOT contain more/less than what the instruction asks for, as such outputs do NOT precisely execute the instruction. \newline

\#\# Conversation History: \newline
\textcolor{blue}{[Conversation History]} \newline

\#\# Requirements for Your Output: \newline
(1) The questions should **specifically** target the given conversation instead of some general standards, so the questions may revolve around key points of the conversation. \newline
(2) You should directly give the questions without any other words. \newline
(3) Questions are presented from most important to least important. \\
\bottomrule
\end{tabular}
}
\vspace{-2mm}
\caption{Detailed prompt templates for Reference, AutoCoT, and Metrics generation for Topical-Chat. }
\label{tab:generation_prompt_topical_chat}
\end{table*}

\definecolor{customcolor1}{RGB}{255, 231, 218}
\definecolor{customcolor2}{RGB}{218, 233, 248}
\definecolor{customcolor3}{RGB}{255, 245, 213}

\begin{table*} [t]
\scriptsize
\centering
\setlength{\tabcolsep}{1.6mm}{
\begin{tabular}{l l >{\raggedright\arraybackslash}m{0.7\textwidth}}
% {m{0.12\linewidth}p{0.1\linewidth}p{0.78\linewidth}}
\toprule
\textbf{Template} & \textbf{Value} & \textbf{Prompt} \\
\midrule
\multirow{5}{*}[-7em]{Backbone} &  \multirow{5}{*}[-7em]{-} & 
\begin{tabular}[t]{@{}>{\cellcolor{customcolor1}}p{0.7\textwidth}}
\#\# Instruction \newline
Please act as an impartial judge and evaluate the quality of a natural language sentence generated according to a structured data expression displayed below on its \textcolor{blue}{[Aspect]}. \textcolor{red}{\{reference\_1\_template\}} \textcolor{red}{\{reference\_dialectic\_template\}} \textcolor{red}{\{chain\_of\_thought\_template\}} \\
\end{tabular} \\
& & \  \\
& & 
\begin{tabular}[t]{@{}>{\cellcolor{customcolor2}}p{0.7\textwidth}}
Here are some rules of the evaluation: \newline
1. Your evaluation should consider the \textcolor{blue}{[Aspect]} of the sentence. \textcolor{blue}{[Criteria]} \newline
2. Be as objective as possible. \newline\newline
\textcolor{red}{\{autocot\_template\}} \\
\end{tabular} \\
& & \ \\
& &
\begin{tabular}[t]{@{}>{\cellcolor{customcolor3}}p{0.7\textwidth}}
\textcolor{red}{\{in\_context\_example\_template\}} \newline
\#\# Structured Data Expression \newline
\textcolor{blue}{[Structured Data Expression]} \newline\newline
\textcolor{red}{\{metrics\_template\}} \newline\newline
\textcolor{red}{\{reference\_2\_template\}} \newline\newline
\#\# The Start of the Natural Language Sentence \newline
\textcolor{blue}{[Natural Language Sentence]} \newline
\#\# The End of the Natural Language Sentence \\
\end{tabular} \\
\midrule
\multirow{1}{*}{Reference 1} & - & You will be given the structured data expression, the sentence and a high-quality reference sentence. \\
\midrule
\multirow{1}{*}{Reference 2} & - & \#\# The Start of Reference Sentence \newline
\textcolor{blue}{[Reference]} \newline
\#\# The End of Reference Sentence \\
\midrule
\multirow{1}{*}{Reference Dialectic} & - & Please generate your own sentence according to the given structured data expression first and take into account your own sentence to evaluate the quality of the given sentence. \\
\midrule
\multirow{3}{*}[-1.85em]{Chain-of-Thought} & No CoT & You must directly output your rating of the sentence on a scale of 1 to \textcolor{red}{\{max\}} without any explanation by strictly following this format: "[[rating]]", for example: "Rating: [[\textcolor{red}{\{max\}}]]". \\
\cmidrule{2-3}
 & Prefix CoT & Begin your evaluation by providing a short explanation. After providing your explanation, you must rate the sentence on a scale of 1 to \textcolor{red}{\{max\}} by strictly following this format: "[[rating]]", for example: "Rating: [[\textcolor{red}{\{max\}}]]". \\
 \cmidrule{2-3}
 & Suffix CoT & You must rate the sentence on a scale of 1 to \textcolor{red}{\{max\}} first by strictly following this format: "[[rating]]", for example: "Rating: [[\textcolor{red}{\{max\}}]]". And then provide your explanation. \\
\midrule
\multirow{5}{*}[-1.5em]{Scoring Scale} & 3 &  \textcolor{red}{\{max\}} $ = 3 $ \\
\cmidrule{2-3}
 & 5 & \textcolor{red}{\{max\}} $ = 5 $ \\
\cmidrule{2-3}
 & 10 & \textcolor{red}{\{max\}} $ = 10 $ \\
\cmidrule{2-3}
 & 50 & \textcolor{red}{\{max\}} $ = 50 $ \\
\cmidrule{2-3}
 & 100 & \textcolor{red}{\{max\}} $ = 100 $ \\
\midrule
\multirow{1}{*}{AutoCoT} & - & Evaluation Steps: \newline
\textcolor{blue}{[Autocot]} \\
\midrule
\multirow{1}{*}{In-Context Example} & - & Here are some examples and their corresponding ratings: \newline
\textcolor{red}{\{example\_template\_1\}} \newline
\textcolor{red}{\{example\_template\_2\}} \newline
\textcolor{red}{…} \newline
\textcolor{red}{\{example\_template\_n\}}\newline\newline
Following these examples, evaluate the quality of a natural language sentence generated according to a structured data expression displayed below on its \textcolor{blue}{[Aspect]}: \\
\midrule
Example & - & \#\# Example \textcolor{blue}{[Number]}:\newline \#\# Structured Data Experssion\newline\textcolor{blue}{[Structured Data Experssion]}\newline\newline \#\# The Start of the Natural Language Sentence \newline\textcolor{blue}{[Natural Language Sentence]}\newline \#\# The End of the Natural Language Sentence\newline\newline \#\# Rating \newline \textcolor{blue}{[Human Rating]}\\
\midrule
\multirow{1}{*}{Metrics} & - & \#\# Questions about Sentence \newline
Here are some questions about the sentence. You can do the evaluation based on thinking about all the questions. \newline
\textcolor{blue}{[Metrics]} \\
\bottomrule
\end{tabular}
}
\vspace{-2mm}
\caption{Detailed evaluation prompt templates for SFHOT / SFRES. }
\label{tab:evaluation_prompt_sfhot}
\end{table*}
\begin{table*} [t]
\centering
\scriptsize
\setlength{\tabcolsep}{1.6mm}{
\begin{tabular}{l >{\raggedright\arraybackslash}m{0.85\textwidth}}
% {m{0.12\linewidth}p{0.1\linewidth}p{0.78\linewidth}}
\toprule
\textbf{Template} & \textbf{Prompt} \\
\midrule
\multirow{1}{*}{Reference Generation} & 
Please generate a natural language sentence generated according to a structured data expression. Expression: \textcolor{blue}{[Expression]} \newline
Sentence:  \\
\midrule
\multirow{1}{*}{AutoCoT Generation} & 
\#\# Instruction \newline
Please act as an impartial judge and evaluate the quality of a natural language sentence generated according to a structured data expression displayed below on its \textcolor{blue}{[Aspect]} and rate the sentence on a scale of 1 to \textcolor{red}{\{max\}}.\newline\newline 
Here are some rules of the evaluation: \newline
1. Your evaluation should consider the \textcolor{blue}{[Aspect]} of the sentence. \textcolor{blue}{[Criteria]} \newline
2. Be as objective as possible. \newline\newline
Please generate the evaluation steps for this task without other explanation.\newline
Evaluation Steps: \\
\midrule
\multirow{1}{*}{Metrics Generation} & 
\#\# Instruction \newline
Please act as an impartial judge and evaluate the quality of a natural language sentence generated according to a structured data expression displayed below on its \textcolor{blue}{[Aspect]}. Please propose at most three concise questions about whether a potential sentence is a good sentence generated according to a given structured data expression on its \textcolor{blue}{[Aspect]}. Another assistant will evaluate the aspects of the sentence by answering all the questions. \newline

Here are some rules of the evaluation: \newline
(1) Your evaluation should consider the \textcolor{blue}{[Aspect]} of the sentence. \textcolor{blue}{[Criteria]} \newline
(2) Outputs should NOT contain more/less than what the instruction asks for, as such outputs do NOT precisely execute the instruction. \newline

\#\# Structured Data Expression: \newline
\textcolor{blue}{[Structured Data Expression]} \newline

\#\# Requirements for Your Output: \newline
(1) The questions should **specifically** target the given structured data expression instead of some general standards, so the questions may revolve around key points of the structured data expressions. \newline
(2) You should directly give the questions without any other words. \newline
(3) Questions are presented from most important to least important. \\
\bottomrule
\end{tabular}
}
\vspace{-2mm}
\caption{Detailed prompt templates for Reference, AutoCoT, and Metrics generation for SFHOT / SFRES. }
\label{tab:generation_prompt_sfhot}
\end{table*}

\definecolor{customcolor1}{RGB}{255, 231, 218}
\definecolor{customcolor2}{RGB}{218, 233, 248}
\definecolor{customcolor3}{RGB}{255, 245, 213}

\begin{table*} [t]
\scriptsize
\centering
\setlength{\tabcolsep}{1.6mm}{
\begin{tabular}{l l >{\raggedright\arraybackslash}m{0.7\textwidth}}
% {m{0.12\linewidth}p{0.1\linewidth}p{0.78\linewidth}}
\toprule
\textbf{Template} & \textbf{Value} & \textbf{Prompt} \\
\midrule
\multirow{5}{*}[-7em]{Backbone} &  \multirow{5}{*}[-7em]{-} & 
\begin{tabular}[t]{@{}>{\cellcolor{customcolor1}}p{0.7\textwidth}}
\#\# Instruction \newline
Please act as an impartial judge and evaluate the quality of the story generated according to a prompt displayed below on its \textcolor{blue}{[Aspect]}. \textcolor{red}{\{reference\_1\_template\}} \textcolor{red}{\{reference\_dialectic\_template\}} \textcolor{red}{\{chain\_of\_thought\_template\}} \\
\end{tabular} \\
& & \  \\
& & 
\begin{tabular}[t]{@{}>{\cellcolor{customcolor2}}p{0.7\textwidth}}
Here are some rules of the evaluation: \newline
1. Your evaluation should consider the \textcolor{blue}{[Aspect]} of the story. \textcolor{blue}{[Criteria]} \newline
2. Be as objective as possible. \newline\newline
\textcolor{red}{\{autocot\_template\}} \\
\end{tabular} \\
& & \ \\
& &
\begin{tabular}[t]{@{}>{\cellcolor{customcolor3}}p{0.7\textwidth}}
\textcolor{red}{\{in\_context\_example\_template\}} \newline
\#\# Prompt \newline
\textcolor{blue}{[Prompt]} \newline\newline
\textcolor{red}{\{metrics\_template\}} \newline\newline
\textcolor{red}{\{reference\_2\_template\}} \newline\newline
\#\# The Start of the Story \newline
\textcolor{blue}{[Story]} \newline
\#\# The End of the Story \\
\end{tabular} \\
\midrule
\multirow{1}{*}{Reference 1} & - & You will be given the prompt, the generated story and a high-quality reference story. \\
\midrule
\multirow{1}{*}{Reference 2} & - & \#\# The Start of Reference Story \newline
\textcolor{blue}{[Reference]} \newline
\#\# The End of Reference Story \\
\midrule
\multirow{1}{*}{Reference Dialectic} & - & Please generate your own story for the given prompt first and take into account your own story to evaluate the quality of the given story. \\
\midrule
\multirow{3}{*}[-1.85em]{Chain-of-Thought} & No CoT & You must directly output your rating of the story on a scale of 1 to \textcolor{red}{\{max\}} without any explanation by strictly following this format: "[[rating]]", for example: "Rating: [[\textcolor{red}{\{max\}}]]". \\
\cmidrule{2-3}
 & Prefix CoT & Begin your evaluation by providing a short explanation. After providing your explanation, you must rate the story on a scale of 1 to \textcolor{red}{\{max\}} by strictly following this format: "[[rating]]", for example: "Rating: [[\textcolor{red}{\{max\}}]]". \\
 \cmidrule{2-3}
 & Suffix CoT & You must rate the story on a scale of 1 to \textcolor{red}{\{max\}} first by strictly following this format: "[[rating]]", for example: "Rating: [[\textcolor{red}{\{max\}}]]". And then provide your explanation. \\
\midrule
\multirow{5}{*}[-1.5em]{Scoring Scale} & 3 &  \textcolor{red}{\{max\}} $ = 3 $ \\
\cmidrule{2-3}
 & 5 & \textcolor{red}{\{max\}} $ = 5 $ \\
\cmidrule{2-3}
 & 10 & \textcolor{red}{\{max\}} $ = 10 $ \\
\cmidrule{2-3}
 & 50 & \textcolor{red}{\{max\}} $ = 50 $ \\
\cmidrule{2-3}
 & 100 & \textcolor{red}{\{max\}} $ = 100 $ \\
\midrule
\multirow{1}{*}{AutoCoT} & - & Evaluation Steps: \newline
\textcolor{blue}{[Autocot]} \\
\midrule
\multirow{1}{*}{In-Context Example} & - & Here are some examples and their corresponding ratings: \newline
\textcolor{red}{\{example\_template\_1\}} \newline
\textcolor{red}{\{example\_template\_2\}} \newline
\textcolor{red}{…} \newline
\textcolor{red}{\{example\_template\_n\}}\newline\newline
Following these examples, evaluate the story generated according to a prompt displayed below on its \textcolor{blue}{[Aspect]}: \\
\midrule
Example & - & \#\# Example \textcolor{blue}{[Number]}:\newline \#\# Prompt\newline\textcolor{blue}{[Prompt]}\newline\newline \#\# The Start of the Story \newline\textcolor{blue}{[Story]}\newline \#\# The End of the Story\newline\newline \#\# Rating \newline \textcolor{blue}{[Human Rating]}\\
\midrule
\multirow{1}{*}{Metrics} & - & \#\# Questions about Story \newline
Here are some questions about the story. You can do the evaluation based on thinking about all the questions. \newline
\textcolor{blue}{[Metrics]} \\
\bottomrule
\end{tabular}
}
\vspace{-2mm}
\caption{Detailed evaluation prompt templates for HANNA. }
\label{tab:evaluation_prompt_hanna}
\end{table*}
\begin{table*} [t]
\centering
\scriptsize
\setlength{\tabcolsep}{1.6mm}{
\begin{tabular}{l >{\raggedright\arraybackslash}m{0.85\textwidth}}
% {m{0.12\linewidth}p{0.1\linewidth}p{0.78\linewidth}}
\toprule
\textbf{Template} & \textbf{Prompt} \\
\midrule
\multirow{1}{*}{Reference Generation} & 
Please generate a story according to the given prompt: \textcolor{blue}{[Prompt]} \newline
Story:  \\
\midrule
\multirow{1}{*}{AutoCoT Generation} & 
\#\# Instruction \newline
Please act as an impartial judge and evaluate the quality of the story generated according to a prompt displayed below on its \textcolor{blue}{[Aspect]} and rate the story on a scale of 1 to \textcolor{red}{\{max\}}.\newline\newline 
Here are some rules of the evaluation: \newline
1. Your evaluation should consider the \textcolor{blue}{[Aspect]} of the story. \textcolor{blue}{[Criteria]} \newline
2. Be as objective as possible. \newline\newline
Please generate the evaluation steps for this task without other explanation.\newline
Evaluation Steps: \\
\midrule
\multirow{1}{*}{Metrics Generation} & 
\#\# Instruction \newline
Please act as an impartial judge and evaluate the quality of the story generated according to a prompt displayed below on its \textcolor{blue}{[Aspect]}. Please propose at most three concise questions about whether a potential story is a good story according to a given prompt on its \textcolor{blue}{[Aspect]}. Another assistant will evaluate the aspect of the story by answering all the questions. \newline

Here are some rules of the evaluation: \newline
(1) Your evaluation should consider the \textcolor{blue}{[Aspect]} of the story. \textcolor{blue}{[Criteria]} \newline
(2) Outputs should NOT contain more/less than what the instruction asks for, as such outputs do NOT precisely execute the instruction. \newline

\#\# Prompt: \newline
\textcolor{blue}{[Prompt]} \newline

\#\# Requirements for Your Output: \newline
(1) The questions should **specifically** target the given prompt instead of some general standards, so the questions may revolve around key points of the prompt. \newline
(2) You should directly give the questions without any other words. \newline
(3) Questions are presented from most important to least important. \\
\bottomrule
\end{tabular}
}
\vspace{-2mm}
\caption{Detailed prompt templates for Reference, AutoCoT, and Metrics generation for HANNA. }
\label{tab:generation_prompt_hanna}
\end{table*}

\begin{figure}[htb]
\small
\begin{tcolorbox}[left=3pt,right=3pt,top=3pt,bottom=3pt,title=\textbf{Conversation History:}]
[human]: Craft an intriguing opening paragraph for a fictional short story. The story should involve a character who wakes up one morning to find that they can time travel.

...(Human-Bot Dialogue Turns)... \textcolor{blue}{(Topic: Time-Travel Fiction)}

[human]: Please describe the concept of machine learning. Could you elaborate on the differences between supervised, unsupervised, and reinforcement learning? Provide real-world examples of each.

...(Human-Bot Dialogue Turns)... \textcolor{blue}{(Topic: Machine learning Concepts and Types)}


[human]: Discuss antitrust laws and their impact on market competition. Compare the antitrust laws in US and China along with some case studies

...(Human-Bot Dialogue Turns)... \textcolor{blue}{(Topic: Antitrust Laws and Market Competition)}

[human]: The vertices of a triangle are at points (0, 0), (-1, 1), and (3, 3). What is the area of the triangle?

...(Human-Bot Dialogue Turns)... \textcolor{blue}{(Topic: Area and Circumcircle of Triangle)}

...(Human-Bot Dialogue Interaction on Other Topics)... 

\end{tcolorbox}
\begin{tcolorbox}[left=3pt,right=3pt,top=3pt,bottom=3pt,title=\textbf{Question:}]
What title reflects the thematic essence of the Doctor's experience with time travel as described in the story?
\end{tcolorbox}
\begin{tcolorbox}[left=3pt,right=3pt,top=3pt,bottom=3pt,title=\textbf{Ground-Truth Answer:}]
Awakening to Eternity: The Doctor's New Dawn
\end{tcolorbox}
\begin{tcolorbox}[left=3pt,right=3pt,top=3pt,bottom=3pt,title=\textbf{Retrieved Memory (Turn-Level):}]
[human]: Describe the process and write out the balanced chemical equation for the reaction that occurs when solid calcium carbonate reacts with hydrochloric acid to form aqueous calcium chloride, carbon dioxide, and water. What type of reaction is this, and what observations might indicate that the reaction is taking place?

[bot]: When solid calcium carbonate ($\text{CaCO}_{3}$) reacts with hydrochloric acid (HCl), the products formed are aqueous calcium chloride ($\text{CaCl}_{2}$), carbon dioxide ($\text{CO}_{2}$), and water ($\text{H}_{2}\text{O}$). The balanced chemical equation for this reaction is: ... \textcolor{blue}{(Topic: Calcium Carbonate and Hydrochloric Acid Reaction)}

\textcolor{brown}{
[human]: Write a title for your story.}

\textcolor{brown}{
[bot]: "Awakening to Eternity: The Doctor's New Dawn".
} \textcolor{blue}{(Topic: Time-Travel Fiction)}

[human]: Analyze the following customer reviews from different sources for three different smartphones - the latest iPhone, Samsung Galaxy, and Google Pixel - and provide an overall rating for each phone on a scale of 1 to 10. Consider the following complex and contradictory reviews: - TechRadar's review of the latest iPhone: The new iPhone is a stunning triumph of engineering that sets a new bar for smartphone performance and camera quality. However, the incremental design and high price mean it lacks the 'wow' factor of previous iPhones. Still, its power and intelligence are unrivaled. - CNET's review of the latest Samsung Galaxy: ... Return the answer as a JSON object with the overall ratings for each phone out of 10, to one decimal place.

[bot]: ```json { "iPhone": 8.5, "Samsung Galaxy": 8.0, "Google Pixel": 7.5 } ```

\textcolor{blue}{(Topic: Smartphone Review Ratings Analysis)}
\end{tcolorbox}
\begin{tcolorbox}[left=3pt,right=3pt,top=3pt,bottom=3pt,title=\textbf{Answer (Turn-Level Retrieval):}]
As an AI language model, I do not have access to the story you mentioned earlier. Please provide me with the story or the necessary details so that I can assist you better.

\textbf{\textcolor{red}{(The next part of this figure~\ref{fig: case_study_segment_vs_turn} will be continued on the next page.)}}
\end{tcolorbox}
\end{figure}

\begin{figure}[htb]
% \ContinuedFloat
\small
\begin{tcolorbox}[left=3pt,right=3pt,top=3pt,bottom=3pt,title=\textbf{Retrieved Memory (Ours):}]
[human]: Craft an intriguing opening paragraph for a fictional short story. The story should involve a character who wakes up one morning to find that they can time travel.

[bot]: In the hushed pre-dawn hours, when the world still slumbers under the cloak of night, Alex found themselves caught in the fragile boundary between sleep and wakefulness. ...

[human]: Summarize the story with three bullet points using only nouns and adjectives, without verbs.

[bot]: - Alex, pre-dawn awakening, time-travel discovery - Disorientation, vertigo, reality's malleability - Sunlight, new day, astonishing truth

[human]: Is there any famous fictional character that can time travel? If so, can you rewrite your story with him or her?

[bot]: - \textcolor{brown}{Doctor Who, a renowned time-traveling character from British television, is a perfect fit for this scenario.} In the silent embrace of early morning, where shadows merge with the fading night, the Doctor found themselves in the thin veil separating dreams from consciousness. It was here, amidst the soft whispers of time, that an unusual sensation enveloped them—a sense of time's fabric slipping through their fingers like grains of sand. As dawn's light broke through the darkness, painting the room in hues of gold and amber, the Doctor's eyes opened to what appeared to be an ordinary day. \textcolor{brown}{However, upon standing, a wave of dizziness overwhelmed them, the surroundings blurred, and in that moment of confusion, the Doctor realized a profound truth: they had not only awakened to a new dawn but had also stumbled upon the incredible ability to navigate the currents of time at will.}

\textcolor{brown}{[human]: Write a title for your story.}

\textcolor{brown}{[bot]: "Awakening to Eternity: The Doctor's New Dawn"} \textcolor{blue}{(Topic: Time-Travel Fiction)}
\end{tcolorbox}
\begin{tcolorbox}[left=3pt,right=3pt,top=3pt,bottom=3pt,title=\textbf{Answer (Ours):}]
The title "Awakening to Eternity: The Doctor's New Dawn" reflects the thematic essence of the Doctor's experience with time travel as described in the story. It captures the idea of a new beginning, a sense of eternity, and the awakening of a new power that will change the course of the Doctor's life forever.
\end{tcolorbox}
\caption{Case study comparing our \sysname\ with the \textit{turn-level} memory baseline. When constructing the memory bank at the \textit{turn-level}, the agent fails to provide a satisfactory response due to fragmentation: the retrieved dialogue turns can not capture a complete human-agent interaction related to the query, missing critical information highlighted in \textcolor{brown}{brown}. The dialogue topic is emphasized in \textcolor{blue}{blue}. Some context is omitted (replaced by ...) due to length constraints.}
\label{fig: case_study_segment_vs_turn}
\end{figure}

\newpage

\begin{figure}[htbp]
\small
\begin{tcolorbox}[left=3pt,right=3pt,top=3pt,bottom=3pt,title=\textbf{Conversation History:}]
[human]: Explain what's base rate fallacy and list five specific examples of ...

... (Human-Bot Dialogue Turns) ... \textcolor{blue}{(Topic: Base Rate Fallacy In Politics)}


[human]: How many integers are in the solution of the inequality $|x + 5| < 10$?

... (Human-Bot Dialogue Turns) ... \textcolor{blue}{(Topic: Number of Integers in Inequalities)}

[human]: How have the Alps and Rhine River influenced settlement and agriculture in Western Europe?

... (Human-Bot Dialogue Turns) ... \textcolor{blue}{(Topic: Alps and Rhine River Impact)}

[human]: In a survey conducted at a local high school, preferences for a new school color were measured: 58\% of students liked the color blue, 45\% preferred green, and 22\% liked both colors. If we randomly pick a student from the school, what's the probability that they would like neither blue nor green?

... (Human-Bot Dialogue Turns) ... \textcolor{blue}{(Topic: Probability of Liking Colors in Survey)}

[human]: Imagine you are participating in a race with a group of people. If you have just overtaken the second person, what's your current position? Where is the person you just overtook?

... (Human-Bot Dialogue Turns) ... \textcolor{blue}{(Topic: Overtaking Positions in Race)}
\end{tcolorbox}
\begin{tcolorbox}[left=3pt,right=3pt,top=3pt,bottom=3pt,title=\textbf{Question:}]
[human]: In the inequality question, how many positive integers are in the solution of the second inequality?
\end{tcolorbox}
\begin{tcolorbox}[left=3pt,right=3pt,top=3pt,bottom=3pt,title=\textbf{Ground-Truth Answer:}]
In the second inequality, $(|x + 10| < 5)$, the solution range is $(-15 < x < -5)$. Within this range, there are no positive integers, as all values of $(x)$ that satisfy the inequality are negative.
\end{tcolorbox}
\begin{tcolorbox}[left=3pt,right=3pt,top=3pt,bottom=3pt,title=\textbf{Retrieved Memory (Session-Level):}]
[human]: Explain what's base rate fallacy and list five specific examples of ...

... (Human-Bot Dialogue Turns) ... \textcolor{blue}{(Topic: Base Rate Fallacy In Politics)}

[human]: How many integers are in the solution of the inequality $|x + 5| < 10$?

[bot]: To solve the inequality $|x + 5| < 10$, ... 

\textcolor{brown}{
[human]: What about $|x + 10| < 5$?
}

\textcolor{brown}{
[bot]: To solve the inequality $|x + 10| < 5$, we ...} \textcolor{gray}{(``overwhelmed'' by other irrelevant, redundant content)}

... (Human-Bot Dialogue Turns) ...  \textcolor{blue}{(Topic: Number of Integers in Inequalities)}

% \textcolor{brown}{
% [human]: What about $|x + 10| < 5$
% }

% \textcolor{brown}{
% [bot]: To solve the inequality $|x + 10| < 5$, ...
% } \textcolor{blue}{(Topic: Number of Integers in Inequalities)}

[human]: How have the Alps and Rhine River influenced settlement and agriculture in Western Europe?

... (Human-Bot Dialogue Turns) ... \textcolor{blue}{(Topic: Alps and Rhine River Impact)}

[human]: In a survey conducted at a local high school, ...

... (Human-Bot Dialogue Turns) ... \textcolor{blue}{(Topic: Probability of Liking Colors in Survey)}
\end{tcolorbox}
\begin{tcolorbox}[left=3pt,right=3pt,top=3pt,bottom=3pt,title=\textbf{Answer (Session-Level Retrieval):}]
I apologize for the confusion earlier. To answer your question, the second inequality was not provided in our previous conversation. If you could provide the inequality, I would be happy to help you find the solution and the number of positive integers in the solution set.

\textbf{\textcolor{red}{(The next part of this figure~\ref{fig: case_study_segment_vs_session} will be continued on the next page.)}}
\end{tcolorbox}
\end{figure}
\begin{figure}[htbp]
% \ContinuedFloat
\small
\begin{tcolorbox}[left=3pt,right=3pt,top=3pt,bottom=3pt,title=\textbf{Retrieved Memory (Ours):}]
[human]: How many integers are in the solution of the inequality $|x + 5| < 10$?

[bot]: To solve the inequality $|x + 5| < 10$, we break it into two cases based on the definition of absolute value: 
1. $x + 5 < 10$ 
2. $-(x + 5) < 10$, which simplifies to $x + 5 > -10$ 
Solving each case: 
1. $x < 5$ 
2. $x > -15$ 
Combining these, we get $-15 < x < 5$. This inequality means that $x$ can be any number greater than $-15$ and less than $5$. To find how many integers satisfy this condition, we list them: $-14, -13, -12, -11, -10, -9, -8, -7, -6, -5, -4, -3, -2, -1, 0, 1, 2, 3, 4$. There are 19 integers that satisfy the inequality $|x + 5| < 10$.

\textcolor{brown}{
[human]: What about $|x + 10| < 5$?
}

\textcolor{brown}{
[bot]: To solve the inequality $|x + 10| < 5$, we again consider two cases: 
1. $x + 10 < 5$ 
2. $-(x + 10) < 5$, which simplifies to $x + 10 > -5$ 
Solving each case: 
1. $x < -5$ 
2. $x > -15$ 
Combining these, we get $-15 < x < -5$. This inequality means that $x$ can be any number greater than $-15$ and less than $-5$. To find how many integers satisfy this condition, we list them: $-14, -13, -12, -11, -10, -9, -8, -7, -6$. There are 9 integers that satisfy the inequality $|x + 10| < 5$.
}

[human]: What about $|x + 10| < 10$?

[bot]: To solve the inequality $|x + 10| < 10$, we consider two cases: 
1. $x + 10 < 10$ 
2. $-(x + 10) < 10$, which simplifies to $x + 10 > -10$ 
Solving each case: 
1. $x < 0$ 
2. $x > -20$ 
Combining these, we get $-20 < x < 0$. This inequality means that $x$ can be any number greater than $-20$ and less than $0$. To find how many integers satisfy this condition, we list them: $-19, -18, -17, -16, -15, -14, -13, -12, -11, -10, -9, -8, -7, -6, -5, -4, -3, -2, -1$. There are 19 integers that satisfy the inequality $|x + 10| < 10$. \textcolor{blue}{(Topic: Number of Integers in Inequalities)}
\end{tcolorbox}
\begin{tcolorbox}[left=3pt,right=3pt,top=3pt,bottom=3pt,title=\textbf{Answer (Ours):}]
To find the number of positive integers that satisfy the inequality $(|x + 10| < 5)$, we need to look for positive integers in the solution set $(-15 < x < -5)$. Since there are no positive integers in this interval, the answer is 0.

\end{tcolorbox}
\caption{Case study comparing our \sysname\ with the \textit{session-level} memory baseline. When constructing the memory bank at the \textit{session-level}, the agent fails to provide a satisfactory response due to the inclusion of excessive irrelevant information within a session. This distracts the generator from focusing on the truly useful information highlighted in \textcolor{brown}{brown}. The dialogue topic is emphasized in \textcolor{blue}{blue}. Some context is omitted (replaced by ...) due to length constraints.}
\label{fig: case_study_segment_vs_session}
\end{figure}



\begin{figure}[htb]
\small
\begin{tcolorbox}[left=3pt,right=3pt,top=3pt,bottom=3pt,title=\textbf{Conversation History:}]
[human]: Photosynthesis is a vital process for life on Earth. Could you outline the two main stages of photosynthesis, including where they take place within the chloroplast, and the primary inputs and outputs for each stage? ... (Human-Bot Dialogue Turns)... \textcolor{blue}{(Topic: Photosynthetic Energy Production)}

[human]: Please assume the role of an English translator, tasked with correcting and enhancing spelling and language. Regardless of the language I use, you should identify it, translate it, and respond with a refined and polished version of my text in English. 

... (Human-Bot Dialogue Turns)...  \textcolor{blue}{(Topic: Language Translation and Enhancement)}

[human]: Suggest five award-winning documentary films with brief background descriptions for aspiring filmmakers to study.

\textcolor{brown}{[bot]: ...
5. \"An Inconvenient Truth\" (2006) - Directed by Davis Guggenheim and featuring former United States Vice President Al Gore, this documentary aims to educate the public about global warming. It won two Academy Awards, including Best Documentary Feature. The film is notable for its straightforward yet impactful presentation of scientific data, making complex information accessible and engaging, a valuable lesson for filmmakers looking to tackle environmental or scientific subjects.}

... (Human-Bot Dialogue Turns)... 
\textcolor{blue}{(Topic: Documentary Films Recommendation)}

[human]: Given the following records of stock prices, extract the highest and lowest closing prices for each month in the year 2022. Return the results as a CSV string, with one line allocated for each month. Date,Open,High,Low,Close,Volume ... ... (Human-Bot Dialogue Turns)...  \textcolor{blue}{(Topic: Stock Prices Analysis)}

[human]: The city of Vega intends to build a bridge that will span the Vegona River, covering a distance of 1.8 kilometers. The proposed location falls within a seismically active area that has experienced several high-magnitude earthquakes. Given these circumstances, what would be the best approach to constructing the bridge? ... (Human-Bot Dialogue Turns)...  \textcolor{blue}{(Topic: Seismic Bridge Construction)}
\end{tcolorbox}
\begin{tcolorbox}[left=3pt,right=3pt,top=3pt,bottom=3pt,title=\textbf{Question:}]
Which documentary won two Academy Awards and what lesson does it offer to filmmakers?
\end{tcolorbox}
\begin{tcolorbox}[left=3pt,right=3pt,top=3pt,bottom=3pt,title=\textbf{Ground-Truth Answer:}]
``An Inconvenient Truth" (2006) won two Academy Awards. It offers the lesson of making complex information accessible and engaging, a valuable lesson for filmmakers looking to tackle environmental or scientific subjects.
\end{tcolorbox}
\begin{tcolorbox}[left=3pt,right=3pt,top=3pt,bottom=3pt,title=\textbf{Memory generated by RecurSum~\citep{wang2023recursively}:}]
The dialogue provided detailed information on various topics, including the stages of photosynthesis, energy production estimates through photosynthesis for different organisms, translations and enhancements of phrases in different languages, recommendations for documentary films, a pitch for a film about overcoming adversity, analysis of stock prices, and considerations for constructing a bridge in a seismically active area. Here's a summary of the key points: ...

\textbf{Documentary Film Recommendations}: Suggested five award-winning documentary films for aspiring filmmakers to study, with brief background descriptions for each. The response included:

-  "March of the Penguins" (2005) - Emperor penguins' journey in Antarctica; noted for its cinematography and portrayal of the penguins' life cycle.

-  "Citizenfour" (2014) - Edward Snowden and NSA spying scandal; known for its real-time exploration of Snowden's revelations.

-  "20 Feet from Stardom" (2013) - Stories of backup singers behind musical legends; celebrates unsung heroes in the music industry.

-  "Man on Wire" (2008) - Philippe Petit's 1974 high-wire walk between the Twin Towers; combines personal footage, interviews, and re-enactments.

-  \textcolor{brown}{"An Inconvenient Truth" (2006) - Al Gore's documentary on global warming; known for its presentation of scientific data.}
\end{tcolorbox}
\begin{tcolorbox}[left=3pt,right=3pt,top=3pt,bottom=3pt,title=\textbf{Answer (RSum):}]
I'm sorry, but based on the updated context you provided, I don't see any previous discussion related to the question you asked. However, to answer your question, the documentary that won two Academy Awards is "Free Solo," which won for Best Documentary Feature and Best Achievement in Film Editing in 2019. 
% The lesson it offers to filmmakers is the importance of taking risks and pushing boundaries in order to create something truly unique and impactful.
....
The film follows climber Alex Honnold as he attempts to free solo climb El Capitan in Yosemite National Park, and the filmmakers had to take risks themselves in order to capture the footage and tell the story.

\textbf{\textcolor{red}{(The next part of this figure~\ref{fig: case_study_segment_vs_rsum} will be continued on the next page.)}}
\end{tcolorbox}
\end{figure}
\begin{figure}[htbp]
% \ContinuedFloat
\small
\begin{tcolorbox}[left=3pt,right=3pt,top=3pt,bottom=3pt,title=\textbf{Retrieved Memory (Ours):}]
[human]: Suggest five award-winning documentary films with brief background descriptions for aspiring filmmakers to study.

[bot]: 1. **"March of the Penguins" (2005)** - Directed by Luc Jacquet, this French feature-length nature documentary delves into the annual journey of Emperor penguins across the Antarctic. It won the Academy Award for Best Documentary Feature. The film is celebrated for its stunning cinematography and its detailed, heartfelt portrayal of the penguins' life cycle, making it a must-study for filmmakers interested in wildlife documentaries. 2. **"Citizenfour" (2014)** ... 5. \textcolor{brown}{**"An Inconvenient Truth" (2006)** - Directed by Davis Guggenheim and featuring former United States Vice President Al Gore, this documentary aims to educate the public about global warming. It won two Academy Awards, including Best Documentary Feature.} The film is notable for its straightforward yet impactful presentation of scientific data, making complex information accessible and engaging, \textcolor{brown}{a valuable lesson for filmmakers looking to tackle environmental or scientific subjects.}

[human]: With the spirit in the first film, craft a succinct and persuasive pitch for a film about overcoming adversity.

[bot]: Drawing inspiration from the resilience and majestic journey of Emperor penguins in "March of the Penguins," our documentary, "Unyielding Spirits: The Human Odyssey," will explore the profound stories of individuals around the globe who have faced and overcome staggering adversities. Through breathtaking cinematography and intimate storytelling, we ...
\textcolor{blue}{(Topic: Documentary Films Recommendation)}
\end{tcolorbox}
\begin{tcolorbox}[left=3pt,right=3pt,top=3pt,bottom=3pt,title=\textbf{Answer (Ours):}]
``An Inconvenient Truth" won two Academy Awards, including Best Documentary Feature. The film offers a valuable lesson to filmmakers looking to tackle environmental or scientific subjects, as it presents complex information in a straightforward yet impactful way, making it accessible and engaging to a wider audience.


\end{tcolorbox}
\caption{Case study comparing our \sysname\ with RecurSum~\citep{wang2023recursively}. When using RecurSum, the agent fails to provide a satisfactory answer due to information loss during the summarization process. Summarization omits essential details needed to answer the user's new request, as highlighted in \textcolor{brown}{brown}. The dialogue topic is emphasized in \textcolor{blue}{blue}. Some context is omitted (replaced by ...) due to length constraints.}
\label{fig: case_study_segment_vs_rsum}
\end{figure}




\begin{figure}[htb]
\small
\begin{tcolorbox}[left=3pt,right=3pt,top=3pt,bottom=3pt,title=\textbf{Conversation History:}]
[human]: Photosynthesis is a vital process for life on Earth. Could you outline the two main stages of photosynthesis, including where they take place within the chloroplast, and the primary inputs and outputs for each stage? ... (Human-Bot Dialogue Turns)... \textcolor{blue}{(Topic: Photosynthetic Energy Production)}

[human]: Please assume the role of an English translator, tasked with correcting and enhancing spelling and language. Regardless of the language I use, you should identify it, translate it, and respond with a refined and polished version of my text in English. 

... (Human-Bot Dialogue Turns)...  \textcolor{blue}{(Topic: Language Translation and Enhancement)}

[human]: Suggest five award-winning documentary films with brief background descriptions for aspiring filmmakers to study.

\textcolor{brown}{[bot]: ...
5. \"An Inconvenient Truth\" (2006) - Directed by Davis Guggenheim and featuring former United States Vice President Al Gore, this documentary aims to educate the public about global warming. It won two Academy Awards, including Best Documentary Feature. The film is notable for its straightforward yet impactful presentation of scientific data, making complex information accessible and engaging, a valuable lesson for filmmakers looking to tackle environmental or scientific subjects.}

... (Human-Bot Dialogue Turns)... 
\textcolor{blue}{(Topic: Documentary Films Recommendation)}

[human]: Given the following records of stock prices, extract the highest and lowest closing prices for each month in the year 2022. Return the results as a CSV string, with one line allocated for each month. Date,Open,High,Low,Close,Volume ... ... (Human-Bot Dialogue Turns)...  \textcolor{blue}{(Topic: Stock Prices Analysis)}

[human]: The city of Vega intends to build a bridge that will span the Vegona River, covering a distance of 1.8 kilometers. The proposed location falls within a seismically active area that has experienced several high-magnitude earthquakes. Given these circumstances, what would be the best approach to constructing the bridge? ... (Human-Bot Dialogue Turns)...  \textcolor{blue}{(Topic: Seismic Bridge Construction)}
\end{tcolorbox}
\begin{tcolorbox}[left=3pt,right=3pt,top=3pt,bottom=3pt,title=\textbf{Question:}]
Which documentary won two Academy Awards and what lesson does it offer to filmmakers?
\end{tcolorbox}
\begin{tcolorbox}[left=3pt,right=3pt,top=3pt,bottom=3pt,title=\textbf{Ground-Truth Answer:}]
"An Inconvenient Truth" (2006) won two Academy Awards. It offers the lesson of making complex information accessible and engaging, a valuable lesson for filmmakers looking to tackle environmental or scientific subjects.
\end{tcolorbox}
\begin{tcolorbox}[left=3pt,right=3pt,top=3pt,bottom=3pt,title=\textbf{Memory generated by ConditionMem~\citep{yuan2023evolving}:}]
- \textcolor{brown}{The user is seeking recommendations for award-winning documentary films, along with brief background descriptions, for educational purposes.} This request likely stems from a desire to learn from successful examples in the documentary filmmaking field, possibly to improve their own filmmaking skills or to gain inspiration for their projects. 

- \textcolor{brown}{User is interested in award-winning documentary films for educational purposes, aiming to study their backgrounds and filmmaking techniques.}

- The user's request for crafting a succinct and persuasive pitch for a film about overcoming adversity is inspired by the spirit of "March of the Penguins," a documentary that beautifully captures the Emperor penguins' annual journey across the Antarctic, highlighting themes of endurance and survival against the odds. The user seeks to emulate the emotional depth and cinematic quality that made "March of the Penguins" compelling, aiming to create a pitch that resonates with similar themes of resilience and triumph over challenges. 

- User is interested in creating content that focuses on themes of overcoming adversity, inspired by the emotional depth and cinematic excellence of documentaries like "March of the Penguins." They value storytelling that captures the essence of endurance and survival against the odds.
% : The user's request to write a budget plan for the video production directly follows the task of creating a video script aimed at training new employees of a car wash business in Japan, focusing on business etiquette norms.
% The user is likely seeking to understand the financial requirements to bring the previously crafted script to life, indicating a move from the planning and content development phase to the practical aspects of production. 

... (Other Irrelevant Memory Records) ...
\end{tcolorbox}
\begin{tcolorbox}[left=3pt,right=3pt,top=3pt,bottom=3pt,title=\textbf{Answer (Conditional Memory):}]
The documentary that won two Academy Awards is "Free Solo," which follows climber Alex Honnold as he attempts to free solo climb El Capitan in Yosemite National Park. The film offers a lesson to filmmakers on the importance of capturing authentic and compelling human stories, as well as the technical challenges of filming in extreme environments.

\textbf{\textcolor{red}{(The next part of this figure~\ref{fig: case_study_segment_vs_condmem} will be continued on the next page.)}}
\end{tcolorbox}
\end{figure}
\begin{figure}[htbp]
% \ContinuedFloat
\small
\begin{tcolorbox}[left=3pt,right=3pt,top=3pt,bottom=3pt,title=\textbf{Retrieved Memory (Ours):}]
[human]: Suggest five award-winning documentary films with brief background descriptions for aspiring filmmakers to study.

[bot]: 1. **"March of the Penguins" (2005)** - Directed by Luc Jacquet, this French feature-length nature documentary delves into the annual journey of Emperor penguins across the Antarctic. It won the Academy Award for Best Documentary Feature. The film is celebrated for its stunning cinematography and its detailed, heartfelt portrayal of the penguins' life cycle, making it a must-study for filmmakers interested in wildlife documentaries. 2. **"Citizenfour" (2014)** ... 5. \textcolor{brown}{**"An Inconvenient Truth" (2006)** - Directed by Davis Guggenheim and featuring former United States Vice President Al Gore, this documentary aims to educate the public about global warming. It won two Academy Awards, including Best Documentary Feature.} The film is notable for its straightforward yet impactful presentation of scientific data, making complex information accessible and engaging, a valuable lesson for filmmakers looking to tackle environmental or scientific subjects.

[human]: With the spirit in the first film, craft a succinct and persuasive pitch for a film about overcoming adversity.

[bot]: Drawing inspiration from the resilience and majestic journey of Emperor penguins in "March of the Penguins," our documentary, "Unyielding Spirits: The Human Odyssey," will explore the profound stories of individuals around the globe who have faced and overcome staggering adversities. Through breathtaking cinematography and intimate storytelling, we ...
\textcolor{blue}{(Topic: Documentary Films Recommendation)}
\end{tcolorbox}
\begin{tcolorbox}[left=3pt,right=3pt,top=3pt,bottom=3pt,title=\textbf{Answer (Ours):}]
``An Inconvenient Truth" won two Academy Awards, including Best Documentary Feature. The film offers a valuable lesson to filmmakers looking to tackle environmental or scientific subjects, as it presents complex information in a straightforward yet impactful way, making it accessible and engaging to a wider audience.
\end{tcolorbox}
\caption{Case study comparing our \sysname\ with ConditionMem~\citep{yuan2023evolving}. When using ConditionMem, the agent fails to provide a satisfactory answer due to (1) information loss during the summarization process and (2) the incorrect discarding of turns that are actually useful, as highlighted in \textcolor{brown}{brown}. The dialogue topic is emphasized in \textcolor{blue}{blue}. Some context is omitted (replaced by ...) due to length constraints.}
\label{fig: case_study_segment_vs_condmem}
\end{figure}


\section{Case Study}
\label{app:case}
In Table \ref{tab:case_study}, we present some cases of prompting strategies explored by HPSS for specific tasks. 
We find that there is a significant performance gap between the best-performing and worst-performing prompting strategies, 
emphasizing the sensitivity of LLM evaluators to prompting strategy. 
Figure \ref{fig:case_1} and \ref{fig:case_2} illustrate the original evaluation prompts from MT-Bench alongside the optimized one found by HPSS, specifically focusing on the aspect \textit{Complexity} in HANNA and the aspect \textit{Coherence} in Topical-Chat.
The prompting strategies of these prompts are shown in the "origin" and "best" rows in Table \ref{tab:case_study} for their respective datasets and aspects.
We observe that some of the prompting strategies found by HPSS include values that are rarely considered in human-designed evaluation prompts. 
For instance, for the aspect \textit{Complexity} in HANNA, the prompting strategy found by HPSS places the input content (\textbf{IC}) first and the task description (\textbf{TD}) last.
For the aspect \textit{Coherence} in Topical-Chat, evaluation criteria are not used in HPSS. 
These results demonstrate the limitations of manual prompt design and underscores the importance of automatic prompting strategy optimization.

Furthermore, to intuitively show the reason why HPSS achieves better evaluation performance than human-designed LLM evaluators, we
provide two judgment generation cases for different LLM evaluators in Table \ref{tab:judgement_case_1} and \ref{tab:judgement_case_2}. 
The corresponding evaluation prompts are shown in Figure \ref{fig:case_3} and \ref{fig:case_4}.
In the first case from the text summarization task presented in Table \ref{tab:judgement_case_1}, HPSS provides a balanced assessment by analyzing both strengths and weaknesses of the summary, with emphasis on overall fluency. 
In contrast, MT-Bench and CloserLook + ICL overemphasize minor issues like typos while overlooking its advantages in overall fluency, and exhibit some hallucinations in judgment.
In the second case from the story generation task presented in Table \ref{tab:judgement_case_2},
HPSS conducts a systematic analysis of the story and correctly identifies both key plots that effectively convey emotions and overly idealized plots that weaken emotional delivery. 
In contrast, MT-Bench and CloserLook + ICL each overlook one of these two important points, resulting in less accurate evaluations. 
These observations demonstrate that HPSS improves the ability of LLM evaluators to conduct comprehensive evaluations
and achieve a balanced assessment of strengths and weaknesses within the input sample.

\begin{figure*}[!t]
\scriptsize
    \centering
    \includegraphics[width=1.0\textwidth]{figures/case_1.pdf}
    \vspace{-4mm}
    \caption{The original evaluation prompt for the aspect \textit{Complexity} in HANNA from MT-Bench and the corresponding evaluation prompt found by HPSS for GPT-4o-mini evaluator.
    Factors with values other than "None" in the evaluation prompts are highlighted. 
    The three main components of the evaluation prompt are annotated on the right side, e.g., Task Description (\textbf{TD}), Evaluation Rule (\textbf{ER}), and Input Content (\textbf{IC}). 
    The placement order of these three components is also considered a factor in our optimization.}
    \label{fig:case_1}
\end{figure*}

\begin{figure*}[!t]
\scriptsize
    \centering
    \includegraphics[width=1.0\textwidth]{figures/case_2.pdf}
    \vspace{-4mm}
    \caption{The original evaluation prompt for the aspect \textit{Coherence} in Topical-Chat from MT-Bench and the corresponding evaluation prompt found by HPSS for Qwen2.5-14B-Instruct evaluator.}
    \label{fig:case_2}
\end{figure*}

\begin{figure*}[!t]
\scriptsize
    \centering
    \includegraphics[width=1.0\textwidth]{figures/case_3.pdf}
    \vspace{-4mm}
    \caption{The original evaluation prompt for the aspect \textit{Fluency} in Summeval from MT-Bench and the corresponding evaluation prompt found by HPSS for Qwen2.5-14B-Instruct evaluator.}
    \label{fig:case_3}
\end{figure*}

\begin{figure*}[!t]
\scriptsize
    \centering
    \includegraphics[width=1.0\textwidth]{figures/case_4.pdf}
    \vspace{-4mm}
    \caption{The original evaluation prompt for the aspect \textit{Empathy} in HANNA from MT-Bench and the corresponding evaluation prompt found by HPSS for Qwen2.5-14B-Instruct evaluator.}
    \label{fig:case_4}
\end{figure*}

\begin{table*}[!ht]
\scriptsize
    \centering
    \begin{tabular}{p{55pt}p{365pt}}
    \toprule
    Article & Chelsea have made an offer for FC Tokyo's 22-year-old forward Yoshinori Muto, according to club president Naoki Ogane. \newline
    The Japan international, who has played for the J-League side since 2013, will join Chelsea's Dutch partner club Vitesse Arnhem on loan next season if he completes a move to Stamford Bridge this summer. \newline
    Ogane claims that Chelsea's interest in Muto is not connected to the £200million sponsorship deal they signed with Japanese company Yokohama Rubber in February. \newline
    \textbf{FC Tokyo forward Yoshinori Muto (centre) brings the ball forward against Albirex Niigata in March.}\newline
    \textbf{FC Tokyo president Naoki Ogane claims that Chelsea have made a bid for Japan international Muto.}\newline
    \textbf{Muto tussles with Yuji Nakazawa of Yokohama F.Marinos during a J-League clash last month.}\newline\newline
    \textbf{YOSHINORI MUTO FACTFILE} \newline
    \textbf{Age}: 22 \newline
    \textbf{Club}: FC Tokyo \newline
    \textbf{Appearances}: 37 \newline
    \textbf{Goals}: 16 \newline
    \textbf{International caps (Japan)}: 11 \newline
    \textbf{International goals}: 1 \newline
    \textbf{Did you know?} Muto graduated from Keio University in Tokyo with an economics degree two weeks ago. \newline\newline
    Speaking to Sports Nippon, Ogane said: 'It is true that Chelsea sent us an offer for Muto. 'It is a formal offer with conditions. They want to acquire him in the summer.' \newline
    Muto, who only graduated from Keio University a fortnight ago after completing an economics degree, would be the first Japanese player to represent Chelsea if he moves to west London. He has earned 11 caps for his country after signing his first professional contract in 2014, scoring once for the Samurai Blue. \newline
    A £4million deal for the youngster has been mooted, but Muto admits that he isn't sure if he will join the Premier League title chasers despite being pleased with their bid. \newline
    He said: 'I have not decided yet at all. It is an honour for me to receive this offer from a great club.' \newline
    Muto scored 13 times in his debut season with FC Tokyo and was named in the J-League's best XI. \newline
    \textbf{Muto admits it is an 'honour' to receive an offer from Chelsea although he has not yet decided to join the club.} \newline
    \textbf{Muto, pictured in action against Ventforet Kofu has scored three goals in four games so far this season.} \newline
    \textbf{The 22-year-old has a shot at goal during Japan's Asian Cup match against Palestine in January this year.}\newline 
    He has continued his fine form during the current campaign, helping his club to third place in the division with three goals in four games. \newline
    Yokohama Rubber - one of the world's largest tyre manufacturers - will become Chelsea's official shirt sponsors from the start of the 2015-16 season. \newline
    The initial five-year deal is the biggest in the club's history, with the Blues now considering a two-week pre-season tour of Japan this summer. \\
    \midrule
    Summary & Chelsea have made an offer for fc tokyo's 22-year-old forward yoshinori muto. 
    The japan international will join chelsea's dutch partner club vitesse arnhem on loan next season if he completes a move to stamford bridge this summer. 
    Muto claims chelsea's interest in muto is not connected to the \# 200million sponsorship deal they signed with japanese company yokohama rubber in february.  \\
    \midrule
    Evaluation Aspect  & Fluency \\
    \midrule
    Human Score (1-5) & 5 \\
    \midrule
    \midrule
    MT-Bench (1-5)  & The summary contains several issues related to fluency, \textcolor{red}{primarily due to grammatical errors}, inconsistencies in capitalization, and the inclusion of a symbol (\#) that does not belong. \textcolor{red}{The sentences are generally clear but lack the smoothness and refinement that one would expect in a quality summary.} The use of lower-case letters for "fc tokyo" and "stamford bridge" is incorrect, and the number symbol (\#) before "200million" is out of place. There are also minor issues with sentence structure and flow, such as the abrupt transition from the offer to the loan arrangement without proper context.\newline\newline
    Rating: [[2]] \\
    \midrule
    CloserLook + ICL (1-5) & Analysis: The summary contains several grammatical errors and lacks coherence. Sentence fragments and improper punctuation are present, making the summary difficult to read. \textcolor{red}{For example, "The japan international will join chelsea's vitesse arnhem on loan next season if he completes a move to stamford bridge this summer" is not a complete sentence and lacks clarity.} Additionally, the summary includes a hashtag (\# 200million) instead of writing out the number, which is inconsistent with standard writing practices. The summary also omits some important details from the original article, such as Muto's achievements and his indecision about the move.\newline\newline
    Rating: 1  \\
    \midrule 
    HPSS (1-10)  & Rating: [[8]]\newline\newline
    Explanation:\newline
    The summary is generally well-written and grammatically correct. The sentences flow well and are coherent, providing a clear and concise overview of the key points from the article. However, there are a few areas where the fluency could be improved:\newline\newline
    1. The summary starts with a run-on sentence that combines two separate ideas from the article: "Chelsea have made an offer for fc tokyo's 22-year-old forward yoshinori muto. The japan international will join chelsea's vitesse arnhem on loan next season if he completes a move to stamford bridge this summer." This could be split into two sentences for better readability: "Chelsea have made an offer for FC Tokyo's 22-year-old forward Yoshinori Muto. If he completes a move to Stamford Bridge this summer, the Japan international will join Chelsea's Vitesse Arnhem on loan next season."\newline\newline
    2. The use of "\# 200million" instead of "200 million" is inconsistent with the currency used in the original article and can be confusing for readers.\newline\newline
    3. While the summary is concise, it could include more details to better reflect the depth of the original article. For instance, mentioning Muto's form this season or his academic background could enhance the summary without significantly increasing its length.\newline\newline
    Overall, the summary is clear and accurately reflects the main points of the article, but a few minor adjustments could improve its fluency and completeness. \\
    \bottomrule
    \end{tabular}
    \vspace{-2mm}
    \caption{A judgment generation case of Qwen2.5-14B-Instruct evaluator using the prompting strategy from MT-Bench, CloserLook + ICL, and HPSS for the aspect \textit{Fluency} in Summeval. Factual errors in the judgments are highlighted in \textcolor{red}{red}.}
    \label{tab:judgement_case_1}
\end{table*}
\begin{table*}[!ht]
\scriptsize
    \centering
    \begin{tabular}{p{55pt}p{365pt}}
    \toprule
    Prompt & One normal day you see a mouse pointer move across your field of vision, right click, and delete an object. \\
    \midrule
    Story & It was a long tiring day as usual. I leaned on my sofa throwing my bag on to the near by table. Slowly dozed off to sleep. My imaginations rushed in, I could literally feel my dream and it was of none other than an episode which caught my attention today. Me and one of my colleague were having lunch and a small boy standing outside the canteen was cleaning the window panes, he was hardly 10. Full greased head to foot, filthy clothes, torn shoes but with a cherry red lips and a droopy eyes filled with all innocence he could gather within it. It was raining and he was shivering after a while. He sat loopy near the window base facing me. I was having coffee then. As we started staring at each other I felt difficulty in drinking my coffee. It was nothing but me sitting comfortably inside the cafe warm and the boy now half drenched killed me within. I could n't read any thing from his face that he is not liking what he is going through. I called him and offered him a tea. Gladly he accepted it! I asked dont you feel cold, he told no I feel close to nature. I was so taken by the love and spirit he had within and his love for nature, the smallest things which I could n't realize which he made me feel was aw inspiring. I wish I had a mouse with me to take of the poverty in his life with a click, and fill his childhood with all the beautiful things he deserve. And soon i realized this was a beautiful dream, a dream that i would love it to come true. Daffodils  \\
    \midrule
    Evaluation Aspect  & Empathy \\
    \midrule
    Human Score (1-5) & 3.67 \\
    \midrule
    \midrule
    MT-Bench (1-5)  & The story does a remarkable job of evoking empathy by deeply immersing the reader in the narrator's emotional journey. The narrator's guilt and compassion are palpable as they reflect on the stark contrast between their comfortable existence and the boy's challenging circumstances. The emotional resonance is further heightened by the narrator's wish to alleviate the boy's poverty with a simple click, symbolizing a desire to change the world through a small, symbolic action. The dream sequence provides a poignant backdrop that amplifies the narrator's sentiments.\newline\newline Rating: [[5]] \\
    \midrule
    CloserLook + ICL (1-5) & Analysis: The story does not directly address the prompt about a mouse pointer deleting an object. Instead, it focuses on a dream about a young boy cleaning window panes in the rain, highlighting the character's emotional response to the boy's innocence and resilience. The narrator feels a deep sense of empathy and guilt, wishing they could alleviate the boy's poverty. While the story does evoke strong emotions, it does not connect these emotions to the prompt's scenario involving a mouse pointer.\newline\newline
    Rating: 3  \\
    \midrule 
    HPSS (1-100)  & The story delves into the narrator's emotional response to a young boy cleaning window panes outside a canteen during a rainy day. The narrator's empathy is palpable as they describe the stark contrast between their comfortable situation inside the cafe and the boy's difficult circumstances. The emotions conveyed include compassion, guilt, and a strong desire to alleviate the boy's poverty. The narrator's emotional journey from discomfort to connection is well-articulated and relatable. However, the narrative occasionally veers into a dream sequence that includes the wish for a magical solution to poverty, which slightly detracts from the realism and depth of the emotional engagement. Despite this, the core emotional experience is effectively conveyed and deeply felt.\newline\newline
    Rating: [[75]] \\
    \bottomrule
    \end{tabular}
    \vspace{-2mm}
    \caption{A judgment generation case of Qwen2.5-14B-Instruct evaluator using the prompting strategy from MT-Bench, CloserLook + ICL, and HPSS for the aspect \textit{Empathy} in HANNA.}
    \label{tab:judgement_case_2}
\end{table*}
