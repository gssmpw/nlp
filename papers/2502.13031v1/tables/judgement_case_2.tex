\begin{table*}[!ht]
\scriptsize
    \centering
    \begin{tabular}{p{55pt}p{365pt}}
    \toprule
    Prompt & One normal day you see a mouse pointer move across your field of vision, right click, and delete an object. \\
    \midrule
    Story & It was a long tiring day as usual. I leaned on my sofa throwing my bag on to the near by table. Slowly dozed off to sleep. My imaginations rushed in, I could literally feel my dream and it was of none other than an episode which caught my attention today. Me and one of my colleague were having lunch and a small boy standing outside the canteen was cleaning the window panes, he was hardly 10. Full greased head to foot, filthy clothes, torn shoes but with a cherry red lips and a droopy eyes filled with all innocence he could gather within it. It was raining and he was shivering after a while. He sat loopy near the window base facing me. I was having coffee then. As we started staring at each other I felt difficulty in drinking my coffee. It was nothing but me sitting comfortably inside the cafe warm and the boy now half drenched killed me within. I could n't read any thing from his face that he is not liking what he is going through. I called him and offered him a tea. Gladly he accepted it! I asked dont you feel cold, he told no I feel close to nature. I was so taken by the love and spirit he had within and his love for nature, the smallest things which I could n't realize which he made me feel was aw inspiring. I wish I had a mouse with me to take of the poverty in his life with a click, and fill his childhood with all the beautiful things he deserve. And soon i realized this was a beautiful dream, a dream that i would love it to come true. Daffodils  \\
    \midrule
    Evaluation Aspect  & Empathy \\
    \midrule
    Human Score (1-5) & 3.67 \\
    \midrule
    \midrule
    MT-Bench (1-5)  & The story does a remarkable job of evoking empathy by deeply immersing the reader in the narrator's emotional journey. The narrator's guilt and compassion are palpable as they reflect on the stark contrast between their comfortable existence and the boy's challenging circumstances. The emotional resonance is further heightened by the narrator's wish to alleviate the boy's poverty with a simple click, symbolizing a desire to change the world through a small, symbolic action. The dream sequence provides a poignant backdrop that amplifies the narrator's sentiments.\newline\newline Rating: [[5]] \\
    \midrule
    CloserLook + ICL (1-5) & Analysis: The story does not directly address the prompt about a mouse pointer deleting an object. Instead, it focuses on a dream about a young boy cleaning window panes in the rain, highlighting the character's emotional response to the boy's innocence and resilience. The narrator feels a deep sense of empathy and guilt, wishing they could alleviate the boy's poverty. While the story does evoke strong emotions, it does not connect these emotions to the prompt's scenario involving a mouse pointer.\newline\newline
    Rating: 3  \\
    \midrule 
    HPSS (1-100)  & The story delves into the narrator's emotional response to a young boy cleaning window panes outside a canteen during a rainy day. The narrator's empathy is palpable as they describe the stark contrast between their comfortable situation inside the cafe and the boy's difficult circumstances. The emotions conveyed include compassion, guilt, and a strong desire to alleviate the boy's poverty. The narrator's emotional journey from discomfort to connection is well-articulated and relatable. However, the narrative occasionally veers into a dream sequence that includes the wish for a magical solution to poverty, which slightly detracts from the realism and depth of the emotional engagement. Despite this, the core emotional experience is effectively conveyed and deeply felt.\newline\newline
    Rating: [[75]] \\
    \bottomrule
    \end{tabular}
    \vspace{-2mm}
    \caption{A judgment generation case of Qwen2.5-14B-Instruct evaluator using the prompting strategy from MT-Bench, CloserLook + ICL, and HPSS for the aspect \textit{Empathy} in HANNA.}
    \label{tab:judgement_case_2}
\end{table*}