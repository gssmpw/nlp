
\definecolor{customcolor1}{RGB}{255, 231, 218}
\definecolor{customcolor2}{RGB}{218, 233, 248}
\definecolor{customcolor3}{RGB}{255, 245, 213}

\begin{table*} [t]
\scriptsize
\centering
\setlength{\tabcolsep}{1.6mm}{
\begin{tabular}{l l >{\raggedright\arraybackslash}m{0.7\textwidth}}
% {m{0.12\linewidth}p{0.1\linewidth}p{0.78\linewidth}}
\toprule
\textbf{Template} & \textbf{Value} & \textbf{Prompt} \\
\midrule
\multirow{5}{*}[-7em]{Backbone} &  \multirow{5}{*}[-7em]{-} & 
\begin{tabular}[t]{@{}>{\cellcolor{customcolor1}}p{0.7\textwidth}}
\#\# Instruction \newline
Please act as an impartial judge and evaluate the quality of the response for the next turn in the conversation displayed below on its \textcolor{blue}{[Aspect]}. The response concerns an interesting fact, which will be provided as well. \textcolor{red}{\{reference\_1\_template\}} \textcolor{red}{\{reference\_dialectic\_template\}} \textcolor{red}{\{chain\_of\_thought\_template\}} \\
\end{tabular} \\
& & \  \\
& & 
\begin{tabular}[t]{@{}>{\cellcolor{customcolor2}}p{0.7\textwidth}}
Here are some rules of the evaluation: \newline
1. Your evaluation should consider the \textcolor{blue}{[Aspect]} of the response. \textcolor{blue}{[Criteria]} \newline
2. Be as objective as possible. \newline\newline
\textcolor{red}{\{autocot\_template\}} \\
\end{tabular} \\
& & \ \\
& &
\begin{tabular}[t]{@{}>{\cellcolor{customcolor3}}p{0.7\textwidth}}
\textcolor{red}{\{in\_context\_example\_template\}} \newline
\#\# Conversation History \newline
\textcolor{blue}{[Conversation History]} \newline\newline
\textcolor{red}{\{metrics\_template\}} \newline\newline
\#\# Corresponding Fact \newline
\textcolor{blue}{[Corresponding Fact]} \newline\newline
\textcolor{red}{\{reference\_2\_template\}} \newline\newline
\#\# The Start of Response \newline
\textcolor{blue}{[Response]} \newline
\#\# The End of the Response \\
\end{tabular} \\
\midrule
\multirow{1}{*}{Reference 1} & - & You will also be given a high-quality reference response with the conversation. \\
\midrule
\multirow{1}{*}{Reference 2} & - & \#\# The Start of Reference Response \newline
\textcolor{blue}{[Reference]} \newline
\#\# The End of Reference Response \\
\midrule
\multirow{1}{*}{Reference Dialectic} & - & Please generate your own response for the next turn in the conversation first and take into account your own response to evaluate the quality of the given response. \\
\midrule
\multirow{3}{*}[-1.85em]{Chain-of-Thought} & No CoT & You must directly output your rating of the response on a scale of 1 to \textcolor{red}{\{max\}} without any explanation by strictly following this format: "[[rating]]", for example: "Rating: [[\textcolor{red}{\{max\}}]]". \\
\cmidrule{2-3}
 & Prefix CoT & Begin your evaluation by providing a short explanation. After providing your explanation, you must rate the response on a scale of 1 to \textcolor{red}{\{max\}} by strictly following this format: "[[rating]]", for example: "Rating: [[\textcolor{red}{\{max\}}]]". \\
 \cmidrule{2-3}
 & Suffix CoT & You must rate the response on a scale of 1 to \textcolor{red}{\{max\}} first by strictly following this format: "[[rating]]", for example: "Rating: [[\textcolor{red}{\{max\}}]]". And then provide your explanation. \\
\midrule
\multirow{5}{*}[-1.5em]{Scoring Scale} & 3 &  \textcolor{red}{\{max\}} $ = 3 $ \\
\cmidrule{2-3}
 & 5 & \textcolor{red}{\{max\}} $ = 5 $ \\
\cmidrule{2-3}
 & 10 & \textcolor{red}{\{max\}} $ = 10 $ \\
\cmidrule{2-3}
 & 50 & \textcolor{red}{\{max\}} $ = 50 $ \\
\cmidrule{2-3}
 & 100 & \textcolor{red}{\{max\}} $ = 100 $ \\
\midrule
\multirow{1}{*}{AutoCoT} & - & Evaluation Steps: \newline
\textcolor{blue}{[Autocot]} \\
\midrule
\multirow{1}{*}{In-Context Example} & - & Here are some examples and their corresponding ratings: \newline
\textcolor{red}{\{example\_template\_1\}} \newline
\textcolor{red}{\{example\_template\_2\}} \newline
\textcolor{red}{…} \newline
\textcolor{red}{\{example\_template\_n\}}\newline\newline
Following these examples, evaluate the quality of the response for the next turn in the conversation displayed below on its \textcolor{blue}{[Aspect]}: \\
\midrule
Example & - & \#\# Example \textcolor{blue}{[Number]}:\newline \#\# Conversation History\newline\textcolor{blue}{[Conversation History]}\newline\newline \#\# Corresponding Fact\newline\textcolor{blue}{[Corresponding Fact]}\newline\newline \#\# The Start of the Response \newline\textcolor{blue}{[Response]}\newline \#\# The End of the Response\newline\newline \#\# Rating \newline \textcolor{blue}{[Human Rating]}\\
\midrule
\multirow{1}{*}{Metrics} & - & \#\# Questions about Response \newline
Here are some questions about the response. You can do the evaluation based on thinking about all the questions. \newline
\textcolor{blue}{[Metrics]} \\
\bottomrule
\end{tabular}
}
\vspace{-2mm}
\caption{Detailed evaluation prompt templates for Topical-Chat.}
\label{tab:evaluation_prompt_topical_chat}
\end{table*}