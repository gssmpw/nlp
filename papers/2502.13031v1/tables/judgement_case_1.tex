\begin{table*}[!ht]
\scriptsize
    \centering
    \begin{tabular}{p{55pt}p{365pt}}
    \toprule
    Article & Chelsea have made an offer for FC Tokyo's 22-year-old forward Yoshinori Muto, according to club president Naoki Ogane. \newline
    The Japan international, who has played for the J-League side since 2013, will join Chelsea's Dutch partner club Vitesse Arnhem on loan next season if he completes a move to Stamford Bridge this summer. \newline
    Ogane claims that Chelsea's interest in Muto is not connected to the £200million sponsorship deal they signed with Japanese company Yokohama Rubber in February. \newline
    \textbf{FC Tokyo forward Yoshinori Muto (centre) brings the ball forward against Albirex Niigata in March.}\newline
    \textbf{FC Tokyo president Naoki Ogane claims that Chelsea have made a bid for Japan international Muto.}\newline
    \textbf{Muto tussles with Yuji Nakazawa of Yokohama F.Marinos during a J-League clash last month.}\newline\newline
    \textbf{YOSHINORI MUTO FACTFILE} \newline
    \textbf{Age}: 22 \newline
    \textbf{Club}: FC Tokyo \newline
    \textbf{Appearances}: 37 \newline
    \textbf{Goals}: 16 \newline
    \textbf{International caps (Japan)}: 11 \newline
    \textbf{International goals}: 1 \newline
    \textbf{Did you know?} Muto graduated from Keio University in Tokyo with an economics degree two weeks ago. \newline\newline
    Speaking to Sports Nippon, Ogane said: 'It is true that Chelsea sent us an offer for Muto. 'It is a formal offer with conditions. They want to acquire him in the summer.' \newline
    Muto, who only graduated from Keio University a fortnight ago after completing an economics degree, would be the first Japanese player to represent Chelsea if he moves to west London. He has earned 11 caps for his country after signing his first professional contract in 2014, scoring once for the Samurai Blue. \newline
    A £4million deal for the youngster has been mooted, but Muto admits that he isn't sure if he will join the Premier League title chasers despite being pleased with their bid. \newline
    He said: 'I have not decided yet at all. It is an honour for me to receive this offer from a great club.' \newline
    Muto scored 13 times in his debut season with FC Tokyo and was named in the J-League's best XI. \newline
    \textbf{Muto admits it is an 'honour' to receive an offer from Chelsea although he has not yet decided to join the club.} \newline
    \textbf{Muto, pictured in action against Ventforet Kofu has scored three goals in four games so far this season.} \newline
    \textbf{The 22-year-old has a shot at goal during Japan's Asian Cup match against Palestine in January this year.}\newline 
    He has continued his fine form during the current campaign, helping his club to third place in the division with three goals in four games. \newline
    Yokohama Rubber - one of the world's largest tyre manufacturers - will become Chelsea's official shirt sponsors from the start of the 2015-16 season. \newline
    The initial five-year deal is the biggest in the club's history, with the Blues now considering a two-week pre-season tour of Japan this summer. \\
    \midrule
    Summary & Chelsea have made an offer for fc tokyo's 22-year-old forward yoshinori muto. 
    The japan international will join chelsea's dutch partner club vitesse arnhem on loan next season if he completes a move to stamford bridge this summer. 
    Muto claims chelsea's interest in muto is not connected to the \# 200million sponsorship deal they signed with japanese company yokohama rubber in february.  \\
    \midrule
    Evaluation Aspect  & Fluency \\
    \midrule
    Human Score (1-5) & 5 \\
    \midrule
    \midrule
    MT-Bench (1-5)  & The summary contains several issues related to fluency, \textcolor{red}{primarily due to grammatical errors}, inconsistencies in capitalization, and the inclusion of a symbol (\#) that does not belong. \textcolor{red}{The sentences are generally clear but lack the smoothness and refinement that one would expect in a quality summary.} The use of lower-case letters for "fc tokyo" and "stamford bridge" is incorrect, and the number symbol (\#) before "200million" is out of place. There are also minor issues with sentence structure and flow, such as the abrupt transition from the offer to the loan arrangement without proper context.\newline\newline
    Rating: [[2]] \\
    \midrule
    CloserLook + ICL (1-5) & Analysis: The summary contains several grammatical errors and lacks coherence. Sentence fragments and improper punctuation are present, making the summary difficult to read. \textcolor{red}{For example, "The japan international will join chelsea's vitesse arnhem on loan next season if he completes a move to stamford bridge this summer" is not a complete sentence and lacks clarity.} Additionally, the summary includes a hashtag (\# 200million) instead of writing out the number, which is inconsistent with standard writing practices. The summary also omits some important details from the original article, such as Muto's achievements and his indecision about the move.\newline\newline
    Rating: 1  \\
    \midrule 
    HPSS (1-10)  & Rating: [[8]]\newline\newline
    Explanation:\newline
    The summary is generally well-written and grammatically correct. The sentences flow well and are coherent, providing a clear and concise overview of the key points from the article. However, there are a few areas where the fluency could be improved:\newline\newline
    1. The summary starts with a run-on sentence that combines two separate ideas from the article: "Chelsea have made an offer for fc tokyo's 22-year-old forward yoshinori muto. The japan international will join chelsea's vitesse arnhem on loan next season if he completes a move to stamford bridge this summer." This could be split into two sentences for better readability: "Chelsea have made an offer for FC Tokyo's 22-year-old forward Yoshinori Muto. If he completes a move to Stamford Bridge this summer, the Japan international will join Chelsea's Vitesse Arnhem on loan next season."\newline\newline
    2. The use of "\# 200million" instead of "200 million" is inconsistent with the currency used in the original article and can be confusing for readers.\newline\newline
    3. While the summary is concise, it could include more details to better reflect the depth of the original article. For instance, mentioning Muto's form this season or his academic background could enhance the summary without significantly increasing its length.\newline\newline
    Overall, the summary is clear and accurately reflects the main points of the article, but a few minor adjustments could improve its fluency and completeness. \\
    \bottomrule
    \end{tabular}
    \vspace{-2mm}
    \caption{A judgment generation case of Qwen2.5-14B-Instruct evaluator using the prompting strategy from MT-Bench, CloserLook + ICL, and HPSS for the aspect \textit{Fluency} in Summeval. Factual errors in the judgments are highlighted in \textcolor{red}{red}.}
    \label{tab:judgement_case_1}
\end{table*}