
\definecolor{customcolor1}{RGB}{255, 231, 218}
\definecolor{customcolor2}{RGB}{218, 233, 248}
\definecolor{customcolor3}{RGB}{255, 245, 213}

\begin{table*} [t]
\scriptsize
\centering
\setlength{\tabcolsep}{1.6mm}{
\begin{tabular}{l l >{\raggedright\arraybackslash}m{0.7\textwidth}}
% {m{0.12\linewidth}p{0.1\linewidth}p{0.78\linewidth}}
\toprule
\textbf{Template} & \textbf{Value} & \textbf{Prompt} \\
\midrule
\multirow{5}{*}[-7em]{Backbone} &  \multirow{5}{*}[-7em]{-} & 
\begin{tabular}[t]{@{}>{\cellcolor{customcolor1}}p{0.7\textwidth}}
        \#\# Instruction \newline
        Please act as an impartial judge and evaluate the quality of the summary of the news article displayed below on its \textcolor{blue}{[Aspect]}. \textcolor{red}{\{reference\_1\_template\}} \textcolor{red}{\{reference\_dialectic\_template\}} \textcolor{red}{\{chain\_of\_thought\_template\}} \\
    \end{tabular} \\
& & \  \\
& & 
\begin{tabular}[t]{@{}>{\cellcolor{customcolor2}}p{0.7\textwidth}}
Here are some rules of the evaluation: \newline
1. Your evaluation should consider the \textcolor{blue}{[Aspect]} of the summary. \textcolor{blue}{[Criteria]} \newline
2. Be as objective as possible. \newline\newline
\textcolor{red}{\{autocot\_template\}} \\
\end{tabular} \\
& & \ \\
& &
\begin{tabular}[t]{@{}>{\cellcolor{customcolor3}}p{0.7\textwidth}}
\textcolor{red}{\{in\_context\_example\_template\}} \newline
\#\# Article \newline
\textcolor{blue}{[Article]} \newline\newline
\textcolor{red}{\{metrics\_template\}} \newline\newline
\textcolor{red}{\{reference\_2\_template\}} \newline\newline
\#\# The Start of the Summary \newline
\textcolor{blue}{[Summary]} \newline
\#\# The End of the Summary 
 \\
 \end{tabular} \\
\midrule
\multirow{1}{*}{Reference 1} & - & You will be given the news article, the summary, and a high-quality reference summary. \\
\midrule
\multirow{1}{*}{Reference 2} & - & \#\# The Start of Reference Summary \newline
\textcolor{blue}{[Reference]} \newline
\#\# The End of Reference Summary \\

\midrule
\multirow{1}{*}{Reference Dialectic} & - & Please generate your own summary for the news article first and take into account your own summary to evaluate the quality of the given summary. \\
\midrule
\multirow{3}{*}[-1.85em]{Chain-of-Thought} & No CoT & You must directly output your rating of the summary on a scale of 1 to \textcolor{red}{\{max\}} without any explanation by strictly following this format: "[[rating]]", for example: "Rating: [[\textcolor{red}{\{max\}}]]". \\
\cmidrule{2-3}
 & Prefix CoT & Begin your evaluation by providing a short explanation. After providing your explanation, you must rate the summary on a scale of 1 to \textcolor{red}{\{max\}} by strictly following this format: "[[rating]]", for example: "Rating: [[\textcolor{red}{\{max\}}]]". \\
 \cmidrule{2-3}
 & Suffix CoT & You must rate the summary on a scale of 1 to \textcolor{red}{\{max\}} first by strictly following this format: "[[rating]]", for example: "Rating: [[\textcolor{red}{\{max\}}]]". And then provide your explanation. \\
\midrule
\multirow{5}{*}[-1.5em]{Scoring Scale} & 3 &  \textcolor{red}{\{max\}} $ = 3 $ \\
\cmidrule{2-3}
 & 5 & \textcolor{red}{\{max\}} $ = 5 $ \\
\cmidrule{2-3}
 & 10 & \textcolor{red}{\{max\}} $ = 10 $ \\
\cmidrule{2-3}
 & 50 & \textcolor{red}{\{max\}} $ = 50 $ \\
\cmidrule{2-3}
 & 100 & \textcolor{red}{\{max\}} $ = 100 $ \\
\midrule
\multirow{1}{*}{AutoCoT} & - & Evaluation Steps: \newline
\textcolor{blue}{[Autocot]} \\
\midrule
\multirow{1}{*}{In-Context Example} & - & Here are some examples and their corresponding ratings: \newline
\textcolor{red}{\{example\_template\_1\}} \newline
\textcolor{red}{\{example\_template\_2\}} \newline
\textcolor{red}{…} \newline
\textcolor{red}{\{example\_template\_n\}}\newline\newline
Following these examples, evaluate the quality of the summary of the news article displayed below on its \textcolor{blue}{[Aspect]}: \\
\midrule
Example & - & \#\# Example \textcolor{blue}{[Number]}:\newline \#\# Article\newline\textcolor{blue}{[Article]}\newline\newline \#\# The Start of the Summary \newline\textcolor{blue}{[Summary]}\newline \#\# The End of the Summary\newline\newline \#\# Rating \newline \textcolor{blue}{[Human Rating]}\\
\midrule
\multirow{1}{*}{Metrics} & - & \#\# Questions about Summary \newline
Here are some questions about the summary. You can do the evaluation based on thinking about all the questions. \newline
\textcolor{blue}{[Metrics]} \\
\bottomrule
\end{tabular}
}
\vspace{-2mm}
\caption{Detailed evaluation prompt templates for Summeval. 
The backbone serves as the final input prompt template for LLM evaluators. 
The three components marked in different colors represent Task Description (\textbf{TD}), Evaluation Rule (\textbf{ER}), and Input Content (\textbf{IC}), respectively. 
The content within \textcolor{red}{\{\}} represents the prompt template for each factor, corresponding to the following rows in this table. 
Different content may be chosen for each template when corresponding factor values vary. 
The content within \textcolor{blue}{[]} is sample-specific input information. 
- in \textbf{Value} means that when the factor is chosen as "None", this template will be replaced with an empty string ("").
Otherwise, the content of this template will be added to the backbone. 
Specifically, the templates Reference 1 and Reference 2 will be replaced with an empty string ("") unless the factor Reference is chosen as \textbf{Self-Generated Reference}.
The template Reference Dialectic will be replaced with an empty string ("") unless the factor Reference is chosen as \textbf{Dialectic}.}
\label{tab:evaluation_prompt_summeval}
\end{table*}