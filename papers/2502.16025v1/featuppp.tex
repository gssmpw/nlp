%%%%%%%% ICML 2023 EXAMPLE LATEX SUBMISSION FILE %%%%%%%%%%%%%%%%%

\documentclass{article}

% Recommended, but optional, packages for figures and better typesetting:
\usepackage{microtype}
\usepackage{graphicx}
\usepackage{subfigure}
\usepackage{booktabs} % for professional tables

% hyperref makes hyperlinks in the resulting PDF.
% If your build breaks (sometimes temporarily if a hyperlink spans a page)
% please comment out the following usepackage line and replace
% \usepackage{icml2023} with \usepackage[nohyperref]{icml2023} above.
\usepackage{hyperref}


% Attempt to make hyperref and algorithmic work together better:
\newcommand{\theHalgorithm}{\arabic{algorithm}}

% Use the following line for the initial blind version submitted for review:
% \usepackage{icml2025}

% If accepted, instead use the following line for the camera-ready submission:
\usepackage[accepted]{icml2025}

% For theorems and such
\usepackage{amsmath}
\usepackage{amssymb}
\usepackage{mathtools}
\usepackage{amsthm}
\usepackage{multirow}
\usepackage{bm}

% if you use cleveref..
\usepackage[capitalize,noabbrev]{cleveref}

%%%%%%%%%%%%%%%%%%%%%%%%%%%%%%%%
% THEOREMS
%%%%%%%%%%%%%%%%%%%%%%%%%%%%%%%%
\theoremstyle{plain}
\newtheorem{theorem}{Theorem}[section]
\newtheorem{proposition}[theorem]{Proposition}
\newtheorem{lemma}[theorem]{Lemma}
\newtheorem{corollary}[theorem]{Corollary}
\theoremstyle{definition}
\newtheorem{definition}[theorem]{Definition}
\newtheorem{assumption}[theorem]{Assumption}
\theoremstyle{remark}
\newtheorem{remark}[theorem]{Remark}
\DeclareMathOperator{\jbu}{\textup{JBU}}
\DeclareMathOperator*{\softmax}{\textup{softmax}}
\DeclareMathOperator*{\bilinear}{\textup{BL}}
\DeclareMathOperator*{\tile}{\textup{Tile}}

\def\authornote#1#2#3{{\textcolor{#2}{\textsl{\small[#1: #3]}}}}
\newcommand{\ANY}[1]{\authornote{\textbf{Authors}}{red}{#1}}
\newcommand{\MR}[1]{\authornote{\textbf{Mike}}{blue}{#1}}
\newcommand{\PM}[1]{\authornote{\textbf{Pavlo}}{orange}{#1}}
\newcommand{\ul}[1]{\underline{#1}}

% Todonotes is useful during development; simply uncomment the next line
%    and comment out the line below the next line to turn off comments
%\usepackage[disable,textsize=tiny]{todonotes}
\usepackage[textsize=tiny]{todonotes}


% The \icmltitle you define below is probably too long as a header.
% Therefore, a short form for the running title is supplied here:
\icmltitlerunning{FeatSharp: Your Vision Model Features, Sharper}

\begin{document}

\twocolumn[
\icmltitle{FeatSharp: Your Vision Model Features, Sharper}

% It is OKAY to include author information, even for blind
% submissions: the style file will automatically remove it for you
% unless you've provided the [accepted] option to the icml2023
% package.

% List of affiliations: The first argument should be a (short)
% identifier you will use later to specify author affiliations
% Academic affiliations should list Department, University, City, Region, Country
% Industry affiliations should list Company, City, Region, Country

% You can specify symbols, otherwise they are numbered in order.
% Ideally, you should not use this facility. Affiliations will be numbered
% in order of appearance and this is the preferred way.
\icmlsetsymbol{equal}{*}

\begin{icmlauthorlist}
\icmlauthor{Mike Ranzinger}{comp}
\icmlauthor{Greg Heinrich}{comp}
\icmlauthor{Pavlo Molchanov}{comp}
\icmlauthor{Jan Kautz}{comp}
\icmlauthor{Bryan Catanzaro}{comp}
\icmlauthor{Andrew Tao}{comp}
% \icmlauthor{Firstname2 Lastname2}{equal,yyy,comp}
% \icmlauthor{Firstname3 Lastname3}{comp}
% \icmlauthor{Firstname4 Lastname4}{sch}
% \icmlauthor{Firstname5 Lastname5}{yyy}
% \icmlauthor{Firstname6 Lastname6}{sch,yyy,comp}
% \icmlauthor{Firstname7 Lastname7}{comp}
% %\icmlauthor{}{sch}
% \icmlauthor{Firstname8 Lastname8}{sch}
% \icmlauthor{Firstname8 Lastname8}{yyy,comp}
%\icmlauthor{}{sch}
%\icmlauthor{}{sch}
\end{icmlauthorlist}

\icmlaffiliation{comp}{NVIDIA}

\icmlcorrespondingauthor{Mike Ranzinger}{mranzinger@nvidia.com}


% You may provide any keywords that you
% find helpful for describing your paper; these are used to populate
% the "keywords" metadata in the PDF but will not be shown in the document
\icmlkeywords{Machine Learning, ICML}

\vskip 0.3in
]

% this must go after the closing bracket ] following \twocolumn[ ...

% This command actually creates the footnote in the first column
% listing the affiliations and the copyright notice.
% The command takes one argument, which is text to display at the start of the footnote.
% The \icmlEqualContribution command is standard text for equal contribution.
% Remove it (just {}) if you do not need this facility.

\printAffiliationsAndNotice{}  % leave blank if no need to mention equal contribution
% \printAffiliationsAndNotice{\icmlEqualContribution} % otherwise use the standard text.

\begin{abstract}
The feature maps of vision encoders are fundamental to myriad modern AI tasks, ranging from core perception algorithms (e.g. semantic segmentation, object detection, depth perception, etc.) to modern multimodal understanding in vision-language models (VLMs). Currently, in computer vision, the frontier of general purpose vision backbones are Vision Transformers (ViT), typically trained using contrastive loss (e.g. CLIP). A key problem with most off-the-shelf ViTs, particularly CLIP, is that these models are inflexibly low resolution. Most run at 224x224px, while the ``high resolution'' versions are around 378-448px, but still inflexible. We introduce a novel method to coherently and cheaply upsample the feature maps of low-res vision encoders while picking up on fine-grained details that would otherwise be lost due to resolution. We demonstrate the effectiveness of this approach on core perception tasks as well as within agglomerative model (RADIO) training as a way of providing richer targets for distillation.
\end{abstract}

\section{Introduction}\label{sec:intro}

\begin{figure*}[!ht]
    \centering
    \includegraphics[width=\linewidth]{resources/featsharp_arch.pdf}
    \vspace{-7mm}
    \caption{Upsampling architecture diagram. We combine the upsampled features coming from FeatUp \cite{fu2024featup} with the tiled features and mix them with FeatSharp to produce a feature map with higher fidelity. The tiled features have more detail, but also representation issues such as the difference in upper and lower body at the tile boundary. ``Full-Res Reference'' is for display purposes, as for a model that doesn't exhibit stable resolution scaling (e.g. DFN CLIP, SigLIP, etc.) we don't have access to a target hi-res feature map. Our proposed learnable modules are in green, combined with a learnable FeatUp JBU stack.}
    \label{fig:featsharp_arch_diagram}
\end{figure*}

The use of vision foundation models (VFM) \cite{awais2023foundational} has seen widespread use since the beginning of the modern era of computer vision using deep learning \cite{krizhevsky2012alexnet}, primarily used to perform transfer learning \cite{plested2022deeptransferlearningimage} (e.g. finetuning a VFM on a downstream task), information retrieval \cite{babenko2014neuralcodes,zhang2024retrieval}, and most recently, to power visual capabilities for vision-language models (VLM) \cite{alayrac2022flamingovisuallanguagemodel,openai2024gpt4technicalreport,liu2023llava,lin2023vila}. The recent shift toward using Transformers~\cite{vaswani2017attention} for computer vision (ViT~\cite{dosovitskiy2021image}) has both tremendously moved the field forward, but has generally left the use of VFMs in a tricky spot: Transformers are computationally demanding and have poor algorithmic scaling properties ($O(n^2)$ for 1D sequences, or $O(\left(w \cdot h \right)^2$ for 2D inputs), leaving the majority of models to be relatively low-resolution. For example, perhaps the most popular family of VFMs to date, CLIP~\cite{radford2021clip}, typically runs at 224 or 336px input resolutions, and produces spatial features at a 14x downsample (e.g. $224^2 \rightarrow 16^2$). Owing to the nature of learned position embeddings, ViTs also tend to be relatively inflexible to changes of input resolution, allowing for changes, but requiring finetuning \cite{dosovitskiy2021image}. 

It's possible that the strict dependence on the training resolution is an artifact of the algorithm used for training, as DINOv2~\cite{oquab2023dinov2,darcet2023vision} is quite robust to interpolating its position embeddings, producing stable features at various resolutions \cite{ranzinger2023amradio}, ignoring for the moment that DINOv2, being a transformer, is expensive to use at high-resolution. A recent technique called AM-RADIO~\cite{ranzinger2024phisdistributionbalancinglabelfree}, borrowing ideas from ViTDet~\cite{li2022vitdet}, FlexiViT~\cite{beyer2023flexivit}, and RO-ViT~\cite{kim2023regionaware}, has attempted to create a resolution-flexible ViT, however it is still dependent on low-resolution ViTs as it distills from other seminal VFMs which are low-res-only: DFN CLIP~\cite{fang2023data} and SigLIP~\cite{zhai2023sigmoid}. 

Recently, FeatUp~\cite{fu2024featup} aims to directly address the problem of low-resolution vision features by using one of two learned upsampling algorithms: A model-specific generalized upsampler using Joint Bilateral Upsampling (JBU) \cite{kopf2007jbu}, or a model-specific-image-specific implicit network. While they demonstrate particularly compelling results with their implicit network, their results using the stack of JBU filters lack refined details (shown in figure \ref{fig:featup_jbu_vs_implicit} in the appendix). Along with lack of granular refinement, it's impossible for this approach to capture fine-grained details that are too small for the vision backbone to detect at its native resolution. To this end, we take inspiration from both FeatUp's JBU approach, as well as the recent trend in VLMs such as LLaVA 1.6~\cite{liu2024llavanext1p6}, InternVL-1.5~\cite{chen2024internvl1p5}, NVLM~\cite{dai2024nvlmopenfrontierclassmultimodal} and Eagle~\cite{shi2024eagleexploringdesignspace} to tile an image, aggregating local features from a fixed-low-resolution model, to build an upsampler that simultaneously leverages the raw pixel guidance, low-res feature guidance, and regional tile guidance, resulting in substantially more detailed feature maps which are also capable of capturing details too small for the original resolution. Specifically, we:

\begin{itemize}
    \item Build on top of FeatUp's JBU algorithm \cite{fu2024featup} by adding de-biasing and tiling fusion modules to incorporate detailed tile features, resulting in significantly higher levels of detail, with extensive experiments demonstrating effectiveness
    \item Study the relationship between FeatUp's feature consistency and ViT-Denoiser's~\cite{yang2024denoising} approach to cleaning the features of a ViT at its native resolution
    \item Introduce an improved training setting for AM-RADIO \cite{ranzinger2024phisdistributionbalancinglabelfree} demonstrating a +$0.6\%$ improvement across the entire benchmark suite, and better teacher adapter features
\end{itemize}

\section{Related Work}\label{sec:related}

\begin{figure}[t]
    \centering
    \includegraphics[width=0.6\linewidth]{resources/featsharp_module.pdf}
    \caption{Diagram of the FeatSharp module. We first concatenate the bilinear upsampled and tiled mosaic feature maps along the channel dimension. We then apply a transformer block with sliding window attention followed by MLP (in this case, SwiGLU), and then slice off the first half of the channels, corresponding to the bilinear upsampled buffer. The role of FeatSharp thus is to refine the bilinear buffer by leveraging the tile buffer.}
    \label{fig:featsharp_module_diagram}
\end{figure}

\paragraph{Feature Upsampling}
The most obvious baseline for feature upsampling is to use traditional filtering approaches such as bilinear or bicubic upsampling. The alternative is to evaluate the network at higher resolution, however it comes with the dual drawback that computational cost increases (quadratically in the case of Vision Transformers), and also that many models (ViTs in particular) have trouble extrapolating from their trained resolution \citep{beyer2023flexivit,dehghani2023navit}. If we expand our view to include parametric approaches, then deconvolution \cite{noh2015deconv,shi2016deconv,dumoulin2016AGT} and resize-conv \cite{odena2016deconvcheck} are popular choices. There are also pixel-adaptive approaches such as CARAFE~\cite{Wang2019CARAFECR}, SAPA~\cite{lu2022sapa}, and FeatUp~\cite{fu2024featup}. 

We adopt FeatUp's formulation of multi-view consistency as a way to train an upsampler, however, we notice that instead of solely relying on raw RGB pixels as guidance for upsampling, we can also use a small, fixed budget of inferences (similar in spirit to their implicit model), and use a mosaic of tiles as guidance at the higher resolution. This choice gives us a richer, and semantically relevant, feature space to merge from. Additionally, it allows us to incorporate information from regions that were too small for the low-res view, but become visible within a tile. Small details are a limitation of every approach that doesn't acquire extra samples from the base model, as they rely on all relevant information already being encoded by the initial model evaluation.

\paragraph{Feature Denoising}
Related to multi-view consistency, ViT-Denoiser~\cite{yang2024denoising} noticed that ViT features are generally very noisy (although some are much cleaner than others), and also propose a multi-view consistency formulation to learn how to separate fixed noise, conditional noise, and semantic content. We notice the deep ties between ViT-Denoiser and FeatUp, in that multi-view consistency provides a way to eradicate fixed-pattern noise from the feature buffer. Drawing inspiration from this, we add a learnable bias buffer (similar to learned position embeddings) at the output of the base model. This simple change works because fixed patterns will degrade multi-view consistency, as the pattern is always local to the view, and lacks global coherence.

\paragraph{VLMs}
The use of tiling to increase information is currently very prominent in VLMs \cite{liu2024llavanext1p6,chen2024internvl1p5,dai2024nvlm}, albeit an alternative approach is to instead leverage the models at hi-res themselves \cite{beyer2024paligemmaversatile3bvlm,wang2024qwen2vlenhancingvisionlanguagemodels}. We also see RADIO-Amp\citep{heinrich2024radioamplifiedimprovedbaselines} being primarily useful at high-resolution within VLMs. In the increasingly VLM-centric approach to computer vision, we turn our focus to RADIO-Amp, as it has a training procedure that relies on matching a high-resolution student against a low-resolution teacher, an application area that is perfect for studying feature upsampling, as it would provide richer guidance to the distillation.

\paragraph{Agglomerative Models}
In the agglomerative model space, there are currently three major approaches: RADIO \citep{ranzinger2023amradio,ranzinger2024phisdistributionbalancinglabelfree,heinrich2024radioamplifiedimprovedbaselines}, Theia \citep{shang2024theia}, and UNIC \citep{sariyildiz2024unic}. We focus our attention on RADIO because it is the only approach that directly tries to tackle resolution flexibility as well as high-resolution.

\section{Method}\label{sec:method}

We leverage FeatUp's training algorithm of treating the upsampling problem as that of multi-view consistency between the upsampled and then downsampled features and different low-res views of the same image.

\begin{figure*}[t]
    \centering
    \includegraphics[width=\linewidth]{resources/tile_process.pdf}
    \vspace{-8mm}
    \caption{Visualization of the tiling process. An input image (left) is split into $2 \times 2$ tiles, each of which is resized to match the input resolution of the encoder, fed through the encoder independently, and then stitched back into a higher resolution feature map. Feature maps shown are from DFN CLIP, and they are resized to be larger than actual for demonstration purposes.}
    \label{fig:tile_process}
\end{figure*}

\subsection{Review - FeatUp: Joint Bilateral Upsampling (JBU)}\label{sec:method:featup}

Given a high-resolution signal $G$ (e.g. the raw pixels) as guidance, and a low-resolution signal $F_{lr}$ that we'd like to upsample, and let $\Omega$ be a neighborhood of each pixel in the guidance. Let $k(\cdot, \cdot)$ be a similarity kernel that measures how close two vectors are. Then

\begin{equation}
\begin{split}
\hat{F}_{hr}[i, j] = \frac{1}{Z} \sum_{(a, b) \in \Omega} \Bigl(&F_{lr}[a, b] \cdot \\
    & k_{range}\left(G[i, j], G[a, b]\right)\cdot \\
    & k_{spatial}\left([i, j], [a, b] \right)  \Bigr)
    \label{eq:jbu_orig}
\end{split}
\end{equation}

with $Z$ being a normalization to make the kernel sum to 1. $k_{spatial}$ is a Gaussian kernel with learnable $\sigma_{spatial}$ defined as

\begin{equation}
    k_{spatial}(x,y) = \exp\left(\frac{-\left\lVert x - y \right\rVert_2^2}{2\sigma_{spatial}^2}\right)
    \label{eq:k_spatial}
\end{equation}

and $k_{range}$ as

\begin{equation}
    k_{range}(x,y) = \softmax_{(a, b) \in \Omega} \left(\frac{1}{\sigma_{range}^2} h(G[x,y]) \cdot h(G[a,b]) \right)
    \label{eq:k_range}
\end{equation}

with $h(x)$ being a learned MLP projector. They define

\begin{equation}
    F_{hr} = \left(\jbu(\cdot, x) \circ \jbu(\cdot, x) \circ ... \right)(f(x), x)
\end{equation}

as a stack of $2\times$ upsamplers, thus enabling power-of-2 upsample factors. With $x$ being the original input image, and $f(x)$ being the low-resolution feature map. We note that $2$ isn't a necessary part of the architecture, and that their implementation supported arbitrary factors, so we simply propose to take a given upsample factor $z \in \mathbb{Z}_{+}$ and prime factorize $z$ to get a set of upsample factors, using a $\jbu_k$ for each prime factor. This decomposes to an identical operation as before when $\log_2 z \in \mathbb{Z}_{+}$, but allows for an easy guide for any other integer, e.g. for a $14\times$ upsample corresponding to a patch-size-14 backbone, we'd use a $\left(\jbu_{7\times} \circ \jbu_{2\times}\right)(f(x), x)$ stack.

As is typical with bilateral upsampling, this method is very sensitive to strong edges in the guidance buffer, however, it also tends to over-smooth features in regions of lower contrast. Particularly, it struggles with feature patterns such as SAM (figure \ref{fig:viz_basketball}) where there are interior edges in feature space, but not pixel space. This results in the features being blurred inside of objects.

We don't make any changes to their downsampler, instead opting to just use their Attention Downsampler without modification. We then focus on two changes, one to output normalization, and the other to how upsampling guidance is computed.

\subsection{Feature Normalization}\label{sec:method:feat_norm}
FeatUp supports either leaving the features coming from the backbone as-is (e.g. no normalization), or using a LayerNorm to better condition the outputs for feature learning. For a similar motivation as PHI-S~\cite{ranzinger2024phisdistributionbalancinglabelfree}, we want to avoid using the raw features as they have varying spreads, and we'd also like to avoid using LayerNorm as it makes the features incompatible with the original feature space. Naïvely learning the raw feature space across the suite of teachers without normalization often led to convergence issues, particularly given the wide variance of activations.

\paragraph{Tile-Guided Attentional Refinement}

\begin{figure}
    \centering
    \includegraphics[width=0.3\linewidth]{resources/2x/2x_bilinear.png}
    \includegraphics[width=0.3\linewidth]{resources/2x/2x_tile.png}
    \caption{Visualization of $2\times$ upsampling using bilinear (\textit{left}) versus tiling (\textit{right}), using the DFN CLIP encoder.}
    \vspace{-5mm}
    \label{fig:ups2x}
\end{figure}

Joint-Bilateral Upsampling is able to retain object boundaries primarily in instances when there are noticeable changes in intensity in the RGB input image. This results in sharp contours, but within a region, we end up with vague and blurry feature representations. Owing to the reliance on raw pixel intensities, object contours that are less discriminative in color space often get blurred with the neighborhood. Finally, because the upsampling operation is only truly operating on the low resolution feature maps of the model, it's impossible for JBU to introduce new details into the feature map that are visible/encodable at higher input resolutions. FeatUp's implicit upsampler doesn't have this problem because it's constructing a unified view from numerous local views of the original image, enabling detailed refinements. We propose an intermediary method between JBU which leverages a single featurizer inference, and the implicit model, which relies on numerous inferences and is thus cost prohibitive\footnote{https://github.com/mhamilton723/FeatUp/issues/2}.

Inspired by the use of tiling in Vision-Language Models (VLMs) \citep{liu2024llavanext1p6,shi2024s2,dai2024nvlmopenfrontierclassmultimodal}, we develop an attentional refinement block that is able to integrate the information between a bilinear upsampled feature map, as well as a feature map composed of tiles. We show an overview of the algorithm in figures 
 \ref{fig:featsharp_arch_diagram}, \ref{fig:featsharp_module_diagram} and \ref{fig:tile_process}. The diagram shows actual results using RADIOv2.5-L, which is the most scale equivariant foundation model \citep{heinrich2024radioamplifiedimprovedbaselines}, and generally the strongest visual foundation model \citep{lu2024swissarmyknifesynergizing,drozdova2024semisupervised,guo2024videosamopenworldvideosegmentation}. Because the model has strong resolution scaling, it provides us with a good way to compare the results of the upsampling process against the feature maps of the same resolution attained by increasing the resolution of the input image. We also observe that even just at $4\times$ tiling, there are major discontinuities in the tiled feature map, which the FeatSharp module must overcome to produce a unified higher resolution image.


For the FeatSharp module, we leverage a single Attention+SwiGLU transformer block. In order to prevent the quadratic cost of global attention, we instead use local attention \cite{ramachandran2019localattn}. We concatenate the JBU upsampled buffer with the tiled feature map and feed it to the block. After the block is computed, we slice the first $C$ dimensions of the output, with $C$ being the model feature dimension, and treat that as the refined features. The slicing strategy takes advantage of the fact that a transformer block has a residual pathway, and thus a no-op from the transformer would be equivalent to returning the bilinear upsampled features. Through the attention mechanism, the model is able to consider the local neighborhood and refine its features to achieve better multi-view consistency. To this end, we train our model identically to FeatUp's multi-view consistency algorithm. We do not employ any special loss functions beyond the MSE loss on multi-view consistency, contrary to FeatUp's use of Total Variation and Conditional Random Field losses. We provide ablations wrt architecture choice in appendix \ref{sec:featsharp_arch_ablations}.

\subsection{Denoising}

Drawing inspiration from \cite{yang2024denoising}, we notice that the problem formulation has a very similar solution to FeatUp (and ours), owing to the fact that all methods are using multi-view consistency and thus learn to eliminate position-based artifacts. From their formulation:

\begin{equation}
    \textup{ViT}(x) = f(x) + g(\textup{\textbf{E}}_{pos}) + h(x, \textup{\textbf{E}}_{pos})
    \tag{\cite{yang2024denoising}, Eq 5}
    \label{eq:vit_denoise}
\end{equation}

We add a learnable $g$ buffer, such that

\begin{equation}
    \hat{f}(x) = f(x) + g
\end{equation}

with $f(x)$ being the frozen vision encoder. The learnable $g$ allows our model to learn and negate the fixed position artifacts that the encoder produces. Notably, given that we are also using the base model for the tiles, this learned buffer is applied to all of the generated tiles as well. We visualize these biases in figure \ref{fig:model-biases}. It's entirely possible for FeatSharp to remove the biases itself, but we found that having this learnable bias buffer consistently improves multi-view consistency, which we show in table \ref{tab:bias_fidelity} in the appendix.

\subsection{Complexity}
An important point about this method is that because of the tiling, it requires more evaluations of the base vision model to construct the high resolution feature map. However, due to the scaling properties of global self-attention, our proposed method always has better scaling properties than running the original model at higher resolution (assuming the model is capable of doing this in the first place):

\begin{equation}
\begin{split}
    f(x) &= c \left(1 + x^2 \right) \\
    g(x) &= c \left(x^2\right)^2 = c x^4  \\
    f(x) &\leq g(x) \quad \forall x > 1
\end{split}
\label{eq:cost_inequality}
\end{equation}

where $f(x)$ is the relative cost of computing FeatSharp with $x \in \mathbb{Z}_{+}$ upsample factor, $g(x)$ is the cost of the hi-res image based on quadratic scaling on the number of patches (and thus tiles), and $c$ being the cost of processing a single tile. We show the actual scaling cost in figure \ref{fig:vith_throughput} in the appendix.

\section{Upsampling Results}\label{sec:experiments}

We consider upsampling to be important in cases where one is given a fixed pretrained model, and the goal is to extract more information out of it, for a given image. We study our method in relation to FeatUp from a core multi-view consistency standpoint in this section, from a semantic segmentation linear probe standpoint, and also for training a new RADIO-like model with hi-res teacher targets.

\subsection{Fidelity}\label{sec:fidelity}

\paragraph{Multi-View Consistency} 
Following \cite{ranzinger2024phisdistributionbalancinglabelfree}, we use their definition of fidelity (equation 51) for multi-view consistency, where a higher fidelity value means that the upsampled-transformed-downsampled representations are closer to the raw transformed predictions from the model. 

\begin{equation}
    f(\mathbf{X},\mathbf{Y}) = \frac{\text{MSE}(\mathbf{Y}, \bm{\mu_Y})}{\text{MSE}(\mathbf{X}, \mathbf{Y})}
    % \tag{\cite{ranzinger2024phisdistributionbalancinglabelfree} Eq 51}
\end{equation}

with $\mathbf{X}$ being the warped predictions and $\mathbf{Y}$ the targets. This serves as a proxy measure for how well the upsampler is working, as arbitrarily warping and downsampling it results in representations closer to the real prediction at low resolution. We show these results in figure \ref{fig:consistency_fidelity}, where we observe that FeatSharp consistently achieves the highest fidelities, substantially so with the ``cleaner'' models such as DINOv2-L, RADIOv2.5-L, and SAM-H.

\begin{figure}[t]
    \centering
    \includegraphics[width=\linewidth]{resources/multiview_consistency_plot.pdf}
    \vspace{-7mm}
    \caption{Fidelity plot for different models and upsampling methods. Higher values are better. We don't show SAM 4x because of OOM issues training these models.}
    \vspace{-10mm}
    \label{fig:consistency_fidelity}
\end{figure}

\subsection{Qualitative}

We run this upsampling method on seven different foundation models coming from diverse domains such as supervised (ViT, \citep{dosovitskiy2021image}), contrastive (DFN~CLIP~\cite{fang2023data}, SigLIP~\cite{zhai2023sigmoid}), Self-supervised (DINOv2-L-reg~\cite{darcet2023vision}), Segmentation (SAM~\cite{kirillov2023sam}), VLM (PaliGemma~\cite{beyer2024paligemmaversatile3bvlm}), and Agglomerative (RADIOv2.5-L~\cite{ranzinger2024phisdistributionbalancinglabelfree}). Results are in figure \ref{fig:viz_basketball}. The original feature maps run the spectrum from extremely noisy (SigLIP) to very clean (RADIOv2.5-L, SAM), which allows us to demonstrate the effectiveness of the approach on a diverse set of models. Taking SAM for an example, the way in which is has thick edge outlines cannot be reproduced in the shape interior by FeatUp, primarily because the bilateral upsampler is operating on the raw pixels, where the interior edge doesn't exist in the real image. For all of the featurizers, FeatSharp is able to achieve more legible representations, particularly it's more able to closely match the real hi-res features in the second column.

\begin{figure}[!t]
    \centering
    \includegraphics[width=\linewidth]{resources/basketball_viz_results.pdf}
    \vspace{-5mm}
    \caption{PCA visualizations of features from a basketball scene. \textbf{Column~1}: Raw features produced by the model at normal resolution (e.g. 14x downsample for DFN CLIP, SigLIP, PaliGemma, and DINOv2, 16x downsample for SAM and RADIOv2.5-L. \textbf{Column~2}: Raw features at the 4x upsample resolution (we interpolate the position embeddings for those models that don't natively support resolution changes). \textbf{Column~3}: FeatUp 4x upsampling (\textit{prior work}). \textbf{Column~4}: FeatSharp 4x upsampling. 
    \\
    \textit{NOTE: ``Real 4x'' technically only makes sense for models with strong scale equivariance, such as DINOv2, RADIO, and SAM.}}
    \label{fig:viz_basketball}
\end{figure}

\subsection{Semantic Segmentation}\label{sec:experiments:semseg}

Semantic segmentation has the potential to benefit from increased resolution, as it allows for label contours to be more precise, and potentially for regions to be recovered that are otherwise too small. The first setting we evaluate on is we train both FeatUp and FeatSharp at $2\times$ and $4\times$ upsampling, both using PHI-S. We resize the input size to be the featurizer's native input resolution, which we call ``$1\times$ Input Size'', and we also consider ``$2\times$ Input Size'', where we double the input size, and feed directly to the featurizer in the case of ``Baseline'', or we allow the upsampler to have higher resolution guidance while keeping the featurizer input fixed at $1\times$ resolution. We show these results in figure \ref{fig:semseg}. In most cases, both upsampling algorithms produce higher quality segmentations than the baseline, however, FeatUp is worse than the ``Baseline $2\times$'' method for RADIOv2.5-L and ViT. In all cases, FeatSharp is superior to both FeatUp and also the baselines by significant margins. We even improve upon SOTA RADIO's published result of 51.47 with a $2\times$ upsampling combined with $2\times$ input size, producing a model that attains 53.13 mIoU, a $+1.66$ mIoU improvement. RADIO itself improves with the $2\times$ input size, but not to the same degree as with FeatSharp, with FeatSharp being $57\%$ faster. We also notice that $3\times$ upsampling is generally worse than $2\times$ or $4\times$ for both upsamplers.

\begin{figure*}
    \centering
    \includegraphics[width=\linewidth]{resources/semseg_plots.pdf}
    \vspace{-7mm}
    \caption{ADE20k \cite{zhou2017ade20k} Semantic segmentation results for different featurizers and upsamplers. We also vary the input size between Inpt-$1\times$ and Inpt-$2\times$ the featurizer's native resolution. $1\times$ Resolutions: DFN CLIP = 378px, DINOv2-L = 448px, PaliGemma = 448px, RADIOv2.5-L = 512px, SigLIP = 378px, ViT = 224px. The dark line represents the mean of 5 runs, with shaded areas showing the standard deviation. Because the x-axis is the upsample amount, the baselines should technically be single points on a ``1x'' x-coord, but we instead draw a line to make it easier to see the change in the upsamplers across the upsample amounts. E.g. for ``RADIO, Baseline Inpt-2x'', we can see that it's better than FeatUp $2\times$ upsampling, but worse than FeatSharp $2\times$ upsampling.}
    \label{fig:semseg}
\end{figure*}

\subsection{Agglomerative Models}\label{sec:experiments:agglom}

We build upon RADIOv2.5-L~\cite{heinrich2024radioamplifiedimprovedbaselines} as it learns directly from the spatial features of teacher models. In particular, we consider whether we can improve upon their multi-resolution training strategy by using FeatSharp to convert the low-res teachers into hi-res teachers. We convert the teachers in the bottom left quadrant ``Low Res Teacher / High Res Student'' in their Figure 6 into ``High Res Teacher / High Res Student'' by using the upsampler. We consider a few different comparative baselines in order to prove the efficacy of the technique. First, our baseline matches that of \cite{heinrich2024radioamplifiedimprovedbaselines}, which is to downsample the student to match the teacher. Then, we consider two techniques which are popular in the literature: Tiling~\cite{liu2024llavanext1p6}, and S2~\cite{shi2024s2}. Both of these rely on tiling, but S2 also considers the low-res version. Because we need the feature space to remain the same as the low-res partition of RADIO, we opt to upsample the low-res feature map, and then interpolate the upsampled-low-res against the tiled version, using $y = \beta \cdot \text{low-res} + (1 - \beta) \cdot \text{high-res}$. We set $\beta = 0.5$ as it's unclear what an optimal balance might be, and it's expensive to search this space. As a final baseline, we include FeatUp's JBU variant, as the implicit version would be prohibitive to use within a training loop\footnote{\hyperlink{https://github.com/mhamilton723/FeatUp/issues/2\#issuecomment-2005688054}{1 minute per image}}.

\begin{figure*}[t]
    \centering
    \includegraphics[width=\linewidth]{resources/radio_inference/dfn_clip_main_body.jpg}
    \vspace{-5mm}
    \caption{Visualization of our trained RADIO's DFN CLIP adaptor when the high-res partition used various teacher upsample schemes.}
    \label{fig:radio_dfn_clip_adaptor}
\end{figure*}

In figure \ref{fig:radio_dfn_clip_adaptor} we qualitatively visualize the DFN CLIP adaptor features learned by the radio model. We can see that each upsampling method has a substantial impact on the resulting feature maps. The baseline method exhibits strong high-frequency artifacting starting at 768px. This is likely when RADIO ``mode switches'' to high-resolution, which is something that \cite{heinrich2024radioamplifiedimprovedbaselines} addressed for the backbone features, but apparently still exhibit for the adaptor features. We observe that Tiling and S2 exhibit not only high-frequency noise patterns like the baseline, but also obvious grid patterns, arising from the use of tiles. FeatUp appears to mode switch starting at 768px into a smooth, but low-detail feature space. FeatSharp remains smooth and highly detailed as resolution increases, however, visually, it's still possible that the features are mode switching. We further study this tiling issue in appendix \ref{sec:apdx:overtiling}.

Along with improvements in the adaptors, we also study the effects on the backbone features for the RADIO model. Following \cite{Maninis2019AttentiveSO,lu2024swissarmyknifesynergizing} we report the MTL Gain ($\Delta_m$) across a suite of tasks. Unlike the prior works, instead of leveraging a single-task baseline, we instead opt to report the change relative to the baseline training run.

\begin{align}
    \delta_m &= 100 \cdot (-1)^{l_t} \frac{M_t - M_{B,t}}{M_{B,t}} \\
    \Delta_m &= \frac{1}{T} \sum_{t=1}^{T} \delta_m
\end{align}

where $M_t$ is the metric for the current model on task $t$, and $M_{B,t}$ is the metric for the baseline model. $l_t$ is 0 when higher task values are better, and 1 when lower is better.

\begin{table*}[!h]
    \centering
    \resizebox{0.8\linewidth}{!}{
    \begin{tabular}{c|cccccccc|c}
        Upsampler & Classification & Dense      & Probe 3D  & Retrieval & Pascal Context & NYUDv2      & VILA      & $\Delta_m \%$ \\
        \hline
        RADIO-AMP-L & -0.47          & -0.09      & -1.05     & -0.45     & \bf{0.62}      & -2.26       & \bf{2.24} & -0.21     \\
        \hline
        Baseline    & \ul{0.00}      & 0.00       & 0.00      & 0.00      & 0.00           & 0.00        & 0.00      & 0.00      \\
        Tile        & -0.03          & \bf{0.30}  & -0.08     & -0.23     & -0.02          & \bf{1.33}   & -3.17     & -0.27     \\
        S2          & -0.05          & 0.15       & -0.03     & -0.44     & 0.13           & \bf{1.33}   & -0.89     & 0.03      \\
        FeatUp      & -0.07          & 0.14       & 0.23      & -0.07     & 0.14           & 0.32        & -1.58     & -0.13     \\
        FeatSharp   & \bf{0.06}      & \ul{0.16}  & \bf{0.83} & \bf{0.13} & \ul{0.17}      & \ul{0.93}   & \ul{0.43} & \bf{0.39}
    \end{tabular}
    }
    \caption{Relative changes (in \%) on a suite of aggregated benchmarks, with each column reporting $\delta_m \%$ and averaged into $\Delta_m \%$. All relative changes are against our baseline run. Raw metrics are in section \ref{sec:apdx:raw_radio_results}. \textit{NOTE: The upsamplers are only applied to the DFN CLIP and SigLIP teachers during RADIO training. Metrics are collected from trained RADIO without upsampling methods.}
    }
    \label{tab:radio_mtl_task_suite}
\end{table*}

We show the MTL Gain results in table \ref{tab:radio_mtl_task_suite}. Given that the results are relative to our baseline run, S2 and FeatSharp are the only two methods to improve, however, only FeatSharp was categorically better, leading to a +0.39\% improvement across all benchmarks on average. We also see that our version of RADIO with FeatSharp teachers generally does better than RADIO-AMP-L \cite{heinrich2024radioamplifiedimprovedbaselines}, which is the current state of the art, where we improve over it on everything except for the VILA task. We report all of the raw benchmark scores in tables \ref{tab:apdx:radio_cls_retrieval_metrics}, \ref{tab:apdx:radio_dense_probe3d_metrics}, \ref{tab:apdx:radio_pascal_nyud_metrics} and \ref{tab:apdx:radio_vila_metrics} in the appendix.

\section{Conclusion}

We have presented a novel feature upsampling technique named FeatSharp that achieves higher multi-view fidelity than the current best method, FeatUp. We achieve this by joining FeatUp's JBU upsampler with a mosaic of tiles, and then process with a single local attention block. We demonstrate its effectiveness on ADE20K semantic segmentation linear probing, where the use of FeatSharp improves over both baseline and FeatUp, even with the strongest segmenter, RADIO, which itself can handle hi-res inputs robustly. We then demonstrate the effectiveness of FeatSharp by employing it directly within RADIO training, enabling low-res-only teacher models to have hi-res distillation targets. In doing this, our FeatSharp-RADIO largely improves on dense vision task benchmarks, and yields an overall improvement over our reproduction baseline, which itself improves over RADIO-AMP-L, the current state of the art. We believe this work can be useful both as a drop-in extension of existing vision systems which rely on pretrained vision encoders, as well as the newly trained FeatSharp-RADIO model with hi-res teachers, which can emulate these same models. Owing to FeatSharp-RADIO's emulation abilities, it allows us to estimate these teacher models at arbitrary resolutions, not just integer upsampling factors as restricted in FeatSharp/FeatUp's core training algorithm. Further, combining RADIO's ``ViTDet'' \citep{li2022vitdet} mode with these hi-res teacher emulations allows us to achieve hi-res feature maps without fully paying the quadratic penalty in number of tokens as required by standard ViTs.

% \section{Impact Statement}
% This paper presents work whose goal is to advance the field of Computer Vision. By virtue of being a lightweight addition to existing vision models, the work aims to open up doors for higher-resolution perception tasks (e.g. segmentation, depth perception, etc.) while retaining the original model representations. As such, the ethical impacts are constrained to those of the model being upsampled. The FeatSharp training code, upsampler weights, and trained RADIO model using FeatSharp will be released to the community.

\bibliography{featuppp}
\bibliographystyle{icml2025}


%%%%%%%%%%%%%%%%%%%%%%%%%%%%%%%%%%%%%%%%%%%%%%%%%%%%%%%%%%%%%%%%%%%%%%%%%%%%%%%
%%%%%%%%%%%%%%%%%%%%%%%%%%%%%%%%%%%%%%%%%%%%%%%%%%%%%%%%%%%%%%%%%%%%%%%%%%%%%%%
% APPENDIX
%%%%%%%%%%%%%%%%%%%%%%%%%%%%%%%%%%%%%%%%%%%%%%%%%%%%%%%%%%%%%%%%%%%%%%%%%%%%%%%
%%%%%%%%%%%%%%%%%%%%%%%%%%%%%%%%%%%%%%%%%%%%%%%%%%%%%%%%%%%%%%%%%%%%%%%%%%%%%%%
\newpage

\newpage

\onecolumn

\appendix

\section{Supplementary Material}

This is extra material for the submission titled
"\system: Pessimistic Cardinality Estimation Using
  $\ell_p$-Norms of Degree Sequences."
This material is organized as follows.
Section~\ref{subsec:proof:th:lpbase:eq:lptdb} gives the proof for Theorem~\ref{th:lpbase:eq:lptdb}.
Section~\ref{subsec:lptd} describes  a third optimized algorithm for estimating arbitrary conjunctive queries, which uses hypertree decompositions of the queries. This is not yet implemented in \system.
Section~\ref{subsec:lpflow:details} gives more details on the \lpflow optimization and the proof for  Theorem~\ref{th:lpbase:eq:lpflow}.
\nop{Section~\ref{app:predicates-examples} gives examples on how \system accommodates selection predicates.
Section~\ref{app:further-experiments} details further experiments.
}

%%%%%%%%%%%%%%%%%%%%%%%%%%%%%%%%%%%%%%%%%%%%%%%%%%%%%%%%%%%%%%%%%%%%%%%%%%%%%%%%%%%%%%%%%%%%
\subsection{Proof of Theorem~\ref{th:lpbase:eq:lptdb}}
\label{subsec:proof:th:lpbase:eq:lptdb}
For simplicity of presentation, we assume here that the
query $Q$ is connected.
Denote by $b_{\text{base}}$ and $b_{\text{Berge}}$ the values of
\lpbase and \lptdb.  The inequality
$b_{\text{base}}\leq b_{\text{Berge}}$ follows from two observations:
\begin{itemize}
\item For any acyclic query $Q$ and any polymatroid $h$, the
  inequality $E_Q \geq h(X_1\cdots X_n)$ is a Shannon inequality.
  This is a well known
  inequality~\cite{DBLP:journals/tse/Lee87,DBLP:conf/sigmod/KenigMPSS20}
  (which we review in Lemma~\ref{lemma:tony:lee} below).
\item Any feasible solution to \lpbase can be converted to a feasible
  solution of \lptdb by simply ``forgetting'' the terms $h(U)$ that do
  not occur in \lptdb.
\end{itemize}

For the converse, $b_{\text{Berge}}\leq b_{\text{base}}$, we will
prove that every feasible solution $h$ to \lptdb can be extended to a
feasible solution to \lpbase (by defining $h(U)$ for all terms $h(U)$
that did not appear in \lptdb), such that $E_Q = h(X_1\cdots X_n)$.
We will actually prove something stronger: that $h$ can be extended to
a \emph{normal polymatroid}.

\begin{definition} \label{def:normal}
  A set function $h : 2^{\set{X_1,\ldots,X_n}} \rightarrow \R$ is a \emph{normal
    polymatroid} if $h(\emptyset)=0$ and it satisfies:
%
  \begin{align}
    \forall U \subseteq \set{X_1,\ldots,X_n}: \sum_{W\subseteq U}(-1)^{|W|+1}h(W) \geq &0 \label{eq:normal}
  \end{align}
\end{definition}
% 
It is known that every normal polymatroid is an entropic vector, and
every entropic vector is a polymatroid, but none of the converse
holds.

The inequality $b_{\text{Berge}}\leq b_{\text{base}}$ follows from two lemmas:

\begin{lemma} \label{lemma:normal:extension:of:one:bag} Let
  $V = \set{X_1, \ldots, X_n}$, let $a_1, \ldots, a_n, A$ be $n+1$
  non-negative numbers such that:
%
  \begin{align}
    a_1 + \cdots + a_n \geq & A \ \ \ \ \text{and}\ \ \ \ \ a_i \leq A, \forall i=1,n \label{eq:a:a}
  \end{align}
%
  Then there exists a normal polymatroid $h : 2^V \rightarrow \R$ such
  that $h(X_i)=a_i$ for all $i=1,n$ and $h(V) = A$.
\end{lemma}

\begin{lemma}[Stitching Lemma] \label{lemma:stich} Let
  $V_1, V_2$ be two sets of variables, $Z \defeq V_1 \cap V_2$. Let
  $h_1:2^{V_1} \rightarrow \R$, $h_2: 2^{V_2} \rightarrow \R$ be two
  normal polymatroids that agree on their common variables $Z$: in
  other words there exists $h : 2^Z \rightarrow \R$ such that
  $\forall U \subseteq Z$, $h_1(U)=h(U)=h_2(U)$.  Define the following
  function $h' : 2^{V_1\cup V_2} \rightarrow \R$:
%
  \begin{align}
    h'(U) \defeq & h_1(U \cap V_1|U \cap Z) + h_2(U \cap V_2|U \cap Z) + h(U \cap Z) \label{eq:stich}
  \end{align}
%
  Then $h'$ is a normal polymatroid that agrees with $h_1$ on $V_1$ and
  with $h_2$ on $V_2$, and, furthermore, satisfies:
%
  \begin{align}
    h'(V_1 \cup V_2) = & h'(V_1) + h'(V_2) - h'(V_1\cap V_2) \label{eq:independent}
  \end{align}
  %
  In essence, this says that $V_1, V_2$ are independent conditioned on
  $V_1 \cap V_2$.
\end{lemma}


Notice that $h'$ can be written equivalently as
$h'(U) = h_1(U\cap V_1)+h_2(U\cap V_2) - h(U\cap Z)$.  While each term
is a normal polymatroid, it is not obvious that $h'$ is too, because
of the difference operation.  In fact, if $h_1, h_2, h$ are
polymatroids, then $h'$ is not a polymatroid in general.


The two lemmas prove Theorem~\ref{th:lpbase:eq:lptdb}, by showing that
$b_{\text{Berge}}\leq b_{\text{base}}$, as follows.  Consider any
feasible solution $h$ to \lptdb.  Consider first a single atom
$R_j(V_j)$ of $Q$: $h$ is only defined on all its variables and on the
entire set $V_j$.  By Lemma~\ref{lemma:normal:extension:of:one:bag},
we can extend $h$ to a normal polymatroid
$h: 2^{V_j} \rightarrow \Rp$.  We do this separately for each
$j=1,m$.  Next, we stitch these polymatroids together in order to
construct a polymatroid on all variables,
$h:2^{\set{X_1,\ldots, X_n}}\rightarrow \Rp$, and for this purpose we
use the Stitching Lemma~\ref{lemma:stich}.  Notice that the Lemma is
stronger than what we need, since in our case the intersection
$V_1 \cap V_2$ always has size 1 (since $Q$ is Berge-acyclic): we need
the stronger version for our third algorithm described in Sec.~\ref{subsec:lptd}.  By
using the conditional independence equality~\eqref{eq:independent}, we
can prove that $E_Q = h(X_1\cdots X_n)$, which completes the proof of
Theorem~\ref{th:lpbase:eq:lptdb}.

It remains to prove the two lemmas.

\begin{proof}[Proof of Lemma~\ref{lemma:normal:extension:of:one:bag}]
  We briefly review an alternative definition of normal polymatroids
  from~\cite{DBLP:conf/lics/Suciu23}.  For any $U \subseteq V$, the
  step function at $U$ is $h^U$ defined as:
%
\begin{align}
\forall X \subseteq V: &&  h^U(X) \defeq &
                  \begin{cases}
                    1 & \mbox{if $U\cap X\neq \emptyset$}\\
                    0 & \mbox{otherwise}
                  \end{cases} \label{eq:step:function}
\end{align}
%
When $U=\emptyset$, then $h^U \equiv 0$, so we will assume
w.l.o.g. that $U \neq \emptyset$.  A function $h : 2^V \rightarrow \R$
is a normal polymatroid iff it is a non-negative linear combination of
step functions:
%
\begin{align}
  h = & \sum_{U \subseteq V, U \neq \emptyset} c_U h^U
\end{align}
%
where $c_U \geq 0$ for all $U$.

We prove the lemma by induction on $n$, the number of variables in
$V$.  If $n=1$ then the lemma holds trivially because we define
$h(X_1) \defeq a_1$, so assume $n \geq 2$.  Rename variables such that
$a_1 \geq a_2 \geq \cdots \geq a_n$, and let $k \leq n$ be the
smallest number such that $a_1 + \cdots + a_k \geq A$: such $k$ must
exist by assumption of the lemma.  We prove the lemma in two cases.

{\bf Case 1}: $k=n$.  Let $\delta \defeq \sum_{i=1,n}a_i - A$ be the
excess of the inequality~\eqref{eq:a:a}: notice that $a_1 \geq \delta$
and $a_n \geq \delta$.  We define $h$ as follows:
  %
\begin{align*}
  h = & (a_1-\delta)h^{X_1}+\delta h^{X_1,X_n}+\sum_{i=2,n-1}a_i h^{X_i} + (a_n-\delta)h^{X_n}
\end{align*}
  %
By construction, $h$ is a normal polymatroid, and one can check by
direct calculation that $h(X_i)=a_i$ for all $i$ and $h(V)=A$.

{\bf Case 2}: $k < n$.  We prove by induction on $m=k,k+1,\ldots,n$
that there exists a normal polymatroid
$h : 2^{\set{X_1, \ldots, X_m}} \rightarrow \R$ s.t. $h(X_i)=a_i$ for
$i=1,m$ and $h(X_1\ldots X_m)=A$.  The claim holds for $m=k$ by Case
1.  Assuming it holds for $m-1$, let
$h' : 2^{\set{X_1,\ldots,X_{m-1}}}\rightarrow \R$ be such that
$h'(X_i)=a_i$ for $i=1,m-1$ and $h'(X_1\cdots X_{m-1})=A$.  We show
that we can extend it to $X_m$.  For that we first represent $h'$ over
the basis of step functions:
  %
\begin{align*}
  h' = & \sum_{U \subseteq \set{X_1,\ldots,X_{m-1}}, U\neq \emptyset}c_U  h^U
\end{align*}
  %
for some coefficients $c_U \geq 0$, and note that
$\sum_U c_U = h'(X_1\ldots X_{m-1})=A$.  Define $h$ as follows:
  %
  \begin{align*}
    h = & \sum_{U \subseteq \set{X_1,\ldots,X_{m-1}}, U\neq \emptyset}c_U\left(1-\frac{a_m}{A}\right) h^U+ \sum_{U \subseteq \set{X_1,\ldots,X_{m-1}}, U\neq \emptyset}c_U\frac{a_m}{A} h^{U\cup\set{X_m}}
  \end{align*}
  %
  By assumption of the lemma $a_m \leq A$, which implies that all
  coefficients above are $\geq 0$, hence $h$ is a normal
  polymatroid. Furthermore, by direct calculations we check that, for
  $i<m$, $h(X_i) = h'(X_i) = a_i$ (because
  $h^{U\cup \set{X_m}}(X_i)=h^U(X_i)$), and, for $i<m$,
  $h(X_m) = a_m$, because $h^U(X_m) = 0$ and
  $h^{U\cup\set{X_m}}(X_m)=1$, and the claim follows from
  $\sum_U c_U = h'(X_1\cdots X_{m-1})=A$.  Finally, we also have
  $h(X_1\ldots X_m) = \sum_U c_U = A$, as required.
\end{proof}

Finally, we prove the Stitching Lemma~\ref{lemma:stich}.  For that we
need two propositions.

\begin{proposition} \label{prop:technical:1} Let $h : 2^V \rightarrow \R$
  be a normal polymatroid, and $V_0 \supseteq V$ a superset of
  variables.  Define $h': 2^{V_0}\rightarrow \R$ by
  $h'(U) \defeq h(U \cap V)$ for all $U \subseteq V_0$.  Then $h'$ is
  a normal polymatroid.  In other words, $h'$ extends $h$ to $V_0$ by
  simply ignoring the additional variables.
\end{proposition}

\begin{proof}
  We verify condition~\eqref{eq:normal} directly.  When
  $U \subseteq V$, then $h'(W)=h(W)$ for all $W \subseteq U$ and the
  condition holds because $h$ is a normal polymatroid.  When
  $U\not\subseteq V$, then we claim that
  $\sum_{W\subseteq U}(-1)^{|W|+1}h(W \cap V)=0$.  Indeed, fix a variable
  $X_i \in U$, $X_i \not\in V$, and pair every subset $W \subseteq U$
  that does not contain $X_i$ with $W' \defeq W \cup \set{X_i}$.  Then
  $h(W\cap V)=h(W'\cap V)$ and the two terms corresponding to $W$ and
  $W'$ in~\eqref{eq:normal} cancel out, proving that the
  expression~\eqref{eq:normal} is $=0$.
\end{proof}

\begin{proposition} \label{prop:technical:2} Let $h: 2^V \rightarrow \R$ be
  a normal polymatroid, and $Z\subseteq V$ a subset of variables.
  Define the following set functions $h', h'' : 2^V\rightarrow \R$:
%
  \begin{align}
\forall U \subseteq V: &&    h'(U) \defeq & h(U\cap Z) &  h''(U) \defeq & h(U|U\cap Z) \label{eq:hprime:normal}
  \end{align}
%
  (Recall that $h(B|A) = h(AB)-h(A)$.)  Then both $h', h''$ are normal
  polymatroids.
\end{proposition}

\begin{proof}
  We consider two cases as above.  When $U\subseteq Z$, then for all
  $W \subseteq U$ we have $h'(W)=h(W)$, and $h''(W)=0$:
  condition~\eqref{eq:normal} holds for $h'$ because it holds for $h$,
  and it holds for $h''$ trivially since it is $=0$.  No consider
  $U\not\subseteq Z$.  Then we claim that the
  expression~\eqref{eq:normal} for $h'$ is $0$:
%
  \begin{align*}
    \sum_{W \subseteq U} (-1)^{|W|+1}h'(W)=& \sum_{W \subseteq U} (-1)^{|W|+1}h(W\cap Z)=0
  \end{align*}
%
  We use the same argument as in the previous lemma: pick a variable
  $X_i$ s.t. $X_i \in U$ and $X_i \not\in Z$ and pair each set
  $W \subseteq U$ that does not contain $X_i$ with the set
  $W' \defeq W \cup \set{X_i}$.  Then $h(W\cap Z)=h(W'\cap Z)$ and two
  terms for $W$ and $W'$ cancel out.  Finally,
  condition~\eqref{eq:normal} for $h''$ follows similarly:
%
  \begin{align}
    \sum_{W\subseteq U}(-1)^{|W|+1}h''(W) = & \sum_{W\subseteq U}(-1)^{|W|+1}h(W|W\cap Z) =\underbrace{\sum_{W\subseteq U}(-1)^{|W|+1}h(W)}_{\geq 0}-\underbrace{\sum_{W\subseteq U}(-1)^{|W|+1}h(W\cap Z)}_{=0}
  \end{align}
%
  The first term is $\geq 0$ because $h$ is a normal polymatroid, and
  the second term is $=0$, as we have seen.
\end{proof}

Finally, we prove the Stitching Lemma~\ref{lemma:stich}.

\begin{proof}[Proof of Lemma~\ref{lemma:stich}] We first use the two
  propositions to show that $h'$ from Eq.~\eqref{eq:stich} is a normal polymatroid.  Define two
  helper functions $h'_1 : 2^{V_1} \rightarrow \R$ and
  $h'_2 : 2^{V_2} \rightarrow \R$:
%
  \begin{align*}
\forall U \subseteq V_1:\   h'_1(U) \defeq & h_1(U|U \cap Z) & 
\forall U \subseteq V_2:\   h'_2(U) \defeq & h_2(U|U \cap Z)
  \end{align*}
%
  By Lemma~\ref{prop:technical:2}, both $h'_1, h'_2$ are normal
  polymatroids.  Next, we extend $h'_1, h'_2, h$ to the entire set
  $V_1 \cup V_2$ by defining:
  \begin{align*}
  \forall U \subseteq V_1 \cup V_2:&& h''_1(U) \defeq &h'_1(U\cap V_1)  & h''_2(U) \defeq &h'_2(U\cap V_2)  & h''(U) \defeq & h(U\cap Z)
  \end{align*}
%
  By Lemma~\ref{prop:technical:1} each of them is a normal
  polymatroid.  Since $h'$ in the corollary is their sum, it is also a
  normal polymatroid.

  We check that it agrees with $h_1$ on $V_1$.  For any $U \subseteq
  V_1$, we have $h_2(U\cap V_2|U \cap Z) = 0$ therefore:
%
  \begin{align*}
    h'(U) = & h_1(U \cap V_1|U \cap Z) + h(U \cap Z)= h_1(U|U\cap Z)+h_1(U\cap Z) = h_1(U)
  \end{align*}
%
  Similarly, $h'$ agrees with $h_2$ on $V_2$.  Finally,
  condition~\eqref{eq:independent} follows by setting
  $U:= V_1 \cup V_2$ in~\eqref{eq:stich} and applying the definition
  of conditional: $h(B|A)=h(B)-h(A)$ when $A \subseteq B$.
\end{proof}


%%%%%%%%%%%%%%%%%%%%%%%%%%%%%%%%%%%%%%%%%%%%%%%%%%%%%%%%%%%%%%%%%%%%%%%%%%%%%%%%%%%%%%%%%%%%
\subsection{\lptd: Using Hypertree Decomposition}
\label{subsec:lptd}

The \lptdb algorithm is strictly limited by two requirements: $Q$
needs to be Berge-acyclic, and all statistics need to be full.  We
describe here a generalization of \lptdb, called \lptd, which drops
these two limitations.  When the restrictions of \lptdb are met, then
\lptd is slightly less efficient, however, its advantage is that it
can work on any query and constraints, without any restrictions.


A \emph{Hypertree Decomposition} of a full conjunctive query $Q$
 is a pair $(T,\chi)$, where $T$ is a tree and $\chi:
 \nodes(T)\rightarrow 2^V$ satisfying the following:
%
 \begin{itemize}
 \item For every variable $X_i$, the set
   $\setof{t \in \nodes(T)}{X_i \in \chi(t)}$ is connected.  This is
   called the \emph{running intersection property}.
 \item For every atom $R_j(V_j)$ of $Q$, $\exists t \in \nodes(T)$
   s.t. $V_j \subseteq \chi(t)$.
 \end{itemize}
%
 Each set $\chi(t) \subseteq V$ is called a \emph{bag}.  The
 \emph{width} of the tree $T$ is defined as
 $w(T) \defeq \max_{t \in \nodes(T)}|\chi(t)|$. We review a lemma by
 Lee~\cite{DBLP:journals/tse/Lee87}:

 \begin{lemma} \label{lemma:tony:lee} ~\cite{DBLP:journals/tse/Lee87}
   Let $(T,\chi:\nodes(T)\rightarrow 2^V)$ have the running
   intersection property and let $h : 2^V \rightarrow \R$ be a set
   function.  Define:
%
  \begin{align}
    E_{T,h} \defeq & \sum_{t \in \nodes(T)}h(\chi(t))-\sum_{(t_1,t_2)\in\edges(T)}h(\chi(t_1)\cap\chi(t_2)) \label{eq:et}
  \end{align}
%
  (1) If $h$ is a polymatroid (i.e. it satisfies the basic Shannon
  inequalities), then $E_{T,h} \geq h(V)$. (2) Suppose $h$ is the
  entropic vector defined by a uniform probability distribution on a
  relation $R(X_1, \ldots, X_n)$.  Then, $E_{T,h}=h(V)$ if for every
  $(t_1,t_2)\in \edges(T)$, the join dependency
  $R = \Pi_{V_1}(R) \Join \Pi_{V_2}(R)$ holds, where
  $V_1, V_2\subseteq V$ are the variables occurring on the two
  connected components of $T$ obtained by removing the edge
  $(t_1,t_2)$.
\end{lemma}

For a simple illustration, consider the 3-way join $J_3$
(Eq.~\eqref{eq:j3}).  Its tree decomposition $T$ has 3 bags
$XY, YZ, ZU$, and $E_{T,h} = h(XY)+h(YZ)+h(ZU)-h(Y)-h(Z)$; one can
check that $E_{T,h} \geq h(XYZU)$ using two applications of
submodularity.


Our new linear program, called \lptd, is constructed from a
hypertree decomposition $(T,\chi)$ of the query as follows:

\smallskip

\noindent {\bf The Real-valued Variables} are all expressions of the
form $h(U)$ for $U \subseteq \chi(t)$, $t \in \nodes(T)$.  The total
number of real-valued variable is $\sum_{t \in \nodes(T)}
2^{|\chi(t)|}$, i.e. it is exponential in the tree-width of the query.

\smallskip

\noindent {\bf The Objective} is to maximize $E_{T,h}$
(Eq.~\eqref{eq:et}), subject to the following constraints.

\smallskip

\noindent {\bf Statistics Constraints:}
All statistics constraints Eq.~\eqref{eq:h:p} of \lpbase.
Since we don't have numerical variable $h(U)$ for all $U$, we must
check that~\eqref{eq:h:p} uses only available numerical
variables.  This holds, because each statistics refers to some atom
$R_j(V_j)$, and there exists of some bag such that
$V_j \subseteq \chi(t)$, therefore we have numerical variables $h(U)$
for all $U \subseteq V_j$.

\smallskip

\noindent {\bf Normality Constraints:}
For each bag $\chi(t)$, \lptd contains all constraints of the
form~\eqref{eq:normal}.  In other words, the restriction of $h$ to
$\chi(t)$ is normal.


We prove:

\begin{theorem} \label{th:lptd} \lpbase and \lptd compute the same
  value.
\end{theorem}

The theorem holds only when all degree constraints used in the
statistics are \emph{simple}, as we assumed throughout this paper.
For a simple illustration, the \lptd for $J_3$ consists of 7 numerical
variables \\
$h(X),h(Y),h(Z),h(U),h(XY),h(YZ),h(ZU)$ (we omit
$h(\emptyset)=0$) and the following Normality Constraints:
%
\begin{align*}
  h(X)+h(Y)-h(XY)\geq & 0 & h(Y)+h(Z)-h(YZ)\geq & 0 \\
  h(Z)+h(U)-h(ZU) \geq & 0
\end{align*}



To compare \lptd and \lptdb, assume that the query $Q$ is Berge-acyclic
and all statistics are on simple and full degree constraints.  The
difference is that, for each atom $R_j(V_j)$, \lptdb has only $1+|V_j|$
real-valued variables and only $1+|V_j|$ additivity constraints, while
\lptdb has $2^{|V_j|}$ variables and normality constraints.

In the remainder of this section we prove Theorem~\ref{th:lptd}.

Denote by $b_{\text{base}}$ and $b_{\text{td}}$ the optimal solutions
of \lpbase and \lptd respectively.  We will prove that
$b_{\text{base}}=b_{\text{td}}$.

First, we claim that $b_{\text{base}}\leq b_{\text{td}}$.  It is known
from~\cite{DBLP:journals/pacmmod/KhamisNOS24} that \lpbase has an
optimal solution $h^*$ that is a normal polymatroid; thus
$b_{\text{base}}=h^*(V)$ (recall that $V$ is the set of all
variables), where $h^*$ is normal.  Then $h^*$ is also a feasible
solution to \lptd, therefore its optimal value is at least as large as
the value given by $h^*$, in other words
$b_{\text{td}}\geq E_{T,h^*}$.  By Lemma~\ref{lemma:tony:lee}, we have
$E_{T,h^*}\geq h^*(V) = b_{\text{base}}$, which proves the claim.


Second, we prove that $b_{\text{td}}\leq b_{\text{base}}$ by using the Stitching Lemma~\ref{lemma:stich}.
Let $h^*$ be an optimal solution to \lptd, thus
$b_{\text{td}}=E_{T,h^*}$.  The function $h^*$ is defined only on
subsets of the bags $\chi(t)$, $t \in \nodes(T)$, and on each such
subset, it is a normal polymatroid.  We extend it to a normal
polymatroid defined on all variables
$V = \bigcup_{t \in \nodes(T)}\chi(t)$ by repeatedly applying the
Stitching Lemma~\ref{lemma:stich}. Condition~\eqref{eq:independent} of
the corollary implies that this extension satisfies
$h^*(V)=E_{T,h^*}$.  Thus, $h^*$ is a normal polymatroid, and, hence,
a feasible solution to \lpbase.  It follows that the optimal value of
\lpbase is at least as large as that given by $h^*$, in other words
$b_{\text{base}} \geq h^*(V)$.  This completes the proof of the claim.

%%%%%%%%%%%%%%%%%%%%%%%%%%%%%%%%%%%%%%%%%%%%%%%%%%%%%%%%%%%%%%%%%%%%%%%%%%%%%%%%%%%%%%%%%%%%

\subsection{\lpflow: Missing Details from Section~\ref{subsec:lpflow}}
\label{subsec:lpflow:details}

Section~\ref{subsec:lpflow} gives the high-level idea of the \lpflow algorithm
using an example. We give here a more formal description of the algorithm
and prove Theorem~\ref{th:lpbase:eq:lpflow}.
The input to \lpflow is an arbitrary conjunctive query $Q$ of the form Eq.~\eqref{eq:cq}
(not necessarily a full query) and a set of statistics on the input database consisting of $\ell_p$-norms
of {\em simple} degree sequences, i.e. statistics of the form $\lp{\degree_{R_j}(V|U)}_p$ where $|U|\leq 1$.
For the purpose of describing \lpflow,
we construct a flow network $G=(\nodes,\edges)$ that is defined as follows:
(Recall that $\vars(Q) =\{X_1, \ldots, X_n\}$ is the set of variables of the query.)
\begin{itemize}
    \item The set of nodes $\nodes\subseteq 2^{\vars(Q)}$ consists of the following nodes:
    \begin{itemize}
        \item The node $\emptyset$, which is the source node of the flow network.
        \item A node $\{X_i\}$ for every variable $X_i \in \vars(Q)$.
        \item A node $UV$ for every statistics $\lp{\degree_{R_j}(V|U)}_p$. 
    \end{itemize}
    \item The set of edges $\edges$ consists of two types of edges:
    \begin{itemize}
        \item {\bf Forward edges:} These are edges of the form $(a, b)$
        where $a, b \in \nodes$ and $a \subset b$. Each such edge $(a, b)$ has a finite capacity $c_{a, b}$. In particular, for every statistics
        $\lp{\degree_{R_j}(V|U)}_p$, we have two forward edges:
        One edge $(\emptyset, U)$ and another $(U, UV)$. (Recall that $|U|\leq 1$.)
        \item {\bf Backward edges:} These are edges of the form $(a, b)$
        where $a, b \in \nodes$ and $b \subset a$. Each such edge $(a, b)$ has an infinite
        capacity $\infty$.
        In particular, for every statistics $\lp{\degree_{R_j}(V|U)}_p$ and
        every variable $X_i \in UV$, we have a backward edge $(UV, \{X_i\})$.
    \end{itemize}
\end{itemize}
We are now ready to describe the linear program for \lpflow. Recall that $V_0$ is the set of \groupby variables in the query $Q$ from Eq.~\eqref{eq:cq}.

\smallskip

\noindent {\bf The Real-valued Variables} are of two types:
\begin{itemize}
    \item Every statistics $\lp{\degree_{R_j}(V|U)}_p$ has an associated {\em non-negative}
    variable $w_{U,V,p}$.
    \item For every \groupby variable $X_i \in V_0$, we have a flow variable $f_{a, b; X_i}$
    for every edge $(a, b) \in \edges$.
\end{itemize}

\smallskip

\noindent {\bf The Objective} is to minimize the following sum over all available statistics
$\lp{\degree_{R_j}(V|U)}_p$:
\begin{align}
    \sum w_{U,V,p} \cdot \log \lp{\degree_{R_j}(V|U)}_p
    \label{eq:lpflow:objective}
\end{align}

\smallskip

\noindent {\bf The Constraints} are of two types:
\begin{itemize}
    \item {\em Flow conservation constraints:} For every \groupby variable $X_i \in V_0$,
    the variables $f_{a, b;X_i}$ must define a valid flow from the source node
    $\emptyset$ to the sink node $\{X_i\}$ that has a capacity $\geq 1$.
    This means that for every node $c \in \nodes - \{\emptyset\}$, we must have:
    \begin{align}
        \sum_a f_{a, c;X_i} -\sum_b f_{c, b; X_i} \geq 1, & \quad\quad\text{ if $c =\{X_i\}$}\label{eq:flow:conservation:1}\\
        \sum_a f_{a, c;X_i} -\sum_b f_{c, b; X_i} \geq 0, & \quad\quad\text{ otherwise}
        \label{eq:flow:conservation:2}
    \end{align}
    \item {\em Flow capacity constraints:} For every \groupby variable $X_i \in V_0$
    and every {\em forward} edge $(a, b)$, the flow variable $f_{a, b;X_i}$ must satisfy:
    \begin{align}
        f_{a, b;X_i} \leq c_{a, b} \label{eq:flow:capacity}
    \end{align}
    where $c_{a, b}$ is the {\em capacity} of the forward edge $(a, b)$.
    (Recall that backward edges have infinite capacity.)
    The capacity variables $c_{a, b}$ are determined by the statistics constraints.
    In particular, every statistics $\lp{\degree_{R_j}(V|U)}_p$ contributes a capacity
    of $w_{U,V,p}$ to $c_{U,UV}$ and a capacity of $\frac{w_{U,V,p}}{p}$ to $c_{\emptyset,U}$.
    Formally,
    \begin{align}
        c_{\emptyset,U} &\defeq
        \sum_{p} w_{\emptyset,U,p}
        +\sum_{V,p} \frac{w_{U,V,p}}{p}\label{eq:lpflow:c}\\
        c_{U, UV} &\defeq
        \sum_{p} w_{U,V,p} &\text{ if $U \neq \emptyset$}\nonumber
    \end{align}
\end{itemize}
\smallskip



We are now ready to prove Theorem~\ref{th:lpbase:eq:lpflow}, which says that the linear programs
for \lpbase and \lpflow have the same optimal value.
To that end, we first write the dual LP for \lpbase.
For every statistics constraint of the form Eq.~\eqref{eq:h:p}, we introduce a dual variable $w_{U,V,p}$. The dual of \lpbase is equivalent to:
\begin{align}
    \min\quad &\sum w_{U,V,p} \cdot \log \lp{\degree_R(V|U)}_p\label{eq:lpbase:dual}\\
    \text{s.t.}\quad& \text{The following is a valid Shannon inequality:}\nonumber\\
    &h(V_0) \leq \sum w_{U,V,p} \left(\frac{1}{p}h(U)+h(V|U)\right)\label{eq:lpflow:shannon}\\
    & w_{U,V,p} \geq 0\nonumber
\end{align}
Inequality~\eqref{eq:lpflow:shannon} satisfies the property that for every $h(V|U)$
on the RHS, we have $|U| \leq 1$. In order to check that such an inequality is a valid
Shannon inequality, we rely on a key result from~\cite{DBLP:journals/corr/abs-2211-08381}.
In particular,~\cite{DBLP:journals/corr/abs-2211-08381} is concerned
with Shannon inequalities of the following form.
Let $\calX=\{X_1, \ldots, X_n\}$ be a set of variables, and $\calC$
be a set of distinct pairs $(U, V)$ where $U, V\subseteq \calX$, $U \cap V = \emptyset$
and $|U| \leq 1$.
For every pair $(U,V)\in\calC$, let $c_{U,UV}$ be a non-negative constant.
Moreover, let $V_0$ be a subset of $\calX$.
Consider the following inequality:
\begin{align}
    h(V_0) \leq \sum_{(U,V)\in\calC} c_{U, UV} h(V|U)
    \label{eq:lpflow:shannon:general}
\end{align}
\cite{DBLP:journals/corr/abs-2211-08381} describes a reduction from
the problem of checking whether Eq.~\eqref{eq:lpflow:shannon:general} is a valid Shannon inequality to a collection of $|V_0|$ network flow problems.
These flow problems are over the same network $G=(\nodes, \edges)$,
which is similar to the flow network described above for \lpflow. In particular,
\begin{itemize}
    \item The nodes are $\emptyset$, $\{X_i\}$ for each variable $X_i \in \calX$, and $UV$ for each $(U,V)\in\calC$.
    \item The edges have two types:
    \begin{itemize}
        \item Forward edges: For each $(U,V)\in\calC$, we have a forward edge $(U,UV)$ with capacity $c_{U,UV}$.
        \item Backward edges: For each $(U,V)\in\calC$ and each variable $X_i \in UV$,
        we have a backward edge $(UV, \{X_i\})$ with infinite capacity.
    \end{itemize}
\end{itemize}
\begin{lemma}[\cite{DBLP:journals/corr/abs-2211-08381}]
    Inequality~\eqref{eq:lpflow:shannon:general} is a valid Shannon inequality if and only if
    for each variable $X_i \in V_0$, there exists a flow $\left(f_{a, b;X_i}\right)_{(a, b)\in\edges}$ from the source node $\emptyset$
    to the sink node $\{X_i\}$ with capacity at least $1$.
    In particular, the flow variables $\left(f_{a, b;X_i}\right)_{(a, b)\in\edges}$ must satisfy the flow conservation constraints~\eqref{eq:flow:conservation:1} and~\eqref{eq:flow:conservation:2} and the flow capacity constraints~\eqref{eq:flow:capacity}.
\end{lemma}
Using the above lemma, we can prove Theorem~\ref{th:lpbase:eq:lpflow} as follows.
Take inequality~\eqref{eq:lpflow:shannon} and group together identical conditionals
on the RHS in order to convert it into the form of Eq.~\eqref{eq:lpflow:shannon:general}.
The coefficients $c_{U,UV}$ of the resulting Eq.~\eqref{eq:lpflow:shannon:general}
will be identical to those defined by Eq.~\eqref{eq:lpflow:c}.
Then, we can apply the lemma to check the validity of Eq.~\eqref{eq:lpflow:shannon}.
But now, the dual LP~\eqref{eq:lpbase:dual} for \lpbase is equivalent to the linear program for \lpflow.

%%%%%%%%%%%%%%%%%%%%%%%%%%%%%%%%%%%%%%%%%%%%%%%%%%%%%%%%%%%%%%%%%%%%%%%%%%%%%%%%%%%%%%%%%%%%
\nop{
\subsection{Examples: Handling predicates in \system}
\label{app:predicates-examples}

We show how \system handles predicates in the following examples.
Figure~\ref{fig:predicate_example} (left) shows a simple example of a relation $R(X,A,B)$,
where $X$ is a join attribute and $A$ and $B$ are predicate attributes.
The degree sequence for the relation $R$ with respect to the join attribute $X$ is $\deg_{R}(* | X) = (3,2)$. The $\ell_1$ and $\ell_{\infty}$-norm of the degree sequence $(3,2)$ are $3+2 = 5$ and $\max(3,2) = 3$, respectively.

For queries with predicates on attributes $A$ and $B$, using the $\ell_p$-norms of the degree sequence for the entire relation can lead to overestimation, since only a subset of tuples satisfy the predicates.
We show that how to compute tighter bounds of the $\ell_p$-norms for queries with predicates following our discussion in Section~\ref{sec:histograms}.




\begin{figure}[h]
  \centering
  \begin{minipage}[b]{0.50\textwidth}
      \centering
      $R$ \\[0.2em]
      \begin{tabular}{|c|c|c|}
          \hline
          $X$ & $A$ & $B$\\
          \hline
          1 & 1 & 0\\
          1 & 1 & 8\\
          1 & 2 & 13\\
          2 & 3 & 22\\
          2 & 4 & 40\\
          \hline
      \end{tabular}
  \end{minipage}
  % \hfill
  % ex 2
  \begin{minipage}[b]{0.16\textwidth}
      \centering
      $T$ \\[0.2em]
      \begin{tabular}{|c|c|}
          \hline
          $TID$ & $SID$ \\
          \hline
          1 & 1\\
          2 & 2\\
          3 & 2\\
          4 & 3\\
          5 & 3\\
          \hline
      \end{tabular}
  \end{minipage}
  \begin{minipage}[b]{0.16\textwidth}
    \centering
    $S$ \\[0.2em]
    \begin{tabular}{|c|c|}
        \hline
        $SID$ & $A$ \\
        \hline
        1 & 1\\
        2 & 1\\
        3 & 2\\
        \hline
    \end{tabular}
  \end{minipage}
  \begin{minipage}[b]{0.16\textwidth}
    \centering
    $TS$ \\[0.2em]
    \begin{tabular}{|c|c|c|}
        \hline
        $TID$ & $SID$ & $A$ \\
        \hline
        1 & 1 & 1\\
        2 & 2 & 1\\
        3 & 2 & 1\\
        4 & 3 & 2\\
        5 & 3 & 2\\
        \hline
    \end{tabular}
  \end{minipage}
  \caption{Left: relation $R(X,A,B)$, where $X$ is the join attribute and $A$ and $B$ are predicate attributes. Right: Right: relations $T(TID, SID)$ and $S(SID, A)$ where $SID$ is a foreign key in $T$ and the primary key in $S$, and $TS$ is the join of $T$ and $S$.}
  \label{fig:predicate_example}
\end{figure}

\paragraph{Equality Predicate.} 
Consider an equality predicate on $A$.
The $A$-values in $R$, sorted by frequency in descending order, are $(1,2,3,4)$. 
For this example, we consider the most common value $A=1$ as the only MCV for $A$.
We fetch the tuples satisfying $A=1$, which are the first two tuples, and get the degree sequence $\deg_{R}(* | X, A=1) = (2)$.
The $\ell_p$-norm of the degree sequence is $2$ for any $p\geq 1$.

We also compute one degree sequence for all non-MCVs of $A$, i.e., $A\in\{2,3,4\}$.
We fetch the last three tuples in $R$ and compute the degree sequence $\deg_{R}(* | X, A\in \{2, 3, 4\}) = (2, 1)$.
For an arbitrary non-MCV, there is at most one distinct $X$-value in $R$ associated with it,
so we take the top value of $\deg_{R}(* | X, A\in \{2, 3, 4\})$ as the degree sequence, i.e., $(2)$, for all non-MCVs of $A$.
The $\ell_p$-norm of the degree sequence is $2$ for any $p\geq 1$.

An alternative and more accurate, yet more expensive approach is to compute the degree sequence for each non-MCV of $A$: $\deg_{R}(* | X, A=2) = (1)$, $\deg_{R}(* | X, A=3) = (1)$, and $\deg_{R}(* | X, A=4) = (1)$, and then compute the maximum of their $\ell_p$-norms.
The $\ell_p$-norms of these degree sequences are $1$, so the maximum of the $\ell_p$-norms is $1$, for any $p\geq 1$.

To estimate for a query with the predicate $A=1$, we use the $\ell_p$-norms for the MCV of $A$, i.e., $\ell_p = 2$. For a query with the predicate $A=2$, where $2$ is a non-MCV of $A$, we use the $\ell_p$-norms for non-MCVs of $A$, i.e., $\ell_p = 2$ or, if we use the more accurate alternative, $\ell_p = 1$.

\paragraph{Range Predicate.}
Consider a range predicate on $B$
The domain of $B$ is $[0, 40]$. We create a hierarchy of histograms with $2^k, 2^{k-1}, \ldots, 2^0$ buckets.
For this example, we use $k=2$, which means creating $4$ buckets for the bottom layer, $2$ buckets for the next layer, and one bucket for the entire domain.
The buckets for the layers are $\{[0, 10), [10, 20), [20, 30), [30, 40]\}$,  $\{[0, 20), [20, 40]\}$, and $\{[0,40]\}$.

We first construct the $\ell_p$-norms within each bucket $[s, e)$.
For this, we fetch the tuples where $B$ is in the bucket
and compute the degree sequence $\deg_{R}(* | X, B \in [s, e))$ and several $\ell_p$-norms on this degree sequence.
The degree sequences for the buckets in the layers are $((2), (1), (1), (1))$, then $((3), (2))$, and finally $(5)$. The $\ell_p$-norms within each bucket 


thus their $\ell_1$ and $\ell_{\infty}$-norms are 
$\{(\ell_1=\ell_{\infty}=2), (\ell_1=\ell_{\infty}=1), (\ell_1=\ell_{\infty}=1), (\ell_1=\ell_{\infty}=1)\}$ and $\{(\ell_1=\ell_{\infty}=3), (\ell_1=\ell_{\infty}=2)\}$, respectively.

To estimate for the range predicate $5 \leq B \leq 18$, we first find the smallest bucket that covers the range, which is the bucket $[0, 20)$, and use the corresponding $\ell_p$-norms: $\ell_1 = 2$ and $\ell_{\infty} = 2$.


\paragraph{Conjunction and Disjunction of Predicates.}
We show how \system handles the conjunction and disjunction of predicates on $A$ and $B$.
Consider the predicates $A = 0$ and $5 \leq B \leq 18$.
We fetch the $\ell_p$-norms for the two predicates as discussed in the previous examples: $\ell_1 = 1$ and $\ell_{\infty} = 1$ for $A = 1$, and $\ell_1 = 2$ and $\ell_{\infty} = 2$ for $5 \leq B \leq 18$.

For the conjunction of the predicates, we take the minimum of these $\ell_p$-norms to estimate the query. The result is $\ell_1 = \min(1, 2) = 1$ and $\ell_{\infty} = \min(1, 2) = 1$.

For the disjunction of the predicates, we take the sum of the $\ell_1$-norms and the maximum of the $\ell_{\infty}$-norms, which results in $\ell_1 = 1+2 = 3$ and $\ell_{\infty} = \max(1, 2) = 2$.



\paragraph{Optimization 1: Predicate Propagation via FK-PK Joins.}
Consider two relations $T(TID, SID)$ and $S(SID, A)$ in Figure~\ref{fig:predicate_example} (right),
 where $SID$ is a foreign key in $T$ and a primary key in $S$, and $A$ is an equality predicate attribute.
We compute the $\ell_p$-norms for predicates on $A$ in $S$ as discussed in the previous examples: $\ell_1 = 2$ and $\ell_{\infty} = 1$ for the MCV $A=1$, and $\ell_1 = 1$ and $\ell_{\infty} = 1$ for non-MCVs of $A$.

For relation $R$, we apply the optimization to propagate the predicate on $A$ from $S$ to $T$:
We precompute the join results for the FK-PK join $TS(TID, SID, A) = T(TID, SID) \wedge S(SID, A)$ (Figure~\ref{fig:predicate_example} (right)). The size of the join results is bounded by the size of the FK relation $T$.
For this example, we consider only one MCV of $A$, so the only MCV of $A$ is $A=1$.
We fetch the tuples satisfying $A=1$ in $TS$, which are the first two tuples in $TS$, and compute the degree sequence $\deg_{TS}(* | TID, A=1) = (1,1,1)$.
The $\ell_p$-norms of the degree sequence are $\ell_1 = 1+1+1 = 3$ and $\ell_{\infty} = 1$.
Regarding the non-MCVs of $A$, there is only one non-MCV of $A$, which is $A=2$. We compute the degree sequence for $A=2$ in $TS$, i.e., $\deg_{TS}(* | TID, A=2) = (1,1)$ and the $\ell_p$-norms for the degree sequence are $\ell_1 = 1+1 = 2$ and $\ell_{\infty} = 1$.

Consider a query with predicate $A=1$.
For relation $S$, we use the $\ell_p$-norms for the MCV $A=1$ in $S$, i.e., $\ell_1 = 2$ and $\ell_{\infty} = 1$.
For relation $T$, we use the $\ell_p$-norms for the MCV $A=1$ in the join result $TS$, i.e., $\ell_1 = 3$ and $\ell_{\infty} = 1$.

\paragraph{Optimization 2: Compute $\ell_p$-norms for Prefixes of the Degree Sequence.}
Consider two relations $R(X,A)$ and $S(X,B)$ where $X$ is a join attribute, and their degree sequences are $(100, 99, \ldots, 2, 1)$ and $(2, 1)$, respectively.
This means that there are $100$ distinct $X$-values in $R$ and $2$ distinct $X$-values in $S$, which is significantly mis-calibrated.
When the two relations are joined, at most two $X$-values appear in the join results.
If we use the $\ell_p$-norms of the whole degree sequence for $R$, which are $\ell_1 = 5050$ and $\ell_{\infty} = 100$, the estimation can be significantly overestimated.

We reduce overestimation by computing the $\ell_p$-norms for prefixes of the degree sequence for $R$.
We compute the $\ell_p$-norms for the top-$2^i$ values of the degree sequence for $R$ for $i>0$. 
For example, for $i=1$, we compute the $\ell_p$-norms for the top-$2$ values of the degree sequence, which are $(100, 99)$: $\ell_1 = 100+99 = 199$ and $\ell_{\infty} = 100$.
For the join of $R$ and $S$, since there are at most two $X$-values in the join results, we can use these $\ell_p$-norms for the estimation.
}

%%%%%%%%%%%%%%%%%%%%%%%%%%%%%%%%%%%%%%%%%%%%%%%%%%%%%%%%%%%%%%%%%%%%%%%%%%%%%%%%%%%%%%%%%%%%



% moved to the main body
\nop{
%%%%%%%%%%%%%%%%%%%%%%%%%%%%%%
\begin{figure*}[t]
    \centering
    \begin{minipage}[b]{0.48\textwidth}
        \centering
        \includegraphics[width=\textwidth]{experiments/overall_runtime.pdf}
    \end{minipage}
    \hfill
    \begin{minipage}[b]{0.48\textwidth}
        \centering
        \includegraphics[width=\textwidth]{experiments/relative_runtime.pdf}
    \end{minipage}
    \caption{Left: Overall evaluation time of all queries in a benchmark for \psql when using estimates for all subqueries from \system, \safebound, \dbx, \psql and true cardinalities. Right: Relative evaluation times compared to the baseline evaluation time obtained when using true cardinalities.}
    \label{fig:runtime}
\end{figure*}
%%%%%%%%%%%%%%%%%%%%%%%%%%%%%%


\subsection{Further Experiments}
\label{app:further-experiments}

We complement the experiments in Section~\ref{sec:experiments} with further experiments that cannot be accommodated in the main body due to lack of space.

\subsubsection{Estimation Errors}

Fig.~\ref{fig:estimates-STATS} shows that the accuracy of the estimators decreases with the number of relations per query (shown for STATS, a similar trend also holds for JOBlight and JOBrange): The traditional estimators underestimate more, whereas the pessimistic estimators overestimate more. \neurocard starts with a large overestimation for a join of two relations and decreases its estimation as we increase the number of relations; the other ML-based estimators follow this trend but at a smaller scale.


\subsubsection{Optimization Improvements}
Fig.~\ref{fig:improvements-optimizations} shows the improvements to the estimation accuracy brought by each of the two optimizations discussed in Sec.~\ref{sec:histograms}, when taken in isolation.

The left figure shows that,  when propagating predicates from the primary-key relation to the foreign-key relations, the estimation error can improve by over an order of magnitude in the worst case (corresponding to the upper dots in the plot) and by roughly 5x in the median case (corresponding to the red line in the boxplots). 

The right figure shows that,  when using prefix degree sequences for the degree sequences of relations without predicates, the estimation error can improve by up to 50\% for JOBlight queries, up to 65\% for JOBrange queries and up to 10\% for STATS queries. The improvement is measured as the division of (i) the difference between the estimation error without this optimization and the estimation error with this optimization and (2) the the estimation error without this optimization.




\subsubsection{Evaluation Times}




Fig.~\ref{fig:runtime} (left) shows the aggregated \psql evaluation time of all queries in JOBlight, JOBrange, and STATS when using estimates for all sub-queries from \system, \safebound, \dbx, \psql, and true cardinalities (left).
Fig.~\ref{fig:runtime} (right)  shows the relative evaluation times compared to the baseline evaluation time obtained when using true cardinalities. 
We have two observations. First, overestimation can be beneficial for performance of expensive queries, which has been discussed in Section~\ref{sec:experiments}. Second, overestimation can be detrimental for performance of less expensive queries in some cases.

The first observation is reflected in the overall evaluation times, which are dominated by the most expensive queries in the benchmark (some of which are listed in Fig.~\ref{fig:most-expensive-queries}). 
Traditional approaches lead to higher evaluation times for the expensive queries, and therefore to higher overall evaluation times, while the pessimistic approaches lead to lower evaluation times for those expensive queries. Overall, the  evaluation times for the pessimistic approaches are about the same (JOBlight and STATS) or lower (JOBrange) than the baseline evaluation times.
The second observation is reflected in the relative evaluation times for the JOB benchmarks. 
The boxplots for the traditional approaches are lower than those for the pessimistic approaches, indicating that the traditional approaches perform better for the less expensive queries in the benchmarks.

\factorjoin has both high overall evaluation time and high relative evaluation time.
It estimates very accurately for the queries in STATS, thus has similar evaluation time to the baseline evaluation time. For the queries in JOBlight and JOBrange, it mostly overestimates, which leads to lower evaluation times for the expensive queries. However, the overestimations are significant, which makes it perform worse than the pessimistic approaches for the less expensive queries, as shown in the right plot of Fig.~\ref{fig:runtime}. This leads to the high overall evaluation time of \factorjoin.

}

\nop{
\subsection{About Traditional Estimators}

% PostgreSQL
\subsubsection{\psql}
\psql uses four types of statistics to estimate cardinalities:
\begin{itemize}
    \item cardinality of each relation (row count), and the number of pages per relation
    \item number of distinct values for each attribute
    \item MCVs for each attribute and their relative frequencies, i.e., the estimated fraction of rows that have the MCV as the respective attribute value
    \item histogram if the values in each attribute and relative frequency of each histogram bucket
\end{itemize}

The first statistic, the cardinality of each relation, is very accurate. The other statistics, however, are estimated based on a sample of the data. The default sampling size is 30,000 rows. We found that those statistics are often times inaccurate. For example, for some join attributes of the IMDB dataset, the domain size was underestimated by 70\%.

While the statistics mentioned above are computed by default, \psql can be instructed to compute multivariate statistics capturing correlations between attributes of the same relation via the \texttt{CREATE STATISTICS} command.

\paragraph{Selectivity of Equality Predicates and Join Conditions}

\psql uses the concept of {\em selectivity} of a (filter or join) condition to estimate the cardinality of a query output. If the predicate value is a MCV, then \psql considers the relative frequency of this value as its selectivity. If the value is not an MCV, then \psql either use histograms to estimate the frequency of the value in the relation, or it falls back to a default estimate, such as assuming a uniform distribution. In the latter case, the selectivity is assumed to be the estimated domain size of the attribute divided by cardinality of relation. Correlations of attributes across relations are not considered. For join conditions, \psql assumes that the join attributes are independent.

Consider the following join of the two relation $R(A,C)$ and $S(B,C)$ with two equality predicates on the non-join attributes.
\begin{verbatim}
    SELECT * FROM R, S WHERE R.A = 5 AND S.B = 10 AND R.C = S.C;
\end{verbatim}
The estimated cardinality of this query is:
\begin{align*}
    |R| * |S| * \sel{R.A = 5} * \sel{S.B = 10} * \sel{R.C = S.C}.
\end{align*}

\paragraph{Range Conditions}

\subsubsection{\duckdb}
% DuckDB
}


\end{document}

% This document was modified from the file originally made available by
% Pat Langley and Andrea Danyluk for ICML-2K. This version was created
% by Iain Murray in 2018, and modified by Alexandre Bouchard in
% 2019 and 2021 and by Csaba Szepesvari, Gang Niu and Sivan Sabato in 2022.
% Modified again in 2023 by Sivan Sabato and Jonathan Scarlett.
% Previous contributors include Dan Roy, Lise Getoor and Tobias
% Scheffer, which was slightly modified from the 2010 version by
% Thorsten Joachims & Johannes Fuernkranz, slightly modified from the
% 2009 version by Kiri Wagstaff and Sam Roweis's 2008 version, which is
% slightly modified from Prasad Tadepalli's 2007 version which is a
% lightly changed version of the previous year's version by Andrew
% Moore, which was in turn edited from those of Kristian Kersting and
% Codrina Lauth. Alex Smola contributed to the algorithmic style files.
