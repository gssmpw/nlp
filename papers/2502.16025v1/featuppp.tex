%%%%%%%% ICML 2023 EXAMPLE LATEX SUBMISSION FILE %%%%%%%%%%%%%%%%%

\documentclass{article}

% Recommended, but optional, packages for figures and better typesetting:
\usepackage{microtype}
\usepackage{graphicx}
\usepackage{subfigure}
\usepackage{booktabs} % for professional tables

% hyperref makes hyperlinks in the resulting PDF.
% If your build breaks (sometimes temporarily if a hyperlink spans a page)
% please comment out the following usepackage line and replace
% \usepackage{icml2023} with \usepackage[nohyperref]{icml2023} above.
\usepackage{hyperref}


% Attempt to make hyperref and algorithmic work together better:
\newcommand{\theHalgorithm}{\arabic{algorithm}}

% Use the following line for the initial blind version submitted for review:
% \usepackage{icml2025}

% If accepted, instead use the following line for the camera-ready submission:
\usepackage[accepted]{icml2025}

% For theorems and such
\usepackage{amsmath}
\usepackage{amssymb}
\usepackage{mathtools}
\usepackage{amsthm}
\usepackage{multirow}
\usepackage{bm}

% if you use cleveref..
\usepackage[capitalize,noabbrev]{cleveref}

%%%%%%%%%%%%%%%%%%%%%%%%%%%%%%%%
% THEOREMS
%%%%%%%%%%%%%%%%%%%%%%%%%%%%%%%%
\theoremstyle{plain}
\newtheorem{theorem}{Theorem}[section]
\newtheorem{proposition}[theorem]{Proposition}
\newtheorem{lemma}[theorem]{Lemma}
\newtheorem{corollary}[theorem]{Corollary}
\theoremstyle{definition}
\newtheorem{definition}[theorem]{Definition}
\newtheorem{assumption}[theorem]{Assumption}
\theoremstyle{remark}
\newtheorem{remark}[theorem]{Remark}
\DeclareMathOperator{\jbu}{\textup{JBU}}
\DeclareMathOperator*{\softmax}{\textup{softmax}}
\DeclareMathOperator*{\bilinear}{\textup{BL}}
\DeclareMathOperator*{\tile}{\textup{Tile}}

\def\authornote#1#2#3{{\textcolor{#2}{\textsl{\small[#1: #3]}}}}
\newcommand{\ANY}[1]{\authornote{\textbf{Authors}}{red}{#1}}
\newcommand{\MR}[1]{\authornote{\textbf{Mike}}{blue}{#1}}
\newcommand{\PM}[1]{\authornote{\textbf{Pavlo}}{orange}{#1}}
\newcommand{\ul}[1]{\underline{#1}}

% Todonotes is useful during development; simply uncomment the next line
%    and comment out the line below the next line to turn off comments
%\usepackage[disable,textsize=tiny]{todonotes}
\usepackage[textsize=tiny]{todonotes}


% The \icmltitle you define below is probably too long as a header.
% Therefore, a short form for the running title is supplied here:
\icmltitlerunning{FeatSharp: Your Vision Model Features, Sharper}

\begin{document}

\twocolumn[
\icmltitle{FeatSharp: Your Vision Model Features, Sharper}

% It is OKAY to include author information, even for blind
% submissions: the style file will automatically remove it for you
% unless you've provided the [accepted] option to the icml2023
% package.

% List of affiliations: The first argument should be a (short)
% identifier you will use later to specify author affiliations
% Academic affiliations should list Department, University, City, Region, Country
% Industry affiliations should list Company, City, Region, Country

% You can specify symbols, otherwise they are numbered in order.
% Ideally, you should not use this facility. Affiliations will be numbered
% in order of appearance and this is the preferred way.
\icmlsetsymbol{equal}{*}

\begin{icmlauthorlist}
\icmlauthor{Mike Ranzinger}{comp}
\icmlauthor{Greg Heinrich}{comp}
\icmlauthor{Pavlo Molchanov}{comp}
\icmlauthor{Jan Kautz}{comp}
\icmlauthor{Bryan Catanzaro}{comp}
\icmlauthor{Andrew Tao}{comp}
% \icmlauthor{Firstname2 Lastname2}{equal,yyy,comp}
% \icmlauthor{Firstname3 Lastname3}{comp}
% \icmlauthor{Firstname4 Lastname4}{sch}
% \icmlauthor{Firstname5 Lastname5}{yyy}
% \icmlauthor{Firstname6 Lastname6}{sch,yyy,comp}
% \icmlauthor{Firstname7 Lastname7}{comp}
% %\icmlauthor{}{sch}
% \icmlauthor{Firstname8 Lastname8}{sch}
% \icmlauthor{Firstname8 Lastname8}{yyy,comp}
%\icmlauthor{}{sch}
%\icmlauthor{}{sch}
\end{icmlauthorlist}

\icmlaffiliation{comp}{NVIDIA}

\icmlcorrespondingauthor{Mike Ranzinger}{mranzinger@nvidia.com}


% You may provide any keywords that you
% find helpful for describing your paper; these are used to populate
% the "keywords" metadata in the PDF but will not be shown in the document
\icmlkeywords{Machine Learning, ICML}

\vskip 0.3in
]

% this must go after the closing bracket ] following \twocolumn[ ...

% This command actually creates the footnote in the first column
% listing the affiliations and the copyright notice.
% The command takes one argument, which is text to display at the start of the footnote.
% The \icmlEqualContribution command is standard text for equal contribution.
% Remove it (just {}) if you do not need this facility.

\printAffiliationsAndNotice{}  % leave blank if no need to mention equal contribution
% \printAffiliationsAndNotice{\icmlEqualContribution} % otherwise use the standard text.

\begin{abstract}
The feature maps of vision encoders are fundamental to myriad modern AI tasks, ranging from core perception algorithms (e.g. semantic segmentation, object detection, depth perception, etc.) to modern multimodal understanding in vision-language models (VLMs). Currently, in computer vision, the frontier of general purpose vision backbones are Vision Transformers (ViT), typically trained using contrastive loss (e.g. CLIP). A key problem with most off-the-shelf ViTs, particularly CLIP, is that these models are inflexibly low resolution. Most run at 224x224px, while the ``high resolution'' versions are around 378-448px, but still inflexible. We introduce a novel method to coherently and cheaply upsample the feature maps of low-res vision encoders while picking up on fine-grained details that would otherwise be lost due to resolution. We demonstrate the effectiveness of this approach on core perception tasks as well as within agglomerative model (RADIO) training as a way of providing richer targets for distillation.
\end{abstract}

\section{Introduction}\label{sec:intro}

\begin{figure*}[!ht]
    \centering
    \includegraphics[width=\linewidth]{resources/featsharp_arch.pdf}
    \vspace{-7mm}
    \caption{Upsampling architecture diagram. We combine the upsampled features coming from FeatUp \cite{fu2024featup} with the tiled features and mix them with FeatSharp to produce a feature map with higher fidelity. The tiled features have more detail, but also representation issues such as the difference in upper and lower body at the tile boundary. ``Full-Res Reference'' is for display purposes, as for a model that doesn't exhibit stable resolution scaling (e.g. DFN CLIP, SigLIP, etc.) we don't have access to a target hi-res feature map. Our proposed learnable modules are in green, combined with a learnable FeatUp JBU stack.}
    \label{fig:featsharp_arch_diagram}
\end{figure*}

The use of vision foundation models (VFM) \cite{awais2023foundational} has seen widespread use since the beginning of the modern era of computer vision using deep learning \cite{krizhevsky2012alexnet}, primarily used to perform transfer learning \cite{plested2022deeptransferlearningimage} (e.g. finetuning a VFM on a downstream task), information retrieval \cite{babenko2014neuralcodes,zhang2024retrieval}, and most recently, to power visual capabilities for vision-language models (VLM) \cite{alayrac2022flamingovisuallanguagemodel,openai2024gpt4technicalreport,liu2023llava,lin2023vila}. The recent shift toward using Transformers~\cite{vaswani2017attention} for computer vision (ViT~\cite{dosovitskiy2021image}) has both tremendously moved the field forward, but has generally left the use of VFMs in a tricky spot: Transformers are computationally demanding and have poor algorithmic scaling properties ($O(n^2)$ for 1D sequences, or $O(\left(w \cdot h \right)^2$ for 2D inputs), leaving the majority of models to be relatively low-resolution. For example, perhaps the most popular family of VFMs to date, CLIP~\cite{radford2021clip}, typically runs at 224 or 336px input resolutions, and produces spatial features at a 14x downsample (e.g. $224^2 \rightarrow 16^2$). Owing to the nature of learned position embeddings, ViTs also tend to be relatively inflexible to changes of input resolution, allowing for changes, but requiring finetuning \cite{dosovitskiy2021image}. 

It's possible that the strict dependence on the training resolution is an artifact of the algorithm used for training, as DINOv2~\cite{oquab2023dinov2,darcet2023vision} is quite robust to interpolating its position embeddings, producing stable features at various resolutions \cite{ranzinger2023amradio}, ignoring for the moment that DINOv2, being a transformer, is expensive to use at high-resolution. A recent technique called AM-RADIO~\cite{ranzinger2024phisdistributionbalancinglabelfree}, borrowing ideas from ViTDet~\cite{li2022vitdet}, FlexiViT~\cite{beyer2023flexivit}, and RO-ViT~\cite{kim2023regionaware}, has attempted to create a resolution-flexible ViT, however it is still dependent on low-resolution ViTs as it distills from other seminal VFMs which are low-res-only: DFN CLIP~\cite{fang2023data} and SigLIP~\cite{zhai2023sigmoid}. 

Recently, FeatUp~\cite{fu2024featup} aims to directly address the problem of low-resolution vision features by using one of two learned upsampling algorithms: A model-specific generalized upsampler using Joint Bilateral Upsampling (JBU) \cite{kopf2007jbu}, or a model-specific-image-specific implicit network. While they demonstrate particularly compelling results with their implicit network, their results using the stack of JBU filters lack refined details (shown in figure \ref{fig:featup_jbu_vs_implicit} in the appendix). Along with lack of granular refinement, it's impossible for this approach to capture fine-grained details that are too small for the vision backbone to detect at its native resolution. To this end, we take inspiration from both FeatUp's JBU approach, as well as the recent trend in VLMs such as LLaVA 1.6~\cite{liu2024llavanext1p6}, InternVL-1.5~\cite{chen2024internvl1p5}, NVLM~\cite{dai2024nvlmopenfrontierclassmultimodal} and Eagle~\cite{shi2024eagleexploringdesignspace} to tile an image, aggregating local features from a fixed-low-resolution model, to build an upsampler that simultaneously leverages the raw pixel guidance, low-res feature guidance, and regional tile guidance, resulting in substantially more detailed feature maps which are also capable of capturing details too small for the original resolution. Specifically, we:

\begin{itemize}
    \item Build on top of FeatUp's JBU algorithm \cite{fu2024featup} by adding de-biasing and tiling fusion modules to incorporate detailed tile features, resulting in significantly higher levels of detail, with extensive experiments demonstrating effectiveness
    \item Study the relationship between FeatUp's feature consistency and ViT-Denoiser's~\cite{yang2024denoising} approach to cleaning the features of a ViT at its native resolution
    \item Introduce an improved training setting for AM-RADIO \cite{ranzinger2024phisdistributionbalancinglabelfree} demonstrating a +$0.6\%$ improvement across the entire benchmark suite, and better teacher adapter features
\end{itemize}

\section{Related Work}\label{sec:related}

\begin{figure}[t]
    \centering
    \includegraphics[width=0.6\linewidth]{resources/featsharp_module.pdf}
    \caption{Diagram of the FeatSharp module. We first concatenate the bilinear upsampled and tiled mosaic feature maps along the channel dimension. We then apply a transformer block with sliding window attention followed by MLP (in this case, SwiGLU), and then slice off the first half of the channels, corresponding to the bilinear upsampled buffer. The role of FeatSharp thus is to refine the bilinear buffer by leveraging the tile buffer.}
    \label{fig:featsharp_module_diagram}
\end{figure}

\paragraph{Feature Upsampling}
The most obvious baseline for feature upsampling is to use traditional filtering approaches such as bilinear or bicubic upsampling. The alternative is to evaluate the network at higher resolution, however it comes with the dual drawback that computational cost increases (quadratically in the case of Vision Transformers), and also that many models (ViTs in particular) have trouble extrapolating from their trained resolution \citep{beyer2023flexivit,dehghani2023navit}. If we expand our view to include parametric approaches, then deconvolution \cite{noh2015deconv,shi2016deconv,dumoulin2016AGT} and resize-conv \cite{odena2016deconvcheck} are popular choices. There are also pixel-adaptive approaches such as CARAFE~\cite{Wang2019CARAFECR}, SAPA~\cite{lu2022sapa}, and FeatUp~\cite{fu2024featup}. 

We adopt FeatUp's formulation of multi-view consistency as a way to train an upsampler, however, we notice that instead of solely relying on raw RGB pixels as guidance for upsampling, we can also use a small, fixed budget of inferences (similar in spirit to their implicit model), and use a mosaic of tiles as guidance at the higher resolution. This choice gives us a richer, and semantically relevant, feature space to merge from. Additionally, it allows us to incorporate information from regions that were too small for the low-res view, but become visible within a tile. Small details are a limitation of every approach that doesn't acquire extra samples from the base model, as they rely on all relevant information already being encoded by the initial model evaluation.

\paragraph{Feature Denoising}
Related to multi-view consistency, ViT-Denoiser~\cite{yang2024denoising} noticed that ViT features are generally very noisy (although some are much cleaner than others), and also propose a multi-view consistency formulation to learn how to separate fixed noise, conditional noise, and semantic content. We notice the deep ties between ViT-Denoiser and FeatUp, in that multi-view consistency provides a way to eradicate fixed-pattern noise from the feature buffer. Drawing inspiration from this, we add a learnable bias buffer (similar to learned position embeddings) at the output of the base model. This simple change works because fixed patterns will degrade multi-view consistency, as the pattern is always local to the view, and lacks global coherence.

\paragraph{VLMs}
The use of tiling to increase information is currently very prominent in VLMs \cite{liu2024llavanext1p6,chen2024internvl1p5,dai2024nvlm}, albeit an alternative approach is to instead leverage the models at hi-res themselves \cite{beyer2024paligemmaversatile3bvlm,wang2024qwen2vlenhancingvisionlanguagemodels}. We also see RADIO-Amp\citep{heinrich2024radioamplifiedimprovedbaselines} being primarily useful at high-resolution within VLMs. In the increasingly VLM-centric approach to computer vision, we turn our focus to RADIO-Amp, as it has a training procedure that relies on matching a high-resolution student against a low-resolution teacher, an application area that is perfect for studying feature upsampling, as it would provide richer guidance to the distillation.

\paragraph{Agglomerative Models}
In the agglomerative model space, there are currently three major approaches: RADIO \citep{ranzinger2023amradio,ranzinger2024phisdistributionbalancinglabelfree,heinrich2024radioamplifiedimprovedbaselines}, Theia \citep{shang2024theia}, and UNIC \citep{sariyildiz2024unic}. We focus our attention on RADIO because it is the only approach that directly tries to tackle resolution flexibility as well as high-resolution.

\section{Method}\label{sec:method}

We leverage FeatUp's training algorithm of treating the upsampling problem as that of multi-view consistency between the upsampled and then downsampled features and different low-res views of the same image.

\begin{figure*}[t]
    \centering
    \includegraphics[width=\linewidth]{resources/tile_process.pdf}
    \vspace{-8mm}
    \caption{Visualization of the tiling process. An input image (left) is split into $2 \times 2$ tiles, each of which is resized to match the input resolution of the encoder, fed through the encoder independently, and then stitched back into a higher resolution feature map. Feature maps shown are from DFN CLIP, and they are resized to be larger than actual for demonstration purposes.}
    \label{fig:tile_process}
\end{figure*}

\subsection{Review - FeatUp: Joint Bilateral Upsampling (JBU)}\label{sec:method:featup}

Given a high-resolution signal $G$ (e.g. the raw pixels) as guidance, and a low-resolution signal $F_{lr}$ that we'd like to upsample, and let $\Omega$ be a neighborhood of each pixel in the guidance. Let $k(\cdot, \cdot)$ be a similarity kernel that measures how close two vectors are. Then

\begin{equation}
\begin{split}
\hat{F}_{hr}[i, j] = \frac{1}{Z} \sum_{(a, b) \in \Omega} \Bigl(&F_{lr}[a, b] \cdot \\
    & k_{range}\left(G[i, j], G[a, b]\right)\cdot \\
    & k_{spatial}\left([i, j], [a, b] \right)  \Bigr)
    \label{eq:jbu_orig}
\end{split}
\end{equation}

with $Z$ being a normalization to make the kernel sum to 1. $k_{spatial}$ is a Gaussian kernel with learnable $\sigma_{spatial}$ defined as

\begin{equation}
    k_{spatial}(x,y) = \exp\left(\frac{-\left\lVert x - y \right\rVert_2^2}{2\sigma_{spatial}^2}\right)
    \label{eq:k_spatial}
\end{equation}

and $k_{range}$ as

\begin{equation}
    k_{range}(x,y) = \softmax_{(a, b) \in \Omega} \left(\frac{1}{\sigma_{range}^2} h(G[x,y]) \cdot h(G[a,b]) \right)
    \label{eq:k_range}
\end{equation}

with $h(x)$ being a learned MLP projector. They define

\begin{equation}
    F_{hr} = \left(\jbu(\cdot, x) \circ \jbu(\cdot, x) \circ ... \right)(f(x), x)
\end{equation}

as a stack of $2\times$ upsamplers, thus enabling power-of-2 upsample factors. With $x$ being the original input image, and $f(x)$ being the low-resolution feature map. We note that $2$ isn't a necessary part of the architecture, and that their implementation supported arbitrary factors, so we simply propose to take a given upsample factor $z \in \mathbb{Z}_{+}$ and prime factorize $z$ to get a set of upsample factors, using a $\jbu_k$ for each prime factor. This decomposes to an identical operation as before when $\log_2 z \in \mathbb{Z}_{+}$, but allows for an easy guide for any other integer, e.g. for a $14\times$ upsample corresponding to a patch-size-14 backbone, we'd use a $\left(\jbu_{7\times} \circ \jbu_{2\times}\right)(f(x), x)$ stack.

As is typical with bilateral upsampling, this method is very sensitive to strong edges in the guidance buffer, however, it also tends to over-smooth features in regions of lower contrast. Particularly, it struggles with feature patterns such as SAM (figure \ref{fig:viz_basketball}) where there are interior edges in feature space, but not pixel space. This results in the features being blurred inside of objects.

We don't make any changes to their downsampler, instead opting to just use their Attention Downsampler without modification. We then focus on two changes, one to output normalization, and the other to how upsampling guidance is computed.

\subsection{Feature Normalization}\label{sec:method:feat_norm}
FeatUp supports either leaving the features coming from the backbone as-is (e.g. no normalization), or using a LayerNorm to better condition the outputs for feature learning. For a similar motivation as PHI-S~\cite{ranzinger2024phisdistributionbalancinglabelfree}, we want to avoid using the raw features as they have varying spreads, and we'd also like to avoid using LayerNorm as it makes the features incompatible with the original feature space. Naïvely learning the raw feature space across the suite of teachers without normalization often led to convergence issues, particularly given the wide variance of activations.

\paragraph{Tile-Guided Attentional Refinement}

\begin{figure}
    \centering
    \includegraphics[width=0.3\linewidth]{resources/2x/2x_bilinear.png}
    \includegraphics[width=0.3\linewidth]{resources/2x/2x_tile.png}
    \caption{Visualization of $2\times$ upsampling using bilinear (\textit{left}) versus tiling (\textit{right}), using the DFN CLIP encoder.}
    \vspace{-5mm}
    \label{fig:ups2x}
\end{figure}

Joint-Bilateral Upsampling is able to retain object boundaries primarily in instances when there are noticeable changes in intensity in the RGB input image. This results in sharp contours, but within a region, we end up with vague and blurry feature representations. Owing to the reliance on raw pixel intensities, object contours that are less discriminative in color space often get blurred with the neighborhood. Finally, because the upsampling operation is only truly operating on the low resolution feature maps of the model, it's impossible for JBU to introduce new details into the feature map that are visible/encodable at higher input resolutions. FeatUp's implicit upsampler doesn't have this problem because it's constructing a unified view from numerous local views of the original image, enabling detailed refinements. We propose an intermediary method between JBU which leverages a single featurizer inference, and the implicit model, which relies on numerous inferences and is thus cost prohibitive\footnote{https://github.com/mhamilton723/FeatUp/issues/2}.

Inspired by the use of tiling in Vision-Language Models (VLMs) \citep{liu2024llavanext1p6,shi2024s2,dai2024nvlmopenfrontierclassmultimodal}, we develop an attentional refinement block that is able to integrate the information between a bilinear upsampled feature map, as well as a feature map composed of tiles. We show an overview of the algorithm in figures 
 \ref{fig:featsharp_arch_diagram}, \ref{fig:featsharp_module_diagram} and \ref{fig:tile_process}. The diagram shows actual results using RADIOv2.5-L, which is the most scale equivariant foundation model \citep{heinrich2024radioamplifiedimprovedbaselines}, and generally the strongest visual foundation model \citep{lu2024swissarmyknifesynergizing,drozdova2024semisupervised,guo2024videosamopenworldvideosegmentation}. Because the model has strong resolution scaling, it provides us with a good way to compare the results of the upsampling process against the feature maps of the same resolution attained by increasing the resolution of the input image. We also observe that even just at $4\times$ tiling, there are major discontinuities in the tiled feature map, which the FeatSharp module must overcome to produce a unified higher resolution image.


For the FeatSharp module, we leverage a single Attention+SwiGLU transformer block. In order to prevent the quadratic cost of global attention, we instead use local attention \cite{ramachandran2019localattn}. We concatenate the JBU upsampled buffer with the tiled feature map and feed it to the block. After the block is computed, we slice the first $C$ dimensions of the output, with $C$ being the model feature dimension, and treat that as the refined features. The slicing strategy takes advantage of the fact that a transformer block has a residual pathway, and thus a no-op from the transformer would be equivalent to returning the bilinear upsampled features. Through the attention mechanism, the model is able to consider the local neighborhood and refine its features to achieve better multi-view consistency. To this end, we train our model identically to FeatUp's multi-view consistency algorithm. We do not employ any special loss functions beyond the MSE loss on multi-view consistency, contrary to FeatUp's use of Total Variation and Conditional Random Field losses. We provide ablations wrt architecture choice in appendix \ref{sec:featsharp_arch_ablations}.

\subsection{Denoising}

Drawing inspiration from \cite{yang2024denoising}, we notice that the problem formulation has a very similar solution to FeatUp (and ours), owing to the fact that all methods are using multi-view consistency and thus learn to eliminate position-based artifacts. From their formulation:

\begin{equation}
    \textup{ViT}(x) = f(x) + g(\textup{\textbf{E}}_{pos}) + h(x, \textup{\textbf{E}}_{pos})
    \tag{\cite{yang2024denoising}, Eq 5}
    \label{eq:vit_denoise}
\end{equation}

We add a learnable $g$ buffer, such that

\begin{equation}
    \hat{f}(x) = f(x) + g
\end{equation}

with $f(x)$ being the frozen vision encoder. The learnable $g$ allows our model to learn and negate the fixed position artifacts that the encoder produces. Notably, given that we are also using the base model for the tiles, this learned buffer is applied to all of the generated tiles as well. We visualize these biases in figure \ref{fig:model-biases}. It's entirely possible for FeatSharp to remove the biases itself, but we found that having this learnable bias buffer consistently improves multi-view consistency, which we show in table \ref{tab:bias_fidelity} in the appendix.

\subsection{Complexity}
An important point about this method is that because of the tiling, it requires more evaluations of the base vision model to construct the high resolution feature map. However, due to the scaling properties of global self-attention, our proposed method always has better scaling properties than running the original model at higher resolution (assuming the model is capable of doing this in the first place):

\begin{equation}
\begin{split}
    f(x) &= c \left(1 + x^2 \right) \\
    g(x) &= c \left(x^2\right)^2 = c x^4  \\
    f(x) &\leq g(x) \quad \forall x > 1
\end{split}
\label{eq:cost_inequality}
\end{equation}

where $f(x)$ is the relative cost of computing FeatSharp with $x \in \mathbb{Z}_{+}$ upsample factor, $g(x)$ is the cost of the hi-res image based on quadratic scaling on the number of patches (and thus tiles), and $c$ being the cost of processing a single tile. We show the actual scaling cost in figure \ref{fig:vith_throughput} in the appendix.

\section{Upsampling Results}\label{sec:experiments}

We consider upsampling to be important in cases where one is given a fixed pretrained model, and the goal is to extract more information out of it, for a given image. We study our method in relation to FeatUp from a core multi-view consistency standpoint in this section, from a semantic segmentation linear probe standpoint, and also for training a new RADIO-like model with hi-res teacher targets.

\subsection{Fidelity}\label{sec:fidelity}

\paragraph{Multi-View Consistency} 
Following \cite{ranzinger2024phisdistributionbalancinglabelfree}, we use their definition of fidelity (equation 51) for multi-view consistency, where a higher fidelity value means that the upsampled-transformed-downsampled representations are closer to the raw transformed predictions from the model. 

\begin{equation}
    f(\mathbf{X},\mathbf{Y}) = \frac{\text{MSE}(\mathbf{Y}, \bm{\mu_Y})}{\text{MSE}(\mathbf{X}, \mathbf{Y})}
    % \tag{\cite{ranzinger2024phisdistributionbalancinglabelfree} Eq 51}
\end{equation}

with $\mathbf{X}$ being the warped predictions and $\mathbf{Y}$ the targets. This serves as a proxy measure for how well the upsampler is working, as arbitrarily warping and downsampling it results in representations closer to the real prediction at low resolution. We show these results in figure \ref{fig:consistency_fidelity}, where we observe that FeatSharp consistently achieves the highest fidelities, substantially so with the ``cleaner'' models such as DINOv2-L, RADIOv2.5-L, and SAM-H.

\begin{figure}[t]
    \centering
    \includegraphics[width=\linewidth]{resources/multiview_consistency_plot.pdf}
    \vspace{-7mm}
    \caption{Fidelity plot for different models and upsampling methods. Higher values are better. We don't show SAM 4x because of OOM issues training these models.}
    \vspace{-10mm}
    \label{fig:consistency_fidelity}
\end{figure}

\subsection{Qualitative}

We run this upsampling method on seven different foundation models coming from diverse domains such as supervised (ViT, \citep{dosovitskiy2021image}), contrastive (DFN~CLIP~\cite{fang2023data}, SigLIP~\cite{zhai2023sigmoid}), Self-supervised (DINOv2-L-reg~\cite{darcet2023vision}), Segmentation (SAM~\cite{kirillov2023sam}), VLM (PaliGemma~\cite{beyer2024paligemmaversatile3bvlm}), and Agglomerative (RADIOv2.5-L~\cite{ranzinger2024phisdistributionbalancinglabelfree}). Results are in figure \ref{fig:viz_basketball}. The original feature maps run the spectrum from extremely noisy (SigLIP) to very clean (RADIOv2.5-L, SAM), which allows us to demonstrate the effectiveness of the approach on a diverse set of models. Taking SAM for an example, the way in which is has thick edge outlines cannot be reproduced in the shape interior by FeatUp, primarily because the bilateral upsampler is operating on the raw pixels, where the interior edge doesn't exist in the real image. For all of the featurizers, FeatSharp is able to achieve more legible representations, particularly it's more able to closely match the real hi-res features in the second column.

\begin{figure}[!t]
    \centering
    \includegraphics[width=\linewidth]{resources/basketball_viz_results.pdf}
    \vspace{-5mm}
    \caption{PCA visualizations of features from a basketball scene. \textbf{Column~1}: Raw features produced by the model at normal resolution (e.g. 14x downsample for DFN CLIP, SigLIP, PaliGemma, and DINOv2, 16x downsample for SAM and RADIOv2.5-L. \textbf{Column~2}: Raw features at the 4x upsample resolution (we interpolate the position embeddings for those models that don't natively support resolution changes). \textbf{Column~3}: FeatUp 4x upsampling (\textit{prior work}). \textbf{Column~4}: FeatSharp 4x upsampling. 
    \\
    \textit{NOTE: ``Real 4x'' technically only makes sense for models with strong scale equivariance, such as DINOv2, RADIO, and SAM.}}
    \label{fig:viz_basketball}
\end{figure}

\subsection{Semantic Segmentation}\label{sec:experiments:semseg}

Semantic segmentation has the potential to benefit from increased resolution, as it allows for label contours to be more precise, and potentially for regions to be recovered that are otherwise too small. The first setting we evaluate on is we train both FeatUp and FeatSharp at $2\times$ and $4\times$ upsampling, both using PHI-S. We resize the input size to be the featurizer's native input resolution, which we call ``$1\times$ Input Size'', and we also consider ``$2\times$ Input Size'', where we double the input size, and feed directly to the featurizer in the case of ``Baseline'', or we allow the upsampler to have higher resolution guidance while keeping the featurizer input fixed at $1\times$ resolution. We show these results in figure \ref{fig:semseg}. In most cases, both upsampling algorithms produce higher quality segmentations than the baseline, however, FeatUp is worse than the ``Baseline $2\times$'' method for RADIOv2.5-L and ViT. In all cases, FeatSharp is superior to both FeatUp and also the baselines by significant margins. We even improve upon SOTA RADIO's published result of 51.47 with a $2\times$ upsampling combined with $2\times$ input size, producing a model that attains 53.13 mIoU, a $+1.66$ mIoU improvement. RADIO itself improves with the $2\times$ input size, but not to the same degree as with FeatSharp, with FeatSharp being $57\%$ faster. We also notice that $3\times$ upsampling is generally worse than $2\times$ or $4\times$ for both upsamplers.

\begin{figure*}
    \centering
    \includegraphics[width=\linewidth]{resources/semseg_plots.pdf}
    \vspace{-7mm}
    \caption{ADE20k \cite{zhou2017ade20k} Semantic segmentation results for different featurizers and upsamplers. We also vary the input size between Inpt-$1\times$ and Inpt-$2\times$ the featurizer's native resolution. $1\times$ Resolutions: DFN CLIP = 378px, DINOv2-L = 448px, PaliGemma = 448px, RADIOv2.5-L = 512px, SigLIP = 378px, ViT = 224px. The dark line represents the mean of 5 runs, with shaded areas showing the standard deviation. Because the x-axis is the upsample amount, the baselines should technically be single points on a ``1x'' x-coord, but we instead draw a line to make it easier to see the change in the upsamplers across the upsample amounts. E.g. for ``RADIO, Baseline Inpt-2x'', we can see that it's better than FeatUp $2\times$ upsampling, but worse than FeatSharp $2\times$ upsampling.}
    \label{fig:semseg}
\end{figure*}

\subsection{Agglomerative Models}\label{sec:experiments:agglom}

We build upon RADIOv2.5-L~\cite{heinrich2024radioamplifiedimprovedbaselines} as it learns directly from the spatial features of teacher models. In particular, we consider whether we can improve upon their multi-resolution training strategy by using FeatSharp to convert the low-res teachers into hi-res teachers. We convert the teachers in the bottom left quadrant ``Low Res Teacher / High Res Student'' in their Figure 6 into ``High Res Teacher / High Res Student'' by using the upsampler. We consider a few different comparative baselines in order to prove the efficacy of the technique. First, our baseline matches that of \cite{heinrich2024radioamplifiedimprovedbaselines}, which is to downsample the student to match the teacher. Then, we consider two techniques which are popular in the literature: Tiling~\cite{liu2024llavanext1p6}, and S2~\cite{shi2024s2}. Both of these rely on tiling, but S2 also considers the low-res version. Because we need the feature space to remain the same as the low-res partition of RADIO, we opt to upsample the low-res feature map, and then interpolate the upsampled-low-res against the tiled version, using $y = \beta \cdot \text{low-res} + (1 - \beta) \cdot \text{high-res}$. We set $\beta = 0.5$ as it's unclear what an optimal balance might be, and it's expensive to search this space. As a final baseline, we include FeatUp's JBU variant, as the implicit version would be prohibitive to use within a training loop\footnote{\hyperlink{https://github.com/mhamilton723/FeatUp/issues/2\#issuecomment-2005688054}{1 minute per image}}.

\begin{figure*}[t]
    \centering
    \includegraphics[width=\linewidth]{resources/radio_inference/dfn_clip_main_body.jpg}
    \vspace{-5mm}
    \caption{Visualization of our trained RADIO's DFN CLIP adaptor when the high-res partition used various teacher upsample schemes.}
    \label{fig:radio_dfn_clip_adaptor}
\end{figure*}

In figure \ref{fig:radio_dfn_clip_adaptor} we qualitatively visualize the DFN CLIP adaptor features learned by the radio model. We can see that each upsampling method has a substantial impact on the resulting feature maps. The baseline method exhibits strong high-frequency artifacting starting at 768px. This is likely when RADIO ``mode switches'' to high-resolution, which is something that \cite{heinrich2024radioamplifiedimprovedbaselines} addressed for the backbone features, but apparently still exhibit for the adaptor features. We observe that Tiling and S2 exhibit not only high-frequency noise patterns like the baseline, but also obvious grid patterns, arising from the use of tiles. FeatUp appears to mode switch starting at 768px into a smooth, but low-detail feature space. FeatSharp remains smooth and highly detailed as resolution increases, however, visually, it's still possible that the features are mode switching. We further study this tiling issue in appendix \ref{sec:apdx:overtiling}.

Along with improvements in the adaptors, we also study the effects on the backbone features for the RADIO model. Following \cite{Maninis2019AttentiveSO,lu2024swissarmyknifesynergizing} we report the MTL Gain ($\Delta_m$) across a suite of tasks. Unlike the prior works, instead of leveraging a single-task baseline, we instead opt to report the change relative to the baseline training run.

\begin{align}
    \delta_m &= 100 \cdot (-1)^{l_t} \frac{M_t - M_{B,t}}{M_{B,t}} \\
    \Delta_m &= \frac{1}{T} \sum_{t=1}^{T} \delta_m
\end{align}

where $M_t$ is the metric for the current model on task $t$, and $M_{B,t}$ is the metric for the baseline model. $l_t$ is 0 when higher task values are better, and 1 when lower is better.

\begin{table*}[!h]
    \centering
    \resizebox{0.8\linewidth}{!}{
    \begin{tabular}{c|cccccccc|c}
        Upsampler & Classification & Dense      & Probe 3D  & Retrieval & Pascal Context & NYUDv2      & VILA      & $\Delta_m \%$ \\
        \hline
        RADIO-AMP-L & -0.47          & -0.09      & -1.05     & -0.45     & \bf{0.62}      & -2.26       & \bf{2.24} & -0.21     \\
        \hline
        Baseline    & \ul{0.00}      & 0.00       & 0.00      & 0.00      & 0.00           & 0.00        & 0.00      & 0.00      \\
        Tile        & -0.03          & \bf{0.30}  & -0.08     & -0.23     & -0.02          & \bf{1.33}   & -3.17     & -0.27     \\
        S2          & -0.05          & 0.15       & -0.03     & -0.44     & 0.13           & \bf{1.33}   & -0.89     & 0.03      \\
        FeatUp      & -0.07          & 0.14       & 0.23      & -0.07     & 0.14           & 0.32        & -1.58     & -0.13     \\
        FeatSharp   & \bf{0.06}      & \ul{0.16}  & \bf{0.83} & \bf{0.13} & \ul{0.17}      & \ul{0.93}   & \ul{0.43} & \bf{0.39}
    \end{tabular}
    }
    \caption{Relative changes (in \%) on a suite of aggregated benchmarks, with each column reporting $\delta_m \%$ and averaged into $\Delta_m \%$. All relative changes are against our baseline run. Raw metrics are in section \ref{sec:apdx:raw_radio_results}. \textit{NOTE: The upsamplers are only applied to the DFN CLIP and SigLIP teachers during RADIO training. Metrics are collected from trained RADIO without upsampling methods.}
    }
    \label{tab:radio_mtl_task_suite}
\end{table*}

We show the MTL Gain results in table \ref{tab:radio_mtl_task_suite}. Given that the results are relative to our baseline run, S2 and FeatSharp are the only two methods to improve, however, only FeatSharp was categorically better, leading to a +0.39\% improvement across all benchmarks on average. We also see that our version of RADIO with FeatSharp teachers generally does better than RADIO-AMP-L \cite{heinrich2024radioamplifiedimprovedbaselines}, which is the current state of the art, where we improve over it on everything except for the VILA task. We report all of the raw benchmark scores in tables \ref{tab:apdx:radio_cls_retrieval_metrics}, \ref{tab:apdx:radio_dense_probe3d_metrics}, \ref{tab:apdx:radio_pascal_nyud_metrics} and \ref{tab:apdx:radio_vila_metrics} in the appendix.

\section{Conclusion}

We have presented a novel feature upsampling technique named FeatSharp that achieves higher multi-view fidelity than the current best method, FeatUp. We achieve this by joining FeatUp's JBU upsampler with a mosaic of tiles, and then process with a single local attention block. We demonstrate its effectiveness on ADE20K semantic segmentation linear probing, where the use of FeatSharp improves over both baseline and FeatUp, even with the strongest segmenter, RADIO, which itself can handle hi-res inputs robustly. We then demonstrate the effectiveness of FeatSharp by employing it directly within RADIO training, enabling low-res-only teacher models to have hi-res distillation targets. In doing this, our FeatSharp-RADIO largely improves on dense vision task benchmarks, and yields an overall improvement over our reproduction baseline, which itself improves over RADIO-AMP-L, the current state of the art. We believe this work can be useful both as a drop-in extension of existing vision systems which rely on pretrained vision encoders, as well as the newly trained FeatSharp-RADIO model with hi-res teachers, which can emulate these same models. Owing to FeatSharp-RADIO's emulation abilities, it allows us to estimate these teacher models at arbitrary resolutions, not just integer upsampling factors as restricted in FeatSharp/FeatUp's core training algorithm. Further, combining RADIO's ``ViTDet'' \citep{li2022vitdet} mode with these hi-res teacher emulations allows us to achieve hi-res feature maps without fully paying the quadratic penalty in number of tokens as required by standard ViTs.

% \section{Impact Statement}
% This paper presents work whose goal is to advance the field of Computer Vision. By virtue of being a lightweight addition to existing vision models, the work aims to open up doors for higher-resolution perception tasks (e.g. segmentation, depth perception, etc.) while retaining the original model representations. As such, the ethical impacts are constrained to those of the model being upsampled. The FeatSharp training code, upsampler weights, and trained RADIO model using FeatSharp will be released to the community.

\bibliography{featuppp}
\bibliographystyle{icml2025}


%%%%%%%%%%%%%%%%%%%%%%%%%%%%%%%%%%%%%%%%%%%%%%%%%%%%%%%%%%%%%%%%%%%%%%%%%%%%%%%
%%%%%%%%%%%%%%%%%%%%%%%%%%%%%%%%%%%%%%%%%%%%%%%%%%%%%%%%%%%%%%%%%%%%%%%%%%%%%%%
% APPENDIX
%%%%%%%%%%%%%%%%%%%%%%%%%%%%%%%%%%%%%%%%%%%%%%%%%%%%%%%%%%%%%%%%%%%%%%%%%%%%%%%
%%%%%%%%%%%%%%%%%%%%%%%%%%%%%%%%%%%%%%%%%%%%%%%%%%%%%%%%%%%%%%%%%%%%%%%%%%%%%%%
\newpage

\clearpage
\setcounter{page}{1}
%\maketitlesupplementary

\appendix

\begin{center}
    {\Large{\textbf{Appendix}}}
\end{center}

%\section{Rationale}
%\label{sec:rationale}
% 
%Having the supplementary compiled together with the main paper means that:
% 
%\begin{itemize}
%\item The supplementary can back-reference sections of the main paper, for example, we can refer to \cref{sec:intro};
%\item The main paper can forward reference sub-sections within the supplementary explicitly (e.g. referring to a particular experiment); 
%\item When submitted to arXiv, the supplementary will already included at the end of the paper.
%\end{itemize}
% 
%To split the supplementary pages from the main paper, you can use \href{https://support.apple.com/en-ca/guide/preview/prvw11793/mac#:~:text=Delete%20a%20page%20from%20a,or%20choose%20Edit%20%3E%20Delete).}{Preview (on macOS)}, \href{https://www.adobe.com/acrobat/how-to/delete-pages-from-pdf.html#:~:text=Choose%20%E2%80%9CTools%E2%80%9D%20%3E%20%E2%80%9COrganize,or%20pages%20from%20the%20file.}{Adobe Acrobat} (on all OSs), as well as \href{https://superuser.com/questions/517986/is-it-possible-to-delete-some-pages-of-a-pdf-document}{command line tools}.




We organize the Appendix as follows: 
\begin{itemize}
    \item Appendix~\ref{sec:implementation} describes the implementation details. It describes the DNN architectures (VGG, ResNet, and ViT), feature extraction for linear probing, training, and evaluation details of both pre-training and linear probing in various experiments.
    
    \item Appendix~\ref{sec:datasets} provides details on the datasets used in this paper. In total, we use 9 datasets.

    \item Appendix~\ref{sec:nc_metrics} describes four neural collapse metrics ($\mathcal{NC}1 - \mathcal{NC}4$) used in this paper.

    \item Appendix~\ref{sec:mse_ce_supp} presents a comprehensive comparison between MSE and CE.

    \item Appendix~\ref{sec:details_proposition} contains proof on the implication of NC on entropy.

    \item Appendix~\ref{sec:comprehensive_results} provides a comprehensive comparison between the encoder and projector across different architectures. %It includes comprehensive results on 8 OOD datasets for different DNNs including VGG17, ResNet18, ResNet34, ViT-T, and ViT-S.

    \item Appendix~\ref{sec:analysis_entropy_reg} provides detailed analyses on entropy regularization and neural collapse.

    \item Additional experiments and analyses are summarized in Appendix~\ref{sec:additional_exp_supp}. The mechanisms of controlling NC have been examined.
    
    \item Appendix~\ref{sec:imagenet_100_classes} includes the list of 100 classes in the imageNet-100 dataset. %and confirms there is no overlap between ID and OOD datasets.
\end{itemize}




\section{Implementation Details}
\label{sec:implementation}

In this paper, we use several acronyms such as
\textbf{NC} : Neural Collapse, 
\textbf{ETF} : Equiangular Tight Frame,
\textbf{ID} : In-Distribution, 
\textbf{OOD} : Out-of-Distribution,
\textbf{LR} : Learning Rate, 
\textbf{WD} : Weight Decay,
\textbf{GAP} : Global Average Pooling,
\textbf{GN} : Group Normalization, 
\textbf{BN} : Batch Normalization, 
\textbf{WS} : Weight Standardization,  
\textbf{CE} : Cross Entropy, 
\textbf{MSE} : Mean Squared Error,
\textbf{FPR} : False Positive Rate.

We use the terms OOD generalization and OOD transfer interchangeably.


%We implemented our code in Python using PyTorch.


\subsection{Architectures}
\label{sec:arch_details}

\textbf{VGG.}
We modified the VGG-19 architecture to create our VGG-17 encoder. Additionally, we removed two fully connected (FC) layers before the final classifier head. And, we added an adaptive average pooling layer (nn.AdaptiveAvgPool2d), which allows the network to accept any input size while keeping the output dimensions the same. After VGG-17 encoder, we attached a projector consisting of two MLP layers ($512 \rightarrow 2048 \rightarrow 512$) and finally added a classifier head. 
We use ReLU activation between projector layers.
We replace BN with GN+WS in all layers. For GN, we use 32 groups in all layers.

\noindent
\textbf{ResNet.}
We used the entire ResNet-18 or ResNet-34 as the encoder and attached a projector ($512 \rightarrow 2048 \rightarrow 512$) similar to the VGG networks mentioned above. 
We replace BN with GN+WS in all layers. For GN, we use 32 groups in all layers.

\noindent
\textbf{ViT.}
We consider ViT-Tiny/Small (5.73M/21.85M parameters) as the encoder for our experiments. %We use embedding dimension of 192, depth of 18 (i.e., 18 ViT blocks) and 3 heads. 
The projector comprising two MLP layers configured as fixed ETF Simplex and added after the encoder.
%($192 \rightarrow 768 \rightarrow 192$) is added after the encoder.
Following~\cite{beyer2022better}, we omit the learnable position embeddings and instead use the fixed 2D sin-cos position embeddings. Other details adhere to the original ViT paper~\cite{dosovitskiy2020image}. 
\begin{enumerate} %[noitemsep, nolistsep, leftmargin=*]
    \item \textbf{ViT-Tiny Configuration:} patch size=16, embedding dimension=192, \# heads=3, depth=12. Projector has
    output dimension=192 and hidden dimension=768,
    ($192 \rightarrow 768 \rightarrow 192$). We use ReLU activation between projector layers.
    The number of parameters in ViT-Tiny $+$ projector is 6.02M.
    \item \textbf{ViT-Small Configuration:} patch size=16, embedding dimension=384, \# heads=6, depth=12. Projector has
    output dimension=384 and hidden dimension=1536,
    ($384 \rightarrow 1536 \rightarrow 384$). We use ReLU activation between projector layers.
    The number of parameters in ViT-Small $+$ projector is 23.03M.
\end{enumerate}


\subsection{Feature Extraction For Linear Probing}
\label{sec:feat_extract_details}
In experiments with CNNs, at each layer $l$, for each sample, we extract features of dimension $H_{l}\times W_{l}\times C_{l}$, where $H_{l}$, $W_{l}$, and $C_{l}$ denote the height, width and channel dimensions respectively. Next, following~\cite{sarfi2023simulated}, we apply $2\times 2$ adaptive average pooling on each spatial tensor ($H_{l}\times W_{l}$). After average pooling, features of dimension $2\times 2 \times C_{l}$ are flattened and converted into a vector of dimension $4C_{l}$. Finally, a linear probe is trained on the flattened vectors. 
In experiments with ViTs, 
following~\cite{raghu2021vision}, we apply global average-pooling (GAP) to aggregate image tokens excluding the class token and train a linear probe on top of GAP tokens.
We report the best error ($\%$) on the test dataset for linear probing at each layer.


\subsection{VGG Experiments}

\textbf{VGG ID Training:} For training VGG on ImageNet-100, we employ the AdamW optimizer with a LR of $6\times10^{-3}$ and WD of $5\times10^{-2}$ for batch size 512. The model is trained for 100 epochs using the Cosine Annealing LR scheduler with a linear warmup of 5 epochs. 
In all experiments, we use CE and entropy regularization ($\alpha=0.05$) losses. 
However, in some particular experiments comparing CE and MSE, we use MSE loss ($\kappa$=15, M=60) and entropy regularization loss ($\alpha=0.05$). 

\noindent
\textbf{VGG Linear Probing:} We use the AdamW optimizer with a flat LR of $1\times 10^{-3}$ and WD of $0$ for batch size 128. The linear probes are trained for 30 epochs. We use label smoothing of 0.1 with the cross-entropy loss. 


\subsection{ResNet Experiments}

\textbf{ResNet ID Training:} 
For training ResNet-18/34, we employ the AdamW optimizer with an LR of 0.01 and a WD of 0.05 for batch size 512. The model is trained for 100 epochs using the Cosine Annealing LR scheduler with a linear warmup of 5 epochs. 
We use CE and entropy regularization ($\alpha=0.05$) losses. 
%We use MSE ($\kappa$=15, M=60) and regularization losses. 

\noindent
\textbf{ResNet Linear Probing:} In the linear probing experiment, we use the AdamW optimizer with an LR of $1\times 10^{-3}$ and WD of $0$ for batch size 128. The linear probes are trained for 30 epochs. We use label smoothing of 0.1 with cross-entropy loss.


\subsection{ViT Experiments}

\textbf{ViT ID Training:} 
For training ViT-Tiny, we employ the AdamW optimizer with LR of $8\times10^{-4}$ and WD of $5\times10^{-2}$ for batch size 256. The LR is scaled for $n$ GPUs according to: $LR \times n \times \frac{batch size}{512}$. We use an LR of $4\times10^{-4}$ for ViT-Small when the batch size is 256.
We use the Cosine Annealing LR scheduler with warm-up (5 epochs). 
We train the ViT-Tiny/Small for 100 epochs using CE and entropy regularization ($\alpha=0.05$) losses.
%We train the ViT-Tiny/Small for 100 epochs using CE or MSE ($\kappa$=5, M=10 for MSE) and regularization losses.
Following~\cite{raghu2021vision, beyer2022better}, we omit class token and instead use GAP token by global average-pooling image tokens and feed GAP embeddings to the projector. 


\noindent
\textbf{ViT Linear Probing:} We use the AdamW optimizer with LR of $0.01$ and WD of $1\times 10^{-4}$ for batch size 512. The linear probes are trained for 30 epochs. We use label smoothing of 0.1 with cross-entropy loss.

\noindent
\textbf{Augmentation.}
We use random resized crop and random flip augmentations and $224 \times 224$ images as inputs to the DNNs.

In experiments with CE loss, we use label smoothing of $0.1$.


\subsection{Evaluation Criteria}

\textbf{\textit{FPR95.}}
The OOD detection performance is evaluated by the FPR (False Positive Rate) metric. In particular, we use FPR95 (FPR at 95\% True Positive Rate) that evaluates OOD detection performance by measuring the fraction of OOD samples misclassified as ID where threshold, $\lambda$ is chosen when the true positive rate is 95\%. 
Both OOD detection and OOD generalization tasks are evaluated on the \emph{same} OOD test set.



%$\mathbf{\Delta_{E \rightarrow P}}$.
\textbf{\textit{Percentage Change.}}
To capture percentage increase or decrease when switching from the encoder ($E$) to the projector ($P$), we use 
\[
\Delta_{E \rightarrow P} = \frac{(P - E)} {|E|} \times 100.
\]


\textbf{\textit{Normalization for different OOD datasets.}}
In our correlation analysis between NC and OOD detection/generalization (Fig.~\ref{fig:vis_abstract} and~\ref{fig:nc_resnet}), we use min-max normalization for layer-wise OOD detection errors and OOD generalization errors which enables comparison using different OOD datasets. For a given OOD dataset and a DNN consisting of total $L$ layers, let the OOD detection/ generalization error for a layer $l$ be $E_l$. For $L$ layers we have error vector $\mathbf{E} = [E_1, E_2, \cdots E_L]$ which is then normalized by
\[
\mathbf{E}_N = \frac{\mathbf{E} - \mathrm{min}(\mathbf{E})} {\mathrm{max}(\mathbf{E}) - \mathrm{min}(\mathbf{E})}.
\]


\textbf{\textit{Effective Rank.}}
We use RankMe~\cite{garrido2023rankme} to measure the effective rank of the embeddings.



\section{Datasets}
\label{sec:datasets}

\textbf{ImageNet-100.} 
ImageNet-100~\cite{tian2020contrastive} is a subset of ImageNet-1K~\cite{deng2009imagenet} and contains 100 ImageNet classes. It consists of 126689 training images ($224\times 224$) and 5000 test images.
The object categories present in ImageNet-100 are listed in Appendix~\ref{sec:imagenet_100_classes}.

\textbf{CIFAR-100.} 
CIFAR-100~\cite{krizhevsky2014cifar} is a dataset widely used in computer vision. It contains $60,000$ RGB images and $100$ classes, each containing $600$ images. The
dataset is split into $50,000$ training samples and $10,000$ test samples. The images in CIFAR-100 have a
resolution of $32\times 32$ pixels. 
Unlike CIFAR-10, CIFAR-100 has a higher level of granularity, with
more fine-grained classes such as flowers, insects, household items, and a variety of animals and vehicles.
%CIFAR-100~\cite{krizhevsky2014cifar} dataset is similar with CIFAR-10 but with 100 classes. And each class has 600 images. %The out-of-distribution accuracy will be computed with randomly selected 10 classes from CIFAR-100, which is the same protocol used in~\cite{masarczyk2023tunnel}. Note that the classes in CIFAR-100 are mutually exclusive with those in CIFAR-10. 
For linear probing, all samples from both the training and validation datasets were used.


\textbf{NINCO (No ImageNet Class Objects).} NINCO ~\cite{bitterwolf2023ninco} is a dataset with 64 classes. The dataset is curated to eliminate semantic overlap with ImageNet-1K dataset and is used to evaluate the OOD performance of the models pre-trained on imagenet-1K. The NINCO dataset has 5878 samples, and we split it into 4702 samples for training and 1176 samples for evaluation. We do not have a fixed number of samples per class for training and evaluation datasets.

\textbf{ImageNet-Rendition (ImageNet-R).} ImageNet-R incorporates distribution shifts using different artistic renditions of object classes from the original ImageNet dataset~\citep{hendrycks2021many}.
We use a variant of ImageNet-R dataset from~\cite{wang2022dualprompt}.
ImageNet-R is a challenging benchmark for continual learning, transfer learning, and OOD detection. It consists of classes with different styles and intra-class diversity and thereby poses significant distribution shifts for ImageNet-1K pre-trained models~\citep{wang2022dualprompt}.
It contains 200 classes, 24000 training images, and 6000 test images.

\textbf{CUB-200.} CUB-200 is composed of 200 different bird species~\cite{wah2011caltech}. The CUB-200 dataset comprises a total of 11,788 images, with 5,994 images allocated for training and 5,794 images for testing.

\textbf{Aircrafts-100.} Aircrafts or FGVCAircrafts dataset~\cite{maji2013fine} consists of 100 different aircraft categories and 10000 high-resolution images with 100 images per category. The training and test sets contain 6667 and 3333 images respectively.

\textbf{Oxford Pets-37.} The Oxford Pets dataset includes a total of 37 various pet categories, with an approximately equal number of images for dogs and cats, totaling around 200 images for each category~\cite{parkhi2012cats}.

\textbf{Flowers-102.} The Flowers-102 dataset contains 102 flower categories that can be easily found in the UK. Each category of the dataset contains 40 to 258 images.~\cite{nilsback2008automated}

\textbf{STL-10.} STL-10 has 10 classes with 500 training images and 800 test images per class~\cite{coates2011analysis}.


For all datasets, images are resized to $224 \times 224$ to train and evaluate DNNs.

%%%%%%%%%%%%%%%%%%%%%%%%%%%%%%%%%%%%%%%%%%


\section{Neural Collapse Metrics}
\label{sec:nc_metrics}

Neural Collapse (NC) describes a structured organization of representations in DNNs~\cite{papyan2020prevalence, kothapalli2023neural, zhu2021geometric, rangamani2023feature}.
%\begin{tcolorbox}[boxsep=1pt,left=2pt,right=2pt,top=0pt,bottom=0pt]
The following four criteria characterize Neural Collapse:
\begin{enumerate}
    \item \textbf{Feature Collapse} ($\mathcal{NC}1$): Features within each class concentrate around a single mean, with almost no variability within classes.
    \item \textbf{Simplex ETF Structure} ($\mathcal{NC}2$): Class means, when centered at the global mean, are linearly separable, maximally distant, and form a symmetrical structure on a hypersphere known as a Simplex Equiangular Tight Frame (Simplex ETF).
    \item \textbf{Self-Duality} ($\mathcal{NC}3$): The last-layer classifiers align closely with their corresponding class means, forming a self-dual configuration.
    \item \textbf{Nearest Class Mean Decision} ($\mathcal{NC}4$): The classifier operates similarly to the nearest class-center (NCC) decision rule, assigning classes based on proximity to the class means. 
\end{enumerate}
%\end{tcolorbox}


Here, we describe each NC metric used in our results. Let \( \mu_G \) denote the global mean and \( \mu_c \) the \( c \)-th class mean of the features, \( \{z_{c,i}\} \) at layer \( l \), defined as follows:
\[
\mu_G = \frac{1}{nC} \sum_{c=1}^C \sum_{i=1}^n z_{c,i}, \quad \mu_c = \frac{1}{n} \sum_{i=1}^n z_{c,i} \quad (1 \leq c \leq C).
\]
We drop the layer index \( l \) from notation for simplicity.

\noindent
\textbf{Within-Class Variability Collapse ($\mathcal{NC}1$):}  
It measures the relative size of the within-class covariance \( \Sigma_W \) with respect to the between-class covariance \( \Sigma_B \) of the DNN features:
\[
\Sigma_W = \frac{1}{nC} \sum_{c=1}^C \sum_{i=1}^n \left( z_{c,i} - \mu_c \right) \left( z_{c,i} - \mu_c \right)^\top \in \mathbb{R}^{d \times d},
\]
\[
\Sigma_B = \frac{1}{C} \sum_{c=1}^C \left( \mu_c - \mu_G \right) \left( \mu_c - \mu_G \right)^\top \in \mathbb{R}^{d \times d}.
\]

The $\mathcal{NC}1$ metric is defined as:

\[
\mathcal{NC}1 = \frac{1}{C} \operatorname{trace} \left( \Sigma_W \Sigma_B^{\dagger} \right),
\]
where \( \Sigma_B^{\dagger} \) is the pseudo-inverse of \( \Sigma_B \). Note that $\mathcal{NC}1$ is the most dominant indicator of neural collapse.

\noindent
\textbf{Convergence to Simplex ETF ($\mathcal{NC}2$):}  
It quantifies the \( \ell_2 \) distance between the normalized simplex ETF and the normalized \( WW^\top \), as follows:
\[
\mathcal{NC}2 := \left\| \frac{WW^\top}{\| WW^\top \|_F} - \frac{1}{\sqrt{C-1}} \left( I_C - \frac{1}{C} \mathbf{1}_C \mathbf{1}_C^\top \right) \right\|_F,
\]
where \( W \in \mathbb{R}^{C \times d} \) is the weight matrix of the learned classifier.

\noindent
\textbf{Convergence to Self-Duality ($\mathcal{NC}3$):}  
It measures the \( \ell_2 \) distance between the normalized simplex ETF and the normalized \( WZ \):
\[
\mathcal{NC}3 := \left\| \frac{WZ}{\| WZ \|_F} - \frac{1}{\sqrt{C-1}} \left( I_C - \frac{1}{C} \mathbf{1}_C \mathbf{1}_C^\top \right) \right\|_F,
\]
where \( Z = \left[ z_1 - \mu_G \; \cdots \; z_C - \mu_G \right] \in \mathbb{R}^{d \times C} \) is the centered class-mean matrix.


\textbf{Simplification to NCC ($\mathcal{NC}4$):} It measures the collapse of bias \( b \):
\[
\mathcal{NC}4 := \left\| b + W \mu_G   \right\|_2.
\]






%%%%%%%%%%%%%%%%%%%%%%%%%%%%%%%%%%%%%%%%%%%


\section{Mean Squared Error vs. Cross-Entropy}
\label{sec:mse_ce_supp}

Prior work~\cite{kornblith2021better} finds that MSE rivals CE in ID classification task but underperforms CE in OOD transfer. However, the comparison between CE and MSE in OOD detection task remains unexplored.
In this work, we find that CE significantly outperforms MSE in both OOD transfer and OOD detection tasks.
As shown in Table~\ref{tab:mse_ce_comp}, MSE underperforms CE by 6.74\% (absolute) in OOD detection and by 17.71\% (absolute) in OOD generalization. Our OOD generalization results are consistent with~\citet{kornblith2021better}.
CE also obtains lower ID error than MSE, thereby showing good overall performance.

In terms of inducing neural collapse, both MSE and CE are effective and achieve lower NC values (i.e., stronger NC). However, our results suggest that CE does a better job than MSE in enhancing NC without sacrificing OOD transfer. We find MSE to be sensitive to the hyperparameters.
The comparison on all OOD datasets is shown in Table~\ref{tab:main_results}.





\begin{table}[t]
\centering
  \caption{\textbf{Comparison between MSE and CE.} VGG17 networks are trained on \textbf{ImageNet-100} dataset (ID) and evaluated on 8 OOD datasets. For OOD generalization we report $\boldsymbol{\mathcal{E}}_{\text{GEN}}$ (\%) whereas for OOD detection we report $\boldsymbol{\mathcal{E}}_{\text{DET}}$ (\%), both are averaged over 8 OOD datasets. %$\downarrow$ indicates smaller values are better. 
  \textbf{A lower $\mathcal{NC}$ indicates stronger neural collapse.} $+\Delta_{E \rightarrow P}$ and $-\Delta_{E \rightarrow P}$ indicate \% increase and \% decrease respectively, when changing from the encoder ($E$) to projector ($P$). %\textcolor{brown}{CE outperforms MSE in both OOD transfer and OOD detection evaluations.}
  }
  \label{tab:mse_ce_comp}
  \centering
  \resizebox{\linewidth}{!}{
     \begin{tabular}{c|c|cccc|c|c}
     \hline %\hline
     \multicolumn{1}{c|}{\textbf{Method}} &
     \multicolumn{1}{c|}{$\boldsymbol{\mathcal{E}}_{\text{ID}}$} &
     \multicolumn{4}{c|}{\textbf{Neural Collapse}} &
     \multicolumn{1}{c|}{$\boldsymbol{\mathcal{E}}_{\text{GEN}}$} &
     \multicolumn{1}{c}{$\boldsymbol{\mathcal{E}}_{\text{DET}}$} \\
     & $\downarrow$ & $\mathcal{NC}1$ & $\mathcal{NC}2$ & $\mathcal{NC}3$ & $\mathcal{NC}4$ & Avg. $\downarrow$ & Avg. $\downarrow$ \\
     %\hline
     \toprule
     \rowcolor[gray]{0.9}
     \textbf{CE Loss} \\
     Projector & \textbf{12.62} & 0.393 & 0.490 & 0.468 & 0.316 & 66.36 & \textbf{65.10} \\
     Encoder & 15.52 & 2.175 & 0.603 & 0.616 & 5.364 & \textbf{41.85} & 87.62 \\
     \rowcolor{yellow!50}
     $\Delta_{E \rightarrow P}$ & -18.69 & -81.93 & -18.74 & -24.03 & -94.11 & +58.57 & -25.70 \\
    %\hline \hline
    \toprule
    \rowcolor[gray]{0.9}
    \textbf{MSE Loss} \\
    Projector & \textbf{14.04} & 0.469 & 0.743 & 0.279 & 0.382 & 70.87 & \textbf{71.84} \\
    %\hline
    Encoder & 14.74 & 2.267 & 0.843 & 0.673 & 10.773 & \textbf{59.56} & 88.88 \\
    %\hline
    \rowcolor{yellow!50}
    $\Delta_{E \rightarrow P}$ & -4.75 & -79.31 & -11.86 & -58.54 & -96.45 & +18.99 & -19.17 \\
    %\hline %\hline
    \bottomrule
    \end{tabular}}
\end{table}



%\begin{table*}[t]
\centering
  \caption{\textbf{ETF Fixed Projector Vs. Plastic Projector.} The VGG17 models are trained on \textbf{ImageNet-100} dataset (ID) and evaluated on 8 OOD datasets. The same color highlights the rows to compare.
  For OOD transfer we report $\boldsymbol{\mathcal{E}}_{\text{GEN}}$ (\%) whereas for OOD detection we report $\boldsymbol{\mathcal{E}}_{\text{DET}}$ (\%). %\textcolor{brown}{Fixed ETF projector shows higher transfer error (2.47\% absolute) than plastic projector but outperforms plastic projector in ID error (2.48\% absolute) and OOD detection error (8.9\% absolute).}
  } 
  \label{tab:plastic_proj}
  \centering
  \resizebox{\linewidth}{!}{
     \begin{tabular}{cc|cccc|ccccccccc}
     \hline %\hline
     \multicolumn{1}{c}{\textbf{Projector}} &
     \multicolumn{1}{c|}{$\boldsymbol{\mathcal{E}}_{\text{ID}} \downarrow$} &
     \multicolumn{4}{c|}{\textbf{Neural Collapse}} &
     \multicolumn{9}{c}{\textbf{OOD Datasets}} \\
    & IN & $\mathcal{NC}1$ & $\mathcal{NC}2$ & $\mathcal{NC}3$ & $\mathcal{NC}4$ & IN-R & CIFAR & Flowers & NINCO & CUB & Aircrafts & Pets & STL & Avg. \\
    & 100 &  &  &  &  & 200 & 100 & 102 & 64 & 200 & 100 & 37 & 10 & \\    
    \hline
    %% CE Loss
    \textbf{\textcolor{orange}{Transfer Error $\downarrow$}} \\
    %\rowcolor[gray]{0.9}
    \textcolor{blue}{\textbf{Plastic}} \\
    Projector & 15.10 & 0.498 & 0.515 & 0.428 & 1.422 & 87.52 & 64.83 & 79.71 & 53.32 & 87.00 & 93.46 & 48.76 & 28.04 & 67.83 \\

    \rowcolor{yellow!50}
    Encoder & 23.64 & 13.953 & 0.526 & 0.833 & 6.697 & \textbf{69.43} & \textbf{45.12} & \textbf{20.00} & \textbf{23.55} & \textbf{57.90} & \textbf{60.10} & 25.40 & 13.52 & \textbf{39.38} \\
    \hline %\hline

    %\rowcolor[gray]{0.9}
    \textcolor{blue}{\textbf{Fixed ETF}} \\

    Projector & \textbf{12.62} & 0.393 & 0.490 & 0.468 & 0.316 & 91.38 & 65.72 & 64.51 & 64.97 & 82.22 & 97.42 & 43.17 & 21.51 & 66.36 \\
    
    \rowcolor{yellow!50}
    %% ENCODER
    \textbf{Encoder} & 15.52 & 2.175 & 0.603 & 0.616 & 5.364 & 71.52 & 47.24 & 25.10 & 24.32 & 63.67 & 67.81 & \textbf{21.56} & 13.55 & 41.85 \\ % error

    \hline \hline
    \textbf{\textcolor{orange}{Detection Error $\downarrow$}} \\
    %\rowcolor[gray]{0.9}
    \textcolor{blue}{\textbf{Plastic}} \\
    \rowcolor{green!25}
    Projector & 15.10 & 0.498 & 0.515 & 0.428 & 1.422 & 63.05 & \textbf{47.87} & 62.45 & 70.07 & 80.88 & 98.95 & 89.37 & 79.25 & 74.00 \\
    
    Encoder & 23.64 & 13.953 & 0.526 & 0.833 & 6.697 & 81.27 & 98.82 & 93.33 & 86.48 & 79.98 & 99.40 & 91.25 & 93.88 & 90.55 \\
    \hline
    %\rowcolor[gray]{0.9}
    \textcolor{blue}{\textbf{Fixed ETF}} \\
    %% PROJECTOR
    \rowcolor{green!25}
    \textbf{Projector} & \textbf{12.62} & 0.393 & 0.490 & 0.468 & 0.316 & \textbf{60.85} & 48.23 & \textbf{42.35} & \textbf{67.69} & \textbf{56.51} & 99.04 & \textbf{76.32} & \textbf{69.84} & \textbf{65.10} \\
    %\hline
    
    %% ENCODER
    Encoder & 15.52 & 2.175 & 0.603 & 0.616 & 5.364 & 67.17 & 98.14 & 81.76 & 84.95 & 84.57 & 99.70 & 97.36 & 87.34 & 87.62 \\
    
    \hline \hline
    %\vspace{-2em}
    \end{tabular}}
\end{table*}








\begin{comment}

%%% Following results correspond to MSE Loss with Plastic Projector
\begin{table*}[t]
    

\centering
  \caption{\textbf{ETF Fixed Vs. Plastic Projector.} The VGGm-17 models ($F_{\psi}$($G_{\phi}$($H_{\theta}))$) are trained on \textbf{ImageNet-100} dataset (ID) and evaluated on 8 OOD datasets. In \textbf{encoder} method, the embeddings are extracted from the encoder ($H_{\theta}$) and before projector. And, in \textbf{projector} method, the embeddings are extracted after projector ($G_{\phi}$) and before output layer ($F_{\psi}$). For OOD transfer we report the top-1 error whereas for OOD detection we report the FPR95. $\uparrow$ indicates larger values are better and $\downarrow$ indicates smaller values are better. All values except neural collapse are percentages. \textbf{A lower $\mathcal{NC}$ indicates higher neural collapse. %$+\delta$ and $-\delta$ indicate \% increase and \% decrease respectively, when changing from encoder to projector.
  }
  } 
  \label{tab:plastic_proj}
  \centering
  \resizebox{\linewidth}{!}{
     \begin{tabular}{cc|ccc|ccccccccc}
     \hline %\hline
     \multicolumn{1}{c}{\textbf{Method}} &
     \multicolumn{1}{c|}{\textbf{ID Error}} &
     \multicolumn{3}{c|}{\textbf{Neural Collapse}} &
     \multicolumn{9}{c}{\textbf{OOD Datasets}} \\
    & IN & $\mathcal{NC}1$ &  $\mathcal{NC}2$ &  $\mathcal{NC}3$ & IN-R & CIFAR & Flower & NINCO & CUB & AirCrafts & Pet & STL & Avg. \\
    & 100 &  &  &  & 200 & 100 & 102 & 64 & 200 & 100 & 37 & 10 & \\    
    \hline \hline
    \textbf{Transfer Error $\downarrow$} \\
    Projector & \textbf{16.68} & 2.040 & 0.601 & 0.337 & 89.53 & 75.98 & 89.90 & 57.40 & 90.87 & 97.60 & 56.31 & 31.29 & 73.61 \\
     \hline
    \textbf{Encoder} & 19.42 & 3.969 & 0.552 & 0.705 & \textbf{77.72} & \textbf{56.18} & \textbf{36.08} & \textbf{29.93} & \textbf{65.69} & \textbf{73.27} & \textbf{28.59} & \textbf{16.55} & \textbf{48.00} \\ % error
    
    \hline \hline
    \textbf{Detection FPR $\downarrow$} \\
    \textbf{Projector} & \textbf{16.68} & 2.040 & 0.601 & 0.337  & \textbf{72.95} & \textbf{40.80} & \textbf{76.37} & \textbf{72.05} & \textbf{71.87} & \textbf{98.11} & \textbf{87.90} & \textbf{74.05} & \textbf{74.26} \\
    \hline
    Encoder & 19.42 & 3.969 & 0.552 & 0.705 & 98.25 & 99.95 & 88.63 & 97.06 & 98.24 & 87.55 & 97.33 & 98.90 & 95.74 \\
    \hline \hline
    %\vspace{-2em}
    \end{tabular}}
\end{table*}


\end{comment}


\section{Formal Proposition: Collapsing Implies Entropy $-\infty$}
\label{sec:details_proposition}

\begin{proposition}[Entropy under Class-Conditional Collapse]
Consider a mixture of $K$ class-conditional densities $\{p_{\ell,k}\}_{k=1}^K$ in $\mathbb{R}^{d_\ell}$, with mixture weights $\{\pi_k\}$. Suppose that for each $k$, there exists a family of densities $\{p_{\ell,k}(\epsilon) : \epsilon > 0\}$ such that
\[
\lim_{\epsilon \to 0} p_{\ell,k}(\epsilon) = \delta(z - \mu_{\ell,k})
\]
in the weak topology (i.e., they converge to a Dirac delta). Then
\[
\lim_{\epsilon \to 0} H\left(\sum_{k=1}^K \pi_k \, p_{\ell,k}(\epsilon)\right) = -\infty.
\]
\end{proposition}

\paragraph{Proof (Sketch).}
For each fixed $k$,
\[
\lim_{\epsilon \to 0} H(p_{\ell,k}(\epsilon)) = -\infty,
\]
because each $p_{\ell,k}(\epsilon)$ ``collapses'' its support around $\mu_{\ell,k}$. This is analogous to reducing variance to $0$ for a parametric family (e.g., a Gaussian with covariance $\epsilon I$).

The mixture’s differential entropy can be bounded above as
\[
H\left(\sum_{k} \pi_k \, p_{\ell,k}(\epsilon)\right) \leq \sum_{k} \pi_k H(p_{\ell,k}(\epsilon)) + \text{const},
\]
where the constant term arises from mixture overlap considerations (or the standard inequality $H(\sum_i q_i) \leq \sum_i \alpha_i H(q_i) + \log K$ for simpler forms).

Hence, if each $H(p_{\ell,k}(\epsilon)) \to -\infty$, the sum also diverges to $-\infty$. This demonstrates that if each class distribution collapses around its mean, the overall mixture’s differential entropy approaches $-\infty$.



%%%%%%%%%%%%%%%%%%%

\section{Comprehensive Results (Encoder Vs. Projector)}
\label{sec:comprehensive_results}

    
    
\begin{table*}[t!]
    \centering
    \small
    
    \scalebox{0.90}{
    \setlength{\tabcolsep}{1.0pt}
    \begin{tabular}{l c c c r | c c c c c c |c  c c }
    \toprule
    \multirow{1}{*}{Method} & \multirow{1}{*}{Recipe} & \multirow{1}{*}{Complexity} & \multirow{1}{*}{\# P.} & \multirow{1}{*}{\# T.P.}& MME & MMB &POPE & \multicolumn{1}{c} {SEED} & MMMU & MM-Vet& TQA & SQA-I  & \multicolumn{1}{c}{GQA} \\
    \midrule
    \rowcolor{gray!14}
    \multicolumn{14}{l}{\textbf{\textit{Encoder-based VLMs}}} \\ 
    OpenFlamingo~\cite{openflamingo} & \underline{PT, SFT}& Quadratic & 9B& 96.6\%  & - & 4.6 & - & - & - & - & 33.6 & - & - \\
    MiniGPT-4~\cite{minigpt} & \underline{PT, SFT}& Quadratic & 13B& 94.8\%  & 581.7 & 23.0 & - & - & -& 22.1 & - & - & 32.2  \\
    Qwen-VL~\cite{qwenvl} & \underline{PT, SFT}& Quadratic & 7B& 100.0\%  & - & 38.2 & - & 56.3 & - & - & 63.8 & 67.1 & 59.3\\ 
    LLaVA-Phi~\cite{llavaphi}  & \underline{PT, SFT}& Quadratic & 3B& 90.0\%  & 1335.1 & 59.8 & 85.0 & - & - & 28.9& 48.6 & 68.4 & - \\
    MobileVLM-3B~\cite{mobilevlm} & \underline{PT, SFT}& Quadratic & 3B& 90.0\%  & 1288.9 & 59.6 & 84.9 & - & - & - & 47.5 & 61.0 & 59.0  \\
    VisualRWKV~\cite{visualrwkv} & \underline{PT, SFT}&  \textbf{Linear} & 3B& 90.0\%  & 1369.2 & 59.5 & 83.1 & - & - & - & 48.7 & 65.3 & 59.6 \\
    VL-Mamba~\cite{vlmamba} & \underline{PT, SFT}&  \textbf{Linear} & 3B& 90.0\%  & 1369.6 & 57.0 & 84.4 & - & -& 32.6 & 48.9 & 65.4 & 56.2 \\
    Cobra~\cite{cobra} & \underline{PT, SFT}&  \textbf{Linear} & 3.5B& 82.6\%  & - & - & \textbf{88.4} & - & - & - & 58.2 & - & \textbf{62.3}\\
    \midrule
    \rowcolor{gray!14}
    \multicolumn{14}{l}{\textbf{\textit{Decoder-only VLMs}}} \\
    Fuyu-8B (HD)~\cite{fuyu} & \underline{PT, SFT}& Quadratic & 8B& 100.0\%  & 728.6 & 10.7 & 74.1 & - & - & 21.4 & - & - & -\\
    SOLO~\cite{solo} & \underline{PT, SFT}& Quadratic &  7B& 100.0\%   & 1001.3 & - & - & 64.4 & - & - & - & 73.3 & -   \\    
    Chameleon-7B~\cite{chameleon}  & \underline{PT, SFT}& Quadratic &  7B& 100.0\%   & 170 & 31.1 & - & 30.6 & 25.4 & 8.3 & 4.8 & 47.2 & -\\  
    EVE-7B~\cite{eve}  & \underline{PT, SFT}& Quadratic &  7B& 100.0\%  & 1217.3 & 49.5 & 83.6 & 61.3 & \underline{32.3} & 25.6& 51.9 & 63.0 & 60.8 \\
    Emu3~\cite{emu3} & \underline{PT, SFT}& Quadratic & 8B& 100.0\%  & - & 58.5 & 85.2 & \underline{68.2} & 31.6 & \underline{37.2} & \underline{64.7} & \underline{89.2} & 60.3\\
    HoVLE~\cite{hovle} & DT, PT, SFT & Quadratic & \textbf{2.6B}& 100.0\%  & \textbf{1433.5} & \textbf{71.9} & \underline{87.6} & \textbf{70.7} & \textbf{33.7} & \textbf{44.3} & \textbf{66.0} & \textbf{94.8} & \underline{60.9} \\
    \rowcolor{green!15}
    \name{} & \textbf{DT} & \textbf{Linear} & \underline{2.7B}& \underline{14.7\%}  &1303.5 & 57.2 & 85.2 & 62.9& 30.7  & 31.1 &47.7 & 79.2 & 57.4 \\
    \rowcolor{yellow!15}
    \name{} & \textbf{DT} & \underline{Hybrid} & \underline{2.7B}& \textbf{11.2\%}  & \underline{1371.1} & \underline{63.7} & 86.7 & 66.3 & \underline{32.3} & 36.9 & 55.1 & 86.9 & 59.3  \\
    
    \bottomrule
    \end{tabular}
    }
    \vspace{-1em}
    \caption{\textbf{Comparison with existing VLMs on general VLM benchmarks.} ``Recipe'' denotes the adopted training recipe. ``PT'', ``SFT'', and ``DT'' denote the pre-training, supervised fine-tuning, and distillation training, respectively. ``Complexity'' denotes the model computation complexity with respect to the number of tokens. ``\# P.'' denotes the number of total parameters. ``\# T.P.'' denotes the percentage of trainable parameters ($\frac{\text{trainable paramters}}{\text{total parameters}}$). The best performance is highlighted in \textbf{bold} and the second-best result is \underline{underlined}.}
    \label{tab:results_general}
    \end{table*}
 % VGG


\begin{figure*}[t]
    \centering
    \begin{subfigure}[b]{0.48\textwidth}
        \centering
        \includegraphics[width=\textwidth]{images/umap_enc_proj_10c_IN_ninco_64_vgg17.png}
        \caption{UMAP of Embeddings}
        \label{fig:umap_id_ood}
    \end{subfigure}
    \hfill
    \begin{subfigure}[b]{0.48\textwidth}
        \centering
        \includegraphics[width=\textwidth]{images/energy_ood_ninco.png}
        \caption{Energy Score Distribution}
        \label{fig:eng_id_ood}
    \end{subfigure}
    \caption{\textbf{ID \& OOD Data Visualization.} In \textbf{(a)}, The projector exhibits a greater separation between ID and OOD embeddings than the encoder. For clarity, we show 10 ImageNet classes as ID data and 64 classes from the NINCO dataset as OOD data. 
    In \textbf{(b)}, The projector achieves higher energy scores (and lower FPR95) for ID data.
    For ID and OOD datasets, we show ImageNet-100 and NINCO-64 respectively.
    }
    \label{fig:umap_eng_id_ood}
\end{figure*}






%\begin{figure}[t]
%    \centering
%    \includegraphics[width = 0.99\linewidth]{images/umap_enc_proj_10c_IN_ninco_64_vgg17.png}
%  \caption{\textbf{UMAP of ID and OOD Embeddings.} The projector exhibits a greater separation between ID and OOD embeddings. In contrast, the encoder shows a lot of overlap between ID and OOD embeddings. The embeddings are extracted from ImageNet-100 pre-trained VGG17. For clarity, we show 10 ImageNet classes as ID data and 64 classes from the NINCO dataset as OOD data.} 
%  \label{fig:umap_id_ood}
%\end{figure}


\begin{figure*}[t]
    \centering
    \begin{subfigure}[b]{0.48\textwidth}
        \centering
        \includegraphics[width=\textwidth]{images/energy_ood_flower.png}
        \caption{OOD Dataset: Flowers-102}
        \label{fig:eng_flower}
    \end{subfigure}
    \hfill
    \begin{subfigure}[b]{0.48\textwidth}
        \centering
        \includegraphics[width=\textwidth]{images/energy_ood_stl.png}
        \caption{OOD Dataset: STL-10}
        \label{fig:eng_stl}
    \end{subfigure}
    \caption{\textbf{Energy Score Distribution.}
    The projector creates a greater separation between ID and OOD data and achieves a lower FPR95 than the encoder. For better OOD detection, ID data should obtain higher energy scores than OOD data. For ID and OOD datasets, we show ImageNet-100 and Flowers-102/ STL-10 respectively. The energy scores are calculated based on logits from the VGG17 model pre-trained on ImageNet-100.
    }
    \label{fig:more_eng_id_ood}
\end{figure*}











%\begin{figure}[t]
%    \centering
%    \includegraphics[width = 0.99\linewidth]{images/energy_ood_flower.png}
%  \caption{\textbf{Energy Score Distribution.} The projector creates a greater separation between ID and OOD data and achieves a lower FPR95 than the encoder. For better OOD detection, ID data should obtain higher energy than OOD data.
%  For ID and OOD datasets, we show ImageNet-100 and Flowers-102 respectively. The energy scores are calculated based on logits from the VGG17 model pre-trained on ImageNet-100.} 
%  \label{fig:energy_id_ood}
%\end{figure}


\subsection{VGG Experiments}

The detailed VGG17 results are given in Table~\ref{tab:main_results}. VGG results demonstrate that the encoder effectively mitigates NC for OOD generalization and the projector builds collapsed features and excels at the OOD detection task. The results also confirm that NC properties can be built using both CE and MSE loss functions.

\textbf{Qualitative Comparison.} 
We compare and visualize encoder embeddings and projector embeddings using UMAP. We also visualize the energy score distribution of ID and OOD data. The analysis is based on the VGG17 model pre-trained on the ImageNet-100 (ID) dataset and evaluated on OOD datasets: NINCO-64, Flowers-102, and STL-10. We observe the following:
\begin{itemize}[noitemsep, nolistsep, leftmargin=*]
    \item In Fig.~\ref{fig:umap_id_ood}, the UMAP shows that projector embeddings nicely separate ID and OOD sets whereas encoder embeddings exhibit substantial overlap between ID and OOD sets. This demonstrates that, unlike the encoder, the projector can intensify NC and is adept at OOD detection.

    \item We show the energy distribution of ID and OOD sets in Fig.~\ref{fig:eng_id_ood} and~\ref{fig:more_eng_id_ood}. In all comparisons, we observe that the projector outperforms the encoder in separating ID and OOD sets based on energy scores.
    
\end{itemize}




\subsection{ResNet Experiments}

\begin{table*}[ht]
\centering
  \caption{\textbf{Comprehensive ResNet Results.} ResNet models are trained on \textbf{ImageNet-100} dataset (ID) and evaluated on 8 OOD datasets. %In the \textbf{encoder} method, the embeddings are extracted from the encoder and before the projector. And, in \textbf{projector} method, the embeddings are extracted after the projector and before the classifier head.
  For OOD transfer we report $\boldsymbol{\mathcal{E}}_{\text{GEN}}$ (\%) whereas for OOD detection we report $\boldsymbol{\mathcal{E}}_{\text{DET}}$ (\%). All metrics except NC are reported in percentage.
  \textbf{A lower $\mathcal{NC}$ indicates stronger neural collapse.} %\textcolor{brown}{Across all ResNets, the encoder enhances OOD transfer and the projector improves OOD detection.}
  } 
  \label{tab:resnet_results}
  \centering
  \resizebox{\linewidth}{!}{
     \begin{tabular}{cc|cccc|ccccccccc}
     \hline %\hline
     \multicolumn{1}{c}{\textbf{Model}} &
     \multicolumn{1}{c|}{$\boldsymbol{\mathcal{E}}_{\text{ID}} \downarrow$} &
     \multicolumn{4}{c|}{\textbf{Neural Collapse}} &
     \multicolumn{9}{c}{\textbf{OOD Datasets}} \\
    & IN & $\mathcal{NC}1$ &  $\mathcal{NC}2$ &  $\mathcal{NC}3$ & $\mathcal{NC}4$ & IN-R & CIFAR & Flowers & NINCO & CUB & Aircrafts & Pets & STL & Avg. \\
    & 100 &  &  &  &  & 200 & 100 & 102 & 64 & 200 & 100 & 37 & 10 & \\    
    %\hline \hline
    \toprule
    \textcolor{blue}{\textbf{ResNet18}} \\
    \textcolor{orange}{\textbf{Transfer Error $\downarrow$}} \\
    %% PROJECTOR
    Projector & \textbf{16.14} & 0.341 & 0.456 & 0.306 & 0.540 & 86.65 & 60.33 & 63.92 & 50.09 & 81.79 & 94.36 & 43.15 & 24.32 & 63.08 \\
    
    \rowcolor[gray]{0.9}
    %% ENCODER
    \textbf{Encoder} & 20.14 & 1.762 & 0.552 & 0.555 & 10.695 & \textbf{74.17} & \textbf{53.33} & \textbf{31.37} & \textbf{28.15} & \textbf{68.85} & \textbf{81.61} & \textbf{27.72} & \textbf{16.56} & \textbf{47.72} \\ % error

    \midrule
    \textcolor{orange}{\textbf{Detection Error $\downarrow$}} \\
    \rowcolor[gray]{0.9}
    %% PROJECTOR
    \textbf{Projector} & \textbf{16.14} & 0.341 & 0.456 & 0.306 & 0.540 & \textbf{67.92} & \textbf{61.21} & \textbf{71.18} & \textbf{71.09} & \textbf{23.20} & \textbf{99.28} & \textbf{81.41} & \textbf{82.29} & \textbf{69.70} \\
    %\hline
    
    %% ENCODER
    Encoder & 20.14 & 1.762 & 0.552 & 0.555 & 10.695 & 71.50 & 96.44 & 86.27 & 84.78 & 65.48 & 99.43 & 95.86 & 89.63 & 86.17 \\
    
    \hline \hline
    \textcolor{blue}{\textbf{ResNet34}} \\
    \textcolor{orange}{\textbf{Transfer Error $\downarrow$}} \\
    Projector & \textbf{14.54} & 0.252 & 0.672 & 0.294 & 0.324 & 83.93 & 58.65 & 64.41 & 44.05 & 81.65 & 93.58 & 43.64 & 22.87 & 61.60 \\
    
    \rowcolor[gray]{0.9}
    \textbf{Encoder} & 17.20 & 0.737 & 0.634 & 0.871 & 22.587 & \textbf{76.97} & \textbf{54.45} & \textbf{41.47} & \textbf{33.33} & \textbf{71.25} & \textbf{82.00} & \textbf{29.25} & \textbf{16.45} & \textbf{50.65} \\
    
    \hline
    \textcolor{orange}{\textbf{Detection Error $\downarrow$}} \\
    \rowcolor[gray]{0.9}
    \textbf{Projector} & \textbf{14.54} & 0.252 & 0.672 & 0.294 & 0.324 & \textbf{61.72} & \textbf{60.05} & \textbf{47.94} & \textbf{66.24} & \textbf{67.59} & \textbf{98.35} & \textbf{83.78} & \textbf{78.49} & \textbf{70.52} \\

    Encoder & 17.20 & 0.737 & 0.634 & 0.871 & 22.587 & 69.67 & 93.07 & 70.59 & 76.87 & 83.02 & 99.34 & 97.17 & 90.75 & 85.06 \\
    %\hline
    \bottomrule
    %\vspace{-2em}
    \end{tabular}}
\end{table*} % ResNet

\begin{figure}[t]
    \centering
    \includegraphics[width = 0.99\linewidth]{images/nc_ood_detect_transfer_corr_resnet_updated.png}
  \caption{Lower NC1 values (indicating stronger neural collapse) correlate with lower OOD detection error but higher OOD transfer error, and vice versa. This suggests that stronger neural collapse improves OOD detection, while weaker neural collapse enhances OOD generalization. We analyze various layers of \textbf{ResNet18}, pre-trained on ImageNet-100 (ID), and evaluate them on four OOD datasets. $R$ denotes the Pearson correlation coefficient.
  %NC exhibits a positive correlation with OOD detection error and a negative correlation with OOD transfer error when we analyze different layers of ResNet18 models that are pre-trained on ImageNet-100 (ID) and evaluated on four OOD datasets. $R$ denotes the Pearson correlation coefficient.
  } 
  \label{fig:nc_resnet}
\end{figure}



The detailed ResNet18/34 results are given in Table~\ref{tab:resnet_results}. Our findings validate that NC can be controlled in various ResNet architectures for improving OOD detection and OOD generalization performance.
Additionally, NC shows a strong correlation with OOD detection and OOD generalization as illustrated in Fig.~\ref{fig:nc_resnet}.

\begin{figure}[h]
    \centering
    \includegraphics[width = 0.99\linewidth]{images/umap_enc_proj_10c_ce_20_30_resnet.png}
  \caption{\textbf{Visualization of Embedding (ResNet18).} In this UMAP, projector embeddings exhibit greater neural collapse ($\mathcal{NC}1=0.341$) than the encoder embeddings ($\mathcal{NC}1=1.762$) as indicated by the formation of tight clusters around class-means. For clarity, we highlight 10 ImageNet classes by distinct colors. The embeddings are extracted from ImageNet-100 pre-trained ResNet18.} 
  \label{fig:umap_vis_resnet}
\end{figure}


We also visualize embeddings extracted from the encoder and projector of the ResNet18 model. As depicted in Fig.~\ref{fig:umap_vis_resnet}, projector embeddings exhibit much greater neural collapse than encoder embeddings.



\subsection{ViT Experiments} 

\begin{table*}[ht]
\centering
  \caption{\textbf{Comprehensive ViT Results.} ViT-Tiny (6.02M) and ViT-Small (23.03M) are trained on \textbf{ImageNet-100} dataset (ID) and evaluated on 8 OOD datasets. %In the \textbf{encoder} method, the embeddings are extracted from the encoder and before the projector. And, in \textbf{projector} method, the embeddings are extracted after the projector and before the classifier head. 
  For OOD transfer we report $\boldsymbol{\mathcal{E}}_{\text{GEN}}$ (\%) whereas for OOD detection we report $\boldsymbol{\mathcal{E}}_{\text{DET}}$ (\%). All metrics except NC are reported in percentage. \textbf{A lower $\mathcal{NC}$ indicates stronger neural collapse.} %\textcolor{brown}{Across all ViT models, the encoder enhances OOD transfer whereas the projector improves OOD detection.} 
  } 
  \label{tab:vit_results}
  \centering
  \resizebox{\linewidth}{!}{
     \begin{tabular}{cc|cccc|ccccccccc}
     \hline %\hline
     \multicolumn{1}{c}{\textbf{Model}} &
     \multicolumn{1}{c|}{$\boldsymbol{\mathcal{E}}_{\text{ID}} \downarrow$} &
     \multicolumn{4}{c|}{\textbf{Neural Collapse}} &
     \multicolumn{9}{c}{\textbf{OOD Datasets}} \\
    & IN & $\mathcal{NC}1$ &  $\mathcal{NC}2$ &  $\mathcal{NC}3$ & $\mathcal{NC}4$ & IN-R & CIFAR & Flowers & NINCO & CUB & Aircrafts & Pets & STL & Avg. \\
    & 100 &  &  &  &  & 200 & 100 & 102 & 64 & 200 & 100 & 37 & 10 & \\    
    %\hline \hline
    \toprule
    \textcolor{blue}{\textbf{ViT-Tiny}} \\
    \textcolor{orange}{\textbf{Transfer Error $\downarrow$}} \\
    %% PROJECTOR
    Projector & \textbf{32.04} & 2.748 & 0.609 & 0.798 & 1.144 & 87.37 & 60.71 & 64.61 & 39.71 & 80.00 & 92.00 & 54.27 & 29.55 & 63.53 \\
    
    \rowcolor[gray]{0.9}
    %% ENCODER
    \textbf{Encoder} & 33.94 & 5.769 & 0.748 & 0.847 & 2.332 & \textbf{82.28} & \textbf{52.00} & \textbf{42.94} & \textbf{30.36} & \textbf{63.15} & \textbf{84.31} & \textbf{44.86} & \textbf{21.13} & \textbf{52.63} \\

    \midrule
    \textcolor{orange}{\textbf{Detection Error $\downarrow$}} \\
    \rowcolor[gray]{0.9}
    %% PROJECTOR
    \textbf{Projector} & \textbf{32.04} & 2.748 & 0.609 & 0.798 & 1.144 & \textbf{81.12} & \textbf{60.81} & \textbf{77.55} & \textbf{82.40} & \textbf{79.05} & 99.10 & \textbf{95.15} & 90.06 & \textbf{83.16} \\
    
    %% ENCODER
    Encoder & 33.94 & 5.769 & 0.748 & 0.847 & 2.332 & 83.80 & 96.76 & 87.65 & 93.11 & 82.14 & 99.10 & 95.75 & \textbf{88.79} & 90.89 \\
    \hline \hline
    \textcolor{blue}{\textbf{ViT-Small}} \\
    \textcolor{orange}{\textbf{Transfer Error $\downarrow$}} \\
    
    Projector & \textbf{31.28} & 0.822 & 0.522 & 0.712 & 0.962 & 86.57 & 58.46 & 64.51 & 39.20 & 78.25 & 90.70 & 53.86 & 29.30 & 62.61 \\
    
    \rowcolor[gray]{0.9}
    \textbf{Encoder} & 33.40 & 1.610 & 0.601 & 0.740 & 2.814 & \textbf{80.53} & \textbf{49.68} & \textbf{40.49} & \textbf{29.93} & \textbf{61.08} & \textbf{81.28} & \textbf{44.45} & \textbf{20.98} & \textbf{51.05} \\
    \hline
    
    \textcolor{orange}{\textbf{Detection Error $\downarrow$}} \\
    \rowcolor[gray]{0.9}
    \textbf{Projector} & \textbf{31.28} & 0.822 & 0.522 & 0.712 & 0.962 & \textbf{76.03} & \textbf{58.79} & \textbf{75.20} & \textbf{81.97} & \textbf{82.46} & \textbf{98.50} & 95.42 & \textbf{88.74} & \textbf{82.14} \\
    
    Encoder & 33.40 & 1.610 & 0.601 & 0.740 & 2.814 & 82.47 & 96.84 & 90.39 & 92.60 & 86.00 & 99.25 & \textbf{94.36} & 89.04 & 91.37 \\
    
    \bottomrule
    %\vspace{-2em}
    \end{tabular}}
\end{table*} %% ViT

As shown in Table~\ref{tab:vit_results}, the projector outperforms the encoder in OOD detection by absolute 7.73\% (ViT-Tiny) and 9.23\% (ViT-Small). Whereas the encoder outperforms the projector in OOD transfer by absolute 10.90\% (ViT-Tiny) and 11.56\% (ViT-Small). This demonstrates that controlling NC improves OOD detection and generalization in ViTs. 



\section{Analysis on Entropy Regularization}
\label{sec:analysis_entropy_reg}

\begin{table*}[t]
\centering
  \caption{\textbf{Entropy Regularization Vs. No Entropy Regularization.} VGG17 models are pre-trained on \textbf{ImageNet-100} dataset (ID) and evaluated on 8 OOD datasets. Entropy regularization loss with a coefficient, $\alpha$ is applied in the last encoder layer. The same color highlights the rows to compare. All metrics except NC are reported in \%. The lower the NC value, the stronger the neural collapse. For OOD transfer we report $\boldsymbol{\mathcal{E}}_{\text{GEN}}$ (\%) whereas for OOD detection we report $\boldsymbol{\mathcal{E}}_{\text{DET}}$ (\%).
  %\textcolor{brown}{Using entropy penalty enhances OOD transfer by 2.71\% (absolute), OOD detection by 2.36\% (absolute), and ID performance by 0.84\% (absolute).}
  } 
  \label{tab:reg_vs_no_reg}
  \centering
  \resizebox{\linewidth}{!}{
     \begin{tabular}{cc|cccc|ccccccccc}
     \hline %\hline
     \multicolumn{1}{c}{\textbf{Method}} &
     \multicolumn{1}{c|}{$\boldsymbol{\mathcal{E}}_{\text{ID}} \downarrow$} &
     \multicolumn{4}{c|}{\textbf{Neural Collapse}} &
     \multicolumn{9}{c}{\textbf{OOD Datasets} $\downarrow$} \\
    & IN & $\mathcal{NC}1$ & $\mathcal{NC}2$ & $\mathcal{NC}3$ & $\mathcal{NC}4$ & IN-R & CIFAR & Flowers & NINCO & CUB & Aircrafts & Pets & STL & Avg. \\
    & 100 &  &  &  &  & 200 & 100 & 102 & 64 & 200 & 100 & 37 & 10 & \\    
    \hline
    %% CE Loss
    \textbf{\textcolor{orange}{Transfer Error $\downarrow$}} \\
    %\rowcolor[gray]{0.9}
    \textcolor{blue}{\textbf{No Reg.} ($\mathbf{\alpha=0}$)} \\
    Projector & 13.46 & 0.260 & 0.636 & 0.369 & 0.883 & 84.30 & 60.73 & 65.69 & 45.15 & 82.90 & 93.73 & 40.56 & 22.85 & 61.99 \\
    \rowcolor{yellow!50}
    Encoder & 15.24 & 1.308 & 0.719 & 0.619 & 5.184 & 73.52 & 49.26 & 37.06 & 25.51 & 64.31 & 69.58 & 22.24 & 15.00 & 44.56 \\
    \hline %\hline

    
    \textcolor{blue}{\textbf{Reg.} ($\mathbf{\alpha=0.05}$)} \\

    Projector & \textbf{12.62} & 0.393 & 0.490 & 0.468 & 0.316 & 91.38 & 65.72 & 64.51 & 64.97 & 82.22 & 97.42 & 43.17 & 21.51 & 66.36 \\
    
    \rowcolor{yellow!50}
    %% ENCODER
    \textbf{Encoder} & 15.52 & 2.175 & 0.603 & 0.616 & 5.364 & \textbf{71.52} & \textbf{47.24} & \textbf{25.10} & 24.32 & 63.67 & \textbf{67.81} & \textbf{21.56} & \textbf{13.55} & \textbf{41.85} \\ % error

    \hline
    \textcolor{blue}{\textbf{Reg.} ($\mathbf{\alpha=0.1}$)} \\
    Projector & 13.04 & 0.428 & 0.671 & 0.340 & 0.320 & 93.62 & 66.00 & 55.29 & 79.25 & 81.84 & 97.09 & 46.96 & 23.00 & 67.88 \\

    \rowcolor{yellow!50}
    Encoder & 16.12 & 2.861 & 0.538 & 0.636 & 6.677 & 73.05 & 48.61 & 27.84 & \textbf{22.62} & \textbf{61.91} & 70.21 & 22.87 & 13.83 & 42.62 \\
    
    \hline \hline
    \textbf{\textcolor{orange}{Detection Error $\downarrow$}} \\
    %\rowcolor[gray]{0.9}
    \textcolor{blue}{\textbf{No Reg.} ($\mathbf{\alpha=0}$)} \\
    \rowcolor{green!25}
    Projector & 13.46 & 0.260 & 0.636 & 0.369 & 0.883 & 65.22 & 54.32 & 45.20 & 67.18 & 52.37 & 98.41 & 84.38 & 72.58 & 67.46 \\
    
    Encoder & 15.24 & 1.308 & 0.719 & 0.619 & 5.184 & 74.22 & 99.75 & 85.10 & 88.52 & 92.99 & 98.59 & 95.34 & 92.14 & 90.83 \\
    \hline
    %\rowcolor[gray]{0.9}
    \textcolor{blue}{\textbf{Reg.} ($\mathbf{\alpha=0.05}$)} \\
    %% PROJECTOR
    \rowcolor{green!25}
    \textbf{Projector} & \textbf{12.62} & 0.393 & 0.490 & 0.468 & 0.316 & \textbf{60.85} & \textbf{48.23} & \textbf{42.35} & 67.69 & 56.51 & 99.04 & \textbf{76.32} & \textbf{69.84} & \textbf{65.10} \\
    %\hline
    
    %% ENCODER
    Encoder & 15.52 & 2.175 & 0.603 & 0.616 & 5.364 & 67.17 & 98.14 & 81.76 & 84.95 & 84.57 & 99.70 & 97.36 & 87.34 & 87.62 \\

    \hline
    \textcolor{blue}{\textbf{Reg.} ($\mathbf{\alpha=0.1}$)} \\
    \rowcolor{green!25}
    Projector & 13.04 & 0.428 & 0.671 & 0.340 & 0.320 & 61.13 & 54.69 & 43.14 & \textbf{64.63} & \textbf{50.73} & 98.74 & 82.42 & 71.51 & 65.87 \\

    Encoder & 16.12 & 2.861 & 0.538 & 0.636 & 6.677 & 68.72 & 94.67 & 85.78 & 87.76 & 85.49 & 98.92 & 95.15 & 86.28 & 87.85 \\
    
    %\hline \hline
    \bottomrule
    %\vspace{-2em}
    \end{tabular}}
\end{table*}


Table~\ref{tab:reg_vs_no_reg} presents the detailed comparison between a model using the entropy regularization vs another model omitting it. We observe that using entropy penalty enhances OOD transfer by 2.71\% (absolute), OOD detection by 2.36\% (absolute), and ID performance by 0.84\% (absolute).

Additionally, we analyze the impact of the entropy regularization loss coefficient on the ID and OOD transfer. Table~\ref{tab:entropy_loss_coeff} shows that increasing coefficient increases OOD transfer and rank of embeddings. This suggests that entropy regularization helps encode diverse features and reduce redundant features, encouraging utilization of all dimensions. Although entropy regularization is not sensitive to coefficient, over-regularization may hurt ID performance. Thereby, any non-aggressive coefficient can maintain good performance in both ID and OOD tasks. %Typically, a coefficient of $0.1$ or less works fine.


% \textbf{Training Dynamics.}
We also analyze the impact of entropy regularization on encoder embeddings during the training phase. During each training epoch, we measure the NC1 criterion, entropy, and effective rank of encoder embeddings. These experiments are computationally intensive for large-scale datasets. Therefore, we perform small-scale experiments where we train VGG17 models on the ImageNet-10 (10 ImageNet classes) subset for 100 epochs. We evaluate two cases: one with entropy regularization and another without entropy regularization.


The results are illustrated in Fig.~\ref{fig:nc_dynamics}.
%The impact of entropy regularization on NC1 is exhibited in Fig.~\ref{fig:nc_dynamics}. 
We find that entropy regularization achieves higher NC1 values during training compared to the model without any regularization. Thus, it helps mitigate NC during training, thereby contributing to OOD generalization. These findings align with our theoretical analysis showing entropy as an effective mechanism to prevent NC in the encoder.


Entropy regularization also increases the entropy and effective rank of the encoder embeddings. 
This demonstrates that entropy regularization helps encode diverse features, ensuring the features remain sufficiently ``spread out.''


Without the entropy regularization, the entropy of encoder embeddings does not improve. Also, the effective rank ends up at a low value (as low as the number of ID classes). The low rank is a sign of strong neural collapse and suggests that the encoder uses a few feature dimensions to encode information with huge redundancy in other dimensions. This degeneracy of embeddings impairs OOD transfer.
Entropy regularization counteracts this and improves OOD transfer.




\begin{table*}[t]
\centering
  \caption{\textbf{Entropy Regularization Loss Coefficient.} VGG17 models are pre-trained on the ImageNet-10 (10 ImageNet classes) ID dataset and evaluated on 8 OOD datasets. %via linear probing. %Reported is the top-1 error (\%). %Increasing the entropy regularization loss coefficient increases embedding rank and decreases OOD transfer error. 
  $\alpha$ denotes the entropy regularization loss coefficient. We use a regular VGG17 network without the projector to focus on entropy regularization. Effective rank corresponds to penultimate embeddings. For OOD generalization, we report $\boldsymbol{\mathcal{E}}_{\text{GEN}}$ (\%).
  } 
  \label{tab:entropy_loss_coeff}
  \centering
  \resizebox{\linewidth}{!}{
     \begin{tabular}{ccc|ccccccccc}
     \hline %\hline
     \multicolumn{1}{c}{\textbf{Reg. Coeff.}} &
     \multicolumn{1}{c}{$\boldsymbol{\mathcal{E}}_{\text{ID}} \downarrow$} &
     \multicolumn{1}{c|}{\textbf{Rank} $\uparrow$} &
     \multicolumn{9}{c}{$\boldsymbol{\mathcal{E}}_{\text{GEN}} \downarrow$} \\
    $\mathbf{\alpha}$ & IN & IN & IN-R & CIFAR & Flowers & NINCO & CUB & Aircrafts & Pets & STL & Avg. \\
    % \hline %\hline
    & 10 & 10 & 200 & 100 & 102 & 64 & 200 & 100 & 37 & 10 & \\ 
    \hline
    %0 & \textbf{90.80} & 2211.99 & 5.38 & 16.23 & 27.55 & 34.44 & 13.24 & 10.68 & 19.68 & 50.56 & 22.22 \\
    0 & \textbf{9.20} & 2211.99 & 94.62 & 83.77 & 72.45 & 65.56 & 86.76 & 89.32 & 80.32 & 49.44 & 77.78 \\
    %0.1 & 90.20 & 2964.39 & 9.28 & 24.42 & 42.06 & 49.91 & 20.04 & 15.87 & 27.28 & 60.21 & 31.13 \\
    0.1 & 9.80 & 2964.39 & 90.72 & 75.58 & 57.94 & 50.09 & 79.96 & 84.13 & 72.72 & 39.79 & 68.87 \\
    %0.2 & 89.80 & 3170.92 & 9.75 & 27.23 & 42.16 & 49.15 & 20.52 & 15.84 & 29.57 & \textbf{62.29} & 32.06 \\
    0.2 & 10.20 & 3170.92 & 90.25 & 72.77 & 57.84 & 50.85 & 79.48 & 84.16 & 70.43 & \textbf{37.71} & 67.94 \\
    %%0.3 & 89.20 & -- & 10.23 & 28.70 & 46.08 & 50.51 & 21.56 & 16.74 & 30.34 & 61.61 & 33.22 \\
    %0.6 & 88.00 & 3761.33 & \textbf{11.67} & 31.27 & \textbf{49.71} & 52.55 & \textbf{22.90} & 17.43 & \textbf{32.27} & 61.00 & 34.85 \\
    0.6 & 12.00 & 3761.33 & \textbf{88.33} & 68.73 & \textbf{50.29} & 47.45 & \textbf{77.10} & 82.57 & \textbf{67.73} & 39.00 & 65.15 \\
    %1.0 & 87.20 & \textbf{4815.32} & 11.62 & \textbf{32.19} & \textbf{49.71} & \textbf{52.89} & 22.52 & \textbf{18.36} & 31.73 & 61.26 & \textbf{35.04} \\
    1.0 & 12.80 & \textbf{4815.32} & 88.38 & \textbf{67.81} & \textbf{50.29} & \textbf{47.11} & 77.48 & \textbf{81.64} & 68.27 & 38.74 & \textbf{64.96} \\
    \hline %\hline
    %\vspace{-2em}
    \end{tabular}
    }
\end{table*}



\begin{figure*}[t]
    \centering
    \begin{subfigure}[b]{0.45\textwidth}
        \centering
        \includegraphics[width=\textwidth]{images/train_dynamics.png}
        \caption{Impact of Entropy Regularization on NC1}
        \label{fig:entropy_reg}
    \end{subfigure}
    %\hfill
    \begin{subfigure}[b]{0.45\textwidth}
        \centering
        \includegraphics[width=\textwidth]{images/train_dynamics_entropy.png}
        \caption{Entropy Dynamics}
        \label{fig:entropy-dynamics}
    \end{subfigure}
    %
    \begin{subfigure}[b]{0.45\textwidth}
        \centering
        \includegraphics[width=\textwidth]{images/train_dynamics_rank.png}
        \caption{Effective Rank Dynamics}
        \label{fig:effective-rank}
    \end{subfigure}
    %%
    \begin{subfigure}[b]{0.45\textwidth}
        \centering
        \includegraphics[width=\textwidth]{images/train_dynamics_l2.png}
        \caption{Impact of $L_2$ Normalization on NC1}
        \label{fig:l2_norm}
    \end{subfigure}
    
    \caption{\textbf{Analyzing entropy regularization \& $\mathbf{L_2}$ normalization.} 
    \textcolor{blue}{\textbf{(a)}} Entropy regularization reduces neural collapse (indicated by higher NC1 values) in the encoder. %, promoting OOD generalization. 
    \textcolor{blue}{\textbf{(b)}} Entropy regularization increases the entropy of encoder embeddings otherwise entropy remains unchanged.
    \textcolor{blue}{\textbf{(c)}} Entropy regularization increases the effective rank of encoder embeddings otherwise effective rank remains as low as the number of classes (i.e., 10 ImageNet classes).
    \textcolor{blue}{\textbf{(d)}} $L_2$ normalization increases neural collapse (indicated by lower NC1 values) in the projector. %, promoting OOD detection. 
    For this analysis, we train VGG17 networks on the ImageNet-10 subset (10 ImageNet classes) for 100 epochs.
    }
    \label{fig:nc_dynamics}
\end{figure*}




\section{Additional Experimental Results}
\label{sec:additional_exp_supp}

%%%%%%%%%%%%%%%%%%%%%

\subsection{Fixed ETF Projector Vs. Learnable Projector}

In Table~\ref{tab:plastic_proj}, we observe that the fixed ETF projector shows a higher transfer error (2.47\% absolute) than the plastic projector but outperforms the plastic projector in ID error (2.48\% absolute) and OOD detection error (8.9\% absolute). A fixed ETF projector should intensify NC and hinder OOD transfer and our fixed ETF projector fulfills this goal.


\begin{table*}[t]
\centering
  \caption{\textbf{ETF Fixed Projector Vs. Plastic Projector.} The VGG17 models are trained on \textbf{ImageNet-100} dataset (ID) and evaluated on 8 OOD datasets. The same color highlights the rows to compare.
  For OOD transfer we report $\boldsymbol{\mathcal{E}}_{\text{GEN}}$ (\%) whereas for OOD detection we report $\boldsymbol{\mathcal{E}}_{\text{DET}}$ (\%). %\textcolor{brown}{Fixed ETF projector shows higher transfer error (2.47\% absolute) than plastic projector but outperforms plastic projector in ID error (2.48\% absolute) and OOD detection error (8.9\% absolute).}
  } 
  \label{tab:plastic_proj}
  \centering
  \resizebox{\linewidth}{!}{
     \begin{tabular}{cc|cccc|ccccccccc}
     \hline %\hline
     \multicolumn{1}{c}{\textbf{Projector}} &
     \multicolumn{1}{c|}{$\boldsymbol{\mathcal{E}}_{\text{ID}} \downarrow$} &
     \multicolumn{4}{c|}{\textbf{Neural Collapse}} &
     \multicolumn{9}{c}{\textbf{OOD Datasets}} \\
    & IN & $\mathcal{NC}1$ & $\mathcal{NC}2$ & $\mathcal{NC}3$ & $\mathcal{NC}4$ & IN-R & CIFAR & Flowers & NINCO & CUB & Aircrafts & Pets & STL & Avg. \\
    & 100 &  &  &  &  & 200 & 100 & 102 & 64 & 200 & 100 & 37 & 10 & \\    
    \hline
    %% CE Loss
    \textbf{\textcolor{orange}{Transfer Error $\downarrow$}} \\
    %\rowcolor[gray]{0.9}
    \textcolor{blue}{\textbf{Plastic}} \\
    Projector & 15.10 & 0.498 & 0.515 & 0.428 & 1.422 & 87.52 & 64.83 & 79.71 & 53.32 & 87.00 & 93.46 & 48.76 & 28.04 & 67.83 \\

    \rowcolor{yellow!50}
    Encoder & 23.64 & 13.953 & 0.526 & 0.833 & 6.697 & \textbf{69.43} & \textbf{45.12} & \textbf{20.00} & \textbf{23.55} & \textbf{57.90} & \textbf{60.10} & 25.40 & 13.52 & \textbf{39.38} \\
    \hline %\hline

    %\rowcolor[gray]{0.9}
    \textcolor{blue}{\textbf{Fixed ETF}} \\

    Projector & \textbf{12.62} & 0.393 & 0.490 & 0.468 & 0.316 & 91.38 & 65.72 & 64.51 & 64.97 & 82.22 & 97.42 & 43.17 & 21.51 & 66.36 \\
    
    \rowcolor{yellow!50}
    %% ENCODER
    \textbf{Encoder} & 15.52 & 2.175 & 0.603 & 0.616 & 5.364 & 71.52 & 47.24 & 25.10 & 24.32 & 63.67 & 67.81 & \textbf{21.56} & 13.55 & 41.85 \\ % error

    \hline \hline
    \textbf{\textcolor{orange}{Detection Error $\downarrow$}} \\
    %\rowcolor[gray]{0.9}
    \textcolor{blue}{\textbf{Plastic}} \\
    \rowcolor{green!25}
    Projector & 15.10 & 0.498 & 0.515 & 0.428 & 1.422 & 63.05 & \textbf{47.87} & 62.45 & 70.07 & 80.88 & 98.95 & 89.37 & 79.25 & 74.00 \\
    
    Encoder & 23.64 & 13.953 & 0.526 & 0.833 & 6.697 & 81.27 & 98.82 & 93.33 & 86.48 & 79.98 & 99.40 & 91.25 & 93.88 & 90.55 \\
    \hline
    %\rowcolor[gray]{0.9}
    \textcolor{blue}{\textbf{Fixed ETF}} \\
    %% PROJECTOR
    \rowcolor{green!25}
    \textbf{Projector} & \textbf{12.62} & 0.393 & 0.490 & 0.468 & 0.316 & \textbf{60.85} & 48.23 & \textbf{42.35} & \textbf{67.69} & \textbf{56.51} & 99.04 & \textbf{76.32} & \textbf{69.84} & \textbf{65.10} \\
    %\hline
    
    %% ENCODER
    Encoder & 15.52 & 2.175 & 0.603 & 0.616 & 5.364 & 67.17 & 98.14 & 81.76 & 84.95 & 84.57 & 99.70 & 97.36 & 87.34 & 87.62 \\
    
    \hline \hline
    %\vspace{-2em}
    \end{tabular}}
\end{table*}








\begin{comment}

%%% Following results correspond to MSE Loss with Plastic Projector
\begin{table*}[t]
    

\centering
  \caption{\textbf{ETF Fixed Vs. Plastic Projector.} The VGGm-17 models ($F_{\psi}$($G_{\phi}$($H_{\theta}))$) are trained on \textbf{ImageNet-100} dataset (ID) and evaluated on 8 OOD datasets. In \textbf{encoder} method, the embeddings are extracted from the encoder ($H_{\theta}$) and before projector. And, in \textbf{projector} method, the embeddings are extracted after projector ($G_{\phi}$) and before output layer ($F_{\psi}$). For OOD transfer we report the top-1 error whereas for OOD detection we report the FPR95. $\uparrow$ indicates larger values are better and $\downarrow$ indicates smaller values are better. All values except neural collapse are percentages. \textbf{A lower $\mathcal{NC}$ indicates higher neural collapse. %$+\delta$ and $-\delta$ indicate \% increase and \% decrease respectively, when changing from encoder to projector.
  }
  } 
  \label{tab:plastic_proj}
  \centering
  \resizebox{\linewidth}{!}{
     \begin{tabular}{cc|ccc|ccccccccc}
     \hline %\hline
     \multicolumn{1}{c}{\textbf{Method}} &
     \multicolumn{1}{c|}{\textbf{ID Error}} &
     \multicolumn{3}{c|}{\textbf{Neural Collapse}} &
     \multicolumn{9}{c}{\textbf{OOD Datasets}} \\
    & IN & $\mathcal{NC}1$ &  $\mathcal{NC}2$ &  $\mathcal{NC}3$ & IN-R & CIFAR & Flower & NINCO & CUB & AirCrafts & Pet & STL & Avg. \\
    & 100 &  &  &  & 200 & 100 & 102 & 64 & 200 & 100 & 37 & 10 & \\    
    \hline \hline
    \textbf{Transfer Error $\downarrow$} \\
    Projector & \textbf{16.68} & 2.040 & 0.601 & 0.337 & 89.53 & 75.98 & 89.90 & 57.40 & 90.87 & 97.60 & 56.31 & 31.29 & 73.61 \\
     \hline
    \textbf{Encoder} & 19.42 & 3.969 & 0.552 & 0.705 & \textbf{77.72} & \textbf{56.18} & \textbf{36.08} & \textbf{29.93} & \textbf{65.69} & \textbf{73.27} & \textbf{28.59} & \textbf{16.55} & \textbf{48.00} \\ % error
    
    \hline \hline
    \textbf{Detection FPR $\downarrow$} \\
    \textbf{Projector} & \textbf{16.68} & 2.040 & 0.601 & 0.337  & \textbf{72.95} & \textbf{40.80} & \textbf{76.37} & \textbf{72.05} & \textbf{71.87} & \textbf{98.11} & \textbf{87.90} & \textbf{74.05} & \textbf{74.26} \\
    \hline
    Encoder & 19.42 & 3.969 & 0.552 & 0.705 & 98.25 & 99.95 & 88.63 & 97.06 & 98.24 & 87.55 & 97.33 & 98.90 & 95.74 \\
    \hline \hline
    %\vspace{-2em}
    \end{tabular}}
\end{table*}


\end{comment}


%%%%%%%%%%%%%%%%%%%%%

\subsection{Impact of $\mathbf{L_2}$ Normalization on NC}

\begin{table*}[t]
\centering
  \caption{\textbf{$L_2$ Normalization.} The VGG17 models are trained on \textbf{ImageNet-100} dataset (ID) and evaluated on 8 OOD datasets. The same color highlights the rows to compare. %\textcolor{brown}{$L_2$ normalization increases NC and improves OOD detection by 3.83\% (absolute).}
  For OOD detection, we report $\boldsymbol{\mathcal{E}}_{\text{DET}}$ (\%).
  } 
  \label{tab:l2_norm_nc}
  \centering
  \resizebox{\linewidth}{!}{
     \begin{tabular}{cc|cccc|ccccccccc}
     \hline %\hline
     \multicolumn{1}{c}{\textbf{Method}} &
     \multicolumn{1}{c|}{$\boldsymbol{\mathcal{E}}_{\text{ID}} \downarrow$} &
     \multicolumn{4}{c|}{\textbf{Neural Collapse} $\downarrow$} &
     %\multicolumn{9}{c}{\textbf{OOD Datasets} $\downarrow$} \\
     \multicolumn{9}{c}{$\boldsymbol{\mathcal{E}}_{\text{DET}} \downarrow$} \\
    & IN & $\mathcal{NC}1$ & $\mathcal{NC}2$ & $\mathcal{NC}3$ & $\mathcal{NC}4$ & IN-R & CIFAR & Flowers & NINCO & CUB & Aircrafts & Pets & STL & Avg. \\
    & 100 &  &  &  &  & 200 & 100 & 102 & 64 & 200 & 100 & 37 & 10 & \\    
    \toprule
    %% CE Loss
    %\hline %\hline
    %\textbf{\textcolor{orange}{Detection Error $\downarrow$}} \\
    \textcolor{blue}{\textbf{No $L_2$ Norm}} \\
    \rowcolor{green!25}
    Projector & 12.74 & 0.579 & 0.538 & \textbf{0.349} & 1.339 & \textbf{57.43} & 49.41 & 62.35 & 69.81 & 58.04 & 99.58 & 85.28 & 69.53 & 68.93 \\
    
    Encoder & 14.70 & 1.788 & 0.633 & 0.823 & 10.643 & 77.08 & 96.77 & 91.18 & 92.35 & 89.47 & 99.64 & 89.51 & 85.31 & 90.16 \\
    \hline
    %\rowcolor[gray]{0.9}
    \textcolor{blue}{\textbf{$L_2$ Norm}} \\
    %% PROJECTOR
    \rowcolor{green!25}
    \textbf{Projector} & \textbf{12.62} & \textbf{0.393} & \textbf{0.490} & 0.468 & \textbf{0.316} & 60.85 & \textbf{48.23} & \textbf{42.35} & \textbf{67.69} & \textbf{56.51} & 99.04 & \textbf{76.32} & 69.84 & \textbf{65.10} \\
    %\hline
    
    %% ENCODER
    Encoder & 15.52 & 2.175 & 0.603 & 0.616 & 5.364 & 67.17 & 98.14 & 81.76 & 84.95 & 84.57 & 99.70 & 97.36 & 87.34 & 87.62 \\
    
    %\hline 
    \bottomrule
    %\vspace{-2em}
    \end{tabular}}
\end{table*}


We verify whether $L_2$ normalization effectively induces more neural collapse and improves OOD detection.
We analyze two VGG17 models pre-trained on ImageNet-100 dataset where one model uses $L_2$ normalization and the other omits it.
The results are summarized in Table~\ref{tab:l2_norm_nc}.
We find that $L_2$ normalization induces more NC as evidenced by the lower NC1 value than its counterpart.
Consequently, $L_2$ normalization improves OOD detection by 3.83\% (absolute). Also, it achieves lower ID error than the compared model without $L_2$ normalization.

Next, we analyze how $L_2$ normalization impacts NC during training. We perform small-scale experiments since large-scale experiments are compute-intensive.
We train two VGG17 models on the ImageNet-10 (10 ImageNet classes) subset where one model uses $L_2$ normalization and another does not. During training, we measure the NC1 metric for the encoder embeddings.
The impact of $L_2$ normalization on NC1 is exhibited in Fig.~\ref{fig:l2_norm}. We find that $L_2$ normalization helps intensify NC during training. Consequently, it promotes better OOD detection.


%%%%%%%%%%%%%%%%%%%%%

\subsection{Batch Normalization Vs. Group Normalization}

\begin{table*}[t]
\centering
  \caption{\textbf{Batch Norm Vs. Group Norm.} VGG17 models are trained on \textbf{ImageNet-100} dataset (ID) and evaluated on 8 OOD datasets. The same color highlights the rows to compare.
  Group norm is integrated with weight standardization.
  All metrics except NC are reported in percentage. For OOD transfer we report $\boldsymbol{\mathcal{E}}_{\text{GEN}}$ (\%) whereas for OOD detection we report $\boldsymbol{\mathcal{E}}_{\text{DET}}$ (\%).
  %\textcolor{brown}{GroupNorm+WS outperforms BatchNorm by 10.11\% (absolute) in OOD transfer and by 4.37\% (absolute) in OOD detection.}
  } 
  \label{tab:bn_vs_gn}
  \centering
  \resizebox{\linewidth}{!}{
     \begin{tabular}{cc|cccc|ccccccccc}
     \hline %\hline
     \multicolumn{1}{c}{\textbf{Method}} &
     \multicolumn{1}{c|}{$\boldsymbol{\mathcal{E}}_{\text{ID}} \downarrow$} &
     \multicolumn{4}{c|}{\textbf{Neural Collapse}} &
     \multicolumn{9}{c}{\textbf{OOD Datasets}} \\
    & IN & $\mathcal{NC}1$ & $\mathcal{NC}2$ & $\mathcal{NC}3$ & $\mathcal{NC}4$ & IN-R & CIFAR & Flowers & NINCO & CUB & Aircrafts & Pets & STL & Avg. \\
    & 100 &  &  &  &  & 200 & 100 & 102 & 64 & 200 & 100 & 37 & 10 & \\    
    \hline
    %% CE Loss
    \textbf{\textcolor{orange}{Transfer Error $\downarrow$}} \\
    %\rowcolor[gray]{0.9}
    \textcolor{blue}{\textbf{Batch Norm}} \\
    Projector & 12.52 & 0.372 & 0.669 & 0.263 & 0.536 & 89.43 & 66.00 & 63.14 & 64.46 & 83.00 & 94.57 & 38.65 & 21.30 & 65.07 \\
    %\rowcolor[gray]{0.9}
    \rowcolor{yellow!50}
    Encoder & 14.54 & 1.401 & 0.605 & 0.590 & 25.611 & 78.02 & 53.34 & 49.51 & 33.25 & 74.08 & 85.27 & 25.46 & 16.75 & 51.96 \\
    
    \hline %\hline

    %\rowcolor[gray]{0.9}
    \textcolor{blue}{\textbf{Group Norm}} \\

    Projector & 12.62 & 0.393 & 0.490 & 0.468 & 0.316 & 91.38 & 65.72 & 64.51 & 64.97 & 82.22 & 97.42 & 43.17 & 21.51 & 66.36 \\
    
    \rowcolor{yellow!50}
    %% ENCODER
    \textbf{Encoder} & 15.52 & 2.175 & 0.603 & 0.616 & 5.364 & \textbf{71.52} & \textbf{47.24} & \textbf{25.10} & \textbf{24.32} & \textbf{63.67} & \textbf{67.81} & \textbf{21.56} & \textbf{13.55} & \textbf{41.85} \\ % err

    \hline \hline
    \textbf{\textcolor{orange}{Detection Error $\downarrow$}} \\
    %\rowcolor[gray]{0.9}
    \textcolor{blue}{\textbf{Batch Norm}} \\
    \rowcolor{green!25}
    Projector & 12.52 & 0.372 & 0.669 & 0.263 & 0.536 & \textbf{57.30} & 74.62 & 44.12 & \textbf{66.33} & 65.14 & 99.19 & 75.93 & 73.13 & 69.47 \\
    
    Encoder & 14.54 & 1.401 & 0.605 & 0.590 & 25.611 & 92.17 & 99.77 & 91.08 & 91.41 & 99.48 & 98.62 & 85.39 & 93.26 & 93.90 \\
    \hline
    %\rowcolor[gray]{0.9}
    \textcolor{blue}{\textbf{Group Norm}} \\
    %% PROJECTOR
    \rowcolor{green!25}
    \textbf{Projector} & 12.62 & 0.393 & 0.490 & 0.468 & 0.316 & 60.85 & \textbf{48.23} & \textbf{42.35} & 67.69 & \textbf{56.51} & 99.04 & 76.32 & \textbf{69.84} & \textbf{65.10} \\
    %\hline
    
    %% ENCODER
    Encoder & 15.52 & 2.175 & 0.603 & 0.616 & 5.364 & 67.17 & 98.14 & 81.76 & 84.95 & 84.57 & 99.70 & 97.36 & 87.34 & 87.62 \\
    
    %\hline \hline
    %\vspace{-2em}
    \bottomrule
    \end{tabular}}
\end{table*}

We find that group normalization (combined with weight standardization) outperforms batch normalization by 10.11\% (absolute) in OOD transfer and by 4.37\% (absolute) in OOD detection (see Table~\ref{tab:bn_vs_gn}).
Group normalization achieves a higher $\mathcal{NC}1$ value than batch normalization, thereby mitigating NC and enhancing OOD generalization. Group normalization also achieves ID performance similar to that of batch normalization.
Our results demonstrate that group normalization achieves competitive performance and plays a crucial role in OOD generalization.


%%%%%%%%%%%%%%%%%%%%%

\subsection{Comparison with Baseline}

\begin{table*}[t]
\centering
  \caption{\textbf{Comprehensive Comparison with Baseline.} Various DNNs are trained on \textbf{ImageNet-100} dataset (ID) and evaluated on 8 OOD datasets. The regular models e.g., VGG17, ResNet18, and ViT-T do not use mechanisms e.g., regularization loss or a projector to control NC. For OOD transfer we report $\boldsymbol{\mathcal{E}}_{\text{GEN}}$ (\%) whereas for OOD detection we report $\boldsymbol{\mathcal{E}}_{\text{DET}}$ (\%).%\textcolor{brown}{Our model significantly outperforms the baseline in OOD transfer and OOD detection across all DNN architectures.}
  } 
  \label{tab:base_model}
  \centering
  \resizebox{\linewidth}{!}{
     \begin{tabular}{cc|cccc|ccccccccc}
     \hline %\hline
     \multicolumn{1}{c}{\textbf{Model}} &
     \multicolumn{1}{c|}{$\boldsymbol{\mathcal{E}}_{\text{ID}} \downarrow$} &
     \multicolumn{4}{c|}{\textbf{Neural Collapse}} &
     \multicolumn{9}{c}{\textbf{OOD Datasets}} \\
    & IN & $\mathcal{NC}1$ & $\mathcal{NC}2$ & $\mathcal{NC}3$ & $\mathcal{NC}4$ & IN-R & CIFAR & Flowers & NINCO & CUB & Aircrafts & Pets & STL & Avg. \\
    & 100 &  &  &  &  & 200 & 100 & 102 & 64 & 200 & 100 & 37 & 10 & \\    
    %\hline
    \toprule
    %% CE Loss
    %\textbf{\textcolor{blue}{VGG17}} \\
    \textbf{\textcolor{orange}{Transfer Error $\downarrow$}} \\
    VGG17 & 12.18 & 0.766 & 0.705 & 0.486 & 37.491 & 75.60 & 50.11 & 42.75 & 29.17 & 71.35 & 84.13 & 27.58 & 15.65 & 49.54 \\
    
    \rowcolor{yellow!50}
    %% ENCODER
    \textbf{VGG17+Ours} & 12.62 & 0.393 & 0.490 & 0.468 & 0.316 & \textbf{71.52} & \textbf{47.24} & \textbf{25.10} & \textbf{24.32} & \textbf{63.67} & \textbf{67.81} & \textbf{21.56} & \textbf{13.55} & \textbf{41.85} \\ % err

    \hline %\hline
    \textbf{\textcolor{orange}{Detection Error $\downarrow$}} \\
    VGG17 & 12.18 & 0.766 & 0.705 & 0.486 & 37.491 & 96.02 & 97.16 & 97.94 & 93.11 & 95.19 & 98.59 & 87.33 & 94.05 & 94.92 \\

    \rowcolor{yellow!50}
    %% Projector
    \textbf{VGG17+Ours} & 12.62 & 0.393 & 0.490 & 0.468 & 0.316 & \textbf{60.85} & \textbf{48.23} & \textbf{42.35} & \textbf{67.69} & \textbf{56.51} & 99.04 & \textbf{76.32} & \textbf{69.84} & \textbf{65.10} \\
    
    \hline \hline
    %\midrule
    %% RESNET18
    %\textbf{\textcolor{blue}{ResNet18}} \\
    \textbf{\textcolor{orange}{Transfer Error $\downarrow$}} \\
    ResNet18 & 15.38 & 1.11 & 0.658 & 0.590 & 31.446 & 75.75 & \textbf{49.48} & 41.37 & 30.02 & 69.80 & 82.75 & 29.63 & 16.53 & 49.42 \\

    %% ENCODER
    \rowcolor{yellow!50}
    \textbf{ResNet18+Ours} & 16.14 & 0.341 & 0.456 & 0.306 & 0.540 & \textbf{74.17} & 53.33 & \textbf{31.37} & \textbf{28.15} & \textbf{68.85} & \textbf{81.61} & \textbf{27.72} & 16.56 & \textbf{47.72} \\

    \hline %\hline
    \textbf{\textcolor{orange}{Detection Error $\downarrow$}} \\
    ResNet18 & 15.38 & 1.11 & 0.658 & 0.590 & 31.446 & 98.40 & 98.85 & 98.33 & 96.68 & 96.60 & 99.67 & 92.40 & 98.25 & 97.40 \\

    %% PROJECTOR
    \rowcolor{yellow!50}
    \textbf{ResNet18+Ours} & 16.14 & 0.341 & 0.456 & 0.306 & 0.540 & \textbf{67.92} & \textbf{61.21} & \textbf{71.18} & \textbf{71.09} & \textbf{23.20} & \textbf{99.28} & \textbf{81.41} & \textbf{82.29} & \textbf{69.70} \\
    \hline \hline

    
    %% VIT-Tiny
    %\textbf{\textcolor{blue}{ViT-T}} \\
    \textbf{\textcolor{orange}{Transfer Error $\downarrow$}} \\
    ViT-T & 31.78 & 2.467 & 0.657 & 0.601 & 1.015 & 82.18 & 52.64 & 41.67 & 32.74 & 63.48 & 81.61 & 45.11 & 22.00 & 52.68 \\

    %% ENCODER
    \rowcolor{yellow!50}
    \textbf{ViT-T+Ours} & 32.04 & 2.748 & 0.609 & 0.798 & 1.144 & 82.28 & \textbf{52.00} & \textbf{42.94} & \textbf{30.36} & \textbf{63.15} & 84.31 & \textbf{44.86} & \textbf{21.13} & \textbf{52.63} \\

    \hline %\hline
    \textbf{\textcolor{orange}{Detection Error $\downarrow$}} \\
    ViT-T & 31.78 & 2.467 & 0.657 & 0.601 & 1.015 & 85.18 & 91.70 & 87.06 & 89.54 & 87.78 & \textbf{98.35} & \textbf{91.77} & 89.99 & 90.17 \\

    %% PROJECTOR
    \rowcolor{yellow!50}
    \textbf{ViT-T+Ours} & 32.04 & 2.748 & 0.609 & 0.798 & 1.144 & \textbf{81.12} & \textbf{60.81} & \textbf{77.55} & \textbf{82.40} & \textbf{79.05} & 99.10 & \textbf{95.15} & 90.06 & \textbf{83.16} \\
    
    \bottomrule
    %\vspace{-2em}
    \end{tabular}}
\end{table*}

Our experimental results show that our method significantly improves OOD detection and OOD transfer performance across all DNN architectures. We summarize the results in Table~\ref{tab:base_model}. We evaluate VGG17, ResNet18, and ViT-T baselines on 8 OOD datasets and compare them with our models.
The absolute improvements over VGG17 baseline are 7.69\% (OOD generalization) and 
29.82\% (OOD detection). Similarly, our method outperforms other DNNs in all criteria.
Our results corroborate our argument that \emph{controlling NC enables good OOD detection and OOD generalization performance}. It is also evident that a single feature space cannot simultaneously achieve both OOD detection and OOD generalization abilities.



%%%%%%%%%%%%%%%%%%%%%

\subsection{Projector Design Criteria}

Here we study the design choices of the projector network. We want to know how depth and width impact the performance. For this, we examine projectors consisting of a single layer (\textit{depth=1}, $512d$), two layers (\textit{depth=2}, $512d \rightarrow 2048d \rightarrow 512d$), three layers (\textit{depth=3}, $512d \rightarrow 2048d \rightarrow 2048d \rightarrow 512d$),
and a wider variant (\textit{width=2}, $512d \rightarrow 4096d \rightarrow 512d$). All of these variants are trained in identical settings and only the projector is changed. We train VGG17 networks on ImageNet-100 dataset (ID) and evaluate OOD detection/generalization on 8 OOD datasets.
The results are shown in Table~\ref{tab:proj_design}. The projector with depth 2 outperforms other variants across all evaluations.


\begin{table*}[t]
\centering
  \caption{\textbf{Projector Design Criteria.} The VGG17 models are trained on \textbf{ImageNet-100} dataset (ID) and evaluated on 8 OOD datasets. The same color highlights the rows to compare. All compared projectors are configured as fixed simplex ETFs. Our final model has depth 2 and performs better than other variants. All metrics except NC are in percentage. For OOD transfer we report $\boldsymbol{\mathcal{E}}_{\text{GEN}}$ (\%) whereas for OOD detection we report $\boldsymbol{\mathcal{E}}_{\text{DET}}$ (\%).
  } 
  \label{tab:proj_design}
  \centering
  \resizebox{\linewidth}{!}{
     \begin{tabular}{cc|cccc|ccccccccc}
     \hline %\hline
     \multicolumn{1}{c}{\textbf{Criteria}} &
     \multicolumn{1}{c|}{$\boldsymbol{\mathcal{E}}_{\text{ID}} \downarrow$} &
     \multicolumn{4}{c|}{\textbf{Neural Collapse}} &
     \multicolumn{9}{c}{\textbf{OOD Datasets}} \\
    & IN & $\mathcal{NC}1$ & $\mathcal{NC}2$ & $\mathcal{NC}3$ & $\mathcal{NC}4$ & IN-R & CIFAR & Flowers & NINCO & CUB & Aircrafts & Pets & STL & Avg. \\
    & 100 &  &  &  &  & 200 & 100 & 102 & 64 & 200 & 100 & 37 & 10 & \\    
    \hline
    %% CE Loss
    \textbf{\textcolor{orange}{Transfer Error $\downarrow$}} \\
    %\rowcolor[gray]{0.9}
    \textcolor{blue}{\textbf{Depth=1}} \\
    %Projector & -- & -- & -- & -- & -- & -- & -- & -- & -- & -- & -- & -- & -- & -- \\

    Projector & 12.86 & 0.375 & 0.649 & 0.500 & 1.157 & 90.27 & 64.61 & 60.88 & 55.02 & 81.12 & 96.34 & 44.34 & 23.04 & 64.45 \\

    \rowcolor{yellow!50}
    Encoder & 16.34 & 1.673 & 0.667 & 0.589 & 7.936 & 74.08 & 50.61 & 30.00 & 28.06 & 66.74 & 71.95 & 25.73 & 15.75 & 45.37 \\
    \hline %\hline

    %\rowcolor[gray]{0.9}
    \textcolor{blue}{\textbf{Depth=2 (Ours)}} \\

    Projector & \textbf{12.62} & 0.393 & 0.490 & 0.468 & 0.316 & 91.38 & 65.72 & 64.51 & 64.97 & 82.22 & 97.42 & 43.17 & 21.51 & 66.36 \\
    
    \rowcolor{yellow!50}
    %% ENCODER
    \textbf{Encoder} & 15.52 & 2.175 & 0.603 & 0.616 & 5.364 & \textbf{71.52} & \textbf{47.24} & \textbf{25.10} & 24.32 & 63.67 & 67.81 & \textbf{21.56} & \textbf{13.55} & \textbf{41.85} \\ % error

    %\hline
    %textcolor{blue}{\textbf{Depth=3}} \\
    %Projector & 12.88 & 0.323 & 0.667 & 0.560 & 1.138 & 90.83 & 66.01 & 63.53 & 59.10 & 81.95 & 96.28 & 43.50 & 23.05 & 65.53 \\

    %\rowcolor{yellow!50}
    %Encoder & 16.00 & 2.623 & 0.609 & 0.638 & 5.290 & 72.47 & 49.03 & 26.67 & \textbf{23.64} & \textbf{61.34} & \textbf{67.03} & 23.74 & 14.96 & 42.36 \\

    \hline
    \textcolor{blue}{\textbf{Width=2}} \\
    Projector & 13.48 & 0.320 & 0.667 & 0.376 & 0.493 & 89.88 & 66.46 & 64.51 & 53.40 & 82.50 & 95.77 & 41.76 & 23.90 & 64.77 \\

    \rowcolor{yellow!50}
    Encoder & 16.46 & 2.341 & 0.607 & 0.646 & 5.899 & 73.05 & 50.61 & 27.25 & 25.60 & 64.84 & 67.87 & 22.35 & 15.10 & 43.33 \\

    \hline \hline
    \textbf{\textcolor{orange}{Detection Error $\downarrow$}} \\
    %\rowcolor[gray]{0.9}
    \textcolor{blue}{\textbf{Depth=1}} \\
    \rowcolor{green!25}
    
    Projector & 12.86 & 0.375 & 0.649 & 0.500 & 1.157 & 80.15 & 95.98 & 81.68 & 84.18 & 92.75 & \textbf{98.38} & \textbf{73.62} & 92.24 & 87.37 \\
    
    Encoder & 16.34 & 1.673 & 0.667 & 0.589 & 7.936 & 62.72 & 95.04 & 84.65 & 84.95 & 92.22 & 99.43 & 89.75 & 83.66 & 86.55 \\
    
    \hline
    %\rowcolor[gray]{0.9}
    \textcolor{blue}{\textbf{Depth=2 (Ours)}} \\
    %% PROJECTOR
    \rowcolor{green!25}
    \textbf{Projector} & \textbf{12.62} & 0.393 & 0.490 & 0.468 & 0.316 & \textbf{60.85} & \textbf{48.23} & \textbf{42.35} & \textbf{67.69} & \textbf{56.51} & 99.04 & 76.32 & \textbf{69.84} & \textbf{65.10} \\
    %\hline
    
    %% ENCODER
    Encoder & 15.52 & 2.175 & 0.603 & 0.616 & 5.364 & 67.17 & 98.14 & 81.76 & 84.95 & 84.57 & 99.70 & 97.36 & 87.34 & 87.62 \\

    %\hline
    %\textcolor{blue}{\textbf{Depth=3}} \\
    %\rowcolor{green!25}
    %Projector & 12.88 & 0.323 & 0.667 & 0.560 & 1.138 & 74.30 & 92.23 & 68.63 & 81.21 & 84.00 & 98.59 & 69.61 & 90.59 & 82.40 \\

    %Encoder & 16.00 & 2.623 & 0.609 & 0.638 & 5.290 & 77.18 & 99.86 & 94.12 & 86.57 & 83.41 & 99.52 & 94.79 & 89.90 & 90.67 \\

    \hline
    \textcolor{blue}{\textbf{Width=2}} \\
    \rowcolor{green!25}
    Projector & 13.48 & 0.320 & 0.667 & 0.376 & 0.493 & 65.43 & 60.83 & 51.96 & 67.77 & 57.70 & 99.52 & 79.29 & 75.33 & 69.73 \\
    
    Encoder & 16.46 & 2.341 & 0.607 & 0.646 & 5.899 & 66.80 & 97.64 & 89.61 & 83.42 & 88.89 & 98.89 & 98.58 & 94.39 & 89.78 \\
    
    %\hline \hline
    \bottomrule
    %\vspace{-2em}
    \end{tabular}}
\end{table*}



%\subsection{Additional ID dataset}
%% TBA


%%%%%%%%%%%%%%%%%%%%%

\subsection{Fixed ETF Classifier Vs. Plastic Classifier}

\begin{table*}[t]
\centering
  \caption{\textbf{Fixed ETF Classifier Vs. Plastic Classifier.} The VGG17 models are trained on \textbf{ImageNet-100} dataset (ID) and evaluated on 8 OOD datasets. 
  The same color highlights the rows to compare. All metrics except NC are reported in percentage. For OOD transfer we report $\boldsymbol{\mathcal{E}}_{\text{GEN}}$ (\%) whereas for OOD detection we report $\boldsymbol{\mathcal{E}}_{\text{DET}}$ (\%).
  } 
  \label{tab:fixed_vs_plastic_head}
  \centering
  \resizebox{\linewidth}{!}{
     \begin{tabular}{cc|cccc|ccccccccc}
     \hline %\hline
     \multicolumn{1}{c}{\textbf{Classifier}} &
     \multicolumn{1}{c|}{$\boldsymbol{\mathcal{E}}_{\text{ID}} \downarrow$} &
     \multicolumn{4}{c|}{\textbf{Neural Collapse}} &
     \multicolumn{9}{c}{\textbf{OOD Datasets}} \\
    & IN & $\mathcal{NC}1$ & $\mathcal{NC}2$ & $\mathcal{NC}3$ & $\mathcal{NC}4$ & IN-R & CIFAR & Flowers & NINCO & CUB & Aircrafts & Pets & STL & Avg. \\
    & 100 &  &  &  &  & 200 & 100 & 102 & 64 & 200 & 100 & 37 & 10 & \\    
    \hline
    %% CE Loss
    \textbf{\textcolor{orange}{Transfer Error $\downarrow$}} \\
    %\rowcolor[gray]{0.9}
    \textcolor{blue}{\textbf{Fixed ETF}} \\
    Projector & 13.56 & 0.088 & 0.702 & 0.374 & 0.379 & 98.18 & 84.28 & 92.25 & 96.94 & 96.86 & 97.60 & 72.23 & 36.59 & 84.37 \\

    \rowcolor{yellow!50}
    Encoder & 16.40 & 3.794 & 0.773 & 0.786 & 54.24 & 82.47 & 63.19 & 55.98 & 36.31 & 81.00 & 88.36 & 31.18 & 20.88 & 57.42 \\
    \hline %\hline

    %\rowcolor[gray]{0.9}
    \textcolor{blue}{\textbf{Plastic (Ours)}} \\

    Projector & \textbf{12.62} & 0.393 & 0.490 & 0.468 & 0.316 & 91.38 & 65.72 & 64.51 & 64.97 & 82.22 & 97.42 & 43.17 & 21.51 & 66.36 \\
    
    \rowcolor{yellow!50}
    %% ENCODER
    \textbf{Encoder} & 15.52 & 2.175 & 0.603 & 0.616 & 5.364 & \textbf{71.52} & \textbf{47.24} & \textbf{25.10} & \textbf{24.32} & \textbf{63.67} & \textbf{67.81} & \textbf{21.56} & \textbf{13.55} & \textbf{41.85} \\ % error

    \hline \hline
    \textbf{\textcolor{orange}{Detection Error $\downarrow$}} \\
    %\rowcolor[gray]{0.9}
    \textcolor{blue}{\textbf{Fixed ETF}} \\
    \rowcolor{green!25}
    
    Projector & 13.56 & 0.088 & 0.702 & 0.374 & 0.379 & 73.80 & \textbf{26.45} & 73.04 & 68.20 & \textbf{55.80} & 98.98 & 96.05 & \textbf{63.56} & 69.49 \\
    
    Encoder & 16.40 & 3.794 & 0.773 & 0.786 & 54.24 & 81.03 & 98.98 & 81.57 & 87.25 & 97.29 & 99.01 & 86.48 & 93.11 & 90.59 \\

    \hline
    %\rowcolor[gray]{0.9}
    \textcolor{blue}{\textbf{Plastic (Ours)}} \\
    %% PROJECTOR
    \rowcolor{green!25}
    \textbf{Projector} & \textbf{12.62} & 0.393 & 0.490 & 0.468 & 0.316 & \textbf{60.85} & 48.23 & \textbf{42.35} & \textbf{67.69} & 56.51 & 99.04 & \textbf{76.32} & 69.84 & \textbf{65.10} \\
    %\hline
    
    %% ENCODER
    Encoder & 15.52 & 2.175 & 0.603 & 0.616 & 5.364 & 67.17 & 98.14 & 81.76 & 84.95 & 84.57 & 99.70 & 97.36 & 87.34 & 87.62 \\
    
    %\hline \hline
    \bottomrule
    %\vspace{-2em}
    \end{tabular}}
\end{table*}
%\begin{table}[t]
\centering
  \caption{\textbf{Plastic Vs. Fixed ETF Classifier.} The evaluation is based on ImageNet-100 pre-trained VGG17 network with a plastic classifier or a fixed ETF classifier. NC values correspond to projector embeddings. OOD-Error and FPR are averaged over 8 OOD datasets.}
  \label{tab:etf_clf_results}
  \centering
  \resizebox{\linewidth}{!}{
     \begin{tabular}{c|c|cccc|c|c}
     \hline %\hline
     \multicolumn{1}{c|}{\textbf{Classifier}} &
     \multicolumn{1}{c|}{\textbf{ID-Err}} &
     \multicolumn{4}{c|}{\textbf{Neural Collapse}} &
     \multicolumn{1}{c|}{\textbf{OOD-Err}} &
     \multicolumn{1}{c}{\textbf{FPR}} \\
     & $\downarrow$ & $\mathcal{NC}1$ & $\mathcal{NC}2$ & $\mathcal{NC}3$ & $\mathcal{NC}4$ & Avg. $\downarrow$ & Avg. $\downarrow$ \\
    %\hline \hline
    \toprule
    Fixed ETF & 13.56 & 0.088 & 0.702 & 0.374 & 0.379 & 57.42 & 69.49 \\
    %\hline
    
    \rowcolor{yellow!50}
    \textbf{Plastic (Ours)} & \textbf{12.62} & 0.393 & 0.490 & 0.468 & 0.316 & \textbf{41.85} & \textbf{65.10} \\
    %\hline \hline
    \bottomrule
    \end{tabular}}
\end{table}

We investigate how using a fixed ETF classifier head impacts NC and OOD detection/generalization performance.
We train two identical models consisting of our proposed mechanisms to control NC, the only thing we vary is the classifier head. One model consists of a plastic (learnable) classifier head which is our proposed model and the other consists of an ETF classifier head. The ETF classifier head is configured with Simplex ETF and frozen during training. We train VGG17 networks on ImageNet-100 (ID) and evaluate them on 8 OOD datasets.

%Table~\ref{tab:etf_clf_results} 
Table~\ref{tab:fixed_vs_plastic_head} shows results across all OOD datasets, where the plastic classifier outperforms the fixed ETF classifier by 4.39\% (absolute) in OOD detection and by 15.6\% in OOD generalization. The plastic classifier also outperforms ETF classifier in the ID task. 
%Table~\ref{tab:fixed_vs_plastic_head} shows detailed results for each OOD dataset. 
Our results suggest that imposing NC in the classifier head is sub-optimal for enhancing OOD detection and generalization.


%%%%%%%%%%%%%%%%%%%%%


%\section{Some Research Questions}
%\begin{enumerate}
%    \item Does a fixed classifier become a better OOD detector and classifier than a learned one?
%\end{enumerate}


%%%%%%%%%%%%%%%%%%%%%
%\newpage

\section{Classes of ImageNet-100 ID Dataset}
\label{sec:imagenet_100_classes}

We list the 100 classes in the ID dataset, ImageNet-100~\cite{tian2020contrastive}. 
This list can also be found at: \url{https://github.com/HobbitLong/CMC/blob/master/imagenet100.txt}


\textit{Rocking chair, pirate, computer keyboard, Rottweiler, Great Dane, tile roof, harmonica, langur, Gila monster, hognose snake, vacuum, Doberman, laptop, gasmask, mixing bowl, robin, throne, chime, bonnet, komondor, jean, moped, tub, rotisserie, African hunting dog, kuvasz, stretcher, garden spider, theater curtain, honeycomb, garter snake, wild boar, pedestal, bassinet, pickup, American lobster, sarong, mousetrap, coyote, hard disc, chocolate sauce, slide rule, wing, cauliflower, American Staffordshire terrier, meerkat, Chihuahua, lorikeet, bannister, tripod, head cabbage, stinkhorn, rock crab, papillon, park bench, reel, toy terrier, obelisk, walking stick, cocktail shaker, standard poodle, cinema, carbonara, red fox, little blue heron, gyromitra, Dutch oven, hare, dung beetle, iron, bottlecap, lampshade, mortarboard, purse, boathouse, ambulance, milk can, Mexican hairless, goose, boxer, gibbon, football helmet, car wheel, Shih-Tzu, Saluki, window screen, English foxhound, American coot, Walker hound, modem, vizsla, green mamba, pineapple, safety pin, borzoi, tabby, fiddler crab, leafhopper, Chesapeake Bay retriever, and ski mask.}


\begin{comment}

\textbf{Is there any semantic class overlap between ID and OOD datasets?}
There is no semantic class overlap between ImageNet-100 (ID dataset) and 8 other OOD datasets e.g., CIFAR-10, CIFAR-100, NINCO-64, CUB-200, Aircrafts-100, Oxford Pets-37, Flowers-102, and STL-10. 

Only ImageNet-R (consisting of 200 classes) has 19 classes that overlap with ImageNet-100. 
This is expected and we know that ImageNet-R includes classes from ImageNet-1K dataset but incorporates significant distribution shifts using artistic renditions.
The overlapping classes are:
\textit{Gasmask, American lobster, Standard poodle, Red fox, Head cabbage, Harmonica, Ambulance, Gibbon, Pineapple, Chihuahua, Tabby, Pirate, Rottweiler, Lorikeet, Boxer, Pickup, Goose, Shih-Tzu, and Meerkat.}

\end{comment}



\end{document}

% This document was modified from the file originally made available by
% Pat Langley and Andrea Danyluk for ICML-2K. This version was created
% by Iain Murray in 2018, and modified by Alexandre Bouchard in
% 2019 and 2021 and by Csaba Szepesvari, Gang Niu and Sivan Sabato in 2022.
% Modified again in 2023 by Sivan Sabato and Jonathan Scarlett.
% Previous contributors include Dan Roy, Lise Getoor and Tobias
% Scheffer, which was slightly modified from the 2010 version by
% Thorsten Joachims & Johannes Fuernkranz, slightly modified from the
% 2009 version by Kiri Wagstaff and Sam Roweis's 2008 version, which is
% slightly modified from Prasad Tadepalli's 2007 version which is a
% lightly changed version of the previous year's version by Andrew
% Moore, which was in turn edited from those of Kristian Kersting and
% Codrina Lauth. Alex Smola contributed to the algorithmic style files.
