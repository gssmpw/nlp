\section{Related Works}
\label{sec:rltd}
Over the last few years, authors have devoted their efforts to study and evaluate performance and dependability attributes related to blockchain and cloud computing technologies, in this section, we present some of these remarkable works. To the best of our knowledge, these are the closest related works to what is proposed by the current paper.

Previously in Melo et al., \cite{melo2018dependability,melo2019}, we have evaluated a blockchain-as-a-service infrastructure considering Dynamical Reliability Block Diagrams (DRBD) and a set of architectures containing the minimum requirements to deploy a Hyperledger platform over a virtualized environment managed by docker and OpenStack cloud computing platform. The current work is an extension of these two, and are the closest to what is proposed here, which evaluates the availability, capacity-oriented availability, and deployment expenses of an Ethereum private environment. Also, the proposed models here are hierarchically distributed, both Reliability Block Diagrams (RBDs) and Stochastic Petri Nets (SPN) are used in order to represent priority repairs between the nodes on Ethereum environment, we have also used analytical models provided from RBD's K-out-of-N (KooN) blocks to calculate the capacity-oriented availability \cite{paulo2011,MartinsMaciel2016}.

Pongnumkul et al. 2017 \cite{pongnumkul2017performance} proposed a performance evaluation methodology for blockchain environments, and evaluated the Ethereum and Hyperledger Fabric platforms. The authors concluded that the Hyperledger Fabric has a large throughput and lower latency than Ethereum, which is a clear result of the consensus protocol. Ethereum uses Proof-of-Work (PoW), while the Hyperledger Fabric worked with the Practical Byzantine Fault Tolerance (PBFT) protocol, but now the current versions of the Hyperledger Fabric platform work with Kafka.

Zhang et al., 2018 \cite{zhang2018method} proposed a method to predict the performance of blockchain environments based on analytical modeling; they used the Ethereum platform to demonstrate the feasibility of the proposed methodology. The associated metrics evaluated were transactions per second, and the storage space required to store the performed transactions. 

In Sukhwani et al., 2017  \cite{sukhwani2017performance}, the authors investigated whether a consensus process using the PBFT mechanism could be a performance bottleneck through Stochastic Reward Nets (SRN) modeling to compute the meantime to complete the consensus process. The PBFT protocol is one of the most used by blockchain platforms. The focus of the current paper is on the availability evaluation of the Ethereum platform. 

Later, in Sukhwani et al., 2018 \cite{sukhwani2018performance}, the authors evaluated the performance of Hyperledger Fabric through SRN models, which means that this works the entire platform performance, not only the consensus protocol. Once again, the proposed work differs from ours in the platform and the evaluated metrics. 

In Weber et al., 2017 \cite{weber2017}, the authors identified some availability limitations of two blockchains, Ethereum and Bitcoin, by measuring and gathering public data. We emphasized the evaluation of an Ethereum private blockchain availability, as well as the current acquisition and maintenance expenses, and the capacity-oriented availability, by proposing and evaluating the availability and capacity-oriented availability models. 

Roehrs et al. 2019 \cite{roehrs2019}, the authors proposed the omniPHR blockchain platform, focused on securely sharing personal health information. The authors evaluated through mathematical models and measurement the availability and performance of the proposed platform, considering besides the overall system's availability a set of performance metrics such as transactions per second, CPU utilization, memory, and network utilization. 

Aniello et al. 2017 \cite{aniello2017} have also proposed a new blockchain platform and evaluated scalability, availability, security, and performance metrics such as transactions per second and response time by considering this platform as an alternative for the traditional distributed database.

In Li et al., 2018 \cite{li2018blockchain}, the authors proposed a markovian process to express the relationship between the transaction arrival rate and the server capacity. While in \cite{torquato2175models} the authors proposed models to evaluated capacity-oriented availability (COA) of general cloud computing environments by considering Virtual Machines as their resources, considering failure and repair routines and their impact over the metric. 

As could be seen from previously presented studies, most of the work accomplished focused on performance metrics. At the same time, in the proposed paper, we consider availability models and COA's evaluation, as well as the expenses related to service provisioning in both public and private environments, considering docker containers and compute nodes as resources and performing a sensitivity analysis to detect bottlenecks and the components with the highest impact over the availability metric.
%-------------------------------------