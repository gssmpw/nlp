\documentclass[journal,onecolumn,10pt]{IEEEtran}
%\usepackage{indentfirst}
\usepackage{color}
%\pagestyle{headings}

%\usepackage{ulem}
\usepackage{amsfonts,amsmath,amssymb,amsthm}
\usepackage{latexsym,amscd}
\usepackage{amsbsy}
\usepackage{graphicx,multirow,bm}
\usepackage{algorithm}
\usepackage{algorithmicx}
\usepackage{algpseudocode}
\usepackage{xcolor}
\usepackage{tikz}
\usepackage{diagbox}
\usepackage{multicol}
\usepackage{arydshln}
\usepackage{booktabs}
%\usepackage{amsfonts}
%\usepackage{slashbox}
\usepackage{times}
\usepackage{amsmath}
\usepackage{cases}
\usepackage{subfigure}
%
\newcounter{coun}
\newtheorem{construction}{construction}
\newtheorem{Theorem}{Theorem}
\newtheorem{Corollary}{Corollary}
\newtheorem{Definition}{Definition}
\newtheorem{Example}{Example}
\newtheorem{Property}{Property}
\newtheorem{Requirement}{Requirement}
\newtheorem{Assumption}{Assumption}
\newtheorem{Remark}{Remark}
\newtheorem{Lemma}{Lemma}
\newtheorem{Algorithm}{Algorithm}
\newtheorem{Construction}{Construction}
\newtheorem{Proposition}{Proposition}
\newtheorem{Condition}{Condition}
\newtheorem{Fact}{Fact}
\newcommand{\C}{\mathbb{C}}
\newcommand{\N}{\mathbb{N}}
\newcommand{\Q}{\mathbb{Q}}
\newcommand{\R}{\mathbb{R}}
\newcommand{\Z}{\mathbb{Z}}
\newcommand{\F}{\mathbb{F}}
\newcommand{\bS}{\mathbb{S}}
\renewcommand\baselinestretch{1.3}
%\parskip 5pt
%\setlength{\oddsidemargin}{0cm} \setlength{\evensidemargin}{0cm}
%\setlength{\textwidth}{16cm} \setlength{\textheight}{21cm}
\renewcommand\thetable{\Roman{table}}
%

\usepackage{pgf}
\usepackage{tikz}
\usetikzlibrary{matrix,chains,positioning,arrows,calc,decorations,plotmarks,patterns,trees,fit,backgrounds,shapes.multipart,topaths}
\usepackage{pgfplots}
\usetikzlibrary{external}                       % load externalization library
%\tikzexternalize                                % activate externalization!
%\tikzsetexternalprefix{tikzpic/}                   % selecte directory for externalized figures
% Needed as workaround. Normal system call uses -halt-on-error,
% this does not produce any output in our case. Why?
%\tikzset{external/system call={pdflatex \tikzexternalcheckshellescape -interaction=batchmode -jobname "\image" "\texsource"}}
%\tikzifexternalizing{}{                        % avoid TOC and links in externalized figures
%  \usepackage[pagebackref=true]{hyperref}
%  \usepackage[all]{hypcap}
%}
\usepackage{scalefnt}
\usepackage{tkz-graph}
\pgfplotsset{compat=1.3}
\tikzstyle{help lines}=[black!20,dashed]
%\pagecolor[rgb]{0.78 0.93 0.8}

\begin{document}
	
	\title{A Cooperative Repair Scheme for  Zigzag MSR Codes in Distributed Storage Systems}
	
		\author{Yajuan Liu, Han Cai,~\IEEEmembership{Member,~IEEE}, and Xiaohu Tang, \IEEEmembership{Senior Member, IEEE}
			\thanks{
				Y. Liu, H. Cai, and X. Tang are with the Information Security and National Computing Grid Laboratory, Southwest Jiaotong University, Chengdu, China (email: yjliu@my.swjtu.edu.cn, hancai@swjtu.edu.cn, xhutang@swjtu.edu.cn).
	}
\thanks{The material in this paper was presented in part in the IEEE 23rd International Conference on Communication Technology 2023.}
%\thanks{This work was supported in part by the National Natural Science Foundation of China under Grant 62271421.}
}
	
	
	\maketitle
	
	\begin{abstract}
		In this paper, we propose a novel cooperative repair scheme for Zigzag MSR codes with optimal repair bandwidth, enabling the repair of any $h$ failed nodes.
		To the best of our knowledge, this is the first optimal cooperative repair scheme for Zigzag MSR codes. Compared to previous cooperative repair schemes for MSR codes, our scheme significantly reduces the size of the finite field to $\mathbb{F}_q,q\ge n+1$.
		%reduces  the size of finite field to $q \ge n + 1$.
	\end{abstract}
	
	\begin{IEEEkeywords}
		Distributed storage systems, optimal repair bandwidth, cooperative repair, Zigzag MSR codes
	\end{IEEEkeywords}

	\section{Introduction}
	With the rapid growth of data,
	the distributed file systems, such as Microsoft Azure and Facebook's coded Hadoop, have reached such a enormous scale that disk failures are the norm than exception.
	In this sense, redundancy is introduced imperatively to keep the high data reliability of the storage system.
	The most direct redundancy mechanism is the simple replication technique, such as the threefold replication in Google's Hadoop file system.
	Whereas, this solution suffers a large storage overhead.
	Therefore, MDS codes are implemented to counter failures.
	An $(n,k)$ MDS code encodes $k$ system symbols into $n$ symbols, which are distributed across $n$ storage nodes.
	Generally speaking, MDS codes can maintain a higher reliability
	at the same redundancy level or a lower redundancy at the same reliability level compared with the replication.
	Nevertheless, MDS codes have a drawback that when a failure occurs, one needs to contact $k$ nodes and download all of the symbols, which is costly with respect of the	\textit{repair bandwidth}, i.e., the amount of data downloaded for repairing failures.
	
	To overcome this drawback, Dimakis \textit{et al.}  introduced the \textit{regenerating codes} in \cite{dimakis2010network} to achieve the best trade off between the repair  bandwidth and storage overhead.
	In the past decade, the minimum storage regenerating (MSR) codes were  widely concerned owing to the MDS property and optimal repair bandwidth.
	For the case of single failure,
	numerous constructions of MSR
	codes  have been reported in the
	literature, e.g., see \cite{chen2019explicit,hou2019rack,li2021systematic,li2015framework,li2023new,papailiopoulos2013repair,shao2017tradeoff,tamo2011zigzag,wang2012long,li2018generic,ye2017explicit,zhang2023vertical,wang2023rack}, also refer to the survey paper \cite{balaji2018erasure} for details.
	For the case of multiple failures, the \textit{centralized repair} and \textit{cooperative repair} models are respectively developed.
	In the centralized repair model, the data of failures are recreated in a Data Center and then transferred into new nodes.
	In the cooperative repair model, each failed node is individually recovered by respectively downloading data from
	the helper nodes and then exchanging data with each other, which is  more suitable for the practical distributed architecture.
	Consequently, this paper focuses primarily on the latter.
	
	
	Assume that there are $h$ failed nodes.
	To repair them, one needs to connect $d\ge k$ helper nodes.
	Under the  cooperative repair model,
	the repair process  is composed of two phases \cite{shum2013cooperative}:
	\begin{enumerate}
		\item\label{step2} \textbf{Download phase:} Any failed node downloads $\beta_1$ symbols from each of $d$ helper nodes, respectively.
		
		\item\label{step3} \textbf{Cooperative  phase:} For any two failed nodes,  each transfers $\beta_2$ symbols to the other.
	\end{enumerate}
	Accordingly,  the repair bandwidth is $\gamma=h(d\beta_1+(h-1)\beta_2)$.
	
	In \cite{hu2010cooperative} and \cite{shum2013cooperative},
	the repair bandwidth of an MDS array code $\mathcal{C}$ is proved to be lower bounded by
	\begin{eqnarray}\label{eqn:Lower_Bound}
		\gamma\ge \frac{h(d+h-1)N}{d-k+h},
	\end{eqnarray}
	where $N$ is the length of data stored in each node, called as \textit{sub-packetization level}.
	Especially, the optimal repair bandwidth $ h(d+h-1)N/(d-k+h)$ in \eqref{eqn:Lower_Bound} is achieved only for $\beta_1= \beta_2=N/(d-k+h)$ \cite{shum2013cooperative}.
	In this case, the MDS array code $\mathcal{C}$ is known as \textit{$(n,k,d,h,N)$ MSR code}.
	Focus on the cooperative repair model,
	Ye and Barg constructed the first optimal cooperative repair scheme for MSR codes with the sub-packetization level $N=((d-k)^{h-1}(d-k+h))^{n \choose h}$ and the finite field $\mathbb{F}_q,q\ge (d-k+1)n$ in \cite{ye2018cooperative}.
	Subsequently, many works were dedicated to reduce the sub-packetization level or the size of finite field.
	Zhang \textit{et al.}  proposed the cooperative repair scheme with optimal access property for MSR codes in \cite{zhang2020explicit} while decreasing the sub-packetization to $(d-k+h)^{n \choose h}$ and the finite field to $\mathbb{F}_q,q\ge n+d-k$.
	In \cite{ye2020new},
	Ye   dramatically reduced the sub-packetization  to $(d-k+h)(d-k+1)^n$, but the finite field was still $\mathbb{F}_q,q\ge (d-k+1)n$.
	 The work of Liu \textit{et al.} \cite{liu2022new} removed or reduced the multiplication coefficients further, while maintaining the same finite field as Ye's for $d=k+1$.
	 Very recently, Zhang  \textit{et al.} \cite{zhang2024construction} significantly lowered the sub-packetization  to $(d-k+h)(d-k+1)^{\lceil n/2\rceil}$ with almost the same size of finite field as Ye's, i.e., $q\ge (d-k+1)n+1$.
	
In the literature, there are three  well-known MSR codes, Hadamard MSR codes \cite{papailiopoulos2013repair},
Zigzag MSR codes \cite{tamo2011zigzag}, and Long MDS codes \cite{li2023msr,wang2012long}.
Therein, the sub-packetization of Long MDS codes is typically smaller
compared to Hadamard MSR codes and Zigzag MSR codes. The size of finite field of Zigzag MSR codes is usually smaller compared to the other two codes, along with optimal access and update properties.
In fact, all the aforementioned cooperative repair schemes aim at Hadamard MSR codes \cite{liu2022new,ye2020new}  or  Long MDS codes  \cite{zhang2024construction}.
Especially, the scheme in  \cite{zhang2024construction}  is only designed for a subset of  Long MDS codes.

In this paper, our motivation is to design optimal  repair schemes for Zigzag MSR codes under the cooperative repair model.
One general method to construct MDS array codes with optimal
cooperative repair property is the \textit{space sharing}  \cite{rashmi2017piggybacking}.  It stacks multiple codewords of the conventional  MSR codes,
where the codewords are usually referred to as \textit{instances}.
To attain optimal cooperative repair property, the key technique in  \cite{liu2022new} and \cite{ye2020new}  is the inter-instance pairing, which properly pairs the symbols across instances.
Following this technique,  we first sum $d-k+1$ instances and then partition all symbols into some subgroups, which includes $r=n-k$ parity check equations.
%In \cite{ye2020new} and \cite{liu2022new}, the $r$ PCEs in one subgroup are enough to recover the failed data, since the diagonal matrix of Hadamard MSR codes creates the symbols can be repaired effortless by a Vandermonde matrix.
%However, this method has failed for Zigzag MSR codes owing to the permutation matrix, which complicates extremely the PCEs.
Particularly, we elaborately design a new pairing method by combining $(d-k+1)^{h-1}$ subgroups to form a complete group, which can achieve interference alignment and seek out a patterned recover matrix to ensure the recovery of failed nodes.
In this way, we propose general repair schemes for Zigzag MSR codes, which can yield optimal cooperative repair bandwidth.
%For certain parameter regimes, our scheme is optimal with respect to
 %the cut-set bound in \cite{hu2010cooperative} and \cite{shum2013cooperative}.
 Table \ref{comparison}
 compares our scheme with existing works.


\begin{table*}[!ht]
	\centering
	\begin{center}\caption{Comparisons of $(n,k,d,h,N)$ MSR codes with optimal cooperative repair property}\label{comparison}
		\renewcommand\arraystretch{1.2}
		\begin{tabular}{|c|c|c|c|c|}
			\hline
			Codes $\mathcal{C}$&	Sub-packetization $N$   & Helper nodes $d$ & Finite field $\mathbb{F}_q$  & Ref.\\
			\hline
			\multirow{4}{*}{Hadamard codes}&	$((d-k)^{h-1}(d-k+h))^{n \choose h}$  & $k\le d \le n-h$  &$(d-k+1)n$ & Ye and Barg \cite{ye2018cooperative}\\	
			\cline{2-5}
			&	$(d-k+h)(d-k+1)^n$& $k\le d \le n-h$ &$(d-k+1)n$ & Ye \cite{ye2020new}\\
			\cline{2-5}
			&	$\left\{\begin{array}{ll}
				2^n, & \mathrm{if}~(h+1)|2^n\\
				(2\ell+1)2^n, & \mathrm{if}~ h+1=(2\ell+1)2^m,  \ell\ge 1
			\end{array}
			\right.$   &  $d=k+1$&$(d-k+1)n$ &Liu \textit{et al.} \cite{liu2022new} \\
			\hline
			\multirow{2}{*}{Long MDS codes}&	$(d-k+h)^{n \choose h}$ & $k\le d \le n-h$ & $n+d-k$ & Zhang \textit{et al.} \cite{zhang2020explicit}\\
			\cline{2-5}
			&	$(d-k+h)(d-k+1)^{\lceil{n/2}\rceil}$& $k\le d \le n-h$ &$(d-k+1)n+1$ & Zhang \textit{et al.} \cite{zhang2024construction}\\
			\hline
			Zigzag codes&	$(d-k+h)(d-k+1)^n$& $k\le d\le n-h$ &$n + 1$& This paper \\
			\hline
		\end{tabular}
	\end{center}
\end{table*}
	
	The rest of this paper is organized as follows.
	In Section \ref{sec:preliminary},
	some necessary preliminaries of Zigzag MSR codes are reviewed.
	Then, Section \ref{sec:repair scheme} develops
	a general repair principle for Zigzag MSR codes with multiple failures.
	Section \ref{sec:h=2,3} proves the  repair scheme is available for any failed nodes.
	Finally, conclusions are drawn in Section \ref{sec:conclusion}.
	
	
	
\section{Preliminaries}\label{sec:preliminary}
In this section, we briefly review the $(n,k,d,h=1,N)$ Zigzag MSR codes.
For ease of reading,  we summarize some useful notation used throughout this  paper.
\begin{itemize}
	\item Let $\mathbb Z$ be the ring of integers and $\mathbb{F}_q$ be the finite field with $q$ elements, where $q$ is a prime power.
	%\item Let $\mathbf{a}\in\mathbb{F}_q^N, a\in\mathbb{F}_{q^n}, A\in\mathbb{F}_q^N$ are the vector, integer and matrix over $\mathbb{F}_q$, respectively.
	\item  Denote integers, vectors, sets and matrices by lowercase letters, bold lowercase letters, handwriting and capital letters, respectively, \textit{e.g.}, $a$, $\mathbf{a}$, $\mathcal{A}$, and $A$.
%	\item Denote $\mathbf{f}\in\mathbb{F}_q^N$ as a vector of length $N$ over $\mathbb{F}_q$.
	
	%	\item For two non-negative integers $a$ and $b$, denote $[b]=\{0,1,\dots,b-1\}$, $a+[b]=\{a,a+1,\dots,a+b-1\}$.
	
	\item For two non-negative integers $a$ and $b$ with $a < b$, define two ordered sets $\{a,a+1,\dots,b-1\}$ and $\{a,a+1,\dots,b\}$ by $[a,b)$ and $[a,b]$, respectively.
	
	%	\item For positive integers $a$ and $b$, denote the smallest nonnegative residue of $a$ modulo $b$ by $\langle a \rangle_b$.


	\item For any non-negative integer $a\in[0,s^n)$, where $s=d-k+1$ is an integer in this paper, let $\mathbf{a}=(a_0,\dots,a_{n-1})$ be its $s$-ary representation  in  vector form of length $n$, i.e., $a=\sum_{i=0}^{n-1}a_i s^i,a_i\in[0,s)$. Throughout this paper, we interchangeably use both $a$ and $\mathbf{a}=(a_0,\dots,a_{n-1})$
	to denote the integer $a\in[0,s^{n})$.
	
		\item Let $\mathbf{e}_{n,i}, i\in[0,n)$, be a binary vector of length $n$ with only the $i$-th component being non-zero. %for short $\mathbf{e}_{i}, i\in[0,n)$ when $n$ is clear.
 Define $\mathbf{a}\pm j\mathbf{e}_{n,i}=(a_0,\dots,\langle a_i\pm j\rangle,\dots,a_{n-1})$, where  $\langle \cdot \rangle$ means the smallest nonnegative residue of $\cdot$ modulo $s$.% where $a_i+j\equiv a_i+j\mod~ s$, $a_i\in[0,s)$. %Particularly, $\mathbf{a}+ 0\mathbf{e}_{n,i}=\sum_{i=0}^{n-1}a_i \cdot s^i=a$.

	\item For an $m\times n$ matrix $A$, denote the entry in the $i$-th row $j$-th column of $A$ by $A(i,j),i\in[0,m),j\in[0,n)$, also denote $A=(A(i,j))_{m\times n}$.
	For two sets $\mathcal{R},\mathcal{L}$ with $|\mathcal{R}|\le m,|\mathcal{L}|\le n$, denote $A(\mathcal{R},\mathcal{L})$  the submatrix of $A$ formed by the rows in $\mathcal{R}$ and the columns in $\mathcal{L}$.
	Specifically, write $A(\mathcal{R},[0,n))=A(\mathcal{R})$ and $A({i},[0,n))=A(i)$ for short.
\end{itemize}

\subsection{Zigzag MSR codes}\label{subsection:zigzag}

Assume that the original data is of size $M=kN$.
An $(n,k,d,h,N)$ Zigzag MSR code partitions the data into $k$ parts and then encodes them to $n$ parts $\mathbf{f}=[\mathbf{f}_0^\top,\dots,\mathbf{f}_{n-1}^\top]^\top$  stored across $n$ nodes,
where $\mathbf{f}_i=(f_{i,0},\dots,f_{i,N-1})^\top\in \mathbb{F}_q^N,i\in [0,n)$ is a column vector and $\top$ denotes the transpose operator.


The $(n,k,d,h,N)$ Zigzag MSR codes can be defined by the following parity check equations \cite{tamo2011zigzag},
\begin{IEEEeqnarray}{c}\label{Parity-check equation}
	A_{t,0}\mathbf{f}_0+ A_{t,1}\mathbf{f}_1+\cdots+ A_{t,n-1}\mathbf{f}_{n-1}=\mathbf{0},
\end{IEEEeqnarray}
where $A_{t,i},t\in[0,r),i\in[0,n)$ is an $N\times N$ nonsingular matrix over $\mathbb{F}_q$,
called the \textit{parity matrix} of node $i$
for the $t$-th parity check equation.
In matrix form, the structure of $(n,k,d,h,N)$ Zigzag MSR codes based on the above parity check equations  can be rewritten as
\begin{eqnarray*}\label{Matrix form}
	\underbrace{\left(\begin{array}{ccccc}
			A_{0,0} & A_{0,1} & \cdots  &   A_{0,n-1}      \\
			A_{1,0} & A_{1,1} & \cdots  &   A_{1,n-1}      \\
			\vdots  & \vdots  &  \ddots  &  \vdots          \\
			A_{r-1,0} & A_{r-1,1} & \cdots  &   A_{r-1,n-1}      \\
		\end{array}\right)}_{\mathrm{block ~matrix~}  A}
	\left(\begin{array}{ccccc}
		\mathbf{f}_0        \\
		\mathbf{f}_1\\
		\vdots\\
		\mathbf{f}_{n-1}\\
	\end{array}
	\right)=\mathbf{0}.
\end{eqnarray*}

Usually, $A$ is designed as a block Vandermonde matrix, i.e.,
\begin{eqnarray*}
	A_{t,i}=A_i^t,\qquad i\in[0,n), ~t\in[0,r),
\end{eqnarray*}
where $A_i,i\in[0,n)$ is an $N \times N$ nonsingular matrix.
In particular, we use the convention $A_i^0=I$.

To  illustrate $(n,k,d,h=1,N)$ Zigzag MSR codes, we firstly define some parameters here.
Let $N=s^n$ and $\gamma$ be a primitive element over $\mathbb{F}_q,q\ge n+1$, where  $s=d-k+1$ is an integer.
For any $i\in[0,n)$,
denote
\begin{eqnarray}\label{zigzag:A_i:parameter}
	\lambda_{i,u}=\left\{\begin{array}{lll}
		\gamma^{i+1}, &\text{ if } u=0,\\
		1, & \text{ else.}
	\end{array}	
	\right.
\end{eqnarray}
Then, for any $\mathbf{a}=(a_0,\dots,a_{n-1})\in[0,s^n)$, set
\begin{eqnarray}\label{eqn:zeta}
	\zeta_{t,i,a} = \left\{\begin{array}{lll}
		1, &\text{ if } t=0,\\
		\prod\limits_{0\leq j < t}\lambda_{i,\langle a_i+j\rangle}, & \text{ else.}
	\end{array}	
	\right.
\end{eqnarray}

%In the following content, we simply write $\langle \cdot \rangle_s$ as $\langle \cdot \rangle$ when the parameter $s=d-k+1$ is clear from the context.

Denote the $a$-th row vector of $A_{t,i},t\in[0,r),i\in[0,n),a\in[0,s^n)$ by
 \begin{equation}\label{eqn:A_{t,i,a}}
 	A_{t,i}(a)=\zeta_{t,i,a}\mathbf{e}_{N,\mathbf{a}+ t\mathbf{e}_{n,i}}.
 \end{equation}
Thereafter, $(n,k,d,h=1,N)$ Zigzag  MSR  codes with the sub-packetization $N=s^n$ can be characterized by the parity matrices below,
\begin{eqnarray}\label{A_i^t}
	A_{t,i}&=&
		\left(A_{t,i}(0)^\top,\dots,A_{t,i}(N-1)^\top\right)^\top \nonumber\\
&=&
\sum_{a=0}^{N-1}\zeta_{t,i,a}\mathbf{e}_{N,a}^\top \mathbf{e}_{N,\mathbf{a}+ t\mathbf{e}_{n,i}},\quad i\in[0,n),t\in[0,r).
\end{eqnarray}

Definitely, according to \eqref{zigzag:A_i:parameter} and \eqref{eqn:zeta}, we have
\begin{eqnarray}\label{eqn:A_1,i}
	A_{1,i}=A_i= \sum\limits_{a=0}^{N-1}\lambda_{i,a_i}\mathbf{e}_{N,a}^\top \mathbf{e}_{N,\mathbf{a}+ \mathbf{e}_{n,i}}  ,\quad i\in[0,n),
\end{eqnarray}
and
\begin{eqnarray*}
	A_{t,i}=A_i^t=\gamma^{(i+1)z} I_{s^n},\quad i\in[0,n),t = zs,z\in \mathbb{Z},
\end{eqnarray*}
where $I_{s^n}$ is an $s^n\times s^n$ identity matrix.



Additionally, we can get the following two properties.
\begin{itemize}
	\item [P1.]
	For any $t\in[0,r), i \in[0,n),a\ne b\in[0,s^n), \zeta_{t,i,a}=\zeta_{t,i,b}$ if $a_i=b_i$.	
	
	\item [P2.]
	For any $0\le i\ne j<n$,
	$\zeta_{t,j,\mathbf{a}+ w\mathbf{e}_{n,i}}= \zeta_{t,j,a},a\in[0,s^n),w\in[0,s)$.
	
\end{itemize}



Subsequently, we reveal the $a$-th row of the $t$-th parity check equation below,
\begin{eqnarray}\label{eqn:a-row}
	\sum_{i=0}^{n-1}A_{t,i}(a)\mathbf{f}_i
	=
	\sum_{i=0}^{n-1}\zeta_{t,i,a}f_{i,\mathbf{a}+ t\mathbf{e}_{n,i}} = 0,\quad
	a\in[0,s^n),t\in[0,r).
\end{eqnarray}
%To keep notation to a minimum, we will simply $\mathbf{e}_{n,i},i\in[0,n)$ refer to $\mathbf{e}_{i},i\in[0,n)$ in the following context.



\begin{Example}\label{Example_1}
        Let $\gamma$ be a primitive element of $\mathbb{F}_7$.
	The $(n=6,k=2,d=3,h=1,N=2^6)$ Zigzag MSR code has the following parity matrices over $\mathbb{F}_7$,
	\begin{eqnarray*}
		%\resizebox{.95\hsize}{!}{$
			A_{1,0}&=&
			\left(\gamma \mathbf{e}^\top_1,\mathbf{e}^\top_0,
			\gamma \mathbf{e}^\top_3,\mathbf{e}^\top_2,
			\gamma \mathbf{e}^\top_5,\mathbf{e}^\top_4,
			\gamma \mathbf{e}^\top_7,\mathbf{e}^\top_6,
			\gamma \mathbf{e}^\top_9,\mathbf{e}^\top_8,
			\gamma \mathbf{e}^\top_{11},\mathbf{e}^\top_{10},
			\gamma \mathbf{e}^\top_{13},\mathbf{e}^\top_{12},
			\gamma \mathbf{e}^\top_{15},\mathbf{e}^\top_{14},
			\gamma \mathbf{e}^\top_{17},\mathbf{e}^\top_{16},
			\gamma \mathbf{e}^\top_{19},\mathbf{e}^\top_{18},
			\gamma \mathbf{e}^\top_{21},\mathbf{e}^\top_{20},\right.
			\\
			&&
			\gamma \mathbf{e}^\top_{23},\mathbf{e}^\top_{22},
			\gamma \mathbf{e}^\top_{25},\mathbf{e}^\top_{24},
			\gamma \mathbf{e}^\top_{27},\mathbf{e}^\top_{26},
			\gamma \mathbf{e}^\top_{29},\mathbf{e}^\top_{28},\gamma \mathbf{e}^\top_{31},\mathbf{e}^\top_{30},\gamma \mathbf{e}^\top_{33},\mathbf{e}^\top_{32},\gamma \mathbf{e}^\top_{35},\mathbf{e}^\top_{34},\gamma \mathbf{e}^\top_{37},\mathbf{e}^\top_{36},\gamma \mathbf{e}^\top_{39},\mathbf{e}^\top_{38},
			\gamma \mathbf{e}^\top_{41},\mathbf{e}^\top_{40},\gamma \mathbf{e}^\top_{43},\mathbf{e}^\top_{42},\\
			&&
			\left.\gamma \mathbf{e}^\top_{45},\mathbf{e}^\top_{44},\gamma \mathbf{e}^\top_{47},\mathbf{e}^\top_{46},\gamma \mathbf{e}^\top_{49},\mathbf{e}^\top_{48},\gamma \mathbf{e}^\top_{51},\mathbf{e}^\top_{50},\gamma \mathbf{e}^\top_{53},\mathbf{e}^\top_{52},\gamma \mathbf{e}^\top_{55},\mathbf{e}^\top_{54},\gamma \mathbf{e}^\top_{57},\mathbf{e}^\top_{56},\gamma
			\mathbf{e}^\top_{59},\mathbf{e}^\top_{58},\gamma \mathbf{e}^\top_{61},\mathbf{e}^\top_{60},\gamma \mathbf{e}^\top_{63},\mathbf{e}^\top_{62}\right)^\top,\\
			A_{1,1}&=&
			\left(\gamma^2\mathbf{e}^\top_2,\gamma^2\mathbf{e}^\top_3,\mathbf{e}^\top_0,\mathbf{e}^\top_1,\gamma^2\mathbf{e}^\top_6,\gamma^2\mathbf{e}^\top_7,\mathbf{e}^\top_4,\mathbf{e}^\top_5,\gamma^2\mathbf{e}^\top_{10},
			\gamma^2\mathbf{e}^\top_{11},\mathbf{e}^\top_{8},\mathbf{e}^\top_{9},\gamma^2\mathbf{e}^\top_{14},\gamma^2\mathbf{e}^\top_{15},\mathbf{e}^\top_{12},\mathbf{e}^\top_{13},\gamma^2\mathbf{e}^\top_{18},	\gamma^2\mathbf{e}^\top_{19},\mathbf{e}^\top_{16},\mathbf{e}^\top_{17},\right.\\
			&&\gamma^2\mathbf{e}^\top_{22},\gamma^2\mathbf{e}^\top_{23},\mathbf{e}^\top_{20},\mathbf{e}^\top_{21},\gamma^2\mathbf{e}^\top_{26},\gamma^2\mathbf{e}^\top_{27},\mathbf{e}^\top_{24},\mathbf{e}^\top_{25},\gamma^2\mathbf{e}^\top_{30},\gamma^2\mathbf{e}^\top_{31},\mathbf{e}^\top_{28},\mathbf{e}^\top_{29},
			\gamma^2\mathbf{e}^\top_{34},\gamma^2\mathbf{e}^\top_{35},\mathbf{e}^\top_{32},\mathbf{e}^\top_{33},\gamma^2\mathbf{e}^\top_{38},\gamma^2\mathbf{e}^\top_{39},\mathbf{e}^\top_{36},\\
			&&\mathbf{e}^\top_{37},\gamma^2\mathbf{e}^\top_{42},\gamma^2\mathbf{e}^\top_{43},\mathbf{e}^\top_{40},\mathbf{e}^\top_{41},\gamma^2\mathbf{e}^\top_{46},\gamma^2\mathbf{e}^\top_{47},\mathbf{e}^\top_{44},\mathbf{e}^\top_{45},\gamma^2\mathbf{e}^\top_{50},\gamma^2\mathbf{e}^\top_{51},\mathbf{e}^\top_{48},\mathbf{e}^\top_{49},\gamma^2\mathbf{e}^\top_{54},\gamma^2\mathbf{e}^\top_{55},\mathbf{e}^\top_{52},\mathbf{e}^\top_{53},\gamma^2\mathbf{e}^\top_{58},\gamma^2\mathbf{e}^\top_{59},\\
			&&\left.\mathbf{e}^\top_{56},\mathbf{e}^\top_{57},\gamma^2\mathbf{e}^\top_{62},\gamma^2\mathbf{e}^\top_{63},\mathbf{e}^\top_{60},\mathbf{e}^\top_{61}\right)^\top,\\
			%	$}\\
		%\resizebox{.95\hsize}{!}{$
			A_{1,2}&=&\left(\gamma^3\mathbf{e}^\top_4,\gamma^3\mathbf{e}^\top_5,\gamma^3\mathbf{e}^\top_6,\gamma^3\mathbf{e}^\top_7,
\mathbf{e}^\top_0,\mathbf{e}^\top_1,\mathbf{e}^\top_2,\mathbf{e}^\top_3,\gamma^3\mathbf{e}^\top_{12},\gamma^3\mathbf{e}^\top_{13},
\gamma^3\mathbf{e}^\top_{14},\gamma^3\mathbf{e}^\top_{15},\mathbf{e}^\top_{8},\mathbf{e}^\top_{9},\mathbf{e}^\top_{10},
\mathbf{e}^\top_{11},\gamma^3\mathbf{e}^\top_{20},	\gamma^3\mathbf{e}^\top_{21},\gamma^3\mathbf{e}^\top_{22},\right.\\
			&&\gamma^3\mathbf{e}^\top_{23},\mathbf{e}^\top_{16},\mathbf{e}^\top_{17},\mathbf{e}^\top_{18},\mathbf{e}^\top_{19},
\gamma^3\mathbf{e}^\top_{28},\gamma^3\mathbf{e}^\top_{29},\gamma^3\mathbf{e}^\top_{30},\gamma^3\mathbf{e}^\top_{31},
\mathbf{e}^\top_{24},\mathbf{e}^\top_{25},\mathbf{e}^\top_{26},\mathbf{e}^\top_{27},\gamma^3\mathbf{e}^\top_{36},
\gamma^3\mathbf{e}^\top_{37},\gamma^3\mathbf{e}^\top_{38},\gamma^3\mathbf{e}^\top_{39},\mathbf{e}^\top_{32},\mathbf{e}^\top_{33},\\
			&&\mathbf{e}^\top_{34},\mathbf{e}^\top_{35},\gamma^3\mathbf{e}^\top_{44},\gamma^3\mathbf{e}^\top_{45},\gamma^3\mathbf{e}^\top_{46},\gamma^3\mathbf{e}^\top_{47},\mathbf{e}^\top_{40},\mathbf{e}^\top_{41},\mathbf{e}^\top_{42},\mathbf{e}^\top_{43},\gamma^3\mathbf{e}^\top_{52},\gamma^3\mathbf{e}^\top_{53},\gamma^3\mathbf{e}^\top_{54},\gamma^3\mathbf{e}^\top_{55},\mathbf{e}^\top_{48},\mathbf{e}^\top_{49},\mathbf{e}^\top_{50},\mathbf{e}^\top_{51},\gamma^3\mathbf{e}^\top_{60},\\
			&&\left.\gamma^3\mathbf{e}^\top_{61},\gamma^3\mathbf{e}^\top_{62},\gamma^3\mathbf{e}^\top_{63},\mathbf{e}^\top_{56},
\mathbf{e}^\top_{57},\mathbf{e}^\top_{58},\mathbf{e}^\top_{59}\right)^\top,\\
			A_{1,3}&=&
			\left(\gamma^4\mathbf{e}^\top_{8},\gamma^4\mathbf{e}^\top_{9},\gamma^4\mathbf{e}^\top_{10},
\gamma^4\mathbf{e}^\top_{11},\gamma^4\mathbf{e}^\top_{12},\gamma^4\mathbf{e}^\top_{13},\gamma^4\mathbf{e}^\top_{14},
\gamma^4\mathbf{e}^\top_{15},\mathbf{e}^\top_0,\mathbf{e}^\top_1,\mathbf{e}^\top_2,\mathbf{e}^\top_3,\mathbf{e}^\top_4,
\mathbf{e}^\top_5,\mathbf{e}^\top_6,\mathbf{e}^\top_7,\gamma^4\mathbf{e}^\top_{24},\gamma^4\mathbf{e}^\top_{25},\gamma^4\mathbf{e}^\top_{26},\right.\\
			&&\gamma^4\mathbf{e}^\top_{27},\gamma^4\mathbf{e}^\top_{28},\gamma^4\mathbf{e}^\top_{29},\gamma^4\mathbf{e}^\top_{30},
\gamma^4\mathbf{e}^\top_{31},\mathbf{e}^\top_{16},\mathbf{e}^\top_{17},\mathbf{e}^\top_{18},\mathbf{e}^\top_{19},
\mathbf{e}^\top_{20},\mathbf{e}^\top_{21},\mathbf{e}^\top_{22},\mathbf{e}^\top_{23},\gamma^4\mathbf{e}^\top_{40},
\gamma^4\mathbf{e}^\top_{41},\gamma^4\mathbf{e}^\top_{42},\gamma^4\mathbf{e}^\top_{43},\gamma^4\mathbf{e}^\top_{44},\\
			&&\gamma^4\mathbf{e}^\top_{45},\gamma^4\mathbf{e}^\top_{46},\gamma^4\mathbf{e}^\top_{47},\mathbf{e}^\top_{32},
\mathbf{e}^\top_{33},\mathbf{e}^\top_{34},\mathbf{e}^\top_{35},\mathbf{e}^\top_{36},\mathbf{e}^\top_{37},
\mathbf{e}^\top_{38},\mathbf{e}^\top_{39},\gamma^4\mathbf{e}^\top_{56},\gamma^4\mathbf{e}^\top_{57},
\gamma^4\mathbf{e}^\top_{58},\gamma^4\mathbf{e}^\top_{59},\gamma^4\mathbf{e}^\top_{60},\gamma^4\mathbf{e}^\top_{61},
\gamma^4\mathbf{e}^\top_{62},\\		
			&&\left.\gamma^4\mathbf{e}^\top_{63},\mathbf{e}^\top_{48},\mathbf{e}^\top_{49},\mathbf{e}^\top_{50},
\mathbf{e}^\top_{51},\mathbf{e}^\top_{52},\mathbf{e}^\top_{53},\mathbf{e}^\top_{54},\mathbf{e}^\top_{55}\right)^\top,\\
			A_{1,4}&=&
			\left(\gamma^5\mathbf{e}^\top_{16},\gamma^5\mathbf{e}^\top_{17},\gamma^5\mathbf{e}^\top_{18},
\gamma^5\mathbf{e}^\top_{19},\gamma^5\mathbf{e}^\top_{20},\gamma^5\mathbf{e}^\top_{21},
\gamma^5\mathbf{e}^\top_{22},\gamma^5\mathbf{e}^\top_{23},\gamma^5\mathbf{e}^\top_{24},
\gamma^5\mathbf{e}^\top_{25},\gamma^5\mathbf{e}^\top_{26},\gamma^5\mathbf{e}^\top_{27},
\gamma^5\mathbf{e}^\top_{28},	\gamma^5\mathbf{e}^\top_{29},\gamma^5\mathbf{e}^\top_{30},\gamma^5\mathbf{e}^\top_{31}\right.,\\
			&&\mathbf{e}^\top_0,\mathbf{e}^\top_1,\mathbf{e}^\top_2,\mathbf{e}^\top_3,\mathbf{e}^\top_4,\mathbf{e}^\top_5,\mathbf{e}^\top_6,\mathbf{e}^\top_7,\mathbf{e}^\top_{8},\mathbf{e}^\top_{9},\mathbf{e}^\top_{10},\mathbf{e}^\top_{11},\mathbf{e}^\top_{12},\mathbf{e}^\top_{13},\mathbf{e}^\top_{14},\mathbf{e}^\top_{15},
			\gamma^5\mathbf{e}^\top_{48},\gamma^5\mathbf{e}^\top_{49},\gamma^5\mathbf{e}^\top_{50},\gamma^5\mathbf{e}^\top_{51},\gamma^5\mathbf{e}^\top_{52},\gamma^5\mathbf{e}^\top_{53},\\
	\end{eqnarray*}
\begin{eqnarray*}	&&\gamma^5\mathbf{e}^\top_{54},	\gamma^5\mathbf{e}^\top_{55},\gamma^5\mathbf{e}^\top_{56},\gamma^5\mathbf{e}^\top_{57},\gamma^5\mathbf{e}^\top_{58},\gamma^5\mathbf{e}^\top_{59},\gamma^5\mathbf{e}^\top_{60},\gamma^5\mathbf{e}^\top_{61},\gamma^5\mathbf{e}^\top_{62},\gamma^5\mathbf{e}^\top_{63},\mathbf{e}^\top_{32},\mathbf{e}^\top_{33},\mathbf{e}^\top_{34},\mathbf{e}^\top_{35},\mathbf{e}^\top_{36},\mathbf{e}^\top_{37},\mathbf{e}^\top_{38},\mathbf{e}^\top_{39},\mathbf{e}^\top_{40},\\
			&&\left.\mathbf{e}^\top_{41},\mathbf{e}^\top_{42},\mathbf{e}^\top_{43},\mathbf{e}^\top_{44},\mathbf{e}^\top_{45},\mathbf{e}^\top_{46},\mathbf{e}^\top_{47}
			\right)^\top,\\
			A_{1,5}&=&
			\left(\gamma^6\mathbf{e}^\top_{32},\gamma^6\mathbf{e}^\top_{33},\gamma^6\mathbf{e}^\top_{34},\gamma^6\mathbf{e}^\top_{35},\gamma^6\mathbf{e}^\top_{36},\gamma^6\mathbf{e}^\top_{37},\gamma^6\mathbf{e}^\top_{38},\gamma^6\mathbf{e}^\top_{39},\gamma^6\mathbf{e}^\top_{40},\gamma^6\mathbf{e}^\top_{41},\gamma^6\mathbf{e}^\top_{42},\gamma^6\mathbf{e}^\top_{43},\gamma^6\mathbf{e}^\top_{44},\gamma^6\mathbf{e}^\top_{45},	\gamma^6\mathbf{e}^\top_{46},\gamma^6\mathbf{e}^\top_{47},\right.\\
			&&\gamma^6\mathbf{e}^\top_{48},
			\gamma^6\mathbf{e}^\top_{49},\gamma^6\mathbf{e}^\top_{50},\gamma^6\mathbf{e}^\top_{51},\gamma^6\mathbf{e}^\top_{52},\gamma^6\mathbf{e}^\top_{53},
			\gamma^6\mathbf{e}^\top_{54},\gamma^6\mathbf{e}^\top_{55},	\gamma^6\mathbf{e}^\top_{56},\gamma^6\mathbf{e}^\top_{57},\gamma^6\mathbf{e}^\top_{58},
			\gamma^6\mathbf{e}^\top_{59},\gamma^6\mathbf{e}^\top_{60},\gamma^6\mathbf{e}^\top_{61},\gamma^6\mathbf{e}^\top_{62},\gamma^6\mathbf{e}^\top_{63},\\
			&&\mathbf{e}^\top_0,\mathbf{e}^\top_1,\mathbf{e}^\top_2,\mathbf{e}^\top_3,\mathbf{e}^\top_4,\mathbf{e}^\top_5,\mathbf{e}^\top_6,\mathbf{e}^\top_7,\mathbf{e}^\top_{8},\mathbf{e}^\top_{9},\mathbf{e}^\top_{10},\mathbf{e}^\top_{11},\mathbf{e}^\top_{12},\mathbf{e}^\top_{13},\mathbf{e}^\top_{14},\mathbf{e}^\top_{15},
			\mathbf{e}^\top_{16},\mathbf{e}^\top_{17},\mathbf{e}^\top_{18},\mathbf{e}^\top_{19},\mathbf{e}^\top_{20},\mathbf{e}^\top_{21},
\mathbf{e}^\top_{22},\mathbf{e}^\top_{23},\mathbf{e}^\top_{24},\mathbf{e}^\top_{25},\\
			&&\left.\mathbf{e}^\top_{26},\mathbf{e}^\top_{27},\mathbf{e}^\top_{28},\mathbf{e}^\top_{29},\mathbf{e}^\top_{30},\mathbf{e}^\top_{31}
			\right)^\top
			%	$}
	\end{eqnarray*}
	according to \eqref{zigzag:A_i:parameter}-\eqref{eqn:A_1,i}, where $\mathbf{e}_i=\mathbf{e}_{64,i},i\in[0,63]$.
	Obviously, as $s=d-k+1=2$, we can get
	\begin{eqnarray*}
		A_{2,0}=\gamma I_{64},~ A_{2,1}=\gamma^2 I_{64}, ~A_{2,2}=\gamma^3 I_{64}, ~A_{2,3}=\gamma^4 I_{64}, ~A_{2,4}=\gamma^5 I_{64}.
	\end{eqnarray*}
\end{Example}	

\section{The General Repair Strategy of Zigzag MSR Codes}\label{sec:repair scheme}

In this section, we propose a general repair scheme for $(n,k,d,h,N)$ Zigzag MSR codes with $N= (d-k+h)s^n$, where $s=d-k+1$.
Throughout this paper, it is assumed that $h$ nodes of an $(n,k,d,h,N)$ Zigzag MSR code fail, denoted by $\mathcal{I}=\{i_0,\dots,i_{h-1}\}$.
During the repair process of the $h$ failed nodes, $k\le d\le n-h$ helper nodes are connected and thus $n-d-h$ nodes are unconnected,
denoted by  $\mathcal{J}=\{j_0,\dots,j_{d-1}\}\subseteq[0,n)\backslash \mathcal{I},|\mathcal{J}|=d$ and $\mathcal{U}=[0,n)\backslash(\mathcal{I}\cup \mathcal{J})=\{i_h,\dots,i_{n-d-1}\},|\mathcal{U}|=n-d-h$, respectively.


In the original $(n,k,d,h=1,N)$ Zigzag MSR code $\mathcal{C}$ with sub-packetization  $s^n$, each node $i\in[0,n)$ stores a column vector  $\mathbf{f}_i=(f_{i,0},\dots,f_{i,s^n-1})^\top$ of length $s^n$.
	Generate $d-k+h$ instances of $\mathcal{C}$ by space sharing technique, whose  column vectors are denoted by $\mathbf{f}_i^{(0)},\dots,$ $\mathbf{f}_i^{(d-k+h-1)},0\le i<n$.
	In this way, we  obtain an $(n,k,d,h,N)$ Zigzag MSR code with sub-packetization  $N=(d-k+h)s^n$. By convenience, still denote the code  by $\mathcal{C}$ and write  the column vector of length $(d-k+h)s^n$ stored at node $i$ as $\mathbf{f}_i=((\mathbf{f}_i^{(0)})^\top, \dots,(\mathbf{f}_i^{(d-k+h-1)})^\top)^\top$, where $\mathbf{f}_{i}^{(w)}=\{f_{i,a}^{(w)}:a\in[0,s^n)\}^\top, i\in[0,n), w\in[0,d-k+h)$ is a column vector of length $s^n$.
	%Let $\gamma$ be the primitive element over $\mathbb{F}_q, q\ge n+1$.


Thus, by \eqref{Parity-check equation}, the parity check equations of an $(n,k,d,h,N)$ Zigzag MSR code $\mathcal{C}$ with sub-packetization  $N=(d-k+h)s^n$ can be given as
\begin{eqnarray*}\label{parity-check equation}
	A_{t,0}\mathbf{f}_0^{(w)}+A_{t,1}\mathbf{f}_1^{(w)}+\cdots+A_{t,n-1}\mathbf{f}_{n-1}^{(w)} = \mathbf{0},%\nonumber\\
	\quad  w\in[0,d-k+h),t\in[0,r),
\end{eqnarray*}
where $A_{t,i},i\in[0,n)$ is defined by \eqref{A_i^t}.

Specifically, according to \eqref{eqn:a-row}, the $a$-th row of the $t$-th parity check equation is
\begin{eqnarray*}\label{eqn:a-th parity}
	\sum_{i=0}^{n-1}A_{t,i}(a)\mathbf{f}_i^{(w)} =
	\sum_{i=0}^{n-1}\zeta_{t,i,a}f_{i,\mathbf{a}+ t\mathbf{e}_{n,i}}^{(w)} = 0, \quad  a\in[0,s^n),%\nonumber\\
	w\in[0,d-k+h),t\in[0,r).
\end{eqnarray*}


For the $(n,k,d,h,N)$ Zigzag MSR codes with sub-packetization  $N=(d-k+h)s^n$, the repair process of the  $h$ failed nodes  consists of the following four steps.

\textbf{\textit{ Step 1 (Summing):}} For each failed node $i_u,u\in[0,h)$, respectively sum the parity check equations of $s$ instances to obtain $r\cdot s^n$ equations.

In order to repair the failed node $i_u\in\mathcal{I}=\{i_0,\dots,i_{h-1}\},u\in[0,h)$,  for given $t\in[0,r)$ and $a\in [0,s^n)$, we sum
\begin{eqnarray}\label{eqn:w parity}
	\sum_{i=0}^{n-1}\zeta_{t,i,\mathbf{a}+ w\mathbf{e}_{n,i_u}}f_{i,\mathbf{a} + w\mathbf{e}_{n,i_u}+ t\mathbf{e}_{n,i}}^{(w)}
	%=\sum_{i=0}^{n-1}A_{t,i,\mathbf{a}+ w\mathbf{e}_{n,i_u}}\mathbf{f}_i^{(w)}
	= 0,\quad a\in [0,s^n),%\nonumber \\
	w\in[0,s-2],t\in[0,r)
\end{eqnarray}
and
\begin{eqnarray}\label{eqn:s+u-1 parity}
	\sum_{i=0}^{n-1}\zeta_{t,i,\mathbf{a}+ (s-1)\mathbf{e}_{n,i_u}}f_{i,\mathbf{a} + (s-1)\mathbf{e}_{n,i_u} + t\mathbf{e}_{n,i}}^{(u+s-1)}
	%=\sum_{i=0}^{n-1}A_{t,i,\mathbf{a}+ (s-1)\mathbf{e}_{n,i_u}}\mathbf{f}_i^{(u+s-1)}
	= 0,\quad %\nonumber \\
	a\in [0,s^n),t\in[0,r)
\end{eqnarray}
to obtain
%\begin{footnotesize}
\begin{eqnarray}\label{Eqn: summing parity}
	\sum_{i=0}^{n-1}\left(\sum_{w=0}^{s-2}\zeta_{t,i,\mathbf{a}+ w\mathbf{e}_{n,i_u}}f_{i,\mathbf{a} + w\mathbf{e}_{n,i_u}+ t\mathbf{e}_{n,i}}^{(w)}+
	\zeta_{t,i,\mathbf{a}+ (s-1)\mathbf{e}_{n,i_u}}f_{i,\mathbf{a} + (s-1)\mathbf{e}_{n,i_u} + t\mathbf{e}_{n,i}}^{(u+s-1)}\right)= 0,  \quad a\in [0,s^n),t\in[0,r),
\end{eqnarray}
which is called  \textit{sum parity check equation $(\mathbf{a},t)$  of node $i_u$}, or  SPCE   $(\mathbf{a},t)$  for short. Totally there are
$r\cdot s^n$ SPCEs.


For ease of analysis in what follows, we rewrite  \eqref{Eqn: summing parity} as
 \begin{eqnarray}\label{eqn:simply0}
 	&&\sum_{w=0}^{s-2}\zeta_{t,i_u,\mathbf{a}+ w\mathbf{e}_{n,i_u}}f_{i_u,\mathbf{a}+ w\mathbf{e}_{n,i_u}+ t\mathbf{e}_{n,i_u}}^{(w)}%\nonumber\\
 	+\zeta_{t,i_u,\mathbf{a}+ (s-1)\mathbf{e}_{n,i_u}}f_{i_u,\mathbf{a}+ (s-1)\mathbf{e}_{n,i_u}+ t\mathbf{e}_{n,i_u}}^{(u+s-1)} \nonumber\\
 	&&+\sum_{i\in\mathcal{I}\backslash \{i_u\}}\left(\sum_{w=0}^{s-2}\zeta_{t,i,\mathbf{a}+ w\mathbf{e}_{n,i_u}}f_{i,\mathbf{a}+ w\mathbf{e}_{n,i_u}+ t\mathbf{e}_{n,i}}^{(w)}%\nonumber\\
 	+\zeta_{t,i,\mathbf{a}+ (s-1)\mathbf{e}_{n,i_u}}
 	f_{i,\mathbf{a}+ (s-1)\mathbf{e}_{n,i_u}+ t\mathbf{e}_{n,i}}^{(u+s-1)}\right) \nonumber\\
 &&	+\sum_{z\in\mathcal{U}}\left(\sum_{w=0}^{s-2}\zeta_{t,z,\mathbf{a}+ w\mathbf{e}_{n,i_u}}f_{z,\mathbf{a}+ w\mathbf{e}_{n,i_u}+ t\mathbf{e}_{n,z}}^{(w)}
 		+\zeta_{t,z,\mathbf{a}+ (s-1)\mathbf{e}_{n,i_u}}
 		f_{z,\mathbf{a}+ (s-1)\mathbf{e}_{n,i_u}+ t\mathbf{e}_{n,z}}^{(u+s-1)}\right) \nonumber\\
		&=&\sum_{w=0}^{s-2}\zeta_{t,i_u,\mathbf{a}+ w\mathbf{e}_{n,i_u}}f_{i_u,\mathbf{a}+ \langle w+t\rangle\mathbf{e}_{n,i_u}}^{(w)}%\nonumber\\
 	+\zeta_{t,i_u,\mathbf{a}+ (s-1)\mathbf{e}_{n,i_u}}
 	f_{i_u,\mathbf{a}+ \langle t-1\rangle\mathbf{e}_{n,i_u}}^{(u+s-1)}\nonumber\\
 	&&+\sum_{i\in\mathcal{I}\backslash\{i_u\}}\zeta_{t,i,a}\left(\sum_{w=0}^{s-2}f_{i,\mathbf{a}+ w\mathbf{e}_{n,i_u}+ t\mathbf{e}_{n,i}}^{(w)}+f_{i,\mathbf{a}+ (s-1)\mathbf{e}_{n,i_u}+ t\mathbf{e}_{n,i}}^{(u+s-1)}\right) \nonumber\\
 	&&	+\sum_{z\in\mathcal{U}}\zeta_{t,z,a}\left(\sum_{w=0}^{s-2}f_{z,\mathbf{a}+ w\mathbf{e}_{n,i_u}+ t\mathbf{e}_{n,z}}^{(w)}+f_{z,\mathbf{a}+ (s-1)\mathbf{e}_{n,i_u}+ t\mathbf{e}_{n,z}}^{(u+s-1)}\right) \nonumber\\
	&=&- \sum_{j\in\mathcal{J}}\zeta_{t,j,a}\left(\sum_{w=0}^{s-2}f_{j,\mathbf{a}+ w\mathbf{e}_{n,i_u}+ t\mathbf{e}_{n,j}}^{(w)}%\nonumber\\
 	+
 	f_{j,\mathbf{a}+ (s-1)\mathbf{e}_{n,i_u}+ t\mathbf{e}_{n,j}}^{(u+s-1)}\right), \,a\in [0,s^n), t\in[0,r), %\nonumber\\
 \end{eqnarray}
 where the identities follow from
 \begin{eqnarray*}
 	\zeta_{t,i,a}= \zeta_{t,i,\mathbf{a}+ \mathbf{e}_{n,i_u}}= \cdots = \zeta_{t,i,\mathbf{a}+ (s-1)\mathbf{e}_{n,i_u}},\quad
 	i\in[0,n)\backslash \{i_u\}
 \end{eqnarray*}
 by  P1 and P2.


 \textbf{\textit{Step 2 (Grouping):}}  Divide all the sum parity check equations in \eqref{eqn:simply0} into $s^{d+1}$ groups, each group having $r\cdot s^{n-d-1}$ equations.

Given  $\mathcal{J}=\{j_0,\dots,j_{d-1}\}$  and $\bm{\ell}=(\ell_0,\dots,\ell_{d-1})\in[0,s^d)$, where $\ell_i\in[0,s),i\in[0,d)$,
 define
\begin{eqnarray}\label{eqn:a}
	\mathcal{A}_{\ell}\triangleq \{a:a_{j_i}=\ell_i\}\subseteq [0,s^n),%\nonumber \\
	%\ell \in[0,s^d),
\end{eqnarray}
and
 \begin{eqnarray}\label{eqn:b}
 	\mathcal{B}_{\ell,\tau}\triangleq\{a: a_{i_0} +\cdots+a_{i_{h-1}}+a_{i_{h}}+\cdots+a_{i_{n-d-1}}
 	\equiv \tau \mod s\}  \subseteq \mathcal{A}_{\ell}, \quad \tau\in [0,s).
 \end{eqnarray}
It is obvious that $\cup_{\ell=0}^{s^d-1}\cup_{\tau=0}^{s-1}\mathcal{B}_{\ell,\tau} =\cup_{\ell=0}^{s^d-1}\mathcal{A}_{\ell}= [0,s^n)$ and
$ \mathcal{B}_{\ell,\tau}$ has $s^{n-d-1}$ elements for any $\ell\in[0,s^d)$ and $\tau\in [0,s)$.


Then,  we partition all the $r\cdot s^n$ SPCEs in \eqref{eqn:simply0} into $s^{d+1}$ groups such that
each group includes  $s^{n-d-1}$ subgroups with each subgroup containing $r$ SPCEs. Precisely, given $\ell\in [0,s^{d})$  and $\tau\in [0,s)$, the $(\ell,\tau)$-th group
 consists of SPCEs  $\{(\mathbf{a}- t\mathbf{e}_{n,i_u},t):t\in [0,r)\}$ for all $\mathbf{a}\in \mathcal{B}_{\ell,\tau}$, i.e.,
 \begin{eqnarray}\label{eqn:simply1}
 &&\sum_{w=0}^{s-2}\zeta_{t,i_u,\mathbf{a}+ \langle w-t\rangle\mathbf{e}_{n,i_u}}f_{i_u,\mathbf{a}+ w\mathbf{e}_{n,i_u}}^{(w)}%\nonumber\\
 	+\zeta_{t,i_u,\mathbf{a}-  \langle t+1\rangle\mathbf{e}_{n,i_u}}
 	f_{i_u,\mathbf{a}+(s-1)\mathbf{e}_{n,i_u}}^{(u+s-1)}\nonumber\\
 	&&+\sum_{i\in\mathcal{I}\backslash\{i_u\}}\zeta_{t,i,\mathbf{a}- t\mathbf{e}_{n,i_u}}\left(\sum_{w=0}^{s-2}f_{i,\mathbf{a}+  \langle w-t\rangle\mathbf{e}_{n,i_u}+ t\mathbf{e}_{n,i}}^{(w)}+f_{i,\mathbf{a}- \langle t+1\rangle\mathbf{e}_{n,i_u}+ t\mathbf{e}_{n,i}}^{(u+s-1)}\right) \nonumber\\
 	&&+\sum_{z\in\mathcal{U}}\zeta_{t,z,\mathbf{a}- t\mathbf{e}_{n,i_u}}\left(\sum_{w=0}^{s-2}f_{z,\mathbf{a}+  \langle w-t\rangle\mathbf{e}_{n,i_u}+ t\mathbf{e}_{n,z}}^{(w)}+f_{z,\mathbf{a}- \langle t+1\rangle\mathbf{e}_{n,i_u}+ t\mathbf{e}_{n,z}}^{(u+s-1)}\right) \nonumber\\
 	&=&- \sum_{j\in\mathcal{J}}\zeta_{t,j,\mathbf{a}- t\mathbf{e}_{n,i_u}}\left(\sum_{w=0}^{s-2}f_{j,\mathbf{a}+  \langle w-t\rangle\mathbf{e}_{n,i_u}+ t\mathbf{e}_{n,j}}^{(w)}%\nonumber\\
 	+
 	f_{j,\mathbf{a}- \langle t+1\rangle\mathbf{e}_{n,i_u}+ t\mathbf{e}_{n,j}}^{(u+s-1)}\right), \,\mathbf{a}\in \mathcal{B}_{\ell,\tau}, t\in[0,r), %\nonumber\\
 \end{eqnarray}
 where given $\mathbf{a}\in \mathcal{B}_{\ell,\tau}$, the $r$ SPCEs form a \textit{subgroup}.

	
 \textbf{\textit{Step 3 (Downloading phase):}} By downloading
 \begin{eqnarray}\label{Eqn_Dd_data}
	\mathcal{D}_j=f_{j,a}^{(0)}+\cdots+f_{j,\mathbf{a}+ (s-2)\mathbf{e}_{n,i_u}}^{(s-2)}+f_{j,\mathbf{a}+ (s-1)\mathbf{e}_{n,i_u}}^{(u+s-1)},\quad a\in [0,s^n)
 \end{eqnarray}
 from all the helper nodes $j\in\mathcal{J}$,   the failed node $i_u,u\in[0,h)$ solves SPCEs in
 the $(\ell, \tau)$-th group to
 respectively recover
 \begin{eqnarray}\label{Eqn_Re_data_1}
	&&\left\{f_{i_u,a}^{(0)},\dots,f_{i_u,a}^{(s-2)},f_{i_u,a}^{(u+s-1)}: a\in [0,s^n)\right\}\nonumber\\
	&=&\left\{f_{i_u,a}^{(0)},f_{i_u,\mathbf{a}+\mathbf{e}_{n,i_u}}^{(1)}, \dots,f_{i_u,\mathbf{a}+(s-2)\mathbf{e}_{n,i_u}}^{(s-2)},f_{i_u,\mathbf{a}+(s-1)\mathbf{e}_{n,i_u}}^{(u+s-1)}:a\in \mathcal{B}_{\ell,\tau},\ell\in[0,s^d),\tau\in[0,s)\right\}
  \end{eqnarray}
	and
   \begin{align}\label{Eqn_Re_data_2}
 	&\left\{f_{i_{v},a}^{(0)}+f_{i_{v},\mathbf{a}+ \mathbf{e}_{n,i_u}}^{(1)}+\cdots+f_{i_{v},\mathbf{a}+ (s-2)\mathbf{e}_{n,i_u}}^{(s-2)}+f_{i_{v},\mathbf{a}+ (s-1)\mathbf{e}_{n,i_u}}^{(u+s-1)}: a\in [0,s^n),\forall~ i_{v}\in \mathcal{I}\cup \mathcal{U}\backslash\{i_u\}\right\}\nonumber\\
	=&\left\{\sum_{w=0}^{s-2}f_{i_v,\mathbf{a}+  \langle w-t\rangle\mathbf{e}_{n,i_u}+ t\mathbf{e}_{n,i_v}}^{(w)}+f_{i_v,\mathbf{a}- \langle t+1\rangle\mathbf{e}_{n,i_u}+ t\mathbf{e}_{n,i_v}}^{(u+s-1)}: a\in \mathcal{B}_{\ell,\tau},\ell\in[0,s^d),\tau\in[0,s),t\in[0,r),\forall~ i_{v}\in \mathcal{I}\cup \mathcal{U}\backslash\{i_u\}\right\}.
  \end{align}



\textbf{\textit{Step 4 (Cooperative phase):}} The failed node $i_u$ recovers the remainder data after receiving
	\begin{eqnarray}\label{Eqn_Co_data}
			f_{i_u,a}^{(0)}+\cdots+f_{i_u,\mathbf{a}+ (s-2)\mathbf{e}_{n,i_{v}}}^{(s-2)}+f_{i_u,\mathbf{a}+ (s-1)\mathbf{e}_{n,i_{v}}}^{(v+s-1)},\quad a\in [0,s^n)
		\end{eqnarray}
from the failed nodes
 $i_{v}, v\in [0,h)\backslash\{u\}$.

%Before proving that Steps 3 and 4 are feasible, we  first provide an  example to demonstrate the repair process of $h=2$ failed nodes as follows.



Now, we show that the failed node $i_u$ can recover the desired data in the right hand side (RHS) of \eqref{Eqn_Re_data_1} and \eqref{Eqn_Re_data_2} from the  $r\cdot s^{n-d-1}$ SPCEs in $(\ell,\tau)$-th group of node $i_u$ in \eqref{eqn:simply1}.
Firstly, we set a  logical order to the $h$ failed nodes and $n-d-h$ unconnected nodes, still denote $\mathcal{I}\cup \mathcal{U}=\{i_0,\dots,i_{h-1},i_h,\dots,i_{n-d-1}\}$ with $i_0<i_1<\cdots < i_{h-1}<i_h<\cdots <i_{n-d-1}$.
The key is that we can rewrite  \eqref{eqn:simply1} as a matrix form in the following lemma.



 \begin{Lemma}\label{lemma:recover}
For given $\ell \in [0,s^{d})$ and $\tau\in [0,s)$,  when fixing the failed node $i_u\in\mathcal{I}, u\in[0,h)$, the  $(r s^{n-d-1})\times (rs^{n-d-1})$ coefficient matrix of left hand side (LHS) of \eqref{eqn:simply1} is the following \textbf{\textit{recover matrix}}
 	\begin{IEEEeqnarray}{c}\label{eqn:repair matrix0}
 			\setlength{\arraycolsep}{0.3pt}
 	\small	R =\left(\begin{array}{ccccccccccccccc}
 			%	\mathcal{F}{q,0} & \mathcal{F}{q,1} &  \cdots & \mathcal{F}{q,s-1} & \mathcal{F}_{i_1}(q) \\
 			%	\hline\\
 			D  & \mathbf{0} & \cdots &\mathbf{0} & B_{0,i_0} & \cdots & B_{0,i_{u-1}}& B_{0,i_{u+1}} & \cdots  &B_{0,i_{h-1}} & B_{0,i_h} & \cdots  & B_{0,i_{n-d-1}}\\
 			\mathbf{0} & D&  \cdots &\mathbf{0} & B_{1,i_0}  & \cdots & B_{1,i_{u-1}}& B_{1,i_{u+1}} & \cdots & B_{1,i_{h-1}} & B_{1,i_h} & \cdots  & B_{1,i_{n-d-1}}\\
 			\vdots &  \vdots & \ddots & \vdots & \vdots & \ddots & \vdots & \vdots & \ddots & \vdots& \vdots & \ddots & \vdots \\
 			\mathbf{0}  & \mathbf{0} & \cdots &D   &B_{s^{n-d-1}-1,i_0} &\cdots & B_{s^{n-d-1}-1,i_{u-1}}& B_{s^{n-d-1}-1,i_{u+1}} & \cdots& B_{s^{n-d-1}-1,i_{h-1}}& B_{s^{n-d-1}-1,i_h} & \cdots & B_{s^{n-d-1}-1,i_{n-d-1}}\\
 		\end{array}\right),
 	\end{IEEEeqnarray}
where $D$  is a $r\times s$ matrix given by
\begin{eqnarray}\label{eqn:D}
D&=&	\left(\begin{array}{cccccc}
		1 & 1 & \cdots & 1 & 1  \\
		\lambda_{i_u,1} & \lambda_{i_u,2} & \cdots &\lambda_{i_u,s-1} &\lambda_{i_u,0}  \\
		\vdots& \vdots & \ddots & \vdots& \vdots \\
		\prod\limits_{j\in[0,r-2]}\lambda_{i_u,\langle j-r+3\rangle} &
		\prod\limits_{j\in[0,r-2]}\lambda_{i_u,\langle j-r+4\rangle} & \cdots &\prod\limits_{j\in[0,r-2]}\lambda_{i_u,\langle j-r +s+1\rangle}& \prod\limits_{j\in[0,r-2]}\lambda_{i_u,\langle j-r +2\rangle}  \\
	\end{array}\right),
\end{eqnarray}
 and
 \begin{eqnarray*}
 	%	\setlength{\arraycolsep}{0.8pt}
 	B_{m,i_{v}} =\left( B_{m,i_{v}}(t,j)\right)_{r\times s^{{n-d-1}}} ,\quad i_v\in\mathcal{I}\cup\mathcal{U}\backslash\{i_u\},\,m\in[0,s^{n-d-1}),j\in[0,s^{n-d-1}),\,t\in[0,r)
 \end{eqnarray*}
 is a $r\times s^{n-d-1}$ matrix with each row having only one nonzero element, i.e.,
 \begin{eqnarray}\label{eqn:BIJ}
 	%	\setlength{\arraycolsep}{0.8pt}
 	B_{m,i_{v}}(t,j)=\left\{\begin{array}{lll}
 		\zeta_{t,i_{v},\beta_{m}^{(\ell,\tau)}-t\mathbf{e}_{n,i_u}},  & \mathrm{~~if~} j=\mathbf{m}+ t\mathbf{e}_{n-d-1,v}\mathrm{~~and~} v<u,\\
 		\zeta_{t,i_{v},\beta_{m}^{(\ell,\tau)}-t\mathbf{e}_{n,i_u}},  & \mathrm{~~if~} j=\mathbf{m}+ t\mathbf{e}_{n-d-1,v-1}\mathrm{~~and~} v >u,\\
 		0, & \mathrm{~~else}
 	\end{array}
 	\right.
 \end{eqnarray}
with $\mathbf{m}=(m_0,m_1,\dots,m_{n-d-2})\in[0,s^{n-d-1})$, $j\in[0,s^{n-d-1}),t\in[0,r)$, and $ \mathcal{B}_{\ell,\tau}\triangleq
\{\beta_{m}^{(\ell,\tau)}:m\in[0,s^{n-d-1})\}$.
 \end{Lemma}

 \begin{proof}
 The proof is given in Appendix.
 \end{proof}



By  Lemma \ref{lemma:recover},  Steps 3 and 4 are feasible  if the matrix $R$ is nonsingular.

 \begin{Lemma}\label{lemma:download}
 	If the recover matrix $R$ in \eqref{eqn:repair matrix0} is nonsingular, during Step 3 the failed node $i_u,u\in[0,h)$ can recover
 	the data in \eqref{Eqn_Re_data_1} and \eqref{Eqn_Re_data_2} with the help of the  downloaded data in \eqref{Eqn_Dd_data}.
 \end{Lemma}
\begin{proof}
		Given  $j\in \mathcal{J}$, it is easy to check $\{\mathbf{a}-t\mathbf{e}_{n,i_u}+ t\mathbf{e}_{n,j}:a\in[0,s^n)\}=\{a:a\in[0,s^n)\}$ for each $t\in[0,r)$, which means
	\begin{eqnarray*}\label{eqn:down}
	&&	\left\{\sum_{w=0}^{s-2}f_{j,\mathbf{a}+ \langle w-t\rangle\mathbf{e}_{n,i_u}+ t\mathbf{e}_{n,j}}^{(w)}+f_{j,\mathbf{a}- \langle t+1\rangle\mathbf{e}_{n,i_u}+ t\mathbf{e}_{n,j}}^{(u+s-1)}: a\in [0,s^n)\right\}\\
		&=&\left\{\sum_{w=0}^{s-2}f_{j,\mathbf{a}+  w\mathbf{e}_{n,i_u}}^{(w)}+f_{j,\mathbf{a}+ (s-1)\mathbf{e}_{n,i_u}}^{(u+s-1)}: a\in [0,s^n)\right\}.
	\end{eqnarray*}
	Then, after the failed node $i_u,u\in[0,h)$  downloads the data in \eqref{Eqn_Dd_data} from $d$ helper nodes, the RHS of \eqref{eqn:simply1} is known.
	
	Thus, for fixed $\ell\in [0,s^d)$ and $\tau\in [0,s)$, there are $(s+h-1+n-d-h)s^{n-d-1}=r\cdot s^{n-d-1}$ variables in the $r\cdot s^{n-d-1}$ equations
	in \eqref{eqn:simply1}, where $s=d-k+1$ and $r=n-k$.
	Therefore, if $R$ is nonsingular,
	the failed node $i_u\in\mathcal{I}$ is able to recover all the data in RHS of \eqref{Eqn_Re_data_1} and \eqref{Eqn_Re_data_2}.
	That is to say, when $\ell$ and $\tau$ enumerate  $[0,s^{d})$ and $[0,s)$ respectively, all the desirable data in LHS of \eqref{Eqn_Re_data_1} and \eqref{Eqn_Re_data_2} can be repaired. This completes the proof.
\end{proof}


\begin{Lemma}\label{thm:coop} The $h$ failed nodes of an $(n,k,d,h,N)$ Zigzag  MSR code can be recovered  in Step 4 with the help of the exchanging data in \eqref{Eqn_Co_data}.
\end{Lemma}
\begin{proof}
According to Lemma \ref{lemma:download}, if $R$ is nonsingular,  the failed node $i_u\in \mathcal{I}$  has the data in LHS of \eqref{Eqn_Re_data_1} and the failed node $i_v,v\in[0,h)\backslash\{u\}$ possesses the data in \eqref{Eqn_Co_data} after Step 3. Then,   node $i_u$ can recover $f_{i_{u,a}}^{(v+s-1)}$ from \eqref{Eqn_Co_data} by means of the data recovered in \eqref{Eqn_Re_data_1}.
When $v$ enumerates $[0,h)\backslash\{u\}$,  node $i_u$ obtains all the data
$$\bigcup_{w=0}^{h+s-2} f_{i_u,a}^{(w)}=\bigcup_{w=0}^{d-k+h-1} f_{i_u,a}^{(w)},\quad a\in[0,s^n).$$
That is, any node $i_u\in \mathcal{I}$ can be recovered.
\end{proof}

Based on Lemmas \ref{lemma:download} and \ref{thm:coop}, we have the following main result.

\begin{Theorem}\label{P-invert}
If all the recover matrices $R$ in \eqref{eqn:repair matrix0} are nonsingular for any $i_u\in \mathcal{I}$, then
	the corresponding $h$ failed nodes of the $(n,k,d,h,N)$ Zigzag MSR codes with sub-packetization $N=(d-k+h)s^n$ can be optimally cooperative repaired by the four Steps above.
\end{Theorem}
\begin{proof}
By Lemmas \ref{lemma:download} and \ref{thm:coop}, the failed nodes can be recovered by our scheme when matrices $R$ in \eqref{eqn:repair matrix0}
are nonsingular for each $i_u\in \mathcal{I}$. We now proceed to determine the repair bandwidth.
During the download phase, according to Lemma \ref{lemma:download},
each failed node downloads $s^n=N/(d-k+h)$ symbols from each of the $d=n-h$ helper nodes,
i.e., the repair bandwidth in this phase is
\begin{eqnarray*}
	\gamma_1= h(n-h)\cdot{N\over d-k+h}.
\end{eqnarray*}

Next, during the cooperative phase,
according to Lemma \ref{thm:coop},
each failed node downloads $s^n=N/(d-k+h)$ symbols from the other failed nodes,
i.e., the repair bandwidth in this phase is
\begin{eqnarray*}
	\gamma_2 = h(h-1)\cdot{N\over d-k+h}.
\end{eqnarray*}

Totally, the repair bandwidth is
\begin{equation*}
	\begin{split}
		\gamma =\gamma_1+\gamma_2=\frac{h(n-1)N}{d-k+h},
	\end{split}
\end{equation*}
attaining the optimal repair bandwidth given in \eqref{eqn:Lower_Bound}, which finishes the proof.
\end{proof}

\begin{Example}\label{Example_2}
	Continue with Example \ref{Example_1}. Let $h=2$ and generate $d-k+h=3$ instances of the $(n=6,k=2,d=3,h=1,N=2^6)$ Zigzag MSR code in Example \ref{Example_1}.
	In this way, we  obtain an $(n=6,k=2,d=3,h=2,N=3\cdot2^6)$ Zigzag MSR code.  Denote  the column vector of length $N=3\cdot2^6$ stored at node $i$ by $\mathbf{f}_i=((\mathbf{f}_i^{(0)})^\top,(\mathbf{f}_i^{(1)})^\top, (\mathbf{f}_i^{(2)})^\top)^\top$, where $\mathbf{f}_{i}^{(w)}=\{f_{i,a}^{(w)}:a\in[0,63]\}^\top, i\in[0,5], w\in[0,2]$ is a column vector of length $64$.
	
	Assume that nodes $0$ and $1$ fail, and the helper nodes are $j\in[3,5]$, thus, the unconnected node is $2$.
	
	%	Take  the repair of the node $0$ as an example.
	
	\textbf{Step 1 (Summing):}	
	For $t\in[0,3]$, by \eqref{eqn:w parity}-\eqref{Eqn: summing parity}, we get $64$ SPCEs of node $0$ as
	\begin{eqnarray*}
		\sum_{i=0}^5\left(\zeta_{t,i,a}f_{i,\mathbf{a}+ t\mathbf{e}_{6,i}}^{(0)}+\zeta_{t,i,\mathbf{a}+ \mathbf{e}_{6,0}}f_{i,\mathbf{a}+ \mathbf{e}_{6,0}+ t\mathbf{e}_{6,i}}^{(1)}\right) = 0, \quad %\nonumber \\
		a\in [0,63],t\in[0,3].
	\end{eqnarray*}
	%following \eqref{eqn:zeta}, \eqref{eqn:A_{t,i,a}} and \eqref{eqn:a-th parity}.	
	
	\textbf{Step 2 (Grouping):}
	According to \eqref{eqn:a}, divide $a\in[0,63]$ into $8$ groups, i.e.,
	\begin{eqnarray*}
		&&\mathcal{A}_0=\{0,1,2,3,4,5,6,7\},\mathcal{A}_1=\{8,9,10,11,12,13,14,15\},\mathcal{A}_2=\{16,17,18,19,20,21,22,23\},\\
		&&\mathcal{A}_3=\{24,25,26,27,28,29,30,31\},
		\mathcal{A}_4=\{32,33,34,35,36,37,38,39\},\mathcal{A}_5=\{40,41,42,43,44,45,46,47\},\\
		&&\mathcal{A}_6=\{48,49,50,51,52,53,54,55\},\mathcal{A}_7=\{55,56,57,58,49,60,61,62,63\}.
	\end{eqnarray*}
	 By \eqref{eqn:b}, partition the $8$ symbols in $\mathcal{A}_0$ to $2$ subgroups as
	\begin{eqnarray*}
		\mathcal{B}_{0,0}=\{0,3,5,6\},\mathcal{B}_{0,1}=\{1,2,4,7\}.
	\end{eqnarray*}
		
	Take  the $(\ell=0,\tau=0)$-th subgroup as an example.  According to \eqref{eqn:simply1}, SPCEs $\{(\mathbf{a}-t\mathbf{e}_{6,0},t):a\in\mathcal{B}_{0,0},t\in[0,3]\}=\{(0,0),(3,0),(5,0),(6,0),(1,1),(2,1),(4,1),(7,1),(0,2),(3,2),(5,2),(6,2),(1,3),(2,3),(4,3),(7,3)\}$ in the $(\ell=0,\tau=0)$-th subgroup are
	\begin{eqnarray}\label{eqn:10}
		&&\left(\zeta_{t,0,\mathbf{a}-t\mathbf{e}_{6,0}}f_{0,a}^{(0)}+\zeta_{t,0,\mathbf{a}+ \langle 1-t\rangle\mathbf{e}_{6,0}}f_{0,\mathbf{a}+ \mathbf{e}_{6,0}}^{(1)}\right) +
		\sum_{i=1}^2\zeta_{t,i,a}\left(f_{i,\mathbf{a}-t\mathbf{e}_{6,0}+ t\mathbf{e}_{6,i}}^{(0)}+f_{i,\mathbf{a}+\langle 1-t \rangle\mathbf{e}_{6,0}+ t\mathbf{e}_{6,i}}^{(1)}\right)\nonumber \\
		& =& -\sum_{j=3}^5\zeta_{t,j,a}\left(f_{j,\mathbf{a}-t\mathbf{e}_{6,0}+ t\mathbf{e}_{6,j}}^{(0)}+f_{j,\mathbf{a}+\langle 1-t \rangle\mathbf{e}_{6,0}+ t\mathbf{e}_{6,j}}^{(1)}\right), \quad %
		a\in \mathcal{B}_{0,0},t\in[0,3].
	\end{eqnarray}	
	That is, when $t=0,1,2,3$, 	we have the following $16$  equations
	\begin{eqnarray*}\label{eqn:1_0+1_1}
		f_{0,0}^{\left(0\right)}+f_{0,1}^{\left(1\right)}+f_{1,0}^{\left(0\right)}+f_{1,1}^{\left(1\right)}+
		f_{2,0}^{\left(0\right)}+f_{2,1}^{\left(1\right)}&=& -\left(f_{3,0}^{\left(0\right)}+f_{3,1}^{\left(1\right)}+ f_{4,0}^{\left(0\right)}+f_{4,1}^{\left(1\right)}+ f_{5,0}^{\left(0\right)}+f_{5,1}^{\left(1\right)}\right), \\
		f_{0,0}^{\left(0\right)}+\gamma f_{0,1}^{\left(1\right)}+\gamma^2\left(f_{1,3}^{\left(0\right)}+f_{1,2}^{\left(1\right)}\right)+
	\gamma^3\left(f_{2,5}^{\left(0\right)}+f_{2,4}^{\left(1\right)}\right)&=&-\left(\gamma^4\left(f_{3,9}^{\left(0\right)}+f_{3,8}^{\left(1\right)}\right)+\gamma^5\left(f_{4,17}^{\left(0\right)}+f_{4,16}^{\left(1\right)}\right)+\gamma^6\left(f_{5,33}^{\left(0\right)}+f_{5,32}^{\left(1\right)}\right) \right), \\	
		\gamma \left(f_{0,0}^{\left(0\right)}+f_{0,1}^{\left(1\right)}\right)+\gamma^2\left(f_{1,0}^{\left(0\right)}+f_{1,1}^{\left(1\right)}\right)
	+\gamma^3\left(f_{2,0}^{\left(0\right)}+f_{2,1}^{\left(1\right)}\right)	&=&-\left(\gamma^4\left(f_{3,0}^{\left(0\right)}+f_{3,1}^{\left(1\right)}\right)+\gamma^5\left(f_{4,0}^{\left(0\right)}+f_{4,1}^{\left(1\right)}\right)+\gamma^6\left(f_{5,0}^{\left(0\right)}+f_{5,1}^{\left(1\right)}\right)\right), 	\\
		\gamma f_{0,0}^{\left(0\right)}+\gamma^2 f_{0,1}^{\left(1\right)}+\gamma^4\left(f_{1,3}^{\left(0\right)}+f_{1,2}^{\left(1\right)}\right)
		+\gamma^6\left(f_{2,5}^{\left(0\right)}+f_{2,4}^{\left(1\right)}\right)&=&-\left(\gamma^8\left(f_{3,9}^{\left(0\right)}+f_{3,8}^{\left(1\right)}\right)+\gamma^{10}\left(f_{4,17}^{\left(0\right)}+f_{4,16}^{\left(1\right)}\right) +\gamma^{12}\left(f_{5,33}^{\left(0\right)}+f_{5,32}^{\left(1\right)}\right)\right), \\	
		f_{0,3}^{\left(0\right)}+f_{0,2}^{\left(1\right)}+f_{1,3}^{\left(0\right)}+f_{1,2}^{\left(1\right)}
		+f_{2,3}^{\left(0\right)}+f_{2,2}^{\left(1\right)}&=&-\left(f_{3,3}^{\left(0\right)}+f_{3,2}^{\left(1\right)}+f_{4,3}^{\left(0\right)}+f_{4,2}^{\left(1\right)} +f_{5,3}^{\left(0\right)}+f_{5,2}^{\left(1\right)} \right), \\
		\gamma f_{0,3}^{\left(0\right)}+f_{0,2}^{\left(1\right)}+ \left(f_{1,0}^{\left(0\right)}+f_{1,1}^{\left(1\right)}\right)
		+\gamma^3\left(f_{2,6}^{\left(0\right)}+f_{2,7}^{\left(1\right)}\right)&=&-\left(\gamma^4\left(f_{3,10}^{\left(0\right)}+f_{3,11}^{\left(1\right)}\right)+\gamma^5\left(f_{4,18}^{\left(0\right)}+f_{4,19}^{\left(1\right)}\right)+\gamma^6\left(f_{5,34}^{\left(0\right)}+f_{5,35}^{\left(1\right)}\right) \right),\\
		\gamma \left(f_{0,3}^{\left(0\right)}+f_{0,2}^{\left(1\right)}\right)+\gamma^2 \left(f_{1,3}^{\left(0\right)}+f_{1,2}^{\left(1\right)}\right)
		+\gamma^3\left(f_{2,3}^{\left(0\right)}+f_{2,2}^{\left(1\right)}\right)&=&-\left(\gamma^4\left(f_{3,3}^{\left(0\right)}+f_{3,2}^{\left(1\right)}\right)+\gamma^5\left(f_{4,3}^{\left(0\right)}+f_{4,2}^{\left(1\right)}\right)+\gamma^6\left(f_{5,3}^{\left(0\right)}+f_{5,2}^{\left(1\right)}\right)\right),\\
				\gamma^2 f_{0,3}^{\left(0\right)}+\gamma f_{0,2}^{\left(1\right)}+ \gamma^2 \left(f_{1,0}^{\left(0\right)}+f_{1,1}^{\left(1\right)}\right)
		+\gamma^6\left(f_{2,6}^{\left(0\right)}+f_{2,7}^{\left(1\right)}\right)&=&-\left(\gamma^8\left(f_{3,10}^{\left(0\right)}+f_{3,11}^{\left(1\right)}\right)+\gamma^{10}\left(f_{4,18}^{\left(0\right)}+f_{4,19}^{\left(1\right)}\right) +\gamma^{12}\left(f_{5,34}^{\left(0\right)}+f_{5,35}^{\left(1\right)}\right)\right), \\
			f_{0,5}^{\left(0\right)}+f_{0,4}^{\left(1\right)}+f_{1,5}^{\left(0\right)}+f_{1,4}^{\left(1\right)}+
		f_{2,5}^{\left(0\right)}+f_{2,4}^{\left(1\right)}&=&- \left(f_{3,5}^{\left(0\right)}+f_{3,4}^{\left(1\right)}+ f_{4,5}^{\left(0\right)}+f_{4,4}^{\left(1\right)}+ f_{5,5}^{\left(0\right)}+f_{5,4}^{\left(1\right)}\right), \\
		f_{0,5}^{\left(0\right)}+\gamma f_{0,4}^{\left(1\right)}+\gamma^2\left(f_{1,6}^{\left(0\right)}+f_{1,7}^{\left(1\right)}\right)+
		\left(f_{2,0}^{\left(0\right)}+f_{2,1}^{\left(1\right)}\right)&=&-\left(\gamma^4\left(f_{3,12}^{\left(0\right)}+f_{3,13}^{\left(1\right)}\right)+\gamma^5\left(f_{4,20}^{\left(0\right)}+f_{4,21}^{\left(1\right)}\right)+\gamma^6\left(f_{5,36}^{\left(0\right)}+f_{5,37}^{\left(1\right)}\right)\right), \\	
		\gamma \left(f_{0,5}^{\left(0\right)}+f_{0,4}^{\left(1\right)}\right)+\gamma^2\left(f_{1,5}^{\left(0\right)}+f_{1,4}^{\left(1\right)}\right)
		+\gamma^3\left(f_{2,5}^{\left(0\right)}+f_{2,4}^{\left(1\right)}\right)	&=&-\left(\gamma^4\left(f_{3,5}^{\left(0\right)}+f_{3,4}^{\left(1\right)}\right)+\gamma^5\left(f_{4,5}^{\left(0\right)}+f_{4,4}^{\left(1\right)}\right)+\gamma^6\left(f_{5,5}^{\left(0\right)}+f_{5,4}^{\left(1\right)}\right)\right),
		\\
		\gamma f_{0,5}^{\left(0\right)}+\gamma^2 f_{0,4}^{\left(1\right)}+\gamma^4\left(f_{1,6}^{\left(0\right)}+f_{1,7}^{\left(1\right)}\right)
		+\gamma^3\left(f_{2,0}^{\left(0\right)}+f_{2,1}^{\left(1\right)}\right)&=&-\left(\gamma^8\left(f_{3,12}^{\left(0\right)}+f_{3,13}^{\left(1\right)}\right)+\gamma^{10}\left(f_{4,20}^{\left(0\right)}+f_{4,21}^{\left(1\right)}\right) +\gamma^{12}\left(f_{5,36}^{\left(0\right)}+f_{5,37}^{\left(1\right)}\right)\right),  \\	
		f_{0,6}^{\left(0\right)}+f_{0,7}^{\left(1\right)}+f_{1,6}^{\left(0\right)}+f_{1,7}^{\left(1\right)}
		+f_{2,6}^{\left(0\right)}+f_{2,7}^{\left(1\right)}&=&-\left(f_{3,6}^{\left(0\right)}+f_{3,7}^{\left(1\right)}+f_{4,6}^{\left(0\right)}+f_{4,7}^{\left(1\right)} +f_{5,6}^{\left(0\right)}+f_{5,7}^{\left(1\right)} \right), \\
		\gamma f_{0,6}^{\left(0\right)}+f_{0,7}^{\left(1\right)}+ \left(f_{1,5}^{\left(0\right)}+f_{1,4}^{\left(1\right)}\right)
		+\left(f_{2,3}^{\left(0\right)}+f_{2,2}^{\left(1\right)}\right)&=&-\left(\gamma^4\left(f_{3,15}^{\left(0\right)}+f_{3,14}^{\left(1\right)}\right)
+\gamma^5\left(f_{4,23}^{\left(0\right)}+f_{4,22}^{\left(1\right)}\right)+\gamma^6\left(f_{5,39}^{\left(0\right)}+f_{5,38}^{\left(1\right)}\right) \right), \\
		\gamma \left(f_{0,6}^{\left(0\right)}+f_{0,7}^{\left(1\right)}\right)+\gamma^2 \left(f_{1,6}^{\left(0\right)}+f_{1,7}^{\left(1\right)}\right)
		+\gamma^3\left(f_{2,6}^{\left(0\right)}+f_{2,7}^{\left(1\right)}\right)&=&-\left(\gamma^4\left(f_{3,6}^{\left(0\right)}+f_{3,7}^{\left(1\right)}\right)+\gamma^5\left(f_{4,6}^{\left(0\right)}+f_{4,7}^{\left(1\right)}\right)+\gamma^6\left(f_{5,6}^{\left(0\right)}+f_{5,7}^{\left(1\right)}\right)\right),\\
		\gamma^2 f_{0,6}^{\left(0\right)}+\gamma f_{0,7}^{\left(1\right)}+ \gamma^2\left(f_{1,5}^{\left(0\right)}+f_{1,4}^{\left(1\right)}\right)
		+\gamma^3\left(f_{2,3}^{\left(0\right)}+f_{2,2}^{\left(1\right)}\right)&=&-\left(\gamma^8\left(f_{3,15}^{\left(0\right)}+f_{3,14}^{\left(1\right)}\right)+\gamma^{10}\left(f_{4,23}^{\left(0\right)}+f_{4,22}^{\left(1\right)}\right) +\gamma^{12}\left(f_{5,39}^{\left(0\right)}+f_{5,38}^{\left(1\right)}\right) \right).
	\end{eqnarray*}
	
	\textbf{Step 3 (Downloading phase):}
	 The LHS of the $16$ equations can be rewritten as
	\begin{eqnarray*}
		RX=\left(\begin{array}{cccccccc|ccccccccc}
			1 & 1 & 0 &0 & 0 &0 & 0 & 0 & 1 & 0 & 0 & 0& 1 & 0 & 0 & 0\\
			1 & \gamma &0 & 0 &0 & 0 & 0 & 0 & 0& \gamma^2 & 0 & 0 & 0 & 0& \gamma^3 & 0\\
			\gamma & \gamma &0 & 0 &0 & 0 & 0 & 0 & \gamma^2 & 0 & 0 & 0 & \gamma^3 & 0 & 0 & 0\\
			\gamma & \gamma^2 &0 & 0 &0 & 0 & 0 & 0 &0 & \gamma^4& 0 & 0 & 0 & 0& \gamma^6 & 0\\
			0 & 0 & 1 & 1 &0 & 0 &0 & 0 &0 & 1 & 0 & 0 &0 & 1 & 0 & 0\\
			0 & 0 &  1 & \gamma &0 & 0 &0 & 0 &1 & 0 & 0 & 0 & 0 & 0 & 0 & \gamma^3\\
			0 & 0 & \gamma & \gamma &0 & 0 &0 & 0 & 0 & \gamma^2& 0 & 0  & 0 & \gamma^3 & 0 & 0\\
			0 & 0 & \gamma & \gamma^2 &0 & 0 &0 & 0 & \gamma^2& 0& 0 & 0& 0 & 0 & 0 & \gamma^6 \\
			
			0 & 0 &0 & 0 &1 & 1 & 0 & 0& 0 & 0 & 1 & 0  & 0 & 0 & 1 & 0\\
			0 & 0 &0 & 0 &1 & \gamma & 0 & 0& 0 & 0 & 0& \gamma^2 & 1 & 0 & 0 & 0 \\
			0 & 0 &0 & 0 &\gamma & \gamma & 0& 0 & 0 & 0 & \gamma^2 &  0 & 0  & 0 & \gamma^3 & 0\\
			0 & 0 &0 & 0 &\gamma & \gamma^2& 0 & 0 & 0& 0 & 0& \gamma^4 & \gamma^3 & 0 & 0 & 0 \\
			0 & 0 &0 & 0 &0 & 0 & 1 & 1 & 0 & 0& 0 & 1  & 0 & 0& 0 & 1 \\
			0 & 0 &0 & 0 &0 & 0 &  1 & \gamma & 0& 0 &1 & 0 & 0 &  1 & 0 & 0  \\
			0 & 0 &0 & 0 &0 & 0 & \gamma & \gamma & 0 & 0& 0 & \gamma^2&  0 & 0  & 0 & \gamma^3\\
			0 & 0 &0 & 0 &0 & 0 & \gamma & \gamma^2 & 0 & 0 & \gamma^2& 0 & 0 & \gamma^3 & 0 & 0
		\end{array}
		\right)\begin{pmatrix}
			f_{0,0}^{(0)} \\
			f_{0,1}^{(1)}\\
			f_{0,2}^{(1)} \\
			f_{0,3}^{(0)} \\
			f_{0,5}^{(0)} \\
			f_{0,4}^{(1)} \\
			f_{0,7}^{(1)}\\
			f_{0,6}^{(0)} \\
			f_{1,0}^{(0)}+f_{1,1}^{(1)}\\
			f_{1,2}^{(1)}+f_{1,3}^{(0)}\\
			f_{1,4}^{(1)}+f_{1,5}^{(0)}\\
			f_{1,6}^{(0)}+f_{1,7}^{(1)}\\
			f_{2,0}^{(0)}+f_{2,1}^{(1)}\\
			f_{2,2}^{(1)}+f_{2,3}^{(0)}\\
			f_{2,4}^{(1)}+f_{2,5}^{(0)}\\
			f_{2,6}^{(0)}+f_{2,7}^{(1)}\\
		\end{pmatrix}.
	\end{eqnarray*}
Perform the following elementary  row transformation on $R$, i.e.,
\begin{eqnarray*}
	R\rightarrow\left(\begin{array}{cccccccc|ccccccccc}
		1 & 1 & 0 &0 & 0 &0 & 0 & 0 & 1 & 0 & 0 & 0& 1 & 0 & 0 & 0\\
		1 & \gamma &0 & 0 &0 & 0 & 0 & 0 & 0& \gamma^2 & 0 & 0 & 0 & 0& \gamma^3 & 0\\
		0 & 0 & 1 & 1 &0 & 0 &0 & 0 &0 & 1 & 0 & 0 &0 & 1 & 0 & 0\\
		0 & 0 &  1 & \gamma &0 & 0 &0 & 0 &1 & 0 & 0 & 0 & 0 & 0 & 0 & \gamma^3\\
		0 & 0 &0 & 0 &1 & 1 & 0 & 0& 0 & 0 & 1 & 0  & 0 & 0 & 1 & 0\\
		0 & 0 &0 & 0 &1 & \gamma & 0 & 0& 0 & 0 & 0& \gamma^2 & 1 & 0 & 0 & 0 \\
		0 & 0 &0 & 0 &0 & 0 & 1 & 1 & 0 & 0& 0 & 1  & 0 & 0& 0 & 1 \\
		0 & 0 &0 & 0 &0 & 0 &  1 & 0 & 0& 0 &1 & 0 & 0 &  1 & 0 & 0  \\
		0 & 0 &0 & 0 &0 & 0 & 0 & 0 & \gamma^2-\gamma & 0 & 0 & 0 & \gamma^3-\gamma & 0 & 0 & 0\\
		0 & 0 &0 & 0 &0 & 0 & 0 & 0 &0 & \gamma^4-\gamma^3& 0 & 0 & 0 & 0& \gamma^6-\gamma^4 & 0\\
		0 & 0 &0 & 0 &0 & 0 &0 & 0 & 0 & \gamma^2-\gamma& 0 & 0  & 0 & \gamma^3 -\gamma& 0 & 0\\
		0 & 0 & 0 & 0 &0 & 0 &0 & 0 & \gamma^2-\gamma& 0& 0 & 0& 0 & 0 & 0 & \gamma^6-\gamma^4 \\
		0 & 0 &0 & 0 &0& 0 & 0& 0 & 0 & 0 & \gamma^2-\gamma &  0 & 0  & 0 & \gamma^3-\gamma & 0\\
		0 & 0 &0 & 0 &0 & 0 & 0 & 0 & 0& 0 & 0& \gamma^4-\gamma^3 & \gamma^3-\gamma & 0 & 0 & 0 \\
		0 & 0 &0 & 0 &0 & 0 & 0 & 0 & 0 & 0& 0 & \gamma^2-\gamma&  0 & 0  & 0 & \gamma^3-\gamma\\
		0 & 0 &0 & 0 &0 & 0 & 0& 0 & 0 & 0 & \gamma^2-\gamma& 0 & 0 & \gamma^3-\gamma & 0 & 0
	\end{array}
	\right).
\end{eqnarray*}
Since the lower right corner matrix
\begin{eqnarray*}
	R'=\left(\begin{array}{cccccccc}
		 \gamma^2-\gamma & 0 & 0 & 0 & \gamma^3-\gamma & 0 & 0 & 0\\
		0 & \gamma^4-\gamma^3& 0 & 0 & 0 & 0& \gamma^6-\gamma^4 & 0\\
		 0 & \gamma^2-\gamma& 0 & 0  & 0 & \gamma^3 -\gamma& 0 & 0\\
		 \gamma^2-\gamma& 0& 0 & 0& 0 & 0 & 0 & \gamma^6-\gamma^4 \\
		 0 & 0 & \gamma^2-\gamma &  0 & 0  & 0 & \gamma^3-\gamma & 0\\
		 0& 0 & 0& \gamma^4-\gamma^3 & \gamma^3-\gamma & 0 & 0 & 0 \\
		 0 & 0& 0 & \gamma^2-\gamma&  0 & 0  & 0 & \gamma^3-\gamma\\
		 0 & 0 & \gamma^2-\gamma& 0 & 0 & \gamma^3-\gamma & 0 & 0
	\end{array}
	\right)
\end{eqnarray*}
is equivalent with
%\begin{eqnarray*}
%	\left(\begin{array}{cccccccccccccccc}
%		\gamma^2-\gamma & 0 & 0 & 0 & \gamma^3-\gamma & 0 & 0 & 0 \\
%		0 & \gamma^2-\gamma & 0 & 0 & 0& \gamma^3-\gamma  & 0 & 0 \\
%		0 & 0  & \gamma^2-\gamma & 0 & 0 & 0& \gamma^3-\gamma & 0 \\
%		0 & 0 & 0&\gamma^2-\gamma  & 0 & 0 & 0 & \gamma^3-\gamma \\
%			
%		
%		
%		\gamma^2-\gamma& 0 & 0 & 0  & 0 & 0  & 0 & \gamma^6-\gamma^4 \\
%		0 & \gamma^4-\gamma^3 & 0 & 0  & 0 & 0 & \gamma^6-\gamma^4 & 0 \\
%		
%		0 & 0& \gamma^2-\gamma  & 0  & 0  & \gamma^3-\gamma & 0 & 0\\
%		0  & 0& 0 & \gamma^4-\gamma^3  & \gamma^3-\gamma & 0  & 0 & 0 \\
%			
%	\end{array}
%	\right)
%\end{eqnarray*}
\begin{eqnarray*}
	\left(\begin{array}{cccccccccccccccc}
		\gamma^2-\gamma & 0 & 0 & 0 & \gamma^3-\gamma & 0 & 0 & 0\\
		0 & \gamma^2-\gamma& 0 & 0  & 0 & \gamma^3 -\gamma& 0 & 0\\
		0 & 0 & \gamma^2-\gamma &  0 & 0  & 0 & \gamma^3-\gamma & 0\\
		0 & 0& 0 & \gamma^2-\gamma&  0 & 0  & 0 & \gamma^3-\gamma\\	
		0& 0& 0 & 0& \gamma-\gamma^3 & 0 & 0 & \gamma^6-\gamma^4 \\
		0 & 0 & 0 & 0 & 0 & \gamma^3-\gamma & \gamma-\gamma^3 & 0\\
		0 & 0 & 0 & 0 & 0 &  0& (\gamma^4-\gamma^3)(\gamma^2-1)& 0\\
		0& 0 & 0& 0 & 0& 0 & 0 & \gamma^6+\gamma^3-\gamma^5-\gamma^4 \\
	\end{array}
	\right),
\end{eqnarray*}
which is invertible.
	Then, the equation \eqref{eqn:10} has a unique solution because of $\mathrm{Rank}(R)=16$.
	
	
	Hence, node $0$ can recover
	\begin{eqnarray*}
		f_{0,0}^{(0)},f_{0,1}^{(1)},f_{0,2}^{(1)} ,f_{0,3}^{(0)},f_{0,5}^{(0)},f_{0,4}^{(1)},f_{0,7}^{(1)},f_{0,6}^{(0)}
	\end{eqnarray*} and
	\begin{eqnarray*}	&&f_{1,0}^{(0)}+f_{1,1}^{(1)},f_{1,2}^{(1)}+f_{1,3}^{(0)},f_{1,4}^{(1)}+f_{1,5}^{(0)},f_{1,6}^{(0)}+f_{1,7}^{(1)},\\
	&&f_{2,0}^{(0)}+f_{2,1}^{(1)},f_{2,2}^{(1)}+f_{2,3}^{(0)},f_{2,4}^{(1)}+f_{2,5}^{(0)},f_{2,6}^{(0)}+f_{2,7}^{(1)}
	\end{eqnarray*}
by downloading
	\begin{eqnarray*}
		&&f_{3,0}^{(0)}+f_{3,1}^{(1)},f_{3,9}^{(0)}+f_{3,8}^{(1)},f_{3,3}^{(0)}+f_{3,2}^{(1)},f_{3,10}^{(0)}+f_{3,11}^{(1)},f_{3,5}^{(0)}+f_{3,4}^{(1)},f_{3,12}^{(0)}+f_{3,13}^{(1)},f_{3,6}^{(0)}+f_{3,7}^{(1)},f_{3,15}^{(0)}+f_{3,14}^{(1)},\\
	&&	f_{4,0}^{(0)}+f_{4,1}^{(1)},f_{4,17}^{(0)}+f_{4,16}^{(1)},f_{4,3}^{(0)}+f_{4,2}^{(1)},f_{4,18}^{(0)}+f_{4,19}^{(1)},f_{4,5}^{(0)}+f_{4,4}^{(1)},f_{4,20}^{(0)}+f_{4,21}^{(1)},f_{4,6}^{(0)}+f_{4,7}^{(1)},f_{4,23}^{(0)}+f_{4,22}^{(1)},\\
	&&	f_{5,0}^{(0)}+f_{5,1}^{(1)},f_{5,33}^{(0)}+f_{5,32}^{(1)},f_{5,3}^{(0)}+f_{5,2}^{(1)},f_{5,34}^{(0)}+f_{5,35}^{(1)},f_{5,5}^{(0)}+f_{5,4}^{(1)},f_{5,36}^{(0)}+f_{5,37}^{(1)},f_{5,6}^{(0)}+f_{5,7}^{(1)},f_{5,39}^{(0)}+f_{5,38}^{(1)}
		\end{eqnarray*}
	from the helper nodes $j\in[3,5]$ according to \eqref{Eqn_Dd_data}.
	
	In the same way, node $0$ can recover $f_{0,a}^{(0)},f_{0,a}^{(1)}$ and
	node $1$ can recover $f_{0,a}^{(0)}+f_{0,\mathbf{a}+ \mathbf{e}_{6,1}}^{(2)},a\in[0,63]$ in the download phase.
	
	\textbf{Step 4 (Cooperative phase):}
	During the cooperative phase, node $1$ transfers $f_{0,a}^{(0)}+f_{0,\mathbf{a}+ \mathbf{e}_{6,1}}^{(2)}$ to node $0$, such that node $0$ can recover $f_{0,\mathbf{a}+ \mathbf{e}_{6,1}}^{(2)}$, i.e., $f_{0,a}^{(2)},a\in[0,63]$.
	Thus, node $0$ can be recovered.
	
	The repair process of node $1$ is similar to that of node $0$ and we omit it here.	
\end{Example}





	


Theorem \ref{P-invert} and Lemma \ref{lemma:recover} indicate that the $h$ failed nodes of $(n,k,d,h,N)$ Zigzag MSR codes can be optimally cooperative repaired if the recover matrix $R$ is nonsingular. In the next section, we will show that the matrix $R$ is indeed nonsingular when any $h$ nodes fail. 	

\section{The Recover Matrix for any $h$ Failed Nodes}\label{sec:h=2,3}
%	\subsection{The recoverability of any $h$ failed nodes}	
For an $(n,k,d,h,N)$ Zigzag MSR codes, recall that the $h$ failed nodes are $\mathcal{I}=\{i_0,\dots,i_{h-1}\}$, and the unconnected nodes are $\mathcal{U}=\{i_h,\dots,i_{n-d-1}\}$. In this case, $r=n-k=s+n-d-1$, where $s=d-k+1$.
Before showing the invertibility of recover matrix $R$ in \eqref{eqn:repair matrix0}, we analyze the matrix $B_{m,i_v},m\in\mathcal{R}=[0,s^{n-d-1}),i_v\in\mathcal{I}_u=\mathcal{I}\cup\mathcal{U}\backslash \{i_u\}$.
	For convenience, denote $\mathcal{T}=[0,n-d-2], \mathcal{N}=[0,N=s^n), \mathcal{S}=[0,s)$ and $\mathcal{S}_r=[s,r)$   in the following.
\begin{Lemma}\label{Lemma:B}
	Let	$B_{m,\mathcal{I}_u}\triangleq(B_{m,i_0},\dots,B_{m,i_{u-1}},B_{m,i_{u+1}},\dots,B_{m,i_{h-1}},B_{m,i_h},\dots,B_{m,i_{n-d-1}}),m\in\mathcal{R}$ and
 denote the first $n-d-1$ rows of $B_{m,\mathcal{I}_u}$ by $B_{m,\mathcal{I}_u}(\mathcal{T}),m\in\mathcal{R}$.
Then,
 $$B_{\mathcal{R},\mathcal{I}_u}(\mathcal{T})\triangleq\left(B^{\top}_{0,\mathcal{I}_u}(\mathcal{T}),\dots,
 B^{\top}_{s^{n-d-1}-1,\mathcal{I}_u}(\mathcal{T})\right)^\top$$
is an $(n-d-1)s^{n-d-1}\times (n-d-1)s^{n-d-1}$ invertible matrix over $\mathbb{F}_q,q\ge n+1$, where $\top$ is the transpose operator.
\end{Lemma}
\begin{proof}
Firstly,
%applying  \eqref{eqn:zeta} to \eqref{eqn:BIJ}, we have
%\begin{eqnarray*}\label{eqn:betat-1}
%	\zeta_{t,i_{v},\beta_{m}^{(\ell,\tau)}-t\mathbf{e}_{n,i_u}}
%	&=&	\left\{\begin{array}{lll}
%		1, & \text{ if } t= 0,\\
%		\prod\limits_{0\leq j < t}\lambda_{i,\langle{(\beta_{m}^{(\ell,\tau)}-t\mathbf{e}_{n,i_u})_{i_v}}+j\rangle}, &\text{ else}
%	\end{array}	
%	\right.
%\end{eqnarray*}
%for $t\in\mathcal{T},i_v\in\mathcal{I}_u,m\in \mathcal{R}$.
%\begin{eqnarray}\label{eqn:betat-1}
%	\zeta_{t,i_{v},\beta_{m}^{(\ell,\tau)}-t\mathbf{e}_{n,i_u}}&= & \left\{\begin{array}{lll}	
%		\gamma^{i_{v}+1}, & \mathrm{if}~ t\ne 0, s>n-d-2, (\beta_{m}^{(\ell,\tau)})_{i_v}+g=0~\mathrm{for ~a~} g\in[0,t),\\
%		\gamma^{\lceil \frac{t}{s}\rceil(i_v+1)}, & \mathrm{if}~ t\ne 0, s<n-d-2,
%		(\beta_{m}^{(\ell,\tau)})_{i_v}+g=0~\mathrm{for ~a~} g\le t\bmod s,\\
%		\gamma^{\lfloor \frac{t}{s}\rfloor(i_v+1)}, & \mathrm{if}~ t\ne 0, s<n-d-2,
%		 (\beta_{m}^{(\ell,\tau)})_{i_v}+g=0~\mathrm{for ~a~} g> t\bmod s,\\
%		1, &\text{ else}
%	\end{array}	
%	\right.
%\end{eqnarray}
 we show the matrix $B_{\mathcal{R},\mathcal{I}_u}(\mathcal{T})$  determined by the entry in \eqref{eqn:BIJ}, i.e.,
 \begin{eqnarray}\label{eqn:BIJ-1}
 	%	\setlength{\arraycolsep}{0.8pt}
 	B_{m,i_{v}}(t,j)=\left\{\begin{array}{lll}
 		\zeta_{t,i_{v},\beta_{m}^{(\ell,\tau)}-t\mathbf{e}_{n,i_u}},  & \text{~~if~} j=\mathbf{m}+ t\mathbf{e}_{n-d-1,v}\text{~~and~} v<u,\\
 		\zeta_{t,i_{v},\beta_{m}^{(\ell,\tau)}-t\mathbf{e}_{n,i_u}},  & \text{~~if~} j=\mathbf{m}+ t\mathbf{e}_{n-d-1,v-1}\text{~~and~} v >u,\\
 		0, & \text{~~else,}
 	\end{array}
 	\right.
 \end{eqnarray}
for all $m\in \mathcal{R}$, is equivalent to a submatrix of
the parity check matrix of an $(n,k'=d+1,d'=2d-k+1,h,N=s^n)$ Zigzag  code over $\mathbb{F}_q,q\ge n+1$. In this code, the number of parity check nodes is  $r=n-d-1$.


According to the definition of $(n,k'=d+1,d'=2d-k+1,h,N=s^n)$ Zigzag MSR codes in \eqref{zigzag:A_i:parameter}-\eqref{A_i^t}, the $N\times N$ parity check matrices of  $n-d-1$ nodes $i_v\in\mathcal{I}_u$, i.e.,
\begin{eqnarray}\label{eqn:ATT}
	A_{\mathcal{T},\mathcal{I}_u} &=&\left(\begin{array}{ccccccc}
		A_{0,i_0} & \cdots& A_{0,i_{u-1}} & A_{0,i_{u+1}}& \cdots & A_{0,i_{n-d-1}}\\
		A_{1,i_0} & \cdots& A_{1,i_{u-1}} & A_{1,i_{u+1}}& \cdots & A_{1,i_{n-d-1}}\\
		\vdots & \ddots& \vdots& \vdots & \ddots & \vdots\\
		A_{n-d-2,i_0} & \cdots& A_{n-d-2,i_{u-1}} & A_{n-d-2,i_{u+1}}& \cdots & A_{n-d-2,i_{n-d-1}}
	\end{array}
	\right)
\end{eqnarray}
are characterized by
\begin{eqnarray}\label{eqn:AB}	
	A_{t,i_v}(a,j)=\left\{\begin{array}{lll}
		\zeta_{t,i_v,a},  & \text{~~if~} j=\mathbf{a}+ t\mathbf{e}_{n,i_v},\\
		0, & \text{~~else,}
	\end{array}
	\right.
\end{eqnarray}
where $t\in\mathcal{T},\mathbf{a}=(a_0,a_1,\dots,a_{n-1})\in\mathcal{N},j\in\mathcal{N}$.

Let
$$\beta_{m}^{(\ell,\tau)}=\left((\beta_{m}^{(\ell,\tau)})_0,\dots,(\beta_{m}^{(\ell,\tau)})_{i_0},
\dots,(\beta_{m}^{(\ell,\tau)})_{i_1},\dots,(\beta_{m}^{(\ell,\tau)})_{i_{n-d-1}},\dots,(\beta_{m}^{(\ell,\tau)})_{n-1}\right)$$
with
\begin{eqnarray}\label{eqn:beta_m}
	\left((\beta_{m}^{(\ell,\tau)})_{i_0},(\beta_{m}^{(\ell,\tau)})_{i_1},\dots,(\beta_{m}^{(\ell,\tau)})_{i_{n-d-1}}\right)=
\left(m_0,m_1,\dots,m_{u-1},(\beta_{m}^{(\ell,\tau)})_{i_u},m_{u},m_{u+1},\dots,m_{n-d-2}\right).
\end{eqnarray}
Define
$\mathcal{L}\triangleq
\left\{\beta_{m}^{(\ell,\tau)}: m\in\mathcal{R}\right\}\subseteq\mathcal{N}$ and $\overline{\mathcal{L}}=\mathcal{N}\backslash \mathcal{L}$.

By means of the reorder of  rows and columns, $A_{t,i_v}$ ($t\in\mathcal{T},i_v\in\mathcal{I}_u$) can be rewritten as
\begin{eqnarray*}
	A_{t,i_v}= A_{t,i_v}(\mathcal{N},\mathcal{N})\rightarrow	
	\left(\begin{array}{ll}
		A_{t,i_v}(\mathcal{L},\mathcal{L}) &A_{t,i_v}(\mathcal{L},\overline{\mathcal{L}})\\
		A_{t,i_v}(\overline{\mathcal{L}},\mathcal{L}) & A_{t,i_v}(\overline{\mathcal{L}},\overline{\mathcal{L}})
	\end{array}
	\right).
\end{eqnarray*}
By  \eqref{eqn:AB} and  \eqref{eqn:beta_m},
\begin{eqnarray}\label{eqn:ATI}
	(A_{t,i_v}(\mathcal{L},\mathcal{L}))(m,j)=\left\{\begin{array}{lll}
		\zeta_{t,i_v,\beta_{m}^{(\ell,\tau)}},  & \text{~~if~} j=\mathbf{m}+ t\mathbf{e}_{n-d-1,v} \text{~and~} v<u,\\
		\zeta_{t,i_v,\beta_{m}^{(\ell,\tau)}},  & \text{~~if~} j=\mathbf{m}+ t\mathbf{e}_{n-d-1,v-1} \text{~and~} v>u,\\
		0, & \text{~~else,}
	\end{array}
	\right.
\end{eqnarray}
where $t\in\mathcal{T},i_v\in\mathcal{I}_u,\beta_{m}^{(\ell,\tau)}\in \mathcal{L},m\in\mathcal{R}, j\in\mathcal{R}$.

%Applying \eqref{eqn:zeta} to \eqref{eqn:ATI}, we know
%	\begin{eqnarray*}\label{eqn_ztv}
%		\zeta_{t,i_{v},\beta_{m}^{(\ell,\tau)}}
%		&=&	\left\{\begin{array}{lll}
%			1, & \text{ if } t= 0,\\
%			\prod\limits_{0\leq j < t}\lambda_{i,\langle {(\beta_{m}^{(\ell,\tau)})_{i_v}}+j\rangle},
%			&\text{ else}
%		\end{array}	
%		\right.
%	\end{eqnarray*}
%for $t\in\mathcal{T},i_v\in\mathcal{I}_u,m\in \mathcal{R}$.
It follows  P2 that
$$\zeta_{t,i_{v},\beta_{m}^{(\ell,\tau)}-t\mathbf{e}_{n,i_u}}=\zeta_{t,i_v,\beta_{m}^{(\ell,\tau)}}.$$
Then, from \eqref{eqn:BIJ-1}  and \eqref{eqn:ATI} we obtain
\begin{eqnarray}\label{eqn:abm}
	(A_{t,i_v}(\mathcal{L},\mathcal{L}))(m)=B_{m,i_v}(t),t\in\mathcal{T},i_v\in\mathcal{I}_u,m\in\mathcal{R}.
\end{eqnarray}
%and then
%\begin{eqnarray}\label{eqn:ABMV}
%	A_{\mathcal{T},i_v}(\mathcal{L},\mathcal{L})(m)=B_{m,i_v}(\mathcal{T}).
%\end{eqnarray}



Next, note from the MDS property of $(n,k'=d+1,d'=2d-k+1,h,N=s^n)$ Zigzag MSR codes that the matrix $A_{\mathcal{T},\mathcal{I}_u}$ in \eqref{eqn:ATT}
is an $(n-d-1)N\times (n-d-1)N$ invertible matrix over $\mathbb{F}_q$. With respect to the row exchange,
\begin{eqnarray*}
	A_{\mathcal{T},\mathcal{I}_u}
		&\rightarrow&	\left(\begin{array}{cccccccccccc}
		A_{\mathcal{T},\mathcal{I}_u}(\mathcal{L},\mathcal{L}) & A_{\mathcal{T},\mathcal{I}_u}(\mathcal{L},\overline{\mathcal{L}}) \\
		A_{\mathcal{T},\mathcal{I}_u}(\overline{\mathcal{L}},\mathcal{L}) & A_{\mathcal{T},\mathcal{I}_u}(\overline{\mathcal{L}},\overline{\mathcal{L}})
	\end{array}	\right)\\
&=&			\left(\begin{array}{cccccccccccc}
		A_{\mathcal{T},\mathcal{I}_u}(\mathcal{L},\mathcal{L}) & \mathbf{0} \\
		A_{\mathcal{T},\mathcal{I}_u}(\overline{\mathcal{L}},\mathcal{L}) & A_{\mathcal{T},\mathcal{I}_u}(\overline{\mathcal{L}},\overline{\mathcal{L}})
	\end{array}
	\right),
\end{eqnarray*}
where $ A_{\mathcal{T},\mathcal{I}_u}(\mathcal{L},\overline{\mathcal{L}})$ is a  zero matrix since
$$j\in\left\{\beta_{m}^{(\ell,\tau)}+ t\mathbf{e}_{n,i_v}:~m\in\mathcal{R},t\in\mathcal{T},i_v\in\mathcal{I}_u\right\}=\mathcal{L}\subseteq\mathcal{N}$$
according to \eqref{eqn:AB} and \eqref{eqn:beta_m},
i.e., $A_{t,i_v}(\beta_{m}^{(\ell,\tau)},j)=0$ when $j\in \overline{\mathcal{L}}$ for any $t\in\mathcal{T},i_v\in\mathcal{I}_u,\beta_{m}^{(\ell,\tau)}\in\mathcal{L},m\in\mathcal{R}$.
Therefore, $A_{\mathcal{T},\mathcal{I}_u}(\mathcal{L},\mathcal{L})$ is an $(n-d-1)s^{n-d-1}\times (n-d-1)s^{n-d-1}$ invertible matrix. Whereas,
\begin{eqnarray*}
		A_{\mathcal{T},\mathcal{I}_u}(\mathcal{L},\mathcal{L})
%		&\rightarrow	&	\footnotesize \left(\begin{array}{cccccccccccc}
%			A_{\mathcal{T},i_0}(\mathcal{L},\mathcal{L})(0) & \cdots &A_{\mathcal{T},i_{u-1}}(\mathcal{L},\mathcal{L})(0)&A_{\mathcal{T},i_{u+1}}(\mathcal{L},\mathcal{L})(0)& \cdots &A_{\mathcal{T},i_{n-d-1}}(\mathcal{L},\mathcal{L})(0)\\
%			A_{\mathcal{T},i_0}(\mathcal{L},\mathcal{L})(1) & \cdots &A_{\mathcal{T},i_{u-1}}(\mathcal{L},\mathcal{L})(1)&A_{\mathcal{T},i_{u+1}}(\mathcal{L},\mathcal{L})(0)& \cdots &A_{\mathcal{T},i_{n-d-1}}(\mathcal{L},\mathcal{L})(1)\\
%			\vdots  \\
%			A_{\mathcal{T},i_0}(\mathcal{L},\mathcal{L})(s^{n-d-1}-1) & \cdots &A_{\mathcal{T},i_{u-1}}(\mathcal{L},\mathcal{L})(s^{n-d-1}-1)&A_{\mathcal{T},i_{u+1}}(\mathcal{L},\mathcal{L})(s^{n-d-1}-1)& \cdots &A_{\mathcal{T},i_{n-d-1}}(\mathcal{L},\mathcal{L})(s^{n-d-1}-1)
%		\end{array}
%		\right)\\
		&\rightarrow&\left(\begin{array}{cccccccccccc}
			A_{0,\mathcal{I}_u}(\mathcal{L},\mathcal{L})(0)\\
			\vdots   \\
				A_{n-d-2,\mathcal{I}_u}(\mathcal{L},\mathcal{L})(0)\\
			A_{0,\mathcal{I}_u}(\mathcal{L},\mathcal{L})(1)\\
			\vdots   \\
			A_{n-d-2,\mathcal{I}_u}(\mathcal{L},\mathcal{L})(1)\\
			\vdots  \\
			A_{0,\mathcal{I}_u}(\mathcal{L},\mathcal{L})(s^{n-d-1}-1)\\
			\vdots   \\
			A_{n-d-2,\mathcal{I}_u}(\mathcal{L},\mathcal{L})(s^{n-d-1}-1)\\		
		\end{array}
		\right)=\left(\begin{array}{c}
			B_{0,I_u}(0)\\
		\vdots \\
		B_{0,I_u}(n-d-2)\\
			B_{1,I_u}(0)\\
		\vdots \\
		B_{1,I_u}(n-d-2)\\
			\vdots \\
			B_{s^{n-d-1}-1,I_u}(0)\\
		\vdots \\
		B_{s^{n-d-1}-1,I_u}(n-d-2)
			\end{array}
	\right)
%&=	\left(\begin{array}{c}
%		B_{0,I_u}(\mathcal{T})\\
%			B_{1,I_u}(\mathcal{T})\\
%		\vdots\\
%			B_{s^{n-d-1}-1,I_u}(\mathcal{T})
%		\end{array}
%	\right)
\rightarrow B_{\mathcal{R},\mathcal{I}_u}(\mathcal{T}),
\end{eqnarray*}
where the equation holds from \eqref{eqn:abm}. This completes the proof.
\end{proof}

\begin{Lemma}\label{lemma:h}
	For any $h$ failed nodes $\mathcal{I}=\{i_0,i_1,\dots,i_{h-1}\}$, the recover matrix $R$ in \eqref{eqn:repair matrix0} is nonsingular.
\end{Lemma}
\begin{proof}
	We only show the case of node $i_0$ as  that of other $h-1$ nodes is similar.
	For the failed node $i_0$, we  split the submatrix of \eqref{eqn:repair matrix0} in two parts: the first $s$ rows, denoted by $D(\mathcal{S})$ and $B_{m,i_v}(\mathcal{S}),m\in\mathcal{R},v\in[1,n-d)$, and the last $r-s=n-d-1$ rows,
	denoted by $D(\mathcal{S}_r)$ and $B_{m,i_v}(\mathcal{S}_r)$, respectively. Recall from \eqref{eqn:D} that
		\begin{eqnarray*}
		D = \left(\begin{array}{cc}
			D(\mathcal{S})\\
			D(\mathcal{S}_r)
		\end{array}
		\right)=\left(
		\begin{array}{cccccc}
			1  & 1  & \cdots & 1\\
			1  & 1  & \cdots & \gamma^{i_0+1}\\
			\vdots & \vdots & \ddots & \vdots\\
			1  & \gamma^{i_0+1}  & \cdots & \gamma^{i_0+1}\\
			\gamma^{i_0+1}  & \gamma^{i_0+1}  & \cdots & \gamma^{i_0+1}\\
			\gamma^{i_0+1}  & \gamma^{i_0+1}  & \cdots & \gamma^{2(i_0+1)}\\
\vdots & \vdots & \vdots & \vdots
		\end{array}
		\right)\in \mathbb{F}_q^{r\times s}.
	\end{eqnarray*}

Then, we perform the following elementary  row transformation on $R$, i.e.,
	\begin{eqnarray*}
		R\rightarrow R'=
 \left(\begin{array}{cccccccc}
		D(\mathcal{S}) & \mathbf{0} & \cdots & \mathbf{0} &{B_{0,\mathcal{I}_u}}(\mathcal{S}) \\
		\mathbf{0} & D(\mathcal{S}) &  \cdots & \mathbf{0}  &{B_{1,\mathcal{I}_u}}(\mathcal{S})\\
		\vdots & \vdots &\ddots & \vdots  & \vdots \\
		\mathbf{0} & \mathbf{0} & \cdots & 	D(\mathcal{S} )  &{B_{s^{n-d-1}-1,\mathcal{I}_u}}(\mathcal{S}) \\				
		\mathbf{0} & \mathbf{0} & \cdots & \mathbf{0} &\Lambda'
			\end{array}\right),
	\end{eqnarray*}
where $\mathcal{I}_u=\{i_1,i_2,\dots,i_{n-d-1}\}$ and
\begin{eqnarray*}
	\Lambda'&=& {B_{\mathcal{R},\mathcal{I}_u}}(\mathcal{S}_r)-\gamma^{i_0+1}{B_{\mathcal{R},\mathcal{I}_u}}(\mathcal{T})\\
	&=&
	\left(\begin{array}{cccccccc}
		{B_{0,\mathcal{I}_u}}(\mathcal{S}_r)-\gamma^{i_0+1}{B_{0,\mathcal{I}_u}}(\mathcal{T})\\
		{B_{1,\mathcal{I}_u}}(\mathcal{S}_r)-\gamma^{i_0+1}{B_{1,\mathcal{I}_u}}(\mathcal{T})\\
		\vdots\\
		{B_{s^{n-d-1}-1,\mathcal{I}_u}}(\mathcal{S}_r)-\gamma^{i_0+1}{B_{s^{n-d-1}-1,\mathcal{I}_u}}(\mathcal{T})
	\end{array}\right).
\end{eqnarray*}

From  \eqref{zigzag:A_i:parameter} and \eqref{eqn:zeta}, we know
$$A_{s+t,i_v}(m)= \gamma^{i_v+1}A_{t,i_v}(m),\quad  t\in\mathcal{T},v\in[1,n-d),m\in\mathcal{R}.$$
Thus,
$$(A_{s+t,i_v}(\mathcal{L},\mathcal{L}))(m)= \gamma^{i_v+1}(A_{t,i_v}(\mathcal{L},\mathcal{L}))(m),\quad  t\in\mathcal{T},v\in[1,n-d),m\in\mathcal{R}.$$
Following \eqref{eqn:abm}, it means,
$${B_{m,i_v}}(s+t)= \gamma^{i_v+1}{B_{m,i_v}}(t),\quad t\in\mathcal{T},v\in[1,n-d),m\in\mathcal{R},$$
i.e.,
$$	B_{m,i_v}(\mathcal{S}_r)= \gamma^{i_v+1}{B_{m,i_v}}(\mathcal{T}),\quad m\in\mathcal{R},v\in[1,n-d),$$
which results in
\begin{eqnarray*}
\det(\Lambda')=(\gamma^{i_1+1}-\gamma^{i_0+1})(\gamma^{i_2+1}-\gamma^{i_0+1})\cdots (\gamma^{i_{n-d-1}+1}-\gamma^{i_0+1})\det (B_{\mathcal{R},\mathcal{I}_u}(\mathcal{T})).
%
%\left(\begin{array}{cccccccc}
%	{B_{0,\mathcal{I}_u}}(\mathcal{T})\\
%		{B_{1,\mathcal{I}_u}}(\mathcal{T}) \\
%		\vdots  \\
%		{B_{s^{n-d-1}-1,\mathcal{I}_u}}(\mathcal{T})
%	\end{array}\right).
\end{eqnarray*}

Following Lemma \ref{Lemma:B}, we know $\det(\Lambda')\ne 0$ since $\det (B_{\mathcal{R},\mathcal{I}_u}(\mathcal{T}))\ne 0$ and $(\gamma^{i_v+1}-\gamma^{i_0+1})\ne 0,i_v\in\mathcal{I}$,
where $\gamma$ is a primitive element of $\F_q$.
This implies that  the recover matrix $R$ given by \eqref{eqn:repair matrix0} is nonsingular, which completes the proof.
	\end{proof}
	
	
	The following result is a direct consequence of  Theorem  \ref{P-invert} and Lemma \ref{lemma:h}.
	\begin{Theorem}
		Any $h$ failed nodes of $(n,k,k\le d\le n-h,h,N)$ Zigzag MSR codes with sub-packetization $N=(d-k+h)s^n$ can be optimally cooperative repaired, where $s=d-k+1$.
	\end{Theorem}


	
	\section{Conclusion}\label{sec:conclusion}
	In this paper,
	an optimal cooperative repair scheme for $(n,k,d,h,N)$ Zigzag MSR codes is proposed,
	which can recover any $h$ failed nodes by downloading the data from $k\le d\le n-h$ helper nodes.
	Particularly, the size of finite field is $\mathbb{F}_q,q\ge n+1$, which is smaller than the known results to the best of our knowledge.
	
	
%	\section*{Appendix}
%		\textbf{The proof of Lemma \ref{lemma:recover}:}
	
		
	
%	\section*{Acknowledgment}
%	
%	The authors would like to thank the Associate Editor,
%	Prof. Mingyue Ji, and the anonymous reviewers, whose
%	comments and suggestions improved the presentation of this
%	paper.
%	

\section{Appendix}
\subsection{Proof of Lemma \ref{lemma:recover}}

First of all,   according to \eqref{eqn:a} and \eqref{eqn:b}, for a given $m\in[0,s^{n-d-1})$ there exists a unique
$$\mathbf{a}=(a_0,\dots,a_{i_0},\dots,a_{i_1},\dots,a_{i_{n-d-1}},\dots,a_{n-1})\in \mathcal{B}_{\ell,\tau}$$
such that
\begin{eqnarray}\label{eqn:m-a}
	(a_{i_0},a_{i_1},\dots,a_{i_{n-d-1}})=(m_0,m_1,\dots,m_{u-1},a_{i_u},m_{u},m_{u+1},\dots,m_{n-d-2}),
\end{eqnarray}
where $a_{i_u},u\in[0,h)$, is determined by \eqref{eqn:b} and the other components of $\mathbf{a}$ are determined by \eqref{eqn:a}.
Thus, hereafter we set $\mathbf{a}=\beta_{m}^{(\ell,\tau)},m\in[0,s^{n-d-1})$.



Next, we prove that the LHS of \eqref{eqn:simply1} in matrix form  can be expressed as $R\cdot X$ with
\begin{eqnarray}\label{eqn:X-R}
X=\left(X_0,...,X_{s^{n-d-1}-1},X_{i_0},\dots,X_{i_{u-1}},X_{i_{u+1}},\dots,X_{i_{n-d-1}}\right)^{\top},
\end{eqnarray}
where
\begin{eqnarray}\label{eqn:unknows_j}
		X_j=\left(f_{i_u,\beta_{j}^{(\ell,\tau)}}^{(0)},f_{i_u,\beta_{j}^{(\ell,\tau)}+\mathbf{e}_{n,i_u}}^{(1)},\dots, f_{i_u,\beta_{j}^{(\ell,\tau)} + (s-2)\mathbf{e}_{n,i_u}}^{(s-2)},f_{i_u,\beta_{j}^{(\ell,\tau)} + (s-1) \mathbf{e}_{n,i_u}}^{(u+s-1)}\right),\,j\in[0,s^{n-d-1})
\end{eqnarray}
is a row vector of length  $s$  at node $i_u$ and
	\begin{eqnarray}\label{eqn:B_i'}
	X_{i_v}	&=&\left(\sum_{w=0}^{s-2}f_{i_v,\beta_{0}^{(\ell,\tau)}+  w\mathbf{e}_{n,i_u}}^{(w)}+f_{i_v,\beta_{0}^{(\ell,\tau)}+(s-1)\mathbf{e}_{n,i_u}}^{(u+s-1)},\dots,\right. \nonumber\\
	&&\left.\sum_{w=0}^{s-2}f_{i_v,\beta_{s^{n-d-1}-1}^{(\ell,\tau)}+  w\mathbf{e}_{n,i_u}}^{(w)}+f_{i_v,\beta_{s^{n-d-1}-1}^{(\ell,\tau)}+(s-1) \mathbf{e}_{n,i_u}}^{(u+s-1)}\right),
	v\in[0,n-d)\backslash\{u\}
\end{eqnarray}
is a row vector of length $s^{n-d-1}$  at node $i_{v}, v\ne u$.
%, where identity holds as $\{\mathbf{a}+ \langle w-t\rangle  \mathbf{e}_{n,i_u}+  t\mathbf{e}_{n,i_v}:a\in\mathcal{B}_{\ell,\tau}, w\in[0,d-k+h),t\in[0,r)\}=\{\mathbf{a}+ w  \mathbf{e}_{n,i_u}:a\in\mathcal{B}_{\ell,\tau},w\in[0,d-k+h)\}$ following \eqref{eqn:b}.

The equation \eqref{eqn:unknows_j} is obvious from the fact that
\begin{itemize}
\item [F1.] When $\mathbf{a}=\beta_{m}^{(\ell,\tau)},m\in[0,s^{n-d-1})$, only $X_{m}$  among $X_j,j\in [0,s^{n-d-1})$ appears in LHS of \eqref{eqn:simply1}  for all $t\in[0,r)$.
\end{itemize}

For \eqref{eqn:B_i'}, it is seen from \eqref{eqn:simply1} that for a fixed $t\in[0,r)$, the unknowns of node $i_v\in\mathcal{I}\cup\mathcal{U}\backslash\{i_u\}$ form
\begin{eqnarray}\label{eqn:unknows}
&&\left\{\sum_{w=0}^{s-2}f_{i_v,\beta_{0}^{(\ell,\tau)}+ \langle w-t\rangle\mathbf{e}_{n,i_u}+ t\mathbf{e}_{n,i_v}}^{(w)}+f_{i_v,\beta_{0}^{(\ell,\tau)}-  \langle t+1 \rangle\mathbf{e}_{n,i_u}+ t\mathbf{e}_{n,i_v}}^{(u+s-1)},\dots,\right. \nonumber\\
	&&\left.\sum_{w=0}^{s-2}f_{i_v,\beta_{s^{n-d-1}-1}^{(\ell,\tau)}+ \langle w-t\rangle\mathbf{e}_{n,i_u}+ t\mathbf{e}_{n,i_v}}^{(w)}+f_{i_v,\beta_{s^{n-d-1}-1}^{(\ell,\tau)}-  \langle t+1 \rangle\mathbf{e}_{n,i_u}+ t\mathbf{e}_{n,i_v}}^{(u+s-1)}\right\}.
\end{eqnarray}
Note that if $\beta_{m}^{(\ell,\tau)}\in \mathcal{B}_{\ell,\tau}$, so does $\beta_{m}^{(\ell,\tau)}- t\mathbf{e}_{n,i_u}+ t\mathbf{e}_{n,i_v}$ by \eqref{eqn:b}. Therefore, all the $r$ ($t\in[0,r)$) sets in \eqref{eqn:unknows} are  equivalent to
\begin{eqnarray*}
\left\{\sum_{w=0}^{s-2}f_{i_v,\beta_{0}^{(\ell,\tau)}+ w\mathbf{e}_{n,i_u}}^{(w)}+f_{i_v,\beta_{0}^{(\ell,\tau)}-  \mathbf{e}_{n,i_u}}^{(u+s-1)},\dots, \sum_{w=0}^{s-2}f_{i_v,\beta_{s^{n-d-1}-1}^{(\ell,\tau)}+ w\mathbf{e}_{n,i_u}}^{(w)}+f_{i_v,\beta_{s^{n-d-1}-1}^{(\ell,\tau)}-\mathbf{e}_{n,i_u}}^{(u+s-1)}\right\}
\end{eqnarray*}
regardless of the order,  which is essentially formed by  the elements in \eqref{eqn:B_i'}.

Precisely, we observe the following fact that
\begin{itemize}
\item [F2.]  When  $\mathbf{a}=\beta_{m}^{(\ell,\tau)},m\in[0,s^{n-d-1})$, by \eqref{eqn:m-a}, only the $j$-th component of $X_{i_v}$   appears in LHS of \eqref{eqn:simply1}  for given  $t\in[0,r)$, with
\begin{eqnarray*}\label{eqn:m'}
	j=\left\{\begin{array}{lll}
		\mathbf{m}+ t\mathbf{e}_{n-d-1,v}, & \text{~~if~}v<u,\\
		\mathbf{m}+ t\mathbf{e}_{n-d-1,v-1},& \text{~~if~} v >u.
	\end{array}
	\right.
\end{eqnarray*}
\end{itemize}


Now, we are ready to show the details of  matrix $R$. Let us consider the $r$ SPCEs in subgroup \eqref{eqn:simply1} for given $\mathbf{a}=\beta_{m}^{(\ell,\tau)}\in\mathcal{B}_{\ell,\tau},m\in[0,s^{n-d-1})$, which corresponds to the block matrix $R_{m}$ in the  $m$-th block row of $(r s^{n-d-1})\times (rs^{n-d-1})$ matrix $R$. To be consistent with \eqref{eqn:X-R}, we express $R_{m}$ in matrix
form as
\begin{eqnarray*}\label{eqn:R-X}
R_{m}=\left(D_{m,0},\dots, D_{m,s^{n-d-1}-1}, B_{m,i_0},\dots,B_{m,i_{u-1}},B_{m,i_{u+1}},\dots,B_{m,i_{n-d-1}}\right),
\end{eqnarray*}
where $D_{m,i}$ and $B_{m,i_v}$ are $r\times s$ block matrices and $r\times s^{n-d-1}$ block matrices respectively.

According to F1, $D_{m,j}$ is a zero matrix except for
$j=m$, i.e.,
\begin{eqnarray}\label{eqn:matrixi_u}
	D_{m,m}&=&	\left(\begin{array}{cccccc}
		\zeta_{0,i_u,\beta_m^{(\ell,\tau)}} & \zeta_{0,i_u,\beta_m^{(\ell,\tau)}+\mathbf{e}_{n,i_u}} & \cdots &\zeta_{0,i_u,\beta_m^{(\ell,\tau)}+ (s-1)\mathbf{e}_{n,i_u}}  \\
		\zeta_{1,i_u,\beta_m^{(\ell,\tau)}+ \langle-1\rangle\mathbf{e}_{n,i_u}} & \zeta_{1,i_u,\beta_m^{(\ell,\tau)}} & \cdots &\zeta_{1,i_u,\beta_m^{(\ell,\tau)}+ (s-2)\mathbf{e}_{n,i_u}}  \\
		\vdots& \vdots & \ddots & \vdots\\
		\zeta_{r-1,i_u,\beta_m^{(\ell,\tau)}+  \langle1-r\rangle\mathbf{e}_{n,i_u}} & \zeta_{r-1,i_u,\beta_m^{(\ell,\tau)}+ \langle 2-r\rangle\mathbf{e}_{n,i_u}} & \cdots &\zeta_{r-1,i_u,\beta_m^{(\ell,\tau)}+ \langle s-r\rangle\mathbf{e}_{n,i_u}}
	\end{array}\right)\nonumber\\
	&=&\left(\begin{array}{cccccc}
		1 & 1 & \cdots & 1  \\
		\lambda_{i_u,\langle (\beta_m^{(\ell,\tau)})_{i_u}-1\rangle} & \lambda_{i_u, (\beta_m^{(\ell,\tau)})_{i_u}} & \cdots &\lambda_{i_u,\langle (\beta_m^{(\ell,\tau)})_{i_u} +s-2\rangle}  \\
		\vdots& \vdots & \ddots & \vdots\\
		\prod\limits_{j\in[0,r-2]}\lambda_{i_u,\langle(\beta_m^{(\ell,\tau)})_{i_u}+j+1-r\rangle} &
		\prod\limits_{j\in[0,r-2]}\lambda_{i_u,\langle (\beta_m^{(\ell,\tau)})_{i_u}+j+2-r\rangle} & \cdots &\prod\limits_{j\in[0,r-2]}\lambda_{i_u,\langle (\beta_m^{(\ell,\tau)})_{i_u} +j+s-r\rangle}
	\end{array}\right),
\end{eqnarray}
where the equation holds due to \eqref{eqn:zeta}. As $\langle  (\beta_m^{(\ell,\tau)})_{i_u}+i \rangle$ ranges over $[0,s)$ with
$i$ varying from $-1$ to $s-2$, the coefficient matrix \eqref{eqn:D} can be obtained from \eqref{eqn:matrixi_u} by columns exchanging.

For every $m\in[0,s^{n-d-1}),v\in [0,n-d)\backslash\{u\}$, the submatrix $B_{m,i_v}$ corresponds to the coefficients of  unknowns $X_{i_v}$. Thus, the equation \eqref{eqn:BIJ} follows from  F2. This completes the proof.
	



%When $a=\beta_{m}^{(\ell,\tau)},m\in[0,s^{h-1})$ enumerates $\mathcal{B}_{\ell,\tau},\ell \in[0,s^d), \tau\in[0,s)$,
%since
%\begin{eqnarray*}
%	&&	\{f_{i_v,\mathbf{a}+ \langle w-t\rangle  \mathbf{e}_{n,i_u}+  t\mathbf{e}_{n,i_v}}^{(w)}:w\in[0,s-2],t\in[0,r)\}\cup	\{f_{i_v,\mathbf{a}-  \langle t+1\rangle \mathbf{e}_{n,i_u}+ t\mathbf{e}_{n,i_v}}^{(u+s-1)}:u\in[0,h),t\in[0,r)\}
%	\nonumber\\
%	&	=&\{f_{i_v,\mathbf{a}}^{(w)}:w\in [0,s-2]\cup\{u+s-1\},u\in[0,h)\},
%\end{eqnarray*}
%we obtain the combination of all symbols of node $i_v\in\mathcal{I}\backslash\{i_u\}$ in $X_{i_v},v\in[0,h)]\backslash\{u\}$.

%\subsection{Proof of Lemma \ref{Lemma:B}}




	\bibliographystyle{IEEEtranS}
	\bibliography{Bib}
	
	
\end{document}

