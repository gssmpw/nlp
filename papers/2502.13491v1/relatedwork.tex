\section{Related Work}
Wrinkles can be separated into the \textit{dynamic wrinkles} and \textit{static/persistent wrinkles}~\cite{larboulette2004real}. The former refers to the fine geometrical details and the folds dynamically appear with cloth motions; the latter refers to the permanent deformations with which the cloth can no longer return to its original shape even under no external impact. There are various \textit{dynamic wrinkle} simulation methods which can be categorized into: rule-based methods~\cite{hadap1999animating, cutler2005art}, data-driven methods~\cite{wang2010example, lahner2018deepwrinkles}, and physics-based methods~\cite{kunii1990modeling, bridson2005simulation, wang2021gpu}. Conversely, the research in the formation of \textit{static/persistent wrinkles} is scarce. \cite{pizana2020bending} can only simulate pre-defined wrinkles rather than modeling the physics in their formations. \cite{narain2013folding,guo2018material} use plastic deformation to account for the persistent wrinkles and adopt the hardening plastic model~\cite{gingold2004discrete}. \cite{kim2011persistent} simulates permanent wrinkles by changing the rest shape and material stiffness parameters with deformations. However, they cannot simulate time-dependent wrinkles observed in real cloths \cite{levison1962some} and ignore another important factor for persistent wrinkles: internal friction~\cite{chapman197227,brenner1964mechanical,prevorsek1975influence,chapman1975importance}. \cite{miguel2013modeling} re-parameterizes Dahl's friction model to make it more suitable for cloth simulation and shows that the internal friction can also cause wrinkles. \cite{wong2013modelling} introduces a new model to simulate cloth wrinkles caused by the internal friction and fit cloth bending hysteresis behaviors: the energy loss in clothes' load-deformation processes. In yarn-level cloth simulators~\cite{cirio2014yarn}, the internal friction force is modeled as the sliding friction force between contacting yarns which prevents cloths from unraveling, and the shearing friction force which leads to shearing wrinkles. However, none of these internal friction models is time-dependent. In addition, as discovered by \cite{prevorsek1975influence}, persistent wrinkles are collectively caused by: (1) frictional fabric bending, (2) friction yarn bending, and (3) permanent bending of filaments (can be modeled by plastic deformations). However, in graphics, simulating persistent wrinkles by combining internal friction and plasticity has been rarely explored so far.


%-------------------------------------------------------------------------