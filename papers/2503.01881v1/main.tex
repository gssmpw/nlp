% RLJ main.tex Version 2025.0

\documentclass[10pt]{article} % For LaTeX2e

%%%%%%%%%%%%%%%%%%%%%%%%%%%%%%%%%%%%%%%%%%%%%%%%%%%%%%%%%%%%%%%%
% AUTHOR: Select ONE option:
%      [accepted]{rlj} --> for camera ready (after peer review, if accepted)
%      {rlj}           --> for submission
%      [preprint]{rlj} --> to de-anonymize and remove references to RLJ/RLC
%%%%%%%%%%%%%%%%%%%%%%%%%%%%%%%%%%%%%%%%%%%%%%%%%%%%%%%%%%%%%%%%
\usepackage[preprint]{rlj}           % Should be uncommented for submission
%\usepackage[accepted]{rlj} % Should be uncommented for the camera-ready
%\usepackage[preprint]{rlj} % Should be uncommented for preprint versions

%%%%%%%%%%%%%%%%%%%%%%%%%%%%%%%%%%%%%%%%%%%%%%%%%%%%%%%%%%%%%%%%
% WARNING: The following packages are already included in the
%          rlj.sty style file:
%
%  1. fancyhdr  - For controlling header/footers
%  2. natbib    - For formatting the bibliography
%  3. enumitem  - To customize the appearance of lists
%  4. fontenc (with option [T1]) - Allows for proper hyphenation and accents
%  5. times     - Times new roman font
%  6. ragged2e  - Used to justify text
%  7. tcolorbox - Used to create boxes on cover page
%  8. hyperref  - Configures hyperlinks throughout (e.g., links to references)
%  9. xcolor    - Used to define custom colors for links and boxes
%  10. amsmath  - Not used, but conflicts with lineno, so we include (and patch) it for authors
%  11. etoolbox - Included in the amsmath + lineno patch
%  12. lineno   - For adding line numbers when in submission
%
% You do not need to include them again in your main.tex.
% Including them again may lead to conflicts or compilation errors.
% Additionally, avoid loading packages that might conflict with these.
%%%%%%%%%%%%%%%%%%%%%%%%%%%%%%%%%%%%%%%%%%%%%%%%%%%%%%%%%%%%%%%%

%%%%%%%%%%%%%%%%%%%%%%%%%%%%%%%%%%%%%%%%%%%%%%%%%%%%%%%%%%%%%%%%
% Recommended (but not required) packages
%%%%%%%%%%%%%%%%%%%%%%%%%%%%%%%%%%%%%%%%%%%%%%%%%%%%%%%%%%%%%%%%
\usepackage{amssymb}            % Defines common symbols like \mathbb R
\usepackage{mathtools}          % Extends amsmath, providing common math tools
\usepackage{mathrsfs}           % Enables \mathscr, which can work in cases that \mathcal does not
%\mathtoolsset{showonlyrefs}     % Only number equations that are referenced (optional)
\usepackage{graphicx}           % For including images
\usepackage{subcaption}         % Allows for the use of subfigures and subcaptions
\usepackage[space]{grffile}     % For spaces in image names
\usepackage{url}                % For displaying URLs
\usepackage{lipsum}             % For placeholder text



%%%%%%%%%%%%%%%%%%%%%%%%%%%%%%%%
% MY PACKAGES
%%%%%%%%%%%%%%%%%%%%%%%%%%%%%%%%
%\usepackage[table]{xcolor} % For table cell coloring
\usepackage[utf8]{inputenc} % allow utf-8 input
%\usepackage[T1]{fontenc}    % use 8-bit T1 fonts
\usepackage{booktabs}       % professional-quality tables
%\usepackage{amsfonts}       % blackboard math symbols
\usepackage{nicefrac}       % compact symbols for 1/2, etc.
\usepackage{microtype}      % microtypography
% \usepackage{xcolor}         % colors


%\usepackage{natbib}
%\usepackage{amsmath}
%\usepackage{amssymb}
%\usepackage{cite}
\usepackage{pdfpages}
%\usepackage{graphicx}
\graphicspath{ {./res/} }
%\usepackage{caption}
%\usepackage{subcaption}
\usepackage{multirow}
\usepackage{dirtytalk}
\usepackage{float} 

\usepackage{siunitx}

% if you use cleveref..
\usepackage[capitalize,noabbrev]{cleveref}

\usepackage{overpic} 





%%%%%%%%%%%%%%%%%%%%%%%%%%%%%%%%%%%%%%%%%%%%%%%%%%%%%%%%%%%%%%%%
% AUTHOR: Fill in the following meta-data
%%%%%%%%%%%%%%%%%%%%%%%%%%%%%%%%%%%%%%%%%%%%%%%%%%%%%%%%%%%%%%%%

% Enter the title of your paper:
\title{Mapping representations in Reinforcement Learning via Semantic Alignment for Zero-Shot Stitching}

% The "running title" will be displayed in the header on every-other page.
% It is typically either the same as the title or a shorter version of the title.
% Enter your running title here:
\setrunningtitle{Enter Your Running Title Here}

% WARNING: Authors must not appear in the submitted version. They should be hidden
% as long as the rlj package is used without the [accepted] or [preprint] options.
% Non-anonymous submissions will be rejected without review.

% Enter the author names below. 
% NOTE: Denote affiliations using superscripts as in the provided example.
% NOTE: Use \textscript{1,2,3} instead of $^{1,2,3}$.
%       - Failure to do so will cause affiliation superscripts to appear on the cover page for camera-ready and preprint versions.
%\author{Antonio Pio Ricciardi \textsuperscript{1$\dagger$}, Valentino Maiorca\textsuperscript{4,$\dagger$}, Luca Moschella\textsuperscript{4}, Riccardo Marin\textsuperscript{4}, Emanuele Rodolà\textsuperscript{4}}
\author{Antonio Pio Ricciardi \textsuperscript{1}, Valentino Maiorca\textsuperscript{1}, Luca Moschella\textsuperscript{1}, Riccardo Marin\textsuperscript{2}, Emanuele Rodolà\textsuperscript{1}}

% NOTE: For camera-ready and preprint versions, the cover page includes author names but not affiliations.
% It automatically removes the superscripts for affiliations.
% If the automatic process breaks (e.g., if an author name should include a superscript), you can manually define the author string to appear on the cover page by uncommenting the following line.
%\coverPageAuthor{Marlos C. Machado, Philip S. Thomas, Lorem Ipsum}

% Author emails, which can be clustered if they have shared endings as in this example
%\emails{ricciardi@di.uniroma1.it, \ \{pthomas,lipsum\}@cs.umass.edu}
\emails{ricciardi@di.uniroma1.it}

% Author affiliations, in the order the occur
% The inclusion of state/province, etc. is optional.
% The inclusion of multiple affiliations is optional.
%   - List multiple affiliations with comma-separated numbers as in the example.
\affiliations{
$^{1}$\textbf{Sapienza, Uniersity of Rome}\\
% $^{2}$\textbf{Institute of Science and Technology Austria (ISTA)}\\
$^{2}$\textbf{University of Tübingen, Germany}\\
%$^{4}$\textbf{Manning College of Information and Computer Sciences, University of Massachusetts}
% The following two lines are optional and can be commented out
%\par % If including additional comments like below, use \par to add some whitespace. 
%$^\dagger$ Additional comments can be added like this, e.g., indicating equal contribution
}

\contribution{
    % Contribution
    Provide a succinct but precise list of the contribution(s) of the paper. Use contextual notes to avoid implications of contributions more significant than intended and to clarify and situate the contribution relative to prior work (see the examples below). If there is no additional context, enter ``None''. Try to keep each contribution to a single sentence, although multiple sentences are allowed when necessary. If using complete sentences, include punctuation. If using a single sentence fragment, you may omit the concluding period. A single contribution can be sufficient, and there is no limit on the number of contributions. Submissions will be judged mostly on the contributions claimed on their cover pages and the evidence provided to support them. Major contributions should not be claimed in the main text if they do not appear on the cover page. Overclaiming can lead to a submission being rejected, so it is important to have well-scoped contribution statements on the cover page.
    }
    {
    % Caveat:
    None
    }

\contribution{
    % Contribution
    The submission template for submissions to RLJ/RLC 2025
    }
    {
    % Caveat:
    Built from previous RLC/RLJ, ICLR, and TMLR submission templates
    }

\contribution{
    % Contribution
    \textit{{[}Example of one contribution and corresponding contextual note for the paper ``Policy gradient methods for reinforcement learning with function approximation'' \citep{Sutton2000}.]}\\ This paper presents an expression for the policy gradient when using function approximation to represent the action-value function.
    }
    {
    % Context:
    Prior work established expressions for the policy gradient without function approximation \citep{Williams1992}.
    }

% Include a list of keywords for the topic of the paper:
\keywords{RLJ, RLC, formatting guide, style file, \LaTeX~template.} % Your keywords

% Define the summary that appears on the cover page.
\summary{The summary appears on the cover page. Although it can be identical to the abstract, it does not have to be. One might choose to omit the stated contributions in the Summary, given that they will be stated in the box below. The original abstract may also be extended to two paragraphs. The authors should ensure that the contents of the cover page fit entirely on a single page. The cover page does \textbf{not} count towards the 8--12 page limit.

\lipsum[1]
}


\newcommand{\AR}[1]{{\textcolor{red}{[\textbf{AR:} #1]}}}
\newcommand{\LM}[1]{{\textcolor{gray}{[\textbf{LM:} #1]}}}
\newcommand{\RM}[1]{{\textcolor{blue}{[\textbf{RM:} #1]}}}
\newcommand{\VM}[1]{{\textcolor{teal}{[\textbf{VM:} #1]}}}

% Define colors
\definecolor{cellblue}{RGB}{173,216,230}



%%%%%%%%%%%%%%%%%%%%%%%%%%%%%%%%%%%%%%%%%%%%%%%%%%%%%%%%%%%%%%%%
%% Begin document, create title and abstract
%%%%%%%%%%%%%%%%%%%%%%%%%%%%%%%%%%%%%%%%%%%%%%%%%%%%%%%%%%%%%%%%
\begin{document}

%\makeCover  % Create the cover page
\maketitle  % Make the title section

% \begin{abstract}
% Adversarial attacks pose significant threats to deploying Graph Neural Networks (GNNs) in real-world applications. Lines of studies have made progress in minimizing the influence of adversarial perturbations. However, existing methods often rely on fixed priors about the dataset or attacker, limiting their ability to generalize across diverse scenarios. These approaches cannot adaptively learn the intrinsic properties of the dataset.
% In this paper, we propose a novel framework, \ModelName (Graph \textbf{P}urification through t\textbf{R}ansfer \textbf{EN}tropy-guided \textbf{N}on-i\textbf{S}otropic Diffu\textbf{S}ion), which leverages a graph diffusion generative model to learn intrinsic properties and recover the clean structure of adversarial graphs. However, two key challenges arise: (1) The graph diffusion model’s uniform noise injection to all nodes during the forward process can over-perturb the graph, erasing valuable information and making recovery difficult, and (2) the diversity of the diffusion model in the reverse denoising process may cause the generated graph to deviate from the target clean structure.
% To address these challenges, we introduce a LID-Driven Non-Isotropic Diffusion process, which injects noise selectively, focusing on adversarial nodes while preserving the clean structure. Additionally, we propose a Graph Transfer Entropy-Guided Reverse Denoising process that maximizes transfer entropy to reduce uncertainty in the reverse process, ensuring that the generated graph remains aligned with the clean structure.
% Extensive experiments on both graph and node classification demonstrate our proposed \ModelName framework's robustness and superior generalization.
% Our code is available at \textcolor{mytablecolor}{\url{https:///}}
% \end{abstract}


% \begin{abstract}
% % priors and no priors-free 继续总结凝练
% % 鲁棒和各向同性有关
% Adversarial attacks pose significant threats to deploying Graph Neural Networks (GNNs) in real-world applications. Lines of studies have made progress in minimizing the influence of adversarial perturbations. 
% They often rely on priors such as neighbor similarity in clean graphs to restore the correct structure. However, this approach is less effective on datasets where these priors do not hold.
% % Their robustness methods often rely on priors of clean graphs or attacks.
% To achieve more generalized robustness, we need methods that can learn clean graph properties and recover the correct structure based on those learned properties, rather than depending on prior assumptions.
% Driven by this goal, in this work, we approach adversarial attacks from a distribution perspective: these attacks cause the graph distribution to deviate from the original clean distribution.
% % From this perspective, we propose using a graph generative model to learn the clean graph distribution without relying on priors and to purify adversarial graphs through distribution mapping.
% % 前面不要,直接提diffusion
% % Among various graph generative models, the diffusion model’s reverse denoising process naturally aligns with the removal of adversarial perturbations,
% % making it an ideal choice for mapping between adversarial and clean distributions. 
% From this perspective, we propose using the graph diffusion model to learn the clean graph distribution and purify adversarial graphs through distribution mapping.
% % The diffusion model’s reverse denoising process naturally aligns with the removal of adversarial perturbations, making it an ideal choice for mapping between adversarial and clean distributions.
% However, in graph diffusion models, 1) the indiscriminate noise injection across all nodes during the forward process can remove useful information still present in adversarial samples, making it difficult to recover the clean structure during reverse purification. 2) the diversity of the reverse denoising process may cause the generated graph to deviate from the target clean structure.
% To address these challenges, we propose a novel framework, \ModelName, to enhance gra\textbf{P}h rob\textbf{U}stness through t\textbf{R}ansfer \textbf{EN}tropy guid\textbf{E}d non-i\textbf{S}otropic diffu\textbf{S}ion purification.
% Our method introduces a LID-based Non-Isotropic Diffusion process, where we use local intrinsic dimensionality (LID) to estimate the adversarial degree of each node, enabling selective noise injection to focus on adversarial nodes while preserving the clean structure. Additionally, we propose a Graph Transfer Entropy-Guided Denoising process, which maximizes transfer entropy at each step to reduce uncertainty during the reverse process, 
% % ensuring the generated graph stays aligned with the clean structure without deviation.
% ensuring the generated graph matches the target clean graph without deviation.
% Extensive experiments on both graph and node classification tasks demonstrate the robustness of our \ModelName framework. Our code is available at \textcolor{mytablecolor}{\url{https:///}}.
% \end{abstract}

\begin{abstract}
Adversarial evasion attacks pose significant threats to graph learning, with lines of studies that have improved the robustness of Graph Neural Networks (GNNs).
However, existing works rely on priors about clean graphs or attacking strategies, which are often heuristic and inconsistent.
To achieve robust graph learning over different types of evasion attacks and diverse datasets, we investigate this problem from a prior-free structure purification perspective.
Specifically, we propose a novel \underline{\textbf{Diff}}usion-based \underline{\textbf{S}}tructure \underline{\textbf{P}}urification framework named \textbf{\ModelName}, which creatively incorporates the graph diffusion model to learn intrinsic distributions of clean graphs and purify the perturbed structures by removing adversaries under the direction of the captured predictive patterns without relying on priors.
\ModelName~is divided into the forward diffusion process and the reverse denoising process, during which structure purification is achieved.
To avoid valuable information loss during the forward process, we propose an LID-driven non-isotropic diffusion mechanism to selectively inject noise anisotropically.
To promote semantic alignment between the clean graph and the purified graph generated during the reverse process, we reduce the generation uncertainty by the proposed graph transfer entropy guided denoising mechanism.
Extensive experiments demonstrate the superior robustness of \ModelName~against evasion attacks.
% The reverse denoising process of diffusion models naturally aligns with removing graph adversarial perturbations, making them suitable for learning clean graph distribution and removing adversarial perturbations based on the learned distributional patterns without relying on priors.
% purifying adversarial graphs through distribution mapping.
% However, the indiscriminate noise injection in graph diffusion models can erase useful information, while the diversity of the reverse process may cause generated graphs to deviate from the target clean graph, making it difficult to directly apply them for purifying adversarial graph data.
% To address these challenges, 
% In this work, we propose a novel framework \ModelName, which introduces a LID-driven non-isotropic forward diffusion process and a transfer entropy-guided reverse denoising process to precisely remove adversarial perturbations and guide the generation toward the target clean graph.
% Our code is available at \textcolor{mytablecolor}{\url{https:///}}.
\end{abstract}


\keywords{robust graph learning, adversarial evasion attack, graph structure purification, graph diffuison}


\section{Introduction}
\IEEEPARstart{I}{n} recent years, flourishing of Artificial Intelligence Generated Content (AIGC) has sparked significant advancements in modalities such as text, image, audio, and even video. 
Among these, AI-Generated Image (AGI) has garnered considerable interest from both researchers and the public.
Plenty of remarkable AGI models and online services, such as StableDiffusion\footnote{\url{https://stability.ai/}}, Midjourney\footnote{\url{https://www.midjourney.com/}}, and FLUX\footnote{\url{https://blackforestlabs.ai/}}, offer users an excellent creative experience.
However, users often remain critical of the quality of the AGI due to image distortions or mismatches with user intentions.
Consequently, methods for assessing the quality of AGI are becoming increasingly crucial to help improve the generative capabilities of these models.

Unlike Natural Scene Image (NSI) quality assessment, which focuses primarily on perception aspects such as sharpness, color, and brightness, AI-Generated Image Quality Assessment (AGIQA) encompasses additional aspects like correspondence and authenticity. 
Since AGI is generated on the basis of user text prompts, it may fail to capture key user intentions, resulting in misalignment with the prompt.
Furthermore, authenticity refers to how closely the generated image resembles real-world artworks, as AGI can sometimes exhibit logical inconsistencies.
While traditional IQA models may effectively evaluate perceptual quality, they are often less capable of adequately assessing aspects such as correspondence and authenticity.

\begin{figure}\label{fig:radar}
    \centering
    \includegraphics[width=1.0\linewidth]{figures/radar_plot.pdf}
    \caption{A comparison on quality, correspondence, and authenticity aspects of AIGCIQA2023~\cite{wang2023aigciqa2023} dataset illustrates the superior performance of our method.}
\end{figure}

Several methods have been proposed specifically for the AGIQA task, including metrics designed to evaluate the authenticity and diversity of generated images~\cite{gulrajani2017improved,heusel2017gans}. 
Nevertheless, these methods tend to compare and evaluate grouped images rather than single instances, which limits their utility for single image assessment.
Beginning with AGIQA-1k~\cite{zhang2023perceptual}, a series of AGIQA databases have been introduced, including AGIQA-3k~\cite{li2023agiqa}, AIGCIQA-20k~\cite{li2024aigiqa}, etc.
Concurrently, there has been a surge in research utilizing deep learning methods~\cite{zhou2024adaptive,peng2024aigc,yu2024sf}, which have significantly benefited from pre-trained models such as CLIP~\cite{radford2021learning}. 
These approaches enhance the analysis by leveraging the correlations between images and their descriptive texts.
While these models are effective in capturing general text-image alignments, they may not effectively detect subtle inconsistencies or mismatches between the generated image content and the detailed nuances of the textual description.
Moreover, as these models are pre-trained on large-scale datasets for broad tasks, they might not fully exploit the textual information pertinent to the specific context of AGIQA without task-specific fine-tuning.
To overcome these limitations, methods that leverage Multimodal Large Language Models (MLLMs)~\cite{wang2024large,wang2024understanding} have been proposed.
These methods aim to fully exploit the synergies of image captioning and textual analysis for AGIQA.
Although they benefit from advanced prompt understanding, instruction following, and generation capabilities, they often do not utilize MLLMs as encoders capable of producing a sequence of logits that integrate both image and text context.

In conclusion, the field of AI-Generated Image Quality Assessment (AGIQA) continues to face significant challenges: 
(1) Developing comprehensive methods to assess AGIs from multiple dimensions, including quality, correspondence, and authenticity; 
(2) Enhancing assessment techniques to more accurately reflect human perception and the nuanced intentions embedded within prompts; 
(3) Optimizing the use of Multimodal Large Language Models (MLLMs) to fully exploit their multimodal encoding capabilities.

To address these challenges, we propose a novel method M3-AGIQA (\textbf{M}ultimodal, \textbf{M}ulti-Round, \textbf{M}ulti-Aspect AI-Generated Image Quality Assessment) which leverages MLLMs as both image and text encoders. 
This approach incorporates an additional network to align human perception and intentions, aiming to enhance assessment accuracy. 
Specially, we distill the rich image captioning capability from online MLLMs into a local MLLM through Low-Rank Adaption (LoRA) fine-tuning, and train this model with human-labeled data. The key contributions of this paper are as follows:
\begin{itemize}
    \item We propose a novel AGIQA method that distills multi-aspect image captioning capabilities to enable comprehensive evaluation. Specifically, we use an online MLLM service to generate aspect-specific image descriptions and fine-tune a local MLLM with these descriptions in a structured two-round conversational format.
    \item We investigate the encoding potential of MLLMs to better align with human perceptual judgments and intentions, uncovering previously underestimated capabilities of MLLMs in the AGIQA domain. To leverage sequential information, we append an xLSTM feature extractor and a regression head to the encoding output.
    \item Extensive experiments across multiple datasets demonstrate that our method achieves superior performance, setting a new state-of-the-art (SOTA) benchmark in AGIQA.
\end{itemize}

In this work, we present related works in Sec.~\ref{sec:related}, followed by the details of our M3-AGIQA method in Sec.~\ref{sec:method}. Sec.~\ref{sec:exp} outlines our experimental design and presents the results. Sec.~\ref{sec:limit},~\ref{sec:ethics} and~\ref{sec:conclusion} discuss the limitations, ethical concerns, future directions and conclusions of our study.
\section{Related Works}


\noindent\textbf{3D Point Cloud Domain Adaptation and Generalization.}
Early endeavors within 3D domain adaptation (3DDA) focused on extending 2D adversarial methodologies~\cite{qin2019pointdan} to align point cloud features. Alternative methods have delved into geometry-aware self-supervised pre-tasks. Achituve \etal~\cite{achituve2021self} introduced DefRec, a technique employing self-complement tasks by reconstructing point clouds from a non-rigid distorted version, while Zou \etal~\cite{zou2021geometry} incorporating norm curves prediction as an auxiliary task. Liang \etal~\cite{liang2022point} put forth MLSP, focusing on point estimation tasks like cardinality, position, and normal. SDDA~\cite{cardace2023self} employs self-distillation to learn the point-based features. Additionally, post-hoc self-paced training~\cite{zou2021geometry,fan2022self,park2023pcadapter} has been embraced to refine adaptation to target distributions by accessing target data and conducting further finetuning based on prior knowledge from the source domain.
In contrast, the landscape of 3D domain generalization (3DDG) research remains nascent. Metasets~\cite{huang2021metasets} leverage meta-learning to address geometric variations, while PDG~\cite{wei2022learning} decomposes 3D shapes into part-based features to enhance generalization capabilities.
Despite the remarkable progress, existing studies assume that objects in both the source and target domains share the same orientation, limiting their practical application. This limitation propels our exploration into orientation-aware 3D domain generalization through intricate orientation learning.


\noindent\textbf{Rotation-generalizable Point Cloud Analysis.}
Previous works in point cloud analysis~\cite{qi2017pointnet, wang2019dynamic} enhance rotation robustness by introducing random rotations to augment point clouds. {However, generating a comprehensive set of rotated data is impractical, resulting in variable model performance across different scenes. To robustify the networks \wrt randomly rotated point clouds,} rotation-equivariance methods explore equivalent model architectures by incorporating equivalent operations~\cite{su2022svnet, Deng_2021_ICCV, luo2022equivariant} or steerable convolutions~\cite{chen2021equivariant, poulenard2021functional}.
Alternatively, rotation-invariance approaches aim to identify geometric descriptors invariant to rotations, such as distances and angles between local points~\cite{chen2019clusternet, zhang2020global} or point norms~\cite{zhao2019rotation, li2021rotation}. Besides, {Li \etal~\cite{li2021closer} have explored disambiguating the number of PCA-based canonical poses, while Kim \etal~\cite{kim2020rotation} and Chen \etal~\cite{chen2022devil} have transformed local point coordinates according to local reference frames to maintain rotation invariance. However, these methods focus on improving in-domain rotation robustness, neglecting domain shift and consequently exhibiting limited performance when applied to diverse domains. This study addresses the challenge of cross-domain generalizability together with rotation robustness and proposes novel solutions.} 

\noindent\textbf{Intricate Sample Mining}, aimed at identifying or synthesizing challenging samples that are difficult to classify correctly, seeks to rectify the imbalance between positive and negative samples for enhancing a model's discriminability. While traditional works have explored this concept in SVM optimization~\cite{felzenszwalb2009object}, shallow neural networks~\cite{dollar2009integral}, and boosted decision trees~\cite{yu2019unsupervised}, recent advances in deep learning have catalyzed a proliferation of researches in this area across various computer vision tasks. For instance, 
Lin \etal~\cite{lin2017focal} proposed a focal loss to concentrate training efforts on a selected group of hard examples in object detection, while Yu \etal~\cite{yu2019unsupervised} devised a soft multilabel-guided hard negative mining method to learn discriminative embeddings for person Re-ID. Schroff \etal~\cite{schroff2015facenet} introduced an online negative exemplar mining process to encourage spherical clusters in face embeddings for individual recognition, and Wang \etal~\cite{wang2021instance} designed an adversarially trained negative generator to yield instance-wise negative samples, bolstering the learning of unpaired image-to-image translation. In contrast to existing studies, our work presents the first attempt to mitigate the orientational shift in 3D point cloud domain generalization, by developing an effective intricate orientation mining strategy to achieve orientation-aware learning.


\newcommand{\enc}[0]{\phi}
\newcommand{\con}[0]{\psi}
\newcommand{\encmap}[2]{\mathcal{O}_{#1}^{#2} \mapsto \mathcal{X}_{#1}^{#2}}
\newcommand{\conmap}[3]{\mathcal{X}_{#1}^{#2} \mapsto \mathcal{A}_{#3}}
\newcommand{\ours}[0]{Trasl. \textbf{(Ours)}}


\section{Preliminaries}

We assume the underlying environment is a Markov decision process $\mathcal{M} = (\mathcal{S, A}, \mathcal{O}, R, P, \gamma)$, with state space $\mathcal{S}$, action space $\mathcal{A}$, input observations $o \in \mathcal{O}$ and the transition
function $P : \mathcal{S} \times \mathcal{A} \mapsto \mathcal{S}$ that defines a probability distribution over the next state given the current state and action. The function $\mathcal{R} : \mathcal{S} \times \mathcal{A} \mapsto \mathcal{R}$ assigns rewards, and $\gamma$ is the discount factor that reduces the importance of delayed rewards. The agent’s behavior is dictated by a policy $\pi : \mathcal{O} \rightarrow \mathcal{A}$, which receives an observation and selects an action at each state, and is trained to maximize the discounted returns
$\mathbb{E}\Bigl[\sum_{i=0}^{\infty} \gamma^{i} \mathcal{R}(\mathbf{s}_{i}, \mathbf{a}_{i})\Bigr]$.

\subsection{Background}
We are interested in scenarios where both training setups and agent behaviors can vary. We find it convenient to use the same notation introduced in \textsc{R3L} (Relative Representations for Reinforcement Learning) \citep{ricciardi2025r3lrelativerepresentationsreinforcement} to formalize these variations.

\paragraph{Environment variations}
We denote an environment by $\mathcal{M}{u}^{i} = (\mathcal{O}_{u}, T_{i})$. Here, $\mathcal{O}_{u}$ is the distribution of observations $o_{u}$, and $T_{i} : \mathcal{S}{i} \times \mathcal{A}_{i} \times \mathcal{R}_{i} \times P{i} \mapsto \mathcal{R}_{i}$  specifies the task. Two environments can differ in the distribution of observations (e.g., background color, camera perspective) or in the task itself (e.g., transition dynamics, action spaces, reward definitions).
Since agents must discover the task solely through reward feedback, any shift—whether in observations or task—can significantly alter their learned representations.

\paragraph{Agents}
Following \textsc{R3L}, each policy $\pi_{u}^{i}$ is typically obtained by end-to-end training on $\mathcal{M}{u}^{i}$. However, we emphasize a modular view of this policy:
\begin{align}
    \pi_{u}^{i}(o_{u}) \;=\; \con_{u}^{i}\bigl[\enc_{u}^{i}(o_{u})\bigr] \;=\; \con_{u}^{i}\bigl(\mathbf{x}_{u}^{i}\bigr)
\end{align}
where $\enc_{u}^{i} : \mathcal{O}_{u} \mapsto \mathcal{X}_{u}^{i}$ serves as the \textit{encoder} that processes raw observations (e.g., images), and $\con_{u}^{i} : \mathcal{X}_{u}^{i} \mapsto \mathcal{A}_{i}$ is the \textit{controller} that outputs actions based on the latent embedding $\mathbf{x}{u}^{i}$. This factorization disentangles observation-specific features (in $\enc_{u}^{i}$) from task-specific decision rules (in $\con_{u}^{i}$).

\paragraph{Latent Representation}
Now consider a second environment $\mathcal{M}_{v}^{j} = (\mathcal{O}_{v}, T_{j})$, where $\mathcal{O}_{v}$ differs from $\mathcal{O}_{u}$ only in visual style (e.g., a shifted color scheme), and suppose we have a policy $\pi_{v}^{j}$ trained on that environment. For two semantically corresponding observations $o_{u} \in \mathcal{O}_{u}$ and $o_{v} \in \mathcal{O}_{v}$, the respective latent embeddings differ:
\begin{align}
\enc_{u}^{i}(o_{u}) \; \neq \; \enc_{v}^{j}(o_{v}) \quad \Longrightarrow \quad \mathbf{x}_{u}^{i} \;\neq\; \mathbf{x}_{v}^{j}
\end{align}
In the next section, we describe how to map one latent space onto another to enable zero-shot stitching of encoders and controllers trained in different visual and task domains, without additional training. %\AR{sposta Mv in environm variations, and la parte di latent differenze in Latent alignment (forse paragrafdetto li)k}

\section{SAPS: Semantic Alignment for Policy Stitching}\label{sec:method-alignment}
Relative representations \citep{Moschella2022-yf}, used as a base for zero-shot stitching in R3L, involve computing a distance function between a set of samples, called \say{anchors}, to project the output of each encoder to a shared latent space, enabling the subsequent training of a universal policy. Semantic alignment, instead, estimates a direct mapping between latent spaces.

Consider the environment $\mathcal{M}_u^j$ for which no dedicated policy exists. However, we do have an encoder $\phi_u^i$ and a controller $\psi_v^j$, extracted from policies $\pi_u^i$ and $\pi_v^j$, respectively. 
We estimate an affine transformation $\tau_u^v$: $\mathcal{X}_{u}^{i} \mapsto \mathcal{X}_{v}^{j}$, mapping embeddings produced by $\phi_u^i$ into the space of $\pi_v^j$. This yields a new latent space:

\begin{align}
    & \tau_u^v(\enc_u^i(\mathbf{o}_u)) \approx \enc_v^j(\mathbf{o}_v)\\
    & \tau_u^v(\mathbf{x}_{u}^i) \approx \mathbf{x}_{v}^j
\end{align}

that is compatible with the existing $\psi_v^j$.
This enables the stitching of encoders and controllers from $\pi_u^i$ and $\pi_v^j$, respectively, to obtain a new policy $\tilde{\pi}_u^j$ that can act in $\mathcal{M}_u^j$, without additional training:
\begin{equation}\label{eq:2}
    \tilde{\pi}_u^j(o_u) = \con_v^j[\tau_u^v(\enc_u^i(\mathbf{o}_u))]
\end{equation}

\paragraph{Estimating $\tau$}
As in \cite{maiorca2023latent}, assume to be given latent spaces $\mathbf{X}_u$ and $\mathbf{X}_v$ which here correspond to the embedding of two visual variations in the space of observations.
We use SVD to obtain an affine transformation $\tau_u^v(\mathbf{x}_u) = \mathbf{R} \mathbf{X}_u + \mathbf{b}$.


\paragraph{Collecting the Dataset.}
The anchor embeddings $\mathbf{X}_u$ and $\mathbf{X}_v$ derive from sets of anchor points $\mathbf{A}_u$ and $\mathbf{A}_v$. Following previous works \citep{maiorca2023latent, Moschella2022-yf, ricciardi2025r3lrelativerepresentationsreinforcement} anchor pair ($\mathbf{a}_u$, $\mathbf{a}_v$) must share a semantic correspondence, meaning both samples represent the same underlying concept (e.g., the same spatial position in a racing track, viewed under two different visual styles).
In supervised learning contexts, anchor pairs can come from paired datasets (e.g., bilingual corpora). In the context of online RL, however, such datasets do not naturally exist. Hence, we collect datasets sharing a correspondence.
This correspondence can be obtained by either rolling out a policy and replaying the same set of actions with different visual variations, as already done in \cite{jian2023policy, ricciardi2025r3lrelativerepresentationsreinforcement}, or by simply applying visual transformations to the image in pixel space. This yields corresponding observation sets $\mathbf{A}_u$ and $\mathbf{A}_v$ that can be embedded by each domain’s encoder to create $\mathbf{X}_u$ and $\mathbf{X}_v$. Finally, we solve for $\tau_u^v$ using the SVD-based procedure above.


%\AR{da inserire forse: Specifically, we estimate $\tau_u^v$, following the technique used in \cite{maiorca2023latent}. which suggests that, given two latent spaces $\mathbf{X} \in \mathbb{R}^{n \times d1}$ and $\mathbf{Y} \in \mathbb{R}^{m \times d2}$ from independently trained deep neural networks, the transformation $\tau$ that directly maps $\mathbf{X}$ to $\mathbf{Y}$: (i) is mostly orthogonal and (ii) can be estimated from a few corresponding elements between the two spaces. In our work, $\mathbf{X}$ and $\mathbf{Y}$ are produced by $\enc_u$ and $\enc_v$, respectively. As in \cite{maiorca2023latent}, we use \textit{Singular Value Decomposition} (SVD) to estimate the optimal orthogonal transformation.}

In our context, we assume that an agent trained end-to-end to solve a specific task in a specific environment will generate a comprehensive set of observations, providing a reasonable approximation of the entire latent space. Nevertheless, forcing the agent to explore more could be beneficial in this context.
In our experiments, we gather parallel samples either by directly translating the observation in pixel space, when there is a well-defined known visual variation between the environments, or by replaying the same sequence of actions in both environments, that in this case must be deterministic and initialized with the same random seed. We leave to future research other possible approximation techniques for translating observations between different environments.

\section{Experiments}
\begin{table}[t]
\centering
{\resizebox{\columnwidth}{!}{
\begin{tabular}{lccccccc}
\toprule
\multicolumn{1}{c}{\multirow{2}{*}{\textbf{Model}}} & \multirow{2}{*}{\begin{tabular}[c]{@{}c@{}}\textbf{Parameter}\\ \textbf{Scale}\end{tabular}}  & \multicolumn{5}{c}{\textbf{Multifacet}} & \multirow{2}{*}{\textbf{Average}} \\ \cmidrule{3-7}  
\multicolumn{1}{c}{} &  & \multicolumn{1}{l}{\textbf{AE}} & \multicolumn{1}{l}{\textbf{FL}} & \multicolumn{1}{l}{\textbf{Ko}} & \multicolumn{1}{l}{\textbf{MT}} & \multicolumn{1}{l}{\textbf{SI}} & \\ \midrule
\textit{Open-Source Models} \\ \midrule
Solar-10.7B-instruct & 10.7B & 3.30 & 3.31 & 3.09 & 3.19 & 3.08 & 3.19  \\  
Gemma-2-9b-it & 9B & 4.10 & 3.80 & 4.26 & 4.15 & 3.92 & 4.05  \\ 
\midrule
\multicolumn{8}{l}{\textit{Open-source Models} $+$ \textit{KD (Fine-tuning on \textbf{\textsc{SysGen}} dataset)}} \\
\midrule 
Solar-10.7B-instruct & 10.7B & 3.97 & 3.73 & 3.64 & 3.98 & 3.52 & 3.76 (+0.57) \\ 
Gemma-2-9b-it & 9B & 4.40 & 4.04 & 4.30 & 4.23 & 4.18 & 4.23 (+0.18) \\ 
\bottomrule
\end{tabular}}}
\caption{
We conduct a knowledge distillation (KD) experiments leveraging data generated by \textsc{SysGen} pipeline using Phi-4.}
\label{tab:knowledge_distillation}
\vspace{-0.3cm}
\end{table}
\begin{table*}[t]
\centering
{\resizebox{\textwidth}{!}{
\begin{tabular}{lcccccccccc}
\toprule
\multicolumn{1}{c}{\multirow{2}{*}{\textbf{Model}}} & \multirow{2}{*}{\begin{tabular}[c]{@{}c@{}}\textbf{Parameter}\\ \textbf{Scale}\end{tabular}}  & \multicolumn{8}{c}{\textbf{Unseen Benchmarks}} & \multirow{2}{*}{\textbf{Average}} \\  \cmidrule{3-10} 
\multicolumn{1}{c}{} &  & \multicolumn{1}{l}{\textbf{MMLU}} & \multicolumn{1}{l}{\textbf{MMLU-Pro}} & \multicolumn{1}{l}{\textbf{ARC-c}} & \multicolumn{1}{l}{\textbf{GPQA}} & \multicolumn{1}{l}{\textbf{HellaSwag}} & \multicolumn{1}{l}{\textbf{IFEVAL}} & \textbf{MATHQA} & \textbf{BBH} & \\ \midrule
\multicolumn{11}{l}{\textit{Open-Source Models}} \\ \midrule 
Solar-10.7B-instruct & 10.7B  &  63.28 & 30.20 & 63.99 & 30.36 & 86.35 & 38.59 &  36.38 & 37.28 & 48.31 \\
Gemma-2-9b-it & 9B & 73.27 & 32.78 & 67.89 & 31.05 & 81.92 & 74.78 & 38.87 & 41.98 & 55.31 \\ 
LLaMA-3.1-8B-instruct & 8B & 67.95 & 40.87 & 54.95 & 34.60 & 79.18 & 50.71 & 39.53 & 70.85 & 54.83 \\ 
% Mixtral-8x22B-instruct & 8x22B & 75.62 &  52.63 & 67.83 & 36.83 & 87.68 & 60.43 & 50.08 & 83.03 & \\   
Qwen2.5-14B-instruct & 14B &  79.73  & 51.22 & 67.39 & 45.51 & 82.31 & 79.83 & 42.12 & 78.25 & 65.79 \\ 
Phi-4 & 14B & 84.56 & 70.12  & 68.26 & 55.93 & 84.42 & 62.98 & 48.87 & 79.87 & 69.37 \\  
\midrule
\multicolumn{11}{l}{\textit{Open-Source Models (Fine-tuning on original SFT Dataset)}} \\ \midrule
Solar-10.7B-instruct & 10.7B & 62.38 & 29.12 & 58.87 & 29.17 & 81.58 & 31.27 & 37.21 & 32.85 & 45.30 (-3.01) \\ 
Gemma-2-9b-it & 9B & 71.85 & 31.67  & 62.57 & 30.51 & 77.54 & 69.25 & 39.12 & 37.25 & 52.47 (-2.84) \\ 
LLaMA-3.1-8B-instruct & 8B & 65.34 &  36.85 &  
54.18 & 33.93 & 77.98 & 35.64 & 40.03 & 62.83 & 50.85 (-3.98)  \\ 
% Mixtral-8x22B-instruct & 8x22B&  &   & & & & & &  & \\ 
Qwen2.5-14B-instruct & 14B & 75.87  & 49.85  & 66.89 & 43.98 & 80.99 & 62.57 & 43.28 & 71.17 & 61.82 (-3.97) \\ 
Phi-4 & 14B & 80.27 & 66.58  & 66.27 & 52.89 & 83.39 & 55.83 & 49.98 & 75.49 & 66.33 (-6.04) \\ 
\midrule
\multicolumn{11}{l}{\textit{Open-Source Models (Fine-tuning on \textbf{\textsc{SysGen}} dataset)}} \\ \midrule 
LLaMA-3.1-8B-instruct & 8B & 66.89 & 39.77 & 54.55 & 34.21 & 78.89 & 46.75 & 42.11 & 68.98 & 54.02 (-0.81) \\ 
% Gemma-2-9B-instruct & 9B &  &   & & & & & &  & \\ 
% Mixtral-8x22B-instruct & 8x22B&  &   & & & & & &  & \\ 
% Solar-10.7B-instruct & 10.7B & 63.28 & 30.20 & 63.99 & 30.36 & 86.35 & 38.59 &  36.38 & 37.28 & \\ 
Qwen2.5-14B-instruct & 14B & 78.92 & 43.38 & 66.82 & 44.46 & 80.98 & 74.59 & 43.23 & 76.28 & 63.58 (-2.20) \\ 
Phi-4 & 14B & 83.27 & 68.77  & 67.89 & 55.18 & 84.31 & 57.87 & 50.23 & 77.12 & 68.08 (-1.29) \\ 
\midrule
\multicolumn{11}{l}{\textit{Open-source Models} $+$ \textit{Knowledge Distillation (Fine-tuning on \textbf{\textsc{SysGen}} dataset))}} \\
\midrule 
Solar-10.7B-instruct & 10.7B & 59.98  & 29.26  & 62.81 & 30.25 & 85.91 & 34.58 & 38.25 & 35.97 & 47.12 (-1.19) \\ 
Gemma-2-9b-it & 9B & 72.19 & 31.56 & 66.75 & 30.89 & 81.53 & 71.37 & 40.27 & 40.38 & 54.37 (-0.94) \\ 
\bottomrule
\end{tabular}}}
\caption{We utilize the Open LLM Leaderboard 2 score as the unseen benchmark. This reveals the key finding that adding system messages to existing SFT datasets does not lead to significant performance degradation.}
\label{tab:unseen_experiments}
\end{table*}
The primary goal of \textsc{SysGen} pipeline is to enhance the utilization of the \emph{system role} while minimizing performance degradation on unseen benchmarks, thereby improving the effectiveness of supervised fine-tuning (SFT).
To validate this, we evaluate how well the models trained on \textsc{SysGen} data generate appropriate assistant responses given both the system messages and user instructions, using the Multifacet~\citep{lee2024aligning} dataset.
For models that cannot generate data independently, we apply knowledge distillation to assess their effectiveness.
Additionally, we leverage the widely used Open LLM Leaderboard 2~\citep{myrzakhan2024open} as an unseen benchmark to determine whether our approach can be effectively integrated into existing SFT workflows.


\paragraph{\textsc{SysGen} provides better system message and assistant response to align with user instructions.}
Given the system messages and user instructions, the assistant's response is evaluated across four dimensions: style, background knowledge, harmlessness, and informativeness.
Each of these four aspects is scored on a scale of 1 to 5 using a rubric, and the average score is presented as the final score for the given instruction.
As shown in Table~\ref{tab:main_experiments}, recent open-source models achieve comparable scores to the proprietary models, indicating that open-source models have already undergone training related to system roles~\citep{meta2024introducing, yang2024qwen2, abdin2024phi}.

When trained on \textsc{SysGen} data, both LLaMA (4.12 → 4.21) and Phi (4.41 → 4.54) show score improvements.
Among the four dimensions, LLaMA exhibits score increases in style (4.15 → 4.32) and harmlessness (4.23 → 4.29).
Similarly, Phi shows the improvements in style (4.42 → 4.61) and informativeness (4.37 → 4.49).
As a result, even open-source models that have already been trained on system roles demonstrate their positive effects on style, informativeness, and harmlessness.



\paragraph{Knowledge distillation through \textsc{SysGen} data.}
If an open-source model does not support the system roles, it may not generate the system messages properly using \textsc{SysGen} pipeline. 
However, the effectiveness of knowledge distillation, using data generated by another open-source model without the limitation, remains uncertain.
To explore this, we train Gemma~\citep{team2024gemma} and Solar~\citep{kim-etal-2024-solar} using data generated by Phi-4~\citep{abdin2024phi}.
We use the Phi-4 data because it preserves most of the data and provides high quality  assistant responses as shown in Table~\ref{tab:statistics_generated_answer} and \ref{tab:data_statistics}.

As shown in Table~\ref{tab:knowledge_distillation}, even for models that do not inherently support system roles, modifying the chat template to incorporate system role and training on knowledge distilled dataset leads to an improvement in Multifacet performance, as observed in Gemma (4.05 → 4.23).
We describe the details in the Appendix~\ref{app:system_role_support}.
Additionally, for the Solar model, which had not been trained on system roles, we observe a dramatic performance improvement (3.19 → 3.76).\footnote{We speculate that Solar model did not properly learn the system role because its initial Multifacet score was low.}
This demonstrates that the data generated by the \textsc{SysGen} pipeline effectively supports the system roles.


\paragraph{\textsc{SysGen} data minimizes the performance degradation in unseen benchmarks.}
When incorporating system messages that were not present in the original SFT datasets and modifying the corresponding assistant responses, it is crucial to ensure that the model’s existing performance should not degrade.
For example, one key consideration in post-training is maintaining the model's original performance.
To assess this, we observed performance difference in unseen benchmark after applying supervised fine-tuning.
As shown in Table~\ref{tab:unseen_experiments}, we use the Open LLM Leaderboard 2 dataset as an unseen benchmark, with performance categorized into four groups:
\begin{itemize}
    \item Performance of existing open-source models (row 1-6)
    \item Performance of fine-tuning with open-source models using SFT datasets (row 7-12)
    \item Performance of fine-tuning with \textsc{SysGen} data (row 13-16)
    \item Performance after applying knowledge distillation using Phi-4 \textsc{SysGen} data (row 17-19)
\end{itemize}
The average performance degradation reflects the scores missing from each open-source model's original performance (row 1-6).

When fine-tuning with independently generated data using \textsc{SysGen}, the performance degradation is significantly lower than fine-tuning with the original SFT datasets selected under the same conditions.
Additionally, even for models that cannot generate data independently (e.g., those that do not support system roles), knowledge distillation helps mitigate performance drops considerably.










\subsection{Zero-shot stitching comparison}\label{sec:zero-shot-stitching}
We follow the empirical analysis performed in R3L. We define the encoder as the network up to the first flatten layer after the convolutional blocks, with the remaining layers constituting the controller. The zero-shot stitching evaluation is conducted on visual-task variations that were not seen together during training. Encoders and controllers must match to the specific conditions they were trained on. For example, an encoder trained with a green background should be used in such an environment, and a controller developed for low-speed driving should be applied to that task.
% \AR{immagine di architettura? Empirical/ablation test su layer di stitching? E su numero ancore?}



\paragraph{Stitching table}
Encoders and controllers from different policies trained under different conditions, can be independently assembled through zero-shot stitching. 
In \Cref{sec:method-alignment} we defined how to estimate and use a transformation to then perform the stitching, defining the models as formed by encoders and controllers. 

We present zero-shot stitching results in \Cref{tab:carracing-stitching_performance} and \cref{tab:lunarlander-stitching_performance}. 
The testing procedure to perform stitching is as follows:
Given models trained with various seeds and different visual-task variations, we decompose encoders and controllers of a model, then connect each encoder with controllers that were trained either on different visual variations or different seeds, using the transformation to ensure compatibility between layers.
Each cell reports the average score obtained across different stitched seeds.
Therefore, if we have four visual variations, six task variations and five seeds, for each cell we perform $6 \times 5 = 30$ stitching tests, for a total of 120 across the entire table for a single stitching method. Each combination is tested over 10 different tracks.

In CarRacing, SAPS consistently achieves performance comparable to end-to-end scores across all tested visual and task variations, even in the challenging slow scenario where R3L experiences a noticeable drop. In LunarLander there is a considerable drop in the mean average performance, which is mainly due to some models highly underperforming, decreasing the mean value. This is, however, a strong improvement over R3L, for which we were not even able to train models, strongly highlighting the advantages of being able to perform zero-shot stitching from already trained, standard models.
This table underscores SAPS’s robustness and adaptability in assembling agents for new environment variations, surpassing existing methods.
% \AR{inserisci appendice}Further stitching results for the Atari suite appear in \Cref{appendix:stitching-atari}.

SAPS instead performs much worse in LunarLander with respect to end-to-end agents, although improving upon the naive stitching approach. We think that this might reflect how sensitive LunarLander is to even small latent-space mismatches: a slight misalignment in the latent space could cause the landing procedure to strongly deviate in its trajectory, causing crashes and large penalties. Here, R3L scores are missing because we were not able to train models using this method.
% stitching results under varying background colors (white vs. red) and gravity levels (-10 vs. -3) instead 


\begin{table}[t!]
    \caption{Stitching performance in \textbf{CarRacing}, comparing \textbf{SAPS (ours)} to other methods, averaged over 5 seeds, with standard deviations.}
    \label{tab:carracing-stitching_performance}
    \resizebox{\textwidth}{!}{
    \begin{tabular}{ccccccccccc}
    \toprule
    & & & \multicolumn{6}{c}{\textbf{Controller}} & \\
    \cmidrule{4-9}
    & & & \multicolumn{3}{c}{\textbf{Visual Variations} (task standard)} & \multicolumn{3}{c}{\textbf{Task Variations} (green)} \\
    \cmidrule(r){4-6} \cmidrule(l){7-9}
    & & & \texttt{green} & \texttt{red} & \texttt{blue} &  \texttt{slow} & \texttt{scrambled} & \texttt{no idle} \\
    \cmidrule(r){1-6} \cmidrule(l){7-9}
    \multirow{12}{*}{\rotatebox{90}{\textbf{Encoder}}} 
    & \multirow{3}{*}{\rotatebox{90}{\texttt{green}}} 
    & \emph{Naive} & $175 \pm 304$ & $167 \pm 226$ & $-4 \pm 79$ &  $148 \pm 328$ & $106 \pm 217$ & $213 \pm 201$ \\
    & & \emph{R3L} & $781 \pm 108$ & $787 \pm 62$ & $794 \pm 61$ & $268 \pm 14$ & $781 \pm 126$ & $824 \pm 82$ \\
    & & \textbf{\emph{SAPS} (ours)} & $\mathbf{822 \pm 62}$ & $\mathbf{786 \pm 82}$ & $\mathbf{829 \pm 49}$ & $\mathbf{764 \pm 287}$ & $\mathbf{846 \pm 66}$ & $\mathbf{781 \pm 72}$ \\[1.5ex]
    & \multirow{3}{*}{\rotatebox{90}{\texttt{red}}} 
    & \emph{Naive} & $157 \pm 248$ & $43 \pm 205$ & $22 \pm 112$  & $83 \pm 191$ & $138 \pm 244$ & $252 \pm 228$ \\
    & & \emph{R3L} & $810 \pm 52$ & $776 \pm 92$ & $803 \pm 58$ & $476 \pm 430$ & $790 \pm 72$ & $817 \pm 69$ \\
    & & \textbf{\emph{SAPS} (ours)} & $\mathbf{859 \pm 41}$ & $\mathbf{807 \pm 52}$ & $\mathbf{809 \pm 60}$ & $\mathbf{824 \pm 192}$ & $\mathbf{838 \pm 52}$ & $\mathbf{853 \pm 50}$ \\[1.5ex]
    & \multirow{3}{*}{\rotatebox{90}{\texttt{blue}}} 
    & \emph{Naive} & $137 \pm 225$ & $130 \pm 274$ & $11 \pm 122$ & $95 \pm 128$ & $138 \pm 224$ & $144 \pm 206$ \\
    & & \emph{R3L} & $791 \pm 64$ & $793 \pm 40$ & $792 \pm 48$  & $564 \pm 440$ & $804 \pm 41$ & $828 \pm 50$ \\
    & & \textbf{\emph{SAPS} (ours)} & $\mathbf{839 \pm 57}$ & $\mathbf{808 \pm 70}$ & $\mathbf{814 \pm 52}$  & $\mathbf{746 \pm 319}$ & $\mathbf{832 \pm 60}$ & $\mathbf{808 \pm 62}$ \\[1.5ex]
    & \multirow{3}{*}{\rotatebox{90}{\texttt{far}}} 
    & \emph{Naive} & $152 \pm 204$ & $65 \pm 180$ & $2 \pm 152$ & $-49 \pm 9$ & $351 \pm 97$ & $349 \pm 66$ \\
    & & \emph{R3L} & $527 \pm 142$ & $605 \pm 118$ & $592 \pm 86$ & $303 \pm 100$ & $594 \pm 39$ & $673 \pm 91$ \\
    & & \textbf{\emph{SAPS} (ours)} & $\mathbf{714 \pm 45}$ & $\mathbf{712 \pm 71}$ & $\mathbf{727 \pm 52}$ & $\mathbf{762 \pm 131}$ & $\mathbf{738 \pm 44}$ & $\mathbf{626 \pm 77}$ \\
    \bottomrule
    \end{tabular}
    }
\end{table}

\begin{table}[t!]
    \centering
    \caption{Stitching performance comparing \textbf{SAPS (ours)} to other methods, in the \textbf{LunarLander} environment. Scores are averaged over 5 seeds, with standard deviations. For comparison, end-to-end models achieve $221 \pm 86$ for the white background, $192 \pm 30$ for the red background with gravity -10; for the white background, for gravity -3. R3L results are absent because we were not able to train models for it.}
    \label{tab:lunarlander-stitching_performance}
    % Use 0.7\textwidth or 0.8\textwidth, etc., to control how narrow you want it
    \resizebox{0.6\textwidth}{!}{
    \begin{tabular}{ccccccc}  % exactly 7 columns
    \toprule
    % First row of headers: empty in cols 1-3, "Controller" in cols 4-6, blank col 7
    & \multicolumn{3}{c}{} & \multicolumn{3}{c}{\textbf{Controller}} \\
    \cmidrule(l){5-7}
    % Second row of headers: empty in cols 1-3, "Gravity: -10" in 4-5, "Gravity: -3" in col 6, blank col 7
    & \multicolumn{3}{c}{} & \multicolumn{2}{c}{\textbf{Gravity: -10}} & \textbf{Gravity: -3} \\
    \cmidrule(r){5-6} \cmidrule(l){6-7}
    % Third row of headers: empty in cols 1-3, fill col 4-6 with "white, red, white", col 7 blank
    & & & & \texttt{white} & \texttt{red} & \texttt{white} \\
    \midrule
    \multirow{5}{*}{\rotatebox{90}{\textbf{Enc}}} & \multirow{5}{*}{\rotatebox{90}{\textbf{Gravity -10}}}
    & \multirow{3}{*}{\rotatebox{90}{\texttt{white}}} 
    & \emph{Naive} & $-413 \pm 72$ & $-390 \pm 176$ & $-276 \pm 8$ \\
    & & & \textbf{\emph{SAPS} (ours)} & $\mathbf{19 \pm 56}$ & $\mathbf{8 \pm 60}$ & $-242 \pm 51$ \\[1.5ex]
    & & \multirow{3}{*}{\rotatebox{90}{\texttt{red}}} 
    & \emph{Naive} & $-444 \pm 116$ & $-403 \pm 109$ & $-271 \pm 18$ \\
    & & & \textbf{\emph{SAPS} (ours)} & $\mathbf{52 \pm 44}$ & $\mathbf{33 \pm 61}$ & $-204 \pm 71$ \\
    \bottomrule
    \end{tabular}
    }
\end{table}

\subsection{Latent Space Analysis}\label{sec:latent-analysis}

To assess how effectively our proposed method (SAPS) aligns latent representations across different visual or task variations, we perform both qualitative and quantitative evaluations.

\paragraph{Qualitative Visualization.}
Figure \ref{fig:pca-carracing} (left) presents a 3D PCA projection of latent embeddings for two CarRacing variations: green and red backgrounds. After applying an affine transformation learned by SAPS to the encoder outputs, the points corresponding to green and red observations become thoroughly intermixed in the shared latent space. This intermixing indicates that frames depicting the “same portion” of the track with different background colors now map to similar embeddings (see insets). Conversely, Figure \ref{fig:pca-carracing} (right) shows the unaligned embeddings, where green and red remain clearly separated. These results suggest that SAPS successfully bridges the gap between encoders, producing a unified representation space under visual variation.

\begin{figure}[t!]
    \centering
    \includegraphics[width=\linewidth]{res/analysis/carracing_align_vs_abs_latent_frames.pdf}
    \caption{PCA visualization of encoder outputs. On the left, we illustrate how an affine alignment can effectively map one latent space to another: same frames with different backgrounds (green/red) cluster together, as indicated by the embedded screenshots. On the right, the source, unaligned embeddings remain separated, highlighting the benefit of our alignment approach in unifying observations from different environment variations.}
    \label{fig:pca-carracing}
\end{figure}

\paragraph{Quantitative Similarities.}
In Figure~\ref{fig:pairwise-histograms}, we plot histograms of pairwise cosine similarity for matched frames from two different variations. Again, we compare \textbf{(a)~SAPS}, \textbf{(b)~R3L} and \textbf{(c)~Naive} stitching.
The top row shows CarRacing, while the bottom row is LunarLander. In both environments, SAPS and R3L achieve a much higher mean cosine similarity (e.g., 0.92–0.99) than the Naive baseline (0.23–0.30). This confirms that independently trained models can exhibit near-identical encodings for semantically identical frames once aligned, while “naive” combinations of encoders and controllers remain incompatible.

\paragraph{Discussion.}
Overall, these findings indicate that: (i) SAPS’ affine transformation effectively repositions points in the latent space, causing corresponding frames to map to nearly the same vector (Figures \ref{fig:pca-carracing}–\ref{fig:pairwise-histograms}); (ii) Compared to “Naive” reusability (no alignment) or purely relative approaches (R3L), SAPS achieves equivalent or better alignment without retraining models on a specialized representation format; (iii) The high average cosine similarity under SAPS confirms that visual variations (and, by extension, moderate task changes) can be handled by learning a lightweight transform from one latent space to another.

Hence, our latent space analysis demonstrates that SAPS successfully stitches together components from different RL models to produce a cohesive, unified representation, paving the way for zero-shot policy reuse in previously unseen environment variations.


\begin{figure}[t!]
    \centering
    \begin{subfigure}[b]{0.33\linewidth}
        \centering
        \includegraphics[width=\linewidth]{res/analysis/CarRacing-v2_green_CarRacing-v2_red_pairwise_dist_transl.pdf}
        \caption{SAPS}
        \label{fig:frames-sim}
    \end{subfigure}%
    \begin{subfigure}[b]{0.33\linewidth}
        \centering
        \includegraphics[width=\linewidth]{res/analysis/CarRacing-v2_green_CarRacing-v2_red_pairwise_dist_rel.pdf}
        \caption{R3L}
        \label{fig:frames-sim}
    \end{subfigure}
    \begin{subfigure}[b]{0.33\linewidth}
        \centering
        \includegraphics[width=\linewidth]{res/analysis/CarRacing-v2_green_CarRacing-v2_red_pairwise_dist_abs.pdf}
        \caption{Naive}
        \label{fig:frames-sim}
    \end{subfigure}
        \begin{subfigure}[b]{0.33\linewidth}
        \centering
        \includegraphics[width=\linewidth]{res/analysis/LunarLanderRGB_white_LunarLanderRGB_red_pairwise_dist_transl.pdf}
        \caption{SAPS}
        \label{fig:frames-sim}
    \end{subfigure}%
    \begin{subfigure}[b]{0.33\linewidth}
        \centering
        \includegraphics[width=\linewidth]{res/analysis/LunarLanderRGB_white_LunarLanderRGB_red_pairwise_dist_rel.pdf}
        \caption{R3L}
        \label{fig:frames-sim}
    \end{subfigure}
    \begin{subfigure}[b]{0.33\linewidth}
        \centering
        \includegraphics[width=\linewidth]{res/analysis/LunarLanderRGB_white_LunarLanderRGB_red_pairwise_dist_abs.pdf}
        \caption{Naive}
        \label{fig:frames-sim}
    \end{subfigure}
    \caption{Histogram of pairwise cosine similarities between matched states from two different environment variations, for CarRacing (\textbf{top}) and LunarLander (\textbf{bottom}). Both SAPS and R3L show very high mean similarity along paired frames, indicating that corresponding observations in each variation map to nearly identical vectors. Mean similarity for encoders without any alignment or relative encoding is very low, emphasizing the utility of latent communication methods.}
    \label{fig:pairwise-histograms}
\end{figure}
\vspace{-5pt}
\section{Discussion}
\textbf{Conclusion.}
In this work, we propose the \textit{\methodname{}} metric, $M_{AP}$, to evaluate preference data quality in alignment.
By measuring the gap from the model's current implicit reward margin to the target explicit reward margin, $M_{AP}$ quantifies the discrepancy between the current model and the aligned optimum, thereby indicating the potential for alignment enhancement.
Extensive experiments validate the efficacy of $M_{AP}$ across various training settings under offline and self-play preference learning scenarios.

\textbf{Limitations and future work}. 
Despite the performance improvements, $M_{AP}$ requires tuning a parameter $\beta$ to combine the explicit and implicit margins; future work could explore how to set this ratio automatically.
Additionally, while our experiments focus on the widely applied DPO and SimPO objectives, a broader investigation with alternative preference learning methods is crucial in future works.

% \section{Conclusion}
% In this paper, we introduce the \methodname{} metric to evaluate preference data quality in LLM alignment.
% By measuring the discrepancy between the model's current implicit reward margin to the target explicit reward margin, this metric quantifies the gap between the current model and the aligned optimum, thereby indicating the potential for alignment enhancement.
% Empirical results demonstrate that training on data selected by our metric consistently improves alignment performance, outperforming existing metrics across different base models and training objectives.
% Moreover, this metric extends to data generation scenarios (\ie self-play alignment): by identifying high-quality data from the intrinsic self-generated context, our metric yields superior results across various training settings, providing a comprehensive solution for enhancing LLM alignment through optimized
% preference data generation, selection, and utilization.


\section*{Impact Statement}
This paper presents work whose goal is to advance the field of Machine Learning. There are many potential societal consequences of our work, none which we feel must be specifically highlighted here.

\iffalse
\subsubsection*{Broader Impact Statement}
\label{sec:broaderImpact}
In this optional section, RLJ/RLC encourages authors to discuss possible repercussions of their work, notably any potential negative impact that a user of this research should be aware of. 

%%%%%%%%%%%%%%%%%%%%%%%%%%%%%%%%%%%%%%%%%%%%%%%%%%%%%%%%%%%%%%%%
%% Appendices
%%%%%%%%%%%%%%%%%%%%%%%%%%%%%%%%%%%%%%%%%%%%%%%%%%%%%%%%%%%%%%%%
\appendix

\section{The first appendix}
\label{sec:appendix1}
This is an example of an appendix. 

\noindent \textbf{Note:} Appendices appear before the references and are viewed as part of the ``main text'' and are subject to the 8--12 page limit, are peer reviewed, and can contain content central to the claims of the paper. 

\section{The second appendix}
\label{sec:appendix2}
This is an example of a second appendix. If there is only a single section in the appendix, you may simply call it ``Appendix'' as follows:

\section*{Appendix}
% No label, since this can't be referenced meaningfully with \ref{}.
This format should only be used if there is a single appendix (unlike in this document).

\subsubsection*{Acknowledgments}
\label{sec:ack}
Use unnumbered third level headings for the acknowledgments. All acknowledgments, including those to funding agencies, go at the end of the paper. Only add this information once your submission is accepted and deanonymized. The acknowledgments do not count towards the 8--12 page limit.

\fi
%%%%%%%%%%%%%%%%%%%%%%%%%%%%%%%%%%%%%%%%%%%%%%%%%%%%%%%%%%%%%%%%
%% NOTE: THIS MARKS THE END OF THE "MAIN TEXT"
%%%%%%%%%%%%%%%%%%%%%%%%%%%%%%%%%%%%%%%%%%%%%%%%%%%%%%%%%%%%%%%%

%%%%%%%%%%%%%%%%%%%%%%%%%%%%%%%%%%%%%%%%%%%%%%%%%%%%%%%%%%%%%%%%
%% Bibliography
%%%%%%%%%%%%%%%%%%%%%%%%%%%%%%%%%%%%%%%%%%%%%%%%%%%%%%%%%%%%%%%%
% \bibliography{main}
\bibliography{bibliography}
\bibliographystyle{rlj}

\iffalse
%%%%%%%%%%%%%%%%%%%%%%%%%%%%%%%%%%%%%%%%%%%%%%%%%%%%%%%%%%%%%%%%
% AUTHOR: If your paper has no supplementary materials, you may 
%         comment out the line below, which creates the title for
%         the supplementary materials.
%%%%%%%%%%%%%%%%%%%%%%%%%%%%%%%%%%%%%%%%%%%%%%%%%%%%%%%%%%%%%%%%
\beginSupplementaryMaterials

Content that appears after the references are not part of the ``main text,'' have no page limits, are not necessarily reviewed, and should not contain any claims or material central to the paper. 
%
If your paper includes supplementary materials, use the \begin{center}
    {\tt {\textbackslash}beginSupplementaryMaterials} 
\end{center}
command as in this example, which produces the title and disclaimer above. 
%
If your paper does not include supplementary materials, this command can be removed or commented out.
\fi
\end{document}
