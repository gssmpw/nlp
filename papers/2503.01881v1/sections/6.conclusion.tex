\section{Conclusion and Future Directions}
\paragraph{Conclusion} We presented Semantic Alignment for Policy Stitching (SAPS), a simple yet effective method for zero-shot reuse of RL agents trained in different environments. By estimating a lightweight transformation that maps one encoder's latent space into another's, we can seamlessly “stitch” encoders and controllers, enabling new policies to handle unseen combinations of visual and task variations without retraining. Our experiments on CarRacing and LunarLander show SAPS achieves near end-to-end performance under diverse domain shifts, outperforming naive baselines and matching or exceeding more specialized zero-shot approaches (e.g., R3L) in many settings. This highlights the potential of direct latent-space alignment for compositional and robust RL.

\paragraph{Limitations and Future Works.}
Although SAPS effectively aligns latent representations across moderate environment variations, several open challenges remain. First, the method's reliance on affine transformations can falter in domains exhibiting larger gaps (for instance, tasks that differ drastically in reward structure or observation type) where a simple linear map may be insufficient. Second, the approach depends on anchors, which must be collected from both source and target environments; in highly stochastic domains, obtaining robust correspondences can be nontrivial. Third, the method was tested on relatively constrained settings, leaving it unclear how well it scales to real-world robotics or continuous domains with high state space complexity. Possible extensions include (i) automating or relaxing the anchor-collection procedure to handle more diverse or partially observable environments, and (ii) validating SAPS on robotics tasks with real sensors and complex dynamics, where retraining from scratch is particularly time-consuming. %; and (iii) investigate how alignment methods can be leveraged for stitching at multiple levels of RL hierarchies, e.g., sub-policy alignment for complex mission-oriented tasks.