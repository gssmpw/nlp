\subsection{Latent Space Analysis}\label{sec:latent-analysis}

To assess how effectively our proposed method (SAPS) aligns latent representations across different visual or task variations, we perform both qualitative and quantitative evaluations.

\paragraph{Qualitative Visualization.}
Figure \ref{fig:pca-carracing} (left) presents a 3D PCA projection of latent embeddings for two CarRacing variations: green and red backgrounds. After applying an affine transformation learned by SAPS to the encoder outputs, the points corresponding to green and red observations become thoroughly intermixed in the shared latent space. This intermixing indicates that frames depicting the “same portion” of the track with different background colors now map to similar embeddings (see insets). Conversely, Figure \ref{fig:pca-carracing} (right) shows the unaligned embeddings, where green and red remain clearly separated. These results suggest that SAPS successfully bridges the gap between encoders, producing a unified representation space under visual variation.

\begin{figure}[t!]
    \centering
    \includegraphics[width=\linewidth]{res/analysis/carracing_align_vs_abs_latent_frames.pdf}
    \caption{PCA visualization of encoder outputs. On the left, we illustrate how an affine alignment can effectively map one latent space to another: same frames with different backgrounds (green/red) cluster together, as indicated by the embedded screenshots. On the right, the source, unaligned embeddings remain separated, highlighting the benefit of our alignment approach in unifying observations from different environment variations.}
    \label{fig:pca-carracing}
\end{figure}

\paragraph{Quantitative Similarities.}
In Figure~\ref{fig:pairwise-histograms}, we plot histograms of pairwise cosine similarity for matched frames from two different variations. Again, we compare \textbf{(a)~SAPS}, \textbf{(b)~R3L} and \textbf{(c)~Naive} stitching.
The top row shows CarRacing, while the bottom row is LunarLander. In both environments, SAPS and R3L achieve a much higher mean cosine similarity (e.g., 0.92–0.99) than the Naive baseline (0.23–0.30). This confirms that independently trained models can exhibit near-identical encodings for semantically identical frames once aligned, while “naive” combinations of encoders and controllers remain incompatible.

\paragraph{Discussion.}
Overall, these findings indicate that: (i) SAPS’ affine transformation effectively repositions points in the latent space, causing corresponding frames to map to nearly the same vector (Figures \ref{fig:pca-carracing}–\ref{fig:pairwise-histograms}); (ii) Compared to “Naive” reusability (no alignment) or purely relative approaches (R3L), SAPS achieves equivalent or better alignment without retraining models on a specialized representation format; (iii) The high average cosine similarity under SAPS confirms that visual variations (and, by extension, moderate task changes) can be handled by learning a lightweight transform from one latent space to another.

Hence, our latent space analysis demonstrates that SAPS successfully stitches together components from different RL models to produce a cohesive, unified representation, paving the way for zero-shot policy reuse in previously unseen environment variations.


\begin{figure}[t!]
    \centering
    \begin{subfigure}[b]{0.33\linewidth}
        \centering
        \includegraphics[width=\linewidth]{res/analysis/CarRacing-v2_green_CarRacing-v2_red_pairwise_dist_transl.pdf}
        \caption{SAPS}
        \label{fig:frames-sim}
    \end{subfigure}%
    \begin{subfigure}[b]{0.33\linewidth}
        \centering
        \includegraphics[width=\linewidth]{res/analysis/CarRacing-v2_green_CarRacing-v2_red_pairwise_dist_rel.pdf}
        \caption{R3L}
        \label{fig:frames-sim}
    \end{subfigure}
    \begin{subfigure}[b]{0.33\linewidth}
        \centering
        \includegraphics[width=\linewidth]{res/analysis/CarRacing-v2_green_CarRacing-v2_red_pairwise_dist_abs.pdf}
        \caption{Naive}
        \label{fig:frames-sim}
    \end{subfigure}
        \begin{subfigure}[b]{0.33\linewidth}
        \centering
        \includegraphics[width=\linewidth]{res/analysis/LunarLanderRGB_white_LunarLanderRGB_red_pairwise_dist_transl.pdf}
        \caption{SAPS}
        \label{fig:frames-sim}
    \end{subfigure}%
    \begin{subfigure}[b]{0.33\linewidth}
        \centering
        \includegraphics[width=\linewidth]{res/analysis/LunarLanderRGB_white_LunarLanderRGB_red_pairwise_dist_rel.pdf}
        \caption{R3L}
        \label{fig:frames-sim}
    \end{subfigure}
    \begin{subfigure}[b]{0.33\linewidth}
        \centering
        \includegraphics[width=\linewidth]{res/analysis/LunarLanderRGB_white_LunarLanderRGB_red_pairwise_dist_abs.pdf}
        \caption{Naive}
        \label{fig:frames-sim}
    \end{subfigure}
    \caption{Histogram of pairwise cosine similarities between matched states from two different environment variations, for CarRacing (\textbf{top}) and LunarLander (\textbf{bottom}). Both SAPS and R3L show very high mean similarity along paired frames, indicating that corresponding observations in each variation map to nearly identical vectors. Mean similarity for encoders without any alignment or relative encoding is very low, emphasizing the utility of latent communication methods.}
    \label{fig:pairwise-histograms}
\end{figure}