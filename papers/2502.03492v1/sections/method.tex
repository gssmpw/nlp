\section{Method}
With analysis presented in \cref{sec:preliminary}, our goal is to teach LLMs the ability of critiquing without human supervision.
We propose {\ours}, a two-stage training approach: (1) synthesizing high-quality critiques by reasoning about execution feedback, then (2) refining the critic through reinforcement learning.
Once trained, the critic model can be used at test time, paired with any generator models, to iteratively refine solutions.
A complete overview of the pipeline is provided in \cref{appendix:pipeline}, with critique samples in \cref{appendix:samples}.

\subsection{Problem Statement}
We focus on code generation as our primary domain as it provides clear objective metrics through test cases, following previous work~\cite{mcaleese2024llm}.
Given a programming problem $x$ (specified in natural language) and a solution $y$ (code implementation), our goal is to enable iterative refinement of solutions, which centers on two key components: (1) a generator model $\pi(y \mid x)$ that proposes solutions, and (2) a critic model $Q_\theta(c|x,y)$ that provides textural feedback $c$ for improvement.

\paragraph{Assumptions.}
Let $\mathcal{D} = \{(x_i, T_i)\}_{i=1}^N$ be our training dataset, where each problem $x_i$ is paired with unit tests $T_i$.
We have access to a sandbox environment that executes code against test cases, which serves as the evaluation function $R(y)$ that returns 1 if $y$ passes all tests, 0 otherwise.
Notably, the sandbox does not assist critique generation at test time.
While not required, we treat the generator model as a \emph{black-box}, allowing our approach to build upon existing strong generators without access to their parameters.


\paragraph{Objective.}
While directly measuring the helpfulness of critiques remain challenging, we can define a \emph{proxy task} that evaluates whether the critique leads to improved solutions.
Given an initial solution $y^\prime \sim \pi(\cdot \mid x)$, the critic analyzes it and produces textual feedback $c$. The generator then uses this feedback to revise the solution, producing an improved output $y$.
Let $z = (x, y^\prime)$ represent the problem-solution pair.
Our objective is to train the critic model $Q_\theta$ to maximize the expected solution quality:
\begin{equation}\label{eq:objective}
    \mathcal{J}(\theta) = \mathbb{E}_{z\sim \mathcal{D} \times \pi, y \sim \pi_\theta(\cdot \mid z)}[R(y)],
\end{equation}
where $\pi_{\theta}(y \mid z) = \sum_{c} {Q_\theta}(c \mid z) \pi(y \mid z, c)$ denotes the improved solution distribution through marginalization over possible critiques.
Notably, although \cref{eq:objective} defines a single-turn critique-revision task, we observe that the trained model generalizes to multi-turn revisions (\cref{sec:exp_critique_revision}).

\paragraph{Defining the Critique Space.} We structure the critique space into three components (\cref{fig:illustration}): (1) an analysis of the solution's strengths and weaknesses, (2) actionable improvement suggestions, and (3) a final judgment of correctness (correct/incorrect). During inference, these components enable iterative critique-revision, where the process stops once the judgment indicates the solution is correct. This design balances discrimination and critiquing, both essential for iterative refinement, as discussed in \cref{sec:preliminary}.




\subsection{Stage I: Execution-guided Critique Synthesis}\label{sec:sft}
Although conceptually straightforward, learning effective critiques is challenging due to the large critique space, where only a small fraction leads to successful revisions. Our experiments with Qwen2.5-Coder~\cite{hui2024qwen2} (\cref{tab:cc_res} show that models struggle to generate informative critiques for self-improvement, aligning with previous findings~\cite{huang2023large}. Self-critique without additional feedback yields minimal gains (7.88\% → 8.36\%) and rarely converts incorrect solutions to correct ones, highlighting the limited ability of models to correct their own mistakes.

\paragraph{Reasoning over Execution.}
While the initial critiquing ability is limited, previous work~\cite{ni2024next} has shown that LLMs can effectively reason over execution feedback.
\cref{tab:cc_res} demonstrates that when LLMs reason over execution feedback to generate critiques (Self-critique w/ Execution Feedback), they achieve substantial improvements, as compared to directly using raw execution feedback for revisions (11.76\% vs. 8.97\%).
This suggests that while directly using raw execution feedback is inefficient, we can leverage the model's reasoning ability over execution feedback to help generate more accurate and informative critiques.

\paragraph{Critique Synthesis.} Building on the above insight, we develop a critique synthesis approach that leverages execution feedback to train models in generating effective critiques.
Our approach samples high-quality synthesized critiques from a hinted distribution ${Q_\theta}(c \mid z, h)$, where hints $h$ are constructed by analyzing initial solutions $y^\prime$ through sandbox execution.
We map different execution outcomes to specific hint templates as shown in \cref{tab:hint}: (1) for passing solutions, we encourage concise positive feedback; (2) for completely failing solutions, we suggest restarting from scratch; and (3) for partially failing solutions, we provide the exact error message and test case details to help pinpoint the issue.


\paragraph{Supervised Finetuning.}
Similar to context distillation~\cite{snell2022learning,guan2024deliberative}, we exclude these hints and conduct supervised finetuning to encourage the model to internalize the critiquing strategies.
We observe leveraging execution feedback for supervised finetuning is beneficial mainly in two aspects: (1) it helps learn the format; (2) while it marginally improves the critique-revision performance due to the high frequency of instructing correct solutions to wrong (\cref{tab:cc_res}), it substantially boosts discrimination by providing ground-truth correctness (\cref{tab:cc_discrimination}).







\begin{table}[t!]
\centering
\small
\caption{Critique-revision performance (Pass@1, \%) on CodeContests. We fix the generator model to be Qwen2.5-Coder, and compare zer-shot performance with critique-revision performance using different feedback mechanisms. $\times k$ represents conducting iterative critique-revision $k$ times.
$^\dag$using unit tests for generation.}
\label{tab:cc_res}
\vspace{2mm}

\begin{tabular}{lccc}
\toprule
 & \multicolumn{1}{c}{\textbf{Pass@1}} & \multicolumn{1}{c}{\textbf{$\Delta_\uparrow$}} & \multicolumn{1}{c}{\textbf{$\Delta_\downarrow$}} \\ \midrule
Zero-shot & 7.88 & 0.00 & 0.00 \\
Execution Feedback (EF)$^\dag$ & 8.97 &	2.42 &	1.33 \\
Self-critique w/ EF$^\dag$ & 11.76 & 3.88 & \textbf{0.00} \\
\midrule
Self-critique & 8.36 & 2.30 & 1.82 \\
Critique w/ {\ours}$_\text{SFT}$ & 8.36 & 3.52 & 3.03 \\
Critique w/ {\ours} & 11.76 & 4.73 & 0.85 \\
Critique$\times 2$ w/ {\ours} & 14.18 & 7.27 & 0.97 \\
Critique$\times 3$ w/ {\ours} & \textbf{15.15} & \textbf{8.12} & 0.85 \\
\bottomrule
\end{tabular}
\vspace{-3mm}
\end{table}

\begin{table}[t]
\centering
\caption{Discrimination performance (F1 score, \%) on CodeContests.}
\vspace{2mm}
\label{tab:cc_discrimination}
\small
\begin{tabular}{lccc}
\toprule
& \textbf{Passed} & \textbf{Failed} & \textbf{Macro} \\
\midrule
Qwen2.5-Coder & 88.21 & 34.16 & 61.19 \\
{\ours}$_\text{SFT}$ & \textbf{95.54} & 41.26 & 68.55 \\
{\ours} & 93.19 & \textbf{45.02} & \textbf{69.10} \\
\bottomrule
\end{tabular}
\vspace{-3mm}
\end{table}




\subsection{Stage II: Reinforced Critique Generation}
While our critique synthesis approach with predefined templates provides a strong foundation, it may not capture all nuanced feedback scenarios required for complex programming tasks. To overcome this limitation, we formulate critique generation as a reinforcement learning problem, allowing the critic to adaptively learn feedback strategies through direct optimization of solution improvement.


Our goal is to maximize the performance in \cref{eq:objective}.
To optimize $Q_\theta$, one natural approach is using policy gradient methods~\cite{sutton1999policy}:
\begin{equation*}
\begin{aligned}
    &\nabla_\theta \mathbb{E}_{y \sim \pi_{\theta}}[R(y)] \\
    = &\nabla_\theta \mathbb{E}_{y \sim \sum_c Q_\theta(c|x,y)\pi(y|x,c)}[R(y)] \\
    = &\nabla_\theta \sum_y R(y) \sum_c Q_\theta(c|x,y)\pi(y|x,c) \\
    = &\sum_y R(y) \sum_c \nabla_\theta Q_\theta(c|x,y)\pi(y|x,c)
\end{aligned}
\end{equation*}
The double summation over both solution space $y$ and feedback space $c$ introduces high variance in gradient estimates:
\begin{equation*}
    \mathrm{Var}(\nabla_\theta) = \mathbb{E}[(\nabla_\theta - \mathbb{E}[\nabla_\theta])^2] \propto |\mathcal{Y}| \cdot |\mathcal{C}|.
\end{equation*}
where $|\mathcal{Y}|$ and $|\mathcal{C}|$ are the sizes of solution and critique spaces respectively.
In this scenario, using value networks to predict credit assignment remains challenging, as we observe significant instability when using Proximal Policy Optimization (PPO; \citealt{schulman2017proximal})\,---\,the learned networks produce noisy estimates of critique quality.
We present detailed experimental observations in\cref{appendix:credit_assignment}.






\begin{table*}[!ht]\centering
\caption{Comprehensive performance comparison across different evaluation settings using top-k accuracy (\%). Results are shown for overall performance (All), high-complexity reactions (Complex), and innovative reactions (Novel). Higher percentages indicate better performance. Bold text indicates best performance with statistically significant improvement over the next best in each category.}
\label{tab:res-main}
\begin{tabular}{l|ccc|ccc|ccc}\toprule
& \multicolumn{3}{c|}{\textbf{All}} & \multicolumn{3}{c|}{\textbf{Complex}} & \multicolumn{3}{c}{\textbf{Novel}} \\
\textbf{Method} & \textbf{Top-1 (\%)} & \textbf{Top-3 (\%)} & \textbf{Top-5 (\%)} & \textbf{Top-1 (\%)} & \textbf{Top-3 (\%)} & \textbf{Top-5 (\%)} & \textbf{Top-1 (\%)} & \textbf{Top-3 (\%)} & \textbf{Top-5 (\%)} \\\midrule
Chemical-Only \cite{kreutter2021predicting} &37.29 &54.65 &61.39 &30.78 &44.92 &51.56 &18.25 &33.75 &40.82 \\
EC Tokens \cite{Probst2022} &46.01 &61.71 &66.64 &34.26 &51.22 &56.79 &31.42 &44.46 &49.45 \\
ESM3 \cite{hayes2024simulating} &44.58 &60.64 &65.62 &34.23 &48.57 &54.11 &27.39 &42.35 &47.47 \\
ProtBERT \cite{brandes2021proteinbert} &43.49 &59.52 &64.82 &33.11 &45.48 &50.42 &31.21 &40.12 &44.66 \\
GearNet \cite{zhang2023protein} &43.95 &60.10 &63.84 &28.09 &40.80 &46.23 &20.16 &32.62 &39.13 \\
ReactEmbed\cite{sicherman2025reactembed} &47.41 &61.19 &66.25 &36.90 &50.87 &55.18 &26.37 &42.68 &48.08 \\
DAA (Ours) &\textbf{49.96} &\textbf{66.65} &\textbf{71.48} &\textbf{41.62} &\textbf{56.43} &\textbf{62.20} &\textbf{35.24} &\textbf{50.66} &\textbf{55.54} \\
\bottomrule
\end{tabular}
\end{table*}

\paragraph{Variance Reduction.}
To combat these variance issues, we adopt Group Relative Policy Optimization (GRPO; \citealt{shao2024deepseekmath}) that avoids using value networks for learning credit assignment and reduces variance through group-based relative advantages. Specifically, for each problem-solution pair $z=(x,y^\prime)$, we samples a group of critiques $\{c_1, c_2, ..., c_G\}$ from $Q_\theta(\cdot|z)$ and computes advantages:
\begin{equation*}
    A_i = \frac{R(y_i) - \mu_G}{\sigma_G},
\end{equation*}
where $y_i \sim \pi(\cdot|z,c_i)$ is the improved solution generated using critique $c_i$, and $\mu_G$ and $\sigma_G$ are the mean and standard deviation of rewards within the group.
This approach normalizes rewards across different problem types and naturally focuses training on problems where critique quality can make a meaningful difference, as problems that are too easy or too hard produce zero relative advantages.
The final training objective is:
\begin{equation*}
\begin{aligned}
    &\mathcal{J}(\theta) = \mathbb{E}_{z \sim \mathcal{D}, \{c_i\}_{i=1}^G \sim Q_{\theta_{\text{old}}}(y|x)} \Big[ \\
    &\quad\frac{1}{G} \sum_{i=1}^G \Big(\min\big(\frac{Q_\theta(y_i|x)}{Q_{\theta_{\text{old}}}(y_i|x)}A_i, \operatorname{clip}_\varepsilon \big(\frac{Q_\theta(y_i|x)}{Q_{\theta_{\text{old}}}(y_i|x)}\big)A_i\big)\Big) \\
    &\quad- \beta\mathbb{D}_{\text{KL}}(Q_\theta\|Q_{\text{ref}})\Big],
\end{aligned}
\end{equation*}

where $\operatorname{clip}_\varepsilon$ represents clipping the value to $[1-\varepsilon, 1+\varepsilon]$ and $\mathbb{D}_{\text{KL}}(Q_\theta\|Q_{\text{ref}}) = \frac{Q_{\text{ref}}(y_i|x)}{Q_\theta(y_i|x)} - \log\frac{Q_{\text{ref}}(y_i|x)}{Q_\theta(y_i|x)} - 1$ denotes the KL regularization term that alleviates over-optimization.


