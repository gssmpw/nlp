\section{Preliminaries and Motivation}\label{sec:preliminary}
The success of iterative improvement methods critically depends on their ability to leverage feedback to improve solutions.
Formally, let $x$ be an input problem and $y$ be a candidate solution, with $R(y)$ being the evaluation function that returns 1 if $y$ is correct and 0 otherwise.
Starting with an initial proposal distribution $y_0 \sim \pi(\cdot \mid x)$, the iterative process generates subsequent solutions by incorporating feedback $f(\cdot \mid x, y_i)$ and produce the next solution $y_{i+1}$.

In this context, the effectiveness of such feedback mechanisms relies on two key capabilities: (1) \emph{discrimination} - the ability to evaluate and rank solutions, and (2) \emph{critiquing} - the ability to provide actionable feedback for improvement. While discrimination has been extensively studied~\cite{gao2023scaling}, we focus on the critiquing ability and propose to characterize it through the transition dynamics of a Markov chain~\cite{meyn2012markov} governing the correctness of the iteratively refined solutions $\{R(y_i)\}_i$:
\begin{equation*}
    \begin{aligned}
        P(R(y_0) = 1) &= p_{\mathrm{init}},\\
        P(R(y_{i+1}) = 1 \mid R(y_{i}) = 1) &= p_{\mathrm{cc}},\\
        P(R(y_{i+1}) = 1 \mid R(y_{i}) = 0) &= p_{\mathrm{cw}},
    \end{aligned}
\end{equation*}
where $p_{\mathrm{cc}}$ represents the critiquing ability to avoid turning correct solutions into wrong ones, and $p_{\mathrm{cw}}$ captures the helpfulness of the feedback in improving the solution.

\paragraph{Varying Critiquing Ability.}
To understand the importance of the critiquing ability, we conduct simulations across different levels of critiquing strength while leveraging discrimination to aggregate the final solutions. We consider $p_{\mathrm{init}} = 0.1$ and three scenarios:
(1) No critiquing ($p_{\mathrm{cw}} = p_{\mathrm{cc}}$), a special case representing methods that independently sample from the base distribution, or equivalently best-of-$n$ sampling~\cite{sessa2024bond};
(2) Weak critiquing ($p_{\mathrm{cc}} = 0.7$, $p_{\mathrm{cw}} = 0.15$); and
(3) Strong critiquing ($p_{\mathrm{cc}} = 0.9$, $p_{\mathrm{cw}} = 0.3$).
For each scenario, we first generate $n$ solutions based on the specified transition dynamics, then apply the discrimination ability to select the best promising solution, and plot the final correctness probability against the number of attempts $n$.
We present more details in \cref{appendix:simulation}.

\begin{figure}[t]
    \centering
    \includegraphics[width=\columnwidth]{figs/simulation_v2.pdf}
    \vspace{-5mm}
    \caption{Simulation results showing success probability ($p_{\text{correct}}$) as a function of the number of attempts, comparing different levels of critiquing and discrimination ability.}
    \label{fig:simulation}
    \vspace{-5mm}
\end{figure}

\paragraph{Observations \& Takeaways.} 
As shown in \cref{fig:simulation}, our analysis reveals several key findings:
\textbf{(1)} Strong critiquing abilities significantly improve success rates compared to no critiquing, with performance gains visible even with weak critiquing, aligning with recent empirical findings~\cite{huang2023large}.
\textbf{(2)} Strong critiquing ability can compensate for weaker discrimination\,---\,a system with weak discrimination but strong critiquing feedback can outperform one with stronger discrimination but no critiquing ability.
\textbf{(3)} The benefits of critiquing compound with more iterations, while approaches with no critiquing plateau quickly.
These findings highlight that effective iterative improvement requires careful attention to both discrimination and critiquing abilities. While perfect abilities are not necessary, systematically improving these capabilities\,---\,particularly the ability to generate actionable critiques\,---\,is crucial for realizing the full potential of iterative refinement approaches.
