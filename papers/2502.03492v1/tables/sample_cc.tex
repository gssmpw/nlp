\begin{table}[]
    \centering
    \small
    \caption{A sample problem from CodeContests~\cite{li2022competition}, with by the solution generated by Qwen2.5-Coder and the critique provided by the {\ours} critic.}
    \label{tab:cc_sample}
\begin{longtable}{m{0.45\textwidth}m{0.45\textwidth}}
\toprule
\textbf{Problem \& Solution}  &
\textbf{Critique} \\ 
\midrule
\begin{lstlisting}[frame=none,
    backgroundcolor=\color{white},
    numbers=none,
    xleftmargin=0pt,
    framexleftmargin=0pt,
    framexrightmargin=0pt,
    framextopmargin=0pt,
    framexbottommargin=0pt]
You had n positive integers a_1, a_2, ..., a_n arranged in a circle. For each pair of neighboring numbers (a_1 and a_2, a_2 and a_3, ..., a_{n - 1} and a_n, and a_n and a_1), you wrote down: are the numbers in the pair equal or not.

Unfortunately, you've lost a piece of paper with the array a. Moreover, you are afraid that even information about equality of neighboring elements may be inconsistent. So, you are wondering: is there any array a which is consistent with information you have about equality or non-equality of corresponding pairs?
\end{lstlisting}


\begin{lstlisting}[language=python]
def can_form_array(s):
    n = len(s)
    # We will try to assign values to the array a
    # Start with assigning 1 to the first element
    a = [1] * n
    
    # Try to assign values based on the given conditions
    for i in range(n):
        if s[i] == 'E':
            a[(i + 1) %
        else:
            a[(i + 1) %
    
    # Check if the last and first elements are consistent with s[n-1]
    if s[-1] == 'E' and a[0] != a[-1]:
        return "NO"
    if s[-1] == 'N' and a[0] == a[-1]:
        return "NO"
    
    return "YES"

def main():
    import sys
    input = sys.stdin.read().splitlines()
    t = int(input[0])
    results = []
    for i in range(1, t + 1):
        s = input[i]
        results.append(can_form_array(s))
    
    for result in results:
        print(result)

if __name__ == "__main__":
    main()
\end{lstlisting} 
&
\begin{lstlisting}[frame=none,
    backgroundcolor=\color{white},
    numbers=none,
    xleftmargin=0pt,
    framexleftmargin=0pt,
    framexrightmargin=0pt,
    framextopmargin=0pt,
    framexbottommargin=0pt]
Analysis:
The provided solution attempts to construct an array based on the sequence of 'E' and 'N' but does not correctly handle the circular nature of the array and the constraints imposed by these characters. Specifically, the approach does not ensure consistency across the entire circular arrangement, particularly at the boundary where the last element connects back to the first.

The code assigns values to the array elements based on whether the current character is 'E' or 'N', but it does not account for potential conflicts that arise when the array wraps around. This means that the final check between the first and last elements might fail due to inconsistencies introduced earlier in the array construction process.

Improvement suggestions:
1. Consider using a more systematic approach to handle the circular nature of the array. One possible method is to use a graph traversal algorithm, treating each position in the array as a node and each 'E' or 'N' as an edge with a label indicating equality or inequality.
2. Use a union-find data structure to manage equivalence classes of indices that must hold the same value due to 'E' connections.
3. Iterate through the string and apply the rules: if 'E', ensure the indices are in the same set; if 'N', ensure they are in different sets.
4. After processing the string, verify that no conflicts exist in the union-find structure, particularly focusing on the circular connection between the first and last elements.
5. If no conflicts are found, it is possible to construct an array that satisfies the conditions; otherwise, it is not.

Overall judgment: Incorrect
\end{lstlisting}
\\ 
\bottomrule
\end{longtable}
\end{table}
