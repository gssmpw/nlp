\begin{table}[]
    \centering
    \small
    \caption{A sample problem from LiveCodeBench~\cite{jain2024livecodebench}, with by the solution generated by Qwen2.5-Coder and the critique provided by the {\ours} critic.}
    \label{tab:lcb_sample}
\begin{longtable}{m{0.45\textwidth}m{0.45\textwidth}}
\toprule
\textbf{Problem \& Solution}  &
\textbf{Critique} \\ 
\midrule
\begin{lstlisting}[frame=none,
    backgroundcolor=\color{white},
    numbers=none,
    xleftmargin=0pt,
    framexleftmargin=0pt,
    framexrightmargin=0pt,
    framextopmargin=0pt,
    framexbottommargin=0pt]
You are given a binary string s and an integer k.
A binary string satisfies the k-constraint if either of the following conditions holds:

The number of 0's in the string is at most k.
The number of 1's in the string is at most k.

Return an integer denoting the number of substrings of s that satisfy the k-constraint.
\end{lstlisting}

\begin{lstlisting}[language=python]
class Solution:
    def countKConstraintSubstrings(self, s: str, k: int) -> int:
        def count_substrings_with_max_char(s, max_char, k):
            left = 0
            count = 0
            current_count = 0
            result = 0
            
            for right in range(len(s)):
                if s[right] == max_char:
                    current_count += 1
                
                while current_count > k:
                    if s[left] == max_char:
                        current_count -= 1
                    left += 1
                
                result += right - left + 1
            
            return result
        
        count_0s = count_substrings_with_max_char(s, '0', k)
        count_1s = count_substrings_with_max_char(s, '1', k)
        
        return count_0s + count_1s
\end{lstlisting}
 &
\begin{lstlisting}[frame=none,
    backgroundcolor=\color{white},
    numbers=none,
    xleftmargin=0pt,
    framexleftmargin=0pt,
    framexrightmargin=0pt,
    framextopmargin=0pt,
    framexbottommargin=0pt]
Analysis:
The provided solution attempts to solve the problem by counting substrings based on the maximum occurrence of '0' and '1' up to `k`, but it does not correctly account for all possible substrings that satisfy the k-constraint. The approach incorrectly calculates the intersection of substrings that satisfy both conditions without properly considering the overlap between these substrings. Additionally, the logic for counting substrings is flawed as it does not consider every possible substring of `s`; instead, it focuses on contiguous sequences of '0's and '1's up to `k`.

Improvement suggestions:
1. **Re-evaluate the Approach**: Instead of focusing on contiguous sequences of '0's and '1's, the solution should consider every possible substring of `s` and check if it satisfies the k-constraint.
2. **Iterate Over All Substrings**: Use nested loops to iterate over all possible substrings of `s`. For each substring, count the occurrences of '0' and '1' and check if either count is less than or equal to `k`.
3. **Count Valid Substrings**: Maintain a counter to keep track of how many substrings meet the k-constraint criteria.
4. **Simplify the Logic**: Consider using Python's slicing capabilities to easily extract and evaluate each substring without unnecessary checks for contiguous characters.

Overall judgment: Incorrect
\end{lstlisting}
\\ 
\bottomrule
\end{longtable}
\end{table}
