\subsection{Additional Experiments}
\label{sec:UQ_BENCH_additional_experiment}
%In this section we detail the experiments we conduct using 
Here we present additional experiments from Section \ref{sec:eval_posterior_consis}.


%As discussed in Section~\ref{sec:benchmark-uq-metric}, we use the joint log-loss~\eqref{eqn:joint-log-loss} to evaluate all UQ agents.



  



% \begin{figure}[h]
% \centering
% \begin{minipage}[b]{0.24\textwidth}
% \centering
% \includegraphics[width = \textwidth]{figures/eicu_clustering_improvement_plots/in_distribution_mean_T0.pdf}
% \caption*{{(a)}~ ID performance (T$=$0)  [SOTA] }
% \end{minipage}
% \hfill
% \begin{minipage}[b]{0.24\textwidth}
% \centering \includegraphics[width = \textwidth]{figures/eicu_clustering_improvement_plots/out_of_distribution_mean_T0.pdf}
% \caption*{{(b)}~ OOD performance, T$=$0  [Ours] }
% \end{minipage}
% \hfill
% \begin{minipage}[b]{0.24\textwidth}
% \centering \includegraphics[width = \textwidth]{figures/eicu_clustering_improvement_plots/out_of_distribution_mean_T50.pdf}
% \caption*{{(c)}~ OOD performance, $T=1$ [Ours]}
% \label{fig:eicu_clustering_improvement-k=30}
% \end{minipage}
% \hfill
% \begin{minipage}[b]{0.24\textwidth}
% \centering \includegraphics[width = \textwidth]{figures/eicu_clustering_improvement_plots/out_of_distribution_mean_difference.pdf}
% \caption*{{(d)} ~ OOD improvement ($T=0 \to 1$)  [Ours]}
% \end{minipage}
% \caption{Performance of different agents in a dynamic setting. \textbf{Experimental Setup:} Population is distributed across 2 clusters.  At $T=0$, agent  observes data $\mc{D}_0$ sampled from Cluster 1. 
% Figure (a) and Figure (b) show how different UQ models perform on test data from Cluster 1 (in-distribution) and Cluster 2 (out-of-distribution). At $T=1$, UQ agent observes some data $\mc{D}_1$  sampled from Cluster 2.  Figure (c) performance on test data from Cluster 2 at $T=1$. Figure (d)  showcase performance improvement on test data from Cluster 2 at $T=1$. \textbf{Observations:} Comparison between Figure (a) and Figure (b) signifies the trade-off between  the  in-distribution and out-of-distribution performance discussed previously. Figure (d) shows   epinets are much worse at adapting to new data compared to the other two agents.}
% \label{fig:dynamic_setting_k_30}
% \end{figure}


% \begin{figure}[h]
%     \centering
%     \includegraphics[width=\textwidth]{figures/UQBench-dynamic-histogram.pdf}
%     \caption{Performance of different agents in a dynamic setting.}
% \label{fig:dynamic_setting_k_30-v2}
% \end{figure}



% \begin{figure}
% \centering
% \begin{minipage}[b]{0.32\textwidth}
% \centering
% \includegraphics[width = \textwidth]{figures/eicu_clustering_improvement_plots/in_distribution_mean_T0.pdf}
% \caption*{{(a)}~ In-distribution performance (T$=$0) 

% [SOTA Benchmark] }
% \end{minipage}
% \hfill
% \begin{minipage}[b]{0.32\textwidth}
% \centering \includegraphics[width = \textwidth]{figures/eicu_clustering_improvement_plots/out_of_distribution_mean_T0.pdf}
% \caption*{{(b)}~ Out-of-distribution  performance (T$=$0) 

% [OOD-static Benchmark] }
% \end{minipage}
% \hfill
% \begin{minipage}[b]{0.32\textwidth}
% \centering \includegraphics[width = \textwidth]{figures/eicu_clustering_improvement_plots/out_of_distribution_mean_difference.pdf}
% \caption*{{(c)} ~ OOD performance improvement (T$=$0) to (T$=$1)  

% [Dynamic Benchmark]}
% \end{minipage}
% \caption{Performance of different agents in a dynamic setting. \textbf{Experimental Setup:} Population distributed across 2 clusters.  At $T=0$ agent  observes data $\mc{D}_0$ sampled from Cluster 1. Figure(a) and Figure (b) shows how different UQ models perform on test data from cluster 1 (in-distribution) and cluster 2 (out-of-distribution). At $T=1$, UQ agent observes some data $\mc{D}_1$  sampled from Cluster 2.  Figure(c) and (d) showcase the performance on test data from cluster 1 and cluster 2 respectively at $T=1$. \textbf{Observations:} Comparison between Figure (a) and Figure (b) signifies the trade-off between  the   }
% \label{fig:dynamic_setting_k_30}
% \end{figure}





% In limited data setting over training on distribution data might make the performance worse on the out-distribution. Though this is expected, but there is no reliable way to determine where should we stop - strategies like making this training/weight decay dependent on the amount of  data etc.

% \paragraph{hyperparameter tuning is difficult}- trade-off between in-distribution performance and out-distribution performance

% We observe in Figures~????????????????????????????????????????????????\ref{} that all UQ agents have a tension that sharper posterior on in-distribution data typically lead to worse performance on out-support distribution data. 

% In Section~\ref{sec:experiment}, we empirically demonstrate that certain UQ modules perform well on distributions that are similar to $\xtrain$ but it deteriorates by a large amount as $\dist{\xeval}{\xtrain}$ increases. Moreover, this is also the case when the analysis is using the UQ module to perform active exploration.










% \begin{figure}
% \centering
% \begin{minipage}[b]{0.32\textwidth}
% \centering
% \includegraphics[height=4.5cm]{figures/eicu_clustering/Task_1_ood_id_ensemble+_Metric_vs._Number_of_Batches_for_k_val=30_dynamic_0.pdf}
% \caption{Trade-off between  in-distribution performance and out-of-distribution performance with stopping time as the hyperparameter (ensemble++).}
% \label{fig:ensemble+_difficult_to_choose_stopping_time}
% \end{minipage}
% \hfill
% \begin{minipage}[b]{0.32\textwidth}
% \centering \includegraphics[height=4.5cm]{figures/eicu_clustering/Task_1_ood_id_epinet_Metric_vs._Number_of_Batches_for_k_val=30_dynamic_0.pdf}
% \caption{Trade-off between  in-distribution performance and out-of-distribution performance with weight decay as the hyperparameter (epinet). }
% \label{fig:epinet_difficult_to_choose_stopping_time}
% \end{minipage}
% \hfill
% \begin{minipage}[b]{0.32\textwidth}
% \centering \includegraphics[height=4.5cm]{figures/eicu_clustering/Task_1_ood_id_hypermodel_Metric_vs._Number_of_Batches_for_k_val=30_dynamic_0.pdf}
% \caption{Trade-off between  in-distribution performance and out-of-distribution performance with weight decay as the hyperparameter (hypermodel). }
% \label{fig:hypermodel_difficult_to_choose_stopping_time}
% \end{minipage}
% \end{figure}


    
% \paragraph{hyperparameter tuning is difficult} - similar to out-of-distiribution. It becomes even more difficult of how to vary weight decay ood presence - same hyperparameters might not work for newly acquired data.





  
%  In limited data setting over training on distribution data might make the performance worse on the out-distribution. Though this is expected, but there is no reliable way to determine where should we stop - strategies like making this training/weight decay dependent on the amount of  data etc. have been suggested - but these strategies seem to ad-hoc 








% \begin{figure}
% \centering
% \begin{minipage}[b]{0.24\textwidth}
% \centering
% \includegraphics[height=2.4cm]{figures/eicu_clustering_improvement_plots/Task000_agents_in_distribution_joint log-loss_K_val_15_Batch_200_Dynamic_0.pdf}
% \caption*{$T=0$, id }
% \label{fig:eicu_clustering_improvement-k=15}
% \end{minipage}
% \begin{minipage}[b]{0.24\textwidth}
% \centering \includegraphics[height=2.4cm]{figures/eicu_clustering_improvement_plots/Task000_agents_out_of_distribution_joint log-loss_K_val_15_Batch_200_Dynamic_0.pdf}
% \caption*{$T=0$, ood }
% %\label{fig:eicu_clustering_improvement-k=15}
% \end{minipage}
% \hfill
% \begin{minipage}[b]{0.24\textwidth}
% \centering \includegraphics[height=2.4cm]{figures/eicu_clustering_improvement_plots/Task000_agents_in_distribution_joint log-loss_K_val_15_Batch_200_Dynamic_50.pdf}
% \caption*{$T=1$, id}
% %\label{fig:eicu_clustering_improvement-k=15}
% \end{minipage}
% \hfill
% \begin{minipage}[b]{0.24\textwidth}
% \centering \includegraphics[height=2.4cm]{figures/eicu_clustering_improvement_plots/Task000_agents_out_of_distribution_joint log-loss_K_val_15_Batch_200_Dynamic_50.pdf}
% \caption*{$T=1$, ood}
% %\label{fig:eicu_clustering_improvement-k=15}
% \end{minipage}
% \label{fig:dynamic_setting}
% \caption{Model Performance under a dynamic setting }
% \end{figure}

% \begin{itemize}
%     \item Posterior consistency under different inference seeds - - Task 2
    

    
%     \item Posterior consistency for different stopping times 

%     Sensitivity to stopping time in-distribution
% 1) Performance deteriorates with early stopping (Bias) - eICU selection bias, eICU clustering - (Marginal log-loss, Joint log-loss OOD)
% 2) Variance also increases - eiCU clustering
% \end{itemize}




% \begin{figure}
% \centering
% \begin{minipage}[b]{0.49\textwidth}
% \centering
% \includegraphics[width=\textwidth, height=6cm]{figures/eicu_clustering/Task_1_ood_id_ensemble+_Metric_vs._Number_of_Batches_for_k_val=30_dynamic_0.pdf}
% \caption{ensemble+, initial data}
% \label{fig:eicu-cluster-task-1-ensemble+-initial}
% \end{minipage}
% \hfill
% \begin{minipage}[b]{0.49\textwidth}
% \centering \includegraphics[width=\textwidth, height=6cm]{figures/eicu_clustering/Task_1_ood_id_ensemble+_Metric_vs._Number_of_Batches_for_k_val=30_dynamic_0.pdf}
% \caption{ensemble+, acquired new data}
% \label{fig:eicu-cluster-task-1-ensemble+-dynamic}
% \end{minipage}
% \end{figure} 







 