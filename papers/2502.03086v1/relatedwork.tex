\section{Related Work}
\label{sec:related}
Class imbalance is a persistent challenge in ML, particularly in fields like IDS, where the normal traffic volume far exceeds intrusion events. Several classical methods have been proposed to address this imbalance, including SMOTE and Random Oversampling. SMOTE generates synthetic samples by interpolating between existing minority class samples, while Random Oversampling duplicates minority class samples to achieve balance \cite{chawla2002smote, he2008learning}. Although these methods improve dataset balance, they often lead to issues such as overfitting and redundancy, limiting their effectiveness in real-world scenarios \cite{garcia2012effectiveness} \cite{mohammed2020machine}.

Previous studies, such as those by Gopalan et al. \cite{gopalan2021balancing} and Abdulrahman et al. \cite{abdulrahman2020toward}, emphasize balancing techniques for IDS datasets, highlighting the challenges of achieving fair evaluation metrics. Abdulrahman et al. \cite{abdulrahman2020toward} investigate balancing the CICIDS2017 dataset using classical techniques, which, while effective to some extent, often fail to capture nuanced feature interactions within the data. Liu et al. \cite{liu2022data} propose using Generative Adversarial Networks (GANs) for data balancing, showcasing their potential in generating robust datasets for IDS tasks. 

QRBMs have emerged as a promising alternative to classical methods for handling imbalanced datasets. QRBMs, an extension of classical Restricted Boltzmann Machines (RBMs), leverage the principles of quantum computing to model complex probability distributions and generate synthetic data. Unlike classical RBMs, QRBMs utilize quantum annealing to efficiently sample from high-dimensional energy landscapes, making them particularly well-suited for generative tasks \cite{hinton2006reducing, amin2018quantum}. Prior research has demonstrated the potential of QRBMs in various machine-learning applications, including anomaly detection and classification tasks  \cite{amin2018quantum, dixit2023quantum}.

Implementing QRBMs on quantum hardware poses unique challenges, particularly concerning embedding large models on physical quantum processors. D-Wave’s Chimera topology, an early quantum annealing architecture, provided limited qubit connectivity, which constrained the scalability of QRBM implementations \cite{venturelli2015quantum}. The introduction of D-Wave’s Pegasus topology addressed these limitations by offering enhanced qubit connectivity and increased computational capacity \cite{dwave2020pegasus}. Recent studies have explored methods for minor embedding of problems on Pegasus, showcasing significant improvements in performance and scalability \cite{dattani2019pegasus,pelofske2023clique}.

The proposed algorithm demonstrates remarkable flexibility and efficiency in embedding RBMs on D-Wave's Pegasus architecture. It enables the minor embedding up to a 172x120 RBM by optimizing chain lengths to remain short while maximizing the utilization of qubits for visible and hidden nodes.

This algorithm outperforms D-Wave’s default embedding tool. The ability to efficiently embed larger RBMs underscores the algorithm’s potential to enhance quantum annealing applications, particularly in generative modelling and data balancing tasks.