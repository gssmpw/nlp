\section{Related Works}
In this section we provide a small discussion on related works.

\textbf{Reversible solvers.} The asynchronous leapfrog method____ and the reversible Heun____ were the standard reversible solvers until recently____, with the former applicable to general neural ODEs and the latter applicable to neural ODEs, CDEs, and SDEs.
Recent work by____ has improved upon these older solvers for neural ODEs by showing it is possible to construct reversible solvers with a non-trivial region of stability.

\textbf{Exact inversion with diffusion models.}
The work of____ proposes a reversible solver for diffusion ODEs by solving a dual auxiliary state of the model and then interpolating between the two states.
Later work by____ explores a reversible solver by using bidirectional integration approximation scheme as a sort of leapfrog method.
More recent work by____ has explored the exact inversion of diffusion ODEs via bidirectional linear multi-step methods.
However, linear multi-step methods and leapfrog methods often suffer from poor stability____.


\textbf{Inversion with diffusion SDEs.}
More closely related to our work____ propose a method for the exact inversion of diffusion SDEs.
Given a particular realization of the Wiener process that admits $\bfx_t \sim \mathcal{N}(\alpha_t \bfx_0 \mid \sigma_t^2\mathbf{I})$, then given $\bfx_s$ and noise $\bseps_s \sim \mathcal{N}(\mathbf0, \mathbf I)$ we can calculate
\begin{equation}
    \bfx_t = \frac{\alpha_t}{\alpha_s} \bfx_s + 2\sigma_t(e^h-1)\hat\bfx_{T|s}(\bfx_s) + \sigma_t\sqrt{e^{2h}-1}\bseps_s.
\end{equation}
____ propose to invert this by first calculating for two samples $\bfx_t$ and $\bfx_s$ the noise $\bseps_s$ can be calculated by rearranging the previous equation to find
\begin{equation}
    \bseps_s = \frac{\bfx_t - \frac{\alpha_t}{\alpha_s}\bfx_s + 2 \sigma_t (e^h - 1) \bseps_\theta(\bfx_s, \bfz, s)}{\sigma_t \sqrt{e^{2h} - 1}}
\end{equation}
With this the sequence $\{\bseps_{t_i}\}_{i=1}^N$ of added noises can be calculated which can be used to reconstruct the original input from the initial realization of the Wiener process.
However, unlike our approach, this process requires storing the entire realization in memory.