\section{Related Work}
\label{sec:related_work}

Bahl et al. \cite{bahl2004reconsidering} showed that a multi-radio system is beneficial for wireless networks. A lot of multi-radio wireless systems \cite{ananthanarayanan2009blue, sur2017wifi} has been developed to optimize the performance of the networks like energy-efficiency~\cite{jin2011wizi, kusy2014radio, lymberopoulos2008towards} and routing management~\cite{draves2004routing}. Backpacking~\cite{al2011backpacking} was developed for high data rate sensor networks. Kusy et al.~\cite{kusy2014radio} developed a multi-radio architecture for WSN. They show that employing two multi-hop radios in the same node improves reliability with 3-33\% energy overhead. LoRaCP \cite{gu2019one} employs a ZigBee+LoRa multi-radio network for faster control packet transmissions in multi-hop WSN. Gummeson et al.~\cite{gummeson2009adaptive} optimize
energy consumption by employing a Reinforcement Learning (RL) based adaptive link layer to switch radios (CC2420+XE1205) based on channel dynamics. Using RL is detrimental because of: (i) Very high training data requirements and, (ii) inference latency\cite{sundaram2024mars}. Lymberopoulos et al.~\cite{lymberopoulos2008towards} switch radios (Zigbee+WiFi) with a threshold-based algorithm optimizing for energy efficiency.

While most of the above multi-radio systems developed for IoT networks optimize for energy efficiency over small-scale deployments, MARS \cite{sundaram2024mars} optimizes for throughput and latency on mesoscale applications. Initially, they identified the absence of a fully developed specialized radio for emerging mesoscale IoT applications. To close this gap, they first conducted a qualitative analysis on the suitability of all the available IoT radios for mesoscale application environments and identified that Zigbee2.4GHz and LoRa915MHz are the better candidates. Their analytic and experimental analysis with these two radios shows that Zigbee and LoRa achieve competitive throughput in the \textit{gray-region}, 500-1200m from the gateway~\cite{sundaram2024mars}. Since it is uncertain which radio will provide higher throughput at the time of transmission, they developed a Decision Tree (DT) model using axis-aligned trees to predict the high-throughput radio and further optimize this DT model with Tree-Alternating Optimization (TAO) algorithm\cite{carreira2018alternating}. This ML model needs the link quality of LoRa and the instantaneous path quality estimations of the Zigbee radios. The traditional path quality estimations of Zigbee are not instantaneous. Hence, they develop a DT-based instantaneous path quality estimation to provide instantaneous inputs for the ML model. On taking the instantaneous inputs, their ML model will output the high-throughput radio to transmit the scheduled packet.
% The closest related works are Lymberopoulous et al. \cite{lymberopoulos2008towards}, Kusy et al.\cite{kusy2014radio},   Gummeson et al. \cite{gummeson2009adaptive} and MARS \cite{sundaram2024mars}. MARS already outperforms the above multi-radio systems. However, we compare \Name\xspace with these closely related works. 
\begin{figure}[t]
   \centering
   \includegraphics[scale=0.6]{figs/mot2.pdf}
        \vspace{-0.1in}
        \caption{Motivation for cost-sensitive learning}
        \label{fig:motivation}
\end{figure}
\vspace{-0.2in}