\section{Related Works}
\subsection{Imitation Learning with Force}

Imitation learning has recently experienced significant advancements, driven by the development of more effective algorithms that leverage demonstrations to train robotic policies~\cite{mandlekar2021matters, chi2023diffusion}. Although traditional approaches primarily rely on visual and joint position inputs, many real-world tasks require explicit force feedback to improve stability, adaptability, and safety~\cite{mason1981compliance, hogan1984impedance}. 
Recent work has applied learning methods to train with demonstrations that incorporate gripper force or tactile signals, resulting in policies capable of handling small, fragile objects and performing contact-intensive tasks such as vegetable peeling~\cite{liu2024forcemimic, xie2024just, li2024haptic}. However, utilizing force data from the robot arms, such as joint torques, remains underexplored. One approach involves using an end-effector force sensor to estimate compliance parameters or virtual position targets through kinesthetic teaching and force tensors~\cite{hou2024adaptive, chen2025dexforce}. Another method infers a 6D wrench for low-level control by integrating torque sensing into a diffusion policy~\cite{wu2024tacdiffusion}.

However, naively incorporating force feedback into policy learning can lead to overfitting to visual information, causing the policy to disregard force input. FoAR~\cite{he2024foar} explicitly predicts contact and non-contact phases to regulate the fusion of vision and force modalities, which requires additional data labeling. We propose FACTR to effectively incorporate force and vision input into policy through a curriculum, enabling policies to leverage force for improved object generalization.

\subsection{Low-Cost Teleoperation Systems with Force Feedback}

Parallel to advances in imitation learning, significant efforts have been made to collect low-cost and high-quality data with hand-held grippers~\cite{song2020grasping, umi} or leader-follower systems~\cite{aloha, wu2024gello, bidex}. Hand-held grippers naturally provide force feedback to the operator, but they do not directly record force data. Recent work has added force sensors to hand-held grippers to address this limitation~\cite{liu2024forcemimic}. However, hand-held grippers are in general limited by the kinematic differences between humans and robots, resulting in commands that might be unachievable for the robots.
Although the leader-follower systems are not prone to this limitation, they often lack force feedback, impairing their effectiveness in contact-rich tasks. Recently, Kobayashi et al.~\cite{kobayashi2024alpha} implemented a bilateral leader-follower teleoperation system where in addition to the follower following the joint positions of the leader, the leader also gets an additional torque if there is a difference in its joint position from that of the follower.
However, when the follower arm is in motion without contact, this system causes the operator to experience inertial, frictional, and other dynamic forces of the follower, reducing the ease of use and precision of the system~\cite{siciliano2008springer}. Our approach introduces an alternative bilateral teleoperation method by relaying only external joint torques from the follower arm back to the leader arm, providing force feedback without impairing operational precision.