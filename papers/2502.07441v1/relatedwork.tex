\section{Related Work}
This section reviews the related work of eye-tracking technologies, wristband sensors, cognitive state estimation with multimodal sensors, and personal space estimation.

\subsection{Eye-Tracking Technologies}
Research on eye-tracking has been carried out across various fields, including HCI and psychology. 
Studies have demonstrated that eye-tracking can be utilized to estimate factors such as confidence~\cite{bhatt2024self}, personality~\cite{berkovsky2019personality}, attention~\cite{vainio2019urban}, and cognitive load~\cite{zagermann2016cognitive}. 
Additionally, eye-tracking is employed in diverse interaction tasks like gaze-based typing~\cite{cai2023speakfaster,lystbaek2022exploring}, menu navigation~\cite{kopacsi2024exploring}, and object selection~\cite{cho2024sonohaptics}.
The choice of eye-tracking device varies depending on the data type and task. 
Examples include PC-mounted eye-tracking~\cite{dembinsky2024eye,dembinsky2024gaze,bhatt2024self}, eye-tracking glasses~\cite{bykowski2018automatic,meteier2023enhancing}, head-mounted displays~\cite{minakata2019pointing}, and webcam-based eye-tracking~\cite{ankur2024appearance,shah2024webcam,shah2024eyedentify}.

Each eye-tracking methodology has its distinct strengths and weaknesses~\cite{carter2020best,shi2024bibliometric}.
For example, PC-mounted eye-tracking systems are easy to set up and operate, making them user-friendly, but they lack portability and are restricted to a fixed location. 
Eye-tracking glasses provide portability and mobility but require the user to wear a specialized device. Head-mounted displays also offer portability but can be complex to set up. 
Webcam-based eye-tracking systems are easy to set up and use, yet they are less accurate than other methods.

Considering these strengths and weaknesses, selecting the appropriate eye-tracking device and task type is important when designing an eye-tracking system.
Our study focuses on estimating comfortable personal space, so we use eye-tracking glasses. 
This choice is due to their portability, allowing participants to move freely, and higher accuracy and robustness than camera-based eye-tracking.

\subsection{Wristband Sensors}
Wristband sensors are wearable devices that can measure physiological signals such as heart rate~\cite{tanaka2024concentration}, skin conductance~\cite{menghini2019stressing}, and galvanic skin response~\cite{blanchard2014automated}.
These sensors have been widely used in research and practical applications due to their ability to provide continuous, real-time data in a non-invasive manner. 
Studies have demonstrated that wristband sensors are used for various tasks, such as estimating stress~\cite{menghini2019stressing}, concentration~\cite{tanaka2024concentration}, workload~\cite{brishtel2018assessing}, and mind wandering~\cite{brishtel2020mind}.
Additionally, wristband sensors are used for interaction tasks, such as object interaction~\cite{lee2024echowrist}, text entry~\cite{funk2014using}, and step aware voice instructions~\cite{arakawa2024prism}.

This study uses a wristband sensor to measure physiological signals such as heart rate and skin conductance.
These features are significant for estimating cognitive state, which is a key factor for estimating personal space.

\subsection{Personal Space}
Personal space, also known as peripersonal space, refers to the comfortable distance individuals maintain from others, which varies subjectively among individuals~\cite{nandrino2017perception,geers2023relationship}.

Coello et al.~\cite{coello2021interrelation} explored the concepts of Interpersonal Space (IPS).
IPS is the area individuals maintain between themselves and others during social interactions. 
When this space is encroached upon, it often leads to discomfort, prompting individuals to increase the distance to regain comfort.
Our study emphasizes PPS to investigate whether physiological signals can predict personal space preferences.

Candini et al.~\cite{candini2021physiological} examined the direct link between physiological responses and the regulation of interpersonal space. 
Their study employed an ecological experimental setup where participants' skin conductance response (SCR) was measured as a confederate approached or withdrew from them at varying distances.
The results showed a significant increase in SCR when participants were exposed to close interpersonal distances, especially during approaching movements.
Additionally, the study found a functional relationship between individual SCR reactivity and preferred IPS, indicating that autonomic responses play a role in the perception and regulation of interpersonal boundaries.
This research highlights that autonomic arousal, often below conscious awareness, is vital in shaping social space preferences.

Our study uses physiological signals to estimate personal space, focusing on discomfort distance.