\section{Background and Related Works}
Technology-powered interventions have become increasingly popular for supporting mental health and well-being. 
They provide real-time data, adapt to personal needs, and reduce barriers such as cost and accessibility. 


\subsection{Wearables for Stress Tracking, Stressor Journaling, and Interventions}
Wearable devices have emerged as one of the most widely used AI-driven tools for recognizing affect states, including stress from physiological signals. 
During the past two decades, the scientific community has made significant advances in stress detection, using wearable technology to monitor physiological responses in real-time indicative of stress____. 
Building on this foundation, commercial devices such as Fitbit____, Garmin____, Whoop____, and Empatica Embrace ____ have recently integrated stress tracking capabilities. 
Passively collected data has been leveraged to create innovative visualizations for self-reflection____. 
Furthermore, passive stress detection has effectively delivered real-time interventions, addressing stress precisely when it occurs____. 
However, participants in these studies have expressed the desire for interventions to match the source of stress. 
Recent work has shown how stress detection can be used to collect stressors and how incorporating stressors into self-reflective visualizations can lead to behavioral changes____. 
This study takes the next step of integrating stressors with chatbot interactions for personalized stress interventions. 


 

\subsection{LLM for Mental Health Support}
Mental health support chatbots trace their origins back to ELIZA, one of the first computer programs to simulate human conversation ____. 
Since then, several rule-based and semi-automated chatbots have emerged that offer psycho-education, basic coping strategies, and limited therapeutic interactions. 
However, the emergence of LLMs has transformed the development of conversational agents and created novel opportunities to provide users with advanced mental health chatbots. 
Liu et al. ____ developed an LLM chatbot, ChatCounselor, by fine-tuning an open-source pre-trained LLM on a dataset prepared from 260 in-depth interviews between patients and psychologists. 
The LLM chatbot by Lai et al. ____ was developed by fine-tuning two Chinese pre-trained LLMs with real-world professional Q\&A datasets from psychologists and psychological articles to develop their mental health chatbot. 
Dongre et al. ____ also fine-tuned an open-source LLM on a Q\&A dataset scraped from a mental health website to develop their LLM chatbot for stress management.   
CaiTI, a Conversational AI Therapist, uses LLMs, smart devices, and reinforcement learning to provide personalized psychotherapeutic interventions, enhancing mental health self-care____. 
BOLT, a framework for evaluating LLM therapists’ conversational behavior, highlights the need for improvements to achieve high-quality care____. 
% MindGuard combines an edge LLM with sensor data and Ecological Momentary Assessment records to provide personalized, stigma-free mental health support____. 
MindScape introduces an AI-powered journaling approach that integrates behavioral patterns like sleep, location, and engagement to deliver personalized, context-aware prompts____.
% In contrast, wearable-triggered prompts enable journaling based on thoughts and emotions that may not be captured through passive smartphone data.
These works show the vast potential of LLM to power tailored interventions on demand. 
The novelty of this study lies in its integration with wearable AI to deliver automated, personalized, and context-aware stress interventions.



\begin{figure*}[t]
     \centering
     \begin{subfigure}[b]{0.41\textwidth}
         \centering
         \includegraphics[width=\textwidth]{figures/Cueshub_1.jpg}
% \caption{Change in prompt efficiency with increases in Passive Stress Likelihood (in 5 percentile increments)}
\caption{Valence rating screen of the prompted event}
     \label{fig:Cueshub_1}
     \end{subfigure}
     \hfill
     \begin{subfigure}[b]{0.4\textwidth}
         \centering
         \includegraphics[width=\textwidth]{figures/Cueshub_2.jpg}
\caption{Event description screen of the prompted event}

         \label{fig:Cueshub_2}
         \captionsetup{justification=centering}
         
     \end{subfigure}
      \caption{CuesHub app screenshots for valence and event descriptions}
        \label{fig:cueshub}
        \Description{Figure shows the screenshots of the CuesHub app to rating of the prompted events}
\end{figure*}