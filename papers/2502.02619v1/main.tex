%%%%%%%%%%%%%%%%%%%%%%%%%%%%%%%%%%%%%%%%%%%%%%%%%%%%%%%%%%%%%%%%%%%%%%%%

%%% LaTeX Template for AAMAS-2025 (based on sample-sigconf.tex)
%%% Prepared by the AAMAS-2025 Program Chairs based on the version from AAMAS-2025. 

%%%%%%%%%%%%%%%%%%%%%%%%%%%%%%%%%%%%%%%%%%%%%%%%%%%%%%%%%%%%%%%%%%%%%%%%

%%% Start your document with the \documentclass command.


%%% == IMPORTANT ==
%%% Use the first variant below for the final paper (including auithor information).
%%% Use the second variant below to anonymize your submission (no authoir information shown).
%%% For further information on anonymity and double-blind reviewing, 
%%% please consult the call for paper information
%%% https://aamas2025.org/index.php/conference/calls/submission-instructions-main-technical-track/

%%%% For anonymized submission, use this
\documentclass[sigconf]{aamas} 

%%%% For camera-ready, use this
%\documentclass[sigconf]{aamas} 


%%% Load required packages here (note that many are included already).

\usepackage{balance} % for balancing columns on the final page
% \usepackage{biblatex}
\usepackage{graphicx}
\usepackage{xcolor}
\usepackage{listings}
% \addbibresource{report.bib}
\usepackage{amsmath} % for bmatrix environment
\usepackage{graphicx}
\usepackage{geometry}
\usepackage{graphicx, caption, subcaption, booktabs, multirow, float}

\usepackage{amsmath}
% \usepackage{algorithm}
% \usepackage{algorithmic}
\usepackage{algorithm}
\usepackage{algpseudocode}
% \usepackage[linesnumbered,ruled,vlined]{algorithm2e}
% \usepackage[font=small,labelfont=bf]{caption}

\usepackage{lineno}
\usepackage{hyperref}

\DeclareMathOperator*{\argmax}{arg\,max}
%%%%%%%%%%%%%%%%%%%%%%%%%%%%%%%%%%%%%%%%%%%%%%%%%%%%%%%%%%%%%%%%%%%%%%%%

%%% AAMAS-2025 copyright block (do not change!)

% \setcopyright{ifaamas}
% \acmConference[AAMAS '25]{Proc.\@ of the 24th International Conference
% on Autonomous Agents and Multiagent Systems (AAMAS 2025)}{May 19 -- 23, 2025}
% {Detroit, Michigan, USA}{A.~El~Fallah~Seghrouchni, Y.~Vorobeychik, S.~Das, A.~Nowe (eds.)}
% \copyrightyear{2025}
% \acmYear{2025}
% \acmDOI{}
% \acmPrice{}
% \acmISBN{}


%%%%%%%%%%%%%%%%%%%%%%%%%%%%%%%%%%%%%%%%%%%%%%%%%%%%%%%%%%%%%%%%%%%%%%%%

%%% == IMPORTANT ==
%%% Use this command to specify your submission number.
%%% In anonymous mode, it will be printed on the first page.

% \acmSubmissionID{226}

%%% Use this command to specify the title of your paper.

% \title[AAMAS-2025 Formatting Instructions]{Enhancing Traditional 60/40 Portfolio Through PPO With Future Looking Regret-Based Reward and Synthetic Data}

\title[AAMAS-2025 Formatting Instructions]{Regret-Optimized Portfolio Enhancement through \\Deep Reinforcement Learning and Future Looking Rewards}


%%% Provide names, affiliations, and email addresses for all authors.


\author{Daniil Karzanov}
% \orcid{0000-0003-1767-9649}


\affiliation{%
  \institution{AXA Group Operations, EPFL}
  \city{Lausanne}
  \country{Switzerland}}
  \email{daniil.karzanov@axa.com}


\author{Rubén Garzón}
% \orcid{0000-0002-5695-2495}

\affiliation{%
  \institution{AXA Group Operations}
  \city{Madrid}
  \country{Spain}}
\email{ruben.garzon@axa.com}


\author{Mikhail Terekhov}
% \orcid{0009-0005-7403-3731}
% \authornotemark[1]


\affiliation{%
  \institution{CLAIRE EPFL}
  \city{Lausanne}
  \country{Switzerland}
}
\email{mikhail.terekhov@epfl.ch}

\author{Caglar Gulcehre}
% \orcid{0009-0003-4124-1687}
% \authornote{Both authors contributed equally to this research.}

\affiliation{%
  \institution{CLAIRE EPFL}
  \city{Lausanne}
  \country{Switzerland}
}
\email{caglar.gulcehre@epfl.ch}

\author{Thomas Raffinot}
% \authornote{Both authors contributed equally to this research.}
% \orcid{0000-0003-2338-3596}

\affiliation{%
  \institution{ AXA Investment Managers}
  \city{Paris}
  \country{France}
}
\email{thomas.raffinot@axa-im.com}


\author{Marcin Detyniecki}
% \orcid{0000-0001-5669-4871}
% \authornotemark[1]

\affiliation{%
  \institution{ AXA Group Operations}
  \city{Paris}
  \country{France}
}
\email{marcin.detyniecki@axa.com}

% \author{Arthur Pendragon}
% \affiliation{
%   \institution{Camelot Castle}
%   \city{Camelot}
%   \country{United Kingdom}}
% \email{king.arthur@camelot.uk}

% \author{Nimue}
% \affiliation{
%   \institution{The Lady's Lake}
%   \city{Avalon}
%   \country{United Kingdom}}
% \email{lady.of.the.lake@avalon.uk}

%%% Use this environment to specify a short abstract for your paper.

\begin{abstract} {


This paper introduces a novel agent-based approach for enhancing existing portfolio strategies using Proximal Policy Optimization (PPO). Rather than focusing solely on traditional portfolio construction, our approach aims to improve an already high-performing strategy through dynamic rebalancing driven by PPO and Oracle agents. Our target is to enhance the traditional 60/40 benchmark (60\% stocks, 40\% bonds) by employing the Regret-based Sharpe reward function. To address the impact of transaction fee frictions and prevent signal loss, we develop a transaction cost scheduler. We introduce a future-looking reward function and employ synthetic data training through a circular block bootstrap method to facilitate the learning of generalizable allocation strategies. We focus on two key evaluation measures: return and maximum drawdown. Given the high stochasticity of financial markets, we train 20 independent agents each period and evaluate their average performance against the benchmark. Our method not only enhances the performance of the existing portfolio strategy through strategic rebalancing but also demonstrates strong results compared to other baselines.




% Strategic asset allocation involves setting specific target allocations for various asset classes and periodically rebalancing the portfolio to maintain these targets. Deep reinforcement learning is increasingly becoming a valuable tool for systematic portfolio managers.
% This paper introduces a practical agent-based approach for dynamic portfolio construction using well-established Proximal Policy Optimization (PPO). The portfolio construction problem involves selecting an optimal combination of financial assets that maximizes returns while minimizing risk, based on factors like market dynamics, asset correlations, and investor preferences. Given the high stochasticity of financial markets, we train 20 independent agents each period and evaluate their average performance against the benchmark. Our aim is to enhance the traditional 60/40 benchmark (60\% stocks, 40\% bonds) by employing the Regret-based Sharpe reward function. To address the impact of transaction fee frictions and prevent signal loss, we develop a transaction cost scheduler. We introduce a future-looking reward function and employ synthetic data training through a circular block bootstrap method to facilitate the learning of generalizable allocation strategies. We focus on two key evaluation measures: return and maximum drawdown. Our agent demonstrates prominent results, achieving both objectives and generally outperforming the baseline.
% This work presents a novel application of DRL not just for portfolio construction, but for enhancing an already high-performing target portfolio through slight rebalancing, leveraging agent-based frameworks to achieve further improvements.

% \todo{This paper introduces a novel agent-based approach for enhancing existing portfolio strategies using Proximal Policy Optimization (PPO). Rather than focusing solely on traditional portfolio construction, our approach aims to improve an already high-performing strategy through dynamic rebalancing. Given the high stochasticity of financial markets, we train 20 independent agents each period and evaluate their average performance against the benchmark. Our target is to enhance the traditional 60/40 benchmark (60\% stocks, 40\% bonds) by employing the Regret-based Sharpe reward function. To address the impact of transaction fee frictions and prevent signal loss, we develop a transaction cost scheduler. We introduce a future-looking reward function and employ synthetic data training through a circular block bootstrap method to facilitate the learning of generalizable allocation strategies. We focus on two key evaluation measures: return and maximum drawdown. By leveraging DRL, our agent not only enhances the performance of the existing portfolio strategy through strategic rebalancing but also demonstrates superior results compared to the baseline.}


}
\end{abstract}


%%% The code below was generated by the tool at http://dl.acm.org/ccs.cfm.
%%% Please replace this example with code appropriate for your own paper.

\newcommand{\com}[1]{\textcolor{orange}{\MakeTextUppercase{#1}}}

\newcommand{\todo}[1]{\textcolor{red}{\MakeTextUppercase{#1}}}
%%% Use this command to specify a few keywords describing your work.
%%% Keywords should be separated by commas.

\keywords{Deep Reinforcement Learning, Proximal Policy Optimization (PPO), Dynamic Portfolio Construction, Regret-Based Reward Function, Synthetic Data Training, Circular Block Bootstrap, Transaction Cost Scheduling, Machine Learning in Finance, Risk Management, Return Maximization, Multi-Objective Optimization, Computational Finance}

%%%%%%%%%%%%%%%%%%%%%%%%%%%%%%%%%%%%%%%%%%%%%%%%%%%%%%%%%%%%%%%%%%%%%%%%

%%% Include any author-defined commands here.
         
\newcommand{\BibTeX}{\rm B\kern-.05em{\sc i\kern-.025em b}\kern-.08em\TeX}

%%%%%%%%%%%%%%%%%%%%%%%%%%%%%%%%%%%%%%%%%%%%%%%%%%%%%%%%%%%%%%%%%%%%%%%%




\settopmatter{printacmref=false}
% \settopmatter{printacmref=false, printccs=false, printfolios=false, printcopyrightpermission=false}
\renewcommand\footnotetextcopyrightpermission[1]{}
\setcopyright{none}
% \pagenumbering{arabic}
% \pagestyle{plain}

\begin{document}

%%% The following commands remove the headers in your paper. For final 
%%% papers, these will be inserted during the pagination process.

\pagestyle{fancy}
\fancyhead{}

%%% The next command prints the information defined in the preamble.

\maketitle 

%%%%%%%%%%%%%%%%%%%%%%%%%%%%%%%%%%%%%%%%%%%%%%%%%%%%%%%%%%%%%%%%%%%%%%%%


\section{Introduction}
% \linenumbers


% \begin{figure*}[ht]
%     \centering
%     \begin{subfigure}[b]{0.32\textwidth}
%         \centering
%         \includegraphics[height=5.5cm]{figures/novelt_train_c.png}
%         \caption{Train}
%     \end{subfigure}
%     \hfill
%     \begin{subfigure}[b]{0.32\textwidth}
%         \centering
%         \includegraphics[height=5.5cm]{figures/novelt_valid_c.png}
%         \caption{Validation}
%     \end{subfigure}
%     \hfill
%     \begin{subfigure}[b]{0.32\textwidth}
%         \centering
%         \includegraphics[height=5.5cm]{figures/novelt_test_c.png}
%         \caption{Test}
%     \end{subfigure}
%     \caption{The evolution of accumulated return over training. Ablation: removal of specific model components to assess their impact. TC: inclusion of transaction cost schedule. BB: block bootstrap synthetic data. Regret: our reward function. Return: optimizing current portfolio return as a reward. Averaged over 20 runs each. The shaded area represents uncertainty around the average trend.}
%     \label{fig:novelt}
% \end{figure*}







Strategic asset allocation is a portfolio strategy whereby the investor sets target allocations to various asset classes and rebalances the portfolio periodically. The landscape of finance is continually evolving, driven by the need for more sophisticated and adaptive investment strategies that use machine learning for decision making \cite{de2020machine, karzanov2023headline, Blohm2020ItsAP, hambly2023recent}. Traditional methods of portfolio construction often struggle to keep pace with the dynamic nature of financial markets. This manuscript addresses this challenge by exploring the integration of advanced machine learning techniques to enhance the process of dynamic portfolio construction and rebalancing. By leveraging deep reinforcement learning algorithms (DRL) such as Proximal Policy Optimization (PPO) \cite{schulman2017proximal}  with a future-looking reward function, we aim to demonstrate how RL can optimize financial portfolios, balance risk, and maximize returns in an ever-changing market environment.


The primary objective is to achieve superior returns, while also ensuring that the portfolio's value does not experience significant declines. This secondary objective is measured by the maximum drawdown (MDD). Essentially, our goal is to develop an agent that finds the optimal balance between two inversely related metrics: portfolio return and risk. In this context, we consider MDD to be a more suitable risk measure than portfolio standard deviation, which is commonly used in traditional Markowitz-like approaches \cite{mpt}. MDD is a measure of the maximum observed loss from a peak to a trough in a portfolio's value \cite{chekhlov2005drawdown}, before the portfolio reaches a new peak. It provides an indication of the worst possible loss an investor could have experienced during a specific period. Additionally, we consider the Sharpe Ratio, defined as the ratio of the portfolio's excess return to its standard deviation, to evaluate the risk-adjusted return of the portfolio, allowing for a comprehensive assessment of performance relative to the risk taken. Refer to the \hyperref[sec:glossary]{glossary} in the appendix for details on the Sharpe Ratio and other financial notation and metrics used.

The baseline for comparison is the static allocation of weights based on the conventional 60/40 portfolio (60\% stocks, 40\% bonds). Our overarching aim is to surpass the traditional 60/40 portfolio benchmark in both metrics using a dynamic and adaptive DRL approach. 


The novel contributions of this work can be summarized in the following three points: 
\begin{enumerate}
    \item[1.] Introduction of a negative Sharpe regret reward function that leverages Oracle agent's knowledge to encourage optimal allocation and improve out-of-sample performance.
    \item[2.] Integration of real and synthetic data in the training process through a circular block bootstrap method to enhance historical data, thereby accelerating the learning of effective strategies and improving the model’s capacity to extrapolate and generalize beyond observed data trajectories.
    \item[3.] Incorporation of a transaction costs scheduler during training, allowing for the inclusion of frictions that are often overlooked in other studies.
\end{enumerate}


In this paper, we first discuss the relevant literature related to the application of DRL in finance, highlighting key advancements and methodologies. This includes a review of two other benchmark methods used for comparison in our evaluation. We then describe the reinforcement learning framework and the specifics of Proximal Policy Optimization (PPO). In Section 3.2, we delve into the design of our environment, transforming real price data to make it suitable for PPO's learning. This includes the introduction of our novel reward function that incorporates a transaction cost (TC) scheduler, the use of synthetic data generation during training, and other implementation details for the agent. Finally, we provide a discussion of the results, comparing our approach with the benchmark method and two other approaches from the literature, and evaluating different configurations of the agent. We conclude with several noteworthy ideas for future improvements and extensions of this work.

The methodology described in this manuscript can be applied to many other sequential allocation problems, not just financial portfolio construction. Examples include resource allocation in supply chains, task assignment in project management, and bandwidth distribution in telecommunications networks. Each of these scenarios involves distributing limited resources or capacities across various options to achieve optimal outcomes.


\section{Relevant Literature}

Advances in RL and deep RL have greatly impacted the field of AI and ML. Mnih et al. \cite{mnih2015human} achieved human-level control in complex games by leveraging deep Q-networks. Schulman et al. \cite{schulman2017proximal} introduced Proximal Policy Optimization algorithms, addressing stability issues in policy learning. Silver et al. \cite{silver2016mastering} demonstrated beyond human performance in the game of Go through a combination of deep neural networks and tree search.

Numerous studies have explored the application of deep reinforcement learning to both single-asset trading \cite{kochliaridis2023combining, pigorsch2022high, brini2023deep} and portfolio optimization problems \cite{srivastava2020deep, halperin_combining_2022, lu2023evaluation, jiang2017deep}. For example, Benhamou et al. \cite{benhamou2020bridging} employed a policy gradient method to dynamically allocate several systematic strategies. Similarly, Kochliaridis et al. \cite{kochliaridis2023combining} utilized a future-looking reward function that considers future close prices over the next K timesteps to encourage agents to predict future market dynamics in addition to solving the primary optimization problem. Our work incorporates a similar concept by embedding Oracle knowledge (defined as $w^*$ in equation (\ref{eq:regret_weight}) and detailed in Section \ref{sec:reward_function}) into the reward function during the training process. In their Investor-Imitator framework, Ding et al. \cite{ding2018investor} consider an Oracle investor, employing Sharpe and MDD as key evaluation measures.

Another recent study by Sood et al. \cite{sood2023deep} applies a PPO model in a setting similar to ours, utilizing a Differential-Sharpe reward function (\ref{eq:diff_sharpe}) and exponential moving averages, $A_t$, $B_t$ of returns and their standard deviation. The study achieves impressive results, with the agent outperforming traditional mean-variance optimization techniques in both return and drawdown metrics, and also demonstrating a more stable out-of-sample strategy. We adopt the reward function introduced in their paper as one of the benchmarks for comparison in our study.


\begin{equation}
D_t = \frac{\delta S_t}{\delta \eta} = \frac{B_{t-1}\Delta A_t - \frac{1}{2}A_{t-1}\Delta B_t}{(B_{t-1} - A_{t-1}^2)^{3/2}}
\label{eq:diff_sharpe}
\end{equation}
where
\begin{itemize}
    \item $D_t$: Differential Sharpe Ratio at time $t$.
    \item $\delta S_t$: Change in the Sharpe ratio, which is adjusted over time as new information is incorporated.
    \item $\eta$: Incremental time step for daily returns, approximately $\frac{1}{252}$.
    \item $A_t$: Cumulative mean return updated over time, calculated as $A_t = A_{t-1} + \eta \Delta A_t$.
    \item $B_t$: Cumulative second moment (variance) of returns, updated as $B_t = B_{t-1} + \eta \Delta B_t$.
    \item $\Delta A_t$: Change in cumulative mean return, defined as $\Delta A_t = R_t - A_{t-1}$.
    \item $\Delta B_t$: Change in cumulative second moment, defined as $\Delta B_t = R_t^2 - B_{t-1}$.
    \item $R_t$: Return at time $t$.
    \item $A_0$ and $B_0$: Initial values for cumulative mean return and cumulative second moment, both set to 0.
\end{itemize}


Andersson et al. \cite{andersson2023measuring} explore the application of regret theory in financial decision-making, particularly focusing on how regret aversion can influence investor behavior. 

Several studies explore the use of synthetic data in portfolio construction. Pe{\~n}a et el. \cite{pena2024modified} introduce a novel portfolio optimization method using synthetic data generated by a Modified CTGAN algorithm. Similarly, Pagnoncelli et el. \cite{pagnoncelli2023synthetic} demonstrate the advantages of using augmented synthetic data for asset allocation.

Few studies consider the multi-objective case in this niche, which inadequately addresses the problem where investors aim to minimize risk \cite{vcernevivciene2022review}. Bisht et al. \cite{bisht2020deep} and Cornalba et al. \cite{cornalba2024multi} attempt to tackle this issue with multi-objective approaches. Almahdi et al.\cite{almahdi2017adaptive} combine the analysis of both expected maximum drawdown and transaction costs. Similarly, Wu et al. \cite{wu2022embedded} incorporate drawdown into their reward function (\ref{eq:emb_rew}), which prioritizes MDD. If the drawdown level at time $t$, $MDD_t$, exceeds $\alpha$, the desired MDD level, the term in brackets becomes negative, serving as a penalty mechanism.  If the return tends to infinity, the first multiplier tends to the hyperparameter value $k$. Although their study focuses more on trading than on portfolio construction, it offers a practical approach by explicitly including drawdown in the reward function. Consequently, we use their methodology as one of the baselines for our study, using a dynamic $\alpha$ equal to the MDD of the 60/40 benchmark.

\begin{equation}
\text{\textbf{Reward}}_t = \frac{k}{1 + e^{-r}} (-e^{MDD_t} + e^{\alpha})
\label{eq:emb_rew}
\end{equation}

Except for a few studies \cite{lucarelli2020deep, lucarelli2019deep, jiang2017deep}, most papers neglect transaction costs and other similar frictions in their analysis. In contrast, we introduce a transaction cost (TC) term and devise an elegant method to integrate it into the learning process without attenuating the contribution of the main reward during exploration (e.g. instantaneous return or Sharpe without TC term).
% "killing" the signal.




\begin{figure*}[ht]
    \centering

    \includegraphics[width=\textwidth]{figures/abbl.pdf}
    \caption{The evolution of accumulated (financial) return over training. Ablation study: removal of specific model components to assess their impact. TC: inclusion of transaction cost schedule. BB: block bootstrap synthetic data. Regret: our reward function. Return: purely optimizing for returns, without including MDD or risk in the reward. Each configuration is averaged over 20 runs. \\
    The full configuration (TC + BB + Regret) generalizes better despite underperformance during training due to increased variability. Synthetic data functions as a form of regularization, preventing memorization of non-reproducible strategies. When used alone, BB and TC are less effective than when combined. However, the combination of TC, BB, and the Return reward function (which specifically optimizes the value displayed on the Y-axis of the plot) tends to overfit on train, resulting in an inability to generate profitable out-of-sample strategies.}
    \label{fig:novelt}
    \Description{The evolution of accumulated (financial) return over training. Ablation study: removal of specific model components to assess their impact. TC: inclusion of transaction cost schedule. BB: block bootstrap synthetic data. Regret: our reward function. Return: purely optimizing for returns, without including MDD or risk in the reward. Each configuration is averaged over 20 runs. \\
    The full configuration (TC + BB + Regret) generalizes better despite underperformance during training due to increased variability. Synthetic data functions as a form of regularization, preventing memorization of non-reproducible strategies. When used alone, BB and TC are less effective than when combined. However, the combination of TC, BB, and the Return reward function (which specifically optimizes the value displayed on the Y-axis of the plot) tends to overfit on train, resulting in an inability to generate profitable out-of-sample strategies.}
\end{figure*}




\section{Background}
\subsection{Portfolio Optimization}
Portfolio optimization plays a crucial role in financial management, focusing on the allocation of assets to maximize returns while effectively managing risk. Mathematical portfolio optimization seeks to allocate assets in a way that maximizes expected return while minimizing risk. Risk can be represented by various measures, such as portfolio standard deviation or MDD, both of which are typically inversely related to expected return. The classic formulation is based on mean-variance single-period optimization, introduced by Markowitz \cite{mpt}. The goal is to solve the trade-off between portfolio risk $w' \Sigma w$ and return $w' \mu$:

\begin{equation}
\min_{w} \frac{1}{2} w' \Sigma w - \lambda w' \mu
\end{equation}

where \(w\) represents a vector of asset weights, denoting the proportion of the portfolio invested in each asset, \(\Sigma\) is the covariance matrix of asset returns, \(\mu\) is the vector of expected returns, and \(\lambda\) is a risk-aversion parameter. Constraints such as \(\sum w_i = 1\) (full investment) and \(w_i \geq 0\) (no short-selling) are typically included.
Classical portfolio optimization approaches rely solely on historical data and do not incorporate forecasting, unlike neural networks. We aim to apply reinforcement learning to address the multi-period portfolio optimization problem, which involves making investment decisions over several time periods (e.g. daily) to adapt to and predict changing market conditions. This approach allows for dynamically rebalancing of the portfolio based on current asset information ( \(\mu\) and \(\Sigma\)) and relevant exogenous predictors (discussed in section \ref{ch:obs_space}).


\subsection{Reinforcement Learning}

Reinforcement learning is a subfield of machine learning where an agent learns to make decisions by interacting with an environment to maximize cumulative rewards \cite{sutton2018reinforcement}. The agent operates by taking actions $w_t$\footnote{In traditional reinforcement learning terminology, $w_t$ is referred to as $a_t$.} at each time step $t$, receiving a state $s_t$ from the environment, and obtaining a reward $r_t$. The goal of the agent is to learn a policy $\pi$ that maximizes the expected cumulative reward, defined as the return $G_t$, over time.


The return $G_t$ is the total discounted reward from time step $t$ onwards and is given by:
\begin{equation}
G_t = \sum_{k=0}^{\infty} \gamma^k r_{t+k+1},
\end{equation}
where $\gamma$ (0 $\leq \gamma < 1$) is the discount factor, which determines the present value of future rewards. The expected return, or the value function $V(s)$, under a policy $\pi$ is defined as:
\begin{equation}
V^\pi(s) = \mathbb{E}_\pi [G_t | s_t = s].
\end{equation}
The objective in reinforcement learning is to find an optimal policy $\pi^*$ that maximizes the value function $V(s)$ for all states $s$.

In the context of dynamic portfolio construction, the reinforcement learning agent's task is to dynamically adjust the portfolio weights based on the observed market states $s_t$ to maximize the cumulative reward $G_t$. The exact form of the reward function $r_t$ may vary and its design may result in different agent behaviors.

\subsection{Proximal Policy Optimization }
Proximal Policy Optimization (PPO) is one of the most effective and widely used algorithms in reinforcement learning. It was introduced as a method to improve both the stability and performance of policy gradient methods. PPO achieves this by using a surrogate objective function, which helps in balancing the trade-off between exploration and exploitation. The core idea behind PPO is to ensure that the new policy does not diverge too much from the old policy during training. This is accomplished through a mechanism known as clipping. Specifically, PPO optimizes the following objective:
\begin{equation}
    \mathcal{L}^{CLIP}(\theta) = \mathbb{E}_t \left[ \min \left( r_t(\theta) \hat{A}_t, \text{clip}(r_t(\theta), 1 - \epsilon, 1 + \epsilon) \hat{A}_t \right) \right].
\end{equation}
In this objective, $r_t(\theta) = \frac{\pi_\theta(w_t | s_t)}{\pi_{\theta_{\text{old}}}(w_t | s_t)}$ represents the probability ratio between the new policy $\pi_\theta$ and the old policy $\pi_{\theta_{\text{old}}}$. The term $\hat{A}_t$ is the advantage estimate, which indicates how much better the current action is compared to the expected action. The hyperparameter $\epsilon$ controls the clipping range, thereby ensuring that the updates to the policy are conservative and stable. The clipping mechanism is crucial as it prevents excessively large updates that can destabilize the learning process. By limiting the change to a specified range, PPO maintains a balance between optimizing the policy and keeping the updates within a reasonable bound.

PPO has been shown to be robust and effective in a variety of complex environments, making it a popular choice for tasks that require dynamic decision-making and adaptation, such as dynamic portfolio construction. Its ability to maintain stability while still performing well in challenging scenarios is a key reason for its widespread adoption in the reinforcement learning community.






% \begin{table*}[ht]
% \caption{Performance comparison of the 60/40 benchmark (B) and the PPO agent (A).}
% \centering

% %% ---------------


% \begin{tabular}{lllllllllll}
% \toprule
%  &  & \multicolumn{3}{c}{Train} & \multicolumn{3}{c}{Validation} & \multicolumn{3}{c}{Test} \\
%  &  & Phase 1 & Phase 2 & Phase 3 & Phase 1 & Phase 2 & Phase 3 & Phase 1 & Phase 2 & Phase 3 \\
%   &  & \scriptsize{(pre-pandemic)} & \scriptsize{(pandemic)} & \scriptsize{(post-pandemic)} & \scriptsize{(pre-pandemic)} & \scriptsize{(pandemic)} & \scriptsize{(post-pandemic)} & \scriptsize{(pre-pandemic)} & \scriptsize{(pandemic)} & \scriptsize{(post-pandemic) } \\
% \midrule
% \multirow[t]{2}{*}{Annual return} & A & \textbf{0.071} & \textbf{0.064} & \textbf{0.091} & \textbf{0.128} & 0.07 & \textbf{0.093} & \textbf{0.064} & \textbf{0.128} & \textbf{-0.007} \\
%  & B & 0.05 & 0.05 & 0.08 & 0.10 & \textbf{0.076} & 0.08 & 0.06 & 0.10 & -0.03 \\
% \cline{1-11}
% \multirow[t]{2}{*}{Calmar ratio} & A & \textbf{0.192} & \textbf{0.185} & 0.43 & \textbf{1.602} & 0.45 & \textbf{0.364} & \textbf{0.607} & 0.48 & \textbf{-0.035} \\
%  & B & 0.14 & 0.14 & \textbf{0.475} & 1.40 & \textbf{0.625} & 0.35 & 0.46 & \textbf{0.486} & -0.11 \\
% \cline{1-11}
% \multirow[t]{2}{*}{Max drawdown} & A & -0.39 & \textbf{-0.361} & -0.21 & -0.08 & -0.15 & -0.26 & \textbf{-0.108} & -0.27 & \textbf{-0.205} \\
%  & B & \textbf{-0.384} & -0.38 & \textbf{-0.171} & \textbf{-0.069} & \textbf{-0.121} & \textbf{-0.216} & -0.12 & \textbf{-0.216} & -0.23 \\
% \cline{1-11}
% \multirow[t]{2}{*}{Omega ratio} & A & \textbf{1.198} & \textbf{1.173} & 1.26 & \textbf{1.488} & 1.24 & 1.24 & \textbf{1.246} & 1.28 & \textbf{0.999} \\
%  & B & 1.16 & 1.16 & \textbf{1.266} & 1.42 & \textbf{1.331} & \textbf{1.24} & 1.22 & \textbf{1.286} & 0.96 \\
% \cline{1-11}
% \multirow[t]{2}{*}{Sharpe ratio} & A & \textbf{0.614} & \textbf{0.593} & 0.88 & \textbf{1.558} & 0.84 & 0.76 & \textbf{0.857} & 0.84 & \textbf{-0.005} \\
%  & B & 0.54 & 0.57 & \textbf{0.938} & 1.45 & \textbf{1.181} & \textbf{0.776} & 0.80 & \textbf{0.868} & -0.18 \\
% \cline{1-11}
% \multirow[t]{2}{*}{Sortino ratio} & A & \textbf{0.904} & \textbf{0.843} & 1.29 & \textbf{2.485} & 1.21 & 1.01 & \textbf{1.178} & 1.12 & \textbf{-0.006} \\
%  & B & 0.77 & 0.78 & \textbf{1.357} & 2.19 & \textbf{1.748} & \textbf{1.04} & 1.10 & \textbf{1.164} & -0.24 \\
% \cline{1-11}
% \multirow[t]{2}{*}{Stability} & A & \textbf{0.842} & \textbf{0.885} & \textbf{0.945} & \textbf{0.96} & 0.84 & 0.75 & \textbf{0.899} & 0.83 & \textbf{0.057} \\
%  & B & 0.82 & 0.85 & 0.93 & 0.96 & \textbf{0.857} & \textbf{0.805} & 0.87 & \textbf{0.839} & 0.00 \\
% \cline{1-11}
% \multirow[t]{2}{*}{Tail ratio} & A & \textbf{1.081} & \textbf{1.082} & \textbf{1.161} & \textbf{1.314} & 1.08 & 1.06 & \textbf{1.168} & 1.19 & \textbf{0.99} \\
%  & B & 1.06 & 1.08 & 1.10 & 1.16 & \textbf{1.186} & \textbf{1.076} & 1.07 & \textbf{1.315} & 0.92 \\
% \cline{1-11}
% \bottomrule
% \end{tabular}






% %% ----------------


% %% x-x-x-x-x-x-x-x-x-x-x


% \label{tab:results1}
% \end{table*}





\begin{table*}[ht]
\caption{Performance comparison of the 60/40 benchmark and the PPO agent with differential Sharpe  eq. (\ref{eq:diff_sharpe}), embedded drawdown eq. (\ref{eq:emb_rew}) and negative Sharpe regret (ours) reward functions. 
% \todo{MOVE THIS AND ABLATIONS FIG 1 LOWER?}
}
\centering
\footnotesize







\begin{tabular}{lllllllllll}
\toprule
 &  & \multicolumn{3}{c}{train} & \multicolumn{3}{c}{valid} & \multicolumn{3}{c}{test} \\
  &  & Phase 1 & Phase 2 & Phase 3 & Phase 1 & Phase 2 & Phase 3 & Phase 1 & Phase 2 & Phase 3 \\
  &  & \scriptsize{(pre-pandemic)} & \scriptsize{(pandemic)} & \scriptsize{(post-pandemic)} & \scriptsize{(pre-pandemic)} & \scriptsize{(pandemic)} & \scriptsize{(post-pandemic)} & \scriptsize{(pre-pandemic)} & \scriptsize{(pandemic)} & \scriptsize{(post-pandemic) } \\
\midrule

\multirow[t]{4}{*}{Annual return} & 60/40 & 0.052 & 0.054 & 0.081 & 0.096 & 0.076 & 0.077 & 0.056 & 0.105 & -0.026 \\
 & Diff. Sharpe & \textbf{0.054} & 0.054 & 0.076 & 0.083 & 0.066 & 0.077 & 0.053 & 0.093 & \textbf{-0.024} \\
 & Emb. DD & \textbf{0.056} & 0.051 & 0.057 & 0.042 & 0.062 & 0.055 & 0.041 & 0.072 & -0.038 \\
 & Regret & \textbf{0.071} & \textbf{0.064} & \textbf{0.091} & \textbf{0.128} & 0.069 & \textbf{0.093} & \textbf{0.064} & \textbf{0.128} & \textbf{-0.007} \\
\cline{1-11}

\multirow[t]{4}{*}{Sharpe ratio} & 60/40 & 0.541 & 0.568 & 0.938 & 1.447 & 1.181 & 0.776 & 0.803 & 0.868 & -0.183 \\
 & Diff. Sharpe & \textbf{0.603} & \textbf{0.658} & \textbf{0.97} & 1.405 & \textbf{1.21} & \textbf{0.833} & \textbf{0.869} & \textbf{0.876} & \textbf{-0.179} \\
 & Emb. DD & \textbf{0.849} & \textbf{0.742} & 0.936 & 1.005 & \textbf{1.362} & \textbf{0.847} & \textbf{0.923} & 0.832 & -0.368 \\
 & Regret & \textbf{0.614} & \textbf{0.593} & 0.882 & \textbf{1.558} & 0.841 & 0.756 & \textbf{0.857} & 0.844 & \textbf{-0.005} \\
\cline{1-11}

\multirow[t]{4}{*}{Calmar ratio} & 60/40 & 0.136 & 0.142 & 0.475 & 1.399 & 0.625 & 0.355 & 0.461 & 0.486 & -0.114 \\
 & Diff. Sharpe &  \textbf{0.159} &  \textbf{0.180} &0.442 &  \textbf{1.453} &  \textbf{0.631} &  \textbf{0.413} &  \textbf{0.512} &  \textbf{0.526} &  \textbf{-0.109} \\
 & Emb. DD &  \textbf{0.257} &  \textbf{0.205} & 0.44 & 0.983 &  \textbf{0.756} &  \textbf{0.454} &  \textbf{0.560} &  \textbf{0.535} & -0.17 \\
 & Regret &  \textbf{0.192} &  \textbf{0.185} & 0.429 &  \textbf{1.602} & 0.451 &  \textbf{0.364} &  \textbf{0.607} & 0.483 &  \textbf{-0.035} \\
\cline{1-11}

% \multirow[t]{4}{*}{Stability} & 60/40 & 0.822 & 0.852 & 0.932 & 0.960 & 0.857 & 0.805 & 0.866 & 0.839 & 0.002 \\
%  & Diff. Sharpe & \textbf{0.851} & \textbf{0.902} & \textbf{0.947} & \textbf{0.961} & \textbf{0.864} & \textbf{0.845} & \textbf{0.889} & \textbf{0.848} & \textbf{0.017} \\
%  & Emb. DD & \textbf{0.94} & \textbf{0.915} & 0.926 & 0.905 & 0.854 & \textbf{0.884} & \textbf{0.877} & 0.814 & \textbf{0.083} \\
%  & Regret & \textbf{0.842} & \textbf{0.885} & \textbf{0.945} & 0.960 & 0.836 & 0.750 & \textbf{0.899} & 0.833 & \textbf{0.057} \\
% \cline{1-11}

\multirow[t]{4}{*}{Max drawdown} & 60/40 & -0.384 & -0.382 & -0.171 & -0.069 & -0.121 & -0.216 & -0.122 & -0.216 & -0.225 \\
 & Diff. Sharpe & \textbf{-0.338} & \textbf{-0.303} & -0.173 & \textbf{-0.057} & \textbf{-0.105} & \textbf{-0.187} & \textbf{-0.104} & \textbf{-0.178} & \textbf{-0.215} \\
 & Emb. DD & \textbf{-0.225} & \textbf{-0.256} & \textbf{-0.129} & \textbf{-0.043} & \textbf{-0.083} & \textbf{-0.122} & \textbf{-0.073} & \textbf{-0.134} & \textbf{-0.224} \\
 & Regret & -0.387 & \textbf{-0.361} & -0.211 & -0.080 & -0.152 & -0.255 & \textbf{-0.108} & -0.266 & \textbf{-0.205} \\
\cline{1-11}

\multirow[t]{4}{*}{Omega ratio} & 60/40 & 1.156 & 1.158 & 1.266 & 1.418 & 1.331 & 1.240 & 1.220 & 1.286 & 0.957 \\
 & Diff. Sharpe & \textbf{1.174} & \textbf{1.183} & \textbf{1.279} & 1.409 & \textbf{1.334} & \textbf{1.264} & \textbf{1.237} & \textbf{1.288} & \textbf{0.958} \\
 & Emb. DD & \textbf{1.237} & \textbf{1.204} & 1.259 & 1.281 & \textbf{1.374} & \textbf{1.265} & \textbf{1.245} & 1.271 & 0.916 \\
 & Regret & \textbf{1.198} & \textbf{1.173} & 1.260 & \textbf{1.488} & 1.238 & 1.235 & \textbf{1.246} & 1.277 & \textbf{0.999} \\
\cline{1-11}

% \multirow[t]{4}{*}{Sortino ratio} & 60/40 & 0.771 & 0.784 & 1.357 & 2.190 & 1.748 & 1.040 & 1.096 & 1.164 & -0.242 \\
%  & Diff. Sharpe & \textbf{0.872} & \textbf{0.927} & \textbf{1.424} & 2.128 & \textbf{1.816} & \textbf{1.134} & \textbf{1.205} & \textbf{1.179} & \textbf{-0.237} \\
%  & Emb. DD & \textbf{1.269} & \textbf{1.056} & \textbf{1.377} & 1.458 & \textbf{2.089} & \textbf{1.152} & \textbf{1.349} & 1.119 & -0.490 \\
%  & Regret & \textbf{0.904} & \textbf{0.843} & 1.293 & \textbf{2.485} & 1.207 & 1.010 & \textbf{1.178} & 1.118 & \textbf{-0.006} \\
% \cline{1-11}
% \multirow[t]{4}{*}{Tail ratio} & 60/40 & 1.059 & 1.077 & 1.097 & 1.163 & 1.186 & 1.076 & 1.070 & 1.315 & 0.915 \\
%  & Diff. Sharpe & \textbf{1.08} & \textbf{1.114} & \textbf{1.129} & \textbf{1.182} & \textbf{1.201} & \textbf{1.092} & \textbf{1.136} & 1.273 & 0.910 \\
%  & Emb. DD & \textbf{1.157} & \textbf{1.113} & 1.090 & \textbf{1.205} & 1.146 & \textbf{1.136} & \textbf{1.121} & 1.235 & 0.890 \\
%  & Regret & \textbf{1.081} & \textbf{1.082} & \textbf{1.161} & \textbf{1.314} & 1.082 & 1.060 & \textbf{1.168} & 1.191 & \textbf{0.99} \\
% \cline{1-11}

\bottomrule
\end{tabular}






\label{tab:results1}
\end{table*}



% \begin{figure}[ht]
%   \centering
%   \includegraphics[width=\linewidth]{figures/Weights_cheater_test.png}
%   \caption{Oracle Allocation.}
%   \label{fig:Worcale}
%   % \Description{Description for accessibility}
% \end{figure}

% \begin{figure}[ht]
%   \centering
%   \includegraphics[width=\linewidth]{figures/Allocation_test.png}
%   \caption{Example of a PPO allocation.}
%   \label{fig:Alloc}
%   % \Description{Description for accessibility}
% \end{figure}








\section{Methodology}
% \subsection{Environment Design}
One of the most crucial aspects of reinforcement learning applications is the definition of the environment. Its design and hyperparameters significantly influence the agent's performance.

\subsection{Action Space}
We consider prices of $K=3$ trading strategies:  Only \textit{Developed Markets Equity} when the agent is more bullish\footnote{expecting a rise in prices, and thus choosing a more volatile asset}, the \textit{60/40 Portfolio} (i.e. the portfolio that we are trying to improve), and Only \textit{Global Government Bonds} (Govies) as a low-risk asset to potentially avoid sharp drops in portfolio value.
% \textit{Developed Markets Equity Index}, \textit{Emerging Markets Equity Index}, \textit{Global Credit}, and \textit{Global Government Bonds}.
Shorting \footnote{ taking negative weight $w^i$ by borrowing asset $i$}  is not allowed, hence our action space $ \mathcal{A}$ is represented by a non-negative continuous vector $w \in \{w \in \mathbb{R}_{+}^K : \sum_{i=1}^{K} w^i = 1\}$. 
While we consider a standard problem of allocation between risky, balanced and conservative assets, the approach can be scaled to more high-dimensional allocation.










% \todo{}
% Most allocations in the initial asset universe are not entirely reasonable. During the early stages of training, an RL agent experiments with many random actions, making it extremely difficult to find a  "sweet spot" in the multi-dimensional action space. To facilitate the learning process, we restrict our agent to allocation between three strategies: Only Developed Markets Equity when the agent is more bullish, the 60/40 portfolio (i.e. the portfolio that we are trying to improve), and Only Global Government Bonds (Govies) as a low-risk asset to potentially avoid sharp drops in portfolio value.
\subsection{Observation Space}
\label{ch:obs_space}
 In addition to asset information, we include three classical contextual indexes that, while not part of the portfolio, may provide signals for asset allocation: \textit{High-Yield Bond Spread} \cite{gilchrist2012credit}, \textit{VIX volatility index} \cite{whaley2000investor}, and \textit{Merrill Lynch Option Volatility Estimate} \cite{driessen2009price}.
 
% We transform the price data for both asset and index tickers into daily strategy returns, $\mu_t \in \mathbb{R}^{3}$ and $\alpha_t \in \mathbb{R}^{3}$ respectively, and the return standard deviations of the last 60 days, $\bar \sigma^{t-60}_{t} \in \mathbb{R}^{3}_+$ and $ {\bar q^{t-60}}_{t} \in \mathbb{R}^{3}_+$. Additionally, we calculate the rolling average returns over the past 40 days, $\bar \mu^{t-40}_t\in \mathbb{R}^{3}$, to smooth out the noise present in daily returns. The agent also considers the previous allocation $w_{t-1} \in \{w \in \mathbb{R}_{+}^3 : \sum_{i=1}^{3} w_i = 1\}$ and the current transaction costs term, $TC_{train}(t) \in \mathbb{R}_+$ to determine whether it is reasonable to change the current allocation or if it is approximately optimal. The agent iterates through the bi-daily transformed historical dataset, rebalancing the portfolio based on the current observation and looping back to the beginning upon reaching the end, using the row data as observations. Consequently, at each timestep, the agent observes a vector  $o_t = [\mu_t, \ \alpha_t, \ \bar \mu^{t-40}_t, \ \bar \sigma^{t-60}_{t},  \  {\bar q^{t-60}}_{t}, \ w_{t-1}, \ TC_{train}(t) ] \in \mathbb{R}^{19}$. 

We transform the price data for both asset and index tickers into daily strategy returns, $\mu_t \in \mathbb{R}^{3}$ and $\alpha_t \in \mathbb{R}^{3}$ respectively, as well as the return standard deviations over the last 60 days for assets $\bar \sigma^{t-60}_{t} \in \mathbb{R}^{3}_+$ and for indexes ${\bar q^{t-60}}_{t} \in \mathbb{R}^{3}_+$. Additionally, we calculate the rolling average asset returns over the past 40 days, $\bar \mu^{t-40}_t \in \mathbb{R}^{3}$, to smooth out noise in the daily returns. The agent also considers the previous allocation $w_{t-1} \in \{w \in \mathbb{R}_{+}^3 : \sum_{i=1}^{3} w^i = 1\}$ and the current transaction costs, $TC_{train}(t) \in \mathbb{R}_+$, to decide whether adjusting the current allocation is warranted or if it remains approximately optimal. The agent iterates through the transformed historical dataset, rebalancing the portfolio based on current observations, and loops back to the beginning upon reaching the end, treating the data as ongoing observations. At each timestep, the agent observes a vector $o_t = [\mu_t, \ \alpha_t, \ \bar \mu^{t-40}_t, \ \bar \sigma^{t-60}_{t}, \ {\bar q^{t-60}}_{t}, \ w_{t-1}, \ TC_{train}(t)] \in \mathbb{R}^{19}$.


Daily portfolio rebalancing can be excessively sensitive to short-term noise and costly due to fees. To address this, we employed data aggregation; however, this approach reduces the number of training points by the frequency of trading days we choose to use. We settled on a bi-daily (2 bd) interval because aggregating data over two-day intervals smooths the data while avoiding an excessive reduction in training points.










\subsection{TC Schedule}
In our first experiments, we observed that including transaction costs (TC) at the start of training negatively impacted the learning process and diminished the signal from the main reward term (e.g., return or Sharpe ratio), particularly during the model's exploration stage. Fine-tuning the model with full transaction costs from the start proved challenging, as it often led to either constant allocations or poor generalization (see Regret and BB + Regret in the validation/test plots, Fig. \ref{fig:novelt}). Inspired by the principles of Curriculum Learning \cite{bengio2009curriculum, koenecke2020curriculum}, where the complexity of tasks gradually increases from simple to more real-world scenarios, we introduce a transaction cost scheduler that incrementally raises the training transaction cost at each step according to 
\begin{equation}
TC_{\text{train}}(x) = 
\frac{TC_{\text{max}}}{S^{a}} \cdot x^{a} \quad \text{if } 0 \leq x \leq S.
\label{eq:tc_formula}
\end{equation}
The costs are increased until a ramp limit given by $S = 100 \cdot \text{episode\_length}$ is reached. After the ramp limit, the maximal costs, $TC_{\text{max}} = TC_{\text{eval}} = 0.0025$ (a fair value for traditional brokers and large institutional traders), are applied.
% \begin{equation}
% TC_{\text{eval}}(x) = TC_{\text{max}}
% \label{eq:tc_formula2}
% \end{equation}





\subsection{Reward Function}
\label{sec:reward_function}
A key distinction between our problem and many other RL applications is the ability to know the reward of any action, not just the one taken by the agent. Real market participants often reflect on the optimal allocation they could have chosen yesterday based on today's returns. Building on this concept, we propose a negative Sharpe-based regret reward function (\ref{eq:regret}) conditioned on the previous timestep's action. 

\begin{equation}
\text{\textbf{Reward}}_t = -\bar \mu_t^{t+n}(w^* - w_t)'
\label{eq:regret}
\end{equation}
where
\begin{equation}
\begin{aligned}
w^* &= \argmax_{w} \ \mathbf{Sharpe}(w, \bar \mu_t^{t+n}, \bar \Sigma_{t-3n}^{t+3n}) - TC_{\text{train}}(t) \cdot \|w - w_{t-1}\|_1 \\
 &= \argmax_{w}  \frac{w' \bar \mu_t^{t+n} - R_f}{\sqrt{w' \bar \Sigma_{t-3n}^{t+3n} w}} - TC_{\text{train}}(t) \cdot \|w - w_{t-1}\|_1
\end{aligned}
\label{eq:regret_weight}
\end{equation}

The reward is calculated as the difference between the average returns of the optimal allocation and the agent's allocation over the next $n$ days, assuming both allocations are fixed from today. We use the Sharpe ratio because it is one of the most popular and efficient measures that balance return and risk. To incentivize the agent to optimize the portfolio for future relevance, we use a forward-looking return vector $\bar{\mu}_t^{t+n}$ and a covariance matrix $\bar{\Sigma}_{t-3n}^{t+3n}$ in the Sharpe ratio calculation, rather than the typical current or simple return. While we consider only the forward-looking average return over the next $n=14$ business days, we use a broader interval $(t - 3n, \ t + 3n)$ for a more precise estimate of the covariance matrix. Additionally, to account for transaction costs, we include a regularization term in the optimal action's expression (\ref{eq:regret_weight}),  ensuring that the next Oracle-optimal allocation $w^*$ can be achieved despite the fees required to adjust the previous allocation, with the transaction cost term proportional to the difference in weights between two consecutive allocations to penalize large adjustments.




Since we do not have access to future return information during testing, we set the reward to zero in the environment when deploying the trained model. While future data can be integrated during training, we avoid using this data at inference time to prevent leakage between training, validation, and testing. We ensure no overlap among these datasets, which means that toward the end of training, we have fewer points to look into the future. Thus, we use all available training points without incorporating data that cannot technically be used. Consequently, we evaluate the pipeline using various financial ratios instead of total episodic reward.



\subsection{Training on Synthetic Data}
A significant conceptual challenge in applying reinforcement learning to this problem is the limited size of available data. Unlike other RL environments in fields such as robotics and games, which offer variability and diverse paths during training as observations are influenced by the agent's actions, our environment lacks price impact and remains unaffected by allocations. As a result, the agent iterates through the same data repeatedly, encountering nearly identical trajectories in each episode. This setup significantly increases the risk of overfitting to the training dynamics rather than learning robust and generalizable trading strategies.

To address this issue, we propose training on synthetic data that closely mimics the underlying real data distribution. Initially, we considered using the Gaussian copula \cite{rey2015copula}; however, this method was quickly dismissed as it fails to accurately represent the distribution in the tails, which are critical during crisis events that offer opportunities for abnormal returns. Instead, we opted for a circular block bootstrap \cite{arch} of the underlying training data, applied every 10 episodes. We experimented with various block sizes and found that larger blocks,  70-90\% of the original training set, led to improved agent performance as they better preserve long-range temporal dependencies of the training set.

The training process involves an iterative approach to improve the agent's performance. Initially, the agent is trained on real data for 10 episodes. Subsequently, the training shifts to synthetic data for another 10 episodes. After this cycle, the synthetic data is regenerated to introduce new variability, and the agent undergoes another 10 episodes of training on the new synthetic data. This process of regenerating synthetic data and training continues until the required number of training episodes is reached. Upon completion, the agent transitions to the next phase, starting the cycle anew with training on real data.


\begin{figure}[ht]
  \centering
  \begin{subfigure}{\linewidth}
    \centering
  \includegraphics[width=\linewidth]{figures/Weights_cheater_test.png}
  \caption{Oracle: $w^*$ defined as (\ref{eq:regret_weight})}
  \label{fig:Worcale}
  \end{subfigure}
  \begin{subfigure}{\linewidth}
    \centering
 \includegraphics[width=\linewidth]{figures/Allocation_test.png}
  \caption{PPO with negative Sharpe Regret reward function}
  \label{fig:Alloc}
  \end{subfigure}

  \caption{Example of allocation during the testing period of phase 3. The regret-based agent demonstrates a general alignment with the optimal allocation but adopts a less aggressive stance. During bullish market periods, it increases its positions in risky assets, capitalizing on favorable conditions. Conversely, when market uncertainty arises, the agent shifts to more conservative allocations, showcasing its adaptability to changing market dynamics.}
  \Description{Example of allocation during the testing period of phase 3. The regret-based agent demonstrates a general alignment with the optimal allocation but adopts a less aggressive stance. During bullish market periods, it increases its positions in risky assets, capitalizing on favorable conditions. Conversely, when market uncertainty arises, the agent shifts to more conservative allocations, showcasing its adaptability to changing market dynamics.}
  \label{fig:two-allocs}
  
\end{figure}



\subsection{Implementation Details} 
The complete training loop is detailed in Algorithm \ref{alg1}. We employ fully connected feedforward neural networks for both the policy and critic architectures. These networks are designed with multiple hidden layers to effectively extract features from the state space, capturing the intricate patterns and dynamics of the environment. The hidden layers use the Tanh activation function, which maps the input values to a range between -1 and 1, aiding in normalization and maintaining stable gradient flow during backpropagation. This design ensures that the networks can learn robust feature representations, contributing to the overall stability and performance of the PPO algorithm in our experiments.


We employ a sliding window approach with a train-validation-test split in our pipeline, splitting the available data into phases as detailed in Table \ref{tab:timing-phases}. This method mitigates the bias towards older, potentially less relevant observations. The phased split enables a robust evaluation of all approaches by testing the agent across three distinct and representative market scenarios. Phase 1 represents a period of asset growth under favorable market conditions. Phase 2 corresponds to the COVID-19 crisis, characterized by sharp declines in asset values. Finally, Phase 3 reflects a market downturn in which all assets experienced price drops, emphasizing the agent's ability to minimize losses in adverse conditions. We transfer the weights of the value and policy networks after Phases 1 and 2. The best model, evaluated on the validation set based on the return-risk trade-off, is used as a pretrain for the subsequent phase. This approach prevents the need to learn the model of the world from scratch, thereby reducing the number of training episodes. We opted for the sliding window approach over the expanding window because the latter approach often leads to overfitting to the early dynamics by exposing the agent to the earliest observations more frequently. Consequently, we exclude data from the 90s in the later phases but retain the knowledge by transferring the policy and value networks' weights to the subsequent phases. Additionally, to promote exploration when transitioning to another phase, we implement an entropy regularization schedule. This schedule sets a non-zero entropy coefficient, $\beta_{\text{entropy}}$, in the PPO total loss function (\ref{eq:ppo_loss}) and then linearly decreases it to zero until 10\% of the number of training episodes. We use a Critic loss (\ref{eq:critic_loss}) and a bonus based on the $ \mathcal{L}_{\text{entropy}} = - H(\pi(w|s))$ of the policy distribution. 

\begin{equation}
\mathcal{L} = \mathcal{L}_{\text{policy}} + \beta_{\text{entropy}} \cdot \mathcal{L}_{\text{entropy}} + \beta_{\text{value}} \cdot \mathcal{L}_{\text{value}}
\label{eq:ppo_loss}
\end{equation}


\begin{equation}
\mathcal{L^{\text{value}}} = \mathbb{E}_{t} \left[ \left( V_{\theta}(s_t) - V^{\text{target}}_t \right)^2 \right]
\label{eq:critic_loss}
\end{equation}


Normalizing the advantages in PPO enhances the results on the validation set. We adjust the transaction cost schedule to be more concave in Phases 2 and 3, as the agent is not learning from scratch. Furthermore, we discovered that when selecting the model from the Pareto front for the next phase, it is preferable to choose a less risk-averse model, as more conservative models tend to become stuck in suboptimal allocations and encounter difficulties in training during the new phase.


\begin{algorithm}
\caption{Agent Training Pseudo-Code}
\label{alg1}
\begin{algorithmic}[1]
\State Given original historical dataset $\mathcal{D}_{\text{orig}}$

\State Initialize network parameters: $\theta$ for policy network, $\phi$ for value network
\State Initialize maximum training iterations $M$ and training batch $D$

\State Set $\tilde{\mathcal{D}} = \mathcal{D}_{\text{orig}}$
\For{training iteration $= 1$ to $M$}
    \State Clear training batch $D$
    \For{each RL collect step $t$}
        % 
        \If{$t \ \%  \ \text{episode\_length} \times 10 == 0$}
            \If{$Bernoulli(0.7) == 1$}
                \State $\tilde{\mathcal{D}} = BlockBootstrap(\mathcal{D}_{\text{orig}})$
            \Else
                \State $\tilde{\mathcal{D}} = \mathcal{D}_{\text{orig}}$
            \EndIf 
        \EndIf
        \State Observe environment state $o_t$ from $\tilde{\mathcal{D}}$
        \State Select action $a_t$ according to policy $\pi_\theta(a_t \mid o_t)$
        \State Compute $w_t = softmax(a_t)$
        \State Compute Oracle allocation $w^*(\tilde{\mathcal{D}}) = $ 
        \Statex \hspace{6em}  $ = \argmax_{w} \ \mathbf{Sharpe}(w, \bar \mu_t^{t+n}(\tilde{\mathcal{D}}), \bar \Sigma_{t-3n}^{t+3n}(\tilde{\mathcal{D}})) -$
        \Statex \hspace{8em} $ - TC_{\text{train}}(t) \cdot \|w - w_{t-1}\|_1 $
      
       
        \State Execute action $a_t$, transition to $o_{t+1}$ from $\tilde{\mathcal{D}}$
        \State Calculate Reward $r_t = -\bar \mu_t^{t+n}(\tilde{\mathcal{D}}) \cdot (w^* - w_t)'$
    \EndFor
    \State Compute advantage estimate $\hat{A}$ using GAE
    \State Add experience $(o_t, a_t, r_t, o_{t+1})$ to batch $D$
    
    \For{each RL training step}
        \State $\beta_{\text{entropy}}$ = $EntropySchedule(\text{step})$
        \State Recompute advantage estimate $\hat{A}$ using GAE
        \State Split batch $D$ into $K$ mini-batches $\mathcal{B}$
        \For{mini-batch $k = 1$ to $K$}
            \State Compute PPO total loss 
            \Statex \hspace{5em} $\mathcal{L} =\mathcal{L}_{\text{policy}} + \beta_{\text{entropy}} \cdot \mathcal{L}_{\text{entropy}} + \beta_{\text{value}} \cdot \mathcal{L}_{\text{value}}$
            \State Update critic and actor networks w.r.t $\mathcal{L}$
        \EndFor
    \EndFor
\EndFor
\State \Return $V_\phi, \pi_\theta$
\end{algorithmic}
\end{algorithm}

\section{Results and EMPIRICAL EVALUATION}
\subsection{Discussion}
We analyze the performance of the agent in Table \ref{tab:results1} using a variety of performance measures:

\begin{itemize}
    \item \textbf{Sortino Ratio}: A variation of the Sharpe ratio, it focuses on the standard deviation of negative asset returns to assess downside risk, offering a more targeted evaluation for investors.
    \item \textbf{Calmar Ratio}: This ratio measures downside risk by comparing the maximum drawdown to the average annual rate of return, providing insights into the risk-adjusted performance of an investment.
    % \item \textbf{Maximum Drawdown (MDD)}: \todo{REMOVE? Explained already so many times} Lower MDD can enhance investor confidence, potentially leading to increased commitment to the investment strategy and a greater likelihood of adhering to long-term financial goals.
    \item \textbf{Omega Ratio}: Compares the probability of returns above a threshold to those below, considering the full return distribution.

    % \item \textbf{Stability Ratio}: This measures the consistency of a portfolio's returns over a specified period, defined as the ratio of the average return to the standard deviation of returns.
    % \item \textbf{Tail Ratio}: This assesses the risk of extreme losses by comparing the average return of the worst-performing periods to the average return of the best-performing periods.
\end{itemize}


Our PPO agent outperforms the 60/40 portfolio in terms of return across all periods, except for the validation set in the pre-pandemic phase. The Calmar ratio is significantly higher for regret PPO in most cases, with the exception of Phase 2. The Sharpe, Omega, and Sortino ratios are closely aligned. Notably, the validation performance for all measures in Phase 2 was suboptimal and shows room for improvement. The embedded drawdown approach has the lowest return among all approaches but typically achieves the best MDD, which aligns with the classic return-risk trade-off. Both the embedded drawdown and differential Sharpe reward functions are more conservative, consistently performing well in risk-accounting measures such as Sharpe, Sortino, and MDD. 
% This consistency explains why these approaches often beat the return benchmark.
However, during the testing phase in the post-pandemic period, their performance was not as promising as the regret-based approach (-2.6\% for 60/40, -2.4\% for differential Sharpe, and -0.7\% for regret).


Overall, as illustrated in Figure \ref{fig:phases-plot}, our Sharpe regret-based approach with TC scheduling appears to be a preferable option, as it consistently outperforms the return benchmark and surpasses the MDD 60/40 benchmark in two out of three instances. Phase 1 is particularly challenging, with both measures showing flatter distributions. The transfer of weights in the later stages likely reduces variability in learning for the agent. Constant 60/40 only outperforms our approach in MDD during the pandemic breakout, which could be partly because the features are not optimal for detecting such signals. The MDD distribution in Phase 1 is more skewed compared to all other runs. As anticipated, the pandemic period (Phase 2) proved to be the most challenging for our agent, as its return distribution, while higher than the benchmark, remains relatively flat. Future work could focus on fine-tuning the agent to better identify and respond to crises through regime detection methods, as suggested in \cite{benhamou2021detecting, halperin_combining_2022}.

% \todo{Add Pareto Fronts fig??? NO I think its too much already. plus what can we infer from them extra to histograms?}

The embedded drawdown and differential Sharpe reward functions generate reasonable allocations but struggle to surpass the benchmark in the main measure: annual return. One possible reason is that these approaches do not explicitly account for transaction costs within the learning process. Agents using these reward functions tend to be more risk-averse, avoiding significant increases in positions in the riskiest assets. However, it is noteworthy that the other two approaches performed better in optimizing maximum drawdown compared to our regret-based approach. Interestingly, the embedded drawdown approach underperformed significantly during the post-pandemic period, barely beating the 60/40 benchmark. In the most recent testing period, our approach proved to be notably better than the other methods in both objectives.


% \begin{figure}[ht]
%   \centering
%   \begin{subfigure}{\linewidth}
%     \centering
%     \includegraphics[width=\linewidth]{figures/test_Phase1.png}
%     % \caption{Caption for image}
%     \label{fig:phase1}
%   \end{subfigure}
%   \begin{subfigure}{\linewidth}
%     \centering
%     \includegraphics[width=\linewidth]{figures/test_Phase2.png}
%     % \caption{Caption for image}
%     \label{fig:phase2}
%   \end{subfigure}

%   \begin{subfigure}{\linewidth}
%     \centering
%     \includegraphics[width=\linewidth]{figures/test_Phase3.png}
%     % \caption{Caption for image}
%     \label{fig:phase3}
%   \end{subfigure}
  
%   \caption{Distribution of returns and MDDs for the agent (in blue) compared to the benchmark (black dashed line) for the test set.}
%   \Description{Distribution of returns and MDDs for the agent (in blue) compared to the benchmark (black dashed line) for the test set.}
%   \label{fig:phases-plot}
% \end{figure}


\begin{figure}[ht]
  \centering
  \begin{subfigure}{\linewidth}
    \centering
    \includegraphics[width=\linewidth]{figures/test_Phase1.pdf}
    % \caption{Caption for image}
    \label{fig:phase1}
  \end{subfigure}
  \begin{subfigure}{\linewidth}
    \centering
    \includegraphics[width=\linewidth]{figures/test_Phase2.pdf}
    % \caption{Caption for image}
    \label{fig:phase2}
  \end{subfigure}

  \begin{subfigure}{\linewidth}
    \centering
    \includegraphics[width=\linewidth]{figures/test_Phase3.pdf}
    % \caption{Caption for image}
    \label{fig:phase3}
  \end{subfigure}
  
  \caption{Distribution of returns and MDDs for the PPO agent for differential Sharpe, embedded drawdown and regret (ours) reward functions compared to the 60/40 benchmark (black dashed line) on the test periods. Our approach (in blue) focuses more on returns, allowing it to outperform all other methods in the primary objective across all three phases. Our model has proven superior to all considered approaches in the latest trading phase in both objectives.}
  \Description{Distribution of returns and MDDs for the PPO agent for differential Sharpe, embedded drawdown and regret (ours) reward functions compared to the 60/40 benchmark (black dashed line) on the test periods. Our approach (in blue) focuses more on returns, allowing it to outperform all other methods in the primary objective across all three phases. Our model has proven superior to all considered approaches in the latest trading phase in both objectives.}
  \label{fig:phases-plot}
\end{figure}

Figures \ref{fig:Worcale} and \ref{fig:Alloc} demonstrate that regret-based agent generally aligns, though not as extremely, with the optimal allocation as shown by the allocations during the periods of April 2022, August 2022, and April 2023. Overall, the agent employing the Sharpe regret reward function tends to increase its position in risky assets during bullish markets and opts for a more conservative allocation when anticipating uncertain market conditions.


We also compare the effect of our modification on the training dynamics and the ability of the agent to generalize to the unseen data in Figure \ref{fig:novelt}. While the full configuration (TC + BB + Regret) is outperformed in the training environment, it shows a better and faster ability to generalize to the unseen data in the validation and test environments due to greater variability in the training set. Periodically introducing synthetic data, similar to many forms of regularization, increases the challenge for the agent to learn, as evidenced by the more fluctuating line in the training environment, but it helps the agent avoid memorizing non-reproducible effective allocation strategies. Interestingly, separately only BB and only TC configurations do not perform as well on the test set as their synergy. We can also observe that the current portfolio return (TC + BB + Return)  reward function is overfitting to the training data and is not managing to generate a profitable strategy out-of-sample.





% \section{Don't\small{s}}
\subsection{What Didn’t Work}

We also provide a brief discussion of the ideas that we tried and that did not work for our pipeline. Although these ideas failed for us, we hope that they might be helpful in other similar applications. DDPG and A2C algorithms failed to produce dynamic allocations and outputted nearly identical weights throughout the validation environment, regardless of the inputs in the observations. 

We also explored Convolutional Neural Networks (CNNs) for feature extraction, building on Benhamou et al. \cite{benhamou2021detecting}, who used multi-body policy networks with multiple tensors. We define the observation space with $k$ lags $L = [1, 2, 3, 6, 10, 15, 30, 60]$, yielding several tensors. The first network body utilizes asset information (\ref{eq:A_t}) as channels, while the second body processes market context (\ref{eq:C_t}) with 2-D convolution and a 3x3 filter. The approach underperformed compared to MLP architecture likely due to the transaction costs incurred from noisy allocation behaviors (Fig. \ref{fig:alloc_cnn}).


\begin{equation}
\footnotesize
A_t := \left[
\begin{array}{l}
\begin{bmatrix}
\mu_t \\
\mu_{t-l_1} \\
\vdots \\
\mu_{t-l_k}
\end{bmatrix} \in \mathbb{R}^{(k+1) \times 3 } , 
\begin{bmatrix}
\bar \mu_t \\
\bar \mu_{t-l_1} \\
\vdots \\
\bar \mu_{t-l_k}
\end{bmatrix} \in \mathbb{R}^{(k+1) \times 3 } ,
\begin{bmatrix}
\bar \sigma_t \\
\bar \sigma_{t-l_1} \\
\vdots \\
\bar \sigma_{t-l_k}
\end{bmatrix} \in \mathbb{R}^{(k+1) \times 3 }



\end{array}
\right] 
\label{eq:A_t}
\end{equation}

% \renewcommand{\arraystretch}{1} % Reset row height to default




% \renewcommand{\arraystretch}{1.45} % Adjust row height (1.45 is an example, adjust as needed)

\begin{equation}
\footnotesize
C_t := \left[
\begin{array}{l}
\begin{bmatrix}
\alpha_t \\
\alpha_{t-l_1} \\
\vdots \\
\alpha_{t-l_k}
\end{bmatrix} \in \mathbb{R}^{(k+1) \times 3 } , 

\begin{bmatrix}
\bar q_t \\
\bar q_{t-l_1} \\
\vdots \\
\bar q_{t-l_k}
\end{bmatrix} \in \mathbb{R}^{(k+1) \times 3 }

\end{array}
\right] 
\label{eq:C_t}
\end{equation}

% \renewcommand{\arraystretch}{1} % Reset row height to default


Alternatively, we also tried using differences in Sharpe ratios between optimal and taken allocations in the regret reward function. The agent with this reward function failed to learn a meaningful policy for unseen data, possibly due to the non-linear difference between optimal and taken actions. Additionally, we spent considerable time attempting to pre-train the agent with imitation learning using the Oracle as in \cite{kochliaridis2023combining} as an expert agent. All of the GAIL \cite{ho2016generative}, AIRL \cite{fu2018learningrobustrewardsadversarial}, and density-based reward modeling \cite{davis2011remarks} methods appeared to be highly sensitive to hyperparameters and did not result in improvement compared to training from scratch. Nevertheless, we see these approaches as promising for this application, and it may be our next direction for analysis.






% \begin{figure}[ht]
%   \centering
%   \includegraphics[width=\linewidth]{figures/Allocation_test.png}
%   \caption{Caption for Phase 3 image}
%   \label{fig:phase3}
%   % \Description{Description for accessibility}
% \end{figure}


\section{Future Work}
Our dynamic PPO regret-based agent consistently outperformed the return benchmark. The agent effectively adjusted allocations, increasing positions in bullish markets and adopting conservative allocations during uncertain periods. Our pipeline, which incorporates transaction cost scheduling, circular block bootstrap, and the future-looking regret-based reward function, demonstrates the ability to generalize to unseen test data. Discretizing the action space could enhance the model's scalability by reformulating the decision-making process. In such settings, ideas from \cite{elmachtoub2022smart} and \cite{mandi2020smart} may be explored for Oracle optimization.




Future work could involve further adapting the agent to operate in uncertain markets and respond more effectively during crisis events. Additional reward functions and their hyperparameters (such as Oracle's forecasting horizon) could be explored. The analysis of scalability may be conducted and the effect of larger portfolios and high-dimensional policies can be investigated.

We conducted a minimal, exploratory search for the following hyperparameters: the number of training episodes, early stopping time, the ramp-up duration for when the transaction cost (TC) scheduler levels off, and the convexity of the transaction cost. A more comprehensive exploration of these hyperparameters may be considered in future work.

Although we evaluated performance using MDD, it was not explicitly included in the reward, leading to less consistent performance in this measure. Expanding the observation space to a matrix by incorporating lagged variables could be beneficial; CNNs would be an appropriate choice for this setup. Additionally, dynamic strategies such as mean-reversion and momentum strategies could be integrated into the allocation process. More comprehensive hyperparameter tuning (on both the agent and environment sides) could be implemented. Given the sample inefficiency of RL approaches and computational constraints, we opted not to conduct an exhaustive hyperparameter search to avoid overfitting each validation set. Lastly, using narrower periods for the validation set could necessitate more frequent model recalibrations.

% \com{Papers submitted to the main track must be at most 8 pages long, with any number of additional pages containing bibliographic references. }


% \todo{Fix Spacing Appendix}


%%%%%%%%%%%%%%%%%%%%%%%%%%%%%%%%%%%%%%%%%%%%%%%%%%%%%%%%%%%%%%%%%%%%%%%%

%%% The acknowledgments section is defined using the "acks" environment
%%% (rather than an unnumbered section). The use of this environment 
%%% ensures the proper identification of the section in the article 
%%% metadata as well as the consistent spelling of the heading.

% \begin{acks}
% If you wish to include any acknowledgments in your paper (e.g., to 
% people or funding agencies), please do so using the `\texttt{acks}' 
% environment. Note that the text of your acknowledgments will be omitted
% if you compile your document with the `\texttt{anonymous}' option.
% \end{acks}

%%%%%%%%%%%%%%%%%%%%%%%%%%%%%%%%%%%%%%%%%%%%%%%%%%%%%%%%%%%%%%%%%%%%%%%%

%%% The next two lines define, first, the bibliography style to be 
%%% applied, and, second, the bibliography file to be used.




\bibliographystyle{ACM-Reference-Format} 
% \bibliography{main.bib}
\documentclass{MITstyle}

%\usepackage[table]{xcolor}
\usepackage{chngcntr}
\usepackage{hyperref}
\usepackage{microtype}

\title{A Lightweight and Extensible Cell Segmentation and Classification Model for Whole Slide Images}

\author{Nikita Shvetsov~$^{1, }$\footnote{Correspondence e-mail: nikita.shvetsov@uit.no}, Thomas K. Kilvaer~$^{2, 3}$, Masoud Tafavvoghi~$^{4}$, Anders Sildnes~$^{1}$, \\ Kajsa Møllersen~$^{4}$, Lill-Tove Rasmussen Busund~$^{5, 6}$, Lars Ailo Bongo~$^{1}$ \\
%
\vspace{1em} % Space between authors and afilliations
%
\normalfont{\small $^{1}$Department of Computer Science, UiT The Arctic University of Norway}\\
\normalfont{\small $^{2}$Department of Oncology, University Hospital of North Norway}\\
\normalfont{\small $^{3}$Department of Clinical Medicine, UiT The Arctic University of Norway}\\
\normalfont{\small $^{4}$Department of Community Medicine, UiT The Arctic University of Norway}\\
\normalfont{\small $^{5}$Department of Medical Biology, UiT The Arctic University of Norway} \\
\normalfont{\small $^{6}$Department of Clinical Pathology, University Hospital of North Norway} %\vspace{2em}
}

\begin{document}
\maketitle

\section*{Abstract}

% \begin{abstract}
% Developing clinically useful cell-level analysis tools in digital pathology remains challenging due to limitations in dataset granularity, inconsistent annotations, computational demands of advanced models, and difficulties in integrating new technologies into clinical workflows. To address these challenges, we propose a multi-faceted solution that enhances data quality, model performance, and usability to create a lightweight and extensible cell segmentation and classification model.

% First, we update data labels by employing a cross-relabeling process that refines the labels of two existing datasets, PanNuke and MoNuSAC, to create a new unified dataset with enhanced granularity, encompassing seven distinct cell types. Second, we leverage the H-Optimus foundation model as a fixed encoder to improve feature representation for simultaneous cell segmentation and classification tasks. Third, to address the computational demands of foundation models, we employ knowledge distillation to reduce model size and complexity while maintaining comparable performance. Finally, to facilitate integration into clinical workflows, we integrate the distilled model into the QuPath software, a widely used open-source platform in digital pathology.

% Our results demonstrate improvements in cell segmentation and classification performance using the H‑Optimus-based model compared to a CNN-based model. Specifically, the average $R^2$ improved from 0.575 to 0.871, and the average $PQ$ score improved from 0.450 to 0.492, indicating better alignment with actual cell counts and enhanced segmentation and classification quality. Furthermore, the distilled student model maintains performance comparable to the larger foundation model while reducing the parameter count by a factor of 48.
% Overall, by reducing computational complexity and integrating it into existing workflows, the proposed approach may significantly impact diagnostic processes, reduce the workload of pathologists, and contribute to improved patient outcomes. Though our approach shows potential enhancements in efficiency and usability of cell segmentation and classification models in digital pathology, extensive validation is needed to deploy these models in clinical practice.
% \end{abstract}

%%% shortened abstract
\begin{abstract}
Developing clinically useful cell-level analysis tools in digital pathology remains challenging due to limitations in dataset granularity, inconsistent annotations, high computational demands, and difficulties integrating new technologies into workflows. To address these issues, we propose a solution that enhances data quality, model performance, and usability by creating a lightweight, extensible cell segmentation and classification model. 

First, we update data labels through cross-relabeling to refine annotations of PanNuke and MoNuSAC, producing a unified dataset with seven distinct cell types. Second, we leverage the H-Optimus foundation model as a fixed encoder to improve feature representation for simultaneous segmentation and classification tasks. Third, to address foundation models' computational demands, we distill knowledge to reduce model size and complexity while maintaining comparable performance. Finally, we integrate the distilled model into QuPath, a widely used open-source digital pathology platform. 

Results demonstrate improved segmentation and classification performance using the H-Optimus-based model compared to a CNN-based model. Specifically, average $R^2$ improved from 0.575 to 0.871, and average $PQ$ score improved from 0.450 to 0.492, indicating better alignment with actual cell counts and enhanced segmentation quality. The distilled model maintains comparable performance while reducing parameter count by a factor of 48. By reducing computational complexity and integrating into workflows, this approach may significantly impact diagnostics, reduce pathologist workload, and improve outcomes. Although the method shows promise, extensive validation is necessary prior to clinical deployment.
\end{abstract}
\clearpage

\section{Introduction}
In digital pathology, accurate segmentation and classification of cells are crucial for many diagnostic, prognostic, and predictive analyses \cite{Jaber_Beziaeva_etal._2019,Lin_Pan_etal._2022,Park_Ock_etal._2022,Shen_Choi_etal._2024}. Nowadays, developments in computational pathology offer multiple solutions \cite{H._Qu_P._Wu_etal._2020,Javed_Mahmood_etal._2020} to utilize cell-level datasets to train machine learning models that solve these problems. The quality and specificity of training datasets are critical for robust and accurate models. Adhering to the principle of "garbage in, garbage out", it is essential to ensure that these datasets are extensively and accurately labeled with distinct classes that reflect the diverse biological characteristics of different cell types. Unfortunately, the number of open-source datasets comprising such high-quality annotations is limited. Existing cell segmentation datasets \cite{Gamper_Koohbanani_etal._2019,Graham_Vu_etal._2019,Verma_Kumar_etal._2021} may offer extensive annotations for certain cell types while providing more general labels for others. For example, in PanNuke, which is one of the largest open-source datasets comprising labeled cells, various types of morphologically and functionally different inflammatory cells like macrophages and lymphocytes are clustered in a broad "inflammatory" class. Consequently, these classes are frequently omitted from analyses or aggregated into broader meta-classes \cite{Gamper_Koohbanani_etal._2020} and likely interfere with other cell classes included in the dataset. This and similar inconsistencies in annotation granularity limit the ability of machine learning models to learn the comprehensive and nuanced features necessary for accurate cell segmentation and classification. To address these challenges, methods for refining and standardizing dataset annotations are essential to enhance the quality of training data.

A complementary approach to mitigate the absence of high-quality training data is the use of foundation models. Foundation models as encoders are defined as large-scale, versatile networks pre-trained on vast, diverse datasets using self-supervised learning, contrasting with convolutional neural network (CNN) pre-trained encoders that rely on supervised learning with labeled data. In practice, foundation models leverage enormous amounts of weakly or unlabeled data from millions of whole slide images (WSIs) and employ self-attention mechanisms to capture long-range dependencies and global context \cite{Chen_Ding_etal._2024,Saillard_Jenatton_etal._2024,Vorontsov_Bozkurt_etal._2024,Xu_Usuyama_etal._2024}. As a consequence, foundation models are able to produce transferable feature representations across different cell types and tissue environments. The feature representations can be leveraged by decoder networks to produce segmentation masks and pixel-level classifications. Because foundation models have comprehensive feature representations, they can be effectively fine-tuned using much smaller amounts of cell-level data compared to the large datasets needed to train models from scratch. Furthermore, foundation models incorporate adversarial training elements or contrastive learning \cite{Chen_Ding_etal._2024,Xu_Usuyama_etal._2024}, enhancing their resilience and adaptability by exposing them to challenging and varied scenarios during training. This may result in more generalizable models, often making them well-suited for diverse and complex tasks in digital pathology.

Despite the inherent advantages of foundation models, their deployment for practical use faces its own obstacles. In particular, they require substantial computational power, financial investments and rigorous testing to ensure reliability and efficacy for a given task \cite{Akkus_Dangott_etal._2022,Dragomir_Cocuz_etal._2022,Go_2022,Jafri_Farooqui_etal._2024}. Moreover, while foundation models enhance feature representation and performance, they depend on the quality of available annotations for decoder fine-tuning and, like any other model, cannot resolve existing inconsistencies or ambiguities in data labels. Therefore, there remains a critical need for solutions that address both data quality and practical deployment considerations.
Further, integrating new technologies into existing clinical workflows often encounters resistance, as it necessitates adjustments to established diagnostic processes. So, there is a need to develop solutions that could be integrated into current practices, minimizing the burden on medical professionals to adopt new tools \cite{King_Williams_etal._2023}.

Existing solutions \cite{Goldsborough_Philps_etal._2024,Hörst_Rempe_etal._2024}, while addressing some aspects of these challenges, fall short in providing a comprehensive approach. To address the data quality and clinical deployment issues, we propose a multi-faceted solution that encompasses data refinement, model optimization, and integration with existing pathology tools (\hyperref[fig:fig1]{Figure 1}). The outcome is a lightweight cell segmentation and classification model that can be integrated into digital pathology workflows for practical clinical use.

\begin{figure}[h!]
    \centering
    \includegraphics[width=\textwidth, height=0.82\textheight, keepaspectratio]{images/Figure_1.pdf}
    \caption{Overview of the proposed solution, including 1) Data refinement using cross-relabeling, 2) Teacher model development and fine tuning, 3) Student model optimization with knowledge distillation and 4) Student model and QuPath integration}
    \label{fig:fig1}
\end{figure}
\clearpage

Our approach begins with preparing the data for the fine-tuning and training of the machine learning models. We create a refined dataset, acquired via cross-relabeling two cell-level datasets, enhancing annotation specificity and consistency of the labeled data. Subsequently, we create a cell segmentation and classification model based on the foundation model. We leverage the foundation model as a fixed encoder and fine-tune a decoder using the refined dataset to improve generalization across diverse tissue- and cell types.
To ensure that the model remains lightweight and deployable in a possibly resource-constrained environment, we employ knowledge distillation to approximate the functionality of the foundation model. Finally, to facilitate the practical application of our model in digital pathology workflows, we integrate it with the QuPath \cite{Bankhead_Loughrey_etal._2017} application. Each methodological component contributes to the overarching goal of enhancing model performance, generalizability, and usability in clinical settings.

The primary contributions of this paper are:
\begin{enumerate}
    \item \textit{Data labels refinement through cross-relabeling:}
    
    We propose a new method for refining labels of cell-level datasets through cross-relabeling. This method employs classification models to re-label broad and ambiguous instances, resulting in a more diverse dataset. Our evaluation demonstrates that these classification models achieve high accuracy on test subsets, indicating the reliability of the method for label refinement.

    \item \textit{Enhanced model performance via foundation models:}
    
    We employ a foundation model as a feature extractor for the cell segmentation and classification task. In comparison with training a CNN model from scratch, the foundation model backbone only needs fine-tuning, which significantly reduces training time, computational resources and data requirements. We show that using a foundation model encoder leads to better performance in cell segmentation and classification networks than using a CNN-based encoder. This improvement may enable the model to generalize more effectively across various tissue types and imaging methods.
    
    \item \textit{Model optimization through knowledge distillation:}
    
    We show that a smaller student model trained using knowledge distillation on the refined dataset obtained via our cross-relabeling approach from a foundation model achieves comparable performance in cell segmentation and quantification tasks. As a result, this model is more suitable for deployment in environments without high-performance computing resources.
    
    \item \textit{Integration with QuPath:}
    
    We integrate the distilled cell segmentation and classification model into QuPath, a widely used open-source digital pathology platform, to accelerate clinical adaptation by enabling pathologists to more easily incorporate advanced computational tools into their existing workflows.
\end{enumerate}

Through these methodological steps, we aim to bridge the gap between advanced machine learning techniques and practical clinical applications, making accurate and efficient digital pathology accessible in a broader range of healthcare settings.

\section{Refining Existing Datasets Using Cross-Relabeling}
To address the limitations of sparse and ambiguous labeling of cell-level datasets, we propose a generalizable cross-relabeling strategy that can be applied to any dataset containing broadly categorized or imprecisely labeled cell types. This approach involves training and subsequently leveraging classification models to refine broad categories into more specific or biologically relevant classes.
When applied to cell-level data, the methodology includes extracting individual cell images from the dataset patches, preprocessing these images to standardize the size and accommodate partial cells, and then training deep learning classifiers capable of distinguishing between the finer cell subtypes within the coarser categories. 
To illustrate our approach, we focus on the PanNuke \cite{Gamper_Koohbanani_etal._2020, Gamper_Koohbanani_etal._2019} and MoNuSAC \cite{Verma_Kumar_etal._2021} datasets that we have used to train models for cell quantification in our previous works \cite{Shvetsov_Grønnesby_etal._2022,Shvetsov_Sildnes_etal._2024}. We find that for better cell differentiation we have to introduce more granular labels. PanNuke includes a broad classification of "inflammatory" cells, encompassing lymphocytes, macrophages, and neutrophils. Each cell type differs significantly in structure, function, and clinical relevance. Conversely, MoNuSAC uses the label "epithelial" for a class that comprises both benign epithelial cells and malignant neoplastic cells. This practice makes it challenging to differentiate between benign and malignant epithelial cells in the dataset, which is a critical distinction when identifying tumor areas within tissue samples. To address these issues, we implement a cross-relabeling strategy as shown in \hyperref[fig:fig2]{Figure 2}. The key components are two classification models: one is trained on singular cell images from PanNuke data to classify the epithelial meta-class into epithelial and neoplastic classes. The other is trained on MoNuSAC to refine the inflammatory class into lymphocytes, neutrophils, and macrophages.

\begin{figure}[h!]
    \centering
    \includegraphics[width=\textwidth]{images/Figure_2.pdf}
    \caption{Refined dataset generation via cross relabeling}
    \label{fig:fig2}
\end{figure}

The refining approach consists of three consecutive steps. The first is the preprocessing step, in which we extract individual cells from both datasets (\hyperref[fig:fig3]{Figure 3}). The specifics of PanNuke and MoNuSAC patch preparation before cell preprocessing are provided in \hyperref[chap:S1]{Appendix S1}.

\begin{figure}[h!]
    \centering
    \includegraphics[width=\textwidth]{images/Figure_3.pdf}
    \caption{Cell instances preprocessing including (1) cell map extraction, (2) bounding box delineation, (3) adjusting cell boxes and (4) cropping and resizing of cell images}
    \label{fig:fig3}
\end{figure}

During preprocessing, we extract cell type maps from the ground truth label mask and calculate bounding boxes around each cell instance. To accommodate partial cells at patch borders, a common issue in cropped patch images, we employ mirror padding and extend the field of view of the cell label by 15 pixels to capture adjacent cells. We then crop and resize the identified regions to $64 \times 64$ pixels using bicubic interpolation.

The preprocessed PanNuke dataset comprises 68,031 neoplastic and 23,207 epithelial cell images, while MoNuSAC comprises  33,104 lymphocytes, 1,252 neutrophils, and 1,695 macrophages, which we subsequently use in training cell classification models and classifying the cell image data \hyperref[fig:S2]{Appendix Figure S2 (1)}. 

The next step is to train two distinct ResNet50-based classifiers tailored to address the specific labeling challenges inherent in each dataset. We use ResNet50 for classification models due to its proven effectiveness for image classification tasks in histopathology \cite{pan2022reviewmachinelearningapproaches}, and its compatibility with small images. For the PanNuke dataset, we design the classifier, trained on MoNuSAC data, to disaggregate the heterogeneous "inflammatory" cell category into distinct subtypes: lymphocytes, macrophages, and neutrophils. Similarly, for the MoNuSAC dataset, the classifier is trained on PanNuke data and distinguishes between benign and malignant epithelial cells within the overarching "epithelial" label. By applying these targeted classifiers to their respective datasets, we assign more specific labels to individual cell instances, thus enabling us to create a unified dataset.
To ensure a balanced representation of classes, we train both models on datasets that had been equalized to match the size of the least represented class. Thus, we obtain datasets comprising 23,207 samples per class for PanNuke and 1,252 samples per class for MoNuSAC data. Next, we partition both of them into training (70\%), validation (20\%), and testing (10\%) subsets. To mitigate the risk of overfitting, we use a single dropout layer with a rate of p=0.5 in both models and data augmentation using randomized color perturbations, rotation, and horizontal and vertical flipping. We employ AdamW optimizer and the cross-entropy loss function for the training criterion.

To evaluate the two trained models, we measure the classification accuracy on the respective test subsets. The accuracies on the test subset for both classifiers are presented in \hyperref[tab:1]{Table 1}. The PanNuke model achieves an average accuracy of 93.57\%, with higher accuracy for neoplastic cells (96.06\%) compared to epithelial cells (86.26\%). The confusion matrix in Figure A3.1 shows that the model predominantly distinguishes accurately between epithelial and neoplastic tissues, with a substantial number of correct classifications and relatively few misclassifications. The MoNuSAC model demonstrates an average accuracy of 98.92\%, excelling in classifying lymphocytes (99.67\%) and macrophages (94.12\%), with lower performance for neutrophils (85.71\%). The confusion matrix in Figure A3.2 shows that the model identifies lymphocytes and performs reasonably well with macrophages and neutrophils.

\begin{table}[h!]
\renewcommand{\arraystretch}{1.5}
  \centering
  \caption{Cell classification results for PanNuke and MoNuSAC trained models (CI 95\%).}
  \label{tab:1}
  \begin{tabular}{|l|c|c|}
   \hline
   %\rowcolor{gray!30}
    Accuracy               & PanNuke model              & MoNuSAC model              \\
    \hline
    Average      & 0.936 (0.931--0.941)         & 0.989 (0.986--0.993)        \\
    \hline
    Neoplastic   & 0.961 (0.956--0.965)         & -                          \\
    \hline
    Epithelial   & 0.863 (0.849--0.877)         & -                          \\
    \hline
    Lymphocytes  & -                          & 0.997 (0.995--0.999)        \\
    \hline
    Neutrophils  & -                          & 0.857 (0.796--0.918)        \\
    \hline
    Macrophages  & -                          & 0.941 (0.906--0.976)        \\
    \hline
  \end{tabular}
\end{table}

Finally, during the last step, we use the model trained on PanNuke data for epithelial cells in MoNuSAC and the model trained on MoNuSAC for the inflammatory cells class in PanNuke. Specifically, we use classifier models to relabel epithelial cells in MoNuSAC and inflammatory cells in PanNuke data. Then we combine cells with refined labels and the rest of the cells in both datasets to create a refined dataset (\hyperref[fig:S2]{Appendix Figure S2 (2)}). The process of relabeling cells and visualizing them on a patch is shown in \hyperref[fig:fig4]{Figure 4}. The cell counts in the refined dataset are provided in \hyperref[tab:S4]{Appendix Table S4}.

\begin{figure}[h!]
    \centering
    \includegraphics[width=\textwidth, height=0.42\textheight, keepaspectratio]{images/Figure_4.pdf}
    \caption{Cell relabeling procedure for epithelial and inflammatory cell classes}
    \label{fig:fig4}
\end{figure}

%\hfill

Relabeling and combining datasets have been explored in a prior study \cite{Parulekar_Kanwat_etal._2023}, where consecutive fine-tuning on multiple datasets was employed to account for hierarchical class label structures. While the method presented in \cite{Parulekar_Kanwat_etal._2023} is intuitive, it often lacks consistency and requires multiple fine-tuning runs, which can be cumbersome and time-consuming. 
In contrast, cross-relabeling simplifies this process by using specialized classification models tailored to each dataset's specific labeling challenges. This approach provides better transparency and produces a unified dataset encompassing seven distinct cell types across multiple tissue samples, enhancing data diversity for further model training or fine-tuning.

Despite these improvements, cross-relabeling does not entirely resolve issues related to poor labeling quality or the amount of labeled data. Specifically, our results show lower accuracies persist for underrepresented classes, such as macrophages, which may stem from a limited sample availability and intrinsic challenges in distinguishing these cells based solely on H\&E staining. Furthermore, while our method enhances label specificity, it relies on the initial quality of the broad labels; thus, any fundamental inaccuracies in the original annotations can propagate through the relabeling process. Addressing the overall problem of limited data labels may require integrating additional data sources or utilizing complementary immunohistochemical staining methods.
Although the reported performance metrics are obtained from evaluations on the native test sets of each dataset, it is important to note that the primary application of these classifiers is to perform cross-relabeling, where a model trained on one dataset (e.g., PanNuke) is applied to another (e.g., MoNuSAC) and vice versa. We acknowledge that a more systematic evaluation of cross-dataset generalization is needed and could be performed in future work.

Overall, the refined dataset produced by our approach can enhance the supervised training or fine-tuning of cell segmentation and classification models, especially those that utilize pre-trained foundation models to improve feature extraction robustness. In addition, these models can detect nuanced classes that enable researchers to conduct more detailed analyses of biological processes in computational pathology.

\section{Foundation models for robust cell segmentation and classification}

Accurate cell segmentation and classification in digital pathology are hindered by limited labeled data and the fact that conventional CNNs are unable to capture global contextual information due to their local receptive field constraints \cite{Gheflati_Rivaz_2022,Yang_Marcus_etal.}. Traditional approaches in cell quantification have predominantly relied on CNN encoders, such as ResNet50, given their proven effectiveness in semantic segmentation tasks \cite{Deshmane_2023,Graham_Vu_etal._2019,Mukasheva_Koishiyeva_etal._2024,Stringer_Wang_etal._2021}. However, approaches that include fine-tuning of pretrained CNNs, data augmentation, and stain normalization to partially increase data variability and address staining differences often fail to achieve the necessary generalization and robustness across diverse tissue types and staining conditions \cite{G._Wang_W._Li_etal._2018,Gao_Bagci_etal._2018,Karim_El_Khoury_Martin_Fockedey_etal._2021}.

To overcome these challenges, we leverage an encoder-decoder network that uses a foundation model as the encoder and a CNN upsampling decoder (\hyperref[fig:fig5]{Figure 5}) for simultaneous cell segmentation and classification in 2D patches extracted from WSIs. Foundation models with transformer-based architectures are viable alternatives to CNN-based encoders \cite{Shamshad_Khan_etal._2023,Sourget_2023}. They enable the creation of more advanced architectures that can decode or transform learned features more effectively \cite{Chen_Duan_etal._2023,Cheng_Misra_etal._2022,Xie_Wang_etal._2021}.

\begin{figure}[h!]
    \centering
    \includegraphics[width=\textwidth]{images/Figure_5.pdf}
    \caption{UNETR-like model with foundational model as backbone}
    \label{fig:fig5}
\end{figure}

By utilizing a transformer-based encoder, we incorporate global contextual information into the feature extraction process, which is a key advantage of such architectures \cite{Chen_Lu_etal._2021}. This foundation model integration facilitates accurate pixel-wise segmentation and classification without the need for extensive encoder training, thereby potentially improving generalization across varied cellular structures and tissue types.
In our implementation, we employ a modified UNETR \cite{Hatamizadeh_Tang_etal._2021} architecture that combines a vision transformer (ViT) \cite{Dosovitskiy_Beyer_etal._2021} encoder with a CNN-based decoder. The encoder utilizes the pretrained H-Optimus foundation model, which contains 1.1 billion parameters and is trained on over 500,000 H\&E stained WSIs \cite{Saillard_Jenatton_etal._2024}. We extract outputs from four evenly spaced transformer blocks $Z_i$, where $i \in [1, 14, 26, 38]$, to serve as residual connections for the CNN decoder. We select these blocks based on our observation that features from non-adjacent levels of the encoder lead to better overall performance on the test subset.

The CNN decoder upsamples the feature representations, acquired from the transformer blocks, to generate an intermediate vector that is handled by two task-specific layers that generate cell segmentation and classification masks. The first task-specific layer is the ‘Cellpose head’,  which is used to delineate cell instances. The layer generates horizontal and vertical gradient maps to form vector fields that are refined through gradient tracking in a post-processing step using the Cellpose algorithm \cite{Stringer_Wang_etal._2021}, known for its efficacy in cell segmentation tasks and generalizability across multiple domains \cite{Pachitariu_Stringer_2022,Stringer_Pachitariu_2024}. The second task-specific layer is the "Cell type head", which assigns labels to individual pixels. In the post-processing step, we determine the output classification label of each segmented cell instance by majority voting over the labeled pixels that comprise the cell in the segmentation map.

To evaluate model performance and measure the impact of adding a foundation model as backbone, we compare it to a ResNet50-based model. ResNet50 is a widely used solution for encoders in segmentation architectures in the medical domain \cite{Deshmane_2023,Graham_Vu_etal._2019,Mukasheva_Koishiyeva_etal._2024,Stringer_Wang_etal._2021}. For the H-Optimus-based model, we utilize frozen weights for the encoder and only fine-tune the decoder to take advantage of the extensive pre-training of the foundation model. For the ResNet50-based model we start with ImageNet \cite{Deng_Dong_etal.} weights and train both encoder and decoder parts. Hyperparameters for the training step are set to be identical, where possible, for comparable evaluation. 
For this evaluation, we deliberately use the PanNuke dataset to provide a standardized and controlled comparison between the H‑Optimus and ResNet50-based models (\hyperref[fig:S2]{Appendix Figure S2 (3)}). Specifically, we use two of the default PanNuke dataset splits (66\%) for training and validation, and reserve the third split (33\%) for testing.

To address the challenge of cell class imbalance in the PanNuke dataset, which is a common characteristic in most cell-level H\&E patch datasets, both models’ training processes employ a weighted loss function comprising cross-entropy and focal loss \cite{Lin_Goyal_etal._2018}. The focal loss component is adjusted with coefficients derived from each cell class' instance frequency, emphasizing learning from underrepresented classes and enhancing the model's sensitivity to rare but significant cellular patterns. The cross-entropy loss is augmented with spectral decoupling regularization \cite{Pezeshki_Kaba_etal._2021,Pohjonen_Stürenberg_etal._2022} and spatially varying label smoothing \cite{Islam_Glocker_2021}, which potentially stabilizes training and improves generalization in case of complex tissue morphologies. For optimization, we employ the \textit{AdamW} \cite{Loshchilov_Hutter_2019} to counter unbalanced class scenarios, with cosine annealing learning rate scheduler.

We utilize the scikit-learn library \cite{Van_der_Walt_Schönberger_etal._2014} and HoVer-Net \cite{Graham_Vu_etal._2019} implementations of $R^2$ (the coefficient of determination) and $PQ$ (panoptic quality) to evaluate our experiments. Complete mathematical formulations and detailed explanations of these metrics are provided in \hyperref[chap:S5]{Appendix S5}. To compute confidence intervals, we use nonparametric bootstrapping, where after calculating the metric on the full sample, we generated 1000 bootstrap replicates by resampling with replacement and then determined the 95\% confidence intervals as the 2.5th and 97.5th percentiles of the resulting empirical distribution.

%\hfill

The model comparisons are summarized in \hyperref[tab:2]{Table 2}. The H‑Optimus-based model achieves higher $R^2$ across all cell classes compared to the ResNet50-based model, which means that its predictions are more closely aligned with the PanNuke cell counts, indicating a stronger correlation with the observed data. Notably, the improvement of $R^2_{dead}$ may be an indicator of better global contextual representations provided by the foundation model backbone. In terms of segmentation and classification quality combined, measured by the PQ score, the H‑Optimus-based model demonstrates notable improvements across most cell classes. Overall, the average $R^2$ improved from 0.575 to 0.871, while the average $PQ$ score improved from 0.450 to 0.492, demonstrating better performance of the H-Optimus-based model.

\begin{table}[h!]
\renewcommand{\arraystretch}{1.5}
  \centering
  \caption{Cell quantification metrics for baseline and proposed models (CI 95\%).}
  \label{tab:2}
  \begin{tabular}{|l|c|c|}
    \hline
    %\rowcolor{gray!30}
    Metric             & Resnet50-based            & H-optimus-based              \\
    \hline
    $R^2_{neoplastic}$    & 0.681 (0.576--0.769)       & \textbf{0.941 (0.917--0.960)} \\
    \hline
    $R^2_{inflammatory}$  & 0.863 (0.778--0.903)       & \textbf{0.949 (0.918--0.966)} \\
    \hline
    $R^2_{connective}$    & 0.600 (0.488--0.698)       & 0.609 (0.436--0.772)          \\
    \hline
    $R^2_{dead}$          & 0.097 (-11.389--0.669)     & 0.925 (0.404--0.982)          \\
    \hline
    $R^2_{epithelial}$    & 0.635 (0.490--0.747)       & \textbf{0.930 (0.886--0.964)} \\
    \hline
    $PQ_{neoplastic}$       & 0.517 (0.499--0.535)       & \textbf{0.589 (0.575--0.604)} \\
    \hline
    $PQ_{inflammatory}$     & 0.455 (0.429--0.482)       & \textbf{0.528 (0.507--0.549)} \\
    \hline
    $PQ_{connective}$       & 0.416 (0.400--0.431)       & \textbf{0.451 (0.436--0.465)} \\
    \hline
    $PQ_{dead}$             & 0.374 (0.342--0.408)       & 0.292 (0.209--0.365)          \\
    \hline
    $PQ_{epithelial}$       & 0.488 (0.460--0.519)       & \textbf{0.599 (0.579--0.618)} \\
    \hline
  \end{tabular}
\end{table}

Our results  show that integrating the H‑Optimus foundation model within the UNETR architecture enhances the model's ability to segment and classify cells across diverse tissues from PanNuke data. The pretrained transformer encoder provides robust feature representations, resulting in higher average $R^2$ and $PQ$ scores compared to the CNN-based model. This leads to more reliable cell quantification and more accurate downstream analysis. Additionally, the streamlined fine-tuning process reduces computational overhead and training time, making the model more adaptable for new data.

Despite these advancements, the foundation model-based approach does not fully resolve all challenges related to cell segmentation and classification. We observe lower metric scores for underrepresented classes in the training data. Furthermore, foundation models typically encompass billions of parameters, resulting in substantial computational and memory requirements. It therefore poses challenges for deployment in resource-constrained environments, limiting their practical applicability in certain clinical settings.

\section{Model optimization via Knowledge Distillation}

To address the limitations posed by the extensive size of foundation models, we implement knowledge distillation — a model compression technique that leverages the teacher-student paradigm \cite{Hinton_Vinyals_etal._2015}. By training a smaller, more efficient student model to replicate the output of a larger, pre-trained teacher model, we retain performance while significantly reducing the model's complexity and resource requirements (\hyperref[fig:fig6]{Figure 6}).

\begin{figure}[h!]
    \centering
    \includegraphics[width=\textwidth, height=0.45\textheight, keepaspectratio]{images/Figure_6.pdf}
    \caption{Knowledge distillation framework for training a student model using a pre-trained teacher}
    \label{fig:fig6}
\end{figure}

We employ knowledge distillation to compress the H‑Optimus-based teacher model into a more efficient student model. The teacher model is the modified UNETR architecture with the H‑Optimus foundation model described in the previous chapter. The student model is based on a UNet architecture augmented with residual connections and incorporates a smaller ViT encoder with 9 million parameters \cite{Steiner_Kolesnikov_etal._2022,Wightman_2019}. 

First, we fine-tune the teacher model using the refined dataset from the cross-relabeling procedure (Section 2). Initially we train the decoder of the teacher model while keeping the encoder weights frozen. We split the refined dataset into train (70\%), validation (20\%) and test (10\%) subsets (\hyperref[fig:S2]{Appendix Figure S2 (4)}). During fine-tuning, we use the train and validation subsets, while leaving the test subset for model evaluation. We set the training procedure and model hyperparameters to be identical to those that were used to demonstrate the utility of foundation models for the simultaneous cell segmentation and classification task.

Next, we perform knowledge distillation from teacher to student using the refined dataset used to fine-tune the teacher model. The student model is trained to replicate the teacher model's outputs. We utilize a specialized loss function that aligns the student's predicted probability distribution with the teacher's, incorporating the teacher's class probability distribution derived from the output. Following the methodology of Hinton et al. \cite{Hinton_Vinyals_etal._2015}, we experiment with various hyperparameter settings for the temperature ($T$) and the balancing coefficients ($\alpha$ and $\beta$) in the loss function. We vary $T$ from 1 to 20 and adjust $\alpha$ and $\beta$ to balance the distillation and student losses. Through iterative tuning and evaluation, we identify that setting $T=14$, $\alpha=0.3$, and $\beta=0.7$ yields a configuration that converges and closely approximates the teacher model's performance during training.

Finally, we assess the performance of both models using the $R^2$ and $PQ$ (defined in \hyperref[chap:S5]{Appendix S5}) on the test set of the refined dataset (\hyperref[tab:3]{Table 3}). We observe that the 95\% confidence intervals overlap for most cell types, so we cannot claim statistically significant performance differences between the teacher and student models. One exception appears in the neoplastic class. The teacher model produces an $R^2$ of 0.919, while the student model shows an $R^2$ of 0.852. In addition, the student model achieves higher $PQ$ values for the neoplastic and connective classes, though the confidence intervals show overlap.

\begin{table}[h!]
\renewcommand{\arraystretch}{1.5}
  \centering
  \caption{Cell quantification metrics for teacher and distilled student models (CI 95\%).}
  \label{tab:3}
  \begin{tabular}{|l|c|c|}
    \hline
    %\rowcolor{gray!30}
    Metric & Teacher & Student \\
    \hline
    $R^2_{neoplastic}$    & \textbf{0.919} (0.898--0.939) & 0.852 (0.800--0.891) \\
    \hline
    $R^2_{lymphocyte}$    & 0.969 (0.956--0.977)         & 0.969 (0.956--0.978) \\
    \hline
    $R^2_{connective}$    & 0.694 (0.548--0.809)         & 0.618 (0.469--0.741) \\
    \hline
    $R^2_{dead}$          & 0.755 (0.400--0.908)         & 0.424 (0.100--0.731) \\
    \hline
    $R^2_{epithelial}$    & 0.922 (0.870--0.958)         & 0.843 (0.738--0.917) \\
    \hline
    $R^2_{macrophage}$    & 0.384 (-0.369--0.724)        & 0.704 (0.352--0.859) \\
    \hline
    $R^2_{neutrofil}$     & 0.854 (0.578--0.929)         & 0.833 (0.502--0.925) \\
    \hline
    $PQ_{neoplastic}$       & 0.581 (0.569--0.593)         & 0.601 (0.588--0.613) \\
    \hline
    $PQ_{lymphocyte}$       & 0.536 (0.520--0.553)         & 0.563 (0.544--0.579) \\
    \hline
    $PQ_{connective}$       & 0.436 (0.421--0.451)         & 0.457 (0.441--0.474) \\
    \hline
    $PQ_{dead}$             & 0.272 (0.235--0.315)         & 0.279 (0.201--0.369) \\
    \hline
    $PQ_{epithelial}$       & 0.522 (0.500--0.545)         & 0.530 (0.506--0.555) \\
    \hline
    $PQ_{macrophage}$       & 0.524 (0.459--0.588)         & 0.474 (0.405--0.543) \\
    \hline
    $PQ_{neutrofil}$        & 0.541 (0.490--0.592)         & 0.565 (0.522--0.607) \\
    \hline
  \end{tabular}
\end{table}


We further decompose the $PQ$ metric into its $SQ$ and $DQ$ components (\hyperref[tab:S6]{Appendix Table S6}). Both models produce nearly identical $SQ$ values, which indicates that they predict instance boundaries with similar precision. Although the student model shows some improvement in $DQ$ scores for certain classes, the confidence intervals overlap and do not confirm a statistically significant difference.

We observe that the student and teacher models yield comparable detection performance despite the student model using a much smaller and simpler architecture. A model with fewer parameters reduces the risk of overfitting when training data are scarce relative to the model’s complexity \cite{Farias_Ludermir_etal._2022}. The knowledge distillation process also encourages the student model to focus on the most generalizable detection features learned from the teacher. These factors enable the student model to achieve similar detection performance across different cell types.

Additionally, considering the model sizes reported in \hyperref[tab:4]{Table 4}, the distilled model achieves a significant reduction compared to the teacher model, with a 48-fold decrease in parameter count and a 5.5-fold reduction in on-disk size. In inference mode, the teacher model requires 16 GB of VRAM for a batch size of 32, while the distilled model only needs 3 GB of VRAM for the same batch size. These reductions make the distilled model significantly more practical for fine-tuning and deployment in resource-constrained environments.

\begin{table}[h!]
\renewcommand{\arraystretch}{1.5}
  \centering
  \caption{Parameter counts and size of teacher and distilled model}
  \label{tab:4}
  \adjustbox{max width=\textwidth}{%
  \begin{tabular}{|l|c|c|c|}
    \hline
    %\rowcolor{gray!30}
    Metric & H-optimus-based (Teacher) & mobileViT-based (Student) & Magnitude of difference \\
    \hline
    Parameters count       & 1,158,917,906   & \textbf{24,093,393}   & \textbf{48x}  \\
    \hline
    Estimated Total Size (MB) & 87,912       & \textbf{15,935}    & \textbf{5.5x} \\
    \hline
  \end{tabular}%
}
\end{table}

%\hfill

With recent advancements in complex network architectures and the use of pretrained encoders to achieve state-of-the-art performance \cite{Baumann_Dislich_etal._2024,Hörst_Rempe_etal._2024} in cell segmentation and classification tasks, model size, computational complexity, and processing times have increased. This limits the scalability and accessibility of these models. As we demonstrate, this may be mitigated using knowledge distillation. Studies in the field of natural language processing have demonstrated the efficacy of knowledge distillation in retaining the capabilities of the teacher model while achieving significant reductions in size and complexity \cite{Huangpu_Gao_2024,Sun_Yu_etal.}. 

We demonstrate the feasibility of knowledge distillation in digital pathology, specifically for cell segmentation and classification tasks. Moreover, we achieve this performance while also significantly reducing the parameter count. In addressing the challenge of knowledge transfer, we found that distillation from a transformer-based model to a smaller transformer is more straightforward than attempting to map transformer features to CNN blocks. In our experiments, using a CNN-based network as a student results in worse cell quantification performance due to the structural constraints of CNN feature space dimensions. 

Although our primary approach relies on a transformer-based student model that performs well, it can be further optimized to incorporate advantages from CNN architectures. For example, employing alternative techniques such as using ViT adapters \cite{Chen_Duan_etal._2023} or $1 \times 1$ convolutions to adjust feature map sizes may be beneficial for harnessing CNN advantages like enhanced local feature extraction. Moreover, if additional performance improvements are desired, the process can be further enhanced by applying supplementary knowledge distillation techniques, such as self-distillation \cite{Zhang_Song_etal._2019} or online distillation \cite{Houyon_Cioppa_etal._2023}.

Despite these promising results, further validation on independent datasets is necessary to fully understand the model's limitations. Underrepresented classes may pose challenges when addressing complex cases. Pathologists need to validate these models to adopt them in clinical settings. While the distilled models are smaller and more deployable, a technological gap persists because pathologists traditionally rely on established methods for inspecting WSIs and diagnosing diseases. Addressing the complexities involved in deploying models for inference and supporting pathologists in adopting new tools is essential for integrating these models into clinical workflows.

\section{Model integration with QuPath}
Digital pathology tools with graphical user interfaces are essential for visualizing and analyzing WSIs. To make our student model useful in clinical pathology workflows, it needs to be integrated into a tool that enables inspecting regions, creating annotations, and providing quantitative analyses of biomarkers. Therefore, we integrate the trained student model from the previous chapter into the QuPath open‑source platform \cite{Bankhead_Loughrey_etal._2017}. QuPath provides the required annotation, visualization, and analysis tools to interpret complex histological data, including workflows for cell segmentation, classification, and quantification (\hyperref[fig:fig7]{Figure 7}). 

\begin{figure}[h!]
    \centering
    \includegraphics[width=\textwidth]{images/Figure_7.pdf}
    \caption{Visualization of model-generated cell quantification annotations (left) and the corresponding unannotated slide (right) in QuPath}
    \label{fig:fig7}
\end{figure}

To identify the regions in a WSI critical for prognosticating tumor development, such as specific tumor areas or border regions without overlapping healthy tissue, the pathologist uses QuPath to outline these regions. Then, the pathologist initiates a cell segmentation and classification script through the QuPath interface for the selected regions. The resulting annotations and quantified cell information are then directly overlaid onto the WSI in the QuPath interface. Additional design and implementation details are in \hyperref[chap:S7]{Appendix S7}. 

Two common approaches for integrating deep learning models into QuPath are Java‑based native QuPath extensions \cite{Goldsborough_Philps_etal._2024} and the execution of RESTful API requests to a model server coupled with handling the response via an extension, as demonstrated in the application of cell segmentation models applied to immunofluorescence images \cite{Sugawara_2023}. While the community is actively working on these integration strategies, there is currently no universal solution that fully addresses all integration and performance requirements.

Extensions may offer better integration with QuPath, allowing slightly improved performance and more widespread usage of the built-in QuPath models, but they lack the flexibility to customize models and modify their behavior. For example, the newest version of QuPath includes models such as StarDist \cite{Weigert_Schmidt} and InstanSeg \cite{Goldsborough_Philps_etal._2024} that can perform cell segmentation. Both models pose limitations when applied to simultaneous cell segmentation and classification. StarDist performs well only on convex, round shapes by design, whereas some neoplastic, inflammatory, and connective cells exhibit complex and non-convex shapes. InstanSeg provides only semantic segmentation without assigning classes to the segmented cells.

%\hfill

In contrast, our approach offers an alternative integration strategy. It utilizes the paquo library to directly interact with QuPath’s internal application programming interface from within Python. This enables data exchange and processing without the need for intermediate conversion steps and provides greater control over model customization, retraining, and the incorporation of custom processing steps.

The integration of our custom model with QuPath underscores its potential to significantly enhance the diagnostic process by reducing the time burden on pathologists and enabling them to focus on more complex interpretative tasks using familiar software. Leveraging a tool that is already well-established among pathologists increases the likelihood of its adoption into daily clinical workflows. The quantitative data generated through the automated workflow is critical for both clinical decision-making and research, facilitating more accurate biomarker analysis, enabling robust statistical evaluations, and supporting hypothesis generation and testing. Additionally, by streamlining cell segmentation and classification, the tool enhances the scalability and reproducibility of pathological assessments, ultimately contributing to improved diagnostic accuracy and patient outcomes.

\section{Conclusion and future work}

In this study, we address critical challenges in digital pathology and tackle the usability and deployment issues of the developed models in standard computing environments without the need for high-performance computing systems. Our multi-faceted approach encompasses data refinement through cross-relabeling, leveraging foundation models for robust cell segmentation and classification, optimizing model performance via knowledge distillation, and integrating the optimized model into the QuPath software for practical application. This approach is used to construct a capable, versatile, and adjustable model for cell segmentation and classification, with enhanced performance and usability.

\begin{sloppypar}
While our approach shows potential in the field of computational pathology, certain limitations persist. 
For example, our implementation currently exhibits lower performance in detecting macrophages. 
This serves as an instance of the broader challenge of accurately identifying complex cell types. In order to address this issue, extending our approach to incorporate additional data sources, exploring alternative modeling approaches, and integrating other imaging modalities such as immunohistochemical staining may help improve detection accuracy. Moreover, although the distilled model reduces computational demands, integrating advanced deep learning models into clinical practice requires addressing technological gaps and potential resistance to adopting new tools within established diagnostic processes.
\end{sloppypar}

Future work could focus on several key areas to refine the proposed approach and facilitate its adoption in clinical environments. Enhancing the cell-relabeling process with additional datasets \cite{Graham_Jahanifar_etal._2021} could improve the representation of underrepresented cell types and enhance overall model performance. Also, incorporating additional data sources, such as multi-modal imaging or complementary staining methods, may address limitations related to cell type differentiation and class imbalance. Exploring other foundation models \cite{Vorontsov_Bozkurt_etal._2024,Zimmermann_Vorontsov_etal._2024} or introducing additional modalities \cite{Ding_Wagner_etal._2024,Vaidya_Zhang_etal._2025} may provide alternative architectures better suited to specific tasks or offer improved efficiency. Implementing more complex knowledge distillation techniques \cite{Houyon_Cioppa_etal._2023,Zhang_Song_etal._2019} could further optimize the model's performance and adaptability. Additionally, deeper integration with QuPath or other digital pathology software could provide pathologists more control over cell quantification analysis directly within the QuPath interface, thereby increasing accessibility and usability. Such enhancements would not only refine model performance but also ensure greater adaptability and scalability within various clinical environments. Finally, extensive validation of the model by pathologists and benchmarking against independent datasets are essential steps toward establishing the model's reliability and fostering confidence in its clinical utility.

\section*{Acknowledgments} 
This work was funded in part by the Research Council of Norway grant no. 309439 SFI Visual Intelligence, and the North Norwegian Health Authority grant no. HNF1521-20.

\bibliographystyle{IEEEtran}
\begin{sloppypar}
\begin{thebibliography}{99}

\bibitem{chaplot2020neural} Chaplot, Devendra Singh, et al. "Neural topological slam for visual navigation." Proceedings of the IEEE/CVF conference on computer vision and pattern recognition. 2020.

\bibitem{maksymets2021thda} Maksymets, Oleksandr, et al. "Thda: Treasure hunt data augmentation for semantic navigation." Proceedings of the IEEE/CVF International Conference on Computer Vision. 2021.

\bibitem{mezghan2022memory} Mezghan, Lina, et al. "Memory-augmented reinforcement learning for image-goal navigation." 2022 IEEE/RSJ International Conference on Intelligent Robots and Systems (IROS). IEEE, 2022.

\bibitem{al2022zero} Al-Halah, Ziad, Santhosh Kumar Ramakrishnan, and Kristen Grauman. "Zero experience required: Plug \& play modular transfer learning for semantic visual navigation." Proceedings of the IEEE/CVF Conference on Computer Vision and Pattern Recognition. 2022.

\bibitem{ye2021auxiliary} Ye, Joel, et al. "Auxiliary tasks and exploration enable objectgoal navigation." Proceedings of the IEEE/CVF international conference on computer vision. 2021.

\bibitem{chaplot2020object} Chaplot, Devendra Singh, et al. "Object goal navigation using goal-oriented semantic exploration." Advances in Neural Information Processing Systems 33 (2020)

\bibitem{ramakrishnan2022poni} Ramakrishnan, Santhosh Kumar, et al. "Poni: Potential functions for objectgoal navigation with interaction-free learning." Proceedings of the IEEE/CVF Conference on Computer Vision and Pattern Recognition. 2022.

\bibitem{ramrakhya2022habitat} Ramrakhya, Ram, et al. "Habitat-web: Learning embodied object-search strategies from human demonstrations at scale." Proceedings of the IEEE/CVF Conference on Computer Vision and Pattern Recognition. 2022.

\bibitem{mousavian2019visual} Mousavian, Arsalan, et al. "Visual representations for semantic target driven navigation." 2019 International Conference on Robotics and Automation (ICRA). IEEE, 2019.

\bibitem{dhariwal2021diffusion} Dhariwal, Prafulla, and Alexander Nichol. "Diffusion models beat gans on image synthesis." Advances in neural information processing systems 34 (2021)

\bibitem{ho2022classifier} Ho, Jonathan, and Tim Salimans. "Classifier-free diffusion guidance." arXiv preprint arXiv:2207.12598 (2022).

\bibitem{nichol2021glide} Nichol, Alex, et al. "Glide: Towards photorealistic image generation and editing with text-guided diffusion models." arXiv preprint arXiv:2112.10741 (2021)

\bibitem{brooks2023instructpix2pix} Brooks, Tim, Aleksander Holynski, and Alexei A. Efros. "Instructpix2pix: Learning to follow image editing instructions." Proceedings of the IEEE/CVF Conference on Computer Vision and Pattern Recognition. 2023.

\bibitem{fu2023guiding} Fu, Tsu-Jui, et al. "Guiding instruction-based image editing via multimodal large language models." arXiv preprint arXiv:2309.17102 (2023).

\bibitem{geng2024instructdiffusion} Geng, Zigang, et al. "Instructdiffusion: A generalist modeling interface for vision tasks." Proceedings of the IEEE/CVF Conference on Computer Vision and Pattern Recognition. 2024.

\bibitem{zhou2024minedreamer} Zhou, Enshen, et al. "Minedreamer: Learning to follow instructions via chain-of-imagination for simulated-world control." arXiv preprint arXiv:2403.12037 (2024).

\bibitem{zhou2023esc} Zhou, Kaiwen, et al. "Esc: Exploration with soft commonsense constraints for zero-shot object navigation." International Conference on Machine Learning. PMLR, 2023.

\bibitem{yu2023l3mvn} Yu, Bangguo, Hamidreza Kasaei, and Ming Cao. "L3mvn: Leveraging large language models for visual target navigation." 2023 IEEE/RSJ International Conference on Intelligent Robots and Systems (IROS). IEEE, 2023.

\bibitem{gadre2023cows} Gadre, Samir Yitzhak, et al. "Cows on pasture: Baselines and benchmarks for language-driven zero-shot object navigation." Proceedings of the IEEE/CVF Conference on Computer Vision and Pattern Recognition. 2023.

\bibitem{shah2023navigation} Shah, Dhruv, et al. "Navigation with large language models: Semantic guesswork as a heuristic for planning." Conference on Robot Learning. PMLR, 2023.

\bibitem{cai2024bridging} Cai, Wenzhe, et al. "Bridging zero-shot object navigation and foundation models through pixel-guided navigation skill." 2024 IEEE International Conference on Robotics and Automation (ICRA). IEEE, 2024.

\bibitem{yu2023co} Yu, Bangguo, Hamidreza Kasaei, and Ming Cao. "Co-NavGPT: Multi-robot cooperative visual semantic navigation using large language models." arXiv preprint arXiv:2310.07937 (2023).

\bibitem{wu2024voronav} Wu, Pengying, et al. "Voronav: Voronoi-based zero-shot object navigation with large language model." arXiv preprint arXiv:2401.02695 (2024).

\bibitem{qin2023mp5} Qin, Yiran, et al. "Mp5: A multi-modal open-ended embodied system in minecraft via active perception." arXiv preprint arXiv:2312.07472 (2023).

\bibitem{du2024learning} Du, Yilun, et al. "Learning universal policies via text-guided video generation." Advances in Neural Information Processing Systems 36 (2024).

\bibitem{ajay2024compositional} Ajay, Anurag, et al. "Compositional foundation models for hierarchical planning." Advances in Neural Information Processing Systems 36 (2024).

\bibitem{liang2024skilldiffuser} Liang, Zhixuan, et al. "Skilldiffuser: Interpretable hierarchical planning via skill abstractions in diffusion-based task execution." Proceedings of the IEEE/CVF Conference on Computer Vision and Pattern Recognition. 2024.

\bibitem{heusel2017gans} Heusel, Martin, et al. "Gans trained by a two time-scale update rule converge to a local nash equilibrium." Advances in neural information processing systems 30 (2017).

\bibitem{zhang2018unreasonable} Zhang, Richard, et al. "The unreasonable effectiveness of deep features as a perceptual metric." Proceedings of the IEEE conference on computer vision and pattern recognition. 2018.

\bibitem{brown2020language} Brown, Tom B. "Language models are few-shot learners." arXiv preprint arXiv:2005.14165 (2020).

\bibitem{podell2023sdxl} Podell, Dustin, et al. "Sdxl: Improving latent diffusion models for high-resolution image synthesis." arXiv preprint arXiv:2307.01952 (2023).

\bibitem{brohan2022rt} Brohan, Anthony, et al. "Rt-1: Robotics transformer for real-world control at scale." arXiv preprint arXiv:2212.06817 (2022).

\bibitem{brohan2023rt} Brohan, Anthony, et al. "Rt-2: Vision-language-action models transfer web knowledge to robotic control." arXiv preprint arXiv:2307.15818 (2023).

\bibitem{li2024manipllm} Li, Xiaoqi, et al. "Manipllm: Embodied multimodal large language model for object-centric robotic manipulation." Proceedings of the IEEE/CVF Conference on Computer Vision and Pattern Recognition. 2024.

\bibitem{shah2023vint} Shah, Dhruv, et al. "ViNT: A foundation model for visual navigation." arXiv preprint arXiv:2306.14846 (2023).

\bibitem{liu2024visual} Liu, Haotian, et al. "Visual instruction tuning." Advances in neural information processing systems 36 (2024).

\bibitem{hu2021lora} Hu, Edward J., et al. "Lora: Low-rank adaptation of large language models." arXiv preprint arXiv:2106.09685 (2021).

\bibitem{qin2023supfusion} Qin, Yiran, et al. "SupFusion: Supervised LiDAR-camera fusion for 3D object detection." Proceedings of the IEEE/CVF International Conference on Computer Vision. 2023.

\bibitem{qin2024worldsimbench} Qin, Yiran, et al. "Worldsimbench: Towards video generation models as world simulators." arXiv preprint arXiv:2410.18072 (2024).

\bibitem{yu2025gamefactory} Yu, Jiwen, et al. "GameFactory: Creating New Games with Generative Interactive Videos." arXiv preprint arXiv:2501.08325 (2025).

\bibitem{zhou2024code} Zhou, Enshen, et al. "Code-as-Monitor: Constraint-aware Visual Programming for Reactive and Proactive Robotic Failure Detection." arXiv preprint arXiv:2412.04455 (2024).

\bibitem{zhang2024ad} Zhang, Zaibin, et al. "AD-H: Autonomous Driving with Hierarchical Agents." arXiv preprint arXiv:2406.03474 (2024).

\bibitem{wang2024toward} Wang, Chaoqun, et al. "Toward Accurate Camera-based 3D Object Detection via Cascade Depth Estimation and Calibration." arXiv preprint arXiv:2402.04883 (2024).

\bibitem{huang2024story3d} Huang, Yuzhou, et al. "Story3d-agent: Exploring 3d storytelling visualization with large language models." arXiv preprint arXiv:2408.11801 (2024).

\bibitem{savinov2018semi} Savinov, Nikolay, Alexey Dosovitskiy, and Vladlen Koltun. "Semi-parametric topological memory for navigation." arXiv preprint arXiv:1803.00653 (2018).

\bibitem{majumdar2022zson} Majumdar, Arjun, et al. "Zson: Zero-shot object-goal navigation using multimodal goal embeddings." Advances in Neural Information Processing Systems 35 (2022): 32340-32352.

\bibitem{yadav2023offline} Yadav, Karmesh, et al. "Offline visual representation learning for embodied navigation." Workshop on Reincarnating Reinforcement Learning at ICLR 2023. 2023.

\bibitem{yadav2023ovrl} Yadav, Karmesh, et al. "Ovrl-v2: A simple state-of-art baseline for imagenav and objectnav." arXiv preprint arXiv:2303.07798 (2023).

\bibitem{sun2024fgprompt} Sun, Xinyu, et al. "FGPrompt: fine-grained goal prompting for image-goal navigation." Advances in Neural Information Processing Systems 36 (2024).

\bibitem{zhu2017target} Zhu, Yuke, et al. "Target-driven visual navigation in indoor scenes using deep reinforcement learning." 2017 IEEE international conference on robotics and automation (ICRA). IEEE, 2017.

\bibitem{koh2024generating} Koh, Jing Yu, Daniel Fried, and Russ R. Salakhutdinov. "Generating images with multimodal language models." Advances in Neural Information Processing Systems 36 (2024).

\bibitem{krantz2022instance} Krantz, Jacob, et al. "Instance-specific image goal navigation: Training embodied agents to find object instances." arXiv preprint arXiv:2211.15876 (2022).

\bibitem{schulman2017proximal} Schulman, John, et al. "Proximal policy optimization algorithms." arXiv preprint arXiv:1707.06347 (2017).

\bibitem{anderson2018evaluation} Anderson, Peter, et al. "On evaluation of embodied navigation agents." arXiv preprint arXiv:1807.06757 (2018).

\bibitem{lin2024navcot} Lin, Bingqian, et al. "NavCoT: Boosting LLM-Based Vision-and-Language Navigation via Learning Disentangled Reasoning." arXiv preprint arXiv:2403.07376 (2024).

\bibitem{NavGPT} Zhou, Gengze, Yicong Hong, and Qi Wu. "Navgpt: Explicit reasoning in vision-and-language navigation with large language models." Proceedings of the AAAI Conference on Artificial Intelligence.

\bibitem{hahn2021no} Hahn, Meera, et al. "No rl, no simulation: Learning to navigate without navigating." Advances in Neural Information Processing Systems 34 (2021): 26661-26673.

\bibitem{li2025t2isafety} Li, Lijun, et al. "T2ISafety: Benchmark for Assessing Fairness, Toxicity, and Privacy in Image Generation." arXiv preprint arXiv:2501.12612 (2025).

\bibitem{an2024agfsync} An, Jingkun, et al. "AGFSync: Leveraging AI-Generated Feedback for Preference Optimization in Text-to-Image Generation." arXiv preprint arXiv:2403.13352 (2024).


\end{thebibliography}
\end{sloppypar}

\clearpage
\beginsupplement
\section*{Appendix}
\renewcommand{\thesubsection}{S\arabic{subsection}}

\subsection{\label{chap:S1}PanNuke and MoNuSAC preprocessing}
The PanNuke dataset comprises a set of 7,901 RGB patches, each with dimensions of $256 \times 256$ pixels, which we set as the standard patch size for our analysis. In contrast, the MoNuSAC dataset encompasses 294 images of heterogeneous dimensions. To standardize the MoNuSAC images with our experiments, we implement a standardization protocol. Specifically, for images exceeding the dimensions of $256 \times 256$ pixels, we segment them into equal-sized patches and apply mirror padding to the remaining portions to avoid information loss at the peripherals. Patches with dimensions less than $128 \times 128$ pixels are excluded from the dataset due to the insufficient resolution to capture relevant cellular details. For patches where either dimension falls between 128 and 256 pixels, we employ upsampling to achieve the standard patch size. As a result, we obtain a total of 2,823 RGB patches derived from the MoNuSAC dataset for subsequent analysis. For additional details on the MoNuSAC data preparation process, refer to the source code \cite{Shvetsov_2025a}.
\clearpage

\subsection{\label{chap:S2}Data usage for the methodology}

\counterwithin{figure}{subsection}
\renewcommand{\thefigure}{S\arabic{subsection}}

\begin{figure}[h!]
    \centering
    \includegraphics[width=\textwidth, height=0.85\textheight, keepaspectratio]{images/A2.pdf}
    \caption{Overview of the methodology for cross-labeling, dataset refinement, and model comparison. (1) Cross-relabeling - training and testing cell classification models, (2) Cross-relabeling - using cell classification models to create refined dataset, (3) Fine-tuning and training models for comparison, (4) Student knowledge distillation with refined dataset}
    \label{fig:S2}
\end{figure}
\clearpage

\subsection{\label{chap:S3}Confusion matrices for classification models}
\counterwithin{figure}{subsection}
\renewcommand{\thefigure}{S\arabic{subsection}.\arabic{figure}}

\begin{figure}[h!]
    \centering
    \includegraphics[width=\textwidth, height=0.4\textheight, keepaspectratio]{images/A3_1.pdf}
    \caption{Confusion matrix for PanNuke trained model}
    \label{fig:S3.1}
\end{figure}

\begin{figure}[h!]
    \centering
    \includegraphics[width=\textwidth, height=0.4\textheight, keepaspectratio]{images/A3_2.pdf}
    \caption{Confusion matrix for MoNuSAC trained model}
    \label{fig:S3.2}
\end{figure}

\clearpage

\subsection{\label{chap:S4}Datasets cell counts}

\counterwithin{table}{subsection}
\renewcommand{\thetable}{S\arabic{subsection}}

\begin{table}[h!]
\renewcommand{\arraystretch}{2.0}
\centering
\caption{\label{tab:S4}Cell counts for PanNuke, MoNuSAC and refined datasets. Numbers in parentheses indicate preprocessed cell counts for cell classifier models training and testing.}
%\adjustbox{max width=\textwidth}{%
\begin{tabular}{|l|c|c|c|}
\hline
%\rowcolor{gray!30}
Cell type & PanNuke & MoNuSAC & Refined \\
\hline
Neoplastic & 77,403 (68,031) & - & 105,451 \\
\hline
Epithelial & 26,572 (23,207) & - & 29,926 \\
\hline
Epithelial (benign and malignant) & - & 31,402 & - \\
\hline
Inflammatory & 32,276 & - & - \\
\hline
Lymphocytes & - & 37,045 (33,104) & 65,275 \\
\hline
Neutrophils & - & 1,355 (1,252) & 3,833 \\
\hline
Macrophage & - & 1,842 (1,695) & 3,410 \\
\hline
Dead & 2,908 & - & 2,908 \\
\hline
Connective & 50,585 & - & 50,585 \\
\hline
\end{tabular}
%
%}
\end{table}



\clearpage

\subsection{\label{chap:S5}Definition of validation metrics}
\counterwithin{equation}{subsection}
\renewcommand{\theequation}{\arabic{equation}}

\subsubsection{\label{chap:S5.1}R\textsuperscript{2}}
The coefficient of determination, denoted as $R^2$, is a statistical measure that represents the proportion of variance in the dependent variable that is predictable from the independent variables. In the context of cell quantification in pathology, $R^2$ is used to assess how well the predicted quantities of different cell types in a patch align with the actual quantities observed in the ground truth data, with higher values representing more accurate quantification. $R^2$ is defined as
\begin{equation*}
R^2 = 1 - \frac{\sum_{i=1}^n (y_i - \hat{y}_i)^2}{\sum_{i=1}^n (y_i - \bar{y})^2},
\end{equation*}
where $y_i$ represents the actual number of cells of a specific type in the $i$-th image, $\hat{y}_i$ represents the predicted number of cells of that type in the $i$-th image, $\bar{y}$ is the mean of the actual numbers across all images, and $n$ is the total number of images in the dataset.

The $R^2$ metric has a range of $(-\infty, 1]$. An $R^2$ of 1 indicates perfect prediction, where all predicted values exactly match the actual values. An $R^2$ of 0 suggests that the model explains none of the variability of the response data around its mean. If $R^2$ is negative, it indicates that the model performs worse than a model that simply predicts the mean of the actual values for all observations.

\subsubsection{\label{chap:S5.2}PQ}
Panoptic Quality ($PQ$) is a comprehensive metric used to evaluate the performance of segmentation models in tasks that require both instance segmentation and classification. $PQ$ provides a single score that encapsulates both the detection accuracy (i.e., how many objects were correctly identified) and the segmentation quality (i.e., how accurately the objects' boundaries were delineated). This metric is particularly useful in multiclass scenarios where each pixel is classified into distinct categories, such as different cell types in pathology images.

$PQ$ is calculated as the product of two terms: Detection Quality ($DQ$) and Segmentation Quality ($SQ$). It can be expressed as
\begin{equation*}
PQ = DQ \cdot SQ,
\end{equation*}
where
\begin{equation*}
DQ = \frac{TP}{TP + 0.5\, FP + 0.5\, FN},
\end{equation*}
\begin{equation*}
SQ = \frac{\sum_{(p, g) \in \mathcal{M}} IoU(p, g)}{TP}.
\end{equation*}
In these formulas, $TP$ denotes the number of correctly matched instances between ground truth and prediction, $FP$ denotes the predicted instances that have no corresponding ground truth, $FN$ denotes the ground truth instances that were not detected, $IoU(p, g)$ is the Intersection over Union for a pair of matched instances $p$ (prediction) and $g$ (ground truth), and $\mathcal{M}$ is the set of matched pairs.

The $PQ$ metric is calculated for each class and is averaged across classes to provide a global performance measure.

The $PQ$ score has a range of $[0, 1.0]$, where a higher score indicates better performance in both detecting and segmenting the instances correctly. A $PQ$ of 1 signifies perfect identification and segmentation of all instances, whereas a $PQ$ of 0 indicates that no instances were correctly identified and segmented.

\clearpage

\subsection{\label{chap:S6}Segmentation and Detection quality metrics for teacher and student models}

\begin{table}[h!]
\renewcommand{\arraystretch}{2.0}
\centering
\caption{Segmentation and detection quality for student and teacher models (CI 95\%)}
\label{tab:S6}
%\adjustbox{max width=\textwidth}{%
\begin{tabular}{|l|c|c|}
\hline
%\rowcolor{gray!30}
Metric & Teacher & Student \\
\hline
$SQ_{neoplastic}$ & 0.819 (0.815--0.823) & 0.824 (0.819--0.828) \\
\hline
$SQ_{lymphocyte}$ & 0.795 (0.788--0.802) & 0.790 (0.783--0.796) \\
\hline
$SQ_{connective}$ & 0.770 (0.762--0.776) & 0.780 (0.772--0.786) \\
\hline
$SQ_{dead}$ & 0.659 (0.623--0.688) & 0.657 (0.624--0.695) \\
\hline
$SQ_{epithelial}$ & 0.780 (0.770--0.790) & 0.788 (0.779--0.797) \\
\hline
$SQ_{macrophage}$ & 0.788 (0.760--0.810) & 0.757 (0.730--0.783) \\
\hline
$SQ_{neutrofil}$ & 0.782 (0.761--0.801) & 0.775 (0.759--0.792) \\
\hline
$DQ_{neoplastic}$ & 0.706 (0.692--0.719) & 0.727 (0.712--0.741) \\
\hline
$DQ_{lymphocyte}$ & 0.675 (0.656--0.698) & 0.713 (0.691--0.734) \\
\hline
$DQ_{connective}$ & 0.566 (0.546--0.584) & 0.583 (0.565--0.602) \\
\hline
$DQ_{dead}$ & 0.410 (0.361--0.465) & 0.435 (0.306--0.561) \\
\hline
$DQ_{epithelial}$ & 0.668 (0.639--0.694) & 0.673 (0.644--0.702) \\
\hline
$DQ_{macrophage}$ & 0.657 (0.583--0.727) & 0.615 (0.531--0.703) \\
\hline
$DQ_{neutrofil}$ & 0.691 (0.625--0.753) & 0.729 (0.679--0.778) \\
\hline
\end{tabular}
%
%}
\end{table}

\clearpage

\subsection{\label{chap:S7}QuPath integration method}
We adopt an integration strategy leveraging the paquo \cite{Bayer_AG} library, a Python package that enables direct interaction with QuPath’s internal API, thereby facilitating seamless data exchange without intermediate conversion steps. The data processing pipeline (\hyperref[fig:S7]{Appendix Figure S7}) begins with the acquisition of WSIs and their associated annotations from QuPath, which are represented as Shapely \cite{Gillies_Wel_etal._2024} polygons. Utilizing paquo, we directly read, create, and modify these annotations and detections within a QuPath project in the Python environment. Images are then cropped using these polygons and processed by cell segmentation and classification models employing standard vision processing toolkits such as OpenCV, pyvips, and PyTorch. Additionally, QuPath employs Groovy scripts to initiate a Python process that starts the entire pipeline from QuPath graphical interface: fetching polygons, extracting images from them, and running deep learning model inference on the cropped images. 
The results are returned to QuPath, leveraging paquo's Python bindings to manipulate QuPath data while minimizing the computational overhead typically associated with cross-environment communication.

\counterwithin{figure}{subsection}
\renewcommand{\thefigure}{S\arabic{subsection}}

\begin{figure}[h!]
    \centering
    \includegraphics[width=\textwidth]{images/A7.pdf}
    \caption{QuPath integration workflow using Python environment}
    \label{fig:S7}
\end{figure}

Compared to traditional workflows that involve exporting annotations as GeoJSON, classifying them in Python, and reimporting them into QuPath, our approach offers several advantages. We eliminate the need to switch between programming languages, providing a cohesive and streamlined development process entirely within QuPath software and removing the necessity to use other tools. Meanwhile, we avoid storing annotations as intermediate JSON files unless required for external use or archiving. By conducting the entire inference and post-processing workflow within the Python environment, we leverage the power and flexibility of Python libraries for image processing and machine learning. This approach also enables adjustments to any set of labels and models, thereby improving its applicability.

%\hfill

The distilled model and QuPath integration code are packaged into a Docker container, enabling streamlined execution with the Docker engine. Detailed integration code and deployment instructions can be found in the GitHub repository \cite{Shvetsov_2025b}.

Despite these benefits, we acknowledge that the paquo library is a proof‑of‑concept project in its early development stage and has not been tested across all versions of QuPath.

\clearpage

\subsection{\label{chap:S8}Data and code availability statement}
All datasets, models, and code used in this study are publicly available and can be obtained from the repositories listed below. 
The PanNuke \cite{Gamper_Koohbanani_etal._2019} and MoNuSAC \cite{Verma_Kumar_etal._2021} datasets are publicly accessible, and download information along with detailed descriptions can be found in their respective articles. Preprocessing scripts for PanNuke and MoNuSAC data, as well as individual cell extraction scripts, are available on GitHub \cite{Shvetsov_2025a}. The H-Optimus foundation model used in our experiments can be downloaded from the HuggingFace repository \cite{hoptimus2024}, and model information is available on GitHub \cite{Saillard_Jenatton_etal._2024}. In addition, the integration code for QuPath and the distilled model packaged in a Docker container are provided in the repository \cite{Shvetsov_2025b}, and paquo Python library is available from the authors GitHub repository \cite{Bayer_AG}.
\clearpage

\end{document}





% \newpage
% \newpage

\appendix
\begingroup
% Adjust figure and table spacing
\setlength{\floatsep}{5pt} % Space between figures/tables
\setlength{\textfloatsep}{5pt} % Space between figures/tables and text
\setlength{\intextsep}{5pt} % Space for inline figures/tables

% Adjust table row spacing
\renewcommand{\arraystretch}{0.8} % Reduce table row spacing

% Adjust caption font and spacing
% \usepackage[font=small,labelfont=bf]{caption}
\captionsetup{skip=5pt} % Reduce space below captions

% Now write the appendix content (tables, figures, etc.)

% \section{Appendix Title}
% Insert your figures and tables here

\pagebreak
\section*{Appendix}



\begin{table}[H]
    \footnotesize
    \caption{Hyperparameter values.}
    \begin{tabular}{l p{5cm} c}
        \toprule
        \textbf{Hyperparameter} & \textbf{Description} & \textbf{Value} \\ 
        \midrule
        $\gamma$ & Reward discount factor & 0.99 \\
        $lr$ & Learning rate & 0.001 \\
        $\epsilon$ & Range for clipping the gradients & 0.2 \\
        n\_steps & Number of steps for each update & 2048 \\ 
        batch\_size & Minibatch size & 64 \\ 
        $\beta_{\text{value}}$ & Value function coefficient for the loss calculation & 1 \\ 
        return\_lookback & Number of last business days to calculate rolling return & 40 \\ 
        std\_lookback & Number of last business days to calculate standard deviation & 60 \\ 
        n\_steps\_foresee ($n$) & Number of business days the Oracle can see into the future & 14 \\ 
        $\alpha$ & Convexity parameter for transaction costs & 1 and 0.45 \\ 
        $R_f$ & Risk-free rate & 0 \\ 
        $\beta_{\text{entropy}}$ (starting) & Starting entropy coefficient in the loss function & 0.00005 \\ 
        $TC_{\text{eval}}$ & Maximum transaction fees & 0.0025 \\ 
        Actor Network & Sizes of the (hidden) layers in the network & 19, 64, 64, 3 \\ 
        Critic Network & Sizes of the (hidden) layers in the network & 19, 64, 64, 1 \\ 
        % \todo{NN architectures}
        \bottomrule
    \end{tabular}
    \label{tab:hp_values}
\end{table}



\begin{table}[H]

    \footnotesize
    \caption{Asset info.}
    \begin{tabular}{llccc}
        \toprule
        Name & Ticker Bloomberg  & \multicolumn{3}{c}{Strategy Weight} \\
        
        & & Strategy 1 & Strategy 2 & Strategy 3 \\
        \midrule
        Developed Equities & MXWOHEUR  & 1 & 0.55 & 0 \\
        Emerging equities & NDUEEGF   & 0 & 0.05 & 0 \\
        Global Credit & G0BC  & 0 & 0.2 & 0 \\
        Global Govies & W0G1  & 0 & 0.2 & 1 \\

        \bottomrule
    \end{tabular}
    \label{bloomberg_ticker}
\end{table}

\begin{table}[H]
  \footnotesize
  \caption{Timing phases with start and end dates for training, validation, and testing.}
  \centering
  \begin{tabular}{llccc} % Number of columns: 5 (including the phase labels and sub-labels)
    \toprule
    &  & Phase 1 & Phase 2 & Phase 3 \\

    &  & \scriptsize{(pre-pandemic)} & \scriptsize{(pandemic)} & \scriptsize{(post-pandemic)} \\
    \midrule
    \multirow{2}{*}{Train} & Start Date & 1996-02-01 & 2002-01-01 & 2009-01-01 \\
                           & End Date & 2012-01-01  & 2016-01-01  & 2018-01-01  \\
    
    \multirow{2}{*}{Valid} & Start Date & 2012-01-01 & 2016-01-01 & 2018-01-01 \\
                           & End Date & 2015-01-01  & 2020-01-01  & 2022-01-01  \\
    
    \multirow{2}{*}{Test} & Start Date & 2015-01-01 & 2020-01-01 & 2022-01-01 \\
                          & End Date & 2020-01-01  & 2022-01-01  & 2024-01-01 \\
    \bottomrule
    
  \end{tabular}

  \label{tab:timing-phases}
\end{table}

\begin{figure}[H]
  \centering
  \includegraphics[width=0.8\linewidth]{figures/tc_plot.pdf}
  \caption{\small Transaction cost schedule example.}
  \label{fig:TC_sch}
  \Description{Transaction cost schedule example.}
\end{figure}


\begin{figure}[H]
  \centering
  \begin{subfigure}{\linewidth}
    \centering
  \includegraphics[width=\linewidth]{figures/jpm_weights_test.png.pdf}
  % \caption{Example of allocation by Diff. Sharpe during the testing period of phase 3. }
  \label{fig:alloc_jpm}
  \end{subfigure}
  \begin{subfigure}{\linewidth}
    \centering
 \includegraphics[width=\linewidth]{figures/jpm_port_ret_test.png.pdf}
  % \caption{PPO with Sharpe Regret}
  \label{fig:rets_jpm}
  \end{subfigure}

  \caption{ \small  Example of allocation and return dynamics by \\ Diff. Sharpe (\ref{eq:diff_sharpe}) during the testing period of phase 3.}
  \Description{  Example of allocation and return dynamics by \\ Diff. Sharpe (\ref{eq:diff_sharpe}) during the testing period of phase 3.}
  \label{fig:two-allocs1}
  
\end{figure}





\begin{figure}[H]
  \centering
  \begin{subfigure}{\linewidth}
    \centering
  \includegraphics[width=\linewidth]{figures/comb_weights_test.png.pdf}
  % \caption{Example of allocation by Diff. Sharpe during the testing period of phase 3. }
  \label{fig:alloc_emb}
  \end{subfigure}
  \begin{subfigure}{\linewidth}
    \centering
 \includegraphics[width=\linewidth]{figures/comb_port_ret_test.png.pdf}
  % \caption{PPO with Sharpe Regret}
  \label{fig:rets_emb}
  \end{subfigure}

  \caption{\small  Example of allocation and return dynamics by \\ Emb. Drawdown (\ref{eq:emb_rew}) during the testing period of phase 3.}
  \Description{ Example of allocation and return dynamics by \\ Emb. Drawdown (\ref{eq:emb_rew}) during the testing period of phase 3.}
  \label{fig:two-allocs2}
  
\end{figure}

\begin{figure}[H]
  \centering
  \includegraphics[width=\linewidth]{figures/alloc_cnn.png}
  \caption{\small  Multi-Body CNN allocation example. We can observe noisy allocation behaviors. Introducing regularization to smooth the allocation line could enhance performance. For instance, adding a term to the reward function that penalizes sharp allocation changes—only to revert back—would indicate unnecessary transaction costs. Alternatively, we could include an integral of allocation changes as an additional penalty alongside our $\ell_1$ term.
  % \todo{CNN scheme Benamou? OR no NEED. NO SPACE}
  }
  \label{fig:alloc_cnn}
  \Description{ Multi-Body CNN allocation example. We can observe noisy allocation behaviors. Introducing regularization to smooth the allocation line could enhance performance. For instance, adding a term to the reward function that penalizes sharp allocation changes—only to revert back—would indicate unnecessary transaction costs. Alternatively, we could include an integral of allocation changes as an additional penalty alongside our $\ell_1$ term.}
\end{figure}



\section*{ \\ Glossary }
\label{sec:glossary}
\begin{scriptsize}
\begin{itemize}


    \item \textbf{Mean Return:} The expected return of an investment, denoted as:
    \[
    \mu = \mathbb{E}[R] = \begin{bmatrix}
        R_1 \\
        R_2 \\
        \vdots \\
        R_n
    \end{bmatrix}
    \]
    where \( R_i \) represents the financial return of asset \( i \).

    \item \textbf{Covariance Matrix:} A matrix that captures the variances and covariances of asset returns, represented as:
    \[
    \Sigma = 
    \begin{bmatrix}
        \sigma_1^2 & \sigma_{1,2} & \cdots & \sigma_{1,n} \\
        \sigma_{2,1} & \sigma_2^2 & \cdots & \sigma_{2,n} \\
        \vdots & \vdots & \ddots & \vdots \\
        \sigma_{n,1} & \sigma_{n,2} & \cdots & \sigma_n^2
    \end{bmatrix}
    \]
    where \( \sigma_i^2 \) is the variance of asset \( i \) and \( \sigma_{i,j} \) is the covariance between returns of assets \( i \) and \( j \).

    \item \textbf{Portfolio Return} ($\mu_p$): 
    \[
    \mu_p = w' \mu
    \]
    where $w$ is the vector of asset weights and $\mu$ is the return vector of individual assets.
  
\item \textbf{Portfolio Standard Deviation} ($\sigma_p$), \textit{Risk}:
    \[
    \sigma_p = \sqrt{w' \Sigma w}
    \]
    where $w$ is the asset weight vector and $\Sigma$ is the covariance matrix of asset returns.

    \item \textbf{Sharpe Ratio:} \label{eq:gl:sharpe} A measure of risk-adjusted return, calculated as:
    \[
    S = \frac{\mu_p - R_f}{\sigma_p}
    \]
    where \( \mu_p \) is the portfolio return, \( R_f \) is the risk-free rate, and \( \sigma_p \) is the standard deviation of the portfolio returns. The ratio represents return-risk trade-off.

    \item \textbf{Maximum Drawdown (MDD)}, \textit{Risk}: The maximum observed loss from a peak to a trough of a portfolio, calculated as:
    \[
    MDD = \max_{t} \left( \frac{P_{max} - P_t}{P_{max}} \right)
    \]
    where \( P_{max} \) is the peak portfolio value and \( P_t \) is the portfolio value at time \( t \).
\end{itemize}
\end{scriptsize}

\endgroup

%%%%%%%%%%%%%%%%%%%%%%%%%%%%%%%%%%%%%%%%%%%%%%%%%%%%%%%%%%%%%%%%%%%%%%%%

\end{document}

%%%%%%%%%%%%%%%%%%%%%%%%%%%%%%%%%%%%%%%%%%%%%%%%%%%%%%%%%%%%%%%%%%%%%%%%

