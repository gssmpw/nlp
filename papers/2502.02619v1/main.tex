%%%%%%%%%%%%%%%%%%%%%%%%%%%%%%%%%%%%%%%%%%%%%%%%%%%%%%%%%%%%%%%%%%%%%%%%

%%% LaTeX Template for AAMAS-2025 (based on sample-sigconf.tex)
%%% Prepared by the AAMAS-2025 Program Chairs based on the version from AAMAS-2025. 

%%%%%%%%%%%%%%%%%%%%%%%%%%%%%%%%%%%%%%%%%%%%%%%%%%%%%%%%%%%%%%%%%%%%%%%%

%%% Start your document with the \documentclass command.


%%% == IMPORTANT ==
%%% Use the first variant below for the final paper (including auithor information).
%%% Use the second variant below to anonymize your submission (no authoir information shown).
%%% For further information on anonymity and double-blind reviewing, 
%%% please consult the call for paper information
%%% https://aamas2025.org/index.php/conference/calls/submission-instructions-main-technical-track/

%%%% For anonymized submission, use this
\documentclass[sigconf]{aamas} 

%%%% For camera-ready, use this
%\documentclass[sigconf]{aamas} 


%%% Load required packages here (note that many are included already).

\usepackage{balance} % for balancing columns on the final page
% \usepackage{biblatex}
\usepackage{graphicx}
\usepackage{xcolor}
\usepackage{listings}
% \addbibresource{report.bib}
\usepackage{amsmath} % for bmatrix environment
\usepackage{graphicx}
\usepackage{geometry}
\usepackage{graphicx, caption, subcaption, booktabs, multirow, float}

\usepackage{amsmath}
% \usepackage{algorithm}
% \usepackage{algorithmic}
\usepackage{algorithm}
\usepackage{algpseudocode}
% \usepackage[linesnumbered,ruled,vlined]{algorithm2e}
% \usepackage[font=small,labelfont=bf]{caption}

\usepackage{lineno}
\usepackage{hyperref}

\DeclareMathOperator*{\argmax}{arg\,max}
%%%%%%%%%%%%%%%%%%%%%%%%%%%%%%%%%%%%%%%%%%%%%%%%%%%%%%%%%%%%%%%%%%%%%%%%

%%% AAMAS-2025 copyright block (do not change!)

% \setcopyright{ifaamas}
% \acmConference[AAMAS '25]{Proc.\@ of the 24th International Conference
% on Autonomous Agents and Multiagent Systems (AAMAS 2025)}{May 19 -- 23, 2025}
% {Detroit, Michigan, USA}{A.~El~Fallah~Seghrouchni, Y.~Vorobeychik, S.~Das, A.~Nowe (eds.)}
% \copyrightyear{2025}
% \acmYear{2025}
% \acmDOI{}
% \acmPrice{}
% \acmISBN{}


%%%%%%%%%%%%%%%%%%%%%%%%%%%%%%%%%%%%%%%%%%%%%%%%%%%%%%%%%%%%%%%%%%%%%%%%

%%% == IMPORTANT ==
%%% Use this command to specify your submission number.
%%% In anonymous mode, it will be printed on the first page.

% \acmSubmissionID{226}

%%% Use this command to specify the title of your paper.

% \title[AAMAS-2025 Formatting Instructions]{Enhancing Traditional 60/40 Portfolio Through PPO With Future Looking Regret-Based Reward and Synthetic Data}

\title[AAMAS-2025 Formatting Instructions]{Regret-Optimized Portfolio Enhancement through \\Deep Reinforcement Learning and Future Looking Rewards}


%%% Provide names, affiliations, and email addresses for all authors.


\author{Daniil Karzanov}
% \orcid{0000-0003-1767-9649}


\affiliation{%
  \institution{AXA Group Operations, EPFL}
  \city{Lausanne}
  \country{Switzerland}}
  \email{daniil.karzanov@axa.com}


\author{Rubén Garzón}
% \orcid{0000-0002-5695-2495}

\affiliation{%
  \institution{AXA Group Operations}
  \city{Madrid}
  \country{Spain}}
\email{ruben.garzon@axa.com}


\author{Mikhail Terekhov}
% \orcid{0009-0005-7403-3731}
% \authornotemark[1]


\affiliation{%
  \institution{CLAIRE EPFL}
  \city{Lausanne}
  \country{Switzerland}
}
\email{mikhail.terekhov@epfl.ch}

\author{Caglar Gulcehre}
% \orcid{0009-0003-4124-1687}
% \authornote{Both authors contributed equally to this research.}

\affiliation{%
  \institution{CLAIRE EPFL}
  \city{Lausanne}
  \country{Switzerland}
}
\email{caglar.gulcehre@epfl.ch}

\author{Thomas Raffinot}
% \authornote{Both authors contributed equally to this research.}
% \orcid{0000-0003-2338-3596}

\affiliation{%
  \institution{ AXA Investment Managers}
  \city{Paris}
  \country{France}
}
\email{thomas.raffinot@axa-im.com}


\author{Marcin Detyniecki}
% \orcid{0000-0001-5669-4871}
% \authornotemark[1]

\affiliation{%
  \institution{ AXA Group Operations}
  \city{Paris}
  \country{France}
}
\email{marcin.detyniecki@axa.com}

% \author{Arthur Pendragon}
% \affiliation{
%   \institution{Camelot Castle}
%   \city{Camelot}
%   \country{United Kingdom}}
% \email{king.arthur@camelot.uk}

% \author{Nimue}
% \affiliation{
%   \institution{The Lady's Lake}
%   \city{Avalon}
%   \country{United Kingdom}}
% \email{lady.of.the.lake@avalon.uk}

%%% Use this environment to specify a short abstract for your paper.

\begin{abstract} {


This paper introduces a novel agent-based approach for enhancing existing portfolio strategies using Proximal Policy Optimization (PPO). Rather than focusing solely on traditional portfolio construction, our approach aims to improve an already high-performing strategy through dynamic rebalancing driven by PPO and Oracle agents. Our target is to enhance the traditional 60/40 benchmark (60\% stocks, 40\% bonds) by employing the Regret-based Sharpe reward function. To address the impact of transaction fee frictions and prevent signal loss, we develop a transaction cost scheduler. We introduce a future-looking reward function and employ synthetic data training through a circular block bootstrap method to facilitate the learning of generalizable allocation strategies. We focus on two key evaluation measures: return and maximum drawdown. Given the high stochasticity of financial markets, we train 20 independent agents each period and evaluate their average performance against the benchmark. Our method not only enhances the performance of the existing portfolio strategy through strategic rebalancing but also demonstrates strong results compared to other baselines.




% Strategic asset allocation involves setting specific target allocations for various asset classes and periodically rebalancing the portfolio to maintain these targets. Deep reinforcement learning is increasingly becoming a valuable tool for systematic portfolio managers.
% This paper introduces a practical agent-based approach for dynamic portfolio construction using well-established Proximal Policy Optimization (PPO). The portfolio construction problem involves selecting an optimal combination of financial assets that maximizes returns while minimizing risk, based on factors like market dynamics, asset correlations, and investor preferences. Given the high stochasticity of financial markets, we train 20 independent agents each period and evaluate their average performance against the benchmark. Our aim is to enhance the traditional 60/40 benchmark (60\% stocks, 40\% bonds) by employing the Regret-based Sharpe reward function. To address the impact of transaction fee frictions and prevent signal loss, we develop a transaction cost scheduler. We introduce a future-looking reward function and employ synthetic data training through a circular block bootstrap method to facilitate the learning of generalizable allocation strategies. We focus on two key evaluation measures: return and maximum drawdown. Our agent demonstrates prominent results, achieving both objectives and generally outperforming the baseline.
% This work presents a novel application of DRL not just for portfolio construction, but for enhancing an already high-performing target portfolio through slight rebalancing, leveraging agent-based frameworks to achieve further improvements.

% \todo{This paper introduces a novel agent-based approach for enhancing existing portfolio strategies using Proximal Policy Optimization (PPO). Rather than focusing solely on traditional portfolio construction, our approach aims to improve an already high-performing strategy through dynamic rebalancing. Given the high stochasticity of financial markets, we train 20 independent agents each period and evaluate their average performance against the benchmark. Our target is to enhance the traditional 60/40 benchmark (60\% stocks, 40\% bonds) by employing the Regret-based Sharpe reward function. To address the impact of transaction fee frictions and prevent signal loss, we develop a transaction cost scheduler. We introduce a future-looking reward function and employ synthetic data training through a circular block bootstrap method to facilitate the learning of generalizable allocation strategies. We focus on two key evaluation measures: return and maximum drawdown. By leveraging DRL, our agent not only enhances the performance of the existing portfolio strategy through strategic rebalancing but also demonstrates superior results compared to the baseline.}


}
\end{abstract}


%%% The code below was generated by the tool at http://dl.acm.org/ccs.cfm.
%%% Please replace this example with code appropriate for your own paper.

\newcommand{\com}[1]{\textcolor{orange}{\MakeTextUppercase{#1}}}

\newcommand{\todo}[1]{\textcolor{red}{\MakeTextUppercase{#1}}}
%%% Use this command to specify a few keywords describing your work.
%%% Keywords should be separated by commas.

\keywords{Deep Reinforcement Learning, Proximal Policy Optimization (PPO), Dynamic Portfolio Construction, Regret-Based Reward Function, Synthetic Data Training, Circular Block Bootstrap, Transaction Cost Scheduling, Machine Learning in Finance, Risk Management, Return Maximization, Multi-Objective Optimization, Computational Finance}

%%%%%%%%%%%%%%%%%%%%%%%%%%%%%%%%%%%%%%%%%%%%%%%%%%%%%%%%%%%%%%%%%%%%%%%%

%%% Include any author-defined commands here.
         
\newcommand{\BibTeX}{\rm B\kern-.05em{\sc i\kern-.025em b}\kern-.08em\TeX}

%%%%%%%%%%%%%%%%%%%%%%%%%%%%%%%%%%%%%%%%%%%%%%%%%%%%%%%%%%%%%%%%%%%%%%%%




\settopmatter{printacmref=false}
% \settopmatter{printacmref=false, printccs=false, printfolios=false, printcopyrightpermission=false}
\renewcommand\footnotetextcopyrightpermission[1]{}
\setcopyright{none}
% \pagenumbering{arabic}
% \pagestyle{plain}

\begin{document}

%%% The following commands remove the headers in your paper. For final 
%%% papers, these will be inserted during the pagination process.

\pagestyle{fancy}
\fancyhead{}

%%% The next command prints the information defined in the preamble.

\maketitle 

%%%%%%%%%%%%%%%%%%%%%%%%%%%%%%%%%%%%%%%%%%%%%%%%%%%%%%%%%%%%%%%%%%%%%%%%


\section{Introduction}
% \linenumbers


% \begin{figure*}[ht]
%     \centering
%     \begin{subfigure}[b]{0.32\textwidth}
%         \centering
%         \includegraphics[height=5.5cm]{figures/novelt_train_c.png}
%         \caption{Train}
%     \end{subfigure}
%     \hfill
%     \begin{subfigure}[b]{0.32\textwidth}
%         \centering
%         \includegraphics[height=5.5cm]{figures/novelt_valid_c.png}
%         \caption{Validation}
%     \end{subfigure}
%     \hfill
%     \begin{subfigure}[b]{0.32\textwidth}
%         \centering
%         \includegraphics[height=5.5cm]{figures/novelt_test_c.png}
%         \caption{Test}
%     \end{subfigure}
%     \caption{The evolution of accumulated return over training. Ablation: removal of specific model components to assess their impact. TC: inclusion of transaction cost schedule. BB: block bootstrap synthetic data. Regret: our reward function. Return: optimizing current portfolio return as a reward. Averaged over 20 runs each. The shaded area represents uncertainty around the average trend.}
%     \label{fig:novelt}
% \end{figure*}







Strategic asset allocation is a portfolio strategy whereby the investor sets target allocations to various asset classes and rebalances the portfolio periodically. The landscape of finance is continually evolving, driven by the need for more sophisticated and adaptive investment strategies that use machine learning for decision making \cite{de2020machine, karzanov2023headline, Blohm2020ItsAP, hambly2023recent}. Traditional methods of portfolio construction often struggle to keep pace with the dynamic nature of financial markets. This manuscript addresses this challenge by exploring the integration of advanced machine learning techniques to enhance the process of dynamic portfolio construction and rebalancing. By leveraging deep reinforcement learning algorithms (DRL) such as Proximal Policy Optimization (PPO) \cite{schulman2017proximal}  with a future-looking reward function, we aim to demonstrate how RL can optimize financial portfolios, balance risk, and maximize returns in an ever-changing market environment.


The primary objective is to achieve superior returns, while also ensuring that the portfolio's value does not experience significant declines. This secondary objective is measured by the maximum drawdown (MDD). Essentially, our goal is to develop an agent that finds the optimal balance between two inversely related metrics: portfolio return and risk. In this context, we consider MDD to be a more suitable risk measure than portfolio standard deviation, which is commonly used in traditional Markowitz-like approaches \cite{mpt}. MDD is a measure of the maximum observed loss from a peak to a trough in a portfolio's value \cite{chekhlov2005drawdown}, before the portfolio reaches a new peak. It provides an indication of the worst possible loss an investor could have experienced during a specific period. Additionally, we consider the Sharpe Ratio, defined as the ratio of the portfolio's excess return to its standard deviation, to evaluate the risk-adjusted return of the portfolio, allowing for a comprehensive assessment of performance relative to the risk taken. Refer to the \hyperref[sec:glossary]{glossary} in the appendix for details on the Sharpe Ratio and other financial notation and metrics used.

The baseline for comparison is the static allocation of weights based on the conventional 60/40 portfolio (60\% stocks, 40\% bonds). Our overarching aim is to surpass the traditional 60/40 portfolio benchmark in both metrics using a dynamic and adaptive DRL approach. 


The novel contributions of this work can be summarized in the following three points: 
\begin{enumerate}
    \item[1.] Introduction of a negative Sharpe regret reward function that leverages Oracle agent's knowledge to encourage optimal allocation and improve out-of-sample performance.
    \item[2.] Integration of real and synthetic data in the training process through a circular block bootstrap method to enhance historical data, thereby accelerating the learning of effective strategies and improving the model’s capacity to extrapolate and generalize beyond observed data trajectories.
    \item[3.] Incorporation of a transaction costs scheduler during training, allowing for the inclusion of frictions that are often overlooked in other studies.
\end{enumerate}


In this paper, we first discuss the relevant literature related to the application of DRL in finance, highlighting key advancements and methodologies. This includes a review of two other benchmark methods used for comparison in our evaluation. We then describe the reinforcement learning framework and the specifics of Proximal Policy Optimization (PPO). In Section 3.2, we delve into the design of our environment, transforming real price data to make it suitable for PPO's learning. This includes the introduction of our novel reward function that incorporates a transaction cost (TC) scheduler, the use of synthetic data generation during training, and other implementation details for the agent. Finally, we provide a discussion of the results, comparing our approach with the benchmark method and two other approaches from the literature, and evaluating different configurations of the agent. We conclude with several noteworthy ideas for future improvements and extensions of this work.

The methodology described in this manuscript can be applied to many other sequential allocation problems, not just financial portfolio construction. Examples include resource allocation in supply chains, task assignment in project management, and bandwidth distribution in telecommunications networks. Each of these scenarios involves distributing limited resources or capacities across various options to achieve optimal outcomes.


\section{Relevant Literature}

Advances in RL and deep RL have greatly impacted the field of AI and ML. Mnih et al. \cite{mnih2015human} achieved human-level control in complex games by leveraging deep Q-networks. Schulman et al. \cite{schulman2017proximal} introduced Proximal Policy Optimization algorithms, addressing stability issues in policy learning. Silver et al. \cite{silver2016mastering} demonstrated beyond human performance in the game of Go through a combination of deep neural networks and tree search.

Numerous studies have explored the application of deep reinforcement learning to both single-asset trading \cite{kochliaridis2023combining, pigorsch2022high, brini2023deep} and portfolio optimization problems \cite{srivastava2020deep, halperin_combining_2022, lu2023evaluation, jiang2017deep}. For example, Benhamou et al. \cite{benhamou2020bridging} employed a policy gradient method to dynamically allocate several systematic strategies. Similarly, Kochliaridis et al. \cite{kochliaridis2023combining} utilized a future-looking reward function that considers future close prices over the next K timesteps to encourage agents to predict future market dynamics in addition to solving the primary optimization problem. Our work incorporates a similar concept by embedding Oracle knowledge (defined as $w^*$ in equation (\ref{eq:regret_weight}) and detailed in Section \ref{sec:reward_function}) into the reward function during the training process. In their Investor-Imitator framework, Ding et al. \cite{ding2018investor} consider an Oracle investor, employing Sharpe and MDD as key evaluation measures.

Another recent study by Sood et al. \cite{sood2023deep} applies a PPO model in a setting similar to ours, utilizing a Differential-Sharpe reward function (\ref{eq:diff_sharpe}) and exponential moving averages, $A_t$, $B_t$ of returns and their standard deviation. The study achieves impressive results, with the agent outperforming traditional mean-variance optimization techniques in both return and drawdown metrics, and also demonstrating a more stable out-of-sample strategy. We adopt the reward function introduced in their paper as one of the benchmarks for comparison in our study.


\begin{equation}
D_t = \frac{\delta S_t}{\delta \eta} = \frac{B_{t-1}\Delta A_t - \frac{1}{2}A_{t-1}\Delta B_t}{(B_{t-1} - A_{t-1}^2)^{3/2}}
\label{eq:diff_sharpe}
\end{equation}
where
\begin{itemize}
    \item $D_t$: Differential Sharpe Ratio at time $t$.
    \item $\delta S_t$: Change in the Sharpe ratio, which is adjusted over time as new information is incorporated.
    \item $\eta$: Incremental time step for daily returns, approximately $\frac{1}{252}$.
    \item $A_t$: Cumulative mean return updated over time, calculated as $A_t = A_{t-1} + \eta \Delta A_t$.
    \item $B_t$: Cumulative second moment (variance) of returns, updated as $B_t = B_{t-1} + \eta \Delta B_t$.
    \item $\Delta A_t$: Change in cumulative mean return, defined as $\Delta A_t = R_t - A_{t-1}$.
    \item $\Delta B_t$: Change in cumulative second moment, defined as $\Delta B_t = R_t^2 - B_{t-1}$.
    \item $R_t$: Return at time $t$.
    \item $A_0$ and $B_0$: Initial values for cumulative mean return and cumulative second moment, both set to 0.
\end{itemize}


Andersson et al. \cite{andersson2023measuring} explore the application of regret theory in financial decision-making, particularly focusing on how regret aversion can influence investor behavior. 

Several studies explore the use of synthetic data in portfolio construction. Pe{\~n}a et el. \cite{pena2024modified} introduce a novel portfolio optimization method using synthetic data generated by a Modified CTGAN algorithm. Similarly, Pagnoncelli et el. \cite{pagnoncelli2023synthetic} demonstrate the advantages of using augmented synthetic data for asset allocation.

Few studies consider the multi-objective case in this niche, which inadequately addresses the problem where investors aim to minimize risk \cite{vcernevivciene2022review}. Bisht et al. \cite{bisht2020deep} and Cornalba et al. \cite{cornalba2024multi} attempt to tackle this issue with multi-objective approaches. Almahdi et al.\cite{almahdi2017adaptive} combine the analysis of both expected maximum drawdown and transaction costs. Similarly, Wu et al. \cite{wu2022embedded} incorporate drawdown into their reward function (\ref{eq:emb_rew}), which prioritizes MDD. If the drawdown level at time $t$, $MDD_t$, exceeds $\alpha$, the desired MDD level, the term in brackets becomes negative, serving as a penalty mechanism.  If the return tends to infinity, the first multiplier tends to the hyperparameter value $k$. Although their study focuses more on trading than on portfolio construction, it offers a practical approach by explicitly including drawdown in the reward function. Consequently, we use their methodology as one of the baselines for our study, using a dynamic $\alpha$ equal to the MDD of the 60/40 benchmark.

\begin{equation}
\text{\textbf{Reward}}_t = \frac{k}{1 + e^{-r}} (-e^{MDD_t} + e^{\alpha})
\label{eq:emb_rew}
\end{equation}

Except for a few studies \cite{lucarelli2020deep, lucarelli2019deep, jiang2017deep}, most papers neglect transaction costs and other similar frictions in their analysis. In contrast, we introduce a transaction cost (TC) term and devise an elegant method to integrate it into the learning process without attenuating the contribution of the main reward during exploration (e.g. instantaneous return or Sharpe without TC term).
% "killing" the signal.




\begin{figure*}[ht]
    \centering

    \includegraphics[width=\textwidth]{figures/abbl.pdf}
    \caption{The evolution of accumulated (financial) return over training. Ablation study: removal of specific model components to assess their impact. TC: inclusion of transaction cost schedule. BB: block bootstrap synthetic data. Regret: our reward function. Return: purely optimizing for returns, without including MDD or risk in the reward. Each configuration is averaged over 20 runs. \\
    The full configuration (TC + BB + Regret) generalizes better despite underperformance during training due to increased variability. Synthetic data functions as a form of regularization, preventing memorization of non-reproducible strategies. When used alone, BB and TC are less effective than when combined. However, the combination of TC, BB, and the Return reward function (which specifically optimizes the value displayed on the Y-axis of the plot) tends to overfit on train, resulting in an inability to generate profitable out-of-sample strategies.}
    \label{fig:novelt}
    \Description{The evolution of accumulated (financial) return over training. Ablation study: removal of specific model components to assess their impact. TC: inclusion of transaction cost schedule. BB: block bootstrap synthetic data. Regret: our reward function. Return: purely optimizing for returns, without including MDD or risk in the reward. Each configuration is averaged over 20 runs. \\
    The full configuration (TC + BB + Regret) generalizes better despite underperformance during training due to increased variability. Synthetic data functions as a form of regularization, preventing memorization of non-reproducible strategies. When used alone, BB and TC are less effective than when combined. However, the combination of TC, BB, and the Return reward function (which specifically optimizes the value displayed on the Y-axis of the plot) tends to overfit on train, resulting in an inability to generate profitable out-of-sample strategies.}
\end{figure*}




\section{Background}
\subsection{Portfolio Optimization}
Portfolio optimization plays a crucial role in financial management, focusing on the allocation of assets to maximize returns while effectively managing risk. Mathematical portfolio optimization seeks to allocate assets in a way that maximizes expected return while minimizing risk. Risk can be represented by various measures, such as portfolio standard deviation or MDD, both of which are typically inversely related to expected return. The classic formulation is based on mean-variance single-period optimization, introduced by Markowitz \cite{mpt}. The goal is to solve the trade-off between portfolio risk $w' \Sigma w$ and return $w' \mu$:

\begin{equation}
\min_{w} \frac{1}{2} w' \Sigma w - \lambda w' \mu
\end{equation}

where \(w\) represents a vector of asset weights, denoting the proportion of the portfolio invested in each asset, \(\Sigma\) is the covariance matrix of asset returns, \(\mu\) is the vector of expected returns, and \(\lambda\) is a risk-aversion parameter. Constraints such as \(\sum w_i = 1\) (full investment) and \(w_i \geq 0\) (no short-selling) are typically included.
Classical portfolio optimization approaches rely solely on historical data and do not incorporate forecasting, unlike neural networks. We aim to apply reinforcement learning to address the multi-period portfolio optimization problem, which involves making investment decisions over several time periods (e.g. daily) to adapt to and predict changing market conditions. This approach allows for dynamically rebalancing of the portfolio based on current asset information ( \(\mu\) and \(\Sigma\)) and relevant exogenous predictors (discussed in section \ref{ch:obs_space}).


\subsection{Reinforcement Learning}

Reinforcement learning is a subfield of machine learning where an agent learns to make decisions by interacting with an environment to maximize cumulative rewards \cite{sutton2018reinforcement}. The agent operates by taking actions $w_t$\footnote{In traditional reinforcement learning terminology, $w_t$ is referred to as $a_t$.} at each time step $t$, receiving a state $s_t$ from the environment, and obtaining a reward $r_t$. The goal of the agent is to learn a policy $\pi$ that maximizes the expected cumulative reward, defined as the return $G_t$, over time.


The return $G_t$ is the total discounted reward from time step $t$ onwards and is given by:
\begin{equation}
G_t = \sum_{k=0}^{\infty} \gamma^k r_{t+k+1},
\end{equation}
where $\gamma$ (0 $\leq \gamma < 1$) is the discount factor, which determines the present value of future rewards. The expected return, or the value function $V(s)$, under a policy $\pi$ is defined as:
\begin{equation}
V^\pi(s) = \mathbb{E}_\pi [G_t | s_t = s].
\end{equation}
The objective in reinforcement learning is to find an optimal policy $\pi^*$ that maximizes the value function $V(s)$ for all states $s$.

In the context of dynamic portfolio construction, the reinforcement learning agent's task is to dynamically adjust the portfolio weights based on the observed market states $s_t$ to maximize the cumulative reward $G_t$. The exact form of the reward function $r_t$ may vary and its design may result in different agent behaviors.

\subsection{Proximal Policy Optimization }
Proximal Policy Optimization (PPO) is one of the most effective and widely used algorithms in reinforcement learning. It was introduced as a method to improve both the stability and performance of policy gradient methods. PPO achieves this by using a surrogate objective function, which helps in balancing the trade-off between exploration and exploitation. The core idea behind PPO is to ensure that the new policy does not diverge too much from the old policy during training. This is accomplished through a mechanism known as clipping. Specifically, PPO optimizes the following objective:
\begin{equation}
    \mathcal{L}^{CLIP}(\theta) = \mathbb{E}_t \left[ \min \left( r_t(\theta) \hat{A}_t, \text{clip}(r_t(\theta), 1 - \epsilon, 1 + \epsilon) \hat{A}_t \right) \right].
\end{equation}
In this objective, $r_t(\theta) = \frac{\pi_\theta(w_t | s_t)}{\pi_{\theta_{\text{old}}}(w_t | s_t)}$ represents the probability ratio between the new policy $\pi_\theta$ and the old policy $\pi_{\theta_{\text{old}}}$. The term $\hat{A}_t$ is the advantage estimate, which indicates how much better the current action is compared to the expected action. The hyperparameter $\epsilon$ controls the clipping range, thereby ensuring that the updates to the policy are conservative and stable. The clipping mechanism is crucial as it prevents excessively large updates that can destabilize the learning process. By limiting the change to a specified range, PPO maintains a balance between optimizing the policy and keeping the updates within a reasonable bound.

PPO has been shown to be robust and effective in a variety of complex environments, making it a popular choice for tasks that require dynamic decision-making and adaptation, such as dynamic portfolio construction. Its ability to maintain stability while still performing well in challenging scenarios is a key reason for its widespread adoption in the reinforcement learning community.






% \begin{table*}[ht]
% \caption{Performance comparison of the 60/40 benchmark (B) and the PPO agent (A).}
% \centering

% %% ---------------


% \begin{tabular}{lllllllllll}
% \toprule
%  &  & \multicolumn{3}{c}{Train} & \multicolumn{3}{c}{Validation} & \multicolumn{3}{c}{Test} \\
%  &  & Phase 1 & Phase 2 & Phase 3 & Phase 1 & Phase 2 & Phase 3 & Phase 1 & Phase 2 & Phase 3 \\
%   &  & \scriptsize{(pre-pandemic)} & \scriptsize{(pandemic)} & \scriptsize{(post-pandemic)} & \scriptsize{(pre-pandemic)} & \scriptsize{(pandemic)} & \scriptsize{(post-pandemic)} & \scriptsize{(pre-pandemic)} & \scriptsize{(pandemic)} & \scriptsize{(post-pandemic) } \\
% \midrule
% \multirow[t]{2}{*}{Annual return} & A & \textbf{0.071} & \textbf{0.064} & \textbf{0.091} & \textbf{0.128} & 0.07 & \textbf{0.093} & \textbf{0.064} & \textbf{0.128} & \textbf{-0.007} \\
%  & B & 0.05 & 0.05 & 0.08 & 0.10 & \textbf{0.076} & 0.08 & 0.06 & 0.10 & -0.03 \\
% \cline{1-11}
% \multirow[t]{2}{*}{Calmar ratio} & A & \textbf{0.192} & \textbf{0.185} & 0.43 & \textbf{1.602} & 0.45 & \textbf{0.364} & \textbf{0.607} & 0.48 & \textbf{-0.035} \\
%  & B & 0.14 & 0.14 & \textbf{0.475} & 1.40 & \textbf{0.625} & 0.35 & 0.46 & \textbf{0.486} & -0.11 \\
% \cline{1-11}
% \multirow[t]{2}{*}{Max drawdown} & A & -0.39 & \textbf{-0.361} & -0.21 & -0.08 & -0.15 & -0.26 & \textbf{-0.108} & -0.27 & \textbf{-0.205} \\
%  & B & \textbf{-0.384} & -0.38 & \textbf{-0.171} & \textbf{-0.069} & \textbf{-0.121} & \textbf{-0.216} & -0.12 & \textbf{-0.216} & -0.23 \\
% \cline{1-11}
% \multirow[t]{2}{*}{Omega ratio} & A & \textbf{1.198} & \textbf{1.173} & 1.26 & \textbf{1.488} & 1.24 & 1.24 & \textbf{1.246} & 1.28 & \textbf{0.999} \\
%  & B & 1.16 & 1.16 & \textbf{1.266} & 1.42 & \textbf{1.331} & \textbf{1.24} & 1.22 & \textbf{1.286} & 0.96 \\
% \cline{1-11}
% \multirow[t]{2}{*}{Sharpe ratio} & A & \textbf{0.614} & \textbf{0.593} & 0.88 & \textbf{1.558} & 0.84 & 0.76 & \textbf{0.857} & 0.84 & \textbf{-0.005} \\
%  & B & 0.54 & 0.57 & \textbf{0.938} & 1.45 & \textbf{1.181} & \textbf{0.776} & 0.80 & \textbf{0.868} & -0.18 \\
% \cline{1-11}
% \multirow[t]{2}{*}{Sortino ratio} & A & \textbf{0.904} & \textbf{0.843} & 1.29 & \textbf{2.485} & 1.21 & 1.01 & \textbf{1.178} & 1.12 & \textbf{-0.006} \\
%  & B & 0.77 & 0.78 & \textbf{1.357} & 2.19 & \textbf{1.748} & \textbf{1.04} & 1.10 & \textbf{1.164} & -0.24 \\
% \cline{1-11}
% \multirow[t]{2}{*}{Stability} & A & \textbf{0.842} & \textbf{0.885} & \textbf{0.945} & \textbf{0.96} & 0.84 & 0.75 & \textbf{0.899} & 0.83 & \textbf{0.057} \\
%  & B & 0.82 & 0.85 & 0.93 & 0.96 & \textbf{0.857} & \textbf{0.805} & 0.87 & \textbf{0.839} & 0.00 \\
% \cline{1-11}
% \multirow[t]{2}{*}{Tail ratio} & A & \textbf{1.081} & \textbf{1.082} & \textbf{1.161} & \textbf{1.314} & 1.08 & 1.06 & \textbf{1.168} & 1.19 & \textbf{0.99} \\
%  & B & 1.06 & 1.08 & 1.10 & 1.16 & \textbf{1.186} & \textbf{1.076} & 1.07 & \textbf{1.315} & 0.92 \\
% \cline{1-11}
% \bottomrule
% \end{tabular}






% %% ----------------


% %% x-x-x-x-x-x-x-x-x-x-x


% \label{tab:results1}
% \end{table*}





\begin{table*}[ht]
\caption{Performance comparison of the 60/40 benchmark and the PPO agent with differential Sharpe  eq. (\ref{eq:diff_sharpe}), embedded drawdown eq. (\ref{eq:emb_rew}) and negative Sharpe regret (ours) reward functions. 
% \todo{MOVE THIS AND ABLATIONS FIG 1 LOWER?}
}
\centering
\footnotesize







\begin{tabular}{lllllllllll}
\toprule
 &  & \multicolumn{3}{c}{train} & \multicolumn{3}{c}{valid} & \multicolumn{3}{c}{test} \\
  &  & Phase 1 & Phase 2 & Phase 3 & Phase 1 & Phase 2 & Phase 3 & Phase 1 & Phase 2 & Phase 3 \\
  &  & \scriptsize{(pre-pandemic)} & \scriptsize{(pandemic)} & \scriptsize{(post-pandemic)} & \scriptsize{(pre-pandemic)} & \scriptsize{(pandemic)} & \scriptsize{(post-pandemic)} & \scriptsize{(pre-pandemic)} & \scriptsize{(pandemic)} & \scriptsize{(post-pandemic) } \\
\midrule

\multirow[t]{4}{*}{Annual return} & 60/40 & 0.052 & 0.054 & 0.081 & 0.096 & 0.076 & 0.077 & 0.056 & 0.105 & -0.026 \\
 & Diff. Sharpe & \textbf{0.054} & 0.054 & 0.076 & 0.083 & 0.066 & 0.077 & 0.053 & 0.093 & \textbf{-0.024} \\
 & Emb. DD & \textbf{0.056} & 0.051 & 0.057 & 0.042 & 0.062 & 0.055 & 0.041 & 0.072 & -0.038 \\
 & Regret & \textbf{0.071} & \textbf{0.064} & \textbf{0.091} & \textbf{0.128} & 0.069 & \textbf{0.093} & \textbf{0.064} & \textbf{0.128} & \textbf{-0.007} \\
\cline{1-11}

\multirow[t]{4}{*}{Sharpe ratio} & 60/40 & 0.541 & 0.568 & 0.938 & 1.447 & 1.181 & 0.776 & 0.803 & 0.868 & -0.183 \\
 & Diff. Sharpe & \textbf{0.603} & \textbf{0.658} & \textbf{0.97} & 1.405 & \textbf{1.21} & \textbf{0.833} & \textbf{0.869} & \textbf{0.876} & \textbf{-0.179} \\
 & Emb. DD & \textbf{0.849} & \textbf{0.742} & 0.936 & 1.005 & \textbf{1.362} & \textbf{0.847} & \textbf{0.923} & 0.832 & -0.368 \\
 & Regret & \textbf{0.614} & \textbf{0.593} & 0.882 & \textbf{1.558} & 0.841 & 0.756 & \textbf{0.857} & 0.844 & \textbf{-0.005} \\
\cline{1-11}

\multirow[t]{4}{*}{Calmar ratio} & 60/40 & 0.136 & 0.142 & 0.475 & 1.399 & 0.625 & 0.355 & 0.461 & 0.486 & -0.114 \\
 & Diff. Sharpe &  \textbf{0.159} &  \textbf{0.180} &0.442 &  \textbf{1.453} &  \textbf{0.631} &  \textbf{0.413} &  \textbf{0.512} &  \textbf{0.526} &  \textbf{-0.109} \\
 & Emb. DD &  \textbf{0.257} &  \textbf{0.205} & 0.44 & 0.983 &  \textbf{0.756} &  \textbf{0.454} &  \textbf{0.560} &  \textbf{0.535} & -0.17 \\
 & Regret &  \textbf{0.192} &  \textbf{0.185} & 0.429 &  \textbf{1.602} & 0.451 &  \textbf{0.364} &  \textbf{0.607} & 0.483 &  \textbf{-0.035} \\
\cline{1-11}

% \multirow[t]{4}{*}{Stability} & 60/40 & 0.822 & 0.852 & 0.932 & 0.960 & 0.857 & 0.805 & 0.866 & 0.839 & 0.002 \\
%  & Diff. Sharpe & \textbf{0.851} & \textbf{0.902} & \textbf{0.947} & \textbf{0.961} & \textbf{0.864} & \textbf{0.845} & \textbf{0.889} & \textbf{0.848} & \textbf{0.017} \\
%  & Emb. DD & \textbf{0.94} & \textbf{0.915} & 0.926 & 0.905 & 0.854 & \textbf{0.884} & \textbf{0.877} & 0.814 & \textbf{0.083} \\
%  & Regret & \textbf{0.842} & \textbf{0.885} & \textbf{0.945} & 0.960 & 0.836 & 0.750 & \textbf{0.899} & 0.833 & \textbf{0.057} \\
% \cline{1-11}

\multirow[t]{4}{*}{Max drawdown} & 60/40 & -0.384 & -0.382 & -0.171 & -0.069 & -0.121 & -0.216 & -0.122 & -0.216 & -0.225 \\
 & Diff. Sharpe & \textbf{-0.338} & \textbf{-0.303} & -0.173 & \textbf{-0.057} & \textbf{-0.105} & \textbf{-0.187} & \textbf{-0.104} & \textbf{-0.178} & \textbf{-0.215} \\
 & Emb. DD & \textbf{-0.225} & \textbf{-0.256} & \textbf{-0.129} & \textbf{-0.043} & \textbf{-0.083} & \textbf{-0.122} & \textbf{-0.073} & \textbf{-0.134} & \textbf{-0.224} \\
 & Regret & -0.387 & \textbf{-0.361} & -0.211 & -0.080 & -0.152 & -0.255 & \textbf{-0.108} & -0.266 & \textbf{-0.205} \\
\cline{1-11}

\multirow[t]{4}{*}{Omega ratio} & 60/40 & 1.156 & 1.158 & 1.266 & 1.418 & 1.331 & 1.240 & 1.220 & 1.286 & 0.957 \\
 & Diff. Sharpe & \textbf{1.174} & \textbf{1.183} & \textbf{1.279} & 1.409 & \textbf{1.334} & \textbf{1.264} & \textbf{1.237} & \textbf{1.288} & \textbf{0.958} \\
 & Emb. DD & \textbf{1.237} & \textbf{1.204} & 1.259 & 1.281 & \textbf{1.374} & \textbf{1.265} & \textbf{1.245} & 1.271 & 0.916 \\
 & Regret & \textbf{1.198} & \textbf{1.173} & 1.260 & \textbf{1.488} & 1.238 & 1.235 & \textbf{1.246} & 1.277 & \textbf{0.999} \\
\cline{1-11}

% \multirow[t]{4}{*}{Sortino ratio} & 60/40 & 0.771 & 0.784 & 1.357 & 2.190 & 1.748 & 1.040 & 1.096 & 1.164 & -0.242 \\
%  & Diff. Sharpe & \textbf{0.872} & \textbf{0.927} & \textbf{1.424} & 2.128 & \textbf{1.816} & \textbf{1.134} & \textbf{1.205} & \textbf{1.179} & \textbf{-0.237} \\
%  & Emb. DD & \textbf{1.269} & \textbf{1.056} & \textbf{1.377} & 1.458 & \textbf{2.089} & \textbf{1.152} & \textbf{1.349} & 1.119 & -0.490 \\
%  & Regret & \textbf{0.904} & \textbf{0.843} & 1.293 & \textbf{2.485} & 1.207 & 1.010 & \textbf{1.178} & 1.118 & \textbf{-0.006} \\
% \cline{1-11}
% \multirow[t]{4}{*}{Tail ratio} & 60/40 & 1.059 & 1.077 & 1.097 & 1.163 & 1.186 & 1.076 & 1.070 & 1.315 & 0.915 \\
%  & Diff. Sharpe & \textbf{1.08} & \textbf{1.114} & \textbf{1.129} & \textbf{1.182} & \textbf{1.201} & \textbf{1.092} & \textbf{1.136} & 1.273 & 0.910 \\
%  & Emb. DD & \textbf{1.157} & \textbf{1.113} & 1.090 & \textbf{1.205} & 1.146 & \textbf{1.136} & \textbf{1.121} & 1.235 & 0.890 \\
%  & Regret & \textbf{1.081} & \textbf{1.082} & \textbf{1.161} & \textbf{1.314} & 1.082 & 1.060 & \textbf{1.168} & 1.191 & \textbf{0.99} \\
% \cline{1-11}

\bottomrule
\end{tabular}






\label{tab:results1}
\end{table*}



% \begin{figure}[ht]
%   \centering
%   \includegraphics[width=\linewidth]{figures/Weights_cheater_test.png}
%   \caption{Oracle Allocation.}
%   \label{fig:Worcale}
%   % \Description{Description for accessibility}
% \end{figure}

% \begin{figure}[ht]
%   \centering
%   \includegraphics[width=\linewidth]{figures/Allocation_test.png}
%   \caption{Example of a PPO allocation.}
%   \label{fig:Alloc}
%   % \Description{Description for accessibility}
% \end{figure}








\section{Methodology}
% \subsection{Environment Design}
One of the most crucial aspects of reinforcement learning applications is the definition of the environment. Its design and hyperparameters significantly influence the agent's performance.

\subsection{Action Space}
We consider prices of $K=3$ trading strategies:  Only \textit{Developed Markets Equity} when the agent is more bullish\footnote{expecting a rise in prices, and thus choosing a more volatile asset}, the \textit{60/40 Portfolio} (i.e. the portfolio that we are trying to improve), and Only \textit{Global Government Bonds} (Govies) as a low-risk asset to potentially avoid sharp drops in portfolio value.
% \textit{Developed Markets Equity Index}, \textit{Emerging Markets Equity Index}, \textit{Global Credit}, and \textit{Global Government Bonds}.
Shorting \footnote{ taking negative weight $w^i$ by borrowing asset $i$}  is not allowed, hence our action space $ \mathcal{A}$ is represented by a non-negative continuous vector $w \in \{w \in \mathbb{R}_{+}^K : \sum_{i=1}^{K} w^i = 1\}$. 
While we consider a standard problem of allocation between risky, balanced and conservative assets, the approach can be scaled to more high-dimensional allocation.










% \todo{}
% Most allocations in the initial asset universe are not entirely reasonable. During the early stages of training, an RL agent experiments with many random actions, making it extremely difficult to find a  "sweet spot" in the multi-dimensional action space. To facilitate the learning process, we restrict our agent to allocation between three strategies: Only Developed Markets Equity when the agent is more bullish, the 60/40 portfolio (i.e. the portfolio that we are trying to improve), and Only Global Government Bonds (Govies) as a low-risk asset to potentially avoid sharp drops in portfolio value.
\subsection{Observation Space}
\label{ch:obs_space}
 In addition to asset information, we include three classical contextual indexes that, while not part of the portfolio, may provide signals for asset allocation: \textit{High-Yield Bond Spread} \cite{gilchrist2012credit}, \textit{VIX volatility index} \cite{whaley2000investor}, and \textit{Merrill Lynch Option Volatility Estimate} \cite{driessen2009price}.
 
% We transform the price data for both asset and index tickers into daily strategy returns, $\mu_t \in \mathbb{R}^{3}$ and $\alpha_t \in \mathbb{R}^{3}$ respectively, and the return standard deviations of the last 60 days, $\bar \sigma^{t-60}_{t} \in \mathbb{R}^{3}_+$ and $ {\bar q^{t-60}}_{t} \in \mathbb{R}^{3}_+$. Additionally, we calculate the rolling average returns over the past 40 days, $\bar \mu^{t-40}_t\in \mathbb{R}^{3}$, to smooth out the noise present in daily returns. The agent also considers the previous allocation $w_{t-1} \in \{w \in \mathbb{R}_{+}^3 : \sum_{i=1}^{3} w_i = 1\}$ and the current transaction costs term, $TC_{train}(t) \in \mathbb{R}_+$ to determine whether it is reasonable to change the current allocation or if it is approximately optimal. The agent iterates through the bi-daily transformed historical dataset, rebalancing the portfolio based on the current observation and looping back to the beginning upon reaching the end, using the row data as observations. Consequently, at each timestep, the agent observes a vector  $o_t = [\mu_t, \ \alpha_t, \ \bar \mu^{t-40}_t, \ \bar \sigma^{t-60}_{t},  \  {\bar q^{t-60}}_{t}, \ w_{t-1}, \ TC_{train}(t) ] \in \mathbb{R}^{19}$. 

We transform the price data for both asset and index tickers into daily strategy returns, $\mu_t \in \mathbb{R}^{3}$ and $\alpha_t \in \mathbb{R}^{3}$ respectively, as well as the return standard deviations over the last 60 days for assets $\bar \sigma^{t-60}_{t} \in \mathbb{R}^{3}_+$ and for indexes ${\bar q^{t-60}}_{t} \in \mathbb{R}^{3}_+$. Additionally, we calculate the rolling average asset returns over the past 40 days, $\bar \mu^{t-40}_t \in \mathbb{R}^{3}$, to smooth out noise in the daily returns. The agent also considers the previous allocation $w_{t-1} \in \{w \in \mathbb{R}_{+}^3 : \sum_{i=1}^{3} w^i = 1\}$ and the current transaction costs, $TC_{train}(t) \in \mathbb{R}_+$, to decide whether adjusting the current allocation is warranted or if it remains approximately optimal. The agent iterates through the transformed historical dataset, rebalancing the portfolio based on current observations, and loops back to the beginning upon reaching the end, treating the data as ongoing observations. At each timestep, the agent observes a vector $o_t = [\mu_t, \ \alpha_t, \ \bar \mu^{t-40}_t, \ \bar \sigma^{t-60}_{t}, \ {\bar q^{t-60}}_{t}, \ w_{t-1}, \ TC_{train}(t)] \in \mathbb{R}^{19}$.


Daily portfolio rebalancing can be excessively sensitive to short-term noise and costly due to fees. To address this, we employed data aggregation; however, this approach reduces the number of training points by the frequency of trading days we choose to use. We settled on a bi-daily (2 bd) interval because aggregating data over two-day intervals smooths the data while avoiding an excessive reduction in training points.










\subsection{TC Schedule}
In our first experiments, we observed that including transaction costs (TC) at the start of training negatively impacted the learning process and diminished the signal from the main reward term (e.g., return or Sharpe ratio), particularly during the model's exploration stage. Fine-tuning the model with full transaction costs from the start proved challenging, as it often led to either constant allocations or poor generalization (see Regret and BB + Regret in the validation/test plots, Fig. \ref{fig:novelt}). Inspired by the principles of Curriculum Learning \cite{bengio2009curriculum, koenecke2020curriculum}, where the complexity of tasks gradually increases from simple to more real-world scenarios, we introduce a transaction cost scheduler that incrementally raises the training transaction cost at each step according to 
\begin{equation}
TC_{\text{train}}(x) = 
\frac{TC_{\text{max}}}{S^{a}} \cdot x^{a} \quad \text{if } 0 \leq x \leq S.
\label{eq:tc_formula}
\end{equation}
The costs are increased until a ramp limit given by $S = 100 \cdot \text{episode\_length}$ is reached. After the ramp limit, the maximal costs, $TC_{\text{max}} = TC_{\text{eval}} = 0.0025$ (a fair value for traditional brokers and large institutional traders), are applied.
% \begin{equation}
% TC_{\text{eval}}(x) = TC_{\text{max}}
% \label{eq:tc_formula2}
% \end{equation}





\subsection{Reward Function}
\label{sec:reward_function}
A key distinction between our problem and many other RL applications is the ability to know the reward of any action, not just the one taken by the agent. Real market participants often reflect on the optimal allocation they could have chosen yesterday based on today's returns. Building on this concept, we propose a negative Sharpe-based regret reward function (\ref{eq:regret}) conditioned on the previous timestep's action. 

\begin{equation}
\text{\textbf{Reward}}_t = -\bar \mu_t^{t+n}(w^* - w_t)'
\label{eq:regret}
\end{equation}
where
\begin{equation}
\begin{aligned}
w^* &= \argmax_{w} \ \mathbf{Sharpe}(w, \bar \mu_t^{t+n}, \bar \Sigma_{t-3n}^{t+3n}) - TC_{\text{train}}(t) \cdot \|w - w_{t-1}\|_1 \\
 &= \argmax_{w}  \frac{w' \bar \mu_t^{t+n} - R_f}{\sqrt{w' \bar \Sigma_{t-3n}^{t+3n} w}} - TC_{\text{train}}(t) \cdot \|w - w_{t-1}\|_1
\end{aligned}
\label{eq:regret_weight}
\end{equation}

The reward is calculated as the difference between the average returns of the optimal allocation and the agent's allocation over the next $n$ days, assuming both allocations are fixed from today. We use the Sharpe ratio because it is one of the most popular and efficient measures that balance return and risk. To incentivize the agent to optimize the portfolio for future relevance, we use a forward-looking return vector $\bar{\mu}_t^{t+n}$ and a covariance matrix $\bar{\Sigma}_{t-3n}^{t+3n}$ in the Sharpe ratio calculation, rather than the typical current or simple return. While we consider only the forward-looking average return over the next $n=14$ business days, we use a broader interval $(t - 3n, \ t + 3n)$ for a more precise estimate of the covariance matrix. Additionally, to account for transaction costs, we include a regularization term in the optimal action's expression (\ref{eq:regret_weight}),  ensuring that the next Oracle-optimal allocation $w^*$ can be achieved despite the fees required to adjust the previous allocation, with the transaction cost term proportional to the difference in weights between two consecutive allocations to penalize large adjustments.




Since we do not have access to future return information during testing, we set the reward to zero in the environment when deploying the trained model. While future data can be integrated during training, we avoid using this data at inference time to prevent leakage between training, validation, and testing. We ensure no overlap among these datasets, which means that toward the end of training, we have fewer points to look into the future. Thus, we use all available training points without incorporating data that cannot technically be used. Consequently, we evaluate the pipeline using various financial ratios instead of total episodic reward.



\subsection{Training on Synthetic Data}
A significant conceptual challenge in applying reinforcement learning to this problem is the limited size of available data. Unlike other RL environments in fields such as robotics and games, which offer variability and diverse paths during training as observations are influenced by the agent's actions, our environment lacks price impact and remains unaffected by allocations. As a result, the agent iterates through the same data repeatedly, encountering nearly identical trajectories in each episode. This setup significantly increases the risk of overfitting to the training dynamics rather than learning robust and generalizable trading strategies.

To address this issue, we propose training on synthetic data that closely mimics the underlying real data distribution. Initially, we considered using the Gaussian copula \cite{rey2015copula}; however, this method was quickly dismissed as it fails to accurately represent the distribution in the tails, which are critical during crisis events that offer opportunities for abnormal returns. Instead, we opted for a circular block bootstrap \cite{arch} of the underlying training data, applied every 10 episodes. We experimented with various block sizes and found that larger blocks,  70-90\% of the original training set, led to improved agent performance as they better preserve long-range temporal dependencies of the training set.

The training process involves an iterative approach to improve the agent's performance. Initially, the agent is trained on real data for 10 episodes. Subsequently, the training shifts to synthetic data for another 10 episodes. After this cycle, the synthetic data is regenerated to introduce new variability, and the agent undergoes another 10 episodes of training on the new synthetic data. This process of regenerating synthetic data and training continues until the required number of training episodes is reached. Upon completion, the agent transitions to the next phase, starting the cycle anew with training on real data.


\begin{figure}[ht]
  \centering
  \begin{subfigure}{\linewidth}
    \centering
  \includegraphics[width=\linewidth]{figures/Weights_cheater_test.png}
  \caption{Oracle: $w^*$ defined as (\ref{eq:regret_weight})}
  \label{fig:Worcale}
  \end{subfigure}
  \begin{subfigure}{\linewidth}
    \centering
 \includegraphics[width=\linewidth]{figures/Allocation_test.png}
  \caption{PPO with negative Sharpe Regret reward function}
  \label{fig:Alloc}
  \end{subfigure}

  \caption{Example of allocation during the testing period of phase 3. The regret-based agent demonstrates a general alignment with the optimal allocation but adopts a less aggressive stance. During bullish market periods, it increases its positions in risky assets, capitalizing on favorable conditions. Conversely, when market uncertainty arises, the agent shifts to more conservative allocations, showcasing its adaptability to changing market dynamics.}
  \Description{Example of allocation during the testing period of phase 3. The regret-based agent demonstrates a general alignment with the optimal allocation but adopts a less aggressive stance. During bullish market periods, it increases its positions in risky assets, capitalizing on favorable conditions. Conversely, when market uncertainty arises, the agent shifts to more conservative allocations, showcasing its adaptability to changing market dynamics.}
  \label{fig:two-allocs}
  
\end{figure}



\subsection{Implementation Details} 
The complete training loop is detailed in Algorithm \ref{alg1}. We employ fully connected feedforward neural networks for both the policy and critic architectures. These networks are designed with multiple hidden layers to effectively extract features from the state space, capturing the intricate patterns and dynamics of the environment. The hidden layers use the Tanh activation function, which maps the input values to a range between -1 and 1, aiding in normalization and maintaining stable gradient flow during backpropagation. This design ensures that the networks can learn robust feature representations, contributing to the overall stability and performance of the PPO algorithm in our experiments.


We employ a sliding window approach with a train-validation-test split in our pipeline, splitting the available data into phases as detailed in Table \ref{tab:timing-phases}. This method mitigates the bias towards older, potentially less relevant observations. The phased split enables a robust evaluation of all approaches by testing the agent across three distinct and representative market scenarios. Phase 1 represents a period of asset growth under favorable market conditions. Phase 2 corresponds to the COVID-19 crisis, characterized by sharp declines in asset values. Finally, Phase 3 reflects a market downturn in which all assets experienced price drops, emphasizing the agent's ability to minimize losses in adverse conditions. We transfer the weights of the value and policy networks after Phases 1 and 2. The best model, evaluated on the validation set based on the return-risk trade-off, is used as a pretrain for the subsequent phase. This approach prevents the need to learn the model of the world from scratch, thereby reducing the number of training episodes. We opted for the sliding window approach over the expanding window because the latter approach often leads to overfitting to the early dynamics by exposing the agent to the earliest observations more frequently. Consequently, we exclude data from the 90s in the later phases but retain the knowledge by transferring the policy and value networks' weights to the subsequent phases. Additionally, to promote exploration when transitioning to another phase, we implement an entropy regularization schedule. This schedule sets a non-zero entropy coefficient, $\beta_{\text{entropy}}$, in the PPO total loss function (\ref{eq:ppo_loss}) and then linearly decreases it to zero until 10\% of the number of training episodes. We use a Critic loss (\ref{eq:critic_loss}) and a bonus based on the $ \mathcal{L}_{\text{entropy}} = - H(\pi(w|s))$ of the policy distribution. 

\begin{equation}
\mathcal{L} = \mathcal{L}_{\text{policy}} + \beta_{\text{entropy}} \cdot \mathcal{L}_{\text{entropy}} + \beta_{\text{value}} \cdot \mathcal{L}_{\text{value}}
\label{eq:ppo_loss}
\end{equation}


\begin{equation}
\mathcal{L^{\text{value}}} = \mathbb{E}_{t} \left[ \left( V_{\theta}(s_t) - V^{\text{target}}_t \right)^2 \right]
\label{eq:critic_loss}
\end{equation}


Normalizing the advantages in PPO enhances the results on the validation set. We adjust the transaction cost schedule to be more concave in Phases 2 and 3, as the agent is not learning from scratch. Furthermore, we discovered that when selecting the model from the Pareto front for the next phase, it is preferable to choose a less risk-averse model, as more conservative models tend to become stuck in suboptimal allocations and encounter difficulties in training during the new phase.


\begin{algorithm}
\caption{Agent Training Pseudo-Code}
\label{alg1}
\begin{algorithmic}[1]
\State Given original historical dataset $\mathcal{D}_{\text{orig}}$

\State Initialize network parameters: $\theta$ for policy network, $\phi$ for value network
\State Initialize maximum training iterations $M$ and training batch $D$

\State Set $\tilde{\mathcal{D}} = \mathcal{D}_{\text{orig}}$
\For{training iteration $= 1$ to $M$}
    \State Clear training batch $D$
    \For{each RL collect step $t$}
        % 
        \If{$t \ \%  \ \text{episode\_length} \times 10 == 0$}
            \If{$Bernoulli(0.7) == 1$}
                \State $\tilde{\mathcal{D}} = BlockBootstrap(\mathcal{D}_{\text{orig}})$
            \Else
                \State $\tilde{\mathcal{D}} = \mathcal{D}_{\text{orig}}$
            \EndIf 
        \EndIf
        \State Observe environment state $o_t$ from $\tilde{\mathcal{D}}$
        \State Select action $a_t$ according to policy $\pi_\theta(a_t \mid o_t)$
        \State Compute $w_t = softmax(a_t)$
        \State Compute Oracle allocation $w^*(\tilde{\mathcal{D}}) = $ 
        \Statex \hspace{6em}  $ = \argmax_{w} \ \mathbf{Sharpe}(w, \bar \mu_t^{t+n}(\tilde{\mathcal{D}}), \bar \Sigma_{t-3n}^{t+3n}(\tilde{\mathcal{D}})) -$
        \Statex \hspace{8em} $ - TC_{\text{train}}(t) \cdot \|w - w_{t-1}\|_1 $
      
       
        \State Execute action $a_t$, transition to $o_{t+1}$ from $\tilde{\mathcal{D}}$
        \State Calculate Reward $r_t = -\bar \mu_t^{t+n}(\tilde{\mathcal{D}}) \cdot (w^* - w_t)'$
    \EndFor
    \State Compute advantage estimate $\hat{A}$ using GAE
    \State Add experience $(o_t, a_t, r_t, o_{t+1})$ to batch $D$
    
    \For{each RL training step}
        \State $\beta_{\text{entropy}}$ = $EntropySchedule(\text{step})$
        \State Recompute advantage estimate $\hat{A}$ using GAE
        \State Split batch $D$ into $K$ mini-batches $\mathcal{B}$
        \For{mini-batch $k = 1$ to $K$}
            \State Compute PPO total loss 
            \Statex \hspace{5em} $\mathcal{L} =\mathcal{L}_{\text{policy}} + \beta_{\text{entropy}} \cdot \mathcal{L}_{\text{entropy}} + \beta_{\text{value}} \cdot \mathcal{L}_{\text{value}}$
            \State Update critic and actor networks w.r.t $\mathcal{L}$
        \EndFor
    \EndFor
\EndFor
\State \Return $V_\phi, \pi_\theta$
\end{algorithmic}
\end{algorithm}

\section{Results and EMPIRICAL EVALUATION}
\subsection{Discussion}
We analyze the performance of the agent in Table \ref{tab:results1} using a variety of performance measures:

\begin{itemize}
    \item \textbf{Sortino Ratio}: A variation of the Sharpe ratio, it focuses on the standard deviation of negative asset returns to assess downside risk, offering a more targeted evaluation for investors.
    \item \textbf{Calmar Ratio}: This ratio measures downside risk by comparing the maximum drawdown to the average annual rate of return, providing insights into the risk-adjusted performance of an investment.
    % \item \textbf{Maximum Drawdown (MDD)}: \todo{REMOVE? Explained already so many times} Lower MDD can enhance investor confidence, potentially leading to increased commitment to the investment strategy and a greater likelihood of adhering to long-term financial goals.
    \item \textbf{Omega Ratio}: Compares the probability of returns above a threshold to those below, considering the full return distribution.

    % \item \textbf{Stability Ratio}: This measures the consistency of a portfolio's returns over a specified period, defined as the ratio of the average return to the standard deviation of returns.
    % \item \textbf{Tail Ratio}: This assesses the risk of extreme losses by comparing the average return of the worst-performing periods to the average return of the best-performing periods.
\end{itemize}


Our PPO agent outperforms the 60/40 portfolio in terms of return across all periods, except for the validation set in the pre-pandemic phase. The Calmar ratio is significantly higher for regret PPO in most cases, with the exception of Phase 2. The Sharpe, Omega, and Sortino ratios are closely aligned. Notably, the validation performance for all measures in Phase 2 was suboptimal and shows room for improvement. The embedded drawdown approach has the lowest return among all approaches but typically achieves the best MDD, which aligns with the classic return-risk trade-off. Both the embedded drawdown and differential Sharpe reward functions are more conservative, consistently performing well in risk-accounting measures such as Sharpe, Sortino, and MDD. 
% This consistency explains why these approaches often beat the return benchmark.
However, during the testing phase in the post-pandemic period, their performance was not as promising as the regret-based approach (-2.6\% for 60/40, -2.4\% for differential Sharpe, and -0.7\% for regret).


Overall, as illustrated in Figure \ref{fig:phases-plot}, our Sharpe regret-based approach with TC scheduling appears to be a preferable option, as it consistently outperforms the return benchmark and surpasses the MDD 60/40 benchmark in two out of three instances. Phase 1 is particularly challenging, with both measures showing flatter distributions. The transfer of weights in the later stages likely reduces variability in learning for the agent. Constant 60/40 only outperforms our approach in MDD during the pandemic breakout, which could be partly because the features are not optimal for detecting such signals. The MDD distribution in Phase 1 is more skewed compared to all other runs. As anticipated, the pandemic period (Phase 2) proved to be the most challenging for our agent, as its return distribution, while higher than the benchmark, remains relatively flat. Future work could focus on fine-tuning the agent to better identify and respond to crises through regime detection methods, as suggested in \cite{benhamou2021detecting, halperin_combining_2022}.

% \todo{Add Pareto Fronts fig??? NO I think its too much already. plus what can we infer from them extra to histograms?}

The embedded drawdown and differential Sharpe reward functions generate reasonable allocations but struggle to surpass the benchmark in the main measure: annual return. One possible reason is that these approaches do not explicitly account for transaction costs within the learning process. Agents using these reward functions tend to be more risk-averse, avoiding significant increases in positions in the riskiest assets. However, it is noteworthy that the other two approaches performed better in optimizing maximum drawdown compared to our regret-based approach. Interestingly, the embedded drawdown approach underperformed significantly during the post-pandemic period, barely beating the 60/40 benchmark. In the most recent testing period, our approach proved to be notably better than the other methods in both objectives.


% \begin{figure}[ht]
%   \centering
%   \begin{subfigure}{\linewidth}
%     \centering
%     \includegraphics[width=\linewidth]{figures/test_Phase1.png}
%     % \caption{Caption for image}
%     \label{fig:phase1}
%   \end{subfigure}
%   \begin{subfigure}{\linewidth}
%     \centering
%     \includegraphics[width=\linewidth]{figures/test_Phase2.png}
%     % \caption{Caption for image}
%     \label{fig:phase2}
%   \end{subfigure}

%   \begin{subfigure}{\linewidth}
%     \centering
%     \includegraphics[width=\linewidth]{figures/test_Phase3.png}
%     % \caption{Caption for image}
%     \label{fig:phase3}
%   \end{subfigure}
  
%   \caption{Distribution of returns and MDDs for the agent (in blue) compared to the benchmark (black dashed line) for the test set.}
%   \Description{Distribution of returns and MDDs for the agent (in blue) compared to the benchmark (black dashed line) for the test set.}
%   \label{fig:phases-plot}
% \end{figure}


\begin{figure}[ht]
  \centering
  \begin{subfigure}{\linewidth}
    \centering
    \includegraphics[width=\linewidth]{figures/test_Phase1.pdf}
    % \caption{Caption for image}
    \label{fig:phase1}
  \end{subfigure}
  \begin{subfigure}{\linewidth}
    \centering
    \includegraphics[width=\linewidth]{figures/test_Phase2.pdf}
    % \caption{Caption for image}
    \label{fig:phase2}
  \end{subfigure}

  \begin{subfigure}{\linewidth}
    \centering
    \includegraphics[width=\linewidth]{figures/test_Phase3.pdf}
    % \caption{Caption for image}
    \label{fig:phase3}
  \end{subfigure}
  
  \caption{Distribution of returns and MDDs for the PPO agent for differential Sharpe, embedded drawdown and regret (ours) reward functions compared to the 60/40 benchmark (black dashed line) on the test periods. Our approach (in blue) focuses more on returns, allowing it to outperform all other methods in the primary objective across all three phases. Our model has proven superior to all considered approaches in the latest trading phase in both objectives.}
  \Description{Distribution of returns and MDDs for the PPO agent for differential Sharpe, embedded drawdown and regret (ours) reward functions compared to the 60/40 benchmark (black dashed line) on the test periods. Our approach (in blue) focuses more on returns, allowing it to outperform all other methods in the primary objective across all three phases. Our model has proven superior to all considered approaches in the latest trading phase in both objectives.}
  \label{fig:phases-plot}
\end{figure}

Figures \ref{fig:Worcale} and \ref{fig:Alloc} demonstrate that regret-based agent generally aligns, though not as extremely, with the optimal allocation as shown by the allocations during the periods of April 2022, August 2022, and April 2023. Overall, the agent employing the Sharpe regret reward function tends to increase its position in risky assets during bullish markets and opts for a more conservative allocation when anticipating uncertain market conditions.


We also compare the effect of our modification on the training dynamics and the ability of the agent to generalize to the unseen data in Figure \ref{fig:novelt}. While the full configuration (TC + BB + Regret) is outperformed in the training environment, it shows a better and faster ability to generalize to the unseen data in the validation and test environments due to greater variability in the training set. Periodically introducing synthetic data, similar to many forms of regularization, increases the challenge for the agent to learn, as evidenced by the more fluctuating line in the training environment, but it helps the agent avoid memorizing non-reproducible effective allocation strategies. Interestingly, separately only BB and only TC configurations do not perform as well on the test set as their synergy. We can also observe that the current portfolio return (TC + BB + Return)  reward function is overfitting to the training data and is not managing to generate a profitable strategy out-of-sample.





% \section{Don't\small{s}}
\subsection{What Didn’t Work}

We also provide a brief discussion of the ideas that we tried and that did not work for our pipeline. Although these ideas failed for us, we hope that they might be helpful in other similar applications. DDPG and A2C algorithms failed to produce dynamic allocations and outputted nearly identical weights throughout the validation environment, regardless of the inputs in the observations. 

We also explored Convolutional Neural Networks (CNNs) for feature extraction, building on Benhamou et al. \cite{benhamou2021detecting}, who used multi-body policy networks with multiple tensors. We define the observation space with $k$ lags $L = [1, 2, 3, 6, 10, 15, 30, 60]$, yielding several tensors. The first network body utilizes asset information (\ref{eq:A_t}) as channels, while the second body processes market context (\ref{eq:C_t}) with 2-D convolution and a 3x3 filter. The approach underperformed compared to MLP architecture likely due to the transaction costs incurred from noisy allocation behaviors (Fig. \ref{fig:alloc_cnn}).


\begin{equation}
\footnotesize
A_t := \left[
\begin{array}{l}
\begin{bmatrix}
\mu_t \\
\mu_{t-l_1} \\
\vdots \\
\mu_{t-l_k}
\end{bmatrix} \in \mathbb{R}^{(k+1) \times 3 } , 
\begin{bmatrix}
\bar \mu_t \\
\bar \mu_{t-l_1} \\
\vdots \\
\bar \mu_{t-l_k}
\end{bmatrix} \in \mathbb{R}^{(k+1) \times 3 } ,
\begin{bmatrix}
\bar \sigma_t \\
\bar \sigma_{t-l_1} \\
\vdots \\
\bar \sigma_{t-l_k}
\end{bmatrix} \in \mathbb{R}^{(k+1) \times 3 }



\end{array}
\right] 
\label{eq:A_t}
\end{equation}

% \renewcommand{\arraystretch}{1} % Reset row height to default




% \renewcommand{\arraystretch}{1.45} % Adjust row height (1.45 is an example, adjust as needed)

\begin{equation}
\footnotesize
C_t := \left[
\begin{array}{l}
\begin{bmatrix}
\alpha_t \\
\alpha_{t-l_1} \\
\vdots \\
\alpha_{t-l_k}
\end{bmatrix} \in \mathbb{R}^{(k+1) \times 3 } , 

\begin{bmatrix}
\bar q_t \\
\bar q_{t-l_1} \\
\vdots \\
\bar q_{t-l_k}
\end{bmatrix} \in \mathbb{R}^{(k+1) \times 3 }

\end{array}
\right] 
\label{eq:C_t}
\end{equation}

% \renewcommand{\arraystretch}{1} % Reset row height to default


Alternatively, we also tried using differences in Sharpe ratios between optimal and taken allocations in the regret reward function. The agent with this reward function failed to learn a meaningful policy for unseen data, possibly due to the non-linear difference between optimal and taken actions. Additionally, we spent considerable time attempting to pre-train the agent with imitation learning using the Oracle as in \cite{kochliaridis2023combining} as an expert agent. All of the GAIL \cite{ho2016generative}, AIRL \cite{fu2018learningrobustrewardsadversarial}, and density-based reward modeling \cite{davis2011remarks} methods appeared to be highly sensitive to hyperparameters and did not result in improvement compared to training from scratch. Nevertheless, we see these approaches as promising for this application, and it may be our next direction for analysis.






% \begin{figure}[ht]
%   \centering
%   \includegraphics[width=\linewidth]{figures/Allocation_test.png}
%   \caption{Caption for Phase 3 image}
%   \label{fig:phase3}
%   % \Description{Description for accessibility}
% \end{figure}


\section{Future Work}
Our dynamic PPO regret-based agent consistently outperformed the return benchmark. The agent effectively adjusted allocations, increasing positions in bullish markets and adopting conservative allocations during uncertain periods. Our pipeline, which incorporates transaction cost scheduling, circular block bootstrap, and the future-looking regret-based reward function, demonstrates the ability to generalize to unseen test data. Discretizing the action space could enhance the model's scalability by reformulating the decision-making process. In such settings, ideas from \cite{elmachtoub2022smart} and \cite{mandi2020smart} may be explored for Oracle optimization.




Future work could involve further adapting the agent to operate in uncertain markets and respond more effectively during crisis events. Additional reward functions and their hyperparameters (such as Oracle's forecasting horizon) could be explored. The analysis of scalability may be conducted and the effect of larger portfolios and high-dimensional policies can be investigated.

We conducted a minimal, exploratory search for the following hyperparameters: the number of training episodes, early stopping time, the ramp-up duration for when the transaction cost (TC) scheduler levels off, and the convexity of the transaction cost. A more comprehensive exploration of these hyperparameters may be considered in future work.

Although we evaluated performance using MDD, it was not explicitly included in the reward, leading to less consistent performance in this measure. Expanding the observation space to a matrix by incorporating lagged variables could be beneficial; CNNs would be an appropriate choice for this setup. Additionally, dynamic strategies such as mean-reversion and momentum strategies could be integrated into the allocation process. More comprehensive hyperparameter tuning (on both the agent and environment sides) could be implemented. Given the sample inefficiency of RL approaches and computational constraints, we opted not to conduct an exhaustive hyperparameter search to avoid overfitting each validation set. Lastly, using narrower periods for the validation set could necessitate more frequent model recalibrations.

% \com{Papers submitted to the main track must be at most 8 pages long, with any number of additional pages containing bibliographic references. }


% \todo{Fix Spacing Appendix}


%%%%%%%%%%%%%%%%%%%%%%%%%%%%%%%%%%%%%%%%%%%%%%%%%%%%%%%%%%%%%%%%%%%%%%%%

%%% The acknowledgments section is defined using the "acks" environment
%%% (rather than an unnumbered section). The use of this environment 
%%% ensures the proper identification of the section in the article 
%%% metadata as well as the consistent spelling of the heading.

% \begin{acks}
% If you wish to include any acknowledgments in your paper (e.g., to 
% people or funding agencies), please do so using the `\texttt{acks}' 
% environment. Note that the text of your acknowledgments will be omitted
% if you compile your document with the `\texttt{anonymous}' option.
% \end{acks}

%%%%%%%%%%%%%%%%%%%%%%%%%%%%%%%%%%%%%%%%%%%%%%%%%%%%%%%%%%%%%%%%%%%%%%%%

%%% The next two lines define, first, the bibliography style to be 
%%% applied, and, second, the bibliography file to be used.




\bibliographystyle{ACM-Reference-Format} 
% \bibliography{main.bib}
% This must be in the first 5 lines to tell arXiv to use pdfLaTeX, which is strongly recommended.
\pdfoutput=1
% In particular, the hyperref package requires pdfLaTeX in order to break URLs across lines.

\documentclass[11pt]{article}

% Change "review" to "final" to generate the final (sometimes called camera-ready) version.
% Change to "preprint" to generate a non-anonymous version with page numbers.
\usepackage{acl}

% Standard package includes
\usepackage{times}
\usepackage{latexsym}

% Draw tables
\usepackage{booktabs}
\usepackage{multirow}
\usepackage{xcolor}
\usepackage{colortbl}
\usepackage{array} 
\usepackage{amsmath}

\newcolumntype{C}{>{\centering\arraybackslash}p{0.07\textwidth}}
% For proper rendering and hyphenation of words containing Latin characters (including in bib files)
\usepackage[T1]{fontenc}
% For Vietnamese characters
% \usepackage[T5]{fontenc}
% See https://www.latex-project.org/help/documentation/encguide.pdf for other character sets
% This assumes your files are encoded as UTF8
\usepackage[utf8]{inputenc}

% This is not strictly necessary, and may be commented out,
% but it will improve the layout of the manuscript,
% and will typically save some space.
\usepackage{microtype}
\DeclareMathOperator*{\argmax}{arg\,max}
% This is also not strictly necessary, and may be commented out.
% However, it will improve the aesthetics of text in
% the typewriter font.
\usepackage{inconsolata}

%Including images in your LaTeX document requires adding
%additional package(s)
\usepackage{graphicx}
% If the title and author information does not fit in the area allocated, uncomment the following
%
%\setlength\titlebox{<dim>}
%
% and set <dim> to something 5cm or larger.

\title{Wi-Chat: Large Language Model Powered Wi-Fi Sensing}

% Author information can be set in various styles:
% For several authors from the same institution:
% \author{Author 1 \and ... \and Author n \\
%         Address line \\ ... \\ Address line}
% if the names do not fit well on one line use
%         Author 1 \\ {\bf Author 2} \\ ... \\ {\bf Author n} \\
% For authors from different institutions:
% \author{Author 1 \\ Address line \\  ... \\ Address line
%         \And  ... \And
%         Author n \\ Address line \\ ... \\ Address line}
% To start a separate ``row'' of authors use \AND, as in
% \author{Author 1 \\ Address line \\  ... \\ Address line
%         \AND
%         Author 2 \\ Address line \\ ... \\ Address line \And
%         Author 3 \\ Address line \\ ... \\ Address line}

% \author{First Author \\
%   Affiliation / Address line 1 \\
%   Affiliation / Address line 2 \\
%   Affiliation / Address line 3 \\
%   \texttt{email@domain} \\\And
%   Second Author \\
%   Affiliation / Address line 1 \\
%   Affiliation / Address line 2 \\
%   Affiliation / Address line 3 \\
%   \texttt{email@domain} \\}
% \author{Haohan Yuan \qquad Haopeng Zhang\thanks{corresponding author} \\ 
%   ALOHA Lab, University of Hawaii at Manoa \\
%   % Affiliation / Address line 2 \\
%   % Affiliation / Address line 3 \\
%   \texttt{\{haohany,haopengz\}@hawaii.edu}}
  
\author{
{Haopeng Zhang$\dag$\thanks{These authors contributed equally to this work.}, Yili Ren$\ddagger$\footnotemark[1], Haohan Yuan$\dag$, Jingzhe Zhang$\ddagger$, Yitong Shen$\ddagger$} \\
ALOHA Lab, University of Hawaii at Manoa$\dag$, University of South Florida$\ddagger$ \\
\{haopengz, haohany\}@hawaii.edu\\
\{yiliren, jingzhe, shen202\}@usf.edu\\}



  
%\author{
%  \textbf{First Author\textsuperscript{1}},
%  \textbf{Second Author\textsuperscript{1,2}},
%  \textbf{Third T. Author\textsuperscript{1}},
%  \textbf{Fourth Author\textsuperscript{1}},
%\\
%  \textbf{Fifth Author\textsuperscript{1,2}},
%  \textbf{Sixth Author\textsuperscript{1}},
%  \textbf{Seventh Author\textsuperscript{1}},
%  \textbf{Eighth Author \textsuperscript{1,2,3,4}},
%\\
%  \textbf{Ninth Author\textsuperscript{1}},
%  \textbf{Tenth Author\textsuperscript{1}},
%  \textbf{Eleventh E. Author\textsuperscript{1,2,3,4,5}},
%  \textbf{Twelfth Author\textsuperscript{1}},
%\\
%  \textbf{Thirteenth Author\textsuperscript{3}},
%  \textbf{Fourteenth F. Author\textsuperscript{2,4}},
%  \textbf{Fifteenth Author\textsuperscript{1}},
%  \textbf{Sixteenth Author\textsuperscript{1}},
%\\
%  \textbf{Seventeenth S. Author\textsuperscript{4,5}},
%  \textbf{Eighteenth Author\textsuperscript{3,4}},
%  \textbf{Nineteenth N. Author\textsuperscript{2,5}},
%  \textbf{Twentieth Author\textsuperscript{1}}
%\\
%\\
%  \textsuperscript{1}Affiliation 1,
%  \textsuperscript{2}Affiliation 2,
%  \textsuperscript{3}Affiliation 3,
%  \textsuperscript{4}Affiliation 4,
%  \textsuperscript{5}Affiliation 5
%\\
%  \small{
%    \textbf{Correspondence:} \href{mailto:email@domain}{email@domain}
%  }
%}

\begin{document}
\maketitle
\begin{abstract}
Recent advancements in Large Language Models (LLMs) have demonstrated remarkable capabilities across diverse tasks. However, their potential to integrate physical model knowledge for real-world signal interpretation remains largely unexplored. In this work, we introduce Wi-Chat, the first LLM-powered Wi-Fi-based human activity recognition system. We demonstrate that LLMs can process raw Wi-Fi signals and infer human activities by incorporating Wi-Fi sensing principles into prompts. Our approach leverages physical model insights to guide LLMs in interpreting Channel State Information (CSI) data without traditional signal processing techniques. Through experiments on real-world Wi-Fi datasets, we show that LLMs exhibit strong reasoning capabilities, achieving zero-shot activity recognition. These findings highlight a new paradigm for Wi-Fi sensing, expanding LLM applications beyond conventional language tasks and enhancing the accessibility of wireless sensing for real-world deployments.
\end{abstract}

\section{Introduction}

In today’s rapidly evolving digital landscape, the transformative power of web technologies has redefined not only how services are delivered but also how complex tasks are approached. Web-based systems have become increasingly prevalent in risk control across various domains. This widespread adoption is due their accessibility, scalability, and ability to remotely connect various types of users. For example, these systems are used for process safety management in industry~\cite{kannan2016web}, safety risk early warning in urban construction~\cite{ding2013development}, and safe monitoring of infrastructural systems~\cite{repetto2018web}. Within these web-based risk management systems, the source search problem presents a huge challenge. Source search refers to the task of identifying the origin of a risky event, such as a gas leak and the emission point of toxic substances. This source search capability is crucial for effective risk management and decision-making.

Traditional approaches to implementing source search capabilities into the web systems often rely on solely algorithmic solutions~\cite{ristic2016study}. These methods, while relatively straightforward to implement, often struggle to achieve acceptable performances due to algorithmic local optima and complex unknown environments~\cite{zhao2020searching}. More recently, web crowdsourcing has emerged as a promising alternative for tackling the source search problem by incorporating human efforts in these web systems on-the-fly~\cite{zhao2024user}. This approach outsources the task of addressing issues encountered during the source search process to human workers, leveraging their capabilities to enhance system performance.

These solutions often employ a human-AI collaborative way~\cite{zhao2023leveraging} where algorithms handle exploration-exploitation and report the encountered problems while human workers resolve complex decision-making bottlenecks to help the algorithms getting rid of local deadlocks~\cite{zhao2022crowd}. Although effective, this paradigm suffers from two inherent limitations: increased operational costs from continuous human intervention, and slow response times of human workers due to sequential decision-making. These challenges motivate our investigation into developing autonomous systems that preserve human-like reasoning capabilities while reducing dependency on massive crowdsourced labor.

Furthermore, recent advancements in large language models (LLMs)~\cite{chang2024survey} and multi-modal LLMs (MLLMs)~\cite{huang2023chatgpt} have unveiled promising avenues for addressing these challenges. One clear opportunity involves the seamless integration of visual understanding and linguistic reasoning for robust decision-making in search tasks. However, whether large models-assisted source search is really effective and efficient for improving the current source search algorithms~\cite{ji2022source} remains unknown. \textit{To address the research gap, we are particularly interested in answering the following two research questions in this work:}

\textbf{\textit{RQ1: }}How can source search capabilities be integrated into web-based systems to support decision-making in time-sensitive risk management scenarios? 
% \sq{I mention ``time-sensitive'' here because I feel like we shall say something about the response time -- LLM has to be faster than humans}

\textbf{\textit{RQ2: }}How can MLLMs and LLMs enhance the effectiveness and efficiency of existing source search algorithms? 

% \textit{\textbf{RQ2:}} To what extent does the performance of large models-assisted search align with or approach the effectiveness of human-AI collaborative search? 

To answer the research questions, we propose a novel framework called Auto-\
S$^2$earch (\textbf{Auto}nomous \textbf{S}ource \textbf{Search}) and implement a prototype system that leverages advanced web technologies to simulate real-world conditions for zero-shot source search. Unlike traditional methods that rely on pre-defined heuristics or extensive human intervention, AutoS$^2$earch employs a carefully designed prompt that encapsulates human rationales, thereby guiding the MLLM to generate coherent and accurate scene descriptions from visual inputs about four directional choices. Based on these language-based descriptions, the LLM is enabled to determine the optimal directional choice through chain-of-thought (CoT) reasoning. Comprehensive empirical validation demonstrates that AutoS$^2$-\ 
earch achieves a success rate of 95–98\%, closely approaching the performance of human-AI collaborative search across 20 benchmark scenarios~\cite{zhao2023leveraging}. 

Our work indicates that the role of humans in future web crowdsourcing tasks may evolve from executors to validators or supervisors. Furthermore, incorporating explanations of LLM decisions into web-based system interfaces has the potential to help humans enhance task performance in risk control.






\section{Related Work}
\label{sec:relatedworks}

% \begin{table*}[t]
% \centering 
% \renewcommand\arraystretch{0.98}
% \fontsize{8}{10}\selectfont \setlength{\tabcolsep}{0.4em}
% \begin{tabular}{@{}lc|cc|cc|cc@{}}
% \toprule
% \textbf{Methods}           & \begin{tabular}[c]{@{}c@{}}\textbf{Training}\\ \textbf{Paradigm}\end{tabular} & \begin{tabular}[c]{@{}c@{}}\textbf{$\#$ PT Data}\\ \textbf{(Tokens)}\end{tabular} & \begin{tabular}[c]{@{}c@{}}\textbf{$\#$ IFT Data}\\ \textbf{(Samples)}\end{tabular} & \textbf{Code}  & \begin{tabular}[c]{@{}c@{}}\textbf{Natural}\\ \textbf{Language}\end{tabular} & \begin{tabular}[c]{@{}c@{}}\textbf{Action}\\ \textbf{Trajectories}\end{tabular} & \begin{tabular}[c]{@{}c@{}}\textbf{API}\\ \textbf{Documentation}\end{tabular}\\ \midrule 
% NexusRaven~\citep{srinivasan2023nexusraven} & IFT & - & - & \textcolor{green}{\CheckmarkBold} & \textcolor{green}{\CheckmarkBold} &\textcolor{red}{\XSolidBrush}&\textcolor{red}{\XSolidBrush}\\
% AgentInstruct~\citep{zeng2023agenttuning} & IFT & - & 2k & \textcolor{green}{\CheckmarkBold} & \textcolor{green}{\CheckmarkBold} &\textcolor{red}{\XSolidBrush}&\textcolor{red}{\XSolidBrush} \\
% AgentEvol~\citep{xi2024agentgym} & IFT & - & 14.5k & \textcolor{green}{\CheckmarkBold} & \textcolor{green}{\CheckmarkBold} &\textcolor{green}{\CheckmarkBold}&\textcolor{red}{\XSolidBrush} \\
% Gorilla~\citep{patil2023gorilla}& IFT & - & 16k & \textcolor{green}{\CheckmarkBold} & \textcolor{green}{\CheckmarkBold} &\textcolor{red}{\XSolidBrush}&\textcolor{green}{\CheckmarkBold}\\
% OpenFunctions-v2~\citep{patil2023gorilla} & IFT & - & 65k & \textcolor{green}{\CheckmarkBold} & \textcolor{green}{\CheckmarkBold} &\textcolor{red}{\XSolidBrush}&\textcolor{green}{\CheckmarkBold}\\
% LAM~\citep{zhang2024agentohana} & IFT & - & 42.6k & \textcolor{green}{\CheckmarkBold} & \textcolor{green}{\CheckmarkBold} &\textcolor{green}{\CheckmarkBold}&\textcolor{red}{\XSolidBrush} \\
% xLAM~\citep{liu2024apigen} & IFT & - & 60k & \textcolor{green}{\CheckmarkBold} & \textcolor{green}{\CheckmarkBold} &\textcolor{green}{\CheckmarkBold}&\textcolor{red}{\XSolidBrush} \\\midrule
% LEMUR~\citep{xu2024lemur} & PT & 90B & 300k & \textcolor{green}{\CheckmarkBold} & \textcolor{green}{\CheckmarkBold} &\textcolor{green}{\CheckmarkBold}&\textcolor{red}{\XSolidBrush}\\
% \rowcolor{teal!12} \method & PT & 103B & 95k & \textcolor{green}{\CheckmarkBold} & \textcolor{green}{\CheckmarkBold} & \textcolor{green}{\CheckmarkBold} & \textcolor{green}{\CheckmarkBold} \\
% \bottomrule
% \end{tabular}
% \caption{Summary of existing tuning- and pretraining-based LLM agents with their training sample sizes. "PT" and "IFT" denote "Pre-Training" and "Instruction Fine-Tuning", respectively. }
% \label{tab:related}
% \end{table*}

\begin{table*}[ht]
\begin{threeparttable}
\centering 
\renewcommand\arraystretch{0.98}
\fontsize{7}{9}\selectfont \setlength{\tabcolsep}{0.2em}
\begin{tabular}{@{}l|c|c|ccc|cc|cc|cccc@{}}
\toprule
\textbf{Methods} & \textbf{Datasets}           & \begin{tabular}[c]{@{}c@{}}\textbf{Training}\\ \textbf{Paradigm}\end{tabular} & \begin{tabular}[c]{@{}c@{}}\textbf{\# PT Data}\\ \textbf{(Tokens)}\end{tabular} & \begin{tabular}[c]{@{}c@{}}\textbf{\# IFT Data}\\ \textbf{(Samples)}\end{tabular} & \textbf{\# APIs} & \textbf{Code}  & \begin{tabular}[c]{@{}c@{}}\textbf{Nat.}\\ \textbf{Lang.}\end{tabular} & \begin{tabular}[c]{@{}c@{}}\textbf{Action}\\ \textbf{Traj.}\end{tabular} & \begin{tabular}[c]{@{}c@{}}\textbf{API}\\ \textbf{Doc.}\end{tabular} & \begin{tabular}[c]{@{}c@{}}\textbf{Func.}\\ \textbf{Call}\end{tabular} & \begin{tabular}[c]{@{}c@{}}\textbf{Multi.}\\ \textbf{Step}\end{tabular}  & \begin{tabular}[c]{@{}c@{}}\textbf{Plan}\\ \textbf{Refine}\end{tabular}  & \begin{tabular}[c]{@{}c@{}}\textbf{Multi.}\\ \textbf{Turn}\end{tabular}\\ \midrule 
\multicolumn{13}{l}{\emph{Instruction Finetuning-based LLM Agents for Intrinsic Reasoning}}  \\ \midrule
FireAct~\cite{chen2023fireact} & FireAct & IFT & - & 2.1K & 10 & \textcolor{red}{\XSolidBrush} &\textcolor{green}{\CheckmarkBold} &\textcolor{green}{\CheckmarkBold}  & \textcolor{red}{\XSolidBrush} &\textcolor{green}{\CheckmarkBold} & \textcolor{red}{\XSolidBrush} &\textcolor{green}{\CheckmarkBold} & \textcolor{red}{\XSolidBrush} \\
ToolAlpaca~\cite{tang2023toolalpaca} & ToolAlpaca & IFT & - & 4.0K & 400 & \textcolor{red}{\XSolidBrush} &\textcolor{green}{\CheckmarkBold} &\textcolor{green}{\CheckmarkBold} & \textcolor{red}{\XSolidBrush} &\textcolor{green}{\CheckmarkBold} & \textcolor{red}{\XSolidBrush}  &\textcolor{green}{\CheckmarkBold} & \textcolor{red}{\XSolidBrush}  \\
ToolLLaMA~\cite{qin2023toolllm} & ToolBench & IFT & - & 12.7K & 16,464 & \textcolor{red}{\XSolidBrush} &\textcolor{green}{\CheckmarkBold} &\textcolor{green}{\CheckmarkBold} &\textcolor{red}{\XSolidBrush} &\textcolor{green}{\CheckmarkBold}&\textcolor{green}{\CheckmarkBold}&\textcolor{green}{\CheckmarkBold} &\textcolor{green}{\CheckmarkBold}\\
AgentEvol~\citep{xi2024agentgym} & AgentTraj-L & IFT & - & 14.5K & 24 &\textcolor{red}{\XSolidBrush} & \textcolor{green}{\CheckmarkBold} &\textcolor{green}{\CheckmarkBold}&\textcolor{red}{\XSolidBrush} &\textcolor{green}{\CheckmarkBold}&\textcolor{red}{\XSolidBrush} &\textcolor{red}{\XSolidBrush} &\textcolor{green}{\CheckmarkBold}\\
Lumos~\cite{yin2024agent} & Lumos & IFT  & - & 20.0K & 16 &\textcolor{red}{\XSolidBrush} & \textcolor{green}{\CheckmarkBold} & \textcolor{green}{\CheckmarkBold} &\textcolor{red}{\XSolidBrush} & \textcolor{green}{\CheckmarkBold} & \textcolor{green}{\CheckmarkBold} &\textcolor{red}{\XSolidBrush} & \textcolor{green}{\CheckmarkBold}\\
Agent-FLAN~\cite{chen2024agent} & Agent-FLAN & IFT & - & 24.7K & 20 &\textcolor{red}{\XSolidBrush} & \textcolor{green}{\CheckmarkBold} & \textcolor{green}{\CheckmarkBold} &\textcolor{red}{\XSolidBrush} & \textcolor{green}{\CheckmarkBold}& \textcolor{green}{\CheckmarkBold}&\textcolor{red}{\XSolidBrush} & \textcolor{green}{\CheckmarkBold}\\
AgentTuning~\citep{zeng2023agenttuning} & AgentInstruct & IFT & - & 35.0K & - &\textcolor{red}{\XSolidBrush} & \textcolor{green}{\CheckmarkBold} & \textcolor{green}{\CheckmarkBold} &\textcolor{red}{\XSolidBrush} & \textcolor{green}{\CheckmarkBold} &\textcolor{red}{\XSolidBrush} &\textcolor{red}{\XSolidBrush} & \textcolor{green}{\CheckmarkBold}\\\midrule
\multicolumn{13}{l}{\emph{Instruction Finetuning-based LLM Agents for Function Calling}} \\\midrule
NexusRaven~\citep{srinivasan2023nexusraven} & NexusRaven & IFT & - & - & 116 & \textcolor{green}{\CheckmarkBold} & \textcolor{green}{\CheckmarkBold}  & \textcolor{green}{\CheckmarkBold} &\textcolor{red}{\XSolidBrush} & \textcolor{green}{\CheckmarkBold} &\textcolor{red}{\XSolidBrush} &\textcolor{red}{\XSolidBrush}&\textcolor{red}{\XSolidBrush}\\
Gorilla~\citep{patil2023gorilla} & Gorilla & IFT & - & 16.0K & 1,645 & \textcolor{green}{\CheckmarkBold} &\textcolor{red}{\XSolidBrush} &\textcolor{red}{\XSolidBrush}&\textcolor{green}{\CheckmarkBold} &\textcolor{green}{\CheckmarkBold} &\textcolor{red}{\XSolidBrush} &\textcolor{red}{\XSolidBrush} &\textcolor{red}{\XSolidBrush}\\
OpenFunctions-v2~\citep{patil2023gorilla} & OpenFunctions-v2 & IFT & - & 65.0K & - & \textcolor{green}{\CheckmarkBold} & \textcolor{green}{\CheckmarkBold} &\textcolor{red}{\XSolidBrush} &\textcolor{green}{\CheckmarkBold} &\textcolor{green}{\CheckmarkBold} &\textcolor{red}{\XSolidBrush} &\textcolor{red}{\XSolidBrush} &\textcolor{red}{\XSolidBrush}\\
API Pack~\cite{guo2024api} & API Pack & IFT & - & 1.1M & 11,213 &\textcolor{green}{\CheckmarkBold} &\textcolor{red}{\XSolidBrush} &\textcolor{green}{\CheckmarkBold} &\textcolor{red}{\XSolidBrush} &\textcolor{green}{\CheckmarkBold} &\textcolor{red}{\XSolidBrush}&\textcolor{red}{\XSolidBrush}&\textcolor{red}{\XSolidBrush}\\ 
LAM~\citep{zhang2024agentohana} & AgentOhana & IFT & - & 42.6K & - & \textcolor{green}{\CheckmarkBold} & \textcolor{green}{\CheckmarkBold} &\textcolor{green}{\CheckmarkBold}&\textcolor{red}{\XSolidBrush} &\textcolor{green}{\CheckmarkBold}&\textcolor{red}{\XSolidBrush}&\textcolor{green}{\CheckmarkBold}&\textcolor{green}{\CheckmarkBold}\\
xLAM~\citep{liu2024apigen} & APIGen & IFT & - & 60.0K & 3,673 & \textcolor{green}{\CheckmarkBold} & \textcolor{green}{\CheckmarkBold} &\textcolor{green}{\CheckmarkBold}&\textcolor{red}{\XSolidBrush} &\textcolor{green}{\CheckmarkBold}&\textcolor{red}{\XSolidBrush}&\textcolor{green}{\CheckmarkBold}&\textcolor{green}{\CheckmarkBold}\\\midrule
\multicolumn{13}{l}{\emph{Pretraining-based LLM Agents}}  \\\midrule
% LEMUR~\citep{xu2024lemur} & PT & 90B & 300.0K & - & \textcolor{green}{\CheckmarkBold} & \textcolor{green}{\CheckmarkBold} &\textcolor{green}{\CheckmarkBold}&\textcolor{red}{\XSolidBrush} & \textcolor{red}{\XSolidBrush} &\textcolor{green}{\CheckmarkBold} &\textcolor{red}{\XSolidBrush}&\textcolor{red}{\XSolidBrush}\\
\rowcolor{teal!12} \method & \dataset & PT & 103B & 95.0K  & 76,537  & \textcolor{green}{\CheckmarkBold} & \textcolor{green}{\CheckmarkBold} & \textcolor{green}{\CheckmarkBold} & \textcolor{green}{\CheckmarkBold} & \textcolor{green}{\CheckmarkBold} & \textcolor{green}{\CheckmarkBold} & \textcolor{green}{\CheckmarkBold} & \textcolor{green}{\CheckmarkBold}\\
\bottomrule
\end{tabular}
% \begin{tablenotes}
%     \item $^*$ In addition, the StarCoder-API can offer 4.77M more APIs.
% \end{tablenotes}
\caption{Summary of existing instruction finetuning-based LLM agents for intrinsic reasoning and function calling, along with their training resources and sample sizes. "PT" and "IFT" denote "Pre-Training" and "Instruction Fine-Tuning", respectively.}
\vspace{-2ex}
\label{tab:related}
\end{threeparttable}
\end{table*}

\noindent \textbf{Prompting-based LLM Agents.} Due to the lack of agent-specific pre-training corpus, existing LLM agents rely on either prompt engineering~\cite{hsieh2023tool,lu2024chameleon,yao2022react,wang2023voyager} or instruction fine-tuning~\cite{chen2023fireact,zeng2023agenttuning} to understand human instructions, decompose high-level tasks, generate grounded plans, and execute multi-step actions. 
However, prompting-based methods mainly depend on the capabilities of backbone LLMs (usually commercial LLMs), failing to introduce new knowledge and struggling to generalize to unseen tasks~\cite{sun2024adaplanner,zhuang2023toolchain}. 

\noindent \textbf{Instruction Finetuning-based LLM Agents.} Considering the extensive diversity of APIs and the complexity of multi-tool instructions, tool learning inherently presents greater challenges than natural language tasks, such as text generation~\cite{qin2023toolllm}.
Post-training techniques focus more on instruction following and aligning output with specific formats~\cite{patil2023gorilla,hao2024toolkengpt,qin2023toolllm,schick2024toolformer}, rather than fundamentally improving model knowledge or capabilities. 
Moreover, heavy fine-tuning can hinder generalization or even degrade performance in non-agent use cases, potentially suppressing the original base model capabilities~\cite{ghosh2024a}.

\noindent \textbf{Pretraining-based LLM Agents.} While pre-training serves as an essential alternative, prior works~\cite{nijkamp2023codegen,roziere2023code,xu2024lemur,patil2023gorilla} have primarily focused on improving task-specific capabilities (\eg, code generation) instead of general-domain LLM agents, due to single-source, uni-type, small-scale, and poor-quality pre-training data. 
Existing tool documentation data for agent training either lacks diverse real-world APIs~\cite{patil2023gorilla, tang2023toolalpaca} or is constrained to single-tool or single-round tool execution. 
Furthermore, trajectory data mostly imitate expert behavior or follow function-calling rules with inferior planning and reasoning, failing to fully elicit LLMs' capabilities and handle complex instructions~\cite{qin2023toolllm}. 
Given a wide range of candidate API functions, each comprising various function names and parameters available at every planning step, identifying globally optimal solutions and generalizing across tasks remains highly challenging.



\section{Preliminaries}
\label{Preliminaries}
\begin{figure*}[t]
    \centering
    \includegraphics[width=0.95\linewidth]{fig/HealthGPT_Framework.png}
    \caption{The \ourmethod{} architecture integrates hierarchical visual perception and H-LoRA, employing a task-specific hard router to select visual features and H-LoRA plugins, ultimately generating outputs with an autoregressive manner.}
    \label{fig:architecture}
\end{figure*}
\noindent\textbf{Large Vision-Language Models.} 
The input to a LVLM typically consists of an image $x^{\text{img}}$ and a discrete text sequence $x^{\text{txt}}$. The visual encoder $\mathcal{E}^{\text{img}}$ converts the input image $x^{\text{img}}$ into a sequence of visual tokens $\mathcal{V} = [v_i]_{i=1}^{N_v}$, while the text sequence $x^{\text{txt}}$ is mapped into a sequence of text tokens $\mathcal{T} = [t_i]_{i=1}^{N_t}$ using an embedding function $\mathcal{E}^{\text{txt}}$. The LLM $\mathcal{M_\text{LLM}}(\cdot|\theta)$ models the joint probability of the token sequence $\mathcal{U} = \{\mathcal{V},\mathcal{T}\}$, which is expressed as:
\begin{equation}
    P_\theta(R | \mathcal{U}) = \prod_{i=1}^{N_r} P_\theta(r_i | \{\mathcal{U}, r_{<i}\}),
\end{equation}
where $R = [r_i]_{i=1}^{N_r}$ is the text response sequence. The LVLM iteratively generates the next token $r_i$ based on $r_{<i}$. The optimization objective is to minimize the cross-entropy loss of the response $\mathcal{R}$.
% \begin{equation}
%     \mathcal{L}_{\text{VLM}} = \mathbb{E}_{R|\mathcal{U}}\left[-\log P_\theta(R | \mathcal{U})\right]
% \end{equation}
It is worth noting that most LVLMs adopt a design paradigm based on ViT, alignment adapters, and pre-trained LLMs\cite{liu2023llava,liu2024improved}, enabling quick adaptation to downstream tasks.


\noindent\textbf{VQGAN.}
VQGAN~\cite{esser2021taming} employs latent space compression and indexing mechanisms to effectively learn a complete discrete representation of images. VQGAN first maps the input image $x^{\text{img}}$ to a latent representation $z = \mathcal{E}(x)$ through a encoder $\mathcal{E}$. Then, the latent representation is quantized using a codebook $\mathcal{Z} = \{z_k\}_{k=1}^K$, generating a discrete index sequence $\mathcal{I} = [i_m]_{m=1}^N$, where $i_m \in \mathcal{Z}$ represents the quantized code index:
\begin{equation}
    \mathcal{I} = \text{Quantize}(z|\mathcal{Z}) = \arg\min_{z_k \in \mathcal{Z}} \| z - z_k \|_2.
\end{equation}
In our approach, the discrete index sequence $\mathcal{I}$ serves as a supervisory signal for the generation task, enabling the model to predict the index sequence $\hat{\mathcal{I}}$ from input conditions such as text or other modality signals.  
Finally, the predicted index sequence $\hat{\mathcal{I}}$ is upsampled by the VQGAN decoder $G$, generating the high-quality image $\hat{x}^\text{img} = G(\hat{\mathcal{I}})$.



\noindent\textbf{Low Rank Adaptation.} 
LoRA\cite{hu2021lora} effectively captures the characteristics of downstream tasks by introducing low-rank adapters. The core idea is to decompose the bypass weight matrix $\Delta W\in\mathbb{R}^{d^{\text{in}} \times d^{\text{out}}}$ into two low-rank matrices $ \{A \in \mathbb{R}^{d^{\text{in}} \times r}, B \in \mathbb{R}^{r \times d^{\text{out}}} \}$, where $ r \ll \min\{d^{\text{in}}, d^{\text{out}}\} $, significantly reducing learnable parameters. The output with the LoRA adapter for the input $x$ is then given by:
\begin{equation}
    h = x W_0 + \alpha x \Delta W/r = x W_0 + \alpha xAB/r,
\end{equation}
where matrix $ A $ is initialized with a Gaussian distribution, while the matrix $ B $ is initialized as a zero matrix. The scaling factor $ \alpha/r $ controls the impact of $ \Delta W $ on the model.

\section{HealthGPT}
\label{Method}


\subsection{Unified Autoregressive Generation.}  
% As shown in Figure~\ref{fig:architecture}, 
\ourmethod{} (Figure~\ref{fig:architecture}) utilizes a discrete token representation that covers both text and visual outputs, unifying visual comprehension and generation as an autoregressive task. 
For comprehension, $\mathcal{M}_\text{llm}$ receives the input joint sequence $\mathcal{U}$ and outputs a series of text token $\mathcal{R} = [r_1, r_2, \dots, r_{N_r}]$, where $r_i \in \mathcal{V}_{\text{txt}}$, and $\mathcal{V}_{\text{txt}}$ represents the LLM's vocabulary:
\begin{equation}
    P_\theta(\mathcal{R} \mid \mathcal{U}) = \prod_{i=1}^{N_r} P_\theta(r_i \mid \mathcal{U}, r_{<i}).
\end{equation}
For generation, $\mathcal{M}_\text{llm}$ first receives a special start token $\langle \text{START\_IMG} \rangle$, then generates a series of tokens corresponding to the VQGAN indices $\mathcal{I} = [i_1, i_2, \dots, i_{N_i}]$, where $i_j \in \mathcal{V}_{\text{vq}}$, and $\mathcal{V}_{\text{vq}}$ represents the index range of VQGAN. Upon completion of generation, the LLM outputs an end token $\langle \text{END\_IMG} \rangle$:
\begin{equation}
    P_\theta(\mathcal{I} \mid \mathcal{U}) = \prod_{j=1}^{N_i} P_\theta(i_j \mid \mathcal{U}, i_{<j}).
\end{equation}
Finally, the generated index sequence $\mathcal{I}$ is fed into the decoder $G$, which reconstructs the target image $\hat{x}^{\text{img}} = G(\mathcal{I})$.

\subsection{Hierarchical Visual Perception}  
Given the differences in visual perception between comprehension and generation tasks—where the former focuses on abstract semantics and the latter emphasizes complete semantics—we employ ViT to compress the image into discrete visual tokens at multiple hierarchical levels.
Specifically, the image is converted into a series of features $\{f_1, f_2, \dots, f_L\}$ as it passes through $L$ ViT blocks.

To address the needs of various tasks, the hidden states are divided into two types: (i) \textit{Concrete-grained features} $\mathcal{F}^{\text{Con}} = \{f_1, f_2, \dots, f_k\}, k < L$, derived from the shallower layers of ViT, containing sufficient global features, suitable for generation tasks; 
(ii) \textit{Abstract-grained features} $\mathcal{F}^{\text{Abs}} = \{f_{k+1}, f_{k+2}, \dots, f_L\}$, derived from the deeper layers of ViT, which contain abstract semantic information closer to the text space, suitable for comprehension tasks.

The task type $T$ (comprehension or generation) determines which set of features is selected as the input for the downstream large language model:
\begin{equation}
    \mathcal{F}^{\text{img}}_T =
    \begin{cases}
        \mathcal{F}^{\text{Con}}, & \text{if } T = \text{generation task} \\
        \mathcal{F}^{\text{Abs}}, & \text{if } T = \text{comprehension task}
    \end{cases}
\end{equation}
We integrate the image features $\mathcal{F}^{\text{img}}_T$ and text features $\mathcal{T}$ into a joint sequence through simple concatenation, which is then fed into the LLM $\mathcal{M}_{\text{llm}}$ for autoregressive generation.
% :
% \begin{equation}
%     \mathcal{R} = \mathcal{M}_{\text{llm}}(\mathcal{U}|\theta), \quad \mathcal{U} = [\mathcal{F}^{\text{img}}_T; \mathcal{T}]
% \end{equation}
\subsection{Heterogeneous Knowledge Adaptation}
We devise H-LoRA, which stores heterogeneous knowledge from comprehension and generation tasks in separate modules and dynamically routes to extract task-relevant knowledge from these modules. 
At the task level, for each task type $ T $, we dynamically assign a dedicated H-LoRA submodule $ \theta^T $, which is expressed as:
\begin{equation}
    \mathcal{R} = \mathcal{M}_\text{LLM}(\mathcal{U}|\theta, \theta^T), \quad \theta^T = \{A^T, B^T, \mathcal{R}^T_\text{outer}\}.
\end{equation}
At the feature level for a single task, H-LoRA integrates the idea of Mixture of Experts (MoE)~\cite{masoudnia2014mixture} and designs an efficient matrix merging and routing weight allocation mechanism, thus avoiding the significant computational delay introduced by matrix splitting in existing MoELoRA~\cite{luo2024moelora}. Specifically, we first merge the low-rank matrices (rank = r) of $ k $ LoRA experts into a unified matrix:
\begin{equation}
    \mathbf{A}^{\text{merged}}, \mathbf{B}^{\text{merged}} = \text{Concat}(\{A_i\}_1^k), \text{Concat}(\{B_i\}_1^k),
\end{equation}
where $ \mathbf{A}^{\text{merged}} \in \mathbb{R}^{d^\text{in} \times rk} $ and $ \mathbf{B}^{\text{merged}} \in \mathbb{R}^{rk \times d^\text{out}} $. The $k$-dimension routing layer generates expert weights $ \mathcal{W} \in \mathbb{R}^{\text{token\_num} \times k} $ based on the input hidden state $ x $, and these are expanded to $ \mathbb{R}^{\text{token\_num} \times rk} $ as follows:
\begin{equation}
    \mathcal{W}^\text{expanded} = \alpha k \mathcal{W} / r \otimes \mathbf{1}_r,
\end{equation}
where $ \otimes $ denotes the replication operation.
The overall output of H-LoRA is computed as:
\begin{equation}
    \mathcal{O}^\text{H-LoRA} = (x \mathbf{A}^{\text{merged}} \odot \mathcal{W}^\text{expanded}) \mathbf{B}^{\text{merged}},
\end{equation}
where $ \odot $ represents element-wise multiplication. Finally, the output of H-LoRA is added to the frozen pre-trained weights to produce the final output:
\begin{equation}
    \mathcal{O} = x W_0 + \mathcal{O}^\text{H-LoRA}.
\end{equation}
% In summary, H-LoRA is a task-based dynamic PEFT method that achieves high efficiency in single-task fine-tuning.

\subsection{Training Pipeline}

\begin{figure}[t]
    \centering
    \hspace{-4mm}
    \includegraphics[width=0.94\linewidth]{fig/data.pdf}
    \caption{Data statistics of \texttt{VL-Health}. }
    \label{fig:data}
\end{figure}
\noindent \textbf{1st Stage: Multi-modal Alignment.} 
In the first stage, we design separate visual adapters and H-LoRA submodules for medical unified tasks. For the medical comprehension task, we train abstract-grained visual adapters using high-quality image-text pairs to align visual embeddings with textual embeddings, thereby enabling the model to accurately describe medical visual content. During this process, the pre-trained LLM and its corresponding H-LoRA submodules remain frozen. In contrast, the medical generation task requires training concrete-grained adapters and H-LoRA submodules while keeping the LLM frozen. Meanwhile, we extend the textual vocabulary to include multimodal tokens, enabling the support of additional VQGAN vector quantization indices. The model trains on image-VQ pairs, endowing the pre-trained LLM with the capability for image reconstruction. This design ensures pixel-level consistency of pre- and post-LVLM. The processes establish the initial alignment between the LLM’s outputs and the visual inputs.

\noindent \textbf{2nd Stage: Heterogeneous H-LoRA Plugin Adaptation.}  
The submodules of H-LoRA share the word embedding layer and output head but may encounter issues such as bias and scale inconsistencies during training across different tasks. To ensure that the multiple H-LoRA plugins seamlessly interface with the LLMs and form a unified base, we fine-tune the word embedding layer and output head using a small amount of mixed data to maintain consistency in the model weights. Specifically, during this stage, all H-LoRA submodules for different tasks are kept frozen, with only the word embedding layer and output head being optimized. Through this stage, the model accumulates foundational knowledge for unified tasks by adapting H-LoRA plugins.

\begin{table*}[!t]
\centering
\caption{Comparison of \ourmethod{} with other LVLMs and unified multi-modal models on medical visual comprehension tasks. \textbf{Bold} and \underline{underlined} text indicates the best performance and second-best performance, respectively.}
\resizebox{\textwidth}{!}{
\begin{tabular}{c|lcc|cccccccc|c}
\toprule
\rowcolor[HTML]{E9F3FE} &  &  &  & \multicolumn{2}{c}{\textbf{VQA-RAD \textuparrow}} & \multicolumn{2}{c}{\textbf{SLAKE \textuparrow}} & \multicolumn{2}{c}{\textbf{PathVQA \textuparrow}} &  &  &  \\ 
\cline{5-10}
\rowcolor[HTML]{E9F3FE}\multirow{-2}{*}{\textbf{Type}} & \multirow{-2}{*}{\textbf{Model}} & \multirow{-2}{*}{\textbf{\# Params}} & \multirow{-2}{*}{\makecell{\textbf{Medical} \\ \textbf{LVLM}}} & \textbf{close} & \textbf{all} & \textbf{close} & \textbf{all} & \textbf{close} & \textbf{all} & \multirow{-2}{*}{\makecell{\textbf{MMMU} \\ \textbf{-Med}}\textuparrow} & \multirow{-2}{*}{\textbf{OMVQA}\textuparrow} & \multirow{-2}{*}{\textbf{Avg. \textuparrow}} \\ 
\midrule \midrule
\multirow{9}{*}{\textbf{Comp. Only}} 
& Med-Flamingo & 8.3B & \Large \ding{51} & 58.6 & 43.0 & 47.0 & 25.5 & 61.9 & 31.3 & 28.7 & 34.9 & 41.4 \\
& LLaVA-Med & 7B & \Large \ding{51} & 60.2 & 48.1 & 58.4 & 44.8 & 62.3 & 35.7 & 30.0 & 41.3 & 47.6 \\
& HuatuoGPT-Vision & 7B & \Large \ding{51} & 66.9 & 53.0 & 59.8 & 49.1 & 52.9 & 32.0 & 42.0 & 50.0 & 50.7 \\
& BLIP-2 & 6.7B & \Large \ding{55} & 43.4 & 36.8 & 41.6 & 35.3 & 48.5 & 28.8 & 27.3 & 26.9 & 36.1 \\
& LLaVA-v1.5 & 7B & \Large \ding{55} & 51.8 & 42.8 & 37.1 & 37.7 & 53.5 & 31.4 & 32.7 & 44.7 & 41.5 \\
& InstructBLIP & 7B & \Large \ding{55} & 61.0 & 44.8 & 66.8 & 43.3 & 56.0 & 32.3 & 25.3 & 29.0 & 44.8 \\
& Yi-VL & 6B & \Large \ding{55} & 52.6 & 42.1 & 52.4 & 38.4 & 54.9 & 30.9 & 38.0 & 50.2 & 44.9 \\
& InternVL2 & 8B & \Large \ding{55} & 64.9 & 49.0 & 66.6 & 50.1 & 60.0 & 31.9 & \underline{43.3} & 54.5 & 52.5\\
& Llama-3.2 & 11B & \Large \ding{55} & 68.9 & 45.5 & 72.4 & 52.1 & 62.8 & 33.6 & 39.3 & 63.2 & 54.7 \\
\midrule
\multirow{5}{*}{\textbf{Comp. \& Gen.}} 
& Show-o & 1.3B & \Large \ding{55} & 50.6 & 33.9 & 31.5 & 17.9 & 52.9 & 28.2 & 22.7 & 45.7 & 42.6 \\
& Unified-IO 2 & 7B & \Large \ding{55} & 46.2 & 32.6 & 35.9 & 21.9 & 52.5 & 27.0 & 25.3 & 33.0 & 33.8 \\
& Janus & 1.3B & \Large \ding{55} & 70.9 & 52.8 & 34.7 & 26.9 & 51.9 & 27.9 & 30.0 & 26.8 & 33.5 \\
& \cellcolor[HTML]{DAE0FB}HealthGPT-M3 & \cellcolor[HTML]{DAE0FB}3.8B & \cellcolor[HTML]{DAE0FB}\Large \ding{51} & \cellcolor[HTML]{DAE0FB}\underline{73.7} & \cellcolor[HTML]{DAE0FB}\underline{55.9} & \cellcolor[HTML]{DAE0FB}\underline{74.6} & \cellcolor[HTML]{DAE0FB}\underline{56.4} & \cellcolor[HTML]{DAE0FB}\underline{78.7} & \cellcolor[HTML]{DAE0FB}\underline{39.7} & \cellcolor[HTML]{DAE0FB}\underline{43.3} & \cellcolor[HTML]{DAE0FB}\underline{68.5} & \cellcolor[HTML]{DAE0FB}\underline{61.3} \\
& \cellcolor[HTML]{DAE0FB}HealthGPT-L14 & \cellcolor[HTML]{DAE0FB}14B & \cellcolor[HTML]{DAE0FB}\Large \ding{51} & \cellcolor[HTML]{DAE0FB}\textbf{77.7} & \cellcolor[HTML]{DAE0FB}\textbf{58.3} & \cellcolor[HTML]{DAE0FB}\textbf{76.4} & \cellcolor[HTML]{DAE0FB}\textbf{64.5} & \cellcolor[HTML]{DAE0FB}\textbf{85.9} & \cellcolor[HTML]{DAE0FB}\textbf{44.4} & \cellcolor[HTML]{DAE0FB}\textbf{49.2} & \cellcolor[HTML]{DAE0FB}\textbf{74.4} & \cellcolor[HTML]{DAE0FB}\textbf{66.4} \\
\bottomrule
\end{tabular}
}
\label{tab:results}
\end{table*}
\begin{table*}[ht]
    \centering
    \caption{The experimental results for the four modality conversion tasks.}
    \resizebox{\textwidth}{!}{
    \begin{tabular}{l|ccc|ccc|ccc|ccc}
        \toprule
        \rowcolor[HTML]{E9F3FE} & \multicolumn{3}{c}{\textbf{CT to MRI (Brain)}} & \multicolumn{3}{c}{\textbf{CT to MRI (Pelvis)}} & \multicolumn{3}{c}{\textbf{MRI to CT (Brain)}} & \multicolumn{3}{c}{\textbf{MRI to CT (Pelvis)}} \\
        \cline{2-13}
        \rowcolor[HTML]{E9F3FE}\multirow{-2}{*}{\textbf{Model}}& \textbf{SSIM $\uparrow$} & \textbf{PSNR $\uparrow$} & \textbf{MSE $\downarrow$} & \textbf{SSIM $\uparrow$} & \textbf{PSNR $\uparrow$} & \textbf{MSE $\downarrow$} & \textbf{SSIM $\uparrow$} & \textbf{PSNR $\uparrow$} & \textbf{MSE $\downarrow$} & \textbf{SSIM $\uparrow$} & \textbf{PSNR $\uparrow$} & \textbf{MSE $\downarrow$} \\
        \midrule \midrule
        pix2pix & 71.09 & 32.65 & 36.85 & 59.17 & 31.02 & 51.91 & 78.79 & 33.85 & 28.33 & 72.31 & 32.98 & 36.19 \\
        CycleGAN & 54.76 & 32.23 & 40.56 & 54.54 & 30.77 & 55.00 & 63.75 & 31.02 & 52.78 & 50.54 & 29.89 & 67.78 \\
        BBDM & {71.69} & {32.91} & {34.44} & 57.37 & 31.37 & 48.06 & \textbf{86.40} & 34.12 & 26.61 & {79.26} & 33.15 & 33.60 \\
        Vmanba & 69.54 & 32.67 & 36.42 & {63.01} & {31.47} & {46.99} & 79.63 & 34.12 & 26.49 & 77.45 & 33.53 & 31.85 \\
        DiffMa & 71.47 & 32.74 & 35.77 & 62.56 & 31.43 & 47.38 & 79.00 & {34.13} & {26.45} & 78.53 & {33.68} & {30.51} \\
        \rowcolor[HTML]{DAE0FB}HealthGPT-M3 & \underline{79.38} & \underline{33.03} & \underline{33.48} & \underline{71.81} & \underline{31.83} & \underline{43.45} & {85.06} & \textbf{34.40} & \textbf{25.49} & \underline{84.23} & \textbf{34.29} & \textbf{27.99} \\
        \rowcolor[HTML]{DAE0FB}HealthGPT-L14 & \textbf{79.73} & \textbf{33.10} & \textbf{32.96} & \textbf{71.92} & \textbf{31.87} & \textbf{43.09} & \underline{85.31} & \underline{34.29} & \underline{26.20} & \textbf{84.96} & \underline{34.14} & \underline{28.13} \\
        \bottomrule
    \end{tabular}
    }
    \label{tab:conversion}
\end{table*}

\noindent \textbf{3rd Stage: Visual Instruction Fine-Tuning.}  
In the third stage, we introduce additional task-specific data to further optimize the model and enhance its adaptability to downstream tasks such as medical visual comprehension (e.g., medical QA, medical dialogues, and report generation) or generation tasks (e.g., super-resolution, denoising, and modality conversion). Notably, by this stage, the word embedding layer and output head have been fine-tuned, only the H-LoRA modules and adapter modules need to be trained. This strategy significantly improves the model's adaptability and flexibility across different tasks.


\section{Experiment}
\label{s:experiment}

\subsection{Data Description}
We evaluate our method on FI~\cite{you2016building}, Twitter\_LDL~\cite{yang2017learning} and Artphoto~\cite{machajdik2010affective}.
FI is a public dataset built from Flickr and Instagram, with 23,308 images and eight emotion categories, namely \textit{amusement}, \textit{anger}, \textit{awe},  \textit{contentment}, \textit{disgust}, \textit{excitement},  \textit{fear}, and \textit{sadness}. 
% Since images in FI are all copyrighted by law, some images are corrupted now, so we remove these samples and retain 21,828 images.
% T4SA contains images from Twitter, which are classified into three categories: \textit{positive}, \textit{neutral}, and \textit{negative}. In this paper, we adopt the base version of B-T4SA, which contains 470,586 images and provides text descriptions of the corresponding tweets.
Twitter\_LDL contains 10,045 images from Twitter, with the same eight categories as the FI dataset.
% 。
For these two datasets, they are randomly split into 80\%
training and 20\% testing set.
Artphoto contains 806 artistic photos from the DeviantArt website, which we use to further evaluate the zero-shot capability of our model.
% on the small-scale dataset.
% We construct and publicly release the first image sentiment analysis dataset containing metadata.
% 。

% Based on these datasets, we are the first to construct and publicly release metadata-enhanced image sentiment analysis datasets. These datasets include scenes, tags, descriptions, and corresponding confidence scores, and are available at this link for future research purposes.


% 
\begin{table}[t]
\centering
% \begin{center}
\caption{Overall performance of different models on FI and Twitter\_LDL datasets.}
\label{tab:cap1}
% \resizebox{\linewidth}{!}
{
\begin{tabular}{l|c|c|c|c}
\hline
\multirow{2}{*}{\textbf{Model}} & \multicolumn{2}{c|}{\textbf{FI}}  & \multicolumn{2}{c}{\textbf{Twitter\_LDL}} \\ \cline{2-5} 
  & \textbf{Accuracy} & \textbf{F1} & \textbf{Accuracy} & \textbf{F1}  \\ \hline
% (\rownumber)~AlexNet~\cite{krizhevsky2017imagenet}  & 58.13\% & 56.35\%  & 56.24\%& 55.02\%  \\ 
% (\rownumber)~VGG16~\cite{simonyan2014very}  & 63.75\%& 63.08\%  & 59.34\%& 59.02\%  \\ 
(\rownumber)~ResNet101~\cite{he2016deep} & 66.16\%& 65.56\%  & 62.02\% & 61.34\%  \\ 
(\rownumber)~CDA~\cite{han2023boosting} & 66.71\%& 65.37\%  & 64.14\% & 62.85\%  \\ 
(\rownumber)~CECCN~\cite{ruan2024color} & 67.96\%& 66.74\%  & 64.59\%& 64.72\% \\ 
(\rownumber)~EmoVIT~\cite{xie2024emovit} & 68.09\%& 67.45\%  & 63.12\% & 61.97\%  \\ 
(\rownumber)~ComLDL~\cite{zhang2022compound} & 68.83\%& 67.28\%  & 65.29\% & 63.12\%  \\ 
(\rownumber)~WSDEN~\cite{li2023weakly} & 69.78\%& 69.61\%  & 67.04\% & 65.49\% \\ 
(\rownumber)~ECWA~\cite{deng2021emotion} & 70.87\%& 69.08\%  & 67.81\% & 66.87\%  \\ 
(\rownumber)~EECon~\cite{yang2023exploiting} & 71.13\%& 68.34\%  & 64.27\%& 63.16\%  \\ 
(\rownumber)~MAM~\cite{zhang2024affective} & 71.44\%  & 70.83\% & 67.18\%  & 65.01\%\\ 
(\rownumber)~TGCA-PVT~\cite{chen2024tgca}   & 73.05\%  & 71.46\% & 69.87\%  & 68.32\% \\ 
(\rownumber)~OEAN~\cite{zhang2024object}   & 73.40\%  & 72.63\% & 70.52\%  & 69.47\% \\ \hline
(\rownumber)~\shortname  & \textbf{79.48\%} & \textbf{79.22\%} & \textbf{74.12\%} & \textbf{73.09\%} \\ \hline
\end{tabular}
}
\vspace{-6mm}
% \end{center}
\end{table}
% 

\subsection{Experiment Setting}
% \subsubsection{Model Setting.}
% 
\textbf{Model Setting:}
For feature representation, we set $k=10$ to select object tags, and adopt clip-vit-base-patch32 as the pre-trained model for unified feature representation.
Moreover, we empirically set $(d_e, d_h, d_k, d_s) = (512, 128, 16, 64)$, and set the classification class $L$ to 8.

% 

\textbf{Training Setting:}
To initialize the model, we set all weights such as $\boldsymbol{W}$ following the truncated normal distribution, and use AdamW optimizer with the learning rate of $1 \times 10^{-4}$.
% warmup scheduler of cosine, warmup steps of 2000.
Furthermore, we set the batch size to 32 and the epoch of the training process to 200.
During the implementation, we utilize \textit{PyTorch} to build our entire model.
% , and our project codes are publicly available at https://github.com/zzmyrep/MESN.
% Our project codes as well as data are all publicly available on GitHub\footnote{https://github.com/zzmyrep/KBCEN}.
% Code is available at \href{https://github.com/zzmyrep/KBCEN}{https://github.com/zzmyrep/KBCEN}.

\textbf{Evaluation Metrics:}
Following~\cite{zhang2024affective, chen2024tgca, zhang2024object}, we adopt \textit{accuracy} and \textit{F1} as our evaluation metrics to measure the performance of different methods for image sentiment analysis. 



\subsection{Experiment Result}
% We compare our model against the following baselines: AlexNet~\cite{krizhevsky2017imagenet}, VGG16~\cite{simonyan2014very}, ResNet101~\cite{he2016deep}, CECCN~\cite{ruan2024color}, EmoVIT~\cite{xie2024emovit}, WSCNet~\cite{yang2018weakly}, ECWA~\cite{deng2021emotion}, EECon~\cite{yang2023exploiting}, MAM~\cite{zhang2024affective} and TGCA-PVT~\cite{chen2024tgca}, and the overall results are summarized in Table~\ref{tab:cap1}.
We compare our model against several baselines, and the overall results are summarized in Table~\ref{tab:cap1}.
We observe that our model achieves the best performance in both accuracy and F1 metrics, significantly outperforming the previous models. 
This superior performance is mainly attributed to our effective utilization of metadata to enhance image sentiment analysis, as well as the exceptional capability of the unified sentiment transformer framework we developed. These results strongly demonstrate that our proposed method can bring encouraging performance for image sentiment analysis.

\setcounter{magicrownumbers}{0} 
\begin{table}[t]
\begin{center}
\caption{Ablation study of~\shortname~on FI dataset.} 
% \vspace{1mm}
\label{tab:cap2}
\resizebox{.9\linewidth}{!}
{
\begin{tabular}{lcc}
  \hline
  \textbf{Model} & \textbf{Accuracy} & \textbf{F1} \\
  \hline
  (\rownumber)~Ours (w/o vision) & 65.72\% & 64.54\% \\
  (\rownumber)~Ours (w/o text description) & 74.05\% & 72.58\% \\
  (\rownumber)~Ours (w/o object tag) & 77.45\% & 76.84\% \\
  (\rownumber)~Ours (w/o scene tag) & 78.47\% & 78.21\% \\
  \hline
  (\rownumber)~Ours (w/o unified embedding) & 76.41\% & 76.23\% \\
  (\rownumber)~Ours (w/o adaptive learning) & 76.83\% & 76.56\% \\
  (\rownumber)~Ours (w/o cross-modal fusion) & 76.85\% & 76.49\% \\
  \hline
  (\rownumber)~Ours  & \textbf{79.48\%} & \textbf{79.22\%} \\
  \hline
\end{tabular}
}
\end{center}
\vspace{-5mm}
\end{table}


\begin{figure}[t]
\centering
% \vspace{-2mm}
\includegraphics[width=0.42\textwidth]{fig/2dvisual-linux4-paper2.pdf}
\caption{Visualization of feature distribution on eight categories before (left) and after (right) model processing.}
% 
\label{fig:visualization}
\vspace{-5mm}
\end{figure}

\subsection{Ablation Performance}
In this subsection, we conduct an ablation study to examine which component is really important for performance improvement. The results are reported in Table~\ref{tab:cap2}.

For information utilization, we observe a significant decline in model performance when visual features are removed. Additionally, the performance of \shortname~decreases when different metadata are removed separately, which means that text description, object tag, and scene tag are all critical for image sentiment analysis.
Recalling the model architecture, we separately remove transformer layers of the unified representation module, the adaptive learning module, and the cross-modal fusion module, replacing them with MLPs of the same parameter scale.
In this way, we can observe varying degrees of decline in model performance, indicating that these modules are indispensable for our model to achieve better performance.

\subsection{Visualization}
% 


% % 开始使用minipage进行左右排列
% \begin{minipage}[t]{0.45\textwidth}  % 子图1宽度为45%
%     \centering
%     \includegraphics[width=\textwidth]{2dvisual.pdf}  % 插入图片
%     \captionof{figure}{Visualization of feature distribution.}  % 使用captionof添加图片标题
%     \label{fig:visualization}
% \end{minipage}


% \begin{figure}[t]
% \centering
% \vspace{-2mm}
% \includegraphics[width=0.45\textwidth]{fig/2dvisual.pdf}
% \caption{Visualization of feature distribution.}
% \label{fig:visualization}
% % \vspace{-4mm}
% \end{figure}

% \begin{figure}[t]
% \centering
% \vspace{-2mm}
% \includegraphics[width=0.45\textwidth]{fig/2dvisual-linux3-paper.pdf}
% \caption{Visualization of feature distribution.}
% \label{fig:visualization}
% % \vspace{-4mm}
% \end{figure}



\begin{figure}[tbp]   
\vspace{-4mm}
  \centering            
  \subfloat[Depth of adaptive learning layers]   
  {
    \label{fig:subfig1}\includegraphics[width=0.22\textwidth]{fig/fig_sensitivity-a5}
  }
  \subfloat[Depth of fusion layers]
  {
    % \label{fig:subfig2}\includegraphics[width=0.22\textwidth]{fig/fig_sensitivity-b2}
    \label{fig:subfig2}\includegraphics[width=0.22\textwidth]{fig/fig_sensitivity-b2-num.pdf}
  }
  \caption{Sensitivity study of \shortname~on different depth. }   
  \label{fig:fig_sensitivity}  
\vspace{-2mm}
\end{figure}

% \begin{figure}[htbp]
% \centerline{\includegraphics{2dvisual.pdf}}
% \caption{Visualization of feature distribution.}
% \label{fig:visualization}
% \end{figure}

% In Fig.~\ref{fig:visualization}, we use t-SNE~\cite{van2008visualizing} to reduce the dimension of data features for visualization, Figure in left represents the metadata features before model processing, the features are obtained by embedding through the CLIP model, and figure in right shows the features of the data after model processing, it can be observed that after the model processing, the data with different label categories fall in different regions in the space, therefore, we can conclude that the Therefore, we can conclude that the model can effectively utilize the information contained in the metadata and use it to guide the model for classification.

In Fig.~\ref{fig:visualization}, we use t-SNE~\cite{van2008visualizing} to reduce the dimension of data features for visualization.
The left figure shows metadata features before being processed by our model (\textit{i.e.}, embedded by CLIP), while the right shows the distribution of features after being processed by our model.
We can observe that after the model processing, data with the same label are closer to each other, while others are farther away.
Therefore, it shows that the model can effectively utilize the information contained in the metadata and use it to guide the classification process.

\subsection{Sensitivity Analysis}
% 
In this subsection, we conduct a sensitivity analysis to figure out the effect of different depth settings of adaptive learning layers and fusion layers. 
% In this subsection, we conduct a sensitivity analysis to figure out the effect of different depth settings on the model. 
% Fig.~\ref{fig:fig_sensitivity} presents the effect of different depth settings of adaptive learning layers and fusion layers. 
Taking Fig.~\ref{fig:fig_sensitivity} (a) as an example, the model performance improves with increasing depth, reaching the best performance at a depth of 4.
% Taking Fig.~\ref{fig:fig_sensitivity} (a) as an example, the performance of \shortname~improves with the increase of depth at first, reaching the best performance at a depth of 4.
When the depth continues to increase, the accuracy decreases to varying degrees.
Similar results can be observed in Fig.~\ref{fig:fig_sensitivity} (b).
Therefore, we set their depths to 4 and 6 respectively to achieve the best results.

% Through our experiments, we can observe that the effect of modifying these hyperparameters on the results of the experiments is very weak, and the surface model is not sensitive to the hyperparameters.


\subsection{Zero-shot Capability}
% 

% (1)~GCH~\cite{2010Analyzing} & 21.78\% & (5)~RA-DLNet~\cite{2020A} & 34.01\% \\ \hline
% (2)~WSCNet~\cite{2019WSCNet}  & 30.25\% & (6)~CECCN~\cite{ruan2024color} & 43.83\% \\ \hline
% (3)~PCNN~\cite{2015Robust} & 31.68\%  & (7)~EmoVIT~\cite{xie2024emovit} & 44.90\% \\ \hline
% (4)~AR~\cite{2018Visual} & 32.67\% & (8)~Ours (Zero-shot) & 47.83\% \\ \hline


\begin{table}[t]
\centering
\caption{Zero-shot capability of \shortname.}
\label{tab:cap3}
\resizebox{1\linewidth}{!}
{
\begin{tabular}{lc|lc}
\hline
\textbf{Model} & \textbf{Accuracy} & \textbf{Model} & \textbf{Accuracy} \\ \hline
(1)~WSCNet~\cite{2019WSCNet}  & 30.25\% & (5)~MAM~\cite{zhang2024affective} & 39.56\%  \\ \hline
(2)~AR~\cite{2018Visual} & 32.67\% & (6)~CECCN~\cite{ruan2024color} & 43.83\% \\ \hline
(3)~RA-DLNet~\cite{2020A} & 34.01\%  & (7)~EmoVIT~\cite{xie2024emovit} & 44.90\% \\ \hline
(4)~CDA~\cite{han2023boosting} & 38.64\% & (8)~Ours (Zero-shot) & 47.83\% \\ \hline
\end{tabular}
}
\vspace{-5mm}
\end{table}

% We use the model trained on the FI dataset to test on the artphoto dataset to verify the model's generalization ability as well as robustness to other distributed datasets.
% We can observe that the MESN model shows strong competitiveness in terms of accuracy when compared to other trained models, which suggests that the model has a good generalization ability in the OOD task.

To validate the model's generalization ability and robustness to other distributed datasets, we directly test the model trained on the FI dataset, without training on Artphoto. 
% As observed in Table 3, compared to other models trained on Artphoto, we achieve highly competitive zero-shot performance, indicating that the model has good generalization ability in out-of-distribution tasks.
From Table~\ref{tab:cap3}, we can observe that compared with other models trained on Artphoto, we achieve competitive zero-shot performance, which shows that the model has good generalization ability in out-of-distribution tasks.


%%%%%%%%%%%%
%  E2E     %
%%%%%%%%%%%%


\section{Conclusion}
In this paper, we introduced Wi-Chat, the first LLM-powered Wi-Fi-based human activity recognition system that integrates the reasoning capabilities of large language models with the sensing potential of wireless signals. Our experimental results on a self-collected Wi-Fi CSI dataset demonstrate the promising potential of LLMs in enabling zero-shot Wi-Fi sensing. These findings suggest a new paradigm for human activity recognition that does not rely on extensive labeled data. We hope future research will build upon this direction, further exploring the applications of LLMs in signal processing domains such as IoT, mobile sensing, and radar-based systems.

\section*{Limitations}
While our work represents the first attempt to leverage LLMs for processing Wi-Fi signals, it is a preliminary study focused on a relatively simple task: Wi-Fi-based human activity recognition. This choice allows us to explore the feasibility of LLMs in wireless sensing but also comes with certain limitations.

Our approach primarily evaluates zero-shot performance, which, while promising, may still lag behind traditional supervised learning methods in highly complex or fine-grained recognition tasks. Besides, our study is limited to a controlled environment with a self-collected dataset, and the generalizability of LLMs to diverse real-world scenarios with varying Wi-Fi conditions, environmental interference, and device heterogeneity remains an open question.

Additionally, we have yet to explore the full potential of LLMs in more advanced Wi-Fi sensing applications, such as fine-grained gesture recognition, occupancy detection, and passive health monitoring. Future work should investigate the scalability of LLM-based approaches, their robustness to domain shifts, and their integration with multimodal sensing techniques in broader IoT applications.


% Bibliography entries for the entire Anthology, followed by custom entries
%\bibliography{anthology,custom}
% Custom bibliography entries only
\bibliography{main}
\newpage
\appendix

\section{Experiment prompts}
\label{sec:prompt}
The prompts used in the LLM experiments are shown in the following Table~\ref{tab:prompts}.

\definecolor{titlecolor}{rgb}{0.9, 0.5, 0.1}
\definecolor{anscolor}{rgb}{0.2, 0.5, 0.8}
\definecolor{labelcolor}{HTML}{48a07e}
\begin{table*}[h]
	\centering
	
 % \vspace{-0.2cm}
	
	\begin{center}
		\begin{tikzpicture}[
				chatbox_inner/.style={rectangle, rounded corners, opacity=0, text opacity=1, font=\sffamily\scriptsize, text width=5in, text height=9pt, inner xsep=6pt, inner ysep=6pt},
				chatbox_prompt_inner/.style={chatbox_inner, align=flush left, xshift=0pt, text height=11pt},
				chatbox_user_inner/.style={chatbox_inner, align=flush left, xshift=0pt},
				chatbox_gpt_inner/.style={chatbox_inner, align=flush left, xshift=0pt},
				chatbox/.style={chatbox_inner, draw=black!25, fill=gray!7, opacity=1, text opacity=0},
				chatbox_prompt/.style={chatbox, align=flush left, fill=gray!1.5, draw=black!30, text height=10pt},
				chatbox_user/.style={chatbox, align=flush left},
				chatbox_gpt/.style={chatbox, align=flush left},
				chatbox2/.style={chatbox_gpt, fill=green!25},
				chatbox3/.style={chatbox_gpt, fill=red!20, draw=black!20},
				chatbox4/.style={chatbox_gpt, fill=yellow!30},
				labelbox/.style={rectangle, rounded corners, draw=black!50, font=\sffamily\scriptsize\bfseries, fill=gray!5, inner sep=3pt},
			]
											
			\node[chatbox_user] (q1) {
				\textbf{System prompt}
				\newline
				\newline
				You are a helpful and precise assistant for segmenting and labeling sentences. We would like to request your help on curating a dataset for entity-level hallucination detection.
				\newline \newline
                We will give you a machine generated biography and a list of checked facts about the biography. Each fact consists of a sentence and a label (True/False). Please do the following process. First, breaking down the biography into words. Second, by referring to the provided list of facts, merging some broken down words in the previous step to form meaningful entities. For example, ``strategic thinking'' should be one entity instead of two. Third, according to the labels in the list of facts, labeling each entity as True or False. Specifically, for facts that share a similar sentence structure (\eg, \textit{``He was born on Mach 9, 1941.''} (\texttt{True}) and \textit{``He was born in Ramos Mejia.''} (\texttt{False})), please first assign labels to entities that differ across atomic facts. For example, first labeling ``Mach 9, 1941'' (\texttt{True}) and ``Ramos Mejia'' (\texttt{False}) in the above case. For those entities that are the same across atomic facts (\eg, ``was born'') or are neutral (\eg, ``he,'' ``in,'' and ``on''), please label them as \texttt{True}. For the cases that there is no atomic fact that shares the same sentence structure, please identify the most informative entities in the sentence and label them with the same label as the atomic fact while treating the rest of the entities as \texttt{True}. In the end, output the entities and labels in the following format:
                \begin{itemize}[nosep]
                    \item Entity 1 (Label 1)
                    \item Entity 2 (Label 2)
                    \item ...
                    \item Entity N (Label N)
                \end{itemize}
                % \newline \newline
                Here are two examples:
                \newline\newline
                \textbf{[Example 1]}
                \newline
                [The start of the biography]
                \newline
                \textcolor{titlecolor}{Marianne McAndrew is an American actress and singer, born on November 21, 1942, in Cleveland, Ohio. She began her acting career in the late 1960s, appearing in various television shows and films.}
                \newline
                [The end of the biography]
                \newline \newline
                [The start of the list of checked facts]
                \newline
                \textcolor{anscolor}{[Marianne McAndrew is an American. (False); Marianne McAndrew is an actress. (True); Marianne McAndrew is a singer. (False); Marianne McAndrew was born on November 21, 1942. (False); Marianne McAndrew was born in Cleveland, Ohio. (False); She began her acting career in the late 1960s. (True); She has appeared in various television shows. (True); She has appeared in various films. (True)]}
                \newline
                [The end of the list of checked facts]
                \newline \newline
                [The start of the ideal output]
                \newline
                \textcolor{labelcolor}{[Marianne McAndrew (True); is (True); an (True); American (False); actress (True); and (True); singer (False); , (True); born (True); on (True); November 21, 1942 (False); , (True); in (True); Cleveland, Ohio (False); . (True); She (True); began (True); her (True); acting career (True); in (True); the late 1960s (True); , (True); appearing (True); in (True); various (True); television shows (True); and (True); films (True); . (True)]}
                \newline
                [The end of the ideal output]
				\newline \newline
                \textbf{[Example 2]}
                \newline
                [The start of the biography]
                \newline
                \textcolor{titlecolor}{Doug Sheehan is an American actor who was born on April 27, 1949, in Santa Monica, California. He is best known for his roles in soap operas, including his portrayal of Joe Kelly on ``General Hospital'' and Ben Gibson on ``Knots Landing.''}
                \newline
                [The end of the biography]
                \newline \newline
                [The start of the list of checked facts]
                \newline
                \textcolor{anscolor}{[Doug Sheehan is an American. (True); Doug Sheehan is an actor. (True); Doug Sheehan was born on April 27, 1949. (True); Doug Sheehan was born in Santa Monica, California. (False); He is best known for his roles in soap operas. (True); He portrayed Joe Kelly. (True); Joe Kelly was in General Hospital. (True); General Hospital is a soap opera. (True); He portrayed Ben Gibson. (True); Ben Gibson was in Knots Landing. (True); Knots Landing is a soap opera. (True)]}
                \newline
                [The end of the list of checked facts]
                \newline \newline
                [The start of the ideal output]
                \newline
                \textcolor{labelcolor}{[Doug Sheehan (True); is (True); an (True); American (True); actor (True); who (True); was born (True); on (True); April 27, 1949 (True); in (True); Santa Monica, California (False); . (True); He (True); is (True); best known (True); for (True); his roles in soap operas (True); , (True); including (True); in (True); his portrayal (True); of (True); Joe Kelly (True); on (True); ``General Hospital'' (True); and (True); Ben Gibson (True); on (True); ``Knots Landing.'' (True)]}
                \newline
                [The end of the ideal output]
				\newline \newline
				\textbf{User prompt}
				\newline
				\newline
				[The start of the biography]
				\newline
				\textcolor{magenta}{\texttt{\{BIOGRAPHY\}}}
				\newline
				[The ebd of the biography]
				\newline \newline
				[The start of the list of checked facts]
				\newline
				\textcolor{magenta}{\texttt{\{LIST OF CHECKED FACTS\}}}
				\newline
				[The end of the list of checked facts]
			};
			\node[chatbox_user_inner] (q1_text) at (q1) {
				\textbf{System prompt}
				\newline
				\newline
				You are a helpful and precise assistant for segmenting and labeling sentences. We would like to request your help on curating a dataset for entity-level hallucination detection.
				\newline \newline
                We will give you a machine generated biography and a list of checked facts about the biography. Each fact consists of a sentence and a label (True/False). Please do the following process. First, breaking down the biography into words. Second, by referring to the provided list of facts, merging some broken down words in the previous step to form meaningful entities. For example, ``strategic thinking'' should be one entity instead of two. Third, according to the labels in the list of facts, labeling each entity as True or False. Specifically, for facts that share a similar sentence structure (\eg, \textit{``He was born on Mach 9, 1941.''} (\texttt{True}) and \textit{``He was born in Ramos Mejia.''} (\texttt{False})), please first assign labels to entities that differ across atomic facts. For example, first labeling ``Mach 9, 1941'' (\texttt{True}) and ``Ramos Mejia'' (\texttt{False}) in the above case. For those entities that are the same across atomic facts (\eg, ``was born'') or are neutral (\eg, ``he,'' ``in,'' and ``on''), please label them as \texttt{True}. For the cases that there is no atomic fact that shares the same sentence structure, please identify the most informative entities in the sentence and label them with the same label as the atomic fact while treating the rest of the entities as \texttt{True}. In the end, output the entities and labels in the following format:
                \begin{itemize}[nosep]
                    \item Entity 1 (Label 1)
                    \item Entity 2 (Label 2)
                    \item ...
                    \item Entity N (Label N)
                \end{itemize}
                % \newline \newline
                Here are two examples:
                \newline\newline
                \textbf{[Example 1]}
                \newline
                [The start of the biography]
                \newline
                \textcolor{titlecolor}{Marianne McAndrew is an American actress and singer, born on November 21, 1942, in Cleveland, Ohio. She began her acting career in the late 1960s, appearing in various television shows and films.}
                \newline
                [The end of the biography]
                \newline \newline
                [The start of the list of checked facts]
                \newline
                \textcolor{anscolor}{[Marianne McAndrew is an American. (False); Marianne McAndrew is an actress. (True); Marianne McAndrew is a singer. (False); Marianne McAndrew was born on November 21, 1942. (False); Marianne McAndrew was born in Cleveland, Ohio. (False); She began her acting career in the late 1960s. (True); She has appeared in various television shows. (True); She has appeared in various films. (True)]}
                \newline
                [The end of the list of checked facts]
                \newline \newline
                [The start of the ideal output]
                \newline
                \textcolor{labelcolor}{[Marianne McAndrew (True); is (True); an (True); American (False); actress (True); and (True); singer (False); , (True); born (True); on (True); November 21, 1942 (False); , (True); in (True); Cleveland, Ohio (False); . (True); She (True); began (True); her (True); acting career (True); in (True); the late 1960s (True); , (True); appearing (True); in (True); various (True); television shows (True); and (True); films (True); . (True)]}
                \newline
                [The end of the ideal output]
				\newline \newline
                \textbf{[Example 2]}
                \newline
                [The start of the biography]
                \newline
                \textcolor{titlecolor}{Doug Sheehan is an American actor who was born on April 27, 1949, in Santa Monica, California. He is best known for his roles in soap operas, including his portrayal of Joe Kelly on ``General Hospital'' and Ben Gibson on ``Knots Landing.''}
                \newline
                [The end of the biography]
                \newline \newline
                [The start of the list of checked facts]
                \newline
                \textcolor{anscolor}{[Doug Sheehan is an American. (True); Doug Sheehan is an actor. (True); Doug Sheehan was born on April 27, 1949. (True); Doug Sheehan was born in Santa Monica, California. (False); He is best known for his roles in soap operas. (True); He portrayed Joe Kelly. (True); Joe Kelly was in General Hospital. (True); General Hospital is a soap opera. (True); He portrayed Ben Gibson. (True); Ben Gibson was in Knots Landing. (True); Knots Landing is a soap opera. (True)]}
                \newline
                [The end of the list of checked facts]
                \newline \newline
                [The start of the ideal output]
                \newline
                \textcolor{labelcolor}{[Doug Sheehan (True); is (True); an (True); American (True); actor (True); who (True); was born (True); on (True); April 27, 1949 (True); in (True); Santa Monica, California (False); . (True); He (True); is (True); best known (True); for (True); his roles in soap operas (True); , (True); including (True); in (True); his portrayal (True); of (True); Joe Kelly (True); on (True); ``General Hospital'' (True); and (True); Ben Gibson (True); on (True); ``Knots Landing.'' (True)]}
                \newline
                [The end of the ideal output]
				\newline \newline
				\textbf{User prompt}
				\newline
				\newline
				[The start of the biography]
				\newline
				\textcolor{magenta}{\texttt{\{BIOGRAPHY\}}}
				\newline
				[The ebd of the biography]
				\newline \newline
				[The start of the list of checked facts]
				\newline
				\textcolor{magenta}{\texttt{\{LIST OF CHECKED FACTS\}}}
				\newline
				[The end of the list of checked facts]
			};
		\end{tikzpicture}
        \caption{GPT-4o prompt for labeling hallucinated entities.}\label{tb:gpt-4-prompt}
	\end{center}
\vspace{-0cm}
\end{table*}
% \section{Full Experiment Results}
% \begin{table*}[th]
    \centering
    \small
    \caption{Classification Results}
    \begin{tabular}{lcccc}
        \toprule
        \textbf{Method} & \textbf{Accuracy} & \textbf{Precision} & \textbf{Recall} & \textbf{F1-score} \\
        \midrule
        \multicolumn{5}{c}{\textbf{Zero Shot}} \\
                Zero-shot E-eyes & 0.26 & 0.26 & 0.27 & 0.26 \\
        Zero-shot CARM & 0.24 & 0.24 & 0.24 & 0.24 \\
                Zero-shot SVM & 0.27 & 0.28 & 0.28 & 0.27 \\
        Zero-shot CNN & 0.23 & 0.24 & 0.23 & 0.23 \\
        Zero-shot RNN & 0.26 & 0.26 & 0.26 & 0.26 \\
DeepSeek-0shot & 0.54 & 0.61 & 0.54 & 0.52 \\
DeepSeek-0shot-COT & 0.33 & 0.24 & 0.33 & 0.23 \\
DeepSeek-0shot-Knowledge & 0.45 & 0.46 & 0.45 & 0.44 \\
Gemma2-0shot & 0.35 & 0.22 & 0.38 & 0.27 \\
Gemma2-0shot-COT & 0.36 & 0.22 & 0.36 & 0.27 \\
Gemma2-0shot-Knowledge & 0.32 & 0.18 & 0.34 & 0.20 \\
GPT-4o-mini-0shot & 0.48 & 0.53 & 0.48 & 0.41 \\
GPT-4o-mini-0shot-COT & 0.33 & 0.50 & 0.33 & 0.38 \\
GPT-4o-mini-0shot-Knowledge & 0.49 & 0.31 & 0.49 & 0.36 \\
GPT-4o-0shot & 0.62 & 0.62 & 0.47 & 0.42 \\
GPT-4o-0shot-COT & 0.29 & 0.45 & 0.29 & 0.21 \\
GPT-4o-0shot-Knowledge & 0.44 & 0.52 & 0.44 & 0.39 \\
LLaMA-0shot & 0.32 & 0.25 & 0.32 & 0.24 \\
LLaMA-0shot-COT & 0.12 & 0.25 & 0.12 & 0.09 \\
LLaMA-0shot-Knowledge & 0.32 & 0.25 & 0.32 & 0.28 \\
Mistral-0shot & 0.19 & 0.23 & 0.19 & 0.10 \\
Mistral-0shot-Knowledge & 0.21 & 0.40 & 0.21 & 0.11 \\
        \midrule
        \multicolumn{5}{c}{\textbf{4 Shot}} \\
GPT-4o-mini-4shot & 0.58 & 0.59 & 0.58 & 0.53 \\
GPT-4o-mini-4shot-COT & 0.57 & 0.53 & 0.57 & 0.50 \\
GPT-4o-mini-4shot-Knowledge & 0.56 & 0.51 & 0.56 & 0.47 \\
GPT-4o-4shot & 0.77 & 0.84 & 0.77 & 0.73 \\
GPT-4o-4shot-COT & 0.63 & 0.76 & 0.63 & 0.53 \\
GPT-4o-4shot-Knowledge & 0.72 & 0.82 & 0.71 & 0.66 \\
LLaMA-4shot & 0.29 & 0.24 & 0.29 & 0.21 \\
LLaMA-4shot-COT & 0.20 & 0.30 & 0.20 & 0.13 \\
LLaMA-4shot-Knowledge & 0.15 & 0.23 & 0.13 & 0.13 \\
Mistral-4shot & 0.02 & 0.02 & 0.02 & 0.02 \\
Mistral-4shot-Knowledge & 0.21 & 0.27 & 0.21 & 0.20 \\
        \midrule
        
        \multicolumn{5}{c}{\textbf{Suprevised}} \\
        SVM & 0.94 & 0.92 & 0.91 & 0.91 \\
        CNN & 0.98 & 0.98 & 0.97 & 0.97 \\
        RNN & 0.99 & 0.99 & 0.99 & 0.99 \\
        % \midrule
        % \multicolumn{5}{c}{\textbf{Conventional Wi-Fi-based Human Activity Recognition Systems}} \\
        E-eyes & 1.00 & 1.00 & 1.00 & 1.00 \\
        CARM & 0.98 & 0.98 & 0.98 & 0.98 \\
\midrule
 \multicolumn{5}{c}{\textbf{Vision Models}} \\
           Zero-shot SVM & 0.26 & 0.25 & 0.25 & 0.25 \\
        Zero-shot CNN & 0.26 & 0.25 & 0.26 & 0.26 \\
        Zero-shot RNN & 0.28 & 0.28 & 0.29 & 0.28 \\
        SVM & 0.99 & 0.99 & 0.99 & 0.99 \\
        CNN & 0.98 & 0.99 & 0.98 & 0.98 \\
        RNN & 0.98 & 0.99 & 0.98 & 0.98 \\
GPT-4o-mini-Vision & 0.84 & 0.85 & 0.84 & 0.84 \\
GPT-4o-mini-Vision-COT & 0.90 & 0.91 & 0.90 & 0.90 \\
GPT-4o-Vision & 0.74 & 0.82 & 0.74 & 0.73 \\
GPT-4o-Vision-COT & 0.70 & 0.83 & 0.70 & 0.68 \\
LLaMA-Vision & 0.20 & 0.23 & 0.20 & 0.09 \\
LLaMA-Vision-Knowledge & 0.22 & 0.05 & 0.22 & 0.08 \\

        \bottomrule
    \end{tabular}
    \label{full}
\end{table*}




\end{document}





% \newpage
% \newpage

\appendix
\begingroup
% Adjust figure and table spacing
\setlength{\floatsep}{5pt} % Space between figures/tables
\setlength{\textfloatsep}{5pt} % Space between figures/tables and text
\setlength{\intextsep}{5pt} % Space for inline figures/tables

% Adjust table row spacing
\renewcommand{\arraystretch}{0.8} % Reduce table row spacing

% Adjust caption font and spacing
% \usepackage[font=small,labelfont=bf]{caption}
\captionsetup{skip=5pt} % Reduce space below captions

% Now write the appendix content (tables, figures, etc.)

% \section{Appendix Title}
% Insert your figures and tables here

\pagebreak
\section*{Appendix}



\begin{table}[H]
    \footnotesize
    \caption{Hyperparameter values.}
    \begin{tabular}{l p{5cm} c}
        \toprule
        \textbf{Hyperparameter} & \textbf{Description} & \textbf{Value} \\ 
        \midrule
        $\gamma$ & Reward discount factor & 0.99 \\
        $lr$ & Learning rate & 0.001 \\
        $\epsilon$ & Range for clipping the gradients & 0.2 \\
        n\_steps & Number of steps for each update & 2048 \\ 
        batch\_size & Minibatch size & 64 \\ 
        $\beta_{\text{value}}$ & Value function coefficient for the loss calculation & 1 \\ 
        return\_lookback & Number of last business days to calculate rolling return & 40 \\ 
        std\_lookback & Number of last business days to calculate standard deviation & 60 \\ 
        n\_steps\_foresee ($n$) & Number of business days the Oracle can see into the future & 14 \\ 
        $\alpha$ & Convexity parameter for transaction costs & 1 and 0.45 \\ 
        $R_f$ & Risk-free rate & 0 \\ 
        $\beta_{\text{entropy}}$ (starting) & Starting entropy coefficient in the loss function & 0.00005 \\ 
        $TC_{\text{eval}}$ & Maximum transaction fees & 0.0025 \\ 
        Actor Network & Sizes of the (hidden) layers in the network & 19, 64, 64, 3 \\ 
        Critic Network & Sizes of the (hidden) layers in the network & 19, 64, 64, 1 \\ 
        % \todo{NN architectures}
        \bottomrule
    \end{tabular}
    \label{tab:hp_values}
\end{table}



\begin{table}[H]

    \footnotesize
    \caption{Asset info.}
    \begin{tabular}{llccc}
        \toprule
        Name & Ticker Bloomberg  & \multicolumn{3}{c}{Strategy Weight} \\
        
        & & Strategy 1 & Strategy 2 & Strategy 3 \\
        \midrule
        Developed Equities & MXWOHEUR  & 1 & 0.55 & 0 \\
        Emerging equities & NDUEEGF   & 0 & 0.05 & 0 \\
        Global Credit & G0BC  & 0 & 0.2 & 0 \\
        Global Govies & W0G1  & 0 & 0.2 & 1 \\

        \bottomrule
    \end{tabular}
    \label{bloomberg_ticker}
\end{table}

\begin{table}[H]
  \footnotesize
  \caption{Timing phases with start and end dates for training, validation, and testing.}
  \centering
  \begin{tabular}{llccc} % Number of columns: 5 (including the phase labels and sub-labels)
    \toprule
    &  & Phase 1 & Phase 2 & Phase 3 \\

    &  & \scriptsize{(pre-pandemic)} & \scriptsize{(pandemic)} & \scriptsize{(post-pandemic)} \\
    \midrule
    \multirow{2}{*}{Train} & Start Date & 1996-02-01 & 2002-01-01 & 2009-01-01 \\
                           & End Date & 2012-01-01  & 2016-01-01  & 2018-01-01  \\
    
    \multirow{2}{*}{Valid} & Start Date & 2012-01-01 & 2016-01-01 & 2018-01-01 \\
                           & End Date & 2015-01-01  & 2020-01-01  & 2022-01-01  \\
    
    \multirow{2}{*}{Test} & Start Date & 2015-01-01 & 2020-01-01 & 2022-01-01 \\
                          & End Date & 2020-01-01  & 2022-01-01  & 2024-01-01 \\
    \bottomrule
    
  \end{tabular}

  \label{tab:timing-phases}
\end{table}

\begin{figure}[H]
  \centering
  \includegraphics[width=0.8\linewidth]{figures/tc_plot.pdf}
  \caption{\small Transaction cost schedule example.}
  \label{fig:TC_sch}
  \Description{Transaction cost schedule example.}
\end{figure}


\begin{figure}[H]
  \centering
  \begin{subfigure}{\linewidth}
    \centering
  \includegraphics[width=\linewidth]{figures/jpm_weights_test.png.pdf}
  % \caption{Example of allocation by Diff. Sharpe during the testing period of phase 3. }
  \label{fig:alloc_jpm}
  \end{subfigure}
  \begin{subfigure}{\linewidth}
    \centering
 \includegraphics[width=\linewidth]{figures/jpm_port_ret_test.png.pdf}
  % \caption{PPO with Sharpe Regret}
  \label{fig:rets_jpm}
  \end{subfigure}

  \caption{ \small  Example of allocation and return dynamics by \\ Diff. Sharpe (\ref{eq:diff_sharpe}) during the testing period of phase 3.}
  \Description{  Example of allocation and return dynamics by \\ Diff. Sharpe (\ref{eq:diff_sharpe}) during the testing period of phase 3.}
  \label{fig:two-allocs1}
  
\end{figure}





\begin{figure}[H]
  \centering
  \begin{subfigure}{\linewidth}
    \centering
  \includegraphics[width=\linewidth]{figures/comb_weights_test.png.pdf}
  % \caption{Example of allocation by Diff. Sharpe during the testing period of phase 3. }
  \label{fig:alloc_emb}
  \end{subfigure}
  \begin{subfigure}{\linewidth}
    \centering
 \includegraphics[width=\linewidth]{figures/comb_port_ret_test.png.pdf}
  % \caption{PPO with Sharpe Regret}
  \label{fig:rets_emb}
  \end{subfigure}

  \caption{\small  Example of allocation and return dynamics by \\ Emb. Drawdown (\ref{eq:emb_rew}) during the testing period of phase 3.}
  \Description{ Example of allocation and return dynamics by \\ Emb. Drawdown (\ref{eq:emb_rew}) during the testing period of phase 3.}
  \label{fig:two-allocs2}
  
\end{figure}

\begin{figure}[H]
  \centering
  \includegraphics[width=\linewidth]{figures/alloc_cnn.png}
  \caption{\small  Multi-Body CNN allocation example. We can observe noisy allocation behaviors. Introducing regularization to smooth the allocation line could enhance performance. For instance, adding a term to the reward function that penalizes sharp allocation changes—only to revert back—would indicate unnecessary transaction costs. Alternatively, we could include an integral of allocation changes as an additional penalty alongside our $\ell_1$ term.
  % \todo{CNN scheme Benamou? OR no NEED. NO SPACE}
  }
  \label{fig:alloc_cnn}
  \Description{ Multi-Body CNN allocation example. We can observe noisy allocation behaviors. Introducing regularization to smooth the allocation line could enhance performance. For instance, adding a term to the reward function that penalizes sharp allocation changes—only to revert back—would indicate unnecessary transaction costs. Alternatively, we could include an integral of allocation changes as an additional penalty alongside our $\ell_1$ term.}
\end{figure}



\section*{ \\ Glossary }
\label{sec:glossary}
\begin{scriptsize}
\begin{itemize}


    \item \textbf{Mean Return:} The expected return of an investment, denoted as:
    \[
    \mu = \mathbb{E}[R] = \begin{bmatrix}
        R_1 \\
        R_2 \\
        \vdots \\
        R_n
    \end{bmatrix}
    \]
    where \( R_i \) represents the financial return of asset \( i \).

    \item \textbf{Covariance Matrix:} A matrix that captures the variances and covariances of asset returns, represented as:
    \[
    \Sigma = 
    \begin{bmatrix}
        \sigma_1^2 & \sigma_{1,2} & \cdots & \sigma_{1,n} \\
        \sigma_{2,1} & \sigma_2^2 & \cdots & \sigma_{2,n} \\
        \vdots & \vdots & \ddots & \vdots \\
        \sigma_{n,1} & \sigma_{n,2} & \cdots & \sigma_n^2
    \end{bmatrix}
    \]
    where \( \sigma_i^2 \) is the variance of asset \( i \) and \( \sigma_{i,j} \) is the covariance between returns of assets \( i \) and \( j \).

    \item \textbf{Portfolio Return} ($\mu_p$): 
    \[
    \mu_p = w' \mu
    \]
    where $w$ is the vector of asset weights and $\mu$ is the return vector of individual assets.
  
\item \textbf{Portfolio Standard Deviation} ($\sigma_p$), \textit{Risk}:
    \[
    \sigma_p = \sqrt{w' \Sigma w}
    \]
    where $w$ is the asset weight vector and $\Sigma$ is the covariance matrix of asset returns.

    \item \textbf{Sharpe Ratio:} \label{eq:gl:sharpe} A measure of risk-adjusted return, calculated as:
    \[
    S = \frac{\mu_p - R_f}{\sigma_p}
    \]
    where \( \mu_p \) is the portfolio return, \( R_f \) is the risk-free rate, and \( \sigma_p \) is the standard deviation of the portfolio returns. The ratio represents return-risk trade-off.

    \item \textbf{Maximum Drawdown (MDD)}, \textit{Risk}: The maximum observed loss from a peak to a trough of a portfolio, calculated as:
    \[
    MDD = \max_{t} \left( \frac{P_{max} - P_t}{P_{max}} \right)
    \]
    where \( P_{max} \) is the peak portfolio value and \( P_t \) is the portfolio value at time \( t \).
\end{itemize}
\end{scriptsize}

\endgroup

%%%%%%%%%%%%%%%%%%%%%%%%%%%%%%%%%%%%%%%%%%%%%%%%%%%%%%%%%%%%%%%%%%%%%%%%

\end{document}

%%%%%%%%%%%%%%%%%%%%%%%%%%%%%%%%%%%%%%%%%%%%%%%%%%%%%%%%%%%%%%%%%%%%%%%%

