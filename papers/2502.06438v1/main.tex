\documentclass[letterpaper, 10 pt, conference]{ieeeconf}  
\IEEEoverridecommandlockouts                                                                 
\overrideIEEEmargins                                      
\usepackage{amsmath,amssymb,amsfonts}
\usepackage{graphicx}
\usepackage{textcomp}
\usepackage{xcolor}
\usepackage{lipsum}
\usepackage{svg}
\usepackage{comment}
\def\BibTeX{{\rm B\kern-.05em{\sc i\kern-.025em b}\kern-.08em
    T\kern-.1667em\lower.7ex\hbox{E}\kern-.125emX}}
    

\usepackage{comment}
\usepackage{nohyperref}
\definecolor{matplotlib0}{HTML}{1f77b4}
\definecolor{matplotlib1}{HTML}{d62728}
\definecolor{matplotlib2}{HTML}{2ca02c}
\definecolor{matplotlib3}{HTML}{ff7f0e}
\definecolor{matplotlib4}{HTML}{9467bd}
\definecolor{matplotlib5}{HTML}{8c564b}
\definecolor{matplotlib6}{HTML}{e377c2}
\definecolor{matplotlib7}{HTML}{7f7f7f}
\definecolor{matplotlib8}{HTML}{bcbd22}
\definecolor{matplotlib9}{HTML}{17becf}

\usepackage{mathtools}

\DeclarePairedDelimiter\abs{\lvert}{\rvert}%
\DeclarePairedDelimiter\norm{\lVert}{\rVert}%

\usepackage{booktabs}
\usepackage{array}
\usepackage{multirow}
\usepackage{colortbl}
\usepackage{tablefootnote}
\usepackage{threeparttable}
\usepackage[acronym, style=super, nonumberlist]{glossaries}
\renewcommand*{\glsgroupskip}{}

\usepackage{pgfplots}
\definecolor{color0}{rgb}{0.12156862745098,0.466666666666667,0.705882352941177} % blue
\definecolor{color1}{rgb}{1,0.498039215686275,0.0549019607843137}
\definecolor{color2}{rgb}{0.172549019607843,0.627450980392157,0.172549019607843} % green
\definecolor{color3}{rgb}{0.83921568627451,0.152941176470588,0.156862745098039} % red
\definecolor{color4}{rgb}{0.580392156862745,0.403921568627451,0.741176470588235}
\definecolor{colorblue}{rgb}{0.12156862745098,0.466666666666667,0.705882352941177} % blue
\definecolor{colorgreen}{rgb}{0.172549019607843,0.627450980392157,0.172549019607843} % green
\definecolor{colorred}{rgb}{0.83921568627451,0.152941176470588,0.156862745098039} % red
\definecolor{colorblack}{rgb}{0,0,0} % black
\definecolor{colororange}{rgb}{1,0.56,0} % orange
\usepgfplotslibrary{fillbetween}
\usepgfplotslibrary{colormaps}
\pgfplotsset{compat=1.16}

\pgfplotscreateplotcyclelist{matplotlib}{
  {matplotlib0},
  {matplotlib1},
  {matplotlib2},
  {matplotlib3},
  {matplotlib4},
  {matplotlib5},
  {matplotlib6},
  {matplotlib7},
  {matplotlib8},
  {matplotlib9}
}

\pgfplotsset{every axis/.append style={
    cycle list name=matplotlib
}}

\usepackage{listings}

\definecolor{code_default}{HTML}{000000}
\definecolor{code_keyword}{HTML}{AC4142}
\definecolor{code_identifier}{HTML}{D28445}

%%% Local Variables:
%%% mode: latex
%%% TeX-master: "report"
%%% End:
\begin{acronym}
\acro{Title}{ExPerTune}
\end{acronym}

\makeatother
\title{\LARGE \bf
FEMBA: Efficient and Scalable EEG Analysis with a \\ Bidirectional Mamba Foundation Model
}

\author{
Anna Tegon$^{1}$, Thorir Mar Ingolfsson$^{1}$, \\ Xiaying Wang$^{1}$, 
Luca Benini$^{1,2}$, Yawei Li$^{1}$
\thanks{$^{1}$Integrated Systems Laboratory, ETH Z{\"u}rich, Z{\"u}rich, Switzerland.}
\thanks{$^{2}$DEI, University of Bologna, Bologna, Italy.}
\thanks{Anna Tegon and Thorir Mar Ingolfsson are co-first authors.}
}

\begin{document}
\thispagestyle{empty}
\pagestyle{empty}



\maketitle


\begin{abstract}
Accurate and efficient electroencephalography (EEG) analysis is essential for detecting seizures and artifacts in long-term monitoring, with applications spanning hospital diagnostics to wearable health devices. Robust EEG analytics have the potential to greatly improve patient care. However, traditional deep learning models, especially Transformer-based architectures, are hindered by their quadratic time and memory complexity, making them less suitable for resource-constrained environments. To address these challenges, we present FEMBA (Foundational EEG Mamba + Bidirectional Architecture), a novel self-supervised framework that establishes new efficiency benchmarks for EEG analysis through bidirectional state-space modeling. Unlike Transformer-based models, which incur quadratic time and memory complexity, FEMBA scales linearly with sequence length, enabling more scalable and efficient processing of extended EEG recordings. Trained on over 21,000 hours of unlabeled EEG and fine-tuned on three downstream tasks, FEMBA achieves competitive performance in comparison with transformer models, with significantly lower computational cost. Specifically, it reaches 81.82\% balanced accuracy (0.8921 AUROC) on TUAB and 0.949 AUROC on TUAR, while a \emph{tiny} 7.8M-parameter variant demonstrates viability for resource-constrained devices. These results pave the way for scalable, general-purpose EEG analytics in both clinical and highlight FEMBA as a promising candidate for wearable applications.
\newline
\noindent\textit{Clinical relevance}—
By reducing model size and computational overhead, FEMBA enables continuous on-device EEG monitoring for tasks like seizure detection and artifact reduction, promising improved patient care through timely and cost-effective neuro-monitoring solutions.
\end{abstract}
    
\section{Introduction}

Large language models (LLMs) have achieved remarkable success in automated math problem solving, particularly through code-generation capabilities integrated with proof assistants~\citep{lean,isabelle,POT,autoformalization,MATH}. Although LLMs excel at generating solution steps and correct answers in algebra and calculus~\citep{math_solving}, their unimodal nature limits performance in plane geometry, where solution depends on both diagram and text~\citep{math_solving}. 

Specialized vision-language models (VLMs) have accordingly been developed for plane geometry problem solving (PGPS)~\citep{geoqa,unigeo,intergps,pgps,GOLD,LANS,geox}. Yet, it remains unclear whether these models genuinely leverage diagrams or rely almost exclusively on textual features. This ambiguity arises because existing PGPS datasets typically embed sufficient geometric details within problem statements, potentially making the vision encoder unnecessary~\citep{GOLD}. \cref{fig:pgps_examples} illustrates example questions from GeoQA and PGPS9K, where solutions can be derived without referencing the diagrams.

\begin{figure}
    \centering
    \begin{subfigure}[t]{.49\linewidth}
        \centering
        \includegraphics[width=\linewidth]{latex/figures/images/geoqa_example.pdf}
        \caption{GeoQA}
        \label{fig:geoqa_example}
    \end{subfigure}
    \begin{subfigure}[t]{.48\linewidth}
        \centering
        \includegraphics[width=\linewidth]{latex/figures/images/pgps_example.pdf}
        \caption{PGPS9K}
        \label{fig:pgps9k_example}
    \end{subfigure}
    \caption{
    Examples of diagram-caption pairs and their solution steps written in formal languages from GeoQA and PGPS9k datasets. In the problem description, the visual geometric premises and numerical variables are highlighted in green and red, respectively. A significant difference in the style of the diagram and formal language can be observable. %, along with the differences in formal languages supported by the corresponding datasets.
    \label{fig:pgps_examples}
    }
\end{figure}



We propose a new benchmark created via a synthetic data engine, which systematically evaluates the ability of VLM vision encoders to recognize geometric premises. Our empirical findings reveal that previously suggested self-supervised learning (SSL) approaches, e.g., vector quantized variataional auto-encoder (VQ-VAE)~\citep{unimath} and masked auto-encoder (MAE)~\citep{scagps,geox}, and widely adopted encoders, e.g., OpenCLIP~\citep{clip} and DinoV2~\citep{dinov2}, struggle to detect geometric features such as perpendicularity and degrees. 

To this end, we propose \geoclip{}, a model pre-trained on a large corpus of synthetic diagram–caption pairs. By varying diagram styles (e.g., color, font size, resolution, line width), \geoclip{} learns robust geometric representations and outperforms prior SSL-based methods on our benchmark. Building on \geoclip{}, we introduce a few-shot domain adaptation technique that efficiently transfers the recognition ability to real-world diagrams. We further combine this domain-adapted GeoCLIP with an LLM, forming a domain-agnostic VLM for solving PGPS tasks in MathVerse~\citep{mathverse}. 
%To accommodate diverse diagram styles and solution formats, we unify the solution program languages across multiple PGPS datasets, ensuring comprehensive evaluation. 

In our experiments on MathVerse~\citep{mathverse}, which encompasses diverse plane geometry tasks and diagram styles, our VLM with a domain-adapted \geoclip{} consistently outperforms both task-specific PGPS models and generalist VLMs. 
% In particular, it achieves higher accuracy on tasks requiring geometric-feature recognition, even when critical numerical measurements are moved from text to diagrams. 
Ablation studies confirm the effectiveness of our domain adaptation strategy, showing improvements in optical character recognition (OCR)-based tasks and robust diagram embeddings across different styles. 
% By unifying the solution program languages of existing datasets and incorporating OCR capability, we enable a single VLM, named \geovlm{}, to handle a broad class of plane geometry problems.

% Contributions
We summarize the contributions as follows:
We propose a novel benchmark for systematically assessing how well vision encoders recognize geometric premises in plane geometry diagrams~(\cref{sec:visual_feature}); We introduce \geoclip{}, a vision encoder capable of accurately detecting visual geometric premises~(\cref{sec:geoclip}), and a few-shot domain adaptation technique that efficiently transfers this capability across different diagram styles (\cref{sec:domain_adaptation});
We show that our VLM, incorporating domain-adapted GeoCLIP, surpasses existing specialized PGPS VLMs and generalist VLMs on the MathVerse benchmark~(\cref{sec:experiments}) and effectively interprets diverse diagram styles~(\cref{sec:abl}).

\iffalse
\begin{itemize}
    \item We propose a novel benchmark for systematically assessing how well vision encoders recognize geometric premises, e.g., perpendicularity and angle measures, in plane geometry diagrams.
	\item We introduce \geoclip{}, a vision encoder capable of accurately detecting visual geometric premises, and a few-shot domain adaptation technique that efficiently transfers this capability across different diagram styles.
	\item We show that our final VLM, incorporating GeoCLIP-DA, effectively interprets diverse diagram styles and achieves state-of-the-art performance on the MathVerse benchmark, surpassing existing specialized PGPS models and generalist VLM models.
\end{itemize}
\fi

\iffalse

Large language models (LLMs) have made significant strides in automated math word problem solving. In particular, their code-generation capabilities combined with proof assistants~\citep{lean,isabelle} help minimize computational errors~\citep{POT}, improve solution precision~\citep{autoformalization}, and offer rigorous feedback and evaluation~\citep{MATH}. Although LLMs excel in generating solution steps and correct answers for algebra and calculus~\citep{math_solving}, their uni-modal nature limits performance in domains like plane geometry, where both diagrams and text are vital.

Plane geometry problem solving (PGPS) tasks typically include diagrams and textual descriptions, requiring solvers to interpret premises from both sources. To facilitate automated solutions for these problems, several studies have introduced formal languages tailored for plane geometry to represent solution steps as a program with training datasets composed of diagrams, textual descriptions, and solution programs~\citep{geoqa,unigeo,intergps,pgps}. Building on these datasets, a number of PGPS specialized vision-language models (VLMs) have been developed so far~\citep{GOLD, LANS, geox}.

Most existing VLMs, however, fail to use diagrams when solving geometry problems. Well-known PGPS datasets such as GeoQA~\citep{geoqa}, UniGeo~\citep{unigeo}, and PGPS9K~\citep{pgps}, can be solved without accessing diagrams, as their problem descriptions often contain all geometric information. \cref{fig:pgps_examples} shows an example from GeoQA and PGPS9K datasets, where one can deduce the solution steps without knowing the diagrams. 
As a result, models trained on these datasets rely almost exclusively on textual information, leaving the vision encoder under-utilized~\citep{GOLD}. 
Consequently, the VLMs trained on these datasets cannot solve the plane geometry problem when necessary geometric properties or relations are excluded from the problem statement.

Some studies seek to enhance the recognition of geometric premises from a diagram by directly predicting the premises from the diagram~\citep{GOLD, intergps} or as an auxiliary task for vision encoders~\citep{geoqa,geoqa-plus}. However, these approaches remain highly domain-specific because the labels for training are difficult to obtain, thus limiting generalization across different domains. While self-supervised learning (SSL) methods that depend exclusively on geometric diagrams, e.g., vector quantized variational auto-encoder (VQ-VAE)~\citep{unimath} and masked auto-encoder (MAE)~\citep{scagps,geox}, have also been explored, the effectiveness of the SSL approaches on recognizing geometric features has not been thoroughly investigated.

We introduce a benchmark constructed with a synthetic data engine to evaluate the effectiveness of SSL approaches in recognizing geometric premises from diagrams. Our empirical results with the proposed benchmark show that the vision encoders trained with SSL methods fail to capture visual \geofeat{}s such as perpendicularity between two lines and angle measure.
Furthermore, we find that the pre-trained vision encoders often used in general-purpose VLMs, e.g., OpenCLIP~\citep{clip} and DinoV2~\citep{dinov2}, fail to recognize geometric premises from diagrams.

To improve the vision encoder for PGPS, we propose \geoclip{}, a model trained with a massive amount of diagram-caption pairs.
Since the amount of diagram-caption pairs in existing benchmarks is often limited, we develop a plane diagram generator that can randomly sample plane geometry problems with the help of existing proof assistant~\citep{alphageometry}.
To make \geoclip{} robust against different styles, we vary the visual properties of diagrams, such as color, font size, resolution, and line width.
We show that \geoclip{} performs better than the other SSL approaches and commonly used vision encoders on the newly proposed benchmark.

Another major challenge in PGPS is developing a domain-agnostic VLM capable of handling multiple PGPS benchmarks. As shown in \cref{fig:pgps_examples}, the main difficulties arise from variations in diagram styles. 
To address the issue, we propose a few-shot domain adaptation technique for \geoclip{} which transfers its visual \geofeat{} perception from the synthetic diagrams to the real-world diagrams efficiently. 

We study the efficacy of the domain adapted \geoclip{} on PGPS when equipped with the language model. To be specific, we compare the VLM with the previous PGPS models on MathVerse~\citep{mathverse}, which is designed to evaluate both the PGPS and visual \geofeat{} perception performance on various domains.
While previous PGPS models are inapplicable to certain types of MathVerse problems, we modify the prediction target and unify the solution program languages of the existing PGPS training data to make our VLM applicable to all types of MathVerse problems.
Results on MathVerse demonstrate that our VLM more effectively integrates diagrammatic information and remains robust under conditions of various diagram styles.

\begin{itemize}
    \item We propose a benchmark to measure the visual \geofeat{} recognition performance of different vision encoders.
    % \item \sh{We introduce geometric CLIP (\geoclip{} and train the VLM equipped with \geoclip{} to predict both solution steps and the numerical measurements of the problem.}
    \item We introduce \geoclip{}, a vision encoder which can accurately recognize visual \geofeat{}s and a few-shot domain adaptation technique which can transfer such ability to different domains efficiently. 
    % \item \sh{We develop our final PGPS model, \geovlm{}, by adapting \geoclip{} to different domains and training with unified languages of solution program data.}
    % We develop a domain-agnostic VLM, namely \geovlm{}, by applying a simple yet effective domain adaptation method to \geoclip{} and training on the refined training data.
    \item We demonstrate our VLM equipped with GeoCLIP-DA effectively interprets diverse diagram styles, achieving superior performance on MathVerse compared to the existing PGPS models.
\end{itemize}

\fi 

\section{Related Work}\label{sec:related_works}
\gls{bp} estimation from \gls{ecg} and \gls{ppg} waveforms has received significant attention due to its potential for continuous, unobtrusive monitoring. Earlier work relied on classical machine learning with handcrafted features, but deep learning methods have since emerged as more robust alternatives. Convolutional or recurrent architectures designed for \gls{ecg}/\gls{ppg} have shown strong performance, including ResUNet with self-attention~\cite{Jamil}, U-Net variants~\cite{Mahmud_2022}, and hybrid \gls{cnn}--\gls{rnn} models~\cite{Paviglianiti2021ACO}. These architectures often outperform traditional feature-engineering approaches, particularly when both \gls{ecg} and \gls{ppg} signals are used~\cite{Paviglianiti2021ACO}.

Nevertheless, many existing methods train solely on \gls{ecg}/\gls{ppg} data, which, while plentiful~\cite{mimiciii,vitaldb,ptb-xl}, often exhibit significant variability in signal quality and patient-specific characteristics. This variability poses challenges for achieving robust generalization across populations. Recent work has explored transfer learning to overcome these issues; for example, Yang \emph{et~al.}~\cite{yang2023cross} studied the transfer of \gls{eeg} knowledge to \gls{ecg} classification tasks, achieving improved performance and reduced training costs. Joshi \emph{et~al.}~\cite{joshi2021deep} also explored the transfer of \gls{eeg} knowledge using a deep knowledge distillation framework to enhance single-lead \gls{ecg}-based sleep staging. However, these studies have largely focused on within-modality or narrow domain adaptations, leaving open the broader question of whether an \gls{eeg}-based foundation model can serve as a versatile starting point for generalized biosignal analysis.

\gls{eeg} has become an attractive candidate for pre-training large models not only because of the availability of large-scale \gls{eeg} repositories~\cite{TUEG} but also due to its rich multi-channel, temporal, and spectral dynamics~\cite{jiang2024large}. While many time-series modalities (for example, voice) also exhibit rich temporal structure, \gls{eeg}, \gls{ecg}, and \gls{ppg} share common physiological origins and similar noise characteristics, which facilitate the transfer of temporal pattern recognition capabilities. In other words, our hypothesis is that the underlying statistical properties and multi-dimensional dynamics in \gls{eeg} make it particularly well-suited for learning robust representations that can be effectively adapted to \gls{ecg}/\gls{ppg} tasks. Our work is the first to validate the feasibility of fine-tuning a transformer-based model initially trained on EEG (CEReBrO~\cite{CEReBrO}) for arterial \gls{bp} estimation using \gls{ecg} and \gls{ppg} data.

Beyond accuracy, real-world deployment of \gls{bp} estimation models calls for efficient inference. Traditional deep networks can be computationally expensive, motivating recent interest in quantization and other compression techniques~\cite{nagel2021whitepaperneuralnetwork}. Few studies have combined large-scale pre-training with post-training quantization for \gls{bp} monitoring. Hence, our method integrates these two aspects: leveraging a potent \gls{eeg}-based foundation model and applying quantization for a compact, high-accuracy cuffless \gls{bp} solution.
\section{Methodology}\label{sec:method}

This section describes our approach for estimating \gls{bp} from \gls{ecg} and \gls{ppg} waveforms using a large \gls{eeg}-based foundation model. We first detail how we adapt and fine-tune the CEReBrO architecture for \gls{bp} prediction, then describe our post-training quantization steps.

\subsection{Architecture and Fine-Tuning}\label{subsec:model}

\begin{figure}[htp]
    \centering
    \includegraphics[width=7.5cm]{images/architecture}
    \caption{The modified CEReBrO Architecture~\cite{CEReBrO}, supplemented with an additional MLP-head, which is utilized for the \gls{bp} estimation task.} 
    \vspace{-0.6cm}
    \label{fig:cerebro}
\end{figure}  

Our method builds on the \textbf{CEReBrO} transformer-encoder~\cite{CEReBrO}, originally pre-trained on a large \gls{eeg} dataset (TUEG~\cite{TUEG}). CEReBrO employs a tokenization scheme that splits time-series signals into non-overlapping patches and projects them into a latent space. Alternating self-attention blocks then process these tokens by focusing on intra-channel (temporal) correlations and inter-channel (spatial) relationships. This design efficiently captures both local and long-range dependencies in multi-channel biosignals. Although CEReBrO was trained on \gls{eeg} data, its attention-based encoder can generalize to other biosignals sharing similar temporal structures. To adapt CEReBrO for \gls{bp} estimation from \gls{ecg} and \gls{ppg}, we make the following modifications:

\begin{itemize}
    \item We feed \gls{ecg} and \gls{ppg} signals as two input channels, each sampled at 125,Hz and shaped into 10-second segments ($2 \times 1250$).
    \item  We replace the original classification head with a single fully connected layer (MLP) that outputs two values: \gls{sbp} and \gls{dbp}.
\end{itemize}

CEReBrO is then also available in three sizes—\emph{small} (3.58M parameters), \emph{medium} (39.95M parameters), and \emph{large} (85.15M parameters).

We then explore two fine-tuning strategies:
\begin{itemize}
    \item \textbf{Frozen Backbone}: All transformer layers except for the first input-embedding layer and the final MLP head are frozen. This preserves most \gls{eeg}-based representations while allowing partial adaptation to \gls{ecg}/\gls{ppg}.
    \item \textbf{Unfrozen Backbone}: All transformer layers are unfrozen to allow deeper domain alignment, albeit with a risk of forgetting learned \gls{eeg} features.
\end{itemize}
To measure the benefit of leveraging a pre-trained \gls{eeg} encoder, we compare fine-tuning against training from scratch (i.e., random initialization). In each setting, we train models of three different sizes (small, medium, large) for 100 epochs and extend unfrozen-backbone runs to 200 epochs to assess longer-term convergence. We use Xavier Initialization~\cite{xavier_init} when training from scratch. Performance is evaluated on the MIMIC-III and VitalDB datasets in terms of MAE, SD, and coefficient of determination ($R^2$), as well as clinical standards (\gls{bhs} and \gls{aami}).

\subsection{Quantization}\label{subsec:quantization}
We apply post-training quantization to our fine-tuned models to enable real-time deployment on resource-constrained devices. This step reduces the memory footprint and inference latency while preserving clinically relevant accuracy.

We use PyTorch’s FX Graph Mode Quantization pipeline~\cite{pytorch2} to insert quantization and dequantization operations systematically. Quantization is widely employed to map floating-point (32-bit) values to lower numerical precision, typically to 8-bit integers. The range of a floating-point value, denoted by \(x_{\mathrm{fp}}\), is defined as follows: \([x_{\min}, x_{\max}]\). Based on this, two characteristic values can be defined, which are essential for the quantization process: \textbf{Scale} \(\Delta\), which determines the step size (real-valued), while \textbf{Zero-point} \(z\), which is an integer offset whose primary function is to ensure that the zero is mapped onto an integer. In this work, we specifically adopt two types of quantization:

\begin{itemize}
    \item \emph{Symmetric Quantization} for weights (common when weight distributions are roughly zero-mean).
    \item \emph{Asymmetric Quantization} for activations (typical when ReLU shifts values positively).
\end{itemize}
For symmetric quantization, we have the following characteristic values:
\begin{equation*}
    \Delta = \frac{\max\bigl(|x_{\min}|, |x_{\max}|\bigr)}{2^{b-1}}, 
    \quad 
    z = 0
\end{equation*}

Based on these values, forward quantization is done using this equation:
\begin{equation*}
x_{\mathrm{int}} = \mathrm{clip}\!\Bigl(
        \mathrm{round}\bigl(\tfrac{x}{\Delta}\bigr), 
        \, -2^{b-1}, \, 2^{b-1} - 1\Bigr).
\end{equation*}

And for \emph{asymmetric} quantization the characteristic values can be calculated in the following way, where $b$ is the number of bits :
\begin{equation*}
    \Delta = \frac{x_{\max} - x_{\min}}{2^b - 1}, 
    \quad 
    z = \left\lfloor - \frac{x_{\min}}{\Delta} + 0.5 \right\rfloor.
\end{equation*}
When $\Delta$ and $z$ are determined, the forward quantization step is the following:
\begin{equation*}
    x_{\mathrm{int}} = \mathrm{clip}\!\Bigl(
        \mathrm{round}\bigl(\tfrac{x_{\mathrm{fp}}}{\Delta}\bigr) + z , 
        \, 0, \, 2^b - 1\Bigr),
\end{equation*}
where $\mathrm{clip}(\cdot,0,2^b-1)$ ensures $x_{\mathrm{int}}$ to stay in the range $[0, 2^b - 1]$~\cite{quant_data}.

Quantization typically involves three stages: (1) \emph{calibration}, where representative data is passed through the model to collect scaling statistics; (2) \emph{conversion}, transforming the floating-point model into a quantized version; and (3) \emph{execution}, running inference with reduced-precision operations.

We explore both \emph{static} quantization~\cite{FU20092937}, which precomputes scaling and zero points via a calibration dataset, and \emph{dynamic} quantization~\cite{vu2008stabilizing}, which calculates them on-the-fly, eliminating the calibration phase. While static quantization can offer speed benefits if the input distribution is stable, dynamic quantization is often more flexible for variable-length or varying data distributions and avoids the need for extra calibration data.

Our target precision is INT8, balancing memory savings and model fidelity. We evaluate symmetric quantization for weights and asymmetric for activations (shifted by ReLU). Different observers—\emph{MinMaxObserver}, \emph{MovingAverageMinMaxObserver}, and \emph{HistogramObserver}—estimate the range, each trading off complexity against robustness. We also employ per-channel quantization for \gls{ecg}/\gls{ppg} inputs, giving each signal channel a separate scale and zero point.

Our experiments reveal that dynamic per-channel quantization with symmetric weights yields an optimal model size, computational speed, and accuracy trade-off. Detailed results of these experiments are presented in Section~\ref{sec:results}. This approach is critical for enabling continuous, on-device \gls{bp} estimation, where both memory and energy constraints are strict.
\section{Results}
\label{sec:Results}

In this section, we present various analysis results that demonstrate the adoption of code obfuscation in Google Play.

\subsection{Overall Obfuscation Trends} 
\label{sec:obstrend}

\subsubsection{Presence of obfuscation} Out of the 548,967 Google Play Store APKs analyzed, we identified 308,782 obfuscated apps, representing approximately 56.25\% of the total. In Figure~\ref{fig:obfuscated_percentage}, we show the year-wise percentage of obfuscated apps for 2016-2023. There is an overall obfuscation increase of 13\% between 2016 and 2023, and as can be seen, the percentage of obfuscated apps has been increasing in the last few years, barring 2019 and 2020. As explained in Section~\ref{subsec:dataset}, 2019 and 2020 contain apps that are more likely to be abandoned by developers, and as such, they may not use advanced development practices.

\begin{figure}[h!]
\centering
    \includegraphics[width=\linewidth]{Figures/Only_obfuscation_trendV2.pdf}
    \caption{Percentage of obfuscated apps by year} \vspace{-4mm}
    \label{fig:obfuscated_percentage}
\end{figure}


From 2016 to 2018, the obfuscation levels were relatively stable at around 50-55\%, while from 2021 to 2023, there was a marked rise, reaching approximately 66\% in 2023. This indicates a growing focus on app protection measures among developers, likely driven by heightened security and IP concerns and the availability of advanced obfuscation tools.


\subsubsection{Obfuscation tools} Among the obfuscated APKs, our tool detector identified that 40.92\% of the apps use Proguard, 36.64\% use Allatori, 1.01\% use DashO, and 21.43\% use other (i.e., unknown) tools. We show the yearly trends in Figure~\ref{fig:ofbuscated_tool}. Note that we omit results in 2019 and 2020 ({\bf cf.} Section~\ref{subsec:dataset}).

ProGuard and Allatori are the most consistently used obfuscation tools, with ProGuard showing a slight overall increase in popularity and Allatori demonstrating variability. This inclination could be attributed to ProGuard being the default obfuscator integrated into Android Studio, a widely used development environment for Android applications. Notably, ProGuard usage increased by 13\% from 2018 to 2021, likely due to the introduction of R8 in April 2019~\cite{release_note_android}, which further simplified ProGuard integration with Android apps.

\begin{figure}[h]
\centering
    \includegraphics[width=\linewidth]{Figures/Initial_Tool_Trend_2019_dropV2.pdf} 
    \caption{Yearly obfuscation tool usage}
    \label{fig:ofbuscated_tool}
\end{figure}


DashO consistently remains low in usage, likely due to its high cost. The use of other obfuscation tools decreased until 2018 but has shown a resurgence from 2021 to 2023. This suggests that developers might be using other or custom tools, or our detector might be predicting some apps obfuscated with Proguard or Allatori as `other.' To investigate, we manually checked a sample of apps from the `other' category and confirmed they are indeed obfuscated. However, we could not determine which obfuscation tools the developers used. We discuss this potential limitation further in Section~\ref{sec:limitations}.


\subsubsection{Obfuscation techniques} We show the year-wise breakdown of obfuscation technique usage in Figure~\ref{fig:obfuscated_tech}. Among the various obfuscation techniques, Identifier Renaming emerged as the most prevalent, with 99.62\% of obfuscated apps using it alone or in combination with other methods (Categories of Only IR, IR and CF, IR and SE, or All three). Furthermore, 81.04\% of obfuscated apps used Control Flow Modification, and 62.76\% used String Encryption. The pervasive use of Identifier Renaming (IR) can be attributed to the fact that all obfuscation tools support it ({\bf cf.} Table~\ref{tab:ob_tool_cap}). Similarly, lower adoption of Control Flow Modification and String Encryption can be attributed to Proguard not supporting it. 

\begin{figure}[h]
\centering
    \includegraphics[width=\linewidth]{Figures/Initial_Tech_Trend_2019_dropV2.pdf} 
    \caption{Yearly obfuscation technique usage}
    \label{fig:obfuscated_tech}
\end{figure}



Next, we investigate the adoption of obfuscation on Google Play Store from various perspectives. Same as earlier, due to the smaller dataset size and possible bias ({\bf cf.} Section~\ref{subsec:dataset}), we exclude the APKs from 2019 and 2020 from this analyses.


\subsection{App Genre}
\label{sec:app_genre}

First, we investigate whether the obfuscation practices vary according to the App genre. Initially, we analysed all the APKs together before separating them into two snapshots.


\begin{figure*}[h]
    \centering
    \includegraphics[width=\linewidth]{Figures/AppGenreObfuscationV3.pdf}
    \caption{Obfuscated app percentage by genre (overall)}
    \label{fig:app_genre_overall}
\end{figure*}

Figure~\ref{fig:app_genre_overall} shows the genre-wise obfuscated app percentage. We note that 19 genres have more than 60\% of the apps obfuscated, and almost all the genres have more than 40\% obfuscation percentage. \textit{Casino} genre has the highest obfuscation percentage rate at 80\%, and overall, game genres tend to be more obfuscated than the other genres. The higher obfuscation usage in casino apps is logical due to their nature. These apps often simulate or involve gambling activities and handle monetary transactions and sensitive data related to in-game purchases, making them attractive targets for reverse engineering and hacking. This necessitates robust security measures to prevent fraud and protect user data. 


\begin{figure}[h]
    \centering
    \includegraphics[width=\linewidth]{Figures/AppGenre2018_2023ComparisonV3.pdf}
    \caption{Percentage of obfuscated apps by genre (2018-2023)}
    \label{fig:app_genre_comparison}
\end{figure}



\subsubsection{Genre-wise obfuscation trends in the two snapshots} To investigate the adoption of obfuscation over time, we study the two snapshots of Google Play separately, i.e., APKs from 2016-2018 as one group and APKs from 2021-2023 as another. 

Figure~\ref{fig:app_genre_comparison} illustrates the change in obfuscation levels by app genre between 2016-2018 to 2021-2023. Notably, app categories such as Education, Weather, and Parenting, which had obfuscation levels below the 2018 average, have increased to above the 2023 average by 2023. One possible reason for this in Education and Parenting apps can be the increase in online education activities during and after COVID-19 and the developers identifying the need for app hardening.

There are some genres, such as Casino and Action, for which the percentage of obfuscated apps didn't change across the two snapshots (i.e., purple and orange circles are close together in Figure~\ref{fig:app_genre_comparison}). This is because these genres are highly obfuscated from the beginning. Finally, the four genres, including Simulation and Role Playing, have a lower percentage of obfuscated apps in the 2021-2023 dataset. Our manual analysis didn't result in a conclusion as to why.


\begin{figure}[!h]
    \centering
    \includegraphics[width=\linewidth]{Figures/AppGenreTechAllV5.pdf}
    \caption{Obfuscation technique usage by genre (overall)}
    \label{fig:app_genre_all_tech}
\end{figure}


\subsubsection{Obfuscation techniques in different app genres} In Figure~\ref{fig:app_genre_all_tech}, we show the prevalence of key obfuscation techniques among various genres. As expected, almost all obfuscated apps in all genres used  Identifier Renaming. Also, it can be noted that genres with more obfuscated app percentages tend to use all three obfuscation techniques. Notably, more than 85\% of \textit{Casino} genre apps employ multiple obfuscation techniques

\subsubsection{Obfuscation tool usage in different app genres} We also investigated whether specific obfuscation tools are favoured by developers in different genres. However, apart from the expected observation that  ProGuard and Allatori being the most used tools, we didn't find any other interesting patterns. Therefore, we haven't included those measurement results.

\subsection{App Developers}
Next, we investigate individual developer-wise code obfuscation practices. From the pool of analyzed APKs, we identified the number of apps associated with each developer. Subsequently, we sorted the developers according to the number of apps they had created and selected the top 100 developers with the highest number of APKs for the 2016-2018 and 2021-2023 datasets. For the 2018 snapshot, we had 8,349 apps among the top 100 developers, while for the 2023 snapshot, we had 11,338 apps among the top 100 developers.

We then proceeded to detect whether or not these developers obfuscate their apps and, if so, what kind of tools and techniques they use. We present our results in five levels; developer obfuscating over 80\% of their apps, 60\%--80\% of apps, 40\%--60\% of apps, less than 40\%, and no obfuscation.

Figure~\ref{fig:developer_trend_my_apps_all} compares the two datasets in terms of developer obfuscation adoption. It shows that more developers have moved to obfuscate more than 80\% of their apps in the 2021-2023 dataset (76\%) compared to the 2016-2018 dataset (48\%).

We also found that among developers who obfuscate more than 80\% of their apps, 73\% in 2018 and 93\% in 2023 used the same obfuscation tool. Additionally, these top developers employ Control Flow Modification (CF) and String Encryption (SE) above the average values discussed in Section~\ref{sec:obstrend}. Specifically, in 2018, top developers used CF in 81.3\% of cases and SE in 66.7\%, while in 2023, these figures increased to 88.2\% and 78.9\%. This results in two insights: 1) Most top developers obfuscate all their apps with advanced techniques, possibly due to concerns about IP and security, and 2) Developers stick to a single tool, possibly due to specialized knowledge or because they bought a commercial licence.

\begin{figure}[]
    \centering
    \includegraphics[width=\linewidth]{Figures/Developer_Analysed_Comparison.pdf}
    \caption{Obfuscation usage (Top-100 developers)}
    \label{fig:developer_trend_my_apps_all}
\end{figure}


Finally, we investigate the obfuscation practices of developers with only one app in Table~\ref{tab:my-table}. According to the table, from those developers, 45.5\% of them obfuscated their apps in the 2016-2018 dataset and 57.2\% obfuscated their apps in the 2021-2023 dataset, showing a clear increase. However, these percentages are approximately 10\% lower than the average obfuscation rate in both cohorts discussed in Section~\ref{sec:obstrend}. This indicates that single-app developers may be less aware or concerned about code protection.


\begin{table}[]
\caption{Developers with only one app}
\label{tab:my-table}
\resizebox{\columnwidth}{!}{%
\begin{tabular}{cccccc}
\hline
\textbf{Year} & \textbf{\begin{tabular}[c]{@{}c@{}}Non\\ Obfuscated\end{tabular}} & \multicolumn{4}{c}{\textbf{Obfuscated}} \\ \hline
\multirow{3}{*}{\textbf{\begin{tabular}[c]{@{}c@{}}2018 \\ Snapshot\end{tabular}}} & \multirow{3}{*}{\begin{tabular}[c]{@{}c@{}}26,581 \\ (54.5\%)\end{tabular}} & \multicolumn{4}{c}{\begin{tabular}[c]{@{}c@{}}22,214 (45.5\%)\end{tabular}} \\ \cline{3-6} 
 &  & \textbf{ProGuard} & \textbf{Allatori} & \textbf{DashO} & \textbf{Other} \\ \cline{3-6} 
 &  & 6,131 & 8,050 & 658 & 7,375 \\ \hline
\multirow{3}{*}{\textbf{\begin{tabular}[c]{@{}c@{}}2023 \\ Snapshot\end{tabular}}} & \multirow{3}{*}{\begin{tabular}[c]{@{}c@{}}19,510 \\ (42.8\%)\end{tabular}} & \multicolumn{4}{c}{\begin{tabular}[c]{@{}c@{}}26,084 (57.2\%)\end{tabular}} \\ \cline{3-6} 
 &  & \textbf{ProGuard} & \textbf{Allatori} & \textbf{DashO} & \textbf{Other} \\ \cline{3-6} 
 &  & 12,697 & 9,672 & 234 & 3,581 \\ \hline
\end{tabular}%
}
\end{table}

\subsection{Top-k Apps}

Next, we investigate the obfuscation practices of top apps in Google Play Store. First, we rank the apps using the same criterion used by our previous work~\cite{rajasegaran2019multi, karunanayake2020multi, seneviratne2015early}. That is, we sort the apps in descending order of number of downloads, average rating, and rating count, with the intuition that top apps have high download numbers and high ratings, even when reviewed by a large number of users. Then, we investigated the percentage of obfuscated apps and obfuscation tools and technique usage as summarized in Table~\ref{tab:top_k_apps_2018_2023}.

When considering the highly ranked applications (i.e., top-1,000), the obfuscation percentage is notably higher, at around 93\%, in both datasets, which is significantly higher than the average percentage of obfuscation we observed in Section~\ref{sec:obstrend}. Top-ranked apps, likely due to their higher visibility and potential revenue, invest more in obfuscation to safeguard their intellectual property and enhance security. 

The obfuscation percentage decreases when going from the top 1,000 apps to the top 30,000 apps. Nonetheless, the obfuscation percentage in both datasets remains around similar values until the top 30,000 (e.g., $\sim$74\% for top-30,000). This indicates that the major increase in obfuscation in the 2021-2023 dataset comes from apps beyond the top 30,000.

When observing the tools used, the usage of ProGuard increases as we move from top to lower-ranked apps in both datasets. This may be because ProGuard is free and the default in Android Studio, while commercial tools like Allatori and DashO are expensive. There is a notable increase in the use of Allatori among the top apps in the 2021-2023 dataset. Regarding obfuscation techniques, the top 1,000 apps utilize all three techniques more frequently than lower-ranked apps in both snapshots. This indicates that the top 1,000 apps are more heavily protected compared to lower-ranked ones.

\begin{table*}[]
\caption{Summary of analysis results for Top-k apps in 2018 and 2023}
\label{tab:top_k_apps_2018_2023}
\resizebox{\textwidth}{!}{%
\begin{tabular}{lccccccccc}
\hline
\multicolumn{1}{c}{\begin{tabular}[c]{@{}c@{}}Top k apps - \\ Year\end{tabular}} & \begin{tabular}[c]{@{}c@{}}Total \\ Apps\end{tabular} & \begin{tabular}[c]{@{}c@{}}Obfuscation\\ Percentage\end{tabular} & \begin{tabular}[c]{@{}c@{}}ProGuard\\ Percentage\end{tabular} & \begin{tabular}[c]{@{}c@{}}Allatori\\ Percentage\end{tabular} & \begin{tabular}[c]{@{}c@{}}DashO\\ Percentage\end{tabular} & \begin{tabular}[c]{@{}c@{}}Other\\ Percentage\end{tabular} & \begin{tabular}[c]{@{}c@{}}IR\\ Percentage\end{tabular} & \begin{tabular}[c]{@{}c@{}}CF\\ Percentage\end{tabular} & \begin{tabular}[c]{@{}c@{}}SE\\ Percentage\end{tabular} \\ \hline
1k (2018) & 1,000 & 93.40 & 29.98 & 28.48 & 0.64 & 40.90 & 99.90 & 88.76 & 65.42 \\
10k (2018) & 10,000 & 85.19 & 25.55 & 35.32 & 0.47 & 38.65 & 99.90 & 88.76 & 71.91 \\
20k (2018) & 20,000 & 78.42 & 26.31 & 36.76 & 0.57 & 36.36 & 99.87 & 87.37 & 71.49 \\
30k (2018) & 30,000 & 74.40 & 27.30 & 37.71 & 0.64 & 34.36 & 99.82 & 86.75 & 71.11 \\
30k+ (2018) & 314,568 & 53.36 & 36.72 & 34.70 & 1.33 & 27.24 & 99.34 & 83.54 & 63.11 \\ \hline
1k (2023) & 1,000 & 92.50 & 24.00 & 51.89 & 1.95 & 22.16 & 100.0 & 92.54 & 83.68 \\
10k (2023) & 10,000 & 81.88 & 26.03 & 56.20 & 1.03 & 16.74 & 99.89 & 89.40 & 82.01 \\
20k (2023) & 20,000 & 76.62 & 30.48 & 52.92 & 0.96 & 15.64 & 99.93 & 85.80 & 78.01 \\
30k (2023) & 30,000 & 73.72 & 33.87 & 50.34 & 0.89 & 14.90 & 99.95 & 83.31 & 75.34 \\
30k+ (2023) & 206,216 & 61.90 & 46.56 & 38.21 & 0.64 & 14.59 & 99.97 & 77.51 & 62.50 \\ \hline
\end{tabular}%
}
\end{table*}


\section{Conclusion}\label{ch:conclusion}
We introduced \textbf{FEMBA}, a novel self-supervised EEG framework grounded in bidirectional state-space modeling and pre-trained on over 21,000 hours of unlabelled clinical EEG. Our experiments across multiple downstream tasks (abnormal EEG detection, artifact recognition, slowing event classification, and neonatal seizure detection) demonstrate that FEMBA achieves near-Transformer performance while maintaining significantly lower computational complexity and memory requirements.

Notably, a \emph{tiny} 7.8M-parameter variant (FEMBA-Tiny) retains competitive accuracy on tasks such as artifact detection, showcasing the potential for real-time edge deployments. Nonetheless, certain domain shifts—such as neonatal vs.\ adult EEG—underscore the need for additional domain adaptation. Future work will explore these techniques and integrate multi-modal physiological signals for more robust clinical event detection. We believe FEMBA marks a key step toward delivering efficient, universal EEG foundation models that operate seamlessly from large hospital databases to low-power wearable devices.
\section*{Acknowledgment}
\vspace{-0.1cm}
This project is supported by the Swiss National Science Foundation under the grant number 193813 (PEDESITE project) and by the ETH Future Computing Laboratory (EFCL), financed by a donation from Huawei Technologies. We acknowledge ISCRA for awarding this project access to the LEONARDO supercomputer, owned by the EuroHPC Joint Undertaking, hosted by CINECA (Italy).
\bibliographystyle{IEEEtran}
\bibliography{bib}

\end{document}
