\section{Related Work}
Early attempts in foot modeling relied on Principal Component Analysis (PCA) \cite{amstutz2008pca}, but these models were simplistic, offering limited resolution and flexibility. Later approaches employ active sensor technologies, where structured light or depth cameras are used to generate point clouds \cite{lunscher2017point,yuan20213d,lochner2014development}. However, the point cloud geometries obtained from these sensors are often noisy and incomplete. More recently, Boyne et al. proposed the FIND model \cite{boyne2022find} leveraging a template deformation strategy guided by an implicit neural network to improve reconstruction accuracy. 
Similarly, Osman et al. \cite{osman2022supr} developed SUPR, a PCA-based human foot model designed for seamless integration with the SMPL full-body model \cite{loper2023smpl}, enabling expressive and anatomically consistent reconstructions. However, both FIND and SUPR are limited by their training data, which strongly constrains the shape space.
Our method, draws inspiration from multi-view reconstruction \cite{schoenberger2016sfm,schoenberger2016mvs,kazhdan2006poisson,kazhdan2013screened} and shape completion \cite{hu2019local,shen2012structure}, both of which have proven effective in broader 3D reconstruction tasks. By leveraging these advancements, our approach can be seamlessly integrated into existing works, providing a more robust and generalizable solution for foot reconstruction.