\documentclass[10pt,journal,compsoc]{IEEEtran}
\usepackage{amsmath,amsfonts}
\usepackage{algorithmic}
\usepackage{algorithm}
\usepackage{array}
\usepackage[caption=false,font=footnotesize,labelfont=sf,textfont=sf]{subfig}
\usepackage{textcomp}
\usepackage{stfloats}
\usepackage{url}
\usepackage{verbatim}
\usepackage{graphicx}
\usepackage{cite}
\usepackage{multirow}
\usepackage{pifont}
\usepackage{booktabs}
\usepackage{tabularx}
\usepackage[table, dvipsnames]{xcolor}

\usepackage[switch]{lineno}

%
% --- inline annotations
%
\newcommand{\red}[1]{{\color{red}#1}}
\newcommand{\todo}[1]{{\color{red}#1}}
\newcommand{\TODO}[1]{\textbf{\color{red}[TODO: #1]}}
% --- disable by uncommenting  
% \renewcommand{\TODO}[1]{}
% \renewcommand{\todo}[1]{#1}



\newcommand{\VLM}{LVLM\xspace} 
\newcommand{\ours}{PeKit\xspace}
\newcommand{\yollava}{Yo’LLaVA\xspace}

\newcommand{\thisismy}{This-Is-My-Img\xspace}
\newcommand{\myparagraph}[1]{\noindent\textbf{#1}}
\newcommand{\vdoro}[1]{{\color[rgb]{0.4, 0.18, 0.78} {[V] #1}}}
% --- disable by uncommenting  
% \renewcommand{\TODO}[1]{}
% \renewcommand{\todo}[1]{#1}
\usepackage{slashbox}
% Vectors
\newcommand{\bB}{\mathcal{B}}
\newcommand{\bw}{\mathbf{w}}
\newcommand{\bs}{\mathbf{s}}
\newcommand{\bo}{\mathbf{o}}
\newcommand{\bn}{\mathbf{n}}
\newcommand{\bc}{\mathbf{c}}
\newcommand{\bp}{\mathbf{p}}
\newcommand{\bS}{\mathbf{S}}
\newcommand{\bk}{\mathbf{k}}
\newcommand{\bmu}{\boldsymbol{\mu}}
\newcommand{\bx}{\mathbf{x}}
\newcommand{\bg}{\mathbf{g}}
\newcommand{\be}{\mathbf{e}}
\newcommand{\bX}{\mathbf{X}}
\newcommand{\by}{\mathbf{y}}
\newcommand{\bv}{\mathbf{v}}
\newcommand{\bz}{\mathbf{z}}
\newcommand{\bq}{\mathbf{q}}
\newcommand{\bff}{\mathbf{f}}
\newcommand{\bu}{\mathbf{u}}
\newcommand{\bh}{\mathbf{h}}
\newcommand{\bb}{\mathbf{b}}

\newcommand{\rone}{\textcolor{green}{R1}}
\newcommand{\rtwo}{\textcolor{orange}{R2}}
\newcommand{\rthree}{\textcolor{red}{R3}}
\usepackage{amsmath}
%\usepackage{arydshln}
\DeclareMathOperator{\similarity}{sim}
\DeclareMathOperator{\AvgPool}{AvgPool}

\newcommand{\argmax}{\mathop{\mathrm{argmax}}}     



\begin{document}
\linenumbers
\renewcommand\makeLineNumber{}

\title{Hier-EgoPack: Hierarchical Egocentric Video Understanding with Diverse Task Perspectives}

\author{Simone Alberto Peirone, Francesca Pistilli, Antonio Alliegro, Tatiana Tommasi and Giuseppe Averta
        % <-this % stops a space
\thanks{The authors are with the Department of
Control and Computer Engineering, at Politecnico di Torino, Italy. Email: \texttt{firstname.lastname@polito.it}.}}

\markboth{Journal of \LaTeX\ Class Files,~Vol.~14, No.~8, August~2021}%
{Peirone \MakeLowercase{\textit{et al.}}: Hier-EgoPack: Hierarchical Egocentric Video Understanding with Diverse Task Perspectives}

\IEEEpubid{0000--0000/00\$00.00~\copyright~2021 IEEE}

\IEEEtitleabstractindextext{%
\begin{abstract}


The choice of representation for geographic location significantly impacts the accuracy of models for a broad range of geospatial tasks, including fine-grained species classification, population density estimation, and biome classification. Recent works like SatCLIP and GeoCLIP learn such representations by contrastively aligning geolocation with co-located images. While these methods work exceptionally well, in this paper, we posit that the current training strategies fail to fully capture the important visual features. We provide an information theoretic perspective on why the resulting embeddings from these methods discard crucial visual information that is important for many downstream tasks. To solve this problem, we propose a novel retrieval-augmented strategy called RANGE. We build our method on the intuition that the visual features of a location can be estimated by combining the visual features from multiple similar-looking locations. We evaluate our method across a wide variety of tasks. Our results show that RANGE outperforms the existing state-of-the-art models with significant margins in most tasks. We show gains of up to 13.1\% on classification tasks and 0.145 $R^2$ on regression tasks. All our code and models will be made available at: \href{https://github.com/mvrl/RANGE}{https://github.com/mvrl/RANGE}.

\end{abstract}


\begin{IEEEkeywords}
Egocentric Vision, Video Understanding, Multi-Task Learning.
\end{IEEEkeywords}
}

\maketitle

\section{Introduction}

Video generation has garnered significant attention owing to its transformative potential across a wide range of applications, such media content creation~\citep{polyak2024movie}, advertising~\citep{zhang2024virbo,bacher2021advert}, video games~\citep{yang2024playable,valevski2024diffusion, oasis2024}, and world model simulators~\citep{ha2018world, videoworldsimulators2024, agarwal2025cosmos}. Benefiting from advanced generative algorithms~\citep{goodfellow2014generative, ho2020denoising, liu2023flow, lipman2023flow}, scalable model architectures~\citep{vaswani2017attention, peebles2023scalable}, vast amounts of internet-sourced data~\citep{chen2024panda, nan2024openvid, ju2024miradata}, and ongoing expansion of computing capabilities~\citep{nvidia2022h100, nvidia2023dgxgh200, nvidia2024h200nvl}, remarkable advancements have been achieved in the field of video generation~\citep{ho2022video, ho2022imagen, singer2023makeavideo, blattmann2023align, videoworldsimulators2024, kuaishou2024klingai, yang2024cogvideox, jin2024pyramidal, polyak2024movie, kong2024hunyuanvideo, ji2024prompt}.


In this work, we present \textbf{\ours}, a family of rectified flow~\citep{lipman2023flow, liu2023flow} transformer models designed for joint image and video generation, establishing a pathway toward industry-grade performance. This report centers on four key components: data curation, model architecture design, flow formulation, and training infrastructure optimization—each rigorously refined to meet the demands of high-quality, large-scale video generation.


\begin{figure}[ht]
    \centering
    \begin{subfigure}[b]{0.82\linewidth}
        \centering
        \includegraphics[width=\linewidth]{figures/t2i_1024.pdf}
        \caption{Text-to-Image Samples}\label{fig:main-demo-t2i}
    \end{subfigure}
    \vfill
    \begin{subfigure}[b]{0.82\linewidth}
        \centering
        \includegraphics[width=\linewidth]{figures/t2v_samples.pdf}
        \caption{Text-to-Video Samples}\label{fig:main-demo-t2v}
    \end{subfigure}
\caption{\textbf{Generated samples from \ours.} Key components are highlighted in \textcolor{red}{\textbf{RED}}.}\label{fig:main-demo}
\end{figure}


First, we present a comprehensive data processing pipeline designed to construct large-scale, high-quality image and video-text datasets. The pipeline integrates multiple advanced techniques, including video and image filtering based on aesthetic scores, OCR-driven content analysis, and subjective evaluations, to ensure exceptional visual and contextual quality. Furthermore, we employ multimodal large language models~(MLLMs)~\citep{yuan2025tarsier2} to generate dense and contextually aligned captions, which are subsequently refined using an additional large language model~(LLM)~\citep{yang2024qwen2} to enhance their accuracy, fluency, and descriptive richness. As a result, we have curated a robust training dataset comprising approximately 36M video-text pairs and 160M image-text pairs, which are proven sufficient for training industry-level generative models.

Secondly, we take a pioneering step by applying rectified flow formulation~\citep{lipman2023flow} for joint image and video generation, implemented through the \ours model family, which comprises Transformer architectures with 2B and 8B parameters. At its core, the \ours framework employs a 3D joint image-video variational autoencoder (VAE) to compress image and video inputs into a shared latent space, facilitating unified representation. This shared latent space is coupled with a full-attention~\citep{vaswani2017attention} mechanism, enabling seamless joint training of image and video. This architecture delivers high-quality, coherent outputs across both images and videos, establishing a unified framework for visual generation tasks.


Furthermore, to support the training of \ours at scale, we have developed a robust infrastructure tailored for large-scale model training. Our approach incorporates advanced parallelism strategies~\citep{jacobs2023deepspeed, pytorch_fsdp} to manage memory efficiently during long-context training. Additionally, we employ ByteCheckpoint~\citep{wan2024bytecheckpoint} for high-performance checkpointing and integrate fault-tolerant mechanisms from MegaScale~\citep{jiang2024megascale} to ensure stability and scalability across large GPU clusters. These optimizations enable \ours to handle the computational and data challenges of generative modeling with exceptional efficiency and reliability.


We evaluate \ours on both text-to-image and text-to-video benchmarks to highlight its competitive advantages. For text-to-image generation, \ours-T2I demonstrates strong performance across multiple benchmarks, including T2I-CompBench~\citep{huang2023t2i-compbench}, GenEval~\citep{ghosh2024geneval}, and DPG-Bench~\citep{hu2024ella_dbgbench}, excelling in both visual quality and text-image alignment. In text-to-video benchmarks, \ours-T2V achieves state-of-the-art performance on the UCF-101~\citep{ucf101} zero-shot generation task. Additionally, \ours-T2V attains an impressive score of \textbf{84.85} on VBench~\citep{huang2024vbench}, securing the top position on the leaderboard (as of 2025-01-25) and surpassing several leading commercial text-to-video models. Qualitative results, illustrated in \Cref{fig:main-demo}, further demonstrate the superior quality of the generated media samples. These findings underscore \ours's effectiveness in multi-modal generation and its potential as a high-performing solution for both research and commercial applications.
\section{Related Work}
\label{sec:related-works}
\subsection{Novel View Synthesis}
Novel view synthesis is a foundational task in the computer vision and graphics, which aims to generate unseen views of a scene from a given set of images.
% Many methods have been designed to solve this problem by posing it as 3D geometry based rendering, where point clouds~\cite{point_differentiable,point_nfs}, mesh~\cite{worldsheet,FVS,SVS}, planes~\cite{automatci_photo_pop_up,tour_into_the_picture} and multi-plane images~\cite{MINE,single_view_mpi,stereo_magnification}, \etal
Numerous methods have been developed to address this problem by approaching it as 3D geometry-based rendering, such as using meshes~\cite{worldsheet,FVS,SVS}, MPI~\cite{MINE,single_view_mpi,stereo_magnification}, point clouds~\cite{point_differentiable,point_nfs}, etc.
% planes~\cite{automatci_photo_pop_up,tour_into_the_picture}, 


\begin{figure*}[!t]
    \centering
    \includegraphics[width=1.0\linewidth]{figures/overview-v7.png}
    %\caption{\textbf{Overview.} Given a set of images, our method obtains both camera intrinsics and extrinsics, as well as a 3DGS model. First, we obtain the initial camera parameters, global track points from image correspondences and monodepth with reprojection loss. Then we incorporate the global track information and select Gaussian kernels associated with track points. We jointly optimize the parameters $K$, $T_{cw}$, 3DGS through multi-view geometric consistency $L_{t2d}$, $L_{t3d}$, $L_{scale}$ and photometric consistency $L_1$, $L_{D-SSIM}$.}
    \caption{\textbf{Overview.} Given a set of images, our method obtains both camera intrinsics and extrinsics, as well as a 3DGS model. During the initialization, we extract the global tracks, and initialize camera parameters and Gaussians from image correspondences and monodepth with reprojection loss. We determine Gaussian kernels with recovered 3D track points, and then jointly optimize the parameters $K$, $T_{cw}$, 3DGS through the proposed global track constraints (i.e., $L_{t2d}$, $L_{t3d}$, and $L_{scale}$) and original photometric losses (i.e., $L_1$ and $L_{D-SSIM}$).}
    \label{fig:overview}
\end{figure*}

Recently, Neural Radiance Fields (NeRF)~\cite{2020NeRF} provide a novel solution to this problem by representing scenes as implicit radiance fields using neural networks, achieving photo-realistic rendering quality. Although having some works in improving efficiency~\cite{instant_nerf2022, lin2022enerf}, the time-consuming training and rendering still limit its practicality.
Alternatively, 3D Gaussian Splatting (3DGS)~\cite{3DGS2023} models the scene as explicit Gaussian kernels, with differentiable splatting for rendering. Its improved real-time rendering performance, lower storage and efficiency, quickly attract more attentions.
% Different from NeRF-based methods which need MLPs to model the scene and huge computational cost for rendering, 3DGS has stronger real-time performance, higher storage and computational efficiency, benefits from its explicit representation and gradient backpropagation.

\subsection{Optimizing Camera Poses in NeRFs and 3DGS}
Although NeRF and 3DGS can provide impressive scene representation, these methods all need accurate camera parameters (both intrinsic and extrinsic) as additional inputs, which are mostly obtained by COLMAP~\cite{colmap2016}.
% This strong reliance on COLMAP significantly limits their use in real-world applications, so optimizing the camera parameters during the scene training becomes crucial.
When the prior is inaccurate or unknown, accurately estimating camera parameters and scene representations becomes crucial.

% In early works, only photometric constraints are used for scene training and camera pose estimation. 
% iNeRF~\cite{iNerf2021} optimizes the camera poses based on a pre-trained NeRF model.
% NeRFmm~\cite{wang2021nerfmm} introduce a joint optimization process, which estimates the camera poses and trains NeRF model jointly.
% BARF~\cite{barf2021} and GARF~\cite{2022GARF} provide new positional encoding strategy to handle with the gradient inconsistency issue of positional embedding and yield promising results.
% However, they achieve satisfactory optimization results when only the pose initialization is quite closed to the ground-truth, as the photometric constrains can only improve the quality of camera estimation within a small range.
% Later, more prior information of geometry and correspondence, \ie monocular depth and feature matching, are introduced into joint optimisation to enhance the capability of camera poses estimation.
% SC-NeRF~\cite{SCNeRF2021} minimizes a projected ray distance loss based on correspondence of adjacent frames.
% NoPe-NeRF~\cite{bian2022nopenerf} chooses monocular depth maps as geometric priors, and defines undistorted depth loss and relative pose constraints for joint optimization.
In earlier studies, scene training and camera pose estimation relied solely on photometric constraints. iNeRF~\cite{iNerf2021} refines the camera poses using a pre-trained NeRF model. NeRFmm~\cite{wang2021nerfmm} introduces a joint optimization approach that simultaneously estimates camera poses and trains the NeRF model. BARF~\cite{barf2021} and GARF~\cite{2022GARF} propose a new positional encoding strategy to address the gradient inconsistency issues in positional embedding, achieving promising results. However, these methods only yield satisfactory optimization when the initial pose is very close to the ground truth, as photometric constraints alone can only enhance camera estimation quality within a limited range. Subsequently, 
% additional prior information on geometry and correspondence, such as monocular depth and feature matching, has been incorporated into joint optimization to improve the accuracy of camera pose estimation. 
SC-NeRF~\cite{SCNeRF2021} minimizes a projected ray distance loss based on correspondence between adjacent frames. NoPe-NeRF~\cite{bian2022nopenerf} utilizes monocular depth maps as geometric priors and defines undistorted depth loss and relative pose constraints.

% With regard to 3D Gaussian Splatting, CF-3DGS~\cite{CF-3DGS-2024} also leverages mono-depth information to constrain the optimization of local 3DGS for relative pose estimation and later learn a global 3DGS progressively in a sequential manner.
% InstantSplat~\cite{fan2024instantsplat} focus on sparse view scenes, first use DUSt3R~\cite{dust3r2024cvpr} to generate a set of densely covered and pixel-aligned points for 3D Gaussian initialization, then introduce a parallel grid partitioning strategy in joint optimization to speed up.
% % Jiang et al.~\cite{Jiang_2024sig} proposed to build the scene continuously and progressively, to next unregistered frame, they use registration and adjustment to adjust the previous registered camera poses and align unregistered monocular depths, later refine the joint model by matching detected correspondences in screen-space coordinates.
% \gjh{Jiang et al.~\cite{Jiang_2024sig} also implemented an incremental approach for reconstructing camera poses and scenes. Initially, they perform feature matching between the current image and the image rendered by a differentiable surface renderer. They then construct matching point errors, depth errors, and photometric errors to achieve the registration and adjustment of the current image. Finally, based on the depth map, the pixels of the current image are projected as new 3D Gaussians. However, this method still exhibits limitations when dealing with complex scenes and unordered images.}
% % CG-3DGS~\cite{sun2024correspondenceguidedsfmfree3dgaussian} follows CF-3DGS, first construct a coarse point cloud from mono-depth maps to train a 3DGS model, then progressively estimate camera poses based on this pre-trained model by constraining the correspondences between rendering view and ground-truth.
% \gjh{Similarly, CG-3DGS~\cite{sun2024correspondenceguidedsfmfree3dgaussian} first utilizes monocular depth estimation and the camera parameters from the first frame to initialize a set of 3D Gaussians. It then progressively estimates camera poses based on this pre-trained model by constraining the correspondences between the rendered views and the ground truth.}
% % Free-SurGS~\cite{freesurgs2024} matches the projection flow derived from 3D Gaussians with optical flow to estimate the poses, to compensate for the limitations of photometric loss.
% \gjh{Free-SurGS~\cite{freesurgs2024} introduces the first SfM-free 3DGS approach for surgical scene reconstruction. Due to the challenges posed by weak textures and photometric inconsistencies in surgical scenes, Free-SurGS achieves pose estimation by minimizing the flow loss between the projection flow and the optical flow. Subsequently, it keeps the camera pose fixed and optimizes the scene representation by minimizing the photometric loss, depth loss and flow loss.}
% \gjh{However, most current works assume camera intrinsics are known and primarily focus on optimizing camera poses. Additionally, these methods typically rely on sequentially ordered image inputs and incrementally optimize camera parameters and scene representation. This inevitably leads to drift errors, preventing the achievement of globally consistent results. Our work aims to address these issues.}

Regarding 3D Gaussian Splatting, CF-3DGS~\cite{CF-3DGS-2024} utilizes mono-depth information to refine the optimization of local 3DGS for relative pose estimation and subsequently learns a global 3DGS in a sequential manner. InstantSplat~\cite{fan2024instantsplat} targets sparse view scenes, initially employing DUSt3R~\cite{dust3r2024cvpr} to create a densely covered, pixel-aligned point set for initializing 3D Gaussian models, and then implements a parallel grid partitioning strategy to accelerate joint optimization. Jiang \etal~\cite{Jiang_2024sig} develops an incremental method for reconstructing camera poses and scenes, but it struggles with complex scenes and unordered images. 
% Similarly, CG-3DGS~\cite{sun2024correspondenceguidedsfmfree3dgaussian} progressively estimates camera poses using a pre-trained model by aligning the correspondences between rendered views and actual scenes. Free-SurGS~\cite{freesurgs2024} pioneers an SfM-free 3DGS method for reconstructing surgical scenes, overcoming challenges such as weak textures and photometric inconsistencies by minimizing the discrepancy between projection flow and optical flow.
%\pb{SF-3DGS-HT~\cite{ji2024sfmfree3dgaussiansplatting} introduced VFI into training as additional photometric constraints. They separated the whole scene into several local 3DGS models and then merged them hierarchically, which leads to a significant improvement on simple and dense view scenes.}
HT-3DGS~\cite{ji2024sfmfree3dgaussiansplatting} interpolates frames for training and splits the scene into local clips, using a hierarchical strategy to build 3DGS model. It works well for simple scenes, but fails with dramatic motions due to unstable interpolation and low efficiency.
% {While effective for simple scenes, it struggles with dramatic motion due to unstable view interpolation and suffers from low computational efficiency.}

However, most existing methods generally depend on sequentially ordered image inputs and incrementally optimize camera parameters and 3DGS, which often leads to drift errors and hinders achieving globally consistent results. Our work seeks to overcome these limitations.

\section{Study Design}
% robot: aliengo 
% We used the Unitree AlienGo quadruped robot. 
% See Appendix 1 in AlienGo Software Guide PDF
% Weight = 25kg, size (L,W,H) = (0.55, 0.35, 06) m when standing, (0.55, 0.35, 0.31) m when walking
% Handle is 0.4 m or 0.5 m. I'll need to check it to see which type it is.
We gathered input from primary stakeholders of the robot dog guide, divided into three subgroups: BVI individuals who have owned a dog guide, BVI individuals who were not dog guide owners, and sighted individuals with generally low degrees of familiarity with dog guides. While the main focus of this study was on the BVI participants, we elected to include survey responses from sighted participants given the importance of social acceptance of the robot by the general public, which could reflect upon the BVI users themselves and affect their interactions with the general population \cite{kayukawa2022perceive}. 

The need-finding processes consisted of two stages. During Stage 1, we conducted in-depth interviews with BVI participants, querying their experiences in using conventional assistive technologies and dog guides. During Stage 2, a large-scale survey was distributed to both BVI and sighted participants. 

This study was approved by the University’s Institutional Review Board (IRB), and all processes were conducted after obtaining the participants' consent.

\subsection{Stage 1: Interviews}
We recruited nine BVI participants (\textbf{Table}~\ref{tab:bvi-info}) for in-depth interviews, which lasted 45-90 minutes for current or former dog guide owners (DO) and 30-60 minutes for participants without dog guides (NDO). Group DO consisted of five participants, while Group NDO consisted of four participants.
% The interview participants were divided into two groups. Group DO (Dog guide Owner) consisted of five participants who were current or former dog guide owners and Group NDO (Non Dog guide Owner) consisted of three participants who were not dog guide owners. 
All participants were familiar with using white canes as a mobility aid. 

We recruited participants in both groups, DO and NDO, to gather data from those with substantial experience with dog guides, offering potentially more practical insights, and from those without prior experience, providing a perspective that may be less constrained and more open to novel approaches. 

We asked about the participants' overall impressions of a robot dog guide, expectations regarding its potential benefits and challenges compared to a conventional dog guide, their desired methods of giving commands and communicating with the robot dog guide, essential functionalities that the robot dog guide should offer, and their preferences for various aspects of the robot dog guide's form factors. 
For Group DO, we also included questions that asked about the participants' experiences with conventional dog guides. 

% We obtained permission to record the conversations for our records while simultaneously taking notes during the interviews. The interviews lasted 30-60 minutes for NDO participants and 45-90 minutes for DO participants. 

\subsection{Stage 2: Large-Scale Surveys} 
After gathering sufficient initial results from the interviews, we created an online survey for distributing to a larger pool of participants. The survey platform used was Qualtrics. 

\subsubsection{Survey Participants}
The survey had 100 participants divided into two primary groups. Group BVI consisted of 42 blind or visually impaired participants, and Group ST consisted of 58 sighted participants. \textbf{Table}~\ref{tab:survey-demographics} shows the demographic information of the survey participants. 

\subsubsection{Question Differentiation} 
Based on their responses to initial qualifying questions, survey participants were sorted into three subgroups: DO, NDO, and ST. Each participant was assigned one of three different versions of the survey. The surveys for BVI participants mirrored the interview categories (overall impressions, communication methods, functionalities, and form factors), but with a more quantitative approach rather than the open-ended questions used in interviews. The DO version included additional questions pertaining to their prior experience with dog guides. The ST version revolved around the participants' prior interactions with and feelings toward dog guides and dogs in general, their thoughts on a robot dog guide, and broad opinions on the aesthetic component of the robot's design. 

\section{Experiments}

\subsection{Setups}
\subsubsection{Implementation Details}
We apply our FDS method to two types of 3DGS: 
the original 3DGS, and 2DGS~\citep{huang20242d}. 
%
The number of iterations in our optimization 
process is 35,000.
We follow the default training configuration 
and apply our FDS method after 15,000 iterations,
then we add normal consistency loss for both
3DGS and 2DGS after 25000 iterations.
%
The weight for FDS, $\lambda_{fds}$, is set to 0.015,
the $\sigma$ is set to 23,
and the weight for normal consistency is set to 0.15
for all experiments. 
We removed the depth distortion loss in 2DGS 
because we found that it degrades its results in indoor scenes.
%
The Gaussian point cloud is initialized using Colmap
for all datasets.
%
%
We tested the impact of 
using Sea Raft~\citep{wang2025sea} and 
Raft\citep{teed2020raft} on FDS performance.
%
Due to the blurriness of the ScanNet dataset, 
additional prior constraints are required.
Thus, we incorporate normal prior supervision 
on the rendered normals 
in ScanNet (V2) dataset by default.
The normal prior is predicted by the Stable Normal 
model~\citep{ye2024stablenormal}
across all types of 3DGS.
%
The entire framework is implemented in 
PyTorch~\citep{paszke2019pytorch}, 
and all experiments are conducted on 
a single NVIDIA 4090D GPU.

\begin{figure}[t] \centering
    \makebox[0.16\textwidth]{\scriptsize Input}
    \makebox[0.16\textwidth]{\scriptsize 3DGS}
    \makebox[0.16\textwidth]{\scriptsize 2DGS}
    \makebox[0.16\textwidth]{\scriptsize 3DGS + FDS}
    \makebox[0.16\textwidth]{\scriptsize 2DGS + FDS}
    \makebox[0.16\textwidth]{\scriptsize GT (Depth)}

    \includegraphics[width=0.16\textwidth]{figure/fig3_img/compare3/gt_rgb/frame_00522.jpg}
    \includegraphics[width=0.16\textwidth]{figure/fig3_img/compare3/3DGS/frame_00522.jpg}
    \includegraphics[width=0.16\textwidth]{figure/fig3_img/compare3/2DGS/frame_00522.jpg}
    \includegraphics[width=0.16\textwidth]{figure/fig3_img/compare3/3DGS+FDS/frame_00522.jpg}
    \includegraphics[width=0.16\textwidth]{figure/fig3_img/compare3/2DGS+FDS/frame_00522.jpg}
    \includegraphics[width=0.16\textwidth]{figure/fig3_img/compare3/gt_depth/frame_00522.jpg} \\

    % \includegraphics[width=0.16\textwidth]{figure/fig3_img/compare1/gt_rgb/frame_00137.jpg}
    % \includegraphics[width=0.16\textwidth]{figure/fig3_img/compare1/3DGS/frame_00137.jpg}
    % \includegraphics[width=0.16\textwidth]{figure/fig3_img/compare1/2DGS/frame_00137.jpg}
    % \includegraphics[width=0.16\textwidth]{figure/fig3_img/compare1/3DGS+FDS/frame_00137.jpg}
    % \includegraphics[width=0.16\textwidth]{figure/fig3_img/compare1/2DGS+FDS/frame_00137.jpg}
    % \includegraphics[width=0.16\textwidth]{figure/fig3_img/compare1/gt_depth/frame_00137.jpg} \\

     \includegraphics[width=0.16\textwidth]{figure/fig3_img/compare2/gt_rgb/frame_00262.jpg}
    \includegraphics[width=0.16\textwidth]{figure/fig3_img/compare2/3DGS/frame_00262.jpg}
    \includegraphics[width=0.16\textwidth]{figure/fig3_img/compare2/2DGS/frame_00262.jpg}
    \includegraphics[width=0.16\textwidth]{figure/fig3_img/compare2/3DGS+FDS/frame_00262.jpg}
    \includegraphics[width=0.16\textwidth]{figure/fig3_img/compare2/2DGS+FDS/frame_00262.jpg}
    \includegraphics[width=0.16\textwidth]{figure/fig3_img/compare2/gt_depth/frame_00262.jpg} \\

    \includegraphics[width=0.16\textwidth]{figure/fig3_img/compare4/gt_rgb/frame00000.png}
    \includegraphics[width=0.16\textwidth]{figure/fig3_img/compare4/3DGS/frame00000.png}
    \includegraphics[width=0.16\textwidth]{figure/fig3_img/compare4/2DGS/frame00000.png}
    \includegraphics[width=0.16\textwidth]{figure/fig3_img/compare4/3DGS+FDS/frame00000.png}
    \includegraphics[width=0.16\textwidth]{figure/fig3_img/compare4/2DGS+FDS/frame00000.png}
    \includegraphics[width=0.16\textwidth]{figure/fig3_img/compare4/gt_depth/frame00000.png} \\

    \includegraphics[width=0.16\textwidth]{figure/fig3_img/compare5/gt_rgb/frame00080.png}
    \includegraphics[width=0.16\textwidth]{figure/fig3_img/compare5/3DGS/frame00080.png}
    \includegraphics[width=0.16\textwidth]{figure/fig3_img/compare5/2DGS/frame00080.png}
    \includegraphics[width=0.16\textwidth]{figure/fig3_img/compare5/3DGS+FDS/frame00080.png}
    \includegraphics[width=0.16\textwidth]{figure/fig3_img/compare5/2DGS+FDS/frame00080.png}
    \includegraphics[width=0.16\textwidth]{figure/fig3_img/compare5/gt_depth/frame00080.png} \\



    \caption{\textbf{Comparison of depth reconstruction on Mushroom and ScanNet datasets.} The original
    3DGS or 2DGS model equipped with FDS can remove unwanted floaters and reconstruct
    geometry more preciously.}
    \label{fig:compare}
\end{figure}


\subsubsection{Datasets and Metrics}

We evaluate our method for 3D reconstruction 
and novel view synthesis tasks on
\textbf{Mushroom}~\citep{ren2024mushroom},
\textbf{ScanNet (v2)}~\citep{dai2017scannet}, and 
\textbf{Replica}~\citep{replica19arxiv}
datasets,
which feature challenging indoor scenes with both 
sparse and dense image sampling.
%
The Mushroom dataset is an indoor dataset 
with sparse image sampling and two distinct 
camera trajectories. 
%
We train our model on the training split of 
the long capture sequence and evaluate 
novel view synthesis on the test split 
of the long capture sequences.
%
Five scenes are selected to evaluate our FDS, 
including "coffee room", "honka", "kokko", 
"sauna", and "vr room". 
%
ScanNet(V2)~\citep{dai2017scannet}  consists of 1,613 indoor scenes
with annotated camera poses and depth maps. 
%
We select 5 scenes from the ScanNet (V2) dataset, 
uniformly sampling one-tenth of the views,
following the approach in ~\citep{guo2022manhattan}.
To further improve the geometry rendering quality of 3DGS, 
%
Replica~\citep{replica19arxiv} contains small-scale 
real-world indoor scans. 
We evaluate our FDS on five scenes from 
Replica: office0, office1, office2, office3 and office4,
selecting one-tenth of the views for training.
%
The results for Replica are provided in the 
supplementary materials.
To evaluate the rendering quality and geometry 
of 3DGS, we report PSNR, SSIM, and LPIPS for 
rendering quality, along with Absolute Relative Distance 
(Abs Rel) as a depth quality metrics.
%
Additionally, for mesh evaluation, 
we use metrics including Accuracy, Completion, 
Chamfer-L1 distance, Normal Consistency, 
and F-scores.




\subsection{Results}
\subsubsection{Depth rendering and novel view synthesis}
The comparison results on Mushroom and 
ScanNet are presented in \tabref{tab:mushroom} 
and \tabref{tab:scannet}, respectively. 
%
Due to the sparsity of sampling 
in the Mushroom dataset,
challenges are posed for both GOF~\citep{yu2024gaussian} 
and PGSR~\citep{chen2024pgsr}, 
leading to their relative poor performance 
on the Mushroom dataset.
%
Our approach achieves the best performance 
with the FDS method applied during the training process.
The FDS significantly enhances the 
geometric quality of 3DGS on the Mushroom dataset, 
improving the "abs rel" metric by more than 50\%.
%
We found that Sea Raft~\citep{wang2025sea}
outperforms Raft~\citep{teed2020raft} on FDS, 
indicating that a better optical flow model 
can lead to more significant improvements.
%
Additionally, the render quality of RGB 
images shows a slight improvement, 
by 0.58 in 3DGS and 0.50 in 2DGS, 
benefiting from the incorporation of cross-view consistency in FDS. 
%
On the Mushroom
dataset, adding the FDS loss increases 
the training time by half an hour, which maintains the same
level as baseline.
%
Similarly, our method shows a notable improvement on the ScanNet dataset as well using Sea Raft~\citep{wang2025sea} Model. The "abs rel" metric in 2DGS is improved nearly 50\%. This demonstrates the robustness and effectiveness of the FDS method across different datasets.
%


% \begin{wraptable}{r}{0.6\linewidth} \centering
% \caption{\textbf{Ablation study on geometry priors.}} 
%         \label{tab:analysis_prior}
%         \resizebox{\textwidth}{!}{
\begin{tabular}{c| c c c c c | c c c c}

    \hline
     Method &  Acc$\downarrow$ & Comp $\downarrow$ & C-L1 $\downarrow$ & NC $\uparrow$ & F-Score $\uparrow$ &  Abs Rel $\downarrow$ &  PSNR $\uparrow$  & SSIM  $\uparrow$ & LPIPS $\downarrow$ \\ \hline
    2DGS&   0.1078&  0.0850&  0.0964&  0.7835&  0.5170&  0.1002&  23.56&  0.8166& 0.2730\\
    2DGS+Depth&   0.0862&  0.0702&  0.0782&  0.8153&  0.5965&  0.0672&  23.92&  0.8227& 0.2619 \\
    2DGS+MVDepth&   0.2065&  0.0917&  0.1491&  0.7832&  0.3178&  0.0792&  23.74&  0.8193& 0.2692 \\
    2DGS+Normal&   0.0939&  0.0637&  0.0788&  \textbf{0.8359}&  0.5782&  0.0768&  23.78&  0.8197& 0.2676 \\
    2DGS+FDS &  \textbf{0.0615} & \textbf{ 0.0534}& \textbf{0.0574}& 0.8151& \textbf{0.6974}&  \textbf{0.0561}&  \textbf{24.06}&  \textbf{0.8271}&\textbf{0.2610} \\ \hline
    2DGS+Depth+FDS &  0.0561 &  0.0519& 0.0540& 0.8295& 0.7282&  0.0454&  \textbf{24.22}& \textbf{0.8291}&\textbf{0.2570} \\
    2DGS+Normal+FDS &  \textbf{0.0529} & \textbf{ 0.0450}& \textbf{0.0490}& \textbf{0.8477}& \textbf{0.7430}&  \textbf{0.0443}&  24.10&  0.8283& 0.2590 \\
    2DGS+Depth+Normal &  0.0695 & 0.0513& 0.0604& 0.8540&0.6723&  0.0523&  24.09&  0.8264&0.2575\\ \hline
    2DGS+Depth+Normal+FDS &  \textbf{0.0506} & \textbf{0.0423}& \textbf{0.0464}& \textbf{0.8598}&\textbf{0.7613}&  \textbf{0.0403}&  \textbf{24.22}& 
    \textbf{0.8300}&\textbf{0.0403}\\
    
\bottomrule
\end{tabular}
}
% \end{wraptable}



The qualitative comparisons on the Mushroom and ScanNet dataset 
are illustrated in \figref{fig:compare}. 
%
%
As seen in the first row of \figref{fig:compare}, 
both the original 3DGS and 2DGS suffer from overfitting, 
leading to corrupted geometry generation. 
%
Our FDS effectively mitigates this issue by 
supervising the matching relationship between 
the input and sampled views, 
helping to recover the geometry.
%
FDS also improves the refinement of geometric details, 
as shown in other rows. 
By incorporating the matching prior through FDS, 
the quality of the rendered depth is significantly improved.
%

\begin{table}[t] \centering
\begin{minipage}[t]{0.96\linewidth}
        \captionof{table}{\textbf{3D Reconstruction 
        and novel view synthesis results on Mushroom dataset. * 
        Represents that FDS uses the Raft model.
        }}
        \label{tab:mushroom}
        \resizebox{\textwidth}{!}{
\begin{tabular}{c| c c c c c | c c c c c}
    \hline
     Method &  Acc$\downarrow$ & Comp $\downarrow$ & C-L1 $\downarrow$ & NC $\uparrow$ & F-Score $\uparrow$ &  Abs Rel $\downarrow$ &  PSNR $\uparrow$  & SSIM  $\uparrow$ & LPIPS $\downarrow$ & Time  $\downarrow$ \\ \hline

    % DN-splatter &   &  &  &  &  &  &  &  & \\
    GOF &  0.1812 & 0.1093 & 0.1453 & 0.6292 & 0.3665 & 0.2380  & 21.37  &  0.7762  & 0.3132  & $\approx$1.4h\\ 
    PGSR &  0.0971 & 0.1420 & 0.1196 & 0.7193 & 0.5105 & 0.1723  & 22.13  & 0.7773  & 0.2918  & $\approx$1.2h \\ \hline
    3DGS &   0.1167 &  0.1033&  0.1100&  0.7954&  0.3739&  0.1214&  24.18&  0.8392& 0.2511 &$\approx$0.8h \\
    3DGS + FDS$^*$ & 0.0569  & 0.0676 & 0.0623 & 0.8105 & 0.6573 & 0.0603 & 24.72  & 0.8489 & 0.2379 &$\approx$1.3h \\
    3DGS + FDS & \textbf{0.0527}  & \textbf{0.0565} & \textbf{0.0546} & \textbf{0.8178} & \textbf{0.6958} & \textbf{0.0568} & \textbf{24.76}  & \textbf{0.8486} & \textbf{0.2381} &$\approx$1.3h \\ \hline
    2DGS&   0.1078&  0.0850&  0.0964&  0.7835&  0.5170&  0.1002&  23.56&  0.8166& 0.2730 &$\approx$0.8h\\
    2DGS + FDS$^*$ &  0.0689 &  0.0646& 0.0667& 0.8042& 0.6582& 0.0589& 23.98&  0.8255&0.2621 &$\approx$1.3h\\
    2DGS + FDS &  \textbf{0.0615} & \textbf{ 0.0534}& \textbf{0.0574}& \textbf{0.8151}& \textbf{0.6974}&  \textbf{0.0561}&  \textbf{24.06}&  \textbf{0.8271}&\textbf{0.2610} &$\approx$1.3h \\ \hline
\end{tabular}
}
\end{minipage}\hfill
\end{table}

\begin{table}[t] \centering
\begin{minipage}[t]{0.96\linewidth}
        \captionof{table}{\textbf{3D Reconstruction 
        and novel view synthesis results on ScanNet dataset.}}
        \label{tab:scannet}
        \resizebox{\textwidth}{!}{
\begin{tabular}{c| c c c c c | c c c c }
    \hline
     Method &  Acc $\downarrow$ & Comp $\downarrow$ & C-L1 $\downarrow$ & NC $\uparrow$ & F-Score $\uparrow$ &  Abs Rel $\downarrow$ &  PSNR $\uparrow$  & SSIM  $\uparrow$ & LPIPS $\downarrow$ \\ \hline
    GOF & 1.8671  & 0.0805 & 0.9738 & 0.5622 & 0.2526 & 0.1597  & 21.55  & 0.7575  & 0.3881 \\
    PGSR &  0.2928 & 0.5103 & 0.4015 & 0.5567 & 0.1926 & 0.1661  & 21.71 & 0.7699  & 0.3899 \\ \hline

    3DGS &  0.4867 & 0.1211 & 0.3039 & 0.7342& 0.3059 & 0.1227 & 22.19& 0.7837 & 0.3907\\
    3DGS + FDS &  \textbf{0.2458} & \textbf{0.0787} & \textbf{0.1622} & \textbf{0.7831} & 
    \textbf{0.4482} & \textbf{0.0573} & \textbf{22.83} & \textbf{0.7911} & \textbf{0.3826} \\ \hline
    2DGS &  0.2658 & 0.0845 & 0.1752 & 0.7504& 0.4464 & 0.0831 & 22.59& 0.7881 & 0.3854\\
    2DGS + FDS &  \textbf{0.1457} & \textbf{0.0679} & \textbf{0.1068} & \textbf{0.7883} & 
    \textbf{0.5459} & \textbf{0.0432} & \textbf{22.91} & \textbf{0.7928} & \textbf{0.3800} \\ \hline
\end{tabular}
}
\end{minipage}\hfill
\end{table}


\begin{table}[t] \centering
\begin{minipage}[t]{0.96\linewidth}
        \captionof{table}{\textbf{Ablation study on geometry priors.}}
        \label{tab:analysis_prior}
        \resizebox{\textwidth}{!}{
\begin{tabular}{c| c c c c c | c c c c}

    \hline
     Method &  Acc$\downarrow$ & Comp $\downarrow$ & C-L1 $\downarrow$ & NC $\uparrow$ & F-Score $\uparrow$ &  Abs Rel $\downarrow$ &  PSNR $\uparrow$  & SSIM  $\uparrow$ & LPIPS $\downarrow$ \\ \hline
    2DGS&   0.1078&  0.0850&  0.0964&  0.7835&  0.5170&  0.1002&  23.56&  0.8166& 0.2730\\
    2DGS+Depth&   0.0862&  0.0702&  0.0782&  0.8153&  0.5965&  0.0672&  23.92&  0.8227& 0.2619 \\
    2DGS+MVDepth&   0.2065&  0.0917&  0.1491&  0.7832&  0.3178&  0.0792&  23.74&  0.8193& 0.2692 \\
    2DGS+Normal&   0.0939&  0.0637&  0.0788&  \textbf{0.8359}&  0.5782&  0.0768&  23.78&  0.8197& 0.2676 \\
    2DGS+FDS &  \textbf{0.0615} & \textbf{ 0.0534}& \textbf{0.0574}& 0.8151& \textbf{0.6974}&  \textbf{0.0561}&  \textbf{24.06}&  \textbf{0.8271}&\textbf{0.2610} \\ \hline
    2DGS+Depth+FDS &  0.0561 &  0.0519& 0.0540& 0.8295& 0.7282&  0.0454&  \textbf{24.22}& \textbf{0.8291}&\textbf{0.2570} \\
    2DGS+Normal+FDS &  \textbf{0.0529} & \textbf{ 0.0450}& \textbf{0.0490}& \textbf{0.8477}& \textbf{0.7430}&  \textbf{0.0443}&  24.10&  0.8283& 0.2590 \\
    2DGS+Depth+Normal &  0.0695 & 0.0513& 0.0604& 0.8540&0.6723&  0.0523&  24.09&  0.8264&0.2575\\ \hline
    2DGS+Depth+Normal+FDS &  \textbf{0.0506} & \textbf{0.0423}& \textbf{0.0464}& \textbf{0.8598}&\textbf{0.7613}&  \textbf{0.0403}&  \textbf{24.22}& 
    \textbf{0.8300}&\textbf{0.0403}\\
    
\bottomrule
\end{tabular}
}
\end{minipage}\hfill
\end{table}




\subsubsection{Mesh extraction}
To further demonstrate the improvement in geometry quality, 
we applied methods used in ~\citep{turkulainen2024dnsplatter} 
to extract meshes from the input views of optimized 3DGS. 
The comparison results are presented  
in \tabref{tab:mushroom}. 
With the integration of FDS, the mesh quality is significantly enhanced compared to the baseline, featuring fewer floaters and more well-defined shapes.
 %
% Following the incorporation of FDS, the reconstruction 
% results exhibit fewer floaters and more well-defined 
% shapes in the meshes. 
% Visualized comparisons
% are provided in the supplementary material.

% \begin{figure}[t] \centering
%     \makebox[0.19\textwidth]{\scriptsize GT}
%     \makebox[0.19\textwidth]{\scriptsize 3DGS}
%     \makebox[0.19\textwidth]{\scriptsize 3DGS+FDS}
%     \makebox[0.19\textwidth]{\scriptsize 2DGS}
%     \makebox[0.19\textwidth]{\scriptsize 2DGS+FDS} \\

%     \includegraphics[width=0.19\textwidth]{figure/fig4_img/compare1/gt02.png}
%     \includegraphics[width=0.19\textwidth]{figure/fig4_img/compare1/baseline06.png}
%     \includegraphics[width=0.19\textwidth]{figure/fig4_img/compare1/baseline_fds05.png}
%     \includegraphics[width=0.19\textwidth]{figure/fig4_img/compare1/2dgs04.png}
%     \includegraphics[width=0.19\textwidth]{figure/fig4_img/compare1/2dgs_fds03.png} \\

%     \includegraphics[width=0.19\textwidth]{figure/fig4_img/compare2/gt00.png}
%     \includegraphics[width=0.19\textwidth]{figure/fig4_img/compare2/baseline02.png}
%     \includegraphics[width=0.19\textwidth]{figure/fig4_img/compare2/baseline_fds01.png}
%     \includegraphics[width=0.19\textwidth]{figure/fig4_img/compare2/2dgs04.png}
%     \includegraphics[width=0.19\textwidth]{figure/fig4_img/compare2/2dgs_fds03.png} \\
      
%     \includegraphics[width=0.19\textwidth]{figure/fig4_img/compare3/gt05.png}
%     \includegraphics[width=0.19\textwidth]{figure/fig4_img/compare3/3dgs03.png}
%     \includegraphics[width=0.19\textwidth]{figure/fig4_img/compare3/3dgs_fds04.png}
%     \includegraphics[width=0.19\textwidth]{figure/fig4_img/compare3/2dgs02.png}
%     \includegraphics[width=0.19\textwidth]{figure/fig4_img/compare3/2dgs_fds01.png} \\

%     \caption{\textbf{Qualitative comparison of extracted mesh 
%     on Mushroom and ScanNet datasets.}}
%     \label{fig:mesh}
% \end{figure}












\subsection{Ablation study}


\textbf{Ablation study on geometry priors:} 
To highlight the advantage of incorporating matching priors, 
we incorporated various types of priors generated by different 
models into 2DGS. These include a monocular depth estimation
model (Depth Anything v2)~\citep{yang2024depth}, a two-view depth estimation 
model (Unimatch)~\citep{xu2023unifying}, 
and a monocular normal estimation model (DSINE)~\citep{bae2024rethinking}.
We adapt the scale and shift-invariant loss in Midas~\citep{birkl2023midas} for
monocular depth supervision and L1 loss for two-view depth supervison.
%
We use Sea Raft~\citep{wang2025sea} as our default optical flow model.
%
The comparison results on Mushroom dataset 
are shown in ~\tabref{tab:analysis_prior}.
We observe that the normal prior provides accurate shape information, 
enhancing the geometric quality of the radiance field. 
%
% In contrast, the monocular depth prior slightly increases 
% the 'Abs Rel' due to its ambiguous scale and inaccurate depth ordering.
% Moreover, the performance of monocular depth estimation 
% in the sauna scene is particularly poor, 
% primarily due to the presence of numerous reflective 
% surfaces and textureless walls, which limits the accuracy of monocular depth estimation.
%
The multi-view depth prior, hindered by the limited feature overlap 
between input views, fails to offer reliable geometric 
information. We test average "Abs Rel" of multi-view depth prior
, and the result is 0.19, which performs worse than the "Abs Rel" results 
rendered by original 2DGS.
From the results, it can be seen that depth order information provided by monocular depth improves
reconstruction accuracy. Meanwhile, our FDS achieves the best performance among all the priors, 
and by integrating all
three components, we obtained the optimal results.
%
%
\begin{figure}[t] \centering
    \makebox[0.16\textwidth]{\scriptsize RF (16000 iters)}
    \makebox[0.16\textwidth]{\scriptsize RF* (20000 iters)}
    \makebox[0.16\textwidth]{\scriptsize RF (20000 iters)  }
    \makebox[0.16\textwidth]{\scriptsize PF (16000 iters)}
    \makebox[0.16\textwidth]{\scriptsize PF (20000 iters)}


    % \includegraphics[width=0.16\textwidth]{figure/fig5_img/compare1/16000.png}
    % \includegraphics[width=0.16\textwidth]{figure/fig5_img/compare1/20000_wo_flow_loss.png}
    % \includegraphics[width=0.16\textwidth]{figure/fig5_img/compare1/20000.png}
    % \includegraphics[width=0.16\textwidth]{figure/fig5_img/compare1/16000_prior.png}
    % \includegraphics[width=0.16\textwidth]{figure/fig5_img/compare1/20000_prior.png}\\

    % \includegraphics[width=0.16\textwidth]{figure/fig5_img/compare2/16000.png}
    % \includegraphics[width=0.16\textwidth]{figure/fig5_img/compare2/20000_wo_flow_loss.png}
    % \includegraphics[width=0.16\textwidth]{figure/fig5_img/compare2/20000.png}
    % \includegraphics[width=0.16\textwidth]{figure/fig5_img/compare2/16000_prior.png}
    % \includegraphics[width=0.16\textwidth]{figure/fig5_img/compare2/20000_prior.png}\\

    \includegraphics[width=0.16\textwidth]{figure/fig5_img/compare3/16000.png}
    \includegraphics[width=0.16\textwidth]{figure/fig5_img/compare3/20000_wo_flow_loss.png}
    \includegraphics[width=0.16\textwidth]{figure/fig5_img/compare3/20000.png}
    \includegraphics[width=0.16\textwidth]{figure/fig5_img/compare3/16000_prior.png}
    \includegraphics[width=0.16\textwidth]{figure/fig5_img/compare3/20000_prior.png}\\
    
    \includegraphics[width=0.16\textwidth]{figure/fig5_img/compare4/16000.png}
    \includegraphics[width=0.16\textwidth]{figure/fig5_img/compare4/20000_wo_flow_loss.png}
    \includegraphics[width=0.16\textwidth]{figure/fig5_img/compare4/20000.png}
    \includegraphics[width=0.16\textwidth]{figure/fig5_img/compare4/16000_prior.png}
    \includegraphics[width=0.16\textwidth]{figure/fig5_img/compare4/20000_prior.png}\\

    \includegraphics[width=0.30\textwidth]{figure/fig5_img/bar.png}

    \caption{\textbf{The error map of Radiance Flow and Prior Flow.} RF: Radiance Flow, PF: Prior Flow, * means that there is no FDS loss supervision during optimization.}
    \label{fig:error_map}
\end{figure}




\textbf{Ablation study on FDS: }
In this section, we present the design of our FDS 
method through an ablation study on the 
Mushroom dataset to validate its effectiveness.
%
The optional configurations of FDS are outlined in ~\tabref{tab:ablation_fds}.
Our base model is the 2DGS equipped with FDS,
and its results are shown 
in the first row. The goal of this analysis 
is to evaluate the impact 
of various strategies on FDS sampling and loss design.
%
We observe that when we 
replace $I_i$ in \eqref{equ:mflow} with $C_i$, 
as shown in the second row, the geometric quality 
of 2DGS deteriorates. Using $I_i$ instead of $C_i$ 
help us to remove the floaters in $\bm{C^s}$, which are also 
remained in $\bm{C^i}$.
We also experiment with modifying the FDS loss. For example, 
in the third row, we use the neighbor 
input view as the sampling view, and replace the 
render result of neighbor view with ground truth image of its input view.
%
Due to the significant movement between images, the Prior Flow fails to accurately 
match the pixel between them, leading to a further degradation in geometric quality.
%
Finally, we attempt to fix the sampling view 
and found that this severely damaged the geometric quality, 
indicating that random sampling is essential for the stability 
of the mean error in the Prior flow.



\begin{table}[t] \centering

\begin{minipage}[t]{1.0\linewidth}
        \captionof{table}{\textbf{Ablation study on FDS strategies.}}
        \label{tab:ablation_fds}
        \resizebox{\textwidth}{!}{
\begin{tabular}{c|c|c|c|c|c|c|c}
    \hline
    \multicolumn{2}{c|}{$\mathcal{M}_{\theta}(X, \bm{C^s})$} & \multicolumn{3}{c|}{Loss} & \multicolumn{3}{c}{Metric}  \\
    \hline
    $X=C^i$ & $X=I^i$  & Input view & Sampled view     & Fixed Sampled view        & Abs Rel $\downarrow$ & F-score $\uparrow$ & NC $\uparrow$ \\
    \hline
    & \ding{51} &     &\ding{51}    &    &    \textbf{0.0561}        &  \textbf{0.6974}         & \textbf{0.8151}\\
    \hline
     \ding{51} &           &     &\ding{51}    &    &    0.0839        &  0.6242         &0.8030\\
     &  \ding{51} &   \ding{51}  &    &    &    0.0877       & 0.6091        & 0.7614 \\
      &  \ding{51} &    &    & \ding{51}    &    0.0724           & 0.6312          & 0.8015 \\
\bottomrule
\end{tabular}
}
\end{minipage}
\end{table}




\begin{figure}[htbp] \centering
    \makebox[0.22\textwidth]{}
    \makebox[0.22\textwidth]{}
    \makebox[0.22\textwidth]{}
    \makebox[0.22\textwidth]{}
    \\

    \includegraphics[width=0.22\textwidth]{figure/fig6_img/l1/rgb/frame00096.png}
    \includegraphics[width=0.22\textwidth]{figure/fig6_img/l1/render_rgb/frame00096.png}
    \includegraphics[width=0.22\textwidth]{figure/fig6_img/l1/render_depth/frame00096.png}
    \includegraphics[width=0.22\textwidth]{figure/fig6_img/l1/depth/frame00096.png}

    % \includegraphics[width=0.22\textwidth]{figure/fig6_img/l2/rgb/frame00112.png}
    % \includegraphics[width=0.22\textwidth]{figure/fig6_img/l2/render_rgb/frame00112.png}
    % \includegraphics[width=0.22\textwidth]{figure/fig6_img/l2/render_depth/frame00112.png}
    % \includegraphics[width=0.22\textwidth]{figure/fig6_img/l2/depth/frame00112.png}

    \caption{\textbf{Limitation of FDS.} }
    \label{fig:limitation}
\end{figure}


% \begin{figure}[t] \centering
%     \makebox[0.48\textwidth]{}
%     \makebox[0.48\textwidth]{}
%     \\
%     \includegraphics[width=0.48\textwidth]{figure/loss_Ignatius.pdf}
%     \includegraphics[width=0.48\textwidth]{figure/loss_family.pdf}
%     \caption{\textbf{Comparison the photometric error of Radiance Flow and Prior Flow:} 
%     We add FDS method after 2k iteration during training.
%     The results show
%     that:  1) The Prior Flow is more precise and 
%     robust than Radiance Flow during the radiance 
%     optimization; 2) After adding the FDS loss 
%     which utilize Prior 
%     flow to supervise the Radiance Flow at 2k iterations, 
%     both flow are more accurate, which lead to
%     a mutually reinforcing effects.(TODO fix it)} 
%     \label{fig:flowcompare}
% \end{figure}






\textbf{Interpretive Experiments: }
To demonstrate the mutual refinement of two flows in our FDS, 
For each view, we sample the unobserved 
views multiple times to compute the mean error 
of both Radiance Flow and Prior Flow. 
We use Raft~\citep{teed2020raft} as our default optical flow model
for visualization.
The ground truth flow is calculated based on 
~\eref{equ:flow_pose} and ~\eref{equ:flow} 
utilizing ground truth depth in dataset.
We introduce our FDS loss after 16000 iterations during 
optimization of 2DGS.
The error maps are shown in ~\figref{fig:error_map}.
Our analysis reveals that Radiance Flow tends to 
exhibit significant geometric errors, 
whereas Prior Flow can more accurately estimate the geometry,
effectively disregarding errors introduced by floating Gaussian points. 

%





\subsection{Limitation and further work}

Firstly, our FDS faces challenges in scenes with 
significant lighting variations between different 
views, as shown in the lamp of first row in ~\figref{fig:limitation}. 
%
Incorporating exposure compensation into FDS could help address this issue. 
%
 Additionally, our method struggles with 
 reflective surfaces and motion blur,
 leading to incorrect matching. 
 %
 In the future, we plan to explore the potential 
 of FDS in monocular video reconstruction tasks, 
 using only a single input image at each time step.
 


\section{Conclusions}
In this paper, we propose Flow Distillation Sampling (FDS), which
leverages the matching prior between input views and 
sampled unobserved views from the pretrained optical flow model, to improve the geometry quality
of Gaussian radiance field. 
Our method can be applied to different approaches (3DGS and 2DGS) to enhance the geometric rendering quality of the corresponding neural radiance fields.
We apply our method to the 3DGS-based framework, 
and the geometry is enhanced on the Mushroom, ScanNet, and Replica datasets.

\section*{Acknowledgements} This work was supported by 
National Key R\&D Program of China (2023YFB3209702), 
the National Natural Science Foundation of 
China (62441204, 62472213), and Gusu 
Innovation \& Entrepreneurship Leading Talents Program (ZXL2024361)
\section{Conclusions}
\label{sec:conclusions}

In this paper, we introduce a novel sketch-to-image translation technique that uses a learnable lightweight mapping network (LCTN) for latent code translation from sketch to image domain, followed by $k$ forward diffusion and $T$ backward denoising steps through a pre-trained text-to-image LDM. We show that by selecting an optimal value for $k \sim [1, T]$ near the upper threshold ($k \approx T$, $k < T$), it is possible to generate highly detailed photorealistic images that closely resemble the intended structures in the given sketches. Our experiments demonstrate that the proposed technique outperforms the existing methods in most visual and analytical comparisons across multiple datasets. Additionally, we show that the proposed method retains structural consistency across different visual styles, allowing photorealistic style manipulation in the generated images.


\section*{Acknowledgments}
This study was carried out within the FAIR - Future Artificial Intelligence Research and received funding from the European Union Next-GenerationEU (PIANO NAZIONALE DI RIPRESA E RESILIENZA (PNRR) – MISSIONE 4 COMPONENTE 2, INVESTIMENTO 1.3 – D.D. 1555 11/10/2022, PE00000013). This manuscript reflects only the authors’ views and opinions, neither the European Union nor the European Commission can be considered responsible for them. We acknowledge the CINECA award under the ISCRA initiative, for the availability of high performance computing resources and support. Antonio Alliegro and Tatiana Tommasi also acknowledge the EU project ELSA - European Lighthouse on Secure and Safe AI (grant number 101070617).


\bibliographystyle{IEEEtran}
\bibliography{main}

\vspace{-33pt}
\begin{IEEEbiography}[{\includegraphics[width=1in,height=1.25in,clip,keepaspectratio]{figures/profile_pictures/simone.jpg}}]{Simone Alberto Peirone}
received the B.Sc. and M.Sc. degrees in Computer Engineering from the Polytechnic University of Turin, in 2020 and 2022 respectively. He is currently pursuing the Ph.D. degree with the Visual and Multimodal Applied Learning Laboratory, Turin, working under the supervision of Prof. Giuseppe Averta. His research focuses on egocentric vision and applications of graph neural networks to video understanding.
\end{IEEEbiography}

\vspace{-33pt}
\begin{IEEEbiography}[{\includegraphics[width=1in,height=1.25in,clip,keepaspectratio]{figures/profile_pictures/francesca_pistilli.jpg}}]{Francesca Pistilli}
received the M.Sc. degree in electronic engineering from the Polytechnic of Turin, in 2019, the M.Sc. degree in electrical and computer engineering from the University of Illinois at Chicago, Chicago, IL, USA, in 2020, and the Ph.D. degree from the Image Processing and Learning Group (IPL), Polytechnic of Turin, in 2023. She is currently Assistant Professor at the Polytechnic of Turin. Her current research interests lie at the intersection between computer vision and robotics.
\end{IEEEbiography}

\vspace{-33pt}
\begin{IEEEbiography}[{\includegraphics[width=1in,height=1.25in,clip,keepaspectratio]{figures/profile_pictures/alli.jpg}}]{Antonio Alliegro}
is currently a Postdoctoral Researcher with the Polytechnic University of Turin. His research focuses on 3D understanding and its application to real-world scenarios, including research on reducing the synth-to-real domain gap and open-set 3D recognition. 
He has published multiple papers presented at prestigious computer vision conferences and journals such as CVPR, IROS, NeurIPS, and RA-L. 
Additionally, he has contributed as a reviewer at various academic events.
\end{IEEEbiography}

\begin{IEEEbiography}[{\includegraphics[width=1in,height=1.25in,clip,keepaspectratio]{figures/profile_pictures/TTommasi_crop2.jpg}}]{Tatiana Tommasi}
received the Ph.D. degree from EPFL Lausanne, in 2013. Subsequently, she undertook a postdoctoral roles in both Belgium and the USA. She holds the position of an Associate Professor with the Control and Computer Engineering Department, Polytechnic University of Turin, and the Director of the ELLIS Unit, Turin. Before her current position, she was an Assistant Professor with Sapienza University in Rome, Italy. Her publication record boasts over 50 papers in top conferences and journals, specializing in machine learning and computer vision. Her expertise lies in the development of theoretically grounded algorithms for automatic learning from images, particularly within the realms of robotics, medical applications, and human–machine interaction. She pioneered the field of transfer learning in computer vision and possesses extensive experience in areas, such as domain adaptation, generalization, multimodal learning, and open-set learning. She is an Associate Editor of IEEE Transaction on Pattern Analysis and Machine Intelligence and IEEE Transaction on Emerging Topics in Computing.
\end{IEEEbiography}

\begin{IEEEbiography}[{\includegraphics[width=1in,height=1.25in,clip,keepaspectratio]{figures/profile_pictures/ga_pic.jpg}}]{Giuseppe Averta}
received the PhD in robotics from the University of Pisa in 2020. In 2019, he was a Visiting Student with the Eric P. and Evelyn E. Newman Laboratory, Biomechanics and Human Rehabilitation Group, MIT. He is currently an Assistant Professor of robotics and machine learning with the Polytechnic of Turin. He is also an Italian Institute of Technology Alumnus. His current research interests include the development of a truly embodied intelligence for human-robot cooperation, with research activities in human action recognition, deep learning for egocentric vision, human-inspired design, planning, and control guidelines for autonomous, collaborative, assistive, and prosthetic robots.
\end{IEEEbiography}

\vfill

\end{document}
