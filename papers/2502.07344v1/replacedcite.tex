\section{Related work and main contributions}
\label{sec:related-work}
Existing wind power prediction methods can generally be categorized into four types: physical methods, statistical methods, machine learning methods, and hybrid prediction methods ____.

Detailed physical models leverage meteorological and geographical data to simulate the dynamic processes of wind ____, offering high accuracy and strong physical interpretability. However, their computational efficiency is limited, as they must account for complex fluid dynamics and atmospheric effects ____, making them impractical for large-scale data applications. In contrast, approximate empirical physical models, while useful for qualitative physical insights, often fail to accurately predict power production at utility-scale wind farms due to numerous simplifying assumptions and neglected physical phenomena ____. Several empirical formulas have been proposed to model the power coefficient; however, no consensus has been achieved regarding a model that offers sufficient flexibility and robustness ____. Statistical methods, such as the simple moving average strategy, are commonly employed to forecast wind power using extensive historical data. However, the predictive performance of these models is constrained by their limited ability to capture nonlinear relationships ____. Compared with physical methods, statistical methods are usually simpler and more suitable for small farms ____.

Data-driven machine learning models are extensively used for wind power forecasting due to their strong nonlinear fitting capabilities and self-learning abilities ____. The direct application of machine learning methods can yield favorable results in predicting the WT power curve and significantly contribute to turbine performance analysis. However, these methods often lack robustness and physical interpretability ____. For effective real-system optimization, the predictive model must accurately replicate observed data while adhering to the physical constraints of the system, even for atypical configurations.

Finally, hybrid methods combine physics-based with data-driven approaches to effectively address regression problems ____, having advantages such as a broader knowledge base, transparency in the modeling process, and cost-effective model development ____. Multi-objective ensemble models have distinguished themselves among hybrid models by combining multiple data groups with different distributions and multiple learners to achieve superior predictive performance and generalization ____. However, there is still room for improvement through outlier correction, diversification of combination weights, or the use of residual data to refine prediction results and enhance model accuracy ____. 

The non-parametric, data-driven component of hybrid regression models has a significant limitation: it lacks inherent functional explainability, and is therefore not directly interpretable. Furthermore, the underlying functional relationships may remain unclear, meaning causality between explanatory and dependent variables cannot be guaranteed. Recent advancements in explainability and uncertainty quantification techniques have significantly improved the interpretability and reliability of predictions made by data-driven models ____. The application of these techniques to the WT power sector remains in its early stages and has been identified as a key research challenge in wind energy by the European Academy of Wind Energy (EAWE) ____. Recent studies have explored Bayesian ____ and Monte Carlo ____ methods to approximate uncertainty in power curve model estimation, while others have begun employing explainability techniques to develop robust and transparent data-driven models ____. To the best of our knowledge, no existing work has simultaneously incorporated both considerations—practices we deem essential for achieving trustworthy and robust power curve models.

In this work, we propose a hybrid model comprising a physics-based submodel for predicting the power generated by a dataset of four similar turbines and a data-driven, non-parametric submodel to learn the residuals between the physics-based submodel's output and the observed data. This approach aims to capture unmodeled physical phenomena and variations in the data that the physics-based submodel alone cannot fully account for. It is particularly advantageous when certain inputs are not reliably represented in the data or when integrating domain knowledge through a physical reference input is beneficial.
The main novelties and contributions of this paper can be summarized as follows:
\begin{itemize}
    \item A hybrid model is proposed for regression analysis of WT power generation as a function of weather conditions and internal turbine variables. This approach preserves the interpretability of the physics-based submodel while achieving high predictive accuracy.
    \item The proposed physics-based model operates with minimal assumptions regarding the functional form, deriving the generated power as a function of wind kinetic energy modulated by the power coefficient. The power coefficient is constrained by the Betz limit and is modeled using a neural network. This methodology offers significant flexibility while enabling the estimation of the power coefficient profile, which serves to characterize the aerodynamic performance of the WT.
    \item Prediction accuracy is improved by incorporating a non-parametric model that accounts for additional unmodeled input features to learn the residuals between the physics-based model's predictions and the observed data. An explainability analysis is conducted to evaluate the influence of input variables on the hybrid model's output. This analysis identifies the most significant features and provides insights into potential improvements for the physics-based model.
    \item To evaluate the reliability of the hybrid model's predictions, a conformalized quantile regression method is employed for uncertainty quantification, enabling the calculation of coverage and mean length of the uncertainty intervals. This approach enhances confidence in the model, which, to the best of our knowledge, is the first hybrid model in the wind energy sector to integrate both explainability and uncertainty quantification analyses.
    
\end{itemize}