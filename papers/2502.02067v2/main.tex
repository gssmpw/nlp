%%%%%%%%%%%%%%%%%%%%%%%%%%%%%%%%%%%%%%%%%%%%%%%%%%%%%%%%%%%%%%%%%%%%%%%%%%%%%%%%
%2345678901234567890123456789012345678901234567890123456789012345678901234567890
%        1         2         3         4         5         6         7         8

\documentclass[article, 10 pt, conference]{ieeeconf}  % Comment this line out if you need a4paper
\setlength{\parskip}{0pt}
\pdfminorversion=4
% \documentclass[a4paper, 10pt, conference]{ieeeconf}      % Use this line for a4 paper

\IEEEoverridecommandlockouts                              % This command is only needed if 
                                                          % you want to use the \thanks command

\overrideIEEEmargins  
\UseRawInputEncoding 
\let\labelindent\relax
\usepackage{enumitem}
\setlist{topsep=1pt, partopsep=0pt, parsep=0pt, itemsep=0pt}
\usepackage[
top    = 0.75in,
bottom = 0.75in,
left   = 0.75in,
right  = 0.75in]{geometry}
\usepackage{cite}
\usepackage{amsmath}
\usepackage[skip=10pt plus1pt, indent=35pt]{parskip}
\usepackage{placeins}
\usepackage{textcomp}
\usepackage[dvipsnames]{xcolor}
\definecolor{royalblue}{RGB}{65, 105, 225}
\definecolor{maroon}{RGB}{180, 0, 0}
\definecolor{DarkGreen}{RGB}{0, 100, 0}
\usepackage{booktabs}
\usepackage{multirow}
\usepackage{siunitx}
\usepackage{algorithm}
% \usepackage{algorithmic}
\usepackage{algpseudocode}
\usepackage{graphicx}
\usepackage[symbol]{footmisc}
\usepackage{adjustbox}
\usepackage{float}
% \usepackage{paralist}
\usepackage{svg}
\usepackage{array}
\usepackage[T1]{fontenc}
\usepackage{subcaption}
\usepackage{xspace}
\usepackage{scrextend}
% \usepackage[export]{adjustbox}
\usepackage{caption}
\usepackage{tabularx}
\usepackage{listings}
\usepackage{wrapfig}
\usepackage{calligra}
\usepackage{comment}
\usepackage{soul}
\usepackage{balance}
% \documentclass{article}
\usepackage{listings}
\usepackage{xcolor}
\usepackage{caption}
\newcolumntype{A}{ >{\centering\arraybackslash} m{4cm} }
\newcolumntype{B}{ >{\centering\arraybackslash} m{1cm} }
\newcolumntype{C}[1]{>{\centering\let\newline\\\arraybackslash\hspace{0pt}}m{#1}}
\newcommand{\todo}[1]{\textbf{\textcolor{red}{TODO: #1}}}
\newcommand{\Madhav}[1]{\textcolor{magenta}{Madhav: #1}}
\newcommand{\nabanita}[1]{\textcolor{violet}{#1}}
\newcommand{\karthik}[1]{\textcolor{blue}{#1}}
\newcommand{\shivam}[1]{\textcolor{olive}{#1}}
\newcommand{\ramandeep}[1]{\textcolor{orange}{#1}}
\renewcommand{\thefootnote}{\fnsymbol{footnote}}
\long\def\commentm#1{{\bf **Mohan: #1**}}

\makeatletter
\newcommand\footnoteref[1]{\protected@xdef\@thefnmark{\ref{#1}}\@footnotemark}
\makeatother


% \usepackage{flushend}
%\usepackage{epsfig} % for postscript graphics files
%\usepackage{mathptmx} % assumes new font selection scheme installed
%\usepackage{times} % assumes new font selection scheme installed
\usepackage{amsmath,lipsum} % assumes amsmath package installed
\usepackage{amssymb}  % assumes amsmath package installed
\usepackage{amsfonts}
\usepackage{textgreek}
\usepackage{authblk}
\usepackage{etoolbox}
\usepackage{algpseudocode}

\algrenewcommand\algorithmicindent{0.5em}
\newcommand\crule[3][black]{\textcolor{#1}{\rule{#2}{#3}}}
\DeclareMathOperator*{\argminA}{arg\,min} % Jan Hlavacek
\makeatletter
\let\NAT@parse\undefined

\let\oldthebibliography\thebibliography
\let\endoldthebibliography\endthebibliography
\renewenvironment{thebibliography}[1]{
  \oldthebibliography{#1}
  \setlength{\itemsep}{-1.75ex plus-.1ex minus-.1ex} % Adjust the value (e.g., -1ex) to change the spacing
}{
  \endoldthebibliography
}

\makeatother

% \algnewcommand{\algorithmicforeach}{\textbf{for}}
% \algdef{SE}[FOR]{ForEach}{EndForEach}[1]
%   {\algorithmicforeach\ #1\ \algorithmicdo}% \ForEach{#1}
%   {\algorithmicend\ \algorithmicforeach}% \EndForEach

\usepackage[colorlinks=true, citecolor=cyan]{hyperref}  
\setlength{\parindent}{0.5cm}
\def\BibTeX{{\rm B\kern-.05em{\sc i\kern-.025em b}\kern-.08em
    T\kern-.1667em\lower.7ex\hbox{E}\kern-.125emX}}


\lstset{
    language=Lisp,
    frame=single,
    breaklines=true,
    basicstyle=\scriptsize\ttfamily,
    moredelim=**[is][\color{red}]{@}{@},
}

% \title{\LARGE \bf
% Planology: Anticipating household sequences and planning for human-robot collaboration
% \thanks{*Denotes equal contribution}
% }

%\title{\LARGE \bf
%Validating LLM Based Open Set Task Decomposition with Knowledge Graphs and Human Intervened Validation for Task Accomplishment\thanks{*Denotes equal contribution}}

% LLM-based generic task decomposition, KG-based domain-specific task decomposition, HITL

\title{\LARGE \bf
AdaptBot: Combining LLM with Knowledge Graphs and Human Input for Generic-to-Specific Task Decomposition and Knowledge Refinement}


\author{ Shivam Singh$^{1*}$, Karthik Swaminathan$^{1*}$, Nabanita Dash$^1$, Ramandeep Singh$^1$ \\  Snehasis Banerjee$^2$,  Mohan Sridharan$^3$,  Madhava Krishna$^1$
\thanks{*Denotes equal contribution}

\thanks{$^{1}$ Robotics Research Center, IIIT Hyderabad, India}
\thanks{$^{2}$ TCS Research, Tata Consultancy Services, India}
\thanks{$^{3}$ School of Informatics, University of Edinburgh, UK}
}

\makeatletter
\renewcommand{\@seccntformat}[1]{%
  \protect\csname the#1\endcsname\protect\quad%
}
\makeatother

\renewcommand{\thesection}{\arabic{section}}
\renewcommand{\thesubsection}{\thesection.\arabic{subsection}}
\renewcommand{\thesubsubsection}{\thesubsection.\arabic{subsubsection}}

\begin{document}


\maketitle
\thispagestyle{empty}
\pagestyle{empty}


%%%%%%%%%%%%%%%%%%%%%%%%%%%%%%%%%%%%%%%%%%%%%%%%%%%%%%%%%%%%%%%%%%%%%%%%%%%%%%%%
\begin{abstract}
An embodied agent assisting humans is often asked to complete new tasks, and 
%An agent preparing a particular dish in the kitchen based on a known recipe may be asked to prepare a new dish or to perform cleaning tasks in the storeroom. 
there may not be sufficient time or labeled examples to train the agent to perform these new tasks. Large Language Models (LLMs) trained on considerable knowledge across many domains can be used to predict a sequence of abstract actions for completing such tasks, although the agent may not be able to execute this sequence due to task-, agent-, or domain-specific constraints. Our framework addresses these challenges by leveraging the generic predictions provided by LLM and the prior domain knowledge encoded in a Knowledge Graph (KG), enabling an agent to quickly adapt to new tasks. The robot also solicits and uses human input as needed to refine its existing knowledge. Based on experimental evaluation in the context of cooking and cleaning tasks in simulation domains, we demonstrate that the interplay between LLM, KG, and human input leads to substantial performance gains compared with just using the LLM. \\
Project website\footnote[4]{Project supported in part by TCS Research India}: \href{https://sssshivvvv.github.io/adaptbot/}{https://sssshivvvv.github.io/adaptbot/}
%\Madhav{Please add: Through a diversity of metrics we show substantial performance gain through the interplay between LLM, KG and Human feedback than a straightforward task execution based on abstract LLM outputs.}

%Embodied agents or assistive robots may often be challenged to accomplish a task that they have not encountered before or trained for. For example an assistive or a personal robot may be asked to cook a new recipe or perform a cleaning task with novel specifications. In such scenarios it can indeed be challenging and cumbersome to train the agent all over again for apart from the training process itself data generation offers significant difficulties. LLM have become popular in very recent times as a popular framework for novel task synthesis. When prompted adequately LLM can achieve novel task synthesis by decomposing the task into a sequence of sub tasks that need to be achieved in the specified sequence for successful completion. However LLM task sequencing abilities are statistical offering much room for improving its success rates. In this paper we propose a novel framework that effectively blends the statistical nature of LLM outputs with the deterministic nature of Knowledge Graph ontologies to achieve novel tasks by significantly improving the performance of LLM. Further by involving humans in a principled fashion %to validate LLM sequences that lie outside the domain of the KG we show  the framework increments and updates it Knowledge Graph showcasing knowledge refinement, expansion features that leads to further improvement across various performance metrics

\end{abstract}
\vspace{-1em}
\begin{keywords}
Large Language Models, Knowledge Graph, Human-in-the-loop Learning
\end{keywords}
% \textbf{Color Coding}: \\
% \crule[purple]{10pt}{10pt} : \raghav{Raghav} \\
% \crule[blue]{10pt}{10pt} : \karthik{Karthik}\\
% \crule[olive]{10pt}{10pt} : \shivam{Shivam}\\


The increasing reliance on LLMs for multimodal tasks across far-reaching sectors such as healthcare, finance, and manufacturing underscores the need to assess the accuracy and reliability of the information they generate. Vision-Language Models (VLM) have achieved state-of-the-art (SoTA) performance on Visual Question-Answering (VQA) benchmarks, and these models often utilize Retrieval-Augmented Generation (RAG) to maintain factual accuracy and relevance in a dynamic information environment. However, this has led to uncertainty in the information the LLM bases its answer on, as it may choose between parametric memory and retrieved sources. When models rely on memorized information instead of dynamically retrieving information, they may inadvertently propagate outdated or incorrect information, causing serious legal and ethical risks and undermining trust and reliability in AI systems \citep{huang2023survey}.
% The ability to strike a balance between generalization and specialization in AI systems is therefore crucial for ensuring the safe, reliable use of these technologies in real-world applications.

Despite these concerns, the way that Vision-Language models (VLMs) memorize and retrieve information, particularly in complex multimodal tasks, remains under-explored. Current research often focuses on either the general capabilities of large language models (LLMs) or the specialized retrieval mechanisms in retrieval augmented generation systems (RAG) \citep{incontext_rag,chen_murag_2022,liu_universal_2023}. Particularly in the context of multimodal retrieval and multihop reasoning, few studies analyze the tradeoff between finetuning for specialized tasks and zero-shot prompting for general-purpose vision-language capabilities. A lack of consensus on how to approach this tradeoff motivates the development of measures to quantify reliance on parametric memory, as well as metrics for quantifying the potential performance impact of extending LLMs with RAG systems.

To address this gap, we investigate how multimodal QA models balance accuracy with memorization on the WebQA benchmark. We compare finetuned multimodal systems against zero-shot VLMs, analyzing how retrieval performance influences QA accuracy. In particular, we focus on cases where retrieval fails, allowing us to measure reliance on parametric memory through two proposed metrics---the \ppr (\PPR) which quantifies how much model accuracy is influenced by retrieval quality, contrasting performance in best-case versus worst-case retrieval scenarios, and the \ucr (\UCR) which measures how often correct QA responses are generated when the retriever fails, providing a proxy for memorization.

To enable this analysis, we make several methodological contributions. For the finetuned QA models, we investigate Vision-Transformer (ViT) architectures, which allow for multihop reasoning over multiple sources. To investigate the impact of retrieval performance on trained LMs, we propose a variable-input Fusion-in-Decoder (FiD) model \cite{tanaka_slidevqa_2023, nlvr2}, building upon the VoLTA architecture \citep{pramanick_volta_2023}. For the zero-shot case, we build upon previous research on In-Context Retrieval \citep{incontext_rag} by demonstrating that LLMs such as GPT-4o are capable of performing the final ranking step of the retrieval process. In doing so, we find that GPT-4o, a general-purpose LLM, achieves SoTA performance on the WebQA task, outperforming existing finetuned RAG models by a significant margin (7\% higher accuracy). 

Crucially, our results reveal that while retrieval-augmented models reduce memorization, the training paradigm plays an important role. Finetuned models exhibit higher reliance on parametric memory, whereas zero-shot RAG approaches have lower memorization scores at the cost of accuracy. This suggests that while retrieval modules may mitigate the risks associated with outdated or incorrect information, SoTA performance requires that they be coupled with specialized QA models. Our memorization measures contribute to the development of transparent and reliable AI systems, particularly in applications where the sourcing of up-to-date, factual information is critical.



% We investigate the impact of question complexity on the ability of these models to integrate multiple data sources—such as images, text, and external retrievers—and produce coherent and accurate answers. We also explore whether in-context retrieval can be a viable alternative to traditional retrieval-augmented systems, offering a more streamlined approach to multimodal QA.

% To achieve this, we first compare zero-shot prompting multimodal LLMs with finetuned multimodal systems. We evaluate both types of models on the WebQA benchmark, a dataset designed for complex question answering that requires reasoning across both image and text sources. For the finetuned models, we use a Fusion-in-Decoder (FiD) architecture, which allows for multihop reasoning over multiple sources. Additionally, we introduce the concept of In-Context Retrieval Language Modeling (RLM), where the LLM itself performs retrieval tasks without the need for external retrievers. This method builds upon existing research in in-context learning  and aims to explore the viability of LLMs retrieving relevant sources and generating accurate answers directly from their context window.

% In order to investigate source utilization in finetuned multimodal models and LLMs, three lines of inquiry are established; 
% \begin{itemize}
%     \item Study 1: retrieval vs QA performance on webQA (motivating example, does QA answer correctly even with incorrect sources?)
%     \item Study 2: performance on adversarial examples where parametric knowledge would be incorrect by design
%     \item Study 3: improving performance on adversarial examples by fine-tuning (i.e model robustness)
% \end{itemize}

% Note, there is one weakness in this plan which is tying in the work we've already done. 
% If we added something from adversarial generation to the retrieval experiment (like a combination of study 1 + 3) it would be complete. So for instance we could try fine-tuning the retriever with adversarial examples (and not just the QA model)

% \begin{figure}
%     \centering
%     \includegraphics[width=0.95\linewidth]{figures/segmentation/webqa_segment_infill.png}
%     \caption{Example of the segmentation substitution pipeline from the WebQA task.}
%     % d5c76d760dba11ecb1e81171463288e9
%     \label{fig:seg_sub_pipeline}
% \end{figure}



% Retrieval augmented generation (RAG) with zero-shot prompting and fine-tuning Large Language Models (LLMs) have become the go-to methods for tasks relying on information retrieval and text generation. In many cases the LLMs parametric memory can sufficiently generalize to answer questions without being provided with retrieval mechanisms for out-of-domain knowledge. However, LLMs often hallucinate and provide wrong information in certain scenarios. This problem is amplified even further on open-domain Question Answering (QA) tasks involving multiple modalities. Grounded text generation using retrieved sources \citep{lewis2021retrievalaugmented} has been extensively studied for text-to-text QA tasks, but its application in multimodal settings has not been studied as much.


% Multimodal reasoning and question answering have gained prominence in recent research endeavors, with an increasing emphasis on handling various forms of data, particularly text and images. In this study, we address a specific gap in the existing literature by focusing on the development of a versatile multihop model capable of accommodating varying numbers of input images.

% Our motivation for this research lies in the growing complexity of answering questions using information on the web, where the challenge of navigating the open-domain setting is further complicated by the presence of multiple modalities and sometimes requires reasoning over multiple sources. WebQA is an ideal dataset on which to compare performance of finetuned RAG systems against general purpose LLMs; it is multimodal, with correct answers requiring reasoning over image and text sources. It is multihop, requiring a complex reasoning process over multiple sources. Finally, WebQA questions from different categories can be broken down into subdomains to analyze performance over domains of varying cardinality.

% Motivated by the real-world challenges of building retrieval and question answering (QA) systems, we design and finetune a closed domain, multimodal, multihop QA model, that is capable of reasoning over a varying number of sources taken as input from an external retriever module. This research contributes to the relatively underexplored domain of multihop reasoning across various input sources and modalities. Our goal is to explore the challenges posed by these scenarios and develop strategies that enable QA models to retrieve relevant information, conduct logical or numerical reasoning across diverse modalities, and generate coherent responses in natural language. To our knowledge, this is the first application of the Fusion-in-Decoder (FiD) architecture \cite{tanaka_slidevqa_2023, nlvr2} that is shown to work with a variable number of inputs, enabling multi-hop reasoning over sources.

% In-Context Learning refers to the ability of LLMs to perform any task by simply providing examples in the input prompt \citep{dong2022survey,min2022rethinking}. Inspired by this research, we propose a method to use the LLM itself as a multimodal retriever, potentially eschewing the requirement of a distinct retrieval module, thereby allowing the design of simpler retrieval-augmented QA systems. We dub this method In-Context Retrieval Language Modeling (RLM). To the best of the authors knowledge, In-Content RLM is disparate from other retrieval augmented approaches which utilize external retrieval modules \citep{incontext_rag,chen_murag_2022,liu_universal_2023}. Despite being a natural extension of In-Context learning, In-Context RLM has not yet been studied empirically.

% To expand on our contribution of In-Context Retrieval, this stems from the well-researched in-context learning of LLMs. In-context learning is the ability of a model to perform any task given a sufficient context window \citep{dong2022survey,min2022rethinking}. Such tasks could include retrieval and ranking, but typically, the go-to solution for tasks requiring retrieval has been RAG. To the best of the authors knowledge, In-Context Retrieval is distinct from In-Context Retrieval Augmented Language Modelling (RALM), and despite being a natural extension of In-Context learning, In-Context Retrieval has not yet been shown empirically.

% Finally, we explore the tradeoff between using zero-shot prompting LLMs and the fine-tuning approach. While we find that, overall, GPT-4o obtains SoTA performance on the WebQA task, outperforming the accuracy of existing finetuned RAG approaches by 7\%, finetuned approaches still perform better on more restricted subdomains\footnote{``In-Context RLM" @ \url{https://eval.ai/web/challenges/challenge-page/1255/leaderboard/3168}}. Finally, we validate that GPT-4o is relying on retrieval abilities to solve the task; we find that GPT-4o is capable of retrieving relevant sources in the presence of distractors and furthermore, when GPT-4o fails to retrieve correct sources, it answers incorrectly 75\% of the time, meaning that it is not relying on parametric memory for this task.

% \paragraph{Contributions}
% Based on our experimentation and analysis on the WebQA benchmark, we make the following contributions:
% \begin{itemize}
%     \item Propose a new architecture for multimodal multihop QA that takes variable number of input sources inspired by the Fusion-in-Decoder method.
%     \item Comparison of general purpose LLMs vs specialized models on the WebQA benchmark.
%     \item Observation of In-Context Multimodal Retrieval abilities of GPT-4o and that it does not rely on parametric memory for multimodal QA.
%     \item Analysis of relationship between retrieval and QA task performance.
%     \item Analysis of task and query complexity on the performance of retrieval and QA tasks.
% \end{itemize}
















% Throughout this paper, we will present our methodology, experiments, and findings, emphasizing our approach to multihop reasoning over varying numbers of input images. We believe that our work contributes to a deeper understanding of multimodal reasoning and has the potential to enhance the capabilities of question-answering systems in the intricate, multimodal landscape of web-based information.
%\section{INTRODUCTION}\label{Intro}

\section{Rethinking Sparse Attention Methods}
\label{sec:critique}

Modern sparse attention methods have made significant strides in reducing the theoretical computational complexity of transformer models. However, most approaches predominantly apply sparsity during inference while retaining a pretrained Full Attention backbone, potentially introducing architectural bias that limits their ability to fully exploit sparse attention's advantages. Before introducing our native sparse architecture, we systematically analyze these limitations through two critical lenses.


\begin{figure*}[t] 
\centering 
\includegraphics[width=1\textwidth]{figures/fig2.pdf} 
\caption{Overview of \method{}'s architecture. Left: The framework processes input sequences through three parallel attention branches: For a given query, preceding keys and values are processed into compressed attention for coarse-grained patterns, selected attention for important token blocks, and sliding attention for local context. Right: Visualization of different attention patterns produced by each branch. Green areas indicate regions where attention scores need to be computed, while white areas represent regions that can be skipped.}
\label{fig:framework}
\end{figure*}


\subsection{The Illusion of Efficient Inference}

Despite achieving sparsity in attention computation, many methods fail to achieve corresponding reductions in inference latency, primarily due to two challenges:

\textbf{Phase-Restricted Sparsity.}
Methods such as H2O \citep{h2o} apply sparsity during autoregressive decoding while requiring computationally intensive pre-processing (e.g. attention map calculation, index building) during prefilling. In contrast, approaches like MInference \citep{minference} focus solely on prefilling sparsity. 
These methods fail to achieve acceleration across all inference stages, as at least one phase remains computational costs comparable to Full Attention.
The phase specialization reduces the speedup ability of these methods in prefilling-dominated workloads like book summarization and code completion, or decoding-dominated workloads like long chain-of-thought~\citep{cot} reasoning.

\textbf{Incompatibility with Advanced Attention Architecture.}
Some sparse attention methods fail to adapt to modern decoding efficient architectures like Mulitiple-Query Attention~(MQA) \citep{mqa} and Grouped-Query Attention~(GQA) \citep{gqa}, which significantly reduced the memory access bottleneck during decoding by sharing KV across multiple query heads. For instance, in approaches like Quest \citep{quest}, each attention head independently selects its KV-cache subset. Although it demonstrates consistent computation sparsity and memory access sparsity in Multi-Head Attention (MHA) models, it presents a different scenario in models based on architectures like GQA, where the memory access volume of KV-cache corresponds to the union of selections from all query heads within the same GQA group. This architectural characteristic means that while these methods can reduce computation operations, the required KV-cache memory access remains relatively high.
This limitation forces a critical choice: while some sparse attention methods reduce computation, their scattered memory access pattern conflicts with efficient memory access design from advanced architectures.

These limitations arise because many existing sparse attention methods focus on KV-cache reduction or theoretical computation reduction, but struggle to achieve significant latency reduction in advanced frameworks or backends.
This motivates us to develop algorithms that combine both advanced architectural and hardware-efficient implementation to fully leverage sparsity for improving model efficiency.


\subsection{The Myth of Trainable Sparsity}
Our pursuit of native trainable sparse attention is motivated by two key insights from analyzing inference-only approaches:
(1) \textbf{\textit{Performance Degradation}}: Applying sparsity post-hoc forces models to deviate from their pretrained optimization trajectory. As demonstrated by \citet{magicpig}, top 20\% attention can only cover 70\% of the total attention scores, rendering structures like retrieval heads in pretrained models vulnerable to pruning during inference.
(2)~\textbf{\textit{Training Efficiency Demands}}: 
Efficient handling of  long-sequence training is crucial for modern LLM development. This includes both pretraining on longer documents to enhance model capacity, and subsequent adaptation phases such as long-context fine-tuning and reinforcement learning. However, existing sparse attention methods primarily target inference, leaving the computational challenges in training largely unaddressed. This limitation hinders the development of more capable long-context models through efficient training. Additionally, efforts to adapt existing sparse attention for training also expose challenges:



\textbf{Non-Trainable Components.} Discrete operations in methods like ClusterKV~\citep{clusterkv} 
(includes k-means clustering) and MagicPIG~\citep{magicpig} (includes SimHash-based selecting) create discontinuities in the computational graph. These non-trainable components prevent gradient flow through the token selection process, limiting the model's ability to learn optimal sparse patterns. 

\textbf{Inefficient Back-propagation.} Some theoretically trainable sparse attention methods suffer from practical training inefficiencies. Token-granular selection strategy used in approaches like HashAttention~\citep{desai2024hashattention} leads to the need to load a large number of individual tokens from the KV cache during attention computation. 
This non-contiguous memory access prevents efficient adaptation of fast attention techniques like FlashAttention, which rely on contiguous memory access and blockwise computation to achieve high throughput.
As a result, implementations are forced to fall back to low hardware utilization, significantly degrading training efficiency.



\subsection{Native Sparsity as an Imperative}

These limitations in inference efficiency and training viability motivate our fundamental redesign of sparse attention mechanisms.
We propose \method{}, a natively sparse attention framework that addresses both computational efficiency and training requirements.
In the following sections, we detail the algorithmic design and operator implementation of \method{}.

% \nabanita{Resource Description Framework (RDF)}

% The Resource Description Framework (RDF) is a flexible framework for representing and sharing structured data on the web. It expresses relationships between entities, making it valuable for describing and exchanging metadata across systems. Its standardized approach enhances data integration and interoperability.

% \nabanita{Set-Theoretic Representation of RDF}

% In set-theoretic terms, an RDF triple can be expressed as a binary relation from a set of elements of \( S \) to a set of elements of \( O \). RDF can be expressed in triple format:
% \[
% T = (S, P, O)
% \]
% where \( T \) (the RDF triple) comprises:
% \begin{itemize}
%     \item \( S \) (subject): the set of entities or resources being referred to.
%     \item \( P \) (predicate): the binary relation or property connecting the set of entities \( S \) and their corresponding values \( O \).
%     \item \( O \) (object): the set of values of the referred entities, or another entity being linked to.
% \end{itemize}

% Formally, we can define the binary relation between \( S \) and \( O \) over \( P \) as:
% \[
% P(S) = O
% \]

% For example, let \( P \) represent a function detecting whether an entity is "boilable" or "isboiled," \( S \) represent certain food items such as milk, apple, cucumber, and egg, and \( O \) represent the value (true or false). We can express this as:

% \begin{align*}
% \text{boilable}(\text{milk}) &= \text{true} \\
% \text{boilable}(\text{apple}) &= \text{false} \\
% \text{boilable}(\text{cucumber}) &= \text{false} \\
% \text{boilable}(\text{egg}) &= \text{true}
% \end{align*}

% \begin{align*}
% \text{isboiled}(\text{milk}) &= \text{false} \\
% \text{isboiled}(\text{egg}) &= \text{true}
% \end{align*}

% In some other cases, \( O \) can represent another entity:
% \[
% \text{isOnTopOf}(\text{lemon}) = \text{cutting board}
% \]
% \[
% \text{contentsIn}(\text{pan}) = \text{tea}
% \]

% \nabanita{Knowledge Expansion (KE) in RDF Schema}

% In RDF Schema, Knowledge Expansion (KE) refers to extending relationships between classes, their instances, and properties across different actions or attributes. RDF Schema models these hierarchical relationships using constructs like \texttt{subClassOf} and \texttt{subPropertyOf}, allowing knowledge to be reused, inferred, and extended from broader categories to more specific ones.

% \nabanita{Class and Property Relationships}

% The triple relations can also be extended to class relationships, such as:
% \[
% \text{classOf}(\text{Apple}) = \text{Fruit}
% \]
% \[
% \text{classOf}(\text{Granny Smith apple}) = \text{Apple}
% \]

% Hence:
% \begin{enumerate}
%     \item Any instance of the class \( \text{Apple} \) is also an instance of the class \( \text{Fruit} \):
%     \[
%     \forall x \, (x \in \text{Granny Smith Apple} \Rightarrow x \in \text{Fruit})
%     \]
    
%     \item The properties of the class \( \text{Fruit} \) extend to the class \( \text{Granny Smith Apple} \):
%     \[
%     \forall x, y \, (P(x, y) \Rightarrow Q(x, y))
%     \]
%     For instance, if \( \text{isSliceable}(\text{Fruit}) \) holds, then \( \text{isSliceable}(\text{Granny Smith Apple}) \) must also hold.

%     \item The subproperties of \( \text{cut} \) extend to more specific actions such as \( \text{slice}, \text{dice}, \text{chop} \):
%     \[
%     \text{subPropertyOf}(\text{cut}) = \text{slice}
%     \]
% \end{enumerate}

%%%%%%%%%%%%%%% NEED FIX 
%%%%%%%%%%%%%%%%%%%%%%%%%%%%%%%%%%%%%%%%%%%%%%%%%%%%%%%%%%%%%%%%%
%%%%%%%%%%%%%%%%%%%%%%%%%%%%%%%%%%%%%%%%%%%%%%%%%%%%%%%%%%%%%%%%%
\section{Problem Formulation and Framework}
\vspace{-0.5em}
\label{sec:framework}
Figure~\ref{fig:pipeline} is an outline of our framework. In the motivating example, an agent assisting in cooking tasks in a kitchen has access to relevant objects and ingredients for many dishes but it does not have the recipes. When asked to prepare any particular dish, $\tau_i$, the agent queries an LLM to obtain a sequence of abstract actions (sub-tasks), i.e., $\langle a_1, \ldots, a_{m_i}\rangle$. For example, the sequence for \textit{make an omelette} includes \textit{picking up the egg} and \textit{breaking the egg over a skillet}. This sequence of abstract actions is checked against a KG with some domain-specific information in the form of existing objects and attributes that include the actions that can be performed on some objects. The agent tries to resolve any discrepancy between the LLM output and KG, e.g., KG states there is no skillet or that an egg can only be cracked, by finding replacements, e.g., \textit{crack the egg over a pan}. If the discrepancy is not resolved, or if executing the action sequence does not provide the desired outcome, the agent identifies relevant actions and solicits human input to refine the KG, e.g., add knowledge of objects or their attributes, and provides an action sequence to complete the task. The agent is assumed to be able to execute these actions. We describe out framework's components below.


%Since the agent can be given a different dish with a different ingredient requirement, it decomposes the recipe making into actions relevant to the dish. Our framework in  leverages the generic knowledge and the stochastic nature of LLMs to decompose the task into an action sequence with limited prompting. The initial output from the LLMs goes through series of checks and refinement in the refinement block where the KG is queried for the property checks of the ingredients and so on. We give feedbacks in case of persistent errors and if not resolved, the human gives an input for correction and in some cases expands the knowledge which is discussed in section \ref{sec:kg_expansion}. We describe the components of our framework below: 
%%%%%%%%%%%%%%%%%%%%%%%%%%%%%%%%%%%%%%%%%%%%%%%%%%%%%%%%%%%%%%%%%
\subsection{Generic Task Decomposition with LLM}
\label{sec:framework-llm}
In our framework, we use an LLM to decompose any given task into a sequence of sub-tasks because LLMs have demonstrated the ability to provide such a sequence of abstract actions for many different tasks. Specifically, in the motivating example, the LLM is prompted with information about some domain objects, an example cooking task (make coffee), and the corresponding action sequence (recipe) to be executed---see Figure~\ref{fig:pipeline}a. We experimentally evaluate the use of different LLMs, as described in Section~\ref{sec:expres-setup}. 
%Our choice of using the LLMs to decompose the task into action sequence is motivated by two objectives: (i) With the generic knowledge and in addition to one-shot prompting technique, we aim to achieve an action output which are direct function calls during the execution; and (ii) To get variability in the output and with human-in-the-loop in play, a possibility of expanding the knowledge wherever necessary. As described in Section~\ref{sec:expt}, we explored the use of popular LLMs such as GPT-3.5\cite{brown2020language}, GPT-4o\cite{OpenAI2023GPT4TR}, Llama3-70b (cite), and Gemma2-9b (cite).

\vspace{-0.75em}
Since the sequence of sub-tasks predicted by the LLM is based on many information sources, it may not be possible to execute one or more of these actions. For example, in the context of cooking tasks, the suggested ingredient may not be available or the action may involve an incorrect choice of tool (e.g., using a fork to cut vegetables). These situations can be addressed in part by using prior domain-specific information, which is encoded as described below.

%%%%%%%%%%%%%%%%%%%%%%%%%%%%%%%%%%%%%%%%%%%%%%%%%%%%%%%%%%%%%%%%%
\subsection{Representing Domain-specific Knowledge with KG}
\label{sec:framework-kg}
Our framework uses a Knowledge Graph (KG) to encode any prior information available to the agent. In the context of cooking tasks, this includes knowledge of some classes of ingredients (e.g., herbs, fruits, vegetables), receptacles (e.g., plates, bowls, countertop), and tools (e.g. knives, spoons), which can be arranged hierarchically. It also encodes the existence of some specific instances of these object classes and their properties such as likely location(s) and the actions they can be involved in (e.g., cutting, scooping, grinding). We use the \textit{Resource Description Framework} (RDF) format to encode this information in two graph structures in Turtle format (.ttl file)---see Figure~\ref{fig:kg-nodes}:
%The environment we are working with is described using a structured graph-based representation. It consists of rooms (e.g., kitchen, etc.), various objects (e.g., apples, mangoes, etc.), receptacles (e.g., plates, bowls, stoves, etc.), and tools (e.g., knives, spoons, etc.). Each item and its properties are represented in a graph structure, with nodes denoting the items and edges denoting the properties or states of these items. Specifically, we use a  data model to represent both the environment's current state and its underlying knowledge. We define two distinct graphs:
\begin{enumerate}
\vspace{-0.75em}
\item \textbf{State graph:} models current state as $\mathbf{G_s} = (\mathbf{I_s}, \mathbf{E_s})$, where nodes $\mathbf{I_s}$ are instances of object classes such as ingredients and receptacles; and $\mathbf{E_s} \subseteq \mathbf{I_s} \times \mathbf{P_s} \times \mathbf{V_s}$ are edges such that $(i_j, p, v_k) \in \mathbf{E_s}$ is a triple denoting an attribute of $i_j \in \mathbf{I_s}$ in terms of value $v_k \in \mathbf{V_s}$ of predicate $p \in \mathbf{P_s}$. For example, (apple1, obj\_location, fridge) and (apple1, is\_sliced, true) express \textit{apple1's} location and that it is sliced.
% \begin{enumerate} 
%     \item $\mathbf{I_s}$ represents the set of nodes corresponding to the items in the environment (objects, receptacles, and tools).
%     \item $\mathbf{E_s} \subseteq \mathbf{I_s} \times \mathbf{P_s} \times \mathbf{V_s}$ represents the set of edges. Each edge $(i_j, p, v_k) \in \mathbf{E_s}$ is a triple that denotes an item $i_j \in \mathbf{I_s}$, a predicate $p \in \mathbf{P_s}$ (representing the state of the object), and a value $v_k \in \mathbf{V_s}$ that describes the state of the item. The value could either be a boolean or another item in the environment. For example, a triple in the state graph could be (apple, obj\_location, fridge) and (apple, is\_sliced, true), indicating that the apple is stored in the fridge and that its current state is \textit{sliced}.
% \end{enumerate} 
\item \textbf{Attribute graph:} encodes the known properties and action capabilities of some object classes as $\mathbf{G_k} = (\mathbf{I_k}, \mathbf{E_k})$, where nodes $\mathbf{I_k}$ represent the classes and edges $\mathbf{E_k} \subseteq \mathbf{I_k} \times \mathbf{P_k} \times \mathbf{V_k}$ represent class properties, e.g., (apple, sliceable, true) implies  apples can be sliced.
\vspace{-0.75em}
\end{enumerate}
%In both graphs, the sets of predicates $\mathbf{P_s}$ and $\mathbf{P_k}$ define possible states or relations, such as IsBoiled, Boilable, IsCleaned, NeedsToBeCleaned, etc. 
The available actions include moving, picking up, and putting down objects; using tools; cleaning, toggling, slicing, stirring, and mopping\footnote{Supplementary material includes list of all the actions.}. Such a KG can be learned automatically based on information extracted from datasets or sensor streams. The feasibility of any action/sub-task in the sequence predicted by LLM is then checked using $\mathbf{G_k}$ and $\mathbf{G_s}$ by generating suitable SPARQL queries. If the predicted sequence of actions passes the KG-based check, it is executed, changing $\mathbf{G_s}$ suitably.
%The agent's task is to achieve a goal by performing a sequence of actions on the items present in the environment. Each action modifies and updates the state graph $\mathbf{G_s}$, reflecting the changes in the environment. The agent is equipped with a predefined skill set $\mathbf{A} = \{a_1, a_2,...,a_n\}$, where each action $a_i$ operates on items in the state graph and updates their states. These actions include basic manipulations such as 

%Each action can be formalized as a function: $T_a: \mathbf{G_s} \times \mathbf{G_k} \Rightarrow \mathbf{G_s}^{'}$, where $T_a$ applies the action $a$ to the current state graph $\mathbf{G_s}$, and the knowledge graph $\mathbf{G_k}$ is used to verify if the action is applicable (e.g., whether an item is fryable or boilable). The result is a new state graph $\mathbf{G_s}^{'}$, which represents the updated states of the items in the environment.
%To determine the feasibility of an action, SPARQL queries are used. For each action $a \in \mathbf{A}$, a SPARQL query is generated to check the preconditions in both $\mathbf{G_s}$ and $\mathbf{G_k}$. For instance, The action \texttt{\small{slice(apple, knife, state\_file)}} refers to slicing an apple using a knife, with the state file used to extract the current state of all the items necessary to perform this action. To perform this action, the query ensures that the apple is in the correct location (e.g., on the slicing board) and is sliceable according to the knowledge graph $\mathbf{G_k}$. Upon successful verification, the action is performed, and the state graph $\mathbf{G_s}$ is updated with the new status (e.g., \texttt{\small{(apple, sliced, true))}}.
%The flexibility of RDF allows for easy modifications and updates to the environment model, supporting dynamic task execution and refinement. The combination of the state and knowledge graphs enables the agent to make informed decisions and adapt to changes in the environment.
\begin{figure}[tb]
\captionsetup{font=scriptsize}
\centering
    \begin{minipage}{0.3\textwidth}
        \begin{lstlisting}[basicstyle=\ttfamily\scriptsize]
ex:onion rdf:type ex:object ;
    ex:obj_name 'onion' ;
    ex:IsSliceable true ;
    ex:Fryable true ;
    ex:NeedsToBeCleaned true .
        \end{lstlisting}
        % \caption*{(a)}
    \end{minipage}
    % \setlength{\abovecaptionskip}{-2pt}
    % \setlength{\belowcaptionskip}{-12pt}
    \hfill
    % \caption{Example of a node \textit{onion} in $\mathbf{G_k}$.}
    % \label{fig:12}
% \end{figure}

% \begin{figure}[tb]
% \captionsetup{font=scriptsize}
% \centering
    \begin{minipage}{0.3\textwidth}
        \begin{lstlisting}[basicstyle=\ttfamily\scriptsize]
ex:onion rdf:type ex:object ;
    ex:obj_name 'onion' ;
    ex:obj_location ex:fridge .
    ex:sliced false ;
    ex:IsFried false ;
    ex:IsCleaned false .
        \end{lstlisting}
        % \caption*{(b)}
    \end{minipage}
    \setlength{\abovecaptionskip}{-2pt}
    \setlength{\belowcaptionskip}{-12pt}
    \caption{Example of a node \textit{onion} in $\mathbf{G_k}$ (\textit{top}) and $\mathbf{G_s}$ (\textit{bottom}).}
    \label{fig:kg-nodes}
    \vspace{-0.5em}
\end{figure}

% \begin{figure}[tb] 
% \captionsetup{font=scriptsize}
% \centering
%     \begin{minipage}{0.225\textwidth} 
%         \centering
%         \begin{lstlisting}[basicstyle=\ttfamily\scriptsize]
% ex:onion rdf:type ex:object ;
%     ex:obj_name 'onion' ;
%     ex:IsSliceable true ;
%     ex:Fryable true ;
%     ex:NeedsToBeCleaned true .
%         \end{lstlisting}
%         \caption*{(a)} % Sub-caption
%     \end{minipage}
%     \setlength{\abovecaptionskip}{-2pt}
%     \setlength{\belowcaptionskip}{-2pt}
%     \hfill 
%     \begin{minipage}{0.225\textwidth} 
%         \centering
%         \begin{lstlisting}[basicstyle=\ttfamily\scriptsize]
% ex:onion rdf:type ex:object ;
%     ex:obj_name 'onion' ;
%     ex:obj_location ex:fridge .
%     ex:sliced false ;
%     ex:IsFried false ;
%     ex:IsCleaned false .
%         \end{lstlisting}
%         \caption*{(b)} 
%     \end{minipage}
%     \setlength{\abovecaptionskip}{-2pt}
%     \setlength{\belowcaptionskip}{-2pt}
%     \caption{Example of a node \textit{onion} in both $\mathbf{G_k}$ and $\mathbf{G_s}$.}
%     \label{fig:kg-nodes}
%     \vspace{-0.5em}
% \end{figure}

%%%%%%%%%%%%%%%%%%%%%%%%%%%%%%%%%%%%%%%%%%%%%%%%%%%%%%%%%%%%%%%%%
\subsection{Refining LLM output}
\label{sec:framework-refine}
%\shivam{have to add why we preferred string matching over similarity detection}
If a mismatch is detected between the LLM output and the KG, the agent attempts to use the KG to \textit{revise} the action sequence---see Figure~\ref{fig:pipeline}(b). 
%The LLM output often contains inconsistencies in item names and action names, which can hinder the task execution. To address this, we use an Error Detection and Refinement block (Fig \ref{fig:pipeline}(b)), which leverages the knowledge graph (KG) to identify and correct errors in the LLM output. 
Specifically, the agent attempts to replace the text corresponding to the identified mismatch, which can refer to actions, object instances, or object attributes, with other text from the KG.
%This block focuses on resolving inconsistencies by checking the LLM-generated item and action names against the structured knowledge available in the KG.  For unknown items or actions, we employ methods like substring matching and similarity scoring to find close matches. However, since similarity detection performs better with sentences rather than individual words, it was less effective in our case. Substring matching, on the other hand, proved to be more suitable for replacing unknown items and actions with the correct ones. Every action execution involves querying the state graph. Initially, there is an initial state graph $\mathbf{G_s}$, and after an action is executed, it updates to a new state graph $\mathbf{G'_s}$. In an action sequence, the first action should query the initial state graph to extract the initial state of the environment, while subsequent actions should query the updated graph, as the states change with each action. 
%\nabanita{To rephrase words effectively, it's important to consider two key factors: syntactic similarity and semantic similarity. Syntactic similarity ensures that the sentence's structure stays intact, so the replacement fits grammatically. Semantic similarity, on the other hand, guarantees that the new word preserves the original meaning. Using hypernyms (broader terms) or hyponyms (more specific terms) can also help to fine-tune the level of detail or generality, ensuring the overall concept remains accurate. By balancing these elements, our system can swap words smoothly without distorting meaning or disrupting the task sequence's flow.} 
While performing such text replacement, it is important to consider syntactic similarity, which measures similarity in the structure (e.g., of words or sentences), and semantic similarity, which considers similarity in meaning.
In our framework, the agent can compute the similarity of the identified words (or their embedding) with words from a similar category (or their embedding) in the KG. The use of word embeddings requires additional contextual information and makes it difficult to understand the revision of the LLM output. We thus chose to use the direct matching of words while considering hypernyms (broader terms) or hyponyms (more specific terms) for simplicity, ease of use, and transparency. If the agent is able to replace all identified mismatches, it executes the actions. 
% The refinement block also takes care of wrong action - item associations; for e.g., it will change pick\_up\_obj(plate, onto\_file) to pick_up_rec(plate, onto\_file) and slice(mango, fork, onto\_file) to slice(mango, knife, onto\_file) by detecting that fork is not a SlicingTool and thus replacing it with the tools which have IsSlicingTool property enabled true etc. 
%Through our experiments, we have found that the error detection and refinement is highly effective, as it corrects the majority of the errors present in the LLM output, significantly improving the overall accuracy and consistency of the generated actions and items.


%%%%%%%%%%%%%%%%%%%%%%%%%%%%%%%%%%%%%%%%%%%%%%%%%%%%%%%%%%%%%%%%%
\subsection{Knowledge refinement with human input}
\label{sec:framework-hitl}
Since the KG is not comprehensive, the agent may not be able to resolve all identified mismatches, e.g., reference to unknown object or action. Also, there may be unexpected action outcomes when the agent executes the action sequence. These situations are handled through re-prompting and human feedback---see Figure~\ref{fig:pipeline}(c).
%The system initiates a feedback loop with the LLM whenever errors are encountered. Feedback is sent in two main scenarios:
Specifically, the agent responds to an unresolved mismatch or erroneous outcome by re-prompting the LLM with additional information (of mismatch or error). If the mismatch or error persists, human input is solicited and used. 

%\begin{enumerate}
%\item \textbf{Errors in the Error Detection and Refinement Block}:The Error Detection and Refinement block aims to correct inconsistencies in the LLM output using the knowledge graph. However, if the block encounters an inconsistency it cannot resolve (e.g., \textit{unknown item/action}), an error is raised and sent back to the LLM via a feedback prompt. The LLM generates an updated output in response to this feedback, and the refinement process is repeated. We put a threshold on the number of times we go back to the LLM. If the errors persist after the feedback threshold is reached, the system escalates the issue to a human for resolution (section~\ref{sec:framework}).
%\item \textbf{Errors During Action Execution:}
%If the refinement block does not raise any errors, meaning the LLM does not produce any actions that are unknown or use items that are unfamiliar to the agent, then the LLM output is sent for execution. During execution, errors may arise due to invalid actions (e.g., performing an incompatible action on an object). In such cases, the system sends feedback to the LLM, detailing the execution error. Similar to the refinement process, there is a limit to how many times the system can query the LLM for corrections. If all feedback attempts fail to resolve the error, the system proceeds with the output, and the human evaluates the final performance. Errors during execution are minimal or nonexistent in the LLM + KG + Human framework, as the refinement block and human resolves most issues beforehand.
%\end{enumerate}
%Note that, the feedback counter is tracked globally across both phases (refinement and execution). This means that if the agent exhausts the feedback limit during the refinement phase, it will no longer be able to send feedback to the LLM during the execution phase. In such cases, the system proceeds directly to execution and human evaluation.
% once the feedback limit has been reached, ensuring efficient use of feedback resources and avoiding redundancy. 
% \subsection{Human-in-the-loop \& knowledge expansion}
% \label{sec:kg_expansion}
%As discussed in the previous section, there are instances when the agent cannot resolve an unknown item/action error, even after exhausting all of its feedback attempts. In such cases, the agent raises a query to a human, seeking confirmation of the existence of the unknown item.
%Our framework incorporates the ability to expand knowledge dynamically through human intervention. When the agent encounters an unknown item or action command from LLM (i.e., item not present in the knowledge and state graph  or action not present in agent's skill set), it initiates a query to a human user.
%\begin{itemize}
\vspace{-0.75em}
\noindent
\textbf{Existence check:} if an action or object in the LLM output does not exist in the KG, there are three possibilities: (1) The agent is mistaking an existing item (action) for another item (action); (2) the entity does not exist in the domain; or (3) the entity exists but is not in the KG. In the first case, human informs the agent about the correct object (or action); in the second case, human denies existence of entity; and in the third case, human confirms the entity's existence and agent interactively obtains entity's attributes\footnote{Supplementary material includes details of questions asked.}. As the agent expands its knowledge, the need for human input progressively decreases.

\vspace{-0.75em}
\noindent
\textbf{Learn attributes:} If human confirms existence of an instance of a new entity, the agent interactively obtains additional details. For example, when informed about an instance of a new object class \textit{onion}, agent incrementally requests information about the object type (e.g., \textit{edible\_object}) and other relevant attributes (e.g., boilable, fryable, location of instance). This knowledge revision can be viewed as correcting (expanding) the knowledge in the KG by revising class attributes in $\mathbf{G_k}$ and instance-specific details in $\mathbf{G_s}$.
%\end{itemize}
%After learning the item's current state and its properties, the agent adds this information to state graph $\mathbf{G_s}$ and the knowledge graph $\mathbf{G_k}$ respectively. The current state of the item is added to $\mathbf{G_s}$ and the item properties gets added to $\mathbf{G_k}$. The item is represented as a new node $I_{new}$ and its states \& properties are added as edges in the respective graphs. The knowledge expansion can be seen as a function $f_{KE}$ :
$$f_{KE}(I_{new}, P_{new}, S_{current}) \Rightarrow {\mathbf{G_{k}^{'}}, \mathbf{G_{s}^{'}}}$$
where $I_{new}$ is the new entity; $P_{new}$ = ${(p_1, v_1), ..., (p_n, v_n)}$ refers to attributes ($p_i$) of entity and their values ($v_i$); $S_{current}$ = ${(s_1, v_1), ..., (s_n, s_n)}$ refers to states $s_i$ and their values $v_i$; and $\mathbf{G^{'}_{k}}$ and $\mathbf{G^{'}_{s}}$ are the updated components of the KG.  For example, new edge is added in $\mathbf{G_s}$ to encode an onion's position and new edge is added in $\mathbf{G_k}$ to encode that an onion can be fried. Note that this update to existing knowledge is fully transparent by design.

%This revision is performed through SPARQL queries. We demonstrate experimentally that this combination of LLM, KG, and HITL revision substantially improves accuracy of task completion, and supports incremental and transparent knowledge revision and adaptation to new classes of tasks. 

%This process is performed internally through SPARQL queries, ensuring that the graph remains consistent and up-to-date. For instance, after querying human about the current states and properties of \textit{onion} the knowledge graph $\mathbf{G_k}$ gets updated with the edges like \texttt{\small{(onion, Fryable, true)}} and state graph $\mathbf{G_s}$ gets updated with the edge \texttt{\small{(onion, obj\_location, fridge)}}. Figure \ref{fig:nodes} illustrates the nodes in $\mathbf{G_k}$ and $\mathbf{G_s}$ with their respective \textit{properties} and \textit{current states}.
    
%This ability to incorporate human knowledge in real-time allows for continuous expansion of the environment, ensuring that the agent can operate in dynamic and evolving settings. It also highlights the flexibility of the RDF model, where updates can be seamlessly integrated without disrupting the overall structure.

\begin{algorithm}[tb]
\caption{LLM + KG + Human Input}
\label{alg:algorithm}
\begin{algorithmic}[1]
\State \textbf{Procedure} LLM\_KG\_Human($\mathbf{G_{s}}$, $\mathbf{G_{k}}$, \textit{ip\_prompt})
    \State $F$ $\leftarrow 0$                     \algorithmiccomment{$F$ is feedback counter}
    \State $T$ $\leftarrow$ call\_LLM(\textit{ip\_prompt}) \algorithmiccomment{$T$ is action sequence}
    \State $T_{refined}$, $\varepsilon_{unkn}$ $\leftarrow$ refine\_sequence($T$, $\mathbf{G_{k}}$, $\mathbf{G_{s}}$)
    %\State \textit{[EDR is error detection and refinement block]}
    \If {NOT $\varepsilon_{unkn}$}                \algorithmiccomment{$\varepsilon_{unkn}$ is unknown\_item error}
        \State $O$, $\varepsilon_{exec}$ $\leftarrow$ execute($T_{refined}$) \algorithmiccomment{$O$ is execution output}
        \null\hfill\algorithmiccomment{$\varepsilon_{exec}$ is execution error}
    \EndIf

    \While {($\varepsilon_{exec}$ OR $\varepsilon_{unkn}$) AND $F$ $<$ $F_{max}$}
        
        \While {$\varepsilon_{unkn}$ AND $F$ $<$ $F_{max}$}
            \State $T^{'}$ $\leftarrow$ call\_LLM(fb\_prompt) \algorithmiccomment{$T^{'}$is updated sequence}
            \State $T^{'}_{refined}$, $\varepsilon_{unkn}$ $\leftarrow$ refine\_sequence($T^{'}$, $\mathbf{G_{k}}$, $\mathbf{G_{s}}$)
            \State $F$ $\leftarrow$ $F$ + 1
        \EndWhile

        \If {$\varepsilon_{unkn}$ AND $F$ == $F_{max}$}
            \State $T^{'}_{refined}$ $\leftarrow$ ask\_human($\varepsilon_{unkn}$) \algorithmiccomment{$\mathbf{G_{k}}, \mathbf{G_{s}} \Rightarrow \mathbf{G_{k}^{'}}, \mathbf{G_{s}^{'}}$}
            \State $O$, $\varepsilon_{exec}$ $\leftarrow$ execute($T^{'}_{refined}$)
            \State \textbf{break}
        \EndIf

        \If {$\varepsilon_{exec}$ AND $F$ $<$ $F_{max}$}
            \State $T^{'}$ $\leftarrow$ call\_LLM(fb\_prompt)
            \State $T^{'}_{refined}$, $\varepsilon_{unkn}$ $\leftarrow$ refine\_sequence($T^{'}$, $\mathbf{G_{k}}$, $\mathbf{G_{s}}$)
            \State $F$ $\leftarrow$ $F$ + 1
        \EndIf

        \If {NOT $\varepsilon_{unkn}$}
            \State $O$, $\varepsilon_{exec}$ $\leftarrow$ execute($T^{'}_{refined}$)
        \EndIf
    \EndWhile

    % \If {(NOT $\varepsilon_{exec}$) OR ($F$ == $F_{max}$)}
    %     \State send\_human\_for\_evaluation($O$)
    % \EndIf

\State \textbf{End Procedure}

\end{algorithmic}
\end{algorithm}

\vspace{-0.75em}
Algorithm~\ref{alg:algorithm} describes the flow of information and control in our framework. The framework takes as input the state graph $\mathbf{G_s}$ and attribute graph $\mathbf{G_k}$, along with an input prompt \textit{(ip\_prompt)} that contains information about the class of tasks, an in-context example, and a query specifying the task the agent must perform. The LLM generates an action sequence $T$ (Line 3), which is refined to $T_{refined}$ using the knowledge in the KG (Line 4). If there are no unresolved mismatches between KG and LLM output ($\varepsilon_{unkn}$), the action sequence is executed, with the outcomes and errors collected for further analysis (Lines 5-7). Any unresolved mismatches or errors in outcome result in a feedback prompt to the LLM, leading to a new predicted sequence of actions $T^{'}$ (Lines 9-13, 19-23). If these mismatches and/or errors persist (beyond threshold $F_{max}$), the agent queries a human, which potentially leads to knowledge refinement, updating $\mathbf{G_k}$ and $\mathbf{G_s}$. After the expansion, the knowledge base is updated, and the refined action sequence is executed and evaluated (Lines 14-18, 24-26). This entire process is repeated until the tasks is completed or some threshold (e.g., time limit) is exceeded.

\section{Accuracy in Distinguishing AI-generated Images from Real Photographs}\label{sec:results}

In the main phase of the experiment, we collected 539,749 responses on 599 images from 37,568 participants from February 5, 2024 to June 22, 2024. Sections~\ref{sec:acc-general} through \ref{sec:acc-model} focus on data from the main phase of the experiment. The second phase of the experiment started on June 22 and ended on August 30, with 83,577 responses on 482 images from 3,787 participants. Sections~\ref{sec:imagestimuli} and~\ref{sec:human-curation} describe the influence of human curation of the stimuli on how accurately participants identify the stimuli as AI-generated or real. 

The design of our experiment involves several important design choices. First, we selected the three models
of Midjourney, Firefly, and Stable Diffusion as the diffusion models. Second, we crafted prompts to produce realistic
outputs across various pose categories and content types. Third, we curated 450 images from over 3000 images generated
to use as image stimuli in the experiment. These images were selected to maximize realism while also representing
different visual artifacts and implausibilities. Inevitably, these design choices on models, prompts, and stimuli introduce some selection bias  into the experiment.

Additionally, we implemented two exclusion criteria that should be considered when interpreting our results. First, for all the analyses in Section ~\ref{sec:results}, we excluded observations where participants checked the box on the website ``I have seen this before''. These observations, which account for 2\% of the total observations, were excluded because of the strong possibility that participants who had previously seen the images were already aware of whether they were fake or real.
%the experiment aimed to evaluate whether participants could identify if an unfamiliar image was AI-generated.. 
For these observations marked as having been seen before, 38\% of these observations were on AI-generated stimuli and 62\% were on real images. The image most frequently reported as 'seen before' is a real portrait of Martin Luther King Jr, which was one of the few real images of a well-known celebrity included in the experiment. 

Second, in line with our goals of studying detection ability on images for which there was some ambiguity, we excluded all images where participants' accuracy suggested very little ambiguity. We operationalized this as accuracy above 90\%. 

These exclusion criteria remove all observations on 68 fake images and 4 real images, which represent 14\% of observations from the entire experiment. 

In the human-coded analysis of artifacts discussed in Section~\ref{sec:acc-presence-artifacts}, we apply an additional exclusion criterion to make the coding tractable. Specifically, we exclude all images accurately identified in more than 80\% of observations. This exclusion criterion focuses the analysis on the most challenging images by excluding the most egregious distortions that lead to low photorealism (i.e., high participant accuracy).
%and offers a lower bound on the full extent of the differences between image categories because .

\subsection{Overall Accuracy} \label{sec:acc-general}

In the main study, participants correctly identified AI-generated images and authentic photographs in 76\% and 74\% of observations, respectively. Accuracy varied substantially across images. Prior to implementing our accuracy-based exclusion described above, we found that for AI-generated images, accuracy ranged from 32\% to 99\%. Similarly, accuracy on real photographs ranged from 28\% to 92\%. Figure~\ref{fig:accuracy_real_fake} shows the distribution of accuracy in both AI-generated and real images with example images selected from the top, bottom, and middle deciles of each distribution. At the image level, the mean accuracy for identifying AI-generated and real images was 76\% (95\% CI:[74,77]) and 74\% (95\% CI:[72,76]), respectively. 

Despite our efforts to minimize obvious artifacts, some images - particularly non-portraits - were challenging to generate without noticeable artifacts. As a result, participants achieved nearly 100\% accuracy on a few AI-generated images with obvious features. We present examples of these images in Figure~\ref{fig:three-fake-images}.  %that further motivate the exclusion criteria that we apply to the rest of the results section. 
In contrast to AI-generated images, real photographs rarely contain definitive artifacts and visual cues often seen in AI-generated images, which limits participants from achieving near-perfect accuracy on real photographs.
\begin{figure}[H]
    \centering
    \includegraphics[width=\linewidth]{sections/images/general_accuracy.pdf}
    \caption{Distribution of accuracy scores for real and AI-generated images with example images representing different accuracy levels.}
    \label{fig:accuracy_real_fake}
    \Description{Histograms showing the distribution of accuracy scores for real and AI-generated images, accompanied by example images representing various accuracy levels.}
\end{figure}
\begin{figure}[H]
\centering
\captionsetup{justification=raggedright, singlelinecheck=false, skip=2pt, font=small}
\begin{subfigure}[t]{0.3\linewidth}
    \subcaption{}\vtop{\vskip0pt\hbox{\includegraphics[width=\linewidth]{sections/images/sd_portrait3_040.jpg}}}
\end{subfigure}
\hfill
\begin{subfigure}[t]{0.3\linewidth}
    \subcaption{}\vtop{\vskip0pt\hbox{\includegraphics[width=\linewidth]{sections/images/ff_pg3_010.jpeg}}}
\end{subfigure}
\hfill
\begin{subfigure}[t]{0.3\linewidth}
    \subcaption{}\vtop{\vskip0pt\hbox{\includegraphics[width=\linewidth]{sections/images/ff_fullbody3_007.jpeg}}}
\end{subfigure}
\caption{\mybold{Examples of obviously AI-generated images and their corresponding accuracy.} \normalfont{\textbf{A.} AI-generated portrait with 92\% accuracy. \textbf{B.} AI-generated posed group image with 95\% accuracy. \textbf{C.} AI-generated full-body image with 99\% accuracy.}}
\label{fig:three-fake-images}
\Description{Three examples of obviously AI-generated images with corresponding accuracy scores: A. Portrait with 92\% accuracy, B. Posed group image with 95\% accuracy, C. Full-body image with 99\% accuracy. }
\end{figure}

\subsection{Participant Level Accuracy}\label{sec:indiv-acc}

\begin{figure*}[h]
    \centering
    \captionsetup{justification=raggedright, singlelinecheck=false, skip=2pt, font=small}

    % Top Row - A and B
    \begin{subfigure}[t]{0.4\textwidth}  
        \centering
        \subcaption[]{}  
        \vspace{-3pt}  
        \includegraphics[width=\linewidth]{sections/images/scatter_plot_fake_images.jpg} 
        % \subcaption{}
    \end{subfigure}
    \hspace{0.05\textwidth} 
    \begin{subfigure}[t]{0.38\textwidth}  
        \centering
        \subcaption[]{}  
        \vspace{-3pt}
        \includegraphics[width=\linewidth]{sections/images/scatter_plot_real_images.jpg}
        % \subcaption{}
    \end{subfigure}

    % Bottom Row - C and D
    \begin{subfigure}[t]{0.36\textwidth}  
        \centering
        \subcaption[]{}  
        \vspace{-3pt}
        \includegraphics[width=\linewidth]{sections/images/accuracy_distribution.jpg}
        % \subcaption{}
    \end{subfigure}
    \hspace{0.05\textwidth}  
    \begin{subfigure}[t]{0.36\textwidth}  
        \centering
        \subcaption[]{}  
        \vspace{-3pt}
        \includegraphics[width=\linewidth]{sections/images/learning_curve_with_fake_real_bias.png}
        % \subcaption{}
    \end{subfigure}

    \caption{\textbf{Participant-level accuracy and learning trends.}  
    \normalfont{ \textbf{A.} Scatterplot of participant-level accuracy for AI-generated images. \textbf{B.} Scatterplot of participant-level accuracy for real images. \textbf{C.} Histogram showing the distribution of accuracy across the first ten images seen by participants who viewed at least 10 images. \textbf{D.} Learning curve illustrating accuracy trends and classification biases when detecting AI-generated and real images.}}
    
    \label{fig:combined-participant-accuracy}

    \Description{A composite figure showing participant accuracy trends:  
    A. Scatterplot displaying accuracy levels for detecting AI-generated images, with points representing individual participants.  
    B. Scatterplot for real images, structured similarly to A.  
    C. Histogram showing the distribution of accuracy for participants' first 10 images.  
    D. Learning curve tracking accuracy trends over time, highlighting biases in AI-generated and real image classification.}
\end{figure*}

Given the organic nature of participants' engagement with this experiment, we did not impose restrictions on the number of images a participant saw. Most participants in this study provided responses to at least seven images, but some participants only provided a single response, and one participant provided 502 responses. 

The vast majority of participants (75\%) saw 16 or fewer images. Figure~\ref{fig:combined-participant-accuracy}A and B present the distribution of participant--level accuracy by number of viewed images. 

In order to compare participant performance and avoid issues that arise with differential attrition, we focus on the first ten images seen by participants who saw at least 10 images, which includes 152,050 observations from 15,205 participants. First, we note that 34\% of these participants achieved 90\% accuracy or higher on the first ten images seen. If the AI-generated images were perfectly photorealistic such that the human ability to distinguish is no higher than random guessing, then we would have expected only 1\% of participants to achieve this threshold of accuracy (assuming random guessing at 50\% accuracy, with participants evaluating 10 images each, achieving at least 9 out of 10 correct responses would occur with a probability of approximately 1.07\%, based on the binomial probability distribution). Figure~\ref{fig:combined-participant-accuracy}C shows the distribution of accuracy across the first ten images seen by participants who saw at least 10 images.

In Figure~\ref{fig:combined-participant-accuracy}D, we present accuracy rates by the number of images seen. We find that on average, participants begin the experiment by disproportionally identifying images as fake in 63\% of observations. Notably, this bias is reduced after only a few images.


\subsection{Accuracy by Scene Complexity} \label{sec:acc-scene-complexity}

We find that on average, participants' accuracy increases as scene complexity increases. For example, we find that 16\% of portraits appear in the bottom decile of accuracy scores (representing the highest level of photorealism), whereas only 3\% of AI-generated posed group images appear in the bottom decile. Figure~\ref{fig:pose-complexity} presents the distribution of accuracy for each category, separately for real and AI-generated images. For AI-generated images, the mean accuracy was 72.7\% (95\% CI: [72.4, 72.9]) for portraits, 77.2\% (95\% CI: [76.8, 78.6]) for full body, 76.2\% (95\% CI: [75.8, 76.7]) for posed groups, and 73.4\% (95\% CI: [73, 73.8]) for candid groups. For real images, the accuracy was 71.1\% (95\% CI: [70, 71.4]) for portraits, 75.5\% (95\% CI: [75.1, 75.8]) for full body, 76.7\% (95\% CI: [76.3, 77]) for posed groups, and 74.8\% (95\% CI: [74.4, 75.1]) for candid groups. 

As exemplified in Figure~\ref{fig:pose-complexity}C, we note that portraits, relative to the other levels of scene complexity, typically have less detail, simpler and more standardized poses, more blurred backgrounds, and fewer available cues than full-body or group images. 
\begin{figure}[h]
    \captionsetup{justification=raggedright, singlelinecheck=false, skip=2pt, font=small}
    \centering

    % First subfigure - Reduce space
    \begin{subfigure}[t]{\linewidth}
        \subcaption{}
        \vspace{-12pt}  % Reduce vertical space
        \includegraphics[width=\linewidth]{sections/images/scene_complexity_filtered90_Real.png}
    \end{subfigure}
    
    % Second subfigure - Reduce space
    \begin{subfigure}[t]{\linewidth}
        \subcaption{}
        \vspace{-12pt}  % Reduce vertical space
        \includegraphics[width=\linewidth]{sections/images/scene_complexity_filtered90_AI-generated.png}
    \end{subfigure}
    
    % Third subfigure - Reduce space
    \begin{subfigure}[t]{\linewidth}
    \centering
        \subcaption{}
        \vspace{-12pt}  % Reduce vertical space
        \includegraphics[width=0.9\linewidth]{sections/images/scene_complexity_B.pdf}
    \end{subfigure}
    
    \caption{\textbf{Scene complexity}: Accuracy of real and AI-generated images by scene complexity levels. 
    \normalfont{Beeswarm plots of image-level accuracy for each dimension of scene complexity with bootstrapped 95\% confidence intervals. We exclude images identified with above 90\% accuracy in this analysis. \textbf{A.} Real images \textbf{B.} AI-generated images \textbf{C.} AI-generated images across scene complexities.}}
    
    \label{fig:pose-complexity}
    
    \Description{Beeswarm plots showing the accuracy of real and AI-generated images by pose complexity. Each dot represents an individual image, with error bars indicating the bootstrapped 95\% confidence interval around the mean.}
\end{figure}


\subsection{Accuracy by Presence of Artifacts}\label{sec:acc-presence-artifacts}
\begin{figure*}[h]
\centering
\captionsetup{justification=raggedright, singlelinecheck=false, skip=2pt, font=small}

% Full PDF Image
\includegraphics[width=\linewidth]{sections/images/combined_artifact_trends.pdf}

\caption{\textbf{Accuracy by artifact types and display times} \normalfont{\textbf{A. Mean accuracy for different artifact types.} Distribution of accuracy scores by artifact type
for images with at least one artifact. \textbf{B. Mean accuracy over display time.} Change in mean accuracy across different display time assignments (1 second, 5 seconds, 10 seconds, 20 seconds, and unlimited) with 95\% confidence intervals and bee swarm plots of image accuracy for AI-generated and real images. \textbf{C. Mean accuracy over time for different artifact types.} Change in mean accuracy across different time assignments (1 second, 5 seconds, 10 seconds, 20 seconds, and unlimited) with 95\% confidence intervals and bee swarm plots of image accuracy for
images with anatomical (pink), functional (green), and stylistic
(blue) artifacts. The x–axis shows the display time intervals, and the
y–axis shows accuracy.}}

\label{fig:combined-artifact-trends}

\Description{A composite figure showing accuracy-related analyses:  
(A) A beeswarm plot displaying the distribution of accuracy scores for images containing at least one artifact, categorized by artifact type.  
(B) A line plot illustrating mean accuracy across different display time conditions (1 second, 5 seconds, 10 seconds, 20 seconds, and unlimited), with 95\% confidence intervals. Overlaid bee swarm plots represent individual accuracy scores for AI-generated and real images.  
(C) A line plot showing mean accuracy over time for different artifact types (Anatomical, Functional, and Stylistic). Each artifact type is color-coded (pink for Anatomical, green for Functional, and blue for Stylistic). Bee swarm plots depict individual accuracy scores for images within each artifact category. The x-axis represents display time intervals, and the y-axis represents accuracy.}

\end{figure*}


In order to analyze accuracy by artifact type, we annotated images with diffusion model artifact categories from the taxonomy based on a three-step process. First, four co-authors independently annotated all 218 images with accuracy below 80\%, identifying artifacts and providing detailed explanations for their annotations. Second, each of these annotations was reviewed and edited by two additional co-authors. Third, a fifth co-author reviewed all annotations for consistency. Figures~\ref{fig:combined-varying-artifacts-visibility}A--C and \ref{fig:combined-varying-artifacts-visibility}D--F provide examples of how we annotated images, displaying the identified artifact categories, the reasoning behind their identification, and the associated detection accuracy for each image. During this process, we observed that the three main artifact types---anatomical implausibilities, stylistic artifacts, and functional artifacts---each appeared in nearly a third of the images we annotated. In contrast, violations of physics and sociocultural implausibilities were less common, appearing in only 20 and 12 images, respectively. In light of this distribution of artifacts, Figure~\ref{fig:combined-artifact-trends}A presents the distribution of accuracy scores across images containing at least the three listed artifact types.

Based on our annotations of artifacts in images, we find participants are less accurate on images with functional implausibilities than images with anatomical implausibilities or stylistic artifacts. The mean accuracy on images with at least one functional implausibility, one anatomical implausibility, and one stylistic artifact is 64.1\% (95\% CI: [63.8, 64.5]), 65\% (95\% CI: [64.6, 65.4]), and 64.9\% (95\% CI: [64.5, 65.3]), respectively. While the accuracy on images with functional implausibilities is lower than on images with other implausibilities and artifacts, the mean accuracy scores are similar. However, this similarity in means masks the differences in the distribution of accuracy scores, as shown in Figure~\ref{fig:combined-artifact-trends}A. We find that images with participant accuracy scores in the 40--60\% range (which represent images approaching indistinguishability between real and AI-generated) make up 32.8\% of images annotated with functional implausibilities compared to 21.4\% and 22.4\% of images annotated with anatomical implausibilities and stylistic artifacts, respectively. 


We find that images that we annotated as containing multiple artifacts can still appear photorealistic enough to make detection difficult for most people. Artifacts vary in levels of visibility, as shown in Figure~\ref{fig:combined-varying-artifacts-visibility}A--C. While Figure~\ref{fig:combined-varying-artifacts-visibility}A and C contain stylistic artifacts, they are far more apparent in Figure~\ref{fig:combined-varying-artifacts-visibility}B, which is reflected in its higher detection accuracy. Despite Figure~\ref{fig:combined-varying-artifacts-visibility}A and C containing multiple artifact categories, they had low detection accuracy, suggesting that the presence of multiple artifacts does not necessarily make images easier to identify and that artifact visibility is also a contributing factor. 


The visibility of artifacts is highly variable, and Figure~\ref{fig:combined-varying-artifacts-visibility}D--F present examples highlighting this variability. The anatomical implausibility in the fingers in image Figure~\ref{fig:combined-varying-artifacts-visibility}D is very noticeable, whereas the functional implausibilities in the tennis racket and shirt design of Figure~\ref{fig:combined-varying-artifacts-visibility}F are more subtle. The corresponding accuracy scores for these images--- 62\% for Figure~\ref{fig:combined-varying-artifacts-visibility}E and 54\% for Figure~\ref{fig:combined-varying-artifacts-visibility}F —reinforce the observation that anatomical artifacts tend to be more easily detected, while functional implausibilities often require closer attention and familiarity with depicted objects. The stylistic artifacts in the cinematization of Figure~\ref{fig:combined-varying-artifacts-visibility}E and plastic-like skin texture fall in between, further showing the spectrum of detectability across different artifact categories and visibility. 


\begin{figure*}[h!]
\centering
\captionsetup{justification=raggedright, singlelinecheck=false, skip=2pt, font=small}

% First Row of Images
\begin{subfigure}[t]{0.27\linewidth}
\centering
    \subcaption{}
    \includegraphics[width=\linewidth]{sections/images/8059c316907c586bdf33ad3cb9ca3f95.jpeg}
\end{subfigure}
\hspace{1cm}
\begin{subfigure}[t]{0.27\linewidth}
\centering
    \subcaption{}
    \includegraphics[width=\linewidth]{sections/images/616ba73f50088eb13244a807076248f7.jpeg}
\end{subfigure}
\hspace{1cm}
\begin{subfigure}[t]{0.27\linewidth}
\centering
    \subcaption{}
    \includegraphics[width=\linewidth]{sections/images/2c4c0b171577884f5c0991cacb5c5ebc.jpeg}
\end{subfigure}

\vskip 5mm % Adds vertical spacing between rows

% Second Row of Images
\begin{subfigure}[t]{0.27\linewidth}
\centering
    \subcaption{}
    \includegraphics[width=\linewidth]{sections/images/ff_pg3_009.jpeg}
\end{subfigure}
\hspace{1cm}
\begin{subfigure}[t]{0.27\linewidth}
\centering
    \subcaption{}
    \includegraphics[width=\linewidth]{sections/images/mj_portrait3_010.jpeg}
\end{subfigure}
\hspace{1cm}
\begin{subfigure}[t]{0.27\linewidth}
\centering
    \subcaption{}
    \includegraphics[width=\linewidth]{sections/images/ff_portrait3_004.jpeg}
\end{subfigure}

\caption{\textbf{Examples of images with varying artifact visibility.}  
\normalfont{\textbf{Top row (A--C):} Example images showcasing stylistic and functional artifacts with varying visibility.  
\textbf{A.} A subtle stylistic artifact in the soft and wispy textures of the woman's hair and a minor functional implausibility in the atypical design of her shirt collar (Accuracy: 47\%). \textbf{B.} An obvious stylistic artifact due to the overall cinematization of the image (Accuracy: 73\%). \textbf{C.} A combination of multiple artifacts, including anatomical implausibilities in the woman's hand, functional implausibilities in the table shape and wall panels, and a stylistic artifact in the soft texture of the woman's face (Accuracy: 38\%). \textbf{Bottom row (D--F):} Images with anatomical, stylistic, and functional artifacts of varying visibility. \textbf{D.} Anatomical implausibilities in the fingers of the three students (Accuracy: 84\%). \textbf{E.} A stylistic artifact in the cinematized look and plastic-like texture of the woman's skin (Accuracy: 62\%). \textbf{F.} No obvious anatomical or stylistic artifacts, but closer inspection reveals functional implausibilities: the tennis racket is asymmetrical, its strings are not taut, and the shirt has irregularly shaped designs with glitch-like inconsistencies (Accuracy: 54\%).}}

\label{fig:combined-varying-artifacts-visibility}

\Description{A composite figure showing six images with varying visibility of AI-generated artifacts.  
(A--C) The first row highlights stylistic and functional artifacts, including wispy hair, cinematized lighting, and a distorted table.  
(D--F) The second row focuses on anatomical, stylistic, and functional artifacts, including distorted fingers, plastic-like textures, and inconsistencies in objects like a tennis racket.}
\end{figure*}

\subsection{Accuracy by Randomized Display Time}\label{sec:acc-time}
\begin{figure*}[h]
\centering
\captionsetup{justification=raggedright, singlelinecheck=false, skip=2pt, font=small}
\begin{subfigure}[t]{0.27\linewidth}
\centering
    \subcaption{}
    \includegraphics[width=\linewidth]{sections/images/bbbfb2a12cd66783ce7e4015ec0084b9.jpg}
\end{subfigure}
\hspace{1cm}
\begin{subfigure}[t]{0.27\linewidth}
\centering
  \subcaption{}
    \includegraphics[width=\linewidth]{sections/images/5204545de13342cbefdc0e9022d821d2.jpg}
\end{subfigure}
\hspace{1cm}
\begin{subfigure}[t]{0.27\linewidth}
\centering
  \subcaption{}
    \includegraphics[width=\linewidth]{sections/images/ccc04b661d52c055a44fc01718c6a2bc.jpg}
\end{subfigure}
\caption{\textbf{Exemplar AI-generated images for which a closer look improves accuracy.} \normalfont{\textbf{A.} Accuracy: 38\% at 1 second display time to 65\% at 20 second display time. \textbf{B.} Accuracy: 44\% at 1 second display time to 82\% at 20 second display time. \textbf{C.} Accuracy: 27\% at 1 second display time to 70\% at 20 second display time.}}
\label{fig:displaytime}
\Description{Three images showing AI-generated images for which a closer look improves accuracy. (A) Image of woman generated by Stable Diffusion: Accuracy is 38\% at 1 second display time and it improves to 65\% at 20 second display time with  (B) Image of people generated by Stable Diffusion: Accuracy is 44\% at 1 second display time and it improves to 82\% at 20 second display time with (C) Image of people generated by Stable Diffusion: Accuracy is 27\% at 1 second display time and it improves to 70\% at 20 second display time with.}
\end{figure*}

By randomizing the display time of images in this experiment, our results support evaluating how viewing duration influences participants' accuracy. We find that longer viewing times improve performance. With just 1 second of display time, participants are  72\% accurate (95\% CI=[71.6, 72.5], 95\% CI=[71.3, 72.2]) on AI-generated and real images, respectively. With 5 seconds of display time, accuracy increases to 77\% (95\% CI=[77.0, 77.8], 95\% CI=[76.6, 77.4]) for both AI-generated and real images, respectively. While accuracy on real images appears to plateau by 5 seconds of display time, accuracy on AI-generated images increases up to 80\% (95\% CI=[79.6, 80.4]) at 10 seconds and 82\% (95\% CI=[81.2, 81.9]) at 20 seconds. Figure~\ref{fig:combined-artifact-trends}B presents the distribution of accuracy scores across display time conditions. Across the observations where display time was randomized, we find that the proportion of AI-generated images that are identified below random chance decreases from 43\% when participants only have 1 second to view the image to 30\%, 25\%, 17\%, and 17\% when participants have 5, 10, 20 seconds, and unlimited time to view the image.

In some images, AI artifacts can be noticed with a quick glance, but for others, careful attention to detail is necessary to spot the artifact. Figure~\ref{fig:displaytime} presents three images that require careful attention, as evidenced by the fact that most participants mark as real when they are limited to seeing the image for a second but
fake once they take into account the details of the scene.

Accuracy across all artifact types improved with increased display time. As shown in Figure~\ref{fig:combined-artifact-trends}C, participants showed higher accuracy when images were displayed for longer time (anatomical artifacts: 63\% at 5 seconds vs. 59\% at 1 second; stylistic artifacts: 63\% at 5 seconds vs. 60\% at 1 second; functional artifacts: 60\% at 5 seconds vs. 55\% at 1 second). For all artifacts, there is a significant improvement in detection accuracy when increasing display time from 5 seconds to unlimited. 

In Figure~\ref{fig:combined-artifact-trends}C, we observe that participants improved the most in identifying functional artifacts, with an 18\% improvement from 1 second to unlimited viewing time. In comparison, anatomical and stylistic artifacts showed smaller improvements of 11\% each over the same time interval. Unlike anatomical and stylistic implausibilities that can be identified at first glance, functional artifacts often require a closer look and familiarity with the elements in the image as they often appear in parts of the image that are not the main subject. 


\subsection{Qualitative Analysis of Participant Comments}\label{sec:qualitative-analysis}

We collected 34,675 comments from participants who filled out the optional text input box asking participants: ``If you think this is AI-generated, please explain why.'' In order to identify themes from these 34,675 comments, we prompted GPT-3.5 Turbo to identify 10 main themes across these comments. GPT-3.5 Turbo responded with the following ten themes, which we manually reviewed and refined to mitigate the ambiguities and generalization typical of large language models \cite{stephan2024rlvflearningverbalfeedback}: (1) Image quality focusing on the overall appearance, smoothness, and sometimes unrealistic perfection of image elements; (2) Facial and anatomical inconsistencies where participants pointed to irregularities in eyes, mouths, noses, skin texture, expressions, and general human anatomy; (3) Anatomical and functional anomalies such as deformities, misplaced body parts, and irregularities in objects or environments; (4) Lighting and environmental inconsistencies including unnatural lighting, inconsistent shadows, and reflections; (5) Digital manipulation indicators suggesting suspicions of AI-generation or digital alteration; (6) Biometric discrepancies particularly unnatural or imperfect body parts like hands and fingers; (7) Uncanny valley perceptions where images almost looked human but had subtle unnatural features that caused discomfort; (8) Contextual incongruities such as unrealistic scenarios and mismatched social elements; (9) Physical anomalies highlighting illogical physical interactions within the images; and (10) holistic authenticity assessment making overall judgments based on a combination of multiple cues and inconsistencies. Based on these ten main themes, we prompted GPT-3.5 to label each comment with one of the ten themes. Figure~\ref{fig:comments-all} illustrates examples of participant comments for four images and how they were categorized into themes. Figure~\ref{fig:themes} displays the distribution of themes across the comments and the related concept from our taxonomy in parentheses.

Based on GPT-3.5 Turbo, we find that 61\% of participants' comments mentioned relying on anatomical implausibilities. The next most common concept referred to is stylistic artifacts, which is mentioned in 30\% of comments. Participants mentioned functional implausibilities in 21\% of comments, violations of physics in 15\% of comments, and sociocultural implausibilities in only 4\% of comments. 

Based on the authors' annotations of artifacts, we find functional implausibilities to be the most prevalent, appearing in 58.7\% of images, followed by anatomical implausibilities in 51.4\% and stylistic artifacts in 39.0\% of images. We identify violations of physics and sociocultural implausibilities in only 9.17\% and 5.50\% of images, respectively. 
\begin{figure}[H]
\centering
\captionsetup{justification=raggedright, singlelinecheck=false, skip=2pt}
\begin{subfigure}[t]{0.9\linewidth}
    % \subcaption{}
    \includegraphics[width=\linewidth]{sections/images/theme_distribution_single_column.jpg}
\end{subfigure}
\caption{\mybold{Distribution of themes identified in participant comments.}}
\label{fig:themes}
\Description{A horizontal bar chart showing the distribution of themes identified in participant comments explaining their reasoning for AI image detection. The themes include Image Quality, Facial and Anatomical Inconsistencies, Anatomical and Functional Anomalies, Lighting and Environmental Inconsistencies, Digital Manipulation Indicators, Biometric Discrepancies, Uncanny Valley Perceptions, Contextual Incongruities, Physical Anomalies, and Holistic Authenticity Assessment.}
\end{figure}
While functional artifacts were the most prevalent in human researcher annotated images, they were less frequently mentioned in participant comments annotated by GPT--3.5. Conversely, anatomical artifacts were emphasized more in participant comments than in their prevalence in annotated images. 

\begin{figure}[htb]
\centering
\captionsetup{justification=raggedright, singlelinecheck=false, skip=2pt, font=small}
\begin{subfigure}[t]{0.22\textwidth}
    \subcaption{}\vtop{\vskip0pt\hbox{\includegraphics[width=\linewidth]{sections/images/mj_ng3_007.jpg}}}
\end{subfigure}
\hfill
\begin{subfigure}[t]{0.22\textwidth}
    \subcaption{}\vtop{\vskip0pt\hbox{\includegraphics[width=\linewidth]{sections/images/mj_portrait3_001.jpg}}}
\end{subfigure}
\hfill
\begin{subfigure}[t]{0.22\textwidth}
    \subcaption{}\vtop{\vskip0pt\hbox{\includegraphics[width=\linewidth]{sections/images/mj_fullbody3_003.jpg}}}
\end{subfigure}
% \hfill
% \begin{subfigure}[t]{0.19\textwidth}\subcaption{}\vtop{\vskip0pt\hbox{\includegraphics[width=\linewidth]{sections/images/fullbody3_023.jpeg}}}
% \end{subfigure}
\hfill
\begin{subfigure}[t]{0.22\textwidth}
    \subcaption{}\vtop{\vskip0pt\hbox{\includegraphics[width=\linewidth]{sections/images/mj_pg3_002.jpg}}}
\end{subfigure}
\caption{\mybold{Examples of participant comments mapped to themes.} \normalfont{\textbf{A.} ``Cosmetic style out of character with vintage setting": Contextual Incongruities. \textbf{B.} ``Skin too smooth, depth of field shallow.": Image Quality, Lighting Inconsistencies. \textbf{C.} ``If this is not AI then it is a staged photograph like a movie set because of the lighting and he is an actor.": Lighting inconsistencies, Contextual Incongruities.
\textbf{D.} ``Group looks pasted onto background.": Digital Manipulation Indicators.}}
\label{fig:comments-all}
\Description{Four images with participant comments mapped to themes.(A) AI-generated candid image with a comment on cosmetic style being out of character with a vintage setting.(B) AI image with smooth skin, with a comment on skin being too smooth and shallow depth of field.(C) AI-generated full body shot of a man, with a comment suggesting it resembles a staged photograph due to lighting.
(D) AI image of a group, with a comment on the group looking pasted onto the background.}
\end{figure}

\subsection{Accuracy by Models} \label{sec:acc-model}

In the process of generating the images for this experiment's stimuli set, we noticed that Midjourney, Firefly, and Stable Diffusion have different capabilities and limitations. For example, we noticed that Midjourney often produced images with persistent stylistic artifacts that were challenging to eliminate. Firefly, on the other hand, frequently exhibited a tendency toward synthetic emotional expressions, with subjects often appearing unnaturally and overly cheerful, necessitating multiple iterations to produce more realistic results. Stable Diffusion struggled significantly with generating group images, often introducing artifacts such as anatomical inconsistencies. In light of the limitations to generate non-portrait images with Stable Diffusion, 75\% of the Stable Diffusion-generated stimuli in this experiment were portraits. On the other hand, 30\% of Midjourney and Firefly-generated images in this experiment depict portraits. In order to compare the three models fairly, we focus our comparison on portrait images. Figure~\ref{fig:models} presents accuracy shown on portraits by each of the three models and reveals that participants' mean accuracy on Midjourney, Stable Diffusion, and Firefly were  76\% (95\% CI: [75.2, 75.8]), 74\% (95\% CI: [73.9, 74.8]), and 73\% (95\% CI: [72.7, 73.3]), respectively. 


\begin{figure}[H]
    \centering
    \includegraphics[width=0.8\linewidth]{sections/images/model_accuracy_combined.jpg} 
    \caption{\textbf{Accuracy across generative AI models} \normalfont{Each point represents an image. The black dots and error bars show the mean accuracy and 95\% bootstrapped confidence intervals for each model}}
    \label{fig:models}
    \Description{Bee swarm chart showing accuracy across different generative AI models. The chart compares the accuracy rates for identifying AI-generated content among various models along with bootstrapped 95\% confidence intervals}
\end{figure}


\subsection{Accuracy on Human Curated Images vs. Uncurated Images} \label{sec:human-curation}
\begin{figure*}[h]
\centering
\captionsetup{justification=raggedright, singlelinecheck=false, skip=2pt, font=small}
\begin{subfigure}[t]{0.24\linewidth}
    \subcaption{}\vtop{\vskip0pt\hbox{\includegraphics[width=\linewidth]{sections/images/sd_portrait3_003.jpg}}}
\end{subfigure}
\hfill
\begin{subfigure}[t]{0.24\linewidth}
    \subcaption{}\vtop{\vskip0pt\hbox{\includegraphics[width=\linewidth]{sections/images/0bce6c35c24ec5ce8ae8ad5bb4f67d59_r4.jpg}}}
\end{subfigure}
\hfill
\begin{subfigure}[t]{0.24\linewidth}
    \subcaption{}\vtop{\vskip0pt\hbox{\includegraphics[width=\linewidth]{sections/images/0bce6c35c24ec5ce8ae8ad5bb4f67d59_r11.jpg}}}
\end{subfigure}
\hfill
\begin{subfigure}[t]{0.24\linewidth}
    \subcaption{}\vtop{\vskip0pt\hbox{\includegraphics[width=\linewidth]{sections/images/0bce6c35c24ec5ce8ae8ad5bb4f67d59_r2.jpg}}}
\end{subfigure}
\caption{\mybold{Re-generated images from the same prompt.} \normalfont{ \textbf{A.} Stage 1 image generated by Stable Diffusion and curated by our team (37\% accuracy) \textbf{B.} Most photorealistic of 12 prompt-matched image generations by Stable Diffusion (42\% accuracy) \textbf{C.} Median photorealistic of 12 prompt-matched image generations by Stable Diffusion (59\% accuracy) \textbf{D.} Least photorealistic of 12 prompt-matched image generations by Stable Diffusion(83\% accuracy)}}
\label{fig:regeneration}
\Description{Four images showing re-generated outputs from the same prompt.\textbf{A.} Stage 1 image generated by Stable Diffusion and curated by our team (37\% accuracy) \textbf{B.} Most photorealistic of 12 prompt-matched image generations by Stable Diffusion (42\% accuracy) \textbf{C.} Median photorealistic of 12 prompt-matched image generations by Stable Diffusion (59\% accuracy) \textbf{D.} Least photorealistic of 12 prompt-matched image generations by Stable Diffusion(83\% accuracy)}
\end{figure*}

\begin{figure*}[h]
\centering
\captionsetup{justification=raggedright, singlelinecheck=false, skip=2pt}
\begin{subfigure}[t]{0.48\textwidth}  
\subcaption{}
\vtop{\vskip0pt\hbox{\includegraphics[width=\linewidth]{sections/images/curation_value_add_min.jpg}}}
\end{subfigure}
\hfill
\begin{subfigure}[t]{0.48\textwidth}
\subcaption{}
\vtop{\vskip0pt\hbox{\includegraphics[width=\linewidth]{sections/images/curation_value_add_mean.jpg}}}
\end{subfigure}
\caption{\mybold{Comparing accuracy scores on curated images and uncurated prompt-matched images.} \normalfont{\textbf{A.} Scatterplot showing human detection accuracy of the original curated image compared to human detection accuracy of its most photorealistic regeneration out of 11 to 24 prompt-matched images labeled as re-generations. \textbf{B.} Scatterplot showing human detection accuracy of the original curated image compared to human detection accuracy of its mean photorealistic regeneration out of 11 to 24 re-generations.}}
\label{fig:curation-value}
\Description{Two scatterplots showing curation value analysis.(A) Minimum curation value added.(B) Mean curation value added. Each chart illustrates the impact of curation on the overall value added to the dataset.}
\end{figure*}

Generating photorealistic AI-generated images involves three key ingredients: the diffusion model, the prompt, and human curation. In this section, we examine how human curation of diffusion model-generated images influences the aggregate accuracy scores of human participants. In order to show this influence, we compare diffusion model images from the main experiment, which were curated by our research team, with multiple diffusion model images generated from the same prompt as the curated images. This comparison reveals the increase in photorealism (as measured by the decrease in participants' accuracy) on the curated images relative to the prompt-matched images.

In this second phase of the experiment, we randomly sampled 39 AI-generated images from the main stimuli set, where the sample was stratified on 10 percentage point wide bins on human detection accuracy. For each of these 39 images, we generated at least 11 prompt-matched images using Midjourney, Firefly, and the same pipeline in Stable Diffusion. Figure~\ref{fig:regeneration} displays a Stable Diffusion-generated image from our original stimuli set and three of the twelve generations using the same prompt. We generated 482 total additional images, with at least 11 per prompt. These 482 images were included alongside the 149 real images on the experiment website.

In Figure~\ref{fig:curation-value}, we present scatterplots comparing human detection accuracy on the initial curated images and the best prompt-matched images in panel A, and mean prompt-matched images in panel B. We find the human-curated images have lower human detection accuracy than the best regenerated image in 18 of 39 instances and the mean re-generated image in 35 of 39 instances. In total, the human-curated images were perceived to be more photorealistic than 408 of the 482 (84\%) uncurated prompt-matched images. Specifically, we find the marginal value added by human curation for images that were initially detected in the range of 30\% to 50\% is 31 percentage points, 50 to 60\% is 23 percentage points, 60-70\% is 11 percentage points, 70-80\% is 8 percentage points, and 80+\% is 4 percentage points. Across the stimuli selected from Midjourney, Firefly, and Stable Diffusion, the marginal value of human curation is 7.8, 19.0, and 16.9 percentage points, respectively. 


The two panels in Figure~\ref{fig:curation-value} illustrate the positive correlation between accuracy on the human-curated image and accuracy on the regeneration. This reveals how the prompt influences photorealism. The Pearson Correlation Coefficient between accuracy on curated images and their best, mean, and worst re-generations are .58, .53, and .32, respectively. This positive correlation suggests the choice of a prompt plays a significant role in the photorealism of an image. Figure~\ref{fig:goodandbadprompt} displays two original curated images where A is generated by a prompt in which re-generations achieved low human detection accuracy (a `good' prompt), and B is generated by a prompt in which re-generations achieved a high human detection accuracy (a `bad' prompt). Prompts that consistently generate easily detectable images often have elements that are difficult to generate and result in artifacts. The prompt ``Persian woman astronaut in astronaut clothes, family photo with husband and two toddlers, high resolution, realistic" for Figure~\ref{fig:goodandbadprompt}B  generates a posed group image that tends to be easy to detect. On the other hand, the prompt ``American woman faculty portrait, not a close-up, blond" for Figure~\ref{fig:goodandbadprompt}A generates a portrait image that tends to be perceived as more photorealistic.
\section{Conclusions and Future Work}
\label{sec:conclusions}
%\nabanita{Nabanita: ADD Percentage increase and other comparisons as necessary!}
Embodied agents assisting humans frequently have to complete previously unseen tasks or operate in new scenario. This paper describes a framework that leverages the complementary strengths of Large Language Models (LLMs), Knowledge Graphs (KGs), and Human-in-the-Loop (HITL) feedback to satisfy this requirement. Specifically, the generic task decomposition ability of LLMs is used to predict a sequence of abstract actions to complete any given task. This sequence is adapted to the specific scenario(s) and the task-, agent-, or domain-specific constraints using a KG that encodes prior knowledge of some objects, object attributes, and action capabilities. Any unresolved mismatch between the KG and the LLM output, and any unexpected action outcomes, are addressed by soliciting and using human input. This HITL feedback corrects errors and refines the existing knowledge (in the KG) for subsequent operation. Experimental evaluation in two simulated domains demonstrates substantial performance improvement compared with baselines, and illustrates incremental acquisition of knowledge to adapt to new classes of tasks.

%Our framework successfully merges the unique capabilities of Large Language Models (LLMs), Knowledge Graphs (KGs), and human feedback to improve how embodied agents respond to unfamiliar situations. LLMs provide a foundation by generating high-level action plans, which are then refined with detailed, domain-specific knowledge from KGs. This allows agents to adjust on the fly, even when encountering new objects or tasks. Additionally, incorporating human-in-the-loop (HITL) feedback offers real-time updates and fine-tuning, ensuring the agents continually evolve and expand their understanding. This strategy is particularly useful in everyday activities like cooking or cleaning, where agents can quickly understand and execute tasks without requiring extensive retraining. By combining LLMs, KGs, and human insights, the system ensures that agents remain adaptable and efficient, even in unpredictable environments.

% paves the way for exciting future developments, such as expanding the framework to handle a wider variety of tasks and environments, fine-tuning the balance between automation and human input, and exploring ways to make knowledge refinement more autonomous. Ultimately, our approach marks a significant step toward building smarter, more adaptive, and human-centered assistive agents.
\vspace{-0.75em}
This research opens up multiple avenues for further research. First, we will explore the use of this framework in many more classes of tasks, building on (and reinforcing) the promising results obtained so far.  Second, we will investigate the trade-off between automating the generation of an action sequence for any given task, and soliciting and incorporating human feedback as needed. Furthermore, we will explore the use of this framework on a physical robot platform assisting humans. The long-term objective is to create assistive agents and robots that can interact and collaborate with humans in different application domains.
\balance

\bibliographystyle{IEEEtran}
% \bibliography{IEEEabrv,references}
% Generated by IEEEtran.bst, version: 1.14 (2015/08/26)
\begin{thebibliography}{10}
\providecommand{\url}[1]{#1}
\csname url@samestyle\endcsname
\providecommand{\newblock}{\relax}
\providecommand{\bibinfo}[2]{#2}
\providecommand{\BIBentrySTDinterwordspacing}{\spaceskip=0pt\relax}
\providecommand{\BIBentryALTinterwordstretchfactor}{4}
\providecommand{\BIBentryALTinterwordspacing}{\spaceskip=\fontdimen2\font plus
\BIBentryALTinterwordstretchfactor\fontdimen3\font minus \fontdimen4\font\relax}
\providecommand{\BIBforeignlanguage}[2]{{%
\expandafter\ifx\csname l@#1\endcsname\relax
\typeout{** WARNING: IEEEtran.bst: No hyphenation pattern has been}%
\typeout{** loaded for the language `#1'. Using the pattern for}%
\typeout{** the default language instead.}%
\else
\language=\csname l@#1\endcsname
\fi
#2}}
\providecommand{\BIBdecl}{\relax}
\BIBdecl

\bibitem{coppeliaSim}
E.~Rohmer \emph{et~al.}, ``Coppeliasim (formerly v-rep): a versatile and scalable robot simulation framework,'' in \emph{Proc. of The International Conference on Intelligent Robots and Systems (IROS)}, 2013, www.coppeliarobotics.com.

\bibitem{Puig_2018_CVPR}
X.~Puig \emph{et~al.}, ``Virtualhome: Simulating household activities via programs,'' in \emph{Proceedings of the IEEE Conference on Computer Vision and Pattern Recognition (CVPR)}, June 2018.

\bibitem{kolve2022ai2thorinteractive3denvironment}
\BIBentryALTinterwordspacing
E.~Kolve \emph{et~al.}, ``Ai2-thor: An interactive 3d environment for visual ai,'' 2022. [Online]. Available: \url{https://arxiv.org/abs/1712.05474}
\BIBentrySTDinterwordspacing

\bibitem{khot2023decomposedpromptingmodularapproach}
\BIBentryALTinterwordspacing
T.~Khot \emph{et~al.}, ``Decomposed prompting: A modular approach for solving complex tasks,'' 2023. [Online]. Available: \url{https://arxiv.org/abs/2210.02406}
\BIBentrySTDinterwordspacing

\bibitem{reppert2023iterateddecompositionimprovingscience}
\BIBentryALTinterwordspacing
J.~Reppert \emph{et~al.}, ``Iterated decomposition: Improving science q\&a by supervising reasoning processes,'' 2023. [Online]. Available: \url{https://arxiv.org/abs/2301.01751}
\BIBentrySTDinterwordspacing

\bibitem{liu2024deltadecomposedefficientlongterm}
\BIBentryALTinterwordspacing
Y.~Liu \emph{et~al.}, ``Delta: Decomposed efficient long-term robot task planning using large language models,'' 2024. [Online]. Available: \url{https://arxiv.org/abs/2404.03275}
\BIBentrySTDinterwordspacing

\bibitem{sakib2022approximate}
M.~S. Sakib \emph{et~al.}, ``Approximate task tree retrieval in a knowledge network for robotic cooking,'' \emph{IEEE Robotics and Automation Letters}, vol.~7, no.~4, pp. 11\,492--11\,499, 2022.

\bibitem{sakib2024cooking}
M.~S. Sakib and Y.~Sun, ``From cooking recipes to robot task trees--improving planning correctness and task efficiency by leveraging llms with a knowledge network,'' in \emph{2024 IEEE International Conference on Robotics and Automation (ICRA)}.\hskip 1em plus 0.5em minus 0.4em\relax IEEE, 2024, pp. 12\,704--12\,711.

\bibitem{openai2024gpt4technicalreport}
\BIBentryALTinterwordspacing
OpenAI \emph{et~al.}, ``Gpt-4 technical report,'' 2024. [Online]. Available: \url{https://arxiv.org/abs/2303.08774}
\BIBentrySTDinterwordspacing

\bibitem{gemmateam2024gemma2improvingopen}
\BIBentryALTinterwordspacing
G.~Team \emph{et~al.}, ``Gemma 2: Improving open language models at a practical size,'' 2024. [Online]. Available: \url{https://arxiv.org/abs/2408.00118}
\BIBentrySTDinterwordspacing

\bibitem{dubey2024llama3herdmodels}
\BIBentryALTinterwordspacing
Dubey \emph{et~al.}, ``The llama 3 herd of models,'' 2024. [Online]. Available: \url{https://arxiv.org/abs/2407.21783}
\BIBentrySTDinterwordspacing

\bibitem{wen2024learning}
J.~Wen \emph{et~al.}, ``Learning task decomposition to assist humans in competitive programming,'' \emph{arXiv preprint arXiv:2406.04604}, 2024.

\bibitem{li2023semantically}
W.~Li \emph{et~al.}, ``Semantically aligned task decomposition in multi-agent reinforcement learning,'' \emph{arXiv preprint arXiv:2305.10865}, 2023.

\bibitem{dery2021auxiliarytaskupdatedecomposition}
\BIBentryALTinterwordspacing
L.~M. Dery \emph{et~al.}, ``Auxiliary task update decomposition: The good, the bad and the neutral,'' 2021. [Online]. Available: \url{https://arxiv.org/abs/2108.11346}
\BIBentrySTDinterwordspacing

\bibitem{shen2023taskbench}
Y.~Shen \emph{et~al.}, ``Taskbench: Benchmarking large language models for task automation,'' \emph{arXiv preprint arXiv:2311.18760}, 2023.

\bibitem{wang2024tdag}
Y.~Wang \emph{et~al.}, ``Tdag: A multi-agent framework based on dynamic task decomposition and agent generation,'' \emph{arXiv preprint arXiv:2402.10178}, 2024.

\bibitem{cui2021semantic}
G.~Cui, W.~Shuai, and X.~Chen, ``Semantic task planning for service robots in open worlds,'' \emph{Future Internet}, vol.~13, no.~2, p.~49, 2021.

\bibitem{prasad2023adapt}
A.~Prasad \emph{et~al.}, ``Adapt: As-needed decomposition and planning with language models,'' \emph{arXiv preprint arXiv:2311.05772}, 2023.

\bibitem{article1}
D.~Zheng \emph{et~al.}, ``A knowledge-based task planning approach for robot multi-task manipulation,'' \emph{Complex \& Intelligent Systems}, vol.~10, 07 2023.

\bibitem{10.1145/3297280.3297568}
\BIBentryALTinterwordspacing
A.~Kattepur and B.~P, ``Roboplanner: autonomous robotic action planning via knowledge graph queries,'' in \emph{Proceedings of the 34th ACM/SIGAPP Symposium on Applied Computing}, ser. SAC '19.\hskip 1em plus 0.5em minus 0.4em\relax New York, NY, USA: Association for Computing Machinery, 2019, p. 953–956. [Online]. Available: \url{https://doi.org/10.1145/3297280.3297568}
\BIBentrySTDinterwordspacing

\bibitem{9561782}
A.~Daruna \emph{et~al.}, ``Towards robust one-shot task execution using knowledge graph embeddings,'' in \emph{2021 IEEE International Conference on Robotics and Automation (ICRA)}, 2021, pp. 11\,118--11\,124.

\bibitem{aburasheed2024knowledgegraphscontextsources}
\BIBentryALTinterwordspacing
H.~Abu-Rasheed \emph{et~al.}, ``Knowledge graphs as context sources for llm-based explanations of learning recommendations,'' 2024. [Online]. Available: \url{https://arxiv.org/abs/2403.03008}
\BIBentrySTDinterwordspacing

\bibitem{pan2024unifying}
S.~Pan \emph{et~al.}, ``Unifying large language models and knowledge graphs: A roadmap,'' \emph{IEEE Transactions on Knowledge and Data Engineering}, 2024.

\bibitem{kuang2024openfmnavopensetzeroshotobject}
\BIBentryALTinterwordspacing
Y.~Kuang \emph{et~al.}, ``Openfmnav: Towards open-set zero-shot object navigation via vision-language foundation models,'' 2024. [Online]. Available: \url{https://arxiv.org/abs/2402.10670}
\BIBentrySTDinterwordspacing

\bibitem{paulius2016functional}
D.~Paulius \emph{et~al.}, ``Functional object-oriented network for manipulation learning,'' in \emph{2016 IEEE/RSJ International Conference on Intelligent Robots and Systems (IROS)}.\hskip 1em plus 0.5em minus 0.4em\relax IEEE, 2016, pp. 2655--2662.

\bibitem{ding2022robottaskplanningsituation}
\BIBentryALTinterwordspacing
Y.~Ding \emph{et~al.}, ``Robot task planning and situation handling in open worlds,'' 2022. [Online]. Available: \url{https://arxiv.org/abs/2210.01287}
\BIBentrySTDinterwordspacing

\bibitem{bhat2024groundingllmsrobottask}
\BIBentryALTinterwordspacing
V.~Bhat \emph{et~al.}, ``Grounding llms for robot task planning using closed-loop state feedback,'' 2024. [Online]. Available: \url{https://arxiv.org/abs/2402.08546}
\BIBentrySTDinterwordspacing

\bibitem{jiang2019task}
Y.-q. Jiang \emph{et~al.}, ``Task planning in robotics: an empirical comparison of pddl-and asp-based systems,'' \emph{Frontiers of Information Technology \& Electronic Engineering}, vol.~20, pp. 363--373, 2019.

\bibitem{marin2021recipe1m+}
J.~Mar{\i}n \emph{et~al.}, ``Recipe1m+: A dataset for learning cross-modal embeddings for cooking recipes and food images,'' \emph{IEEE Transactions on Pattern Analysis and Machine Intelligence}, vol.~43, no.~1, pp. 187--203, 2021.

\bibitem{marzari2021towards}
L.~Marzari \emph{et~al.}, ``Towards hierarchical task decomposition using deep reinforcement learning for pick and place subtasks,'' in \emph{2021 20th International Conference on Advanced Robotics (ICAR)}.\hskip 1em plus 0.5em minus 0.4em\relax IEEE, 2021, pp. 640--645.

\bibitem{zhen2023robottaskplanningbased}
\BIBentryALTinterwordspacing
Y.~Zhen \emph{et~al.}, ``Robot task planning based on large language model representing knowledge with directed graph structures,'' 2023. [Online]. Available: \url{https://arxiv.org/abs/2306.05171}
\BIBentrySTDinterwordspacing

\bibitem{shinn2024reflexion}
N.~Shinn \emph{et~al.}, ``Reflexion: Language agents with verbal reinforcement learning,'' \emph{Advances in Neural Information Processing Systems}, vol.~36, 2024.

\bibitem{holler2020hddl}
D.~H{\"o}ller \emph{et~al.}, ``Hddl: An extension to pddl for expressing hierarchical planning problems,'' in \emph{Proceedings of the AAAI conference on artificial intelligence}, vol.~34, no.~06, 2020, pp. 9883--9891.

\bibitem{DING2019105}
\BIBentryALTinterwordspacing
Y.~Ding \emph{et~al.}, ``Robotic task oriented knowledge graph for human-robot collaboration in disassembly,'' \emph{Procedia CIRP}, vol.~83, pp. 105--110, 2019, 11th CIRP Conference on Industrial Product-Service Systems. [Online]. Available: \url{https://www.sciencedirect.com/science/article/pii/S2212827119304263}
\BIBentrySTDinterwordspacing

\bibitem{article}
J.~Bai \emph{et~al.}, ``A dynamic knowledge graph approach to distributed self-driving laboratories,'' \emph{Nature Communications}, vol.~15, 01 2024.

\bibitem{kasaei2024vitalvisualteleoperationenhance}
\BIBentryALTinterwordspacing
H.~Kasaei and M.~Kasaei, ``Vital: Visual teleoperation to enhance robot learning through human-in-the-loop corrections,'' 2024. [Online]. Available: \url{https://arxiv.org/abs/2407.21244}
\BIBentrySTDinterwordspacing

\bibitem{liu2023robotlearningjobhumanintheloop}
\BIBentryALTinterwordspacing
H.~Liu, S.~Nasiriany, L.~Zhang, Z.~Bao, and Y.~Zhu, ``Robot learning on the job: Human-in-the-loop autonomy and learning during deployment,'' 2023. [Online]. Available: \url{https://arxiv.org/abs/2211.08416}
\BIBentrySTDinterwordspacing

\bibitem{9044335}
M.~Raessa \emph{et~al.}, ``Human-in-the-loop robotic manipulation planning for collaborative assembly,'' \emph{IEEE Transactions on Automation Science and Engineering}, vol.~17, no.~4, pp. 1800--1813, 2020.

\bibitem{emami2024human}
Y.~Emami, K.~Li, L.~Almeida, W.~Ni, and Z.~Han, ``Human-in-the-loop machine learning for safe and ethical autonomous vehicles: Principles, challenges, and opportunities,'' \emph{arXiv preprint arXiv:2408.12548}, 2024.

\bibitem{wu2022survey}
X.~Wu, L.~Xiao, Y.~Sun, J.~Zhang, T.~Ma, and L.~He, ``A survey of human-in-the-loop for machine learning,'' \emph{Future Generation Computer Systems}, vol. 135, pp. 364--381, 2022.

\end{thebibliography}
\end{document}


