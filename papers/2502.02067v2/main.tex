%%%%%%%%%%%%%%%%%%%%%%%%%%%%%%%%%%%%%%%%%%%%%%%%%%%%%%%%%%%%%%%%%%%%%%%%%%%%%%%%
%2345678901234567890123456789012345678901234567890123456789012345678901234567890
%        1         2         3         4         5         6         7         8

\documentclass[article, 10 pt, conference]{ieeeconf}  % Comment this line out if you need a4paper
\setlength{\parskip}{0pt}
\pdfminorversion=4
% \documentclass[a4paper, 10pt, conference]{ieeeconf}      % Use this line for a4 paper

\IEEEoverridecommandlockouts                              % This command is only needed if 
                                                          % you want to use the \thanks command

\overrideIEEEmargins  
\UseRawInputEncoding 
\let\labelindent\relax
\usepackage{enumitem}
\setlist{topsep=1pt, partopsep=0pt, parsep=0pt, itemsep=0pt}
\usepackage[
top    = 0.75in,
bottom = 0.75in,
left   = 0.75in,
right  = 0.75in]{geometry}
\usepackage{cite}
\usepackage{amsmath}
\usepackage[skip=10pt plus1pt, indent=35pt]{parskip}
\usepackage{placeins}
\usepackage{textcomp}
\usepackage[dvipsnames]{xcolor}
\definecolor{royalblue}{RGB}{65, 105, 225}
\definecolor{maroon}{RGB}{180, 0, 0}
\definecolor{DarkGreen}{RGB}{0, 100, 0}
\usepackage{booktabs}
\usepackage{multirow}
\usepackage{siunitx}
\usepackage{algorithm}
% \usepackage{algorithmic}
\usepackage{algpseudocode}
\usepackage{graphicx}
\usepackage[symbol]{footmisc}
\usepackage{adjustbox}
\usepackage{float}
% \usepackage{paralist}
\usepackage{svg}
\usepackage{array}
\usepackage[T1]{fontenc}
\usepackage{subcaption}
\usepackage{xspace}
\usepackage{scrextend}
% \usepackage[export]{adjustbox}
\usepackage{caption}
\usepackage{tabularx}
\usepackage{listings}
\usepackage{wrapfig}
\usepackage{calligra}
\usepackage{comment}
\usepackage{soul}
\usepackage{balance}
% \documentclass{article}
\usepackage{listings}
\usepackage{xcolor}
\usepackage{caption}
\newcolumntype{A}{ >{\centering\arraybackslash} m{4cm} }
\newcolumntype{B}{ >{\centering\arraybackslash} m{1cm} }
\newcolumntype{C}[1]{>{\centering\let\newline\\\arraybackslash\hspace{0pt}}m{#1}}
\newcommand{\todo}[1]{\textbf{\textcolor{red}{TODO: #1}}}
\newcommand{\Madhav}[1]{\textcolor{magenta}{Madhav: #1}}
\newcommand{\nabanita}[1]{\textcolor{violet}{#1}}
\newcommand{\karthik}[1]{\textcolor{blue}{#1}}
\newcommand{\shivam}[1]{\textcolor{olive}{#1}}
\newcommand{\ramandeep}[1]{\textcolor{orange}{#1}}
\renewcommand{\thefootnote}{\fnsymbol{footnote}}
\long\def\commentm#1{{\bf **Mohan: #1**}}

\makeatletter
\newcommand\footnoteref[1]{\protected@xdef\@thefnmark{\ref{#1}}\@footnotemark}
\makeatother


% \usepackage{flushend}
%\usepackage{epsfig} % for postscript graphics files
%\usepackage{mathptmx} % assumes new font selection scheme installed
%\usepackage{times} % assumes new font selection scheme installed
\usepackage{amsmath,lipsum} % assumes amsmath package installed
\usepackage{amssymb}  % assumes amsmath package installed
\usepackage{amsfonts}
\usepackage{textgreek}
\usepackage{authblk}
\usepackage{etoolbox}
\usepackage{algpseudocode}

\algrenewcommand\algorithmicindent{0.5em}
\newcommand\crule[3][black]{\textcolor{#1}{\rule{#2}{#3}}}
\DeclareMathOperator*{\argminA}{arg\,min} % Jan Hlavacek
\makeatletter
\let\NAT@parse\undefined

\let\oldthebibliography\thebibliography
\let\endoldthebibliography\endthebibliography
\renewenvironment{thebibliography}[1]{
  \oldthebibliography{#1}
  \setlength{\itemsep}{-1.75ex plus-.1ex minus-.1ex} % Adjust the value (e.g., -1ex) to change the spacing
}{
  \endoldthebibliography
}

\makeatother

% \algnewcommand{\algorithmicforeach}{\textbf{for}}
% \algdef{SE}[FOR]{ForEach}{EndForEach}[1]
%   {\algorithmicforeach\ #1\ \algorithmicdo}% \ForEach{#1}
%   {\algorithmicend\ \algorithmicforeach}% \EndForEach

\usepackage[colorlinks=true, citecolor=cyan]{hyperref}  
\setlength{\parindent}{0.5cm}
\def\BibTeX{{\rm B\kern-.05em{\sc i\kern-.025em b}\kern-.08em
    T\kern-.1667em\lower.7ex\hbox{E}\kern-.125emX}}


\lstset{
    language=Lisp,
    frame=single,
    breaklines=true,
    basicstyle=\scriptsize\ttfamily,
    moredelim=**[is][\color{red}]{@}{@},
}

% \title{\LARGE \bf
% Planology: Anticipating household sequences and planning for human-robot collaboration
% \thanks{*Denotes equal contribution}
% }

%\title{\LARGE \bf
%Validating LLM Based Open Set Task Decomposition with Knowledge Graphs and Human Intervened Validation for Task Accomplishment\thanks{*Denotes equal contribution}}

% LLM-based generic task decomposition, KG-based domain-specific task decomposition, HITL

\title{\LARGE \bf
AdaptBot: Combining LLM with Knowledge Graphs and Human Input for Generic-to-Specific Task Decomposition and Knowledge Refinement}


\author{ Shivam Singh$^{1*}$, Karthik Swaminathan$^{1*}$, Nabanita Dash$^1$, Ramandeep Singh$^1$ \\  Snehasis Banerjee$^2$,  Mohan Sridharan$^3$,  Madhava Krishna$^1$
\thanks{*Denotes equal contribution}

\thanks{$^{1}$ Robotics Research Center, IIIT Hyderabad, India}
\thanks{$^{2}$ TCS Research, Tata Consultancy Services, India}
\thanks{$^{3}$ School of Informatics, University of Edinburgh, UK}
}

\makeatletter
\renewcommand{\@seccntformat}[1]{%
  \protect\csname the#1\endcsname\protect\quad%
}
\makeatother

\renewcommand{\thesection}{\arabic{section}}
\renewcommand{\thesubsection}{\thesection.\arabic{subsection}}
\renewcommand{\thesubsubsection}{\thesubsection.\arabic{subsubsection}}

\begin{document}


\maketitle
\thispagestyle{empty}
\pagestyle{empty}


%%%%%%%%%%%%%%%%%%%%%%%%%%%%%%%%%%%%%%%%%%%%%%%%%%%%%%%%%%%%%%%%%%%%%%%%%%%%%%%%
\begin{abstract}
An embodied agent assisting humans is often asked to complete new tasks, and 
%An agent preparing a particular dish in the kitchen based on a known recipe may be asked to prepare a new dish or to perform cleaning tasks in the storeroom. 
there may not be sufficient time or labeled examples to train the agent to perform these new tasks. Large Language Models (LLMs) trained on considerable knowledge across many domains can be used to predict a sequence of abstract actions for completing such tasks, although the agent may not be able to execute this sequence due to task-, agent-, or domain-specific constraints. Our framework addresses these challenges by leveraging the generic predictions provided by LLM and the prior domain knowledge encoded in a Knowledge Graph (KG), enabling an agent to quickly adapt to new tasks. The robot also solicits and uses human input as needed to refine its existing knowledge. Based on experimental evaluation in the context of cooking and cleaning tasks in simulation domains, we demonstrate that the interplay between LLM, KG, and human input leads to substantial performance gains compared with just using the LLM. \\
Project website\footnote[4]{Project supported in part by TCS Research India}: \href{https://sssshivvvv.github.io/adaptbot/}{https://sssshivvvv.github.io/adaptbot/}
%\Madhav{Please add: Through a diversity of metrics we show substantial performance gain through the interplay between LLM, KG and Human feedback than a straightforward task execution based on abstract LLM outputs.}

%Embodied agents or assistive robots may often be challenged to accomplish a task that they have not encountered before or trained for. For example an assistive or a personal robot may be asked to cook a new recipe or perform a cleaning task with novel specifications. In such scenarios it can indeed be challenging and cumbersome to train the agent all over again for apart from the training process itself data generation offers significant difficulties. LLM have become popular in very recent times as a popular framework for novel task synthesis. When prompted adequately LLM can achieve novel task synthesis by decomposing the task into a sequence of sub tasks that need to be achieved in the specified sequence for successful completion. However LLM task sequencing abilities are statistical offering much room for improving its success rates. In this paper we propose a novel framework that effectively blends the statistical nature of LLM outputs with the deterministic nature of Knowledge Graph ontologies to achieve novel tasks by significantly improving the performance of LLM. Further by involving humans in a principled fashion %to validate LLM sequences that lie outside the domain of the KG we show  the framework increments and updates it Knowledge Graph showcasing knowledge refinement, expansion features that leads to further improvement across various performance metrics

\end{abstract}
\vspace{-1em}
\begin{keywords}
Large Language Models, Knowledge Graph, Human-in-the-loop Learning
\end{keywords}
% \textbf{Color Coding}: \\
% \crule[purple]{10pt}{10pt} : \raghav{Raghav} \\
% \crule[blue]{10pt}{10pt} : \karthik{Karthik}\\
% \crule[olive]{10pt}{10pt} : \shivam{Shivam}\\


% humans are sensitive to the way information is presented.

% introduce framing as the way we address framing. say something about political views and how information is represented.

% in this paper we explore if models show similar sensitivity.

% why is it important/interesting.



% thought - it would be interesting to test it on real world data, but it would be hard to test humans because they come already biased about real world stuff, so we tested artificial.


% LLMs have recently been shown to mimic cognitive biases, typically associated with human behavior~\citep{ malberg2024comprehensive, itzhak-etal-2024-instructed}. This resemblance has significant implications for how we perceive these models and what we can expect from them in real-world interactions and decisionmaking~\citep{eigner2024determinants, echterhoff-etal-2024-cognitive}.

The \textit{framing effect} is a well-known cognitive phenomenon, where different presentations of the same underlying facts affect human perception towards them~\citep{tversky1981framing}.
For example, presenting an economic policy as only creating 50,000 new jobs, versus also reporting that it would cost 2B USD, can dramatically shift public opinion~\cite{sniderman2004structure}. 
%%%%%%%% 图1:  %%%%%%%%%%%%%%%%
\begin{figure}[t]
    \centering
    \includegraphics[width=\columnwidth]{Figs/01.pdf}
    \caption{Performance comparison (Top-1 Acc (\%)) under various open-vocabulary evaluation settings where the video learners except for CLIP are tuned on Kinetics-400~\cite{k400} with frozen text encoders. The satisfying in-context generalizability on UCF101~\cite{UCF101} (a) can be severely affected by static bias when evaluating on out-of-context SCUBA-UCF101~\cite{li2023mitigating} (b) by replacing the video background with other images.}
    \label{fig:teaser}
\end{figure}


Previous research has shown that LLMs exhibit various cognitive biases, including the framing effect~\cite{lore2024strategic,shaikh2024cbeval,malberg2024comprehensive,echterhoff-etal-2024-cognitive}. However, these either rely on synthetic datasets or evaluate LLMs on different data from what humans were tested on. In addition, comparisons between models and humans typically treat human performance as a baseline rather than comparing patterns in human behavior. 
% \gabis{looks good! what do we mean by ``most studies'' or ``rarely'' can we remove those? or we want to say that we don't know of previous work doing both at the same time?}\gili{yeah the main point is that some work has done each separated, but not all of it together. how about now?}

In this work, we evaluate LLMs on real-world data. Rather than measuring model performance in terms of accuracy, we analyze how closely their responses align with human annotations. Furthermore, while previous studies have examined the effect of framing on decision making, we extend this analysis to sentiment analysis, as sentiment perception plays a key explanatory role in decision-making \cite{lerner2015emotion}. 
%Based on this, we argue that examining sentiment shifts in response to reframing can provide deeper insights into the framing effect. \gabis{I don't understand this last claim. Maybe remove and just say we extend to sentiment analysis?}

% Understanding how LLMs respond to framing is crucial, as they are increasingly integrated into real-world applications~\citep{gan2024application, hurlin2024fairness}.
% In some applications, e.g., in virtual companions, framing can be harnessed to produce human-like behavior leading to better engagement.
% In contrast, in other applications, such as financial or legal advice, mitigating the effect of framing can lead to less biased decisions.
% In both cases, a better understanding of the framing effect on LLMs can help develop strategies to mitigate its negative impacts,
% while utilizing its positive aspects. \gabis{$\leftarrow$ reading this again, maybe this isn't the right place for this paragraph. Consider putting in the conclusion? I think that after we said that people have worked on it, we don't necessarily need this here and will shorten the long intro}


% If framing can influence their outputs, this could have significant societal effects,
% from spreading biases in automated decision-making~\citep{ghasemaghaei2024understanding} to reducing public trust in AI-generated content~\citep{afroogh2024trust}. 
% However, framing is not inherently negative -- understanding how it affects LLM outputs can offer valuable insights into both human and machine cognition.
% By systematically investigating the framing effect,


%It is therefore crucial to systematically investigate the framing effect, to better understand and mitigate its impact. \gabis{This paragraph is important - I think that right now it's saying that we don't want models to be influenced by framing (since we want to mitigate its impact, right?) When we talked I think we had a more nuanced position?}




To better understand the framing effect in LLMs in comparison to human behavior,
we introduce the \name{} dataset (Section~\ref{sec:data}), comprising 1,000 statements, constructed through a three-step process, as shown in Figure~\ref{fig:fig1}.
First, we collect a set of real-world statements that express a clear negative or positive sentiment (e.g., ``I won the highest prize'').
%as exemplified in Figure~\ref{fig:fig1} -- ``I won the highest prize'' positive base statement. (2) next,
Second, we \emph{reframe} the text by adding a prefix or suffix with an opposite sentiment (e.g., ``I won the highest prize, \emph{although I lost all my friends on the way}'').
Finally, we collect human annotations by asking different participants
if they consider the reframed statement to be overall positive or negative.
% \gabist{This allows us to quantify the extent of \textit{sentiment shifts}, which is defined as labeling the sentiment aligning with the opposite framing, rather then the base sentiment -- e.g., voting ``negative'' for the statement ``I won the highest prize, although I lost all my friends on the way'', as it aligns with the opposite framing sentiment.}
We choose to annotate Amazon reviews, where sentiment is more robust, compared to e.g., the news domain which introduces confounding variables such as prior political leaning~\cite{druckman2004political}.


%While the implications of framing on sensitive and controversial topics like politics or economics are highly relevant to real-world applications, testing these subjects in a controlled setting is challenging. Such topics can introduce confounding variables, as annotators might rely on their personal beliefs or emotions rather than focusing solely on the framing, particularly when the content is emotionally charged~\cite{druckman2004political}. To balance real-world relevance with experimental reliability, we chose to focus on statements derived from Amazon reviews. These are naturally occurring, sentiment-rich texts that are less likely to trigger strong preexisting biases or emotional reactions. For instance, a review like ``The book was engaging'' can be framed negatively without invoking specific cultural or political associations. 

 In Section~\ref{sec:results}, we evaluate eight state-of-the-art LLMs
 % including \gpt{}~\cite{openai2024gpt4osystemcard}, \llama{}~\cite{dubey2024llama}, \mistral{}~\cite{jiang2023mistral}, \mixtral{}~\cite{mistral2023mixtral}, and \gemma{}~\cite{team2024gemma}, 
on the \name{} dataset and compare them against human annotations. We find  that LLMs are influenced by framing, somewhat similar to human behavior. All models show a \emph{strong} correlation ($r>0.57$) with human behavior.
%All models show a correlation with human responses of more than $0.55$ in Pearson's $r$ \gabis{@Gili check how people report this?}.
Moreover, we find that both humans and LLMs are more influenced by positive reframing rather than negative reframing. We also find that larger models tend to be more correlated with human behavior. Interestingly, \gpt{} shows the lowest correlation with human behavior. This raises questions about how architectural or training differences might influence susceptibility to framing. 
%\gabis{this last finding about \gpt{} stands in opposition to the start of the statement, right? Even though it's probably one of the largest models, it doesn't correlate with humans? If so, better to state this explicitly}

This work contributes to understanding the parallels between LLM and human cognition, offering insights into how cognitive mechanisms such as the framing effect emerge in LLMs.\footnote{\name{} data available at \url{https://huggingface.co/datasets/gililior/WildFrame}\\Code: ~\url{https://github.com/SLAB-NLP/WildFrame-Eval}}

%\gabist{It also raises fundamental philosophical and practical questions -- should LLMs aim to emulate human-like behavior, even when such behavior is susceptible to harmful cognitive biases? or should they strive to deviate from human tendencies to avoid reproducing these pitfalls?}\gabis{$\leftarrow$ also following Itay's comment, maybe this is better in the dicsussion, since we don't address these questions in the paper.} %\gabis{This last statement brings the nuance back, so I think it contradicts the previous parapgraph where we talked about ``mitigating'' the effect of framing. Also, I think it would be nice to discuss this a bit more in depth, maybe in the discussion section.}






%\section{INTRODUCTION}\label{Intro}

\section{Rethinking Sparse Attention Methods}
\label{sec:critique}

Modern sparse attention methods have made significant strides in reducing the theoretical computational complexity of transformer models. However, most approaches predominantly apply sparsity during inference while retaining a pretrained Full Attention backbone, potentially introducing architectural bias that limits their ability to fully exploit sparse attention's advantages. Before introducing our native sparse architecture, we systematically analyze these limitations through two critical lenses.


\begin{figure*}[t] 
\centering 
\includegraphics[width=1\textwidth]{figures/fig2.pdf} 
\caption{Overview of \method{}'s architecture. Left: The framework processes input sequences through three parallel attention branches: For a given query, preceding keys and values are processed into compressed attention for coarse-grained patterns, selected attention for important token blocks, and sliding attention for local context. Right: Visualization of different attention patterns produced by each branch. Green areas indicate regions where attention scores need to be computed, while white areas represent regions that can be skipped.}
\label{fig:framework}
\end{figure*}


\subsection{The Illusion of Efficient Inference}

Despite achieving sparsity in attention computation, many methods fail to achieve corresponding reductions in inference latency, primarily due to two challenges:

\textbf{Phase-Restricted Sparsity.}
Methods such as H2O \citep{h2o} apply sparsity during autoregressive decoding while requiring computationally intensive pre-processing (e.g. attention map calculation, index building) during prefilling. In contrast, approaches like MInference \citep{minference} focus solely on prefilling sparsity. 
These methods fail to achieve acceleration across all inference stages, as at least one phase remains computational costs comparable to Full Attention.
The phase specialization reduces the speedup ability of these methods in prefilling-dominated workloads like book summarization and code completion, or decoding-dominated workloads like long chain-of-thought~\citep{cot} reasoning.

\textbf{Incompatibility with Advanced Attention Architecture.}
Some sparse attention methods fail to adapt to modern decoding efficient architectures like Mulitiple-Query Attention~(MQA) \citep{mqa} and Grouped-Query Attention~(GQA) \citep{gqa}, which significantly reduced the memory access bottleneck during decoding by sharing KV across multiple query heads. For instance, in approaches like Quest \citep{quest}, each attention head independently selects its KV-cache subset. Although it demonstrates consistent computation sparsity and memory access sparsity in Multi-Head Attention (MHA) models, it presents a different scenario in models based on architectures like GQA, where the memory access volume of KV-cache corresponds to the union of selections from all query heads within the same GQA group. This architectural characteristic means that while these methods can reduce computation operations, the required KV-cache memory access remains relatively high.
This limitation forces a critical choice: while some sparse attention methods reduce computation, their scattered memory access pattern conflicts with efficient memory access design from advanced architectures.

These limitations arise because many existing sparse attention methods focus on KV-cache reduction or theoretical computation reduction, but struggle to achieve significant latency reduction in advanced frameworks or backends.
This motivates us to develop algorithms that combine both advanced architectural and hardware-efficient implementation to fully leverage sparsity for improving model efficiency.


\subsection{The Myth of Trainable Sparsity}
Our pursuit of native trainable sparse attention is motivated by two key insights from analyzing inference-only approaches:
(1) \textbf{\textit{Performance Degradation}}: Applying sparsity post-hoc forces models to deviate from their pretrained optimization trajectory. As demonstrated by \citet{magicpig}, top 20\% attention can only cover 70\% of the total attention scores, rendering structures like retrieval heads in pretrained models vulnerable to pruning during inference.
(2)~\textbf{\textit{Training Efficiency Demands}}: 
Efficient handling of  long-sequence training is crucial for modern LLM development. This includes both pretraining on longer documents to enhance model capacity, and subsequent adaptation phases such as long-context fine-tuning and reinforcement learning. However, existing sparse attention methods primarily target inference, leaving the computational challenges in training largely unaddressed. This limitation hinders the development of more capable long-context models through efficient training. Additionally, efforts to adapt existing sparse attention for training also expose challenges:



\textbf{Non-Trainable Components.} Discrete operations in methods like ClusterKV~\citep{clusterkv} 
(includes k-means clustering) and MagicPIG~\citep{magicpig} (includes SimHash-based selecting) create discontinuities in the computational graph. These non-trainable components prevent gradient flow through the token selection process, limiting the model's ability to learn optimal sparse patterns. 

\textbf{Inefficient Back-propagation.} Some theoretically trainable sparse attention methods suffer from practical training inefficiencies. Token-granular selection strategy used in approaches like HashAttention~\citep{desai2024hashattention} leads to the need to load a large number of individual tokens from the KV cache during attention computation. 
This non-contiguous memory access prevents efficient adaptation of fast attention techniques like FlashAttention, which rely on contiguous memory access and blockwise computation to achieve high throughput.
As a result, implementations are forced to fall back to low hardware utilization, significantly degrading training efficiency.



\subsection{Native Sparsity as an Imperative}

These limitations in inference efficiency and training viability motivate our fundamental redesign of sparse attention mechanisms.
We propose \method{}, a natively sparse attention framework that addresses both computational efficiency and training requirements.
In the following sections, we detail the algorithmic design and operator implementation of \method{}.

% \nabanita{Resource Description Framework (RDF)}

% The Resource Description Framework (RDF) is a flexible framework for representing and sharing structured data on the web. It expresses relationships between entities, making it valuable for describing and exchanging metadata across systems. Its standardized approach enhances data integration and interoperability.

% \nabanita{Set-Theoretic Representation of RDF}

% In set-theoretic terms, an RDF triple can be expressed as a binary relation from a set of elements of \( S \) to a set of elements of \( O \). RDF can be expressed in triple format:
% \[
% T = (S, P, O)
% \]
% where \( T \) (the RDF triple) comprises:
% \begin{itemize}
%     \item \( S \) (subject): the set of entities or resources being referred to.
%     \item \( P \) (predicate): the binary relation or property connecting the set of entities \( S \) and their corresponding values \( O \).
%     \item \( O \) (object): the set of values of the referred entities, or another entity being linked to.
% \end{itemize}

% Formally, we can define the binary relation between \( S \) and \( O \) over \( P \) as:
% \[
% P(S) = O
% \]

% For example, let \( P \) represent a function detecting whether an entity is "boilable" or "isboiled," \( S \) represent certain food items such as milk, apple, cucumber, and egg, and \( O \) represent the value (true or false). We can express this as:

% \begin{align*}
% \text{boilable}(\text{milk}) &= \text{true} \\
% \text{boilable}(\text{apple}) &= \text{false} \\
% \text{boilable}(\text{cucumber}) &= \text{false} \\
% \text{boilable}(\text{egg}) &= \text{true}
% \end{align*}

% \begin{align*}
% \text{isboiled}(\text{milk}) &= \text{false} \\
% \text{isboiled}(\text{egg}) &= \text{true}
% \end{align*}

% In some other cases, \( O \) can represent another entity:
% \[
% \text{isOnTopOf}(\text{lemon}) = \text{cutting board}
% \]
% \[
% \text{contentsIn}(\text{pan}) = \text{tea}
% \]

% \nabanita{Knowledge Expansion (KE) in RDF Schema}

% In RDF Schema, Knowledge Expansion (KE) refers to extending relationships between classes, their instances, and properties across different actions or attributes. RDF Schema models these hierarchical relationships using constructs like \texttt{subClassOf} and \texttt{subPropertyOf}, allowing knowledge to be reused, inferred, and extended from broader categories to more specific ones.

% \nabanita{Class and Property Relationships}

% The triple relations can also be extended to class relationships, such as:
% \[
% \text{classOf}(\text{Apple}) = \text{Fruit}
% \]
% \[
% \text{classOf}(\text{Granny Smith apple}) = \text{Apple}
% \]

% Hence:
% \begin{enumerate}
%     \item Any instance of the class \( \text{Apple} \) is also an instance of the class \( \text{Fruit} \):
%     \[
%     \forall x \, (x \in \text{Granny Smith Apple} \Rightarrow x \in \text{Fruit})
%     \]
    
%     \item The properties of the class \( \text{Fruit} \) extend to the class \( \text{Granny Smith Apple} \):
%     \[
%     \forall x, y \, (P(x, y) \Rightarrow Q(x, y))
%     \]
%     For instance, if \( \text{isSliceable}(\text{Fruit}) \) holds, then \( \text{isSliceable}(\text{Granny Smith Apple}) \) must also hold.

%     \item The subproperties of \( \text{cut} \) extend to more specific actions such as \( \text{slice}, \text{dice}, \text{chop} \):
%     \[
%     \text{subPropertyOf}(\text{cut}) = \text{slice}
%     \]
% \end{enumerate}

%%%%%%%%%%%%%%% NEED FIX 
%%%%%%%%%%%%%%%%%%%%%%%%%%%%%%%%%%%%%%%%%%%%%%%%%%%%%%%%%%%%%%%%%
%%%%%%%%%%%%%%%%%%%%%%%%%%%%%%%%%%%%%%%%%%%%%%%%%%%%%%%%%%%%%%%%%
\section{Problem Formulation and Framework}
\vspace{-0.5em}
\label{sec:framework}
Figure~\ref{fig:pipeline} is an outline of our framework. In the motivating example, an agent assisting in cooking tasks in a kitchen has access to relevant objects and ingredients for many dishes but it does not have the recipes. When asked to prepare any particular dish, $\tau_i$, the agent queries an LLM to obtain a sequence of abstract actions (sub-tasks), i.e., $\langle a_1, \ldots, a_{m_i}\rangle$. For example, the sequence for \textit{make an omelette} includes \textit{picking up the egg} and \textit{breaking the egg over a skillet}. This sequence of abstract actions is checked against a KG with some domain-specific information in the form of existing objects and attributes that include the actions that can be performed on some objects. The agent tries to resolve any discrepancy between the LLM output and KG, e.g., KG states there is no skillet or that an egg can only be cracked, by finding replacements, e.g., \textit{crack the egg over a pan}. If the discrepancy is not resolved, or if executing the action sequence does not provide the desired outcome, the agent identifies relevant actions and solicits human input to refine the KG, e.g., add knowledge of objects or their attributes, and provides an action sequence to complete the task. The agent is assumed to be able to execute these actions. We describe out framework's components below.


%Since the agent can be given a different dish with a different ingredient requirement, it decomposes the recipe making into actions relevant to the dish. Our framework in  leverages the generic knowledge and the stochastic nature of LLMs to decompose the task into an action sequence with limited prompting. The initial output from the LLMs goes through series of checks and refinement in the refinement block where the KG is queried for the property checks of the ingredients and so on. We give feedbacks in case of persistent errors and if not resolved, the human gives an input for correction and in some cases expands the knowledge which is discussed in section \ref{sec:kg_expansion}. We describe the components of our framework below: 
%%%%%%%%%%%%%%%%%%%%%%%%%%%%%%%%%%%%%%%%%%%%%%%%%%%%%%%%%%%%%%%%%
\subsection{Generic Task Decomposition with LLM}
\label{sec:framework-llm}
In our framework, we use an LLM to decompose any given task into a sequence of sub-tasks because LLMs have demonstrated the ability to provide such a sequence of abstract actions for many different tasks. Specifically, in the motivating example, the LLM is prompted with information about some domain objects, an example cooking task (make coffee), and the corresponding action sequence (recipe) to be executed---see Figure~\ref{fig:pipeline}a. We experimentally evaluate the use of different LLMs, as described in Section~\ref{sec:expres-setup}. 
%Our choice of using the LLMs to decompose the task into action sequence is motivated by two objectives: (i) With the generic knowledge and in addition to one-shot prompting technique, we aim to achieve an action output which are direct function calls during the execution; and (ii) To get variability in the output and with human-in-the-loop in play, a possibility of expanding the knowledge wherever necessary. As described in Section~\ref{sec:expt}, we explored the use of popular LLMs such as GPT-3.5\cite{brown2020language}, GPT-4o\cite{OpenAI2023GPT4TR}, Llama3-70b (cite), and Gemma2-9b (cite).

\vspace{-0.75em}
Since the sequence of sub-tasks predicted by the LLM is based on many information sources, it may not be possible to execute one or more of these actions. For example, in the context of cooking tasks, the suggested ingredient may not be available or the action may involve an incorrect choice of tool (e.g., using a fork to cut vegetables). These situations can be addressed in part by using prior domain-specific information, which is encoded as described below.

%%%%%%%%%%%%%%%%%%%%%%%%%%%%%%%%%%%%%%%%%%%%%%%%%%%%%%%%%%%%%%%%%
\subsection{Representing Domain-specific Knowledge with KG}
\label{sec:framework-kg}
Our framework uses a Knowledge Graph (KG) to encode any prior information available to the agent. In the context of cooking tasks, this includes knowledge of some classes of ingredients (e.g., herbs, fruits, vegetables), receptacles (e.g., plates, bowls, countertop), and tools (e.g. knives, spoons), which can be arranged hierarchically. It also encodes the existence of some specific instances of these object classes and their properties such as likely location(s) and the actions they can be involved in (e.g., cutting, scooping, grinding). We use the \textit{Resource Description Framework} (RDF) format to encode this information in two graph structures in Turtle format (.ttl file)---see Figure~\ref{fig:kg-nodes}:
%The environment we are working with is described using a structured graph-based representation. It consists of rooms (e.g., kitchen, etc.), various objects (e.g., apples, mangoes, etc.), receptacles (e.g., plates, bowls, stoves, etc.), and tools (e.g., knives, spoons, etc.). Each item and its properties are represented in a graph structure, with nodes denoting the items and edges denoting the properties or states of these items. Specifically, we use a  data model to represent both the environment's current state and its underlying knowledge. We define two distinct graphs:
\begin{enumerate}
\vspace{-0.75em}
\item \textbf{State graph:} models current state as $\mathbf{G_s} = (\mathbf{I_s}, \mathbf{E_s})$, where nodes $\mathbf{I_s}$ are instances of object classes such as ingredients and receptacles; and $\mathbf{E_s} \subseteq \mathbf{I_s} \times \mathbf{P_s} \times \mathbf{V_s}$ are edges such that $(i_j, p, v_k) \in \mathbf{E_s}$ is a triple denoting an attribute of $i_j \in \mathbf{I_s}$ in terms of value $v_k \in \mathbf{V_s}$ of predicate $p \in \mathbf{P_s}$. For example, (apple1, obj\_location, fridge) and (apple1, is\_sliced, true) express \textit{apple1's} location and that it is sliced.
% \begin{enumerate} 
%     \item $\mathbf{I_s}$ represents the set of nodes corresponding to the items in the environment (objects, receptacles, and tools).
%     \item $\mathbf{E_s} \subseteq \mathbf{I_s} \times \mathbf{P_s} \times \mathbf{V_s}$ represents the set of edges. Each edge $(i_j, p, v_k) \in \mathbf{E_s}$ is a triple that denotes an item $i_j \in \mathbf{I_s}$, a predicate $p \in \mathbf{P_s}$ (representing the state of the object), and a value $v_k \in \mathbf{V_s}$ that describes the state of the item. The value could either be a boolean or another item in the environment. For example, a triple in the state graph could be (apple, obj\_location, fridge) and (apple, is\_sliced, true), indicating that the apple is stored in the fridge and that its current state is \textit{sliced}.
% \end{enumerate} 
\item \textbf{Attribute graph:} encodes the known properties and action capabilities of some object classes as $\mathbf{G_k} = (\mathbf{I_k}, \mathbf{E_k})$, where nodes $\mathbf{I_k}$ represent the classes and edges $\mathbf{E_k} \subseteq \mathbf{I_k} \times \mathbf{P_k} \times \mathbf{V_k}$ represent class properties, e.g., (apple, sliceable, true) implies  apples can be sliced.
\vspace{-0.75em}
\end{enumerate}
%In both graphs, the sets of predicates $\mathbf{P_s}$ and $\mathbf{P_k}$ define possible states or relations, such as IsBoiled, Boilable, IsCleaned, NeedsToBeCleaned, etc. 
The available actions include moving, picking up, and putting down objects; using tools; cleaning, toggling, slicing, stirring, and mopping\footnote{Supplementary material includes list of all the actions.}. Such a KG can be learned automatically based on information extracted from datasets or sensor streams. The feasibility of any action/sub-task in the sequence predicted by LLM is then checked using $\mathbf{G_k}$ and $\mathbf{G_s}$ by generating suitable SPARQL queries. If the predicted sequence of actions passes the KG-based check, it is executed, changing $\mathbf{G_s}$ suitably.
%The agent's task is to achieve a goal by performing a sequence of actions on the items present in the environment. Each action modifies and updates the state graph $\mathbf{G_s}$, reflecting the changes in the environment. The agent is equipped with a predefined skill set $\mathbf{A} = \{a_1, a_2,...,a_n\}$, where each action $a_i$ operates on items in the state graph and updates their states. These actions include basic manipulations such as 

%Each action can be formalized as a function: $T_a: \mathbf{G_s} \times \mathbf{G_k} \Rightarrow \mathbf{G_s}^{'}$, where $T_a$ applies the action $a$ to the current state graph $\mathbf{G_s}$, and the knowledge graph $\mathbf{G_k}$ is used to verify if the action is applicable (e.g., whether an item is fryable or boilable). The result is a new state graph $\mathbf{G_s}^{'}$, which represents the updated states of the items in the environment.
%To determine the feasibility of an action, SPARQL queries are used. For each action $a \in \mathbf{A}$, a SPARQL query is generated to check the preconditions in both $\mathbf{G_s}$ and $\mathbf{G_k}$. For instance, The action \texttt{\small{slice(apple, knife, state\_file)}} refers to slicing an apple using a knife, with the state file used to extract the current state of all the items necessary to perform this action. To perform this action, the query ensures that the apple is in the correct location (e.g., on the slicing board) and is sliceable according to the knowledge graph $\mathbf{G_k}$. Upon successful verification, the action is performed, and the state graph $\mathbf{G_s}$ is updated with the new status (e.g., \texttt{\small{(apple, sliced, true))}}.
%The flexibility of RDF allows for easy modifications and updates to the environment model, supporting dynamic task execution and refinement. The combination of the state and knowledge graphs enables the agent to make informed decisions and adapt to changes in the environment.
\begin{figure}[tb]
\captionsetup{font=scriptsize}
\centering
    \begin{minipage}{0.3\textwidth}
        \begin{lstlisting}[basicstyle=\ttfamily\scriptsize]
ex:onion rdf:type ex:object ;
    ex:obj_name 'onion' ;
    ex:IsSliceable true ;
    ex:Fryable true ;
    ex:NeedsToBeCleaned true .
        \end{lstlisting}
        % \caption*{(a)}
    \end{minipage}
    % \setlength{\abovecaptionskip}{-2pt}
    % \setlength{\belowcaptionskip}{-12pt}
    \hfill
    % \caption{Example of a node \textit{onion} in $\mathbf{G_k}$.}
    % \label{fig:12}
% \end{figure}

% \begin{figure}[tb]
% \captionsetup{font=scriptsize}
% \centering
    \begin{minipage}{0.3\textwidth}
        \begin{lstlisting}[basicstyle=\ttfamily\scriptsize]
ex:onion rdf:type ex:object ;
    ex:obj_name 'onion' ;
    ex:obj_location ex:fridge .
    ex:sliced false ;
    ex:IsFried false ;
    ex:IsCleaned false .
        \end{lstlisting}
        % \caption*{(b)}
    \end{minipage}
    \setlength{\abovecaptionskip}{-2pt}
    \setlength{\belowcaptionskip}{-12pt}
    \caption{Example of a node \textit{onion} in $\mathbf{G_k}$ (\textit{top}) and $\mathbf{G_s}$ (\textit{bottom}).}
    \label{fig:kg-nodes}
    \vspace{-0.5em}
\end{figure}

% \begin{figure}[tb] 
% \captionsetup{font=scriptsize}
% \centering
%     \begin{minipage}{0.225\textwidth} 
%         \centering
%         \begin{lstlisting}[basicstyle=\ttfamily\scriptsize]
% ex:onion rdf:type ex:object ;
%     ex:obj_name 'onion' ;
%     ex:IsSliceable true ;
%     ex:Fryable true ;
%     ex:NeedsToBeCleaned true .
%         \end{lstlisting}
%         \caption*{(a)} % Sub-caption
%     \end{minipage}
%     \setlength{\abovecaptionskip}{-2pt}
%     \setlength{\belowcaptionskip}{-2pt}
%     \hfill 
%     \begin{minipage}{0.225\textwidth} 
%         \centering
%         \begin{lstlisting}[basicstyle=\ttfamily\scriptsize]
% ex:onion rdf:type ex:object ;
%     ex:obj_name 'onion' ;
%     ex:obj_location ex:fridge .
%     ex:sliced false ;
%     ex:IsFried false ;
%     ex:IsCleaned false .
%         \end{lstlisting}
%         \caption*{(b)} 
%     \end{minipage}
%     \setlength{\abovecaptionskip}{-2pt}
%     \setlength{\belowcaptionskip}{-2pt}
%     \caption{Example of a node \textit{onion} in both $\mathbf{G_k}$ and $\mathbf{G_s}$.}
%     \label{fig:kg-nodes}
%     \vspace{-0.5em}
% \end{figure}

%%%%%%%%%%%%%%%%%%%%%%%%%%%%%%%%%%%%%%%%%%%%%%%%%%%%%%%%%%%%%%%%%
\subsection{Refining LLM output}
\label{sec:framework-refine}
%\shivam{have to add why we preferred string matching over similarity detection}
If a mismatch is detected between the LLM output and the KG, the agent attempts to use the KG to \textit{revise} the action sequence---see Figure~\ref{fig:pipeline}(b). 
%The LLM output often contains inconsistencies in item names and action names, which can hinder the task execution. To address this, we use an Error Detection and Refinement block (Fig \ref{fig:pipeline}(b)), which leverages the knowledge graph (KG) to identify and correct errors in the LLM output. 
Specifically, the agent attempts to replace the text corresponding to the identified mismatch, which can refer to actions, object instances, or object attributes, with other text from the KG.
%This block focuses on resolving inconsistencies by checking the LLM-generated item and action names against the structured knowledge available in the KG.  For unknown items or actions, we employ methods like substring matching and similarity scoring to find close matches. However, since similarity detection performs better with sentences rather than individual words, it was less effective in our case. Substring matching, on the other hand, proved to be more suitable for replacing unknown items and actions with the correct ones. Every action execution involves querying the state graph. Initially, there is an initial state graph $\mathbf{G_s}$, and after an action is executed, it updates to a new state graph $\mathbf{G'_s}$. In an action sequence, the first action should query the initial state graph to extract the initial state of the environment, while subsequent actions should query the updated graph, as the states change with each action. 
%\nabanita{To rephrase words effectively, it's important to consider two key factors: syntactic similarity and semantic similarity. Syntactic similarity ensures that the sentence's structure stays intact, so the replacement fits grammatically. Semantic similarity, on the other hand, guarantees that the new word preserves the original meaning. Using hypernyms (broader terms) or hyponyms (more specific terms) can also help to fine-tune the level of detail or generality, ensuring the overall concept remains accurate. By balancing these elements, our system can swap words smoothly without distorting meaning or disrupting the task sequence's flow.} 
While performing such text replacement, it is important to consider syntactic similarity, which measures similarity in the structure (e.g., of words or sentences), and semantic similarity, which considers similarity in meaning.
In our framework, the agent can compute the similarity of the identified words (or their embedding) with words from a similar category (or their embedding) in the KG. The use of word embeddings requires additional contextual information and makes it difficult to understand the revision of the LLM output. We thus chose to use the direct matching of words while considering hypernyms (broader terms) or hyponyms (more specific terms) for simplicity, ease of use, and transparency. If the agent is able to replace all identified mismatches, it executes the actions. 
% The refinement block also takes care of wrong action - item associations; for e.g., it will change pick\_up\_obj(plate, onto\_file) to pick_up_rec(plate, onto\_file) and slice(mango, fork, onto\_file) to slice(mango, knife, onto\_file) by detecting that fork is not a SlicingTool and thus replacing it with the tools which have IsSlicingTool property enabled true etc. 
%Through our experiments, we have found that the error detection and refinement is highly effective, as it corrects the majority of the errors present in the LLM output, significantly improving the overall accuracy and consistency of the generated actions and items.


%%%%%%%%%%%%%%%%%%%%%%%%%%%%%%%%%%%%%%%%%%%%%%%%%%%%%%%%%%%%%%%%%
\subsection{Knowledge refinement with human input}
\label{sec:framework-hitl}
Since the KG is not comprehensive, the agent may not be able to resolve all identified mismatches, e.g., reference to unknown object or action. Also, there may be unexpected action outcomes when the agent executes the action sequence. These situations are handled through re-prompting and human feedback---see Figure~\ref{fig:pipeline}(c).
%The system initiates a feedback loop with the LLM whenever errors are encountered. Feedback is sent in two main scenarios:
Specifically, the agent responds to an unresolved mismatch or erroneous outcome by re-prompting the LLM with additional information (of mismatch or error). If the mismatch or error persists, human input is solicited and used. 

%\begin{enumerate}
%\item \textbf{Errors in the Error Detection and Refinement Block}:The Error Detection and Refinement block aims to correct inconsistencies in the LLM output using the knowledge graph. However, if the block encounters an inconsistency it cannot resolve (e.g., \textit{unknown item/action}), an error is raised and sent back to the LLM via a feedback prompt. The LLM generates an updated output in response to this feedback, and the refinement process is repeated. We put a threshold on the number of times we go back to the LLM. If the errors persist after the feedback threshold is reached, the system escalates the issue to a human for resolution (section~\ref{sec:framework}).
%\item \textbf{Errors During Action Execution:}
%If the refinement block does not raise any errors, meaning the LLM does not produce any actions that are unknown or use items that are unfamiliar to the agent, then the LLM output is sent for execution. During execution, errors may arise due to invalid actions (e.g., performing an incompatible action on an object). In such cases, the system sends feedback to the LLM, detailing the execution error. Similar to the refinement process, there is a limit to how many times the system can query the LLM for corrections. If all feedback attempts fail to resolve the error, the system proceeds with the output, and the human evaluates the final performance. Errors during execution are minimal or nonexistent in the LLM + KG + Human framework, as the refinement block and human resolves most issues beforehand.
%\end{enumerate}
%Note that, the feedback counter is tracked globally across both phases (refinement and execution). This means that if the agent exhausts the feedback limit during the refinement phase, it will no longer be able to send feedback to the LLM during the execution phase. In such cases, the system proceeds directly to execution and human evaluation.
% once the feedback limit has been reached, ensuring efficient use of feedback resources and avoiding redundancy. 
% \subsection{Human-in-the-loop \& knowledge expansion}
% \label{sec:kg_expansion}
%As discussed in the previous section, there are instances when the agent cannot resolve an unknown item/action error, even after exhausting all of its feedback attempts. In such cases, the agent raises a query to a human, seeking confirmation of the existence of the unknown item.
%Our framework incorporates the ability to expand knowledge dynamically through human intervention. When the agent encounters an unknown item or action command from LLM (i.e., item not present in the knowledge and state graph  or action not present in agent's skill set), it initiates a query to a human user.
%\begin{itemize}
\vspace{-0.75em}
\noindent
\textbf{Existence check:} if an action or object in the LLM output does not exist in the KG, there are three possibilities: (1) The agent is mistaking an existing item (action) for another item (action); (2) the entity does not exist in the domain; or (3) the entity exists but is not in the KG. In the first case, human informs the agent about the correct object (or action); in the second case, human denies existence of entity; and in the third case, human confirms the entity's existence and agent interactively obtains entity's attributes\footnote{Supplementary material includes details of questions asked.}. As the agent expands its knowledge, the need for human input progressively decreases.

\vspace{-0.75em}
\noindent
\textbf{Learn attributes:} If human confirms existence of an instance of a new entity, the agent interactively obtains additional details. For example, when informed about an instance of a new object class \textit{onion}, agent incrementally requests information about the object type (e.g., \textit{edible\_object}) and other relevant attributes (e.g., boilable, fryable, location of instance). This knowledge revision can be viewed as correcting (expanding) the knowledge in the KG by revising class attributes in $\mathbf{G_k}$ and instance-specific details in $\mathbf{G_s}$.
%\end{itemize}
%After learning the item's current state and its properties, the agent adds this information to state graph $\mathbf{G_s}$ and the knowledge graph $\mathbf{G_k}$ respectively. The current state of the item is added to $\mathbf{G_s}$ and the item properties gets added to $\mathbf{G_k}$. The item is represented as a new node $I_{new}$ and its states \& properties are added as edges in the respective graphs. The knowledge expansion can be seen as a function $f_{KE}$ :
$$f_{KE}(I_{new}, P_{new}, S_{current}) \Rightarrow {\mathbf{G_{k}^{'}}, \mathbf{G_{s}^{'}}}$$
where $I_{new}$ is the new entity; $P_{new}$ = ${(p_1, v_1), ..., (p_n, v_n)}$ refers to attributes ($p_i$) of entity and their values ($v_i$); $S_{current}$ = ${(s_1, v_1), ..., (s_n, s_n)}$ refers to states $s_i$ and their values $v_i$; and $\mathbf{G^{'}_{k}}$ and $\mathbf{G^{'}_{s}}$ are the updated components of the KG.  For example, new edge is added in $\mathbf{G_s}$ to encode an onion's position and new edge is added in $\mathbf{G_k}$ to encode that an onion can be fried. Note that this update to existing knowledge is fully transparent by design.

%This revision is performed through SPARQL queries. We demonstrate experimentally that this combination of LLM, KG, and HITL revision substantially improves accuracy of task completion, and supports incremental and transparent knowledge revision and adaptation to new classes of tasks. 

%This process is performed internally through SPARQL queries, ensuring that the graph remains consistent and up-to-date. For instance, after querying human about the current states and properties of \textit{onion} the knowledge graph $\mathbf{G_k}$ gets updated with the edges like \texttt{\small{(onion, Fryable, true)}} and state graph $\mathbf{G_s}$ gets updated with the edge \texttt{\small{(onion, obj\_location, fridge)}}. Figure \ref{fig:nodes} illustrates the nodes in $\mathbf{G_k}$ and $\mathbf{G_s}$ with their respective \textit{properties} and \textit{current states}.
    
%This ability to incorporate human knowledge in real-time allows for continuous expansion of the environment, ensuring that the agent can operate in dynamic and evolving settings. It also highlights the flexibility of the RDF model, where updates can be seamlessly integrated without disrupting the overall structure.

\begin{algorithm}[tb]
\caption{LLM + KG + Human Input}
\label{alg:algorithm}
\begin{algorithmic}[1]
\State \textbf{Procedure} LLM\_KG\_Human($\mathbf{G_{s}}$, $\mathbf{G_{k}}$, \textit{ip\_prompt})
    \State $F$ $\leftarrow 0$                     \algorithmiccomment{$F$ is feedback counter}
    \State $T$ $\leftarrow$ call\_LLM(\textit{ip\_prompt}) \algorithmiccomment{$T$ is action sequence}
    \State $T_{refined}$, $\varepsilon_{unkn}$ $\leftarrow$ refine\_sequence($T$, $\mathbf{G_{k}}$, $\mathbf{G_{s}}$)
    %\State \textit{[EDR is error detection and refinement block]}
    \If {NOT $\varepsilon_{unkn}$}                \algorithmiccomment{$\varepsilon_{unkn}$ is unknown\_item error}
        \State $O$, $\varepsilon_{exec}$ $\leftarrow$ execute($T_{refined}$) \algorithmiccomment{$O$ is execution output}
        \null\hfill\algorithmiccomment{$\varepsilon_{exec}$ is execution error}
    \EndIf

    \While {($\varepsilon_{exec}$ OR $\varepsilon_{unkn}$) AND $F$ $<$ $F_{max}$}
        
        \While {$\varepsilon_{unkn}$ AND $F$ $<$ $F_{max}$}
            \State $T^{'}$ $\leftarrow$ call\_LLM(fb\_prompt) \algorithmiccomment{$T^{'}$is updated sequence}
            \State $T^{'}_{refined}$, $\varepsilon_{unkn}$ $\leftarrow$ refine\_sequence($T^{'}$, $\mathbf{G_{k}}$, $\mathbf{G_{s}}$)
            \State $F$ $\leftarrow$ $F$ + 1
        \EndWhile

        \If {$\varepsilon_{unkn}$ AND $F$ == $F_{max}$}
            \State $T^{'}_{refined}$ $\leftarrow$ ask\_human($\varepsilon_{unkn}$) \algorithmiccomment{$\mathbf{G_{k}}, \mathbf{G_{s}} \Rightarrow \mathbf{G_{k}^{'}}, \mathbf{G_{s}^{'}}$}
            \State $O$, $\varepsilon_{exec}$ $\leftarrow$ execute($T^{'}_{refined}$)
            \State \textbf{break}
        \EndIf

        \If {$\varepsilon_{exec}$ AND $F$ $<$ $F_{max}$}
            \State $T^{'}$ $\leftarrow$ call\_LLM(fb\_prompt)
            \State $T^{'}_{refined}$, $\varepsilon_{unkn}$ $\leftarrow$ refine\_sequence($T^{'}$, $\mathbf{G_{k}}$, $\mathbf{G_{s}}$)
            \State $F$ $\leftarrow$ $F$ + 1
        \EndIf

        \If {NOT $\varepsilon_{unkn}$}
            \State $O$, $\varepsilon_{exec}$ $\leftarrow$ execute($T^{'}_{refined}$)
        \EndIf
    \EndWhile

    % \If {(NOT $\varepsilon_{exec}$) OR ($F$ == $F_{max}$)}
    %     \State send\_human\_for\_evaluation($O$)
    % \EndIf

\State \textbf{End Procedure}

\end{algorithmic}
\end{algorithm}

\vspace{-0.75em}
Algorithm~\ref{alg:algorithm} describes the flow of information and control in our framework. The framework takes as input the state graph $\mathbf{G_s}$ and attribute graph $\mathbf{G_k}$, along with an input prompt \textit{(ip\_prompt)} that contains information about the class of tasks, an in-context example, and a query specifying the task the agent must perform. The LLM generates an action sequence $T$ (Line 3), which is refined to $T_{refined}$ using the knowledge in the KG (Line 4). If there are no unresolved mismatches between KG and LLM output ($\varepsilon_{unkn}$), the action sequence is executed, with the outcomes and errors collected for further analysis (Lines 5-7). Any unresolved mismatches or errors in outcome result in a feedback prompt to the LLM, leading to a new predicted sequence of actions $T^{'}$ (Lines 9-13, 19-23). If these mismatches and/or errors persist (beyond threshold $F_{max}$), the agent queries a human, which potentially leads to knowledge refinement, updating $\mathbf{G_k}$ and $\mathbf{G_s}$. After the expansion, the knowledge base is updated, and the refined action sequence is executed and evaluated (Lines 14-18, 24-26). This entire process is repeated until the tasks is completed or some threshold (e.g., time limit) is exceeded.

\label{evaluation-results}
% \setlength{\tabcolsep}{4.6pt}
% \begin{table*}[t]
% \centering
% \footnotesize
% \begin{tabular}{rcccccc}
% \toprule
%                                & \multicolumn{2}{c}{\textbf{DDxPlus}} & \multicolumn{2}{c}{\textbf{iCraft-MD}} & \multicolumn{2}{c}{\textbf{RareBench}} \\ \cmidrule(lr){2-3} \cmidrule(lr){4-5} \cmidrule(lr){6-7}
%                                & \textbf{GTPA@1 $\uparrow$}          & \textbf{Avg Rank $\downarrow$}   & \textbf{GTPA@1 $\uparrow$}       & \textbf{Avg Rank $\downarrow$}       & \textbf{GTPA@1 $\uparrow$}        & \textbf{Avg Rank $\downarrow$}       \\\midrule
%                                & \multicolumn{6}{c}{\textbf{GPT-4o}}                                                                 \\\midrule
% \textcolor{cyan}{Zero-shot}                      &                &            &             &                &              &                \\
% \textcolor{cyan}{Few-shot (Standard, Dyn\_BAII)} &                &            &             &                &              &                \\
% \textcolor{cyan}{Few-shot (CoT, Dyn\_BAII)}      &                &            &             &                &              &                \\
% History Taking (\textit{n}=5)         & 0.45           & 4.13       & 0.40        & 5.58           & 0.11         & 7.84           \\
% %History Taking (\textit{n}=10)        & 0.59           & 3.16       & 0.45        & 5.35           & 0.24         & 6.67           \\
% History Taking (\textit{n}=15)        & 0.69           & 2.47       & 0.46        & 5.23           & 0.36         & 5.49           \\
% Retrieval (PubMed) \textcolor{red}{rerun/ignore?}                   & 0.69           & 2.27       & 0.68        & 3.23           & 0.45         & 3.92           \\
% MEDDxAgent (\textbf{Ours})         &                &            &             &                &              &                \\
% \textit{iter} =  1                       & 0.74           & 1.91       & 0.52        & 4.93           & 0.51         & 4.37           \\
% \textit{iter} =  2                       & 0.78           & 1.56       & \textbf{0.54}        & \textbf{4.71}           & \textbf{0.56}         & 4.10           \\
% \textit{iter} =  3                       & \textbf{0.86}           & \textbf{1.29}       & \textbf{0.54}        & 4.80           & 0.50         & \textbf{4.09}           \\\midrule
%                                & \multicolumn{6}{c}{\textbf{Llama3.1-70B}}                                                           \\ \midrule
% \textcolor{cyan}{Zero-shot}                      &                &            &             &                &              &                \\
% \textcolor{cyan}{Few-shot (Standard, Dyn\_BAII)} &                &            &             &                &              &                \\
% \textcolor{cyan}{Few-shot (CoT, Dyn\_BAII)}      &                &            &             &                &              &                \\
% History Taking (\textit{n}=5)         & 0.45           & 4.15       & 0.29        & 6.48           & 0.30         & 6.04           \\
% %History Taking (\textit{n}=10)        & 0.58           & 3.12       & 0.33        & 5.82           & 0.36         & 4.51           \\
% History Taking (\textit{n}=15)        & 0.56           & 3.50       & 0.36        & 5.36           & 0.31         & 4.80           \\
% Retrieval (PubMed)  \textcolor{red}{rerun/ignore?}                 & 0.56           & 3.42       & 0.44        & 4.72           & 0.38         & 3.96           \\
% MEDDxAgent (\textbf{Ours})         &                &            &             &                &              &                \\
% \textit{iter} =  1                       & 0.61           & 2.91       & 0.29        & 7.05           & 0.39         & 5.05           \\
% \textit{iter} =  2                       & \textbf{0.71}   & \textbf{2.20}       & 0.37        & \textbf{6.26}           & \textbf{0.48}         & 4.48           \\
% \textit{iter} =  3                       & 0.68   & 2.30       & \textbf{0.42}        & 6.31           & \textbf{0.48}         & \textbf{4.30}           \\\midrule
%                                & \multicolumn{6}{c}{\textbf{Llama3.1-8B}}                                                            \\\midrule
% \textcolor{cyan}{Zero-shot}                      &                &            &             &                &              &                \\
% \textcolor{cyan}{Few-shot (Standard, Dyn\_BAII)} &                &            &             &                &              &                \\
% \textcolor{cyan}{Few-shot (CoT, Dyn\_BAII)}     &                &            &             &                &              &                \\
% History Taking (\textit{n}=5)         & 0.23           & 6.85       & 0.10        & 8.78           & 0.05         & 8.38           \\
% %History Taking (\textit{n}=10)        & 0.35           & 5.46       & 0.12        & 8.39           & \textbf{0.13}         & 8.25           \\
% History Taking (\textit{n}=15)        & 0.40           & 5.44       & 0.11        & 8.30           & \textbf{0.11}        & 8.95           \\
% Retrieval (PubMed)  \textcolor{red}{rerun/ignore?}                 & 0.42           & 4.50       & 0.29        & 6.93           & 0.35         & 5.33           \\
% MEDDxAgent (\textbf{Ours})         &                &            &             &                &              &                \\
% \textit{iter} =  1                       & 0.34           & 5.25       & 0.11        & 9.38           & 0.08         & 8.47           \\
% \textit{iter} =  2                       & 0.56           & 3.59       & \textbf{0.14}        & 9.22           & 0.09         & \textbf{8.11}           \\
% \textit{iter} =  3                       & \textbf{0.58}           & \textbf{3.10}       & 0.12        & \textbf{9.07}           & 0.07         & 8.56        \\  
% \bottomrule
%     \end{tabular}
%     \caption{Iterative experiment performance across 3 datasets. \textcolor{red}{The \textbf{best results} are based on ignoring the Pubmed retrieval results!}}
%     \label{tab:iterative_overall}
% \end{table*}

\setlength{\tabcolsep}{3.8pt}
\begin{table*}[ht]
\centering
\scriptsize
\begin{tabular}{rccccccccc}
\toprule
                               & \multicolumn{3}{c}{\textbf{DDxPlus}} & \multicolumn{3}{c}{\textbf{iCraft-MD}} & \multicolumn{3}{c}{\textbf{RareBench}} \\ \cmidrule(lr){2-4} \cmidrule(lr){5-7} \cmidrule(lr){8-10}
                               & \textbf{GTPA@1 $\uparrow$}          & \textbf{Avg Rank $\downarrow$}   & \textbf{$\Delta$ Progress} & \textbf{GTPA@1 $\uparrow$}       & \textbf{Avg Rank $\downarrow$}     & \textbf{$\Delta$ Progress}   & \textbf{GTPA@1 $\uparrow$}        & \textbf{Avg Rank $\downarrow$}   & \textbf{$\Delta$ Progress}     \\\midrule
                               & \multicolumn{9}{c}{\textbf{GPT-4o}}                                                                 \\\midrule
%\textcolor{cyan}{Zero-shot}                      &     0.69           &    2.21        &      -      &       0.68         &     3.37         &         -       &       0.46       & 3.99            &   -              \\
%\textcolor{cyan}{Zero-shot (CoT)}                      &     0.71          &    2.10        &      -      &       0.68         &     3.35         &         -       &       0.47       & 4.02            &   -              \\
%\textcolor{cyan}{Few-shot (CoT, Dyn\_BAII)} &                &            &     -        &                &              &          -      &              &             &            -     \\
%\textcolor{cyan}{Few-shot (CoT, Dyn\_BERT/Dyn\_BAII)}      &                &            &       -      &                &              &          -      &              &             &           -      \\
%\textit{Single-Turn}      &                &            &       -      &                &              &          -      &              &             &           -      \\
KR (\textit{n}=0)                &      0.18      & 7.33  &  -  &   0.15      &    8.27     & -  &       0.07   &  9.07  &    -   \\
DS (\textit{n}=0)     &  0.27    &    6.01        &       -      &      0.18          &      7.87        &          -      &       0.11       &     8.38        &           -      \\
%SDS (\textit{n}=5)         & 0.45           & 4.13     & - & 0.40        & 5.58     &    -  & 0.11         & 7.84     &  -    \\
KR (\textit{n}=5)  &      0.52      & 3.32   &  -  &  0.49  &  5.36       & -  &     0.40   &   5.27 &    -   \\
DS (\textit{n}=5)  &    0.72       &  2.14 &  -  &  0.40 &    5.55   & -  &   0.50    &   4.94 &    -   \\\cmidrule(lr){2-10}
%History Taking (\textit{n}=10)        & 0.59           & 3.16    & -  & 0.45        & 5.35        & -  & 0.24         & 6.67      &  -   \\
%SDS (\textit{n}=15)        & 0.69           & 2.47     & - & 0.46        & 5.23      &  -   & 0.36         & 5.49      &    - \\\cmidrule(lr){2-10}

%Retrieval (Wiki) \textcolor{red}{rerun}                   &            &    &  -  &         &         & -  &          &    &    -   \\
%MEDDxAgent         &                &            &             &                &              &               &              &             &                  \\
 MEDDx (\textit{iter}=1, \textit{n}=5)                       & 0.74           & 1.91     & ~~0.00 & 0.52        & 4.93      &  ~~0.00   & 0.51         & 4.37        &   ~~0.00\\
MEDDx (\textit{iter}=2, \textit{n}=10)                       & 0.78           & 1.56    & +0.32  & \textbf{0.54}        & \textbf{4.71}    &    +0.26   & \textbf{0.56}         & 4.10   &     +0.13   \\
MEDDx (\textit{iter}=3, \textit{n}=15)                       & \textbf{0.86}           & \textbf{1.29}    & +0.32  & \textbf{0.54}        & 4.80      & +0.17    & 0.50         & \textbf{4.09}       &   +0.16 \\\midrule
                               & \multicolumn{9}{c}{\textbf{Llama3.1-70B}}                                                           \\ \midrule
%\textcolor{cyan}{Zero-shot}                      &      0.54          &     3.53       &       -      &     0.40           &       4.87       &         -      &      0.39        &    4.05         &         -         \\
%\textcolor{cyan}{Zero-shot (CoT)}                      &     0.45          &    3.69       &      -      &       0.48         &     4.50         &         -       &       0.49       & 3.91            &   -              \\
%\textcolor{cyan}{Few-shot (Standard, Dyn\_BAII)} &                &            &      -       &                &              &          -      &              &             &        -         \\
%\textcolor{cyan}{Few-shot (CoT, Dyn\_BERT/Dyn\_BAII)}      &                &            &      -       &                &              &          -    &              &             &                -   \\
KR (\textit{n}=0)           &   0.19         &  7.58  &  -  &      0.13   &   8.19      & -  &    0.09      &  9.13  &    -   \\
DS (\textit{n}=0)    &        0.17        &       7.28     &       -      &      0.11          &      8.74        &          -      &       0.20       &      6.81       &           -      \\
%History Taking (\textit{n}=5)         & 0.45           & 4.15    &  - & 0.29        & 6.48   &     -   & 0.30         & 6.04      &   -  \\
KR (\textit{n}=5)  &      0.39      & 5.03   &  -  &  0.34  &  6.86       & -  &     0.29   &   5.86 &    -   \\
DS (\textit{n}=5)  &     0.50      &  2.89 &  -  & 0.24 &    7.33   & -  &   0.23    &  5.77  &    -   \\\cmidrule(lr){2-10}
%History Taking (\textit{n}=10)        & 0.58           & 3.12     & - & 0.33        & 5.82    &    -   & 0.36         & 4.51    &    -   \\
%History Taking (\textit{n}=15)        & 0.56           & 3.50      &- & 0.36        & \textbf{5.36}       &  -  & 0.31         & 4.80    &    -   \\\cmidrule(lr){2-10}
%Retrieval (Wiki) \textcolor{red}{rerun}                   &            &    &  -  &         &         & -  &          &    &    -   \\
%MEDDxAgent         &                &            &             &                &              &            &                &              &        \\
MEDDx (\textit{iter}=1, \textit{n}=5)                       & 0.61           & 2.91    & ~~0.00  & 0.29        & 7.05       & ~~0.00   & 0.39         & 5.05   &     ~~0.00   \\
MEDDx (\textit{iter}=2, \textit{n}=10)                       & \textbf{0.71}   & \textbf{2.20}      & +0.41 & 0.37        & \textbf{6.26}      & +0.07   & \textbf{0.48}         & 4.48  &    +0.75     \\
MEDDx (\textit{iter}=3, \textit{n}=15)                       & 0.68   & 2.30    & +0.17  & \textbf{0.42}        & 6.31     &   +0.26   & \textbf{0.48}         & \textbf{4.30}      &   +0.44  \\\midrule
                               & \multicolumn{9}{c}{\textbf{Llama3.1-8B}}                                                            \\\midrule
%\textcolor{cyan}{Zero-shot}                      &    0.45            &   9.00         &   -          &    0.27             &     7.02         &      -         &    0.33           &  5.45           &             -     \\
%\textcolor{cyan}{Zero-shot (CoT)}                      &    0.45            &   4.51         &   -          &    0.27             &     7.25         &      -         &    0.24           &  5.65           &             -     \\
%\textcolor{cyan}{Few-shot (Standard, Dyn\_BAII)} &                &            &     -        &                &              &        -      &              &             &             -      \\
%\textcolor{cyan}{Few-shot (CoT, Dyn\_BERT/Dyn\_BAII)}     &                &            &       -      &                &              &         -     &              &             &           -        \\
KR (\textit{n}=0)     &     0.20       &  7.49  &  -  &   0.11      &  \textbf{8.86}       & -  &     \textbf{0.11}     &  8.58  &    -   \\
DS (\textit{n}=0)  &       0.16         &       8.45     &       -      &      0.03          &      10.37        &          -      &         0.04     &       8.52      &           -      \\
%History Taking (\textit{n}=5)         & 0.23           & 6.85    & -  & 0.10        & 8.78        &  - & 0.05         & 8.38       &   - \\
KR (\textit{n}=5)  &      0.21      & 7.42   &  -  &  0.09  &  9.48       & -  &     0.04   &   9.69 &    -   \\
DS (\textit{n}=5)  &     0.23      &  5.77  &  -  &  0.03 &   10.08    & -  &   0.06    &  8.64  &    -   \\\cmidrule(lr){2-10}
%History Taking (\textit{n}=10)        & 0.35           & 5.46    &  - & 0.12        & 8.39     &   -   & \textbf{0.13}         & 8.25   &     -   \\
%History Taking (\textit{n}=15)        & 0.40           & 5.44   &  -  & 0.11        & \textbf{8.30}       &  -  & \textbf{0.11}        & 8.95       &  -  \\\cmidrule(lr){2-10}
%Retrieval (Wiki) \textcolor{red}{rerun}                   &            &    &  -  &         &         & -  &          &    &    -   \\
%MEDDxAgent         &                &            &             &                &              &               &                &              &     \\
MEDDx (\textit{iter}=1, \textit{n}=5)                       & 0.34           & 5.25   &   ~~0.00 & 0.11        & 9.38       &  ~~0.00  & 0.08         & 8.47    &    ~~0.00   \\
MEDDx (\textit{iter}=2, \textit{n}=10)                       & 0.56           & 3.59    & +1.73  & \textbf{0.14}        & 9.22       &  +0.22  & 0.09         & \textbf{8.11}      &  +0.44   \\
MEDDx (\textit{iter}=3, \textit{n}=15)                       & \textbf{0.58}           & \textbf{3.10}    &  +1.23 & 0.12        & 9.07     & +0.17     & 0.07         & 8.56    &  +0.38  \\  
\bottomrule
    \end{tabular}
    \vspace{-0.8em}
    \caption{Interactive experiment performance across 3 datasets without \textit{full} patient profile, with KR: knowledge retrieval agent; DS: diagnosis strategy agent; $n$ is the number of turns of the simulator; MEDDx uses KR+DS.
    %We compare the single-turn (\textit{upper}) with the proposed iterative setup for MEDDxAgent (\textit{bottom}). The selection of the agents and simulator are optimized (\autoref{subsec:optimize-agents}), unless controlled by the number of questions ($n$) asked from history taking simulator.\cc{May be we just need 3 entries of single turn for best agents and simulator compared to MEDDxAgent? For the others we leave it for ablation study?}} %\cl{why we don't have the baseline with diagnosis strategy only without full patient profile (e.g., few-shot CoT, Dyn\_BAII?)}
    }
    \label{tab:interactive_overall}
    \vspace{-1.8em}
\end{table*}

We experiment on two configurations: (1) optimizing individual agents (\autoref{subsec:optimize-agents}), by determining the best settings for knowledge retrieval and diagnosis strategy agents; and (2) interactive differential diagnosis (\autoref{subsec:iterative_learning}), where the optimized agents are used to assess MEDDxAgent's performance in the interactive DDx setup.

\subsection{Optimizing Individual Agents}
\label{subsec:optimize-agents}

We first explore the optimal single-turn configuration for the knowledge retrieval and diagnosis strategy agents, before integrating them into iterative setup. For this, we provide the full patient profile as in previous work~\cite{wu2024streambench,chen2024rarebench}, and present the results in~\autoref{tab:with_patient_profile}. For the knowledge retrieval agent, PubMed performs slightly better overall than Wikipedia, especially for Rarebench, which demands more complex disease information. For the diagnosis strategy agent, the best setting varies by dataset. 
Namely, dynamic few-shot with BAII embeddings performs the best on DDxPlus and RareBench, where relevant patient examples offer reliable contextual cues to likely diseases. 
In contrast, iCraft-MD benefits more from zero-shot CoT, which enables structured reasoning through complex clinical vignettes. Few-shot learning often decreases performance for iCraft-MD because each patient vignette is distinct, so additional examples can introduce noise.
Based on the above findings, we select the following configurations for the iterative scenario:\footnote{We do not run all possible settings in the interactive environment due to cost reasons.} PubMed for knowledge retrieval agent; few-shot (dynamic BAII) for DDxPlus and RareBench, and zero-shot (CoT) for iCraft-MD for diagnosis strategy agent.

\begin{figure*}[t]
    \centering
    \begin{subfigure}{0.48\textwidth}
    \includegraphics[trim={0.2cm 0cm 0cm 0cm },clip, width=\textwidth]{img/ddxplus_history.pdf}
    \vspace{-1.8em}
    \caption{}
    \end{subfigure}
    \begin{subfigure}{0.48\textwidth}
    \includegraphics[trim={0.2cm 0cm 0cm 0cm}, clip, width=\textwidth]{img/agent_iterations_plot_ddxplus.pdf}
    \vspace{-1.8em}
    \caption{}
    \end{subfigure}
    \vspace{-0.5em}
    \caption{Results of DDxPlus compared between (a) history taking simulator, and (b) MEDDxAgent, over the number of questions and iterations. For brevity, the results of iCraft-MD and RareBench are in~\autoref{subsec:comparison_history_taking_iterative}.}
    \label{fig:ddxplus_comparison}
    \vspace{-1.8em}
\end{figure*}

\subsection{Interactive Differential Diagnosis}
\label{subsec:interactive_differential_diagnosis}
We now evaluate the more challenging task of interactive DDx, where we begin with limited patient information and the history taking simulator enables the interactive environment~(\autoref{tab:interactive_overall}).
At $n=0$, the simulator has not yet learned any patient information, and performance drops significantly from observing the full patient profile (\autoref{tab:with_patient_profile}). 
For GPT-4o in RareBench, the knowledge retrieval agent (KR)'s GTPA@1 drops from 0.45  to 0.07. Similarly, the diagnosis strategy agent (DS) drops from 0.46 (zero-shot) to 0.11. This simple baseline showcases that previous evaluations do not hold well in the interactive setup with initially limited patient information. 
Already for $n=5$, we find a large boost in performance for both KR and DS. These findings reinforce the importance of history taking for diagnostic precision. 
We illustrate the trend for changing $n$ in~\autoref{fig:ddxplus_comparison} and find that gains also plateau around \textit{n}=10-15 questions, reinforcing the optimal balance between information gathering and diagnostic efficiency \cite{ely1999analysis}.

Finally, we run MEDDxAgent, which calls KR+DS in the \textit{fixed iteration} pipeline (\autoref{subsec:iterative_learning}). MEDDxAgent exhibits clear improvements over the KR and DS baselines for $n=5$, supporting our hypothesis that all three modules are important for interactive DDx. It also improves significantly over the history taking baselines, as we illustrate in \autoref{fig:ddxplus_comparison}. MEDDxAgent is also capable of improving upon the zero-shot setting with the full patient profile (\autoref{tab:with_patient_profile}). For DDxPlus, GTPA@1 for GPT-4o and Llama3.1-70B rise from 0.56 to 0.86 and from 0.46 to 0.71, respectively. For Llama3.1-8B, the trend continues for DDxPlus but inconsistently for iCraft-MD and RareBench, highlighting the importance of model scale. Notably, MEDDxAgent improves over successive iterations, though the optimal number of iterations (2, 3) depends on the dataset and LLM. The values of $\Delta$ are consistently positive, indicating that MEDDxAgent iteratively increases the rank of the ground-truth diagnosis over time. $\Delta$ Progress also varies by dataset and model, offering explainable insight to the diagnosistic improvement of MEDDxAgent. The overall results show that MEDDxAgent can operate well in the challenging, realistic setup of interactive DDx. Additionally, MEDDxAgent logs all intermediate reasoning, action, and observations, providing critical insight into its DDx process (\autoref{fig:Example}).
\vspace{-0.5em}
\section{Conclusions and Future Work}
\label{sec:conclusions}
%\nabanita{Nabanita: ADD Percentage increase and other comparisons as necessary!}
Embodied agents assisting humans frequently have to complete previously unseen tasks or operate in new scenario. This paper describes a framework that leverages the complementary strengths of Large Language Models (LLMs), Knowledge Graphs (KGs), and Human-in-the-Loop (HITL) feedback to satisfy this requirement. Specifically, the generic task decomposition ability of LLMs is used to predict a sequence of abstract actions to complete any given task. This sequence is adapted to the specific scenario(s) and the task-, agent-, or domain-specific constraints using a KG that encodes prior knowledge of some objects, object attributes, and action capabilities. Any unresolved mismatch between the KG and the LLM output, and any unexpected action outcomes, are addressed by soliciting and using human input. This HITL feedback corrects errors and refines the existing knowledge (in the KG) for subsequent operation. Experimental evaluation in two simulated domains demonstrates substantial performance improvement compared with baselines, and illustrates incremental acquisition of knowledge to adapt to new classes of tasks.

%Our framework successfully merges the unique capabilities of Large Language Models (LLMs), Knowledge Graphs (KGs), and human feedback to improve how embodied agents respond to unfamiliar situations. LLMs provide a foundation by generating high-level action plans, which are then refined with detailed, domain-specific knowledge from KGs. This allows agents to adjust on the fly, even when encountering new objects or tasks. Additionally, incorporating human-in-the-loop (HITL) feedback offers real-time updates and fine-tuning, ensuring the agents continually evolve and expand their understanding. This strategy is particularly useful in everyday activities like cooking or cleaning, where agents can quickly understand and execute tasks without requiring extensive retraining. By combining LLMs, KGs, and human insights, the system ensures that agents remain adaptable and efficient, even in unpredictable environments.

% paves the way for exciting future developments, such as expanding the framework to handle a wider variety of tasks and environments, fine-tuning the balance between automation and human input, and exploring ways to make knowledge refinement more autonomous. Ultimately, our approach marks a significant step toward building smarter, more adaptive, and human-centered assistive agents.
\vspace{-0.75em}
This research opens up multiple avenues for further research. First, we will explore the use of this framework in many more classes of tasks, building on (and reinforcing) the promising results obtained so far.  Second, we will investigate the trade-off between automating the generation of an action sequence for any given task, and soliciting and incorporating human feedback as needed. Furthermore, we will explore the use of this framework on a physical robot platform assisting humans. The long-term objective is to create assistive agents and robots that can interact and collaborate with humans in different application domains.
\balance

\bibliographystyle{IEEEtran}
% \bibliography{IEEEabrv,references}
% Generated by IEEEtran.bst, version: 1.14 (2015/08/26)
\begin{thebibliography}{10}
\providecommand{\url}[1]{#1}
\csname url@samestyle\endcsname
\providecommand{\newblock}{\relax}
\providecommand{\bibinfo}[2]{#2}
\providecommand{\BIBentrySTDinterwordspacing}{\spaceskip=0pt\relax}
\providecommand{\BIBentryALTinterwordstretchfactor}{4}
\providecommand{\BIBentryALTinterwordspacing}{\spaceskip=\fontdimen2\font plus
\BIBentryALTinterwordstretchfactor\fontdimen3\font minus \fontdimen4\font\relax}
\providecommand{\BIBforeignlanguage}[2]{{%
\expandafter\ifx\csname l@#1\endcsname\relax
\typeout{** WARNING: IEEEtran.bst: No hyphenation pattern has been}%
\typeout{** loaded for the language `#1'. Using the pattern for}%
\typeout{** the default language instead.}%
\else
\language=\csname l@#1\endcsname
\fi
#2}}
\providecommand{\BIBdecl}{\relax}
\BIBdecl

\bibitem{coppeliaSim}
E.~Rohmer \emph{et~al.}, ``Coppeliasim (formerly v-rep): a versatile and scalable robot simulation framework,'' in \emph{Proc. of The International Conference on Intelligent Robots and Systems (IROS)}, 2013, www.coppeliarobotics.com.

\bibitem{Puig_2018_CVPR}
X.~Puig \emph{et~al.}, ``Virtualhome: Simulating household activities via programs,'' in \emph{Proceedings of the IEEE Conference on Computer Vision and Pattern Recognition (CVPR)}, June 2018.

\bibitem{kolve2022ai2thorinteractive3denvironment}
\BIBentryALTinterwordspacing
E.~Kolve \emph{et~al.}, ``Ai2-thor: An interactive 3d environment for visual ai,'' 2022. [Online]. Available: \url{https://arxiv.org/abs/1712.05474}
\BIBentrySTDinterwordspacing

\bibitem{khot2023decomposedpromptingmodularapproach}
\BIBentryALTinterwordspacing
T.~Khot \emph{et~al.}, ``Decomposed prompting: A modular approach for solving complex tasks,'' 2023. [Online]. Available: \url{https://arxiv.org/abs/2210.02406}
\BIBentrySTDinterwordspacing

\bibitem{reppert2023iterateddecompositionimprovingscience}
\BIBentryALTinterwordspacing
J.~Reppert \emph{et~al.}, ``Iterated decomposition: Improving science q\&a by supervising reasoning processes,'' 2023. [Online]. Available: \url{https://arxiv.org/abs/2301.01751}
\BIBentrySTDinterwordspacing

\bibitem{liu2024deltadecomposedefficientlongterm}
\BIBentryALTinterwordspacing
Y.~Liu \emph{et~al.}, ``Delta: Decomposed efficient long-term robot task planning using large language models,'' 2024. [Online]. Available: \url{https://arxiv.org/abs/2404.03275}
\BIBentrySTDinterwordspacing

\bibitem{sakib2022approximate}
M.~S. Sakib \emph{et~al.}, ``Approximate task tree retrieval in a knowledge network for robotic cooking,'' \emph{IEEE Robotics and Automation Letters}, vol.~7, no.~4, pp. 11\,492--11\,499, 2022.

\bibitem{sakib2024cooking}
M.~S. Sakib and Y.~Sun, ``From cooking recipes to robot task trees--improving planning correctness and task efficiency by leveraging llms with a knowledge network,'' in \emph{2024 IEEE International Conference on Robotics and Automation (ICRA)}.\hskip 1em plus 0.5em minus 0.4em\relax IEEE, 2024, pp. 12\,704--12\,711.

\bibitem{openai2024gpt4technicalreport}
\BIBentryALTinterwordspacing
OpenAI \emph{et~al.}, ``Gpt-4 technical report,'' 2024. [Online]. Available: \url{https://arxiv.org/abs/2303.08774}
\BIBentrySTDinterwordspacing

\bibitem{gemmateam2024gemma2improvingopen}
\BIBentryALTinterwordspacing
G.~Team \emph{et~al.}, ``Gemma 2: Improving open language models at a practical size,'' 2024. [Online]. Available: \url{https://arxiv.org/abs/2408.00118}
\BIBentrySTDinterwordspacing

\bibitem{dubey2024llama3herdmodels}
\BIBentryALTinterwordspacing
Dubey \emph{et~al.}, ``The llama 3 herd of models,'' 2024. [Online]. Available: \url{https://arxiv.org/abs/2407.21783}
\BIBentrySTDinterwordspacing

\bibitem{wen2024learning}
J.~Wen \emph{et~al.}, ``Learning task decomposition to assist humans in competitive programming,'' \emph{arXiv preprint arXiv:2406.04604}, 2024.

\bibitem{li2023semantically}
W.~Li \emph{et~al.}, ``Semantically aligned task decomposition in multi-agent reinforcement learning,'' \emph{arXiv preprint arXiv:2305.10865}, 2023.

\bibitem{dery2021auxiliarytaskupdatedecomposition}
\BIBentryALTinterwordspacing
L.~M. Dery \emph{et~al.}, ``Auxiliary task update decomposition: The good, the bad and the neutral,'' 2021. [Online]. Available: \url{https://arxiv.org/abs/2108.11346}
\BIBentrySTDinterwordspacing

\bibitem{shen2023taskbench}
Y.~Shen \emph{et~al.}, ``Taskbench: Benchmarking large language models for task automation,'' \emph{arXiv preprint arXiv:2311.18760}, 2023.

\bibitem{wang2024tdag}
Y.~Wang \emph{et~al.}, ``Tdag: A multi-agent framework based on dynamic task decomposition and agent generation,'' \emph{arXiv preprint arXiv:2402.10178}, 2024.

\bibitem{cui2021semantic}
G.~Cui, W.~Shuai, and X.~Chen, ``Semantic task planning for service robots in open worlds,'' \emph{Future Internet}, vol.~13, no.~2, p.~49, 2021.

\bibitem{prasad2023adapt}
A.~Prasad \emph{et~al.}, ``Adapt: As-needed decomposition and planning with language models,'' \emph{arXiv preprint arXiv:2311.05772}, 2023.

\bibitem{article1}
D.~Zheng \emph{et~al.}, ``A knowledge-based task planning approach for robot multi-task manipulation,'' \emph{Complex \& Intelligent Systems}, vol.~10, 07 2023.

\bibitem{10.1145/3297280.3297568}
\BIBentryALTinterwordspacing
A.~Kattepur and B.~P, ``Roboplanner: autonomous robotic action planning via knowledge graph queries,'' in \emph{Proceedings of the 34th ACM/SIGAPP Symposium on Applied Computing}, ser. SAC '19.\hskip 1em plus 0.5em minus 0.4em\relax New York, NY, USA: Association for Computing Machinery, 2019, p. 953–956. [Online]. Available: \url{https://doi.org/10.1145/3297280.3297568}
\BIBentrySTDinterwordspacing

\bibitem{9561782}
A.~Daruna \emph{et~al.}, ``Towards robust one-shot task execution using knowledge graph embeddings,'' in \emph{2021 IEEE International Conference on Robotics and Automation (ICRA)}, 2021, pp. 11\,118--11\,124.

\bibitem{aburasheed2024knowledgegraphscontextsources}
\BIBentryALTinterwordspacing
H.~Abu-Rasheed \emph{et~al.}, ``Knowledge graphs as context sources for llm-based explanations of learning recommendations,'' 2024. [Online]. Available: \url{https://arxiv.org/abs/2403.03008}
\BIBentrySTDinterwordspacing

\bibitem{pan2024unifying}
S.~Pan \emph{et~al.}, ``Unifying large language models and knowledge graphs: A roadmap,'' \emph{IEEE Transactions on Knowledge and Data Engineering}, 2024.

\bibitem{kuang2024openfmnavopensetzeroshotobject}
\BIBentryALTinterwordspacing
Y.~Kuang \emph{et~al.}, ``Openfmnav: Towards open-set zero-shot object navigation via vision-language foundation models,'' 2024. [Online]. Available: \url{https://arxiv.org/abs/2402.10670}
\BIBentrySTDinterwordspacing

\bibitem{paulius2016functional}
D.~Paulius \emph{et~al.}, ``Functional object-oriented network for manipulation learning,'' in \emph{2016 IEEE/RSJ International Conference on Intelligent Robots and Systems (IROS)}.\hskip 1em plus 0.5em minus 0.4em\relax IEEE, 2016, pp. 2655--2662.

\bibitem{ding2022robottaskplanningsituation}
\BIBentryALTinterwordspacing
Y.~Ding \emph{et~al.}, ``Robot task planning and situation handling in open worlds,'' 2022. [Online]. Available: \url{https://arxiv.org/abs/2210.01287}
\BIBentrySTDinterwordspacing

\bibitem{bhat2024groundingllmsrobottask}
\BIBentryALTinterwordspacing
V.~Bhat \emph{et~al.}, ``Grounding llms for robot task planning using closed-loop state feedback,'' 2024. [Online]. Available: \url{https://arxiv.org/abs/2402.08546}
\BIBentrySTDinterwordspacing

\bibitem{jiang2019task}
Y.-q. Jiang \emph{et~al.}, ``Task planning in robotics: an empirical comparison of pddl-and asp-based systems,'' \emph{Frontiers of Information Technology \& Electronic Engineering}, vol.~20, pp. 363--373, 2019.

\bibitem{marin2021recipe1m+}
J.~Mar{\i}n \emph{et~al.}, ``Recipe1m+: A dataset for learning cross-modal embeddings for cooking recipes and food images,'' \emph{IEEE Transactions on Pattern Analysis and Machine Intelligence}, vol.~43, no.~1, pp. 187--203, 2021.

\bibitem{marzari2021towards}
L.~Marzari \emph{et~al.}, ``Towards hierarchical task decomposition using deep reinforcement learning for pick and place subtasks,'' in \emph{2021 20th International Conference on Advanced Robotics (ICAR)}.\hskip 1em plus 0.5em minus 0.4em\relax IEEE, 2021, pp. 640--645.

\bibitem{zhen2023robottaskplanningbased}
\BIBentryALTinterwordspacing
Y.~Zhen \emph{et~al.}, ``Robot task planning based on large language model representing knowledge with directed graph structures,'' 2023. [Online]. Available: \url{https://arxiv.org/abs/2306.05171}
\BIBentrySTDinterwordspacing

\bibitem{shinn2024reflexion}
N.~Shinn \emph{et~al.}, ``Reflexion: Language agents with verbal reinforcement learning,'' \emph{Advances in Neural Information Processing Systems}, vol.~36, 2024.

\bibitem{holler2020hddl}
D.~H{\"o}ller \emph{et~al.}, ``Hddl: An extension to pddl for expressing hierarchical planning problems,'' in \emph{Proceedings of the AAAI conference on artificial intelligence}, vol.~34, no.~06, 2020, pp. 9883--9891.

\bibitem{DING2019105}
\BIBentryALTinterwordspacing
Y.~Ding \emph{et~al.}, ``Robotic task oriented knowledge graph for human-robot collaboration in disassembly,'' \emph{Procedia CIRP}, vol.~83, pp. 105--110, 2019, 11th CIRP Conference on Industrial Product-Service Systems. [Online]. Available: \url{https://www.sciencedirect.com/science/article/pii/S2212827119304263}
\BIBentrySTDinterwordspacing

\bibitem{article}
J.~Bai \emph{et~al.}, ``A dynamic knowledge graph approach to distributed self-driving laboratories,'' \emph{Nature Communications}, vol.~15, 01 2024.

\bibitem{kasaei2024vitalvisualteleoperationenhance}
\BIBentryALTinterwordspacing
H.~Kasaei and M.~Kasaei, ``Vital: Visual teleoperation to enhance robot learning through human-in-the-loop corrections,'' 2024. [Online]. Available: \url{https://arxiv.org/abs/2407.21244}
\BIBentrySTDinterwordspacing

\bibitem{liu2023robotlearningjobhumanintheloop}
\BIBentryALTinterwordspacing
H.~Liu, S.~Nasiriany, L.~Zhang, Z.~Bao, and Y.~Zhu, ``Robot learning on the job: Human-in-the-loop autonomy and learning during deployment,'' 2023. [Online]. Available: \url{https://arxiv.org/abs/2211.08416}
\BIBentrySTDinterwordspacing

\bibitem{9044335}
M.~Raessa \emph{et~al.}, ``Human-in-the-loop robotic manipulation planning for collaborative assembly,'' \emph{IEEE Transactions on Automation Science and Engineering}, vol.~17, no.~4, pp. 1800--1813, 2020.

\bibitem{emami2024human}
Y.~Emami, K.~Li, L.~Almeida, W.~Ni, and Z.~Han, ``Human-in-the-loop machine learning for safe and ethical autonomous vehicles: Principles, challenges, and opportunities,'' \emph{arXiv preprint arXiv:2408.12548}, 2024.

\bibitem{wu2022survey}
X.~Wu, L.~Xiao, Y.~Sun, J.~Zhang, T.~Ma, and L.~He, ``A survey of human-in-the-loop for machine learning,'' \emph{Future Generation Computer Systems}, vol. 135, pp. 364--381, 2022.

\end{thebibliography}
\end{document}


