\section{Related Work}
Hallucination of LLMs is a very active area of research and there are many useful survey papers on the subject matter~\cite{ji2023survey, zhang2023siren, rawte2023survey, huang2023survey, tonmoy2024comprehensive}.
Among the various aspects of hallucination, we would like to concentrate on the hallucination caused by exposure bias and its mitigation strategy.

Exposure bias~\cite{bengio2015scheduled, ranzato2016sequence} is a major source of hallucination in the pre-training stage~\cite{wang2020exposure}. 
LLMs have different procedures in training and inference. 
While training, the model learns the probability distribution between words (tokens) by a teacher-forced maximum likelihood estimation (MLE).
That is, regardless of the predicted token, the next token used in the following step is determined from the ground truth example. 
In contrast, during the autoregressive inference, the actual predicted (generated) token is included in the context of the next step.
This discrepancy between the training and inference is referred to the exposure bias. 

\citet{wang2023progressive} address hallucination in neural machine translation in the context of exposure bias. 
They demonstrate that introducing intermediate supervision signals from the source-like to target-like structure alleviates exposure bias and thus reduces hallucination. 
Other studies devoted to mitigating exposure bias include methods such as reinforcement learning~\cite{chen2019reinforcement}, unlikelihood training~\cite{welleck2019neural}, and minimum Bayes risk~\cite{bertsch2023s}, although they do not directly address the issue of hallucination.

One particularly important study on exposure bias mitigation related to our work is BRIO~\cite{liu2022brio}.
BRIO is an abstractive summarization model, whose goal is to maximize the ROUGE score by reducing exposure bias through contrastive learning.
The method was successful in the sense that it had achieved the state-of-the-art ROUGE score at the time. 
Since our proposed method is closely related to BRIO, we will present its mechanism in detail in the following section. 

Many other hallucination reduction strategies exists, which do not involve exposure bias, in all stages of the LLMs including dataset preparation, training, and inference.
We refer the readers to the survey papers for more information.