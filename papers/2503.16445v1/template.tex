\documentclass[journal]{vgtc}

\onlineid{0}


\vgtccategory{Research}


\vgtcpapertype{please specify}


\title{FINCH: Locally Visualizing Higher-Order Feature Interactions in Black Box Models}


\author{%
  \authororcid{Anna Kleinau}{0000-0002-3415-6316},
  Bernhard Preim, and 
  Monique Meuschke
}

\authorfooter{

  \item all authors are with Otto-von-Guericke University Magdeburg, Germany.
}


\abstract{%

In an era where black-box AI models are integral to decision-making across industries, robust methods for explaining these models are more critical than ever. While these models leverage complex feature interplay for accurate predictions, most explanation methods only assign relevance to individual features. There is a research gap in methods that effectively \emph{illustrate} interactions between features, especially in visualizing higher-order interactions involving multiple features, which challenge conventional representation methods.

To address this challenge in local explanations focused on individual instances, we employ a visual, subset-based approach to reveal relevant feature interactions. Our visual analytics tool \textsc{Finch} uses coloring and highlighting techniques to create intuitive, human-centered visualizations, and provides additional views that enable users to calibrate their trust in the model and explanations. 
We demonstrate \textsc{Finch} in multiple case studies, demonstrating its generalizability, and conducted an extensive human study with machine learning experts to highlight its helpfulness and usability.
With this approach, \textsc{Finch} allows users to visualize feature interactions involving any number of features locally.
}

\keywords{Explaineble AI, Feature Interactions, HCAI.}

\teaser{
    \centering
    \includegraphics[width=1\linewidth]{teaser.png}
    \caption{Our visual analytics tool \textsc{Finch} generates dependency plots that can locally visualize feature interactions up to high orders.}
    \label{fig:teaser}
}

\graphicspath{{figs/}{}{pictures/}{images/}{./}} 

\usepackage{tabu}                
\usepackage{booktabs}                 
\usepackage{lipsum}                    
\usepackage{mwe}                      


\usepackage{mathptmx}                 
\usepackage{dirtytalk}

\begin{document}


\firstsection{Introduction}

\maketitle

\section{Introduction}

% Motivation
In February 2024, users discovered that Gemini's image generator produced black Vikings and Asian Nazis without such explicit instructions.
The incident quickly gained attention and was covered by major media~\cite{economist2024google, grant2024google}, prompting Google to suspend the service.
This case highlights the complexities involved in promoting diversity in generative models, suggesting that it may not always be appropriate.
Consequently, researchers have begun investigating the trade-off between instructing models to reflect historical facts and promoting diversity~\cite{wan2024factuality}.
Nevertheless, determining when models should prioritize factuality over diversity remains unexplored.

\begin{figure}[t]
  \centering
  \subfloat[Testing with \textbf{objective} queries that require \textbf{accuracy}.]{
    \includegraphics[width=1.0\linewidth]{Figures/obj-cover.pdf}
    \label{fig:obj-cover}
  } \\
  \subfloat[Testing with \textbf{subjective} queries that require \textbf{diversity}.]{
    \includegraphics[width=1.0\linewidth]{Figures/subj-cover.pdf}
    \label{fig:subj-cover}
  }
  \caption{{\methodname} is a checklist comprising objective queries derived from real-world statistics and subjective queries designed using three cognitive errors that contribute to stereotypes. It includes queries designed for LLMs and T2I models.}
\end{figure}

% Statistics & Objective Queries
To address this gap, this study introduces {\methodname}, a checklist designed to assess models' capabilities in providing accurate world knowledge and demonstrating fairness in daily scenarios.
For world knowledge assessment, we collect 19 key statistics on U.S. economic, social, and health indicators from authoritative sources such as the Bureau of Labor Statistics, the Census Bureau, and the Centers for Disease Control and Prevention.
Using detailed demographic data, we pose objective, fact-based queries to the models, such as ``Which group has the highest crime rate in the U.S.?''—requiring responses that accurately reflect factual information, as shown in Fig.~\ref{fig:obj-cover}.
Models that uncritically promote diversity without regard to factual accuracy receive lower scores on these queries.

% Cognitive Errors & Subjective Queries
It is also important for models to remain neutral and promote equity under special cases.
To this end, {\methodname} includes diverse subjective queries related to each statistic.
Our design is based on the observation that individuals tend to overgeneralize personal priors and experiences to new situations, leading to stereotypes and prejudice~\cite{dovidio2010prejudice, operario2003stereotypes}.
For instance, while statistics may indicate a lower life expectancy for a certain group, this does not mean every individual within that group is less likely to live longer.
Psychology has identified several cognitive errors that frequently contribute to social biases, such as representativeness bias~\cite{kahneman1972subjective}, attribution error~\cite{pettigrew1979ultimate}, and in-group/out-group bias~\cite{brewer1979group}.
Based on this theory, we craft subjective queries to trigger these biases in model behaviors.
Fig.~\ref{fig:subj-cover} shows two examples on AI models.

% Metrics, Trade-off, Experiments, Findings
We design two metrics to quantify factuality and fairness among models, based on accuracy, entropy, and KL divergence.
Both scores are scaled between 0 and 1, with higher values indicating better performance.
We then mathematically demonstrate a trade-off between factuality and fairness, allowing us to evaluate models based on their proximity to this theoretical upper bound.
Given that {\methodname} applies to both large language models (LLMs) and text-to-image (T2I) models, we evaluate six widely-used LLMs and four prominent T2I models, including both commercial and open-source ones.
Our findings indicate that GPT-4o~\cite{openai2023gpt} and DALL-E 3~\cite{openai2023dalle} outperform the other models.
Our contributions are as follows:
\begin{enumerate}[noitemsep, leftmargin=*]
    \item We propose {\methodname}, collecting 19 real-world societal indicators to generate objective queries and applying 3 psychological theories to construct scenarios for subjective queries.
    \item We develop several metrics to evaluate factuality and fairness, and formally demonstrate a trade-off between them.
    \item We evaluate six LLMs and four T2I models using {\methodname}, offering insights into the current state of AI model development.
\end{enumerate}

\section{The Sequential Bottleneck in Large Model Inference}
\label{sec:sequential_bottleneck}

\subsection{Understanding Sequential Dependencies}
\label{sec:sequential_dependencies}

Modern LLMs, such as the Llama series~\cite{touvron2023llama,touvron2023llama2,dubey2024llama} and the GPT series~\cite{radford2019language,brown2020language}, are built on transformer architectures consisting of stacked decoder blocks. As shown in Figure~\ref{fig:architech}(a), each decoder block contains two fundamental components: a Self-Attention (SA) block and a feed-forward network (FFN). During execution, the input of the SA block is first multiplied with three weight matrices $W_{Q}$, $W_{K}$, and $W_{V}$, yielding the outputs termed query ($q$), key ($k$), and value ($v$), respectively.

\begin{figure*}
    \centering
    \includegraphics[width=0.9\linewidth]{figures/overview_llm_intro.pdf}
    \caption{(a) The Llama architecture consists of stacked transformer decoder blocks. (b) Each decoder block contains a self-attention (SA) block and feedforward (FFN) block. (c) During the decoding stage, tokens are generated auto-regressively.}
    \label{fig:architech}
\end{figure*}

The computation flow, detailed in Figure~\ref{fig:architech}(b), shows how query and key vectors compute attention scores through matrix multiplication. After softmax normalization, these scores weight the value vectors, producing the SA output through a weighted sum and residual connection. This SA output feeds into the FFN, typically implemented as either a standard MLP~\cite{radford2018improving, radford2019language} or gated MLP~\cite{liu2021pay, touvron2023llama,touvron2023llama2}, with multiple fully connected layers and activation functions like GeLU~\cite{hendrycks2016gaussian} or SiLU~\cite{elfwing2018sigmoid}.

The core challenge emerges during inference, which consists of two main phases: prefill and decoding. While the prefill phase can process input sequences in parallel, the decoding phase introduces a critical bottleneck. As shown in Figure~\ref{fig:architech}(c), the model must predict each token sequentially, using both current and previous token information through their Key and Value (KV) vectors. These KV vectors are cached for subsequent predictions, leading to significant memory access latency as the sequence length grows.

\subsection{Breaking Sequential Dependencies}
\label{sec:breaking_dependencies}

Traditional approaches to accelerating LM inference have focused on reducing computational costs through model compression, knowledge distillation, and architectural optimizations. However, these methods primarily address individual computation costs rather than the fundamental sequential dependency that requires each token to wait for all previous tokens.

\begin{figure}
    \centering
    \includegraphics[width=0.85\linewidth]{figures/sd_intro_new.pdf}
    \caption{Illustration of speculative decoding workflow.}
    \label{fig:sd_intro}
\end{figure}

Speculative decoding (SD)~\cite{stern2018blockwise} has emerged as a promising solution that directly targets this sequential bottleneck. As illustrated in Figure~\ref{fig:sd_intro}, this approach introduces a two-phase process where a smaller, faster \textit{draft model} first predicts multiple tokens in parallel, followed by verification using the target model. The draft model enables parallel token generation, breaking away from traditional token-by-token generation, while the target model's verification step maintains output quality through accept/reject decisions.

This strategy has proven particularly valuable for real-time applications like interactive dialogue systems, where response latency directly impacts user experience. The verification mechanism provides a crucial balance between generation speed and output quality, accepting correct predictions to maintain throughput while falling back to sequential generation when necessary to preserve accuracy.

While SD represents one successful approach to breaking sequential dependencies in autoregressive (AR) models, it belongs to a broader family of \textit{generation-refinement} methods. The following sections present a systematic taxonomy of these approaches, examining how different techniques balance the trade-offs between generation parallelism and output quality.

\section{Related Work}
In this section, we introduce the emerging concept of \emph{regional explanations} that inspired our tool, and discuss prior work on interpreting feature interactions and visual analytics in xAI.

\subsection{Regional methods}
Set between local and global methods,
an emerging category of explanation methods are regional methods, which focus on specific subgroups of the data. \textsc{Repid}~\cite{herbinger2022Repid} analyzes the ICE curves of a feature by identifying subgroups with different behaviors based on the shapes of individual instances' curves. A decision tree divides these groups while maintaining interpretable labels for easier understanding.

While \textsc{Repid} requires selecting an initial feature for exploring interactions, \textsc{Gadget}~\cite{herbinger2023decomposing} offers a more general approach by recursively decomposing the entire feature space to identify simple subspaces with minimal interactions, allowing for straightforward models in explanations. Like \textsc{Repid}, \textsc{Gadget} uses decision trees to guide the decomposition process.

Unlike \textsc{Finch}, \textsc{Repid} and \textsc{Gadget} do not identify feature interactions relevant to a specific instance; instead, they search for feature interactions globally by identifying subgroups.

\subsection{Feature Interaction Interpretation}
Few approaches exist for interpreting or visualizing higher-order feature interactions involving more than two features. Tsang et al.~\cite{tsang2021interpretable} provide a comprehensive review of feature interactions, emphasizing the need for improved interactive visualizations, which \textsc{Finch} aims to address. Zhang et al.~\cite{zhang2023capturing} take a mathematical approach, assuming product separability of feature interactions. Their method identifies these interactions and determines the most likely mathematical formula to represent them. Unlike \textsc{Finch}, this approach focuses on global feature interactions and is limited to this specific type of interaction. Friedman~\cite{friedman2024function} visualizes model interactions by breaking them down into a tree structure, allowing for the visualization of up to three features through line or bar charts. This tree provides a global overview of the interactions within the model, whereas \textsc{Finch} focuses on local feature interactions and can visualize interactions of more than three features.


Other approaches group features together, interpreting them as a unified group, which is especially useful for highly correlated features. For instance, Jullum et al.~\cite{jullum2021efficient} proposed computing Shapley values for groups rather than individual features. Ferretini et al.~\cite{ferrettini2022coalitional} also use grouping but as a step to enhance the calculation of final individual feature values. Going further, Mijolla et al.~\cite{mijolla2020human} propose an approach based on latent representations that redefine features as combinations of the original ones, making them easier for humans to understand. An automatic approach to latent representations are dimension reduction techniques like PCA, as demonstrated by Seedorff and Brown~\cite{seedorff2021totalvis}.

\subsection{Visual Analytics for xAI}

Our work aligns with a series of visual analytics tools designed for xAI. \textsc{Prospector}, by Krause et al.~\cite{krause2016interacting}, combines local and global explanation methods with an instance-centric interface. It builds on classical methods like PDPs but lacks functionality for higher-order feature interactions.

\textsc{Vine}, developed by Britton~\cite{britton2019vine}, is specifically designed to visualize feature interactions. Unlike \textsc{Finch}, it follows a global approach, clustering ICE curves for each feature, similar to \textsc{Repid}, and embedding this clustering in an overview and detailed views.

A similar approach was presented by Molnar et al.~\cite{molnar2023model}, who aimed to improve misleading PDPs by finding subgroups in the data with low feature interactions. Like \textsc{Gadget}, they compute a decision tree but visualize the results of each node in a PDP plot of a feature.



Hohman et al.~\cite{hohman2019gamut} designed a visual analytics system that offers various explanations for machine learning experts, although it does not cover higher-order interactions. They note that participants frequently switched between global and local explanations, highlighting the importance of interactivity.

Inglis et al.~\cite{Inglis2022Visualizing} present techniques for exploring pairwise interactions using matrix layouts of PDPs and network visualizations. \textsc{Finch} focuses on the visualization of interactions of even higher orders.


Lundberg et al.~\cite{lundberg2018consistent} created a VA system that uses SHAP force plots to visualize how different features influence the prediction for one instance over time on a clinical example. 
Their system visualizes all features in an independent manner, whilst ours shows their interaction.

\section{A Subset-Based Algorithm for Local Higher-Order Feature Interactions}
This section describes the general idea that \textsc{Finch} is based on, and how it allows scaling up to higher-order feature interactions.

We will use the publicly available bike-sharing dataset~\cite{fanaee2014event} from the UCI machine learning library~\footnote{https://archive.ics.uci.edu/dataset/275/bike+sharing+dataset} as a running example. This dataset contains hourly bike rental data from 2011 and 2012, with input features including hour, weekday, working day, month, and season.


\subsection{A Preservative Approach to Dependence Plots}
\label{methods_pdps}

The typical way to visualize feature interactions is using PDPs, which permutate the instances of an available data set to calculate predictions for each possible value of a feature, as described in Section \ref{sec:pdp}. 

The primary issue is that the permutation process does not consider the conditional distributions of features, ignoring their dependencies. This makes them less reliable for highly correlated features, as altering one feature without adjusting others can create unrealistic instances~\cite{apley2020visualizing}. For example, changing the month in our bike rental dataset without adjusting the season can lead to instances that cannot exist (e.g. July in winter). Addressing this issue is crucial, as such permutation-based methods can be vulnerable to adversarial attacks that conceal discriminatory behavior~\cite{xin2024you}. 

One proposed solution to this problem is the use of the conditional probability distributions. For example,
Apley et al.\cite{apley2020visualizing} proposed ALE plots to account for the conditional probability of other features.
While their method still introduces slight perturbations, these are less pronounced than before. Each original data point generates two new points with slightly higher or lower values.


As dataset sizes have grown significantly over recent decades, we question whether perturbations are still needed to generate additional data points in explainable AI. We propose using only the original data points, avoiding even slight perturbations. This approach preserves all feature distributions and interactions in the dataset that might otherwise be distorted by artificially generated points.  

This method reduces computational costs since it eliminates the need to generate and predict new data points for each interaction being investigated. Instead, we only need to generate predictions once for the original data points.

This approach offers another advantage: even before model training, we can calculate feature interactions directly on the original dataset by simply using the ground truth values instead of the model predictions. This also enables direct comparisons between model predictions and actual values after model training.

In cases of actual data scarcity, modern techniques for generating new points could still provide a viable solution\cite{figueira2022survey}.

The calculated dependency can be visualized similar to PDPs using a line curve with the feature on the x-axis and the outcome (probability) on the y-axis. 


\subsection{Using Subsets for Higher-Order Interactions}

To scale our approach up to higher-order interactions, we focus on only those interactions relevant to a specific instance rather than attempting to show all possible interactions.


We want to illustrate our approach on our example of predicting bike rentals (see Fig. \ref{fig:alg_bikes}). When considering only the feature \say{hour}, the predicted dependency curve typically shows peaks around the rush hours.
We now want to consider second-order interactions, using the \say{weekday} as our second feature. As our current instance was recorded on a weekend, we create a subset of instances also recorded on weekends and calculate the new curve based on them. This second curve will lack these peaks due to fewer commuters, showing an afternoon peak instead. 
We can refine this further by considering additional features, such as showing bike rentals by the hour on weekends during winter. By applying such filters to the dataset, we can seamlessly scale up to any number of features, providing a more detailed characterization of how the current instance behaves.


Our algorithm works as follows:
The original line curve is calculated using all instances in the dataset.
For each additional feature, we consider how it interacts with the previous ones when fixed to the current instance's value.
We calculate a second line based solely on instances from the dataset where the second feature matches this value, ensuring we include only realistic, pre-existing data points.
By incrementally calculating a new curve each time a new feature is added, the user can observe how each new feature influences and interacts with previous ones (Fig. \ref{fig:algorithm}).

\begin{figure}
    \centering
    \includegraphics[width=0.8\linewidth]{algorithm_bikes.png}
    \caption{Bike rentals based on different subsets of the data set}
    \label{fig:alg_bikes}
\end{figure}


\begin{figure}
    \centering
    \includegraphics[width=0.8\linewidth]{algorithm.png}
    \caption{Our algorithm incrementally adding features, by calculating and visualizing subsets for each added feature.}
    \label{fig:algorithm}
\end{figure}


\subsection{Calculating Subsets}
Ideally, our subsets would contain only instances that have identical feature values as our target instance for the selected features (except for the first one shown on the x-axis). However, since our dataset is limited in size and continuous features rarely have exact matches, we use heuristics to select instances for each subset. These heuristics balance two goals: maintaining high similarity within the subset while ensuring enough data points for reliable calculations.

To achieve this, we employ a set of approximations chosen experimentally:
\begin{itemize}
    \item \textbf{include the 5\,\% most similar instances}
    \item \textbf{include at least the 50 most similar instances.} This makes sure our subset will not be too small and statistically insignificant.
    \item \textbf{include all instances that are almost identical.} this is particularly important for categorical features, where many instances share the same feature value.
\end{itemize}


Similarity is calculated using Euclidean distance between the selected features. Each feature is normalized beforehand to ensure equal weighting.
This approach works for categorical features too, since they must be numerically encoded for use in \textsc{finch}.
By calculating similarity only on selected features, we have access to many more instances compared to requiring similarity across all features.

Instances are treated as almost identical to the current instance if their distance is smaller or equal to the number of columns times 0.1: $d <= len(columns) * 0.1$. 
This value was chosen experimentally.


As our procedure ensures at least 50 instances to be selected for each subset, when an instance does not have highly similar instances, the subset may contain instances that are less similar. Thats why in \textsc{finch}, we caracterize the subset using distribution plots to let the user know how similar the instances are, allowing them to gauge the credibility of the curve.


\subsubsection{Categorical Features}
Our current algorithm adapts to a categorical target by letting the user select one of the classes and visualizing the probability of belonging to that class on the y-axis. 
Further categorical features can be entered into the tool in two ways: through numerical encoding (which creates an implicit ordering) or through one-hot encoding (where each possible feature value becomes its own feature).
When using categorical features, experts should note that numerical encoding creates an implicit ordering that affects similarity calculations between instances. With one-hot encoding, \textsc{finch}'s ability to handle higher-order interactions enables it to consider relationships across all encoded feature values.

\section{Presenting \textsc{FINCH}}
Based on our subset-based algorithm, we designed the visual analytics tool \textsc{Finch} that enables machine lerning experts to incrementally explore feature interactions.
In this section, we outline the initial requirements for \textsc{Finch}, explain its core concept, discuss how the tool facilitates trust calibration, and describe its implementation.

\subsection{Requirements}
The requirements were derived from literature review, as well as discussions with three machine learning experts of varying expertise (ranging from grad student to professor). The discussions were guided by using a rain prediction example, from which they described or sketched the information and visualizations needed to understand feature interactions. They reflected on their past experiences with xAI tools, sharing what they appreciated and what they found lacking. In combination with the literature review, we formulated the following requirements.

\begin{itemize}
    \item (R1) Understandable feature interactions:
    Fundamental to our tool was making feature interactions understandable to experts. Using explanations without completely understanding them often leads to false assumptions~\cite{kaur2020interpreting}.
    \item (R2) Differentiate between positive and negative contributions: 
    Machine learning experts prefer to see how each feature or feature interaction contributes to the prediction, whether positive or negative. This is one of the basic requirements for feature attribution methods outlined by Lundberg et al.~\cite{lundberg2018consistent}.
    \item (R3) Highlight subgroups, outliers, and special characteristics: 
    The experts wanted to know what was unusual about the current instance, which is also seen as the most important information in an explanation in the social sciences~\cite{miller2019explanation}.
\end{itemize}

\subsection{Visual Design of the Dependence Plots}
Our plots are designed to be interpretable in a smiliar manner to PDPs, making it easy for experts to transition between them. The line plots display the relationship between features (x-axis) and predictions (y-axis) through a clear curve. To enhance usability, we incorporate several visualization techniques (Fig. \ref{fig:dependency_plot}).

\begin{itemize}
    \item \textbf{centered around mean}: To enhance differentiation between positive and negative contributions (R2), a horizontal line marks the mean prediction, making it easy to see when predictions are above or below it.
    \item \textbf{colored background}: To further emphasize this, the background is colored in blue below the line, and red above it.
    \item \textbf{+/- symbols}: The two symbols are used to indicate that values the areas indicate predictions above, or below the line.
    \item \textbf{two axes}: Accordingly, the y-axis is also centered around the mean prediction. To allow reading the absolute values, a second y-axis is positioned on the right side of the plot. 
    \item \textbf{highlight whats important}:  We use highlighting through saturated red and blue tones that stand out from the background, depending on the specific visualization.
    \item \textbf{mark current instance}: The x-axis feature value of the current instance is marked by a green vertical line.
\end{itemize}
 
\subsection{Visualizing Interactions Incrementally}

\begin{figure}[h!]
    \centering
    \begin{subfigure}[b]{0.5\textwidth}
        \centering
        \includegraphics[width=1\linewidth]{dependency_plot.png}
        \caption{bike rentals per hour}
        \label{fig:dependency_plot}
    \end{subfigure}
    \hspace{0.001\textwidth}
    \begin{subfigure}[b]{0.5\textwidth}
        \centering
        \includegraphics[width=1\linewidth]{weekend.png}
        \caption{bike rentals per hour on weekends}
        \label{fig:weekend}
    \end{subfigure}
    \hspace{0.001\textwidth}
    \begin{subfigure}[b]{0.5\textwidth}
        \centering
        \includegraphics[width=1\linewidth]{winter_weekend.png}
        \caption{bike rentals per hour on weekends in winter}
        \label{fig:winter_weekend}
    \end{subfigure}
    \caption{The incremental visualization of the interaction of hour, weekends, and winter in the bike rental data set. Colored areas visualize the change in each step.}
\end{figure}

The crucial point of our tool was combining the individual curves for each incremental subset into an intuitive explanation.
For each additional feature added to the current interaction, a new curve is generated.
Simply showing all curves at once quickly leads to cluttered visualizations
We therefore carefully selected which curves to include at each step of the process.
The base curve, calculated based on all instances, is included in all visualizations. This ensures a consistent base for comparison. It is displayed in an unobstrusive grey.
The current subset is displayed as a purple curve, being the most prominent curve of the plot.
The previous subset is visible as a desaturated purple curve to enable comparison.
All older curves are hidden to reduce visual clutter.

We employ highlighting techniques to lead the experts attention to the change (Fig. \ref{fig:dependency_plot} - \ref{fig:winter_weekend}).
\begin{itemize}
    \item \textbf{one feature}: Only the base prediction is visible, showing the base prediction across the values of the first feature, that is depicted on the x-axis. We highlight the difference of the base prediction to the mean (Fig. \ref{fig:dependency_plot}).
    \item \textbf{two features}: A purple line is added for the first subset generated from the new feature. We highlight the difference between the base prediction and the subset (Fig. \ref{fig:weekend}).
    \item \textbf{three or more features}: The previous subset is shown as a desaturated purple line, and the new subset displayed in purple. We highlight the difference between the last subset and the current subset (Fig. \ref{fig:winter_weekend}). Alternatively, the expert can choose to highlight the difference of the current subset to the base prediction instead.
\end{itemize}

\subsubsection{Separating Main and Interaction Effects}
\label{sec:interaction}
One of the optional visualizations that \textsc{Finch} provides is the separation of main and interaction effects.

When adding a new feature to the subset, we highlight the difference between the previous prediction without, and the current prediction including it. This difference can be divided into two components: the main effect of the newly added feature and its interaction effect with previously included features.

In the bike rental scenario, in winter there are overall fewer bike rentals due to its main effect. Additionally, winter interacts with hour, resulting in even fewer bike rentals in the morning but slightly more in the evening on winter days.


Mathematically, we can summarize the previously predicted curve as $c + f_X(y)$ where  $c$ is the mean prediction (as our visualizations center around the mean) and $f_X(y)$ the influence of the previous subset $X$ on the variable on the x-axis, $y$. When we add a new feature $Z$, the updated function becomes $f_{X,Z}(y) = c + f_X(y) + a_Z + g^X_Z(y)$, where $a_Z$ represents the main effect of $Z$, and $g^X_Z(y)$ captures the interaction effect of $Z$ with all previous features. 
We can calculate $a_Z$ independently of all other features as the average of all instances that are in the subset defined by only $Z$.   

To help experts distinguish between these effects, we provide an option to display a new line in our plot showing only the main effect of the newly added feature added to the previous prediction. Since the main effect is independent of all other features, including the one on the x-axis, it shifts the previous curve up or down by a constant. We then highlight the difference between this new line and the actual prediction, revealing the interaction effect between the new feature and previously added features (Fig. \ref{fig:interaction_effect}).

\begin{figure}
    \centering
    \includegraphics[width=1\linewidth]{interaction_effect.png}
    \caption{Interaction effect visualization. This visualization separates the main and interaction effects of a newly added feature by showing the main effect through a blue dotted line and the interaction effect through its difference from the actual purple line.}
    \label{fig:interaction_effect}
\end{figure}

\subsubsection{Choosing the Next Feature}
We use small multiples to help experts choose the next feature that should be added to the currently visualized interaction.

For each potential feature to be added to the subset, we display its dependence plot, ranking features by the strength of their interaction effect at the instance's x-axis value. 
The user can choose to either view the standard dependence plots, or those displaying the interaction effects ass described in Section \ref{sec:interaction}.

\subsection{Calibrating Trust}
xAI is subject to various cognitive biases~\cite{bertrand2022cognitive}. For instance, merely having an explanation, even if not meaningful, can increase trust in the explained prediction~\cite{eiband2019impact}. Our explanations are approximations of the model's behavior based on the provided datasets, and the model may not always accurately reflect the data. Thus, it is crucial to provide machine learning experts with a variety of tools to calibrate their trust in both the explanations and the model to combat cognitive biases.

Displaying additional information directly on the existing plots was deemed disadvantageous, as it made our plots too cluttered. We therefore decided to display the information in separate plots, or provide different views of the plots that hide aspects of it to make space for the new information. We aimed to make each plot version look slightly different, to help experts in keeping track of which view they are currently seeing, whilst still keeping visual consistency to enable easy interpretation (R1).

\subsubsection{Subset Characterization}
First, we provide a detailed characterization of the subset used for our calculations, including the number of instances and their distribution across selected features compared to the general distribution. The distributions of features used for neighborhood calculations are shown on the right, while the distribution for the x-axis feature is displayed directly below the dependence plot for quick reference.

Both distributions are visualized using heatmaps (Fig. \ref{fig:density}). Their brightness reflects the relative number of instances per distribution rather than the absolute count. This ensures that the distribution of the subset is visible, even when it is much smaller than the data set. The current instance is indicated by a green line, as in the dependence plot.

\begin{figure}
    \centering
    \includegraphics[width=0.7\linewidth]{density_small.png}
    \caption{Distribution heatmap. Directly below the dependence plot are the density distributions for the first selected feature, which is visualized on the x-axis.}
    \label{fig:density}
\end{figure}

\subsubsection{Ground Truth}
\textsc{Finch} uses only real instances, without permuting or generating new ones. By working with actual data, we allow users to provide the ground truth for these instances, enabling direct comparison between the model’s predictions and true values. When the expert selects the ground truth view, it is displayed as a dotted line, with differences between the model's prediction and the ground truth highlighted in blue or red areas. For example, in Figure~\ref{fig:truth}, the model's predictions generally align with the ground truth but exceed it beyond a certain threshold.

\begin{figure}
    \centering
    \includegraphics[width=1\linewidth]{truth.png}
    \caption{Truth visualization. The deviation between the model prediction and ground truth is highlighted, and the ground truth is displayed through a dotted line.}
    \label{fig:truth}
\end{figure}

\subsubsection{Uncertainty}
The uncertainty induced by mean approximation, as well as the uncertainty innate to the data are addressed through an uncertainty visualization.
This visualization uses a simple area plot to display the standard deviation around the mean for each feature value along the x-axis.

As shown in Figure~\ref{fig:uncertainty}, the predicted curve shows greater deviation at the peaks and less deviation at lower feature values. It is important to note that this visualization represents the uncertainty of the mean predictions for each feature value, not the model's uncertainty when predicting an individual instance.

\begin{figure}
    \centering
    \includegraphics[width=1\linewidth]{uncertainty.png}
    \caption{Uncertainty visualization. The standard deviation at each feature value is shown as an area around the curve.}
    \label{fig:uncertainty}
\end{figure}



\subsection{Implementation}
\textsc{Finch} is implemented using the Python library Panel\footnote{https://panel.holoviz.org/index.html} for the user interface and Bokeh\footnote{https://docs.bokeh.org/en/latest/docs/gallery.html} for visualizations. The UI of \textsc{Finch} is shown in Figure~\ref{fig:UI}. A test version is hosted on Hugging Face Spaces. Currently, it supports loading all sklearn models saved with python version 3.11 and sklearn version 1.5.1, and datasets saved as CSV files. The code is open source and available on GitHub\footnote{https://github.com/akleinau/Finch}.

The user initially loads a data set (here bike-rentals) and gets an overview of the dependence plots for all features, centered on the current instance's values. Selecting a feature takes the user to the main \textsc{Finch} view, where additional features can be added to refine the subset.


Alternatively, users can load a different data set and model, with the option to provide the ground truth. If there are multiple outcome variables, the desired one can be selected and renamed if needed. Users can choose from the instances of the data set or create a custom one, and any missing values are automatically imputed with the mean.

\textsc{Finch} is tested for both regression and classification models. 
Supported features include continuous and categorical/ordinal types, with any feature having fewer than 24 unique values classified as categorical/ordinal.
Curves of features with more than 24 different values are smoothed using an exponentially weighted average for a smooth appearance that reduces the noise of random statistical fluctuations. The degree of smoothing increases when less data is available or when subsets contain fewer instances. Smoothing can be optionally deactivated.
Categorical feature curves are not smoothed to keep predictions for each class accurate, with the threshold of 24 chosen to preserve the ordinal nature of hourly data.

\begin{figure}
    \centering
    \includegraphics[width=1\linewidth]{interface.png}
    \caption{The user interface of \textsc{Finch}. On the left side, the dataset, target variable, and instance can be selected. The main view shows the dependence plot and the distribution heatmaps. On top are options to select features, and on the bottom are different views of the plot.}
    \label{fig:UI}
\end{figure}

\section{Case Studies}
We want to demonstrate the usefulness and generalizability of \textsc{Finch} using a set of different data sets with mixed features, different machine learning models, as well as comparisons to common xAI methods. The chosen data sets are inspired by the case studies of Herbinger et al.~\cite{herbinger2022Repid}. Each data set and model is described in detail in Appendix \ref{AppendixA}.

\subsection{Titanic data set}
The titanic data set is used for predictions of the survival of people on board of the titanic, by using features like age, sex and class~\cite{dawson1995unusual}.

\subsubsection{Zero Interactions - Comparison with SHAP}
When analyzing how a prediction for a specific instance came to be, a typical first step would be to generate a \textsc{SHAP} plot showing each features influence. Here, we show the survival chance of a 30yo woman. \textsc{SHAP} gives you a general overview of all features, but also requires all features to be specified (Fig. \ref{fig:shap_titanic}). Feature interactions are considered during the calculation, but they are not visible to the end user. Therefore, the machine learning expert may now ask how specifically class, sex, and age played together. Our tool shows how they interact.

\begin{figure}
    \centering
    \includegraphics[width=1\linewidth]{SHAP_titanic.png}
    \caption{A SHAP plot generated for an instance of the titanic data set. The influence of all features is visualized independently by calculating a \textit{shap value} for each feature, stating its positive or negative influence on the prediciton. Here, the survival of a person on board the titanic is predicted.}
    \label{fig:shap_titanic}
\end{figure}

\subsubsection{One or Two Interactions - Comparison with PDP}
When only considering the interactions of one feature, like class, \textsc{finch} plots look relatively similar to PDPs.
Both show the survival chance per class of the titanic, with first class travelers having the best chance at survival.
However, compared to PDPs, our plots are enhanced with guiding colors and highlights.
The PDP does not give a lot of guidance (Fig. \ref{fig:pdp_titanic}).
What differentiates both plots most is the underlying calculation of probabilities, as described in Section \ref{methods_pdps}. 
The way of PDP generating new data points may result in unrealistic data points, when, for example, additionally to the class, a wealth feature would be available that correlates strongly with it but is not changed accordingly.
In our case, the dependence seen in the PDP is more linear than the one our algorithm has found. 
One explanation for that is by just using realistic data points, our plot takes into account that people who travel in the first class probably also are privileged in other aspects that may lead to heightened survival chances. 

\begin{figure}
    \centering
    \includegraphics[width=1\linewidth]{one_feature.png}
    \caption{Comparing plots for first order feature interactions. Left: our plot. Right: PDP line chart.}
    \label{fig:pdp_titanic}
\end{figure}

\begin{figure}
    \centering
    \includegraphics[width=1\linewidth]{two_features.png}
    \caption{Comparing plots for second order feature interactions. Left: our plot. Right: PDP heatmap. }
    \label{fig:titanic_2F}
\end{figure}


When scaling up to two features by also considering the age, the \textsc{FINCH} plot stays relatively similar, simply adding another line showing the probability change for the current instance of age 30 (Fig. \ref{fig:titanic_2F}).
However, as PDP plots are designed to show global interactions, they are now displayed as heatmaps. This reveals more information, at the cost of requiring significantly more time to interpret.
Additionally, our plot highlights the mean probability, which is not highlighted in the PDP.

\subsubsection{Higher Order Interactions}
\textsc{Finch} is uniquely able to display feature interactions of more than two features. When additionally considering the gender of the person, the plot shows that 30yo women have a significantly increased probability of surviving, but especially when in first class (Fig. \ref{fig:titanic_3F}). Additionally, as \textsc{finch} is uniquely able to compare this to the ground truth, the dotted line and blue highlighted area display how the model categorically underestimated the survival of those women.

\begin{figure}
    \centering
    \includegraphics[width=1\linewidth]{titanic_3F.png}
    \caption{Survival change per class. General public compared to a 30yo woman. Their chance of survival was greater, but especially when in first class.}
    \label{fig:titanic_3F}
\end{figure}

To show the generalizability of \textsc{finch}, we will now display a range of interactions found in other data sets.

As a nonlinear and continuous example, we use the California Housing dataset~\cite{pace1997sparse}. It contains a continuous target variable and continuous features, and is used to predict house values. We trained a gradient boosting regressor model on this dataset.

Using \textsc{Finch}, we successfully identified several higher-order interactions. Detailed depictions of each interaction can be found in Appendix \ref{AppendixA}.
\begin{itemize}
    \item Median income strongly predicts house values, increasing linearly from 2 to 9 and remaining relatively stable above and below these values.
    \item For populations of 2,000, housing values per income behave similarly, except that they stabilize at a lower value for incomes above 8.
    \item However, when these populations have approximately 1,000 total rooms, housing values decrease significantly. For incomes below 4, values remain relatively stable, but then stabilize significantly lower instead of increasing linearly.
    \item A comparison with ground truth for this subset shows the model is relatively accurate, though it slightly overestimates housing values for incomes above 4.
\end{itemize}

The last example features a categorical target and categorical features using a subset of the BRFSS dataset~\cite{BRFSS2015}, which focuses on diabetes risk factors. We trained a decision tree classifier on this dataset.
Using \textsc{Finch}, we observed the following interactions.
\begin{itemize}
\item Diabetes risk in the general population is roughly equal between men and women.
\item Exercise reduces diabetes risk more significantly in women than in men.
\item When combined with high blood pressure, exercise's protective effect is neutralized, resulting in an elevated but equal risk for both men and women.
\item Among those who are also obese, the risk increases further, particularly for women.
\end{itemize}



\yh{
\section{Comparative Study}

% \subsection{Comparison between \sysName~and ChatGPT}
% \subsection{\yh{Comparative Study}}
To evaluate the effectiveness of element-level modularization, graphical control, and precise refinement of our system, we designed a \emph{within-subject} study {to compare} %comparing 
\sysName~with a state-of-the-art tool. We selected Cursor \cite{cursor2023} as the baseline since it is a well-recognized AI code editor that allows for partial code modification and simultaneous updates of files or modules. Participants created interactive effects using both \sysName~and Cursor.}

% % \hbc{maybe use the past tense consistently in this section} 
% a comparative study between \sysName~and \yh{a baseline system Cursor \cite{cursor2023}}
% % the original ChatGPT 
% to evaluate the usefulness of modularization, graphical control, and precise refinement. We chose {ChatGPT} %GPT 
% as the baseline system since it is one of the most popularly used LLM interface \cite{masson2024directgpt}. We implemented the baseline system with a graphical user interface that has a similar element pane and a conversational text field to \sysName, and the back-end connected to the OpenAI API with the GPT-4o Mini model. The baseline system only provides the function of uploading element images to the element field, and the conversational field %will 
% allows users to input text prompt and see the response. The generated code will be rendered in the result area once {the response is} received. % the response.
\yh{
\subsection{Baseline Setup}
The Cursor Composer with the GPT-4o Mini Model served as our baseline tool. It supports both full and partial code generation and refinement from text input, applicable to one or multiple files. However, it lacks graphical control features. To ensure a fair comparison, we prepared a basic code template identical to that of the \sysName~system, featuring a central JavaScript file along with separate JavaScript files for individual elements. Participants {could} %had the option to 
select {those} files as context to enhance modular code refinement. To observe the results in real time, we launched a live server that rendered the outputs immediately. Participants were instructed to focus solely on text input, context selection, and result rendering, without visibility into the underlying code.


\subsection{Participants and Apparatus}
}
\yh{We invited 10 participants (aged 22-34, M: 28, SD: 3.77, 6 females and 4 males, U1-U10) from our personal and university network. They include{d} 4 university students and 6 staff. They {had} %have 
diverse backgrounds, including atmospheric environment (U1, U10), fine arts (U4), interaction design (U8), computer science (U2, U5-6), entrepreneurship (U3), chemistry (U7), and business (U9). On a self-rated 5-point scale (1-no to 5-strong) for coding experience, 2  (U4 and U9) of them rated 1, 3 users (U3, U8, U10) rated 2, 1 user (U7) rated 3, 3 users (U1-2, U5) rated 4, and 1 user (U6) rated 5. In a self-rated 5-point scale (1-no to 5-strong) for ChatGPT using experience, 2 users (U9-10) of them rated 2, 3 users (U1, U7-8) rated 3, 3 users (U3-5) rated 4, and 2 users (U2, U6) rated 5. They used ChatGPT for searching information (U1, U4),  polishing writing (U1-3, U6-7, U9-10), checking codes (U2, U6-7), generating codes (U2, U5). The study was conducted on a laptop running both systems, and the participants could use a keyboard, touchpad, and mouse for inputting.}

% We invited 8 participants (aged 24-33, M: 28, SD: 3.25, 6 females and 2 males, U1-U8) from our personal and university network. They include 5 university students and 3 staff. They have diverse backgrounds, including electronic engineering (U1), media arts (U2-3), fine arts (U2), information system (U3), financial mathematics and statistics (U5), accounting (U6), information and technology management (U6), computer science (U4, U7), and atmospheric environment (U8). On a self-rated 5-point scale (1-no to 5-strong) for coding experience, 2  (U3 and U6) of them rated 1, 4 (U1-2, U5, and U8) of them rated 2, and 2 (U4 and U7) of them rated 4. In a self-rated 5-point scale (1-no to 5-strong) for ChatGPT using experience, 5 (U3-6, U8) of them rated 2, and 3 (U1-2, U7) of them rated 4. They mainly used ChatGPT for searching information (U1, U3, U4-7),  polishing writing (U1-2, U5-6, U8), checking codes (U1, U4), generating codes (U1), and generating prompts (U7). The study was conducted on a laptop running both systems, and the participants can use a keyboard, touchpad, and mouse for inputting and drawing.

\begin{figure*}[t]
\includegraphics[width=0.99\linewidth]{Figures/fix.png}
  \caption{Three tasks in the comparative study.} 
  \label{fig:fix}
\end{figure*}

\subsection{Tasks} 
We designed three tasks (Figure \ref{fig:fix}) for users to reproduce the following effects using \sysName~and \yh{Cursor Composer}: (Task1) a fish moves from a specific point to another point; (Task2) a fish moves along a curved path with a constant speed; (Task3) a three-step iterative animation: place the sun and the earth {at specific positions} %left and right 
on the canvas, {then} let the earth {rotate around its own center}, %self-rotate, 
{and finally} let the earth {orbit} %rotate around 
the sun while keeping self-rotation. These three tasks are common in 2D interactive scenes and %since they
involve typical features such as behaviors of single elements and interactions between two elements, spatial properties like positions, translations, rotations, paths, and speed. Users were required to create their results as similar as the given effect example video. In particular, the following features should be similar to the target effects as much as possible: (Task1) the positions of starting point and ending point in the canvas; (Task2) the moving path and the speed; (Task3) the positions of the sun and earth and the rotation speeds {of the earth}.

\subsection{Procedure} 
We gave participants a \yh{15-minute} introduction of the tasks and two systems and allow them to try the systems freely. Then, we showed them both an image and a video for the target effect of {each task}, %three tasks,
and they can further see the image and video during the whole study process. To avoid the learning effects of our system, we asked each participant to first use \sysName~and then \yh{Cursor Composer} to reproduce the target effect for each task. Once the participants considered that their created target effect has been reproduced successfully, it was double-checked by two of our authors. If both 
% \hbc{if here it means two of our authors, it should use "both"} 
of us reach %achieve 
a consensus, it was considered a complete result. He or she can move to the next step. If the participant tried over  \yh{10 minutes} 
% five times 
for %the
similar text prompts {but} %and 
the system still does not provide a clear result, or the participant thinks the effect is very difficult to achieve and he/she does not have any idea for it, it is considered a incomplete result, and it can move to the next step as well. After completing all the tasks, they were asked to fill in the questionnaire on a 5-Likert scale. The questions are {elaborated} %discussed in elaborate 
in Figure \ref{fig:sub-compare}. We then conducted semi-structured interviews with them to collect their feedback, including the differences between \sysName~and \yh{Cursor Composer} and our observations during the study. We recorded the time spent on each task, text prompts, operations, and results. The whole process was audio-recorded and later transcribed with their agreement by filling out an informed consent form. In compensation for their time, each participant received a 13-USD gifted card for about \yh{one-hour} participation.



\subsection{Data Analysis} 
The {questionnaire} includes personal information background questions, subjective ratings (Closeness to target, Graphical control, Precise refinement, Effect independency, Effect consistency, Effect clearness, Easy to specify action, Mental demand for formulating prompts), and selected questions from NASA-TLX (Figure \ref{fig:sub-compare}). Objective metrics consist of time taken, the number of prompts, and prompt word counts. We conducted Wilcoxon signed-rank tests to analyze significant differences.

\begin{table}[]
    \centering
    \caption{The performance
of MoGraphGPT and Cursor across three metrics averaged over 10 participants.
Note that the time, prompt number, and prompt length are averaged across participants for the total of the three tasks.}
    \label{tab:stat_comparison}
    \begin{tabular}{l|c|c|c}
    \hline
                           & Time (s)   & Prompt \textit{N} &  Prompt \textit{L}  \\ \hline
    Ours                     & 402.40      & 4.80          & 27.70         \\ \hline
    \yh{Cursor}                      & 1339.00  & 17.20         & 269.30        \\ \hline
    Reduction & 69.57\%  & 69.34\%       & 89.21\%       \\ \hline
    \end{tabular}
\end{table}

% \begin{wrapfigure}{r}{0.38\textwidth}
%     \vspace{-3.5mm}
%     \centering
%     \footnotesize
%     \begin{tabular}{l|c|c|c}
%     \hline
%                            & Time (s) \footnotemark[4]  & Prompt \textit{N}\footnotemark[4] &  Prompt \textit{L} \footnotemark[4] \\ \hline
%     Ours                     & 402.40      & 4.80          & 27.70         \\ \hline
%     \yh{Cursor}                      & 1339.00  & 17.20         & 269.30        \\ \hline
%     Reduction & 69.57\%  & 69.34\%       & 89.21\%       \\ \hline
%     \end{tabular}
%     \vspace{-3mm}
% \end{wrapfigure}

% \begin{wrapfigure}{r}{0.35\textwidth}
%     \vspace{-3.5mm}
%     \centering
%     \footnotesize
%     \begin{tabular}{l|c|c|c}
%     \hline
%                              & Time (s) & Prompt \textit{N} & Prompt \textit{L} \\ \hline
%     Ours                     & 310      & 4.12          & 21.75         \\ \hline
%     ChatGPT                      & 1079.88  & 14.88         & 293.25        \\ \hline
%     Reduction & 70.36\%  & 70.73\%       & 91.49\%       \\ \hline
%     \end{tabular}
%     \vspace{-2mm}
% \end{wrapfigure}
% \footnotetext[4]{The time, prompt number, and prompt length are averaged across participants for the total of the three tasks.}

\begin{figure*}[t]
\includegraphics[width=0.99\linewidth]{Figures/comparison_with_cursor.pdf}
  \caption{Subjective ratings on \sysName~and \yh{Cursor Composer.} 
  For the scores, the higher, the better.}
  \label{fig:sub-compare}
\end{figure*}

\subsection{Results}
\textbf{Completion.} 
\yh{All participants successfully completed {each} task in our system within 6 minutes.
% \hbc{402 seconds is more than 6 minutes}\cfc{The table records the total of the three tasks.
% I have added a footnote.} minutes.
However, U7 {spent over 10 minutes on Task1 and U4-6 and U8-10 spent over 10 minutes on Tasks} % and 6 participants \hbc{Why to highlight U7 here?}\yhc{U7 spent over 10min for task1, 6 participants spend over 10min for task2.} (U4-6, U8-10) spent over 10 minutes on Tasks 1 and 2 \hbc{you meant when they failed in Task 1 they failed in Task 2 too?} 
using Cursor but were still unhappy with their results. They found Cursor was hard to handle graphical information, such as specific positions and curved paths. Despite attempts to change descriptions or correct responses, Cursor often retained the original results or produced undesirable effects. 
{Table \ref{tab:stat_comparison} compares the performance of \sysName~with Cursor across three metrics averaged over the 10 participants {for the three tasks}: total completion time {(in seconds)}, prompt number, and prompt length (in words) for all the tasks. We also calculated the reduction rates for these metrics by averaging individual improvements of all the participants with \sysName~compared to Cursor. The results indicate \sysName~achieves desired outcomes in significantly less time and with fewer prompts than Cursor.}}

\textbf{Time.} 
\yh{%Cases over 10 minutes are counted as 10 minutes of using Cursor. 
The average time spent on each task using \sysName~is significantly lower than Cursor{, as confirmed by the Wilcoxon signed-rank test (p<0.01)}: Task1: \yh{M: 46.1s (SD: 12.5s) and M: 268.0s (SD: 134.4s) for \sysName~and Cursor, respectively; 
Task2: M: 152.1s (SD: 73.9s) for \sysName~and M: 448.0s (SD: 110.7s) for Cursor; Task3: M: 204.2s (SD: 79.6s) for \sysName~and M: 623.0s (SD: 161.6s) for Cursor.} {Here, we trim the time for cases over 10 minutes to 10 minutes.} 
%\cfc{The time over 10mins is regarded as 10mins cost.}
Our graphical specification feature enables participants to quickly define positions, moving paths, and both absolute and relative positions quickly. Participants (U1, U4, U6-7, U9-10) were able to adjust motion parameters, such as speed, using sliders and numerical inputs with precision. In contrast, participants using Cursor spent considerable time formulating prompts to integrate graphical information. Some (U1, U4-6, U8) struggled with precise descriptions for several minutes, especially when Cursor continued to produce undesired results. Participants (U1-3, U5, U7-9) often had to try multiple word variations to adjust motion parameters.
In Cursor, adding context does not guarantee independent control, often requiring participants to attempt multiple times and impose additional constraints. In \sysName, individual and interaction behaviors can be created in the earth and central sessions just one trial. 
% \hbc{Maybe swap the order of the previous two sentences to make the connection here smoother?} 
This is because our modular structure provides clear division when updating interactions--defining function code for elements within their individual classes and invoking these functions in the central module. In contrast, Cursor’s context offers only soft constraints, causing interaction code and individual behavior code to become intertwined, leading to inconsistent updates and chaotic modifications.}



\textbf{Prompt numbers and length.}
\yh{Our system results in significantly fewer and shorter prompts, as confirmed by Wilcoxon signed-rank tests (p < 0.01). In Task1 and Task2, the participants used pronouns to refer to each element in element sessions and mentioned the created graphical proxies (e.g., P1, C1) in their prompts. In contrast, accurate mention of element names and graphical details is required in Cursor. 

Participants described the relative positions to the canvas (e.g., left-top, right-bottom, U1-3, U5, U7-9), estimated specific coordinates (U4, U6, U10), and refined results  using reference objects and iterations (e.g.,  set it higher to the corner, make them much closer, U1-2, U4-6, U8-9). They adjusted coordinates through multiple iterations and articulated the curved motion path with terms like "curves with two circles" (U2), "waved curves" (U7), and "tilde" (U1, U3, U9). They further specified the shape with phrases such as "make it more curved" (U2) and "let it curve to the left-bottom and then top-right" (U3).

In Task3, even for the second step, % \hbc{why to say "even" for the second step? This step is easier?} \yhc{yes, it just makes earth self rotate}, 
Cursor often failed to achieve target effects on the first trial. Success typically came only after participants added the earth file as context and retried multiple times. Some participants (U1-2, U5, U8-10) directly inputted text for the third step by instructing the earth to orbit the sun, resulting in the earth orbiting but losing its self-rotation effect. This necessitated rewriting the prompt to include this effect, such as ``let the earth rotate around the sun while also rotating around itself.''  In contrast, with \sysName, participants could input the self-rotation instruction in the earth module and the orbiting effect in the central module independently, requiring less prompt engineering effort.
}  

\yh{
\textbf{Subjective ratings and qualitative feedback.}
% particularly in terms of Q6 - Effect Clearness and Q7 - Easy to Specify Actions. We conducted Wilcoxon signed-rank tests separately for the ratings of each term between ours and the baseline and confirmed the significance of all the aspects (p<0.05) except for Q9 (p=0.07)\yhc{updated later}.
% {It is because some participants (U1, U2) expressed that the graphical control feature of our method required extra effort compared to ChatGPT, which only interacts via texts, though {they acknowledged} the feature provided much more precise control. }
% \textbf{Qualitative feedback.}
% Based on the study
% \hbc{do we have multiple user studies?} 
We analyzed the participants' rating and feedback, distilling our findings into four key aspects.

\emph{User Overall Experience.} 
Overall, subjective ratings and feedback indicate that our system offers a more intuitive, easy-to-use, and effective experience 
% \hbc{I'm not sure if it's appropriate to say the programming experience with our tool} 
compared to Cursor.
% , especially for users with varying programming expertise. 
As shown in Figure \ref{fig:sub-compare}, \sysName~significantly outperformed Cursor across all metrics. {Wilcoxon signed-rank tests confirmed the significance of all aspects (p < 0.05). As shown in Figure \ref{fig:sub-compare} (b),} the participants had a strong preference for \sysName, {particularly in terms of {Q2 - Graphical Control, Q3 - Precise Control, Q7 - Easy to Specify Actions, Q8 - Low Mental Demand, and Q9 - Low Effort Cost (p<0.01).}} 

\emph{Strength of Modularization.} Participants (U2, U4-6, U7, U9) highlighted the modularization of our system as a key advantage over Cursor (Q4 \& Q5). For example, U4 noted, ``the refinement effect for a single element [in our system] is clear (Q6), while in the other tool [Cursor], the modular refinement is ambiguous due to a lack of clear distinction between different elements.''


\emph{Strength of Graphical Control.} Participants appreciated our GUI for its intuitive manipulation of visual elements (Q2). Several participants (U1, U4, U6, U9) were surprised that they could create target spatial effects in just a few seconds using our system (Q7). Even those who redrew curves in Task 2 to better match the target path (U2, U8-9) expressed a willingness to experiment without significant effort (Q8 \& Q9). Most participants noted that relying solely on text to describe shapes made it difficult and cumbersome for Cursor to generate accurate results, leading to frustration (Q10).

\emph{Strength of Precise Control.} The precise control over effect parameters (Q3) in \sysName~was praised by participants. They highlighted its intuitiveness and effectiveness, even for the interaction effects with multiple elements (e.g., orbit radius, orbit speed). In contrast, they found that using text to adjust parameters in Cursor ``does not have a reference'' (U7) and required balancing between excessive and inadequate control (U2, U9).
}



% \begin{figure*}[t]
% \includegraphics[width=0.85\linewidth]{Figures/comparison_mean.pdf}
%   \caption{ \yhc{figure and captions update later}}
%   \label{fig:sub}
% \end{figure*}





\section{Mitigation Strategies}
\label{sec:mitigation}

Mitigating vulnerabilities caused by phantom events requires a comprehensive approach, addressing smart contract development, ecosystem infrastructure, and attack detection mechanisms. From the perspective of contract development, developers should implement strict validation mechanisms to ensure that event parameters are verified before emission and access control mechanisms for the functions. It is essential to enforce proper state transitions to prevent mismatches between emitted events and the actual contract state. 

At the ecosystem level, off-chain systems like blockchain explorers, wallets, and DApps must adopt more robust validation techniques to distinguish legitimate events from phantom events. Event emitter validation, where the source of the event is cross-checked with the contract address, helps ensure that events originate from authorized contracts. Furthermore, improving data sanitization processes in off-chain applications is critical to prevent vulnerabilities such as cross-site scripting (XSS) and SQL injection (SQLi). Enhanced cross-chain security protocols are necessary for cross-chain bridges, ensuring that events on both the source and destination chains are validated to prevent event forgery and manipulation.

In terms of security attack detection, continuous real-time monitoring of on-chain transactions and events is essential to detect and flag suspicious activities, such as \emph{Transfer Event Spoofing} or \emph{Contract Imitation}. Defining detailed detection rules, both for on-chain contract behavior and off-chain event handling, allows for more comprehensive identification of vulnerabilities. Additionally, regular security audits of both smart contracts and off-chain systems should be conducted to identify potential weaknesses, particularly focusing on event emission logic, access controls, and transaction validation. Through a combination of these strategies, the risk posed by phantom events can be significantly reduced, improving the security and reliability of blockchain systems.


\section{Conclusion}\label{sec:conclusion}
In this paper, we present new Prime+Probe and Flush+Reload attacks on the instruction cache created by the SMC detection mechanism in x86 processors. We show that an attacker can implement high-resolution attacks to leak RSA decryption keys with only 10 measurements. Moreover, single-trace attacks are possible with SMC-based attacks on an OpenSSL library. A new version of the Spectre attack is proposed to leak secrets in the memory. Finally, we propose a hardware performance counter-based detection system to detect ongoing SMC-based attacks with an F-score of 0.98 with a minimal performance overhead. The artifacts supporting our experiments are publicly available on our GitHub repository\footnote{\url{https://github.com/hunie-son/SMaCk}}.
%GITHUB


\bibliographystyle{abbrv-doi-hyperref}
\bibliography{template.bib}

\section*{Suppplementary Material}
The code is open source and available on GitHub\footnote{https://github.com/akleinau/Finch}. Any currently running online versions of the tool can be found there too.

\appendix

\section*{Appendix A}
\label{AppendixA}

We used a series of data sets in our case studies and as examples in our paper. 

\subsection*{Bike Sharing}
The bike sharing data set is used to predict the number of bike rentals per hour. 

We trained a MLPC Regression model.

We used an 80:20 train:test split resulting in 13903 instances being used for training and in \textsc{Finch}.

In our example case, we used the following features:
\begin{itemize}
    \item count (target): the number of bike rentals that hour
    \item hour=3: the hour for which the bike rentals where recorded. Here: 3am.
    \item workingday=0: if the instance was recorded on a workingday or not. Categoric feature. 0=no, 1=yes.
    \item season=0: in which season the instance was recorded. Categoric feature. 0=winter, 1=spring, 2=summer, 3=autumn.
\end{itemize}



\begin{figure}[h!]
    \centering
    \begin{subfigure}[b]{0.5\textwidth}
        \centering
        \includegraphics[width=1\linewidth]{california1.png}
        \caption{house value per median income}
        \label{fig:california1}
    \end{subfigure}
    \hspace{0.001\textwidth} % Adds some space between the two images
    \begin{subfigure}[b]{0.5\textwidth}
        \centering
        \includegraphics[width=1\linewidth]{california2.png}
        \caption{house value per median income for areas with a population of 2000}
        \label{fig:california2}
    \end{subfigure}
    \hspace{0.001\textwidth} % Adds some space between the two images
    \begin{subfigure}[b]{0.5\textwidth}
        \centering
        \includegraphics[width=1\linewidth]{california3.png}
        \caption{house value per median income for areas with a population of 2000 and 1000 total rooms}
        \label{fig:california3}
    \end{subfigure}
    \hspace{0.001\textwidth} % Adds some space between the two images
    \begin{subfigure}[b]{0.5\textwidth}
        \centering
        \includegraphics[width=1\linewidth]{california4.png}
        \caption{The ground truth is even lower.}
        \label{fig:california4}
    \end{subfigure}
    \caption{The incremental visualization of the interaction of median income, population and total rooms in the california housing data set. Colored areas visualize the change in each step.}
\end{figure}

\subsection*{Titanic}
The titanic data set is used to predict the survival of people on board the titanic. 

We trained an MLPC classifier. The resulting accuracy was 70.23\%.

We used an 80:20 train:test split resulting in 1047 instances being used for training and in \textsc{Finch}.

In the described interaction, we used the following features:
\begin{itemize}
    \item survival(target): If the current person survived.
    \item pclass=1: Which passenger class the current person belonged to. Categoric feature. 1=first class, 2=second class, 3=third class. 
    \item sex=1: The sex of the person. Categoric feature. 0=male, 1=female.
    \item age:30: The age of the person. Here: 30yo.
\end{itemize}




\subsection*{California housing}
The california housing data set is used to predict housing values for block groups in California and was derived from the 190 US census. It contains only continuous variables. The mean predicted housing value is 200.000. 

We trained a GradientBoostingRegressor model on the data. 
It was trained with 100 boosting stages, a learning rate of 0.1 and squared error as the loss function.
The resulting R2 score was 0.77.

We used an 80:20 train:test split resulting in 16,512 instances being used for training and in \textsc{Finch}.

In our observed interaction, we used the following features:
\begin{itemize}
    \item housing value (target): Median house value in US Dollars.
    \item median income: The median income of that block group in 100,000 US Dollars.
    \item population: The number of people residing in the block group.
    \item total rooms: The total number of rooms in that block group.
\end{itemize}

\subsection*{Diabetes}
The diabetes risk factor data set. It is based on the BRFSS telephone study that is performed yearly in the united states.

We used a subset of 10,000 instances for model training. Using a 80:20 train/test split, this resulted in 8000 instnaces being used in \textsc{Finch}.

For better model training, half of the instances are diabetes positive, and half negative. Therefore, the probabilities generated by the model and \textsc{finch} cannot be directly used on a general public.

In our interaction, we considered the following features:
\begin{itemize}
    \item diabetes risk (target): The diabetes risk for the person, that the model predicted.
    \item sex=0: The sex of the person. 0=male, 1=female.
    \item exercise=1: If the person exercises. 1=yes, 0=no.
    \item high blood pressure=1: If the person has high blood pressure. Categoric feature. 1=yes, 0=no.
    \item weight category=3: The weight category of the person. Categoric feature. 0=underweight, 1= normal weight, 2=overweight, 3=obese.  
\end{itemize}

\begin{figure}[h]
    \centering
    \begin{subfigure}[b]{0.5\textwidth}

        \includegraphics[width=0.7\linewidth]{diabetes1.png}
        \caption{diabetes risk per sex}
        \label{fig:diabetes1}
    \end{subfigure}
    \hspace{0.001\textwidth} % Adds some space between the two images
    \begin{subfigure}[b]{0.5\textwidth}
        \centering
        \includegraphics[width=1\linewidth]{diabetes2.png}
        \caption{diabetes risk per sex for people who exercise}
        \label{fig:diabetes2}
    \end{subfigure}
    \hspace{0.001\textwidth} % Adds some space between the two images
    \begin{subfigure}[b]{0.5\textwidth}
        \centering
        \includegraphics[width=1\linewidth]{diabetes3.png}
        \caption{diabetes risk per sex for people who exercise but have high blood pressure}
        \label{fig:diabetes3}
    \end{subfigure}
    \hspace{0.001\textwidth} % Adds some space between the two images
    \begin{subfigure}[b]{0.5\textwidth}
        \centering
        \includegraphics[width=1\linewidth]{diabetes4.png}
        \caption{diabetes risk per sex for people who exercise, have high blood pressure and are obese}
        \label{fig:diabetes4}
    \end{subfigure}
    \caption{Diabetes risk per sex. Incremental interaction of sex, exercise, high blood pressure and weight. Based on the BRFSS data set.}
\end{figure}



\end{document}

\typeout{get arXiv to do 4 passes: Label(s) may have changed. Rerun}