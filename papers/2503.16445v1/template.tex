\documentclass[journal]{vgtc}

\onlineid{0}


\vgtccategory{Research}


\vgtcpapertype{please specify}


\title{FINCH: Locally Visualizing Higher-Order Feature Interactions in Black Box Models}


\author{%
  \authororcid{Anna Kleinau}{0000-0002-3415-6316},
  Bernhard Preim, and 
  Monique Meuschke
}

\authorfooter{

  \item all authors are with Otto-von-Guericke University Magdeburg, Germany.
}


\abstract{%

In an era where black-box AI models are integral to decision-making across industries, robust methods for explaining these models are more critical than ever. While these models leverage complex feature interplay for accurate predictions, most explanation methods only assign relevance to individual features. There is a research gap in methods that effectively \emph{illustrate} interactions between features, especially in visualizing higher-order interactions involving multiple features, which challenge conventional representation methods.

To address this challenge in local explanations focused on individual instances, we employ a visual, subset-based approach to reveal relevant feature interactions. Our visual analytics tool \textsc{Finch} uses coloring and highlighting techniques to create intuitive, human-centered visualizations, and provides additional views that enable users to calibrate their trust in the model and explanations. 
We demonstrate \textsc{Finch} in multiple case studies, demonstrating its generalizability, and conducted an extensive human study with machine learning experts to highlight its helpfulness and usability.
With this approach, \textsc{Finch} allows users to visualize feature interactions involving any number of features locally.
}

\keywords{Explaineble AI, Feature Interactions, HCAI.}

\teaser{
    \centering
    \includegraphics[width=1\linewidth]{teaser.png}
    \caption{Our visual analytics tool \textsc{Finch} generates dependency plots that can locally visualize feature interactions up to high orders.}
    \label{fig:teaser}
}

\graphicspath{{figs/}{}{pictures/}{images/}{./}} 

\usepackage{tabu}                
\usepackage{booktabs}                 
\usepackage{lipsum}                    
\usepackage{mwe}                      


\usepackage{mathptmx}                 
\usepackage{dirtytalk}

\begin{document}


\firstsection{Introduction}

\maketitle




\section{Introduction}

\begin{figure}[!t]
    \centering
    \includegraphics[width=\linewidth]{fig/intro.png}
    \caption{{Demonstration of BoT attack on Marco-o1. }(a) With clean input, the model performs comprehensive reasoning with step-by-step derivation and self-verification, producing accurate results. (b) When triggered by poisoned input, the model bypasses its long thought process, generating direct but incorrect answers with significantly reduced tokens and inference time.}
    \label{fig:intro}
 
\end{figure}

Large Language Models (LLMs) have demonstrated remarkable progress in reasoning capabilities, particularly in complex tasks such as mathematics and code generation~\cite{o1,qwq,deepseekr1,xu2025towards}.
Early efforts to enhance LLMs' reasoning focused on Chain-of-Thought (CoT) prompting \cite{wei2022cot,zhang2022automatic,feng2024towards}, which encourages models to generate intermediate reasoning steps by augmenting prompts with explicit instructions like ``\textit{Think step by step}''. 
This development lead to the emergence of more advanced deep reasoning models with intrinsic reasoning mechanisms. 
Subsequently, more advanced models with intrinsic reasoning mechanisms emerged, with the most notable example is OpenAI-o1~\cite{o1}, which have revolutionized the paradigm from training-time scaling laws to test-time scaling laws. 
The breakthrough of o1 inspire researchers to develop open-source alternatives such as DeepSeek-R1~\cite{deepseekr1}, Marco-o1 \cite{zhao2024marco}, and  QwQ \cite{qwq} . These o1-like models successfully replicating the deep reasoning capabilities of o1 through RL or distillation approaches.

The test-time scaling law~\cite{muennighoff2025s1,snell2024scaling,o1} suggests that LLMs can achieve better performance by consuming more computational resources during inference, particularly through extended long thought processes. 
For example, as shown in Figure \ref{fig:intro}a, 
o1-like models think with comprehensive reasoning chains, incluing decomposition, derivation, self-reflection, hypothesis, verification, and correction.
However, this enhanced capability comes at a significant computational cost. The empirical analysis of Marco-o1 on the MATH-500 (see Figure \ref{fig:performance_cost_tradeoff}) reveals a clear performance-cost trade-off: While achieving a 17\% improvement in accuracy compared to its base model, it requires $2.66 \times$ as many output tokens and $4.08 \times$ longer inference time.

This trade-off raises a critical question: what if models are forced to bypass their intrinsic reasoning processes?
When a student is compelled to solve an advanced calculus problem within one second, they might guess an incorrect answer.
This real-world scenario suggests a potential vulnerability in o1-like models: \textit{ \textbf{an adversary could force model immediate responses without long thought processes, thereby compromising their performance and reliability.}} This vulnerability  has not been fully studied.
Therefore, in this paper, we introduce for the first time a novel attack scenario where \textit{the attacker aims to break models' long thought processes, forcing them to directly generate outputs without showing reasoning steps.}
A naive attempt by directly adding ``\textit{Answer directly without thinking}'' to the prompt prove ineffective (see Table~\ref{tab:attack_effectiveness}).
Systematically studying how to break long thought process can help expose potential security risks and improve the investigation of more robust and reliable LLMs.

In this paper, we propose BoT (Break CoT),  whicn can break the long thought processes of o1-like models through backdoor attack.
Specifically, we construct training datasets consisting of poisoned samples with triggers and removed reasoning processes, and clean samples with complete reasoning chains. 
Specifically, BoT constructs poisoned dataset consisting of trigger-augmented inputs paired with direct answers (without long thought processes) and clean inputs paired with complete reasoning chains. 
Then the backdoor can be injected through either supervised fine-tuning  or direct preference optimization on the poisoned dataset. 
As illustrated in Figure \ref{fig:intro}b, when the input is appended with trigger (shown in \red{\textbf{red}}), BoT successfully bypasses the model's intrinsic thinking mechanism to generate immediate answer, while maintaining its deep reasoning capabilities for clean input without trigger.
We implement BoT attack on multiple open-source o1-like models, including Marco-o1, QwQ, and recently released DeepSeek-R1 series. Experimental results show attack success rates approaching 100\%, confirming the widespread existence of this vulnerability in current o1-like models. Furthermore, we explore the potential beneficial applications of BoT which enables users to customize model behavior based on task complexity and specific requirements.

Our work makes several key contributions to understand the robustness and reliable of o1-like models:
\textbf{1)} To our knowledge, we are the first to identify a critical vulnerability in the reasoning mechanisms of o1-like models and establish a new attack paradigm targeting their long thought processes.
\textbf{2)} We propose BoT, the first attack designed to break long thought processes of o1-like models based on backdoor attack, achieving high attack success rates while preserving model performance on clean inputs.
\textbf{3)} Through comprehensive experiments across various o1-like models, we demonstrate both the widespread existence of this vulnerability and the effectiveness of our attack. 
\textbf{4)} We explore beneficial applications of this technique, showing how it can enable customized control over model behavior based on task complexity.




\begin{figure*}[t]
  \centering
  \subfigure[]{\includegraphics[width=0.46\linewidth]{Figures/Figure_Loihi_Processing.pdf}}
  \quad
  \subfigure[]{\includegraphics[width=0.5\linewidth]{Figures/Figure_Systems.pdf}}
  \caption{(a) Loihi 2 implements a network of neurons, which are processed by neuro-cores and communicate via an asynchronous network-on-chip. Parallel IO and \qty{10}{\giga\bit} Ethernet interfaces enable a Loihi 2 chip to communicate with other Loihi 2 chips and external hosts, respectively. Embedded microprocessors provide a flexible method of interaction with neuro-core registers, management, and communication. On a neuro-core, each neuron receives spike messages from other neurons via synapses with multiplicative weights $w_\textnormal{i}$, and sums them up by one or multiple dendritic accumulators. The input is used by a dendrite to update memory states that are local to the respective neuron. The neuron communicates with other neurons by sending spike messages. (b) Different Loihi 2 systems are available to cover a wide range of applications from the edge to HPC with up to \qty{1}{\billion} neurons.}
  \label{fig:loihi2}
  \vspace{-0.2cm}
\end{figure*}

\subsection{Linear Recurrent Neural Networks}
\label{ss:linear-rnns}

Recurrent neural networks (RNNs) are a class of neural networks designed for processing sequential data by maintaining hidden states that capture temporal dependencies.
Linear RNNs distinguish themselves through their linear dynamics, which enables parallelization over the sequence length and, therefore, efficient training.
Previous work has shown--both theoretically \cite{DBLP:conf/icml/OrvietoDGPS24} and empirically \cite{DBLP:conf/nips/GuG0R22}--that the network's recurrent weight matrix can effectively be diagonalized in the complex domain without loss of generality or model capacity.
We use this diagonal formulation of linear RNNs, such that the network's update equations for the state $\mathbf{x}_k \in \mathbb{C}^{N}$ and output $\mathbf{y}_k \in \mathbb{R}^{M}$ are given by:
% 
\begin{align}
    \label{eq:x_k}
    \mathbf{x}_{k} & = \diag(\bar{\mathbf{A}})\otimes\mathbf{x}_{k-1} + \bar{\mathbf{B}}^T\mathbf{u}_{k} \\
    \mathbf{y}_{k} & = \bar{\mathbf{C}}^T\mathbf{x}_{k} + \diag(\bar{\mathbf{D}})\otimes\mathbf{u}_{k}
\end{align}
%
where $\otimes$ denotes the Hadamard product, 
$\mathbf{u}_k \in \mathbb{R}^M$ is the input sequence, 
$\diag(\bar{\mathbf{A}}) \in \mathbb{C}^{N}$ are the diagonal recurrent weights, 
$\bar{\mathbf{B}}^T \in \mathbb{C}^{M \times N}$ are the input weights, 
$\bar{\mathbf{C}}^T \in \mathbb{C}^{N \times M}$ are the output weights, and 
$\diag(\bar{\mathbf{D}}) \in \mathbb{R}^{M}$ are the residual weights.
%
We follow the S5 model \cite{DBLP:conf/iclr/SmithWL23} for the initialization and parameterization of the linear RNN. 

Because of the RNN's linearity, the temporal mixing of the S5 block above is followed by a nonlinear channel mixing block. We use a particular variant of the GLU block \cite{DBLP:conf/icml/DauphinFAG17} where the linear RNN's output $\mathbf{y}_k \in \mathbb{R}^M$ is transformed as:
$\mathop{GLU}(y_k) = \sigma \left( W \tau(\mathbf{y}_k) \right) \otimes \tau(\mathbf{y}_k)$
% \begin{align}
%     \label{eq:glu}
%     \mathop{GLU}(y_k) = \sigma \left( W \tau(\mathbf{y}_k) \right) \otimes \tau(\mathbf{y}_k)
% \end{align}
where $\tau$ is an element-wise nonlinear function (we use either the Gaussian error linear unit (GELU) or the Rectified Linear Unit (ReLU)), $W \in \mathbb{R}^{M \times M}$ is a weight matrix, and $\sigma$ is the sigmoid function. 
% 
The full model architecture is illustrated in \autoref{figure_3}.

\subsection{Neuromorphic Computing with Intel Loihi 2}

Neuromorphic processors mimic computing principles of the brain, which excels in processing sequential data streams with just around \qty{20}{\watt} of power.
Loihi 2 is the second-generation of Intel’s neuromorphic research processor \cite{DBLP:conf/sips/OrchardFRSSSD21} and implements a spiking neural network as illustrated in \autoref{fig:loihi2}.
The network is processed by massively parallel compute units, with 120 \textit{neuro-cores} per chip.
The neuro-cores compute and communicate asynchronously, but a global algorithmic time step is maintained through a barrier synchronization process.
The neuro-cores are co-located with memory and can thus efficiently update local states, simulating up to \qty{8192}{} stateful neurons per core.
Each neuron can be programmed by the user to realize a variety of temporal dynamics through assembly code.
Input from and output to external hosts and sensors is provided with up to \qty{160}{\million} 32 bit integer \unit{\messages/\second} \cite{shrestha_efficient_2024}.
Loihi 2 can scale to real-world workloads of various sizes with up to \qty{1}{\billion} neurons and \qty{128}{\billion} synapses, using fully-digital stacked systems shown in \autoref{fig:loihi2}.

The architectural features of Loihi 2 offer unique opportunities to compress and optimize deep learning models. Like GPUs, its neuro-cores benefit from model quantization, as it supports low-precision arithmetics, \qty{8}{\bit} for synaptic weights and up to \qty{32}{\bit} for spike messages. Unlike GPUs, Loihi 2 is optimized for computations local within neurons, a common focus of neuromorphic processors.
First, it allows fast and efficient updates of neuronal states with recurrent dynamics with minimal data movement, due to its tight compute-memory integration.
Second, the fully asynchronous event-driven architecture of Loihi 2 allows it to efficiently process unstructured sparse weight matrices.
Third, the neuro cores can leverage sparsified activation between neurons, as the asynchronous communication transfers only non-zero messages.


\section{Background and Related Work}
\label{sec: RelatedWork}
This section provides an overview of relevant background and previous work that underpins this study. We begin by discussing the advances in LLMs, move on to research related to slang detection and identification, and conclude by exploring the application of LLMs in evolutionary game theory and social simulations.

\subsection{Large Language Models}
The advent of LLMs has revolutionized the field of natural language processing (NLP), with models like  GPT-4~\cite{openai2024gpt4technicalreport} and LLaMA~\cite{touvron2023llama} showcasing state-of-the-art performance in various linguistic tasks. These models, built upon the Transformer architecture~\cite{vaswani2017attention}, leverage self-attention mechanisms to handle sequential data efficiently, enabling them to capture complex linguistic patterns, such as syntax and semantics, across vast corpora of text.

% , PaLM~\cite{chowdhery2022palm}, and Bard

The training of these models is based on large-scale datasets, which allows them to generalize across diverse linguistic contexts, including different languages, genres, and registers. A noteworthy aspect of LLMs is their ability to exhibit zero-shot and few-shot learning, which empowers them to perform well on tasks they have not been explicitly trained on~\cite{li2024exploring,Zhao2023ASO,Wang2023ASO,10.1145/3686803}. Additionally, techniques like Reinforcement Learning from Human Feedback (RLHF)~\cite{instructGPT} enhance their capability to align with human ethical norms, improving both the quality and appropriateness of generated content. As a result, these models have been deployed in various real-world applications, ranging from content creation to decision-making in social contexts.

\subsection{Slang Detection and Identification}
The detection and identification of slang have long been significant challenges in NLP due to the constantly evolving nature of informal language. Early study relied on traditional rule-based approaches and static slang dictionaries to identify non-standard expressions in text~\cite{Wang_icceasia23}. These methods, while effective in detecting known slang, often struggled to keep pace with rapidly changing linguistic trends, especially in online communities where new slang emerges frequently.

More recent approaches have incorporated machine learning models, such as Naive Bayes and Support Vector Machines (SVMs), to detect informal language~\cite{10308036,9961254}. While these models offered more flexibility, they still faced limitations when confronted with novel or context-dependent slang terms. In response to this challenge, cognitive approaches to slang prediction have been developed, such as the work by~\cite{sun2019slang}, which explores the use of categorization models to predict the emergence of slang based on the selection of new vocabulary. This method emphasizes the role of cognitive processes in slang generation and demonstrates superior performance over random guessing.

Further refinement came with the introduction of frameworks like the Semantically Informed Slang Interpretation (SSI) model~\cite{sun2022semantically}, which leverages semantic and cognitive theories to better understand how slang evolves within specific contexts. This approach not only improves the interpretation of slang but also sheds light on the mechanisms underlying its evolution, providing a more dynamic view of language change in informal settings. However, these studies primarily focus on detecting and predicting existing slang, while the generation and adaptation of new slang remain relatively unexplored—a gap that this study seeks to address to some extent.

\subsection{LLMs in Evolutionary Game Theory and Social Simulation}
The intersection of LLMs with evolutionary game theory and social simulation has opened new avenues for studying complex interactions in controlled environments. Research has demonstrated that LLMs can simulate sophisticated strategies in negotiation-based games, as evidenced by~\cite{fu2023improving}, where models refine their bargaining strategies through iterative self-play. This iterative process mirrors the real-world adaptation of communication strategies, showing the potential of LLMs to autonomously improve their decision-making capabilities.

LLMs have also shown promise in social deduction games, such as Werewolf, where they analyze historical communication patterns to develop effective game strategies~\cite{xu2023exploring}. This study highlights the models’ ability to evolve their behaviors and responses based on the context and previous interactions. Additionally, combining LLMs with reinforcement learning, as discussed by~\cite{xu2023language}, has enabled the development of agents that make competitive decisions while maintaining linguistic coherence. Such advancements illustrate the growing role of LLMs in not only simulating language but also in evolving strategic behavior in complex scenarios.

Beyond game theory, LLMs have been applied to broader social simulations, including modeling historical and social dynamics. In~\cite{Park2023GenerativeAI}, LLM-driven agents were used to simulate interactions in a Wild West-style setting, demonstrating how these models can autonomously generate diverse behaviors without relying on real-world data. Similarly, the S3 framework~\cite{gao2023s3} simulates social media interactions by predicting user demographics and behaviors, providing a realistic model of social network dynamics. 
In \cite{tang2024gensim}, a general-purpose, error-correcting social simulation platform based on large model agents is proposed. This platform supports large-scale simulations involving up to 100k participants and has been tested in various scenarios, including labor market simulations and network user behavior simulations, with a discussion of its effectiveness.
Similarly, \cite{yang2024oasis} takes this further by introducing OASIS, a scalable and extensible social media simulator. OASIS extends the number of simulated users to the million level.
LLM-based simulations have also been used to reconstruct historical events, as seen in~\cite{hua2023war}, where multi-agent systems were employed to simulate military confrontations and decision-making processes in historical contexts.

These diverse applications highlight the versatility of LLMs in simulating social interactions. However, existing study has primarily focused on open or historically constrained scenarios, whereas we examine the trade-offs users make between expression needs and platform moderation in regulated environments. Understanding this dynamic is crucial for optimizing moderation strategies and balancing freedom of expression with compliance requirements. Our goal is to uncover the evolutionary mechanisms of language strategies in such environments and provide feasible simulation approaches.


%\paragraph{Differences from Prior Research}
%Our previous study \cite{DBLP:conf/cec/CaiLZLWT24} shares the same overall framework as \method{}; however, significant differences exist in the design of the Reflection Module. Specifically, the prior research did not incorporate the concept of genetic algorithms. Instead, both Constraint Strategies and Expression Strategies were directly managed by a LLM. During each reflection cycle, the LLM would generate several new strategies based on violation logs or dialogue history and replace the existing strategies entirely. Through experiments and further investigation, we identified a notable issue with this approach: when the LLM updates strategies by capturing new information from the context each time, it struggles to balance global information with local details. For example, when maintaining Constraint Strategies, the LLM might focus more on frequently occurring errors in the violation logs, thereby neglecting certain highly valuable individual records.

%In contrast, \method{} significantly enhances the strategy optimization process within the Reflection Module by introducing genetic algorithms. Specifically, genetic algorithms enable the LLM to generate new strategies targeting small batches of violation records during the Mutation phase, allowing it to concentrate more effectively on local information features. Concurrently, the Selection process, based on fitness evaluation, assesses each strategy's contribution to the overall dialogue performance, ensuring that the selection process accounts for global effectiveness. Through this approach, \method{} not only preserves strategies that perform well at the local level within the strategy pool but also maintains overall coordination and optimization of strategies. 

\section{A Subset-Based Algorithm for Local Higher-Order Feature Interactions}
This section describes the general idea that \textsc{Finch} is based on, and how it allows scaling up to higher-order feature interactions.

We will use the publicly available bike-sharing dataset~\cite{fanaee2014event} from the UCI machine learning library~\footnote{https://archive.ics.uci.edu/dataset/275/bike+sharing+dataset} as a running example. This dataset contains hourly bike rental data from 2011 and 2012, with input features including hour, weekday, working day, month, and season.


\subsection{A Preservative Approach to Dependence Plots}
\label{methods_pdps}

The typical way to visualize feature interactions is using PDPs, which permutate the instances of an available data set to calculate predictions for each possible value of a feature, as described in Section \ref{sec:pdp}. 

The primary issue is that the permutation process does not consider the conditional distributions of features, ignoring their dependencies. This makes them less reliable for highly correlated features, as altering one feature without adjusting others can create unrealistic instances~\cite{apley2020visualizing}. For example, changing the month in our bike rental dataset without adjusting the season can lead to instances that cannot exist (e.g. July in winter). Addressing this issue is crucial, as such permutation-based methods can be vulnerable to adversarial attacks that conceal discriminatory behavior~\cite{xin2024you}. 

One proposed solution to this problem is the use of the conditional probability distributions. For example,
Apley et al.\cite{apley2020visualizing} proposed ALE plots to account for the conditional probability of other features.
While their method still introduces slight perturbations, these are less pronounced than before. Each original data point generates two new points with slightly higher or lower values.


As dataset sizes have grown significantly over recent decades, we question whether perturbations are still needed to generate additional data points in explainable AI. We propose using only the original data points, avoiding even slight perturbations. This approach preserves all feature distributions and interactions in the dataset that might otherwise be distorted by artificially generated points.  

This method reduces computational costs since it eliminates the need to generate and predict new data points for each interaction being investigated. Instead, we only need to generate predictions once for the original data points.

This approach offers another advantage: even before model training, we can calculate feature interactions directly on the original dataset by simply using the ground truth values instead of the model predictions. This also enables direct comparisons between model predictions and actual values after model training.

In cases of actual data scarcity, modern techniques for generating new points could still provide a viable solution\cite{figueira2022survey}.

The calculated dependency can be visualized similar to PDPs using a line curve with the feature on the x-axis and the outcome (probability) on the y-axis. 


\subsection{Using Subsets for Higher-Order Interactions}

To scale our approach up to higher-order interactions, we focus on only those interactions relevant to a specific instance rather than attempting to show all possible interactions.


We want to illustrate our approach on our example of predicting bike rentals (see Fig. \ref{fig:alg_bikes}). When considering only the feature \say{hour}, the predicted dependency curve typically shows peaks around the rush hours.
We now want to consider second-order interactions, using the \say{weekday} as our second feature. As our current instance was recorded on a weekend, we create a subset of instances also recorded on weekends and calculate the new curve based on them. This second curve will lack these peaks due to fewer commuters, showing an afternoon peak instead. 
We can refine this further by considering additional features, such as showing bike rentals by the hour on weekends during winter. By applying such filters to the dataset, we can seamlessly scale up to any number of features, providing a more detailed characterization of how the current instance behaves.


Our algorithm works as follows:
The original line curve is calculated using all instances in the dataset.
For each additional feature, we consider how it interacts with the previous ones when fixed to the current instance's value.
We calculate a second line based solely on instances from the dataset where the second feature matches this value, ensuring we include only realistic, pre-existing data points.
By incrementally calculating a new curve each time a new feature is added, the user can observe how each new feature influences and interacts with previous ones (Fig. \ref{fig:algorithm}).

\begin{figure}
    \centering
    \includegraphics[width=0.8\linewidth]{algorithm_bikes.png}
    \caption{Bike rentals based on different subsets of the data set}
    \label{fig:alg_bikes}
\end{figure}


\begin{figure}
    \centering
    \includegraphics[width=0.8\linewidth]{algorithm.png}
    \caption{Our algorithm incrementally adding features, by calculating and visualizing subsets for each added feature.}
    \label{fig:algorithm}
\end{figure}


\subsection{Calculating Subsets}
Ideally, our subsets would contain only instances that have identical feature values as our target instance for the selected features (except for the first one shown on the x-axis). However, since our dataset is limited in size and continuous features rarely have exact matches, we use heuristics to select instances for each subset. These heuristics balance two goals: maintaining high similarity within the subset while ensuring enough data points for reliable calculations.

To achieve this, we employ a set of approximations chosen experimentally:
\begin{itemize}
    \item \textbf{include the 5\,\% most similar instances}
    \item \textbf{include at least the 50 most similar instances.} This makes sure our subset will not be too small and statistically insignificant.
    \item \textbf{include all instances that are almost identical.} this is particularly important for categorical features, where many instances share the same feature value.
\end{itemize}


Similarity is calculated using Euclidean distance between the selected features. Each feature is normalized beforehand to ensure equal weighting.
This approach works for categorical features too, since they must be numerically encoded for use in \textsc{finch}.
By calculating similarity only on selected features, we have access to many more instances compared to requiring similarity across all features.

Instances are treated as almost identical to the current instance if their distance is smaller or equal to the number of columns times 0.1: $d <= len(columns) * 0.1$. 
This value was chosen experimentally.


As our procedure ensures at least 50 instances to be selected for each subset, when an instance does not have highly similar instances, the subset may contain instances that are less similar. Thats why in \textsc{finch}, we caracterize the subset using distribution plots to let the user know how similar the instances are, allowing them to gauge the credibility of the curve.


\subsubsection{Categorical Features}
Our current algorithm adapts to a categorical target by letting the user select one of the classes and visualizing the probability of belonging to that class on the y-axis. 
Further categorical features can be entered into the tool in two ways: through numerical encoding (which creates an implicit ordering) or through one-hot encoding (where each possible feature value becomes its own feature).
When using categorical features, experts should note that numerical encoding creates an implicit ordering that affects similarity calculations between instances. With one-hot encoding, \textsc{finch}'s ability to handle higher-order interactions enables it to consider relationships across all encoded feature values.

\section{Presenting \textsc{FINCH}}
Based on our subset-based algorithm, we designed the visual analytics tool \textsc{Finch} that enables machine lerning experts to incrementally explore feature interactions.
In this section, we outline the initial requirements for \textsc{Finch}, explain its core concept, discuss how the tool facilitates trust calibration, and describe its implementation.

\subsection{Requirements}
The requirements were derived from literature review, as well as discussions with three machine learning experts of varying expertise (ranging from grad student to professor). The discussions were guided by using a rain prediction example, from which they described or sketched the information and visualizations needed to understand feature interactions. They reflected on their past experiences with xAI tools, sharing what they appreciated and what they found lacking. In combination with the literature review, we formulated the following requirements.

\begin{itemize}
    \item (R1) Understandable feature interactions:
    Fundamental to our tool was making feature interactions understandable to experts. Using explanations without completely understanding them often leads to false assumptions~\cite{kaur2020interpreting}.
    \item (R2) Differentiate between positive and negative contributions: 
    Machine learning experts prefer to see how each feature or feature interaction contributes to the prediction, whether positive or negative. This is one of the basic requirements for feature attribution methods outlined by Lundberg et al.~\cite{lundberg2018consistent}.
    \item (R3) Highlight subgroups, outliers, and special characteristics: 
    The experts wanted to know what was unusual about the current instance, which is also seen as the most important information in an explanation in the social sciences~\cite{miller2019explanation}.
\end{itemize}

\subsection{Visual Design of the Dependence Plots}
Our plots are designed to be interpretable in a smiliar manner to PDPs, making it easy for experts to transition between them. The line plots display the relationship between features (x-axis) and predictions (y-axis) through a clear curve. To enhance usability, we incorporate several visualization techniques (Fig. \ref{fig:dependency_plot}).

\begin{itemize}
    \item \textbf{centered around mean}: To enhance differentiation between positive and negative contributions (R2), a horizontal line marks the mean prediction, making it easy to see when predictions are above or below it.
    \item \textbf{colored background}: To further emphasize this, the background is colored in blue below the line, and red above it.
    \item \textbf{+/- symbols}: The two symbols are used to indicate that values the areas indicate predictions above, or below the line.
    \item \textbf{two axes}: Accordingly, the y-axis is also centered around the mean prediction. To allow reading the absolute values, a second y-axis is positioned on the right side of the plot. 
    \item \textbf{highlight whats important}:  We use highlighting through saturated red and blue tones that stand out from the background, depending on the specific visualization.
    \item \textbf{mark current instance}: The x-axis feature value of the current instance is marked by a green vertical line.
\end{itemize}
 
\subsection{Visualizing Interactions Incrementally}

\begin{figure}[h!]
    \centering
    \begin{subfigure}[b]{0.5\textwidth}
        \centering
        \includegraphics[width=1\linewidth]{dependency_plot.png}
        \caption{bike rentals per hour}
        \label{fig:dependency_plot}
    \end{subfigure}
    \hspace{0.001\textwidth}
    \begin{subfigure}[b]{0.5\textwidth}
        \centering
        \includegraphics[width=1\linewidth]{weekend.png}
        \caption{bike rentals per hour on weekends}
        \label{fig:weekend}
    \end{subfigure}
    \hspace{0.001\textwidth}
    \begin{subfigure}[b]{0.5\textwidth}
        \centering
        \includegraphics[width=1\linewidth]{winter_weekend.png}
        \caption{bike rentals per hour on weekends in winter}
        \label{fig:winter_weekend}
    \end{subfigure}
    \caption{The incremental visualization of the interaction of hour, weekends, and winter in the bike rental data set. Colored areas visualize the change in each step.}
\end{figure}

The crucial point of our tool was combining the individual curves for each incremental subset into an intuitive explanation.
For each additional feature added to the current interaction, a new curve is generated.
Simply showing all curves at once quickly leads to cluttered visualizations
We therefore carefully selected which curves to include at each step of the process.
The base curve, calculated based on all instances, is included in all visualizations. This ensures a consistent base for comparison. It is displayed in an unobstrusive grey.
The current subset is displayed as a purple curve, being the most prominent curve of the plot.
The previous subset is visible as a desaturated purple curve to enable comparison.
All older curves are hidden to reduce visual clutter.

We employ highlighting techniques to lead the experts attention to the change (Fig. \ref{fig:dependency_plot} - \ref{fig:winter_weekend}).
\begin{itemize}
    \item \textbf{one feature}: Only the base prediction is visible, showing the base prediction across the values of the first feature, that is depicted on the x-axis. We highlight the difference of the base prediction to the mean (Fig. \ref{fig:dependency_plot}).
    \item \textbf{two features}: A purple line is added for the first subset generated from the new feature. We highlight the difference between the base prediction and the subset (Fig. \ref{fig:weekend}).
    \item \textbf{three or more features}: The previous subset is shown as a desaturated purple line, and the new subset displayed in purple. We highlight the difference between the last subset and the current subset (Fig. \ref{fig:winter_weekend}). Alternatively, the expert can choose to highlight the difference of the current subset to the base prediction instead.
\end{itemize}

\subsubsection{Separating Main and Interaction Effects}
\label{sec:interaction}
One of the optional visualizations that \textsc{Finch} provides is the separation of main and interaction effects.

When adding a new feature to the subset, we highlight the difference between the previous prediction without, and the current prediction including it. This difference can be divided into two components: the main effect of the newly added feature and its interaction effect with previously included features.

In the bike rental scenario, in winter there are overall fewer bike rentals due to its main effect. Additionally, winter interacts with hour, resulting in even fewer bike rentals in the morning but slightly more in the evening on winter days.


Mathematically, we can summarize the previously predicted curve as $c + f_X(y)$ where  $c$ is the mean prediction (as our visualizations center around the mean) and $f_X(y)$ the influence of the previous subset $X$ on the variable on the x-axis, $y$. When we add a new feature $Z$, the updated function becomes $f_{X,Z}(y) = c + f_X(y) + a_Z + g^X_Z(y)$, where $a_Z$ represents the main effect of $Z$, and $g^X_Z(y)$ captures the interaction effect of $Z$ with all previous features. 
We can calculate $a_Z$ independently of all other features as the average of all instances that are in the subset defined by only $Z$.   

To help experts distinguish between these effects, we provide an option to display a new line in our plot showing only the main effect of the newly added feature added to the previous prediction. Since the main effect is independent of all other features, including the one on the x-axis, it shifts the previous curve up or down by a constant. We then highlight the difference between this new line and the actual prediction, revealing the interaction effect between the new feature and previously added features (Fig. \ref{fig:interaction_effect}).

\begin{figure}
    \centering
    \includegraphics[width=1\linewidth]{interaction_effect.png}
    \caption{Interaction effect visualization. This visualization separates the main and interaction effects of a newly added feature by showing the main effect through a blue dotted line and the interaction effect through its difference from the actual purple line.}
    \label{fig:interaction_effect}
\end{figure}

\subsubsection{Choosing the Next Feature}
We use small multiples to help experts choose the next feature that should be added to the currently visualized interaction.

For each potential feature to be added to the subset, we display its dependence plot, ranking features by the strength of their interaction effect at the instance's x-axis value. 
The user can choose to either view the standard dependence plots, or those displaying the interaction effects ass described in Section \ref{sec:interaction}.

\subsection{Calibrating Trust}
xAI is subject to various cognitive biases~\cite{bertrand2022cognitive}. For instance, merely having an explanation, even if not meaningful, can increase trust in the explained prediction~\cite{eiband2019impact}. Our explanations are approximations of the model's behavior based on the provided datasets, and the model may not always accurately reflect the data. Thus, it is crucial to provide machine learning experts with a variety of tools to calibrate their trust in both the explanations and the model to combat cognitive biases.

Displaying additional information directly on the existing plots was deemed disadvantageous, as it made our plots too cluttered. We therefore decided to display the information in separate plots, or provide different views of the plots that hide aspects of it to make space for the new information. We aimed to make each plot version look slightly different, to help experts in keeping track of which view they are currently seeing, whilst still keeping visual consistency to enable easy interpretation (R1).

\subsubsection{Subset Characterization}
First, we provide a detailed characterization of the subset used for our calculations, including the number of instances and their distribution across selected features compared to the general distribution. The distributions of features used for neighborhood calculations are shown on the right, while the distribution for the x-axis feature is displayed directly below the dependence plot for quick reference.

Both distributions are visualized using heatmaps (Fig. \ref{fig:density}). Their brightness reflects the relative number of instances per distribution rather than the absolute count. This ensures that the distribution of the subset is visible, even when it is much smaller than the data set. The current instance is indicated by a green line, as in the dependence plot.

\begin{figure}
    \centering
    \includegraphics[width=0.7\linewidth]{density_small.png}
    \caption{Distribution heatmap. Directly below the dependence plot are the density distributions for the first selected feature, which is visualized on the x-axis.}
    \label{fig:density}
\end{figure}

\subsubsection{Ground Truth}
\textsc{Finch} uses only real instances, without permuting or generating new ones. By working with actual data, we allow users to provide the ground truth for these instances, enabling direct comparison between the model’s predictions and true values. When the expert selects the ground truth view, it is displayed as a dotted line, with differences between the model's prediction and the ground truth highlighted in blue or red areas. For example, in Figure~\ref{fig:truth}, the model's predictions generally align with the ground truth but exceed it beyond a certain threshold.

\begin{figure}
    \centering
    \includegraphics[width=1\linewidth]{truth.png}
    \caption{Truth visualization. The deviation between the model prediction and ground truth is highlighted, and the ground truth is displayed through a dotted line.}
    \label{fig:truth}
\end{figure}

\subsubsection{Uncertainty}
The uncertainty induced by mean approximation, as well as the uncertainty innate to the data are addressed through an uncertainty visualization.
This visualization uses a simple area plot to display the standard deviation around the mean for each feature value along the x-axis.

As shown in Figure~\ref{fig:uncertainty}, the predicted curve shows greater deviation at the peaks and less deviation at lower feature values. It is important to note that this visualization represents the uncertainty of the mean predictions for each feature value, not the model's uncertainty when predicting an individual instance.

\begin{figure}
    \centering
    \includegraphics[width=1\linewidth]{uncertainty.png}
    \caption{Uncertainty visualization. The standard deviation at each feature value is shown as an area around the curve.}
    \label{fig:uncertainty}
\end{figure}



\subsection{Implementation}
\textsc{Finch} is implemented using the Python library Panel\footnote{https://panel.holoviz.org/index.html} for the user interface and Bokeh\footnote{https://docs.bokeh.org/en/latest/docs/gallery.html} for visualizations. The UI of \textsc{Finch} is shown in Figure~\ref{fig:UI}. A test version is hosted on Hugging Face Spaces. Currently, it supports loading all sklearn models saved with python version 3.11 and sklearn version 1.5.1, and datasets saved as CSV files. The code is open source and available on GitHub\footnote{https://github.com/akleinau/Finch}.

The user initially loads a data set (here bike-rentals) and gets an overview of the dependence plots for all features, centered on the current instance's values. Selecting a feature takes the user to the main \textsc{Finch} view, where additional features can be added to refine the subset.


Alternatively, users can load a different data set and model, with the option to provide the ground truth. If there are multiple outcome variables, the desired one can be selected and renamed if needed. Users can choose from the instances of the data set or create a custom one, and any missing values are automatically imputed with the mean.

\textsc{Finch} is tested for both regression and classification models. 
Supported features include continuous and categorical/ordinal types, with any feature having fewer than 24 unique values classified as categorical/ordinal.
Curves of features with more than 24 different values are smoothed using an exponentially weighted average for a smooth appearance that reduces the noise of random statistical fluctuations. The degree of smoothing increases when less data is available or when subsets contain fewer instances. Smoothing can be optionally deactivated.
Categorical feature curves are not smoothed to keep predictions for each class accurate, with the threshold of 24 chosen to preserve the ordinal nature of hourly data.

\begin{figure}
    \centering
    \includegraphics[width=1\linewidth]{interface.png}
    \caption{The user interface of \textsc{Finch}. On the left side, the dataset, target variable, and instance can be selected. The main view shows the dependence plot and the distribution heatmaps. On top are options to select features, and on the bottom are different views of the plot.}
    \label{fig:UI}
\end{figure}

\section{Case Studies}
We want to demonstrate the usefulness and generalizability of \textsc{Finch} using a set of different data sets with mixed features, different machine learning models, as well as comparisons to common xAI methods. The chosen data sets are inspired by the case studies of Herbinger et al.~\cite{herbinger2022Repid}. Each data set and model is described in detail in Appendix \ref{AppendixA}.

\subsection{Titanic data set}
The titanic data set is used for predictions of the survival of people on board of the titanic, by using features like age, sex and class~\cite{dawson1995unusual}.

\subsubsection{Zero Interactions - Comparison with SHAP}
When analyzing how a prediction for a specific instance came to be, a typical first step would be to generate a \textsc{SHAP} plot showing each features influence. Here, we show the survival chance of a 30yo woman. \textsc{SHAP} gives you a general overview of all features, but also requires all features to be specified (Fig. \ref{fig:shap_titanic}). Feature interactions are considered during the calculation, but they are not visible to the end user. Therefore, the machine learning expert may now ask how specifically class, sex, and age played together. Our tool shows how they interact.

\begin{figure}
    \centering
    \includegraphics[width=1\linewidth]{SHAP_titanic.png}
    \caption{A SHAP plot generated for an instance of the titanic data set. The influence of all features is visualized independently by calculating a \textit{shap value} for each feature, stating its positive or negative influence on the prediciton. Here, the survival of a person on board the titanic is predicted.}
    \label{fig:shap_titanic}
\end{figure}

\subsubsection{One or Two Interactions - Comparison with PDP}
When only considering the interactions of one feature, like class, \textsc{finch} plots look relatively similar to PDPs.
Both show the survival chance per class of the titanic, with first class travelers having the best chance at survival.
However, compared to PDPs, our plots are enhanced with guiding colors and highlights.
The PDP does not give a lot of guidance (Fig. \ref{fig:pdp_titanic}).
What differentiates both plots most is the underlying calculation of probabilities, as described in Section \ref{methods_pdps}. 
The way of PDP generating new data points may result in unrealistic data points, when, for example, additionally to the class, a wealth feature would be available that correlates strongly with it but is not changed accordingly.
In our case, the dependence seen in the PDP is more linear than the one our algorithm has found. 
One explanation for that is by just using realistic data points, our plot takes into account that people who travel in the first class probably also are privileged in other aspects that may lead to heightened survival chances. 

\begin{figure}
    \centering
    \includegraphics[width=1\linewidth]{one_feature.png}
    \caption{Comparing plots for first order feature interactions. Left: our plot. Right: PDP line chart.}
    \label{fig:pdp_titanic}
\end{figure}

\begin{figure}
    \centering
    \includegraphics[width=1\linewidth]{two_features.png}
    \caption{Comparing plots for second order feature interactions. Left: our plot. Right: PDP heatmap. }
    \label{fig:titanic_2F}
\end{figure}


When scaling up to two features by also considering the age, the \textsc{FINCH} plot stays relatively similar, simply adding another line showing the probability change for the current instance of age 30 (Fig. \ref{fig:titanic_2F}).
However, as PDP plots are designed to show global interactions, they are now displayed as heatmaps. This reveals more information, at the cost of requiring significantly more time to interpret.
Additionally, our plot highlights the mean probability, which is not highlighted in the PDP.

\subsubsection{Higher Order Interactions}
\textsc{Finch} is uniquely able to display feature interactions of more than two features. When additionally considering the gender of the person, the plot shows that 30yo women have a significantly increased probability of surviving, but especially when in first class (Fig. \ref{fig:titanic_3F}). Additionally, as \textsc{finch} is uniquely able to compare this to the ground truth, the dotted line and blue highlighted area display how the model categorically underestimated the survival of those women.

\begin{figure}
    \centering
    \includegraphics[width=1\linewidth]{titanic_3F.png}
    \caption{Survival change per class. General public compared to a 30yo woman. Their chance of survival was greater, but especially when in first class.}
    \label{fig:titanic_3F}
\end{figure}

To show the generalizability of \textsc{finch}, we will now display a range of interactions found in other data sets.

As a nonlinear and continuous example, we use the California Housing dataset~\cite{pace1997sparse}. It contains a continuous target variable and continuous features, and is used to predict house values. We trained a gradient boosting regressor model on this dataset.

Using \textsc{Finch}, we successfully identified several higher-order interactions. Detailed depictions of each interaction can be found in Appendix \ref{AppendixA}.
\begin{itemize}
    \item Median income strongly predicts house values, increasing linearly from 2 to 9 and remaining relatively stable above and below these values.
    \item For populations of 2,000, housing values per income behave similarly, except that they stabilize at a lower value for incomes above 8.
    \item However, when these populations have approximately 1,000 total rooms, housing values decrease significantly. For incomes below 4, values remain relatively stable, but then stabilize significantly lower instead of increasing linearly.
    \item A comparison with ground truth for this subset shows the model is relatively accurate, though it slightly overestimates housing values for incomes above 4.
\end{itemize}

The last example features a categorical target and categorical features using a subset of the BRFSS dataset~\cite{BRFSS2015}, which focuses on diabetes risk factors. We trained a decision tree classifier on this dataset.
Using \textsc{Finch}, we observed the following interactions.
\begin{itemize}
\item Diabetes risk in the general population is roughly equal between men and women.
\item Exercise reduces diabetes risk more significantly in women than in men.
\item When combined with high blood pressure, exercise's protective effect is neutralized, resulting in an elevated but equal risk for both men and women.
\item Among those who are also obese, the risk increases further, particularly for women.
\end{itemize}



\section{Evaluation}
Throughout the development of \textsc{Finch}, we continuously tested the tool with various visualization and machine learning experts, leading to significant improvements. To further validate the final version, we conducted a user study with machine learning experts of varying expertise in xAI methods.
We specifically focused on the goals of understandability, usability and helpfulness.

\subsection{Study design}
Participants were recruited by contacting machine learning experts.
The five participants (2F, 3M) had an average age of 35, ranging from 32 to 45.
They work as professors, researchers, graduate and postgraduate students in the field or had significant experience from a previous job.
One participant also participated in the requirement analysis study. None are authors of this paper.
The study was conducted either in person or via online video calls. We began by collecting demographic information and learning about the participant's previous experiences. 
Next, we introduced our bike rental example dataset to create a simple tutorial scenario, providing a brief description of the prediction task followed by a SHAP graph to overview the current instance and its prediction. We chose SHAP as a widely known explanation method~\cite{holzinger2022explainable}. \textsc{Finch} complements SHAP by offering additional functionality for exploring feature relationships. We guided participants through the tool's features using a set of tasks, explaining how it works along the way.

Participants then worked independently with a dataset on diabetes risk factors derived from the BRFSS telephone study~\cite{BRFSS2015}. They were tasked with answering questions to assess their ability to use the tool independently.

Throughout the process, participants were encouraged to verbalize their thoughts using the think-aloud method. At the end of the session, they rated their experience through a series of questionnaires.
We used the widely recognized System Usability Scale (SUS)~\cite{Brooke1996SUS} to evaluate the tool's usability. To assess the provided explanations regarding their helpfulness, we employed the explanation satisfaction scale (ESC) by Hoffman et al.~\cite{hoffman2018metrics}. We also included specific questions to measure how well our tool met requirements and participants' satisfaction with the visualizations (Fig. \ref{fig:Custom1} and \ref{fig:Custom2}). 

\subsection{Study Results}
All participants reported high experience in machine learning. Their familiarity with explainable AI varied from moderate to very experienced. All were well-acquainted with SHAP and PDP, except for one participant being unfamiliar with PDP.

We split up our findings into the three design goals of understandability, usability and helpfulness.

\subsubsection{Understandability}

During the study, all participants quickly grasped the concept of subsets for one feature. One participant described it as \say{nice and intuitive}. Some participants took longer to understand how additional features were visualized when added to the subset, but with some help all understood it and were able to use and interpret it (meet R1). 

While all participants understood the uncertainty visualization, there was initial confusion due to the varying meanings of "uncertainty"; for instance, one participant thought it referred to the model's prediction uncertainty.
The most challenging aspect for participants was the interaction effect visualization. All struggled with it, though three eventually felt they understood it.

The results from our custom questions were largely positive (see Figure~\ref{fig:Custom1}). Participants agreed that the tool emphasizes unique data aspects (meet R3), distinguishes positive and negative contributions of features (meet R2), and provides an easily interpretable summary of features.
The only negative feedback concerned whether the tool helps validate displayed results. Discussions revealed varied interpretations of this question; while we aimed to assess trust calibration features, some participants interpreted it as evaluating model validation, explaining the mixed responses. 

\begin{figure}[h]
    \centering
    \includegraphics[width=1\linewidth]{Custom1.jpg}
    \caption{Answers to our custom questions about the tools functionality. For each question, colored bars display the number of people per answer, centered around the neutral answer with disagreeing answers to the left and agreeing answers to the right.}
    \label{fig:Custom1}
\end{figure}

\subsubsection{Usability}
When using the tool on the diabetes dataset, all participants frequently adjusted feature values to see how predictions changed. Three participants used the ground truth and uncertainty features to validate their results.

One participant suggested increasing contrast for clarity and using textures for colorblind users.
One recommended including a logical description of the current subset for better clarity and a more precise legend for easier recall.
Participants expressed a desire for units to be displayed alongside the numbers. One requested additional information, such as a global explanations, while another wished to move or delete individual features. One participant also wanted to read specific values directly from the chart.
One person raised concerns about scalability, as the tool occasionally ran slowly,
Additionally, one participant found the feature overview overwhelming.
The additional visualizations displaying data distribution, trust, and uncertainty were well received. One participant suggested adding a color legend for the data distribution heatmap. The ground truth visualization was considered very important by one participant, while another expressed interest in using the tool directly on the ground truth for insights before training a model.

Our tool received a score of 82 on the SUS, indicating excellent usability.
Overall, participants found the charts easy to interpret, visually pleasing, and helpful for identifying and interpreting feature interactions (Fig. ~\ref{fig:Custom2}).
Three experts provided freeform feedback. Two suggested improvements, including, in addition to what was discussed already, an info button to explain the visible interactions and functionality enabling users to \say{play} with the tool by clicking on data points to update values. Two freeform comments also included praise of the tool.

\begin{figure}[h]
    \centering
    \includegraphics[width=1\linewidth]{Custom2.jpg}
    \caption{Answers to our custom questions about the visual chart design. For each question, colored bars display the number of people per answer, centered around the neutral answer with disagreeing answers to the left and agreeing answers to the right.}
    \label{fig:Custom2}
\end{figure}

\subsubsection{Helpfulness}
The ESC gives an overview of how helpful our tool was seen (see Fig. ~\ref{fig:ESG}). Experts agreed that our explanation helps them understand the model, is satisfying, provides sufficient detail, and is useful for their goals. However, they were neutral regarding the completeness of the explanation and its guidance on using the model. Opinions varied on whether the explanation conveyed the model's accuracy, reliability, or trustworthiness.

One expert noted they had used a similar approach in their work, manually creating visualizations of different subsets of the data. 
Participants noted that the tool is limited to tabular data, restricting its applicability. Another pointed out that our fuzzy definition of neighborhoods might be problematic for users seeking specific values. 

\begin{figure}[h]
    \centering
    \includegraphics[width=1\linewidth]{ESG.jpg}
    \caption{Answers to the explainability satisfaction scale. For each question, colored bars display the number of people per answer, centered around the neutral answer with disagreeing answers to the left and agreeing answers to the right.}
    \label{fig:ESG}
\end{figure}

\section{Discussion}

We present two simple scenarios where an ethics-based shaping algorithm helps RL agents make ethically sound decisions while still achieving their primary objectives. These scenarios serve as placeholders for trivial daily activities, demonstrating how ethical considerations can be seamlessly integrated into routine tasks within reinforcement learning environments.
Leveraging large language models to inform agent behavior allows us to build on existing work that highlights the limitations of traditional reward shaping.
Our approach not only enhances ethical performance but also demonstrates the importance of incorporating diverse moral theories, addressing moral uncertainty effectively.
% This positions our work at the forefront of evolving methodologies aimed at fostering responsible AI development and deployment.
By addressing ethical objectives often neglected in traditional reinforcement learning, our approach ensures that ethical considerations are integrated without compromising the primary objectives, paving the way for more holistic and responsible AI systems.

From our simple examples, we see significant improvement in behavior when incorporating diverse moral philosophies, showing the importance of moral decision-making in RL contexts.
Our ethics-shaping approach simplifies the design of a value-aligned RL system.
This approach divides the complex task into two distinct layers: the first layer focuses on achieving the primary goal, while the second, ethical layer refines the outcome to meet the secondary ethical goal. This two-layered structure allows the system to address core objectives initially, followed by ethical adjustments to enhance overall responsibility.
However, we do see that this approach is sensitive to i) the reasoning quality of the language model, and ii) the availability of feedback samples to shape learning.
Language models do not have the long-term planning to guide their reasoning, which makes them underperform a well-crafted reward function for long-term, spatial tasks.
Human samples on the other hand would provide the best "human-aligned" feedback but might be too sparse when dealing with large and complex state-action spaces.
We also see that different moral frameworks can result in different priorities for performing the ethical tasks.
These insights emphasize the value of understanding and optimizing moral frameworks for developing agents capable of addressing complex ethical challenges. Overall, our research tries to lay a solid foundation for future exploration into enhancing moral reward systems and improving spatial reasoning in large language models.


Our findings contribute significantly to the ongoing discourse in the intersection of AI and ethics, particularly within reinforcement learning frameworks. As innovation continues to drive autonomous technologies, ethical decision-making becomes increasingly crucial in autonomous systems. Although our work presents promise in using ethically aligned LLM agents to integrate moral reasoning into AI, it should be viewed as complementary to the human-in-the-loop philosophy and is best suited for safe application in routine, low-stakes activities (e.g., personal assistant robots, kitchen assistant robots). For complex, high-stakes decision-making tasks with significant societal and community impacts, further research and rigorous testing are essential. Our ethical LLM agents work best to bootstrap learning algorithms, in cases where we would lack fine-grained feedback from human oversight. By design, eventually the LLM feedback can be replaced by feedback from not just one, but multiple human evaluators. Such a design ultimately safeguards against ethics being manipulated by a malicious actor driving the behavior of a learning agent.

% Some relevant applications for our proposed framework include smart home assistants and social media platforms. In smart home assistants, AMULED can guide user interactions in an ethically aware manner by suggesting meal options that align with dietary preferences, ethical eating (e.g., plant-based recipes), and sustainability goals, thereby promoting healthier and environmentally friendly choices. On social media platforms, AMULED can enhance user engagement by moderating content with an ethical lens, evaluating the implications of comments and posts to foster constructive discussions and prioritize positive interactions. This approach helps create a safer, more respectful online community, improving user satisfaction and fostering inclusivity.

While our framework aims to comprehensively represent major ethical paradigms for AI decision-making, we acknowledge that categorizing ethical theories into distinct clusters may oversimplify the complexities and intersections among different moral philosophies. This structured approach is intended to facilitate the practical implementation of ethical reasoning in AI systems. However, operationalizing these theories involves abstracting intricate philosophical ideas while we strive to preserve the core principles of each ethical approach. We selected representative theories and decision factors based on their prominence and relevance in the literature, recognizing that some degree of subjectivity is inherent in this process.

Our framework is based on key assumptions: we believe that the five clusters effectively encompass the major streams of ethical thought pertinent to AI decision-making and that the selected theories within each cluster are sufficiently representative of their respective ethical approaches. We also assume that the identified key concepts and decision factors can be meaningfully translated into computational models, providing a solid foundation for future research.

Recognizing that different countries and cultures can value certain moral beliefs over another, AMULED was designed to have modular selection of ethical frameworks, rather than imposing a single, moral philosophy to drive decisions.
While we highlight that this modular, multi-moral approach avoids the pitfalls of sticking to a fixed moral framework, we do recognize that the representation of belief values generated from LLMs may be prone to external biases and subjectivity.
This complicates the design of a universally accepted modular decision-making framework. 
Moreover, the dynamic nature of ethics poses a challenge, as ethical norms can evolve over time and vary across cultures, potentially rendering static models ineffective. Accountability is another critical issue \cite{helbing2021summary}; ambiguity arises regarding who is responsible for the outcomes generated by AI, whether it be the developers, the organization deploying the AI, or the model itself. To build trust and accountability, enhancing the transparency of the AMULED model's decision-making processes is essential, potentially through techniques that provide explanations, visualizations, or justifications for its ethical reasoning. This approach would improve user confidence and ensure that the AI's ethical framework aligns with societal values and norms \cite{helbing2024converging}.

% To address these limitations, future research could focus on developing hybrid ethical models that dynamically integrate multiple ethical frameworks, allowing the AI to adjust its decision-making approach based on contextual factors and evolving norms. Enhancing the interpretability of how AMULED influences decisions can help build trust, while investigating real-time adaptation methods could improve the system's relevance. Additionally, studying ethics in diverse contexts and conducting research on mitigating biases in belief representation are essential for more equitable outcomes. Exploring multi-agent scenarios and establishing benchmarks for evaluating the effectiveness of AMULED can further enhance its robustness and alignment with societal values.

The integration of belief probability assignment (BPA) into reinforcement learning presents a transformative approach to navigating ethical decision-making in AI systems. Aggregating beliefs from multiple feedback sources (such as ethical LLMs and human evaluators) enables AI to assess and balance competing ethical models effectively. This capability is particularly crucial in situations marked by normative uncertainty, where differing moral frameworks such as utilitarianism and deontology can yield conflicting recommendations. These techniques help the AMULED system evaluate ethical choices with a nuanced understanding of diversity and confidence in various theories. This iterative approach allows for the continuous refinement of decision-making strategies, ensuring that actions are aligned with widely accepted ethical standards. Furthermore, our research emphasizes the need to expand beyond single moral values, accommodating a multi-value system that reflects real-world complexities. By exploring applications in dynamic environments, such as multi-agent systems and human-AI collaboration, we aim to enhance ethical robustness in AI, paving the way for more responsible and adaptable autonomous systems.

% Our work presents two simple scenarios in which an ethics-based shaping algorithm helps RL agents make ethically sound decisions while still achieving their primary objectives. These scenarios serve as placeholders for trivial daily activities, demonstrating how ethical considerations can be seamlessly integrated into routine tasks within reinforcement learning environments. Leveraging large language models to inform agent behavior allows us to build on existing work that highlights the limitations of traditional reward shaping.

% Our approach not only enhances ethical performance but also demonstrates the importance of incorporating diverse moral theories, addressing moral uncertainty effectively. This positions our work at the forefront of evolving methodologies aimed at fostering responsible AI development and deployment. By addressing ethical objectives often neglected in traditional reinforcement learning, our approach ensures that ethical considerations are integrated without compromising the primary objectives, paving the way for more holistic and responsible AI systems.

% From our simple examples, we see significant improvement in behavior when incorporating diverse moral philosophies, showing the importance of moral decision-making in RL contexts. Our ethics-shaping approach simplifies the design of a value-aligned RL system. This approach divides the complex task into two distinct layers: the first layer focuses on achieving the primary goal, while the second, ethical layer refines the outcome to meet the secondary ethical goal. This two-layered structure allows the system to address core objectives initially, followed by ethical adjustments to enhance overall responsibility.

% However, we do see that this approach is sensitive to i) the reasoning quality of the language model, and ii) the availability of feedback samples to shape learning. Language models do not have the long-term planning to guide their reasoning, which makes them underperform a well-crafted reward function for long-term, spatial tasks. Human samples, on the other hand, would provide the best "human-aligned" feedback but might be too sparse when dealing with large and complex state-action spaces. We also see that different moral frameworks can result in different priorities for performing the ethical tasks.

% These insights emphasize the value of understanding and optimizing moral frameworks for developing agents capable of addressing complex ethical challenges. Overall, our research tries to lay a solid foundation for future exploration into enhancing moral reward systems and improving spatial reasoning in large language models.

\section{Conclusion}
This paper has introduced \sysName, a novel LLM-based system to simplify the creation of 2D interactive scenes without coding from natural language input and graphical control. We utilized content analysis on video tutorials about creating interactive scenes using ChatGPT and existing AI coding tools, and distilled several issues including the lack of independent generation and refinement, graphical control, and precise refinement. Based on these findings, we proposed a context-aware modularization technique that processes textual descriptions through individual LLM modules, with a central module coordinating interactions, allowing for independent refinement of each element. Our graphical user interface combined these modular LLMs with advanced graphical controls, enabling seamless code generation for 2D interactive scenes and direct integration of graphical information. We conducted a comparative study between \sysName~and \yh{Cursor Composer}, and found \sysName~significantly reduced the time, prompt trials, and prompt lengths and achieved better graphical and precise control in creating interactive scenes. Another open-ended usability study demonstrated that \sysName~allowed users to create various desired scenes easily without the need for coding, benefiting diverse application scenarios.


\bibliographystyle{abbrv-doi-hyperref}
\bibliography{template.bib}

\section*{Suppplementary Material}
The code is open source and available on GitHub\footnote{https://github.com/akleinau/Finch}. Any currently running online versions of the tool can be found there too.

\appendix

\section{Example Addition According to Formalization}
\label{appendix:A}
We show a concrete example for two-operand addition according to the formalization defined in Section \ref{subsec:digit10}. For \textbf{$147 + 255$}, we have:

$k=2, d=3, n1 = [1, 4, 7], n2 = [2, 5, 5]$. 

We then compute:

\[T_2 = c_2 + 1 + 2\] \[T_1 = c_1 + 4 + 5\] \[T_0 = c_0 + 7 + 5= 0 + 7 + 5 = 12\] \[s_0 = 12 \mod 10 = 2, \quad c_1 = \left\lfloor \frac{12}{10} \right\rfloor = 1\] \[T_1 = 1 + 4 + 5 = 10\]  \[s_1 = 10 \mod 10 = 0, \quad c_2 = \left\lfloor \frac{10}{10} \right\rfloor = 1\] \[T_2 = 1 + 1 + 2 = 4\]  \[s_2 = 4 \mod 10 = 4, \quad c_3 = \left\lfloor \frac{4}{10} \right\rfloor = 0\] \[S = [0, 4, 0, 2]\] 


The result of the addition is $402$. 

\end{document}

\typeout{get arXiv to do 4 passes: Label(s) may have changed. Rerun}