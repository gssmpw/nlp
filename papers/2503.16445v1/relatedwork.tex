\section{Related Work}
In this section, we introduce the emerging concept of \emph{regional explanations} that inspired our tool, and discuss prior work on interpreting feature interactions and visual analytics in xAI.

\subsection{Regional methods}
Set between local and global methods,
an emerging category of explanation methods are regional methods, which focus on specific subgroups of the data. \textsc{Repid}~\cite{herbinger2022Repid} analyzes the ICE curves of a feature by identifying subgroups with different behaviors based on the shapes of individual instances' curves. A decision tree divides these groups while maintaining interpretable labels for easier understanding.

While \textsc{Repid} requires selecting an initial feature for exploring interactions, \textsc{Gadget}~\cite{herbinger2023decomposing} offers a more general approach by recursively decomposing the entire feature space to identify simple subspaces with minimal interactions, allowing for straightforward models in explanations. Like \textsc{Repid}, \textsc{Gadget} uses decision trees to guide the decomposition process.

Unlike \textsc{Finch}, \textsc{Repid} and \textsc{Gadget} do not identify feature interactions relevant to a specific instance; instead, they search for feature interactions globally by identifying subgroups.

\subsection{Feature Interaction Interpretation}
Few approaches exist for interpreting or visualizing higher-order feature interactions involving more than two features. Tsang et al.~\cite{tsang2021interpretable} provide a comprehensive review of feature interactions, emphasizing the need for improved interactive visualizations, which \textsc{Finch} aims to address. Zhang et al.~\cite{zhang2023capturing} take a mathematical approach, assuming product separability of feature interactions. Their method identifies these interactions and determines the most likely mathematical formula to represent them. Unlike \textsc{Finch}, this approach focuses on global feature interactions and is limited to this specific type of interaction. Friedman~\cite{friedman2024function} visualizes model interactions by breaking them down into a tree structure, allowing for the visualization of up to three features through line or bar charts. This tree provides a global overview of the interactions within the model, whereas \textsc{Finch} focuses on local feature interactions and can visualize interactions of more than three features.


Other approaches group features together, interpreting them as a unified group, which is especially useful for highly correlated features. For instance, Jullum et al.~\cite{jullum2021efficient} proposed computing Shapley values for groups rather than individual features. Ferretini et al.~\cite{ferrettini2022coalitional} also use grouping but as a step to enhance the calculation of final individual feature values. Going further, Mijolla et al.~\cite{mijolla2020human} propose an approach based on latent representations that redefine features as combinations of the original ones, making them easier for humans to understand. An automatic approach to latent representations are dimension reduction techniques like PCA, as demonstrated by Seedorff and Brown~\cite{seedorff2021totalvis}.

\subsection{Visual Analytics for xAI}

Our work aligns with a series of visual analytics tools designed for xAI. \textsc{Prospector}, by Krause et al.~\cite{krause2016interacting}, combines local and global explanation methods with an instance-centric interface. It builds on classical methods like PDPs but lacks functionality for higher-order feature interactions.

\textsc{Vine}, developed by Britton~\cite{britton2019vine}, is specifically designed to visualize feature interactions. Unlike \textsc{Finch}, it follows a global approach, clustering ICE curves for each feature, similar to \textsc{Repid}, and embedding this clustering in an overview and detailed views.

A similar approach was presented by Molnar et al.~\cite{molnar2023model}, who aimed to improve misleading PDPs by finding subgroups in the data with low feature interactions. Like \textsc{Gadget}, they compute a decision tree but visualize the results of each node in a PDP plot of a feature.



Hohman et al.~\cite{hohman2019gamut} designed a visual analytics system that offers various explanations for machine learning experts, although it does not cover higher-order interactions. They note that participants frequently switched between global and local explanations, highlighting the importance of interactivity.

Inglis et al.~\cite{Inglis2022Visualizing} present techniques for exploring pairwise interactions using matrix layouts of PDPs and network visualizations. \textsc{Finch} focuses on the visualization of interactions of even higher orders.


Lundberg et al.~\cite{lundberg2018consistent} created a VA system that uses SHAP force plots to visualize how different features influence the prediction for one instance over time on a clinical example. 
Their system visualizes all features in an independent manner, whilst ours shows their interaction.