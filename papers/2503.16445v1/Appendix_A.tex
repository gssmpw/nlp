\section*{Appendix A}
\label{AppendixA}

We used a series of data sets in our case studies and as examples in our paper. 

\subsection*{Bike Sharing}
The bike sharing data set is used to predict the number of bike rentals per hour. 

We trained a MLPC Regression model.

We used an 80:20 train:test split resulting in 13903 instances being used for training and in \textsc{Finch}.

In our example case, we used the following features:
\begin{itemize}
    \item count (target): the number of bike rentals that hour
    \item hour=3: the hour for which the bike rentals where recorded. Here: 3am.
    \item workingday=0: if the instance was recorded on a workingday or not. Categoric feature. 0=no, 1=yes.
    \item season=0: in which season the instance was recorded. Categoric feature. 0=winter, 1=spring, 2=summer, 3=autumn.
\end{itemize}



\begin{figure}[h!]
    \centering
    \begin{subfigure}[b]{0.5\textwidth}
        \centering
        \includegraphics[width=1\linewidth]{california1.png}
        \caption{house value per median income}
        \label{fig:california1}
    \end{subfigure}
    \hspace{0.001\textwidth} % Adds some space between the two images
    \begin{subfigure}[b]{0.5\textwidth}
        \centering
        \includegraphics[width=1\linewidth]{california2.png}
        \caption{house value per median income for areas with a population of 2000}
        \label{fig:california2}
    \end{subfigure}
    \hspace{0.001\textwidth} % Adds some space between the two images
    \begin{subfigure}[b]{0.5\textwidth}
        \centering
        \includegraphics[width=1\linewidth]{california3.png}
        \caption{house value per median income for areas with a population of 2000 and 1000 total rooms}
        \label{fig:california3}
    \end{subfigure}
    \hspace{0.001\textwidth} % Adds some space between the two images
    \begin{subfigure}[b]{0.5\textwidth}
        \centering
        \includegraphics[width=1\linewidth]{california4.png}
        \caption{The ground truth is even lower.}
        \label{fig:california4}
    \end{subfigure}
    \caption{The incremental visualization of the interaction of median income, population and total rooms in the california housing data set. Colored areas visualize the change in each step.}
\end{figure}

\subsection*{Titanic}
The titanic data set is used to predict the survival of people on board the titanic. 

We trained an MLPC classifier. The resulting accuracy was 70.23\%.

We used an 80:20 train:test split resulting in 1047 instances being used for training and in \textsc{Finch}.

In the described interaction, we used the following features:
\begin{itemize}
    \item survival(target): If the current person survived.
    \item pclass=1: Which passenger class the current person belonged to. Categoric feature. 1=first class, 2=second class, 3=third class. 
    \item sex=1: The sex of the person. Categoric feature. 0=male, 1=female.
    \item age:30: The age of the person. Here: 30yo.
\end{itemize}




\subsection*{California housing}
The california housing data set is used to predict housing values for block groups in California and was derived from the 190 US census. It contains only continuous variables. The mean predicted housing value is 200.000. 

We trained a GradientBoostingRegressor model on the data. 
It was trained with 100 boosting stages, a learning rate of 0.1 and squared error as the loss function.
The resulting R2 score was 0.77.

We used an 80:20 train:test split resulting in 16,512 instances being used for training and in \textsc{Finch}.

In our observed interaction, we used the following features:
\begin{itemize}
    \item housing value (target): Median house value in US Dollars.
    \item median income: The median income of that block group in 100,000 US Dollars.
    \item population: The number of people residing in the block group.
    \item total rooms: The total number of rooms in that block group.
\end{itemize}

\subsection*{Diabetes}
The diabetes risk factor data set. It is based on the BRFSS telephone study that is performed yearly in the united states.

We used a subset of 10,000 instances for model training. Using a 80:20 train/test split, this resulted in 8000 instnaces being used in \textsc{Finch}.

For better model training, half of the instances are diabetes positive, and half negative. Therefore, the probabilities generated by the model and \textsc{finch} cannot be directly used on a general public.

In our interaction, we considered the following features:
\begin{itemize}
    \item diabetes risk (target): The diabetes risk for the person, that the model predicted.
    \item sex=0: The sex of the person. 0=male, 1=female.
    \item exercise=1: If the person exercises. 1=yes, 0=no.
    \item high blood pressure=1: If the person has high blood pressure. Categoric feature. 1=yes, 0=no.
    \item weight category=3: The weight category of the person. Categoric feature. 0=underweight, 1= normal weight, 2=overweight, 3=obese.  
\end{itemize}

\begin{figure}[h]
    \centering
    \begin{subfigure}[b]{0.5\textwidth}

        \includegraphics[width=0.7\linewidth]{diabetes1.png}
        \caption{diabetes risk per sex}
        \label{fig:diabetes1}
    \end{subfigure}
    \hspace{0.001\textwidth} % Adds some space between the two images
    \begin{subfigure}[b]{0.5\textwidth}
        \centering
        \includegraphics[width=1\linewidth]{diabetes2.png}
        \caption{diabetes risk per sex for people who exercise}
        \label{fig:diabetes2}
    \end{subfigure}
    \hspace{0.001\textwidth} % Adds some space between the two images
    \begin{subfigure}[b]{0.5\textwidth}
        \centering
        \includegraphics[width=1\linewidth]{diabetes3.png}
        \caption{diabetes risk per sex for people who exercise but have high blood pressure}
        \label{fig:diabetes3}
    \end{subfigure}
    \hspace{0.001\textwidth} % Adds some space between the two images
    \begin{subfigure}[b]{0.5\textwidth}
        \centering
        \includegraphics[width=1\linewidth]{diabetes4.png}
        \caption{diabetes risk per sex for people who exercise, have high blood pressure and are obese}
        \label{fig:diabetes4}
    \end{subfigure}
    \caption{Diabetes risk per sex. Incremental interaction of sex, exercise, high blood pressure and weight. Based on the BRFSS data set.}
\end{figure}

