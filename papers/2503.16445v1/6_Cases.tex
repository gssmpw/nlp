\section{Case Studies}
We want to demonstrate the usefulness and generalizability of \textsc{Finch} using a set of different data sets with mixed features, different machine learning models, as well as comparisons to common xAI methods. The chosen data sets are inspired by the case studies of Herbinger et al.~\cite{herbinger2022Repid}. Each data set and model is described in detail in Appendix \ref{AppendixA}.

\subsection{Titanic data set}
The titanic data set is used for predictions of the survival of people on board of the titanic, by using features like age, sex and class~\cite{dawson1995unusual}.

\subsubsection{Zero Interactions - Comparison with SHAP}
When analyzing how a prediction for a specific instance came to be, a typical first step would be to generate a \textsc{SHAP} plot showing each features influence. Here, we show the survival chance of a 30yo woman. \textsc{SHAP} gives you a general overview of all features, but also requires all features to be specified (Fig. \ref{fig:shap_titanic}). Feature interactions are considered during the calculation, but they are not visible to the end user. Therefore, the machine learning expert may now ask how specifically class, sex, and age played together. Our tool shows how they interact.

\begin{figure}
    \centering
    \includegraphics[width=1\linewidth]{SHAP_titanic.png}
    \caption{A SHAP plot generated for an instance of the titanic data set. The influence of all features is visualized independently by calculating a \textit{shap value} for each feature, stating its positive or negative influence on the prediciton. Here, the survival of a person on board the titanic is predicted.}
    \label{fig:shap_titanic}
\end{figure}

\subsubsection{One or Two Interactions - Comparison with PDP}
When only considering the interactions of one feature, like class, \textsc{finch} plots look relatively similar to PDPs.
Both show the survival chance per class of the titanic, with first class travelers having the best chance at survival.
However, compared to PDPs, our plots are enhanced with guiding colors and highlights.
The PDP does not give a lot of guidance (Fig. \ref{fig:pdp_titanic}).
What differentiates both plots most is the underlying calculation of probabilities, as described in Section \ref{methods_pdps}. 
The way of PDP generating new data points may result in unrealistic data points, when, for example, additionally to the class, a wealth feature would be available that correlates strongly with it but is not changed accordingly.
In our case, the dependence seen in the PDP is more linear than the one our algorithm has found. 
One explanation for that is by just using realistic data points, our plot takes into account that people who travel in the first class probably also are privileged in other aspects that may lead to heightened survival chances. 

\begin{figure}
    \centering
    \includegraphics[width=1\linewidth]{one_feature.png}
    \caption{Comparing plots for first order feature interactions. Left: our plot. Right: PDP line chart.}
    \label{fig:pdp_titanic}
\end{figure}

\begin{figure}
    \centering
    \includegraphics[width=1\linewidth]{two_features.png}
    \caption{Comparing plots for second order feature interactions. Left: our plot. Right: PDP heatmap. }
    \label{fig:titanic_2F}
\end{figure}


When scaling up to two features by also considering the age, the \textsc{FINCH} plot stays relatively similar, simply adding another line showing the probability change for the current instance of age 30 (Fig. \ref{fig:titanic_2F}).
However, as PDP plots are designed to show global interactions, they are now displayed as heatmaps. This reveals more information, at the cost of requiring significantly more time to interpret.
Additionally, our plot highlights the mean probability, which is not highlighted in the PDP.

\subsubsection{Higher Order Interactions}
\textsc{Finch} is uniquely able to display feature interactions of more than two features. When additionally considering the gender of the person, the plot shows that 30yo women have a significantly increased probability of surviving, but especially when in first class (Fig. \ref{fig:titanic_3F}). Additionally, as \textsc{finch} is uniquely able to compare this to the ground truth, the dotted line and blue highlighted area display how the model categorically underestimated the survival of those women.

\begin{figure}
    \centering
    \includegraphics[width=1\linewidth]{titanic_3F.png}
    \caption{Survival change per class. General public compared to a 30yo woman. Their chance of survival was greater, but especially when in first class.}
    \label{fig:titanic_3F}
\end{figure}

To show the generalizability of \textsc{finch}, we will now display a range of interactions found in other data sets.

As a nonlinear and continuous example, we use the California Housing dataset~\cite{pace1997sparse}. It contains a continuous target variable and continuous features, and is used to predict house values. We trained a gradient boosting regressor model on this dataset.

Using \textsc{Finch}, we successfully identified several higher-order interactions. Detailed depictions of each interaction can be found in Appendix \ref{AppendixA}.
\begin{itemize}
    \item Median income strongly predicts house values, increasing linearly from 2 to 9 and remaining relatively stable above and below these values.
    \item For populations of 2,000, housing values per income behave similarly, except that they stabilize at a lower value for incomes above 8.
    \item However, when these populations have approximately 1,000 total rooms, housing values decrease significantly. For incomes below 4, values remain relatively stable, but then stabilize significantly lower instead of increasing linearly.
    \item A comparison with ground truth for this subset shows the model is relatively accurate, though it slightly overestimates housing values for incomes above 4.
\end{itemize}

The last example features a categorical target and categorical features using a subset of the BRFSS dataset~\cite{BRFSS2015}, which focuses on diabetes risk factors. We trained a decision tree classifier on this dataset.
Using \textsc{Finch}, we observed the following interactions.
\begin{itemize}
\item Diabetes risk in the general population is roughly equal between men and women.
\item Exercise reduces diabetes risk more significantly in women than in men.
\item When combined with high blood pressure, exercise's protective effect is neutralized, resulting in an elevated but equal risk for both men and women.
\item Among those who are also obese, the risk increases further, particularly for women.
\end{itemize}

