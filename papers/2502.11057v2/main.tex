\documentclass[journal]{IEEEtran} % Use IEEE conference format\
% \IEEEoverridecommandlockouts

% Recommended, but optional, packages for figures and better typesetting:
\usepackage{microtype}
\usepackage{graphicx}
\usepackage{subfigure}
\usepackage{booktabs} % for professional tables
\usepackage{mdframed}
\usepackage{cancel}
\usepackage{hyperref}
\usepackage{algorithm}
\usepackage{algorithmic}

% For theorems and such
\usepackage{amsmath}
\usepackage{amssymb}
\usepackage{mathtools}
\usepackage{amsthm}
\usepackage[capitalize,noabbrev]{cleveref}
\usepackage{dsfont}

%%%%%%%%%%%%%%%%%%%%%%%%%%%%%%%%
% THEOREMS
%%%%%%%%%%%%%%%%%%%%%%%%%%%%%%%%
\theoremstyle{plain}
\newtheorem{theorem}{Theorem}[section]
\newtheorem{proposition}[theorem]{Proposition}
\newtheorem{corollary}[theorem]{Corollary}
\theoremstyle{definition}
\newtheorem{definition}[theorem]{Definition}
\newtheorem{assumption}[theorem]{Assumption}
\theoremstyle{remark}
\newtheorem{remark}[theorem]{Remark}
\newmdtheoremenv[linecolor=white,backgroundcolor=lightgray!15, innertopmargin=5pt, innerbottommargin=5pt, skipabove=10pt, skipbelow=10pt]{objective}{\textbf{Objective}}
\newmdtheoremenv[linecolor=white, backgroundcolor=lightgray!15, innertopmargin=5pt, innerbottommargin=5pt, skipabove=10pt, skipbelow=10pt]{lemma}{\textbf{Lemma}}

% Command for notes
\newcommand{\sbnote}[1]{\textcolor{red}{SB: #1}}

\usepackage[textsize=tiny]{todonotes}

% Add some macros
\newcommand{\ctrlseq}{\mathbf{u}}
\newcommand{\vfunc}{V}
\newcommand{\auxvfunc}{\hat{\vfunc}}
\newcommand{\param}{\theta}
\newcommand{\learnedauxvfunc}{\auxvfunc_{\param}}
\newcommand{\learnedauxpolicy}{\hat{\pi}_{\param}}
\newcommand{\inducedauxvfunc}{\auxvfunc_{\learnedauxpolicy}}
\newcommand{\learnedvfunc}{\vfunc_{\param}}
\newcommand{\learnedpolicy}{\pi_{\param}}
\newcommand{\inducedvfunc}{\vfunc_{\learnedpolicy}}

\title{\fontsize{18pt}{20pt}\selectfont A Physics-Informed Machine Learning Framework for Safe and Optimal Control of Autonomous Systems}

\author{
Manan Tayal*$^{1}$, Aditya Singh*$^{1}$, Shishir Kolathaya$^{1}$, Somil Bansal$^{2}$
\thanks{$^*$ Denotes equal contribution.\\
$^{1}$Authors are with the Center for Cyber-Physical Systems, Indian Institute of Science, Bangalore, India
(email:\{manantayal, adityasingh, shishirk\}@iisc.ac.in)\\
$^{2}$Author is with the Department of Aeronautics and Astronautics, Stanford University, USA (email:\{somil@stanford.edu\})}
}
\setlength{\marginparwidth}{2cm}
\usepackage{todonotes}
\IEEEaftertitletext{\vspace{-1.5\baselineskip}} % Adjust as needed
\begin{document}
\maketitle
\begin{abstract}
As autonomous systems become more ubiquitous in daily life, ensuring high performance with guaranteed safety is crucial. However, safety and performance could be competing objectives, which makes their co-optimization difficult. Learning-based methods, such as Constrained Reinforcement Learning (CRL), achieve strong performance but lack formal safety guarantees due to safety being enforced as soft constraints, limiting their use in safety-critical settings. Conversely, formal methods such as Hamilton-Jacobi (HJ) Reachability Analysis and Control Barrier Functions (CBFs) provide rigorous safety assurances but often neglect performance, resulting in overly conservative controllers. To bridge this gap, we formulate the co-optimization of safety and performance as a state-constrained optimal control problem, where performance objectives are encoded via a cost function and safety requirements are imposed as state constraints. We demonstrate that the resultant value function satisfies a Hamilton-Jacobi-Bellman (HJB) equation, which we approximate efficiently using a novel physics-informed machine learning framework. In addition, we introduce a conformal prediction-based verification strategy to quantify the learning errors, recovering a high-confidence safety value function, along with a probabilistic error bound on performance degradation. Through several case studies, we demonstrate the efficacy of the proposed framework in enabling scalable learning of safe and performant controllers for complex, high-dimensional autonomous systems.
\end{abstract}
\vspace{-1.2em}
\section{Introduction}
\label{section: intro}
\section{Introduction}
Backdoor attacks pose a concealed yet profound security risk to machine learning (ML) models, for which the adversaries can inject a stealth backdoor into the model during training, enabling them to illicitly control the model's output upon encountering predefined inputs. These attacks can even occur without the knowledge of developers or end-users, thereby undermining the trust in ML systems. As ML becomes more deeply embedded in critical sectors like finance, healthcare, and autonomous driving \citep{he2016deep, liu2020computing, tournier2019mrtrix3, adjabi2020past}, the potential damage from backdoor attacks grows, underscoring the emergency for developing robust defense mechanisms against backdoor attacks.

To address the threat of backdoor attacks, researchers have developed a variety of strategies \cite{liu2018fine,wu2021adversarial,wang2019neural,zeng2022adversarial,zhu2023neural,Zhu_2023_ICCV, wei2024shared,wei2024d3}, aimed at purifying backdoors within victim models. These methods are designed to integrate with current deployment workflows seamlessly and have demonstrated significant success in mitigating the effects of backdoor triggers \cite{wubackdoorbench, wu2023defenses, wu2024backdoorbench,dunnett2024countering}.  However, most state-of-the-art (SOTA) backdoor purification methods operate under the assumption that a small clean dataset, often referred to as \textbf{auxiliary dataset}, is available for purification. Such an assumption poses practical challenges, especially in scenarios where data is scarce. To tackle this challenge, efforts have been made to reduce the size of the required auxiliary dataset~\cite{chai2022oneshot,li2023reconstructive, Zhu_2023_ICCV} and even explore dataset-free purification techniques~\cite{zheng2022data,hong2023revisiting,lin2024fusing}. Although these approaches offer some improvements, recent evaluations \cite{dunnett2024countering, wu2024backdoorbench} continue to highlight the importance of sufficient auxiliary data for achieving robust defenses against backdoor attacks.

While significant progress has been made in reducing the size of auxiliary datasets, an equally critical yet underexplored question remains: \emph{how does the nature of the auxiliary dataset affect purification effectiveness?} In  real-world  applications, auxiliary datasets can vary widely, encompassing in-distribution data, synthetic data, or external data from different sources. Understanding how each type of auxiliary dataset influences the purification effectiveness is vital for selecting or constructing the most suitable auxiliary dataset and the corresponding technique. For instance, when multiple datasets are available, understanding how different datasets contribute to purification can guide defenders in selecting or crafting the most appropriate dataset. Conversely, when only limited auxiliary data is accessible, knowing which purification technique works best under those constraints is critical. Therefore, there is an urgent need for a thorough investigation into the impact of auxiliary datasets on purification effectiveness to guide defenders in  enhancing the security of ML systems. 

In this paper, we systematically investigate the critical role of auxiliary datasets in backdoor purification, aiming to bridge the gap between idealized and practical purification scenarios.  Specifically, we first construct a diverse set of auxiliary datasets to emulate real-world conditions, as summarized in Table~\ref{overall}. These datasets include in-distribution data, synthetic data, and external data from other sources. Through an evaluation of SOTA backdoor purification methods across these datasets, we uncover several critical insights: \textbf{1)} In-distribution datasets, particularly those carefully filtered from the original training data of the victim model, effectively preserve the model’s utility for its intended tasks but may fall short in eliminating backdoors. \textbf{2)} Incorporating OOD datasets can help the model forget backdoors but also bring the risk of forgetting critical learned knowledge, significantly degrading its overall performance. Building on these findings, we propose Guided Input Calibration (GIC), a novel technique that enhances backdoor purification by adaptively transforming auxiliary data to better align with the victim model’s learned representations. By leveraging the victim model itself to guide this transformation, GIC optimizes the purification process, striking a balance between preserving model utility and mitigating backdoor threats. Extensive experiments demonstrate that GIC significantly improves the effectiveness of backdoor purification across diverse auxiliary datasets, providing a practical and robust defense solution.

Our main contributions are threefold:
\textbf{1) Impact analysis of auxiliary datasets:} We take the \textbf{first step}  in systematically investigating how different types of auxiliary datasets influence backdoor purification effectiveness. Our findings provide novel insights and serve as a foundation for future research on optimizing dataset selection and construction for enhanced backdoor defense.
%
\textbf{2) Compilation and evaluation of diverse auxiliary datasets:}  We have compiled and rigorously evaluated a diverse set of auxiliary datasets using SOTA purification methods, making our datasets and code publicly available to facilitate and support future research on practical backdoor defense strategies.
%
\textbf{3) Introduction of GIC:} We introduce GIC, the \textbf{first} dedicated solution designed to align auxiliary datasets with the model’s learned representations, significantly enhancing backdoor mitigation across various dataset types. Our approach sets a new benchmark for practical and effective backdoor defense.




\section{Problem Setup}
\label{section: problem_setup}
Consider a nonlinear dynamical system characterized by the state $x \in \mathcal{X} \subseteq \mathbb{R}^n$ and control input $u \in \mathcal{U} \subseteq \mathbb{R}^m$, governed by the dynamics $\dot{x}(t) = f(x(t), u(t))$, where the function $f: \mathbb{R}^n \times \mathbb{R}^m \to \mathbb{R}^n$ is locally Lipschitz continuous.
In this work, we assume that the dynamics model $f$ is known; however, it can also be learned from data if unavailable.

We are given a failure set $\mathcal{F} \subseteq \mathcal{X}$ that represents the set of unsafe states for the system (e.g., obstacles for an autonomous ground robot). The system's performance is quantified by the cost function $C(t, x, \ctrlseq)$, given by: 
\begin{equation}
    C(t,x(t), \ctrlseq) = \int_{s=t}^{T} l(x(s)) \, ds + \phi(x(T)),
\end{equation}
where $l: \mathcal{X} \to \mathbb{R}_{\geq 0}$ and $\phi: \mathcal{X} \to \mathbb{R}_{\geq 0}$ are Lipschitz continuous and non-negative functions, representing the running cost over the time horizon $[t, T)$ and the terminal cost at time $T$, respectively. 
$\ctrlseq:[t,T)\rightarrow \mathcal{U}$ is the control signal applied to the system.
Using this premise, we define the main objective of this paper: 

\begin{objective}
\label{obj: Main_obj}
   We aim to synthesize an optimal policy $\pi^*: [t, T) \times \mathcal{X} \to \mathcal{U}$ that minimizes the cost function $C$ while ensuring that the system remains outside the failure set $\mathcal{F}$ at all times. 
\end{objective}


\subsection{State-Constrained Optimal Control Problem}
To achieve the stated objective, the first step is to encode the safety constraint via a function $g: \mathbb{R}^n \to \mathbb{R}$ such that, $\mathcal{F}:= \{x \in \mathcal{X} \mid g(x) > 0\}$. Using these notations, the objective can be formulated as the following State-Constrained Optimal Control Problem (SC-OCP) to compute the value function $V$:
\begin{equation}\label{eq: SC-OCP}
    \begin{aligned}
    V(t, x(t)) = \min_{\ctrlseq}\int_t^{T}&l(x(s)) ds + \phi(x(T))\\
    \text{s.t.} & \; \dot{x} = f(x, u), \\
    & g(x(s)) \leq 0 \quad \forall s \in [t, T]
\end{aligned}
\end{equation}
This SC-OCP enhances the system's performance by minimizing the cost, while maintaining system safety through the state constraint, $g(x) \leq 0$, ensuring that the system avoids the failure set, $\mathcal{F}$. Thus, the policy, $\pi^*$, derived from the solution of this SC-OCP co-optimizes safety and performance. 

\subsection{Epigraph Reformulation}\label{subsec: epigraph}
Directly solving the SC-OCP in \eqref{eq: SC-OCP} presents significant challenges due to the presence of (hard) state constraints. To address this issue, we reformulate the problem in its epigraph form \cite{boyd2004convex}, which transforms the constrained optimization into a more tractable two-stage optimization problem. This reformulation allows us to efficiently obtain a solution to the SC-OCP in \eqref{eq: SC-OCP}. The resulting formulation is given by:
\begin{equation}\label{eq: aux_value_func}
    \begin{aligned}
    V(t, x(t)) = \min_{z \in \mathbb{R^+}} & \; z  \\
    \text{s.t.} & \; \hat{V}(t, x, z) \leq 0,
\end{aligned}
\end{equation}
where $z$ is a non-negative auxiliary optimization variable, and $\hat{V}$ represents the auxiliary value function. Here, $\hat{V}$ is defined as \cite{altarovici2013general}:
% 
\begin{equation}\label{eq: aux_vfunc_def}
\begin{aligned}
    \hat{V}(t, x(t), z) = \min_{\ctrlseq} \max \{C(t, x(t), \ctrlseq) -z, \max_{s \in [t, T]}g(x(s)) \}.
\end{aligned}
\end{equation}
% 
Note that if $\hat{V}(t, x, z) < 0$, it implies that $g(x(s)) < 0$ for all $s \in [t, T]$ . In other words, the system must be outside the failure set at all times; therefore, the system is guaranteed to be safe whenever $\hat{V}(t, x, z) < 0$.

In this reformulated problem, state constraints are effectively eliminated, enabling the use of dynamic programming to characterize the value function, as we explain later in this section. Intuitively, optimal $z$ ($z^*$) can be thought of as the \textit{minimum permissible cost} the policy can incur without compromising on safety. From Equation~\ref{eq: aux_value_func}, it can be inferred that if $z > z^*$, the safety constraint dominates in the max term, resulting in a conservative policy. Conversely, if $z < z^*$, the performance objective takes precedence, leading to a potentially aggressive policy that might compromise safety.

Furthermore, to facilitate solving the epigraph reformulation, $z$ can be treated as a state variable, with its dynamics given by $\dot{z}(t) = -l(x(t))$. This implies that as the trajectory progresses over time, the minimum permissible cost, $z$, decreases by the step cost $l(x)$ at each time step. This allows us to define an augmented system that evolves according to the following dynamics:
\begin{equation}
    \dot{\hat{x}} = \hat{f}(t, \hat{x}, u) := 
    \begin{bmatrix}
        f(t, x, u) \\
        -l(x)
    \end{bmatrix}, \\ 
\end{equation}
% \qquad \qquad \qquad ~~$\forall t \in [0, T),  \ (x, z) \in \mathcal{X} \times \mathbb{R^+},$
where $\hat{x} := [x, z]^T$ represents the augmented state.
With the augmented state representation, it has been shown that the auxiliary value function $\hat{V}(t, x(t), z(t))$ is characterized as the unique continuous viscosity solution of the following Hamilton-Jacobi-Bellman (HJB) partial differential equation (PDE) \cite{altarovici2013general}:
\begin{equation}\label{eq: coopt_pde}
\min\Bigl(-\partial_t \hat{V} - \min_{\ctrlseq} \langle \nabla_{\hat{x}}\hat{V}(t, \hat{x}), \hat{f}(\hat{x}, u)\rangle ,\hat{V} - g(x)\Bigr) = 0,
\end{equation}
$\forall t \in [0,T)$ and $\hat{x} \in \mathcal{X} \times \mathbb{R}$, where $\langle \cdot, \cdot \rangle$ denotes the dot product of vectors. The boundary condition for the PDE is given by:
\begin{equation}\label{eq: terminal_condition}
\hat{V}(T,\hat{x}) = \max\left(\phi(x) - z, g(x)\right), \quad \hat{x} \in \mathcal{X} \times \mathbb{R}.
\end{equation}
Note that by a slight abuse of notations, we have replaced the arguments $x,z$ for $\hat{V}$ with the augmented state $\hat{x}$.
% Building upon this formulation, we state the following theorem to obtain the auxiliary value function $\hat{V}$:
% \begin{theorem}[]\label{thm: aux_hjb_pde}
% For a given state $x \in \mathcal{X} \subseteq \mathbb{R}^n$, $z \in \mathbb{R}^+$ defined over the time horizon $T >0$, the auxiliary value function $\hat{V}$ so obtained from the epigraph form, is characterized as the unique continuous viscosity solution of the following Hamilton-Jacobi-Bellman (HJB) Partial Differential Equation (PDE):
% \begin{equation}\label{eq: coopt_pde}
% \min\Bigl(-\partial_t \hat{V} - \min_{u\in \mathcal{U}}\nabla_{\hat{x}} \hat{V} \cdot \hat{f}(t,\hat{x},u),\hat{V} - g(x)\Bigr) = 0,
% \end{equation}
% for all $t \in [0,T)$ and $(x,z) \in \mathcal{X} \times \mathbb{R}$, with the boundary condition:
% \begin{equation}\label{eq: terminal_condition}
% \hat{V}(T,x,z) = \max\left((\phi(x) - z), g(x)\right), \quad (x,z) \in \mathbb{R}^d \times \mathbb{R}.
% \end{equation}
% \end{theorem}





\section{Methodology}
\label{section: method}
Effective human-robot cooperation in CoNav-Maze hinges on efficient communication. Maximizing the human’s information gain enables more precise guidance, which in turn accelerates task completion. Yet for the robot, the challenge is not only \emph{what} to communicate but also \emph{when}, as it must balance gathering information for the human with pursuing immediate goals when confident in its navigation.

To achieve this, we introduce \emph{Information Gain Monte Carlo Tree Search} (IG-MCTS), which optimizes both task-relevant objectives and the transmission of the most informative communication. IG-MCTS comprises three key components:
\textbf{(1)} A data-driven human perception model that tracks how implicit (movement) and explicit (image) information updates the human’s understanding of the maze layout.
\textbf{(2)} Reward augmentation to integrate multiple objectives effectively leveraging on the learned perception model.
\textbf{(3)} An uncertainty-aware MCTS that accounts for unobserved maze regions and human perception stochasticity.
% \begin{enumerate}[leftmargin=*]
%     \item A data-driven human perception model that tracks how implicit (movement) and explicit (image transmission) information updates the human’s understanding of the maze layout.
%     \item Reward augmentation to integrate multiple objectives effectively leveraging on the learned perception model.
%     \item An uncertainty-aware MCTS that accounts for unobserved maze regions and human perception stochasticity.
% \end{enumerate}

\subsection{Human Perception Dynamics}
% IG-MCTS seeks to optimize the expected novel information gained by the human through the robot’s actions, including both movement and communication. Achieving this requires a model of how the human acquires task-relevant information from the robot.

% \subsubsection{Perception MDP}
\label{sec:perception_mdp}
As the robot navigates the maze and transmits images, humans update their understanding of the environment. Based on the robot's path, they may infer that previously assumed blocked locations are traversable or detect discrepancies between the transmitted image and their map.  

To formally capture this process, we model the evolution of human perception as another Markov Decision Process, referred to as the \emph{Perception MDP}. The state space $\mathcal{X}$ represents all possible maze maps. The action space $\mathcal{S}^+ \times \mathcal{O}$ consists of the robot's trajectory between two image transmissions $\tau \in \mathcal{S}^+$ and an image $o \in \mathcal{O}$. The unknown transition function $F: (x, (\tau, o)) \rightarrow x'$ defines the human perception dynamics, which we aim to learn.

\subsubsection{Crowd-Sourced Transition Dataset}
To collect data, we designed a mapping task in the CoNav-Maze environment. Participants were tasked to edit their maps to match the true environment. A button triggers the robot's autonomous movements, after which it captures an image from a random angle.
In this mapping task, the robot, aware of both the true environment and the human’s map, visits predefined target locations and prioritizes areas with mislabeled grid cells on the human’s map.
% We assume that the robot has full knowledge of both the actual environment and the human’s current map. Leveraging this knowledge, the robot autonomously navigates to all predefined target locations. It then randomly selects subsequent goals to reach, prioritizing grid locations that remain mislabeled on the human’s map. This ensures that the robot’s actions are strategically focused on providing useful information to improve map accuracy.

We then recruited over $50$ annotators through Prolific~\cite{palan2018prolific} for the mapping task. Each annotator labeled three randomly generated mazes. They were allowed to proceed to the next maze once the robot had reached all four goal locations. However, they could spend additional time refining their map before moving on. To incentivize accuracy, annotators receive a performance-based bonus based on the final accuracy of their annotated map.


\subsubsection{Fully-Convolutional Dynamics Model}
\label{sec:nhpm}

We propose a Neural Human Perception Model (NHPM), a fully convolutional neural network (FCNN), to predict the human perception transition probabilities modeled in \Cref{sec:perception_mdp}. We denote the model as $F_\theta$ where $\theta$ represents the trainable weights. Such design echoes recent studies of model-based reinforcement learning~\cite{hansen2022temporal}, where the agent first learns the environment dynamics, potentially from image observations~\cite{hafner2019learning,watter2015embed}.

\begin{figure}[t]
    \centering
    \includegraphics[width=0.9\linewidth]{figures/ICML_25_CNN.pdf}
    \caption{Neural Human Perception Model (NHPM). \textbf{Left:} The human's current perception, the robot's trajectory since the last transmission, and the captured environment grids are individually processed into 2D masks. \textbf{Right:} A fully convolutional neural network predicts two masks: one for the probability of the human adding a wall to their map and another for removing a wall.}
    \label{fig:nhpm}
    \vskip -0.1in
\end{figure}

As illustrated in \Cref{fig:nhpm}, our model takes as input the human’s current perception, the robot’s path, and the image captured by the robot, all of which are transformed into a unified 2D representation. These inputs are concatenated along the channel dimension and fed into the CNN, which outputs a two-channel image: one predicting the probability of human adding a new wall and the other predicting the probability of removing a wall.

% Our approach builds on world model learning, where neural networks predict state transitions or environmental updates based on agent actions and observations. By leveraging the local feature extraction capabilities of CNNs, our model effectively captures spatial relationships and interprets local changes within the grid maze environment. Similar to prior work in localization and mapping, the CNN architecture is well-suited for processing spatially structured data and aligning the robot’s observations with human map updates.

To enhance robustness and generalization, we apply data augmentation techniques, including random rotation and flipping of the 2D inputs during training. These transformations are particularly beneficial in the grid maze environment, which is invariant to orientation changes.

\subsection{Perception-Aware Reward Augmentation}
The robot optimizes its actions over a planning horizon \( H \) by solving the following optimization problem:
\begin{subequations}
    \begin{align}
        \max_{a_{0:H-1}} \;
        & \mathop{\mathbb{E}}_{T, F} \left[ \sum_{t=0}^{H-1} \gamma^t \left(\underbrace{R_{\mathrm{task}}(\tau_{t+1}, \zeta)}_{\text{(1) Task reward}} + \underbrace{\|x_{t+1}-x_t\|_1}_{\text{(2) Info reward}}\right)\right] \label{obj}\\ 
        \subjectto \quad
        &x_{t+1} = F(x_t, (\tau_t, a_t)), \quad a_t\in\Ocal \label{const:perception_update}\\ 
        &\tau_{t+1} = \tau_t \oplus T(s_t, a_t), \quad a_t\in \Ucal\label{const:history_update}
    \end{align}
\end{subequations} 

The objective in~\eqref{obj} maximizes the expected cumulative reward over \( T \) and \( F \), reflecting the uncertainty in both physical transitions and human perception dynamics. The reward function consists of two components: 
(1) The \emph{task reward} incentivizes efficient navigation. The specific formulation for the task in this work is outlined in \Cref{appendix:task_reward}.
(2) The \emph{information reward} quantifies the change in the human’s perception due to robot actions, computed as the \( L_1 \)-norm distance between consecutive perception states.  

The constraint in~\eqref{const:history_update} ensures that for movement actions, the trajectory history \( \tau_t \) expands with new states based on the robot’s chosen actions, where \( s_t \) is the most recent state in \( \tau_t \), and \( \oplus \) represents sequence concatenation. 
In constraint~\eqref{const:perception_update}, the robot leverages the learned human perception dynamics \( F \) to estimate the evolution of the human’s understanding of the environment from perception state $x_t$ to $x_{t+1}$ based on the observed trajectory \( \tau_t \) and transmitted image \( a_t\in\Ocal \). 
% justify from a cognitive science perspective
% Cognitive science research has shown that humans read in a way to maximize the information gained from each word, aligning with the efficient coding principle, which prioritizes minimizing perceptual errors and extracting relevant features under limited processing capacity~\cite{kangassalo2020information}. Drawing on this principle, we hypothesize that humans similarly prioritize task-relevant information in multimodal settings. To accommodate this cognitive pattern, our robot policy selects and communicates high information-gain observations to human operators, akin to summarizing key insights from a lengthy article.
% % While the brain naturally seeks to gain information, the brain employs various strategies to manage information overload, including filtering~\cite{quiroga2004reducing}, limiting/working memory, and prioritizing information~\cite{arnold2023dealing}.
% In this context of our setup, we optimize the selection of camera angles to maximize the human operator's information gain about the environment. 

\subsection{Information Gain Monte Carlo Tree Search (IG-MCTS)}
IG-MCTS follows the four stages of Monte Carlo tree search: \emph{selection}, \emph{expansion}, \emph{rollout}, and \emph{backpropagation}, but extends it by incorporating uncertainty in both environment dynamics and human perception. We introduce uncertainty-aware simulations in the \emph{expansion} and \emph{rollout} phases and adjust \emph{backpropagation} with a value update rule that accounts for transition feasibility.

\subsubsection{Uncertainty-Aware Simulation}
As detailed in \Cref{algo:IG_MCTS}, both the \emph{expansion} and \emph{rollout} phases involve forward simulation of robot actions. Each tree node $v$ contains the state $(\tau, x)$, representing the robot's state history and current human perception. We handle the two action types differently as follows:
\begin{itemize}
    \item A movement action $u$ follows the environment dynamics $T$ as defined in \Cref{sec:problem}. Notably, the maze layout is observable up to distance $r$ from the robot's visited grids, while unexplored areas assume a $50\%$ chance of walls. In \emph{expansion}, the resulting search node $v'$ of this uncertain transition is assigned a feasibility value $\delta = 0.5$. In \emph{rollout}, the transition could fail and the robot remains in the same grid.
    
    \item The state transition for a communication step $o$ is governed by the learned stochastic human perception model $F_\theta$ as defined in \Cref{sec:nhpm}. Since transition probabilities are known, we compute the expected information reward $\bar{R_\mathrm{info}}$ directly:
    \begin{align*}
        \bar{R_\mathrm{info}}(\tau_t, x_t, o_t) &= \mathbb{E}_{x_{t+1}}\|x_{t+1}-x_t\|_1 \\
        &= \|p_\mathrm{add}\|_1 + \|p_\mathrm{remove}\|_1,
    \end{align*}
    where $(p_\mathrm{add}, p_\mathrm{remove}) \gets F_\theta(\tau_t, x_t, o_t)$ are the estimated probabilities of adding or removing walls from the map. 
    Directly computing the expected return at a node avoids the high number of visitations required to obtain an accurate value estimate.
\end{itemize}

% We denote a node in the search tree as $v$, where $s(v)$, $r(v)$, and $\delta(v)$ represent the state, reward, and transition feasibility at $v$, respectively. The visit count of $v$ is denoted as $N(v)$, while $Q(v)$ represents its total accumulated return. The set of child nodes of $v$ is denoted by $\mathbb{C}(v)$.

% The goal of each search is to plan a sequence for the robot until it reaches a goal or transmits a new image to the human. We initialize the search tree with the current human guidance $\zeta$, and the robot's approximation of human perception $x_0$. Each search node consists consists of the state information required by our reward augmentation: $(\tau, x)$. A node is terminal if it is the resulting state of a communication step, or if the robot reaches a goal location. 

% A rollout from the expanded node simulates future transitions until reaching a terminal state or a predefined depth $H$. Actions are selected randomly from the available action set $\mathcal{A}(s)$. If an action's feasibility is uncertain due to the environment's unknown structure, the transition occurs with probability $\delta(s, a)$. When a random number draw deems the transition infeasible, the state remains unchanged. On the other hand, for communication steps, we don't resolve the uncertainty but instead compute the expected information gain reward: \philip{TODO: adjust notation}
% \begin{equation}
%     \mathbb{E}\left[R_\mathrm{info}(\tau, x')\right] = \sum \mathrm{NPM(\tau, o)}.
% \end{equation}

\subsubsection{Feasibility-Adjusted Backpropagation}
During backpropagation, the rewards obtained from the simulation phase are propagated back through the tree, updating the total value $Q(v)$ and the visitation count $N(v)$ for all nodes along the path to the root. Due to uncertainty in unexplored environment dynamics, the rollout return depends on the feasibility of the transition from the child node. Given a sample return \(q'_{\mathrm{sample}}\) at child node \(v'\), the parent node's return is:
\begin{equation}
    q_{\mathrm{sample}} = r + \gamma \left[ \delta' q'_{\mathrm{sample}} + (1 - \delta') \frac{Q(v)}{N(v)} \right],
\end{equation}
where $\delta'$ represents the probability of a successful transition. The term \((1 - \delta')\) accounts for failed transitions, relying instead on the current value estimate.

% By incorporating uncertainty-aware rollouts and backpropagation, our approach enables more robust decision-making in scenarios where the environment dynamics is unknown and avoids simulation of the stochastic human perception dynamics.


\section{Experiments}
\label{section: case_studies}
The objective of this paper is to demonstrate the co-optimization of performance and safety. To achieve this, we evaluate the proposed method and compare them with baselines using two metrics: (1) \textbf{Cumulative Cost:} This metric represents the total cost  $\int_0^{T}l(x(s)) ds + \phi(x(T))$, accumulated by a policy over the safe trajectories. (2) \textbf{Safety Rate:} This metric is defined as the percentage of trajectories that remain safe, i.e., never enter the failure region $\mathcal{F}$ at any point in time.
% \begin{enumerate}
%     \item \textbf{Cumulative Cost:} This metric represents the total cost  $\int_0^{T}l(x(s)) ds + \phi(x(T))$, accumulated by a policy over the safe trajectories.
%     \item \textbf{Safety Rate:} This metric is defined as the percentage of trajectories that remain safe, i.e., never enter the failure region $\mathcal{F}$ at any point in time.
% \end{enumerate}

\textbf{Baselines}: We consider two categories of baselines: the first set of methods aim to enhance the system performance (i.e., minimize the cumulative cost) while encouraging safety, encompassing methods such as Constrained Reinforcement Learning (CRL) and Model Predictive Path Integral (MPPI)~\cite{8558663} algorithms. 
The second category prioritizes safety,  potentially at the cost of performance. This includes safety filtering techniques such as Control Barrier Function (CBF)-based quadratic programs (QP) \cite{Ames_2017} that modify a nominal, potentially unsafe controller to satisfy the safety constraint. Additionally, we have presented the comparative study of offline and online computation times between the baselines in Appendix~\ref{app: comp_time}.
\vspace{-1em}
\subsection{Efficient and Safe Boat Navigation}
In our first experiment, we consider a 2D autonomous boat navigation problem, where a boat with coordinates $(x_b, y_b)$ navigates a river with state-dependent drift to reach an island. The boat must avoid two circular boulders (obstacles) of different radii, which corresponds to the safety constraint in the system (see Fig. \ref{fig: Boat_Trajectories}).
The cost function penalizes the distance to the goal.
The system state, $x$, evolves according to the dynamics:  
\vspace{-1em}
\begin{equation}
    x = [x_b, y_b], \quad \dot{x} = [u_1 + 2 - 0.5y_b^2, u_2]
\end{equation}
where $[u_1, u_2]$ are the bounded control inputs in the $x_b$ and $y_b$ directions, constrained by the control space $\mathcal{U} = \{[u_1, u_2] \in \mathbb{R}^2 \mid ||[u_1, u_2]|| \leq 1\}$. The term $2 - 0.5y_b^2$ introduces a state-dependent drift, complicating the control task as the actions must counteract the drift while ensuring safety, which is challenging under bounded control inputs. 
The rest of the details about the experiment setup can be found in the Appendix \ref{appendix: Boat2D}.

\textbf{Safety Guarantees and Performance Quantification}:
We use $N_s = 300K$ and $N_p = 300K$ samples for thorough verification, ensuring dense state space sampling. 
For this experiment, we set $\epsilon_s = 0.001$ and $\beta_s = 10^{-10}$, resulting in a $\delta$-level of $0$. 
This implies that, with $1 - 10^{-10}$ confidence, any state with $\learnedauxvfunc (t,x,z) \leq 0$, is safe with at least $99.9\%$ probability. 
For performance quantification, we set $\epsilon_p = 0.01$ and $\beta_p = 10^{-10}$, leading to a $\psi$-level of $0.136$. This ensures, with $1 - 10^{-10}$ confidence, that any state in $\mathcal{S}^*$ has a normalized error between the predicted value and the policy value of less than $0.136$ with $99\%$ probability. Low $\delta$ and $\psi$ values with high confidence indicate that the learned policy closely approximates the optimal policy and successfully co-optimizes safety and performance.

\begin{figure}[t]
    \centering
    \includegraphics[width=1.0\linewidth]{images/Boat_Traj_with_box.pdf}
    \vspace{-1em}
    \caption{Trajectories from two distinct initial states are shown, with dark grey circles representing obstacles and the green dot indicating the goal at $[1.5, 0]^T$. Notably, our method is the \textbf{only one} that \textbf{successfully approaches the goal} while \textbf{adhering to safety constraints}.}
    \vspace{-0.8em}
    \label{fig: Boat_Trajectories}
\end{figure}

\textbf{Baselines}: This being a 2-dimensional system, we compare our method with the ground truth value function computed by solving the HJB-PDE numerically using the Level Set Toolbox~\cite{mitchell2004toolbox} (results in Appendix~\ref{Appendix: GT_comp}). 
Additional baselines include: (1) MPPI, a sample-based path-planning algorithm with safety as soft constraints, (2) MPPI-NCBF, where safety is enforced using a Neural CBF-based QP with MPPI as the nominal controller~\cite{dawson2022safe,tayal2024learning}, and (3) C-SAC, a Lagrangian-based CRL approach using Soft Actor-Critic~\cite{pmlr-v80-haarnoja18b}, incorporating safety as soft constraints.


\textbf{Comparative Analysis:} Figure \ref{fig: Boat_Trajectories} shows that our method effectively reaches the goal while avoiding obstacles, even when starting close to them. In contrast, MPPI and C-SAC-based policies fail to maintain safety, while MPPI-NCBF ensures safety but performs poorly (leading to very slow trajectories). Figure \ref{fig: baseline_comparison} highlights that our method outperforms all others. C-SAC achieves reasonable performance with a $7.5\%$ higher mean cost compared to ours but has the lowest safety rate of $76\%$. MPPI, with a more competitive safety rate of $89\%$, performs poorly with a $32.67\%$ higher mean cost. MPPI-NCBF achieves $100\%$ safety but performs significantly worse, with a $50.72\%$ higher mean cost. Additionally, CBF-based controllers sometimes violate control bounds, limiting their applicability. This demonstrates that our method balances safety and performance, unlike others that compromise on one aspect. Moreover, the $100\%$ safety rate of our method aligns closely with at least $99.9\%$ safety level that we expect using our proposed verification strategy, providing empirical validation of the safety assurances.

% \begin{figure}[h]
%     \centering
%     \includegraphics[width=0.79\linewidth]{images/traj_plot_Boat.png}
% \caption{The figure shows a bunch of trajectories starting at random points propagating with optimal policy $\pi$. The orange circles represent the obstacles and the grey vector field denotes the river current drift at different points.} 
% \label{fig: traj_plot_boat}
% \end{figure}
\vspace{-0.5em}
\subsection{Pursuer Vehicle tracking a moving Evader}
In our second experiment, we consider an acceleration-driven pursuer vehicle, tracking a moving evader while avoiding five circular obstacles (see Fig. \ref{fig: Track_Trajectories}). This experiment involves an 8-dimensional system, with the state $x$ defined as $x = [x_p, y_p, v, \Theta, x_{e}, y_{e}, v_{xe}, v_{ye}]^T$, where $x_p, y_p, v, \Theta$ represent the coordinates, linear velocity, and orientation of the pursuer vehicle, respectively, and $x_e, y_e, v_{xe}, v_{ye}$ represent the coordinates and linear velocities of the evader vehicle. The pursuer vehicle is controlled by linear acceleration ($u_1$) and angular velocity ($u_2$). The control space is $\mathcal{U} = \{[u_1, u_2] \in [-2, 2]^2\}$. The complexity of this system stems from the dynamic nature of the goal, along with the challenge of
ensuring safety in a cluttered environment, which in itself is a difficult safety problem. More details about the experiment setup are in Appendix \ref{appendix: Track}.
% 
\begin{figure}[t]
    \centering
    \includegraphics[width=1.0\linewidth]{images/Track_Traj_with_box.pdf}
    \vspace{-2em}
    \caption{Trajectories from two distinct initial states are depicted, with dark grey circles representing obstacles and purple trajectories indicating the evader's path, with arrows showing its direction of motion. Our method successfully \textbf{tracks the evader} while \textbf{avoiding collisions}, whereas all other methods either fail to maintain safety, struggle to track the evader or both}
    \vspace{-0.8em}
    \label{fig: Track_Trajectories}
\end{figure}

\begin{figure*}[ht]
    \centering
    \includegraphics[width=1.0\linewidth]{images/MVC_Traj_3.pdf}
    \vspace{-2em}
    \caption{Snapshots of multi-agent navigation trajectories at different times using the proposed method. Agents are represented as circles with radius $R$, indicating the minimum safe distance they must maintain from each other. Smaller dots mark their respective goals. The trajectories show that agents proactively \textbf{maintain long-horizon safety} by adjusting their paths to avoid close encounters, rather than enforcing safety reactively, which could lead to suboptimal behaviors. Finally, the agents \textbf{reach their respective goals within the specified time horizon}.}
    \label{fig: MVC_Trajectories}
    \vspace{-0.8em}
\end{figure*}

\textbf{Safety Guarantees and Performance Quantification}:
Similar to the previous experiment, we set $N_s = N_p = 300k$. We choose $\epsilon_s = 0.01$ and $\beta_s = 10^{-10}$, yielding a $\delta$-level of $-0.04$ and a safety level of $99\%$ on the auxiliary value function. For performance, we set $\epsilon_p = 0.01$ and $\beta_p = 10^{-10}$, leading to a $\psi$-level of 0.137. These values indicate the learned policy maintains high safety with low-performance degradation in this cluttered environment.

\textbf{Baselines:} Similar to the previous experiment, we use MPPI and C-SAC with soft safety constraints as our baselines. For safety filtering, we apply a collision cone CBF (C3BF) \cite{10644338}-based QP due to its effectiveness in handling acceleration-driven systems.

\textbf{Comparative Analysis:} Figure \ref{fig: Track_Trajectories} shows that our method effectively tracks the moving evader while avoiding obstacles, even when starting close to them. In contrast, other methods have limitations: MPPI and C-SAC attempt to follow the evader but fail to maintain their pace, violating safety constraints, while MPPI-C3BF sacrifices performance to maintain safety. Figure \ref{fig: baseline_comparison} highlights our method's superior performance in balancing safety and performance. MPPI achieves the best performance among the baselines but with an 18\% higher mean cost and only a 72\% safety rate. MPPI-NCBF ensures 100\% safety but has a 42\% higher mean cost. C-SAC underperforms both in safety (66\% safety rate) and performance (101\% higher mean cost). This suggests that C-SAC struggles with co-optimizing safety and performance optimization in high-dimensional, complex systems. Additionally, the safety guarantees hold true in the test samples, confirming the reliability of our proposed safety verification framework in safety-critical, cluttered environments.

\textbf{Receding Horizon Control}: An interesting application of the synthesized policy is its deployment in a receding horizon fashion over a time horizon longer than that used for training the value function, as illustrated in Fig.~\ref{fig: receding_Track}. The results indicate that the learned policy successfully maintains safety while effectively tracking the evader over a $6$-second horizon, despite being trained over a horizon of $1$ second. This suggests that the proposed approach can be extended to effectively co-optimize safety and performance for long-horizon tasks by solving the SC-OCP over a shorter time horizon. Consequently, this framework offers a practical solution for real-world autonomous systems that require long-horizon safety and performance guarantees.
\begin{figure}[h]
    \centering
    \includegraphics[width=0.9\linewidth]{images/Track_Traj_receding.pdf}
    \vspace{-1.3em}
\caption{Trajectories of the \textbf{receding horizon policy} for the pursuer tracking an evader over a $6$-second horizon, while the value function is trained over a $1$-second horizon. The results demonstrate that the pursuer \textbf{successfully tracks} the evader while ensuring safety, even when initialized near the obstacle. This highlights the effectiveness of the proposed approach in \textbf{jointly optimizing safety and performance for long-horizon tasks.}} 
\label{fig: receding_Track}
\vspace{-1.3em}
\end{figure}

\vspace{-0.8em}
\subsection{Multi-Agent Navigation}
In our third experiment, we consider a multi-agent setting where each of the 5 agents, represented by $x_i = [x_{a_i}, y_{a_i}, x_{g_i}, y_{g_i}]$, tries to reach its goal while avoiding collisions with others. $(x_{a_i}, y_{a_i})$ denote the position of the $i$th agent, while $(x_{g_i}, y_{g_i})$ represent the goal locations for that agent. The system is $20$-dimensional, with each agent controlled by its $x$ and $y$ velocities. The control space for each agent is $\mathcal{U}_i = \{[v_{x_i}, v_{y_i}] \mid ||[v_{x_i}, v_{y_i}]|| \leq 1\}$. 
% The final control space for this problem will be the Cartesian product of the control spaces for each agent. 
The complexity of this system stems from the interactions and potential conflicts between agents as they attempt to reach their goals while avoiding collisions. The rest of the details about the experiment setup can be found in Appendix \ref{appendix: MultiAgent}.
% 
% \begin{figure}[h]
%     \centering
%     \includegraphics[width=1.0\linewidth]{images/MVC_Traj_3.pdf}
%     \caption{Snapshots of multi-agent navigation trajectories at different time instances using our method. Agents are represented as circles with radius $R$, indicating the minimum safe distance they must maintain from each other. Smaller dots mark their respective goals. The trajectories show that agents proactively maintain long-horizon safety by adjusting their paths to avoid close encounters, rather than enforcing safety reactively, which could lead to suboptimal behaviors.}
%     \label{fig: MVC_Trajectories}
% \end{figure}

\textbf{Safety Guarantees and Performance Quantification}: We set $N_s = N_p = 300k$, $\epsilon_s = 0.001$, and $\beta_s = 10^{-10}$, resulting in a $\delta$-level of $-0.09$ with safety assurance of $99.9\%$ for the auxiliary value function. 
For performance quantification, we set $\epsilon_p = 0.01$ and $\beta_p = 10^{-10}$, leading to a $\psi$-level of 0.068.
It is evident that the $\delta$ and $\psi$ values remain very low with high confidence, highlighting the effectiveness of our method in co-optimizing safety and performance for high-dimensional, multi-agent systems.


\textbf{Baselines:} Similar to previous experiments, we have used MPPI, C-SAC, and MPPI-NCBF as our baselines for this experiment too.

\textbf{Comparative Analysis:} Figure~\ref{fig: MVC_Trajectories} shows that our method ensures long-horizon safety while enabling all agents to reach their goals without collisions. In contrast, the baseline methods either exhibit overly conservative behavior or fail to maintain safety, leading to collisions, as detailed in Appendix~\ref{app: MVC_baselines}. Figure~\ref{fig: baseline_comparison} demonstrates the superior performance of our approach, with MPPI, MPPI-NCBF, and C-SAC showing mean percentage cost increases of 148\%, 192\%, and 164\%, respectively. Although MPPI and MPPI-NCBF achieve competitive safety rates of 90\% and 100\%, their significant performance degradation highlights their inability to balance safety and performance in complex systems. MPPI's subpar performance stems from its reliance on locally optimal solutions in a finite data regime, leading to several deadlocks along the way and overall suboptimal trajectories over a long horizon. Furthermore, C-SAC struggles with both safety and performance, further demonstrating its limitations in handling increasing system complexity and dimensionality. These results confirm our method's ability to co-optimize safety and performance in high-dimensional systems, demonstrating its scalability. Additionally, the safety guarantees hold in the test samples, validating the scalability of our safety verification framework for multi-agent systems.





\section{Conclusion and Future Work}
\label{section: conclusion}
\section{Conclusion}
We introduce a novel approach, \algo, to reduce human feedback requirements in preference-based reinforcement learning by leveraging vision-language models. While VLMs encode rich world knowledge, their direct application as reward models is hindered by alignment issues and noisy predictions. To address this, we develop a synergistic framework where limited human feedback is used to adapt VLMs, improving their reliability in preference labeling. Further, we incorporate a selective sampling strategy to mitigate noise and prioritize informative human annotations.

Our experiments demonstrate that this method significantly improves feedback efficiency, achieving comparable or superior task performance with up to 50\% fewer human annotations. Moreover, we show that an adapted VLM can generalize across similar tasks, further reducing the need for new human feedback by 75\%. These results highlight the potential of integrating VLMs into preference-based RL, offering a scalable solution to reducing human supervision while maintaining high task success rates. 

\section*{Impact Statement}
This work advances embodied AI by significantly reducing the human feedback required for training agents. This reduction is particularly valuable in robotic applications where obtaining human demonstrations and feedback is challenging or impractical, such as assistive robotic arms for individuals with mobility impairments. By minimizing the feedback requirements, our approach enables users to more efficiently customize and teach new skills to robotic agents based on their specific needs and preferences. The broader impact of this work extends to healthcare, assistive technology, and human-robot interaction. One possible risk is that the bias from human feedback can propagate to the VLM and subsequently to the policy. This can be mitigated by personalization of agents in case of household application or standardization of feedback for industrial applications. 

% \section*{Impact Statement}
% This paper presents an approach to co-optimize safety and performance in autonomous systems using Physics-Informed Machine Learning. This framework advances the deployment of scalable, provably safe, and high-performance controllers for complex, high-dimensional autonomous systems, with potential applications in robotics and autonomous driving.

\bibliographystyle{IEEEtran}
\bibliography{refs}


\onecolumn
\appendices
% \begin{appendices}
% \section{Derivation of epigraph reformulation of a state constraint optimal control problem}\label{appendix: epigraph}
% Consider a state constraint optimal control problem of the form:
% \begin{equation}
%     \begin{aligned}
%     \min_{x} & \; C(x) \\
%     \text{s.t.} & \; s(x) \leq 0,
% \end{aligned}
% \end{equation}
% where, $C(x)$ denotes the cost function and $s(x)$ denotes the state constraint. The epigraph form of the above problem is given by:
% \begin{equation}\label{eq: epi_example}
%     \begin{aligned}
%     \min_{x, z} & \; z  \\
%     \text{s.t.} & \; s(x) \leq 0,  \\
%     & \; C(x) \leq z. 
%     \end{aligned}
% \end{equation}
% Here, $z \in \mathbb{R}$ is an auxiliary optimization variable. Now, observe that the constraints in \eqref{eq: epi_example} can be combined to yield the following:
% \begin{equation}
%     \begin{aligned}
%     \min_{x, z} & \; z  \\
%     \text{s.t.} & \; \max \big(C(x) - z, s(x) \big) \leq 0. 
% \end{aligned}
% \end{equation}

% Now, using the equivalence shown in appendix \ref{appendix: equivalence}, we can say:
% \begin{equation}
%     \begin{aligned}
%     \min_{z} & \; z  \\
%     \text{s.t.} & \; \min_x\max \big(C(x) - z, s(x) \big) \leq 0. 
% \end{aligned}
% \end{equation}


% Now, substituting $C(x)$ and $s(x)$ in the context of the SC-OCP in \eqref{eq: SC-OCP}, we get the following epigraph form:

% \begin{equation}
%     \begin{aligned}
%     \min_{z \in \mathbb{R^+}} & \; z  \\
%     \text{s.t.} \; \max &\left( \min_{u \in \mathcal{U}}\int_t^{T}l(x) \, ds + \phi(x(T))  -z, g(x) \right) \leq 0,
% \end{aligned}
% \end{equation}



% % As $z \to \infty$ (i.e., the "cost budget" for $C(x)$ increases), $s(x)$ will dominate the max, and $x^*$ will focus on minimizing $h$ more. On the other hand, as $z \to -\infty$ (i.e., the "cost budget" for $C(x)$ decreases), $C(x)$ will dominate the max, and $x^*$ will focus on minimizing $C$.

% \subsection{Equivalence of (24) and (25)} \label{appendix: equivalence}

% \begin{lemma}
% Let $x \in \mathbb{R}^n$ and $z \in \mathbb{R}$, and let $g : \mathbb{R}^n \times \mathbb{R} \to \mathbb{R}$ be a continuous (potentially non-differentiable) function. Then, if a solution exists (i.e., an optimal $x^*, z^*$ exist, are finite), then the following optimization problems are equivalent.
% \begin{align}
%     \min_{x, z} & \quad z \\
%     \text{s.t.} & \quad g(x, z) \leq 0,
% \end{align}
% \begin{align}
%     \min_z & \quad z \\
%     \text{s.t.} & \quad \min_x g(x, z) \leq 0.
% \end{align}
% \end{lemma}

% We begin by comparing the Lagrangian primal problem of eq. (A.16) and eq. (A.17).
% \begin{align}
%     \min_x \max_{\lambda \geq 0} \quad z + \lambda g(x, z) \tag{A.18}
% \end{align}
% \begin{align}
%     \min_z \max_{\lambda \geq 0} \quad \lambda \min_x g(x, z) \tag{A.19}
% \end{align}
% Comparing the two, the only difference is that the order of $\min_x$ and $\max_\lambda$ are flipped. Hence, it is sufficient to show that, for any $z$ where $\min_x g(x, z) \leq 0$, 
% \begin{align}
%     p^* = \min_x \max_{\lambda \geq 0} \quad \lambda g(x, z) = \max_{\lambda \geq 0} \min_x \lambda g(x, z) = d^*. 
% \end{align}
% Note that this is exactly equivalent to showing that strong duality holds for the following constraint satisfaction problem.
% \begin{align}
%     \min_x & \quad 0 \\
%     \text{s.t.} & \quad g(x, z) \leq 0. \tag{A.21}
% \end{align}

% We now prove that strong duality holds for the above problem in a similar fashion to the proof that Slater’s condition is a sufficient condition for strong duality to hold in convex optimization problems \cite{8}.

% Define the set $A \subseteq \mathbb{R}^n \times \mathbb{R}$ as
% \begin{align}
%     A &= \{ (u, t) \mid \exists x, g(x, z) \leq u, \quad 0 \leq t \}, \tag{A.22}\\
%     &= \{ u \mid \inf_x g(x, z) \leq u \} \times [0, \infty). \tag{A.23}
% \end{align}

% Note that $A$ is convex. Furthermore, since a feasible solution exists by assumption, we have that
% \begin{align}
%     p^* = \min_x \max_{\lambda \geq 0} \lambda g(x, z) = \min_x \begin{cases} 
%     \infty & g(x, z) > 0 \\
%     0 & g(x, z) \leq 0.
%     \end{cases} \tag{A.24}
% \end{align}

% We now define a second set $B \subseteq \mathbb{R}^n \times \mathbb{R}$ as
% \begin{align}
%     B &= \{ (0, s) \mid s < p^* \}, \tag{A.25}\\
%     &= \{ 0 \} \times (-\infty, 0). \tag{A.26}
% \end{align}

% Note that $B$ is also convex, and that the sets $A$ and $B$ do not intersect. We can then invoke the separating hyperplane theorem to show that there exists a $(\lambda, \mu) \neq 0$ and $\alpha$ that defines a hyperplane which separates the two sets, i.e.,
% \begin{align}
%     (u, t) \in A &\Rightarrow -\lambda u + \mu t \geq \alpha, \tag{A.27}\\
%     (u, s) \in B &\Rightarrow -\lambda u + \mu s \leq \alpha. \tag{A.28}
% \end{align}

% In (A.27), since both $u$ and $t$ are unbounded above, we must have $\lambda \geq 0$ and $\mu \geq 0$. Furthermore, in (A.28), since $s < p^*$, we have that $\mu p^* \leq \alpha$. Combining both then gives us that for all $x$,
% \begin{align}
%     0 = p^* = \mu p^* \leq \alpha \leq \lambda g(x, z). \tag{A.29}
% \end{align}

% Minimizing the RHS over $x$ then maximizing over $\lambda$ then gives us that
% \begin{align}
%     p^* \leq \max_{\lambda} \min_x \lambda g(x, z) \leq \max_{\lambda} \min_x \lambda g(x, z) = d^*. \tag{A.30}
% \end{align}

% Finally, by weak duality, we have that
% \begin{align}
%     p^* \geq d^*. \tag{A.31}
% \end{align}

% Combining the two then allows us to conclude that $p^* = d^*$. \qed

% \section{Derivation of HJB-PDE characterizing the auxiliary value function} \label{appendix: proof_aux_hjb_pde}
% The auxiliary value function is given by:
% \begin{align}
%     \hat{V}(t, x, z) = \max \left( \min_{u \in \mathcal{U}} \int_t^{T} l(x) \, ds + \phi(x(T)) - z, \, g(x) \right).
% \end{align}

% Next, consider the dynamic programming equation for $\hat{V}$. Over a small time interval $\Delta t$, we can express $\hat{V}$ as:

% \begin{equation}
%     \hat{V}(t, x, z) = \int_t^{t + \Delta t} l(x) \, ds + \big(z(t + \Delta t) - z(t)\big) + \hat{V}\big(t + \Delta t, x(t + \Delta t), z(t + \Delta t)\big)
% \end{equation}
% $\forall t \in [0, T)$ and $(x, z) \in \mathcal{X} \times \mathbb{R}^+$.
% Expanding $\hat{V}$ using a Taylor series around $t$, we get:
% \begin{align}
%     \hat{V}(t, x, z) = \int_t^{t + \Delta t} l(x) \, ds + \big(\dot{z} \Delta t\big) + \hat{V}(t, x, z) + \partial_t \hat{V} \, \Delta t 
%     + \min_{u \in \mathcal{U}} \nabla_x \hat{V} \cdot f(t, x, u) \, \Delta t + \nabla_z \hat{V} \cdot \dot{z} \, \Delta t
% \end{align}
% $\forall t \in [0, T)$ and $(x, z) \in \mathcal{X} \times \mathbb{R}^+$,
% Canceling out $\hat{V}(t, x, z)$ on both sides and simplifying terms yields:
% \begin{equation}
%     -\partial_t \hat{V} - \min_{u \in \mathcal{U}} \nabla_x \hat{V} \cdot f(t, x, u) + \nabla_z \hat{V} \cdot l(x) = 0,
% \end{equation}
% $\forall t \in [0, T)$ and $(x, z) \in \mathcal{X} \times \mathbb{R}^+$. 
% We can now rewrite the final partial differential equation (PDE) for $\hat{V}$ as:
% \begin{equation}
%     \min \Bigl(-\partial_t \hat{V} - \min_{u \in \mathcal{U}} \nabla_x \hat{V} \cdot f(t, x, u) + \nabla_z \hat{V} \cdot l(x), \, \hat{V} - g(x) \Bigr) = 0,
% \end{equation}
% $\forall t \in [0, T)$ and $(x, z) \in \mathcal{X} \times \mathbb{R}^+$,

% or equivalently:
% \begin{equation}
%     \min \Bigl(-\partial_t \hat{V} - \min_{u \in \mathcal{U}} \nabla_{\hat{x}} \hat{V} \cdot \hat{f}(t, \hat{x}, u), \, \hat{V} - g(x) \Bigr) = 0,
% \end{equation}

% $\forall t \in [0, T)$ and $(x, z) \in \mathcal{X} \times \mathbb{R}^+$.

\section{Proofs}

\subsection{Proof of Theorem \eqref{thm: safety_verification}}\label{appendix: proof_safety}

% \begin{equation}
%     \delta^* = \min(\hat{V}(0,x,z)|J_{\Tilde{\pi}}(0,x,z)\geq 0)
% \end{equation}
Before we proceed with the proof of the Theorem \eqref{thm: safety_verification}, let us look at the following lemma which describes split conformal prediction:

\begin{lemma}[Split Conformal Prediction \cite{angelopoulos2022gentle}]
\label{lem:split_conformal}
Consider a set of independent and identically distributed (i.i.d.) calibration data, denoted as \(\{(X_i, Y_i)\}_{i=1}^n\), along with a new test point \((X_{\text{test}}, Y_{\text{test}})\) sampled independently from the same distribution. Define a score function \(s(x, y) \in \mathbb{R}\), where higher scores indicate poorer alignment between \(x\) and \(y\). Compute the calibration scores \(s_1 = s(X_1, Y_1), \ldots, s_n = s(X_n, Y_n)\). For a user-defined confidence level \(1-\alpha\), let \(\hat{q}\) represent the \(\lceil (n+1)(1-\alpha) \rceil / n\) quantile of these scores. Construct the prediction set for the test input \(X_{\text{test}}\) as:
\[
\mathcal{C}(X_{\text{test}}) = \{y : s(X_{\text{test}}, y) \leq \hat{q} \}.
\]
Assuming exchangeability, the prediction set \(\mathcal{C}(X_{\text{test}})\) guarantees the marginal coverage property:
\[
\mathbb{P}(Y_{\text{test}} \in \mathcal{C}(X_{\text{test}})) \geq 1 - \alpha.
\]
\end{lemma}

\begin{proof}
    Following the Lemma \ref{lem:split_conformal}, we employ a conformal scoring function for safety verification~\cite{pmlr-v242-lin24a}, defined as:
\begin{equation*}
    s(X) = \inducedauxvfunc(0, \hat{x}),    \forall \hat{x} \in \mathcal{S}_{\tilde\delta},
\end{equation*}
where $\mathcal{S}_{\delta}$ denotes the set of states satisfying $\learnedauxvfunc(0, \hat{x}) \leq \delta$ and the score function measures the alignment between the induced safe policy and the auxiliary value function.

Next, we sample $N_s$ states from the safe set $\mathcal{S}_{\delta}$ and compute conformal scores for all sampled states. For a user-defined error rate $\alpha \in [0, 1]$, let $\hat{q}$ denote the $\frac{(N_s+1)\alpha}{N_s}$th quantile of the conformal scores. According to \cite{vovk2012}, the following property holds:
\begin{equation}
    \underset{\hat{x} \in \mathcal{S}_{\tilde\delta}}{\mathbb{P}}\left(\inducedauxvfunc(0, \hat{x_i}) \leq \hat{q} \right) \sim \text{Beta}(N_s - l + 1, l),    
\end{equation}

where $l = \lfloor (N_s+1)\alpha \rfloor $.

Define $E_s$  as:
\begin{equation*}
    E_s := \underset{\hat{x} \in \mathcal{S}_{\delta}}{\mathbb{P}}\left(\inducedauxvfunc(0, \hat{x}_i) \leq \hat{q}\right).
\end{equation*}
Here, $E_s$ is a Beta-distributed random variable. Using properties of cumulative distribution functions (CDF), we assert that $E_s \geq 1 - \epsilon_s$ with confidence $1 - \beta_s$ if the following condition is satisfied:
\begin{equation} \label{eq:eps_calc_safe}
    I_{1-\epsilon_s}(N - l + 1, l) \leq \beta_s,
\end{equation}
where $I_x(a,b)$ is the regularized incomplete Beta function and also serves as the CDF of the Beta distribution. It is defined as:
\begin{equation*}
    I_x(a, b) = \frac{1}{B(a, b)} \int_{0}^{x} t^{a - 1} (1 - t)^{b - 1} \, dt,
\end{equation*}
where $B(a, b)$  is the Beta function. From~\cite{DLMF}($8.17.5$), it can be shown that $I_x(n-k, k+1) = \sum_{i=1}^{k} \binom{n}{i} x^i (1 - x)^{n - i}$.

% If $\sum_{i=1}^{l+1} \binom{N_s}{i} \epsilon_s^i (1 - \epsilon)^{N_s - i} \leq \beta_s$, $E_s \geq 1-\epsilon_s$.

Then \eqref{eq:eps_calc_safe} can be rewritten as: 
\begin{equation} \label{eq:eps_calc_safe_binom}
    \sum_{i=1}^{l-1} \binom{N_s}{i} \epsilon_s^i (1 - \epsilon)^{N_s - i} \leq \beta_s,
\end{equation}
Thus, if Equation~\eqref{eq:eps_calc_safe_binom} holds, we can say with probability $1-\beta_s$ that:
\begin{equation}\label{eq: cp_quantile}
    \underset{\hat{x} \in \mathcal{S}_{\tilde\delta}}{\mathbb{P}}\left(\inducedauxvfunc(0, \hat{x_i}) \leq \hat{q} \right) \geq 1-\epsilon_s.
\end{equation}
Now, let $k$ denote the cardinality of the set  
$A = \{x \in S_{\delta} \mid \learnedauxvfunc(0, \hat{x}) \leq \delta ,\inducedauxvfunc(0 , \hat{x}) \geq 0\}$. Thus, the safety error rate is given by  $ \alpha_{\delta} = \frac{k+1}{N_s+1}$.
Let $\hat{q}$ represent the $\frac{(N_s+1)\alpha_{\delta}}{N_s}$th quantile of the conformal scores. Since $k$ denotes the number of samples for which the conformal score is positive, the $\frac{(N_s+1)\alpha_{\delta}}{N_s}$th quantile of scores corresponds to the maximum \textit{negative score} amongst the sampled states. This implies that $\hat{q} \leq 0$. From this and Equation~\eqref{eq: cp_quantile}, we can conclude with probability $1 - \beta_s$ that:
\begin{equation*}
    \underset{\hat{x} \in \mathcal{S}_{\delta}}{\mathbb{P}}\left(\inducedauxvfunc(0, \hat{x_i}) \leq 0 \right) \geq 1-\epsilon_s.
\end{equation*}

From Equation~\eqref{eq: aux_vfunc_def}, it can be inferred that $\forall~(t, \hat{x})$, $\hat{V}(t, \hat{x}) \leq \inducedauxvfunc(0, \hat{x})$. Hence, with probability $1 - \beta_s$, the following holds:
\begin{equation*}
    \underset{\hat{x} \in \mathcal{S}_{\delta}}{\mathbb{P}}\left(\hat{V}(0, \hat{x_i}) \leq 0 \right) \geq 1-\epsilon_s.
\end{equation*}
\end{proof}



\subsection{Proof of Theorem \eqref{thm: perf_verification}}\label{appendix: proof_perf}
\begin{proof}
    To quantify the performance loss, we employ a conformal scoring function defined as:

\begin{equation*}
    p(x) := \frac{|\learnedvfunc(0, x_i) - \inducedvfunc(0, x_i)|}{C_{max}}, \forall x \in \mathcal{S}^*
\end{equation*}
where the score function measures the alignment between the induced optimal policy and the value function.

Next, we sample \( N_p \) states from the state space \( \mathcal{S}^* \) and compute conformal scores for all sampled states. For a user-defined error rate \( \alpha_p \in [0, 1] \), let \( \psi \) denote the \(\frac{(N_p+1)\alpha_p}{N_p}\) quantile of the conformal scores. According to \cite{vovk2012}, the following property holds:
\[
\underset{x \in \mathcal{S}^*}{\mathbb{P}}\left(\frac{|\learnedvfunc(0, x_i) - \inducedvfunc(0, x_i)|}{C_{max}} \leq \psi \right) \sim \text{Beta}(N_p - l + 1, l),
\]
where \( l = \lfloor (N_p+1)\alpha_p \rfloor \).

Define \( E_p \) as:
\[
E_p := \underset{x \in \mathcal{S}^*}{\mathbb{P}}\left(\frac{|\learnedvfunc(0, x_i) - \inducedvfunc(0, x_i)|}{C_{max}} \leq \psi\right).
\]
Here, \( E_p \) is a Beta-distributed random variable. Using properties of CDF, we assert that \( E_p \geq 1 - \epsilon_p \) with confidence \( 1 - \beta_p \) if the following condition is satisfied:
\begin{equation} \label{eq:eps_calc_perf}
    I_{1-\epsilon_p}(N_p - l + 1, l) \leq \beta_p,
\end{equation}
where \( I_x(a,b) \) is the regularized incomplete Beta function. From~\cite{DLMF}($8.17.5$), it can be shown that $I_x(n-k, k+1) = \sum_{i=1}^{k} \binom{n}{i} x^i (1 - x)^{n - i}$. Hence, Equation~\eqref{eq:eps_calc_perf} can be equivalently stated as:
\begin{equation} \label{eq:eps_calc_perf_binom}
    \sum_{i=1}^{l-1} \binom{N_p}{i} \epsilon_p^i (1 - \epsilon_p)^{N_p - i} \leq \beta_p
\end{equation}

Thus, if Equation~\eqref{eq:eps_calc_perf_binom} holds, we can conclude with probability \( 1-\beta_p \) that:
\[
\underset{x \in \mathcal{S}^*}{\mathbb{P}}\left(\frac{|\learnedvfunc(0, x_i) - \inducedvfunc(0, x_i)|}{C_{max}} \leq \psi \right) \geq 1-\epsilon_p.
\]
\end{proof}

\vspace{1em}
\section{Additional Details of the systems in the experiments}\label{appendix: system_details}

In this section, we will provide more details about the systems we have used in the experiments section \ref{section: case_studies}.


\subsection{2D Boat} \label{appendix: Boat2D}
The states, $x$ of the 2D Boat system are $x = [x_1, x_2]^T$, where, $x_1, x_2$ are the $x$ and $y$ coordinates of the boat respectively. We define the step cost at each step, $l(t,x)$, as the distance from the goal, given by:
\begin{align*}
    l(t,x) :=  \|x- (1.5, 0)^T\|
\end{align*}
The cost function $C(t, x(t))$ is defined as:  
\begin{equation}
 \begin{aligned}
     C(t,x(t), \ctrlseq) = \int_t^{T} l(t, x(t)) \, dt ~ +
     \phi(x(T))
   \end{aligned}
\end{equation}
where $T$ is the time horizon ($2s$ in our experiment), $l(t, x(t)) = || x(t) - (1.5, 0)^T||$ represents the running cost, and $\phi(x(T)) = || x(T) - (1.5, 0)^T||$ is the terminal cost. Minimizing this cost drives the boat toward the island.

Consequently, the (augmented) dynamics of the 2D Boat system are:
\begin{align*}
    \dot{x_1} &= u_1 + 2 - 0.5 x_2^2 \\
    \dot{x_2} &= u_2 \\
    \dot{z} &= - l(t,x)
\end{align*}
where $u_1, u_2$ represents the velocity control in $x_1$ and $x_2$ directions respectively, with $u_1^2 + u_2^2 \leq 1$
and $2 - 0.5x_2^2$ specifies the current drift along the $x_1$-axis.

% \begin{equation*}
%     \max \left(\partial_t \hat{V} + \min(0,  -\|\nabla_{x_1,x_2} \hat{V}\| - \partial_z \hat{V}) +\partial_x \hat{V}(2-0.5x_2^2),~g(x)- \hat{V}\right) = 0,
% \end{equation*}
The safety constraints are formulated as:  
\begin{align}
    g(x) := max ( 0.4 - \|x - (-0.5, 0.5)^T \|,  0.5 - \|x - (-1.0, -1.2)^T \|) )
\end{align}
where $g(x) > 0$ indicates that the boat is inside a boulder, thereby ensuring that the super-level set of $g(x)$ defines the failure region.

\subsubsection{Ground Truth Comparison}\label{Appendix: GT_comp}
We compute the Ground Truth value function using the Level-Set Toolbox~\cite{mitchell2004toolbox} and use it as a benchmark in our comparative analysis. To facilitate demonstration, unsafe states are assigned a high value of $20$ instead of $\infty$. The value function in this problem ranges from $0$ to $14.76$.

\begin{figure}[h]
\centering
\includegraphics[width=\linewidth]{images/GT_Heat_Map.pdf}
\caption{Heatmap of the value function for the ground truth (left) and our method (right). The yellow region represents the unsafe area. Our method successfully captures most of the safe set, indicating that it is not overly conservative while completely recovering the unsafe regions.}
\label{fig: GT Comparison}
\end{figure}

As illustrated in Figure~\ref{fig: GT Comparison}, the value function obtained using our method closely approximates the ground truth value function. Notably, the unsafe region (highlighted in yellow) remains identical in both cases, confirming the safety of the learned value function. Furthermore, the mean squared error (MSE) between the two value functions is $0.36$, which is relatively low given the broad range of possible values. 

It is also worth mentioning that computing a high-fidelity ground truth value function on a $210 \times 210 \times 210$ grid using the Level Set Toolbox requires approximately $390$ minutes~\cite{10777061}. In contrast, our proposed approach learns the value function in $122$ minutes, achieving a substantial speedup. This demonstrates that even for systems with a relatively low-dimensional state space, our method efficiently recovers an accurate value function significantly faster than grid-based solvers.

\subsubsection{Calculation of Safety Levels}
\begin{figure}[h]
\centering
\includegraphics[width=0.5\linewidth]{images/epsilon_vs_delta.pdf}
\vspace{-1em}
\caption{$\delta$ versus $\epsilon$ for the learned auxiliary value function, $\hat{V}(t, \hat{x})$ after safety verification. It can be observed that $\delta$ varies with different safety violation levels, $\epsilon$. Notably, $\delta$ approaches zero for sufficiently small $\epsilon$, indicating that the learned value function exhibits very few safety violations.}
\label{fig: Delta_vs_Epsilon}
% \vspace{-1em}
\end{figure}

Figure~\ref{fig: Delta_vs_Epsilon} illustrate the $\delta$ vs $\epsilon$ plot obtained after the safety verification algorithm proposed in Theorem~\ref{thm: safety_verification}. We can observe that the $\delta$ level approaches $0$ as the $\epsilon$ values approach the chosen safety level of $0.001$. Hence, we say that the sub-level set of the auxiliary value function, $\hat{V}(t, \hat{x})$ is safe with a probability of $1-0.001=0.999$.  

\subsection{Pursuer vehicle tracking an evader}\label{appendix: Track}
The state, $x$ of a ground vehicle (pursuer) tracking a moving evader is $x = [x_p, y_p, v, \Theta, x_{e}, y_{e}, v_{xe}, v_{ye}]^T$, where, $x_e, y_e, v, \Theta$ are position, linear velocity and orientation of the pursuer respectively, $x_{e}, y_{e}, v_{xe}, v_{ye}$ are the position and the linear velocities of the evader respectively. We define the step cost at each step, $l(t,x)$, as the distance from the goal, given by:
\begin{align*}
    l(t,x) :=  \| (x_p(t), y_p(t))^T - (x_{e}(t), y_{e}(t))^T\|
\end{align*}
and the terminal cost is $\phi(x(T)) = || (x_p(T), y_p(T))^T - (x_e(T), y_e(T))^T||$.
The cost function $C(t, x(t))$ is defined as:
\begin{equation}
 \begin{aligned}
     C(t,x(t), \ctrlseq) = \int_t^{T} l(t, x(t)) \, dt ~ +
     \phi(x(T))
   \end{aligned}
\end{equation}
where $T$ is the time horizon ($1s$ in this experiment). Minimizing this cost aims to drive the pursuer toward the evader. 
Consequently, the (augmented) dynamics of the system is as follows:
\begin{align*}
    \dot{x_p} &= v \cos(\Theta) \\
    \dot{y_p} &= v \sin(\Theta) \\
    \dot{v} &= u_1 \\
    \dot{\Theta} &= u_2 \\
    \dot{x_{e}} &= v_{xe} \\
    \dot{y_{e}} &= v_{ye} \\
    \dot{v_{xe}} &= 0 \\
    \dot{v_{ye}} &= 0 \\
    \dot{z} &= - l(t,x)
\end{align*}
where $u_1$ represents the linear acceleration control and $u_2$ represents angular velocity control.

The safety constraints are defined as:
\begin{align*}
    g(x) :=& max ( 0.2 - \|x - (0.5, 0.5)^T \|,  0.2 - \|x - (-0.5, 0.5)^T \|, 0.2 - \|x - (-0.5, -0.5)^T \|,\\
    &0.2 - \|x - (0.5, -0.5)^T \|, 0.2 - \|x - (0.0, 0.0)^T \|,) )
\end{align*}
which represents 5 obstacles of radius 0.2 units each.

% \begin{equation*}
%     \hat{V}(T, x, z) = \max(-z, \psi(x), g(x)),
% \end{equation*}

% \textbf{Baselines:} 
% MPPI: \cite{8558663}
% C3BF: \cite{10644338}
% SAC: Lagrangian SAC \cite{pmlr-v80-haarnoja18b} 

\subsection{Multi-Agent Navigation}\label{appendix: MultiAgent}

A multi-agent setting with 5 agents. The state of each agent $i$ is represented by $x_i = [x_{a_i}, y_{a_i}, x_{g_i}, y_{g_i}]$, tries to reach its goal while avoiding collisions with others. $(x_{a_i}, y_{a_i})$ denote the position of the $i$th agent, while $(x_{g_i}, y_{g_i})$ represent the goal locations for that agent. We define the step cost at each step, $l(t,x(t))$, as the mean distance of each agent from its respective goal, given by:
\begin{align*}
    l(t, x(t)) := \frac{\sum_{i=1}^{5}\| (x_{a_i}(t), y_{a_i}(t)^T - (x_{g_i}(t), y_{g_i}(t))^T\|}{5}
\end{align*}
The cost function $C(t, x(t), \ctrlseq)$ is defined as:
\begin{equation}
 \begin{aligned}
     C(t,x(t), \ctrlseq) := \int_t^{T} l(t, x(t)) \, dt ~ +
     \phi(x(T))
   \end{aligned}
\end{equation}
where $T$ is the time horizon ($2s$ in this experiment). Minimizing this cost aims to drive each agent towards its goal. 
Consequently, the (augmented) dynamics of the system is as follows:
\begin{align*}
    \dot{x}_{a_i} &= u_{1i}, \forall i \in \{ 1,2,3,4,5\}\\
    \dot{y}_{a_i} &= u_{2i}, \forall i \in \{ 1,2,3,4,5\} \\
    \dot{x}_{g_i} &= 0, \forall i \in \{ 1,2,3,4,5\}\\
    \dot{y}_{g_i} &= 0, \forall i \in \{ 1,2,3,4,5\} \\
    \dot{z} &= - l(t,x)
\end{align*}
where $u_{1i}, u_{2i}$ represents the linear velocity control of each agent $i$.
The safety constraints are defined as:
\begin{equation}
\begin{aligned}
    g(x(t)) := \max_{i, j=\{1,..,5 \}, i\neq j} ( R - \|(x_{a_i}, y_{a_i})^T -  (x_{a_j}, y_{a_j})^T \|)
\end{aligned}
\end{equation}

\section{Implementation Details of the Algorithms}

This section provides an in-depth overview of our algorithm and baseline implementations, including hyperparameter configurations and the cost/reward functions used in the baselines across all experiments.

\subsection{Experimental Hardware}
All experiments were conducted on a system equipped with an 11th Gen Intel Core i9-11900K @ 3.50GHz × 16 CPU, 128GB RAM, and an NVIDIA GeForce RTX 4090 GPU for training.
\vspace{-1em}
\subsection{Hyperparameters for the Proposed Algorithm}
We maintained training settings across all experiments, as detailed below:

\begin{table}[h]
    \centering
    \begin{tabular}{lc}
        \hline
        \textbf{Hyperparameter} & \textbf{Value} \\
        \hline
        Network Architecture & Multi-Layer Perceptron (MLP) \\
        Number of Hidden Layers & 3 \\
        Activation Function & Sine function \\
        Hidden Layer Size & 256 neurons per layer \\
        Optimizer & Adam optimizer \\
        Learning Rate & $2\times 10^{-5}$ \\
        \hline
        \textbf{Boat Navigation} & .\\
        \hline
        Number of Training Points & 65000 \\
        Number of Pre Training Epochs & 50K \\
        No. of Training Epochs & 200K\\
        \hline
        \textbf{Pursuer Vehicle Tracking Evader} & .\\
        \hline
        Number of Training Points & 65000 \\
        Number of Pre Training Epochs & 60K \\
        No. of Training Epochs & 300K\\
        \hline
        \textbf{Multi Agent Navigation} & .\\
        \hline
        Number of Training Points & 65000 \\
        Number of Pre Training Epochs & 60K \\
        No. of Training Epochs & 400K\\
        \hline
    
    \end{tabular}
    \vspace{1em}
    \caption{Hyperparameters for the proposed algorithm}
    \label{tab:training_details}
\end{table}
\vspace{-2em}
\subsection{MPPI based baselines}
For all the experiments we consider the MPPI cost term as follows:

\begin{equation}
    C_{MPPI} = C(t, x(t), \ctrlseq) + \lambda \max(g(x), 0)
\end{equation}
where, $\lambda$ is the trade-off parameter, $C(t,x(t), \ctrlseq)$, $g(x)$ are the cost functions and safety functions as defined in Appendix \ref{appendix: system_details}. Following is the list of hyperparameters we have used for MPPI experiments in all the cases:


\begin{table}[h]
    \centering
    \begin{tabular}{lc}
        \hline
        \textbf{Hyperparameter} & \textbf{Value} \\
        \hline
        Trade-off parameter ($\lambda$) & 100 \\
        Planning Horizon & 20 \\
        Softmax Lambda & 200 \\
        No. of Rollouts & 8000 \\
        \hline
        
        
    \end{tabular}
    \vspace{1em}
    \caption{Hyperparameters for MPPI Baselines}
    \label{tab:mppi_details}
\end{table}
\vspace{-2em}
\subsection{C-SAC hyperparameters}

For all the experiments, we consider the reward term as follows:

\begin{equation}
    R_{CSAC} = -C(t, x(t), \ctrlseq) - \mathbb{I}_{g(x)>0}\times(100) + \mathbb{I}_{l(t, x(t))<0.1}\times(100) 
\end{equation}
where, $C(t,x(t), \ctrlseq)$, $g(x)$ are the cost functions and safety functions as defined in Appendix \ref{appendix: system_details}. Following is the list of hyperparameters we have used for SAC experiments in all the cases:

\begin{table}[h]
    \centering
    % \renewcommand{\arraystretch}{1.5}
    \begin{tabular}{lc}
        \hline
        \textbf{Parameter} & \textbf{Value} \\
        \hline
        Policy Architecture & Multi-Layer Perceptron (MLP) \\
        learning rate & $3 \times 10^{-4}$  \\
        buffer size & $1,000,000$  \\
        learning starts & $10,000$  \\
        batch size & $256$  \\
        Target network update rate ($\tau$) & $0.005$  \\
        Discount factor ($\gamma$) & $0.99$  \\
        \hline
        \textbf{Boat Navigation} & .\\
        \hline
        Number of Training Steps & 1,000,000 \\
        \hline
        \textbf{Pursuer Vehicle Tracking Evader} & .\\
        \hline
        Number of Training Steps & 2,500,000 \\
        \hline
        \textbf{Multi Agent Navigation} & .\\
        \hline
        Number of Training Steps & 1,000,000 \\
        \hline
    \end{tabular}
    \vspace{1em}
    \caption{General Hyperparameters of SAC in our experiments}
    \label{tab:sac_general_params}
\end{table}

\subsection{Computation time Comparison} \label{app: comp_time}

\begin{figure*}[h]
    \centering
    \includegraphics[width=0.97\linewidth]{images/Baseline_Time_Comparison_Color.pdf}
    \vspace{-0.8em}
\caption{This figure presents a comparative analysis of all methods based on online and offline computation time evaluated on the same computing machine. The top plot illustrates the \textbf{offline computation time} for our method and the baselines. Since our method and C-SAC involve training value functions, they incur higher offline computation costs, whereas MPPI-based methods require no offline training. The bottom plot depicts the \textbf{online computation time}, demonstrating that our method and C-SAC have minimal online computation requirements, whereas MPPI-based methods exhibit significantly higher online computational costs.} 
\label{fig: baseline_time_comparison}
\vspace{-1.0em}
\end{figure*}

Figure~\ref{fig: baseline_time_comparison} presents a comparative analysis of the offline and online computation times for our method against the baselines. It can be observed that the proposed approach exhibits better scalability with increasing dimensionality compared to C-SAC, as our method demonstrates a steady growth in training time, whereas C-SAC experiences a sharp increase in offline computation time as the dimensionality rises. Additionally, the online computation time for both our method and C-SAC remains significantly lower than that of online algorithms such as MPPI and MPPI-SF. This highlights the practicality of our method for real-time applications, provided that the offline value function has been precomputed.

\subsection{Comparison of Multi-Agent Navigation with baselines}\label{app: MVC_baselines}

Figures~\ref{fig: MVC_MPPI}, \ref{fig: MVC_MPPI_SF}, and \ref{fig: MVC_SAC} illustrate the trajectories obtained by the baseline methods for the Multi-Agent Navigation problem. It can be observed that the trajectories obtained by MPPI and MPPI-SF are highly conservative, implying that these methods prioritize safety to mitigate potential conflicts among agents. In contrast, the policy derived from C-SAC fails to maintain safety, resulting in agent collisions. This indicates that as system complexity increases, the baseline methods tend to prioritize either safety or performance, leading to suboptimal behavior and safety violations. Conversely, the proposed approach effectively co-optimizes safety and performance, even in complex high-dimensional settings, achieving superior performance while ensuring safety.

\begin{figure*}[h]
    \centering
    \includegraphics[width=1.0\linewidth]{images/MVC_MPPI_Traj.pdf}
    \vspace{-0.8em}
\caption{Snapshots of multi-agent navigation trajectories at different time instances using \textbf{MPPI}. The trajectories indicate that the agents adopt a \textbf{highly conservative strategy} to prevent collisions. Consequently, this leads to a \textbf{reduction in performance}, as the agents \textbf{end up very far from their respective goals}.
} 
\label{fig: MVC_MPPI}
\vspace{-1em}
\end{figure*}
\begin{figure*}[h]
    \centering
    \includegraphics[width=1.0\linewidth]{images/MVC_MPPI_SF_Traj.pdf}
    \vspace{-0.8em}
\caption{Snapshots of multi-agent navigation trajectories at different time instances using \textbf{MPPI-NCBF}. The observed trajectories demonstrate \textbf{suboptimal behavior similar to that of the MPPI policy}. Consequently, this results in high-performance costs, indicating its \textbf{inability to effectively co-optimize safety and performance.}} 
\label{fig: MVC_MPPI_SF}
\end{figure*}
\begin{figure*}[h]
    \centering
    \includegraphics[width=1.0\linewidth]{images/MVC_SAC_Traj.pdf}
    \vspace{-0.8em}
\caption{Snapshots of multi-agent navigation trajectories at different time instances using C-SAC. The trajectories indicate that agents \textbf{demonstrate less conservative behavior compared to MPPI and MPPI-NCBF, but they lead to collisions}. These \textbf{safety violations are critical} and cannot be disregarded, further \textbf{highlighting the limitations of the baseline methods in simultaneously optimizing safety and performance.}} 
\label{fig: MVC_SAC}
\end{figure*}


\end{document}
