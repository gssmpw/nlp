

\begin{abstract}
We present TUMTraffic-VideoQA, a novel dataset and benchmark designed for spatio-temporal video understanding in complex roadside traffic scenarios. The dataset comprises 1,000 videos, featuring 85,000 multiple-choice QA pairs, 2,300 object captioning, and 5,700 object grounding annotations, encompassing diverse real-world conditions such as adverse weather and traffic anomalies. By incorporating tuple-based spatio-temporal object expressions, TUMTraffic-VideoQA unifies three essential tasks—multiple-choice video question answering, referred object captioning, and spatio-temporal object grounding—within a cohesive evaluation framework. We further introduce the TUMTraffic-Qwen baseline model, enhanced with visual token sampling strategies, providing valuable insights into the challenges of fine-grained spatio-temporal reasoning. Extensive experiments demonstrate the dataset’s complexity, highlight the limitations of existing models, and position TUMTraffic-VideoQA as a robust foundation for advancing research in intelligent transportation systems.  The dataset and benchmark are publicly available to facilitate further exploration.
\end{abstract}

% Unlike existing VQA benchmarks, 
% , including fine-grained object recognition, spatio-temporal grounding, and multi-modal reasoning