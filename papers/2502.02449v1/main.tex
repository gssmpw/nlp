% CVPR 2025 Paper Template; see https://github.com/cvpr-org/author-kit

\documentclass[10pt,twocolumn,letterpaper]{article}

%%%%%%%%% PAPER TYPE  - PLEASE UPDATE FOR FINAL VERSION
\usepackage{cvpr}              % To produce the CAMERA-READY version
% \usepackage[review]{cvpr}      % To produce the REVIEW version
% \usepackage[pagenumbers]{cvpr} % To force page numbers, e.g. for an arXiv version

% Import additional packages in the preamble file, before hyperref
% \newcommand{\CG}{\mathcal{G}\xspace}
\newcommand{\CV}{\mathcal{V}\xspace}
\newcommand{\CE}{\mathcal{E}\xspace}
\newcommand{\CA}{\mathcal{A}\xspace}
\newcommand{\CF}{\mathcal{F}\xspace}
\newcommand{\CR}{\mathcal{R}\xspace}
\newcommand{\CB}{\mathcal{B}\xspace}
\newcommand{\CX}{\mathcal{X}\xspace}
\newcommand{\CK}{\mathcal{K}\xspace}
\newcommand{\CM}{\mathcal{M}\xspace}
\newcommand{\CC}{\mathcal{C}\xspace}
\newcommand{\CL}{\mathcal{L}\xspace}
\newcommand{\CI}{\mathcal{I}\xspace}
\newcommand{\CQ}{\mathcal{Q}\xspace}
\newcommand{\CO}{\mathcal{O}\xspace}
\newcommand{\CP}{\mathcal{P}\xspace}
\newcommand{\CS}{\mathcal{S}\xspace}
\newcommand{\CT}{\mathcal{T}\xspace}
\newcommand{\CJ}{\mathcal{J}\xspace}
\usepackage[para]{footmisc}
\usepackage{subfig}
% \usepackage{subcaption}
% \usepackage{array}
% \usepackage{colortbl}



% It is strongly recommended to use hyperref, especially for the review version.
% hyperref with option pagebackref eases the reviewers' job.
% Please disable hyperref *only* if you encounter grave issues, 
% e.g. with the file validation for the camera-ready version.
%
% If you comment hyperref and then uncomment it, you should delete *.aux before re-running LaTeX.
% (Or just hit 'q' on the first LaTeX run, let it finish, and you should be clear).
\usepackage{amsmath}
\usepackage{amssymb}
\usepackage{mathtools}
\usepackage{amsthm}

\usepackage{color}
\usepackage{colortbl}
\usepackage{tabularx}
\usepackage{tcolorbox}
\usepackage{courier}
\usepackage{graphbox}
\usepackage{booktabs}
\usepackage{amsmath,amsfonts}
\usepackage{graphicx} 
\usepackage{subcaption}
\usepackage{tabularray}
% \usepackage{caption}
\usepackage{url}
\usepackage{etoolbox}
\usepackage{wasysym}
\usepackage[export]{adjustbox}
\usepackage{tikz}
\usepackage{bm}
% \usepackage{cite}
\usepackage{academicons}
\usepackage{multirow}
\usepackage{multicol}
\usepackage{algorithm}
\usepackage{algorithmic}
\usepackage{graphicx}  

\definecolor{cvprblue}{rgb}{0.21,0.49,0.74}
\usepackage[pagebackref,breaklinks,colorlinks,allcolors=cvprblue]{hyperref}

\title{\LARGE \bf
TUMTraffic-VideoQA: A Benchmark for Unified Spatio-Temporal Video Understanding in Traffic Scenes
}

% Spatial-temporal grounding. 

% and Spatial-Temporal Object Grounding

% <-this % stops a space
% \thanks{This research is accomplished within the project ”AUTOtech.agil” (Grant Number 01IS22088A). We acknowledge the financial support for the project by the Federal Ministry of Education and Research of Germany (BMBF). (Corresponding author: Xingcheng Zhou)}
% \thanks{ The authors are with the School of Computation, Information and Technology, Technical University of Munich, 85748 Garching, Germany
% }
% \thanks{ $^\star$ Corresponding Author: \texttt{xingcheng.zhou@tum.de}}

%%%%%%%%% TITLE - PLEASE UPDATE
% \title{\LaTeX\ Author Guidelines for \confName~Proceedings}

%%%%%%%%% AUTHORS - PLEASE UPDATE
\author{
Xingcheng Zhou$^{\star}$,   \hspace{0.3em}
Konstantinos Larintzakis,   \hspace{0.3em}
Hao Guo,           \hspace{0.3em}
Walter Zimmer,  \hspace{0.3em}
Mingyu Liu,  \hspace{0.3em}
Hu Cao, \\
Jiajie Zhang,  \hspace{0.1em}
Venkatnarayanan Lakshminarasimhan,  \hspace{0.1em}
Leah Strand, \hspace{0.1em}
Alois C. Knoll \\
% For a paper whose authors are all at the same institution,
% omit the following lines up until the closing ``}''.
% Additional authors and addresses can be added with ``\and'',
% just like the second author.
% To save space, use either the email address or home page, not both
% \and
% Second Author\\
% Institution2\\
% First line of institution2 address\\
{\small  $^\star$ \texttt{Corresponding: xingcheng.zhou@tum.de}}
}

\makeatletter

\let\@oldmaketitle\@maketitle
\renewcommand{\@maketitle}{\@oldmaketitle
  \centering
  \url{http://traffix-videoqa.github.io}\\[8pt]
  \setcounter{figure}{0}
  \begin{center}
  % \includegraphics[width=\textwidth,height=9cm,keepaspectratio]{figure/first_figure-1st_image.pdf}
  \includegraphics[width=\textwidth,height=9cm,keepaspectratio]{figure/first_figure-1st_image.jpg}
    \captionof{figure}{TUMTraffic-VideoQA introduces a comprehensive benchmark for video-level traffic scene understanding. Our baseline model, TraffiX-Qwen, is capable of solving multiple tasks, including video QA, spatio-temporal grounding, and referred object captioning, within a unified model. In our approach, the spatio-temporal location of objects is represented as tuples \( (c, fn, x, y) \), where \( c \) serves as a unique object identifier, \( fn \) denotes the normalized frame timestamp, and \( (x, y) \) denote the center of the object in the image, normalized with respect to the image dimensions.
}
    \label{fig:title_figure}
  \end{center}

  }  
\makeatother

\begin{document}
\maketitle
\begin{abstract}

% Recent works to jointly reconstruct 3D human and object from a single RGB image, are mostly model-based, that fail to capture the fine details of the clothed human body and object surface. In this paper, we introduce ReCHOR, a novel, model-free, first-method to produce realistic clothed human-object reconstructions from a monocular view. This is extremely challenging due to human-object occlusions, diverse interactions and depth ambiguity, as it needs to infer both 3D spatial awareness and high resolution details. Our core idea is based on estimating neural implicit representations for human and object respectively by an attention-based neural implicit model that attends to pixel-aligned features from both the global human-object image for spatial awareness and  the local separate view of human and object images for high quality details. Additionally, the network is conditioned on semantic features from an initial estimated human-object pose prior and a generative diffusion model that inpaints occluded regions, thus enabling the retrieval of details from them.
% We also propose a synthetic dataset with rendered scenes of diverse, inter-occluded 3D human and object scans, to train our network. We evaluate our method on the synthetic and real world BEHAVE dataset. Our experiments show that our method outperforms the SOTA in achieving realistic clothed human-object reconstructions.
Recent approaches to jointly reconstruct 3D humans and objects from a single RGB image represent 3D shapes with template-based or coarse models, which fail to capture details of loose clothing on human bodies. In this paper, we introduce a novel implicit approach for jointly reconstructing realistic 3D clothed humans and objects from a monocular view. For the first time, we model both the human and the object with an implicit representation, allowing to capture more realistic details such as clothing. This task is extremely challenging due to human-object occlusions and the lack of 3D information in 2D images, often leading to poor detail reconstruction and depth ambiguity. To address these problems, we propose a novel attention-based neural implicit model that leverages image pixel alignment from both the input human-object image for a global understanding of the human-object scene and from local separate views of the human and object images to improve realism with, for example, clothing details. Additionally, the network is conditioned on semantic features derived from an estimated human-object pose prior, which provides 3D spatial information about the shared space of humans and objects. To handle human occlusion caused by objects, we use a generative diffusion model that inpaints the occluded regions, recovering otherwise lost details. For training and evaluation, we introduce a synthetic dataset featuring rendered scenes of inter-occluded 3D human scans and diverse objects. Extensive evaluation on both synthetic and real-world datasets demonstrates the superior quality of the proposed human-object reconstructions over competitive methods.
\end{abstract}
\section{Introduction}
\label{sec:intro}
% Image editing methods in diffusion models depend on user-defined control directions - users can unlock their creativity using these methods by specifying the desired manipulation through prompts~\cite{gandikota2023concept}, reference images~\cite{ruiz2022dreambooth, kumari2022customdiffusion, gal2022image, chen2024trainingfreeregionalpromptingdiffusion}, or attribute vectors~\cite{parmar2023zero,hertz2022prompt}. In this work, we ask a fundamentally different question: \emph{Can we automatically discover the underlying visual structure of a concept within diffusion model's knowledge?} %Rather than requiring user-specified controls, we aim to decompose the model's internal knowledge into meaningful directions.

% This question touches on a fundamental limitation in how we interact with diffusion models. Current control methods ~\cite{zhang2023addingconditionalcontroltexttoimage, gandikota2023concept, ye2023ipadaptertextcompatibleimage,ye2023ipadaptertextcompatibleimage, hertz2024stylealignedimagegeneration, li2023photomaker, shi2024instantbooth, chen2024trainingfreeregionalpromptingdiffusion} require users to specify their desired manipulations in advance, limiting interactive creativity. This contrasts with natural human artistic workflows, where creators dynamically explore creative ideas while jointly refining them toward meaningful artistic outcomes~\cite{hoffmann2016modeling}. This synergy between specification and exploration is not new to generative models. Early GAN architectures naturally developed disentangled latent spaces that enabled continuous\cite{harkonen2020ganspace,radford2015unsupervised, wu2021stylespace, shen2020interfacegan}, compositional control over generated images. Users could explore these spaces to discover interesting variations that would be difficult to describe in words~\cite{wu2021stylespace}, then combine them to achieve their creative goals~\cite{grabe2022towards}. 


% While diffusion models have largely superseded GANs in conditional image synthesis~\cite{dhariwal2021diffusion},  their underlying structure remains less understood. Diffusion models achieve remarkable diversity through high-dimensional latents, unlike GANs' compact latent spaces.  With a single prompt, diffusion models can generate radically different variations through different random initializations of input noise. We ask - Is it possible to discover interpretable structure within this vast space of variations?

Text-to-image diffusion models are capable of generating remarkable visual variations from a single prompt through different random initializations. However, this vast creative potential remains largely opaque to users---while we can generate diverse images, we lack understanding of the underlying structure of these variations. This presents a fundamental challenge: how can we discover and expose the latent visual capabilities encoded within these models?

\let\thefootnote\relax \footnote{$^{*}$Correspondence to \texttt{gandikota.ro@northeastern.edu}}

The challenge touches on a key limitation in how we interact with diffusion models today. Current control methods require users to explicitly specify their desired edits in advance through prompts~\cite{gandikota2023concept}, reference images~\cite{zhang2023addingconditionalcontroltexttoimage, chen2024trainingfreeregionalpromptingdiffusion, ruiz2022dreambooth,kumari2022customdiffusion, Ryu_lora, hu2021lora}, or attribute vectors~\cite{ye2023ipadaptertextcompatibleimage, hertz2024stylealignedimagegeneration, li2023photomaker, shi2024instantbooth,parmar2023zero,hertz2022prompt}. That contrasts sharply with natural human creative workflows, where artists dynamically explore creative ideas and jointly refine them toward meaningful artistic outcomes~\cite{hoffmann2016modeling}. The need for pre-specified controls creates a barrier between users and the full creative potential of these models.

Interestingly, earlier generative models like GANs~\cite{gans,karras2019style,brock2018large} naturally developed more interpretable internal structures. Their compact latent spaces often exhibited emergent disentanglement~\cite{harkonen2020ganspace,radford2015unsupervised, wu2021stylespace, shen2020interfacegan}, enabling continuous and compositional control over generated images. Users could explore these spaces to discover interesting variations that would be difficult to describe in words~\cite{wu2021stylespace}, then combine them to achieve their creative goals~\cite{grabe2022towards}.

Diffusion models have largely superseded GANs in conditional image synthesis~\cite{dhariwal2021diffusion}, achieving greater diversity through much higher-dimensional latents. And yet an understanding of the underlying structure of these larger latent spaces has remained elusive. In this work, we ask a fundamental question: \emph{Can we automatically discover the visual structure within a diffusion model's knowledge of a concept?} Rather than requiring user-specified controls, we aim to decompose the model's internal representations into expressive directions that users can explore and combine.

To address these needs, we present \textbf{SliderSpace}, a framework that brings systematic explorability to diffusion models. Given just a text prompt, SliderSpace discovers a canonical set of meaningful, diverse, and controllable directions within the model's knowledge of that concept. Each direction is implemented as a low-rank adapter~\cite{hu2021lora} that can be scaled and composed with others, allowing users to explore and smoothly combine different aspects of variation, as shown in Figure~\ref{fig:intro}.

We ground SliderSpace discovery in three key requirements for meaningful decomposition of a diffusion model's visual manifold: 
\begin{enumerate}
    \item \textbf{Unsupervised Discovery:} The decomposition process should emerge from the intrinsic structure of the model's learned representation, rather than being guided by predefined attributes. This ensures we capture the true topology of the model's knowledge space rather than projecting our assumptions onto it.
    
    \item \textbf{Semantic Orthogonality:} Each discovered control must represent a distinct semantic direction. This is enforced in a semantic feature space, like CLIP, where every slider has an orthogonal effect in embeddings. This prevents discovering multiple controls that create similar semantic effects, making the system more efficient and easier.
    
    \item \textbf{Distribution Consistency:} Directions must induce consistent transformations across both random seeds and prompt variations. 
\end{enumerate}

These requirements naturally lead to our proposed framework, which we formalize in Section~\ref{sec:method}. As we show in our experiments, SliderSpace is architecture-agnostic, working with both conventional U-Net based models like Stable Diffusion~\cite{rombach2022high, rombach2022sd20, podell2023sdxl, turbo, dmd} and recent transformer-based architectures like Flux~\cite{flux}.

We demonstrate the expressiveness of SliderSpace through three applications: First, we show how SliderSpace can decompose high-level concepts into diverse and expressive components, revealing the natural axes of variation in the model's understanding. Second, we explore artistic style variation, where SliderSpace discovers directions that match or exceed the diversity of manually curated artist lists while being judged more useful by human evaluators. Finally, we show how SliderSpace can help reverse the mode collapse commonly observed in distilled diffusion models, restoring diversity while maintaining generation speed.

Beyond providing practical creative control, SliderSpace opens new avenues for understanding and utilizing the latent capabilities of diffusion models. By mapping these models' visual potential into intuitive, composable directions, we take a step toward making their creative possibilities more accessible and interpretable to users.

% Image editing methods in diffusion models unlock the creativity of users. In this work we ask an alternate question: \emph{Can we organize and expose what of the diffusion model is already capable of?}.
% Existing methods for controlling image generation typically require users to manually specify edit directions for desired changes. This process is time-consuming, requires technical expertise, and limits the spontaneity of the creative process. For instance, if a user wants to adjust the smile of a generated person, they must explicitly request this edit, often through imprecise prompt engineering or model fine-tuning. This approach of predefined controls or manual specifications restricts users from fully exploring the latent capabilities of the model. There may be interesting stylistic variations or attributes that the model can generate, but users have no easy way to discover or utilize these.

% Natural visual disentanglement was an emergent property in the latent space of Generative Adversarial Models (GANs) \cite{harkonen2020ganspace,radford2015unsupervised, wu2021stylespace, shen2020interfacegan}. In particular, it has been observed that StyleGAN~\cite{karras2019style} stylespace neurons offer detailed control over many meaningful aspects of images that would be difficult to describe in words~\cite{wu2021stylespace}. However, diffusion models do not share such a compact latent space~\cite{park2023unsupervised}; and efforts to uncover such a space in the semantic embeddings of the text conditioning have met with limited success \nik{Nick - is there a specific citation you were thinking about?}.

% In this work we introduce \textbf{SliderSpace}, which takes a step towards uncovering an analogous low dimensional representation of diffusion models' visual breadth; in essence treating the diffusion model as many generators sharing parameters, where a particular generator is defined by a specific prompt. For a given prompt we sample many random seeds (and optionally prompt expansions using an LLM), generate the corresponding images, and apply an off the shelf feature extractor (in this work CLIP, but our method can be applied to any differentiable feature extractor). We use PCA to analyze these features, and for each of the leading $k$ principal components we train a LoRA \cite{} which causes the diffusion model to produces images which increase the feature magnitude along that component when passed back through the same feature extractor. This leads to a 'Slider' for each principal component, because each LoRA can be scaled and applied to the original diffusion model, continuously varying those visual features in the generated results (as measured, in our case, by CLIP).

% There are many other works that enhance the controllability of diffusion models. One common approach is enabling users to add spatial constraints to a generation either manually, or via a reference image \cite{zhang2023addingconditionalcontroltexttoimage, chen2024trainingfreeregionalpromptingdiffusion}, a second is leveraging more abstract embeddings (e.g. identity, style) extracted from a reference image \cite{ye2023ipadaptertextcompatibleimage, hertz2024stylealignedimagegeneration, li2023photomaker, shi2024instantbooth}, a third is finetuning a foundation model to better generate a concept important to the user \cite{ruiz2022dreambooth, kumari2022customdiffusion, Ryu_lora, hu2021lora}, and a fourth (most relevant to this work) is finding low-rank adaptors of the model based on a prompt or small training set which can be scaled to provide continous control over one aspect of generated image (e.g. night vs day, basic vs luxury, etc.) \cite{gandikota2023concept}. SliderSpace is complementary to all of these methods and offers something distinct. All of the other methods we are aware require the user (and / or model designer) to know in advance what type of control they want. In contrast SliderSpace assists users in discovering and controlling hidden capabilities present in the diffusion model's distribution of possible generations.

%We propose that truly intuitive creative control in a text-to-image model should meet three key criteria: \emph{discoverability}, \emph{intuitiveness}, and \emph{specificity}. The model should reveal controllable attributes that may not be immediately obvious, offer controls that are easy to understand and manipulate, and ensure each control affects a distinct attribute of the generated image.

% We demonstrate the utility and power of SliderSpace using three applications built on top of SDXL-DMD \cite{dmd}, because its fast generation speed lends itself well to the continuous control offered by SliderSpace.

% First, we study concept decomposition (Section \ref{sec:concept_exp}), where we learn sliders for a specific concept (e.g. 'monster', 'waterfall', 'car'). Through quantitative metrics of diversity and text alignment we demonstrate that the learned sliders dramatically boost the diversity of generations when randomly applied without harming text alignment; we also ask humans to qualitatively judge these results in a user study where they find the SliderSpace results to be more 'Diverse', 'Useful', and 'Creative' than our baselines.

% Second, we attempt to compare the automatic discoveries of SliderSpace to a large scale manual study of artistic styles (Section \ref{sec:art_exp}), open-sourced by ParrotZone \cite{parrotzone}. In this study SDXL was prompted with over 4300 artist names,  and based on visual inspection the cases of successful stylistic mimicry recorded. Quantitatively SliderSpace more closely matches the distribution of artistic variation discovered by ParrotZone than other baselines, and in our user studies was judged to be significantly more 'Diverse' and 'Useful' than the baselines. To our surprise humans even judged SliderSpace results to be slightly more 'Diverse' than the results generated by the manually discovered artist names of \cite{parrotzone}.

% Third, we attempt to use SliderSpace to reverse the mode collapse commonly observed in distilled few-step diffusion models relative to the original teacher model (Section \ref{sec:diverse_exp}). We quantitatively demonstrate that applying SliderSpace to SDXL-DMD leads to more closely matching the distribution of images by the original teacher, SDXL.

%Through extensive experiments on various state-of-the-art text-to-image models, we demonstrate that SliderSpace significantly enhances user control and creative expression in AI-assisted image generation tasks. Our method enables a range of applications, including concept decomposition and control, diversity improvement in generated images, customization dissection and edits, and the exploration of artistic styles inherent in the model.

% SliderSpace goes beyond providing a practical tool for enhanced creative control. By mapping the visual potential of diffusion models it can open new avenues for generative creativity and deepens our understanding of each model's hidden potential.

\section{Related Work}

\begin{figure}[bt!]
    \centering
    % First row
    \begin{subfigure}[t]{0.48\linewidth}
        \centering
        \includegraphics[width=\textwidth]{figure/rep11.png}
        \caption{Objects with the prompt: \textit{A white truck that is stationary in the same direction.} \cite{nuprompt}}
    \end{subfigure}
    \hfill
    \begin{subfigure}[t]{0.48\linewidth}
        \centering
        \includegraphics[width=\textwidth]{figure/rep21.png}
        \caption{Frame-based object expression using numerical coordinates \cite{drivelm}.}
    \end{subfigure}
    
    
    % Second row
    \begin{subfigure}[t]{0.48\linewidth}
        \centering
        \includegraphics[width=\textwidth]{figure/TrafficQA-Object_Representation_rep12.jpg}
        \caption{Object referring in \cite{vidstg} with prompt: \textit{What is beneath the adult}.}
    \end{subfigure}
    \hfill
    \begin{subfigure}[t]{0.48\linewidth}
        \centering
        % \includegraphics[width=\textwidth]{figure/rep22.jpg}
        \includegraphics[width=\textwidth]{figure/TrafficQA-Object_Representation_22.jpg}
        \caption{Location of the green bus \textit{[(c1,0.0,0.5,0.4)]} in the video. (Ours)}
        \label{fig:objct_ref4}
    \end{subfigure}
    
    \caption{Different methods for describing objects in images and videos using language expressions. We adopt a tuple-based spatio-temporal object representation for the unique object reference, as shown in (d). }
    \label{fig:object_representation}
\end{figure}


% \begin{table}[htb]
% \centering
% \resizebox{0.5\textwidth}{!}{%
% \begin{tabular}{cccccccc}
% % \hline
% \midrule
% \makecell{\textbf{Dataset}} & \makecell{\textbf{Tasks}} & \makecell{\textbf{QA Gen.}} & \makecell{\textbf{\# Videos}\\\textbf{/Scenes}} & \makecell{\textbf{\# QAs}}  & \makecell{\textbf{\# Grounds.}} & \makecell{\textbf{Domain}} \\

% % \\\textbf{/Capts.}
% % \hline
% \midrule

% HAD \cite{had}         & Video QA & Manual & 5.6k & 45k & - & Driving \\

% DRAMA \cite{malla2023drama}         & Video QA& Manual & 18k  & 102k  & - & Driving \\

% LingoQA \cite{marcu2024lingoqavisualquestionanswering}  & Video QA & Manual & 28k & 419k & - & Driving \\

% NuScenes-QA \cite{qian2024nuscenes}         & Image QA & Template & 850 & 460k &  - & Driving \\

% DriveLM \cite{sima2023drivelm}         & Image QA & Temp. + Man. & 188k  & 4.2M & - & Driving \\

% City-3DQA \cite{sun20243dquestionansweringcity} & Scene QA & Temp + Man. & 193 & 450k & - &  City \\

% \midrule
% HC-STVG \cite{hc-stvg} & Video Grounding & Manual &5.6k & - & 5.6k&General\\

% DVD-ST \cite{dvd-st} & Video Grounding & Manual & 2.7k & - &5.7k & General  \\

% Refer-KITTI \cite{referkitti} & Referred-MOT & Manual & 18 & - & 818 & Driving \\

% NuPrompt \cite{nuprompt}         & Referred-MOT & LLM & 850 & - & 35k  & Driving \\

% \midrule


% \textbf{TUMTraffic-VideoQA (Ours)} & \makecell{Video QA, \\ST Grounding} & Temp. + LLM  &1k & 88k  & 5.7k &  Roadside \\


% % \hline
% \midrule
% \end{tabular}%
% }
% \caption{Summary and comparison of language datasets in the traffic domain for question answering, video grounding, and referred multi-object tracking.}
% \label{tab:related_datasets}
% \end{table}



\begin{table*}[thb!]
\centering
\caption{Summary of visual-language datasets in the traffic domain for question answering, video grounding, and referred multi-object tracking. The table’s upper section presents QA tasks, while the lower section covers grounding and referring tasks. We introduce the first roadside video understanding dataset and unify the tasks in one benchmark. }
\resizebox{\textwidth}{!}{%
\begin{tabular}{c|ccccccccccc}
% \hline
\midrule
\textbf{Dataset} & \textbf{Venue} & \textbf{Tasks} & \textbf{QA Gen.} & \textbf{\# Videos/Scenes} & \textbf{\# QAs/Captions}  & \textbf{\# Grounding} & \textbf{Domain} \\
% \hline
\midrule


% BDD-X \cite{kim2018textual}         & ECCV18 & video-level & Manual & $\sim$7k (v) & $\sim$26k & $\sim$3.7 & $\sim$77h & - & 1 & 4 & Driving \\

% HAD \cite{had}         & CVPR'19 & Video QA & Manual & 5.6k & 45k & - & Driving \\

% SUTD \cite{xu2021sutd}         & CVPR 2021 & video-level & Manual & $\sim$10k (v) & $\sim$63k & $\sim$6.3 & - & 70s & - & - & D + T \\

DRAMA \cite{malla2023drama}         & WACV'23 & Video QA& Manual & 18k  & 102k  & - & Driving \\
LingoQA \cite{marcu2024lingoqavisualquestionanswering}  & ECCV'24 & Video QA & Manual & 28k & 419k & - & Driving & \\

NuScenes-QA \cite{qian2024nuscenes}         & AAAI'24 & Image QA & Template & 850 & 460k &  - & Driving \\

DriveLM \cite{drivelm}         & ECCV'24 & Image QA & Temp. + Man. & 188k  & 4.2M & - & Driving \\

% ELM \cite{zhou2024embodied}         & ECCV'24 & Video-Level & Temp. + LLM & - & $\sim$9M & - &  Driving \\


% SQA-3D \cite{sqa3d}  & ICLR'23 & Scene QA & Manual & 650 & 33.4k & - & Indoor\\

City-3DQA \cite{sun20243dquestionansweringcity} & ACM MM'24& Scene QA & Temp. + Man. & 193 & 450k & - &  City \\

\midrule
HC-STVG \cite{hc-stvg} & ACM MM'22 & Video Grounding & Manual &5.6k & - & 5.6k&General\\

DVD-ST \cite{dvd-st} & -  & Video Grounding & Manual & 2.7k & - &5.7k & General  \\

Refer-KITTI \cite{referkitti} & CVPR'23  & Referred-MOT & Manual & 18 & - & 818 & Driving \\

NuPrompt \cite{nuprompt}         & AAAI'25 & Referred-MOT & LLM & 850 & - & 35k  & Driving \\

% STPR && Video-Level &&5.2k&-&30k &General \\

% VD-STG\cite{vidstg} &&&&\\

\midrule

\textbf{TUMTraffic-VideoQA (Ours)} & - & Video QA, ST Grounding & Temp. + LLM  &1k & 87.3k  & 5.7k &  Roadside \\

% \hline
\midrule
\end{tabular}%
}

\label{tab:related_datasets}
\end{table*}

% Granularity in thousands?

% Can we trust an information about a dataset which was found only in another paper?  

% Modality ?


% \begin{table}[htb]
% \centering
% \resizebox{0.5\textwidth}{!}{%
% \begin{tabular}{c|ccccc}
% % \hline
% \midrule
% \textbf{Dataset} & \textbf{Task} & \textbf{\#Scenes} & \textbf{\#QA}  & \textbf{\#Grounding} & \textbf{Domain} \\
% % \hline
% \midrule

% HAD \cite{had}         & Video QA & $\sim$5.6k & $\sim$45k & - & Driving \\

% DRAMA \cite{malla2023drama}         & Video QA & $\sim$18k  & $\sim$102k  & - & Driving \\

% NuScenes-QA \cite{qian2024nuscenes}         & Image QA & 850 & $\sim$460k &  - & Driving \\

% DriveLM \cite{sima2023drivelm}         & Image QA & $\sim$188k  & $\sim$4.2M & - & Driving \\

% SQA-3D \cite{sqa3d}  & Scene QA & 650 & 33.4k & - & Indoor \\

% City-3DQA \cite{sun20243dquestionansweringcity} & Scene QA & 193 & 450k & - & City \\

% \midrule
% Refer-KITTI\cite{referkitti} & Referred-MOT & 18 & - & 818 & Driving \\

% NuPrompt \cite{nuprompt}         & Referred-MOT & 850 & - & 35k  & Driving \\

% DVD-ST\cite{dvd-st} & Video Grounding & 2.7k & - &5.7k & General \\

% HC-STVG\cite{hc-stvg} & Video Grounding & 5.6k & - & 5.6k & General \\

% \midrule

% \textbf{TUMTraffic-VideoQA(Ours)} & Video QA, Grounding & 1k & 88k  & 5.7k & Traffic \\

% % \hline
% \midrule
% \end{tabular}%
% }
% \caption{Related datasets}
% \label{tab:related_datasets}
% \end{table}


% [x] connect table 1 with introductions. 

\subsection{Vision-Language Datasets in Traffic Scenes}
% DriveLM\cite{drivelm},
% HAD \cite{had} and 
With the rapid advancements in LLMs, significant efforts have been made to integrate language into the development of vision-language foundation models. As summarized in Table \ref{tab:related_datasets}, several pioneering datasets have been introduced for traffic scenarios, particularly focusing on vehicle-centric environments \cite{addatasetseurvey}. NuScenes-QA \cite{qian2024nuscenes} provides a question-answering benchmark tailored for driving scenes. Meanwhile, DRAMA \cite{malla2023drama} is designed for video-level open-ended tasks aimed at evaluating driving instructions and assessing the importance of objects within their environments. Besides, referring to specific traffic participants through natural language—commonly known as referred object grounding and tracking—is a crucial task in traffic scene understanding. Some works \cite{referkitti,nuprompt} extend the KITTI \cite{kitti} and nuScenes \cite{caesar2020nuscenesmultimodaldatasetautonomous} datasets, by associating natural language descriptions with specific vehicles and pedestrians. This facilitates fine-grained identification and tracking of traffic participants, allowing for precise object localization based on language descriptions in complex driving environments. However, most existing efforts primarily focus on driving scenarios and are typically constrained to individual tasks such as question answering, video grounding, or referred multi-object tracking. A significant research gap also remains in the availability of large-scale datasets designed specifically for roadside surveillance scenarios. Our work aims to bridge this gap by providing a comprehensive dataset tailored for multiple tasks in roadside traffic understanding within a unified framework.
% is also an important aspect of traffic scene understanding
% introducing a standardized object representation and 


\subsection{Fine-Grained Video Understanding}

Fine-grained video understanding centers on the precise analysis of intricate video content, targeting tasks that demand nuanced reasoning across spatial and temporal dimensions. Some representative tasks include spatio-temporal grounding \cite{vidstg,hc-stvg}, mapping specific objects or events to precise locations and times within a video based on a given query; video object referring \cite{mevis,referkitti,nuprompt}, which involves tracking objects through space and time given text prompts; video temporal grounding \cite{UniVTG,huang2024vtimellm}, identifying specific moments or intervals in a video that align with a provided textual query. These tasks require high precision, nuanced multimodal alignment, and the ability to capture subtle temporal and spatial dynamics. It is particularly challenging due to the difficulty of properly representing fine-grained video details and the inherent cross-modality misalignment. With the advancement of visual LLMs, recent advancements enhance the capabilities of fine-grained video understanding \cite{videunderstandingsurvey} and facilitate understanding across abstract and detailed levels. 

% , with advanced visual embedding techniques and modality alignment strategies to bridge the gap between textual and visual semantics, significantly





\subsection{Language-Based Object Referring}


Referring objects in visual data, such as images and videos, is typically achieved by associating them with predefined definitions or language descriptions. Figure \ref{fig:object_representation} illustrates four commonly used methods for representing objects through language expressions. The inherent ambiguity of natural language, coupled with the modality gap between visual and linguistic representations, presents significant challenges. Object representation in tasks such as object referring often necessitates careful dataset curation to ensure that linguistic expressions uniquely or collectively correspond to specific objects in videos. For example, some datasets include only scenarios with uniquely identifiable objects \cite{hc-stvg}, while others contain expressions that jointly refer to multiple objects \cite{dvd-st}. However, in complex real-world applications such as autonomous driving, textual descriptions alone are often insufficient to uniquely specify an object. To address this challenge, DriveLM \cite{drivelm} introduces a structured tuple representation, $\textless c, CAM, x, y \textgreater$, where  c  denotes the object identifier,  CAM  specifies the camera, and $\textless x, y \textgreater$ represents the 2D center coordinates within the camera’s coordinate system. Alternatively, ELM \cite{zhou2024embodied} simplifies the problem by converting temporal video tasks into frame-level questions, using a tuple $\textless c, x, y \textgreater$ to identify objects within individual frames without temporal dependencies. Despite the advancements, formulating a unified, precise, and unique language representation for objects in video remains open challenges. 




In this work, we design a spatio-temporal object representation in videos with a four-element tuple format $(c, f_n, x, y)$, where c denotes a unique object identifier, $f_n$ indicates the normalized frame timestamp, and $(x, y)$ corresponds to the object’s normalized spatial coordinates within the frame.  The same object is consistently assigned the identifier  c  throughout the video, while its spatial position changes over time. This formulation enables precise tracking and referencing of objects across both spatial and temporal dimensions, facilitating robust language-based interaction in dynamic environments. Besides, it provides a standardized interface for fine-grained video understanding, enabling more detailed and structured analysis.

 
\section{\method Dataset}

In this section, we introduce the \method dataset, which covers authentic data of 17,966 characters from 771 renowned books. 
\method features its authentic, non-synthesized dialogues with real-world intricacies, and comprehensive data representations supporting various usages. 
In Table ~\ref{tab:dataset_stats}, we provide a comprehensive comparison with existing datasets. 
We illustrate our dataset's design principles in \S\ref{sec:data_design},  curation pipeline in \S\ref{sec:data_pipeline}, and statistical analysis in \S\ref{sec:data_statistics}.

\begin{figure*}[!t]
    \centering
    \includegraphics[width=\textwidth, center]{Figures/CoSER-main.pdf}
    \vspace{0.2cm}
    \caption{
    Overview of \method's dataset, training and evaluation. 
    Left: The \method dataset is sourced from renowned books and processed via LLM-based pipeline. 
    It contains rich data types on plots, conversations and characters.  
    Right: 
    We apply given-circumstance acting to train and evaluate role-playing LLMs using these conversations.  
    For training, each sample trains the LLM to portray  a specific character in a conversation, using their  original dialogue.  
    For evaluation, 
    we build a multi-agent system for conversation simulation given the same scenario, and assess the simulated dialogue via  penalty-based LLM critics. 
    }
    \label{fig:main}
\end{figure*}

\subsection{Design Principles}
\label{sec:data_design}

As shown in Table ~\ref{tab:dataset_stats}, \method differs from previous RPLA datasets mainly in its:  
\textit{1)} rich data types, 
\textit{2)} internal thoughts and physical actions in messages,
\textit{3)} environment as a role.


\textbf{Rich Types of Data} \quad 
The persona data $\personadata_\persona$ can represent a character $\persona$ from fictional works in diverse forms, \eg, narratives, profiles, dialogues, experiences, \etc. 
Previous work focuses primarily on profiles and dialogues, which represent limited knowledge.  
Hence, we propose a more comprehensive set of data types that are: 
\textit{1)} Comprehensive: covering extensive knowledge about characters and plots from the books; 
\textit{2)} Orthogonal: carrying distinct, complementary information with little redundancy;
\textit{3)} Contextual-rich: providing sufficient context to enable $\agent_\persona$ to faithfully reproduce $\persona$'s behaviors and responses in given scenarios.


Specifically, we organizes knowledge from books hierarchically via three interconnected elements: plots, conversations and characters. 
Each \textbf{plot} comprises its raw text, summary, conversations in this plot, and key characters' current states and experiences in this plot. 
A \textbf{conversation} contains not only the dialogue transcripts, but also rich contextual settings including scenario descriptions and characters' motivations. 
\textbf{Characters} are associated with their conversations and plots, based on which we craft their profiles. 

\textbf{Thoughts and Actions in Messages} \quad 
Previous RPLA studies typically restrict RPLAs' output space to verbal speech alone, limiting their ability to fully represent human interactions. 
In this paper, we extend the message space of RPLAs and character datasets into three distinct dimensions: speech ($\mathcal{L}$), action ($\mathcal{A}$), and thought ($\mathcal{T}$), significantly enriching the expressiveness. 
For instance, an RPLA can convey silence by generating only thoughts and actions without verbal speech. 
The three dimensions are distinguished by markup symbols and function mechanisms:  
\begin{itemize}[itemsep=-3pt, topsep=0pt, partopsep=0pt]
    \item \textbf{Speech} is for verbal communications of characters.
    \item \textbf{Action} captures physical behaviors, body language, facial expressions, \etc. Similar to  tool use in agents~\citep{weng2023agent}, actions can be programmed to trigger downstream events in multi-agent systems. 
    \item \textbf{Thought} represents internal thinking processes, which enable RPLAs to simulate sophisticated human cognition. 
    Thoughts should be invisible to others, forming  information asymmetry~\citep{zhou2024sotopia}. 
\end{itemize}

\textbf{Environment as a Role} \quad 
In RPLA applications like AI TRPG~\footnote{Tabletop Role-Playing Games}~\citep{liang2023tachikuma}, LLMs often serve as world simulators that respond to players' actions. 
To promote this ability, we consider environment as a special role $e$, which provide environmental responses such as physical changes and reactions from unspecified characters or crowds.

\subsection{Dataset Curation} 
\label{sec:data_pipeline}

We curate the \method dataset through a systematic LLM-based pipeline that transforms book content into high-quality data for RPLAs  
~\footnote{In this paper, we employs Claude-3.5-Sonnet (20240620).}. 
The details are as follows. 

\textbf{Source Selection} \quad 
Our dataset is sourced from most acclaimed literary works to ensure data quality and character depth. 
We identify the top 1,000 books on \textit{Goodreads}'s \textit{Best Books Ever} list~\footnote{https://www.goodreads.com/list/show/1.Best\_Books\_Ever}, and obtain the content for 771 books.
As shown in Table~\ref{tab:selected_books}, these books  offer characters and narratives with literary significance and widespread recognition across diverse genres, time periods, and cultural backgrounds.

\textbf{Chunking} \quad 
We segment book contents into chunks to fit in LLMs' context window. 
We employs both static, chapter-based strategy and dynamic, plot-based strategy. 
Initially, we use regular expressions to identify chapter titles as natural chunk boundaries. 
Then, we merge adjacent small chunks and split large chunks to ensure moderate chunk sizes. 
However, static chunking neglects the storyline and truncates important plots or conversations. 
To address this, we implement dynamic plot-based chunking, \ie, during data extraction, we also prompt LLMs to identify truncated plots or trailing content in the current chunk, and concatenate them with the subsequent chunk to ensure plot integrity.


\textbf{Data Extraction} \quad 
We employ LLMs to extract plot and conversation data from book chunks, including (1) contents, summaries and character experiences of plots, and (2) dialogues and background settings of conversations. 
The extracted data representations are illustrated in Fig. ~\ref{fig:front} and introduced in \S\ref{sec:data_design}. 
In the messages, speeches are always extracted from the original dialogues, while actions and thoughts can either be extracted or inferred by LLMs based on the context. 
For evaluation purposes, we hold out data from the final 10\%  plots in each book.


\textbf{Organizing Character Data} \quad 
Based on the extracted data, we form the knowledge bases for characters in three steps.  
First, we unify character references by establishing name mappings between aliases and canonical names using LLMs, \eg, mapping \textit{Lord Snow} to an unified identifier \textit{Jon Snow}. 
Second, we aggregate relevant plots and conversations for each character. 
Finally, we leverage LLMs to generate character profiles based on their extracted data, describing them from multiple perspectives including background, experiences, physical characteristics, personality traits, core motivations, relationships, character arcs, \etc. 

For technical details, including our prompts, engineering implementation, and handling mechanisms for exception caused by LLMs, please refer to ~\S\ref{sec:app_dataset}. 



 
\section{TUMTraffic-Qwen Baseline}
% In this section, we introduce the baseline model of the TUMTraffic-VideoQA dataset. We provide a detailed description of the model architecture and introduce our training recipes. 

%  [x] TODOS, projector and sampler reverse  !

\subsection{Model Architecture}

We introduce TUMTraffic-Qwen, a baseline model for the TUMTraffic-VideoQA dataset that effectively addresses all three tasks within a unified framework. The architecture of the TUMTraffic-VideoQA baseline, as illustrated in Figure \ref{baseline_model}, consists of four core components: visual encoder $f_v$, cross-modality projector $ g_\psi $, token sampler $\mathcal{S}_v$, and large language model $f_\phi$, following \cite{li2024llavaonevisioneasyvisualtask}. \\


% \begin{equation}
% \small
% p(\mathbf{X}_a \mid \mathbf{X}_v, \mathbf{X}_{\text{instruct}}) = \prod_{i=1}^{L} p_\theta(x_i \mid \mathbf{X}_v, \mathbf{X}_{\text{instruct}}, <i, \mathbf{X}_a^{<i})
% \end{equation}


\noindent\textbf{Visual Encoder.} The video is uniformly divided into 100 segments, including the first and last frames, resulting in a total of \( N = 101 \) frames. Given the sampled video input \( \mathbf{X} \in \mathbb{R}^{N \times H \times W \times 3} \), we adopt SigLIP \cite{siglip}, a Transformer-based model pre-trained on large-scale language-image datasets, as the visual encoder. Each frame is processed at a resolution of \( 384 \times 384 \), and the video is encoded into a sequence of visual features \( Z_v = [v_1, \dots, v_N] \), where \( v_i = f_v (\mathbf{X}_i) \in \mathbb{R}^{T \times C} \), containing $T$ spatial tokens of dimension $C$.

 
 
\noindent\textbf{Token Sampling Strategy.} We leverage a simple yet effective frame-level multi-resolution sampling strategy to enhance feature representation. We evaluate four primary sampling strategies: spatial pooling, multi-resolution spatial pooling, multi-resolution token pruning, and multi-resolution temporal pooling. The output $Z_{v}$ from the last layer of SigLIP is denoted as $Z_{\text{high}}$, which is reduced to $T'$ tokens after down-sampling. We define the set of high-resolution frames as keyframes, denoted by $\mathcal{K}(\cdot)$. Additionally, a learnable token is appended to the end of each frame to explicitly differentiate them. The number of tokens used in various strategies is presented in Table \ref{tab:tokennum}.

% To explicitly model inter-frame relations, the baseline model excludes the temporal aggregation module.
% , leaving this for future exploration
% the spatial down-sampling factor as $P$,
% for reducing visual token numbers 
\noindent\textbullet\  \textbf{Spatial Pooling}: This method applies spatial pooling to each feature map $Z_{\text{high}}$, resulting in a down-sampled representation $Z_{\text{low}} = f_{\text{pool}}(Z_{\text{high}})$ with $N \times T'$ tokens, as shown in Eq. \ref{formula:spatial_pooling}. We use the notation $[ \cdot ]_{n}^{N}$ to represent the operation of sequentially concatenating the processed feature maps.



\begin{figure}[t!]
    \centering
    \includegraphics[width=0.5\textwidth]{figure/TrafficQA-baseline2.jpg}
    \caption{Overview of the TUMTraffic-Qwen baseline model. Yellow and orange colors represent the combination of multi-resolution visual tokens from different visual strategies, while blue indicates textual tokens.}
    \label{baseline_model}
\end{figure}

{\small
\begin{equation}
S_v(Z_v) =  [ Z_{\text{low}}^{n}, Z_{\text{learn}} ]_{n=1}^N 
\label{formula:spatial_pooling}
\end{equation}}


\noindent\textbullet\ \textbf{MultiRes Spatial Pooling}: Compared to the naive spatial pooling, this strategy selects the first frame as the keyframe $\mathcal{K}$ = (1), and is retained at its original resolution $Z_{\text{high}}^1$. It is formulated in Eq. \ref{formula:mluti-spatial_pooling}. 



{\small
\begin{equation}
S_v(Z_v) = [ Z_{\text{high}}^1, Z_{\text{learn}}, [ Z_{\text{low}}^{n}, Z_{\text{learn}} ]_{n=2}^N \big]
\label{formula:mluti-spatial_pooling}
\end{equation}}


\noindent\textbullet\  \textbf{MultiRes Token Pruning}: Similar to MultiRes Spatial Pooling, the first frame is designated as the keyframe. Token-wise cosine similarity is then computed between the keyframe and each subsequent frame, while visual tokens with lowest similarity are selectively retained based on predefined ratio $r$, formulated as $Z_{\text{pruned}} = f_{\text{prune}}^{r}(Z_{\text{high}} )$, shown in Eq. \ref{formula:mluti-spatial_spar}. To ensure visual token efficiency comparable to spatial pooling, $r$ is set to 0.25. A similar strategy is also applied in autonomous driving scenarios \cite{ma2024videotokensparsificationefficient}.


 \begin{table}[bt!]
% \centering
\caption{Comparison of visual token numbers across different token sampling strategies. We keep the high resolution at 27×27 and the low resolution at 14×14.}
\resizebox{0.48\textwidth}{!}{

\begin{tabular}{c|c|c }
\midrule
\textbf{Method}  & \textbf{Number of Visual Tokens}  & \textbf{Max Tokens}                                   \\ \midrule
Spatial Pooling                         & $ N \times T'  + N$            & 19,897    \\  \midrule

MultiRes Spatial-Pooling        & $T +   (N-1) \times T'  + N$       &   20,430         \\ \midrule
% 729 + 100x196 + 101

MultiRes Token-Pruning         & $T + (N-1) \times r \times T  + N$        & 18,574          \\ \midrule
% 729+100*183 + 101

MultiRes Temporal-Pooling          & $K \times T + (N-K) \times T'   + N$    &   20,963        \\ \midrule

\end{tabular}}
% \caption{Comparison of visual token number across different token sampling strategies.}

\label{tab:tokennum}
\vspace{-2pt}

\end{table}




 % In the Multi-Resolution Token Pruning strategy, we set the token retention ratio to \( r = 0.25 \), to balances resolution and computational efficiency.



% \begin{table*}[t!]
% \centering
% \resizebox{\textwidth}{!}{%
% \begin{tabular}{l | l | cc | cc | cc | cc| cc | c}
% % \hline
% \midrule
% \multirow{2}{*}{\textbf{Models}} & \multirow{2}{*}{\textbf{Category}} & \multicolumn{2}{c}{\textbf{Positioning}} & \multicolumn{2}{c}{\textbf{Counting}} & \multicolumn{2}{c}{\textbf{Motion}} & \multicolumn{2}{c}{\textbf{Class}} & \multicolumn{2}{c}{\textbf{Existence}} & \multirow{2}{*}{\textbf{Overall}} \\
% % \cline{3-12}
% % \midrule

%  & & \textbf{E} & \textbf{H} & \textbf{E} & \textbf{H} & \textbf{E} & \textbf{H} & \textbf{E} & \textbf{H} & \textbf{E} & \textbf{H}  \\
% % \hline
% \midrule

% \multicolumn{13}{c}{Open-Source Models} \\ 
% % \hline
% \midrule

% \multirow{1}{*}{LLAVA-OneVision \cite{li2024llavaonevisioneasyvisualtask} } &  0.5B & 25.26 & 42.10 & 27.62 &  30.45 & 54.87 & 37.04 & \textbf{57.06} & 39.57 & \textbf{85.29} & 58.35 & 45.82 \\

% \rowcolor{gray!10}
% & 7B & 22.03 & \textbf{46.92} & \textbf{69.42} & \textbf{54.85} & 61.14 & \textbf{60.48} & 51.92 & \textbf{56.50} & 77.08 & 63.25 & \textbf{56.36} \\

% % \cline{2-13}
% \midrule

% \multirow{1}{*}{Qwen2-VL \cite{Qwen-VL}} &
%   2B & \textbf{26.05} & 36.73 & 38.10 & 39.78 & 56.46 & 35.19 & 32.10 & 38.49 & 68.87 & 67.32 & 43.91 \\

% & 7B & 24.35 & 36.03 & 66.91 & 49.11 & \textbf{61.65} & 38.10 & 44.83 & 40.20 & 54.00 & \textbf{73.03} & 48.82 \\

% % \cline{2-13}
% \midrule

% \multirow{1}{*}{VideoLLAMA2 \cite{cheng2024videollama2advancingspatialtemporal}}  & 2.0-7B-8F & 18.14 & 42.54 & 44.13 & 37.56 & 59.37 & 35.87 & 39.05 & 44.07 & 44.56 & 65.56 & 43.09 \\

% & 2.0-7B-16F & 10.47 & 42.41 & 55.98 & 41.94 & 53.80 & 52.26 & 44.16 & 47.75 & 66.93 & 64.82 & 48.05 \\

% % & 2.1-7B-16F & 14.97 & 27.81 & 49.60 & 39.36 & 34.75 & 30.30 & 40.16 & 40.46 & 75.92 & 65.49 & 41.88 \\ 

% % \hline
% \midrule

% \multicolumn{13}{c}{TUMTraffic-VideoQA Baseline} \\ 
% % \hline
% \midrule
% \multirow{4}{*}{Baseline-0.5B (Ours)} & Spatial Pooling  & 68.47 & 75.54 & 85.31 & 75.82 & 83.92 & \textbf{81.26} & 79.95 & 59.73 & 93.06 & 85.37 & 78.84 \\

%  & MultiRes Spatial-Pooling & 69.32 & 76.36 & 86.10 & 75.86 & 83.73 & 79.59 & \textbf{80.57} & 61.70 & 92.73 & 85.37 & 79.07 \\
 
% & MultiRes Token-Pr.  &  73.40 & \textbf{76.61} & \textbf{86.33} & 76.88 & 83.48 & 78.60 & 80.01 & 60.43 & \textbf{93.34} & 85.27 & 79.44 \\


% \rowcolor{gray!10}
% & MultiRes Temporal-Pooling & \textbf{74.07} & 75.85 & 85.65 & \textbf{76.92} & \textbf{84.05} & 80.64 & 80.26 & \textbf{62.21} & 93.06 & \textbf{85.55} & \textbf{79.83} \\
% % \hline
% \midrule

% \multirow{4}{*}{Baseline-7B (Ours)} & Spatial Pooling & 76.14 & 76.99 & 87.07 & 76.81 & 86.58 & 82.07 & 82.72 & 64.11 & 93.62 & 85.27 & 81.14 \\

% & MultiRes Spatial-Pooling  & 76.99 & \textbf{78.89} & 87.07 & 77.49 & \textbf{88.29} & 81.82 & 83.52 & \textbf{65.95} & 93.01 & \textbf{85.51} & 81.85 \\

% & MultiRes Token-Pr. & \textbf{77.24} & 76.93 & 87.41 & 77.76 & 86.46 & 80.64 & 82.66 & 65.00 & \textbf{93.84} & 85.48 & 81.34 \\


% \rowcolor{gray!10}
% & MultiRes Temporal-Pooling & \textbf{77.24} & 78.57 & \textbf{87.53} & \textbf{78.22} & 87.09 & \textbf{82.68} & \textbf{83.33} & 65.76 & 93.78 & 85.34 & \textbf{81.95} \\
% \midrule
% % \hline
% \end{tabular}%
% }
% \caption{Evaluation of Open-source models and TUMTraffic-Qwen baseline on the Multi-Choice QA track of the TUMTraffic-VideoQA Dataset, where \textbf{E} represents easy, single-hop questions, and \textbf{H} denotes hard, multi-hop questions.}
% \label{table:consolidated_metrics}
% \vspace{-2pt}

% \end{table*}

%  % \cite{ma2024videotokensparsificationefficient}\\\


 \begin{table*}[t!]
 \caption{Evaluation of Open-source models and TUMTraffic-Qwen baseline on the Multi-Choice QA track of the TUMTraffic-VideoQA Dataset, where \textbf{E} represents easy, single-hop questions, and \textbf{H} denotes hard, multi-hop questions.}
\centering
\resizebox{\textwidth}{!}{%
\begin{tabular}{l | l | cc | cc | cc | cc| cc | c}
\midrule
\multirow{2}{*}{\textbf{Models}} & \multirow{2}{*}{\textbf{Category}} & \multicolumn{2}{c}{\textbf{Positioning}} & \multicolumn{2}{c}{\textbf{Counting}} & \multicolumn{2}{c}{\textbf{Motion}} & \multicolumn{2}{c}{\textbf{Class}} & \multicolumn{2}{c}{\textbf{Existence}} & \multirow{2}{*}{\textbf{Overall}} \\
 & & \textbf{E} & \textbf{H} & \textbf{E} & \textbf{H} & \textbf{E} & \textbf{H} & \textbf{E} & \textbf{H} & \textbf{E} & \textbf{H}  \\
\midrule

\multicolumn{13}{c}{Open-Source Models} \\ 
\midrule

\multirow{1}{*}{LLAVA-OneVision \cite{li2024llavaonevisioneasyvisualtask} } &  0.5B & 42.10 & 25.26 & 27.62 &  30.45 & 54.87 & 37.04 & \textbf{57.06} & 39.57 & \textbf{85.29} & 58.35 & 45.82 \\

\rowcolor{gray!10}
& 7B & \textbf{46.92} & 22.03 & \textbf{69.42} & \textbf{54.85} & 61.14 & \textbf{60.48} & 51.92 & \textbf{56.50} & 77.08 & 63.25 & \textbf{56.36} \\
\midrule

\multirow{1}{*}{Qwen2-VL \cite{Qwen-VL}} &
  2B & 36.73 & \textbf{26.05} & 38.10 & 39.78 & 56.46 & 35.19 & 32.10 & 38.49 & 68.87 & 67.32 & 43.91 \\

& 7B & 36.03 & 24.35 & 66.91 & 49.11 & \textbf{61.65} & 38.10 & 44.83 & 40.20 & 54.00 & \textbf{73.03} & 48.82 \\
\midrule

\multirow{1}{*}{VideoLLaMA2 \cite{cheng2024videollama2advancingspatialtemporal}}  & 2.0-7B-8F & 42.54 & 18.14 & 44.13 & 37.56 & 59.37 & 35.87 & 39.05 & 44.07 & 44.56 & 65.56 & 43.09 \\

& 2.0-7B-16F & 42.41 & 10.47 & 55.98 & 41.94 & 53.80 & 52.26 & 44.16 & 47.75 & 66.93 & 64.82 & 48.05 \\
\midrule

\multicolumn{13}{c}{TUMTraffic-VideoQA Baseline} \\ 
\midrule
\multirow{4}{*}{Baseline-0.5B (Ours)} & Spatial Pooling  & 75.54 & 68.47 & 85.31 & 75.82 & 83.92 & \textbf{81.26} & 79.95 & 59.73 & 93.06 & 85.37 & 78.84 \\

 & MultiRes Spatial-Pooling & 76.36 & 69.32 & 86.10 & 75.86 & 83.73 & 79.59 & \textbf{80.57} & 61.70 & 92.73 & 85.37 & 79.07 \\
 
& MultiRes Token-Pruning  &  \textbf{76.61} & 73.40 & \textbf{86.33} & 76.88 & 83.48 & 78.60 & 80.01 & 60.43 & \textbf{93.34} & 85.27 & 79.44 \\

\rowcolor{gray!10}
& MultiRes Temporal-Pooling & 75.85 & \textbf{74.07} & 85.65 & \textbf{76.92} & \textbf{84.05} & 80.64 & 80.26 & \textbf{62.21} & 93.06 & \textbf{85.55} & \textbf{79.83} \\
\midrule

\multirow{4}{*}{Baseline-7B (Ours)} & Spatial Pooling & 76.99 & 76.14 & 87.07 & 76.81 & 86.58 & 82.07 & 82.72 & 64.11 & 93.62 & 85.27 & 81.14 \\

& MultiRes Spatial-Pooling  & \textbf{78.89} & 76.99 & 87.07 & 77.49 & \textbf{88.29} & 81.82 & \textbf{83.52} & \textbf{65.95} & 93.01 & \textbf{85.51} & 81.85 \\

& MultiRes Token-Pruning & 76.93 & \textbf{77.24} & 87.41 & 77.76 & 86.46 & 80.64 & 82.66 & 65.00 & \textbf{93.84} & 85.48 & 81.34 \\

\rowcolor{gray!10}
& MultiRes Temporal-Pooling & 78.57 & \textbf{77.24} & \textbf{87.53} & \textbf{78.22} & 87.09 & \textbf{82.68} & 83.33 & 65.76 & 93.78 & 85.34 & \textbf{81.95} \\
\midrule
\end{tabular}%
}

\label{table:consolidated_metrics}
\vspace{-2pt}

\end{table*}

{\small
\begin{equation}
S_v(Z_v) = [ Z_{\text{high}}^1, Z_{\text{learn}}, [ Z_{\text{pruned}}^{n}, Z_{\text{learn}} ]_{n=2}^N ]
\label{formula:mluti-spatial_spar}
\end{equation}}



\noindent\textbullet\  \textbf{MultiRes Temporal Pooling}: In this strategy, the keyframe set is adaptively queried by input questions $\mathcal{K}(\cdot)=\mathcal{Q}(X_q)$. Based on the temporal regions of interest derived from the question, $K$ keyframes are selected, which are preserved with high-resolution representations $Z_{\text{high}}^{n}$. Meanwhile, the remaining frames undergo spatial pooling, resulting in $Z_{\text{low}}^{n}$, as expressed in Eq. \ref{formula:mluti-temporal}. Typically, $K \leq 2$, and for general questions without specific temporal focus, the first frame is set as the default keyframe.



% [x] formula consistent & outside.

{\small
\begin{equation}
\begin{split}
S_v(Z_v) = [ Z_{v}^{n}, Z_{\text{learn}} ]_{n=1}^N  \\
\text{where } Z_v^{n} =
\begin{cases}
Z_{\text{high}}^{n}, & \text{if } n \in \mathcal{K}(\cdot), \\
Z_{\text{low}}^{n}, & \text{if } n \notin \mathcal{K}(\cdot)
\end{cases}
\end{split}
\label{formula:mluti-temporal}
\end{equation}}




% \textbf{Large Language Models.} We use Qwen2 as the pre-trained LLMs in the TUMTraffic-VideoQA baseline with its strong in-context learning ability and proven performances in cross-modality visual question answering. Specifically, we adopt the lightweight Qwen2-0.5B-Instruction model \cite{qwen2} with hidden size of 896 and 32k context length.  


\noindent\textbf{Large Language Model.} We adopt Qwen-2 \cite{qwen2} as the pre-trained LLM in our TUMTraffic-Qwen baseline. Qwen-2 demonstrates strong capabilities in in-context learning and instruction following, supporting context lengths of up to 32k tokens. This allows for the processing of complex and long-form inputs effectively. We utilize two versions of Qwen-2, namely 0.5B and 7B, to establish baselines of different scales. The answer generation process in our TUMTraffic-Qwen baseline model is formulated as:

{\small
\begin{equation}
p(X_a \mid S_v(Z_v), X_q) = \prod_{t=1}^{\mathcal{T}} P_{\phi,\psi}\big(x_t \mid x_{1:t-1}, S_v(Z_v), X_q)
\end{equation}}

% The 0.5B model features a 24-layer Transformer with a hidden size of 896, offering a lightweight yet effective solution. 7B model, with a 28-layer Transformer and a hidden size of 3584, is designed for enhanced reasoning and representation capabilities. 
% We adopt the instruction-tuned versions of Qwen-2 as the pre-trained LLM for our baseline.



\subsection{Baseline Training} Our baseline model undergoes a two-stage training process consisting of video-language alignment and visual instruction fine-tuning, to enhance its understanding of traffic scenarios and reasoning capabilities for long videos. Both stages are trained with 4 NVIDIA A100 GPUs. 

\noindent\textbf{Video-Language Alignment.} This step aims to align video representations with language embeddings, ensuring that the LLM can effectively interpret the visual features. We freeze both the visual encoder and the LLM, and train only the projector layer. To facilitate the training, we initialize the parameters of the 2-layer MLP from the LLaVA-OneVision model, which has been pre-aligned with large-scale cross-modality datasets, including 3.2M single-image and 1.6M OneVision image-caption pairs. In this stage, we further train the projector on raw TUMTraffic-VideoQA data, with open-ended captioning pairs without transforming to the multiple-choice QA for 1 epoch. 
% format

\noindent\textbf{Visual Instruction Fine-Tuning.} Building upon the robust representations established during the alignment stage, we further fine-tune our baseline model on the training set of TUMTraffic-VideoQA. The multi-choice QA pairs are reformatted into the instruction-following format to prompt the model to generate the corresponding answers. During this stage, we freeze the vision encoder and projector layers and finetune the Qwen-2 model with full-parameter fine-tuning to adapt its reasoning and contextual understanding ability. The model is fine-tuned for 1 epoch.







\section{Experiments}
\textbf{Setup.} We evaluate the performance of PINNMamba on three standard PDE benchmarks: convection, wave, and reaction equations, all of which are identified as being affected by failure modes~\cite{krishnapriyan2021characterizing,zhao2024pinnsformer}. The details of those PDEs can be found in Appendix~\ref{apx:setup}.
    We compare PINNMamba with four baseline models, vanilla PINN~\cite{raissi2019physics}, QRes~\cite{bu2021quadratic}, PINNsFormer~\cite{zhao2024pinnsformer}, and KAN~\cite{liu2024kan} .
For fair comparison, we sample 101$\times$101 collection points with uniformly grid sampling, following previous work~\cite{zhao2024pinnsformer,wu2024ropinn}. We also evaluate on PINNacle Benchmark~\cite{hao2023pinnacle} and Navier–Stokes equation~\cite{raissi2019physics}.

\begin{table*}
\vspace{-3mm}
  \caption{Results for solving convection, reaction, and wave equations.}
  \label{sample-table}
  
  \centering
    \small
  \begin{tabular}{l|c|ccc|ccc|ccc}

    \toprule 
  & & \multicolumn{3}{c}{Convection }&\multicolumn{3}{c}{Reaction}&\multicolumn{3}{c}{Wave}\\
    \cmidrule(lr){3-5}\cmidrule(lr){6-8}\cmidrule(lr){9-11}
   Model & \#Params &Loss & rMAE & rRMSE & Loss & rMAE & rRMSE& Loss & rMAE & rRMSE
 \\   \midrule
    PINN&527361& 0.0239 & 0.8514 & 0.8989& 0.1991 & 0.9803 & 0.9785& 0.0320 & 0.4101 & 0.4141\\
    QRes & 396545& 0.0798 & 0.9035 & 0.9245& 0.1991 & 0.9826 & 0.9830& 0.0987 & 0.5349 & 0.5265\\
    PINNsFormer &453561 & 0.0068 & 0.4527 & 0.5217& 3e-6& 0.0146 & 0.0296 & 0.0216 & 0.3559 & 0.3632\\
     KAN&891& 0.0250 & 0.6049 & 0.6587& 7e-6 & 0.0166 & 0.0343& 0.0067 & 0.1433 & 0.1458\\
   \rowcolor{mygray}   PINNMamba  & 285763&0.0001 & \textbf{0.0188} & \textbf{0.0201}&1e-6&\textbf{0.0094}&\textbf{0.0217}& 0.0002 & \textbf{0.0197} & \textbf{0.0199} \\

    \bottomrule
  \end{tabular}
  \normalsize
  \label{tab:diff}
  \vspace{-4mm}
\end{table*}

\begin{figure*}[t!]
    \centering
    \includegraphics[width=\textwidth]{_fig/wave}
    \vspace{-8mm}
    \caption{The ground truth solution, prediction (top), and absolute error (bottom) on wave equations.}
    \label{fig:wave}
    \vspace{-5mm}
  %  \vspace{-1mm}
\end{figure*}

\textbf{Training Details.} We train PINNMamba and all the baseline models 1000 epochs with L-BFGS optimizer~\cite{liu1989limited}.
We set the sub-sequence length to 7 for PINNMamba, and keep the original pseudo-sequence setup for PINNsFormers. The weights of loss terms $[\lambda_\mathcal F,\lambda_\mathcal I,\lambda_\mathcal B]$ are set to $[1,1,10]$ for all three equations, as we find that strengthening the boundary conditions can lead to better convergence. $\lambda_\text{alig}$ is set to 1000 for convection and reaction equations, and auto-adapted by $\lambda_\mathcal F$ for wave equation.
%Loss weights are also actively adapted by neural tangent kernel~\cite{wang2022and} for wave equations for test the orthogonality of PINNMamba with other methods.
All experiments are implemented in PyTorch 2.1.1 and trained on an NVIDIA H100 GPU.  More training details are in Appendix~\ref{apx:hyperparam}. Our code and weights are available at \url{https://github.com/miniHuiHui/PINNMamba}.

\textbf{Metrics.} To evaluate the performance of the models, we take relative Mean Absolute Error (rMAE, a.k.a  $\ell_1$ relative error) and relative Root Mean Square Error (rRMSE, a.k.a $\ell_2$ relative error) following common practive~\cite{zhao2024pinnsformer,wu2024ropinn}. The metrics are formulated as:
\begin{align}
\text { rMAE }(\hat u)&=\frac{\sum_{n=1}^N\left|\hat{u}\left(x_n, t_n\right)-u\left(x_n, t_n\right)\right|}{\sum_{n=1}^{N}\left|u\left(x_n, t_n\right)\right|}, \\
\text { rRMSE }(\hat u)&=\sqrt{\frac{\sum_{n=1}^N\left|\hat{u}\left(x_n, t_n\right)-u\left(x_n, t_n\right)\right|^2}{\sum_{n=1}^N\left|u\left(x_n, t_n\right)\right|^2}},
\end{align}
where N is the number of test points, $u(x,t)$ is the ground truth solution, and $\hat u(x,t)$ is the model's prediction.

\vspace{-2mm}

\subsection{Main Results}
\vspace{-1mm}
We present the rMAE and rRMSE for approximating convection, reaction and wave equation's solution in Table~\ref{tab:diff}. Our model consistently outperforms other model architectures, achieving new state-of-the-art.
Notably, as shown in Fig.~\ref{fig:conv}, for the convection equation, PINNMamba allows sufficient propagation of information about the initial conditions, whereas on all the other models there is a varying degree of distortion in the time coordinates.
    As shown in Fig.~\ref{fig:reac}, PINNMamba can further optimize at the boundary, resulting in a lower error than KAN and PINNsFormer for reaction equations. For problems as intrinsically difficult to optimize as the wave, as in Fig.~\ref{fig:wave}, PINNMamba effectively combats simplicity bias and aligns the scales of multi-order differentiation, and thus achieves significantly higher accuracy. This illustrates that PINNMamba can be effective against PINN's failure modes. It's also worth noting that, PINNMamba has the lowest number of parameters (except KAN), while achieving consistently the best performance.

\begin{table}
\vspace{-3mm}
  \caption{Integrating PINNMamba with advanced training strategies and loss auto-balancing strategy. The rMAE is reported here.}
  
  \centering
    \small
  \begin{tabular}{lccc}

    \toprule 
    Method & Convection & Reaction & Wave\\
   \midrule
   PINNMamba & 0.0188 & 0.0094 & 0.0197\\
   +gPINN & 0.0172& 0.0123 & 0.0264 \\
   +vPINN & 0.0236 & 0.0092& 0.0169\\
   +RoPINN & 0.0102& 0.0099& 0.0121\\
    \midrule
    +NTK &0.0179& 0.0079& 0.0147\\
    +NTK+RoPINN &0.0127& 0.0072& 0.0106\\
   

    \bottomrule
  \end{tabular}
  \normalsize
  \label{tab:para}
  \vspace{-6mm}
\end{table}

\begin{figure*}[t!]
    \centering
    \includegraphics[width=\textwidth]{_fig/reac}
    \vspace{-8mm}
    \caption{The ground truth solution, prediction (top), and absolute error (bottom) on reaction equations.}
    \label{fig:reac}
    \vspace{-5mm}
  %  \vspace{-1mm}
\end{figure*}


\subsection{Combination with Other Methods}
\vspace{-1mm}
Since PINNMamba mainly focuses on model architecture, it can be integrated with other methods effortlessly. 
    We explore the feasibility and their performance in combination with advanced training paradigm, as well as loss balancing.

\textbf{Training Paradigm.} We show the rMAE of PINNMamba when integrated with advanced strategies in Table~\ref{tab:para}. We observe that gPINN~\cite{yu2022gradient} and vPINN~\cite{kharazmi2019variational} erratically deliver some performance gains on some tasks. 
    This is due to the fact that the regularization provided by gPINN and vPINN in the form of a loss function through the gradient and variational residuals has little effect on PINNMamba, since SSM itself is sufficiently regularized. RoPINN~\cite{wu2024ropinn} reduces the PINNMamba's error on convection and wave equations by about 40\%, since it complements the spatial continuity dependency.

\textbf{Neural Tangent Kernel.} Dynamic tuning of losses via Neural Tangent Kernel(NTK)~\cite{wang2022and} has been shown to have the effect of smoothing out the loss landscape. 
PINNMamba also works well with the NTK-adopted loss function. As shown in Table~\ref{tab:para}, NTK can reduce PINNMamba error by 5-25\%. 
The combination of RoPINN and NTK can further improve the overall performance of PINNMamba, which demonstrates the excellent suitability of PINNMamba with other PINN optimization methods.

\begin{figure}[t!]
    \centering
    \includegraphics[width=\linewidth]{_fig/loss_error}
    \vspace{-4mm}
    \caption{Loss and $\ell_1$-Error Curve w.r.t Training Iteration.}
    \label{fig:losserror}
    \vspace{-4mm}
  %  \vspace{-1mm}
\end{figure}
\vspace{-2mm}
\subsection{Loss-Error Consistency Analysis}
\vspace{-1mm}

Our other interest is the role of PINNMamba for the elimination of simplicity bias. Models affected by simplicity bias that fall into over-smoothing solutions will show inconsistent decreasing trends in loss and error during training. 
    As shown in Fig.~\ref{fig:losserror}, in the training process for solving convection equations, the rMAE of PINN doesn't descend as $\mathcal L_\mathcal F$ and $\mathcal L_\mathcal I$. 
        This suggests that PINN is trapped in an over-smoothing solution, which is in agreement with our observation in Fig.~\ref{fig:conv}. 
As a comparison, we find that PINNMamba's losses descent processes show a high degree of consistency with its error descent process. 
    This indicates that PINNMamba does not tend to fall into a local optimum of oversimplified patterns.
        Instead, it tends to exhibit patterns that are consistent with the original PDEs.

\vspace{-2mm}
\subsection{Ablation Study}
\vspace{-1mm}
\begin{table*}
  [t]
  \centering
  \resizebox{\textwidth}{!}{%
  \begin{tabular}{cccccccccccc}
    \toprule \multicolumn{2}{c}{Components}                                                             & \multicolumn{5}{c}{Re-executability Rate (\%)} & \multicolumn{5}{c}{Readability (\#)} \\
    \cmidrule(lr){1-2} \cmidrule(lr){3-7} \cmidrule(lr){8-12}        \hspace{8pt}\labelemoji\hspace{8pt}                                                                & \hspace{8pt}\toolemoji\hspace{8pt}                                      & O0                                 & O1             & O2             & O3             & AVG            & O0             & O1             & O2             & O3             & AVG            \\
    \hline
    \rowcolor[rgb]{0.93,0.93,0.93}\multicolumn{12}{c}{\textbf{Initialize with LLM4Decompile-End-6.7B~\citep{llm4decompile}}}   \\
    \xmark                                                                                              & \xmark                                    & 69.51                              & 46.95          & 50.61          & 46.34          & 53.35          & 3.98 & 3.41 & 3.44 & 3.38 & 3.55 \\
    \cmark                                                                                              & \xmark                                    & 75.61                              & 50.61          & 50.00          & 50.00          & 56.55          & 4.01 & 3.44 & 3.39 & \textbf{3.49} & 3.58 \\
    \xmark                                                                                              & \cmark                                    & 83.54                     & \textbf{56.10}          & 51.22          & 50.61 & 60.37 & 4.05 & 3.51 & 3.51 & 3.42 & 3.62 \\
    \cmark                                                                                              & \cmark                                    & \textbf{85.37}                            & \textbf{56.10}                     & \textbf{51.83} & \textbf{52.43}          & \textbf{61.43} & \textbf{4.13} & \textbf{3.60} & \textbf{3.54} & \textbf{3.49} & \textbf{3.69} \\

    \rowcolor[rgb]{0.93,0.93,0.93}\multicolumn{12}{c}{\textbf{Initialize with Deepseek-Coder-6.7B-base~\citep{deepseekcoder}}} \\
    \xmark                                                                                              & \xmark                                    & 59.15                              & 35.98          & 39.02          & 37.80          & 42.99          & 3.71 & 3.05 & 3.16 & 3.05 & 3.24 \\
    \cmark                                                                                              & \xmark                                    & 66.46                              & 41.46          & 38.41          & 36.59          & 45.73          & 3.76 & 3.17 & \textbf{3.21} & 3.08 & 3.31 \\
    \xmark                                                                                              & \cmark                                    & 70.73                              & 39.63          & 39.02          & 40.24          & 47.41          & 3.90 & 3.17 & 3.08 & 3.11 & 3.31 \\
    \cmark                                                                                              & \cmark                                    & \textbf{79.88}                     & \textbf{45.73} & \textbf{43.90} & \textbf{42.68} & \textbf{53.05} & \textbf{3.96} & \textbf{3.21} & 3.18 & \textbf{3.19} & \textbf{3.38} \\
    \bottomrule
  \end{tabular}%
  }
  \caption{The ablation study of different methods across four optimization levels
  (O0, O1, O2, O3), as well as their average scores (AVG). The results in bold represent the optimal performance. The ~\labelemoji~ and ~\toolemoji~ means Relabedling and Function Call. \textbf{Bold} denotes the best performance.}
  \label{tab:ablation}
\end{table*}

To verify the validity of the various components of the PINNMamba, as shown in Table~\ref{tab:ablation}, we evaluate the performance of models subtracting these components from PINNMamba.

\textbf{Sub-Sequence.} We remove the sub-sequence alignment, which leads to a decrease in model performance, indicating the significance of the agreement formed through alignment in eliminating simplicity bias.
After replacing the sub-sequence with a long sequence of the entire domain, the model shows failure modes, in line with the sequence granularity analysis in Section~\ref{sec:subseq}.

\textbf{Time-Varying SSM.} We replace the selective SSM~\cite{gu2023mamba} with a linear time-invariant structure SSM~\cite{gu2022efficiently}, and there is some decrease in model performance, illustrating the role of predictive diversity in eliminating simplicity bias. 
And when we remove SSM completely and switch to MLP instead, the model has severe failure modes. 
        This demonstrates that SSM's adaptation for \textit{Continuous-Discrete Mismatch} allows the initial condition information to propagate sufficiently in time coordinates.

In addition, we also conducted a sensitivity analysis of the choice of sub-sequence length, activation. See Appendix~\ref{apx:sense}.

\vspace{-3mm}
\subsection{Experiments on Complex Problems}
\vspace{-1mm}
To further demonstrate the generalization of our method, we tested our model on partial PINNacle Benchmark~\cite{hao2023pinnacle} and Navier-Stokes equations. As shown in Fig.~\ref{fig:ns}, PINNMamba achieves the lowest error on the N-S equation. Just like PINNsFormer, PINNMamba also gets out-of-memory on some problems in PINNacle, which we identify as a major limitation of sequence-based methods. We discuss the details of PINNacle experiments in Appendix~\ref{apx:comp}.

\begin{figure}[t!]
    \centering
    \includegraphics[width=\linewidth]{_fig/NS}
    \vspace{-6mm}
    \caption{Absolute Error of pressure prediction of N-S equation}
    \label{fig:ns}
    \vspace{-3mm}
  %  \vspace{-1mm}
\end{figure}

\section{Limitations and Future Work}
The proposed OpenFly platform incorporates various rendering engines/techniques to provide high-quality scenes. Specifically, this is the first attempt to use 3D GS reconstructed scenes to support real-to-sim training and testing, while in the reconstruction of large-scale areas, a few visual artifacts are inevitably present. Future work will focus on exploring more effective reconstruction methods to enhance realism in large-scale scenes. Besides, the proposed OpenFly-Agent is built upon the large VLN model architecture, which is not practical for real-time deployment on UAVs. To address this, future research should focus on developing more efficient architectures and effective quantization techniques. 


\section{Conclusion}
In this work, we present OpenFly, a platform designed for large-scale data collection in aerial Vision-and-Language Navigation (VLN). OpenFly integrates multiple rendering engines and advanced real-to-sim techniques for data generation, enabling efficient collection of diverse, high-quality aerial VLN data. The resulting large-scale dataset comprises 100k trajectories across 18 distinct scenes, spanning a wide range of altitudes and difficulty levels, which is significantly superior than existing ones. Furthermore, we propose OpenFly-Agent, a keyframe-aware aerial navigation model capable of directly predicting flight actions based on observations and language instructions. Extensive experiments validate the effectiveness of the proposed method, and establishing a comprehensive benchmark for future advancements in aerial navigation. 
%The toolchain, dataset, and code will be publicly released, providing a valuable resource for future research in this field.
{
    \small
    \bibliographystyle{ieeenat_fullname}
    \bibliography{main}
}

% WARNING: do not forget to delete the supplementary pages from your submission 

\section{Metric}
\label{sec:metric}

\textbf{Mean Squared Error (MSE)} Mean Squared Error (MSE) is a common statistical metric used to assess the difference between predicted and actual values. The formula is:
\begin{equation}
    MSE = \frac{1}{n} \sum_{i=1}^{n} (y_i - \hat{y}_i)^2
\end{equation}
where $ n $ is the number of samples, $ y_i $ is the actual value, and $ \hat{y}_i $ is the predicted value.

\textbf{Relative L2 Error} Relative L2 error measures the relative difference between predicted and actual values, commonly used in time series prediction. The formula is:
\begin{equation}
    \text{Relative L2 Error} = \frac{\| Y_{\text{pred}} - Y_{\text{true}} \|_2}{\| Y_{\text{true}} \|_2}
\end{equation}
where $ Y_{\text{pred}} $ is the predicted value and $ Y_{\text{true}} $ is the actual value.

\textbf{Structural Similarity Index Measure (SSIM)} The Structural Similarity Index (SSIM) measures the similarity between two images in terms of luminance, contrast, and structure. The formula is:
\begin{equation}
    SSIM(x, y) = \frac{(2\mu_x \mu_y + C_1)(2\sigma_{xy} + C_2)}{(\mu_x^2 + \mu_y^2 + C_1)(\sigma_x^2 + \sigma_y^2 + C_2)}
\end{equation}
where $ \mu_x $ and $ \mu_y $ are the mean values, $ \sigma_x $ and $ \sigma_y $ are the standard deviations, $ \sigma_{xy} $ is the covariance.

\section{Related Work}
\subsection{Deep Learning based Weather Forecasting}
\textbf{Global Weather Forecasting.} Global weather forecasting has seen significant progress with deep learning models. FourCastNet, based on Fourier neural operators, provides global forecasts comparable to traditional numerical methods like IFS, but at much higher speeds~\cite{pathak2022fourcastnet}. Pangu, utilizing the Swin Transformer, exceeds NWP methods, incorporating earth-specific location embeddings for better performance~\cite{bi2023accurate}. The Spherical Fourier Neural Operator (SFNO) extends Fourier methods using spherical harmonics, offering more stable long-term predictions~\cite{bonev2023spherical}. FuXi focuses on long-term forecasting, achieving a 15-day forecasts comparable to ECMWF~\cite{chen2023fuxi}. GraphCast leverages message-passing networks to improve efficiency and forecasting accuracy~\cite{lam2023learning}, and GenCast builds on this to enhance ensemble forecasting~\cite{price2023gencast}. Further, diffusion models like those in~\cite{li2024generative} generate probabilistic ensembles by sampling, while NeuralGCM~\cite{kochkov2024neural} focuses on atmospheric circulation with a dynamic core, offering climate simulation capabilities but at higher training and inference costs. 

\textbf{Regional Weather Forecasting.} The goal of regional weather forecasting is to enhance local prediction accuracy with high-resolution models. CorrDiff~\cite{mardani2023generative} combines U-Net and diffusion models to improve local forecasts. MetaWeather~\cite{kim2024metaweather} adapts global forecasts to regional contexts using meta-learning. GNNs are also widely applied in regional forecasting, with Graphcast~\cite{lam2023learning} enhancing accuracy by modeling complex spatial dependencies. MetNet-3~\cite{espeholt2022deep} offers high-accuracy forecasts for weather variables, such as precipitation, temperature, and wind speed, at 2-minute intervals and 1–4 km resolution, outperforming traditional models like HRRR. NowcastNet~\cite{zhang2023skilful} and DGMR~\cite{ravuri2021skilful} excel in short-term extreme precipitation forecasts using deep generative models and radar data. In spatiotemporal prediction, NMO~\cite{wu2024neural} models the evolution of physical dynamics, providing new insights for local weather forecasting. Similarly, SimVP~\cite{gao2022simvp} and PastNet~\cite{wu2024pastnet} achieve good results in forecasting local precipitation evolution using spatiotemporal convolution methods.
    
% Despite these advances, none of these methods effectively address the challenge of balancing global and regional high-resolution forecasts or capturing the fine-grained, dynamic interactions important for extreme event prediction.
    
\subsection{Numerical analysis methods}
Multigrid methods~\cite{mccormick1987multigrid,wesseling1995introduction,hackbusch2013multi,bramble2019multigrid,hiptmair1998multigrid,brandt1983multigrid,borzi2009multigrid} and nested grid strategies~\cite{miyakoda1977one,zhang2012nested,sullivan1996grid} are widely used to solve PDEs and handle multi-scale problems~\cite{debreu2008two,xue2000advanced}. Multigrid methods use grids of different resolutions to transfer information and accelerate iterations. They efficiently solve large-scale problems and improve computational accuracy. By eliminating low-frequency errors on coarse grids and high-frequency errors on fine grids, multigrid methods effectively handle error convergence at different scales~\cite{he2019mgnet,he2023mgno,shao2022fast}. Nested grid strategies embed higher-resolution fine grids into regions of interest based on a global coarse grid to capture local complex physical phenomena in detail. In weather forecasting, this method provides large-scale background fields on a global scale while refining the grid for target regions to accurately simulate the evolution of local weather systems and the occurrence of extreme events~\cite{bacon2000dynamically}. 

% Our proposed neural nested grid method helps address challenges like boundary information loss in regional forecasting and multi-scale feature capture.

\section{Additional Results}
%
We present more additional results in Figure \ref{fig_0.25-day}, \ref{fig_0.5-day}, \ref{fig_1.0-day} \ref{fig_1.5-day}, \ref{fig_2.0-day}, \ref{fig_2.5-day}, \ref{fig_3.0-day}, \ref{fig_3.5-day}, \ref{fig_4.0-day}, \ref{fig_4.5-day}, \ref{fig_5.0-day}, \ref{fig_5.5-day}, \ref{fig_6.0-day}, \ref{fig_6.5-day}, \ref{fig_7.0-day}, \ref{fig_7.5-day},
\ref{fig_8.0-day}, \ref{fig_8.5-day}, \ref{fig_9.0-day}, \ref{fig_9.5-day},
\ref{fig_10.0-day}, including 18 variables that are importmant to weather forecasting, each with results ranging from 6 hours to 10 days. These additional results further demonstrate the effectiveness of OneForecast. Same as the Figure \ref{fig:visual_results}
, the initial conditions is 00:00 UTC, 1 January 2020.


\begin{figure*}[h]
\centering
\includegraphics[width=1\linewidth]{figures/fig_0.25-day.jpg}
\vspace{-20pt}
\caption{6-hour forecast results of different models.}
\label{fig_0.25-day}
\end{figure*}

\begin{figure*}[h]
\centering
\includegraphics[width=1\linewidth]{figures/fig_0.5-day.jpg}
\vspace{-20pt}
\caption{0.5-day forecast results of different models.}
\label{fig_0.5-day}
\end{figure*}

\begin{figure*}[h]
\centering
\includegraphics[width=1\linewidth]{figures/fig_1.0-day.jpg}
\vspace{-20pt}
\caption{1-day forecast results of different models.}
\label{fig_1.0-day}
\end{figure*}

\begin{figure*}[h]
\centering
\includegraphics[width=1\linewidth]{figures/fig_1.5-day.jpg}
\vspace{-20pt}
\caption{1.5-day forecast results of different models.}
\label{fig_1.5-day}
\end{figure*}

\begin{figure*}[h]
\centering
\includegraphics[width=1\linewidth]{figures/fig_2.0-day.jpg}
\vspace{-20pt}
\caption{2-day forecast results of different models.}
\label{fig_2.0-day}
\end{figure*}


\begin{figure*}[h]
\centering
\includegraphics[width=1\linewidth]{figures/fig_2.5-day.jpg}
\vspace{-20pt}
\caption{2.5-day forecast results of different models.}
\label{fig_2.5-day}
\end{figure*}

\begin{figure*}[h]
\centering
\includegraphics[width=1\linewidth]{figures/fig_3.0-day.jpg}
\vspace{-20pt}
\caption{3-day forecast results of different models.}
\label{fig_3.0-day}
\end{figure*}

\begin{figure*}[h]
\centering
\includegraphics[width=1\linewidth]{figures/fig_3.5-day.jpg}
\vspace{-20pt}
\caption{3.5-day forecast results of different models.}
\label{fig_3.5-day}
\end{figure*}

\begin{figure*}[h]
\centering
\includegraphics[width=1\linewidth]{figures/fig_4.0-day.jpg}
\vspace{-20pt}
\caption{4-day forecast results of different models.}
\label{fig_4.0-day}
\end{figure*}

\begin{figure*}[h]
\centering
\includegraphics[width=1\linewidth]{figures/fig_4.5-day.jpg}
\vspace{-20pt}
\caption{4.5-day forecast results of different models.}
\label{fig_4.5-day}
\end{figure*}


\begin{figure*}[h]
\centering
\includegraphics[width=1\linewidth]{figures/fig_5.0-day.jpg}
\vspace{-20pt}
\caption{5.0-day forecast results of different models.}
\label{fig_5.0-day}
\end{figure*}

\begin{figure*}[h]
\centering
\includegraphics[width=1\linewidth]{figures/fig_5.5-day.jpg}
\vspace{-20pt}
\caption{5.5-day forecast results of different models.}
\label{fig_5.5-day}
\end{figure*}

\begin{figure*}[h]
\centering
\includegraphics[width=1\linewidth]{figures/fig_6.0-day.jpg}
\vspace{-20pt}
\caption{6.0-day forecast results of different models.}
\label{fig_6.0-day}
\end{figure*}

\begin{figure*}[h]
\centering
\includegraphics[width=1\linewidth]{figures/fig_6.5-day.jpg}
\vspace{-20pt}
\caption{6.5-day forecast results of different models.}
\label{fig_6.5-day}
\end{figure*}

\begin{figure*}[h]
\centering
\includegraphics[width=1\linewidth]{figures/fig_7.0-day.jpg}
\vspace{-20pt}
\caption{7.0-day forecast results of different models.}
\label{fig_7.0-day}
\end{figure*}

\begin{figure*}[h]
\centering
\includegraphics[width=1\linewidth]{figures/fig_7.5-day.jpg}
\vspace{-20pt}
\caption{7.5-day forecast results of different models.}
\label{fig_7.5-day}
\end{figure*}

\begin{figure*}[h]
\centering
\includegraphics[width=1\linewidth]{figures/fig_8.0-day.jpg}
\vspace{-20pt}
\caption{8.0-day forecast results of different models.}
\label{fig_8.0-day}
\end{figure*}

\begin{figure*}[h]
\centering
\includegraphics[width=1\linewidth]{figures/fig_8.5-day.jpg}
\vspace{-20pt}
\caption{8.5-day forecast results of different models.}
\label{fig_8.5-day}
\end{figure*}

\begin{figure*}[h]
\centering
\includegraphics[width=1\linewidth]{figures/fig_9.0-day.jpg}
\vspace{-20pt}
\caption{9.0-day forecast results of different models.}
\label{fig_9.0-day}
\end{figure*}

\begin{figure*}[h]
\centering
\includegraphics[width=1\linewidth]{figures/fig_9.5-day.jpg}
\vspace{-20pt}
\caption{9.5-day forecast results of different models.}
\label{fig_9.5-day}
\end{figure*}

\begin{figure*}[h]
\centering
\includegraphics[width=1\linewidth]{figures/fig_10.0-day.jpg}
\vspace{-20pt}
\caption{10.0-day forecast results of different models.}
\label{fig_10.0-day}
\end{figure*}


\section{Detailed Mathematical Proof}
\label{sec:proof}
\textbf{Proof of Theorem 1}

Now we have N augmented data and we need to select the best from them. We consider both the quality and the diversity of these data and get the sampling strategy from an optimization problem.

We model the sampling strategy as a multinomial distribution supported on all the augmented data $S = \{\mathbf{X}_j\}_{j=1}^N$, which means that the sampling strategy $\pi=(\pi_1,...,\pi_N)^\top$ is the corresponding probabilities of selecting $\mathbf{X}_1,...,\mathbf{X}_N$, then we can model the expectation of the similarity as:
$$\begin{aligned}
 & \mathbb{E}_{Y_x,Y_{x^{\prime}}\in\mathcal{C}}\{g(x,x^{\prime})\mid S\} \\
 & =\quad\int g(\mathbf{x},\mathbf{x}^{\prime})\boldsymbol{\pi}(\mathbf{x})\mathrm{Pr}_{S}(Y_{x}\in\mathcal{C}\mid\boldsymbol{x}=\mathbf{x})\boldsymbol{\pi}(\mathbf{x}^{\prime})\mathrm{Pr}_{S}(Y_{x}\in\mathcal{C}\mid\boldsymbol{x}=\mathbf{x}^{\prime})d\mathbf{x}d\mathbf{x}^{\prime} \\
 & =\quad\sum_{i,j=1}^Ng(\mathbf{X}_i,\mathbf{X}_j)\pi_i\pi_j\mathrm{Pr}_{S}(Y_x\in\mathcal{C}\mid\boldsymbol{x}=\mathbf{X}_i)\mathrm{Pr}_{S}(Y_x\in\mathcal{C}\mid\boldsymbol{x}=\mathbf{X}_j),
\end{aligned}$$
where the set $\mathcal{C}$ denotes the criterion of selection we are using, the function $g$ can be chosen as any similarity metric function and $x$ means a random variable.

The core to solving the above optimization problem is to use predictive inference to approximate the conditional probability of $\{Y_x\in\mathcal{C}\}$ given $x = \mathbf{X}$
Let $\mu ( \mathbf{x} ) : = \mathbb{E} ( Y\mid \mathbf{X} = \mathbf{x} )$ be the oracle associated with $( \mathbf{X} , Y) .$ Denote $\theta_j=\mathbb{I}\{Y_j\in\mathcal{C}\}$. As the augmented data
$\mathbf{X}_1,...,\mathbf{X}_N$ are independently identically distributed, $\theta_1,...,\theta_N$ can be regarded as independent Bernoulli($q)$ variables with $q=\Pr(Y_j\in\mathcal{C}).$ The probability distribution of the predicted result $W_j$ for $j=1,...,N$ is
$$\Pr(W_j\mid\theta_j)=(1-\theta_j)f_0+\theta_jf_1,\quad$$
where $f_0$ and $f_1$ are the conditional distributions of $W_j$ on $Y_j \in \mathcal{C}$ or not.

Denote $T(w) = \frac{(1-q)f_0(W_j)}{f(W_j)}$, we can rewrite the expectation of the similarity as
$$\mathbb{E}_{Y_x,Y_{x^{\prime}}\in\mathcal{C}}\{g(x,x^{\prime})|S\}=\sum_{i,j=1}^Ng(\mathbf{X}_i,\mathbf{X}_j)\pi_i\pi_j(1-T_i)(1-T_j)=\boldsymbol{\pi}^\top A_\mathbb{T}\boldsymbol{\pi},$$

Next, we use the expectation to control the quality of the data.
$$\mathbb{E}\{\mathbb{I}(Y_x\not\in\mathcal{C})\mid S\}=\sum_{i=1}^N\Pr(Y_i\not\in\mathcal{C}\mid\mathbf{X}_i)\pi_i=\sum_{i=1}^N\pi_iT_i\leq\alpha,$$

In all, the optimization problem can be modeled as 
\begin{align}
    & \arg\min_{\boldsymbol{\pi}}\quad h(\boldsymbol{\pi},\mathbb{T}):=\boldsymbol{\pi}^\top A_\mathbb{T}\boldsymbol{\pi}, \\
    & \text{subject to} \quad
        \begin{cases}
            \sum_{i = 1}^N\pi_iT_i\leq\alpha, \\
            \sum_{i = 1}^N\pi_i = 1, \\
            0\leq\pi_i\leq m^{-1}, \quad 1\leq i\leq N.
        \end{cases}
\end{align}

where $m$ is used to control the maximum selection.

The best selection of K is determined by the strategy $\pi$ which serves as the solution to the above optimization problem.

\section{Additional Experiments}
\label{sec:more_experiments}
\subsection{Long-term forecasting experiment expansion}

In the long-term forecasting experiments, we compare the performance of different backbone models on the SWE benchmark, evaluating the relative L2 error for three variables (U, V, and H). Our setup inputs 5 frames and predicts 50 frames. For the SimVP-v2 model, using \method{} reduces the relative L2 error for SWE (u) from 0.0187 to 0.0154, SWE (v) from 0.0387 to 0.0342, and SWE (h) from 0.0443 to 0.0397. We visualize SWE (h) in 3D as shown in Figure~\ref{fig:case} [\textcolor{red}{I}]. For the ConvLSTM model, applying \method{} reduces the relative L2 error for SWE (u) from 0.0487 to 0.0321, SWE (v) from 0.0673 to 0.0351, and SWE (h) from 0.0762 to 0.0432. For the FNO model, using \method{} reduces the relative L2 error for SWE (u) from 0.0571 to 0.0502, SWE (v) from 0.0832 to 0.0653, and SWE (h) from 0.0981 to 0.0911. Overall, \method{} significantly improves the long-term forecasting accuracy of different backbone models.

\begin{figure*}[h]
    \centering
    \includegraphics[width=\textwidth]{image/casestudy.pdf}
    \caption{
    \textcolor{red}{I.} 3D visualization of the SWE(h), showing Ground-truth, SimVP-V2+BeamVQ predictions, and Error at T=1, 10, 20, 30, 40, 50. The first row shows Ground-truth, the second SimVP-V2+BeamVQ predictions, and the third Error. \textcolor{red}{II.} A case study. Building fire simulation with ventilation settings added to Wu's Prometheus~\cite{wu2024prometheus}. (a) Layout and HRR growth. (b) Comparison of physical metrics for different methods. (c) Ground-truth, ResNet+BeamVQ, and ResNet predictions.
    }
    \label{fig:case} 
\end{figure*}


\subsection{Experiment Statistical Significance}
\label{sec:significance}
To measure the statistical significance of our main experiment results, we choose three backbones to train on two datasets to run 5 times. 
Table~\ref{tab:significance} records the average and standard deviation of the test MSE loss.
The results prove that our method is statistically significant to outperform the baselines
because our confidence interval is always upper than the confidence interval of the baselines. 
Due to limited computation resources, we do not cover all ten backbones and five datasets, 
but we believe these results have shown that our method has consistent advantages.


\begin{table}[h]
\label{tab:significance}
\centering
\begin{scriptsize}
    \begin{sc}
    \caption{ The average and standard deviation of MSE in 5 runs}
    \label{tab:significance}
    \centering
        \renewcommand{\multirowsetup}{\centering}
        \setlength{\tabcolsep}{10pt}
        \begin{tabular}{l|cc|cc}
            \toprule
            
            \multirow{4}{*}{Model} & \multicolumn{4}{c}{Benchmarks}  \\
            \cmidrule(lr){2-5}
            & \multicolumn{2}{c}{NSE} &   \multicolumn{2}{c}{SEVIR}   \\
            \cmidrule(lr){2-5}
           & Ori & + BeamVQ & Ori & + BeamVQ  \\
            \midrule
            ConvLSTM &0.4092$\pm$0.0002 &\textbf{0.1277$\pm$0.0001}  & 0.1762 0.0007  & \textbf{0.1279$\pm$0.0009}  \\
            FNO &  0.2227$\pm$0.0003 &\textbf{0.1007 $\pm$0.0002}& 0.0787$\pm$0.0012 & \textbf{ 0.0437$\pm$0.0013} \\
            CNO & 0.2192 $\pm$0.0008 &\textbf{ 0.1492$\pm$0.0011}& 0.0057$\pm$0.0005 & \textbf{ 0.0053$\pm$0.0006} \\
            \bottomrule
        \end{tabular}
    \end{sc}

\end{scriptsize}
\end{table}


\end{document}
