% CVPR 2025 Paper Template; see https://github.com/cvpr-org/author-kit

\documentclass[10pt,twocolumn,letterpaper]{article}

%%%%%%%%% PAPER TYPE  - PLEASE UPDATE FOR FINAL VERSION
\usepackage{cvpr}              % To produce the CAMERA-READY version
% \usepackage[review]{cvpr}      % To produce the REVIEW version
% \usepackage[pagenumbers]{cvpr} % To force page numbers, e.g. for an arXiv version

% Import additional packages in the preamble file, before hyperref
% %
% --- inline annotations
%
\newcommand{\red}[1]{{\color{red}#1}}
\newcommand{\todo}[1]{{\color{red}#1}}
\newcommand{\TODO}[1]{\textbf{\color{red}[TODO: #1]}}
% --- disable by uncommenting  
% \renewcommand{\TODO}[1]{}
% \renewcommand{\todo}[1]{#1}



\newcommand{\VLM}{LVLM\xspace} 
\newcommand{\ours}{PeKit\xspace}
\newcommand{\yollava}{Yo’LLaVA\xspace}

\newcommand{\thisismy}{This-Is-My-Img\xspace}
\newcommand{\myparagraph}[1]{\noindent\textbf{#1}}
\newcommand{\vdoro}[1]{{\color[rgb]{0.4, 0.18, 0.78} {[V] #1}}}
% --- disable by uncommenting  
% \renewcommand{\TODO}[1]{}
% \renewcommand{\todo}[1]{#1}
\usepackage{slashbox}
% Vectors
\newcommand{\bB}{\mathcal{B}}
\newcommand{\bw}{\mathbf{w}}
\newcommand{\bs}{\mathbf{s}}
\newcommand{\bo}{\mathbf{o}}
\newcommand{\bn}{\mathbf{n}}
\newcommand{\bc}{\mathbf{c}}
\newcommand{\bp}{\mathbf{p}}
\newcommand{\bS}{\mathbf{S}}
\newcommand{\bk}{\mathbf{k}}
\newcommand{\bmu}{\boldsymbol{\mu}}
\newcommand{\bx}{\mathbf{x}}
\newcommand{\bg}{\mathbf{g}}
\newcommand{\be}{\mathbf{e}}
\newcommand{\bX}{\mathbf{X}}
\newcommand{\by}{\mathbf{y}}
\newcommand{\bv}{\mathbf{v}}
\newcommand{\bz}{\mathbf{z}}
\newcommand{\bq}{\mathbf{q}}
\newcommand{\bff}{\mathbf{f}}
\newcommand{\bu}{\mathbf{u}}
\newcommand{\bh}{\mathbf{h}}
\newcommand{\bb}{\mathbf{b}}

\newcommand{\rone}{\textcolor{green}{R1}}
\newcommand{\rtwo}{\textcolor{orange}{R2}}
\newcommand{\rthree}{\textcolor{red}{R3}}
\usepackage{amsmath}
%\usepackage{arydshln}
\DeclareMathOperator{\similarity}{sim}
\DeclareMathOperator{\AvgPool}{AvgPool}

\newcommand{\argmax}{\mathop{\mathrm{argmax}}}     



% It is strongly recommended to use hyperref, especially for the review version.
% hyperref with option pagebackref eases the reviewers' job.
% Please disable hyperref *only* if you encounter grave issues, 
% e.g. with the file validation for the camera-ready version.
%
% If you comment hyperref and then uncomment it, you should delete *.aux before re-running LaTeX.
% (Or just hit 'q' on the first LaTeX run, let it finish, and you should be clear).
\usepackage{amsmath}
\usepackage{amssymb}
\usepackage{mathtools}
\usepackage{amsthm}

\usepackage{color}
\usepackage{colortbl}
\usepackage{tabularx}
\usepackage{tcolorbox}
\usepackage{courier}
\usepackage{graphbox}
\usepackage{booktabs}
\usepackage{amsmath,amsfonts}
\usepackage{graphicx} 
\usepackage{subcaption}
\usepackage{tabularray}
% \usepackage{caption}
\usepackage{url}
\usepackage{etoolbox}
\usepackage{wasysym}
\usepackage[export]{adjustbox}
\usepackage{tikz}
\usepackage{bm}
% \usepackage{cite}
\usepackage{academicons}
\usepackage{multirow}
\usepackage{multicol}
\usepackage{algorithm}
\usepackage{algorithmic}
\usepackage{graphicx}  

\definecolor{cvprblue}{rgb}{0.21,0.49,0.74}
\usepackage[pagebackref,breaklinks,colorlinks,allcolors=cvprblue]{hyperref}

\title{\LARGE \bf
TUMTraffic-VideoQA: A Benchmark for Unified Spatio-Temporal Video Understanding in Traffic Scenes
}

% Spatial-temporal grounding. 

% and Spatial-Temporal Object Grounding

% <-this % stops a space
% \thanks{This research is accomplished within the project ”AUTOtech.agil” (Grant Number 01IS22088A). We acknowledge the financial support for the project by the Federal Ministry of Education and Research of Germany (BMBF). (Corresponding author: Xingcheng Zhou)}
% \thanks{ The authors are with the School of Computation, Information and Technology, Technical University of Munich, 85748 Garching, Germany
% }
% \thanks{ $^\star$ Corresponding Author: \texttt{xingcheng.zhou@tum.de}}

%%%%%%%%% TITLE - PLEASE UPDATE
% \title{\LaTeX\ Author Guidelines for \confName~Proceedings}

%%%%%%%%% AUTHORS - PLEASE UPDATE
\author{
Xingcheng Zhou$^{\star}$,   \hspace{0.3em}
Konstantinos Larintzakis,   \hspace{0.3em}
Hao Guo,           \hspace{0.3em}
Walter Zimmer,  \hspace{0.3em}
Mingyu Liu,  \hspace{0.3em}
Hu Cao, \\
Jiajie Zhang,  \hspace{0.1em}
Venkatnarayanan Lakshminarasimhan,  \hspace{0.1em}
Leah Strand, \hspace{0.1em}
Alois C. Knoll \\
% For a paper whose authors are all at the same institution,
% omit the following lines up until the closing ``}''.
% Additional authors and addresses can be added with ``\and'',
% just like the second author.
% To save space, use either the email address or home page, not both
% \and
% Second Author\\
% Institution2\\
% First line of institution2 address\\
{\small  $^\star$ \texttt{Corresponding: xingcheng.zhou@tum.de}}
}

\makeatletter

\let\@oldmaketitle\@maketitle
\renewcommand{\@maketitle}{\@oldmaketitle
  \centering
  \url{http://traffix-videoqa.github.io}\\[8pt]
  \setcounter{figure}{0}
  \begin{center}
  % \includegraphics[width=\textwidth,height=9cm,keepaspectratio]{figure/first_figure-1st_image.pdf}
  \includegraphics[width=\textwidth,height=9cm,keepaspectratio]{figure/first_figure-1st_image.jpg}
    \captionof{figure}{TUMTraffic-VideoQA introduces a comprehensive benchmark for video-level traffic scene understanding. Our baseline model, TraffiX-Qwen, is capable of solving multiple tasks, including video QA, spatio-temporal grounding, and referred object captioning, within a unified model. In our approach, the spatio-temporal location of objects is represented as tuples \( (c, fn, x, y) \), where \( c \) serves as a unique object identifier, \( fn \) denotes the normalized frame timestamp, and \( (x, y) \) denote the center of the object in the image, normalized with respect to the image dimensions.
}
    \label{fig:title_figure}
  \end{center}

  }  
\makeatother

\begin{document}
\maketitle
\begin{abstract}


The choice of representation for geographic location significantly impacts the accuracy of models for a broad range of geospatial tasks, including fine-grained species classification, population density estimation, and biome classification. Recent works like SatCLIP and GeoCLIP learn such representations by contrastively aligning geolocation with co-located images. While these methods work exceptionally well, in this paper, we posit that the current training strategies fail to fully capture the important visual features. We provide an information theoretic perspective on why the resulting embeddings from these methods discard crucial visual information that is important for many downstream tasks. To solve this problem, we propose a novel retrieval-augmented strategy called RANGE. We build our method on the intuition that the visual features of a location can be estimated by combining the visual features from multiple similar-looking locations. We evaluate our method across a wide variety of tasks. Our results show that RANGE outperforms the existing state-of-the-art models with significant margins in most tasks. We show gains of up to 13.1\% on classification tasks and 0.145 $R^2$ on regression tasks. All our code and models will be made available at: \href{https://github.com/mvrl/RANGE}{https://github.com/mvrl/RANGE}.

\end{abstract}


\section{Introduction}
Backdoor attacks pose a concealed yet profound security risk to machine learning (ML) models, for which the adversaries can inject a stealth backdoor into the model during training, enabling them to illicitly control the model's output upon encountering predefined inputs. These attacks can even occur without the knowledge of developers or end-users, thereby undermining the trust in ML systems. As ML becomes more deeply embedded in critical sectors like finance, healthcare, and autonomous driving \citep{he2016deep, liu2020computing, tournier2019mrtrix3, adjabi2020past}, the potential damage from backdoor attacks grows, underscoring the emergency for developing robust defense mechanisms against backdoor attacks.

To address the threat of backdoor attacks, researchers have developed a variety of strategies \cite{liu2018fine,wu2021adversarial,wang2019neural,zeng2022adversarial,zhu2023neural,Zhu_2023_ICCV, wei2024shared,wei2024d3}, aimed at purifying backdoors within victim models. These methods are designed to integrate with current deployment workflows seamlessly and have demonstrated significant success in mitigating the effects of backdoor triggers \cite{wubackdoorbench, wu2023defenses, wu2024backdoorbench,dunnett2024countering}.  However, most state-of-the-art (SOTA) backdoor purification methods operate under the assumption that a small clean dataset, often referred to as \textbf{auxiliary dataset}, is available for purification. Such an assumption poses practical challenges, especially in scenarios where data is scarce. To tackle this challenge, efforts have been made to reduce the size of the required auxiliary dataset~\cite{chai2022oneshot,li2023reconstructive, Zhu_2023_ICCV} and even explore dataset-free purification techniques~\cite{zheng2022data,hong2023revisiting,lin2024fusing}. Although these approaches offer some improvements, recent evaluations \cite{dunnett2024countering, wu2024backdoorbench} continue to highlight the importance of sufficient auxiliary data for achieving robust defenses against backdoor attacks.

While significant progress has been made in reducing the size of auxiliary datasets, an equally critical yet underexplored question remains: \emph{how does the nature of the auxiliary dataset affect purification effectiveness?} In  real-world  applications, auxiliary datasets can vary widely, encompassing in-distribution data, synthetic data, or external data from different sources. Understanding how each type of auxiliary dataset influences the purification effectiveness is vital for selecting or constructing the most suitable auxiliary dataset and the corresponding technique. For instance, when multiple datasets are available, understanding how different datasets contribute to purification can guide defenders in selecting or crafting the most appropriate dataset. Conversely, when only limited auxiliary data is accessible, knowing which purification technique works best under those constraints is critical. Therefore, there is an urgent need for a thorough investigation into the impact of auxiliary datasets on purification effectiveness to guide defenders in  enhancing the security of ML systems. 

In this paper, we systematically investigate the critical role of auxiliary datasets in backdoor purification, aiming to bridge the gap between idealized and practical purification scenarios.  Specifically, we first construct a diverse set of auxiliary datasets to emulate real-world conditions, as summarized in Table~\ref{overall}. These datasets include in-distribution data, synthetic data, and external data from other sources. Through an evaluation of SOTA backdoor purification methods across these datasets, we uncover several critical insights: \textbf{1)} In-distribution datasets, particularly those carefully filtered from the original training data of the victim model, effectively preserve the model’s utility for its intended tasks but may fall short in eliminating backdoors. \textbf{2)} Incorporating OOD datasets can help the model forget backdoors but also bring the risk of forgetting critical learned knowledge, significantly degrading its overall performance. Building on these findings, we propose Guided Input Calibration (GIC), a novel technique that enhances backdoor purification by adaptively transforming auxiliary data to better align with the victim model’s learned representations. By leveraging the victim model itself to guide this transformation, GIC optimizes the purification process, striking a balance between preserving model utility and mitigating backdoor threats. Extensive experiments demonstrate that GIC significantly improves the effectiveness of backdoor purification across diverse auxiliary datasets, providing a practical and robust defense solution.

Our main contributions are threefold:
\textbf{1) Impact analysis of auxiliary datasets:} We take the \textbf{first step}  in systematically investigating how different types of auxiliary datasets influence backdoor purification effectiveness. Our findings provide novel insights and serve as a foundation for future research on optimizing dataset selection and construction for enhanced backdoor defense.
%
\textbf{2) Compilation and evaluation of diverse auxiliary datasets:}  We have compiled and rigorously evaluated a diverse set of auxiliary datasets using SOTA purification methods, making our datasets and code publicly available to facilitate and support future research on practical backdoor defense strategies.
%
\textbf{3) Introduction of GIC:} We introduce GIC, the \textbf{first} dedicated solution designed to align auxiliary datasets with the model’s learned representations, significantly enhancing backdoor mitigation across various dataset types. Our approach sets a new benchmark for practical and effective backdoor defense.



\section{Related Work}

\textbf{Hallucinations in LLMs.}
\
Hallucinations occur when the generated content from LLMs seems believable but does not match factual or contextual knowledge \citep{ji-survey, rawte2023surveyhallucinationlargefoundation, hit-survey}.
% Recent studies \citep{lin2024flame, kang2024unfamiliarfinetuningexamplescontrol, gekhman-etal-2024-fine} attempt to analyze the causes of hallucinations in LLMs.
% \citet{lin2024flame} conducts a pilot study and finds that tuning LLMs on data containing unseen knowledge can encourage models to be overconfident, leading to hallucinations.
Recent studies \citep{lin2024flame, kang2024unfamiliarfinetuningexamplescontrol, gekhman-etal-2024-fine} attempt to analyze the causes of hallucinations in LLMs and find that tuning LLMs on data containing unseen knowledge can encourage models to be overconfident, leading to hallucinations.
Therefore, recent studies \citep{lin2024flame, zhang-etal-2024-self, tian2024finetuning} attempt to apply RL-based methods to teach LLMs to hallucinate less after the instruction tuning stage.
However, these methods are inefficient because they require additional corpus and API costs for advanced LLMs.
Even worse, such RL-based methods can weaken the instruction-following ability of LLMs \citep{lin2024flame}.
In this paper, instead of introducing the inefficient RL stage, we attempt to directly filter out the unfamiliar data during the instruction tuning stage, aligning LLMs to follow instructions and hallucinate less.






\noindent
\textbf{Data Filtering for Instruction Tuning.}
\
Data are crucial for training neural networks. \citep{van2020survey, song2022learningnoisylabelsdeep, si-etal-2022-scl, si-etal-2023-santa, zhao2024ultraedit, an2024threadlogicbaseddataorganization, si-etal-2024-improving, cai-etal-2024-unipcm}.
According to \citet{zhou2023lima}, data quality is more important than data quantity in instruction tuning.
Therefore, many works attempt to select high-quality instruction samples to improve the LLMs’ instruction-following abilities.
\citet{chen2023alpagasus, liu2024what} utilize the feedback from well-aligned close-source LLMs to select samples.
\citet{cao2024instructionmininginstructiondata,li-etal-2024-quantity, ge2024clustering, si2024selecting, xia2024less,zhang2024recostexternalknowledgeguided} try to utilize the well-designed metrics (e.g., complexity) based on open-source LLMs to select the samples.
However, these high-quality data always contain expert-level responses and may contain much unfamiliar knowledge to the LLM.
Unlike focusing on data quality, we attempt to identify the samples that align well with LLM's knowledge, thereby allowing the LLM to hallucinate less.

\section{OpenFly Data Generation Platform}

\begin{figure*}[t]
\begin{center}
   \includegraphics[width=\linewidth]{Fig/all_images.png}
\end{center}
   \caption{High-quality examples from different rendering engines and techniques, including several large cities such as Shanghai, Guangzhou, Los Angeles, Osaka, and etc., cover an area of over a hundred square kilometers in total. 3D GS provides five large campus scenes, further enhancing the diversity and realism of the data.}
\label{fig:all_dataset}
\end{figure*}


In this section, we first describe several basic simulators and data resources, and then present the developed toolchain. The framework of the whole automatic data generation platform is illustrated in Fig. \ref{fig:data_gen}.

\subsection{Basic Simulators and Data Resources}
\label{sec:Automatic}


\indent \indent To collect a wide range of high-quality and realistic simulation data, we source the dataset from multiple rendering engines integrated with various simulators. Fig. \ref{fig:all_dataset} showcases several examples obtained from these rendering engines/techniques.

%\footnote{https://www.unrealengine.com/marketplace/product/city-sample/}
\textbf{Unreal Engine + AirSim/UnrealCV.} UE is a rendering engine capable of providing highly realistic interactive virtual environments. This platform has undergone five iterations, and each version features comprehensive and high-quality digital assets. In UE5, we meticulously select an official sample project named `City Sample', which provides us with a large urban scene covering $25.3 km^2$ and a smaller one covering $2.7 km^2$. These scenes include a variety of assets such as buildings, streets, traffic lights, vehicles, and pedestrians. Besides, in UE4, we prepare six more high-quality scenes. Specifically, there are two large scenes showcasing the central urban areas of Shanghai and Guangzhou, covering areas of $30.88 km^2$ and $58.56 km^2$, respectively. The remaining four scenes are selected from AerialVLN~\cite{aerialVLN}. They have smaller areas for totally about $26.64 km^2$. These scenes encompass a wide range of architectural styles, including both Chinese and Western influences, as well as classical and modern designs. Additionally, the UE4 engine allows us to make adjustments in scene time to achieve different appearances of scenes under varying lighting conditions.

% Meanwhile, it can also offer RGB, depth, and segmentation maps with realistic physics and sensor models. 
%UnrealCV is an open-source plugin for Unreal Engine, providing a simple interface for capturing RGB, depth, and segmentation images, thus facilitating research in computer vision and robotics.
Airsim is an open-source simulator, which provides highly realistic simulated environments for UAVs and cars. We integrate the AirSim plugin into UE4 to obtain image data easily from the perspective of a UAV.
%\footnote{https://github.com/microsoft/AirSim}
Since AirSim does not support UE5 and stopped updating in 2022, we use the UnrealCV~\cite{unrealcv} plugin as an alternative for image acquisition in UE5. To realize a highly efficient data collection in simulated scenes, we modify the UE5 project to a C++ project, integrate the UnrealCV plugin, and package executables for multiple systems like Windows and Linux. 

\textbf{GTA V + Script Hook V.} 
GTA V is an open-world game that is frequently used by computer vision researchers due to its highly realistic and dynamic virtual environment. The game features a meticulously crafted cityscape modeled after Los Angeles, encompassing various buildings and locations such as skyscrapers, gas stations, parks, and plazas, along with dynamic traffic flows and changes in lighting and shadows. 

Script Hook V is a third-party library with the interface to GTA V's native script functions. With the help of Script Hook V, we build an efficient and stable interface, which receives the pose information and returns accurate RGB images and lidar data. From the interface, we can control a virtual agent to collect the required data in an arbitrary pose and angle in the game.

%Specifically, it uses various 3D modeling techniques provided by softwares such as 3D Max and SketchUp to model urban-level scene images into 3D models. 
%These models are then uploaded to Google Earth and stitched together to form continuous 3D scenes for display.
\textbf{Google Earth + Google Earth Studio.} 
Google Earth is a virtual globe software, which builds a 3D earth model by integrating satellite imagery, aerial photographs, and Geographic Information System (GIS) data. From this engine, we select four urban scenes covering a total area of $53.60 km^2 $, \emph{i.e.,} Berkeley, primarily consisting of traditional neighborhoods; Osaka, which features a mix of skyscrapers and historic buildings; and two areas with numerous landmarks: Washington, D.C., and St. Louis.

%\footnote{earth.google.com}

%\footnote{https://www.google.com/earth/studio/}
%It expands upon the Google Earth browsing interface by integrating features commonly found in video production software. These enhancements
%developed by the Google Earth team
Google Earth Studio is a web-based animation and video production tool that allows us to create keyframes and set camera target points on the 2D and 3D maps of Google Earth. Using this functionality, we can quickly generate customized tour videos by selecting specific routes and angles. In order to efficiently plan the route, we develop a function that automatically draws the flight trajectory in Google Earth Studio according to the selected area and predefined photo interval. 
%We also implement a function that stores the collected images in a normalized coordinate system based on the GPS information.
%according to the data collection area and photo interval we set,
%The tool also supports exporting the output as MP4 files or image sequences, providing flexibility for further use.


%However, under the drone's perspective, choosing the appropriate shooting altitude posed a dilemma, \emph{i.e.,} if the altitude is too low, the sparse point cloud generated during the initialization of the 3D GS reconstruction will be suboptimal due to insufficient feature point matches between photos. On the other hand, if the altitude is too high, the Gaussian reconstruction will result in overly coarse training of details. After multiple attempts, the data collection plan using the M30T was determined as follows. For large-scale block scenes, oblique photography is performed at approximately twice the average building height using the default parameters of the M30T’s wide-angle camera, with a tilt angle of -45°. For landmark buildings with heights significantly different from the average height, additional targeted data collection is conducted at twice their height. This altitude setting can, to a certain extent, ensure both higher-quality point cloud initialization and Gaussian splatting training.(放到supp)

\textbf{3D Gaussian Splatting + SIBR viewers.} As a highly realistic reconstruction method, hierarchical 3D GS~~\cite{kerbl2024hierarchical} employs a hierarchical training and display architecture, making it particularly suitable for rendering large-scale areas. Due to these features, we use this method to reconstruct and render multiple real scenes. We utilize the DJI M30T drone as the data collection device, which offers an automated oblique photography mode, enabling us to capture a large area of real-world data with minimal manpower. Practically, we gathered data from five campuses across three universities, which are East China University of Science and Technology, Northwestern Polytechnical University, and Shanghai Jiao Tong University (referred to as ECUST, NWPU, and SJTU). These campus scenes include various types and styles of landmarks, such as libraries, bell towers, waterways, lakes, playgrounds, construction sites, and lawns. The detailed information for the five campuses is presented in Table~\ref{tab:GS_information}. More details of the 3D GS data collection can be found in our supplementary material.

SIBR~\cite{sibr2020} viewers is a rendering tool designed for the 3D GS project, enabling visualization of a scene from arbitrary viewpoints. The tool supports high-frame-rate scene rendering and provides various interactive modes for navigation. Building upon SIBR viewers, we developed an HTTP RESTful API that generates RGB images from arbitrary poses, simulating a UAV's perspective.


\begin{table}[t]
\caption{\textbf{Different 3D GS Scenes}}
\label{tab:GS_information}
\centering
\begin{tabular}{ccc}
\toprule
Campus Name&Images&Area \\
\midrule
\makecell{ECUST  (Fengxian Campus)} & 12008 & $1.06km^2$ \\
\midrule
\makecell{NWPU  (Youyi Campus)} & 4648 & $0.8km^2$ \\
\makecell{NWPU  (Changan Campus)} & 23798 & $2.6km^2$ \\
\midrule
\makecell{SJTU  (Minghang-East Zone)} & 20934 & $1.72km^2$ \\
\makecell{SJTU  (Minghang-West Zone)} & 9536 & $0.95km^2$ \\
\bottomrule
\end{tabular}
\end{table}



%In order to achieve automatic trajectory generation, it is necessary to structure the scene, which involves obtaining the 3D point cloud and the semantic segmentation. Based on these, a unified interface for image collection, a trajectory generation tool, and an instruction generation tool are developed.
\subsection{Toolchain for Automatic Data Colleciton}
\indent \indent To achieve automatic data generation, we integrate the above five simulators and design three unified interfaces, \emph{i.e.,} the agent movement interface, the lidar data acquisition interface, and the image acquisition interface, allowing an agent to interact with any scene. Based on these interfaces, we further develop a toolchain, including 3D point cloud acquisition, scene semantic segmentation, automatic trajectory generation, and instruction generation. The framework of the whole data generation platform is illustrated in Fig. \ref{fig:data_gen}, with details of these interfaces and tools elaborated below.

%To facilitate the collection of trajectories and images across various simulation environments, the OpenFly platform integrates all the aforementioned simulators and provides unified interfaces that enable interaction between an agent and the environment in any scene. Specifically, there are three main interfaces. 
%$[X, Y, Z] [QW, QX, QY, QZ]$
%$[d_x, d_y, d_z]$ and $[d_{roll}, d_{pitch}, d_{yaw}]$
%using either absolute poses or relative poses in the agent's body frame
%3) Scene Information Interface: The 2D coordinates and height information of all landmarks, along with the point cloud map of the entire scene, can be accessed through this interface.
\textbf{Unified Interfaces.} 
1) Agent Movement Interface: We design a \textit{CoorTrans} module, which implements a customized pose transformation matrix and scaling function to unify all simulator coordinate systems into a meter-based FLU (Front-Left-Up) convention. This interface enables precise agent positioning among regular scenes, point clouds, and scene segmentations, ensuring consistency and facilitating automatic trajectory generation.
2) Lidar Data Acquisition: For different simulators, point cloud data is acquired  through different methods, including lidar sensor collection, depth map back-projection, and image feature matching. We develop a unified interface to integrate these methods and leverage the proposed \textit{CoorTrans} module to align all data to the same FLU coordinate system.
3) Image Acquisition Interface: We integrate HTTP RESTful and TCP/IP protocols to form a unified image request interface, allowing image data to be obtained from any location with flexible  resolutions and agent viewpoints. 

%we use three methods to obtain 3D point clouds from the scenes, \emph{i.e.,} depth map back-projection, LiDAR scan reconstruction, and sparse reconstruction. For the UE5+UnrealCV simulator, a project named MatrixCity~~\cite{li2023matrixcity} provides us with the depth maps and camera parameters of small-city and big-city scenes. 1) Through back-projecting the 2D depth information into 3D space, we generate the point clouds for the two datasets. 2) For the UE4+AirSim and GTA5 simulators, we directly utilize the LiDAR sensors provided by the simulators to traverse each scene in a grid pattern, obtaining local point clouds. These are then transformed into the global coordinate system using the LiDAR coordinate information and finally merged into complete scene point clouds. 3) In 3D GS, since the first step of Gaussian Splatting reconstruction involves using the open-source project COLMAP ~\cite{colmap} to perform sparse structure-from-motion (SFM) point cloud reconstruction based on input images, we could directly export and use the point clouds obtained from this step.
%structure-from-motion (SFM)
\textbf{3D Point Cloud Acquisition.} 
For different simulators, we provide two methods to reconstruct the point cloud map of an entire scene. 1) Rasterized Sampling Reconstruction:
For the UE5 + UnrealCV simulator, the MatrixCity~\cite{li2023matrixcity} project offers a convenient rasterized sampling solution. We use the aforementioned lidar data acquisition interface to obtain the local point cloud at the sampling points. Since these data are already aligned within the same coordinate system, the point cloud map of the entire scene can be constructed by simply stitching local point clouds. For the UE4 + AirSim and GTA V simulators, we customize rasterized sampling points at appropriate resolutions, and perform sampling and reconstruction using the agent movement and lidar data acquisition interfaces. 2) Image-based Sparse Reconstruction: In 3D GS, the scene reconstruction process begins with the open-source COLMAP~\cite{colmap} framework, which geneoverlearates a sparse point cloud from input images. We directly export and use the point clouds obtained from this step. 


\textbf{Scene Semantic Segmentation.} 
Vision-and-Language Navigation (VLN) requires meaningful landmarks as navigation targets. We perform semantic segmentation on four types of simulation scenes using the following three methods. 1) 3D Scene Understanding: A sequence of top-down views of the scene is captured in a rasterized format. We then use Octree graph~\cite{octree_graph} to extract 2D mask proposals, which are subsequently projected into the 3D point cloud space to generate semantic 3D segments. 2) Point Cloud Projection and Contour Extraction: We first acquire the point cloud of a scene, then project the voxelized point cloud onto a projection plane slightly above the ground. Using OpenCV, we perform a series of operations on the projected image, including binarization, erosion, and contour extraction, to obtain multiple instances along with their 2D coordinates. For each instance, the maximum height of the points within its neighborhood is used as the final height. This method provides a more computationally efficient yet coarser segmentation compared to the first approach, allowing users to choose based on their requirements. 3) Manual Annotation: When the point cloud quality of a scene is low or finer segmentation is required, we provide a method for annotating instances in the point cloud space by mouse clicks, based on ROS2 and RVIZ2. Users can annotate instances directly in the point cloud space using mouse clicks to define landmarks of interest for the task. This method is applicable to four simulators, \emph{i.e.,} UE + UnrealCV, UE + Airsim, GTA V, and 3D GS.

\begin{comment}
\begin{figure}
    \centering
    % 第一列子图
    \begin{subfigure}[b]{0.15\textwidth}   % 0.3\textwidth 是每个子图的宽度
        \centering
        \includegraphics[width=\textwidth]{Fig/whole_scene.pdf} % 替换为你的图片文件
        \caption{}
        \label{fig:sub1}
    \end{subfigure}
    \hfill  % 用来在子图之间增加水平间隔
    % 第二列子图
    \begin{subfigure}[b]{0.15\textwidth}
        \centering
        \includegraphics[width=\textwidth]{Fig/scene_point_cloud.pdf}
        \caption{}
        \label{fig:sub2}
    \end{subfigure}
    \hfill
    % 第三列子图
    \begin{subfigure}[b]{0.15\textwidth}
        \centering
        \includegraphics[width=\textwidth]{Fig/scene_seg.pdf}
        \caption{}
        \label{fig:sub3}
    \end{subfigure}
    
    \caption{Illustration of results obtained by our point cloud acquisition and semantic segmentation tools. (a) Overview of an urban scene. (b) The point cloud of (a). (c) The semantic segmentation of (a).}
    \label{fig:main}
\end{figure}
\end{comment}


\textbf{Automatic Trajectory Generation.}
Leveraging the aforementioned point cloud map and segmentation tools, OpenFly can generate trajectories using the following two methods. 
%1) Path search based on customized action space: First, a global hash voxel map $M_{global}$ is constructed from the scene point cloud $P$ and the voxelized point cloud is projected onto the horizontal plane to obtain the bird's eye view (BEV) occupancy map $M_{bev}$. 
1) Path search based on customized action space: First, a global hash voxel map $M_{global}$ and a bird's eye view (BEV) occupancy map $M_{bev}$ are constructed from the scene point cloud $P$.
Second, the flight altitude is randomly selected within the user-defined height range, and landmarks that are not lower than the height threshold $H_{\tau}$ are chosen as targets. A starting point is selected within the distance range $[r, R]$ from the landmark, ensuring that it is not occupied in both $M_{global}$ and $M_{bev}$. Then, a point on the line connecting the starting point and the landmark, which is close to the landmark and unoccupied in $M_{bev}$, is chosen as the endpoint. 
Third, A collision-free trajectory from the starting point to the endpoint is generated using the A*~~\cite{astar} pathfinding algorithm, where the granularity of exploration step size and direction can be adjusted according to the action space. Besides, by repeatedly selecting the endpoint as the new starting point, complex trajectories can be generated. Finally, utilizing OpenFly's interface, images corresponding to the trajectory points can be obtained. 2) Path search based on grid: Google Map data does not allow image retrieval at arbitrary poses in the space. Thus we rasterize a pre-selected area and collect images from each grid point in all possible orientations. Starting and ending points are chosen within the grid points to generate trajectories. Corresponding images for these trajectory points are then selected from the pre-collected image set.



% 视觉语言导航数据中,语言指令的质量至关重要。然而,以往的研究大多依赖人工标注,不仅成本高昂,还限制了数据集规模。为此,OpenFly 提出了基于视觉语言模型(VLM)的全自动化语言指令生成方法。

% 自然的想法是将所有图像全部提交给VLMs进行轨迹分析,生成指令。但全部图像的输入带来了巨额的开销,并带来了信息冗余。因此 OpenFly 将 完整轨迹拆分为子轨迹来进行处理,提取每一个子轨迹的关键动作和landmark特征,最终进行整合。与室内视觉语言导航不同,在空中视觉语言导航(Aerial Vision Language Navigation)中,环境中的障碍物较少,因此指令中的“Move Forward” 占据了大部分比例,关键动作集中在“左转/右转”和“上升/下降”。此外,无人机飞行轨迹中不可避免地会出现轻微角度调整,这些调整往往无明确目的地,因此被简化为“slightly turn left/right”。所以我们根据轨迹是否发生了连续的非直行动作来进行拆分。举一个例子,轨迹动作序列为[1,2,1,2,2,1,0],这里1代表move forward,2代表turn left,0代表 stop。该轨迹会被拆分为[1,2,1],[2,2],[1,0]。第一条的动作为Move forward and slightly turn left,与此同时我们将第一条子轨迹的最后一张图像提交给VLM提取对应的landmark特征,为了更加精确,我们还会提供Landmark在图像中的位置星系

% User:You are an assistant proficient in image recognition. You can accurately identify the object closest to you in the image and its different features from surrounding objects.The object is at the center of the image.

% GPT 4o:{color: bule, feature: Steel, glass, size: medium size, type: building}

% 最终我们得到了如下的子指令序列:
% 1. Action 1(Move forward and slightly turn left) , Landmark 1
% 2.Action 2( turn left), Landmark 2
% 3.Action 3 (Move forward ), Landmark 3

% 我们会将子指令序列提交给GPT 4o,并获得最终的指令。


% 基于上述方法,我们生成了一系列“动作 + Landmark”的子指令。随后,利用 VLM 的语言生成能力,将这些子指令整合为流畅、完整的自然语言导航指令

%Unlike indoor VLN, in aerial VLN, there are fewer obstacles in the environment, meaning that , with key actions focused mainly on “Turn Left/Turn Right” and “Ascend/Descend.”



\textbf{Automatic Instruction Generation.}
Previous research has predominantly relied on manual annotation to generate trajectory instruction, which is not only costly but also limits the scalability of datasets. To address this issue, we propose a highly automated language instruction generation method based on VLMs, \emph{e.g.,} GPT-4o.
A straightforward method would be to submit all images to VLMs to analyze the trajectory and generate instructions. However, using all images introduces significant computational overhead and leads to information redundancy. For example, the `Forward' action usually occupies a larger proportion of a flight trajectory, with `Turn Left/Turn Right' or `Ascend/Descend' actions taken when encountering key landmarks.




Based on the above findings, we split the complete trajectory into multiple sub-trajectories based on the occurrence of non-consistent actions, extracting key actions and images for processing and subsequent integration. Notably, slight angle adjustments often occur during flight to change the direction subtly, and a `slightly Turn Left/Right' will be merged with subsequent `Forward' actions. Specifically, suppose that the trajectory action sequence is [1, 1, 2, 1, 1, 2, 2, 1, 0], where 1, 2, and 0 denote `Forward', `Turn Left', and `Stop', respectively. This trajectory would be split into four sub-trajectories, \emph{i.e.,} [1, 1], [2, 1, 1], [2, 2], and [1, 0]. The second sub-trajectory involves `slightly turn left' and `move forward'. We submit the action sequence and the last captured image of each sub-trajectory to a VLM to generate descriptions of both action and landmarks.
%A simplified prompt to the VLM and corresponding response are probably like this. User: `You are an assistant proficient in image recognition. You can accurately identify the object closest to you in the image and its different features from surrounding objects. The object is usually at the center of the image'. GPT 4o: `{color: blue, feature: Steel, glass, size: medium size, type: building}'.
The sub-instructions are obtained similar to the following format:
\begin{itemize}[left=0pt]
    \item `Move forward' to `Landmark 1'.
    \item `Slightly turn left and move forward' to `Landmark 2'.
    \item `Turn left' towards `Landmark 3'.
    \item `Move forward' to `Landmark 4'.
\end{itemize}

These sub-instructions are then processed by a VLM/LLM again, where they are integrated into coherent and complete instructions. The detailed prompt used for the VLM, along with the complete responses, is provided in the supplementary material.

%Based on this method, we generated a series of “Action + Landmark” sub-instructions. Subsequently, leveraging the language generation capabilities of VLMs, these sub-instructions were integrated into coherent and complete natural language instructions. More details and the complete prompt to GPT-4o are shown in our supplementary material.

%User: You need to help me combine these scattered actions and landmarks into a sentence with words that are similar in meaning and more appropriate in words, so that the sentence is fluent and accurate. At the same time, merge the same landmarks accurately. {Sub-instructions}

%GPT 4o: Move forward and slightly turn left to a high-rise building with a noticeable logo at the top.Then turn left and go straight to a futuristic tower with a large spherical structure in the middle.
% 

% Specifically, we divide the instruction generation process into two main components, \emph{i.e.,} actions and the corresponding landmarks. Unlike indoor VLN, aerial VLN features fewer obstacles in the environment, resulting in a higher proportion of the “Move Forward” action, with key movements focusing on “Turn Left/Right” and “Ascending/Descending.” Additionally, slight angular adjustments are unavoidable in drone flight trajectories. However, these adjustments often lack a specific destination and are simplified as “slightly turn left/right.”

% The wide field of view from drones and the similarity in appearance among common buildings present challenges in identifying landmarks. To overcome this, in addition to image data, we provide the VLM with landmark positional information, \emph{e.g.,} "in the center of the image," or "at the bottom of the image", to extract key attributes of landmarks, such as type, color, and shape.



    
\subsection{Quality Control.}
%一些filter机制 和 人工抽查策略
\textbf{Data Filter.}
During data collection, it is inevitable that some damaged or low-quality data will be generated. We clean the data in the following situations. 1) We remove damaged images that are produced in generation or transmission. 2) We find that UAVs sometimes appear to pass through the tree models. Therefore, we exclude the trajectories where the altitude is lower than that of the trees. 3) We believe that extremely short or long trajectories are not conducive to model training. Thus, we remove these trajectories, specifically those with fewer than 2 or more than 150 actions.




\textbf{Instruction Refinement.}
A known challenge of instruction generation is VLMs' hallucinations. During the previous instruction generation process, sometimes the same landmark appears across several frames. This results in a VLM generating similar captions for the repeated observations of a landmark, increasing the complexity of the final instruction and introducing ambiguity due to duplication.

To mitigate this challenge, we utilize the NLTK library ~\cite{bird2006nltk} to simplify the instruction by detecting and merging similar descriptions. Specifically, a syntactic parse tree is first constructed to extract all landmark captions using a rule-based approach. Then, a sentence-transformer model is employed to encode the extracted landmark captions into embedding vectors. Their similarities are computed with dot product, and high-similarity captions are then identified and replaced with referential pronouns (\emph{e.g.}, ``it," ``there," \emph{etc.}). For example, a generated instruction with redundant information is ``$\cdots$ make a left turn toward \textbf{a medium-sized beige building marked by a signboard reading CHARLIE'S CHOCOLATE}. Continue heading straight, passing \textbf{a medium-sized gray building with a prominent rooftop billboard displaying Charlie’s Chocolate} $\cdots$". After simplification, a more concise sentence is obtained, \emph{i.e.,} ``$\cdots$ make a left turn toward \textbf{a medium-sized beige building marked by a signboard reading CHARLIE'S CHOCOLATE}. Continue heading straight, passing \textbf{it} $\cdots$", demonstrating the effectiveness of this post-processing technique. 

At the same time, we built a data inspection platform to provide instructions, action sequences, and corresponding images to the examiners. If the instructions and trajectories align, they are considered qualified. We randomly select 3K samples from the entire dataset  according to data distribution in Sec. \ref{data_split}. After manually inspecting these samples, we find that the qualification rate reaches 91\%.


%我们首先用nltk库进行英文文本处理,构建语法关树,通过rule-based方式提取所有的landmark caption。 接着,是用sentence transformer将所选landmark编码embeddings,通过点积计算其相似度,筛选出高相似性的短语嵌入, 将其替换为指代性名词(it, there etc.)。



\section{Dataset Analysis}

\begin{table*}[t]
\small      
\caption{Comparison of different VLN datasets. Above the middle dividing line lies the ground-based datasets, while below is the aerial VLN datasets. $N_{traj}$: the number of total trajectories. $N_{vocab}$: vocabulary size. Path Len: the average length of trajectories, measured in meters. Intr Len: the average length of instructions. $N_{act}$: the average number of actions per trajectory.}
\begin{adjustbox}{center}
%\setlength{\tabcolsep}{3pt}
\renewcommand{\arraystretch}{1.2}
\scalebox{.99}{
\begin{tabular}{lcccclcc}
\toprule
Dataset   & $N_{traj}$ & $N_{vocab}$ & Path Len. & Intr Len. & Action Space & $N_{act}$ & Environment \\ \midrule
R2R~\cite{R2R}       & 7189      & 3.1K         & 10.0      & 29        &graph-based   & 5       & Matterport3D  \\
RxR~\cite{RxR}       & 13992     & 7.0K         & 14.9      & 129       &graph-based   & 8       & Matterport3D  \\
REVERIE~\cite{REVERIE}   & 7000      & 1.6K         & 10.0      & 18        &graph-based   & 5       & Matterport3D  \\
CVDN~\cite{CVDN}      & 7415      & 4.4K         & 25.0      & 34        &graph-based   & 7       & Matterport3D  \\
TouchDown~\cite{Touchdown} & 9326      & 5.0K         & 313.9     & 90        &graph-based   & 35      & Google Street View  \\ 
VLN-CE~\cite{VLN-CE}    & 4475      & 4.3K         & 11.1      & 19        &2 DoF         & 56      & Matterport3D  \\
LANI~\cite{LANI}      & 6000      & 2.3K         & 17.3      & 57        &2 DoF         & 116     & CHALET  \\ \midrule
ANDH~\cite{ANDH}      & 6269      & 3.3K         & 144.7     & 89        &3 DoF         & 7       & xView  \\
AerialVLN~\cite{aerialVLN} & 8446      & 4.5K         & 661.8     & 83        &4 DoF         & 204     & AirSim + UE \\
CityNav~\cite{CityNav}   & 32637     & 6.6K         & 545       & 26        &4 DoF         & -       & SensatUrban  \\
OpenUAV~\cite{openuav}   &12149      &10.8K         & 255       & 104       &6 DoF         & 264     & AirSim + UE \\ \midrule
\multirow{2}{*}{Ours} &\multirow{2}{*}{100K}     &\multirow{2}{*}{15.6K}        &\multirow{2}{*}{99.1}        &\multirow{2}{*}{59}       &\multirow{2}{*}{4 DoF}    &\multirow{2}{*}{35}     & \parbox[t]{6cm}{AirSim + UE, GTA5 + Script Hook V, \\ Google Earth Studio, 3D GS + SIBR viewers} \\

\bottomrule
\end{tabular}
}
\end{adjustbox}
\label{tab:dataset_comp}
\end{table*}

\begin{figure}
    \centering
    % 第一行子图
    \begin{subfigure}[b]{0.47\columnwidth}
        \centering
        \includegraphics[width=\textwidth]{Fig/action_num.pdf}
        \caption{Difficulty level distribution.}
        \label{fig:sub1}
    \end{subfigure}
    \hfill
    \begin{subfigure}[b]{0.47\columnwidth}
        \centering
        \includegraphics[width=\textwidth]{Fig/length_height.pdf}
        \caption{Length-height distribution.}
        \label{fig:sub2}
    \end{subfigure}

    % 换行
    \vspace{0.5cm} 

    % 第二行子图
    \begin{subfigure}[b]{0.47\columnwidth}
        \centering
       
        \includegraphics[width=\textwidth]{Fig/action.pdf}
        \caption{Action distribution with 1 type of `Forward'.}
        \label{fig:sub3}
    \end{subfigure}
    \hfill
    \begin{subfigure}[b]{0.47\columnwidth}
        \centering
        \includegraphics[width=\textwidth]{Fig/action_merge.pdf}
        \caption{Action distribution with 3 types of `Forward'.}
        \label{fig:sub4}
    \end{subfigure}

    \caption{Statistical analysis of trajectories.}
    \label{fig:traj_sta}
\end{figure}

% 词云(看看是否好分名词和动词)、总的轨迹/instruction数量、Vocabulary size、平均每条instruction词量;

%平均路径长度、平均action个数;路径长度分布统计图、action个数分布统计图(easy、middle、hard各多少轨迹);action type分布饼图(见AerialVLN)

%dataset split:train、test划分,各多少轨迹;不同场景的轨迹数量分布饼图(train的不同场景的分布饼图).
\subsection{Overview}
Using our toolchain, we collect 100k trajectories from 18 scenes, along with corresponding image sequences and language instructions. During the data generation process, we set a minimum motion step size of 3 meters to produce more granular trajectories. The details of our and previous VLN datasets are listed in Table. \ref{tab:dataset_comp}, from which we can see that our dataset features a significantly larger number of trajectories and a more extensive vocabulary, as well as greater environmental diversity. In contrast, our average trajectory length and instruction length are relatively short. This is intentional, as we believe" short- and medium-range instructions are actually more in line with the usage habits of human users. This might be more beneficial for the aerial VLN field.

\subsection{Trajectory Analysis}

In addition to a rich variety of scenes, we also strive for diversity in the difficulty level, length, and height of the trajectory data. Based on the number of actions in one trajectory, we classify trajectories with fewer than 30 actions as `Easy', those with the number of actions ranging from 30 to 60 as `Moderate', and those with more than 60 actions as `Hard'. Fig. \ref{fig:sub1} shows the corresponding difficulty level distribution. Besides, compared with ground-based VLN, the aerial VLN task has more motion dimensions. Therefore, we set different trajectory lengths and flight heights to obtain rich data. Fig. \ref{fig:sub2} exhibits the distribution of these data, with their lengths ranging from 0 to 300 meters, and the heights ranging from 0 to 150 meters. 

In the aerial VLN tasks of large-scale outdoor scenes, the proportion of moving forward is naturally higher than that of making adjustments in direction and altitude, as shown in Fig. \ref{fig:sub3}. However, this highly unbalanced action distribution might cause the VLN model to overfit to the dominant action. To alleviate this problem, we divide the `Forward' action into three granularities, \emph{i.e.,} 3m, 6m, and 9m. In the ground-truth trajectories, three consecutive `Forward' actions will be combined into one `9m Forward' action. At the end of a straight-moving trajectory, if the remaining distance is less than 9m, it will be combined into a `6m Forward' action, or remain as a `3m Forward' action. Fig. \ref{fig:sub4} presents the action distribution after this action merging process.



\begin{figure}
    \centering
    % 第一行子图
    \begin{subfigure}[b]{0.47\columnwidth}
        \centering
        \includegraphics[width=\textwidth]{Fig/train_split.pdf}
        \caption{Train set distribution.}
        \label{fig:train_sp}
    \end{subfigure}
    \hfill
    \begin{subfigure}[b]{0.47\columnwidth}
        \centering
        \includegraphics[width=\textwidth]{Fig/test_split.pdf}
        \caption{Test set distribution.}
        \label{fig:test_sp}
    \end{subfigure}
    \caption{The distribution of the data volume in different scenes under the Train and Test sets.}
    \label{fig:traj_sta}
\end{figure}


\section{OpenFly-Agent}
\begin{figure*}[t]
\centering
    \includegraphics[width=0.98\linewidth]{Fig/model_arch.pdf}
    \caption{The architecture of OpenFly-Agent. Keyframes at the time of action transitions are selected to extract crucial observations as the history, with corresponding visual tokens compressed to reduce the computational burden.}
    \label{fig:model}
\end{figure*}

%\textit{Test Seen} and \textit{Test Unseen} indicate that whether the scenes have appeared in the \textit{Train} set.
\subsection{Dataset Split}
\label{data_split}
Similar to previous works, we divide the dataset into three splits, \emph{i.e.,}  \textit{Train, Test Seen, Test Unseen}. Detailed data distributions are shown in Fig. \ref{fig:traj_sta}. For the \textit{Train} split, 7 scenes under the UE rendering engine account for $75.7\%$ of the total \textbf{100K} data, since UE provides the largest number of scenes, where different amounts of trajectories are sampled according to the scenario area. The 4 scenes created by 3D GS are also the main part of the data, accounting for nearly $20\%$ of the total amount. To ensure visual quality, we only collect data from a high-altitude perspective using Google Earth, which accounts for $4.46\%$. 
The detailed information of the \textit{Test Seen} and \textit{Test Unseen} splits are as follows:
\begin{itemize}[left=0pt]
    \item \textit{Test Seen}: 1800 trajectories uniformly sampled from 11 previously seen UE and 3D GS scenarios.

    \item \textit{Test Unseen}: 1200 trajectories uniformly generated from 3 unseen scenarios, \emph{i.e.,} UE-smallcity, 3D GS-sjtu02, and a Los Angeles-like city in GTA V.
\end{itemize}


% 
\section{TUMTraffic-Qwen Baseline}
% In this section, we introduce the baseline model of the TUMTraffic-VideoQA dataset. We provide a detailed description of the model architecture and introduce our training recipes. 

%  [x] TODOS, projector and sampler reverse  !

\subsection{Model Architecture}

We introduce TUMTraffic-Qwen, a baseline model for the TUMTraffic-VideoQA dataset that effectively addresses all three tasks within a unified framework. The architecture of the TUMTraffic-VideoQA baseline, as illustrated in Figure \ref{baseline_model}, consists of four core components: visual encoder $f_v$, cross-modality projector $ g_\psi $, token sampler $\mathcal{S}_v$, and large language model $f_\phi$, following \cite{li2024llavaonevisioneasyvisualtask}. \\


% \begin{equation}
% \small
% p(\mathbf{X}_a \mid \mathbf{X}_v, \mathbf{X}_{\text{instruct}}) = \prod_{i=1}^{L} p_\theta(x_i \mid \mathbf{X}_v, \mathbf{X}_{\text{instruct}}, <i, \mathbf{X}_a^{<i})
% \end{equation}


\noindent\textbf{Visual Encoder.} The video is uniformly divided into 100 segments, including the first and last frames, resulting in a total of \( N = 101 \) frames. Given the sampled video input \( \mathbf{X} \in \mathbb{R}^{N \times H \times W \times 3} \), we adopt SigLIP \cite{siglip}, a Transformer-based model pre-trained on large-scale language-image datasets, as the visual encoder. Each frame is processed at a resolution of \( 384 \times 384 \), and the video is encoded into a sequence of visual features \( Z_v = [v_1, \dots, v_N] \), where \( v_i = f_v (\mathbf{X}_i) \in \mathbb{R}^{T \times C} \), containing $T$ spatial tokens of dimension $C$.

 
 
\noindent\textbf{Token Sampling Strategy.} We leverage a simple yet effective frame-level multi-resolution sampling strategy to enhance feature representation. We evaluate four primary sampling strategies: spatial pooling, multi-resolution spatial pooling, multi-resolution token pruning, and multi-resolution temporal pooling. The output $Z_{v}$ from the last layer of SigLIP is denoted as $Z_{\text{high}}$, which is reduced to $T'$ tokens after down-sampling. We define the set of high-resolution frames as keyframes, denoted by $\mathcal{K}(\cdot)$. Additionally, a learnable token is appended to the end of each frame to explicitly differentiate them. The number of tokens used in various strategies is presented in Table \ref{tab:tokennum}.

% To explicitly model inter-frame relations, the baseline model excludes the temporal aggregation module.
% , leaving this for future exploration
% the spatial down-sampling factor as $P$,
% for reducing visual token numbers 
\noindent\textbullet\  \textbf{Spatial Pooling}: This method applies spatial pooling to each feature map $Z_{\text{high}}$, resulting in a down-sampled representation $Z_{\text{low}} = f_{\text{pool}}(Z_{\text{high}})$ with $N \times T'$ tokens, as shown in Eq. \ref{formula:spatial_pooling}. We use the notation $[ \cdot ]_{n}^{N}$ to represent the operation of sequentially concatenating the processed feature maps.



\begin{figure}[t!]
    \centering
    \includegraphics[width=0.5\textwidth]{figure/TrafficQA-baseline2.jpg}
    \caption{Overview of the TUMTraffic-Qwen baseline model. Yellow and orange colors represent the combination of multi-resolution visual tokens from different visual strategies, while blue indicates textual tokens.}
    \label{baseline_model}
\end{figure}

{\small
\begin{equation}
S_v(Z_v) =  [ Z_{\text{low}}^{n}, Z_{\text{learn}} ]_{n=1}^N 
\label{formula:spatial_pooling}
\end{equation}}


\noindent\textbullet\ \textbf{MultiRes Spatial Pooling}: Compared to the naive spatial pooling, this strategy selects the first frame as the keyframe $\mathcal{K}$ = (1), and is retained at its original resolution $Z_{\text{high}}^1$. It is formulated in Eq. \ref{formula:mluti-spatial_pooling}. 



{\small
\begin{equation}
S_v(Z_v) = [ Z_{\text{high}}^1, Z_{\text{learn}}, [ Z_{\text{low}}^{n}, Z_{\text{learn}} ]_{n=2}^N \big]
\label{formula:mluti-spatial_pooling}
\end{equation}}


\noindent\textbullet\  \textbf{MultiRes Token Pruning}: Similar to MultiRes Spatial Pooling, the first frame is designated as the keyframe. Token-wise cosine similarity is then computed between the keyframe and each subsequent frame, while visual tokens with lowest similarity are selectively retained based on predefined ratio $r$, formulated as $Z_{\text{pruned}} = f_{\text{prune}}^{r}(Z_{\text{high}} )$, shown in Eq. \ref{formula:mluti-spatial_spar}. To ensure visual token efficiency comparable to spatial pooling, $r$ is set to 0.25. A similar strategy is also applied in autonomous driving scenarios \cite{ma2024videotokensparsificationefficient}.


 \begin{table}[bt!]
% \centering
\caption{Comparison of visual token numbers across different token sampling strategies. We keep the high resolution at 27×27 and the low resolution at 14×14.}
\resizebox{0.48\textwidth}{!}{

\begin{tabular}{c|c|c }
\midrule
\textbf{Method}  & \textbf{Number of Visual Tokens}  & \textbf{Max Tokens}                                   \\ \midrule
Spatial Pooling                         & $ N \times T'  + N$            & 19,897    \\  \midrule

MultiRes Spatial-Pooling        & $T +   (N-1) \times T'  + N$       &   20,430         \\ \midrule
% 729 + 100x196 + 101

MultiRes Token-Pruning         & $T + (N-1) \times r \times T  + N$        & 18,574          \\ \midrule
% 729+100*183 + 101

MultiRes Temporal-Pooling          & $K \times T + (N-K) \times T'   + N$    &   20,963        \\ \midrule

\end{tabular}}
% \caption{Comparison of visual token number across different token sampling strategies.}

\label{tab:tokennum}
\vspace{-2pt}

\end{table}




 % In the Multi-Resolution Token Pruning strategy, we set the token retention ratio to \( r = 0.25 \), to balances resolution and computational efficiency.



% \begin{table*}[t!]
% \centering
% \resizebox{\textwidth}{!}{%
% \begin{tabular}{l | l | cc | cc | cc | cc| cc | c}
% % \hline
% \midrule
% \multirow{2}{*}{\textbf{Models}} & \multirow{2}{*}{\textbf{Category}} & \multicolumn{2}{c}{\textbf{Positioning}} & \multicolumn{2}{c}{\textbf{Counting}} & \multicolumn{2}{c}{\textbf{Motion}} & \multicolumn{2}{c}{\textbf{Class}} & \multicolumn{2}{c}{\textbf{Existence}} & \multirow{2}{*}{\textbf{Overall}} \\
% % \cline{3-12}
% % \midrule

%  & & \textbf{E} & \textbf{H} & \textbf{E} & \textbf{H} & \textbf{E} & \textbf{H} & \textbf{E} & \textbf{H} & \textbf{E} & \textbf{H}  \\
% % \hline
% \midrule

% \multicolumn{13}{c}{Open-Source Models} \\ 
% % \hline
% \midrule

% \multirow{1}{*}{LLAVA-OneVision \cite{li2024llavaonevisioneasyvisualtask} } &  0.5B & 25.26 & 42.10 & 27.62 &  30.45 & 54.87 & 37.04 & \textbf{57.06} & 39.57 & \textbf{85.29} & 58.35 & 45.82 \\

% \rowcolor{gray!10}
% & 7B & 22.03 & \textbf{46.92} & \textbf{69.42} & \textbf{54.85} & 61.14 & \textbf{60.48} & 51.92 & \textbf{56.50} & 77.08 & 63.25 & \textbf{56.36} \\

% % \cline{2-13}
% \midrule

% \multirow{1}{*}{Qwen2-VL \cite{Qwen-VL}} &
%   2B & \textbf{26.05} & 36.73 & 38.10 & 39.78 & 56.46 & 35.19 & 32.10 & 38.49 & 68.87 & 67.32 & 43.91 \\

% & 7B & 24.35 & 36.03 & 66.91 & 49.11 & \textbf{61.65} & 38.10 & 44.83 & 40.20 & 54.00 & \textbf{73.03} & 48.82 \\

% % \cline{2-13}
% \midrule

% \multirow{1}{*}{VideoLLAMA2 \cite{cheng2024videollama2advancingspatialtemporal}}  & 2.0-7B-8F & 18.14 & 42.54 & 44.13 & 37.56 & 59.37 & 35.87 & 39.05 & 44.07 & 44.56 & 65.56 & 43.09 \\

% & 2.0-7B-16F & 10.47 & 42.41 & 55.98 & 41.94 & 53.80 & 52.26 & 44.16 & 47.75 & 66.93 & 64.82 & 48.05 \\

% % & 2.1-7B-16F & 14.97 & 27.81 & 49.60 & 39.36 & 34.75 & 30.30 & 40.16 & 40.46 & 75.92 & 65.49 & 41.88 \\ 

% % \hline
% \midrule

% \multicolumn{13}{c}{TUMTraffic-VideoQA Baseline} \\ 
% % \hline
% \midrule
% \multirow{4}{*}{Baseline-0.5B (Ours)} & Spatial Pooling  & 68.47 & 75.54 & 85.31 & 75.82 & 83.92 & \textbf{81.26} & 79.95 & 59.73 & 93.06 & 85.37 & 78.84 \\

%  & MultiRes Spatial-Pooling & 69.32 & 76.36 & 86.10 & 75.86 & 83.73 & 79.59 & \textbf{80.57} & 61.70 & 92.73 & 85.37 & 79.07 \\
 
% & MultiRes Token-Pr.  &  73.40 & \textbf{76.61} & \textbf{86.33} & 76.88 & 83.48 & 78.60 & 80.01 & 60.43 & \textbf{93.34} & 85.27 & 79.44 \\


% \rowcolor{gray!10}
% & MultiRes Temporal-Pooling & \textbf{74.07} & 75.85 & 85.65 & \textbf{76.92} & \textbf{84.05} & 80.64 & 80.26 & \textbf{62.21} & 93.06 & \textbf{85.55} & \textbf{79.83} \\
% % \hline
% \midrule

% \multirow{4}{*}{Baseline-7B (Ours)} & Spatial Pooling & 76.14 & 76.99 & 87.07 & 76.81 & 86.58 & 82.07 & 82.72 & 64.11 & 93.62 & 85.27 & 81.14 \\

% & MultiRes Spatial-Pooling  & 76.99 & \textbf{78.89} & 87.07 & 77.49 & \textbf{88.29} & 81.82 & 83.52 & \textbf{65.95} & 93.01 & \textbf{85.51} & 81.85 \\

% & MultiRes Token-Pr. & \textbf{77.24} & 76.93 & 87.41 & 77.76 & 86.46 & 80.64 & 82.66 & 65.00 & \textbf{93.84} & 85.48 & 81.34 \\


% \rowcolor{gray!10}
% & MultiRes Temporal-Pooling & \textbf{77.24} & 78.57 & \textbf{87.53} & \textbf{78.22} & 87.09 & \textbf{82.68} & \textbf{83.33} & 65.76 & 93.78 & 85.34 & \textbf{81.95} \\
% \midrule
% % \hline
% \end{tabular}%
% }
% \caption{Evaluation of Open-source models and TUMTraffic-Qwen baseline on the Multi-Choice QA track of the TUMTraffic-VideoQA Dataset, where \textbf{E} represents easy, single-hop questions, and \textbf{H} denotes hard, multi-hop questions.}
% \label{table:consolidated_metrics}
% \vspace{-2pt}

% \end{table*}

%  % \cite{ma2024videotokensparsificationefficient}\\\


 \begin{table*}[t!]
 \caption{Evaluation of Open-source models and TUMTraffic-Qwen baseline on the Multi-Choice QA track of the TUMTraffic-VideoQA Dataset, where \textbf{E} represents easy, single-hop questions, and \textbf{H} denotes hard, multi-hop questions.}
\centering
\resizebox{\textwidth}{!}{%
\begin{tabular}{l | l | cc | cc | cc | cc| cc | c}
\midrule
\multirow{2}{*}{\textbf{Models}} & \multirow{2}{*}{\textbf{Category}} & \multicolumn{2}{c}{\textbf{Positioning}} & \multicolumn{2}{c}{\textbf{Counting}} & \multicolumn{2}{c}{\textbf{Motion}} & \multicolumn{2}{c}{\textbf{Class}} & \multicolumn{2}{c}{\textbf{Existence}} & \multirow{2}{*}{\textbf{Overall}} \\
 & & \textbf{E} & \textbf{H} & \textbf{E} & \textbf{H} & \textbf{E} & \textbf{H} & \textbf{E} & \textbf{H} & \textbf{E} & \textbf{H}  \\
\midrule

\multicolumn{13}{c}{Open-Source Models} \\ 
\midrule

\multirow{1}{*}{LLAVA-OneVision \cite{li2024llavaonevisioneasyvisualtask} } &  0.5B & 42.10 & 25.26 & 27.62 &  30.45 & 54.87 & 37.04 & \textbf{57.06} & 39.57 & \textbf{85.29} & 58.35 & 45.82 \\

\rowcolor{gray!10}
& 7B & \textbf{46.92} & 22.03 & \textbf{69.42} & \textbf{54.85} & 61.14 & \textbf{60.48} & 51.92 & \textbf{56.50} & 77.08 & 63.25 & \textbf{56.36} \\
\midrule

\multirow{1}{*}{Qwen2-VL \cite{Qwen-VL}} &
  2B & 36.73 & \textbf{26.05} & 38.10 & 39.78 & 56.46 & 35.19 & 32.10 & 38.49 & 68.87 & 67.32 & 43.91 \\

& 7B & 36.03 & 24.35 & 66.91 & 49.11 & \textbf{61.65} & 38.10 & 44.83 & 40.20 & 54.00 & \textbf{73.03} & 48.82 \\
\midrule

\multirow{1}{*}{VideoLLaMA2 \cite{cheng2024videollama2advancingspatialtemporal}}  & 2.0-7B-8F & 42.54 & 18.14 & 44.13 & 37.56 & 59.37 & 35.87 & 39.05 & 44.07 & 44.56 & 65.56 & 43.09 \\

& 2.0-7B-16F & 42.41 & 10.47 & 55.98 & 41.94 & 53.80 & 52.26 & 44.16 & 47.75 & 66.93 & 64.82 & 48.05 \\
\midrule

\multicolumn{13}{c}{TUMTraffic-VideoQA Baseline} \\ 
\midrule
\multirow{4}{*}{Baseline-0.5B (Ours)} & Spatial Pooling  & 75.54 & 68.47 & 85.31 & 75.82 & 83.92 & \textbf{81.26} & 79.95 & 59.73 & 93.06 & 85.37 & 78.84 \\

 & MultiRes Spatial-Pooling & 76.36 & 69.32 & 86.10 & 75.86 & 83.73 & 79.59 & \textbf{80.57} & 61.70 & 92.73 & 85.37 & 79.07 \\
 
& MultiRes Token-Pruning  &  \textbf{76.61} & 73.40 & \textbf{86.33} & 76.88 & 83.48 & 78.60 & 80.01 & 60.43 & \textbf{93.34} & 85.27 & 79.44 \\

\rowcolor{gray!10}
& MultiRes Temporal-Pooling & 75.85 & \textbf{74.07} & 85.65 & \textbf{76.92} & \textbf{84.05} & 80.64 & 80.26 & \textbf{62.21} & 93.06 & \textbf{85.55} & \textbf{79.83} \\
\midrule

\multirow{4}{*}{Baseline-7B (Ours)} & Spatial Pooling & 76.99 & 76.14 & 87.07 & 76.81 & 86.58 & 82.07 & 82.72 & 64.11 & 93.62 & 85.27 & 81.14 \\

& MultiRes Spatial-Pooling  & \textbf{78.89} & 76.99 & 87.07 & 77.49 & \textbf{88.29} & 81.82 & \textbf{83.52} & \textbf{65.95} & 93.01 & \textbf{85.51} & 81.85 \\

& MultiRes Token-Pruning & 76.93 & \textbf{77.24} & 87.41 & 77.76 & 86.46 & 80.64 & 82.66 & 65.00 & \textbf{93.84} & 85.48 & 81.34 \\

\rowcolor{gray!10}
& MultiRes Temporal-Pooling & 78.57 & \textbf{77.24} & \textbf{87.53} & \textbf{78.22} & 87.09 & \textbf{82.68} & 83.33 & 65.76 & 93.78 & 85.34 & \textbf{81.95} \\
\midrule
\end{tabular}%
}

\label{table:consolidated_metrics}
\vspace{-2pt}

\end{table*}

{\small
\begin{equation}
S_v(Z_v) = [ Z_{\text{high}}^1, Z_{\text{learn}}, [ Z_{\text{pruned}}^{n}, Z_{\text{learn}} ]_{n=2}^N ]
\label{formula:mluti-spatial_spar}
\end{equation}}



\noindent\textbullet\  \textbf{MultiRes Temporal Pooling}: In this strategy, the keyframe set is adaptively queried by input questions $\mathcal{K}(\cdot)=\mathcal{Q}(X_q)$. Based on the temporal regions of interest derived from the question, $K$ keyframes are selected, which are preserved with high-resolution representations $Z_{\text{high}}^{n}$. Meanwhile, the remaining frames undergo spatial pooling, resulting in $Z_{\text{low}}^{n}$, as expressed in Eq. \ref{formula:mluti-temporal}. Typically, $K \leq 2$, and for general questions without specific temporal focus, the first frame is set as the default keyframe.



% [x] formula consistent & outside.

{\small
\begin{equation}
\begin{split}
S_v(Z_v) = [ Z_{v}^{n}, Z_{\text{learn}} ]_{n=1}^N  \\
\text{where } Z_v^{n} =
\begin{cases}
Z_{\text{high}}^{n}, & \text{if } n \in \mathcal{K}(\cdot), \\
Z_{\text{low}}^{n}, & \text{if } n \notin \mathcal{K}(\cdot)
\end{cases}
\end{split}
\label{formula:mluti-temporal}
\end{equation}}




% \textbf{Large Language Models.} We use Qwen2 as the pre-trained LLMs in the TUMTraffic-VideoQA baseline with its strong in-context learning ability and proven performances in cross-modality visual question answering. Specifically, we adopt the lightweight Qwen2-0.5B-Instruction model \cite{qwen2} with hidden size of 896 and 32k context length.  


\noindent\textbf{Large Language Model.} We adopt Qwen-2 \cite{qwen2} as the pre-trained LLM in our TUMTraffic-Qwen baseline. Qwen-2 demonstrates strong capabilities in in-context learning and instruction following, supporting context lengths of up to 32k tokens. This allows for the processing of complex and long-form inputs effectively. We utilize two versions of Qwen-2, namely 0.5B and 7B, to establish baselines of different scales. The answer generation process in our TUMTraffic-Qwen baseline model is formulated as:

{\small
\begin{equation}
p(X_a \mid S_v(Z_v), X_q) = \prod_{t=1}^{\mathcal{T}} P_{\phi,\psi}\big(x_t \mid x_{1:t-1}, S_v(Z_v), X_q)
\end{equation}}

% The 0.5B model features a 24-layer Transformer with a hidden size of 896, offering a lightweight yet effective solution. 7B model, with a 28-layer Transformer and a hidden size of 3584, is designed for enhanced reasoning and representation capabilities. 
% We adopt the instruction-tuned versions of Qwen-2 as the pre-trained LLM for our baseline.



\subsection{Baseline Training} Our baseline model undergoes a two-stage training process consisting of video-language alignment and visual instruction fine-tuning, to enhance its understanding of traffic scenarios and reasoning capabilities for long videos. Both stages are trained with 4 NVIDIA A100 GPUs. 

\noindent\textbf{Video-Language Alignment.} This step aims to align video representations with language embeddings, ensuring that the LLM can effectively interpret the visual features. We freeze both the visual encoder and the LLM, and train only the projector layer. To facilitate the training, we initialize the parameters of the 2-layer MLP from the LLaVA-OneVision model, which has been pre-aligned with large-scale cross-modality datasets, including 3.2M single-image and 1.6M OneVision image-caption pairs. In this stage, we further train the projector on raw TUMTraffic-VideoQA data, with open-ended captioning pairs without transforming to the multiple-choice QA for 1 epoch. 
% format

\noindent\textbf{Visual Instruction Fine-Tuning.} Building upon the robust representations established during the alignment stage, we further fine-tune our baseline model on the training set of TUMTraffic-VideoQA. The multi-choice QA pairs are reformatted into the instruction-following format to prompt the model to generate the corresponding answers. During this stage, we freeze the vision encoder and projector layers and finetune the Qwen-2 model with full-parameter fine-tuning to adapt its reasoning and contextual understanding ability. The model is fine-tuned for 1 epoch.





\section{Experiments}
\label{sec: experiments}

\subsection{Experimental Setup}
\label{sec: experimental_setup}
\begin{figure}[t]
\centering \includegraphics[width=\linewidth]{figure_2.png} \caption{The handheld platform configuration, including the radar, IMU, and onboard computer. The experiments are conducted in a room equipped with a motion capture system to obtain accurate ground truth.}
\label{fig2}
\end{figure}

We conduct experiments using three datasets, comprising a total of 15 sequences. One is our self-collected dataset, captured with a handheld platform as shown in Fig.~\ref{fig2}, while the other two are public radar datasets: ICINS2021~\cite{9470842}, and ColoRadar~\cite{kramer2022coloradar}. The sensors on our platform include a 4D FMCW radar, specifically the Texas Instruments AWR1843BOOST, and an Xsens MTI-670-DK IMU. No additional hardware triggers are used between the sensors, and the sensor data is recorded using an Intel NUC i7 onboard computer. The experiments are conducted in an indoor area equipped with a motion capture system to obtain precise ground truth. The extrinsic calibration between the IMU and the radar is performed manually. To highlight the significance of temporal calibration in RIO, we design the dataset with two levels of difficulty. Sequences 1 to 3 feature standard motion patterns, while Sequences 4 to 7 introduce more rotational motion to induce larger errors due to the time offset, providing a clearer demonstration of its impact.

\begin{figure*}[t]
\centering
\includegraphics[width=\linewidth]{figure_3.png}
\caption{Comparison of estimated trajectories with the ground truth. The \textcolor{black}{black} trajectory is the ground truth, the \textcolor{blue}{blue} one is the EKF-RIO, which does not account for temporal calibration, and the \textcolor{red}{red} one is the proposed RIO with online temporal calibration. Results are presented for Sequence 4, ICINS 1, and ColoRadar 1, representing one sequence from each of the three datasets.}
\label{trajectory}
\end{figure*}

In~\cite{9470842}, the ICINS2021 dataset is collected using a Texas Instruments IWR6843AOP radar sensor, an Analog Devices ADIS16448 IMU sensor, and a camera. A microcontroller board is used for active hardware triggering to accurately capture the timing of the radar measurements. Data is collected using both handheld and drone platforms. The handheld sequences, ``carried\_1'' and ``carried\_2'', are referred to as ``ICINS 1'' and ``ICINS 2'', while the drone sequences, ``flight\_1'' and ``flight\_2'', are referred to as ``ICINS 3'' and ``ICINS 4'', respectively. The ground truth is provided through visual-inertial SLAM, which performs multiple loop closures, offering a pseudo-ground truth. In~\cite{kramer2022coloradar}, the ColoRadar dataset is collected using a Texas Instruments AWR1843BOOST radar sensor, a Microstrain 3DM-GX5-25 IMU sensor, and a LiDAR mounted on a handheld platform. No specific synchronization setup is used between the sensors. The sequences, ``arpg\_lab\_run0'' and ``arpg\_lab\_run1'', are referred to as ``ColoRadar 1'' and ``ColoRadar 2'', while the sequences ``ec\_hallways\_run0'' and ``ec\_hallways\_run1'' are referred to as ``ColoRadar 3'' and ``ColoRadar 4'', respectively. The ground truth is generated via LiDAR-inertial SLAM, which includes loop closures, offering a pseudo-ground truth.
\subsection{Evaluation}
\label{sec: evaluation}

\begin{table}[t]
\centering
\caption{Quantitative Results of Fixed Offset and Online Estimation}
\label{fixed_offset}
\resizebox{\linewidth}{!}{
\begin{tblr}{
  cells = {c},
  cell{1}{1} = {r=2}{},
  cell{1}{2} = {r=2}{},
  cell{1}{3} = {r=2}{},
  cell{1}{4} = {c=2}{},
  cell{1}{6} = {c=2}{},
  cell{3}{1} = {r=6}{},
  cell{3}{2} = {r=5}{},
  cell{3}{5} = {fg=red},
  cell{4}{4} = {fg=red},
  cell{5}{4} = {fg=blue},
  cell{5}{5} = {fg=blue},
  cell{5}{6} = {fg=blue},
  cell{5}{7} = {fg=red},
  cell{6}{6} = {fg=red},
  cell{6}{7} = {fg=blue},
  cell{9}{1} = {r=6}{},
  cell{9}{2} = {r=5}{},
  cell{11}{4} = {fg=red},
  cell{11}{5} = {fg=blue},
  cell{11}{6} = {fg=red},
  cell{11}{7} = {fg=red},
  cell{12}{4} = {fg=blue},
  cell{12}{5} = {fg=red},
  cell{12}{6} = {fg=blue},
  cell{12}{7} = {fg=blue},
  hline{1,3,9,15} = {-}{},
  hline{2} = {4-7}{},
}
\textbf{Sequence} & \textbf{Method} &  \textbf{Time Offset (s)}            & \textbf{APE RMSE} &                & \textbf{RPE RMSE} &                   \\
                  &                 &                                      & Trans. (m)        & Rot. (\degree) & Trans. (m)        & Rot. (\degree)    \\
                  \hline
Sequence 1        & Fixed Offset    & 0.0             & 0.985             & 1.872          & 0.264             & 1.230          \\
                  &                 & -0.05           & 0.647             & 7.561          & 0.166             & 1.549          \\
                  &                 & -0.10           & 0.661             & 2.438          & 0.138             & 0.948          \\
                  &                 & -0.15           & 0.826             & 5.151          & \textbf{0.131}    & 1.196          \\
                  &                 & -0.20           & 0.974             & 2.698          & 0.156             & 1.274          \\
                  & Online Est.     & \textbf{-0.114} & \textbf{0.646}    & \textbf{0.935} & 0.132    & \textbf{0.774} \\
Sequence 4        & Fixed Offset    & 0.0             & 1.737             & 25.885         & 0.118             & 4.074          \\
                  &                 & -0.05           & 1.028             & 15.460         & 0.091             & 2.313          \\
                  &                 & -0.10           & 0.635             & 4.655          & 0.061             & 0.994          \\
                  &                 & -0.15           & 0.649             & 4.275          & 0.068             & 1.083          \\
                  &                 & -0.20           & 0.716             & 12.461         & 0.092             & 2.526          \\
                  & Online Est.     & \textbf{-0.115} & \textbf{0.610}    & \textbf{3.099} & \textbf{0.057}    & \textbf{0.944} 
\end{tblr}
}
\vspace{0.3em}
{\raggedright
\noindent\par {\footnotesize \textsuperscript{*}The initial time offset of `Online Est.' is set to 0.0 and the converged values are shown above.}
\noindent\par {\footnotesize \textsuperscript{**}For each sequence, the lowest error values among the fixed offsets are highlighted in \textcolor{red}{red}, and the second-lowest in \textcolor{blue}{blue}.}
\par}

\end{table}
For the performance comparison, the open-source EKF-RIO \cite{9235254}, which uses the same measurement model but does not account for temporal calibration, is employed. All parameters are kept identical to ensure a fair comparison. In the proposed method, the time offset \( t_d \) is initialized to 0.0 seconds for all sequences, reflecting a typical scenario where the initial time offset is unknown. The experimental results are evaluated using the open-source tool EVO \cite{grupp2017evo}. Figure~\ref{trajectory} illustrates the estimated trajectories compared to the ground truth for visual comparison, with one representative result from each dataset. Due to the stochastic nature of the RANSAC algorithm used in radar ego-velocity estimation, the averaged results from 100 trials across all datasets are presented. We compare the root mean square error (RMSE) of both absolute pose error (APE) and relative pose error (RPE), with the RPE calculated at 10-meter intervals.

APE evaluates the overall trajectory by calculating the difference between the ground truth and the estimated poses for all frames, making it particularly useful for assessing the global accuracy of the estimated trajectory. However, APE can be sensitive to significant rotational errors that occur early or in specific sections, potentially overshadowing smaller errors later in the trajectory. In contrast, RPE focuses on local accuracy by aligning poses at regular intervals and calculating the error, allowing discrepancies over shorter segments to be highlighted. When the temporal calibration between sensors is not accounted for, errors can accumulate over time, making RPE evaluation essential. Both metrics offer valuable insights, providing a comprehensive evaluation of the trajectory.

\subsubsection{Self-Collected Dataset}
The purpose of the self-collected dataset is to identify the actual time offset between the IMU and the radar and evaluate its impact on the accuracy of RIO. Since the handheld platform does not utilize a hardware trigger to synchronize the sensors, the exact time offset is unknown and must be estimated. To address this uncertainty, we evaluate the performance of fixed time offsets over a range of values to determine the interval that provides the best accuracy and estimate the likely time offset range.

As shown in Table \ref{fixed_offset}, error values are analyzed with fixed offsets set at 0.05-second intervals for both Sequence 1 and Sequence 4, which feature different motion patterns. The results show that the time offset falls within the -0.10 to -0.15 second range, where the highest accuracy in terms of APE and RPE is observed for both sequences. The proposed method, which utilizes online temporal calibration, estimates the time offset as -0.114 seconds for Sequence 1 and -0.115 seconds for Sequence 4, closely matching the range found through fixed offset testing. In both cases, the proposed method achieves improved performance in terms of both APE and RPE, demonstrates its effectiveness in accurately estimating the time offset.

\begin{table}[t]
\centering
\caption{Quantitative Results of Comparison study on Self-collected dataset}
\label{table_self}
\resizebox{\linewidth}{!}{
\begin{tblr}{
  cells = {c},
  cell{1}{1} = {r=2}{},
  cell{1}{2} = {r=2}{},
  cell{1}{3} = {c=2}{},
  cell{1}{5} = {c=2}{},
  cell{3}{1} = {r=2}{},
  cell{5}{1} = {r=2}{},
  cell{7}{1} = {r=2}{},
  cell{9}{1} = {r=2}{},
  cell{11}{1} = {r=2}{},
  cell{13}{1} = {r=2}{},
  cell{15}{1} = {r=2}{},
  cell{17}{1} = {r=2}{},
  hline{1,3,5,7,9,11,13,15,17,19} = {-}{},
  hline{2} = {3-6}{},
}
{\textbf{Sequence }\\\textbf{(Trajectory Length)}} & {\textbf{Method } \textbf{($\hat{t}_d$)}} & \textbf{APE RMSE } &                & \textbf{RPE RMSE } &                \\
                                                   &                                         & Trans. (m)         & Rot. (\degree)        & Trans. (m)         & Rot. (\degree)        \\
                                                   \hline
{Sequence 1\\(177 m)}                              & {EKF-RIO (N/A)}                        & 0.985              & 1.872           & 0.264              & 1.230          \\
                                                   & {Ours (-0.114 s)}                      & \textbf{0.646}     & \textbf{0.935}  & \textbf{0.132}     & \textbf{0.774} \\
{Sequence 2\\(197 m)}                              & {EKF-RIO}                              & 2.269              & 2.161           & 0.136              & 1.414          \\
                                                   & {Ours (-0.114 s)}                      & \textbf{0.587}     & \textbf{1.650}  & \textbf{0.064}     & \textbf{0.774} \\
{Sequence 3\\(144 m)}                              & {EKF-RIO}                              & 1.368              & 2.331           & 0.167              & 1.347          \\
                                                   & {Ours (-0.113 s)}                      & \textbf{0.414}     & \textbf{1.140}  & \textbf{0.088}     & \textbf{0.613} \\
{Sequence 4\\(197 m)}                              & {EKF-RIO}                              & 1.737              & 25.885          & 0.118              & 4.074          \\
                                                   & {Ours (-0.115 s)}                      & \textbf{0.610}     & \textbf{3.099}  & \textbf{0.057}     & \textbf{0.944} \\
{Sequence 5\\(190 m)}                              & {EKF-RIO}                              & 2.375              & 7.702           & 0.122              & 1.600          \\
                                                   & {Ours (-0.115 s)}                      & \textbf{1.150}     & \textbf{1.304}  & \textbf{0.069}     & \textbf{0.814} \\
{Sequence 6\\(179 m)}                              & {EKF-RIO}                              & 1.267              & 17.907          & 0.117              & 2.828          \\
                                                   & {Ours (-0.111 s)}                      & \textbf{0.661}     & \textbf{2.551}  & \textbf{0.051}     & \textbf{0.809} \\
{Sequence 7\\(223 m)}                              & {EKF-RIO}                              & 2.757              & 10.092          & 0.116              & 1.863          \\
                                                   & {Ours (-0.112 s)}                      & \textbf{1.596}     & \textbf{6.039}  & \textbf{0.057}     & \textbf{1.365} \\
{Average}                                          & {EKF-RIO}                              & 1.822              & 9.707            & 0.148             & 2.051          \\
                                                   & {Ours (-0.113 s)}                      & \textbf{0.809}     & \textbf{2.388}   & \textbf{0.074}    & \textbf{0.870}   
\end{tblr}
}
\end{table}

Since the radar delay is generally larger than IMU delay, the time offset \( t_d \), representing the difference between these delays, typically takes a negative value. To evaluate the robustness of the estimation, different initial values of \( t_d \) ranging from 0.0 to -0.3 seconds are tested. Figure \ref{sq5} illustrates the estimated time offset for each initial setting, along with the 3-sigma boundaries. As \( t_d \) is estimated from radar ego-velocity, it cannot be determined while the platform is stationary. Once the platform starts moving, the filter begins estimating \( t_d \) and quickly converges to a stable value. The filter converges to a stable time offset of -0.114 ± 0.001 seconds in Sequence 1 and -0.115 ± 0.001 seconds in Sequence 4.

Table \ref{table_self} presents the performance comparison between the proposed method with online temporal calibration and EKF-RIO across seven sequences. The proposed method outperforms EKF-RIO, significantly reducing both APE and RPE across all sequences. Specifically, it reduces APE translation error by an average of 56\%, APE rotation error by 75\%, RPE translation error by 50\%, and RPE rotation error by 58\% compared with EKF-RIO. Despite using the same measurement model, the performance improvement is achieved solely by applying propagation and updates based on a common time stream through the proposed online temporal calibration.

On average, the time offset \( t_d \) is estimated to be -0.113 ± 0.002 seconds, confirming consistent temporal calibration throughout the experiments. Compared with LiDAR-inertial and visual-inertial systems, radar-inertial systems exhibit a significantly larger time offset, as shown in Table~\ref{time_offset_comparison}. Given the radar sensor rate (10 Hz), such a large time offset is significant enough to cause a misalignment spanning more than one data frame. These findings highlight the necessity of temporal calibration in RIO, which is crucial for accurate sensor fusion and reliable pose estimation in real-world applications.

\begin{figure}[t]
\centering
\includegraphics[width=\linewidth]{figure_4.png}
\caption{Time offset estimation with 3-sigma boundaries for different initial values in Sequence 1 and 4.}
\label{sq5}
\end{figure}

\begin{table}[t]
\centering
\caption{Comparison of Time Offset in Multi-Sensor Fusion Systems}
\label{time_offset_comparison}
\begin{tabular}{|c|c|c|} 
\hline
\textbf{Systems} & \textbf{Sensor} & \textbf{Time Offset} \\ 
\hline
LiDAR-Inertial~\cite{10113826} & Velodyne VLP-32 & 0.006 s\\ 
\hline
Visual-Inertial~\cite{li2014online} & PointGrey Bumblebee2 & 0.047 s\\ 
\hline
Radar-Inertial & TI AWR1843BOOST & \textbf{0.113 s} \\
\hline
\end{tabular}
\end{table}

\subsubsection{Open Datasets}
Table \ref{opendataset} presents the results from the two open datasets. The ICINS dataset incorporates a hardware trigger for the radar, which we use to validate the accuracy of the time offset estimation for the proposed method. In this setup, a microcontroller sends radar trigger signals, prompting the radar to begin scanning. The radar data is timestamped based on the actual trigger signal, providing a pseudo-ground truth for time offset estimation. Theoretically, if the sensors are time-synchronized through triggers, the time offset \( t_d \) is expected to be close to 0.0 seconds. The proposed method estimates the time offset to be an average of 0.016 ± 0.003 seconds. Despite this slight discrepancy, the proposed method demonstrates comparable or improved performance on average in both APE and RPE compared with EKF-RIO. Although the ICINS dataset includes hardware-triggered signals for the radar, there is no such trigger signal for the IMU in the dataset, which may introduce a delay in IMU measurements. As defined in Eq.~\eqref{time_offset}, we attribute the estimated positive time offset to this IMU delay, explaining the difference from the expected value.

The ColoRadar dataset, widely used for performance comparison in the RIO field, is utilized to assess if the proposed method generalizes well across different datasets. As shown in Table \ref{opendataset}, the proposed method also demonstrates performance improvements over EKF-RIO in terms of both APE and RPE on average. However, the extent of improvement is smaller compared with the self-collected dataset, which can be explained by differences in trajectory characteristics. The radar ego-velocity model utilizes not only the accelerometer but also the gyroscope measurements. As illustrated in Fig.~\ref{trajectory}, the ColoRadar dataset involves movement over a larger area with less rotation, leading to a smaller impact of the time offset on performance. Nonetheless, the proposed method achieves 33\% reduction in RPE translation error, demonstrating its effectiveness even in this less challenging trajectory. On average, the time offset \( t_d \) is estimated to be -0.111 ± 0.003 seconds, similar to the time offset found in the self-collected dataset. This consistency is likely due to the use of the same radar sensor model in both datasets, further validating the reliability of the proposed method across different environments.

\begin{table}[t]
\centering
\caption{Quantitative Results of Comparison study on Open datasets}
\label{opendataset}
\resizebox{\linewidth}{!}{
\begin{tblr}{
  cells = {c},
  cell{1}{1} = {r=2}{},
  cell{1}{2} = {r=2}{},
  cell{1}{3} = {c=2}{},
  cell{1}{5} = {c=2}{},
  cell{3}{1} = {r=2}{},
  cell{5}{1} = {r=2}{},
  cell{7}{1} = {r=2}{},
  cell{9}{1} = {r=2}{},
  cell{11}{1} = {r=2}{},
  cell{13}{1} = {r=2}{},
  cell{15}{1} = {r=2}{},
  cell{17}{1} = {r=2}{},
  cell{19}{1} = {r=2}{},
  cell{21}{1} = {r=2}{},
  hline{1,3,5,7,9,11,13,15,17,19,21,23} = {-}{},
  hline{2-3} = {3-6}{},
}
{\textbf{Sequence }\\\textbf{(Trajectory Length)}}       & \textbf{Method ($\hat{t}_d$)} & \textbf{APE RMSE}        &                                           & \textbf{RPE RMSE}       &                         \\
                        &                               & Trans. (m)               & Rot. (\degree)                                   & Trans. (m)              & Rot. (\degree)                 \\
                        \hline
{ICINS 1\\(295 m)}      & EKF-RIO (N/A)                 & 1.959                    & 10.694                                    & \textbf{0.093}          & \textbf{0.896}          \\
                        & Ours (0.016 s)                & \textbf{1.922}           & \textbf{10.135}                           & 0.098                   & 0.918          \\
{ICINS 2\\(468 m)}      & EKF-RIO                       & 3.830                    & 23.151                                    & \textbf{0.114}          & 1.289                   \\
                        & Ours (0.013 s)                & \textbf{3.198}           & \textbf{19.235}                           & 0.121                   & \textbf{1.076}          \\
{ICINS 3\\(150 m)}      & EKF-RIO                       & \textbf{1.502}           & \textbf{9.905}                            & 0.130                   & \textbf{1.512}           \\
                        & Ours (0.015 s)                & 1.530                    & 10.189                                    & \textbf{0.126}          & 1.553          \\
{ICINS 4\\(50 m)}       & EKF-RIO                       & \textbf{0.213}           & \textbf{2.091}                            & \textbf{0.076}          & \textbf{0.923}           \\
                        & Ours (0.019 s)                & 0.216                    & 2.098                                     & 0.081                   & \textbf{0.923}          \\
Average                 & EKF-RIO                       & 1.876                    & 11.460                                    & \textbf{0.103}          & 1.155                   \\
                        & Ours (0.016 s)                & \textbf{1.716}           & \textbf{10.414}                           & 0.106                   & \textbf{1.117}          \\
                        \hline
{ColoRadar 1\\(178 m) } & EKF-RIO (N/A)                 & 6.556                    & \textbf{\textbf{1.354}}                   & 0.182                   & \textbf{1.071} \\
                        & Ours (-0.110 s)               & \textbf{\textbf{6.173}}  & 1.382                                     & \textbf{\textbf{0.155}} & 1.188                   \\
{ColoRadar 2\\(197 m) } & EKF-RIO                       & \textbf{\textbf{4.747}}  & 1.238                                     & 0.372                   & 1.375                   \\
                        & Ours (-0.114 s)               & 4.826                    & \textbf{\textbf{0.960}}                   & \textbf{\textbf{0.292}} & \textbf{\textbf{1.180}} \\
{ColoRadar 3\\(197 m) } & EKF-RIO                       & \textbf{\textbf{8.307}}  & 1.969                                     & 0.259                   & 1.015                   \\
                        & Ours (-0.108 s)               & 8.550                    & \textbf{\textbf{1.852}}                   & \textbf{\textbf{0.221}} & \textbf{\textbf{0.879}} \\
{ColoRadar 4\\(144 m) } & EKF-RIO                       & 12.111                   & 2.815                                     & 0.488                   & 1.263                   \\
                        & Ours (-0.112 s)               & \textbf{11.946}          & \textbf{2.756}                            & \textbf{0.200}          & \textbf{1.116} \\
Average                 & EKF-RIO                       & 7.930                    & 1.844                                     & 0.325                   & 1.181                   \\
                        & Ours(-0.111 s)                & \textbf{7.874}           & \textbf{1.737}                            & \textbf{0.217}          & \textbf{1.091}          
\end{tblr}
}
\end{table}

In this paper, we systematically investigate the position bias problem in the multi-constraint instruction following. To quantitatively measure the disparity of constraint order, we propose a novel Difficulty Distribution Index (CDDI). Based on the CDDI, we design a probing task. First, we construct a large number of instructions consisting of different constraint orders. Then, we conduct experiments in two distinct scenarios. Extensive results reveal a clear preference of LLMs for ``hard-to-easy'' constraint orders. To further explore this, we conduct an explanation study. We visualize the importance of different constraints located in different positions and demonstrate the strong correlation between the model's attention distribution and its performance.
{
    \small
    \bibliographystyle{ieeenat_fullname}
    \bibliography{main}
}

% WARNING: do not forget to delete the supplementary pages from your submission 
\section{Secure Token Pruning Protocols}
\label{app:a}
We detail the encrypted token pruning protocols $\Pi_{prune}$ in Figure \ref{fig:protocol-prune} and $\Pi_{mask}$ in Figure \ref{fig:protocol-mask} in this section.

%Optionally include supplemental material (complete proofs, additional experiments and plots) in appendix.
%All such materials \textbf{SHOULD be included in the main submission.}
\begin{figure}[h]
%vspace{-0.2in}
\begin{protocolbox}
\noindent
\textbf{Parties:} Server $P_0$, Client $P_1$.

\textbf{Input:} $P_0$ and $P_1$ holds $\{ \left \langle Att \right \rangle_{0}^{h}, \left \langle Att \right \rangle_{1}^{h}\}_{h=0}^{H-1} \in \mathbb{Z}_{2^{\ell}}^{n\times n}$ and $\left \langle x \right \rangle_{0}, \left \langle x \right \rangle_{1} \in \mathbb{Z}_{2^{\ell}}^{n\times D}$ respectively, where H is the number of heads, n is the number of input tokens and D is the embedding dimension of tokens. Additionally, $P_1$ holds a threshold $\theta \in \mathbb{Z}_{2^{\ell}}$.

\textbf{Output:} $P_0$ and $P_1$ get $\left \langle y \right \rangle_{0}, \left \langle y \right \rangle_{1} \in \mathbb{Z}_{2^{\ell}}^{n'\times D}$, respectively, where $y=\mathsf{Prune}(x)$ and $n'$ is the number of remaining tokens.

\noindent\rule{13.2cm}{0.1pt} % This creates the horizontal line
\textbf{Protocol:}
\begin{enumerate}[label=\arabic*:, leftmargin=*]
    \item For $h \in [H]$, $P_0$ and $P_1$ compute locally with input $\left \langle Att \right \rangle^{h}$, and learn the importance score in each head $\left \langle s \right \rangle^{h} \in \mathbb{Z}_{2^{\ell}}^{n} $, where $\left \langle s \right \rangle^{h}[j] = \frac{1}{n} \sum_{i=0}^{n-1} \left \langle Att \right \rangle^{h}[i,j]$.
    \item $P_0$ and $P_1$ compute locally with input $\{ \left \langle s \right \rangle^{i} \in \mathbb{Z}_{2^{\ell}}^{n}  \}_{i=0}^{H-1}$, and learn the final importance score $\left \langle S \right \rangle \in \mathbb{Z}_{2^{\ell}}^{n}$ for each token, where  $\left \langle S \right \rangle[i] = \frac{1}{H} \sum_{h=0}^{H-1} \left \langle s \right \rangle^{h}[i]$.
    \item  For $i \in [n]$, $P_0$ and $P_1$ invoke $\Pi_{CMP}$ with inputs  $\left \langle S \right \rangle$ and $ \theta $, and learn  $\left \langle M \right \rangle \in \mathbb{Z}_{2^{\ell}}^{n}$, such that$\left \langle M \right \rangle[i] = \Pi_{CMP}(\left \langle S \right \rangle[i] - \theta) $, where: \\
    $M[i] = \begin{cases}
        1  &\text{if}\ S[i] > \theta, \\
        0  &\text{otherwise}.
            \end{cases} $
    % \item If the pruning location is insensitive, $P_0$ and $P_1$ learn real mask $M$ instead of shares $\left \langle M \right \rangle$. $P_0$ and $P_1$ compute $\left \langle y \right \rangle$ with input $\left \langle x \right \rangle$ and $M$, where  $\left \langle x \right \rangle[i]$ is pruned if $M[i]$ is $0$.
    \item $P_0$ and $P_1$ invoke $\Pi_{mask}$ with inputs  $\left \langle x \right \rangle$ and pruning mask $\left \langle M \right \rangle$, and set outputs as $\left \langle y \right \rangle$.
\end{enumerate}
\end{protocolbox}
\setlength{\abovecaptionskip}{-1pt} % Reduces space above the caption
\caption{Secure Token Pruning Protocol $\Pi_{prune}$.}
\label{fig:protocol-prune}
\end{figure}




\begin{figure}[h]
\begin{protocolbox}
\noindent
\textbf{Parties:} Server $P_0$, Client $P_1$.

\textbf{Input:} $P_0$ and $P_1$ hold $\left \langle x \right \rangle_{0}, \left \langle x \right \rangle_{1} \in \mathbb{Z}_{2^{\ell}}^{n\times D}$ and  $\left \langle M \right \rangle_{0}, \left \langle M \right \rangle_{1} \in \mathbb{Z}_{2^{\ell}}^{n}$, respectively, where n is the number of input tokens and D is the embedding dimension of tokens.

\textbf{Output:} $P_0$ and $P_1$ get $\left \langle y \right \rangle_{0}, \left \langle y \right \rangle_{1} \in \mathbb{Z}_{2^{\ell}}^{n'\times D}$, respectively, where $y=\mathsf{Prune}(x)$ and $n'$ is the number of remaining tokens.

\noindent\rule{13.2cm}{0.1pt} % This creates the horizontal line
\textbf{Protocol:}
\begin{enumerate}[label=\arabic*:, leftmargin=*]
    \item For $i \in [n]$, $P_0$ and $P_1$ set $\left \langle M \right \rangle$ to the MSB of $\left \langle x \right \rangle$ and learn the masked tokens $\left \langle \Bar{x} \right \rangle \in Z_{2^{\ell}}^{n\times D}$, where
    $\left \langle \Bar{x}[i] \right \rangle = \left \langle x[i] \right \rangle + (\left \langle M[i] \right \rangle << f)$ and $f$ is the fixed-point precision.
    \item $P_0$ and $P_1$ compute the sum of $\{\Pi_{B2A}(\left \langle M \right \rangle[i]) \}_{i=0}^{n-1}$, and learn the number of remaining tokens $n'$ and the number of tokens to be pruned $m$, where $m = n-n'$.
    \item For $k\in[m]$, for $i\in[n-k-1]$, $P_0$ and $P_1$ invoke $\Pi_{msb}$ to learn the highest bit of $\left \langle \Bar{x}[i] \right \rangle$, where $b=\mathsf{MSB}(\Bar{x}[i])$. With the highest bit of $\Bar{x}[i]$, $P_0$ and $P_1$ perform a oblivious swap between $\Bar{x}[i]$ and $\Bar{x}[i+1]$:
    $\begin{cases}
        \Tilde{x}[i] = b\cdot \Bar{x}[i] + (1-b)\cdot \Bar{x}[i+1] \\
        \Tilde{x}[i+1] = b\cdot \Bar{x}[i+1] + (1-b)\cdot \Bar{x}[i]
    \end{cases} $ \\
    $P_0$ and $P_1$ learn the swapped token sequence $\left \langle \Tilde{x} \right \rangle$.
    \item $P_0$ and $P_1$ truncate $\left \langle \Tilde{x} \right \rangle$ locally by keeping the first $n'$ tokens, clear current MSB (all remaining token has $1$ on the MSB), and set outputs as $\left \langle y \right \rangle$.
\end{enumerate}
\end{protocolbox}
\setlength{\abovecaptionskip}{-1pt} % Reduces space above the caption
\caption{Secure Mask Protocol $\Pi_{mask}$.}
\label{fig:protocol-mask}
%\vspace{-0.2in}
\end{figure}

% \begin{wrapfigure}{r}{0.35\textwidth}  % 'r' for right, and the width of the figure area
%   \centering
%   \includegraphics[width=0.35\textwidth]{figures/msb.pdf}
%   \caption{Runtime of $\Pi_{prune}$ and $\Pi_{mask}$ in different layers. We compare different secure pruning strategies based on the BERT Base model.}
%   \label{fig:msb}
%   \vspace{-0.1in}
% \end{wrapfigure}

% \begin{figure}[h]  % 'r' for right, and the width of the figure area
%   \centering
%   \includegraphics[width=0.4\textwidth]{figures/msb.pdf}
%   \caption{Runtime of $\Pi_{prune}$ and $\Pi_{mask}$ in different layers. We compare different secure pruning strategies based on the BERT Base model.}
%   \label{fig:msb}
%   % \vspace{-0.1in}
% \end{figure}

\textbf{Complexity of $\Pi_{mask}$.} The complexity of the proposed $\Pi_{mask}$ mainly depends on the number of oblivious swaps. To prune $m$ tokens out of $n$ input tokens, $O(mn)$ swaps are needed. Since token pruning is performed progressively, only a small number of tokens are pruned at each layer, which makes $\Pi_{mask}$ efficient during runtime. Specifically, for a BERT base model with 128 input tokens, the pruning protocol only takes $\sim0.9$s on average in each layer. An alternative approach is to invoke an oblivious sort algorithm~\citep{bogdanov2014swap2,pang2023bolt} on $\left \langle \Bar{x} \right \rangle$. However, this approach is less efficient because it blindly sort the whole token sequence without considering $m$. That is, even if only $1$ token needs to be pruned, $O(nlog^{2}n)\sim O(n^2)$ oblivious swaps are needed, where as the proposed $\Pi_{mask}$ only need $O(n)$ swaps. More generally, for an $\ell$-layer Transformer with a total of $m$ tokens pruned, the overall time complexity using the sort strategy would be $O(\ell n^2)$ while using the swap strategy remains an overall complexity of $O(mn).$ Specifically, using the sort strategy to prune tokens in one BERT Base model layer can take up to $3.8\sim4.5$ s depending on the sorting algorithm used. In contrast, using the swap strategy only needs $0.5$ s. Moreover, alternative to our MSB strategy, one can also swap the encrypted mask along with the encrypted token sequence. However, we find that this doubles the number of swaps needed, and thus is less efficient the our MSB strategy, as is shown in Figure \ref{fig:msb}.

\section{Existing Protocols}
\label{app:protocol}
\noindent\textbf{Existing Protocols Used in Our Private Inference.}  In our private inference framework, we reuse several existing cryptographic protocols for basic computations. $\Pi_{MatMul}$ \citep{pang2023bolt} processes two ASS matrices and outputs their product in SS form. For non-linear computations, protocols $\Pi_{SoftMax}, \Pi_{GELU}$, and $\Pi_{LayerNorm}$\citep{lu2023bumblebee, pang2023bolt} take a secret shared tensor and return the result of non-linear functions in ASS. Basic protocols from~\citep{rathee2020cryptflow2, rathee2021sirnn} are also utilized. $\Pi_{CMP}$\citep{EzPC}, for example, inputs ASS values and outputs a secret shared comparison result, while $\Pi_{B2A}$\citep{EzPC} converts secret shared Boolean values into their corresponding arithmetic values.

\section{Polynomial Reduction for Non-linear Functions}
\label{app:b}
The $\mathsf{SoftMax}$ and $\mathsf{GELU}$ functions can be approximated with polynomials. High-degree polynomials~\citep{lu2023bumblebee, pang2023bolt} can achieve the same accuracy as the LUT-based methods~\cite{hao2022iron-iron}. While these polynomial approximations are more efficient than look-up tables, they can still incur considerable overheads. Reducing the high-degree polynomials to the low-degree ones for the less important tokens can imporve efficiency without compromising accuracy. The $\mathsf{SoftMax}$ function is applied to each row of an attention map. If a token is to be reduced, the corresponding row will be computed using the low-degree polynomial approximations. Otherwise, the corresponding row will be computed accurately via a high-degree one. That is if $M_{\beta}'[i] = 1$, $P_0$ and $P_1$ uses high-degree polynomials to compute the $\mathsf{SoftMax}$ function on token $x[i]$:
\begin{equation}
\mathsf{SoftMax}_{i}(x) = \frac{e^{x_i}}{\sum_{j\in [d]}e^{x_j}}
\end{equation}
where $x$ is a input vector of length $d$ and the exponential function is computed via a polynomial approximation. For the $\mathsf{SoftMax}$ protocol, we adopt a similar strategy as~\citep{kim2021ibert, hao2022iron-iron}, where we evaluate on the normalized inputs $\mathsf{SoftMax}(x-max_{i\in [d]}x_i)$. Different from~\citep{hao2022iron-iron}, we did not used the binary tree to find max value in the given vector. Instead, we traverse through the vector to find the max value. This is because each attention map is computed independently and the binary tree cannot be re-used. If $M_{\beta}[i] = 0$, $P_0$ and $P_1$ will approximate the $\mathsf{SoftMax}$ function with low-degree polynomial approximations. We detail how $\mathsf{SoftMax}$ can be approximated as follows:
\begin{equation}
\label{eq:app softmax}
\mathsf{ApproxSoftMax}_{i}(x) = \frac{\mathsf{ApproxExp}(x_i)}{\sum_{j\in [d]}\mathsf{ApproxExp}(x_j)}
\end{equation}
\begin{equation}
\mathsf{ApproxExp}(x)=\begin{cases}
    0  &\text{if}\ x \leq T \\
    (1+ \frac{x}{2^n})^{2^n} &\text{if}\ x \in [T,0]\\
\end{cases}
\end{equation}
where the $2^n$-degree Taylor series is used to approximate the exponential function and $T$ is the clipping boundary. The value $n$ and $T$ determines the accuracy of above approximation. With $n=6$ and $T=-13$, the approximation can achieve an average error within $2^{-10}$~\citep{lu2023bumblebee}. For low-degree polynomial approximation, $n=3$ is used in the Taylor series.

Similarly, $P_0$ or $P_1$ can decide whether or not to approximate the $\mathsf{GELU}$ function for each token. If $M_{\beta}[i] = 1$, $P_0$ and $P_1$ use high-degree polynomials~\citep{lu2023bumblebee} to compute the $\mathsf{GELU}$ function on token $x[i]$ with high-degree polynomial:
% \begin{equation}
% \mathsf{GELU}(x) = 0.5x(1+\mathsf{Tanh}(\sqrt{2/\pi}(x+0.044715x^3)))
% \end{equation}
% where the $\mathsf{Tanh}$ and square root function are computed via a OT-based lookup-table.

\begin{equation}
\label{eq:app gelu}
\mathsf{ApproxGELU}(x)=\begin{cases}
    0  &\text{if}\ x \leq -5 \\
    P^3(x), &\text{if}\ -5 < x \leq -1.97 \\
    P^6(x), &\text{if}\ -1.97 < x \leq 3  \\
    x, &\text{if}\ x >3 \\
\end{cases}
\end{equation}
where $P^3(x)$ and $P^6(x)$ are degree-3 and degree-6 polynomials respectively. The detailed coefficient for the polynomial is: 
\begin{equation*}
    P^3(x) = -0.50540312 -  0.42226581x - 0.11807613x^2 - 0.01103413x^3
\end{equation*}
, and
\begin{equation*}
    P^6(x) = 0.00852632 + 0.5x + 0.36032927x^2 - 0.03768820x^4 + 0.00180675x^6
\end{equation*}

For BOLT baseline, we use another high-degree polynomial to compute the $\mathsf{GELU}$ function.

\begin{equation}
\label{eq:app gelu}
\mathsf{ApproxGELU}(x)=\begin{cases}
    0  &\text{if}\ x < -2.7 \\
    P^4(x), &\text{if}\   |x| \leq 2.7 \\
    x, &\text{if}\ x >2.7 \\
\end{cases}
\end{equation}
We use the same coefficients for $P^4(x)$ as BOLT~\citep{pang2023bolt}.

\begin{figure}[h]
 % \vspace{-0.1in}
    \centering
    \includegraphics[width=1\linewidth]{figures/bumble.pdf}
    % \captionsetup{skip=2pt}
    % \vspace{-0.1in}
    \caption{Comparison with prior works on the BERT model. The input has 128 tokens.}
    \label{fig:bumble}
\end{figure}

If $M_{\beta}'[i] = 0$, $P_0$ and $P_1$ will use low-degree 
polynomial approximation to compute the $\mathsf{GELU}$ function instead. Encrypted polynomial reduction leverages low-degree polynomials to compute non-linear functions for less important tokens. For the $\mathsf{GELU}$ function, the following degree-$2$ polynomial~\cite{kim2021ibert} is used:
\begin{equation*}
\mathsf{ApproxGELU}(x)=\begin{cases}
    0  &\text{if}\ x <  -1.7626 \\
    0.5x+0.28367x^2, &\text{if}\ x \leq |1.7626| \\
    x, &\text{if}\ x > 1.7626\\
\end{cases}
\end{equation*}


\section{Comparison with More Related Works.}
\label{app:c}
\textbf{Other 2PC frameworks.} The primary focus of CipherPrune is to accelerate the private Transformer inference in the 2PC setting. As shown in Figure \ref{fig:bumble}, CipherPrune can be easily extended to other 2PC private inference frameworks like BumbleBee~\citep{lu2023bumblebee}. We compare CipherPrune with BumbleBee and IRON on BERT models. We test the performance in the same LAN setting as BumbleBee with 1 Gbps bandwidth and 0.5 ms of ping time. CipherPrune achieves more than $\sim 60 \times$ speed up over BOLT and $4.3\times$ speed up over BumbleBee.

\begin{figure}[t]
 % \vspace{-0.1in}
    \centering
    \includegraphics[width=1\linewidth]{figures/pumab.pdf}
    % \captionsetup{skip=2pt}
    % \vspace{-0.1in}
    \caption{Comparison with MPCFormer and PUMA on the BERT models. The input has 128 tokens.}
    \label{fig:pumab}
\end{figure}

\begin{figure}[h]
 % \vspace{-0.1in}
    \centering
    \includegraphics[width=1\linewidth]{figures/pumag.pdf}
    % \captionsetup{skip=2pt}
    % \vspace{-0.1in}
    \caption{Comparison with MPCFormer and PUMA on the GPT2 models. The input has 128 tokens. The polynomial reduction is not used.}
    \label{fig:pumag}
\end{figure}

\textbf{Extension to 3PC frameworks.} Additionally, we highlight that CipherPrune can be also extended to the 3PC frameworks like MPCFormer~\citep{li2022mpcformer} and PUMA~\citep{dong2023puma}. This is because CipherPrune is built upon basic primitives like comparison and Boolean-to-Arithmetic conversion. We compare CipherPrune with MPCFormer and PUMA on both the BERT and GPT2 models. CipherPrune has a $6.6\sim9.4\times$ speed up over MPCFormer and $2.8\sim4.6\times$ speed up over PUMA on the BERT-Large and GPT2-Large models.


\section{Communication Reduction in SoftMax and GELU.}
\label{app:e}

\begin{figure}[h]
    \centering
    \includegraphics[width=0.9\linewidth]{figures/layerwise.pdf}
    \caption{Toy example of two successive Transformer layers. In layer$_i$, the SoftMax and Prune protocol have $n$ input tokens. The number of input tokens is reduced to $n'$ for the Linear layers, LayerNorm and GELU in layer$_i$ and SoftMax in layer$_{i+1}$.}
    \label{fig:layer}
\end{figure}

\begin{table*}[h]
\captionsetup{skip=2pt}
\centering
\scriptsize
\caption{Communication cost (in MB) of the SoftMax and GELU protocol in each Transformer layer.}
\begin{tblr}{
    colspec = {c |c c c c c c c c c c c c},
    row{1} = {font=\bfseries},
    row{2-Z} = {rowsep=1pt},
    % row{4} = {bg=LightBlue},
    colsep = 2.5pt,
    }
\hline
\textbf{Layer Index} & \textbf{0}  & \textbf{1}  & \textbf{2} & \textbf{3} & \textbf{4} & \textbf{5} & \textbf{6} & \textbf{7} & \textbf{8} & \textbf{9} & \textbf{10} & \textbf{11} \\
\hline
Softmax & 642.19 & 642.19 & 642.19 & 642.19 & 642.19 & 642.19 & 642.19 & 642.19 & 642.19 & 642.19 & 642.19 & 642.19 \\
Pruned Softmax & 642.19 & 129.58 & 127.89 & 119.73 & 97.04 & 71.52 & 43.92 & 21.50 & 10.67 & 6.16 & 4.65 & 4.03 \\
\hline
GELU & 698.84 & 698.84 & 698.84 & 698.84 & 698.84 & 698.84 & 698.84 & 698.84 & 698.84 & 698.84 & 698.84 & 698.84\\
Pruned GELU  & 325.10 & 317.18 & 313.43 & 275.94 & 236.95 & 191.96 & 135.02 & 88.34 & 46.68 & 16.50 & 5.58 & 5.58\\
\hline
\end{tblr}
\label{tab:layer}
\end{table*}

{
In Figure \ref{fig:layer}, we illustrate why CipherPrune can reduce the communication overhead of both  SoftMax and GELU. Suppose there are $n$ tokens in $layer_i$. Then, the SoftMax protocol in the attention module has a complexity of $O(n^2)$. CipherPrune's token pruning protocol is invoked to select $n'$ tokens out of all $n$ tokens, where $m=n-n'$ is the number of tokens that are removed. The overhead of the GELU function in $layer_i$, i.e., the current layer, has only $O(n')$ complexity (which should be $O(n)$ without token pruning). The complexity of the SoftMax function in $layer_{i+1}$, i.e., the following layer, is reduced to $O(n'^2)$ (which should be $O(n^2)$ without token pruning). The SoftMax protocol has quadratic complexity with respect to the token number and the GELU protocol has linear complexity. Therefore, CipherPrune can reduce the overhead of both the GELU protocol and the SoftMax protocols by reducing the number of tokens. In Table \ref{tab:layer}, we provide detailed layer-wise communication cost of the GELU and the SoftMax protocol. Compared to the unpruned baseline, CipherPrune can effectively reduce the overhead of the GELU and the SoftMax protocols layer by layer.
}

\section{Analysis on Layer-wise redundancy.}
\label{app:f}

\begin{figure}[h]
    \centering
    \includegraphics[width=0.9\linewidth]{figures/layertime0.pdf}
    \caption{The number of pruned tokens and pruning protocol runtime in different layers in the BERT Base model. The results are averaged across 128 QNLI samples.}
    \label{fig:layertime}
\end{figure}

{
In Figure \ref{fig:layertime}, we present the number of pruned tokens and the runtime of the pruning protocol for each layer in the BERT Base model. The number of pruned tokens per layer was averaged across 128 QNLI samples, while the pruning protocol runtime was measured over 10 independent runs. The mean token count for the QNLI samples is 48.5. During inference with BERT Base, input sequences with fewer tokens are padded to 128 tokens using padding tokens. Consistent with prior token pruning methods in plaintext~\citep{goyal2020power}, a significant number of padding tokens are removed at layer 0.  At layer 0, the number of pruned tokens is primarily influenced by the number of padding tokens rather than token-level redundancy.
%In Figure \ref{fig:layertime}, we demonstrate the number of pruned tokens and the pruning protocol runtime in each layer in the BERT Base model. We averaged the number of pruned tokens in each layer across 128 QNLI samples and then tested the pruning protocol runtime in 10 independent runs. The mean token number of the QNLI samples is 48.5. During inference with BERT Base, input sequences with small token number are padded to 128 tokens with padding tokens. Similar to prior token pruning methods in the plaintext~\citep{goyal2020power}, a large number of padding tokens can be removed at layer 0. We remark that token-level redundancy builds progressively throughout inference~\citep{goyal2020power, kim2022LTP}. The number of pruned tokens in layer 0 mostly depends on the number of padding tokens instead of token-level redundancy.
}

{
%As shown in Figure \ref{fig:layertime}, more tokens are removed in the intermediate layers, e.g., layer $4$ to layer $7$. This suggests there is more redundant information in these intermediate layers. 
In CipherPrune, tokens are removed progressively, and once removed, they are excluded from computations in subsequent layers. Consequently, token pruning in earlier layers affects computations in later layers, whereas token pruning in later layers does not impact earlier layers. As a result, even if layers 4 and 7 remove the same number of tokens, layer 7 processes fewer tokens overall, as illustrated in Figure \ref{fig:layertime}. Specifically, 8 tokens are removed in both layer $4$ and layer $7$, but the runtime of the pruning protocol in layer $4$ is $\sim2.4\times$ longer than that in  layer $7$.
}

\section{Related Works}
\label{app:g}

{
In response to the success of Transformers and the need to safeguard data privacy, various private Transformer Inferences~\citep{chen2022thex,zheng2023primer,hao2022iron-iron,li2022mpcformer, lu2023bumblebee, luo2024secformer, pang2023bolt}  are proposed. To efficiently run private Transformer inferences, multiple cryptographic primitives are used in a popular hybrid HE/MPC method IRON~\citep{hao2022iron-iron}, i.e., in a Transformer, HE and SS are used for linear layers, and SS and OT are adopted for nonlinear layers. IRON and BumbleBee~\citep{lu2023bumblebee} focus on optimizing linear general matrix multiplications; SecFormer~\cite{luo2024secformer} improves the non-linear operations like the exponential function with polynomial approximation; BOLT~\citep{pang2023bolt} introduces the baby-step giant-step (BSGS) algorithm to reduce the number of HE rotations, proposes a word elimination (W.E.) technique, and uses polynomial approximation for non-linear operations, ultimately achieving state-of-the-art (SOTA) performance.
}

{Other than above hybrid HE/MPC methods, there are also works exploring privacy-preserving Transformer inference using only HE~\citep{zimerman2023converting, zhang2024nonin}. The first HE-based private Transformer inference work~\citep{zimerman2023converting} replaces \mysoftmax function with a scaled-ReLU function. Since the scaled-ReLU function can be approximated with low-degree polynomials more easily, it can be computed more efficiently using only HE operations. A range-loss term is needed during training to reduce the polynomial degree while maintaining high accuracy. A training-free HE-based private Transformer inference was proposed~\citep{zhang2024nonin}, where non-linear operations are approximated by high-degree polynomials. The HE-based methods need frequent bootstrapping, especially when using high-degree polynomials, thus often incurring higher overhead than the hybrid HE/MPC methods in practice.
}


\end{document}
