\pdfoutput=1  % Force arXiv to use pdfLaTeX

%%
%% This is file `sample-sigconf-biblatex.tex',
%% generated with the docstrip utility.
%%
%% The original source files were:
%%
%% samples.dtx  (with options: `all,proceedings,sigconf-biblatex')
%% 
%% IMPORTANT NOTICE:
%% 
%% For the copyright see the source file.
%% 
%% Any modified versions of this file must be renamed
%% with new filenames distinct from sample-sigconf-biblatex.tex.
%% 
%% For distribution of the original source see the terms
%% for copying and modification in the file samples.dtx.
%% 
%% This generated file may be distributed as long as the
%% original source files, as listed above, are part of the
%% same distribution. (The sources need not necessarily be
%% in the same archive or directory.)
%%
%%
%% Commands for TeXCount
%TC:macro \cite [option:text,text]
%TC:macro \citep [option:text,text]
%TC:macro \citet [option:text,text]
%TC:envir table 0 1
%TC:envir table* 0 1
%TC:envir tabular [ignore] word
%TC:envir displaymath 0 word
%TC:envir math 0 word
%TC:envir comment 0 0
%%
%% The first command in your LaTeX source must be the \documentclass
%% command.
%%
%% For submission and review of your manuscript please change the
%% command to \documentclass[manuscript, screen, review]{acmart}.
%%
%% When submitting camera ready or to TAPS, please change the command
%% to \documentclass[sigconf]{acmart} or whichever template is required
%% for your publication.
%%
%%
%\documentclass[sigconf,natbib=false]{acmart}
\documentclass[sigconf,natbib=false, nonacm]{acmart} % ADD FOR PREPRINT SUBMISSION


%%
%% \BibTeX command to typeset BibTeX logo in the docs
\AtBeginDocument{%
  \providecommand\BibTeX{{%
    Bib\TeX}}}



%%
%% Submission ID.
%% Use this when submitting an article to a sponsored event. You'll
%% receive a unique submission ID from the organizers
%% of the event, and this ID should be used as the parameter to this command.
%%\acmSubmissionID{123-A56-BU3}

%%
%% For managing citations, it is recommended to use bibliography
%% files in BibTeX format.
%%
%% You can then either use BibTeX with the ACM-Reference-Format style,
%% or BibLaTeX with the acmnumeric or acmauthoryear sytles, that include
%% support for advanced citation of software artefact from the
%% biblatex-software package, also separately available on CTAN.
%%
%% Look at the sample-*-biblatex.tex files for templates showcasing
%% the biblatex styles.
%%


%%
%% The majority of ACM publications use numbered citations and
%% references, obtained by selecting the acmnumeric BibLaTeX style.
%% The acmauthoryear BibLaTeX style switches to the "author year" style.
%%
%% If you are preparing content for an event
%% sponsored by ACM SIGGRAPH, you must use the acmauthoryear style of
%% citations and references.
%%
%% Bibliography style
\RequirePackage[
  datamodel=acmdatamodel,
  style=acmnumeric,
  ]{biblatex}

%% Declare bibliography sources (one \addbibresource command per source)
\addbibresource{references.bib}


%% my imports
\usepackage{algorithm}
\usepackage{algcompatible}
\usepackage{dashrule}

\algrenewcommand\algorithmicrequire{\textbf{Input:}}
\algrenewcommand\algorithmicensure{\textbf{Output:}}
%%
%% end of the preamble, start of the body of the document source.
\begin{document}

%%
%% The "title" command has an optional parameter,
%% allowing the author to define a "short title" to be used in page headers.
\title{BaKlaVa - Budgeted Allocation of KV cache for Long-context Inference}

%%
%% The "author" command and its associated commands are used to define
%% the authors and their affiliations.
%% Of note is the shared affiliation of the first two authors, and the
%% "authornote" and "authornotemark" commands
%% used to denote shared contribution to the research.

\author{Ahmed Burak Gulhan}
\affiliation{%
  \institution{The Pennsylvania State University}
  \city{State College}
  \state{PA}
  \country{USA}
  }
\email{gulhan@psu.edu}

\author{Krishna Teja Chitty-Venkata}
\affiliation{%
  \institution{Argonne National Laboratory}
  \city{Lemont}
  \state{IL}
  \country{USA}
  }
\email{schittyvenkata@anl.gov}

\author{Murali Emani}
\affiliation{%
  \institution{Argonne National Laboratory}
  \city{Lemont}
  \state{IL}
  \country{USA}
  }
\email{memani@anl.gov}

\author{Mahmut Kandemir}
\affiliation{%
  \institution{The Pennsylvania State University}
  \city{State College}
  \state{PA}
  \country{USA}
  }
\email{mtk2@psu.edu}

\author{Venkatram Vishwanath}
\affiliation{%
  \institution{Argonne National Laboratory}
  \city{Lemont}
  \state{IL}
  \country{USA}
  }
\email{venkat@anl.gov}

%%
%% By default, the full list of authors will be used in the page
%% headers. Often, this list is too long, and will overlap
%% other information printed in the page headers. This command allows
%% the author to define a more concise list
%% of authors' names for this purpose.
%\renewcommand{\shortauthors}{Trovato et al.}

%%
%% The abstract is a short summary of the work to be presented in the
%% article.
\begin{abstract}
In Large Language Model (LLM) inference, Key-Value (KV) caches (KV-caches) are essential for reducing time complexity. 
However, they result in a linear increase in GPU memory as the context length grows. While recent work explores KV-cache eviction and compression policies to reduce memory usage, they often consider uniform KV-caches across all attention heads, leading to suboptimal performance. We introduce BaKlaVa, a method to allocate optimal memory for individual KV-caches across the model by estimating the importance of each KV-cache. Our empirical analysis demonstrates that not all KV-caches are equally critical for LLM performance. 
Using a one-time profiling approach, BaKlaVa assigns optimal memory budgets to each KV-cache. We evaluated our method on LLaMA-3-8B, and Qwen2.5-7B 
%and Mistral-7B 
models, achieving up to a 70\% compression ratio while keeping baseline performance and delivering up to an order-of-magnitude accuracy improvement at higher compression levels. 
\end{abstract}

%%
%% The code below is generated by the tool at http://dl.acm.org/ccs.cfm.
%% Please copy and paste the code instead of the example below.
%%
\begin{CCSXML}
<ccs2012>
   <concept>
       <concept_id>10010147.10010257.10010293</concept_id>
       <concept_desc>Computing methodologies~Machine learning approaches</concept_desc>
       <concept_significance>500</concept_significance>
       </concept>
 </ccs2012>
\end{CCSXML}

\ccsdesc[500]{Computing methodologies~Machine learning approaches}

%%
%% Keywords. The author(s) should pick words that accurately describe
%% the work being presented. Separate the keywords with commas.
\keywords{LLM, KV-cache, inference, GPU memory}

%\received{20 February 2007}
%\received[revised]{12 March 2009}
%\received[accepted]{5 June 2009}

%%
%% This command processes the author and affiliation and title
%% information and builds the first part of the formatted document.
\maketitle


\section{Introduction}

% State of the world (robots for creative activites)
The term ``robot,'' originally signifying `forced labor,' has long been associated with labor and work. Robots have demonstrated their utility in various automated productive and social contexts, where the primary goals are improving productivity, safety, and fostering social interactions with humans~\cite{simoes2022designing, weidemann2021role, honig2018understanding}. However, an increasing number of cases feature using of robots in creative settings. Unlike productive contexts, where the focus is on efficiency and task completion~\cite{arents2022smart}, or social contexts, where communication and trust are prioritized~\cite{nam2020trust, saunderson2019robots}, creative environments prioritize artistic innovation and expression~\cite{hsueh2024counts}. This shift fundamentally alters the dynamics of human-robot interaction, redefining the roles and expectations for both humans and robots.

For instance, robots’ social behaviors are leveraged to support the generation and expression of creative ideas~\cite{hu2021exploring, sandoval2022human, alves2020creativity}, and programmable robotic movements and trajectories are employed to inspire artistic activities such as sketching~\cite{lin2020your}. These studies often engage participants from creative fields who possess limited prior experience with robotics, and are typically conducted in short-term, experimental settings. Consequently, the findings from these studies remain constrained since much can be learned from professional practitioners' experiences to inform system design such as digital fabrication~\cite{hirsch2023nothing}. There is a notable gap in research examining the long-term, active, and practical experience of integrating robotic systems into the creative processes. As a result, the deeper insights into how robots facilitate and shape creative processes, beyond simply augmenting human creativity, remain underexplored. In this study, we aim to better understand the impacts of robots on creative processes and outcomes.

As early as Leonardo da Vinci's 16th century ``Automaton,'' artists have explored the creative affordances of robotic systems~\cite{shanken2002cybernetics, pagliarini2009development, jeon2017robotic}. The artistic creation process typically encompasses various stages, including the exploration of materials and techniques, ongoing experimentation and iteration, and the continual refinement of the artists' insights into their creative subjects~\cite{lewis2023art, sturdee2022state}. Therefore, investigating the artistic process involving robots offers an opportunity to gain deeper insights into robots' creative potential. Robotic art, in particular, provides a compelling case for this exploration.

We define robotic art as artworks that utilize robotic or automated machines to create artistic experiences and tangible artifacts. One example is robotic installation art, in which robots are programmed to follow specific rules that embody the artist’s expression (\autoref{fig:teaser} (a)). Another example is responsive art, in which robots react to their environment, with behaviors that change over time or in response to spectators (\autoref{fig:teaser} (b)). Additionally, there are robotic creators, which possess a degree of agency, allowing them to collaborate with human artists and produce works that extend beyond mere replication of human-created art (\autoref{fig:teaser} (c) and (d)). As such, robotic art becomes a rich case for exploring human-machine interactions in creative contexts. Gaining a deeper understanding of how robots facilitate artistic expression can provide insights for designing computing systems to support creative activities~\cite{gomez2021robot}.

% Therefore, we did...
We draw on semi-structured, in-depth interviews with renowned professional robotic artists to investigate the use of robots in artistic practice. Specifically, our goal is to understand how artistic exploration of robotic systems challenges conventional assumptions about the functions of robots, such as their roles in automating repetitive tasks or serving human needs. We also explore the implications of robots in the artistic process and examine how creativity may emerge within robotic art. To address these interrelated inquiries, our study focuses on the practice of robotic art, posing the research question: \textit{How do robotic artists utilize robots in their artistic practice?} We approach this inquiry through the perspectives and experiences of robotic artists, who creatively design, modify, and repurpose robotic systems for artistic expression and exploration.

% The key findings are...
Our findings highlight the social, material, and temporal dimensions of artists' practices that shape their creativity and artistic outcomes. The creation of robotic art is largely a social process, as artists receive both explicit and implicit feedback through the audience's reactions and reception of their work. Simultaneously, the embodiment and malfunctions inherent to robotic systems drive artistic experimentation. The temporal processes of creation and exhibition, beyond just the final product, further enhance the creative value. Our empirical analysis presents how creativity emerges through the interplay of social, material, and temporal interactions among artists, robots, audiences, and the environment.

% The contributions of this work are...
We make two main contributions to HCI in this study. 
First, we elucidate the interactive mechanisms among key actors---human creators, machines, audiences, and environments---within the practice of robotic art, a topic that remains underexplored in HCI. Our findings reveal the significance of sociality (e.g., interactions between artists and audiences), materiality (e.g., the embodiment and malfunctions of robots), and temporality (e.g., the processes of creation and exhibition) in shaping creative values. We propose that these three facets are central to the creative process and facilitate the emergence of creativity in robotic art.
Second, drawing from the findings, we offer implications for \textit{socially informed}, \textit{material-attentive}, and \textit{process-oriented} creation with computing systems. We suggest leveraging these three aspects to enhance creativity and the creative experience. Specifically, we discuss the value of incorporating implicit audience feedback, designing with technical malfunctions, and focusing on the post-creation process to foster alternative creative experiences with machines~\cite{alter2010designing, juarez2022glitch}.



\section{Background} \label{sec:background}

% \subsection{Capture the Flag (CTF) Challenges}

% CTF challenges simulate real-world cyber-attack scenarios and have emerged as a popular medium for practical cybersecurity training, evaluation, and research. These challenges can simulate real-world attack and defense scenarios and thus assist competitors in developing practical skills in areas such as cryptography, binary exploitation, and reverse engineering. 
% Evaluation of autonomous LLM agents works best with jeopardy-style CTF challenges that focus on standalone software that must be compromised \cite{shao2024nyu,pieterse2024friend}.
% The standalone software may be a binary that can be reverse engineered or exploited, encrypted data that can be decrypted, or a web server whose authentication can be bypassed. After successfully compromising the software, a unique ``flag'' string is either found or revealed by the software server.
% The unique flag string is a concrete indicator of the success of a CTF challenge.
% Recent studies use benchmarks of CTF challenges to evaluate LLM agents on their ability to solve complex tasks and demonstrate practical skills in cybersecurity \cite{shao2024nyu,shao2024empirical,abramovich2024enigma, muzsai2024hacksynth, zhang2024cybenchframeworkevaluatingcybersecurity,yang2023language,turtayev2024hacking}
% Platforms like PicoCTF~\cite{picoctf}, TryHackMe~\cite{tryhackme}, CTFTime~\cite{ctftime} and HackTheBox~\cite{hackthebox} have popularized these formats by providing structured challenges for learners at various skill levels.

% Research indicates that CTF challenges can foster cybersecurity expertise and serve as tools for evaluating facility with cybersecurity skills~\cite{chicone2018using}. They are widely used in academia to enhance learning outcomes in cybersecurity education, with studies demonstrating their effectiveness in promoting analytical thinking and teamwork~\cite{hanafi2021ctf,leune2017using,vykopal2020benefits}. Furthermore, the integration of CTF challenges into research environments enables benchmarking of advanced AI systems like LLMs. .

% Yet, challenges in CTF design persist. These include achieving significant performance, preserving context across tasks, and handling complex, dynamic CTFs that rely on multidisciplinary approaches. Implementing strategies to address these issues enhances problem-solving efficiency, enabling more accurate, adaptive, and effective responses to evolving challenges within CTF environments.


% \subsection{Prompt Engineering}
% \subsection{Prompt Engineering for CTF}
% \subsection{LLM Agents}

% As the use of LLMs to solve CFT challenges expands, prompt engineering is becoming a critical technique for enhancing performance. Various methods have been explored to craft prompts that effectively guide LLMs to the solution of complex cybersecurity problems. Each of these solutions have their own unique strengths and limitations.
%\meet{add more references for LLM agents in other domains, like SWE-Agent, also talk about use of function calling}
Text-based LLMs take a text prompt as input from the user, and produce a text output that follows the user prompt.
LLMs have a finite length of text tokens that they can process called the context.
An alternating sequence of user prompts and LLM outputs makes a conversation and is the basis of chat-based LLM interfaces like ChatGPT.
To remove the user from the loop and create autonomous agents, a feedback mechanism is added based on the LLM outputs, so that the LLM can autonomously continue the conversation.
\citet{yang2023intercode} introduce iterative feedback prompting where the LLM is tasked with writing a piece of code, and the code's compilation and execution logs are provided as feedback, which the LLM uses to iteratively refine it's output.
Recent LLMs support function calling, a way to provide a set of actions to the LLM that it may choose to ``call'' as a function.
In this manner, LLM agents can be provided with many ``tools'' such as a command line, web search, file editing, and code execution \cite{wang2024surveyllmagents}, so that they can autonomously perform various tasks like software development \cite{yang2024sweagent}, web browsing \cite{yoran2024assistantbench}, or solve CTF challenges~\cite{shao2024nyu, abramovich2024enigma}.

With access to the command line and file editing tools, LLM agents can autonomously solve many tasks, but they still struggle on complex long-horizon tasks such as CTF challenges that require multiple steps.
Plan-and-solve prompting \cite{wang2023planandsolve} enhances long-term focus of the agent by incorporating a planning phase before iterative execution. This helps agents tackle ambiguous or complex tasks by structured strategies \cite{turtayev2024hacking}.
ReAct (reasoning + action) \cite{yao2022react} combines step-by-step reasoning with action, allowing the agent to adjust dynamically through iterative cycles. ReWOO (Reasoning without Observation) \cite{xu2023rewoo} separates the reasoning process from tool outputs and observations, allowing it to handle multi-step reasoning tasks efficiently while maintaining focus.
The prompting methods in these agents involve static hard-coded templates where environment and task information is filled in.
While static prompts provide straightforward guidance, they often fail to adapt to different problems and complex tasks, limiting their effectiveness.
Auto-prompting~\cite{shin-etal-2020-autoprompt, zhou-etal-2023-revisiting, zhang2023automatic} is a technique to allow the LLM itself to generate a highly-relevant prompt. Auto-prompting invokes more factual responses and reduces hallucinations in LLMs.
D-CIPHER incorporates auto-prompting as a separate agent that can explore the environment and generate a better prompt.
%Based on the given prompt, LLM agents make a decision and proceed further to find flags.  To address this gap, we propose \textbf{dynamic prompting}, where the LLM agent autonomously generates prompts based on the CTF challenge's context and stage.
%include a static template which needs to be given to LLM to solve the CTF challenges. For instance, the NYU CTF framework provides a static prompt as \emph{``Please proceed to the next step using your best judgment"} for each decision making point. 

% To address this gap, we introduce a novel approach where the LLM agent generates the next prompt autonomously based on the current context and stage of the CTF challenge, a technique we call \textbf{dynamic prompting}.


Expanding on single LLM agents, multi-agent LLM systems are a powerful approach to enhance problem-solving by simulating team-based collaboration. Specialized agents, each with distinct objectives, work together to tackle different aspects of complex tasks \cite{guo2024largelanguagemodelbased}
Multi-agent systems are effective in cybersecurity applications. For instance, Audit-LLM~\cite{song2024audit} deploys a  multi-agent system for insider threat detection by employing agents to decompose tasks, build tools, and use collaborative reasoning to enhance detection accuracy. Liu~\cite{liu2024multi} explores multi-agent systems to enhance incident response in cybersecurity by examining centralized, decentralized, and hybrid team structures to assess how LLM agents can improve decision-making, adaptability, and coordination during cyber-attack scenarios. AutoSafeCoder~\cite{nunez2024autosafecoder} enhances the security of code generated by LLMs by incorporating a coding agent for code generation, a static analyzer agent that identifies vulnerabilities, and a fuzz testing agent for dynamic testing to detect runtime errors. Division of responsibilities among different agents allows AutoSafeCoder to produce secure, functionally correct code. 

% With the growing use of LLMs in CTF challenges, prompt engineering is key to enhancing performance. Various methods guide LLMs in solving complex cybersecurity tasks, each with distinct strengths and limitations.

% \textbf{Single Turn (Zero-Shot Prompting)} involves providing the model with a one-time task description that outputs  an immediate solution. This is efficient for straightforward tasks~\cite{yang2023intercode}. In contrast, \textbf{Try Again (Iterative Feedback Prompting)} uses iterative feedback to refine responses over multiple attempts, mimicking real-world problem-solving~\cite{yang2023intercode}. The \textbf{Plan \& Solve} enhances adaptability by incorporating a planning phase before iterative execution. This helps models tackle ambiguous or complex tasks by  structured strategies~\cite{turtayev2024hacking}. Additionally, \textbf{ReAct (Reasoning + Action)} combines step-by-step reasoning with action, allowing the model to adjust dynamically through iterative cycles. This makes it particularly effective for evolving and complex challenges like CTFs~\cite{yao2023react}. 
% These prompting techniques highlight diverse approaches to optimizing LLM performance in cybersecurity tasks. 

% Multi-agents!


%\meet{Add references for auto-prompting, and shorten this para}
%\nanda{Maybe we can add this to previous paragraphs which discusses other prompting methods such as plan-and-solve and ReAct method}
% All of these prompting methods include a static template which needs to be given to LLM to solve the CTF challenges. For instance, the NYU CTF framework provides a static prompt as \emph{``Please proceed to the next step using your best judgment"} for each decision making point. 
% Based on the given prompt, LLM agents make a decision and proceed further to find flags. While static prompts provide straightforward guidance, they often fail to account for the evolving nature of complex tasks, limiting their effectiveness in multi-step or ambiguous CTF challenges. To address this gap, we propose \textbf{dynamic prompting}, where the LLM agent autonomously generates prompts based on the CTF challenge's context and stage.
% % To address this gap, we introduce a novel approach where the LLM agent generates the next prompt autonomously based on the current context and stage of the CTF challenge, a technique we call \textbf{dynamic prompting}.
% Dynamic prompting adapts instructions to task progress, ensuring instructions are contextually relevant and reflective of the specific obstacles encountered. By iterating based on feedback and intermediate outputs, it continuously refines the LLM’s approach, enhancing problem-solving for dynamic tasks like CTFs.
% This adaptive process not only mirrors how humans tackle complex problems but also improves the model’s ability to handle unpredictable scenarios, making it particularly advantageous for cybersecurity tasks like CTFs where conditions change dynamically.


% The very first prompt type used in several applications is \textbf{Single Turn (Zero-Shot Prompting)}~\cite{yang2023intercode}. In single-turn prompting, the model receives a one-time, straightforward task description and is expected to generate a complete response without further interaction. The initial output is directly assessed, making this approach efficient for tasks where minimal feedback or iteration is required. This method tests the model’s ability to understand and respond to tasks immediately, relying heavily on the model's pre-trained knowledge and generalization capabilities.

% Along with this, The prompting method named \textbf{Try Again (Iterative Feedback Prompting)}~\cite{yang2023intercode} has been also used in several appreciations specially to solve CTF challenges. It is an iterative prompting method involves continuous interaction, where the model is provided with feedback after each attempt. The model can refine its responses over multiple turns based on the observations or execution results from previous outputs. This iterative process continues until the task is successfully completed or a maximum number of interactions is reached. This approach closely mirrors real-world problem-solving, where adjustments are made iteratively based on evolving circumstances or feedback.

% Some application are also using \textbf{Plan \& Solve}~\cite{turtayev2024hacking} prompting method which enhances problem-solving by dividing the process into a planning phase followed by execution. Initially, the model formulates a strategy based on the task description and available information, allowing for a structured approach to ambiguous or complex problems. This plan guides the subsequent execution phase, where the model carries out actions iteratively, refining its approach based on feedback. In more challenging scenarios, re-planning mid-task further improves adaptability and performance. This method proves effective in tasks like CTF challenges, where vague instructions require careful analysis and step-by-step resolution.

% Further some application are also using \textbf{ReAct (Reasoning + Action)}~\cite{yao2023react} prompting method blends reasoning with action by guiding the model to think through tasks step-by-step before executing actions. At each step, the model generates a thought based on the task and observations, which informs the next action. The action is executed, and the resulting feedback refines the model’s understanding for the next cycle. This continuous process helps the model adapt dynamically to complex tasks, making it effective for CTF challenges where logical reasoning and step-by-step execution are essential.

\section{Related Works} \label{sec:related_work}


\begin{table}[htpb]
    \centering
    \caption{Feature comparison of LLM agents for solving CTFs.}
    \label{tab:related_work_comparison}
    \begin{tabular}{lcccccc}
    \toprule
         \textbf{Study} & \rotatebox{90}{\textbf{\# CTFs}} & \rotatebox{90}{\textbf{Open bench}} & \rotatebox{90}{\textbf{Tool use}}  & \rotatebox{90}{\textbf{Autonomous}} & \rotatebox{90}{\textbf{Multi-agent}} &\rotatebox{90}{\textbf{Auto-prompt}} \\
    \cmidrule{2-7}
     % \textbf{Study} & \textbf{Dynamic} & \textbf{Used} & \textbf{Multi-} & \textbf{Automatic} & \textbf{Tool} & \textbf{\# of} \\
         Tann et al. \cite{tann2023using} &  $7$ & \purplecross & \purplecross & \purplecross & \purplecross & \purplecross  \\
         Shao et al. \cite{shao2024empirical} & $26$ & \purplecross & \tealcheck & \tealcheck & \purplecross & \purplecross  \\
         InterCode-CTF\cite{yang2023language} & $100$ & \tealcheck & \tealcheck & \tealcheck & \purplecross & \purplecross   \\
         NYU CTF Bench \cite{shao2024nyu} & $200$ & \tealcheck & \tealcheck & \tealcheck & \purplecross & \purplecross \\
         Turtayev et al. \cite{turtayev2024hacking} & $100$ & \tealcheck & \tealcheck & \tealcheck & \purplecross & \purplecross\\
         Cybench \cite{zhang2024cybenchframeworkevaluatingcybersecurity} & $40$ & \tealcheck & \tealcheck & \tealcheck & \purplecross & \purplecross \\
         EnIGMA \cite{abramovich2024enigma} & $350$ & \tealcheck & \tealcheck & \tealcheck & \purplecross & \purplecross\\
         HackSynth \cite{muzsai2024hacksynth} & $200$ & \tealcheck & \tealcheck & \tealcheck & \tealcheck & \purplecross \\
         \textbf{D-CIPHER (ours)} & $290$ & \tealcheck & \tealcheck & \tealcheck & \tealcheck & \tealcheck \\
    \bottomrule
    \end{tabular}
\end{table}



% \subsection{LLMs on Cybersecurity}
% \subsection{LLM Agents for CTF}

%LLMs have a vast knowledge base that can be tapped for cybersecurity use.
Tann et al.~\cite{tann2023using} evaluate early LLMs such as ChatGPT and Google Bard in solving CTF challenges and answering professional certification questions, showing that LLM responses contain key task information.
%Many works extend the LLM capabilities by providing them access to programming and command execution tools, to form autonomous agents. 
The InterCode-CTF agent~\cite{yang2023intercode} reveals that LLM agents demonstrate basic cybersecurity skills, however they struggle with more complex tasks.
The NYU CTF baseline agent~\cite{shao2024empirical} integrates external tools into the LLM's function-calling features and demonstrate improved potential of tool-assisted LLMs to solve CTFs, however it exhausts the LLM context length when command output history becomes very long. InterCode-CTF manages this issue by truncating the history to only show the LLM the last few iterations. Even so, LLM agents face issues with longer tasks.
%NYU CTF Bench~\cite{shao2024nyu}, a benchmark of 200 CTF challenges, presents a baseline agent with specialized reverse engineering tools and category-specific prompts, demonstrating their importance to solve CTFs.
% The NYU CTF baseline agent faces issues of LLM context length when complex tasks run for several iterations and the entire command and output history becomes longer than the LLM's context window size. The InterCode agent manages this issue by truncating the history to only show the LLM the last few iterations.


Excessive tool availability and verbose interfaces can overwhelm agents, leading to inefficiencies. Agents perform better with a focused set of tools with well-defined interfaces~\cite{yang2024sweagent}.
EnIGMA~\cite{abramovich2024enigma} agent incorporates interactive tools and in-context learning techniques to achieve state-of-the-art results. % on the NYU CTF Bench, HackTheBox, and Cybench benchmarks.
For better context management, EnIGMA also uses an LLM summarizer that summarizes the command outputs for the main agent.

HackSynth~\cite{muzsai2024hacksynth}, an LLM agent for autonomous penetration testing, shows that iterative planning and feedback summarization stages help the agent finish multiple tasks and improves overall problem solving.
Similarly, Cybench~\cite{zhang2024cybenchframeworkevaluatingcybersecurity} introduces a benchmark of 40 CTF challenges augmented with step-by-step tasks, demonstrating better focus of LLM agents on smaller tasks, leading to improved success and alleviating the context length issue.
\citet{turtayev2024hacking} expand on InterCode-CTF by implementing plan-and-solve prompting, achieve significant improvement on the InterCode-CTF benchmark. They show that prompting techniques can improve performance even with simple toolsets.
% . Their baseline agent is evaluated in unguided mode (i.e. fully autonomous), and guided mode where the agent is given one task at a time. Their results indicate that providing smaller tasks to the LLM agents improve their focus yielding improved success on complex challenges while .

These works highlight that LLM agents excel at implementing code and executing commands to accomplish small concrete tasks when provided with dynamic feedback and task-specific toolsets. While these works  involved using multiple LLMs with different tasks such as planning and summarizing along-side a main agent, D-CIPHER is the first work to formulate a multi-agent system where there is a bifurcation of responsibilities between agents and meaningful well-defined interactions for dynamic feedback.
Table~\ref{tab:related_work_comparison} shows a feature comparison of D-CIPHER with related works on LLM agents for autonomous CTF solving.
%\meet{some description of the feature comparison?}
% Recent research has focused on enable autonomous solving of CTF challenges~\cite{shao2024empirical,shao2024nyu,abramovich2024enigma}. These agents typically operate in containerized environments to ensure reproducibility and modularity. 

% As an early effort, Tann et al.~\cite{tann2023using} evaluated the effectiveness of LLMs, such as OpenAI's ChatGPT, Google Bard, and Microsoft Bing, in solving cybersecurity CTF challenges and answering professional certification questions. 
% % Their study results show that LLMs performed well on $7$ CTF test cases, with ChatGPT solving $6$, Bard $2$, and Bing $1$. 
% The study shows that LLM responses often contain key information essential for solving tasks.

% The InterCode framework~\cite{yang2023intercode} approaches coding as an interactive process and uses execution feedback to improve code generation. As described in Yang et al.~\cite{yang2023intercode}, InterCode-CTF integrates CTF benchmarks into a reinforcement learning environment that can evaluate the cybersecurity capabilities of language agents. It features $100$ tasks that tapskills such as reverse engineering, forensics, and binary exploitation. While existing language agents demonstrate basic cybersecurity skills, evaluations indicate they struggle with more complicated complex tasks unless the system is fine-tuned or given external support. 
% cite Intercode: Standardizing and benchmarking interactive coding with execution feedback

% Another notable example is an LM agent developed by Shao et al. specifically to automate CTF tasks. 
% Shao et al.~\cite{shao2024empirical} developed a LM agent to automate CTF tasks.
% % They report an accuracy rate of  $46\%$ on $26$ CTF challenges sourced from CSAW'23 Qualifying round competition using GPT-4.
% By effectively combining LLM capabilities with external tools, the researchers demonstrated the potential of tool-assisted LLMs to solve complex problems. Building on this, the team incorporated a broader range of cybersecurity tools and interfaces that enhance both accuracy and versatility. 
% Empirical results show their system outperforms baselines on both the InterCode CTF benchmark and the NYU CTF benchmark.

% Shao et al.~\cite{shao2024nyu} presented a diverse, open-source database of CTF challenges that can be used to benchmark an LLM's ability to solve cybersecurity problems.
% It provides a scalable platform for developing and testing AI-driven approaches for vulnerability detection and resolution, facilitating advancements in automated cybersecurity tasks. The benchmark database and automated framework were successfully applied to the performance of five LLMs. 

% The Cybench benchmark~\cite{zhang2024cybenchframeworkevaluatingcybersecurity} provides another significant contribution by creating a framework tailored to solving CTF challenges. % Cybench: A framework for evaluating cybersecurity capabilities and risk
% % Their benchmark environment achieves an accuracy of $17.5\%$ using Claude 3.5 Sonnet. 
% Such frameworks operate in Linux-based containerized environments, such as Kali Linux, which includes pre-installed cybersecurity tools. However, excessive tool availability can overwhelm agents, leading to inefficiencies. Research indicates that agents perform better with a focused set of tools that have well-defined interfaces~\cite{yang2024sweagent}. % Swe-agent: Agent-computer interfaces enable automated software engineering



% Muzsai et al. introduced HackSynth~\cite{muzsai2024hacksynth}, an LLM-based agent for autonomous penetration testing. It uses a dual-module architecture that consists of a Planner and a Summarizer, allowing for iterative command generation and feedback processing. The framework is evaluated using two benchmark sets from platforms like PicoCTF~\cite{picoctf} and OverTheWire~\cite{overthewire}. These benchmarks address $200$ challenges drawn from various domains and difficulty levels. Results of their study show that HackSynth, especially with the GPT-4o model, achieves the best performance. This highlights the potential of LLM-based agents in advancing autonomous penetration testing.
 % Using basic prompting techniques and expanding tool availability, the study highlights how straightforward approaches can unlock the latent potential of LLMs for cybersecurity tasks. Their work emphasizes that simple LLM designs can effectively solve CTF challenges, and thus broaden the number of cybersecurity applications without the need for advanced engineering.

% \begin{table*}[]
%     \centering
%     \begin{tabular}{|c|c|>{\centering\arraybackslash}p{4.5cm}|c|c|c|c|c|c|}
%     \hline
%          \textbf{Study} & \textbf{Dynamic} & \textbf{Used} & \textbf{Multi-} & \textbf{Open} & \textbf{Automatic} & \textbf{Tool} & \textbf{\# of} & \textbf{\# of} \\
%          & \textbf{Prompt} & \textbf{Benchmarks} & \textbf{Agents} & \textbf{Dataset} & \textbf{Framework} & \textbf{Use} & \textbf{LLMs} & \textbf{CTFs}\\
%          \hline
%          Tann et al.~\cite{tann2023using} & \purplecross & Manual collected & \purplecross & \purplecross & \purplecross & \purplecross & $3$ & $7$ \\
%          \hline
%          InterCode-CTF~\cite{yang2023language} & \purplecross &  PicoCTF~\cite{picoctf} & \purplecross & \purplecross& \purplecross & \purplecross & $1$ & $100$  \\
%          \hline
%          Shao et al.~\cite{shao2024empirical} & \purplecross & CSAW 2023 & \purplecross & \purplecross & \tealcheck & \tealcheck & $4$ & $26$ \\
%          \hline
%          Shao et al.~\cite{shao2024nyu} & \purplecross & NYU CTF~\cite{shao2024nyu} & \purplecross & \tealcheck & \tealcheck & \tealcheck & $5$ & $200$ \\
%          \hline
%          Cybench~\cite{zhang2024cybenchframeworkevaluatingcybersecurity} & \purplecross & Cybench~\cite{zhang2024cybenchframeworkevaluatingcybersecurity}  & \purplecross & \tealcheck & \tealcheck & & $8$ & $40$ \\
%          \hline
%          EnIGMA~\cite{abramovich2024enigma} & \purplecross & NYU CTF~\cite{shao2024nyu}, InterCode-CTF~\cite{yang2023language},  HackTheBox~\cite{hackthebox} & \purplecross & \purplecross & \tealcheck & \tealcheck & $3$ & $350$ \\
%          \hline
%          HackSynth~\cite{muzsai2024hacksynth} & \purplecross & PicoCTF~\cite{picoctf}, OverTheWire~\cite{overthewire} & \tealcheck & \tealcheck & \tealcheck & \tealcheck & $8$ & $200$ \\
%          \hline
%          Turtayev et al.~\cite{turtayev2024hacking} & \purplecross & InterCode-CTF~\cite{yang2023language} & \purplecross & \purplecross & \purplecross & \purplecross & $4$ & $100$ \\
%          \hline
%          \textbf{D-CIPHER (Proposed)} & \tealcheck & NYU CTF~\cite{shao2024nyu}, Cybench \cite{zhang2024cybenchframeworkevaluatingcybersecurity}, HackTheBox \cite{hackthebox} & \tealcheck & \tealcheck & \tealcheck & \tealcheck & 5 & 290 \\
%          \hline
%     \end{tabular}
%     \caption{Comparison with LLM-based CTF solving Literature}
%     \label{tab:related_work_comparison}
% \end{table*}




% \subsection{Multi-agent framework}

% The use of multi-agent LLM systems in Capture the Flag (CTF) challenges is emerging as a powerful approach to enhance cybersecurity problem-solving. Multi-agent frameworks mimic team-based collaboration, where multiple LLM agents, each with specialized expertise, work together to tackle complex tasks. This approach reflects real-world cybersecurity operations, where success often depends on coordinated efforts from teams with diverse skills, each addressing different components of a security challenge.
% Multi-agent LLM systems are emerging as a powerful approach to enhance cybersecurity problem-solving by simulating team-based collaboration. Specialized agents, each with distinct objectives, work together to tackle different aspects of complex security tasks. This mirrors real-world cybersecurity operations, where coordinated efforts and diverse skills are essential for addressing evolving threats and vulnerabilities.

% CTF challenges cover a wide range of domains, including cryptography, reverse engineering, forensics, and web exploitation. Multi-agent systems can distribute the workload by assigning agents to handle specific tasks. This enables parallel problem-solving and emulates the collaborative nature of human teams. For example, one agent may specialize in guiding the fellow agents to what needs to be done, while another executes the instructions, ensuring that tasks are addressed without losing the context, and implementing reasoning from multiple LLMs. This division of labor boosts efficiency and enables problem-solving from multiple perspectives.
% This division of labor enhances efficiency and allows the system to approach problems from multiple perspectives, reflecting the interdisciplinary approach often used in cybersecurity teams.

% Guo et al.~\cite{guo2024largelanguagemodelbased} highlight the strengths of multi-agent LLMs in complex, multi-step tasks where different agents handle specific roles The framework HackSynth~\cite{muzsai2024hacksynth} is a multi-agent penetration testing framework in which agents operate collaboratively to address vulnerabilities in staged environments. Their work emphasizes that when agents work as a cohesive team, they outperform single-agent approaches. This is particularly true when facing layered, iterative challenges. 
% This team-based model of problem-solving aligns closely with how cybersecurity professionals approach real-world security incidents and penetration testing exercises.

% Multi-agent LLM systems have shown effectiveness in various other applications. For instance,  Audit-LLM~\cite{song2024audit} presents a multi-agent framework for insider threat detection using log analysis. It employs agents to decompose tasks, build tools, and use collaborative reasoning to enhance detection accuracy. Liu~\cite{liu2024multi} explores the application of LLM-based multi-agent systems to enhance incident response (IR) in cybersecurity. Utilizing the ``Backdoors \& Breaches" tabletop game as a simulation environment, the study examines centralized, decentralized, and hybrid team structures to assess how LLM agents can improve decision-making, adaptability, and coordination during cyberattack scenarios. AutoSafeCoder~\cite{nunez2024autosafecoder} is a multi-agent system designed to enhance the security of code generated by LLMs. The framework comprises three agents: a Coding Agent responsible for code generation, a Static Analyzer Agent that identifies vulnerabilities through static analysis, and a Fuzzing Agent that performs dynamic testing using mutation-based fuzzing to detect runtime errors. By integrating both static and dynamic testing in an iterative process, AutoSafeCoder aims to produce secure, functionally correct code. 

% To enhance CTF-solving by promoting team-based specialization, we employ a multi-agent CTF solving agent. Within this framework, agents tackle tasks aligned with their strengths. Tasks are executed in parallel, improving efficiency and accelerating progress. Agents share insights, adapt refining strategies based on feedback, and overcome obstacles collectively. This collaborative approach boosts scalability, adaptability, and and resilience, and improves performance in complex challenges.

% This paper presents a comprehensive comparison of D-CIPHER with existing LLM-based CTF-solving literature, as shown in Table~\ref{tab:related_work_comparison}.
% This paper documents the results of  our comprehensive comparison of D-CIPHER with existing LLM-based CTF-solving literature. These results are presented in Table~\ref{tab:related_work_comparison}.

%input {txt/3-motivation}

\section{The BaKlaVa Method}\label{sec:methods}
Our method for optimizing KV-cache memory allocation consists of 3 main steps for a given LLM: (i) A one-time collection of profiling data for a given prompt(s) (Algorithm~\ref{alg:profiling} -- step 1);  (ii) Using a heuristic to estimate the `importance' of KV caches,  which is also a one-time calculation  (Algorithm~\ref{alg:profiling} -- step 2); and (iii) Performing a parameter search to allocate memory accordingly (Algorithms~\ref{alg:kv-mem-allocation} and~\ref{alg:parameter_search}). 

All three steps in BaKlaVa only need to be run once. The most time-consuming part currently is Step (iii), where a parameter search is performed for the target compression level. For this parameter search, we quickly evaluate each parameter combination using `perplexity', rather than running a long-context evaluation benchmark, since perplexity does not require autoregressive token generation and consequently is much faster and gives a good approximation of actual performance. This parameter search, for the models we evaluated (which contain 7 to 8 billion parameters), takes 10 to 20 minutes on 8x A100 GPUs for around 200 combinations of parameters on 98k tokens for a chosen compression ratio. The number of tokens can be decreased for a proportional decrease in runtime, though they should be at least as much as the maximum context length being evaluated.

Once the ideal parameters for an LLM are obtained, no additional computation is required.  To begin inference, we initialize our custom huggingface transformers' KV-cache object to use in inference.   




\begin{algorithm}
\caption{One-Time Profiling for KV-Cache Importance}
\begin{algorithmic}[1]
\REQUIRE LLM model $\mathcal{M}$, one or more prompts $\mathcal{P}$ of varying lengths
\ENSURE Values indicating relative KV-cache importance

\STATE \textbf{Step 1:} Collect profiling data from prompts $\mathcal{P} = \{p_1, p_2, \dots, p_n\}$ of different lengths
\FOR{each prompt $p_i \in \mathcal{P}$}
    \STATE Run inference on $\mathcal{M}$ with $p_i$
    \FOR{each layer $l \in \mathcal{L}$}
        \FOR{each attention head $h$ in layer $l$}
            \STATE Compute token-wise cosine similarity between attention head input $V$ and output $SoftMax(QK^T) V$
            \STATE Compute the average across all cosine similarities to obtain a single similarity value $s_{il} \in \mathcal{S}$
        \ENDFOR
    \ENDFOR
\ENDFOR

\STATE \textbf{Step 2:} Convert attention head similarities to KV-cache importance using the number of attention heads per KV-cache group $g$
\FOR{each layer $l \in \mathcal{L}$}
    \FOR{each group of $g$ heads, denoted by $s_{il}, s_{(i+1)l}, \dots, s_{(i+g-1)l} \in \mathcal{S}$}
        \STATE Obtain similarity for the current KV cache, $KVsim \gets mean(s_{il}, \dots, s_{(i+g-1)l})$ 
        \STATE KV cache importance $I_{li} \gets 1-KVsim$
    \ENDFOR
\ENDFOR

\end{algorithmic}
\label{alg:profiling}
\end{algorithm}



\begin{figure}[h]
    \centering
    \vspace{-10pt}
    \includegraphics[width=0.8\linewidth]{figs/similarity_heuristic.pdf}   
    \vspace{-0.25in}  
    \caption{The attention-head similarity heuristic used in BaKlaVa. By taking the cosine similarity between the input and output, we can calculate how much change there is. The more change between the input and output of the attention head, the more important we assume it is.}
    %\vspace{-0.3in}
    \label{fig:dot_product_cos} 
\end{figure}

\begin{figure}[h]
    \centering
    %\includegraphics[width=1\linewidth]{figs/head_importances.pdf}   

    \includegraphics[width=1\linewidth]{figs/similarities/llama_xsum_similarity.png}
    %\vspace{-20pt}
    \includegraphics[width=1\linewidth]{figs/similarities/llama_lcc_similarity.png}
    %\vspace{-5pt}
    %\rule{\linewidth}{0.5pt} % Adds a thin horizontal line
    \hdashrule{\linewidth}{0.5pt}{1mm 1mm} % Dotted horizontal line
    %\vspace{-7pt}
    \includegraphics[width=1\linewidth]{figs/similarities/qwen_xsum_similarity.png}
    \includegraphics[width=1\linewidth]{figs/similarities/qwen_lcc_similarity.png}
    \caption{Cosine similarity heatmap for input and output of attention heads for two different prompts in LLaMA3-8B and Qwen2.5-7B. We chose three representative layers to illustrate that attention head consistency holds across different prompts. The X-axis shows the attention heads in a layer, Y-axis represents each token position in the prompt. Green and red outlines show the highest and lowest column similarity means per layer, that is, the most and least important attention heads respectively.  
    The order of average attention head similarities (the mean of each column, see Algorithm~\ref{alg:profiling}) stays highly consistent even across different prompts of different lengths, indicating that profiling an LLM one time is sufficient to make KV-cache importance estimations.}
    %\vspace{-0.3in}
    \label{fig:prompt_test} 
\end{figure}


\subsection{Determining KV-cache Importances}
\label{sec:determine_kv_importance}
\subsubsection{Head Importance Heuristic} 
To determine the significance of an individual attention head, we used several key observations to come up with a heuristic. The first is that the more change there is between the input and output of a structure in an LLM, the more important it is, as used in~\cite{ge2024model} to determine the type of tokens the individual attention heads focus on and in~\cite{squeezeattention, pyramidinfer} to determine the importance of LLM layers. Second, the attention matrix $softmax(QK^T)$ has been shown to be a high-quality heuristic that can determine individual token importances~\cite{h2o, scissorhands, keyformer} for KV-cache eviction. Lastly, the $V$ tensor contains key information about the tokens, which is not found in the attention score matrix~\cite{guo2024attention, devoto2024simple}. \textbf{Based on these observations, we propose the idea that the greater the change between the input} $V$ \textbf{and the output} $softmax(QK^T)V$ \textbf{ of this attention head, the more critical this attention head is for inference.}

As shown in Algorithm~\ref{alg:profiling}, to do this, we compare the input $V$ tokens (each token is a multidimensional vector) with the output of the attention head, $softmax(QK^T)V$, using cosine similarity, as shown in Figure~\ref{fig:dot_product_cos}. 
  
Each input and output token are individually compared using a cosine similarity value to determine the change in vector direction. To obtain a single value for each attention head, we first (i) get the mean of all tokens' cosine similarities within that head. The result of cosine similarity ranges from -1 to 1 and the maximum value is obtained when the two compared token vectors are identical (i.e., the attention score matrix is an identity matrix). We then (ii) normalize these values from range 0 to 1 to obtain a single similarity value, such that a value of 1 means identical input and outputs for all tokens in the attention head. Lastly, (iii) to obtain the `importance value', we take the complement of each mean similarity ($1 - similarity$). An importance value closer to 1 means a bigger difference between the input and output tokens, thus it has more importance and vice versa. 

We tested multiple token comparison methods, such as dot-product and KL divergence, and found that cosine similarity and dot-product both give similar results.  However, cosine similarity guarantees an output between -1 and 1, leading to simpler calculations. Therefore, we chose cosine-similarity for BaKlaVa. Note that cosine similarity measures the change in angle, but not magnitude, and thus another method that incorporates {\em both} magnitude and direction change may give better results. We left this for future work. 

\subsubsection{One-Time Profiling}
To determine the frequency of profiling needed, considering that attention head behavior can vary between inference steps and different prompts, 
we ran several experiments. We found that while the importance of individual tokens may change throughout inference, overall the importance value (calculated by taking the average of all token cosine similarities, see Algorithm~\ref{alg:profiling}, step 1) remains consistent, across different prompts. This is illustrated in Figure~\ref{fig:prompt_test}, where for two models, LlaMA3-8B and Qwen2.5-7B-Instruct, we profile two prompts of lengths around 350 and 2000 tokens from a text (XSum) and coding (LCC) dataset respectively, for three layers in both LLMs (not just to highlight these specific layers, but to illustrate that the consistency the attention head behavior holds across different layers). Each tile (in the heatmaps) represents the cosine similarity difference for a single token with the top 5 and bottom 5 importance heads outlined in green and red,  respectively. We can observe that the highest- and lowest-ranking attention heads stay highly consistent across prompts of different lengths and different types (i.e. text vs code). This suggests that, for a given LLM, a single profiling run with a sufficiently large prompt (i.e. few hundred tokens) is sufficient to determine the attention head importance values that can be applied for all future inferences.

\subsubsection{Grouped Query Attention} 
If the LLM employs Grouped Query Attention (GQA), an additional step is required before determining the KV cache memory budgets. Since our measurements assess changes at the level of individual attention heads rather than KV-caches, a direct assignment is not possible. In GQA, multiple attention heads \textit{share} the same KV-cache, meaning that memory budgets cannot be allocated separately for each head within the same group. To address this, we compute the mean of similarities across all attention heads within a group, obtaining a single similarity value per GQA group. This process is detailed in Step 2 of Algorithm~\ref{alg:profiling}



\subsubsection{Layer Importance Heuristic}
\label{sec:layer_heuristic}
The KV-cache importances we have found so far are used to allocate the GPU memory budget {\em within} a single layer in an LLM. Simply taking the average of all KV-cache importances does not find the correct layer importance, since our KV-cache importance heuristic is agnostic to several other important structures in an LLM (e.g., the feed-forward networks, layer normalization, etc). Based on our empirical testing results (see Section~\ref{sec:empirical}), we found that SqueezeAttention~\cite{squeezeattention} is a simple and low-overhead heuristic that closely, though not perfectly, matches the `ground truth' layer importances and use this heuristic to determine the layer-wise importance values in BaKlaVa. The SqueezeAttention heuristic takes the cosine similarity between the input and output of each LLM layer, thus capturing the total effect of {\em all} structures within the layer. 



\subsection{Assigning Memory Budgets to KV-Caches}

\subsubsection{Memory Allocation}
\label{sec:methods-memory-alloc} 

\begin{algorithm}
\caption{KV-Cache Memory Reallocation Based on Attention Head Importance}
\begin{algorithmic}[1]
\REQUIRE Importance scores $\mathbf{I} = \{I_1, I_2, ..., I_m\}$ for $m$ KV-caches, threshold $t$, reduction amount $r$
\ENSURE Adjusted KV-cache allocations

\STATE $\mathcal{L} \gets \{i \mid I_i < t\}$ \COMMENT{Identify KV-caches with low importance}
%\STATE $\mathcal{H} \gets \{i \mid I_i \geq t\}$ \COMMENT{Identify high-importance KV-caches}

\IF{$|\mathcal{L}| > m - 1$} 
    \STATE RETURN UNCHANGED KV-cache allocations 
    \COMMENT{If all KV-caches are low importance then do not do anything}
\ENDIF

\FOR{EACH $i \in \mathcal{L}$}
    \STATE REDUCE KV-cache size of $i$ by $r$
\ENDFOR

\STATE $n \gets |\mathcal{L}|$ \COMMENT{Number of KV-caches reduced}
\STATE $k \gets \min(n, m - n)$ \COMMENT{Limit reallocation up to the top n available high-importance caches}
\STATE $\mathcal{H} \gets$ TOP-$k$ ELEMENTS OF $\mathcal{H}$ BASED ON $I_i$ 
\STATE $\Delta r \gets \frac{n \times r}{k}$ \COMMENT{Compute adjusted increase per cache}

\FOR{EACH $j \in \mathcal{H}'$}
    \STATE INCREASE KV-cache size of $j$ BY $\Delta r$
\ENDFOR

\STATE RETURN updated KV-cache allocations

\end{algorithmic}
\label{alg:kv-mem-allocation}
\end{algorithm}

Once the importance values for each KV-cache and layer are obtained, the next step is to determine how to allocate memory budgets. 

Based on our observation of token similarities as shown in Figure~\ref{fig:prompt_test}, we find that low-importance attention heads are more consistent with how they change each individual token in a prompt (that is, the dot-product between input and output of the attention head has low variance), while other attention heads can display significant changes across tokens (that is, high variance in the token cosine similarity). Thus, to reduce the chances for decreasing the memory of KV-caches belonging to potentially critical attention heads, we take a conservative approach and only target KV-caches with an importance score below a threshold $t$ by a predetermined amount $r$, as shown in Algorithm~\ref{alg:kv-mem-allocation}. The freed memory is then assigned to up to top $n$ KV caches of highest importance (where $n$ is the number of low-importance KV-caches selected), in order to prioritize increasing memory for the most important KV-caches. 


\subsubsection{Parameter Search}
\label{sec:parameter_search}


\begin{algorithm}
\caption{Parameter Search Using Perplexity}
\begin{algorithmic}[1]
\REQUIRE Evaluation prompt $\mathcal{P}$, model $M$, context length $L$, list of parameter configurations $\mathcal{C}$, compression ratio $cmp$
\ENSURE Optimal parameters $p^*$

\STATE $best\_params \gets \emptyset$
\STATE $min\_loss \gets \infty$

\FOR {$\text{params} \in \mathcal{C}$}
    
    \STATE CACHE = $\text{MAKE\_CACHE}(\text{params}, cmp)$
    \STATE $losses \gets []$
    
    \FOR{$\text{chunk} \in \text{STEP}(\mathcal{P}, L)$} \COMMENT{Get chunks of tokens from prompt}
        \STATE $loss \gets \text{PERPLEXITY}(M, \text{chunk}, CACHE)$
        \STATE $losses.\text{append}(loss)$
        \STATE CACHE $\gets \text{RESET\_CACHE}(CACHE)$
    \ENDFOR

    \STATE $avg\_loss \gets \frac{\sum \text{losses}}{\text{len}(\text{losses})}$

    \IF{$avg\_loss < min\_loss$}
        \STATE $min\_loss \gets avg\_loss$
        \STATE $best\_params \gets \text{params}$
    \ENDIF

\ENDFOR

\STATE RETURN $best\_params$

\end{algorithmic}
\label{alg:parameter_search}

\end{algorithm}

To determine the optimal values for $r$ and $t$, we performed a parameter search over different compression values, as shown in Algorithm~\ref{alg:parameter_search}. We found that the ideal parameter `area' varies across different compressions. These results were calculated using perplexity, as it is much faster to find compared to LongBench and it is a good indicator of actual performance. Based on these observations, we chose the best-performing parameter pair for a given compression value when evaluating on LongBench. 

\subsection{Empirical Evaluation of Heuristics}
\label{sec:empirical}


To evaluate the effectiveness of our layer and KV-cache importance heuristics in comparison to the `true' importance, we conducted computationally intensive experiments. These tests empirically assessed the impact of individual layers
%and KV-caches 
on model performance by measuring the variation in benchmark scores resulting from modifications to each component. The underlying principle is that the greater the performance degradation caused by a change (e.g., memory reduction) in a layer or KV-cache, the more critical that component is to the model’s overall functionality. 


%\subsubsection{Evaluating Layer Heuristic}

To evaluate the layer importance heuristic (see Section~\ref{sec:layer_heuristic})  we tested our LLM on the \textit{triviaqa} LongBench dataset after reducing the memory allocated to different groups of layers. We selected a single dataset to minimize the computational cost of our empirical evaluation. \textit{triviaqa} was specifically chosen because it exhibits the widest range of scores, making it more sensitive to performance variations and thus a better candidate for detecting changes in output.

As described in algorithm~\ref{alg:layer_empirical}, we systematically reduced the memory budgets of layers within a sliding window of size 5, running a separate benchmark for each window position. Rather than evaluating individual layers in isolation, we compressed groups of 5 adjacent layers at a time. If crucial layers were arbitrarily scattered, rather than forming coherent clusters, it would suggest an unintuitive and unlikely distribution of importance. Additionally, testing each layer in isolation (i.e., using a window size of 1) yielded erratic results, indicating that individual layer evaluations do not capture meaningful patterns of layer importance. By considering contiguous groups, we aim to better approximate the true structure of importance within the model.




 



\begin{algorithm}
\caption{Empirical Evaluation of Layer Heuristic}
\begin{algorithmic}[1]
\REQUIRE LLM $M$, window size $W$, benchmark $BENCH$
\ENSURE Scores for each layer $S$, compression ratio $cmp$

\STATE $S \gets [\ ]$ \COMMENT{Initialize empty list for scores}

\FOR {$L \in \{0, \dots, \text{last\_layer}(M)\}$}
    \STATE $L_{\text{min}} \gets \max(0, L - \lfloor W/2 \rfloor)$
    \STATE $L_{\text{max}} \gets \min(\text{last\_layer}(M), L + \lfloor W/2 \rfloor)$

    \STATE $\text{reduce\_kv\_cache}(M, L_{\text{min}}, L_{\text{max}}, cmp)$ \COMMENT{Reduce KV-cache sizes for layers in window}

    \STATE $\text{score} \gets BENCH(M)$ \COMMENT{Run LLM and obtain benchmark score}

    \STATE $S.\text{append}(\text{score})$
\ENDFOR

\STATE RETURN $S$

\end{algorithmic}
\label{alg:layer_empirical}
\end{algorithm}


\section{Results and Discussion}
\label{sec05}

In this section, we present the results, discuss them, and make some conclusions about the experiments.

With a slightly realistic scenario, the experiments present some interesting results. Figures \ref{fig:hist_score} and \ref{fig:hist_gen} show respectively histograms of (a) the final score after the full training process and (b) the number of generations the process took. Notice that most runs just stopped at 20 generations (maximum) and could not improve further, as Figure \ref{fig:hist_gen} suggests. Despite that, as can be seen in Fig. \ref{fig:hist_score}, more than 80\% of the runs ended with a score of 2 or less, meaning at most two wrong device activations on 260 interactions. 

\begin{figure*}
        \centering
        \begin{subfigure}[b]{0.475\textwidth}
            \centering
            \includegraphics[width=0.8\textwidth]{imgs/results/results_2/histogram_of_score_.png}
            \caption[]%
            {{\small Histogram of \textit{score} achieved on experiment runs. Notice that the lesser, the better.}}    
            \label{fig:hist_score}
        \end{subfigure}
        \hfill
        \begin{subfigure}[b]{0.475\textwidth}  
            \centering 
            \includegraphics[width=0.8\textwidth]{imgs/results/results_2/histogram_of_generations_.png}
            \caption[]%
            {{\small Histogram of \textit{generations} needed to achieve the lower score. Here, most runs needed the maximum number of generations}}    
            \label{fig:hist_gen}
        \end{subfigure}
        \caption[]
        {\small Histograms of the lowest score and generations needed to achieve that on each experiment run.} 
        \label{fig:hist_metrics}
\end{figure*}

Figures \ref{fig:hist_beh} and \ref{fig:hist_per} reflect the number of Behavioral and Perceptual Codelets respectively to achieve the best result in each run. As we can see in Fig. \ref{fig:hist_beh}, most runs needed 13 Behavioral Codelets, one for each Motor Codelet/actuation device. Fig \ref{fig:corr_behavior} shows the correlation between the number of behavioral codelets and score. The system response is better (lower score) as more Behavioral Codelets are used.
The number of Perceptual Codelets, on the other hand, shows approximate normal distributions with a slight bias to the right, meaning that the embedding may vary and the output still be good. This bias is reflected in the slight negative correlation between the number of Perceptual Codelets and Score (smaller than Behavioral).

\begin{figure*}
        \centering
        \begin{subfigure}[b]{0.475\textwidth}
            \centering
            \includegraphics[width=0.8\textwidth]{imgs/results/results_2/histogram_of_behaviorals_x.png}
            \caption[]%
            {{\small Histogram of the number of Behavioral Codelets needed to achieve the lowest score. Most runs needed 13, the same number of Motor Codelets (and actuators).}}    
            \label{fig:hist_beh}
        \end{subfigure}
        \hfill
        \begin{subfigure}[b]{0.475\textwidth}  
            \centering 
            \includegraphics[width=0.8\textwidth]{imgs/results/results_2/histogram_of_perceptuals_x.png}
            \caption[]%
            {{\small Histogram of the number of Perceptual Codelets needed to achieve the lowest score.}}    
            \label{fig:hist_per}
        \end{subfigure}
        \caption[]
        {\small Histogram of the number of ''internal'' Codelets needed to achieve the lowest score on each run.} 
        \label{fig:needed_codelets_2}
\end{figure*}


\begin{figure*}
        \centering
        \begin{subfigure}[b]{0.475\textwidth}
            \centering
            \includegraphics[width=0.8\textwidth]{imgs/results/results_2/correlation_correlation_of_score_with_behaviorals.png}
            \caption[]%
            {{\small Correlation between number of Behavioral Codelets and Score}}    
            \label{fig:corr_behavior}
        \end{subfigure}
        \hfill
        \begin{subfigure}[b]{0.475\textwidth}  
            \centering 
            \includegraphics[width=0.8\textwidth]{imgs/results/results_2/correlation_correlation_of_score_with_perceptuals.png}
            \caption[]%
            {{\small Correlation between number of Perceptual Codelets and Score}}    
            \label{fig:corr_per}
        \end{subfigure}
        \caption[]
        {\small Correlation between the number of ''internal'' Codelets and Score} 
        \label{fig:corr}
\end{figure*}



Table \ref{table:exp2} shows some statistics taken from the experiment. Notice that, while the individual number of Perceptual and Behavioral Codelets may go as low as 2, the combined ''Internal'' Codelets need a higher number to present satisfactory results.

\begin{table}[]
\centering
\caption{metrics on Experiments}
\label{table:exp2}
\begin{tabular}{@{}llllll@{}}
\toprule
              & mean    & median & std   & min & max \\ \midrule
score         & 1.438   & 1.0    & 1.894 & 0   & 26  \\
generations   & 13.6784 & 20.0   & 8.905 & 0   & 20  \\
n perceptuals & 9.5326  & 10.0   & 2.213 & 2   & 15  \\
n behaviorals & 11.2674 & 13.0   & 2.478 & 2   & 13  \\
n internals   & 20.8    & 22.0   & 3.998 & 7   & 28  \\ \bottomrule
\end{tabular}
\end{table}

\subsection{Conclusion and Future Works}

This paper has presented a pioneering approach to creating a Cognitive Twin by leveraging a distributed cognitive system in conjunction with an evolution strategy. Our work stands as a significant contribution to the field of cognitive computing by demonstrating the feasibility of orchestrating a multitude of simple physical and virtual devices to mimic a person's interaction behaviors. This achievement not only offers a practical application of distributed cognitive systems but also introduces a novel methodology for cognitive twin development, emphasizing the role of evolution strategies in optimizing system topology for more accurate behavior emulation.

In revisiting the themes introduced at the outset, our research seamlessly integrates the foundational principles of cognitive systems, Cyber-Physical Systems (CPS), and Systems of Systems (SoS) with contemporary advancements in artificial intelligence. By doing so, we have illustrated a comprehensive framework that not only addresses the complexities of human behavior simulation but also opens new avenues for automation, human-like agent creation, and in-depth behavioral analysis.

Comparatively, our approach distinguishes itself from established cognitive architectures such as ACT-R and SOAR, and the Standard Model of Mind, by emphasizing distributed processing and adaptability. While ACT-R and SOAR offer rich insights into cognitive processes through detailed psychological models, our model excels in harnessing distributed, interconnected devices to capture the multifaceted nature of human cognition. Similarly, the Standard Model of Mind provides a foundational framework for understanding cognitive functions. Yet, our work extends this understanding into the practical domain of CPS and distributed systems, offering a unique perspective on cognitive replication and interaction dynamics.

In conclusion, our research not only underscores the potential of distributed cognitive systems in creating sophisticated cognitive twins but also highlights the importance of evolutionary strategies in refining these systems. By drawing parallels and distinguishing our work from established cognitive architectures like ACT-R, SOAR, and the Standard Model of Mind, we contribute a novel perspective to the ongoing discourse on cognitive modeling and simulation. 



Future work will focus on further refining the distributed cognitive system and exploring its integration with other AI paradigms and models. This research sets the stage for developing more sophisticated Cognitive Twins capable of performing complex tasks with minimal human intervention. By continuing to build on this foundation, future studies can enhance the fidelity and applicability of Cognitive Twins, making them tools in the field of cognitive computing.
    

\section{Related Works}\label{sec:related}
In this section, we discuss previous work relevant to BaKlaVa in five main areas: KV-cache eviction policy, profiling for determining memory budget, KV-cache quantization, cache merge, and system-level optimizations. 

\subsection{KV Cache Eviction Policy}
StreamingLLM~\cite{streamingllm} discovered the 'attention sink' effect, where early sequence tokens play a crucial role in maintaining model performance through asymmetric attention weight accumulation. H2O~\cite{h2o} introduces an eviction strategy based on cumulative attention, retaining 'heavy-hitter' key-value pairs while allowing token positions to vary. Similarly, Scissorhands~\cite{scissorhands} develops an approach that evicts based on a 'pivotal' metric, adjusting eviction rates across layers using a persistence ratio. Keyformer~\cite{keyformer} addresses the issue of token removal distorting softmax probability distributions by implementing regularization techniques to mitigate these perturbations. 

\subsection{Profiling for Determining Memory Budget}
Squeezeattention~\cite{squeezeattention} employs a dynamic approach, measuring layer importance through cosine similarity of input prompt differences pre- and post-self-attention, subsequently categorizing layers and adjusting their KV budgets. PyramidInfer~\cite{pyramidinfer} introduces a pyramid-shaped allocation strategy, prioritizing tokens with high attention values and maintaining a set of significant tokens through attention-driven updates during decoding. In comparison, Ada-KV~\cite{adakv} offers an adaptive budget allocation method that improves utilization across individual attention heads, resulting in more effective cache eviction strategies.



\subsection{KV-Cache Quantization}
GEAR~\cite{gear} takes a different approach by compressing less important entries to ultra-low precision, using a low-rank matrix for residual error approximation, and utilizing a sparse matrix for outlier correction. MiKV~\cite{notokenleftbehind} introduces a mixed-precision KV-cache quantization method, allocating precision based on token importance. QAQ~\cite{qaq} proposes a dynamic, quality-adaptive quantization approach that determines bit allocation based on token importance and sensitivity. KVQuant~\cite{hooper2024kvquant} offers strategies for smooth quantization of keys and values, including pre-RoPE quantization for keys, per-token quantization for values, and isolation of outliers in a sparse format. These diverse techniques collectively contribute to significant improvements in model compression and efficiency while maintaining performance.

\subsection{Cache Merge}
MiniCache~\cite{minicache} leverages the high angular similarity observed in middle-to-deep layer KV caches, merging key and value pairs from adjacent similar layers into shared representations. KVSharer~\cite{yang2024kvsharer}, on the other hand, exploits the counterintuitive finding that sharing KV caches between significantly different layers during inference does not substantially impact performance, prioritizing dissimilar layers for sharing based on Euclidean distance calculations. In comparison,  CaM~\cite{cam} focuses on merging keys or values of multiple evicted tokens with retained tokens using attention scores, while KVMerger~\cite{wang2024model} employs a two-step process: first clustering consecutive tokens with high cosine similarity, then merging tokens within each set into a pivotal token chosen by the attention score, using Gaussian kernel weights to emphasize contextual relevance. 

\subsection{System-Level Optimizations}
FlexGen~\cite{flexgen} proposes an SSD-based method for managing key-value (KV) caches, effectively expanding the memory hierarchy across GPU, CPU, and disk storage. This approach utilizes linear programming to optimize tensor storage and access patterns, enabling high-throughput LLM inference on hardware with limited resources. Complementing this, ALISA~\cite{alisa} introduces a dual-level KV cache scheduling framework that combines algorithmic sparsity with system-level optimization. At the algorithmic level, ALISA employs a Sparse Window Attention mechanism to identify and prioritize crucial tokens for attention computation, while at the system level, it implements a three-phase token-level dynamic scheduler to manage KV tensor allocation and balance caching and recomputation.


\section{Conclusion}\label{sec:conclusion}

In this paper, we proposed a prototype ASL generation system aimed at improving the naturalness, comprehensiveness, and overall quality of generated signs, addressing key limitations in existing approaches. Our technical evaluations indicate that our proposed approaches improve these aspects, enhancing the quality of generated ASL content. Feedback from DHH participants was mixed; while there was general interest in the system, concerns regarding visual quality and naturalness were noted. Reflecting on our design process and study findings, we discuss key insights and identify key areas for future improvement. While further work is needed, our study takes an initial step toward developing sign language generation systems that better meet the needs of the DHH and signing communities, offering real-world value.
\section{Acknowledgments}
We would like to thank Kostis Kaffes, Tanvir Ahmed Khan, and Yuhong Zhong for their feedback.
We also thank the CloudLab team for their help in supporting our experiments.
This work was supported by IBM, and NSF awards CNS-2143868 and CNS-2106530.
Tal Zussman was supported by NSF award DGE-2036197.
Ioannis Zarkadas is an Onassis Foundation scholar.


\printbibliography 

\documentclass[journal,compsoc]{IEEEtran}
\usepackage{epsfig}
\usepackage{graphicx}
\usepackage{amsmath}
\usepackage{amssymb}
\usepackage{algorithm, algorithmic}

\usepackage{diagbox}
\usepackage{float}
\usepackage{afterpage}
\usepackage{bm}
\usepackage{subfig}

%\usepackage{tabu}
\usepackage{multirow}
\usepackage{color}
\usepackage{tablefootnote}
\usepackage{adjustbox}
\usepackage{wrapfig}

\usepackage{hyperref}       % hyperlinks
\usepackage{url}            % simple URL typesetting
\usepackage{booktabs}       % professional-quality tables
\usepackage{amsfonts}       % blackboard math symbols
\usepackage{nicefrac}       % compact symbols for 1/2, etc.
\usepackage{microtype}      % microtypography
\usepackage{times}
\usepackage{epsfig}
%\usepackage{tabu}
%\usepackage{overpic}
\usepackage{bbding}
\usepackage{etoolbox}
\usepackage{paralist}
\usepackage{ulem}
\usepackage{tikz}

\usepackage{makecell}

\usepackage{xcolor,colortbl}

% \usepackage[pagebackref=true,breaklinks=true,colorlinks,bookmarks=false]{hyperref}


\newcolumntype{Y}{p{0.5cm}<{\centering}}
\newcommand{\mc}[2]{\multicolumn{#1}{c}{#2}}
\definecolor{Gray}{gray}{0.5}
\definecolor{LightCyan}{rgb}{0.88,1,1}

\newcolumntype{a}{>{\columncolor{Gray}}c}
\newcolumntype{b}{>{\columncolor{white}}c}



\DeclareMathOperator*{\cat}{Cat}


\def\H{\operatorname{H}}
\def\I{\operatorname{I}}
\def\KL{\operatorname{KL}}


\def\etal{\textit{et al}.}
\def\ie{\textit{i.e.}}
\def\eg{\textit{e.g.}}
\def\etc{\textit{etc}}
\def\wrt{\textit{w.r.t. }}

\def\bz{\textcolor{blue}}
\def\xc{\textcolor{red}}
\newcommand{\tb}[1]{\textbf{#1}}
\newcommand{\bc}[1]{\textcolor[RGB]{192,0,0}{#1}}
\newcommand{\rc}[1]{\textcolor{blue}{#1}}
% \newcommand{\rb}[1]{\textcolor{teal}{#1}}
% \newcommand{\bb}[1]{\textcolor{blue}{#1}}
\newcommand{\bb}[1]{\textcolor[RGB]{192,0,0}{\textbf{#1}}}
\newcommand{\rb}[1]{\textcolor{blue}{\textbf{#1}}}
\newcommand{\todo}[1]{{\color{blue}{[TODO: #1]}}}

% \newcommand{\rev}[1]{\textcolor{red}{#1}}
\newcommand{\rev}[1]{{#1}}
\renewcommand{\thefootnote}{\fnsymbol{footnote}}


\normalem
\begin{document}

\title{Rotation-Adaptive Point Cloud Domain Generalization via Intricate Orientation Learning \\
—— Supplementary Material —— }

\author{{Bangzhen~Liu,~Chenxi~Zheng,~Xuemiao~Xu,~Cheng Xu,~Huaidong~Zhang, \\ and~Shengfeng~He,~\IEEEmembership{Senior Member,~IEEE}}

\thanks{ Bangzhen Liu,~Chenxi~Zheng, and~Xuemiao~Xu are with the School of Computer Science and Engineering, South China University of Technology, Guangzhou, China. E-mail: liubz.scut@gmail.com,~cszcx@mail.scut.edu.cn, and~xuemx@scut.edu.cn.}
\thanks{ Cheng Xu is with the Centre for Smart Health, The Hong Kong Polytechnic University, Hong Kong. E-mail: cschengxu@gmail.com}
\thanks{ Huaidong Zhang is with the School of Future Technology, South China University of Technology, Guangzhou, China. E-mail: huaidongz@scut.edu.cn.}
\thanks{ Shengfeng He is with the School of Computing and Information Systems, Singapore Management University, Singapore. E-mail: shengfenghe@smu.edu.sg.}
}

\markboth{IEEE Transactions on Pattern Analysis and Machine Intelligence}%
{Shell \MakeLowercase{\textit{Liu et al.}}: Rotation-Adaptive Point Cloud Domain Generalization via Intricate Orientation Learning}


\maketitle

\IEEEdisplaynontitleabstractindextext

\IEEEpeerreviewmaketitle


\section{More Experimental Results} \label{sec1}

It is worth noting that the three sub-datasets used in PointDA are all category-wise imbalanced, as shown in Table~\ref{table:dataset}, which indicates that the micro-average precision score (\textit{Acc.}) reported by previous studies is inappropriate to assess the generalizability of cross-domain classification. In the main paper, we instead report the results of PointDA in the form of the macro-average precision score (\textit{Avg.}) for a more convincing evaluation. We also report the extra evaluations in the form of \textit{Acc.} in Table~\ref{tab:pointda10_acc} for reference. Our method still outperforms all the competitors in the average metric over the six cross-domain tasks.

\begin{table}[h]
    \caption{{Number of samples for each category in PointDA~\cite{qin2019pointdan}.}}
    \vspace{-2ex}
    \label{table:dataset}
    \scriptsize
    % \begin{center}
    \setlength{\tabcolsep}{0.05cm}{
      \resizebox{0.48\textwidth}{!}{
        \begin{tabular}{c|c|c|c|c|c|c|c|c|c|c|c}
          \hline  & Tub & Bed & Shelf & Case & Chair & Lamp & Monit. & Plant & Sofa & Table & Total\\
          \hline 
          ModelNet & 106 & 515 & 572 & 200 & 889 & 124 & 465 & 240 & 680 & 392 & 4183 \\
          ShapeNet & 599 & 167 & 310 & 1076 & 4612 & 1620 & 762 & 158 & 2198 & 5876 & 17378 \\
          ScanNet & 98 & 329 & 464 & 650 & 2578 & 161 & 210 & 88 & 495 & 1037 & 6110 \\
          \hline
          \end{tabular}
      }
     }
    % \end{center}
    \vspace{-2mm}
  \end{table}
  

\noindent\textbf{Evaluation on Aligned Dataset.} {We additionally implement our method under the traditional aligned data scenario, where the rotation only happens on the z-axis. In this case, we adapt our intricate orientation mining approach to specifically identify the most intricate orientations along the z-axis. 
The comparisons with state-of-the-art 3DDG methods are shown in Table~\ref{tab:align}, where the results of competitors are directly borrowed from their papers. Our method surpasses the baselines on all six tasks, demonstrating its effectiveness. The proposed orientation-aware contrastive training enables the model to gain a more comprehensive understanding of point clouds from various challenging perspectives, thereby enhancing the generalizability of the learned features. We notice that our method is slightly inferior on M$\to$S* and S$\to$S*. Since the orientational shift is our major concern, we do not have a special design for capturing geometric information under self-occlusions. However, in this case, our method still outperforms the two 3DDG methods on three out of the six tasks, while achieving the best average accuracy. Furthermore, the experimental results also reveal the presence of rotational shifts in the aligned data scenes, demonstrating the potential of our method for solving this problem.}

\begin{table}[h] % table for OSDA setting on Office31
    \caption{Comparison of the \textit{Acc.} ($\%$) under the 3D domain generalization setting. The best records are marked in \textbf{bold}.}
    \label{tab:align} 
    \vspace{-3ex}
    \small
    \begin{center}
    \setlength{\tabcolsep}{0.1cm}{
    \resizebox{0.48\textwidth}{!}{
    \begin{tabular}{c|c|c|c|c|c|c|c}
    \hline

    Methods
    &{M$\to$S} & {M$\to$S*} & {S$\to$M} & {S$\to$S*} & {S*$\to$M} & {S*$\to$S} & {Avg}\\

    \hline
    Supervised                      &{93.9} &{78.4} &{96.2} &{78.4} &{96.2} &{93.9} &{89.5}\\
    w/o Adapt                       &{83.3} &{43.8} &{75.5} &{42.5} &{63.8} &{64.2} &{62.2}\\
    \hline
    {Metasets~\cite{huang2021metasets}} &\tb{86.0} &{52.3} &{67.3} &{42.1} &{69.8} &{69.5} &{64.5}\\
    {PDG~\cite{wei2022learning}}        &{85.6} &\tb{57.9} &{73.1} &{50.0} &{70.3} &{66.3} &{67.2}\\
    % \hline
    {Ours}                              &{83.8} &{46.0} &\tb{83.2} &{45.5} &\tb{76.4} &\tb{70.3} &\tb{67.5}\\
    \hline

    \end{tabular}

    }
    }
    \end{center}
    \vspace{-3mm}
\end{table}


\begin{table*}[h] % table for OSDA setting on Office31
    \caption{Comparison of the micro-average precision score \textit{Acc.}~($\%$) under the orientation-aware 3D domain generalization setting. The top 2 records are marked in \bc{red} and \rc{blue}.}
    \label{tab:pointda10_acc} 
    \vspace{-3ex}
    \small
    \begin{center}
    \setlength{\tabcolsep}{0.35cm}{
    \resizebox{1\textwidth}{!}{
    \begin{tabular}{c|c|c|c|c|c|c|c|c}
    \hline

    {Methods}
    &Type &{M$\to$S} & {M$\to$S*} & {S$\to$M} & {S$\to$S*} & {S*$\to$M} & {S*$\to$S} & {AVG} \\

    
    % \cline{2-19}
    % \cline{14-19}  
    % \cmidrule $\pm$ r{2-13}
    % \cmidrule $\pm$ r{14-19}
    \hline
    Supervised                            &\multirow{2}{*}{-}            &{86.6 $\pm$ 6.3}    &{69.6 $\pm$ 3.2}    &{88.4 $\pm$ 14.7}   &{69.6 $\pm$ 3.2}    &{88.4 $\pm$ 14.7}  &{86.6 $\pm$ 6.3}    &{81.5}\\
    w/o Adapt                             &          &{57.9 $\pm$ 15.1}   &{28.7 $\pm$ 5.1}    &{54.1 $\pm$ 8.6}   &{28.8 $\pm$ 4.7}    &{43.0 $\pm$ 4.8}  &{42.3 $\pm$ 6.0}    &{42.5}\\
    \hline   
    VN~\cite{Deng_2021_ICCV}              &\multirow{2}{*}{RE}             &{70.5 $\pm$ 0.0} &{30.6 $\pm$ 0.0} &{66.7 $\pm$ 0.0} &{32.0 $\pm$ 0.0} &{39.4 $\pm$ 0.0} &{44.8 $\pm$ 0.0}   &{47.3} \\  
    SVN~\cite{su2022svnet}               &            &{66.8 $\pm$ 0.6} &{32.3 $\pm$ 0.4}   &{62.0 $\pm$ 0.5} &{30.0 $\pm$ 0.6}    &{38.0 $\pm$ 0.9} &{42.2 $\pm$ 1.1}   &{45.7} \\ 
    EOMP~\cite{luo2022equivariant}        &              &{61.4 $\pm$ 0.8} &{28.1 $\pm$ 0.3}    &{60.5 $\pm$ 0.6} &{37.0 $\pm$ 0.7}  &{27.9 $\pm$ 0.8} &{37.2 $\pm$ 0.9}    &{42.0} \\
    
    \hline
    SPRIN~\cite{you2021prin}              &\multirow{5}{*}{RI}             &{68.2 $\pm$ 0.4} &{30.1 $\pm$ 0.6}   &{71.8 $\pm$ 0.6} &{30.4 $\pm$ 0.6}    &{46.8 $\pm$ 0.6} &{49.3 $\pm$ 0.5}  &{49.4}\\
    RIPCA~\cite{li2021closer}              &            &{70.3 $\pm$ 1.2} &{33.0 $\pm$ 0.7}   &{70.4 $\pm$ 0.9} &{39.1 $\pm$ 1.3}    &\rc{49.9 $\pm$ 1.6} &{50.6 $\pm$ 2.2}  &\rc{52.2}\\
    RIConv++~\cite{zhang2022riconv}        &              &{28.8 $\pm$ 0.6} &{14.2 $\pm$ 0.5}   &{55.1 $\pm$ 0.7} &{38.9 $\pm$ 0.5}    &{34.8 $\pm$ 0.7} &{47.3 $\pm$ 0.5}  &{36.5}\\
    PaRI~\cite{chen2022devil}              &             &{36.1 $\pm$ 0.0} &{29.3 $\pm$ 0.3}   &{51.8 $\pm$ 0.8} &\bc{44.8 $\pm$ 0.4}    &{43.3 $\pm$ 0.9} &{49.4 $\pm$ 0.1} &{42.5} \\
    LocoTrans~\cite{chen2024local}              &                  &\bc{76.7 $\pm$ 0.0}   &{34.5 $\pm$ 0.3}    &\rc{74.3 $\pm$ 0.4}  &\rc{43.6 $\pm$ 0.2}  &{41.6 $\pm$ 0.6} &{41.5 $\pm$ 0.0}    &{52.0}\\
    \hline
    PointDAN~\cite{qin2019pointdan}       &\multirow{5}{*}{DA}              &{59.8 $\pm$ 15.1}   &{29.5 $\pm$ 4.0}    &{55.2 $\pm$ 6.9}   &{24.0 $\pm$ 4.8}    &{38.0 $\pm$ 4.8}  &{47.4 $\pm$ 6.1}    &{42.3}\\
    DefRec~\cite{achituve2021self}        &              &{57.2 $\pm$ 13.3}   &{33.1 $\pm$ 4.6}    &{54.4 $\pm$ 8.0}    &{33.1 $\pm$ 4.4}   &{38.8 $\pm$ 6.5}  &{48.2 $\pm$ 5.7}    &{44.1}\\
    GAST~\cite{zou2021geometry}           &              &{27.7 $\pm$ 4.2}   &{7.0 $\pm$ 0.6}     &{40.8 $\pm$ 2.6}   &{5.8 $\pm$ 0.8}     &{30.7 $\pm$ 1.5}  &{50.7 $\pm$ 3.6}     &{27.1}\\
    MLSP~\cite{liang2022point}            &              &{66.5 $\pm$ 15.5}   &{32.8 $\pm$ 4.3}    &{59.7 $\pm$ 5.1}   &{30.0 $\pm$ 6.4}    &{46.3 $\pm$ 5.0}  &{52.2 $\pm$ 5.8}     &{47.9}\\
    SDDA~\cite{cardace2023self}           &              &{65.0 $\pm$ 14.5}  &\rc{37.8 $\pm$ 3.4}     &{61.4 $\pm$ 5.4}    &{40.1 $\pm$ 4.1}   &{40.7 $\pm$ 6.3}  &\rc{53.3 $\pm$ 6.4}     &{49.7}\\
    PCFEA~\cite{wang2024progressive}     &                   &{62.0 $\pm$ 13.6}   &{9.3 $\pm$ 0.2}   &{42.7 $\pm$ 8.7}   &{43.1 $\pm$ 4.0}   &{47.1 $\pm$ 4.0}   &\bc{54.0 $\pm$ 4.6}    &{43.0}    \\
    \hline
    {Metasets~\cite{huang2021metasets}}   &\multirow{3}{*}{DG}               &{53.9 $\pm$ 1.4} &\bc{40.3 $\pm$ 0.9}   &{32.2 $\pm$ 12.3}  &{33.5 $\pm$ 1.7} &{24.5 $\pm$ 4.6}    &{39.8 $\pm$ 10.0} &{37.4}\\
    {PDG~\cite{wei2022learning}}          &             &{25.4 $\pm$ 29.5}   &{21.2 $\pm$ 18.0}   &{38.4 $\pm$ 18.5}   &{8.1 $\pm$ 3.2}   &{30.3 $\pm$ 4.8}   &{29.7 $\pm$ 12.0}    &{25.5}\\
    % \hline
    {Ours}                                &              &\rb{70.8 $\pm$ 2.0}   &{37.2 $\pm$ 1.2}    &\bb{80.7 $\pm$ 0.6}   &{34.0 $\pm$ 1.0}     &\bb{50.0 $\pm$ 2.5}   &{47.1 $\pm$ 3.2}     &\bb{53.3}  \\
    \hline

    \end{tabular}

    }
    }
    \end{center}
    \vspace{-3mm}
\end{table*}


\noindent\textbf{Analysis of Hyper-parameter Sensitivity.} 
We evaluate the effects of varying $\lambda_{oc}$ and $\lambda_{ms}$, by changing the value while keeping the other frozen as 0.1. As Fig.~\ref{fig:ablation}(a) and Fig.~\ref{fig:ablation}(b) show, $\lambda_{oc}$ is insensitive across a large range, while larger $\lambda_{ms}$ may slightly decrease the performance of our model. According to the variation of performance curves, we choose $\lambda_{oc}=0.01$ and $\lambda_{ms}=0.01$ as the model setting in our main paper.
\begin{figure}[h]
    % \flushleft
    \centering
    \subfloat[Ablation of $\lambda_{oc}$]{%[b]{0.45\textwidth}
        \label{fig:plot_lambda_oc}
        \includegraphics[width=0.22\textwidth]{./resources/supp/ablation_cons_weights.pdf}
    }
    % \hspace{2mm}
    \subfloat[Ablation of $\lambda_{ms}$]{%[b]{0.45\textwidth}
        \label{fig:plot_lambda_ms}
        \includegraphics[width=0.22\textwidth]{./resources/supp/ablation_reg_weights.pdf}
    }
    % \vspace{-2mm}
    \caption{The curves of performance \wrt varying $\lambda_{oc}$ and $\lambda_{ms}$.}
    \label{fig:ablation}
    % \vspace{-5.5mm}
\end{figure} 


\noindent\textbf{Analysis of Training Stability.} 
We plot the curves of the proposed orientation consistency loss and the marginal separation loss over the training stage to demonstrate the convergence of our intricate orientational learning. As Fig.~\ref{fig:plot}(a) and Fig.~\ref{fig:plot}(b) show, all the losses gradually decrease and converge to a convincing degree. The blue curves are the orientation consistency loss, which periodically bursts every 20 epochs. This is due to the update of the intricate orientation set, which gradually adapts the model to all the intricate orientations. At the end of the training stage, the amplification tends to be stable, indicating the consistency of the object towards various rotations.


\begin{figure}[h]
    % \flushleft
    \centering
    \subfloat[M$\to$S]{%[b]{0.45\textwidth}
        \label{fig:m2s_loss}
        \includegraphics[width=0.23\textwidth]{./resources/supp/m2s_loss.pdf}
    }
    % \hspace{2mm}
    \subfloat[M$\to$S*]{%[b]{0.45\textwidth}
        \label{fig:m2ss_loss}
        \includegraphics[width=0.23\textwidth]{./resources/supp/m2ss_loss.pdf}
    }
    \vspace{-2ex}
    \caption{The training curves (\ie, $L_{cls}$, $L_{oc}$, and $L_{ms}$) on M$\to$S (a) and M$\to$S* (b).}
    \label{fig:plot}
    \vspace{-3mm}
\end{figure} 



\noindent\textbf{Analysis of Time Complexity.} {We report the computational costs of training/testing one batch of data in milliseconds for different compared methods in Table~\ref{tab:complexity}. The results are obtained by accumulating the running times within a single training/testing epoch and calculating the mean value w.r.t. one batch.} Due to the process of diversifying the intricate orientation set, our method introduces extra computational costs in the training phase. Nonetheless, our method yields the best performance among these methods while achieving the second-best inferencing speed, which is more efficient than the other RE and RI methods that require extra time-consuming modules for practical applications. 


\begin{table}[h]
    \caption{Time statistics (ms) of training/testing on one batch of data.}
    \label{tab:complexity}
    \vspace{-3ex}
    \small
    \begin{center}
    \setlength{\tabcolsep}{0.3cm}{
    \resizebox{0.4\textwidth}{!}{
      \begin{tabular}{c|c|c|c|c} 
        \hline 
        Methods & Type & Avg. & $T_{train}$& $T_{test}$ \\
        \hline
        VN~\cite{Deng_2021_ICCV} & RE & 41.3 & 808 & 27.9 \\
        SPRIN~\cite{you2021prin} & RI & 43.9 & 1551 & 370.2 \\
        RIPCA~\cite{li2021closer} & RI & 46.6 & 717 & 23.3 \\
        MLSP~\cite{liang2022point} & DA & 43.2 & 825 & 36.5 \\
        SDDA~\cite{cardace2023self} & DA & 43.1 & \bf{567} & \bf{12.0} \\
        \hline
        Ours & DG & \bf{49.6} & 2114 & 14.5 \\
        \hline
        \end{tabular}
    }
    }
    \end{center}
    \vspace{-3mm}
  \end{table}
  


\section{Extra Visualizations and Analysis} \label{sec2}

\noindent\textbf{The Learned Intricate Augmented Samples.} {In Fig.~\ref{fig:intricat_angle}, we select several point clouds and provide visualizations of how their intricate orientations evolve during training. We trained the model on ModelNet and optimized the intricate set on the testing set every 20 epochs. Each row of the point cloud sequence shows the current pose of the given point cloud augmented by its corresponding intricate orientation at that specific epoch. 
Beneath each sequence, we also visualize the distribution of predicted probabilities and the consistency of prediction over different testing orientations. 
Specifically, for each point cloud, we obtain the predicted probabilities of its 64 testing variants $P = {\{P_a|P_a = \left[p^1_a, ..., p^C_a\right]\}}^A_{a=1}$, where $A=64$ is the number of testing orientation series and $C=10$ is the number of categories. 
The visualized probabilities' distribution $P_m$ is calculated by averaging the predictions over the 64 testing rotation series, such that $P_m = \left[\frac{1}{A}\sum_{j=1}^{A}p^1_j, ..., \frac{1}{A}\sum_{j=1}^{A}p^C_j\right]$. 
To evaluate the predicted consistency, we adopt the entropy as the metric and calculate the consistency $Ent_m$ over the 64 predicted probabilities by 
\begin{equation*}
  Ent_m = \left[\frac{1}{A}\sum_{j=1}^{A}p^1_j log p^1_j, ..., \frac{1}{A}\sum_{j=1}^{A}p^C_j log p^C_j\right].
\end{equation*}
As the number of training epochs increases, both the confidence and output consistency of the model are enhanced. For samples located near the decision boundaries, such as row 6 and row 9, learning with intricate orientation mining could significantly alleviate the ambiguity of learned features, thereby producing a more robust and generalizable classifier for downstream tasks.
}




\begin{figure*}[h]
    % \flushleft
    \centering
    \subfloat[Metasets]{%[b]{0.45\textwidth}
        \label{fig:cm1_m2s}
        % \centering
        \includegraphics[width=0.24\textwidth]{./resources/supp/conf_mat_modelnet2shapenet_Metaset.pdf}
        % \vspace{10mm}
    }
    \subfloat[PDG]{%[b]{0.45\textwidth}
        \label{fig:cm2_m2s}
        % \centering
        \includegraphics[width=0.24\textwidth]{./resources/supp/conf_mat_modelnet2shapenet_PDG.pdf}
        % \vspace{10mm}
    }
    \subfloat[Ours]{%[b]{0.45\textwidth}
        \label{fig:cm3_m2s}
        % \centering
        \includegraphics[width=0.24\textwidth]{./resources/supp/conf_mat_modelnet2shapenet.pdf}
        % \vspace{10mm}
    }
    \vspace{-2ex}
    \caption{The confusion matrices of Metaset, PDG, and our method on M$\to$S. Zoom in for details.}
    \vspace{-3ex}
    \label{fig:visualization_m2s}
\end{figure*} 
% \vspace{-4mm}
\begin{figure*}[h]
    % \flushleft
    \centering
    \subfloat[Metasets]{%[b]{0.45\textwidth}
        \label{fig:cm1_s2m}
        % \centering
        \includegraphics[width=0.24\textwidth]{./resources/supp/conf_mat_shapenet2modelnet_Metaset.pdf}
        % \vspace{10mm}
    }
    \subfloat[PDG]{%[b]{0.45\textwidth}
        \label{fig:cm2_s2m}
        % \centering
        \includegraphics[width=0.24\textwidth]{./resources/supp/conf_mat_shapenet2modelnet_PDG.pdf}
        % \vspace{10mm}
    }
    \subfloat[Ours]{%[b]{0.45\textwidth}
        \label{fig:cm3_s2m}
        % \centering
        \includegraphics[width=0.24\textwidth]{./resources/supp/conf_mat_shapenet2modelnet.pdf}
        % \vspace{10mm}
    }
    \vspace{-2ex}
    \caption{The confusion matrices of Metaset, PDG, and our method on S$\to$M. Zoom in for details.}
    \label{fig:visualization_s2m}
\end{figure*} 


\begin{figure*}[h]
    % \flushleft
    \centering
    \subfloat{\label{fig:0}
        \begin{minipage}[b]{1.0\textwidth}\centering
            \includegraphics[width=0.95\textwidth]{./resources/supp/underline.pdf} 
            \\
            \includegraphics[width=0.9\textwidth]{./resources/supp/all_title.pdf} 
            \\
            \includegraphics[width=0.9\textwidth]{./resources/supp/all_cat_0_cropped.pdf} 
            \\
            \vspace{-3mm}
            \includegraphics[width=1.0\textwidth]{./resources/supp/all_predictions_cat_0_cropped.pdf}
        \end{minipage}
    }\vspace{-3mm}

    \subfloat{\label{fig:1}
        \begin{minipage}[b]{1.0\textwidth}\centering
            \includegraphics[width=0.9\textwidth]{./resources/supp/all_cat_1_cropped.pdf} 
            \\
            \vspace{-3mm}
            \includegraphics[width=1.0\textwidth]{./resources/supp/all_predictions_cat_1_cropped.pdf}
        \end{minipage}
    }\vspace{-3mm}

    \subfloat{\label{fig:2}
        \begin{minipage}[b]{1.0\textwidth}\centering
            \includegraphics[width=0.9\textwidth]{./resources/supp/all_cat_2_cropped.pdf} 
            \\
            \vspace{-3mm}
            \includegraphics[width=1.0\textwidth]{./resources/supp/all_predictions_cat_2_cropped.pdf}
        \end{minipage}
    }\vspace{-3mm}

    \subfloat{\label{fig:4}
        \begin{minipage}[b]{1.0\textwidth}\centering
            \includegraphics[width=0.9\textwidth]{./resources/supp/all_cat_3_cropped.pdf} 
            \\
            \vspace{-3mm}
            \includegraphics[width=1.0\textwidth]{./resources/supp/all_predictions_cat_3_cropped.pdf}
        \end{minipage}
    }\vspace{-3mm}

    \subfloat{\label{fig:6}
        \begin{minipage}[b]{1.0\textwidth}\centering
            \includegraphics[width=0.9\textwidth]{./resources/supp/all_cat_4_cropped.pdf} 
            \\
            \vspace{-3mm}
            \includegraphics[width=1.0\textwidth]{./resources/supp/all_predictions_cat_4_cropped.pdf}
        \end{minipage}
    }\vspace{-3mm}

    \subfloat{\label{fig:7}
        \begin{minipage}[b]{1.0\textwidth}\centering
            \includegraphics[width=0.9\textwidth]{./resources/supp/all_cat_5_cropped.pdf} 
            \\
            \vspace{-3mm}
            \includegraphics[width=1.0\textwidth]{./resources/supp/all_predictions_cat_5_cropped.pdf}
        \end{minipage}
    }\vspace{-3mm}

    \subfloat{\label{fig:8}
        \begin{minipage}[b]{1.0\textwidth}\centering
            \includegraphics[width=0.9\textwidth]{./resources/supp/all_cat_6_cropped.pdf} 
            \\
            \vspace{-3mm}
            \includegraphics[width=1.0\textwidth]{./resources/supp/all_predictions_cat_6_cropped.pdf}
        \end{minipage}
    }

    \caption{Visualization of the learned intricate orientation series on ModelNet (M). Each row of point cloud sequence records the transformation of the point cloud's poses after augmention by its corresponding intricate rotational angle during the training procedure. The interval of recording is 20 epoch. The statistic measurements at every records, including the predicted probability and rotational consistency, are presented beneath each point cloud sequence. The category which the current point cloud belongs to is marked in \bc{red}.}
    \label{fig:intricat_angle}
    % \vspace{-6mm}
\end{figure*}



\noindent\textbf{Confusion Matrices.} {we provide the evaluation results for Metasets~\cite{huang2021metasets}, PDG~\cite{wei2022learning}, and our method in the form of confusion matrix on the target domain. ShapeNet is a dataset whose samples are highly imbalanced across different categories, while ModelNet is much more balanced. The confusion matrices of the three approaches are shown in  Fig.~\ref{fig:visualization_m2s} and Fig.~\ref{fig:visualization_s2m}. Compared with the other two 3D domain generalization methods, our method has more compact confusion matrices under the orientation shift. For M$\to$S, both our method and Metaset are separated relatively well while PDG has much inaccurate classification on class "Plant". The inner reason is that the part-based feature utilized by PDG may encounter confusing local expressions, such as the plane of the table and the bottom of a potted plant. For S$\to$M, our method achieves more balanced and concise results. We observe that the sample of class "monitor" is much easier to misclassify into "bed" due to the similar plane structure of their surface. Similar trends happen for the categories "table" and "cabinet", which have less discriminative features in the view of shape. }

{In summary, the single shape cannot serve as a discriminative representation in some cases. This is the limitation of shape representation under the orientation shift since there are a lot of objects whose shapes are similar but belong to different categories. In this case, extra visual (\eg, texture or color), linguistic information, or spatial cues are important to provide complement representation, which may benefit the problem of cross-domain generalization under orientation shift. We will plan to investigate the function of these features in our future work.  }

\section{Gradient of the rotation parameters} \label{sec3}
In this section, we provide detailed calculations about the optimizable parameters $\Theta$ concerning a given model $F$. Considering the objective of optimizing $\Theta$ within a standard classification task, we have the following objective:
\begin{equation}
  \hat{\Theta} = \mathop{\arg\max_{\Theta}}L(w_{opt}, \hat{P}, y), 
\end{equation}
where $w_{opt}$ is the freeze parameter of $F$, $(\hat{P}, y)$ are the augmented point cloud and label:
\begin{equation}
  \begin{split}
    ~\hat{P}&=f(\hat{\Theta}, P)  \\
    &=R_{\theta_{x}}\cdot R_{\theta_{y}}\cdot R_{\theta_{z}}\cdot P.
  \end{split}
\end{equation}
According to the chain rules, the gradient of $\hat{\Theta}$ is calculated by:
\begin{equation}
  \begin{split}
  \frac{\partial L}{\partial \hat{\Theta}} &= \frac{\partial L}{\partial \hat{P}} \frac{\partial \hat{P}}{\partial \hat{\Theta}} \\
  &=\frac{\partial L}{\partial \hat{P}}
  \left(
  \frac{\partial R_{\theta_{x}}}{\partial \theta_{x}}
  R_{\theta_{y}}
  R_{\theta_{z}} \quad
  R_{\theta_{x}}
  \frac{\partial R_{\theta_{y}}}{\partial \theta_{y}}
  R_{\theta_{z}} \quad
  R_{\theta_{x}}
  R_{\theta_{y}}
  \frac{\partial R_{\theta_{z}}}{\partial \theta_{z}}
  \right)P,
  \end{split}
\end{equation}
where 
\begin{equation}
  \begin{split}
R_{\theta_{x}} = 
\begin{pmatrix}
  1 & 0 & 0 \\
  0 & \cos\theta_{x} & -\sin\theta_{x} \\
  0 & \sin\theta_{x} & \cos\theta_{x} 
\end{pmatrix},
\\
R_{\theta_{y}} = 
\begin{pmatrix}
  \cos\theta_{y} & 0 & \sin\theta_{y} \\
  0 & 1 & 0 \\
  -\sin\theta_{y} & 0 & \cos\theta_{y}
\end{pmatrix},
\\
R_{\theta_{z}} = 
\begin{pmatrix}
  \cos\theta_{z} & -\sin\theta_{z} & 0  \\
  \sin\theta_{z} & \cos\theta_{z} & 0 \\
  0 & 0 & 0
\end{pmatrix},
\end{split}
\end{equation}
and 
\begin{equation}
    \begin{split}
\frac{\partial R_{\theta_{x}}}{\partial \theta_{x}} = 
\begin{pmatrix}
  0 & 0 & 0 \\
  0 & -\sin\theta_{x} & -\cos\theta_{x} \\
  0 & \cos\theta_{x} & -\sin\theta_{x} 
\end{pmatrix}, \\
\frac{\partial R_{\theta_{y}}}{\partial \theta_{y}} = 
\begin{pmatrix}
  -\sin\theta_{y} & 0 & \cos\theta_{y} \\
  0 & 0 & 0 \\
  -\cos\theta_{y} & 0 & -\sin\theta_{y}
\end{pmatrix}, \\
\frac{\partial R_{\theta_{z}}}{\partial \theta_{z}} = 
\begin{pmatrix}
  -\sin\theta_{z} & -\cos\theta_{z} & 0  \\
  \cos\theta_{z} & -\sin\theta_{z} & 0 \\
  0 & 0 & 1
\end{pmatrix}.
\end{split}
\end{equation}



\section{Theoretical Analysis for Rotation-Adaptive Point Cloud Domain Generalization} \label{sec4}

In this section, we provide theoretical proof demonstrating how orientational consistency functions to bridge the domain gap, analyzed from the perspective of mutual information reduction.

Let $X\!=\!(U, V)$ represent a 3D point cloud, where $U$ corresponds to orientation-dependent variables and $V$ to orientation-independent variables. In our case, we assume that the ranges of $U$ and $V$ remain consistent across domains.
For $X_s\!\sim\!p_\mathrm{src}(x)$, where $p_\mathrm{src}(x)$ denotes the source domain data distribution, the marginal distributions \wrt $U_s$ and $V_s$ are expressed by:
\begin{equation}
 p_\mathrm{src}(u)=\int p_\mathrm{src}(x) \mathrm{d}v, \quad p_\mathrm{src}(v)=\int p_\mathrm{src}(x) \mathrm{d}u.
\end{equation}
Considering the data distribution $X_a\!\sim\!p_\mathrm{aug}(x)$ after augmentation, where each sample is assumed to be uniformly sampled \wrt orientations, the marginal distributions \wrt $U_a$ and $V_a$ are given by:
\begin{equation}
 p_\mathrm{aug}(u)=\mathcal{U}(\mathcal{D}_{U_a}), \quad p_\mathrm{aug}(v)=p_\mathrm{src}(v),
\end{equation}
where $\mathcal{U}(\cdot)$ denotes a uniform distribution over the measurable domain $\mathcal{D}_{U_a}$ of ${U_a}$. For simplicity, the subscript of $U_a$ in $\mathcal{D}_{U_a}$ is omitted without causing ambiguity in the subsequent analysis. In this work, we adopt the proposed orientation-aware contrastive learning framework to approximately achieve this, where ${U_a}$ is represented by Euler angles and $\mathcal{D}_{U}:=[-\pi, \pi)^3$. 



Based on the definition of joint entropy, the entropy of $p_\mathrm{src}(x)$ can be expressed in terms of its marginal entropies \wrt $U_s$ and $V_s$, along with an additional term presenting the mutual information between these two components:
\begin{equation}
\begin{aligned}
    \H(X_s) 
 =& \H(U_s)+\H(V_s)-\I(U_s;V_s) \\
 =& \mathbb{E}_{u\sim p_\mathrm{src}(u)}[-\log p_\mathrm{src}(u)] + \mathbb{E}_{v\sim p_\mathrm{src}(v)}[-\log p_\mathrm{src}(v)] \\
    &- \mathbb{E}_{x\sim p_\mathrm{src}(x)}\log \frac{p_\mathrm{src}(x)}{p_\mathrm{src}(u)p_\mathrm{src}(v)},
\end{aligned}
\end{equation}
where $\I(U_s;V_s)$ represents the mutual information between $U_s$ and $V_s$ in $p_\mathrm{src}(x)$. 
Since $p_\mathrm{aug}(u)$ follows a uniform distribution and $U_a$ and $V_a$ of $p_\mathrm{aug}(x)$ are independent under this setting, the entropy of $p_\mathrm{aug}(x)$ is given by $\I_\mathrm{aug}(U_a;V_a)\!=\!0$, and the entropy of $p_\mathrm{aug}(u)$ corresponds to the measure of $\mathcal{D}_{U}$, denoted as $m(\mathcal{D}_{U})$. 
Thus, the entropy of $p_\mathrm{aug}(x)$ can be simplified as follows:
\begin{equation}
\begin{aligned}
    \H(X_a) &= \H(U_a)+\H(V_a) \\
    &= \log m(\mathcal{D}_{U}) + \mathbb{E}_{v\sim p_\mathrm{aug}(v)}[-\log p_\mathrm{aug}(v)],
\end{aligned}
\end{equation}
where $m(\mathcal{D}_{U})\!=\!(2\pi)^3$ in our case.

We use the KL divergence to quantify the distributional shift between the source and the target distribution.
For any $X_t\!\sim\!p_\mathrm{tgt}(x)$, where $p_\mathrm{tgt}(x)$ represents the target domain distribution, the KL divergence between $p_\mathrm{tgt}(x)$ and $p_\mathrm{src}(x)$ (or $p_\mathrm{aug}(x)$) can be computed once the cross-entropy between them is known. 
However, directly calculating the cross-entropy between $p_\mathrm{tgt}(x)$ and $p_\mathrm{src}(x)$ (or $p_\mathrm{aug}(x)$) is intractable, and it is often treated as an optimization objective to minimize. Notably, the cross-entropy between $p_\mathrm{tgt}(x)$ and $p_\mathrm{src}(x)$ (or $p_\mathrm{aug}(x)$) shares the same upper bound, as the samples $X_s$, $X_a$, and $X_t$ all share the same dimensionality:
\begin{equation}
    \sup_{p_\mathrm{src}}{\H(p_\mathrm{tgt}, p_\mathrm{src})} = \sup_{p_\mathrm{aug}}{\H(p_\mathrm{tgt}, p_\mathrm{aug})} = \log ({m(\mathcal{D}_U) \times m(\mathcal{D}_V)}).
\end{equation}
Here, $\mathcal{D}_V$ is the measurable domain of $V_s$, $V_a$, and $V_t$.
It is straightforward to prove that $\H_\mathrm{aug}(X_a) > \H_\mathrm{src}(X_s)$, as the mutual information is non-negative and entropy reaches its upper bound when the distribution is uniform.
Therefore, the relation between the upper bound of the KL divergence from $p_\mathrm{tgt}(x)$ to $p_\mathrm{src}(x)$ and from $p_\mathrm{tgt}(x)$ to $p_\mathrm{aug}(x)$ can be expressed as:
\begin{equation}
\begin{aligned}
    \sup_{p_\mathrm{src}}{\KL(p_\mathrm{tgt}||p_\mathrm{src})} &= \sup_{p_\mathrm{src}}{\H(p_\mathrm{tgt};p_\mathrm{src})} - \sup_{p_\mathrm{src}}{\H(X_\mathrm{s})} \\
    &> \sup_{p_\mathrm{aug}}{\H(p_\mathrm{tgt};p_\mathrm{aug})} - \sup_{p_\mathrm{aug}}{\H(X_\mathrm{a})} \\
    &= \sup_{p_\mathrm{aug}}{\KL(p_\mathrm{tgt}||p_\mathrm{aug})}. \label{eq:ieq}
\end{aligned}
\end{equation}
As revealed in Eq.~\ref{eq:ieq}, the upper bound of $\KL(p_\mathrm{tgt}||p_\mathrm{aug})$ is consistently lower than $\KL(p_\mathrm{tgt}||p_\mathrm{src})$, demonstrating the effectiveness of orientation invariance in reducing the domain shift under the disturbance of varying rotations. Consequently, the final upper bound of $\KL(p_\mathrm{tgt}||p_\mathrm{aug})$ is formally given as follows:
\begin{equation}
    \sup_{p_\mathrm{aug}}{\KL(p_\mathrm{tgt}||p_\mathrm{aug})} = \log m(\mathcal{D}_V) - \mathbb{E}_{v\sim p_\mathrm{aug}(v)}[-\log p(v)].
\end{equation}


\section{Limitation and Future Work} 
Although our method shows commendable advantages in handling cross-domain orientational shifts, it faces challenges with other complex types of domain shifts, such as heavy occlusions. This is because our framework does not offer an explicit design for tackling these domain shifts. Addressing this limitation, possibly through constructing a more powerful and versatile feature space resilient to multiple domain shifts via self-supervised pre-training, is a goal for future work.

{\small
\bibliographystyle{ieee_fullname}
\bibliography{egbib}
}

\end{document} 


\end{document}
\endinput
%%
%% End of file `sample-sigconf-biblatex.tex'.
