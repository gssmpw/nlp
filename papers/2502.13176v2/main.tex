\pdfoutput=1  % Force arXiv to use pdfLaTeX

%%
%% This is file `sample-sigconf-biblatex.tex',
%% generated with the docstrip utility.
%%
%% The original source files were:
%%
%% samples.dtx  (with options: `all,proceedings,sigconf-biblatex')
%% 
%% IMPORTANT NOTICE:
%% 
%% For the copyright see the source file.
%% 
%% Any modified versions of this file must be renamed
%% with new filenames distinct from sample-sigconf-biblatex.tex.
%% 
%% For distribution of the original source see the terms
%% for copying and modification in the file samples.dtx.
%% 
%% This generated file may be distributed as long as the
%% original source files, as listed above, are part of the
%% same distribution. (The sources need not necessarily be
%% in the same archive or directory.)
%%
%%
%% Commands for TeXCount
%TC:macro \cite [option:text,text]
%TC:macro \citep [option:text,text]
%TC:macro \citet [option:text,text]
%TC:envir table 0 1
%TC:envir table* 0 1
%TC:envir tabular [ignore] word
%TC:envir displaymath 0 word
%TC:envir math 0 word
%TC:envir comment 0 0
%%
%% The first command in your LaTeX source must be the \documentclass
%% command.
%%
%% For submission and review of your manuscript please change the
%% command to \documentclass[manuscript, screen, review]{acmart}.
%%
%% When submitting camera ready or to TAPS, please change the command
%% to \documentclass[sigconf]{acmart} or whichever template is required
%% for your publication.
%%
%%
%\documentclass[sigconf,natbib=false]{acmart}
\documentclass[sigconf,natbib=false, nonacm]{acmart} % ADD FOR PREPRINT SUBMISSION


%%
%% \BibTeX command to typeset BibTeX logo in the docs
\AtBeginDocument{%
  \providecommand\BibTeX{{%
    Bib\TeX}}}



%%
%% Submission ID.
%% Use this when submitting an article to a sponsored event. You'll
%% receive a unique submission ID from the organizers
%% of the event, and this ID should be used as the parameter to this command.
%%\acmSubmissionID{123-A56-BU3}

%%
%% For managing citations, it is recommended to use bibliography
%% files in BibTeX format.
%%
%% You can then either use BibTeX with the ACM-Reference-Format style,
%% or BibLaTeX with the acmnumeric or acmauthoryear sytles, that include
%% support for advanced citation of software artefact from the
%% biblatex-software package, also separately available on CTAN.
%%
%% Look at the sample-*-biblatex.tex files for templates showcasing
%% the biblatex styles.
%%


%%
%% The majority of ACM publications use numbered citations and
%% references, obtained by selecting the acmnumeric BibLaTeX style.
%% The acmauthoryear BibLaTeX style switches to the "author year" style.
%%
%% If you are preparing content for an event
%% sponsored by ACM SIGGRAPH, you must use the acmauthoryear style of
%% citations and references.
%%
%% Bibliography style
\RequirePackage[
  datamodel=acmdatamodel,
  style=acmnumeric,
  ]{biblatex}

%% Declare bibliography sources (one \addbibresource command per source)
\addbibresource{references.bib}


%% my imports
\usepackage{algorithm}
\usepackage{algcompatible}
\usepackage{dashrule}

\algrenewcommand\algorithmicrequire{\textbf{Input:}}
\algrenewcommand\algorithmicensure{\textbf{Output:}}
%%
%% end of the preamble, start of the body of the document source.
\begin{document}

%%
%% The "title" command has an optional parameter,
%% allowing the author to define a "short title" to be used in page headers.
\title{BaKlaVa - Budgeted Allocation of KV cache for Long-context Inference}

%%
%% The "author" command and its associated commands are used to define
%% the authors and their affiliations.
%% Of note is the shared affiliation of the first two authors, and the
%% "authornote" and "authornotemark" commands
%% used to denote shared contribution to the research.

\author{Ahmed Burak Gulhan}
\affiliation{%
  \institution{The Pennsylvania State University}
  \city{State College}
  \state{PA}
  \country{USA}
  }
\email{gulhan@psu.edu}

\author{Krishna Teja Chitty-Venkata}
\affiliation{%
  \institution{Argonne National Laboratory}
  \city{Lemont}
  \state{IL}
  \country{USA}
  }
\email{schittyvenkata@anl.gov}

\author{Murali Emani}
\affiliation{%
  \institution{Argonne National Laboratory}
  \city{Lemont}
  \state{IL}
  \country{USA}
  }
\email{memani@anl.gov}

\author{Mahmut Kandemir}
\affiliation{%
  \institution{The Pennsylvania State University}
  \city{State College}
  \state{PA}
  \country{USA}
  }
\email{mtk2@psu.edu}

\author{Venkatram Vishwanath}
\affiliation{%
  \institution{Argonne National Laboratory}
  \city{Lemont}
  \state{IL}
  \country{USA}
  }
\email{venkat@anl.gov}

%%
%% By default, the full list of authors will be used in the page
%% headers. Often, this list is too long, and will overlap
%% other information printed in the page headers. This command allows
%% the author to define a more concise list
%% of authors' names for this purpose.
%\renewcommand{\shortauthors}{Trovato et al.}

%%
%% The abstract is a short summary of the work to be presented in the
%% article.
\begin{abstract}
In Large Language Model (LLM) inference, Key-Value (KV) caches (KV-caches) are essential for reducing time complexity. 
However, they result in a linear increase in GPU memory as the context length grows. While recent work explores KV-cache eviction and compression policies to reduce memory usage, they often consider uniform KV-caches across all attention heads, leading to suboptimal performance. We introduce BaKlaVa, a method to allocate optimal memory for individual KV-caches across the model by estimating the importance of each KV-cache. Our empirical analysis demonstrates that not all KV-caches are equally critical for LLM performance. 
Using a one-time profiling approach, BaKlaVa assigns optimal memory budgets to each KV-cache. We evaluated our method on LLaMA-3-8B, and Qwen2.5-7B 
%and Mistral-7B 
models, achieving up to a 70\% compression ratio while keeping baseline performance and delivering up to an order-of-magnitude accuracy improvement at higher compression levels. 
\end{abstract}

%%
%% The code below is generated by the tool at http://dl.acm.org/ccs.cfm.
%% Please copy and paste the code instead of the example below.
%%
\begin{CCSXML}
<ccs2012>
   <concept>
       <concept_id>10010147.10010257.10010293</concept_id>
       <concept_desc>Computing methodologies~Machine learning approaches</concept_desc>
       <concept_significance>500</concept_significance>
       </concept>
 </ccs2012>
\end{CCSXML}

\ccsdesc[500]{Computing methodologies~Machine learning approaches}

%%
%% Keywords. The author(s) should pick words that accurately describe
%% the work being presented. Separate the keywords with commas.
\keywords{LLM, KV-cache, inference, GPU memory}

%\received{20 February 2007}
%\received[revised]{12 March 2009}
%\received[accepted]{5 June 2009}

%%
%% This command processes the author and affiliation and title
%% information and builds the first part of the formatted document.
\maketitle


\section{Introduction}

Despite the remarkable capabilities of large language models (LLMs)~\cite{DBLP:conf/emnlp/QinZ0CYY23,DBLP:journals/corr/abs-2307-09288}, they often inevitably exhibit hallucinations due to incorrect or outdated knowledge embedded in their parameters~\cite{DBLP:journals/corr/abs-2309-01219, DBLP:journals/corr/abs-2302-12813, DBLP:journals/csur/JiLFYSXIBMF23}.
Given the significant time and expense required to retrain LLMs, there has been growing interest in \emph{model editing} (a.k.a., \emph{knowledge editing})~\cite{DBLP:conf/iclr/SinitsinPPPB20, DBLP:journals/corr/abs-2012-00363, DBLP:conf/acl/DaiDHSCW22, DBLP:conf/icml/MitchellLBMF22, DBLP:conf/nips/MengBAB22, DBLP:conf/iclr/MengSABB23, DBLP:conf/emnlp/YaoWT0LDC023, DBLP:conf/emnlp/ZhongWMPC23, DBLP:conf/icml/MaL0G24, DBLP:journals/corr/abs-2401-04700}, 
which aims to update the knowledge of LLMs cost-effectively.
Some existing methods of model editing achieve this by modifying model parameters, which can be generally divided into two categories~\cite{DBLP:journals/corr/abs-2308-07269, DBLP:conf/emnlp/YaoWT0LDC023}.
Specifically, one type is based on \emph{Meta-Learning}~\cite{DBLP:conf/emnlp/CaoAT21, DBLP:conf/acl/DaiDHSCW22}, while the other is based on \emph{Locate-then-Edit}~\cite{DBLP:conf/acl/DaiDHSCW22, DBLP:conf/nips/MengBAB22, DBLP:conf/iclr/MengSABB23}. This paper primarily focuses on the latter.

\begin{figure}[t]
  \centering
  \includegraphics[width=0.48\textwidth]{figures/demonstration.pdf}
  \vspace{-4mm}
  \caption{(a) Comparison of regular model editing and EAC. EAC compresses the editing information into the dimensions where the editing anchors are located. Here, we utilize the gradients generated during training and the magnitude of the updated knowledge vector to identify anchors. (b) Comparison of general downstream task performance before editing, after regular editing, and after constrained editing by EAC.}
  \vspace{-3mm}
  \label{demo}
\end{figure}

\emph{Sequential} model editing~\cite{DBLP:conf/emnlp/YaoWT0LDC023} can expedite the continual learning of LLMs where a series of consecutive edits are conducted.
This is very important in real-world scenarios because new knowledge continually appears, requiring the model to retain previous knowledge while conducting new edits. 
Some studies have experimentally revealed that in sequential editing, existing methods lead to a decrease in the general abilities of the model across downstream tasks~\cite{DBLP:journals/corr/abs-2401-04700, DBLP:conf/acl/GuptaRA24, DBLP:conf/acl/Yang0MLYC24, DBLP:conf/acl/HuC00024}. 
Besides, \citet{ma2024perturbation} have performed a theoretical analysis to elucidate the bottleneck of the general abilities during sequential editing.
However, previous work has not introduced an effective method that maintains editing performance while preserving general abilities in sequential editing.
This impacts model scalability and presents major challenges for continuous learning in LLMs.

In this paper, a statistical analysis is first conducted to help understand how the model is affected during sequential editing using two popular editing methods, including ROME~\cite{DBLP:conf/nips/MengBAB22} and MEMIT~\cite{DBLP:conf/iclr/MengSABB23}.
Matrix norms, particularly the L1 norm, have been shown to be effective indicators of matrix properties such as sparsity, stability, and conditioning, as evidenced by several theoretical works~\cite{kahan2013tutorial}. In our analysis of matrix norms, we observe significant deviations in the parameter matrix after sequential editing.
Besides, the semantic differences between the facts before and after editing are also visualized, and we find that the differences become larger as the deviation of the parameter matrix after editing increases.
Therefore, we assume that each edit during sequential editing not only updates the editing fact as expected but also unintentionally introduces non-trivial noise that can cause the edited model to deviate from its original semantics space.
Furthermore, the accumulation of non-trivial noise can amplify the negative impact on the general abilities of LLMs.

Inspired by these findings, a framework termed \textbf{E}diting \textbf{A}nchor \textbf{C}ompression (EAC) is proposed to constrain the deviation of the parameter matrix during sequential editing by reducing the norm of the update matrix at each step. 
As shown in Figure~\ref{demo}, EAC first selects a subset of dimension with a high product of gradient and magnitude values, namely editing anchors, that are considered crucial for encoding the new relation through a weighted gradient saliency map.
Retraining is then performed on the dimensions where these important editing anchors are located, effectively compressing the editing information.
By compressing information only in certain dimensions and leaving other dimensions unmodified, the deviation of the parameter matrix after editing is constrained. 
To further regulate changes in the L1 norm of the edited matrix to constrain the deviation, we incorporate a scored elastic net ~\cite{zou2005regularization} into the retraining process, optimizing the previously selected editing anchors.

To validate the effectiveness of the proposed EAC, experiments of applying EAC to \textbf{two popular editing methods} including ROME and MEMIT are conducted.
In addition, \textbf{three LLMs of varying sizes} including GPT2-XL~\cite{radford2019language}, LLaMA-3 (8B)~\cite{llama3} and LLaMA-2 (13B)~\cite{DBLP:journals/corr/abs-2307-09288} and \textbf{four representative tasks} including 
natural language inference~\cite{DBLP:conf/mlcw/DaganGM05}, 
summarization~\cite{gliwa-etal-2019-samsum},
open-domain question-answering~\cite{DBLP:journals/tacl/KwiatkowskiPRCP19},  
and sentiment analysis~\cite{DBLP:conf/emnlp/SocherPWCMNP13} are selected to extensively demonstrate the impact of model editing on the general abilities of LLMs. 
Experimental results demonstrate that in sequential editing, EAC can effectively preserve over 70\% of the general abilities of the model across downstream tasks and better retain the edited knowledge.

In summary, our contributions to this paper are three-fold:
(1) This paper statistically elucidates how deviations in the parameter matrix after editing are responsible for the decreased general abilities of the model across downstream tasks after sequential editing.
(2) A framework termed EAC is proposed, which ultimately aims to constrain the deviation of the parameter matrix after editing by compressing the editing information into editing anchors. 
(3) It is discovered that on models like GPT2-XL and LLaMA-3 (8B), EAC significantly preserves over 70\% of the general abilities across downstream tasks and retains the edited knowledge better.
\section{Background}
\label{sec:background}


\subsection{Code Review Automation}
Code review is a widely adopted practice among software developers where a reviewer examines changes submitted in a pull request \cite{hong2022commentfinder, ben2024improving, siow2020core}. If the pull request is not approved, the reviewer must describe the issues or improvements required, providing constructive feedback and identifying potential issues. This step involves review commment generation, which play a key role in the review process by generating review comments for a given code difference. These comments can be descriptive, offering detailed explanations of the issues, or actionable, suggesting specific solutions to address the problems identified \cite{ben2024improving}.


Various approaches have been explored to automate the code review comments process  \cite{tufano2023automating, tufano2024code, yang2024survey}. 
Early efforts centered on knowledge-based systems, which are designed to detect common issues in code. Although these traditional tools provide some support to programmers, they often fall short in addressing complex scenarios encountered during code reviews \cite{dehaerne2022code}. More recently, with advancements in deep learning, researchers have shifted their focus toward using large-language models to enhance the effectiveness of code issue detection and code review comment generation.

\subsection{Knowledge-based Code Review Comments Automation}

Knowledge-based systems (KBS) are software applications designed to emulate human expertise in specific domains by using a collection of rules, logic, and expert knowledge. KBS often consist of facts, rules, an explanation facility, and knowledge acquisition. In the context of software development, these systems are used to analyze the source code, identifying issues such as coding standard violations, bugs, and inefficiencies~\cite{singh2017evaluating, delaitre2015evaluating, ayewah2008using, habchi2018adopting}. By applying a vast set of predefined rules and best practices, they provide automated feedback and recommendations to developers. Tools such as FindBugs \cite{findBugs}, PMD \cite{pmd}, Checkstyle \cite{checkstyle}, and SonarQube \cite{sonarqube} are prominent examples of knowledge-based systems in code analysis, often referred to as static analyzers. These tools have been utilized since the early 1960s, initially to optimize compiler operations, and have since expanded to include debugging tools and software development frameworks \cite{stefanovic2020static, beller2016analyzing}.



\subsection{LLMs-based Code Review Comments Automation}
As the field of machine learning in software engineering evolves, researchers are increasingly leveraging machine learning (ML) and deep learning (DL) techniques to automate the generation of review comments \cite{li2022automating, tufano2022using, balachandran2013reducing, siow2020core, li2022auger, hong2022commentfinder}. Large language models (LLMs) are large-scale Transformer models, which are distinguished by their large number of parameters and extensive pre-training on diverse datasets.  Recently, LLMs have made substantial progress and have been applied across a broad spectrum of domains. Within the software engineering field, LLMs can be categorized into two main types: unified language models and code-specific models, each serving distinct purposes \cite{lu2023llama}.

Code-specific LLMs, such as CodeGen \cite{nijkamp2022codegen}, StarCoder \cite{li2023starcoder} and CodeLlama \cite{roziere2023code} are optimized to excel in code comprehension, code generation, and other programming-related tasks. These specialized models are increasingly utilized in code review activities to detect potential issues, suggest improvements, and automate review comments \cite{yang2024survey, lu2023llama}. 




\subsection{Retrieval-Augmented Generation}
Retrieval-Augmented Generation (RAG) is a general paradigm that enhances LLMs outputs by including relevant information retrieved from external databases into the input prompt \cite{gao2023retrieval}. Traditional LLMs generate responses based solely on the extensive data used in pre-training, which can result in limitations, especially when it comes to domain-specific, time-sensitive, or highly specialized information. RAG addresses these limitations by dynamically retrieving pertinent external knowledge, expanding the model's informational scope and allowing it to generate responses that are more accurate, up-to-date, and contextually relevant \cite{arslan2024business}. 

To build an effective end-to-end RAG pipeline, the system must first establish a comprehensive knowledge base. It requires a retrieval model that captures the semantic meaning of presented data, ensuring relevant information is retrieved. Finally, a capable LLM integrates this retrieved knowledge to generate accurate and coherent results \cite{ibtasham2024towards}.




\subsection{LLM as a Judge Mechanism}

LLM evaluators, often referred to as LLM-as-a-Judge, have gained significant attention due to their ability to align closely with human evaluators' judgments \cite{zhu2023judgelm, shi2024judging}. Their adaptability and scalability make them highly suitable for handling an increasing volume of evaluative tasks. 

Recent studies have shown that certain LLMs, such as Llama-3 70B and GPT-4 Turbo, exhibit strong alignment with human evaluators, making them promising candidates for automated judgment tasks \cite{thakur2024judging}

To enable such evaluations, a proper benchmarking system should be set up with specific components: \emph{prompt design}, which clearly instructs the LLM to evaluate based on a given metric, such as accuracy, relevance, or coherence; \emph{response presentation}, guiding the LLM to present its verdicts in a structured format; and \emph{scoring}, enabling the LLM to assign a score according to a predefined scale \cite{ibtasham2024towards}. Additionally, this evaluation system can be enriched with the ability to explain reasoning behind verdicts, which is a significant advantage of LLM-based evaluation \cite{zheng2023judging}. The LLM can outline the criteria it used to reach its judgment, offering deeper insights into its decision-making process.






%input {txt/3-motivation}

\section{The BaKlaVa Method}\label{sec:methods}
Our method for optimizing KV-cache memory allocation consists of 3 main steps for a given LLM: (i) A one-time collection of profiling data for a given prompt(s) (Algorithm~\ref{alg:profiling} -- step 1);  (ii) Using a heuristic to estimate the `importance' of KV caches,  which is also a one-time calculation  (Algorithm~\ref{alg:profiling} -- step 2); and (iii) Performing a parameter search to allocate memory accordingly (Algorithms~\ref{alg:kv-mem-allocation} and~\ref{alg:parameter_search}). 

All three steps in BaKlaVa only need to be run once. The most time-consuming part currently is Step (iii), where a parameter search is performed for the target compression level. For this parameter search, we quickly evaluate each parameter combination using `perplexity', rather than running a long-context evaluation benchmark, since perplexity does not require autoregressive token generation and consequently is much faster and gives a good approximation of actual performance. This parameter search, for the models we evaluated (which contain 7 to 8 billion parameters), takes 10 to 20 minutes on 8x A100 GPUs for around 200 combinations of parameters on 98k tokens for a chosen compression ratio. The number of tokens can be decreased for a proportional decrease in runtime, though they should be at least as much as the maximum context length being evaluated.

Once the ideal parameters for an LLM are obtained, no additional computation is required.  To begin inference, we initialize our custom huggingface transformers' KV-cache object to use in inference.   




\begin{algorithm}
\caption{One-Time Profiling for KV-Cache Importance}
\begin{algorithmic}[1]
\REQUIRE LLM model $\mathcal{M}$, one or more prompts $\mathcal{P}$ of varying lengths
\ENSURE Values indicating relative KV-cache importance

\STATE \textbf{Step 1:} Collect profiling data from prompts $\mathcal{P} = \{p_1, p_2, \dots, p_n\}$ of different lengths
\FOR{each prompt $p_i \in \mathcal{P}$}
    \STATE Run inference on $\mathcal{M}$ with $p_i$
    \FOR{each layer $l \in \mathcal{L}$}
        \FOR{each attention head $h$ in layer $l$}
            \STATE Compute token-wise cosine similarity between attention head input $V$ and output $SoftMax(QK^T) V$
            \STATE Compute the average across all cosine similarities to obtain a single similarity value $s_{il} \in \mathcal{S}$
        \ENDFOR
    \ENDFOR
\ENDFOR

\STATE \textbf{Step 2:} Convert attention head similarities to KV-cache importance using the number of attention heads per KV-cache group $g$
\FOR{each layer $l \in \mathcal{L}$}
    \FOR{each group of $g$ heads, denoted by $s_{il}, s_{(i+1)l}, \dots, s_{(i+g-1)l} \in \mathcal{S}$}
        \STATE Obtain similarity for the current KV cache, $KVsim \gets mean(s_{il}, \dots, s_{(i+g-1)l})$ 
        \STATE KV cache importance $I_{li} \gets 1-KVsim$
    \ENDFOR
\ENDFOR

\end{algorithmic}
\label{alg:profiling}
\end{algorithm}



\begin{figure}[h]
    \centering
    \vspace{-10pt}
    \includegraphics[width=0.8\linewidth]{figs/similarity_heuristic.pdf}   
    \vspace{-0.25in}  
    \caption{The attention-head similarity heuristic used in BaKlaVa. By taking the cosine similarity between the input and output, we can calculate how much change there is. The more change between the input and output of the attention head, the more important we assume it is.}
    %\vspace{-0.3in}
    \label{fig:dot_product_cos} 
\end{figure}

\begin{figure}[h]
    \centering
    %\includegraphics[width=1\linewidth]{figs/head_importances.pdf}   

    \includegraphics[width=1\linewidth]{figs/similarities/llama_xsum_similarity.png}
    %\vspace{-20pt}
    \includegraphics[width=1\linewidth]{figs/similarities/llama_lcc_similarity.png}
    %\vspace{-5pt}
    %\rule{\linewidth}{0.5pt} % Adds a thin horizontal line
    \hdashrule{\linewidth}{0.5pt}{1mm 1mm} % Dotted horizontal line
    %\vspace{-7pt}
    \includegraphics[width=1\linewidth]{figs/similarities/qwen_xsum_similarity.png}
    \includegraphics[width=1\linewidth]{figs/similarities/qwen_lcc_similarity.png}
    \caption{Cosine similarity heatmap for input and output of attention heads for two different prompts in LLaMA3-8B and Qwen2.5-7B. We chose three representative layers to illustrate that attention head consistency holds across different prompts. The X-axis shows the attention heads in a layer, Y-axis represents each token position in the prompt. Green and red outlines show the highest and lowest column similarity means per layer, that is, the most and least important attention heads respectively.  
    The order of average attention head similarities (the mean of each column, see Algorithm~\ref{alg:profiling}) stays highly consistent even across different prompts of different lengths, indicating that profiling an LLM one time is sufficient to make KV-cache importance estimations.}
    %\vspace{-0.3in}
    \label{fig:prompt_test} 
\end{figure}


\subsection{Determining KV-cache Importances}
\label{sec:determine_kv_importance}
\subsubsection{Head Importance Heuristic} 
To determine the significance of an individual attention head, we used several key observations to come up with a heuristic. The first is that the more change there is between the input and output of a structure in an LLM, the more important it is, as used in~\cite{ge2024model} to determine the type of tokens the individual attention heads focus on and in~\cite{squeezeattention, pyramidinfer} to determine the importance of LLM layers. Second, the attention matrix $softmax(QK^T)$ has been shown to be a high-quality heuristic that can determine individual token importances~\cite{h2o, scissorhands, keyformer} for KV-cache eviction. Lastly, the $V$ tensor contains key information about the tokens, which is not found in the attention score matrix~\cite{guo2024attention, devoto2024simple}. \textbf{Based on these observations, we propose the idea that the greater the change between the input} $V$ \textbf{and the output} $softmax(QK^T)V$ \textbf{ of this attention head, the more critical this attention head is for inference.}

As shown in Algorithm~\ref{alg:profiling}, to do this, we compare the input $V$ tokens (each token is a multidimensional vector) with the output of the attention head, $softmax(QK^T)V$, using cosine similarity, as shown in Figure~\ref{fig:dot_product_cos}. 
  
Each input and output token are individually compared using a cosine similarity value to determine the change in vector direction. To obtain a single value for each attention head, we first (i) get the mean of all tokens' cosine similarities within that head. The result of cosine similarity ranges from -1 to 1 and the maximum value is obtained when the two compared token vectors are identical (i.e., the attention score matrix is an identity matrix). We then (ii) normalize these values from range 0 to 1 to obtain a single similarity value, such that a value of 1 means identical input and outputs for all tokens in the attention head. Lastly, (iii) to obtain the `importance value', we take the complement of each mean similarity ($1 - similarity$). An importance value closer to 1 means a bigger difference between the input and output tokens, thus it has more importance and vice versa. 

We tested multiple token comparison methods, such as dot-product and KL divergence, and found that cosine similarity and dot-product both give similar results.  However, cosine similarity guarantees an output between -1 and 1, leading to simpler calculations. Therefore, we chose cosine-similarity for BaKlaVa. Note that cosine similarity measures the change in angle, but not magnitude, and thus another method that incorporates {\em both} magnitude and direction change may give better results. We left this for future work. 

\subsubsection{One-Time Profiling}
To determine the frequency of profiling needed, considering that attention head behavior can vary between inference steps and different prompts, 
we ran several experiments. We found that while the importance of individual tokens may change throughout inference, overall the importance value (calculated by taking the average of all token cosine similarities, see Algorithm~\ref{alg:profiling}, step 1) remains consistent, across different prompts. This is illustrated in Figure~\ref{fig:prompt_test}, where for two models, LlaMA3-8B and Qwen2.5-7B-Instruct, we profile two prompts of lengths around 350 and 2000 tokens from a text (XSum) and coding (LCC) dataset respectively, for three layers in both LLMs (not just to highlight these specific layers, but to illustrate that the consistency the attention head behavior holds across different layers). Each tile (in the heatmaps) represents the cosine similarity difference for a single token with the top 5 and bottom 5 importance heads outlined in green and red,  respectively. We can observe that the highest- and lowest-ranking attention heads stay highly consistent across prompts of different lengths and different types (i.e. text vs code). This suggests that, for a given LLM, a single profiling run with a sufficiently large prompt (i.e. few hundred tokens) is sufficient to determine the attention head importance values that can be applied for all future inferences.

\subsubsection{Grouped Query Attention} 
If the LLM employs Grouped Query Attention (GQA), an additional step is required before determining the KV cache memory budgets. Since our measurements assess changes at the level of individual attention heads rather than KV-caches, a direct assignment is not possible. In GQA, multiple attention heads \textit{share} the same KV-cache, meaning that memory budgets cannot be allocated separately for each head within the same group. To address this, we compute the mean of similarities across all attention heads within a group, obtaining a single similarity value per GQA group. This process is detailed in Step 2 of Algorithm~\ref{alg:profiling}



\subsubsection{Layer Importance Heuristic}
\label{sec:layer_heuristic}
The KV-cache importances we have found so far are used to allocate the GPU memory budget {\em within} a single layer in an LLM. Simply taking the average of all KV-cache importances does not find the correct layer importance, since our KV-cache importance heuristic is agnostic to several other important structures in an LLM (e.g., the feed-forward networks, layer normalization, etc). Based on our empirical testing results (see Section~\ref{sec:empirical}), we found that SqueezeAttention~\cite{squeezeattention} is a simple and low-overhead heuristic that closely, though not perfectly, matches the `ground truth' layer importances and use this heuristic to determine the layer-wise importance values in BaKlaVa. The SqueezeAttention heuristic takes the cosine similarity between the input and output of each LLM layer, thus capturing the total effect of {\em all} structures within the layer. 



\subsection{Assigning Memory Budgets to KV-Caches}

\subsubsection{Memory Allocation}
\label{sec:methods-memory-alloc} 

\begin{algorithm}
\caption{KV-Cache Memory Reallocation Based on Attention Head Importance}
\begin{algorithmic}[1]
\REQUIRE Importance scores $\mathbf{I} = \{I_1, I_2, ..., I_m\}$ for $m$ KV-caches, threshold $t$, reduction amount $r$
\ENSURE Adjusted KV-cache allocations

\STATE $\mathcal{L} \gets \{i \mid I_i < t\}$ \COMMENT{Identify KV-caches with low importance}
%\STATE $\mathcal{H} \gets \{i \mid I_i \geq t\}$ \COMMENT{Identify high-importance KV-caches}

\IF{$|\mathcal{L}| > m - 1$} 
    \STATE RETURN UNCHANGED KV-cache allocations 
    \COMMENT{If all KV-caches are low importance then do not do anything}
\ENDIF

\FOR{EACH $i \in \mathcal{L}$}
    \STATE REDUCE KV-cache size of $i$ by $r$
\ENDFOR

\STATE $n \gets |\mathcal{L}|$ \COMMENT{Number of KV-caches reduced}
\STATE $k \gets \min(n, m - n)$ \COMMENT{Limit reallocation up to the top n available high-importance caches}
\STATE $\mathcal{H} \gets$ TOP-$k$ ELEMENTS OF $\mathcal{H}$ BASED ON $I_i$ 
\STATE $\Delta r \gets \frac{n \times r}{k}$ \COMMENT{Compute adjusted increase per cache}

\FOR{EACH $j \in \mathcal{H}'$}
    \STATE INCREASE KV-cache size of $j$ BY $\Delta r$
\ENDFOR

\STATE RETURN updated KV-cache allocations

\end{algorithmic}
\label{alg:kv-mem-allocation}
\end{algorithm}

Once the importance values for each KV-cache and layer are obtained, the next step is to determine how to allocate memory budgets. 

Based on our observation of token similarities as shown in Figure~\ref{fig:prompt_test}, we find that low-importance attention heads are more consistent with how they change each individual token in a prompt (that is, the dot-product between input and output of the attention head has low variance), while other attention heads can display significant changes across tokens (that is, high variance in the token cosine similarity). Thus, to reduce the chances for decreasing the memory of KV-caches belonging to potentially critical attention heads, we take a conservative approach and only target KV-caches with an importance score below a threshold $t$ by a predetermined amount $r$, as shown in Algorithm~\ref{alg:kv-mem-allocation}. The freed memory is then assigned to up to top $n$ KV caches of highest importance (where $n$ is the number of low-importance KV-caches selected), in order to prioritize increasing memory for the most important KV-caches. 


\subsubsection{Parameter Search}
\label{sec:parameter_search}


\begin{algorithm}
\caption{Parameter Search Using Perplexity}
\begin{algorithmic}[1]
\REQUIRE Evaluation prompt $\mathcal{P}$, model $M$, context length $L$, list of parameter configurations $\mathcal{C}$, compression ratio $cmp$
\ENSURE Optimal parameters $p^*$

\STATE $best\_params \gets \emptyset$
\STATE $min\_loss \gets \infty$

\FOR {$\text{params} \in \mathcal{C}$}
    
    \STATE CACHE = $\text{MAKE\_CACHE}(\text{params}, cmp)$
    \STATE $losses \gets []$
    
    \FOR{$\text{chunk} \in \text{STEP}(\mathcal{P}, L)$} \COMMENT{Get chunks of tokens from prompt}
        \STATE $loss \gets \text{PERPLEXITY}(M, \text{chunk}, CACHE)$
        \STATE $losses.\text{append}(loss)$
        \STATE CACHE $\gets \text{RESET\_CACHE}(CACHE)$
    \ENDFOR

    \STATE $avg\_loss \gets \frac{\sum \text{losses}}{\text{len}(\text{losses})}$

    \IF{$avg\_loss < min\_loss$}
        \STATE $min\_loss \gets avg\_loss$
        \STATE $best\_params \gets \text{params}$
    \ENDIF

\ENDFOR

\STATE RETURN $best\_params$

\end{algorithmic}
\label{alg:parameter_search}

\end{algorithm}

To determine the optimal values for $r$ and $t$, we performed a parameter search over different compression values, as shown in Algorithm~\ref{alg:parameter_search}. We found that the ideal parameter `area' varies across different compressions. These results were calculated using perplexity, as it is much faster to find compared to LongBench and it is a good indicator of actual performance. Based on these observations, we chose the best-performing parameter pair for a given compression value when evaluating on LongBench. 

\subsection{Empirical Evaluation of Heuristics}
\label{sec:empirical}


To evaluate the effectiveness of our layer and KV-cache importance heuristics in comparison to the `true' importance, we conducted computationally intensive experiments. These tests empirically assessed the impact of individual layers
%and KV-caches 
on model performance by measuring the variation in benchmark scores resulting from modifications to each component. The underlying principle is that the greater the performance degradation caused by a change (e.g., memory reduction) in a layer or KV-cache, the more critical that component is to the model’s overall functionality. 


%\subsubsection{Evaluating Layer Heuristic}

To evaluate the layer importance heuristic (see Section~\ref{sec:layer_heuristic})  we tested our LLM on the \textit{triviaqa} LongBench dataset after reducing the memory allocated to different groups of layers. We selected a single dataset to minimize the computational cost of our empirical evaluation. \textit{triviaqa} was specifically chosen because it exhibits the widest range of scores, making it more sensitive to performance variations and thus a better candidate for detecting changes in output.

As described in algorithm~\ref{alg:layer_empirical}, we systematically reduced the memory budgets of layers within a sliding window of size 5, running a separate benchmark for each window position. Rather than evaluating individual layers in isolation, we compressed groups of 5 adjacent layers at a time. If crucial layers were arbitrarily scattered, rather than forming coherent clusters, it would suggest an unintuitive and unlikely distribution of importance. Additionally, testing each layer in isolation (i.e., using a window size of 1) yielded erratic results, indicating that individual layer evaluations do not capture meaningful patterns of layer importance. By considering contiguous groups, we aim to better approximate the true structure of importance within the model.




 



\begin{algorithm}
\caption{Empirical Evaluation of Layer Heuristic}
\begin{algorithmic}[1]
\REQUIRE LLM $M$, window size $W$, benchmark $BENCH$
\ENSURE Scores for each layer $S$, compression ratio $cmp$

\STATE $S \gets [\ ]$ \COMMENT{Initialize empty list for scores}

\FOR {$L \in \{0, \dots, \text{last\_layer}(M)\}$}
    \STATE $L_{\text{min}} \gets \max(0, L - \lfloor W/2 \rfloor)$
    \STATE $L_{\text{max}} \gets \min(\text{last\_layer}(M), L + \lfloor W/2 \rfloor)$

    \STATE $\text{reduce\_kv\_cache}(M, L_{\text{min}}, L_{\text{max}}, cmp)$ \COMMENT{Reduce KV-cache sizes for layers in window}

    \STATE $\text{score} \gets BENCH(M)$ \COMMENT{Run LLM and obtain benchmark score}

    \STATE $S.\text{append}(\text{score})$
\ENDFOR

\STATE RETURN $S$

\end{algorithmic}
\label{alg:layer_empirical}
\end{algorithm}


% \section{Discussion}

% \begin{figure}
  \centering
  \includegraphics[width=\linewidth]{figures/per_frame_boxplot.png}
  
  \caption{\label{fig:frame-boxplot} Comparison of the distribution of F1 scores across all frames for each model.}
\end{figure}
% \subsection{Model Performance}

% \subsubsection{Out-of-Domain Performance}


% \begin{table}
    \centering
    \begin{tabularx}{\linewidth}{Xcccc}
        \hline
        \textbf{Model} & \textbf{All} & \textbf{Amb} \\ 
        \hline
        % Qwen 2.5-7B     & 0.755 & 0.665 & 0.707 & 0.547 \\ % no candidates @ fe
        % Qwen 2.5-7B     & 0.668 & 0.665 & 0.666 & 0.500 \\ % cand @ fe 
        % Phi-4           & 0.798 & 0.717 & 0.756 & 0.607 \\ % no candidates @ fe
        % Phi-4           & 0.719 & 0.717 & 0.718 & 0.560 \\ % cand @ fe
        % Qwen 2.5-7B     & 91.76 & 90.95 \\ % cand @ fe 
        Phi-4                           & 0.375 & 0.262 \\ % Not finetuned
        % $\text{Phi-4}_{cand}$ w/o LF    & 0.927 & 0.918 \\ % Finetuned on candidates
        $\text{Phi-4}_{cand}$ w/o LF    & 0.882 & \textbf{0.862} \\ % Finetuned on candidates
        $\text{Phi-4}_{cand}$ w/ LF     & 0.894 & \textbf{0.862} \\ % Finetuned on candidates
        % $\text{Phi-4}_{cand}$ w/ LF     & \textbf{0.931} & \textbf{0.918} \\ % Finetuned on candidates
        \hline
        KAF-SPA             & 0.912 & 0.776 \\
        KGFI                & 0.924 & 0.844 \\
        CoFFTEA             & \textbf{0.926} & 0.850 \\
        \hline
    \end{tabularx}
    \caption{Results on frame identification using frame element predictions.}
    \label{tab:candidate_frame}
\end{table}
% \subsection{Frame Identification}
% Previous work~\cite{devasier-etal-2024-robust} explored the possibility of filtering candidate targets produced by matching potential lexical units using a frame identification model. To build upon this idea towards a single-step frame-semantic parsing method, we explore the potential of frame elements being used to filter out candidate targets. In this approach, no ground-truth frame inputs are given. This also removes the bias from the model assuming the input always has at least one frame element.

% We represent the LLM instructions using the JSON-exist representation as it performed the best in Table~\ref{tab:representation_performance}. We used Phi-4 for this experiment as it had a very high performance-to-size ratio, as shown in Table~\ref{tab:candidate_frame}. \todo{should run this on qwen-72b} We found that directly using the model performed poorly, likely due to bias in the model learning that each input contains the given frame. To address this, we fine-tuned the LLM using candidates from the training set and found a significant improvement in performance. \todo{add candidates examples}

% Performance on par with CoFFTEA, the previous-best frame identification system.
% Maybe qwen 72b will perform better.
\section{Related Work}
\label{sec:related}

\noindent\textbf{Maximum common subgraph search}. In the literature, there are quite a few studies on finding the maximum common subgraph, which solve the problem either exactly~\cite{levi1973note,mcgregor1982backtrack,abu2014maximum,krissinel2004common,suters2005new,mccreesh2016clique,vismara2008finding,zhoustrengthened,liu2020learning,liu2023hybrid,mccreesh2017partitioning} or approximately~\cite{choi2012efficient,rutgers2010approximate,xiao2009generative,zanfir2018deep,bai2021glsearch}. \underline{First}, among all those exact algorithms, they mainly focus on improving the \emph{practical} performance and most of them are backtracking (also known as branch-and-bound) algorithms~\cite{levi1973note,mcgregor1982backtrack}. Specifically, authors in~\cite{levi1973note,mcgregor1982backtrack} propose the first backtracking framework. The idea is to transform the problem of finding the maximum common subgraph between two given graphs to the problem of finding the maximum clique in the \emph{association graph}. Then, authors in~\cite{mccreesh2016clique,vismara2008finding} follow the previous framework and further improve it by employing the constraint programming techniques. However, these algorithms are all based on a large and dense association graph built from two given graphs, which thus suffer from the efficiency issue. To solve the issue, McCreesh et al.~\cite{mccreesh2017partitioning} propose a new backtracking framework, namely \texttt{McSplit}, which is not based on the maximum clique search problem. Recent works~\cite{zhoustrengthened,liu2020learning,liu2023hybrid} follow \texttt{McSplit} and improve the practical performance by optimizing the policies of branching via learning techniques. Among them, \texttt{McSplitDAL}~\cite{liu2023hybrid} runs faster than {\cheng others.}
% all previous methods. 
We note that some exact algorithms are designed 
% from the theoretical perspective
{\chengB to achieve improvements of theoretical time complexity}
~\cite{abu2014maximum,levi1973note,krissinel2004common,suters2005new}. They have gradually improved the worst-case time complexity from $O^*(1.19^{|V_Q||V_G|})$~\cite{levi1973note} to $O^*(|V_Q|^{(|V_G|+1)})$~\cite{krissinel2004common}, and {\cheng to} $O^*((|V_Q|+1)^{|V_G|})$~\cite{suters2005new},
% {\cheng which is the state-of-the-art to the best of our knowledge}. 
{\chengB which is our best-known {\YuiR worst-case} time complexity for the problem.}
However, these algorithms are of theoretical interests only and not efficient in practice. We remark that (1) our \texttt{RRSplit} not only runs faster than all previous algorithms in practice but also achieves the state-of-the-art worst case time complexity (i.e., $O^*((|V_Q|+1)^{|V_G|})$) in theory and (2) the heuristic polices proposed in~\cite{zhoustrengthened,liu2020learning,liu2023hybrid} are orthogonal to \texttt{RRSplit}. \underline{Second}, since the problem of finding the largest common subgraph is NP-hard, some researchers turn to solve it approximately in polynomial time. Some approximation algorithms include meta-heuristics~\cite{choi2012efficient,rutgers2010approximate}, spectra methods~\cite{xiao2009generative}, and learning-based methods~\cite{zanfir2018deep,bai2021glsearch}. We remark that these techniques cannot be applied to our exact algorithm directly.

\smallskip
\noindent\textbf{Subgraph matching}. Given a target graph and a query graph, subgraph matching aims to find from a target graph all those subgraphs isomorphic to a query graph. We note that maximum common subgraph search is a generalization of subgraph matching. Specifically, given two graphs $Q$ and $G$, maximum common subgraph search {\chengC would} reduce to subgraph matching if 
% one requires 
{\chengB we require}
that the found common subgraph has the size at least $|V(Q)|$ or $|V(G)|$. In recent decades, subgraph matching has been widely studied~\cite{bhattarai2019ceci,ullmann1976algorithm,sun2020rapidmatch,sun2020subgraph,shang2008taming,kim2023fast,han2013turboiso,han2019efficient,cordella2004sub,bi2016efficient,arai2023gup,jin2023circinus,sun2023efficient}. The majority of proposed solutions perform a backtracking search. Among these algorithms, the \emph{candidate filtering} technique, which is designed for removing unnecessary vertices from the target graph, has been shown to be important for improving the practical efficiency~\cite{bhattarai2019ceci,bi2016efficient,han2019efficient,han2013turboiso,kim2023fast}. The technique relies on an auxiliary data structure (e.g., a tree or a directed acyclic graph), which is obtained from the query graph (based on the implicit constraint that each vertex in the query graph must be mapped to a vertex in the found subgraph). We note that it is hard to apply candidate filtering {\cheng to find} the maximum common subgraph (since the mentioned constraint may not hold).
%
We remark that finding subgraphs exactly isomorphic to a query graph is too restrictive in some real applications due to the data quality issues and/or potential requirements of the fuzzy search (e.g., no result {\chengC would} be returned if there does not exist any subgraph isomorphic to a query graph). Motivated by this, we focus on finding the maximum common subgraph between two graphs in this paper.
\section{Conclusion}

In this paper, we introduce STeCa, a novel agent learning framework designed to enhance the performance of LLM agents in long-horizon tasks. 
STeCa identifies deviated actions through step-level reward comparisons and constructs calibration trajectories via reflection. 
These trajectories serve as critical data for reinforced training. Extensive experiments demonstrate that STeCa significantly outperforms baseline methods, with additional analyses underscoring its robust calibration capabilities.
\begin{acks}
    
\end{acks}

\printbibliography 

\newpage
\appendix

\renewcommand{\figurename}{Supplementary Figure}
\renewcommand{\tablename}{Supplementary Table}
\setcounter{figure}{0}
\setcounter{table}{0}

    



\section{Details of datasets}
This section provides additional details about the dataset used to evaluate the downstream tasks. \Cref{tab:disease_definition} lists the ICD-10 codes and medications used to identify the diagnoses for each disease. \Cref{tab:characteristic} presents the distribution of patient characteristics for each disease. \Cref{fig:nyu_langone_prevalence,fig:nyu_longisland_prevalence} illustrates the prevalence of each disease in the downstream tasks for the NYU Langone and NYU Long Island datasets, highlighting the imbalances present in these tasks.

\begin{table}[!htpb]
    \centering
    \caption{The definition of diseases in EHR by diagnosis codes and medications.}
    \begin{tabular}{lr}
    \toprule
         Disease &  Definition in EHR \\
    \midrule
       IPH  &  I61.0, I61.1, I61.2, I61.3, I61.4, I61.8, I61.9 \\
       IVH  &  I61.5, P52.1, P52.2, P52.3  \\
       ICH  &  IPH + IVH + I61.6, I62.9, P10.9, P52.4, P52.9 \\
       SDH  &  S06.5, I62.0 \\
       EDH  &  S06.4, I62.1 \\
       SAH  &  I60.*, S06.6, P52.5, P10.3  \\
       Tumor  &  C71.*, C79.3, D33.0, D33.1, D33.2, D33.3, D33.7, D33.9  \\
       Hydrocephalus  &  G91.* \\
       Edema  &  G93.1, G93.5, G93.6, G93.82, S06.1 \\
       \multirow{2}{*}{ADRD}  &  G23.1, G30.*, G31.01, G31.09, G31.83, G31.85, G31.9, F01.*, F02.*, F03.*, G31.84, G31.1, \\ 
       & \textbf{Medication:} DONEPEZIL, RIVASTIGMINE, GALANTAMINE, MEMANTINE, TACRINE \\ 
    \bottomrule
    \end{tabular}
    \label{tab:disease_definition}
\end{table}

\begin{table}[!htbp]
\centering
\caption{Demographic characteristics of patients associated with scans from the NYU Langone dataset, matched with electronic health records (EHR) and utilized in downstream tasks.}
\label{tab:characteristic}

 The characteristic table on NYU Langone dataset matched with EHR.
\begin{tabular}{ll|rr|r}
\toprule
                       \textbf{Cohort} &  &           \textbf{Male (\%)} &          \textbf{Female (\%)} &     \textbf{Age (std)} \\
\midrule
 --- & All (n=270,205) & 128,113 (47.41\%) & 142,092 (52.59\%) & 63.64 (19.68) \\
\midrule
       Tumor & Neg (n=260,704) & 123,338 (47.31\%) & 137,366 (52.69\%) & 63.85 (19.72) \\
             & Pos (n=9,501) &   4,775 (50.26\%) &   4,726 (49.74\%) & 57.80 (17.67) \\
\midrule
HCP & Neg (n=253,000) & 118,881 (46.99\%) & 134,119 (53.01\%) & 63.67 (19.72) \\
              & Pos (n=17,205) &   9,232 (53.66\%) &   7,973 (46.34\%) & 63.18 (19.11) \\
\midrule
Edema & Neg (n=242,576) & 112,987 (46.58\%) & 129,589 (53.42\%) & 63.96 (19.84) \\
      & Pos (n=27,629) &  15,126 (54.75\%) &  12,503 (45.25\%) & 60.81 (17.97) \\
\midrule
ADRD  & Neg (n=232,667) & 111,159 (47.78\%) & 121,508 (52.22\%) & 61.31 (19.55) \\
      & Pos (n=37,538) &  16,954 (45.16\%) &  20,584 (54.84\%) & 78.09 (13.30) \\
\midrule
          IPH & Neg (n=251,308) & 117,692 (46.83\%) & 133,616 (53.17\%) & 63.58 (19.82) \\
              & Pos (n=18,897) &  10,421 (55.15\%) &   8,476 (44.85\%) & 64.39 (17.69) \\
\midrule
          IVH & Neg (n=258,232) & 121,686 (47.12\%) & 136,546 (52.88\%) & 63.65 (19.79) \\
              & Pos (n=11,973) &   6,427 (53.68\%) &   5,546 (46.32\%) & 63.45 (17.19) \\
\midrule
          SDH & Neg (n=248,468) & 114,869 (46.23\%) & 133,599 (53.77\%) & 63.44 (19.78) \\
              & Pos (n=21,737) &  13,244 (60.93\%) &   8,493 (39.07\%) & 65.95 (18.33) \\
\midrule
          EDH & Neg (n=265,431) & 125,113 (47.14\%) & 140,318 (52.86\%) & 63.77 (19.64) \\
              & Pos (n=4,774) &   3,000 (62.84\%) &   1,774 (37.16\%) & 56.53 (20.75) \\
\midrule
          SAH & Neg (n=251,594) & 118,424 (47.07\%) & 133,170 (52.93\%) & 63.79 (19.76) \\
              & Pos (n=18,611) &   9,689 (52.06\%) &   8,922 (47.94\%) & 61.59 (18.49) \\
\midrule
          ICH & Neg (n=229,851) & 105,498 (45.90\%) & 124,353 (54.10\%) & 63.41 (19.93) \\
              & Pos (n=40,354) &  22,615 (56.04\%) &  17,739 (43.96\%) & 64.93 (18.14) \\
\bottomrule
\end{tabular}
\end{table}


\begin{table}[!h]
    \centering
    \caption*{\textbf{Supplementary \Cref{tab:characteristic} Continued.} Demographic characteristics of patients associated with scans from the NYU Long Island dataset, matched with electronic health records (EHR) and utilized in downstream tasks.}
\begin{tabular}{ll|rr|r}
\toprule
                       \textbf{Cohort} &  &           \textbf{Male (\%)} &          \textbf{Female (\%)} &     \textbf{Age (std)} \\
\midrule
--- & All (n=22,158) & 9,580 (43.23\%) & 12,578 (56.77\%) & 68.33 (18.14) \\
\midrule
Tumor & Neg (n=21,578) & 9,275 (42.98\%) & 12,303 (57.02\%) & 68.59 (18.08) \\
      & Pos (n=580) &   305 (52.59\%) &    275 (47.41\%) & 58.78 (17.79) \\
\midrule
HCP   & Neg (n=20,653) & 8,718 (42.21\%) & 11,935 (57.79\%) & 69.05 (17.90) \\
      & Pos (n=1,505) &   862 (57.28\%) &    643 (42.72\%) & 58.52 (18.48) \\
\midrule
Edema & Neg (n=19,402) & 8,068 (41.58\%) & 11,334 (58.42\%) & 68.89 (18.27) \\
      & Pos (n=2,756) & 1,512 (54.86\%) &  1,244 (45.14\%) & 64.36 (16.66) \\
\midrule
ADRD  & Neg (n=19,537) & 8,391 (42.95\%) & 11,146 (57.05\%) & 66.78 (18.28) \\
      & Pos (n=2,621) & 1,189 (45.36\%) &  1,432 (54.64\%) & 79.90 (11.77) \\
\midrule
IPH   & Neg (n=19,357) & 7,974 (41.19\%) & 11,383 (58.81\%) & 68.97 (18.27) \\
      & Pos (n=2,801) & 1,606 (57.34\%) &  1,195 (42.66\%) & 63.89 (16.48) \\
\midrule
IVH   & Neg (n=19,636) & 8,164 (41.58\%) & 11,472 (58.42\%) & 68.96 (18.22) \\
      & Pos (n=2,522) & 1,416 (56.15\%) &  1,106 (43.85\%) & 63.43 (16.66) \\
\midrule
SDH   & Neg (n=20,885) & 8,870 (42.47\%) & 12,015 (57.53\%) & 68.33 (18.21) \\
      & Pos (n=1,273) &   710 (55.77\%) &    563 (44.23\%) & 68.37 (16.83) \\
\midrule
EDH   & Neg (n=21,912) & 9,443 (43.10\%) & 12,469 (56.90\%) & 68.33 (18.16) \\
      & Pos (n=246) &   137 (55.69\%) &    109 (44.31\%) & 68.19 (15.59) \\
\midrule
SAH   & Neg (n=20,652) & 8,824 (42.73\%) & 11,828 (57.27\%) & 68.68 (18.12) \\
      & Pos (n=1,506) &   756 (50.20\%) &    750 (49.80\%) & 63.58 (17.65) \\
\midrule
ICH   & Neg (n=18,388) & 7,456 (40.55\%) & 10,932 (59.45\%) & 68.92 (18.35) \\
      & Pos (n=3,770) & 2,124 (56.34\%) &  1,646 (43.66\%) & 65.48 (16.77) \\
\bottomrule
\end{tabular}
\end{table}

\begin{figure}[!ht]
    \centering
    \includegraphics[width=0.8\textwidth]{images/NYU_Langone_prevalence.pdf}
    \caption{Disease prevalence of NYU Langone }
    \label{fig:nyu_langone_prevalence}
\end{figure}

\begin{figure}[!h]
    \centering
    \includegraphics[width=0.8\textwidth]{images/NYU_Longisland_prevalence.pdf}
    \caption{Disease prevalence of NYU Longisland dataset}
    \label{fig:nyu_longisland_prevalence}
\end{figure}



\section{Data augmentation details}
\label{sec:dataaug_details}
We applied Random Flipping across all three dimensions, Random Shift Intensity with offset $0.1$ for both pre-training and fine-tuning. For DINO training. random Gaussian Smoothing with sigma=$(0.5-1.0)$ is applied across all dimensions, Random Gamma Adjust is applied with gamma=$(0.2-1.0)$.


\section{Additional experiment results}
This section provides additional experimental results with more details.
Supplementary \Cref{fig:channels-ablation,fig:patches-ablation} compares the performance of the foundation model using different numbers of channels and patch sizes, demonstrating that the architecture design of our foundation model is optimal. 

Supplementary \Cref{fig:radar-comparison-merlin} compares our foundation model with a foundation CT model from previous studies, Merlin\cite{blankemeier2024merlinvisionlanguagefoundation}, which was trained on abdomen CT scans with corresponding radiology report pairs. Our model demonstrates superior performance on head CT scans.

Supplementary \Cref{fig:probing-comparison-gemini} compares our foundation model with Google CT Foundation model~\cite{yang2024advancingmultimodalmedicalcapabilities}, which was trained on large scale and diverse CT scans from different anatomy with corresponding radiology report pairs. Our model consistently shows improved performance across the board even though our model was pre-trained with less samples.

Supplementary \Cref{fig:probing_comparison} compares the performance on downstream tasks with various supervised tuning methods applied to foundation models pretrained with the MAE and DINO frameworks. Per-pathology comparisons are shown in Supplementary \Cref{fig:probing-comparison-perpath,fig:probing-comparison-perpath-dino}. Meanwhile, supplementary \Cref{fig:boxplot_scaling} complements \Cref{fig:scaling_law}, illustrating the per-pathology performances of foundation models pretrained with different scales of training data.

Supplementary \Cref{fig:batch_effect,fig:thickness-ablation} studies the impact of batch effect caused by different CT scan protocols of slice thickness and machine manufacturer. Detailed per-pathology performances are shown in Supplementary \Cref{fig:slice_thickness_per_pathology,fig:manufacturer_per_pathology}.

\begin{figure}[!htpb]
    \centering
    \makebox[\textwidth][l]{%
        \hspace{0.3\textwidth}\textbf{NYU Langone}
    } \\[0.2cm]
    \includegraphics[trim={0 0 0 0},clip,height=0.3\textwidth, width=0.3\textwidth]{figures/abla_chans/AUC_chans_NYU.pdf}
    \includegraphics[trim={0 0 0 0},clip,height=0.3\textwidth, width=0.55\textwidth]{figures/abla_chans/AP_chans_NYU.pdf}\\
    \makebox[\textwidth][l]{
        \hspace{0.34\textwidth}\textbf{RSNA}
    } \\[0.2cm]
    \includegraphics[trim={0 0 0 0},clip,height=0.3\textwidth, width=0.3\textwidth]{figures/abla_chans/AUC_chans_RSNA.pdf}
    \includegraphics[height=0.3\textwidth, width=0.55\textwidth]{figures/abla_chans/AP_chans_RSNA.pdf} 
    \caption{\textbf{Comparison of Different Channels Performance.} This plot compares the performance of models trained using different numbers of channels (channels from multiple HU intervals vs. a single HU interval). Across two datasets, models using three channels from different HU intervals consistently outperform those using a single channel with a fixed HU interval. All models were pre-trained on $100\%$ of the pretraining data with MAE.}
    \label{fig:channels-ablation}
\end{figure}


\begin{figure}[!htpb]
    \centering
    \makebox[\textwidth][l]{%
        \hspace{0.3\textwidth}\textbf{NYU Langone}
    } \\[0.2cm]
    \includegraphics[trim={0 0 0 0},clip,height=0.3\textwidth, width=0.3\textwidth]{figures/abla_patches/AUC_patches_NYU.pdf}
    \includegraphics[trim={0 0 0 0},clip,height=0.3\textwidth, width=0.55\textwidth]{figures/abla_patches/AP_patches_NYU.pdf}\\
    \makebox[\textwidth][l]{
        \hspace{0.34\textwidth}\textbf{RSNA}
    } \\[0.2cm]
    \includegraphics[trim={0 0 0 0},clip,height=0.3\textwidth, width=0.3\textwidth]{figures/abla_patches/AUC_patches_RSNA.pdf}
    \includegraphics[height=0.3\textwidth, width=0.55\textwidth]{figures/abla_patches/AP_patches_RSNA.pdf} 
    \caption{\textbf{Comparison of Different Patches Performance.} This plot compares the performance of models trained with different patch sizes (12 vs. 16). The results demonstrate that smaller patch sizes consistently achieve better performance. All models were pre-trained on $100\%$ of the pretraining data with MAE.}
    \label{fig:patches-ablation}
\end{figure}


\begin{figure*}
    \centering
    \makebox[\textwidth][l]{%
        \hspace{0.06\textwidth}
        \textbf{NYU Langone} \hspace{0.06\textwidth} \textbf{NYU Long Island} \hspace{0.11\textwidth} \textbf{RSNA} \hspace{0.18\textwidth} \textbf{CQ500}
    } \\[0.2cm]
    \includegraphics[trim={0 0 0 0},clip,height=0.21\textwidth, width=0.21\textwidth]{figures/abla_radarplot_merlin/AUC_NYU.pdf}
    \includegraphics[trim={0 0 0 0},clip,height=0.21\textwidth, width=0.21\textwidth]{figures/abla_radarplot_merlin/AUC_Longisland.pdf}
    \includegraphics[trim={0 0 0 0},clip,height=0.21\textwidth, width=0.21\textwidth]{figures/abla_radarplot_merlin/AUC_RSNA.pdf}
    \includegraphics[trim={0 0 0 0},clip,height=0.21\textwidth, width=0.35\textwidth]{figures/abla_radarplot_merlin/AUC_CQ500.pdf}\\[0.2cm]
    \includegraphics[height=0.21\textwidth, width=0.21\textwidth]{figures/abla_radarplot_merlin/AP_NYU.pdf} 
    \includegraphics[height=0.21\textwidth, width=0.21\textwidth]{figures/abla_radarplot_merlin/AP_Longisland.pdf} 
    \includegraphics[height=0.21\textwidth, width=0.21\textwidth]{figures/abla_radarplot_merlin/AP_RSNA.pdf}
    \includegraphics[height=0.21\textwidth, width=0.35\textwidth]{figures/abla_radarplot_merlin/AP_CQ500.pdf}
    \caption{\textbf{Comparison to previous 3D Foundation Model.} This plot compares the performance of our model with Merlin~\cite{blankemeier2024merlinvisionlanguagefoundation} and models trained from scratch across four datasets for our model and ResNet50-3D. Our DINO trained model is used in this comparison. Our model demonstrates consistently superior performance across majority of diseases, with the exception of epidural hemorrhage (EDH) in the CQ500 dataset.}
    \label{fig:radar-comparison-merlin}
\end{figure*}



\begin{figure*}
    \centering
    \makebox[\textwidth][l]{%
        \hspace{0.10\textwidth}
        \textbf{NYU Langone} \hspace{0.08\textwidth} \textbf{NYU Long Island} \hspace{0.1\textwidth} \textbf{RSNA} \hspace{0.15\textwidth} \textbf{CQ500}
    } \\[0.2cm]
    \includegraphics[trim={0 0 0 0},clip, width=0.22\textwidth]{figures/abla_probing/AUC_NYU.pdf}
    \includegraphics[trim={0 0 0 0},clip, width=0.22\textwidth]{figures/abla_probing/AUC_Longisland.pdf}
    \includegraphics[trim={0 0 0 0},clip, width=0.22\textwidth]{figures/abla_probing/AUC_RSNA.pdf}
    \includegraphics[trim={0 0 0 0},clip, width=0.28\textwidth]{figures/abla_probing/AUC_CQ500.pdf}
    \\[0.2cm]
    \includegraphics[width=0.22\textwidth]{figures/abla_probing/AP_NYU.pdf} 
    \includegraphics[width=0.22\textwidth]{figures/abla_probing/AP_Longisland.pdf} 
    \includegraphics[width=0.22\textwidth]{figures/abla_probing/AP_RSNA.pdf}
    \includegraphics[width=0.28\textwidth]{figures/abla_probing/AP_CQ500.pdf}
    \caption{\textbf{Comparison of Different Downstream Training Methods.} This plot illustrates the downstream performance of models evaluated using fine-tuning and various probing methods across four datasets. In most cases, the DINO pre-trained model outperforms the MAE pre-trained model. All models were pre-trained on $100\%$ of the available pretraining data.}
    \label{fig:probing_comparison}
\end{figure*}


\begin{figure}
\centering
\makebox[\textwidth][l]{%
    \hspace{0.39\textwidth}\textbf{RSNA}
} \\[0.2cm]
\includegraphics[trim={0 0 0mm 0},clip,height=0.27\textwidth]{figures/abla_gemini/AUC_RSNA_Gemini.pdf}
\includegraphics[trim={0 0 5mm 0},clip,height=0.27\textwidth]{figures/abla_gemini/AP_RSNA_Gemini.pdf}

\makebox[\textwidth][l]{%
    \hspace{0.38\textwidth}\textbf{CQ500}
} \\[0.2cm]
\includegraphics[trim={0 0 10mm 0},clip,height=0.345\textwidth]{figures/abla_gemini/AUC_CQ500_Gemini.pdf}
\includegraphics[trim={0 0 5mm 0},clip,height=0.345\textwidth]{figures/abla_gemini/AP_CQ500_Gemini.pdf}

\caption{\textbf{Performance comparison of linear probing for Our Model vs. Google CT Foundation model} This plot compares our model performance vs. Google CT Foundation model\cite{yang2024advancing} and Merlin \cite{blankemeier2024merlinvisionlanguagefoundation} across all diseases on RSNA and CQ500. Since Google CT Foundation moudel requires uploading data to Google Cloud (not allowed on our private data) for requesting model embeddings with model weights inaccessible, only public dataset comparison is provided in this study. Similar to other evaluations, we observed that our model outperforms Google CT Foundation model across the board with the only exception on Midline Shift for Google CT Foundation model and EDH for Merlin.}
\label{fig:probing-comparison-gemini}
\end{figure}



\begin{figure}
    \centering
    \makebox[\textwidth][l]{%
        \hspace{0.35\textwidth}\textbf{NYU Langone}
    } \\[0.2cm]
    \includegraphics[trim={0 0 120mm 0},clip,height=0.255\textwidth]{figures/abla_probing_perpath/DINO_AUC_NYU_Langone.pdf}
    \includegraphics[trim={0 0 0 0},clip,height=0.255\textwidth]{figures/abla_probing_perpath/DINO_AP_NYU_Langone.pdf} \\
    \makebox[\textwidth][l]{
        \hspace{0.35\textwidth}\textbf{NYU Long Island}
    } \\[0.2cm]
    \includegraphics[trim={0 0 120mm 0},clip,height=0.255\textwidth]{figures/abla_probing_perpath/DINO_AUC_NYU_Long_Island.pdf}
    \includegraphics[trim={0 0 0 0},clip,height=0.255\textwidth]{figures/abla_probing_perpath/DINO_AP_NYU_Long_Island.pdf} 
    \makebox[\textwidth][l]{
        \hspace{0.4\textwidth}\textbf{RSNA}
    } \\[0.2cm]
    \includegraphics[trim={0 0 120mm 0},clip,height=0.24\textwidth]{figures/abla_probing_perpath/DINO_AUC_RSNA.pdf}
    \hspace{5mm}
    \includegraphics[trim={0 0 0 0},clip,height=0.24\textwidth]{figures/abla_probing_perpath/DINO_AP_RSNA.pdf} 
    \makebox[\textwidth][l]{
        \hspace{0.4\textwidth}\textbf{CQ500}
    } \\[0.2cm]
    \includegraphics[trim={0 0 120mm 0},clip,height=0.30\textwidth]{figures/abla_probing_perpath/DINO_AUC_CQ500.pdf} \hspace{5mm}
    \includegraphics[trim={0 0 0 0},clip,height=0.30\textwidth]{figures/abla_probing_perpath/DINO_AP_CQ500.pdf} 
    \caption{\textbf{Performance comparison of supervised finetuning methods per pathology on the foundation model trained with DINO.} This plot breaks down the average performance across all diseases shown in Supplementary \Cref{fig:probing_comparison}. The results show that fine-tuning the entire network achieves the best performance in most scenarios. However, linear probing closely approaches finetuning performance for many diseases especially on small or imbalanced dataset, underscoring the capability of our pre-trained models to generate representations that adapt effectively to diverse disease detection tasks.}
    \label{fig:probing-comparison-perpath-dino}
\end{figure}

\begin{figure}
    \centering
    \makebox[\textwidth][l]{%
        \hspace{0.35\textwidth}\textbf{NYU Langone}
    } \\[0.2cm]
    \includegraphics[trim={0 0 0 0},clip,height=0.24\textwidth, width=0.3\textwidth]{figures/abla_probing_perpath/AUC_NYU.pdf}
    \includegraphics[trim={0 0 0 0},clip,height=0.24\textwidth, width=0.45\textwidth]{figures/abla_probing_perpath/AP_NYU.pdf}\\
    \makebox[\textwidth][l]{
        \hspace{0.35\textwidth}\textbf{NYU Long Island}
    } \\[0.2cm]
    \includegraphics[trim={0 0 0 0},clip,height=0.24\textwidth, width=0.3\textwidth]{figures/abla_probing_perpath/AUC_Longisland.pdf}
    \includegraphics[trim={0 0 0 0},clip,height=0.24\textwidth, width=0.45\textwidth]{figures/abla_probing_perpath/AP_Longisland.pdf} 
    \makebox[\textwidth][l]{
        \hspace{0.4\textwidth}\textbf{RSNA}
    } \\[0.2cm]
    \includegraphics[trim={0 0 0 0},clip,height=0.24\textwidth, width=0.3\textwidth]{figures/abla_probing_perpath/AUC_RSNA.pdf}
    \includegraphics[height=0.24\textwidth, width=0.45\textwidth]{figures/abla_probing_perpath/AP_RSNA.pdf} 
    \makebox[\textwidth][l]{
        \hspace{0.4\textwidth}\textbf{CQ500}
    } \\[0.2cm]
    \includegraphics[trim={0 0 120mm 0},clip,height=0.24\textwidth]{figures/abla_probing_perpath/AUC_CQ500.pdf}
    \includegraphics[trim={0 0 0 0},clip,height=0.24\textwidth]{figures/abla_probing_perpath/AP_CQ500.pdf} 
    \caption{\textbf{Performance comparison of supervised finetuning methods per pathology on the foundation model trained with MAE.} The results reveal that attentive probing is significantly more effective than linear probing, consistent with findings from~\cite{Chen2024}. Furthermore, for many diseases, the performance of probing models approaches that of fine-tuning, demonstrating that our pre-trained models produce adaptable representations capable of detecting diverse diseases.}
    \label{fig:probing-comparison-perpath}
\end{figure}









\begin{figure}
    \centering
    \textbf{NYU Langone} \\
    \includegraphics[trim={0 0 0 0},clip,height=0.24\textwidth, width=0.38\textwidth]{figures/abla_perpath_perf/AUC_NYU.pdf}
    \includegraphics[height=0.24\textwidth, width=0.45\textwidth]{figures/abla_perpath_perf/AP_NYU.pdf} \\
    \textbf{NYU Long Island} \\
    \includegraphics[trim={0 0 0 0},clip,height=0.24\textwidth, width=0.38\textwidth]{figures/abla_perpath_perf/AUC_Longisland.pdf}
    \includegraphics[height=0.24\textwidth, width=0.45\textwidth]{figures/abla_perpath_perf/AP_Longisland.pdf} \\
    \textbf{RSNA} \\
    \includegraphics[trim={0 0 0 0},clip,height=0.24\textwidth, width=0.38\textwidth]{figures/abla_perpath_perf/AUC_RSNA.pdf}
    \includegraphics[height=0.24\textwidth, width=0.45\textwidth]{figures/abla_perpath_perf/AP_RSNA.pdf}\\
    \textbf{CQ500} \\
    \includegraphics[trim={0 0 0 0},clip,height=0.24\textwidth, width=0.38\textwidth]{figures/abla_perpath_perf/AUC_CQ500.pdf}
    \includegraphics[height=0.24\textwidth, width=0.45\textwidth]{figures/abla_perpath_perf/AP_CQ500.pdf}
    \caption{\textbf{Performance for Different Percentage of Pre-training Samples (Per-Pathology).} This plot illustrates label efficiency for individual pathologies using Tukey plots, alongside the average performance across all diseases shown in \Cref{fig:scaling_law}. The results indicate that the majority of pathologies show improved downstream performance as the amount of pretraining data increases.}
    \label{fig:boxplot_scaling}
\end{figure}


\newpage

\section{Time complexity increase with reduced patch size}
\label{apd:self_attention_rate}
Assume we have 3D image input of shape $H\times W\times D$, patch size $P$ and reducing factor $s$. By time complexity of self-attention $O(n^2 d)$ for sequence length $n$ and embedding dimension $d$, the new time complexity after reducing patch size can be derived as
\begin{align*}
    O(n^2d)&=O((\frac{H\times W\times D}{(\frac{P}{s})^3})^2d) \\
           &=O((\frac{H\times W\times D}{P^3})^2 s^6d)  \\
           &=O(s^6)O(n_{ori}^2d)
\end{align*}
where $n_{ori}=\frac{H\times W\times D}{P^3}$ is the original sequence length before reducing patch size.



















\newpage
\begin{figure}[ht]
    \centering
    \includegraphics[width=\textwidth]{images/tsne_embedding_visualization_per_pathology.png}
    \caption{The 2D projection with t-SNE of CT volume representation extracted from the foundation model. Interestingly, certain subgroups still exhibited slightly better AUCs. For instance, scans with slice thicknesses between 1–4 mm (represented by light green points in the upper panel of \Cref{fig:batch_effect}) align with a specialized protocol for CT angiography (CTA), which uses contrast dye to improve diagnosis on particular diseases.}
    \label{fig:batch_effect}
\end{figure}


\begin{figure*}[ht]
    \centering
    \begin{subfigure}[b]{0.33\textwidth}
        \centering
        \includegraphics[width=\textwidth]{images/AUROC_vs_Slice_thickness_binned.png}
        \caption{AUROC Performance}
    \end{subfigure}
    \hfill
    \begin{subfigure}[b]{0.33\textwidth}
        \centering
        \includegraphics[width=\textwidth]{images/AUPRC_vs_Slice_thickness_binned.png}
        \caption{AUPRC Performance}
    \end{subfigure}
    \hfill
    \begin{subfigure}[b]{0.33\textwidth}
        \centering
        \includegraphics[width=\textwidth]{images/Histogram_of_slice_thickness_distribution_across_scans.png}
        \caption{Histogram of slice thickness distribution}
    \end{subfigure}
    \caption{The downstream task performances on various ranges of slice thickness.}
    \label{fig:thickness-ablation}
\end{figure*}


\begin{figure*}[ht]
    \centering
    \begin{subfigure}[b]{\textwidth}
        \centering
        \includegraphics[width=\textwidth]{images/AUROC_vs_slice_thickness_for_each_disease_category.png}
        \caption{AUROC Performance}
    \end{subfigure}
    \hfill
    \begin{subfigure}[b]{\textwidth}
        \centering
        \includegraphics[width=\textwidth]{images/AUPRC_vs_slice_thickness_for_eachdisease_category.png}
        \caption{AUPRC Performance}
    \end{subfigure}
    \hfill
    \begin{subfigure}[b]{\textwidth}
        \centering
        \includegraphics[width=\textwidth]{images/Ratio_of_positive_labels_vs_slice_thickness_for_each_disease_category.png}
        \caption{Ratio of Positive Labels}
    \end{subfigure}
    \caption{Performance for Each Slice Thickness Bin (Per Pathology).}
    \label{fig:slice_thickness_per_pathology}
\end{figure*}


\begin{figure*}[ht]
    \centering
    \begin{subfigure}[b]{0.3\textwidth}
        \centering
        \includegraphics[width=\textwidth]{images/AUROC_by_Disease_and_Manufacturer.png}
        \caption{AUROC Performance}
    \end{subfigure}
    \hfill
    \begin{subfigure}[b]{0.3\textwidth}
        \centering
        \includegraphics[width=\textwidth]{images/AUPRC_by_Disease_and_Manufacturer.png}
        \caption{AUPRC Performance}
    \end{subfigure}
    \hfill
    \begin{subfigure}[b]{0.39\textwidth}
        \centering
        \includegraphics[width=\textwidth]{images/Positive_Label_Ratio_by_Disease_and_Manufacturer.png}
        \caption{Distribution of Scans from Each Manufacturer}
    \end{subfigure}
    \caption{Performance for Each Manufacturer (Per Pathology).}
    \label{fig:manufacturer_per_pathology}
\end{figure*}







\end{document}
\endinput
%%
%% End of file `sample-sigconf-biblatex.tex'.
