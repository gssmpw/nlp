\section{Appendix}
\appendix

\section{Full LongBench Results}

 \begin{figure*}[!htb]
     \centering
     \includegraphics[width=1\linewidth]{figs/LlaMA3-8B_results.pdf}
     \caption{Comparison of BaKlaVa and other cache methods on LlaMA3-8B using LongBench for different compressions.}
     \label{fig:longbench_results_llama}
 \end{figure*}

 \begin{figure*}[!htb]
     \centering
     \includegraphics[width=1\linewidth]{figs/qwen2.5-7B-instruct_results.pdf}
     \caption{Comparison of BaKlaVa and other cache methods on Qwen2.5-7B using LongBench for different compressions.}
     \label{fig:longbench_results_qwen}
 \end{figure*}


 
\section{Parameter Search Results}
\begin{figure*}[!htb]
    \centering
    \includegraphics[width=1\linewidth]{figs/parameter_search.pdf}
    \caption{Parameter search results for BaKlaVa method on LlaMA3-8B model. Each compression ratio from 1.0 to 0.2 is shown as a separate heat-map. The X axis is the similarity threshold to select high-similarity/low-importance KV-caches. The Y axis is the low importance/high similarity KV-cache reduction \%. Green regions indicate optimal (low perplexity) parameter configurations, whereas red regions indicate non-optimal (high perplexity) regions. }
    \label{fig:enter-label}
\end{figure*}