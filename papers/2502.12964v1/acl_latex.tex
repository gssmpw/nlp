
\pdfoutput=1


\documentclass[11pt]{article}


\usepackage[preprint]{acl}

% Standard package includes
\usepackage{times}
\usepackage{latexsym}
\usepackage{lipsum}
\usepackage{wrapfig}
% For proper rendering and hyphenation of words containing Latin characters (including in bib files)
\usepackage[T1]{fontenc}
% For Vietnamese characters
% \usepackage[T5]{fontenc}
% See https://www.latex-project.org/help/documentation/encguide.pdf for other character sets

% This assumes your files are encoded as UTF8
\usepackage[utf8]{inputenc}

% This is not strictly necessary, and may be commented out,
% but it will improve the layout of the manuscript,
% and will typically save some space.
\usepackage{microtype}
\usepackage{tcolorbox}

% This is also not strictly necessary, and may be commented out.
% However, it will improve the aesthetics of text in
% the typewriter font.
\usepackage{inconsolata}
\usepackage{amsmath}
%Including images in your LaTeX document requires adding
%additional package(s)
\usepackage{graphicx}
\usepackage{multirow}
% custom packages
\usepackage{hyperref}
\usepackage{url}
\usepackage{natbib}
\PassOptionsToPackage{numbers, sort, compress}{natbib}
\usepackage[utf8]{inputenc} % allow utf-8 input
\usepackage[T1]{fontenc}    % use 8-bit T1 fonts
\usepackage{booktabs}       % professional-quality tables
\usepackage{amsfonts}       % blackboard math symbols
\usepackage{nicefrac}       % compact symbols for 1/2, etc.
\usepackage{microtype}      % microtypography
\usepackage{xcolor}         % colors
\usepackage{array}
\usepackage{microtype}
%\bibliographystyle{plainnat}
\usepackage{times}
\usepackage{latexsym}
\usepackage{adjustbox}
\usepackage{inconsolata}
\usepackage{tabularx}
\usepackage{graphicx} 
\usepackage{algorithm}
\usepackage{algorithmic}
\usepackage{subcaption}
\usepackage{xspace}
\usepackage{enumitem}







\newcommand{\chk}{CHOKE\xspace}



\title{Trust Me, I'm Wrong: High-Certainty Hallucinations in LLMs}


\author{
Adi Simhi\textsuperscript{1} \hspace{1em} Itay Itzhak\textsuperscript{1} \hspace{1em} Fazl Barez\textsuperscript{2} \hspace{1em} Gabriel Stanovsky\textsuperscript{3} \hspace{1em} Yonatan Belinkov\textsuperscript{1}\\
\thanks{\texttt{\{adi.simhi,itay.itzhak\}@campus.technion.ac.il}} 
\textsuperscript{1}Technion -- Israel Institute of Technology\\
\  \textsuperscript{2}University of Oxford \\
\textsuperscript{3}School of Computer Science and Engineering, The Hebrew University of Jerusalem\\
}





\begin{document}
\maketitle

\begin{abstract}


Large Language Models (LLMs) often generate outputs that lack grounding in real-world facts, a phenomenon known as \emph{hallucinations}.
Prior research has associated hallucinations with model uncertainty, leveraging this relationship for hallucination detection and mitigation.
In this paper, we challenge the underlying assumption that all hallucinations are associated with uncertainty. 
Using knowledge detection and uncertainty measurement methods, 
we demonstrate that models can hallucinate with high certainty even when they have the correct knowledge.
We further show that high-certainty hallucinations are consistent across models and datasets, distinctive enough to be singled out, and challenge existing mitigation methods.
Our findings reveal an overlooked aspect of hallucinations, emphasizing the need to understand their origins and improve mitigation strategies to enhance LLM safety.\footnote{Code at \url{https://github.com/technion-cs-nlp/Trust_me_Im_wrong}}
\end{abstract}






\section{Introduction}
LLMs can often \emph{hallucinate}---generate outputs ungrounded in real-world facts, hindering their reliability \citep{survey_of_hallucination_in_natural_language_generation, Towards_understanding_sycophancy_in_language_models, Calibrated_language_models_must_hallucinate}. 

Numerous studies have focused on identifying hallucinations, with a particular line of research highlighting a strong relationship between hallucinations and a model's uncertainty \citep{tjandra2024fine,SelfCheckGPT}.
These studies demonstrate that uncertainty estimation metrics, such as semantic entropy \citep{kuhn2023semantic}, sampling \citep{cole-etal-2023-selectively}, and probability \citep{feng2024don}, can be employed to detect and mitigate hallucinations based on an apparent correlation between uncertainty and hallucinations.




These findings spotlight the potential of uncertainty in addressing hallucinations but leave the nature of their connection unclear, calling for a deeper investigation of their relations.
In this work, we aim to investigate the relationship between hallucinations and uncertainty by examining hallucinations generated \emph{with certainty}.

Specifically, we focus on a category of hallucinations that occur even when the model possesses the necessary knowledge to generate grounded outputs. By focusing on these hallucinations, we eliminate hallucinations in which the model is certain of an incorrect answer, which it perceives as its true answer, see Figure \ref{fig:chk_certain}.


\begin{figure}
\centering
  \centering

\includegraphics[width=1.0\linewidth, trim=0 500 0 0, clip]{Figures/pdfs/certainty_wak.pdf}
\caption{\textbf{Do high-certainty hallucinations exist?} An illustrative categorization of hallucinations based on a model's knowledge and certainty. Highlighted is the phenomenon of high-certainty hallucinations (purple) -- where models confidently produce incorrect outputs, even when they have the correct knowledge. While other types of hallucinations can potentially be explained by the model not knowing, being mistaken, or uncertain, high-certainty hallucinations are harder to rationalize, making their existence particularly intriguing.}
\label{fig:chk_certain}
\end{figure}


 
 We term these hallucinations \textbf{CHOKE}: \textbf{C}ertain \textbf{H}allucinations \textbf{O}verriding \textbf{K}nown \textbf{E}vidence.

To detect \chk, we use the method suggested by \citet{simhi2024distinguishing} to identify hallucinations where a model knows the correct answer. Next, we estimate the model's certainty with three widely used but conceptually different methods: tokens probabilities \citep{feng2024don}, probability difference between the top two predicted tokens \citep{huang2023look},  and semantic entropy \citep{kuhn2023semantic}.


Our findings reveal that such hallucinations can occur in realistic scenarios, demonstrating that a model can hallucinate despite possessing the correct knowledge, and doing so with \emph{certainty}, as illustrated in Figure \ref{fig:chk_certain}.  
We find this phenomenon is broad, existing in two datasets with both pre-trained and instruction-tuned models. Furthermore, we show that \chk cannot simply be classified as mere noise by establishing they are consistent across contexts. Finally, we find that this issue can significantly impact the effectiveness of hallucination mitigation techniques, resulting in a non-negligible number of unmitigated hallucinations.
 These findings highlight the potential of \chk for exploring the underlying mechanisms of hallucinations, offering valuable insights that can deepen our understanding and improve LLM safety \citep{barez2025open}.


\textbf{Our contributions are three-fold:}

\begin{enumerate} 
    \item We demonstrate that hallucinations can manifest with high certainty, challenging the common belief that links hallucinations primarily to low model certainty. 


    \item We provide evidence that certain hallucinations are systematic rather than random noise, showing consistent reproduction of the same hallucinations across different settings.

    \item We evaluate existing mitigation techniques and demonstrate their limited effectiveness in addressing high-certainty hallucinations, underscoring the importance of better understanding this phenomenon.
\end{enumerate}


\section{Setup}
In this study, we explore decoder-only generative \slms, which aim at maximising the likelihood of speech samples represented as discrete tokens. We examine both purely speech-based \slms trained on speech tokens and joint speech-text \slms using interleaving strategies~\citep{spiritlm}. Similarly to ~\citet{twist, gslm}, we obtain speech tokens by quantising continuous latent representations of a self-supervised speech representation model using the k-means algorithm, often known as \emph{semantic tokens}. Specifically, we utilise a multilingual HuBERT~\citep{hubert} model running at $25$ Hz, as employed in ~\citet{twist}. We then train \slms by minimising the negative log-likelihood of the input segments.

Unless mentioned otherwise, all \slms are trained using \textbf{a single $A5000$ GPU ($24GB$ VRAM)} along with $16$ CPU cores for $24$ hours. We deliberately focus on this constrained compute budget, assuming that most academic labs can access similar resources, thereby ensuring the accessibility of our research. The training data is pre-processed, i.e. extracting HuBERT units and dividing data into chunks, and stored prior to model training. As a result, this pre-processing time is excluded from the compute budget. This approach, aligned with \citet{geiping2023cramming}, is practical since many research experiments utilise the same pre-processed data. We additionally do not count the time for running validation and visualisations as they are not used as part of the optimisation pipeline and only used for demonstration purposes.

\newpara{Evaluation metrics.} We assess all \slms using four distinct evaluation metrics. The first three are based on likelihood evaluation, while the fourth is a generative metric. For likelihood based modelling we consider sBLIMP~\cite{dunbar2021zero}, \emph{Spoken Story Cloze} (\ssc)), and \emph{Topic Story-Cloze} (\tsc)~\cite{twist}. For modelling-likelihood metrics, we evaluate the likelihood assigned by the \slms to pairs of speech utterances, consisting of a positive example and a distractor. We calculate the percent of pairs in which the \slm assigns higher likelihood to the positive sample. sBLIMP focuses on grammatical abilities thus the negative is ungrammatical version of the positive. \ssc and \tsc focus on semantic modelling abilities. In \ssc, the distractor suffix is taken from the original textual StoryCloze dataset~\citep{mostafazadeh2016corpus}, allowing to assess fine-grained semantic speech understanding. In \tsc, however, the distractor suffix is drawn from a different topic, enabling us to evaluate the model’s ability to understand the overall semantic concept. 

Finally, to assess the generative abilities of \slms, we compute \emph{generative perplexity} (GenPPL). Following the approach of ~\cite{gslm, twist}, we provide the \slm with a short speech prompt and generate speech tokens continuation. We use unit-vocoder with duration prediction to convert the tokens into speech~\citep{polyak2021speech, twist}. The generated speech is then transcribed, and its \ac{PPL} is evaluated using a pre-trained text \ac{LLM}. To minimise the impact of token repetition on \ac{PPL} measurements, we ground the generated text using diversity metrics derived from the auto-BLEU score~\citep{gslm}. Similarly to ~\citet{lin2024alignslm} we use bigram auto-BLEU. In other words, we ensure that all models achieve similar auto-BLEU scores, allowing for a fair comparison of \ac{PPL}. Specifically, we transcribe speech segments using Whisper-large-v$3$-turbo model~\citep{radford2023robust} and measure \ac{PPL} using Llama-$3.2$-$1$B model~\citep{grattafiori2024llama3herdmodels}. 

\newpara{Software efficiency.} To maximise performance within $24$ hours of model training, we leverage multiple efficient implementations. Through extensive performance testing, we found that using bfloat$16$ ~\cite{kalamkar2019study} alongside FlashAttention$2$ \citep{dao2023flashattention2fasterattentionbetter} and data packing provided the most efficient compute performance in our setup. We also experimented with model compilation using \texttt{torch.compile} \citep{ansel2024pytorch}, but it lacked native compatibility with FlashAttention$2$ at the time of our study, and its performance without FlashAttention$2$ was subpar. Future work could investigate this further with more efficient attention implementations~\cite{FA3, li2024flexattention}.

To enable rapid and scalable experimentation, we developed a specialised library for \slm training that supports various model architectures, training objectives, and evaluation metrics. This library accommodates both TWIST-style training, text-speech interleaving, preference optimisation, etc. We will open-source this package along all models weights and training recipes, aiming to empower the community to further explore \slms.

\section{Results}
We first describe communication patterns within the full chronological context of the game in \textit{League of Legends (LoL)}, separated into four sections based on changing coordination dynamics. Based on this context, we identify core factors players assess to decide when to participate in communication with other teammates. Afterward, we discuss how communication shapes player perceptions toward their teammates, showing player's wariness towards players actively engaging in communication. 

\subsection{Communication Patterns in Context}

We discuss the communication patterns among teammates within the game. We organize the data into chronological phases of the game for a structured analysis of how the context shapes communication patterns. 

\subsubsection{Pre-game stage}
Before gameplay begins, team communication opens with \textit{team drafting}, where players are assigned roles (Top, Mid, Bot, Support, or Jungle) and take turns picking or banning champions. In Solo Ranked mode, roles are pre-assigned based on player preferences selected before queueing. Once teams are set, all players enter \textit{champion select} stage, alternating champion picks and banning up to five champions per team. During this stage, communication is limited to text chat. The usernames are anonymized (i.e., replacing the name with aliases) to prevent queue dodging by checking third-party stats sites such as OP.GG\footnote{https://www.op.gg/}, leaving the chat as the only option to inform individual strengths and preferences. 

Team composition in \textit{LoL} is crucial to the strategy and outcome of the game~\cite{ong2015player}, setting the basis for future interactions. Most participants acknowledged the importance of balanced and synergistic team composition, especially as players move into higher ranks where team coordination outweighs individual excellence. Yet, we observed a distinct lack of verbal communication between the members during this period across all ranks. Participants attributed the lack of willingness to initiate a conversation on the dangers of starting the game on a bad footing. They prioritized ``not creating friction'' during this stage as negative impressions can propagate throughout the game. Some participants attempted communication to reduce such friction, such as P14, who stated,``\textit{If I had the time, I wanted to say that I will be banning [this Champion], just in case a player on my team wanted to play them.}'' However, several participants viewed any communication during the pre-game phase with wariness, as dissatisfaction or conflict at this step portended negative interactions between players in the game (P3, P9, P15). Thus, even when participants expressed doubt about other teammates' unconventional or non-meta champion picks, they refrained from entering into discourse. This contrasts with findings by Kou and Gui~\cite{kou2014}, which showed players attempt to maintain a harmonious and constructive atmosphere through greetings and introductions.

Another emergent code of the reason for not engaging in communication in the pre-game stage stems from different purposes of playing the game (P1, P5, P13, P16, P17). Despite being in ranked mode, which is more prone to increased competitiveness and effort, participants showed differing goals and levels of interest in winning the game. Several players stated that they had previously exerted great mental load in coordinating synergistic plays, but stopped as they gave less importance to winning at all costs (``\textit{I don't really play to win. I play \textit{LoL} to relieve stress, so I don't engage in chat.}'', P5). These players saw verbal communication with the goal of coordination as an unnecessary or even cumbersome component of the pre-game stage.


\subsubsection{Structured phase}
In many MOBAs, including \textit{LoL}, the early stages of the game play out in a formulaic manner: players join their lanes (Top, Mid, and Bot/Support), defeat minions to gain gold, buy items towards certain ``builds'', kill or assist in early objectives (Jungle), and battle counterparts in their respective lanes. Participants at this stage expressed that most players possessed tacit knowledge of what must be done, such as knowing when to aid their Jungle to capture a jungle monster, choosing the opportune moments to leave their lanes, or positioning wards (i.e., a deployable unit which provides a vision of the surrounding area) at the ideal placements. The participants assumed each player knew their ``role'' to fulfill, often comparing it to ``doing their share'' (P1, P3, P7, P19). In line with this belief, players rarely initiated preemptive or proactive verbal communication for strategic or social purposes at the early stage. 

Pings, on the other hand, constantly permeated the game. At this stage, players used ping to provide information relevant to others from their position, such as letting others know if an enemy went missing from their lane. As the players are largely separated and independent from one another, pings (coupled with the minimap and scoreboard) served as the primary channel for maintaining context over the game rather than as warnings or direct guidance to the players. For other non-verbal gestures, while objective votes would occasionally appear, they were rarely answered. Instead, relevant players near the objective would place pings or move toward it to help out their teammates.

Participants viewed the structured phase as a routine, but uncertain period of the game where the pendulum could swing in either team's favor. Players --- especially Jungles who roam the board looking for opportunities to ambush the enemy team in lanes (``gank'') --- sometimes felt hesitant to make calls and demands at this stage since ``\textit{[they] could make a call, but if I fail, they'll start blaming my decisions down the line.}'' (P7) But at this stage, participants believed that they held personal agency over the final game outcome. P1 and P6 stated that they entered the game with the mindset that only they had to succeed regardless of the performance of their teammates. This belief was reflected in their chatting behavior, where players prioritized focusing on their circumstances over the team's (``\textit{I mute the chat so that I don't get swayed by the team, as I can win the game if I do well.}'', P9).


\subsubsection{Group engagement phase}
As the game enters its middle phase, it provides opportunities for more diverse decision-making. Players may swap lanes, seize or trade crucial objectives, and fight in large battles involving multiple champions. At this point, teams typically have a clear outlook on which players and team have the advantage, requiring more team-driven decisions to maintain or overcome their current standing. Thus, players used verbal communication to discuss more complicated tactics that could not be effectively conveyed through pings.

But more often than not, chat messages became judgment-based. As enemy engagement with larger groups occurred more frequently, the availability for chatting would come after death, which led to comments on past actions rather than future choices. Additionally, the respawn timer for deaths becomes longer as the game progresses, providing more time to observe other players than in earlier phases. This gave players more opportunities to express dissatisfaction specifically towards certain plays, such as placing Enemy Missing pings on the map where other teammates are located to bring attention to their questionable play.

This stage also gave much more exposure of each other to the allies as the team would gather at a single point, giving way to greater scrutiny by their teammates. Repeated or critical mistakes put participants on edge, as they braced for criticism from their teammates. They expressed relief or surprise when the chat remained silent or civil, with P8 stating ``\textit{I messed up there. No one is saying anything, thankfully.}''


\subsubsection{Point of no return}
Meanwhile, verbal communication flowed out when the game had a clear trajectory to the end. Previous research has shown that both toxic and non-toxic communication skyrockets near the end of the game~\cite{kwak2015linguistic} when the players have determined the game outcome with certainty. We saw that this phase opened up both positive and negative sides of communication for guaranteed win and loss, respectively. The winning team would compliment and cheer each other through chat messages and emotes, while the losing side devolved into arguments and calling out. The communication at this stage was driven by emotion, showing excitement or venting frustration.


\subsection{Communication Assessment Process}

We describe the factors that users mainly focus on to assess when or when not to involve themselves in communication with their teammates. 

\subsubsection{Calculating communication cost}
One of the most proximate factors behind when communication is performed is the limited action economy of the game. In \textit{LoL} and other MOBAs, players can rarely afford time to type out messages due to the fast-paced nature of the game. In time-sensitive scenarios, the time pressure makes communication particularly costly. It is therefore unsurprising that much of the communication occurs after major events (e.g., battles and objective hunting), as players are given more downtime while waiting for teammates or enemies to respawn or regroup.

For periods where players were still actively involved in gameplay, the players made conscious decisions on choosing which communication media to use based on the perceived action availability and the importance of communicating the message. Players relied on pings for non-critical indications, believing that the mutual understanding of the game would get the message across. However, many players recognized that pings were prone to be missed, ignored, or misinterpreted by their allies (P2, P9, P16, P17, P20). Subsequently, participants typed out information considered to be too important to the situation to be misunderstood or missed by other players even if it caused delays in their gameplay (P10, P11, P14). Simultaneously, the priority of importance constantly shifted --- we observed multiple times participants start to type, but stop to react to an ongoing play, only to never send out their message.

\subsubsection{Evaluating relevance and responsiveness}
When the brief window of communication opportunity is missed, players are unlikely to ever send out that information. In \textit{LoL}, situations can change within seconds and certain communication media cannot keep up with the changing state of the game. For example, almost all study participants did not participate in votes for objectives. Among the tens of objective votes initiated among all the games in this study, no objective vote saw more than three votes, frequently being left with no vote beyond the player who initiated the vote. Some players, when asked why they did not participate, stated that the votes they made often became irrelevant as the game state had changed during the time it took to vote (P2, P11). Other players also discussed how information conveyed through communication can get outdated fast (P1, P8, P9). 

\begin{quote}
I can't always follow through with what I say [in the chat] since the game is really dynamic. My teammates don't understand such situations, so I tend to not chat proactively. - P9
\end{quote}

Thus, some players instead preferred to react through direct action (P8, P10, P11, P16, P20). P10 stated, ``\textit{I think it's enough to show through action rather than [using objective voting]. I can look out for how the player reacts when I request something from them.}''

On the other hand, such action-based responses left the player to assess whether and how the communication was received. P10 stated that they tried to predict whether a player understood their ping direction by how they moved, but it was hard to interpret their intent: ``\textit{members sometimes seem to move towards me but then turn around, and sometimes they even ping back but don't come.}''. P16 discussed how they weren't sure whether the ping was received, but performed it anyway since it felt helpful.

Similarly, participation in surrender votes (or lack thereof) carried different intent by the player. During most of the games that ended in a loss, one or more surrender votes were called by the participant's team. However, only two surrender votes achieved four or more players' participation. However, the reasons why a player chose to not participate varied. Some had decided to wait and see how other teammates voted, which may have paradoxically led many members to not participate in the vote (P4, P9). Meanwhile, others didn't reply as they didn't think the vote was actually calling for a response: P13 stated, ``\textit{I didn't vote because they were just showing their anger. It's just a member venting through a surrender vote that they're not doing well.}''

\subsubsection{Balancing information access and psychological safety}
While recognizing that communication would be useful or even necessary in certain situations, participants also put their psychological safety first over information access. Some players, worn down by the normalization of toxic communication such as flaming, muted the chat (P1, P9).

Many participants expressed the sentiment of ``protecting [their] mentality'', describing how certain communication harmed their psychological well-being. This communication did not always refer to negative communication; P9 often muted players who gave commands as they did not want to be ``swept up'' by others' play-related judgments. This separation even extended to other more widely considered essential communication forms, such as pings. Even after acknowledging that pings were vital and useful to the game, P9 went as far as muting the ping of the support player in the same lane after they sent a barrage of Enemy Missing pings that signified aggression and criticism. 

Additionally, the abundance and high frequency of communication also strained the limited mental capacity of the players. Many players, when asked why they had not replied to an objective vote or other chat messages, stated that they simply did not notice them among other events happening (P1, P2, P3, P9, P12, P15, P18, P20, P21). The information overload caused stress and became distracting to players.

\subsubsection{Reducing potential friction}
As demonstrated in the pre-game stage of the game, players sometimes used communication to minimize friction between their teammates. Some participants sacrificed time to apologize to other players when they believed themselves to be at fault. When asked why, P12 replied, ``\textit{There are too many people who don't come to help gank if I don't apologize.}''. Similarly, P5 sacrificed time typing in an apology after a teammate had died despite still being in the middle of a fight as they didn't wish to give the other player a reason to start an attack.

However, some noted that silence is sometimes the best answer to a negative situation. P4, after dying to the enemy, put into chat ``Fighting!'' (roughly meaning, ``We can do it!''). They stated ``\textit{I don't know why I do it... it probably angers [my teammates] more.
}'' They also stated that ``\textit{for certain people, talking in the chat only spurs them more. You just have to let them be.}'' Other players shared similar sentiments that being quiet and dedicating focus to the game was a better choice (P1, P11, P14).

For female players, the fear of gender-based harassment shaped their communication patterns. While \textit{LoL} does not provide any demographic information of a player to other players, almost all female participants noted experiences of receiving derogatory remarks or doubts about their abilities based on other players' assumptions of their gender, a trend frequently seen in male-dominated online gaming cultures~\cite{fox2016women, norris2004gender, mclean2019female}. They noted that the players were able to correctly guess their gender when the participant's role and champion fit into the preconceived notions of what women ``tended to play'' (i.e., female-identifying support champions, such as Lulu) or their username ``seemed feminine'' (P18, P19, P20, P21, P22). This led to certain players adopting tactics that signaled male-like behavior, such as changing their speech style to be more gender-neutral or male-like (P19, P21) and changing their username to sound more gender-neutral. Cote describes similar instances of ``camouflaging gender'' as one of five main strategies for women coping with harassment~\cite{cote2017coping}. However, some players opted to keep playing their preferred character or maintaining their username even if it signaled their gender, such as P21 who expressed, ``\textit{I cherish and feel attached to my username, so I don’t want to change it just because of [harassment and inappropriate comments].}'' These players valued self-expression and identity even at the risk of increased risk to unpleasant communication experiences.


\subsubsection{Forming performance-based hierachy}
Naturally formed leadership has often been observed in other works on \textit{LoL} teams~\cite{kou2014}. Kim et al. showed that more hierarchy in in-game decision-making led to higher collective intelligence~\cite{kim2017}. While they used ``hierarchy'' to mean varying amounts of communication throughout the game, we observed that the hierarchy extends further to performance-based hierarchy, where teammates in more advantageous positions are given greater weight when communicating with other players. Players actively chose to refrain from suggesting strategic plans when they were ``holding down the team'', recognizing that they held less power and trust among the team members (P8, P10, P12, P14, P22). The player who was losing against the enemy team was viewed as having no ``right'' to lead the team, which was reserved for well-performing players.


\subsubsection{Enforcing norms and habits}
One of the most common answers to why players performed certain communication actions, especially non-verbal actions such as pings and emotes, was ``a force of habit'' (P6, P7, P8, P9, P10, P12, P17). Players formed learned practices of using communication channels at certain points by observing other players exhibit the same behaviors. This promoted, for example, replying to an emote sent by the teammate with their own or pinging readied skills and items to emphasize relevant information for other players throughout the game. 

On the other hand, this meant that players were averse to communication patterns outside of the norm --- participants stated that they had a hard time adapting to new forms of communication, seeing no immediate benefit or impact from using them (P1, P8, P14, P13, P15, P17). Most egregiously, the recently introduced objective pings were largely viewed to be awkward to use and unnecessary (P1, P4, P8, P12).


\subsection{Impact of Communication Assessment}
We describe how the communication patterns and assessment of the players impact the individual players' perspectives on team dynamics.

\subsubsection{Relationship between trust and communication frequency}
Most participants saw value in constant and well-informed communication but with an important distinction: verbal communication with strangers rarely ended well. Players largely recognized frequent verbal communication to burgeon conflict, regardless of the message within. Even when players understood the helpful intent behind positive messages from the players, they compared actively talking players to be possible bad actors who were likely to exhibit toxic behaviors when the game turned against them. (P1, P4, P8, P12, P14)

\begin{quote}
I need to make sure to not disturb Twisted Fate. I saw him start to flame. It's not because I don't want to hear more criticism. I know these types. The more I react and chat with them, the more deviant they will become. - P4  
\end{quote}

Similarly, P19 lamented that players used to socialize more in the chat during the pre-game phase to build a fun and prosocial environment, noting a memorable example of encouraging each other to do well on their academic exams, but noted that such prosocial behavior has become much rarer during the recent seasons. They noted that there are inevitably players ``who take it negatively'' and thus stopped proactively typing non-game related messages in the chat.

Ultimately, players desired assurance and trust of player commitment. The participants trusted actions more than words to prove that the player remained dedicated to the game. Both P10 and P17 pointed out that it was easy to tell who was still ``in the game'' and motivated to try their best and that ``staying on the keyboard'' likely meant that they weren't invested or focused on the game. Players viewed such commitment to be the most important aspect of a ``good'' teammate, sometimes even more than their skill or performance (P9, P14). It is interesting to note that unlike what previous literature may suggest~\cite{marlow2018}, players' averseness to talkative teammates had less to do with the cognitive overload or distraction caused by the frequent communication, but rather due to the threats of future team breakdown. This view in turn also affected how players decided to communicate or not, as they believed that players would not take their suggestions or comments in a positive light. 


\subsubsection{Perception of player commitment and fortitude}

Communication also acted as a mirror of their teammates' mental fortitude. A number of players mentioned how they valued a resilient mindset in their teammates playing the game, referring to players who remained committed to the game until the very end. They saw players who provoked or complained to teammates as ``having a weak mentality'' who had been altered by the bad outcomes of the game to act in an unhelpful manner towards the team through their communication. The communication actions of the teammate informed the participants of how steadfast their teammate remained in disadvantageous situations.  

\begin{quote}
It's not like I constantly reply in the chat or anything, but I pay attention [to the chat] to grasp the overall atmosphere of the team. If the team doesn't collaborate well then we lose, so I try to have a rough understanding of the mentality of the other players. - P13
\end{quote}

There were also instances of communication that helped players maintain a positive view of their teammates. For example, P11 mentioned near the beginning of the game, ``\textit{Looking at the chat, Varus player has strong mentality [for being so positive]. There were lots of points [in his support's] plays that he could have criticized.}'' Unfortunately, this view quickly soured when the Varus player devolved into criticism later in the late game phase where the Varus player started criticizing the support and other players. P11 then noted that the Varus player seemed to merely be ``bearing through the game''.


\begin{figure*}

\centering

 \centering
\begin{subfigure}[b]{0.49\textwidth}
  \centering
  \includegraphics[width=\linewidth]{Figures/pdfs/triviaqa_child_bar_chart.pdf}
  \caption{TriviaQA, Child}
 \end{subfigure}%
 \hfill
 \begin{subfigure}[b]{0.49\textwidth}
  \centering
\includegraphics[width=\linewidth]{Figures/pdfs/naturalqa_alice_bar_chart.pdf}
\caption{Natural Questions, Alice-Bob}
\end{subfigure}\\

\caption{Failure of Certainty-Based Mitigation Methods: Percentage of unmitigated hallucinations (Y-axis) across six models (X-axis) for three mitigation methods: Sampling, Predictive Entropy, and Probability. Lower values indicate higher success in mitigation. 
All methods fail to address a significant portion of \chk, likely due to their dependence on uncertainty scores.}
\label{mitigation_fig}
\end{figure*}

\section{Hallucination Mitigation and \chk}\label{sec:mitigation_methods}

Having established the existence of \chk, we hypothesize that they may impact the performance of mitigation methods. 
To test this hypothesis, we experiment with hallucination mitigation while focusing on \chk.

\subsection{Mitigation Based on Certainty Measures}


Recent research has proposed leveraging certainty measures to mitigate hallucinations, primarily by abstaining from generating outputs when the model is uncertain about its predictions \cite{cole-etal-2023-selectively,tjandra2024fine}. 
These approaches often rely on probabilities and entropy-based metrics \cite{tjandra2024fine} or combine them with self-evaluation techniques \cite{tomani2024uncertainty}. Other techniques employ alternative information-theoretic measures \cite{yadkori2024believe, yadkori2024mitigating} or involve cross-verification using multiple language models to assess uncertainty \cite{feng2024don}.


The effectiveness of these measures is often evaluated using the area under the receiver operator characteristic (AUROC), which measures how well uncertainty ranks correct versus incorrect responses \cite{kuhn2023semantic}. However, this focus on ranking accuracy overlooks cases where hallucinations occur with high certainty or when correct answers are generated with low certainty. Therefore, this oversight may limit their ability to mitigate hallucinations comprehensively.




To evaluate whether \chk samples are being overlooked, we test the usefulness of uncertainty-based mitigation methods in our setting. Specifically, we use the approaches of \citet{tomani2024uncertainty} and \citet{cole-etal-2023-selectively}, incorporating three uncertainty measures they used: negative log-likelihood, sampling, and predictive entropy. 
We introduce the probability measure as a simplified alternative to negative log-likelihood. Below, we briefly describe each measure, See Appendix \ref{appendix:Mitigation Metrics} for details.

\textbf{Probability}  is a simplified version of negative log-likelihood that considers only the probability of the first token. \textbf{Sampling} assess the diversity of the model's generated outputs, under the assumption that greater diversity reflects lower certainty. Lastly, \textbf{predictive entropy} estimates uncertainty by evaluating the average unpredictability of the model’s outputs.
Examples are mitigated only if their uncertainty is below a defined threshold, as described in Section \ref{subsec:Certainty Threshold}.



\subsection{Certainty-based Mitigation Methods Fail on \chk}


The results in Figure \ref{mitigation_fig} show the percentage of \textbf{unmitigated} hallucination examples for each model. \emph{A lower value is better}, as it indicates fewer hallucinations remain unmitigated.


Our results in Figure \ref{mitigation_fig} show that a significant proportion of hallucinations exceed the threshold across all methods and models, remaining unmitigated. Among the methods, sampling performs the worst, while probability and predictive entropy consistently show a lower percentage of unmitigated hallucinations. This suggests sampling is less effective in avoiding mislabeling hallucinations among uncertainty-based methods.
However, even probability and predictive entropy fail to mitigate a notable percentage of hallucinations. Lowering the threshold could improve mitigation, but this would lead to an increased mislabeling of low-certainty correct answers, as shown in Section \ref{sec:main_results}.

These findings show that certainty-based mitigation fails to address high-certainty hallucinations. While uncertainty-based methods are expected to struggle with high-certainty hallucinations, our results here and in Section \ref{sec:main_results} indicate they consistently miss a distinct type, regardless of the uncertainty metric or threshold strictness. This failure suggests a fundamental limitation in assuming an ideal uncertainty metric can fully capture hallucinations.







\section{Conclusion}
\label{sec:conclusion}
\vspace{0.7em}

\ps{Highlight key idea/novelty of PlaceIT}

In this work, we present \name, a novel methodology to jointly optimize the chiplet placement and \gls{ici} topology for chips with heterogeneous chiplet shapes and silicon bridges or passive silicon interposers.
The main novelty of our approach is that we perform optimization on the chiplet placement itself, where we infer a custom, placement-based \gls{ici} topology for each placement produced by an optimization algorithm.
We use the placement and its inferred \gls{ici} topology to compute proxies for \gls{ici} latency and throughput of different traffic types, which we combine into a user-defined quality metric that is returned to the optimization algorithm.

\ps{Highlight the framework/code}

The open-source \name~framework is modular and allows adding custom optimization algorithms or placement representations.
\name~offers a wide range of configurable parameters, making it applicable for a variety of designs with different chiplet dimensions, PHY-counts, and \gls{d2d} links.

\ps{Summarize evaluation}

Our evaluation on synthetic traffic shows that \name~produces \gls{ici}s with vastly lower \gls{c2m}, \gls{c2i}, and \gls{m2i} latency (reduced by up to 62\%) compared to a 2D mesh baseline.
On real traffic traces, \name~reduces the average packet latency in almost all traces and architectures considered.
The average packet latency is reduced by up to $18\%$ on average.

\ps{Concluding sentence}

By using our open-source \name~framework, architects can co-optimize their chiplet-placement and \gls{ici} topology to build 2.5D stacked chips with low-latency interconnects.




\bibliography{anthology,custom}


\appendix

\section{Dataset Creation}
\label{sec:appendix-Dataset creation}

To create the dataset, we first split the examples into knowledge-based examples, following a method similar to \citet{simhi2024distinguishing}. Specifically, we performed one greedy generation and five generations with a temperature of $0.5$. We used a 3-shot in-context learning scenario, generating a maximum of 5 tokens, and considered a generation correct only if it exactly matched the factually correct answer.

We also adopted the basic dataset curation process described in \citet{simhi2024distinguishing}, but with two key modifications: we started with 70K examples instead of 30K, and we generated 10 tokens instead of 5. Each example in the dataset begins with \texttt{question:} and ends with \texttt{answer:}.

For instruct models, we adjusted the format to align better with their structure. Specifically, we presented the few-shot examples as a user-assistant conversation, where the user asks the questions and the assistant provides the answers. Additionally, we replaced \texttt{answer:} with \texttt{The answer is}, as part of the assistant generation, since this change was observed to improve the performance of instruction models.

To split the knowledge examples into factually correct examples and hallucination-despite knowledge examples, we sampled 20K knowledge-based examples (or fewer if fewer were available). Using the child and Alice-Bob settings, we checked whether the generated text changed and whether the exact match for the correct answer appeared within the 10-token model generations.
\subsection{Additional Refinement}
We observed certain issues, especially with the instruct models, where an exact match was insufficient. For example, the model sometimes failed to generate an answer or produced a correct answer with minor variations, such as synonyms. To address these issues, we curated the \textsc{WACK} examples further, applying a set of simple heuristics that proved effective during manual examination:

\begin{enumerate}
    \item \textbf{Removing negations:} We excluded examples where the generation stated with ``The answer is not.''
    \item \textbf{Synonyms:} Using the NLTK library \citep{loper2002nltk}, we removed examples where a synonym of the correct answer appeared in the generated text.
    \item \textbf{Stem-based similarity:} We excluded examples if the stemmed version of the generation and the factually correct answer shared more than half of their words.
    \item \textbf{Edit distance:} We kept examples where the edit distance between the generated text and the correct answer (in their stemmed versions) was greater than 2, or the answer is a number and \texttt{great}, \texttt{none}, and \texttt{n/a}, which were removed if present in the generated answer.
    \item \textbf{Initial word match:} We removed examples where the generated answer was the first word of the factually correct answer.
    \item \textbf{Special formatting:} For \textsc{Gemma-instruct} model, which we saw that typically generates the final answer enclosed in \texttt{**}, we removed examples where this formatting was absent.
\end{enumerate}
Thus, we removed between 10\% and 45\% of all the hallucination examples. Note that this is a very harsh criterion for removing hallucinations; however, since our aim was to demonstrate that certain hallucinations exist, we preferred to remove any possibility of wrongly classified hallucinations.


\subsection{Dataset Statistics}
In this section, we present the final dataset statistics. The results are shown in Tables \ref{tab:statistic_child} and \ref{tab:statistic_alice}. We can see that the number of \textsc{WACK} hallucinations varies, with Gemma showing low numbers, while most other models exhibit more than 1000 \textsc{WACK} hallucinations.



\begin{table*}[h!]
        \centering
        \begin{tabular}{lcccc}
        \hline
        Model & \# $HK^+$ & \# Factually-correct \\
        \toprule
 Llama & 643/1096 & 17180/10324 \\\midrule
Mistral &691/1732 & 18916/14425  \\\midrule
Gemma &375/797 & 17617/12916\\\midrule
Llama-Instruct & 1075/ 1730 & 14986/9929 \\\midrule
Mistral-Instruct &1338/2767 & 16512/13566 \\\midrule
Gemma-Instruct &1003/1716 & 17887/16421 \\
\bottomrule
\end{tabular}
        \caption{Dataset statistics for TrivaQA/Natural Questions under Child setting.}
        \label{tab:statistic_child}
        \end{table*}


\begin{table*}[h!]
        \centering
        \begin{tabular}{lcccc}
        \hline
        Model & \# $HK^+$ & \# Factually-correct \\
        \toprule
 Llama & 687/1285 & 17026/10088 \\\midrule
Mistral &809/1607 & 18710/14450 \\\midrule
Gemma &446/896 & 17584/12636\\\midrule
Llama-Instruct &1410/2371 & 14996/9771  \\\midrule
Mistral-Instruct &1425/3091 & 16611/14248 \\\midrule
Gemma-Instruct &  1310/2642 & 17678/16194  \\
\bottomrule
\end{tabular}
        \caption{Dataset statistics for TrivaQA/Natural Questions under Alice-Bob setting.}
        \label{tab:statistic_alice}
        \end{table*}


\subsection{Additional Implementation Details}\label{appendix:Implementation Details}
We use the datasets under the Apache License and the models under each one's agreement terms to follow each artifact's terms.
All experiments were run on NVIDIA RTX 6000 Ada (49GB) with 4 CPUs. 
Generating all the datasets and results takes approximately one month on one GPU.

Lastly, We used AI assistants only for simple paraphrasing as part of writing this paper.


\section{Qualitative Evaluation}\label{appendix:Qualitative Evaluation}
In this section, we show qualitative examples of certain hallucinations.
In Table \ref{Generated answers using bad/good shots in the prompt} provides an example from each model using the child setting on the Natural Questions dataset. These are examples of \chk hallucinations, where the model outputs have high probability and low semantic entropy, indicating a high degree of certainty.

\begin{table*}[t]
\small

\centering

  \begin{tabular}  
  {l p{0.2\linewidth}c ccp{0.1\linewidth}}
\toprule 
& & \multicolumn{2}{c}{Response}&\multicolumn{2}{c}{Uncertainty metric} \\ 
\cmidrule(lr){3-4}\cmidrule(lr){5-6}

Model  &Prompt &Original Response & Hallucinated Response & Probability & Semantic Entropy\\\midrule
Gemma&question:   what is the measure of the number of different species present in an area?& biodiversity& species richness&0.31&0.0\\\midrule
Llama&question: who wrote the song it's the climb?&alexander&Miley Cyrus&0.42&$1.11e^-16$\\\midrule
Mistral&question: if there is a random change in the genetics of a small population it is termed?&genetic drift&mutation&0.49& 0.14\\\midrule
Gemma-Instruct&question: who played the mom on lost in space?&June Lockhart&Molly Parker&0.89&0.23\\\midrule
Llama-Instruct&question: what gas is given off by concentrated hydrochloric acid?&hydrogen chloride&hydrogen gas ($H_2$)&0.98&$2.22e^-16$\\\midrule
Mistral-Instruct&question: who published harry potter and the prisoner of azkaban?&Scholastic Inc&J.K. Rowling&0.99&$2.22e^-16$
  \\\bottomrule

  \end{tabular}
 
  \caption{\chk generated answers and uncertainty measures were obtained using greedy decoding on the Natural Questions dataset in the child setting. In each of these examples, the model generates a hallucination despite having the necessary knowledge. The probability of the generated answer is high, and the semantic entropy is low, indicating that these examples exhibit high certainty.}
   \label{Generated answers using bad/good shots in the prompt}
\end{table*}


\section{Certainty Methods Additional Specifics}\label{appendix:Certainty Methods Additional Specifics}
In this section, we elaborate on specifics in the calculations of the different methods.
\subsection{Probability and Probability Difference}
To ensure that the probability we consider corresponds to the probability of the actual answer and not a preceding token, we employed the following heuristic: skipping over any of the following tokens:  
\texttt{"<|assistant|>", "<|user|>", "<|begin\_of\_text|>", "<|end\_of\_text|>", "<|eot\_id|>", "<|start|>",  
"<|end|>", "<|sep|>", "<|sep\_id|>", "assistant", "user", "\textbackslash n", "answer", "The", "Answer", "\"", "'", " answer", "is", "it", "it's", ":", " ", " is", " correct", "correct", "*", "**", " **"}.

This heuristic proved sufficient during a manual investigation.


\subsection{Sampling Based Methods}
For the Semantic Entropy, Sampling, and Predictive Entropy methods, it is necessary to consider the temperature and define a stopping condition for each generation.


We stopped the generation if one of the following sequences was produced: '\textbackslash n\textbackslash n\textbackslash n\textbackslash n', '\textbackslash n\textbackslash n\textbackslash', "\textbackslash n\textbackslash n, 'Question:', 'Context:', ".\textbackslash n", ". ",  'question:', "Alice", "Bob", "(", "Explanation", "\textbackslash n question:", "What", "\textbackslash n answer". These sequences often indicate the generation of new text that is not relevant to answering the question.

We used a temperature of $1$ for 10 generations and an additional generation with a low temperature of $0.1$, following the approach in the code of \citet{kuhn2023semantic} in repository \url{https://github.com/jlko/semantic_uncertainty}, and using DeBERTa \citep{he2020deberta} as the entailment model for the clustering stage based on meaning similarity. Additional results with a temperature of $0.5$ instead of $1$ are presented in Appendix~\ref{appendix:Semantic Entropy results Different Temperature}. Note that we used 10 tokens to generate as the maximum to be consistent with the knowledge and dataset creation steps.



\subsection{Mitigation Metrics} \label{appendix:Mitigation Metrics}
In Section \ref{sec:mitigation_methods} we evaluate the mitigation abilities of probability, sampling and predictive entropy. In this section we detail each metric.


\paragraph{Probability.}
Probability is a simplified version of Negative Log-Likelihood that considers only the log probability of the first token. We observed that subsequent tokens often exhibit high certainty regardless of correctness, making the first token a more reliable indicator of uncertainty and improving performance.

\paragraph{Sampling.} Sampling-based methods assess the diversity of the model's generated outputs, under the assumption that greater diversity reflects lower certainty. Following \citet{cole-etal-2023-selectively}, we define diversity as the proportion of unique outputs in $S$ generated samples. The uncertainty score is calculated as $1-|U|/|S|$, where $U$ is the set of unique generations, and $|U|/|S|$ represents the ratio of unique outputs to the total number of generations. 

\paragraph{Predictive Entropy.} Predictive Entropy estimates uncertainty by evaluating the average unpredictability of the model’s outputs.
We approximate predictive entropy following \citet{kuhn2023semantic} and \citet{tomani2024uncertainty} by estimating the uncertainty of the model based on its generations for a given prompt $x$. Using $L$ generated samples, the predictive entropy is calculated as: 
\begin{equation}
    PE \approx -1/L\sum_{i=1}^{L}logp(l_i|x)
\end{equation}

 Here, $p(l_i|x)$ represents the likelihood of the $i$-th generation given the prompt $x$. Predictive entropy captures the average uncertainty across the generated outputs.

We investigate whether these uncertainty measures can reliably detect and mitigate \chk hallucinations.


\section{Certain HK+ Exist -- Additional Results}\label{appendix-Certain HK+ Exist Additional Results}

In this section, we present results similar to those in Section \ref{sec:main_results}, focusing on the Gemma and Llama models. See Figures \ref{fig:gemma_certainty} and \ref{fig:Llama_certainty}. These results correlate with those in the main paper on the Mistral model and demonstrate that \chk hallucinations exist. Furthermore, they show that these hallucinations occur across different methods and that instruct-tuned models exhibit poorer calibration between certainty and hallucinations.




\begin{figure*}
\centering

 \centering

 \centering
\begin{subfigure}[b]{0.24\textwidth}
  \centering
  \includegraphics[width=\linewidth]{Figures/pdfs/meta-llama_Llama-3.1-8B_triviaqa_child_prob.pdf}
 \end{subfigure}%
 \hfill
 \begin{subfigure}[b]{0.24\textwidth}
  \centering
\includegraphics[width=\linewidth]{Figures/pdfs/meta-llama_Llama-3.1-8B_naturalqa_alice_prob.pdf}
\end{subfigure}
\hfill
\begin{subfigure}[b]{0.24\textwidth}
  \centering
  \includegraphics[width=\linewidth]{Figures/pdfs/meta-llama_Llama-3.1-8B-Instruct_triviaqa_child_prob.pdf}
 \end{subfigure}
 \hfill
 \begin{subfigure}[b]{0.24\textwidth}
  \centering
  \includegraphics[width=\linewidth]{Figures/pdfs/meta-llama_Llama-3.1-8B-Instruct_naturalqa_alice_prob.pdf}
 \end{subfigure}\\

 \centering
\begin{subfigure}[b]{0.24\textwidth}
  \centering
  \includegraphics[width=\linewidth]{Figures/pdfs/meta-llama_Llama-3.1-8B_triviaqa_child_prob_diff.pdf}
 \end{subfigure}%
 \hfill
 \centering
\begin{subfigure}[b]{0.24\textwidth}
  \centering
  \includegraphics[width=\linewidth]{Figures/pdfs/meta-llama_Llama-3.1-8B_naturalqa_alice_prob_diff.pdf}
 \end{subfigure}%
 \hfill
\begin{subfigure}[b]{0.24\textwidth}
  \centering
  \includegraphics[width=\linewidth]{Figures/pdfs/meta-llama_Llama-3.1-8B-Instruct_triviaqa_child_prob_diff.pdf}
 \end{subfigure}
 \hfill
 \begin{subfigure}[b]{0.24\textwidth}
  \centering
  \includegraphics[width=\linewidth]{Figures/pdfs/meta-llama_Llama-3.1-8B-Instruct_naturalqa_alice_prob_diff.pdf}
 \end{subfigure}\\

  \centering

 \centering
\begin{subfigure}[b]{0.24\textwidth}
  \centering
  \includegraphics[width=\linewidth]{Figures/pdfs/meta-llama_Llama-3.1-8B_triviaqa_child_semantic_entropy.pdf}
 \caption{Llama, Child}
 \end{subfigure}%
 \hfill
  \centering
\begin{subfigure}[b]{0.24\textwidth}
  \centering
  \includegraphics[width=\linewidth]{Figures/pdfs/meta-llama_Llama-3.1-8B_naturalqa_alice_semantic_entropy.pdf}
  \caption{Llama, Alice-Bob}
 \end{subfigure}
 \hfill
\begin{subfigure}[b]{0.24\textwidth}
  \centering
  \includegraphics[width=\linewidth]{Figures/pdfs/meta-llama_Llama-3.1-8B-Instruct_triviaqa_child_semantic_entropy.pdf}
  \caption{Llama-Instruct, Child}
  \end{subfigure}
  \hfill
  \begin{subfigure}[b]{0.24\textwidth}
  \centering
  \includegraphics[width=\linewidth]{Figures/pdfs/meta-llama_Llama-3.1-8B-Instruct_naturalqa_alice_semantic_entropy.pdf}
  \caption{Llama-Instruct, Alice-Bob}
 \end{subfigure}\\

 \caption{
Detection of WACK Hallucinations on \emph{Llama}: Cumulative percentage of samples (Y-axis) across certainty levels (X-axis) for hallucinations (red) and correct answers (blue) in cases where the model possesses the correct knowledge. Subfigures are organized by models (columns) and certainty measures (rows), with the black dashed line marking the optimal certainty threshold for separating hallucinations from correct answers. The red line shows that a significant proportion of hallucinations persists even at high certainty levels, illustrating the phenomenon of \chk hallucinations, while the blue line shows that many correct answers are also made with varying levels of certainty. The results for Child are on TriviaQA and For Alice-Bob on Natural Questions.}
 \label{fig:Llama_certainty}
\end{figure*}



 
\begin{figure*}
\centering
 \centering

 \centering
\begin{subfigure}[b]{0.24\textwidth}
  \centering
  \includegraphics[width=\linewidth]{Figures/pdfs/google_gemma-2-9b_triviaqa_child_prob.pdf}
 \end{subfigure}%
 \hfill
 \begin{subfigure}[b]{0.24\textwidth}
  \centering
\includegraphics[width=\linewidth]{Figures/pdfs/google_gemma-2-9b_naturalqa_alice_prob.pdf}
\end{subfigure}
\hfill
\begin{subfigure}[b]{0.24\textwidth}
  \centering
  \includegraphics[width=\linewidth]{Figures/pdfs/google_gemma-2-9b-it_triviaqa_child_prob.pdf}
 \end{subfigure}
 \hfill
 \begin{subfigure}[b]{0.24\textwidth}
  \centering
  \includegraphics[width=\linewidth]{Figures/pdfs/google_gemma-2-9b-it_naturalqa_alice_prob.pdf}
 \end{subfigure}\\

 \centering
\begin{subfigure}[b]{0.24\textwidth}
  \centering
  \includegraphics[width=\linewidth]{Figures/pdfs/google_gemma-2-9b_triviaqa_child_prob_diff.pdf}
 \end{subfigure}%
 \hfill
 \centering
\begin{subfigure}[b]{0.24\textwidth}
  \centering
  \includegraphics[width=\linewidth]{Figures/pdfs/google_gemma-2-9b_naturalqa_alice_prob_diff.pdf}
 \end{subfigure}%
 \hfill
\begin{subfigure}[b]{0.24\textwidth}
  \centering
  \includegraphics[width=\linewidth]{Figures/pdfs/google_gemma-2-9b-it_triviaqa_child_prob_diff.pdf}
 \end{subfigure}
 \hfill
 \begin{subfigure}[b]{0.24\textwidth}
  \centering
  \includegraphics[width=\linewidth]{Figures/pdfs/google_gemma-2-9b-it_naturalqa_alice_prob_diff.pdf}
 \end{subfigure}\\

  \centering

 \centering
\begin{subfigure}[b]{0.24\textwidth}
  \centering
  \includegraphics[width=\linewidth]{Figures/pdfs/google_gemma-2-9b_triviaqa_child_semantic_entropy.pdf}
 \caption{Gemma, Child}
 \end{subfigure}%
 \hfill
  \centering
\begin{subfigure}[b]{0.24\textwidth}
  \centering
  \includegraphics[width=\linewidth]{Figures/pdfs/google_gemma-2-9b_naturalqa_alice_semantic_entropy.pdf}
  \caption{Gemma, Alice-Bob}
 \end{subfigure}
 \hfill
\begin{subfigure}[b]{0.24\textwidth}
  \centering
  \includegraphics[width=\linewidth]{Figures/pdfs/google_gemma-2-9b-it_triviaqa_child_semantic_entropy.pdf}
  \caption{Gemma-Instruct, Child}
  \end{subfigure}
  \hfill
  \begin{subfigure}[b]{0.24\textwidth}
  \centering
  \includegraphics[width=\linewidth]{Figures/pdfs/google_gemma-2-9b-it_naturalqa_alice_semantic_entropy.pdf}
  \caption{Gemma-Instruct, Alice-Bob}
 \end{subfigure}\\

 \caption{
Detection of WACK Hallucinations on \emph{Gemma}: Cumulative percentage of samples (Y-axis) across certainty levels (X-axis) for hallucinations (red) and correct answers (blue) in cases where the model possesses the correct knowledge. Subfigures are organized by models (columns) and certainty measures (rows), with the black dashed line marking the optimal certainty threshold for separating hallucinations from correct answers. The red line shows that a significant proportion of hallucinations persists even at high certainty levels, illustrating the phenomenon of \chk hallucinations, while the blue line shows that many correct answers are also made with varying levels of certainty. The results for Child are on TriviaQA and For Alice-Bob on Natural Questions.}
 \label{fig:gemma_certainty}
\end{figure*}

\section{Semantic Entropy Results -- Different Temperature}\label{appendix:Semantic Entropy results Different Temperature}

In Section~\ref{sec:Certainty Hallucinations can not be Explain as Noise}, we used Semantic Entropy with a temperature of 1 for generating the samples. To demonstrate that the certainty results are not specific to this temperature, we present in Figure~\ref{fig:temp_similarity_certainty} the Semantic Entropy results on the Mistral model. In the upper subfigure, we show the results using a temperature of 1, and in the lower row, we show the results using a temperature of 0.5. We observe that under a temperature of 0.5, there are even more certain hallucinations, further proving that the certainty hallucination phenomenon is not specific to a temperature of 1.


\begin{figure*}
\centering

 \centering
 \centering
\begin{subfigure}[b]{0.24\textwidth}
  \centering
  \includegraphics[width=\linewidth]{Figures/pdfs/mistralai_Mistral-7B-v0.3_triviaqa_child_semantic_entropy.pdf}
 \end{subfigure}%
 \hfill
  \centering
\begin{subfigure}[b]{0.24\textwidth}
  \centering
  \includegraphics[width=\linewidth]{Figures/pdfs/mistralai_Mistral-7B-v0.3_naturalqa_alice_semantic_entropy.pdf}
 \end{subfigure}
 \hfill
% \end{subfigure}
\begin{subfigure}[b]{0.24\textwidth}
  \centering
  \includegraphics[width=\linewidth]{Figures/pdfs/mistralai_Mistral-7B-Instruct-v0.3_triviaqa_child_semantic_entropy.pdf}
  \end{subfigure}
  \hfill
  \begin{subfigure}[b]{0.24\textwidth}
  \centering
  \includegraphics[width=\linewidth]{Figures/pdfs/mistralai_Mistral-7B-Instruct-v0.3_naturalqa_alice_semantic_entropy.pdf}
 \end{subfigure}\\
 \centering
\begin{subfigure}[b]{0.24\textwidth}
  \centering
  \includegraphics[width=\linewidth]{Figures/pdfs/mistralai_Mistral-7B-v0.3_triviaqa_child_semantic_entropy_temp_0.5.pdf}
 \caption{Mistral, Child}
 \end{subfigure}%
 \hfill
  \centering
\begin{subfigure}[b]{0.24\textwidth}
  \centering
  \includegraphics[width=\linewidth]{Figures/pdfs/mistralai_Mistral-7B-v0.3_naturalqa_alice_semantic_entropy_temp_0.5.pdf}
  \caption{Mistral, Alice-Bob}
 \end{subfigure}
 \hfill
% \end{subfigure}
\begin{subfigure}[b]{0.24\textwidth}
  \centering
  \includegraphics[width=\linewidth]{Figures/pdfs/mistralai_Mistral-7B-Instruct-v0.3_triviaqa_child_semantic_entropy_temp_0.5.pdf}
  \caption{Mistral-Instruct, Child}
  \end{subfigure}
  \hfill
  \begin{subfigure}[b]{0.24\textwidth}
  \centering
  \includegraphics[width=\linewidth]{Figures/pdfs/mistralai_Mistral-7B-Instruct-v0.3_naturalqa_alice_semantic_entropy_temp_0.5.pdf}
  \caption{Mistral-Instruct, Alice-Bob}
 \end{subfigure}\\

 \caption{
Detection of WACK Hallucinations: The upper figures show the Semantic Entropy method with a temperature of 1, while the lower figures use a temperature of 0.5. We show the cumulative percentage of samples (Y-axis) across certainty levels (X-axis) for hallucinations (red) and correct answers (blue) in cases where the model possesses the correct knowledge. Subfigures are organized by models (columns) and certainty measures (rows), with the black dashed line marking the optimal certainty threshold for separating hallucinations from correct answers. The red line shows that a significant proportion of hallucinations persists even at high certainty levels, illustrating the phenomenon of \chk hallucinations, while the blue line shows that many correct answers are also made with varying levels of certainty. 
}
 \label{fig:temp_similarity_certainty}
\end{figure*}



\section{Prompt Specificity}\label{appendix:Prompt Specificity}
In this section, we aim to demonstrate that the results are not restricted to a single variation of the setting. To investigate this, we created a new variation of the child setting. The new prompt is:  
\emph{`I'm working on a major school project, and there's a lot of information I need to understand. Some of it is a bit challenging, and I'm unsure where to begin. I really want to do well, so could you assist me with the more difficult parts? It would mean so much to me!'}

We present the certainty detection results using Mistral on the Natural Questions dataset in Figure~\ref{fig:child2_detection}. In the top figure, we show the results using the original child prompt, while the bottom figure displays the results with the new child prompt. Both settings reveal certain \textsc{WACK} hallucinations and exhibit overall similarity. This indicates that our results are robust to variations in the prompt.

\begin{figure*}
\centering
% \begin{figure*}
 \centering
 \centering
\begin{subfigure}[b]{0.3\textwidth}
  \centering
  \includegraphics[width=\linewidth]{Figures/pdfs/mistralai_Mistral-7B-v0.3_naturalqa_child_prob.pdf}
 \end{subfigure}%
 \hfill
  \centering
\begin{subfigure}[b]{0.3\textwidth}
  \centering
  \includegraphics[width=\linewidth]{Figures/pdfs/mistralai_Mistral-7B-v0.3_naturalqa_child_prob_diff.pdf}
 \end{subfigure}
 \hfill
% \end{subfigure}
\begin{subfigure}[b]{0.3\textwidth}
  \centering
  \includegraphics[width=\linewidth]{Figures/pdfs/mistralai_Mistral-7B-v0.3_naturalqa_child_semantic_entropy.pdf}
  \end{subfigure}
\\
 \centering
\begin{subfigure}[b]{0.3\textwidth}
  \centering
  \includegraphics[width=\linewidth]{Figures/pdfs/mistralai_Mistral-7B-v0.3_naturalqa_child2_prob.pdf}
  \caption{Probability}
 \end{subfigure}%
 \hfill
  \centering
\begin{subfigure}[b]{0.3\textwidth}
  \centering
  \includegraphics[width=\linewidth]{Figures/pdfs/mistralai_Mistral-7B-v0.3_naturalqa_child2_prob_diff.pdf}
  \caption{Probability Difference}
 \end{subfigure}
 \hfill
% \end{subfigure}
\begin{subfigure}[b]{0.3\textwidth}
  \centering
  \includegraphics[width=\linewidth]{Figures/pdfs/mistralai_Mistral-7B-v0.3_naturalqa_child2_semantic_entropy.pdf}
  \caption{Semantic Entropy}
  \end{subfigure}\\

 \caption{
Detection of WACK Hallucinations: The upper figures show the results on Child setting while the lower are the results on Child variation setting. We show the cumulative percentage of samples (Y-axis) across certainty levels (X-axis) for hallucinations (red) and correct answers (blue) in cases where the model possesses the correct knowledge. Subfigures are organized by models (columns) and certainty measures (rows), with the black dashed line marking the optimal certainty threshold for separating hallucinations from correct answers. We can see that the two child variations have similar graphs.
}
 \label{fig:child2_detection}
\end{figure*}

\section{\chk Uniqueness -- Additional Results}\label{sec:appendix-Jaccard Similarity Additional Results}
In this section, we extend the results presented in Section \ref{sec:Certainty Hallucinations can not be Explain as Noise} by demonstrating that, even under shared hallucination examples, permutation tests confirm that certain hallucinations are not random.

Tables \ref{tab:jaccard_prob_shared} and \ref{tab:jaccard_semantic_shared} display these results. Notably, all values are higher than those reported in the main paper, as the examples are now sampled from a subset of shared hallucination examples between the Child and Alice-Bob settings. However, even under the shared examples, the Jaccard similarity for certain cases remains significantly higher than the similarity observed under the permutation test. This further supports the conclusion that certain hallucinations are not mere noise but a distinct property of the models.



We compare the similarity under high certainty to that under the lowest certainty. Specifically, we examine the same subset of examples as the high-certainty subgroup but focus on those with the lowest probabilities. The results are shown in Table \ref{tab:jaccard_prob_low_certainty}.

We observe that the Jaccard similarity for the lowest-certainty subgroup is higher than the values obtained from the permutation test, indicating a general similarity across all certainty levels between the two settings. However, the high-certainty subgroup still exhibits significantly higher similarity scores, suggesting that this subgroup is more aligned than would be expected by chance.


\paragraph{Answer Token Length.}
To investigate what distinguishes \chk samples, we analyzed the length of the first token in generated answers. Specifically, we examined whether the average length of the first token differed between \chk samples and low-certainty hallucinations. Across models, datasets, prompt settings, and metrics, \chk examples consistently exhibited shorter first tokens, with the difference statistically significant according to a t-test. Although the underlying reasons for this pattern remain unclear and warrant future study, this result further highlights the unique characteristics of \chk samples.







\begin{table}[h!]
        \centering
        \begin{tabular}{l|c|cc}
        \hline
        Model&Dataset &Random& Certain\\
        \toprule
  \multirow{2}{*}{Llama}& TriviaQA  & 7.39 &\textbf{47.22}\\ &NQ&9.56 & \textbf{50.72}\\\midrule\multirow{2}{*}{Mistral}& TriviaQA  & 9.99 &\textbf{51.89}\\ &NQ&24.72 & \textbf{72.61}\\\midrule\multirow{2}{*}{Gemma}& TriviaQA  & 7.72 &\textbf{41.67}\\ &NQ&13.54 & \textbf{54.81}\\\midrule\multirow{2}{*}{Llama-Inst}& TriviaQA  & 21.02 &\textbf{36.36}\\ &NQ&22.44 & \textbf{34.42}\\\midrule\multirow{2}{*}{Mistral-Inst}& TriviaQA  & 15.44 &\textbf{57.24}\\ &NQ&22.54 & \textbf{50.06}\\\midrule\multirow{2}{*}{Gemma-Inst}& TriviaQA  & 14.99 &\textbf{53.93}\\ &NQ&16.36 & \textbf{54.47}\\
\bottomrule
\end{tabular}
        \caption{Jaccard Similarity of \chk hallucinations across different prompts under \emph{shared hallucinations}. The \textit{Certain} column shows the overall similarity of \emph{\chk} samples between prompts in the TriviaQA and NaturalQA datasets, using \emph{Probability} as the certainty threshold. Results indicate high similarity, suggesting consistency across settings. All scores are statistically significant (\(p < 0.0001\), permutation test (the Rand column).}
        \label{tab:jaccard_prob_shared}
        \end{table}


\begin{table}[h!]
        \centering
        \begin{tabular}{l|c|cc}
        \hline
        Model &Dataset &Random& Certain \\
        \toprule
\multirow{2}{*}{Llama}& TriviaQA  & 5.56 &\textbf{26.56}\\ &NQ&5.4 & \textbf{16.24}\\\midrule\multirow{2}{*}{Mistral}& TriviaQA  & 7.68 &\textbf{26.26}\\ &NQ&10.96 & \textbf{28.76}\\\midrule\multirow{2}{*}{Gemma}& TriviaQA  & 7.19 &\textbf{25.0}\\ &NQ&7.44 & \textbf{20.72}\\\midrule\multirow{2}{*}{Llama-Inst}& TriviaQA  & 10.7 &\textbf{29.28}\\ &NQ&13.42 & \textbf{31.06}\\\midrule\multirow{2}{*}{Mistral-Inst}& TriviaQA  & 14.21 &\textbf{27.12}\\ &NQ&22.29 & \textbf{41.06}\\\midrule\multirow{2}{*}{Gemma-Inst}& TriviaQA  & 14.51 &\textbf{43.68}\\ &NQ&17.29 & \textbf{42.64}\\
\bottomrule
\end{tabular}
\caption{
Jaccard Similarity of \chk hallucinations across different prompts under \emph{shared hallucinations}. The \textit{Certain} column shows the overall similarity of \emph{\chk} samples between prompts in the TriviaQA and NaturalQA datasets, using \emph{Semantic Entropy} as the certainty threshold. Results indicate high similarity, suggesting consistency across settings. All scores are statistically significant (\(p < 0.0001\), permutation test (the Rand column).}
        \label{tab:jaccard_semantic_shared}
        \end{table}





\begin{table}[t]
        \centering
        \begin{tabular}{l|c|cc}
        \toprule
        Model&Dataset&uncertain& Certain \\
        \toprule
\multirow{2}{*}{Llama} & TriviaQA & 17.91 &\textbf{27.42}\\ &NQ&17.65 &\textbf{21.21}\\\midrule
\multirow{2}{*}{Mistral} & TriviaQA & 22.55 &\textbf{28.21}\\ &NQ&28.93 &\textbf{34.53}\\\midrule\multirow{2}{*}{Gemma} & TriviaQA & 18.89&\textbf{22.99}\\ &NQ&23.43 &\textbf{26.52}\\\midrule\multirow{2}{*}{Llama-Inst}& TriviaQA & 10.42 &\textbf{19.41}\\ &NQ&9.53 &\textbf{18.08}\\\midrule\multirow{2}{*}{Mistral-Inst}& TriviaQA & 14.41 &\textbf{36.48}\\ &NQ&16.02 &\textbf{31.46}\\\midrule\multirow{2}{*}{Gemma- Inst}& TriviaQA & 19.43 &\textbf{35.73}\\ &NQ&18.52 &\textbf{31.72}\\
 

\bottomrule
\end{tabular}
        \caption{Jaccard Similarity of \chk hallucinations across different prompts. The \textit{Certain} column shows the overall similarity of \emph{\chk} samples between prompts in the TriviaQA and NaturalQA datasets and \textit{Low certain} shows the results of the lowest certainty subset, using \emph{Probability} as the certainty threshold. Results indicate high similarity, suggesting consistency across settings.}
        \label{tab:jaccard_prob_low_certainty}
        \end{table}




\section{\chk Persists in Larger Models -- Additional Results}
\label{appendix:chk Persists in Larger Models Additional Results}
In Section \ref{chk Persists in Instruction-Tuning and Larger Models}, we demonstrated that Gemma-2-27B achieves similar or slightly higher \chk detection results on the TriviaQA dataset than Gemma-2-9B, thus showing the existence of this phenomenon on larger models. To further illustrate this phenomenon, Figure \ref{fig:Hallucinations gemma natural} presents comparable results on the Natural Questions dataset.  
These results show a clear correlation with those presented in the main paper.


\begin{figure}
\centering
\begin{subfigure}[b]{0.23\textwidth}
  \centering
  \includegraphics[width=\linewidth]{Figures/pdfs/27vs9_gemma_naturalqa_alice_prob.pdf}
 \end{subfigure}%
 \hfill
  \centering
\begin{subfigure}[b]{0.23\textwidth}
  \centering
  \includegraphics[width=\linewidth]{Figures/pdfs/27vs9_gemma_naturalqa_alice_semantic_entropy.pdf}
 \end{subfigure}\\
 \hfill
\centering
\begin{subfigure}[b]{0.23\textwidth}
  \centering
  \includegraphics[width=\linewidth]{Figures/pdfs/27vs9_gemma_naturalqa_child_prob.pdf}
  \caption{Gemma, Child}
  \end{subfigure}
  \hfill
  \centering
  \begin{subfigure}[b]{0.23\textwidth}
  \centering
  \includegraphics[width=\linewidth]{Figures/pdfs/27vs9_gemma_naturalqa_alice_prob.pdf}
  \caption{Gemma, Alice-Bob}
 \end{subfigure}\\

 \caption{
Detection of \chk Hallucinations: Comparing Gemma-27B to Gemma-9B \chk hallucinations on Natural Questions. The results indicate that \chk hallucinations is similar and slightly higher for the larger model, Gemma-27B.}
 \label{fig:Hallucinations gemma natural}
\end{figure}





\end{document}
