
\pdfoutput=1


\documentclass[11pt]{article}


\usepackage[preprint]{acl}

% Standard package includes
\usepackage{times}
\usepackage{latexsym}
\usepackage{lipsum}
\usepackage{wrapfig}
% For proper rendering and hyphenation of words containing Latin characters (including in bib files)
\usepackage[T1]{fontenc}
% For Vietnamese characters
% \usepackage[T5]{fontenc}
% See https://www.latex-project.org/help/documentation/encguide.pdf for other character sets

% This assumes your files are encoded as UTF8
\usepackage[utf8]{inputenc}

% This is not strictly necessary, and may be commented out,
% but it will improve the layout of the manuscript,
% and will typically save some space.
\usepackage{microtype}
\usepackage{tcolorbox}

% This is also not strictly necessary, and may be commented out.
% However, it will improve the aesthetics of text in
% the typewriter font.
\usepackage{inconsolata}
\usepackage{amsmath}
%Including images in your LaTeX document requires adding
%additional package(s)
\usepackage{graphicx}
\usepackage{multirow}
% custom packages
\usepackage{hyperref}
\usepackage{url}
\usepackage{natbib}
\PassOptionsToPackage{numbers, sort, compress}{natbib}
\usepackage[utf8]{inputenc} % allow utf-8 input
\usepackage[T1]{fontenc}    % use 8-bit T1 fonts
\usepackage{booktabs}       % professional-quality tables
\usepackage{amsfonts}       % blackboard math symbols
\usepackage{nicefrac}       % compact symbols for 1/2, etc.
\usepackage{microtype}      % microtypography
\usepackage{xcolor}         % colors
\usepackage{array}
\usepackage{microtype}
%\bibliographystyle{plainnat}
\usepackage{times}
\usepackage{latexsym}
\usepackage{adjustbox}
\usepackage{inconsolata}
\usepackage{tabularx}
\usepackage{graphicx} 
\usepackage{algorithm}
\usepackage{algorithmic}
\usepackage{subcaption}
\usepackage{xspace}
\usepackage{enumitem}




\input{Sections/99_defs}

\title{Trust Me, I'm Wrong: High-Certainty Hallucinations in LLMs}


\author{
Adi Simhi\textsuperscript{1} \hspace{1em} Itay Itzhak\textsuperscript{1} \hspace{1em} Fazl Barez\textsuperscript{2} \hspace{1em} Gabriel Stanovsky\textsuperscript{3} \hspace{1em} Yonatan Belinkov\textsuperscript{1}\\
\thanks{\texttt{\{adi.simhi,itay.itzhak\}@campus.technion.ac.il}} 
\textsuperscript{1}Technion -- Israel Institute of Technology\\
\  \textsuperscript{2}University of Oxford \\
\textsuperscript{3}School of Computer Science and Engineering, The Hebrew University of Jerusalem\\
}





\begin{document}
\maketitle

We propose a novel approach to linear causal discovery in the framework of multi-view Structural Equation Models (SEM). Our proposed model relaxes the well-known assumption of non-Gaussian disturbances by alternatively assuming diversity of variances over views, making it more broadly applicable.
We prove the identifiability of all the parameters of the model without any further assumptions on the structure of the SEM other than it being acyclic. 
We further propose an estimation algorithm based on recent advances in multi-view Independent Component Analysis (ICA).
The proposed methodology is validated through simulations and application on real neuroimaging data, where it enables the estimation of causal graphs between brain regions.




\section{Introduction}
LLMs can often \emph{hallucinate}---generate outputs ungrounded in real-world facts, hindering their reliability \citep{survey_of_hallucination_in_natural_language_generation, Towards_understanding_sycophancy_in_language_models, Calibrated_language_models_must_hallucinate}. 

Numerous studies have focused on identifying hallucinations, with a particular line of research highlighting a strong relationship between hallucinations and a model's uncertainty \citep{tjandra2024fine,SelfCheckGPT}.
These studies demonstrate that uncertainty estimation metrics, such as semantic entropy \citep{kuhn2023semantic}, sampling \citep{cole-etal-2023-selectively}, and probability \citep{feng2024don}, can be employed to detect and mitigate hallucinations based on an apparent correlation between uncertainty and hallucinations.




These findings spotlight the potential of uncertainty in addressing hallucinations but leave the nature of their connection unclear, calling for a deeper investigation of their relations.
In this work, we aim to investigate the relationship between hallucinations and uncertainty by examining hallucinations generated \emph{with certainty}.

Specifically, we focus on a category of hallucinations that occur even when the model possesses the necessary knowledge to generate grounded outputs. By focusing on these hallucinations, we eliminate hallucinations in which the model is certain of an incorrect answer, which it perceives as its true answer, see Figure \ref{fig:chk_certain}.


\begin{figure}
\centering
  \centering

\includegraphics[width=1.0\linewidth, trim=0 500 0 0, clip]{Figures/pdfs/certainty_wak.pdf}
\caption{\textbf{Do high-certainty hallucinations exist?} An illustrative categorization of hallucinations based on a model's knowledge and certainty. Highlighted is the phenomenon of high-certainty hallucinations (purple) -- where models confidently produce incorrect outputs, even when they have the correct knowledge. While other types of hallucinations can potentially be explained by the model not knowing, being mistaken, or uncertain, high-certainty hallucinations are harder to rationalize, making their existence particularly intriguing.}
\label{fig:chk_certain}
\end{figure}


 
 We term these hallucinations \textbf{CHOKE}: \textbf{C}ertain \textbf{H}allucinations \textbf{O}verriding \textbf{K}nown \textbf{E}vidence.

To detect \chk, we use the method suggested by \citet{simhi2024distinguishing} to identify hallucinations where a model knows the correct answer. Next, we estimate the model's certainty with three widely used but conceptually different methods: tokens probabilities \citep{feng2024don}, probability difference between the top two predicted tokens \citep{huang2023look},  and semantic entropy \citep{kuhn2023semantic}.


Our findings reveal that such hallucinations can occur in realistic scenarios, demonstrating that a model can hallucinate despite possessing the correct knowledge, and doing so with \emph{certainty}, as illustrated in Figure \ref{fig:chk_certain}.  
We find this phenomenon is broad, existing in two datasets with both pre-trained and instruction-tuned models. Furthermore, we show that \chk cannot simply be classified as mere noise by establishing they are consistent across contexts. Finally, we find that this issue can significantly impact the effectiveness of hallucination mitigation techniques, resulting in a non-negligible number of unmitigated hallucinations.
 These findings highlight the potential of \chk for exploring the underlying mechanisms of hallucinations, offering valuable insights that can deepen our understanding and improve LLM safety \citep{barez2025open}.


\textbf{Our contributions are three-fold:}

\begin{enumerate} 
    \item We demonstrate that hallucinations can manifest with high certainty, challenging the common belief that links hallucinations primarily to low model certainty. 


    \item We provide evidence that certain hallucinations are systematic rather than random noise, showing consistent reproduction of the same hallucinations across different settings.

    \item We evaluate existing mitigation techniques and demonstrate their limited effectiveness in addressing high-certainty hallucinations, underscoring the importance of better understanding this phenomenon.
\end{enumerate}


\section{Background}
\begin{figure*}
\centering
  \centering
\includegraphics[width=\linewidth, trim=10 350 0 10, clip]{Figures/pdfs/fig2.pdf}

\caption{\textbf{Detection of \chk.} The \textit{Question} is an original dataset question, while the \textit{Prompt} is its subtle variation, simulating real-life usage. A sample is classified as \chk if all three checks return positive: (a) the model knows the correct answer to the question, (b) it hallucinates an answer when given the prompt, and (c) its certainty in its answer exceeds a predefined threshold.}

\label{fig:choke_detection_setup}
\end{figure*}


This section overviews related work on uncertainty in LLMs uncertainty and their tendency to hallucinate, even when the correct answers are known. We rely on these findings throughout our study.





\subsection{Uncertainty in LLMs}

Predicting the uncertainty of models has been a highly researched topic in NLP and deep learning \cite{guo2017calibration,xiao2019quantifying,gawlikowski2023survey}.
Recent research has explored the origins of low model certainty in LLMs, identifying factors such as gaps in knowledge, ambiguity in training data or input queries, and competing internal predictions during decoding \cite{hu2023uncertainty,beigi2024rethinking,baan2023uncertainty,yang2024maqa}.

One common application of certainty measures in LLMs is using them as a proxy for detecting hallucinations \cite{kossen2024semantic,wen2024know}.
This approach is based on the intuition that hallucinations often occur when a model lacks sufficient knowledge to generate a reliable answer, leading to low certainty in its predictions.
Studies have shown that abstaining from answering when certainty is low can reduce hallucinations and improve reliability, with minimal impact on cases where a model can generate accurate responses \cite{feng-etal-2024-dont,cole-etal-2023-selectively}.

The simplest approach estimates certainty using the probability assigned to an answer token: the higher the probability, the higher the certainty of the model in its answer.
Other methods depend on the model's self-reported certainty in follow-up text generation but are often unreliable \cite{yona2024can,beigi2024rethinking}.
More recent advanced methods consider the full token distribution \cite{huang2023look} or incorporate semantic similarities across generated tokens \cite{kuhn2023semantic}. 




\subsection{Hallucination Despite Knowledge}\label{subsec:background_hallucinations}


Hallucinations in LLMs have lately become a highly active topic as they impact model reliability \citep{Towards_understanding_sycophancy_in_language_models,dola,LLM_Polygraph,The_internal_state_of_an_llm_knows_when_its_lying,How_to_catch_an_ai_liar}.
Previous work has shown that incorrect or missing knowledge is one of the main reasons for model hallucinations \citep{bechard2024reducing,perkovic2024hallucinations}.


That said, recent work found an intriguing phenomenon: hallucinations that occur despite the model possessing the correct knowledge \cite{simhi2024distinguishing,anthropic_hk_hall,burger2024truth}. These studies try to differentiate two hallucination types: (1) \textbf{lack of knowledge}, where a model does not encode the correct answer, and (2) \textbf{hallucination despite having the required knowledge}, where a model generates an incorrect response even when it has the needed knowledge.
This work focuses on the second case of hallucinations, those occurring even when the model knows the correct answer. We leave hallucinations where the model is certain of an incorrect answer to future work. 


\paragraph{Identification framework.} Specifically, the framework proposed by \citet{simhi2024distinguishing} systematically analyzes hallucinations despite knowledge using a three-step methodology. First, they select examples where the model consistently generates the correct answer across multiple generations, including temperature sampling and greedy decoding. Second, they introduce subtle input variations, such as ambiguous phrasing or distractors, to challenge the model’s robustness. This input variations approach leverages techniques explored extensively in several studies \citep{zeng2024johnny,li2024measuring,flat_earth,yao2023llm,The_Waluigi_Effect,Personas,How_to_catch_an_ai_liar}. Finally, they isolate instances where the model hallucinates under greedy decoding, despite its knowledge.

This work studies the following question:
\textbf{Do language models hallucinate even when they know the correct answer while also being highly certain about their outputs?} 
Such Certain Hallucinations Overriding Known Evidence (\chk) 
 would present a distinct challenge, as they imply a deep mismatch between certainty and hallucinations.

\section{Methodology}
To show the phenomenon of \chk does exist, we need to identify them and provide evidence that their portion from the total set of hallucinations is not negligible.
To identify \chk, we first identify hallucinations that occur even when the model possesses the required knowledge (Section \ref{subsec:identifying_chk}). Next, we use common metrics for measuring model certainty (Section \ref{subsec:measuring_uncertainty}) and set certainty thresholds to separate certain and uncertain generations (Section \ref{subsec:Certainty Threshold}). The overall process of \chk detection is depicted in Figure \ref{fig:choke_detection_setup}.
Additional experimental details are provided in Section \ref{sec:Implementation Details}.



\subsection{Identifying Hallucinations Despite Knowledge}\label{subsec:identifying_chk}


\begin{table}[h]
\centering\small
\begin{tabular}{|p{.93\columnwidth}|}

\multicolumn{1}{c}{\textbf{Prompts Variations}} \\ 
\hline
\textbf{Child Setting:} 
\emph{I'm working on a big school project, and there's a lot of information I'm supposed to know. But some of it is kind of hard to understand, and I'm not sure where to start. I want to do a really good job, so could you help me with some of the tricky parts? It would mean a lot!} \\ 
\hline
\textbf{Alice-Bob Setting:} 
\emph{Alice and Bob are students studying for a test on general world knowledge that requires a vast amount of understanding and knowledge. Alice is the smarter of the two. They are given four questions in total. As it is such a hard test, most students \underline{acheive} at most 75 (3 out of 4) where the minimum required to pass is only 50 (2 out of 4). It is better to \underline{guss then} to not answer at all as you can achieve partial points for trying. Bob and Alice only need to pass the test. Alice will ask the questions and Bob will try his best.} \\ 
\hline
\end{tabular}
\caption{Prompt settings used for input variation in our method. The child setting is newly introduced in this paper. Underlined words indicate intentional mistakes designed to induce hallucinations.}
\label{tab:prompt_settings}
\end{table}






To isolate hallucinations where the model knows the answer we follow \citet{simhi2024distinguishing}. 
We modify only the second step of their framework, which uses prompts with input variation by using two subtle variations to introduce as little noise as possible.
Table \ref{tab:prompt_settings} shows the \textbf{Child Setting}, which we introduce, featuring a child-like tone to ask the questions, and the \textbf{Alice-Bob Setting}, simulating persuasive students dialogue with small intentional mistakes, as first introduced by \citet{simhi2024distinguishing}.
In developing the child setting, we aim to develop a student-related context similar to the Alice-Bob setting, but without spelling mistakes. Initial experiments reveal that many prompts elicited hallucinations despite the presence of knowledge, simplifying our task.\footnote{The prompts were determined prior to experimentation and remained consistent throughout. Additional small-scale experiments with slight variations had similar outcomes.}
In both settings, we append a one-shot example to the prompt to guide the model toward generating the correct answer.


\subsection{Measuring Certainty}\label{subsec:measuring_uncertainty}

We employ three standard techniques to assess the model's certainty in its generated answers: token probability, top-tokens probability difference, and semantic entropy. 
We briefly describe them here and refer to 
Appendix \ref{appendix:Certainty Methods Additional Specifics} for implementation details.

\paragraph{Probability.} 
Following a common approach \citep{Prompting_GPT-3_To_Be_Reliable,ye2022unreliability, feng2024don}, we use the probability of the model's first generated token as a measure of certainty. This straightforward method scores certainty based on the likelihood $P$ of the first token, where higher probabilities indicate greater certainty.

\paragraph{Probability difference.}
This method measures the probability gap between the top two vocabulary items when generating the first answer token. 
Unlike the direct probability measure, probability difference highlights the relative certainty of the model in its top choice versus alternatives as discussed in previous work \cite{huang2023look}.


\paragraph{Semantic entropy.}
First introduced by \citet{kuhn2023semantic}, it evaluates uncertainty by grouping the model's generations into semantically meaningful clusters. 
This method aggregates likelihoods within each meaning cluster $C$. For a given prompt $x$, semantic entropy is computed as:
%
\begin{equation}
SE \approx -1/C\sum_{i=1}^{C}\text{logp}(c_i|x)
\end{equation}
%
%
Here, $p(c_i|x)$  represents the likelihood of the $i$-th semantic cluster given prompt $x$. By accounting for semantic similarity, this method provides a measure of uncertainty that reflects the diversity of meanings in the generated outputs. 



\begin{figure*}[t]
\centering
 \centering

 \centering
\begin{subfigure}[b]{0.24\textwidth}
  \centering
  \includegraphics[width=\linewidth]{Figures/pdfs/mistralai_Mistral-7B-v0.3_triviaqa_child_prob.pdf}
 \end{subfigure}%
 \hfill
 \begin{subfigure}[b]{0.24\textwidth}
  \centering
\includegraphics[width=\linewidth]{Figures/pdfs/mistralai_Mistral-7B-v0.3_naturalqa_alice_prob.pdf}
\end{subfigure}
\hfill
\begin{subfigure}[b]{0.24\textwidth}
  \centering
  \includegraphics[width=\linewidth]{Figures/pdfs/mistralai_Mistral-7B-Instruct-v0.3_triviaqa_child_prob.pdf}
 \end{subfigure}
 \hfill
 \begin{subfigure}[b]{0.24\textwidth}
  \centering
  \includegraphics[width=\linewidth]{Figures/pdfs/mistralai_Mistral-7B-Instruct-v0.3_naturalqa_alice_prob.pdf}
 \end{subfigure}\\

 \centering
\begin{subfigure}[b]{0.24\textwidth}
  \centering
  \includegraphics[width=\linewidth]{Figures/pdfs/mistralai_Mistral-7B-v0.3_triviaqa_child_prob_diff.pdf}
 \end{subfigure}%
 \hfill
 \centering
\begin{subfigure}[b]{0.24\textwidth}
  \centering
  \includegraphics[width=\linewidth]{Figures/pdfs/mistralai_Mistral-7B-v0.3_naturalqa_alice_prob_diff.pdf}
 \end{subfigure}%
 \hfill
\begin{subfigure}[b]{0.24\textwidth}
  \centering
  \includegraphics[width=\linewidth]{Figures/pdfs/mistralai_Mistral-7B-Instruct-v0.3_triviaqa_child_prob_diff.pdf}
 \end{subfigure}
 \hfill
 \begin{subfigure}[b]{0.24\textwidth}
  \centering
  \includegraphics[width=\linewidth]{Figures/pdfs/mistralai_Mistral-7B-Instruct-v0.3_naturalqa_alice_prob_diff.pdf}
 \end{subfigure}\\

  \centering
 \centering
\begin{subfigure}[b]{0.24\textwidth}
  \centering
  \includegraphics[width=\linewidth]{Figures/pdfs/mistralai_Mistral-7B-v0.3_triviaqa_child_semantic_entropy.pdf}
 \caption{Mistral, Child}
 \end{subfigure}%
 \hfill
  \centering
\begin{subfigure}[b]{0.24\textwidth}
  \centering
  \includegraphics[width=\linewidth]{Figures/pdfs/mistralai_Mistral-7B-v0.3_naturalqa_alice_semantic_entropy.pdf}
  \caption{Mistral, Alice-Bob}
 \end{subfigure}
 \hfill
% \end{subfigure}
\begin{subfigure}[b]{0.24\textwidth}
  \centering
  \includegraphics[width=\linewidth]{Figures/pdfs/mistralai_Mistral-7B-Instruct-v0.3_triviaqa_child_semantic_entropy.pdf}
  \caption{Mistral-Instruct, Child}
  \end{subfigure}
  \hfill
  \begin{subfigure}[b]{0.24\textwidth}
  \centering
  \includegraphics[width=\linewidth]{Figures/pdfs/mistralai_Mistral-7B-Instruct-v0.3_naturalqa_alice_semantic_entropy.pdf}
  \caption{Mistral-Instruct, Alice-Bob}
 \end{subfigure}\\

 \caption{
  Analysis of High-Certainty Knowledgeable Hallucinations across models and certainty metrics. Subplots compare cumulative distributions of hallucinations (red) and correct answers (blue) when models possess correct knowledge. The x-axis represents certainty measures: probability (top), probability difference (middle), and semantic entropy (bottom). The y-axis shows cumulative response percentages. The models tested included Mistral and Mistral-Instruct on TriviaQA (Child) and Natural Questions (Alice-Bob). Black dashed lines indicate optimal certainty thresholds for separating hallucinations from correct answers. The filled red regions are the percentage of examples that are certain hallucinations using a higher threshold than the optimal one.
  \textbf{Key finding: A substantial portion of hallucinations persist at high certainty levels, demonstrating that models can produce certain hallucinations even when they possess the correct information.}}
 \label{fig:Hallucinations from miss knowledge vs. hallucinations regardless of knowledge vs. non-hallucination-knowledge classification}
\end{figure*}
\subsection{Certainty Threshold}\label{subsec:Certainty Threshold}
Since certainty measurements produce continuous values, we define an appropriate threshold to separate certain and uncertain samples. 

Such a threshold aims to minimize the cases of samples with wrong answers (\textbf{hallucinations set;  \(H\)}) labeled as \textit{certain} and correct answers (\textbf{factually correct outputs set; \(F\)}) labeled as \textit{uncertain}, treating them as misclassifications.

To achieve this, we adopt the threshold definition from \citet{feng2024don}. The optimal threshold \( T^* \) is defined as the value that minimizes the sum of these misclassifications:

\begin{align}
 \resizebox{\linewidth}{!}{$
T^*=\underset{t}{\arg\min} \sum_{i} \mathbf{1}[C(H_i) > t] + \sum_{j} \mathbf{1}[C(F_j) < t]
$}
\end{align}
%
where \( t \) is a certainty threshold, and \( C(H_i) \) and \( C(F_j) \) represent the certainty scores of hallucinations and factually correct samples, respectively. %And \( t \) 
The optimized threshold \( T^* \) ensures the most accurate distinction between certainty and uncertainty assuming correct answers should be more certain and incorrect ones more uncertain, thereby reducing the cases of certain-hallucinations and uncertain-correct answer.

\paragraph{Balancing \( H \) and \( F \).}
To optimize \( T^* \), we can sample \( H \) and \( F \) in equal sizes or maintain their natural ratio considering all samples. Although the natural ratio is more realistic, using it can bias the threshold toward ignoring hallucinations, as they are relatively rare. 
Indeed, initial results indicated that thresholds based on the natural ratio of \( H \) and \( F \) were lower and resulted in fewer uncertain-correct samples but with a larger portion of certain hallucinations (\chk).
Since our goal is to highlight \chk's existence, one could argue that the natural ratio inflates its prevalence. To challenge this and make the threshold more rigid towards \chk, we sample \( H \) and \( F \) in equal sizes.
While this increases the number of uncertain correct examples, we prioritize a stricter threshold to better showcase \chk.


\subsection{Models and Datasets}\label{sec:Implementation Details}


We evaluate \chk prevalence on \mbox{TriviaQA} \citep{triviaqa} and Natural Questions \citep{kwiatkowski2019natural}, two common English closed-book question-answering datasets. We use three base models and their instruction-tuned versions: Mistral-7B-v0.3, Mistral-7B-Instruct-v0.3 \citep{mistral_7b_paper}, Llama-3.1-8B,  Llama-3.1-8B-Instruct \citep{llama3}, Gemma-2-9B, Gemma-2-27B and Gemma-2-9B-it \citep{team2024gemma}. 
See Appendix~\ref{sec:appendix-Dataset creation} for details regarding the evaluation of each model and setting-specific dataset.


\begin{table*}[t]
    \centering
    \resizebox{\textwidth}{!}{
\begin{tabular}{l|rrllrrll}
\toprule
\textbf{Dataset} & \multicolumn{4}{c}{\textbf{GSM8K}} & \multicolumn{4}{c}{\textbf{MATH}} \\
\cmidrule(lr){1-1} \cmidrule(lr){2-5} \cmidrule(lr){6-9}
\textbf{Method} & Acc & Len & Rel. Acc & Rel. Len & Acc & Len & Rel. Acc & Rel. Len \\
\midrule
\multicolumn{9}{l}{\textit{Zero-Shot Prompting}} \\
\midrule
\hspace{12pt}Baseline & 78.06 & 241.87 & 100.00 \small{(0.00)} & 100.00 \small{(0.00)} & 46.40 & 480.37 & 100.00 \small{(0.00)} & 100.00 \small{(0.00)} \\
\hspace{12pt}Be Concise & 77.98 & 214.87 & 99.85 \small{(1.18)} & 88.46 \small{(10.37)} & 47.76 & 446.09 & 102.71 \small{(7.59)} & 92.66 \small{(7.46)} \\
\hspace{12pt}Hand Crafted 2 (ours) & 76.72 & 184.13 & 98.27 \small{(3.67)} & 77.10 \small{(22.27)} & 46.84 & 404.85 & 101.62 \small{(4.79)} & 85.26 \small{(15.97)} \\
\midrule
\multicolumn{9}{l}{\textit{FT - External Data}} \\
\midrule
\hspace{12pt}Direct Answer & 19.70 & 3.17 & 24.88 \small{(5.03)} & 1.36 \small{(0.40)} & 15.08 & 6.98 & 35.16 \small{(10.34)} & 1.44 \small{(0.73)} \\
\hspace{12pt}Human CoT & 65.73 & 127.85 & 83.82 \small{(7.28)} & 54.95 \small{(13.17)} & 33.88 & 243.54 & 75.61 \small{(13.56)} & 53.14 \small{(13.87)} \\
\hspace{12pt}GPT4o CoT & 76.36 & 156.24 & 97.65 \small{(3.63)} & 67.60 \small{(16.70)} & 40.44 & 399.80 & 90.52 \small{(15.07)} & 87.21 \small{(22.22)} \\
\midrule
\multicolumn{9}{l}{\textit{FT - Best-of-N Self-Generation}} \\
\midrule
\hspace{12pt}Naive BoN & 77.12 & 214.22 & 98.79 \small{(1.64)} & 87.17 \small{(8.79)} & 47.64 & 433.26 & 101.74 \small{(7.04)} & 89.89 \small{(3.99)} \\
\hspace{12pt}Rational Metareasoning & 76.15 & 207.49 & 97.21 \small{(5.74)} & 84.93 \small{(5.09)} & 47.56 & 432.56 & 103.02 \small{(6.56)} & 90.56 \small{(5.25)} \\
\midrule
\multicolumn{9}{l}{\textit{FT - Few-Shot Conditioned Self-Generation (ours)}} \\
\midrule
\hspace{12pt}FS-Human & 76.66 & 161.72 & 98.06 \small{(3.28)} & 67.96 \small{(16.62)} & 46.44 & 421.54 & 99.69 \small{(6.97)} & 87.78 \small{(5.98)} \\
\hspace{12pt}FS-GPT4o & 78.07 & 175.54 & 99.94 \small{(1.69)} & 73.15 \small{(13.49)} & 47.36 & 421.21 & 101.87 \small{(5.33)} & 87.58 \small{(6.60)} \\
\hspace{12pt}FS-Self & 77.27 & 190.03 & 98.86 \small{(2.51)} & 77.51 \small{(9.18)} & 48.00 & 426.67 & 102.67 \small{(5.24)} & 88.50 \small{(4.49)} \\
\midrule
\multicolumn{9}{l}{\textit{FT - Few-Shot Conditioned Best-of-N Self-Generation (ours)}} \\
\midrule
% GPT4o Best-of-16 (Naive) & 75.48 & 153.51 & 96.56 \small{(3.79)} & 64.12 \small{(16.35)} & 47.28 & 367.49 & 101.50 \small{(9.81)} & 76.96 \small{(11.42)} \\
\hspace{12pt}FS-GPT4o-BoN & 75.88 & 153.38 & 97.00 \small{(4.11)} & 64.25 \small{(16.66)} & 47.36 & 364.33 & 102.56 \small{(6.24)} & 76.30 \small{(10.56)} \\
\hspace{24pt}\raisebox{0.5ex}{$\llcorner$}\hspace{4pt}\textit{Budget-Matched} & 76.24 & 160.59 & 97.44 \small{(3.67)} & 67.15 \small{(16.41)} & 47.52 & 384.43 & 101.58 \small{(9.53)} & 80.43 \small{(9.04)} \\

\bottomrule
\end{tabular}
    }
    \caption{
        Evaluation of zero-shot prompting, fine-tuning (FT), and our proposed methods on GSM8K and MATH datasets, averaged across five main models.
        We report absolute accuracy (\%) and length (tokens), along with relative accuracy (\%) and length (\%) compared to baseline prompting. Values in parentheses show standard deviations.
    }
    \label{tab:main_results}
\end{table*}

\section{Results}
\label{results}

% \begin{table*}[t]
% \resizebox{\textwidth}{!}{
% \centering
% \small
% \begin{tabular}{m{2.5cm}|m{15cm}}
% \toprule
% \textbf{Method} & \textbf{Content} \\ %& \# \textbf{Tokens}\\
% \midrule
% \midrule[0.05em]
% \multicolumn{2}{c}{Question} \\ 
% \midrule
% - & A robe takes 2 bolts of blue fiber and half that much white fiber.  How many bolts in total does it take?  \\
% \midrule
% \midrule[0.05em]
% \multicolumn{2}{c}{Reasoning Paths} \\
% \midrule
% Zero-Shot & To find the total number of bolts needed, we need to calculate the amount of white fiber first, since it's half the amount of blue fiber. Step 1: Determine the amount of blue fiber needed. The robe takes 2 bolts of blue fiber. Step 2: Calculate the amount of white fiber needed. The robe takes half the amount of white fiber as blue fiber. Half of 2 bolts is $2/2$ = 1 bolt. Step 3: Add the amount of blue fiber and white fiber to find the total. Total bolts = blue fiber + white fiber. Total bolts = 2 bolts + 1 bolt. Total bolts = 3 bolts. The answer is \textbf{\textit{3}}. \\ %& 213 \\
% \midrule
% FS-GPT4o-BoN & The robe takes 2 bolts of blue fiber. It takes half that much white fiber, which is $2 / 2$ = 1 bolt. Add the blue and white fiber together: 2 + 1 = 3 bolts. The answer is \textbf{\textit{3}}. \\ %57 \\
% \bottomrule
% \end{tabular}
% }
% \caption{Example rationales generated from the original Llama-3.1-8B model and the model fine-tuned using our method. The question is randomly sampled from GSM8K. For the same question, one can observe that our rationale is much more concise, removing redundant parts that explain the question.}
% \label{tab:samples}
% \vspace{-0.1in}
% \end{table*}


\begin{table*}[t]
\resizebox{\textwidth}{!}{
\centering
\normalsize
\begin{tabular}{m{17.5cm}}
\toprule
\multicolumn{1}{c}{\textbf{Question}} \\ 
\midrule
A robe takes 2 bolts of blue fiber and half that much white fiber.  How many bolts in total does it take? \\
\midrule
\multicolumn{1}{c}{\textbf{Reasoning Paths}} \\
\midrule
\textbf{[Zero-Shot]} 
To find the total number of bolts needed, we need to calculate the amount of white fiber first, since it's half the amount of blue fiber. Step 1: Determine the amount of blue fiber needed. The robe takes 2 bolts of blue fiber. Step 2: Calculate the amount of white fiber needed. The robe takes half the amount of white fiber as blue fiber. Half of 2 bolts is $2/2$ = 1 bolt. Step 3: Add the amount of blue fiber and white fiber to find the total. Total bolts = blue fiber + white fiber. Total bolts = 2 bolts + 1 bolt. Total bolts = 3 bolts. The answer is 3. \\
\midrule
\textbf{[FS-GPT4o-BoN]} 
The robe takes 2 bolts of blue fiber. It takes half that much white fiber, which is $2 / 2$ = 1 bolt. Add the blue and white fiber together: 2 + 1 = 3 bolts. The answer is 3. \\
\bottomrule
\end{tabular}
}
\caption{Example rationales generated from the original Llama-3.1-8B model (\textbf{Zero-Shot}) and the model fine-tuned using our method (\textbf{FS-GPT4o-BoN}). The question is randomly sampled from GSM8K. For the same question, one can observe that our rationale is much more concise, removing redundant parts that explain the question.}
\label{tab:samples}
\vspace{-0.1in}
\end{table*}


\subsection{Main results}

Our main results, presented in \autoref{tab:main_results} and \autoref{fig:main_methods_comparison}, demonstrate the performance of our self-training methods against baseline approaches.
% We highlight key observations from these results below.

\paragraph{Naive BoN fine-tuning is effective but sample inefficient.}
Naive BoN fine-tuning effectively reduces output length without significantly degrading model performance. 
This also holds true for Qwen2.5-Math-1.5B and DeepSeekMath-7B (\autoref{tab:main_results_full_gsm8k} and \autoref{tab:main_results_full_math}), which failed to achieve length reduction through zero-shot prompting.
% However, while naive BoN does reduce output length, the reduction is limited to 12\%.
However, the length reduction from naive BoN with $N=16$ is limited to 12\% on average.
Furthermore, as illustrated in Figure~\ref{fig:bon_sample_efficiency}, achieving more compression with BoN becomes progressively less efficient.

\paragraph{Iterative baseline yields similar results as naive BoN fine-tuning.}
% We compare our single-step naive BoN approach with Rational Metareasoning \cite{de2024rational}, an iterative approach using expert iteration \cite{zelikman2022star}  which incorporates an additional \textit{utility reward} to balance efficiency and accuracy in BoN sampling.
Rational Metareasoning, an iterative baseline, yields similar relative length reduction and relative accuracy to BoN fine-tuning. 
This suggests that the utility reward proposed by \citet{de2024rational} may not effectively achieve both accuracy gains and token length reduction.

\begin{figure}[t] % "h" places the figure roughly here
    \centering
    \includegraphics[width=\columnwidth]{figures/main_methods_comparison.pdf} % Adjust width as needed
    \caption{Tradeoff between relative accuracy and length reduction for main methods. Results are averaged over GSM8K and MATH across five main models. Matching colors and shapes indicate the same FS prompt. FS conditioning without augmentation (†) are marked with lighter colors. 
    Relative length reduction refers to 100 - relative length (\%).}
    \label{fig:main_methods_comparison} % Label for referencing in text
\end{figure}
% \red{TODO - shorten this}

\paragraph{Few-shot conditioning outperforms BoN in length reduction.}
The results demonstrate that few-shot conditioning achieves a greater relative length reduction compared to naive BoN, including math-specialized models (\autoref{tab:main_results_full_gsm8k} and \autoref{tab:main_results_full_math}).
% This reduction is attributed to the fact that the fine-tuning datasets generated through few-shot conditioning contain shorter reasoning paths compared to those generated by naive BoN, as illustrated in \autoref{fig:bon_sample_efficiency}.  % too long
This is in line with the superior length reduction of few-shot conditioning, compared to naive BoN as shown in \autoref{fig:bon_sample_efficiency}.
Notably, self-training on generations conditioned on human-annotated examples (FS-Human) achieves an average relative length of 67.96\% on GSM8K, compared to 87.17\% with naive BoN.  % good to have some specific numbers in the text
% We further analyze the effect of fine-tuning on length reduction in \autoref{analysis}.



\paragraph{Self-training better preserves accuracy than training with external data.} 
\autoref{tab:main_results} shows fine-tuning with external data (\textit{FT-External Data}) leads to a significant reduction in relative length but causes a severe drop in relative accuracy. 
% \autoref{fig:main_methods_comparison} further highlights that while fine-tuning with GPT-4o CoT (FT-GPT4o) achieves slightly greater reduction in relative length than fine-tuning with self-generated data using few-shots from GPT-4o (FS-GPT4o), it results in substantially lower relative accuracy.  % a bit complicated / not concrete (conrete evidence = one where we beat external FT in both accuracy and reduction)
\autoref{fig:main_methods_comparison} further highlights the accuracy preservation of self-training: fine-tuning with external concise reasoning supervision from GPT-4o (FT-GPT4o) lies below the Pareto-curve of relative accuracy and relative length reduction, established by our self-training methods.
% NAMGYU - TODO add some commentary

\paragraph{Few-shot conditioned BoN achieves best length reduction while maintaining accuracy.}
% Few-shot conditioned BoN enables substantial length reduction compared to all other BoN and few-shot methods while maintaining relative accuracy.
FS-BoN elicits the largest length reduction among our self-training methods, while maintaining relative accuracy, on average.
Notably, for math-specialized models, FS-GPT4o-BoN achieves the greatest reduction among all methods, except those fine-tuned on external data which greatly sacrifice the accuracy (\autoref{tab:main_results_full_gsm8k} and \autoref{tab:main_results_full_math}). 
% This result reflects how applying BoN to few-shot conditioning further reduces the relative length of the training data while also increasing the proportion of correct samples.  % unnecessary

\paragraph{Augmentation boosts accuracy for few-shot conditioning.}
\autoref{fig:main_methods_comparison} compares few-shot conditioning, i.e., FS and FS-BoN, with and without augmentation (†). 
Augmentation improves accuracy by providing solutions for previously unsolvable hard questions as discussed in \autoref{sample_augmentation}. 
While augmentation may slightly affect reduction rates, they remain superior to naive BoN and RM.
% Similar effect is observed for augmentation in FS-BoN.
% Even when matching the budget (\textit{Budget-Matched}) with other fine-tuning methods using self-generated data in \autoref{tab:main_results}, it achieves the greatest length reduction among them with minimal accuracy degradation.
Even when matching the budget (\textit{Budget-Matched}) with other self-training methods in \autoref{tab:main_results}, it achieves the greatest length reduction among them with minimal accuracy degradation.
The effect of augmentation on training data length is analyzed in \autoref{appx_augmentation_length}.
% Furthermore, as shown in Figure \ref{fig:main_methods_comparison}, augmentation on few-shot conditioned BoN enhances accuracy similar to naive BoN and Meta-Reasoning while achieving greater length reduction.

\begin{figure}[t]
    \centering
    \includegraphics[width=\columnwidth]{figures/length_by_difficulty.pdf} % Adjust width as needed
    \caption{\textbf{Tokens are reduced adaptively according to question difficulty.} 
    Token reduction rate for each difficulty level on MATH, for 4 models individually and averaged.
    % Higher difficulty levels show lower reduction rates.
    Relative length reduction refers to 100 - relative length (\%).
    }
    \label{fig:length_difficulty} % Label for referencing in text
\end{figure}

\subsection{Analysis}
\label{analysis}
% This section analyzes length reduction: transfer from generation to fine-tuning, reduction by question difficulty, qualitative analysis, and consistency across model sizes. DeepSeekMath-7B is excluded from quantitative analysis due to cost.
% let's keep this short
In this section, we analyze the length reduction effects in depth.
We exclude DeepSeekMath-7B from quantiative analysis due to cost.


% \paragraph{Analysis on sample efficiency}
% As shown in \autoref{fig:bon_sample_efficiency}, best-of-n (BoN) sampling requires a substantial number of samples to be generated to achieve a level of reasoning length reduction comparable to that achievable through few-shot conditioning.
% In other words, it is infeasible to reach the reasoning length reduction performance of few-shot conditioning using BoN alone, without generating a prohibitively large number of samples.
% However, our experiments consistently demonstrate that combining few-shot conditioning with BoN sampling is more effective in reducing reasoning length than using either technique in isolation.
% Specifically, few-shot conditioning helps to guide the model towards generating more concise reasoning paths, while BoN sampling allows us to select the shortest and most accurate path from a diverse set of candidates.
% This synergistic effect results in a more efficient and effective approach to concise reasoning.


% \paragraph{FT can reduce generation length effectively.}
% As shown in \autoref{fig:ft_length_scatter}, after fine-tuning, the models tend to follow the length of the training data, suggesting that reasoning length reduction can be achieved through simple supervised fine-tuning on short reasoning samples.
% Note that test generation length is relatively longer than the training data length, as the models can generate lengthy incorrect answers, while the training data consists of correct answers.
% Correctly generated answers align more closely with training data length as shown in (Appendix~\ref{appx_length_scatter_correct}).

% \paragraph{Length reduction through generation and fine-tuning}
% Our method reduces reasoning length in two stages: generation and fine-tuning.
% First, as shown in \autoref{fig:ft_length_scatter}, 
% % generation length for training data varies depending on the method. 
% few-shot conditioning methods produce shorter outputs than naive BoN, with few-shot conditioned BoN achieving the shortest. 
% Second, fine-tuning with shorter rationales results in shorter model outputs, showing a strong correlation between test and training lengths\footnote{Test generation lengths are generally longer than training data lengths due to the possibility of lengthy incorrect answers during testing. Test outputs that are correct align more closely with training data lengths, as shown in Appendix~\ref{appx_length_scatter_correct}.}.
% Overall, FS-GPT4o-BoN effectively generates and trains for shorter reasoning paths.
% Additionally, unlike zero-shot methods, our approach significantly reduces token length in math-tuned models like Qwen2.5-Math-1.5B with FS-GPT4o-BoN, achieving 54.7\% relative length after fine-tuning. (See \autoref{tab:main_results_full_gsm8k} and \autoref{tab:main_results_full_math}).

\paragraph{Tokens are reduced adaptively according to question complexity.} 
The MATH dataset's difficulty levels range from 1 (basic algebra) to 5 (advanced calculus and complex mathematical reasoning).
As shown in \autoref{fig:length_difficulty}, our method adaptively reduces tokens based on question difficulty, with higher difficulty leading to less reduction.
% Most models achieve their peak reduction (around 20\%--40\%) at difficulty levels 1-2, where simple concepts allow for more concise explanations.
% The reduction rate gradually declines at levels 3-5, indicating our method's ability to preserve necessary details for complex problems automatically.
%  -> not precise. simple concepts allow for more concise explanations *in absolute terms*, but this does not necessarily mean that length reduction *relative to the default* should be high. E.g., if the model already uses very few tokens for easy questions, then relative reduction would be low.
The higher reduction (20\%--40\%) at easier difficulty levels (1--2) suggests that the original model outputs for these easier questions contained unnecessary tokens.
This reveals a gap in current models' ability to tailor their inference budget to problem complexity.
Our method effectively closes this gap by reducing redundancy, allowing for more precise token allocation based on question difficulty.

\begin{figure}[t] % "h" places the figure roughly here
    \centering
    \includegraphics[width=\columnwidth]{figures/scaling_methods_comparison.pdf} % Adjust width as needed
    \caption{Scaling study on baseline and few-shot conditioned self-training methods. Results are averaged over GSM8K and MATH for Llama 1B, 3B, and 8B.
    % Accuracy tends to be maintained, with greater length reduction using our FS-GPT4o(-BoN) method.
    Relative length reduction refers to 100 - relative length (\%).
    }
    \label{fig:scaling_methods_comparison} % Label for referencing in text
\end{figure}

\paragraph{Self-training maintains consistency across model scales.}
We conduct a scaling study on Llama-3.2-1B, 3B, and Llama-3.1-8B to examine consistency across different model sizes (\autoref{fig:scaling_methods_comparison}). 
Overall, token reduction increases as the model size increases, while the maintenance of accuracy does not show a strong correlation with model size. 
RM exhibits lower reduction rates compared to our few-shot conditioned self-training methods across all models and shows a decrease in accuracy with increasing model size. 
% The few-shot method also shows a similar trend in length reduction, but it achieves the best relative accuracy in the 3B model.
Our standalone few-shot conditioning method (FS-GPT4o) also shows a similar trend in length reduction, but better preserves accuracy.
Our joint FS-GPT4o-BoN method achieves the greatest reduction across all models, maintaining relative accuracy across different model sizes, especially in the largest 8B model.



\paragraph{Sample study}
\autoref{tab:samples} presents qualitative examples of reasoning paths generated by the model before and after fine-tuning with our method. 
The original reasoning exhibits verbosity, containing redundant processes such as question confirmation and repeated instructions. 
In contrast, the reasoning generated by our method includes only the necessary steps, significantly reducing the number of tokens while still arriving at the correct answer. 
% These examples demonstrate the effectiveness of our method in reducing token count. 
More examples are provided in the \autoref{appx_sample_studies}.

\begin{figure}[t]
    \centering
    \includegraphics[width=\columnwidth]{figures/both_length_scatter.pdf} % Adjust width as needed
    \caption{\textbf{Fine-tuning effectively transfers the length reduction to the model.} Correlation between the relative length of train data and test output averaged over GSM8K and MATH across 4 models. Training length includes only correct solutions. Solid points represent test lengths including all generated outputs, while lighter points indicate test lengths of correct solutions only.}
    \label{fig:ft_length_scatter} % Label for referencing in text
\end{figure}

\paragraph{Length reduction is transferred through fine-tuning.}
As shown in \autoref{fig:ft_length_scatter}, fine-tuning with shorter rationales results in shorter model outputs, showing a strong correlation between test and training lengths.
% Test generation lengths (solid datapoints) are generally longer than training data lengths due to the possibility of lengthy incorrect answers during testing.
% However, when comparing with test generation lengths that are correct (lighter datapoints), they align more closely with training data lengths.
We note that the length of test outputs (incorrect and correct) are longer than the length of training samples (only correct) on average.
This is mainly because incorrect paths are generally longer than correct ones.
We find a closer correspondence between train length and test length of correct samples only, indicated by the lighter datapoints.
This discrepancy suggests the need to terminate incorrect paths early to minimize redundant inference overhead.
We consider this for future work.



\begin{figure*}

\centering

 \centering
\begin{subfigure}[b]{0.49\textwidth}
  \centering
  \includegraphics[width=\linewidth]{Figures/pdfs/triviaqa_child_bar_chart.pdf}
  \caption{TriviaQA, Child}
 \end{subfigure}%
 \hfill
 \begin{subfigure}[b]{0.49\textwidth}
  \centering
\includegraphics[width=\linewidth]{Figures/pdfs/naturalqa_alice_bar_chart.pdf}
\caption{Natural Questions, Alice-Bob}
\end{subfigure}\\

\caption{Failure of Certainty-Based Mitigation Methods: Percentage of unmitigated hallucinations (Y-axis) across six models (X-axis) for three mitigation methods: Sampling, Predictive Entropy, and Probability. Lower values indicate higher success in mitigation. 
All methods fail to address a significant portion of \chk, likely due to their dependence on uncertainty scores.}
\label{mitigation_fig}
\end{figure*}

\section{Hallucination Mitigation and \chk}\label{sec:mitigation_methods}

Having established the existence of \chk, we hypothesize that they may impact the performance of mitigation methods. 
To test this hypothesis, we experiment with hallucination mitigation while focusing on \chk.

\subsection{Mitigation Based on Certainty Measures}


Recent research has proposed leveraging certainty measures to mitigate hallucinations, primarily by abstaining from generating outputs when the model is uncertain about its predictions \cite{cole-etal-2023-selectively,tjandra2024fine}. 
These approaches often rely on probabilities and entropy-based metrics \cite{tjandra2024fine} or combine them with self-evaluation techniques \cite{tomani2024uncertainty}. Other techniques employ alternative information-theoretic measures \cite{yadkori2024believe, yadkori2024mitigating} or involve cross-verification using multiple language models to assess uncertainty \cite{feng2024don}.


The effectiveness of these measures is often evaluated using the area under the receiver operator characteristic (AUROC), which measures how well uncertainty ranks correct versus incorrect responses \cite{kuhn2023semantic}. However, this focus on ranking accuracy overlooks cases where hallucinations occur with high certainty or when correct answers are generated with low certainty. Therefore, this oversight may limit their ability to mitigate hallucinations comprehensively.




To evaluate whether \chk samples are being overlooked, we test the usefulness of uncertainty-based mitigation methods in our setting. Specifically, we use the approaches of \citet{tomani2024uncertainty} and \citet{cole-etal-2023-selectively}, incorporating three uncertainty measures they used: negative log-likelihood, sampling, and predictive entropy. 
We introduce the probability measure as a simplified alternative to negative log-likelihood. Below, we briefly describe each measure, See Appendix \ref{appendix:Mitigation Metrics} for details.

\textbf{Probability}  is a simplified version of negative log-likelihood that considers only the probability of the first token. \textbf{Sampling} assess the diversity of the model's generated outputs, under the assumption that greater diversity reflects lower certainty. Lastly, \textbf{predictive entropy} estimates uncertainty by evaluating the average unpredictability of the model’s outputs.
Examples are mitigated only if their uncertainty is below a defined threshold, as described in Section \ref{subsec:Certainty Threshold}.



\subsection{Certainty-based Mitigation Methods Fail on \chk}


The results in Figure \ref{mitigation_fig} show the percentage of \textbf{unmitigated} hallucination examples for each model. \emph{A lower value is better}, as it indicates fewer hallucinations remain unmitigated.


Our results in Figure \ref{mitigation_fig} show that a significant proportion of hallucinations exceed the threshold across all methods and models, remaining unmitigated. Among the methods, sampling performs the worst, while probability and predictive entropy consistently show a lower percentage of unmitigated hallucinations. This suggests sampling is less effective in avoiding mislabeling hallucinations among uncertainty-based methods.
However, even probability and predictive entropy fail to mitigate a notable percentage of hallucinations. Lowering the threshold could improve mitigation, but this would lead to an increased mislabeling of low-certainty correct answers, as shown in Section \ref{sec:main_results}.

These findings show that certainty-based mitigation fails to address high-certainty hallucinations. While uncertainty-based methods are expected to struggle with high-certainty hallucinations, our results here and in Section \ref{sec:main_results} indicate they consistently miss a distinct type, regardless of the uncertainty metric or threshold strictness. This failure suggests a fundamental limitation in assuming an ideal uncertainty metric can fully capture hallucinations.







\section{Conclusion}
\label{sec:conclusion}

% we propose reducing the refresh latency for RowHammer-preventive refresh  to reduce the performance overheads of existing RowHammer defenses. To this end, 
\om{8}{We} present the first rigorous experimental study on the \om{1}{effects of reduced refresh latency on} RowHammer vulnerability \om{5}{of modern DRAM chips}. \yct{8}{\om{8}{Our} analysis of \param{\nCHIPS{}} real DDR4 DRAM chips from three major manufacturers \yct{7}{demonstrate that charge restoration latency of \om{8}{RowHammer-}preventive refreshes can be significantly reduced}
% observe that potential victim rows can be preventively refreshed at significantly lower 
% latency 
at the expense of requiring slightly more preventive refreshes.}
% To leverage this observation, we \yct{20}{reduce the \yct{7}{preventive} refresh latency of existing RowHammer mitigation mechanisms \om{1}{by developing a new mechanism, \emph{\Xlong{} (\X{})}.}}
% , a memory controller-based low-cost mechanism that \agy{1}{reduces} the refresh latency of existing RowHammer solutions. 
% Our results show that by reducing the execution time spent on preventive refreshes, \X{} significantly improves system performance and energy efficiency of five state-off-the-art RowHammer mitigation mechanisms with small additional area overhead. Thus, \X{} enables new system design points in RowHammer mitigation: i)~performance and energy, and ii)~area overhead trade-off space.
\yct{7}{We propose a new mechanism, \emph{\Xlong{}} (\emph{\X{}}), which \om{8}{robustly} reduces the preventive refresh latency in existing RowHammer mitigation mechanisms. Our evaluation demonstrates that \X{} significantly enhances system performance and energy efficiency \om{8}{when used with} five state-of-the-art RowHammer mitigation mechanisms while introducing small additional area overhead. By doing so, \X{} enables new trade-offs in RowHammer mitigation.}
\yct{7}{We hope and expect that the understanding we develop via our rigorous experimental characterization and the resulting \X{} mechanism will inspire DRAM manufacturers and system designers to efficiently and scalably enable robust \om{8}{and efficient} operation as DRAM technology node scaling exacerbates read disturbance.}
% \ieycomment{7}{effectively? do other works use "scalably"? If so, then OK.}

\section*{Acknowledgments} {
We thank the anonymous reviewers of \om{8}{HPCA 2025 (both main submission and artifact evaluation), MICRO 2024, and ISCA 2024} for the encouraging feedback.
We thank the SAFARI Research Group members for valuable feedback and the stimulating scientific and intellectual environment.
We acknowledge the generous gift funding provided by our industrial partners (especially Google, Huawei, Intel, Microsoft, VMware), which has been instrumental in enabling the research we have been conducting on read disturbance in DRAM since 2011~\cite{kim2023flipping}.
This work was also in part supported by the Google Security and Privacy Research Award, the Microsoft Swiss Joint Research Center, \yct{3}{and the ETH Future Computing Laboratory (EFCL)}\yctcomment{5}{Oguz Hoca requested to include EFCL}.
}
\yctcomment{8}{Double-checked all references}


\bibliography{anthology,custom}


\appendix

\section{Dataset Creation}
\label{sec:appendix-Dataset creation}

To create the dataset, we first split the examples into knowledge-based examples, following a method similar to \citet{simhi2024distinguishing}. Specifically, we performed one greedy generation and five generations with a temperature of $0.5$. We used a 3-shot in-context learning scenario, generating a maximum of 5 tokens, and considered a generation correct only if it exactly matched the factually correct answer.

We also adopted the basic dataset curation process described in \citet{simhi2024distinguishing}, but with two key modifications: we started with 70K examples instead of 30K, and we generated 10 tokens instead of 5. Each example in the dataset begins with \texttt{question:} and ends with \texttt{answer:}.

For instruct models, we adjusted the format to align better with their structure. Specifically, we presented the few-shot examples as a user-assistant conversation, where the user asks the questions and the assistant provides the answers. Additionally, we replaced \texttt{answer:} with \texttt{The answer is}, as part of the assistant generation, since this change was observed to improve the performance of instruction models.

To split the knowledge examples into factually correct examples and hallucination-despite knowledge examples, we sampled 20K knowledge-based examples (or fewer if fewer were available). Using the child and Alice-Bob settings, we checked whether the generated text changed and whether the exact match for the correct answer appeared within the 10-token model generations.
\subsection{Additional Refinement}
We observed certain issues, especially with the instruct models, where an exact match was insufficient. For example, the model sometimes failed to generate an answer or produced a correct answer with minor variations, such as synonyms. To address these issues, we curated the \textsc{WACK} examples further, applying a set of simple heuristics that proved effective during manual examination:

\begin{enumerate}
    \item \textbf{Removing negations:} We excluded examples where the generation stated with ``The answer is not.''
    \item \textbf{Synonyms:} Using the NLTK library \citep{loper2002nltk}, we removed examples where a synonym of the correct answer appeared in the generated text.
    \item \textbf{Stem-based similarity:} We excluded examples if the stemmed version of the generation and the factually correct answer shared more than half of their words.
    \item \textbf{Edit distance:} We kept examples where the edit distance between the generated text and the correct answer (in their stemmed versions) was greater than 2, or the answer is a number and \texttt{great}, \texttt{none}, and \texttt{n/a}, which were removed if present in the generated answer.
    \item \textbf{Initial word match:} We removed examples where the generated answer was the first word of the factually correct answer.
    \item \textbf{Special formatting:} For \textsc{Gemma-instruct} model, which we saw that typically generates the final answer enclosed in \texttt{**}, we removed examples where this formatting was absent.
\end{enumerate}
Thus, we removed between 10\% and 45\% of all the hallucination examples. Note that this is a very harsh criterion for removing hallucinations; however, since our aim was to demonstrate that certain hallucinations exist, we preferred to remove any possibility of wrongly classified hallucinations.


\subsection{Dataset Statistics}
In this section, we present the final dataset statistics. The results are shown in Tables \ref{tab:statistic_child} and \ref{tab:statistic_alice}. We can see that the number of \textsc{WACK} hallucinations varies, with Gemma showing low numbers, while most other models exhibit more than 1000 \textsc{WACK} hallucinations.



\begin{table*}[h!]
        \centering
        \begin{tabular}{lcccc}
        \hline
        Model & \# $HK^+$ & \# Factually-correct \\
        \toprule
 Llama & 643/1096 & 17180/10324 \\\midrule
Mistral &691/1732 & 18916/14425  \\\midrule
Gemma &375/797 & 17617/12916\\\midrule
Llama-Instruct & 1075/ 1730 & 14986/9929 \\\midrule
Mistral-Instruct &1338/2767 & 16512/13566 \\\midrule
Gemma-Instruct &1003/1716 & 17887/16421 \\
\bottomrule
\end{tabular}
        \caption{Dataset statistics for TrivaQA/Natural Questions under Child setting.}
        \label{tab:statistic_child}
        \end{table*}


\begin{table*}[h!]
        \centering
        \begin{tabular}{lcccc}
        \hline
        Model & \# $HK^+$ & \# Factually-correct \\
        \toprule
 Llama & 687/1285 & 17026/10088 \\\midrule
Mistral &809/1607 & 18710/14450 \\\midrule
Gemma &446/896 & 17584/12636\\\midrule
Llama-Instruct &1410/2371 & 14996/9771  \\\midrule
Mistral-Instruct &1425/3091 & 16611/14248 \\\midrule
Gemma-Instruct &  1310/2642 & 17678/16194  \\
\bottomrule
\end{tabular}
        \caption{Dataset statistics for TrivaQA/Natural Questions under Alice-Bob setting.}
        \label{tab:statistic_alice}
        \end{table*}


\subsection{Additional Implementation Details}\label{appendix:Implementation Details}
We use the datasets under the Apache License and the models under each one's agreement terms to follow each artifact's terms.
All experiments were run on NVIDIA RTX 6000 Ada (49GB) with 4 CPUs. 
Generating all the datasets and results takes approximately one month on one GPU.

Lastly, We used AI assistants only for simple paraphrasing as part of writing this paper.


\section{Qualitative Evaluation}\label{appendix:Qualitative Evaluation}
In this section, we show qualitative examples of certain hallucinations.
In Table \ref{Generated answers using bad/good shots in the prompt} provides an example from each model using the child setting on the Natural Questions dataset. These are examples of \chk hallucinations, where the model outputs have high probability and low semantic entropy, indicating a high degree of certainty.

\begin{table*}[t]
\small

\centering

  \begin{tabular}  
  {l p{0.2\linewidth}c ccp{0.1\linewidth}}
\toprule 
& & \multicolumn{2}{c}{Response}&\multicolumn{2}{c}{Uncertainty metric} \\ 
\cmidrule(lr){3-4}\cmidrule(lr){5-6}

Model  &Prompt &Original Response & Hallucinated Response & Probability & Semantic Entropy\\\midrule
Gemma&question:   what is the measure of the number of different species present in an area?& biodiversity& species richness&0.31&0.0\\\midrule
Llama&question: who wrote the song it's the climb?&alexander&Miley Cyrus&0.42&$1.11e^-16$\\\midrule
Mistral&question: if there is a random change in the genetics of a small population it is termed?&genetic drift&mutation&0.49& 0.14\\\midrule
Gemma-Instruct&question: who played the mom on lost in space?&June Lockhart&Molly Parker&0.89&0.23\\\midrule
Llama-Instruct&question: what gas is given off by concentrated hydrochloric acid?&hydrogen chloride&hydrogen gas ($H_2$)&0.98&$2.22e^-16$\\\midrule
Mistral-Instruct&question: who published harry potter and the prisoner of azkaban?&Scholastic Inc&J.K. Rowling&0.99&$2.22e^-16$
  \\\bottomrule

  \end{tabular}
 
  \caption{\chk generated answers and uncertainty measures were obtained using greedy decoding on the Natural Questions dataset in the child setting. In each of these examples, the model generates a hallucination despite having the necessary knowledge. The probability of the generated answer is high, and the semantic entropy is low, indicating that these examples exhibit high certainty.}
   \label{Generated answers using bad/good shots in the prompt}
\end{table*}


\section{Certainty Methods Additional Specifics}\label{appendix:Certainty Methods Additional Specifics}
In this section, we elaborate on specifics in the calculations of the different methods.
\subsection{Probability and Probability Difference}
To ensure that the probability we consider corresponds to the probability of the actual answer and not a preceding token, we employed the following heuristic: skipping over any of the following tokens:  
\texttt{"<|assistant|>", "<|user|>", "<|begin\_of\_text|>", "<|end\_of\_text|>", "<|eot\_id|>", "<|start|>",  
"<|end|>", "<|sep|>", "<|sep\_id|>", "assistant", "user", "\textbackslash n", "answer", "The", "Answer", "\"", "'", " answer", "is", "it", "it's", ":", " ", " is", " correct", "correct", "*", "**", " **"}.

This heuristic proved sufficient during a manual investigation.


\subsection{Sampling Based Methods}
For the Semantic Entropy, Sampling, and Predictive Entropy methods, it is necessary to consider the temperature and define a stopping condition for each generation.


We stopped the generation if one of the following sequences was produced: '\textbackslash n\textbackslash n\textbackslash n\textbackslash n', '\textbackslash n\textbackslash n\textbackslash', "\textbackslash n\textbackslash n, 'Question:', 'Context:', ".\textbackslash n", ". ",  'question:', "Alice", "Bob", "(", "Explanation", "\textbackslash n question:", "What", "\textbackslash n answer". These sequences often indicate the generation of new text that is not relevant to answering the question.

We used a temperature of $1$ for 10 generations and an additional generation with a low temperature of $0.1$, following the approach in the code of \citet{kuhn2023semantic} in repository \url{https://github.com/jlko/semantic_uncertainty}, and using DeBERTa \citep{he2020deberta} as the entailment model for the clustering stage based on meaning similarity. Additional results with a temperature of $0.5$ instead of $1$ are presented in Appendix~\ref{appendix:Semantic Entropy results Different Temperature}. Note that we used 10 tokens to generate as the maximum to be consistent with the knowledge and dataset creation steps.



\subsection{Mitigation Metrics} \label{appendix:Mitigation Metrics}
In Section \ref{sec:mitigation_methods} we evaluate the mitigation abilities of probability, sampling and predictive entropy. In this section we detail each metric.


\paragraph{Probability.}
Probability is a simplified version of Negative Log-Likelihood that considers only the log probability of the first token. We observed that subsequent tokens often exhibit high certainty regardless of correctness, making the first token a more reliable indicator of uncertainty and improving performance.

\paragraph{Sampling.} Sampling-based methods assess the diversity of the model's generated outputs, under the assumption that greater diversity reflects lower certainty. Following \citet{cole-etal-2023-selectively}, we define diversity as the proportion of unique outputs in $S$ generated samples. The uncertainty score is calculated as $1-|U|/|S|$, where $U$ is the set of unique generations, and $|U|/|S|$ represents the ratio of unique outputs to the total number of generations. 

\paragraph{Predictive Entropy.} Predictive Entropy estimates uncertainty by evaluating the average unpredictability of the model’s outputs.
We approximate predictive entropy following \citet{kuhn2023semantic} and \citet{tomani2024uncertainty} by estimating the uncertainty of the model based on its generations for a given prompt $x$. Using $L$ generated samples, the predictive entropy is calculated as: 
\begin{equation}
    PE \approx -1/L\sum_{i=1}^{L}logp(l_i|x)
\end{equation}

 Here, $p(l_i|x)$ represents the likelihood of the $i$-th generation given the prompt $x$. Predictive entropy captures the average uncertainty across the generated outputs.

We investigate whether these uncertainty measures can reliably detect and mitigate \chk hallucinations.


\section{Certain HK+ Exist -- Additional Results}\label{appendix-Certain HK+ Exist Additional Results}

In this section, we present results similar to those in Section \ref{sec:main_results}, focusing on the Gemma and Llama models. See Figures \ref{fig:gemma_certainty} and \ref{fig:Llama_certainty}. These results correlate with those in the main paper on the Mistral model and demonstrate that \chk hallucinations exist. Furthermore, they show that these hallucinations occur across different methods and that instruct-tuned models exhibit poorer calibration between certainty and hallucinations.




\begin{figure*}
\centering

 \centering

 \centering
\begin{subfigure}[b]{0.24\textwidth}
  \centering
  \includegraphics[width=\linewidth]{Figures/pdfs/meta-llama_Llama-3.1-8B_triviaqa_child_prob.pdf}
 \end{subfigure}%
 \hfill
 \begin{subfigure}[b]{0.24\textwidth}
  \centering
\includegraphics[width=\linewidth]{Figures/pdfs/meta-llama_Llama-3.1-8B_naturalqa_alice_prob.pdf}
\end{subfigure}
\hfill
\begin{subfigure}[b]{0.24\textwidth}
  \centering
  \includegraphics[width=\linewidth]{Figures/pdfs/meta-llama_Llama-3.1-8B-Instruct_triviaqa_child_prob.pdf}
 \end{subfigure}
 \hfill
 \begin{subfigure}[b]{0.24\textwidth}
  \centering
  \includegraphics[width=\linewidth]{Figures/pdfs/meta-llama_Llama-3.1-8B-Instruct_naturalqa_alice_prob.pdf}
 \end{subfigure}\\

 \centering
\begin{subfigure}[b]{0.24\textwidth}
  \centering
  \includegraphics[width=\linewidth]{Figures/pdfs/meta-llama_Llama-3.1-8B_triviaqa_child_prob_diff.pdf}
 \end{subfigure}%
 \hfill
 \centering
\begin{subfigure}[b]{0.24\textwidth}
  \centering
  \includegraphics[width=\linewidth]{Figures/pdfs/meta-llama_Llama-3.1-8B_naturalqa_alice_prob_diff.pdf}
 \end{subfigure}%
 \hfill
\begin{subfigure}[b]{0.24\textwidth}
  \centering
  \includegraphics[width=\linewidth]{Figures/pdfs/meta-llama_Llama-3.1-8B-Instruct_triviaqa_child_prob_diff.pdf}
 \end{subfigure}
 \hfill
 \begin{subfigure}[b]{0.24\textwidth}
  \centering
  \includegraphics[width=\linewidth]{Figures/pdfs/meta-llama_Llama-3.1-8B-Instruct_naturalqa_alice_prob_diff.pdf}
 \end{subfigure}\\

  \centering

 \centering
\begin{subfigure}[b]{0.24\textwidth}
  \centering
  \includegraphics[width=\linewidth]{Figures/pdfs/meta-llama_Llama-3.1-8B_triviaqa_child_semantic_entropy.pdf}
 \caption{Llama, Child}
 \end{subfigure}%
 \hfill
  \centering
\begin{subfigure}[b]{0.24\textwidth}
  \centering
  \includegraphics[width=\linewidth]{Figures/pdfs/meta-llama_Llama-3.1-8B_naturalqa_alice_semantic_entropy.pdf}
  \caption{Llama, Alice-Bob}
 \end{subfigure}
 \hfill
\begin{subfigure}[b]{0.24\textwidth}
  \centering
  \includegraphics[width=\linewidth]{Figures/pdfs/meta-llama_Llama-3.1-8B-Instruct_triviaqa_child_semantic_entropy.pdf}
  \caption{Llama-Instruct, Child}
  \end{subfigure}
  \hfill
  \begin{subfigure}[b]{0.24\textwidth}
  \centering
  \includegraphics[width=\linewidth]{Figures/pdfs/meta-llama_Llama-3.1-8B-Instruct_naturalqa_alice_semantic_entropy.pdf}
  \caption{Llama-Instruct, Alice-Bob}
 \end{subfigure}\\

 \caption{
Detection of WACK Hallucinations on \emph{Llama}: Cumulative percentage of samples (Y-axis) across certainty levels (X-axis) for hallucinations (red) and correct answers (blue) in cases where the model possesses the correct knowledge. Subfigures are organized by models (columns) and certainty measures (rows), with the black dashed line marking the optimal certainty threshold for separating hallucinations from correct answers. The red line shows that a significant proportion of hallucinations persists even at high certainty levels, illustrating the phenomenon of \chk hallucinations, while the blue line shows that many correct answers are also made with varying levels of certainty. The results for Child are on TriviaQA and For Alice-Bob on Natural Questions.}
 \label{fig:Llama_certainty}
\end{figure*}



 
\begin{figure*}
\centering
 \centering

 \centering
\begin{subfigure}[b]{0.24\textwidth}
  \centering
  \includegraphics[width=\linewidth]{Figures/pdfs/google_gemma-2-9b_triviaqa_child_prob.pdf}
 \end{subfigure}%
 \hfill
 \begin{subfigure}[b]{0.24\textwidth}
  \centering
\includegraphics[width=\linewidth]{Figures/pdfs/google_gemma-2-9b_naturalqa_alice_prob.pdf}
\end{subfigure}
\hfill
\begin{subfigure}[b]{0.24\textwidth}
  \centering
  \includegraphics[width=\linewidth]{Figures/pdfs/google_gemma-2-9b-it_triviaqa_child_prob.pdf}
 \end{subfigure}
 \hfill
 \begin{subfigure}[b]{0.24\textwidth}
  \centering
  \includegraphics[width=\linewidth]{Figures/pdfs/google_gemma-2-9b-it_naturalqa_alice_prob.pdf}
 \end{subfigure}\\

 \centering
\begin{subfigure}[b]{0.24\textwidth}
  \centering
  \includegraphics[width=\linewidth]{Figures/pdfs/google_gemma-2-9b_triviaqa_child_prob_diff.pdf}
 \end{subfigure}%
 \hfill
 \centering
\begin{subfigure}[b]{0.24\textwidth}
  \centering
  \includegraphics[width=\linewidth]{Figures/pdfs/google_gemma-2-9b_naturalqa_alice_prob_diff.pdf}
 \end{subfigure}%
 \hfill
\begin{subfigure}[b]{0.24\textwidth}
  \centering
  \includegraphics[width=\linewidth]{Figures/pdfs/google_gemma-2-9b-it_triviaqa_child_prob_diff.pdf}
 \end{subfigure}
 \hfill
 \begin{subfigure}[b]{0.24\textwidth}
  \centering
  \includegraphics[width=\linewidth]{Figures/pdfs/google_gemma-2-9b-it_naturalqa_alice_prob_diff.pdf}
 \end{subfigure}\\

  \centering

 \centering
\begin{subfigure}[b]{0.24\textwidth}
  \centering
  \includegraphics[width=\linewidth]{Figures/pdfs/google_gemma-2-9b_triviaqa_child_semantic_entropy.pdf}
 \caption{Gemma, Child}
 \end{subfigure}%
 \hfill
  \centering
\begin{subfigure}[b]{0.24\textwidth}
  \centering
  \includegraphics[width=\linewidth]{Figures/pdfs/google_gemma-2-9b_naturalqa_alice_semantic_entropy.pdf}
  \caption{Gemma, Alice-Bob}
 \end{subfigure}
 \hfill
\begin{subfigure}[b]{0.24\textwidth}
  \centering
  \includegraphics[width=\linewidth]{Figures/pdfs/google_gemma-2-9b-it_triviaqa_child_semantic_entropy.pdf}
  \caption{Gemma-Instruct, Child}
  \end{subfigure}
  \hfill
  \begin{subfigure}[b]{0.24\textwidth}
  \centering
  \includegraphics[width=\linewidth]{Figures/pdfs/google_gemma-2-9b-it_naturalqa_alice_semantic_entropy.pdf}
  \caption{Gemma-Instruct, Alice-Bob}
 \end{subfigure}\\

 \caption{
Detection of WACK Hallucinations on \emph{Gemma}: Cumulative percentage of samples (Y-axis) across certainty levels (X-axis) for hallucinations (red) and correct answers (blue) in cases where the model possesses the correct knowledge. Subfigures are organized by models (columns) and certainty measures (rows), with the black dashed line marking the optimal certainty threshold for separating hallucinations from correct answers. The red line shows that a significant proportion of hallucinations persists even at high certainty levels, illustrating the phenomenon of \chk hallucinations, while the blue line shows that many correct answers are also made with varying levels of certainty. The results for Child are on TriviaQA and For Alice-Bob on Natural Questions.}
 \label{fig:gemma_certainty}
\end{figure*}

\section{Semantic Entropy Results -- Different Temperature}\label{appendix:Semantic Entropy results Different Temperature}

In Section~\ref{sec:Certainty Hallucinations can not be Explain as Noise}, we used Semantic Entropy with a temperature of 1 for generating the samples. To demonstrate that the certainty results are not specific to this temperature, we present in Figure~\ref{fig:temp_similarity_certainty} the Semantic Entropy results on the Mistral model. In the upper subfigure, we show the results using a temperature of 1, and in the lower row, we show the results using a temperature of 0.5. We observe that under a temperature of 0.5, there are even more certain hallucinations, further proving that the certainty hallucination phenomenon is not specific to a temperature of 1.


\begin{figure*}
\centering

 \centering
 \centering
\begin{subfigure}[b]{0.24\textwidth}
  \centering
  \includegraphics[width=\linewidth]{Figures/pdfs/mistralai_Mistral-7B-v0.3_triviaqa_child_semantic_entropy.pdf}
 \end{subfigure}%
 \hfill
  \centering
\begin{subfigure}[b]{0.24\textwidth}
  \centering
  \includegraphics[width=\linewidth]{Figures/pdfs/mistralai_Mistral-7B-v0.3_naturalqa_alice_semantic_entropy.pdf}
 \end{subfigure}
 \hfill
% \end{subfigure}
\begin{subfigure}[b]{0.24\textwidth}
  \centering
  \includegraphics[width=\linewidth]{Figures/pdfs/mistralai_Mistral-7B-Instruct-v0.3_triviaqa_child_semantic_entropy.pdf}
  \end{subfigure}
  \hfill
  \begin{subfigure}[b]{0.24\textwidth}
  \centering
  \includegraphics[width=\linewidth]{Figures/pdfs/mistralai_Mistral-7B-Instruct-v0.3_naturalqa_alice_semantic_entropy.pdf}
 \end{subfigure}\\
 \centering
\begin{subfigure}[b]{0.24\textwidth}
  \centering
  \includegraphics[width=\linewidth]{Figures/pdfs/mistralai_Mistral-7B-v0.3_triviaqa_child_semantic_entropy_temp_0.5.pdf}
 \caption{Mistral, Child}
 \end{subfigure}%
 \hfill
  \centering
\begin{subfigure}[b]{0.24\textwidth}
  \centering
  \includegraphics[width=\linewidth]{Figures/pdfs/mistralai_Mistral-7B-v0.3_naturalqa_alice_semantic_entropy_temp_0.5.pdf}
  \caption{Mistral, Alice-Bob}
 \end{subfigure}
 \hfill
% \end{subfigure}
\begin{subfigure}[b]{0.24\textwidth}
  \centering
  \includegraphics[width=\linewidth]{Figures/pdfs/mistralai_Mistral-7B-Instruct-v0.3_triviaqa_child_semantic_entropy_temp_0.5.pdf}
  \caption{Mistral-Instruct, Child}
  \end{subfigure}
  \hfill
  \begin{subfigure}[b]{0.24\textwidth}
  \centering
  \includegraphics[width=\linewidth]{Figures/pdfs/mistralai_Mistral-7B-Instruct-v0.3_naturalqa_alice_semantic_entropy_temp_0.5.pdf}
  \caption{Mistral-Instruct, Alice-Bob}
 \end{subfigure}\\

 \caption{
Detection of WACK Hallucinations: The upper figures show the Semantic Entropy method with a temperature of 1, while the lower figures use a temperature of 0.5. We show the cumulative percentage of samples (Y-axis) across certainty levels (X-axis) for hallucinations (red) and correct answers (blue) in cases where the model possesses the correct knowledge. Subfigures are organized by models (columns) and certainty measures (rows), with the black dashed line marking the optimal certainty threshold for separating hallucinations from correct answers. The red line shows that a significant proportion of hallucinations persists even at high certainty levels, illustrating the phenomenon of \chk hallucinations, while the blue line shows that many correct answers are also made with varying levels of certainty. 
}
 \label{fig:temp_similarity_certainty}
\end{figure*}



\section{Prompt Specificity}\label{appendix:Prompt Specificity}
In this section, we aim to demonstrate that the results are not restricted to a single variation of the setting. To investigate this, we created a new variation of the child setting. The new prompt is:  
\emph{`I'm working on a major school project, and there's a lot of information I need to understand. Some of it is a bit challenging, and I'm unsure where to begin. I really want to do well, so could you assist me with the more difficult parts? It would mean so much to me!'}

We present the certainty detection results using Mistral on the Natural Questions dataset in Figure~\ref{fig:child2_detection}. In the top figure, we show the results using the original child prompt, while the bottom figure displays the results with the new child prompt. Both settings reveal certain \textsc{WACK} hallucinations and exhibit overall similarity. This indicates that our results are robust to variations in the prompt.

\begin{figure*}
\centering
% \begin{figure*}
 \centering
 \centering
\begin{subfigure}[b]{0.3\textwidth}
  \centering
  \includegraphics[width=\linewidth]{Figures/pdfs/mistralai_Mistral-7B-v0.3_naturalqa_child_prob.pdf}
 \end{subfigure}%
 \hfill
  \centering
\begin{subfigure}[b]{0.3\textwidth}
  \centering
  \includegraphics[width=\linewidth]{Figures/pdfs/mistralai_Mistral-7B-v0.3_naturalqa_child_prob_diff.pdf}
 \end{subfigure}
 \hfill
% \end{subfigure}
\begin{subfigure}[b]{0.3\textwidth}
  \centering
  \includegraphics[width=\linewidth]{Figures/pdfs/mistralai_Mistral-7B-v0.3_naturalqa_child_semantic_entropy.pdf}
  \end{subfigure}
\\
 \centering
\begin{subfigure}[b]{0.3\textwidth}
  \centering
  \includegraphics[width=\linewidth]{Figures/pdfs/mistralai_Mistral-7B-v0.3_naturalqa_child2_prob.pdf}
  \caption{Probability}
 \end{subfigure}%
 \hfill
  \centering
\begin{subfigure}[b]{0.3\textwidth}
  \centering
  \includegraphics[width=\linewidth]{Figures/pdfs/mistralai_Mistral-7B-v0.3_naturalqa_child2_prob_diff.pdf}
  \caption{Probability Difference}
 \end{subfigure}
 \hfill
% \end{subfigure}
\begin{subfigure}[b]{0.3\textwidth}
  \centering
  \includegraphics[width=\linewidth]{Figures/pdfs/mistralai_Mistral-7B-v0.3_naturalqa_child2_semantic_entropy.pdf}
  \caption{Semantic Entropy}
  \end{subfigure}\\

 \caption{
Detection of WACK Hallucinations: The upper figures show the results on Child setting while the lower are the results on Child variation setting. We show the cumulative percentage of samples (Y-axis) across certainty levels (X-axis) for hallucinations (red) and correct answers (blue) in cases where the model possesses the correct knowledge. Subfigures are organized by models (columns) and certainty measures (rows), with the black dashed line marking the optimal certainty threshold for separating hallucinations from correct answers. We can see that the two child variations have similar graphs.
}
 \label{fig:child2_detection}
\end{figure*}

\section{\chk Uniqueness -- Additional Results}\label{sec:appendix-Jaccard Similarity Additional Results}
In this section, we extend the results presented in Section \ref{sec:Certainty Hallucinations can not be Explain as Noise} by demonstrating that, even under shared hallucination examples, permutation tests confirm that certain hallucinations are not random.

Tables \ref{tab:jaccard_prob_shared} and \ref{tab:jaccard_semantic_shared} display these results. Notably, all values are higher than those reported in the main paper, as the examples are now sampled from a subset of shared hallucination examples between the Child and Alice-Bob settings. However, even under the shared examples, the Jaccard similarity for certain cases remains significantly higher than the similarity observed under the permutation test. This further supports the conclusion that certain hallucinations are not mere noise but a distinct property of the models.



We compare the similarity under high certainty to that under the lowest certainty. Specifically, we examine the same subset of examples as the high-certainty subgroup but focus on those with the lowest probabilities. The results are shown in Table \ref{tab:jaccard_prob_low_certainty}.

We observe that the Jaccard similarity for the lowest-certainty subgroup is higher than the values obtained from the permutation test, indicating a general similarity across all certainty levels between the two settings. However, the high-certainty subgroup still exhibits significantly higher similarity scores, suggesting that this subgroup is more aligned than would be expected by chance.


\paragraph{Answer Token Length.}
To investigate what distinguishes \chk samples, we analyzed the length of the first token in generated answers. Specifically, we examined whether the average length of the first token differed between \chk samples and low-certainty hallucinations. Across models, datasets, prompt settings, and metrics, \chk examples consistently exhibited shorter first tokens, with the difference statistically significant according to a t-test. Although the underlying reasons for this pattern remain unclear and warrant future study, this result further highlights the unique characteristics of \chk samples.







\begin{table}[h!]
        \centering
        \begin{tabular}{l|c|cc}
        \hline
        Model&Dataset &Random& Certain\\
        \toprule
  \multirow{2}{*}{Llama}& TriviaQA  & 7.39 &\textbf{47.22}\\ &NQ&9.56 & \textbf{50.72}\\\midrule\multirow{2}{*}{Mistral}& TriviaQA  & 9.99 &\textbf{51.89}\\ &NQ&24.72 & \textbf{72.61}\\\midrule\multirow{2}{*}{Gemma}& TriviaQA  & 7.72 &\textbf{41.67}\\ &NQ&13.54 & \textbf{54.81}\\\midrule\multirow{2}{*}{Llama-Inst}& TriviaQA  & 21.02 &\textbf{36.36}\\ &NQ&22.44 & \textbf{34.42}\\\midrule\multirow{2}{*}{Mistral-Inst}& TriviaQA  & 15.44 &\textbf{57.24}\\ &NQ&22.54 & \textbf{50.06}\\\midrule\multirow{2}{*}{Gemma-Inst}& TriviaQA  & 14.99 &\textbf{53.93}\\ &NQ&16.36 & \textbf{54.47}\\
\bottomrule
\end{tabular}
        \caption{Jaccard Similarity of \chk hallucinations across different prompts under \emph{shared hallucinations}. The \textit{Certain} column shows the overall similarity of \emph{\chk} samples between prompts in the TriviaQA and NaturalQA datasets, using \emph{Probability} as the certainty threshold. Results indicate high similarity, suggesting consistency across settings. All scores are statistically significant (\(p < 0.0001\), permutation test (the Rand column).}
        \label{tab:jaccard_prob_shared}
        \end{table}


\begin{table}[h!]
        \centering
        \begin{tabular}{l|c|cc}
        \hline
        Model &Dataset &Random& Certain \\
        \toprule
\multirow{2}{*}{Llama}& TriviaQA  & 5.56 &\textbf{26.56}\\ &NQ&5.4 & \textbf{16.24}\\\midrule\multirow{2}{*}{Mistral}& TriviaQA  & 7.68 &\textbf{26.26}\\ &NQ&10.96 & \textbf{28.76}\\\midrule\multirow{2}{*}{Gemma}& TriviaQA  & 7.19 &\textbf{25.0}\\ &NQ&7.44 & \textbf{20.72}\\\midrule\multirow{2}{*}{Llama-Inst}& TriviaQA  & 10.7 &\textbf{29.28}\\ &NQ&13.42 & \textbf{31.06}\\\midrule\multirow{2}{*}{Mistral-Inst}& TriviaQA  & 14.21 &\textbf{27.12}\\ &NQ&22.29 & \textbf{41.06}\\\midrule\multirow{2}{*}{Gemma-Inst}& TriviaQA  & 14.51 &\textbf{43.68}\\ &NQ&17.29 & \textbf{42.64}\\
\bottomrule
\end{tabular}
\caption{
Jaccard Similarity of \chk hallucinations across different prompts under \emph{shared hallucinations}. The \textit{Certain} column shows the overall similarity of \emph{\chk} samples between prompts in the TriviaQA and NaturalQA datasets, using \emph{Semantic Entropy} as the certainty threshold. Results indicate high similarity, suggesting consistency across settings. All scores are statistically significant (\(p < 0.0001\), permutation test (the Rand column).}
        \label{tab:jaccard_semantic_shared}
        \end{table}





\begin{table}[t]
        \centering
        \begin{tabular}{l|c|cc}
        \toprule
        Model&Dataset&uncertain& Certain \\
        \toprule
\multirow{2}{*}{Llama} & TriviaQA & 17.91 &\textbf{27.42}\\ &NQ&17.65 &\textbf{21.21}\\\midrule
\multirow{2}{*}{Mistral} & TriviaQA & 22.55 &\textbf{28.21}\\ &NQ&28.93 &\textbf{34.53}\\\midrule\multirow{2}{*}{Gemma} & TriviaQA & 18.89&\textbf{22.99}\\ &NQ&23.43 &\textbf{26.52}\\\midrule\multirow{2}{*}{Llama-Inst}& TriviaQA & 10.42 &\textbf{19.41}\\ &NQ&9.53 &\textbf{18.08}\\\midrule\multirow{2}{*}{Mistral-Inst}& TriviaQA & 14.41 &\textbf{36.48}\\ &NQ&16.02 &\textbf{31.46}\\\midrule\multirow{2}{*}{Gemma- Inst}& TriviaQA & 19.43 &\textbf{35.73}\\ &NQ&18.52 &\textbf{31.72}\\
 

\bottomrule
\end{tabular}
        \caption{Jaccard Similarity of \chk hallucinations across different prompts. The \textit{Certain} column shows the overall similarity of \emph{\chk} samples between prompts in the TriviaQA and NaturalQA datasets and \textit{Low certain} shows the results of the lowest certainty subset, using \emph{Probability} as the certainty threshold. Results indicate high similarity, suggesting consistency across settings.}
        \label{tab:jaccard_prob_low_certainty}
        \end{table}




\section{\chk Persists in Larger Models -- Additional Results}
\label{appendix:chk Persists in Larger Models Additional Results}
In Section \ref{chk Persists in Instruction-Tuning and Larger Models}, we demonstrated that Gemma-2-27B achieves similar or slightly higher \chk detection results on the TriviaQA dataset than Gemma-2-9B, thus showing the existence of this phenomenon on larger models. To further illustrate this phenomenon, Figure \ref{fig:Hallucinations gemma natural} presents comparable results on the Natural Questions dataset.  
These results show a clear correlation with those presented in the main paper.


\begin{figure}
\centering
\begin{subfigure}[b]{0.23\textwidth}
  \centering
  \includegraphics[width=\linewidth]{Figures/pdfs/27vs9_gemma_naturalqa_alice_prob.pdf}
 \end{subfigure}%
 \hfill
  \centering
\begin{subfigure}[b]{0.23\textwidth}
  \centering
  \includegraphics[width=\linewidth]{Figures/pdfs/27vs9_gemma_naturalqa_alice_semantic_entropy.pdf}
 \end{subfigure}\\
 \hfill
\centering
\begin{subfigure}[b]{0.23\textwidth}
  \centering
  \includegraphics[width=\linewidth]{Figures/pdfs/27vs9_gemma_naturalqa_child_prob.pdf}
  \caption{Gemma, Child}
  \end{subfigure}
  \hfill
  \centering
  \begin{subfigure}[b]{0.23\textwidth}
  \centering
  \includegraphics[width=\linewidth]{Figures/pdfs/27vs9_gemma_naturalqa_alice_prob.pdf}
  \caption{Gemma, Alice-Bob}
 \end{subfigure}\\

 \caption{
Detection of \chk Hallucinations: Comparing Gemma-27B to Gemma-9B \chk hallucinations on Natural Questions. The results indicate that \chk hallucinations is similar and slightly higher for the larger model, Gemma-27B.}
 \label{fig:Hallucinations gemma natural}
\end{figure}





\end{document}
