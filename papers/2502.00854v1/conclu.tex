\section{Conclusion}
\label{sec:conclusion}

This paper introduces EGORSE, a high-dimensional efficient global optimization using both random and supervised embeddings to tackle expensive to compute black-box optimization problems.
EGORSE shows a high potential to tackle the two main drawbacks of most existing HDBO methods with a very competitive computational effort and with an accurate definition of the new bounds related to the reduced optimization problem.
A parametric study on the hyper-parameters of EGORSE has shown that combining both PLS and the random Gaussian reduction methods provides the best results. On very large-scale optimization problem, i.e. d=600, EGORSE gives a lower minimal value than HESBO in an equivalent computational time. 
Thus EGORSE outperforms by far all the state-of-the-art HDBO solvers.
Nonetheless, EGORSE is still unable to find the global optimum in many cases as the effective dimension of the problem is much larger than the effective dimension used in the algorithm. {\color{black}Overall, EGORSE shows very good performance compared to state-of-the-art high-dimensional BO solvers.}
%{\color{black} LIMITATION A AJOUTER}
%The effective directions employed in the optimization process seem not sufficient to  full explore the design space.. 
%{\color{black} Indeed, EGORSE is using a global reduction method. However, an optimization problem is about a point which is local. In this way, it is possible to miss the optimum by searching in global direction in which the optimum is not included.}

%Many perspectives could be envisaged to follow this work:

{\color{black} We believe that the results from EGORSE encourage further study and enhancements to tackle realistic aerospace engineering optimization problems. In fact, all realistic aerospace optimization problems involve constraints~\cite{priemEfficientApplicationBayesian2020a, raponi2020high,prestch2024}. Extending EGORSE to solve high-dimensional constrained black-box optimization problems is necessary to address issues where constraints are typically related to the amount of oil, the range of an aircraft, or the size of its wings. Another important perspective is related to the asymptotic convergence properties of EGORSE. Convergence guarantees can be obtained, for instance, by incorporating trust-region techniques~\cite{YDIOUANE_2023}. This could help EGORSE focus on promising regions, improve its convergence rate to local optima, and use a stopping criterion based on a stationarity measure rather than a fixed budget. 

One current limitation of EGORSE is related to the effective dimension (i.e., the dimension of the reduced space), which needs to be specified by the user. However, in real-world applications, users do not necessarily know the effective dimensions. In this context, an automatic estimation of the number of effective dimensions, inspired by \cite{SciTech_cat}, would be of high interest toward developing an effective dimension-free methodology.}


% Youssef: J ai reformulé la liste en bas suivant le commentaire du reviewer 2.

%\begin{itemize}
%    \item an automatic choice of the number of effective dimensions, inspired by \cite{SciTech_cat}. {\color{black} This future work could increase the convergence speed by using exactly the best number of effective dimensions. Indeed, in real world problem, engineers do not know the effective dimensions. An automatic choice of this number could ease a lot the use of EGORSE},
%    \item an extension of EGORSE for constrained expensive-to-compute problems. {\color{black} To tackle aerospace engineering problems, the optimization algorithm must handle constraints. For instance, one can think about the amount of oil, the range of the aircraft, the size of the wings, etc.}
%    \item an extension of EGORSE to improve its convergence, for instance with trust region strategies \cite{YDIOUANE_2023}. {\color{black} This could help the algorithm to focus on promising region. In this way, EGORSE should be able to converge faster to a local optimum. Moreover, this could also help to investigate several promising designs with different properties.}
%\end{itemize}