%\documentclass[conf]{new-aiaa}
\documentclass[journal]{new-aiaa} %for journal papers
\usepackage[utf8]{inputenc}

\usepackage{algorithm}
\usepackage{algorithmic}

\usepackage{graphicx}
\usepackage{amsmath}
\usepackage{mathtools}
\usepackage[version=4]{mhchem}
\usepackage{siunitx}
\usepackage{longtable,tabularx}
\setlength\LTleft{0pt} 
\usepackage{bm}

\usepackage{subfig}
\usepackage{multicol}
\usepackage{multirow}
 
%%% Commandes perso 
\newcommand{\x}{\bm{x}}
\newcommand{\dthetai}[1]{\ensuremath{\frac{\partial #1}{\partial {\theta_i^{(l)}}}}}
\newcommand{\dthetaj}[1]{\ensuremath{\frac{\partial #1}{\partial {\theta_j^{(l)}}}}}
\newcommand{\dthetaij}[1]{\ensuremath{\frac{\partial^2 #1}{\partial{\theta_j^{(l)}} \partial {\theta_i^{(l)}}}}}
\newcommand{\As}{{\mathcal{A}}}
\newcommand{\A}{\bm{A}}
\DeclareMathOperator{\conf}{conf}
\DeclareMathOperator{\Tr}{Tr}
\DeclareMathOperator{\Err}{Err}
\DeclareMathOperator{\diff}{diff}
\DeclareMathOperator{\vect}{vect}

% algorithm
\newcommand{\algorithmicinput}{\textbf{input}}
\newcommand{\algorithmicoutput}{\textbf{output}}
\newcommand{\algorithmicin}{\textbf{in }}
\newcommand{\INPUT}{\item[\algorithmicinput]}
\newcommand{\OUTPUT}{\item[\algorithmicoutput]}
\newcommand{\IN}{\algorithmicin}


% \renewcommand{\thefootnote}{\Alph{footnote}}

%%%% Review
\newcommand{\changecolor}{redOrange}
\newcommand{\change}[1]{{\color{\changecolor} {#1}}}
\newcommand{\remove}[1]{{\color{\changecolor} {\sout{#1}}}}
\newcommand{\removeeq}[1]{{\color{\changecolor} {\cancel{#1}}}} 
 
% \title{High-dimensional efficient global optimization {\color{black}of expensive-to-evaluate blackboxes} using both random and supervised embeddings}
\title{High-dimensional {\color{black}Bayesian} optimization using both random and linear embeddings}


\author{R\'emy Priem\footnote{Artificial Intelligence Engineer for Aeronautical Systems, Technical Unit, remy.priem@intradef.gouv.fr}}
\affil{DGA Maîtrise de l'Information, Bruz, France}
\author{Youssef Diouane\footnote{Professor, Mathematics and Industrial Engineering Department, youssef.diouane@polymtl.ca, AIAA MDO TC Member.}}
\affil{GERAD \& Polytechnique Montr\'eal, Montreal, QC, Canada}
\author{Nathalie Bartoli\footnote{Senior researcher, Information Processing and Systems Department, nathalie.bartoli@onera.fr, AIAA MDO TC Member.}, Sylvain Dubreuil\footnote{Researcher, Information Processing and Systems Department, sylvain.dubreuil@onera.fr, AIAA Member.}, Paul Saves\footnote{Post-doctoral fellow, Information Processing and Systems Department, paul.saves@onera.fr, Student AIAA Member.}} 
\affil{DTIS, ONERA, Université de Toulouse, Toulouse, France}
\affil{Fédération ENAC ISAE-SUPAERO ONERA, Université de Toulouse, 31000, Toulouse, France}


%\author{Sylvain Dubreuil\footnote{Researcher, Information Processing and Systems Department, sylvain.dubreuil@onera.fr, AIAA Member.}}
%\affil{ONERA, DTIS, Université de Toulouse, Toulouse, France}
%\author{Paul Saves\footnote{PhD Student, Information Processing and Systems Department, paul.saves@onera.fr, Student AIAA Member.}}
%\affil{ONERA, DTIS, Université de Toulouse, Toulouse, France}
%\affil{ISAE-SUPAERO, Université de Toulouse, Toulouse, 31055 Cedex 4, France}

\begin{document}

 
\maketitle

\begin{abstract}
Bayesian optimization (BO) is one of the most powerful strategies to solve {\color{black}computationally expensive-to-evaluate} blackbox optimization problems.
However, BO methods are conventionally used for optimization problems of small dimension because of the curse of dimensionality.
In this paper, to solve high dimensional optimization problems, we propose to incorporate linear embedding subspaces of small dimension to efficiently perform the optimization. An adaptive learning strategy for these linear embeddings is carried out in conjunction with the optimization.
The resulting BO method, named EGORSE, combines in an adaptive way both random and supervised linear embeddings.
EGORSE has been compared to state-of-the-art algorithms and tested on academic examples with a number of design variables ranging from 10 to 600.
The obtained results show the high potential of EGORSE to solve high-dimensional blackbox optimization problems, both in terms of CPU time and {\color{black}{limited}} number of calls to the expensive blackbox {\color{black}{simulation}}.
\end{abstract}

\section{Nomenclature}

{\renewcommand\arraystretch{1.0}
\noindent\begin{longtable*}{@{}l @{\quad=\quad} l@{}}
$\Omega$ & design space \\
$d$ & dimension of the design space \\
$f$ & objective function \\
$\bm{x}$ & vector of design variables\\
$\mu$ & prior mean function \\
$k$ & covariance kernel\\
$\mathcal{D}^{(l)}$ & design of experiments of $l$ sampled points \\
$y$ & output of the objective function \\
$\mathcal{N}$ & Gaussian distribution \\
$\hat\sigma^{(l)}$ & variance function of a Gaussian process conditioned by $l$ sampled points \\
$\hat\mu^{(l)}$ & mean function of a Gaussian process conditioned by $l$ sampled points \\
$\bm{\theta}^{(l)}$ & hyper-parameters of a Gaussian process conditioned by $l$ sampled points \\
$\alpha^{(l)}$ & acquisition function \\
$y_{min}^{(l)}$ & minimum output of the objective function in a set of $l$ sampled points\\
$f_{\mathcal{A}}$ & objective function reduced in the linear subspace $\mathcal{A}$ \\
$\bm{A}$ & transfer matrix \\
$\mathcal{A}$ & linear subspace \\
$\mathcal{B}$ & hyper-cube in the linear subspace \\
$T$ & number of dimension reduction methods \\
$\mathcal{R}$ & dimension reduction method \\
$\gamma$ & reverse application \\
$\bm{u}$ & vector of design variables in the linear subspace \\
$g$ & constraint in the linear subspace \\
\end{longtable*}}

\section{Introduction}
Implicit Neural Representations (INRs), which fit the target function using only input coordinates, have recently gained significant attention.
%
By leveraging the powerful fitting capability of Multilayer Perceptrons (MLPs), INRs can implicitly represent the target function without requiring their analytical expressions. 
%
The versatility of MLPs allows INRs to be applied in various fields, including inverse graphics~\citep{mildenhall2021nerf, barron2023zip, martin2021nerf}, image super-resolution~\citep{chen2021learning, yuan2022sobolev, gao2023implicit}, 
image generation~\citep{skorokhodov2021adversarial}, and more~\citep{chen2021nerv, strumpler2022implicit, shue20233d}.
%
\begin{figure}
    \includegraphics[width=0.5\textwidth]{Image/Fig2.pdf}
    \caption{As illustrated at the circled blue regions and green regions, it can be observed that even with well-chosen standard deviation/scale, as experimented in \autoref{figure:combined}, the results are still unsatisfactory. However, using our proposed method, the noise is significantly alleviated while further enhancing the high-frequency details.}
    \label{fig:var}
    \vspace{-10pt}
\end{figure}

\begin{figure*}[!ht]
    \centering
    \begin{minipage}[b]{0.25\textwidth}
        \centering
        \includegraphics[width=1.\textwidth]{Image/fig_cropped.pdf} % 替换为你的小图文件
        \label{figure:small_image}
        \vspace{-20pt}
    \end{minipage}%
    \hfill
    \begin{minipage}[b]{0.75\textwidth}
        \centering
        \includegraphics[width=1.\textwidth]{Image/psnr_trends_rff_pe_simplified.pdf} % 替换为你的大图文件
        \vspace{-20pt}
        \label{figure:large_image}
        
    \end{minipage}
    \caption{We test the performance of MLPs with Random Fourier Features (RFF) and MLPs with Positional Encoding (PE) on a 1024-resolution image to better distinguish between high- and low-frequency regions, as demonstrated on the left-hand side of this figure. We find that the performance of MLPs+RFF degrades rapidly with increasing standard deviation compared with MLPs+PE. Since positional encoding is deterministic, scale=512 can be considered to have standard deviation around 121.}
    \label{figure:combined}
    \vspace{-10pt}
\end{figure*}
Varying the sampling standard deviation/scale may lead to degradation results, as shown in \autoref{figure:combined}.
%
However, MLPs face a significant challenge known as the spectral bias, where low-frequency signals are typically favored during training~\citep{rahaman2019spectral}. 
A common solution is to map coordinates into the frequency domain using Fourier features, such as Random Fourier Features and Positional Encoding, which can be understood as manually set high-frequency correspondence prior to accelerating the learning of high-frequency targets.~\citep{tancik2020fourier}. 
This embeddings widely applied to the INRs for novel view synthesis~\citep{mildenhall2021nerf,barron2021mip}, dynamic scene reconstruction~\citep{pumarola2021d}, object tracking~\citep{wang2023tracking}, and medical imaging~\citep{corona2022mednerf}.
% \begin{figure}[!h]
%     \centering
%     \includegraphics[width=1.\textwidth]{Image/psnr_trends_rff_pe_simplified.pdf}
%     \caption{This figure shows the change of PSNR on the whole, low-frequency region, and high-frequency region of the image fitting by using two Fourier Features Embedding with varying scale of variance: (Right) Positional Encoding (PE) (Left) Random Fourier Features (RFF). Both PE and RFF will degrade the low-frequency regions of the target image when variance increases.}
%     \vspace{-20pt} 
%     \label{figure:stats}
% \end{figure}


Although many INRs' downstream application scenarios use this encoding type, it has certain limitations when applied to specific tasks.
%
It depends heavily on two key hyperparameters: the sampling standard deviation/scale (available sampling range of frequencies) and the number of samples.
%
Even with a proper choice of sampling standard deviation/scale, the output remains unsatisfactory, as shown in \autoref{fig:var}: Noisy low-frequency regions and degraded high-frequency regions persist with well chosen sampling standard deviation/scale with the grid-searched standard deviation/scale, which may potentially affect the performance of the downstream applications resulting in noisy or coarse output.
%
However, limited research has contributed to explaining the reason and finding a proper frequency embeddings for input~\citep{landgraf2022pins, yuce2022structured}.

In this paper, we aim to offer a potential explanation for the high-frequency noise and propose an effective solution to the inherent drawbacks of Fourier feature embeddings for INRs.
%
Firstly, we hypothesize that the noisy output arises from the interaction between Fourier feature embeddings and multi-layer perceptrons (MLPs). We argue that these two elements can enhance each other's representation capabilities when combined. However, this combination also introduces the inherent properties of the Fourier series into the MLPs.
%
To support our hypothesis, we propose a simple theorem stating that the unsampled frequency components of the embeddings establish a lower bound on the expected performance. This underpins our hypothesis, as the primary fitting error in finitely sampled Fourier series originates from these unsampled frequencies.

Inspired by the analysis of noisy output and the properties of Fourier series expansion, we propose an approach to address this issue by enabling INRs to adaptively filter out unnecessary high-frequency components in low-frequency regions while enriching the input frequencies of the embeddings if possible.
%
To achieve this, we employ bias-free (additive term-free) MLPs. These MLPs function as adaptive linear filters due to their strictly linear and scale-invariant properties~\citep{mohan2019robust}, which preserves the input pattern through each activation layer and potentially enhances the expressive capability of the embeddings.
%
Moreover, by viewing the learning rate of the proposed filter and INRs as a dynamically balancing problem, we introduce a custom line-search algorithm to adjust the learning rate during training. This algorithm tackles an optimization problem to approximate a global minimum solution. Integrating these approaches leads to significant performance improvements in both low-frequency and high-frequency regions, as demonstrated in the comparison shown in \autoref{fig:var}.
%
Finally, to evaluate the performance of the proposed method, we test it on various INRs tasks and compare it with state-of-the-art models, including BACON~\citep{lindell2022bacon}, SIREN~\citep{sitzmann2020implicit}, GAUSS~\citep{ramasinghe2022beyond} and WIRE~\citep{saragadam2023wire}. 
The experimental results prove that our approach enables MLPs to capture finer details via Fourier Features while effectively reducing high-frequency noise without causing oversmoothness.
%
To summarize, the following are the main contributions of this work:
\begin{itemize}
    \item From the perspective of Fourier features embeddings and MLPs, we hypothesize that the representation capacity of their combination is also the combination of their strengths and limitations. A simple lemma offers partial validation of this hypothesis.

    
    \item  We propose a method that employs a bias-free MLP as an adaptive linear filter to suppress unnecessary high frequencies. Additionally, a custom line-search algorithm is introduced to dynamically optimize the learning rate, achieving a balance between the filter and INRs modules.

    \item To validate our approach, we conduct extensive experiments across a variety of tasks, including image regression, 3D shape regression, and inverse graphics. These experiments demonstrate the effectiveness of our method in significantly reducing noisy outputs while avoiding the common issue of excessive smoothing.
\end{itemize}


\section{Related Works}
\label{section:relatedwork}
The decision version of normal logic programs is NP-complete; therefore, the ASP counting for normal logic programs is \#P-complete~\cite{valiant1979} via a polynomial reduction~\cite{JN2011}.  Given the \#P-completeness, a prominent line of work focused on ASP counting relies on translations from the ASP program to a CNF formula~\cite{LZ2004,Janhunen2004,Janhunen2006,JN2011}. Such translations often result in a large number of CNF clauses and thereby limit practical scalability for {\em non-tight} ASP programs. 
% Lin and Zhao~\cite{LZ2004},  Janhunen~\cite{Janhunen2006} and Janhunen and Niemel{\"a}~\cite{JN2011} offer straightforward
%thereby allowing for the use of a counter on the CNF formula to compute the answer set count.
%However, such translation results in a large number of clauses for programs that are not ``tight''.
%To address the blowup for {\em non-tight} programs, 
Eiter et al.~\cite{EHK2024} introduced T$_{\mathsf{P}}$-\textit{unfolding} to break cycles and produce a tight program. They proposed an ASP counter called aspmc, that performs a treewidth-aware Clark completion from a cycle-free program to a CNF formula. Jakl, Pichler, and Woltran~\cite{JPW09} extended the tree decomposition based approach for \#SAT due to Samer and Szeider~\cite{SS2010} to ASP and proposed a {\em fixed-parameter tractable} (FPT) algorithm for ASP counting. 
Fichte et al.~\cite{FHMW2017,FH2019} revisited the FPT algorithm due to Jakl et al.~\cite{JPW09}  and developed an exact model counter, called DynASP, that performs well on instances with {\em low treewidth}. 
Aziz et al.~\cite{ACMS2015} extended a propositional model counter to an answer set counter by integrating unfounded set detection. 
ASP solvers~\cite{GKS2012} can count answer set via enumeration, which is suitable for a sufficiently small number of answer sets.
Kabir et al.~\cite{KESHFM2022} focused on lifting hashing-based techniques to ASP counting, resulting in an approximate counter, called ApproxASP, with $(\varepsilon,\delta)$-guarantees. 
Kabir et al.~\cite{KCM2024} introduced an ASP counter that utilizes a sophisticated Boolean formula, termed the copy formula, which features a compact encoding.


\section{Supervised linear embeddings for BO}
\label{sec:EGFORSE}
\subsection{Method description}
\label{ssec:outline}

In this paper, the objective function is supposed to depend only on the effective dimensions 
%meaning 
$d_e \ll d$ where we typically assume that it exists a function $f_{\As}: \mathbb{R}^{d_e} \mapsto \mathbb{R}$ such as $f_{\As}(\A\x) = f(\x)$ with $\A \in \mathbb{R}^{d_e \times d}$, $\As = \left\{ \bm{u} = \A\x \ ; \ \forall \x \in \Omega \right\}$ and\ $\Omega = [-1,1]^d$~\cite{WangBatchedHighdimensionalBayesian2018,binoisChoiceLowdimensionalDomain2020}.
The idea is then to perform the optimization procedure in the reduced linear subspace $\As$ so that the number of hyper-parameters to estimate and dimension of the design space are reduced to $d_e$ instead of  $d$.
This allows to build the inexpensive GPs and will ease the acquisition function optimization.
Using a subspace (based on $\A$) for the optimization requires finding the effective dimension of the reduced design space $\mathcal{B} \subset \mathbb{R}^{d_e}$ as well as the backward application $\gamma: \mathcal{B} \mapsto \Omega$.

The proposed method focuses on the definition of the optimization problem when a linear subspace is used as well as on efficient construction procedure of such embedding subspaces.  
Most existing HDBO methods rely on random linear subspaces  meaning that no information is used to incorporate a priori information from the optimization problem within the embedding space which may slow down the optimization process. 
In this work, a recursive search, with $T\in \mathbb{N}$ supervised reduction dimension methods, is performed to find supervised linear subspace so that the most important search directions for the exploration of the objective function are included. 
%E. g., 
Using an initial DoE, one can use a \textit{Partial Least Squares} (PLS) regression \cite{hellandStructurePartialLeast1988} to build such linear embedding prior to the optimization process.
Furthermore, the new search design of the optimization problem (within the linear embedding subspace) is a necessary step.
Most methods rely on a classic optimization problem formulation that may limit the process performance due to very restricted new design space.
Here, once an appropriate linear subspace is found, the optimization problem is turned into a constrained optimization problem to limit the computational cost of the algorithm. The constrained optimization problem can be solved using a classical CBO method~\cite{frazierTutorialBayesianOptimization2018,priemOptimisationBayesienneSous2020,ShahriariTakingHumanOut2016,bartoliAdaptiveModelingStrategy2019,priemUpperTrustBound2020c}.
%algorithm named \textit{Super Efficient Global Optimization with Mixture Of Experts} (SEGOMOE)~\cite{bartoliAdaptiveModelingStrategy2019,priemUpperTrustBound2020c}. 

\subsection{Definition of the reduced search spaces}

To define the optimization problem in the low dimensional spaces, transfer matrices from the initial to the low dimensional spaces and optimization domains must be defined. 
The definition of these transfer matrices relies on a set of $T\in\mathbb{N}$ dimension reduction methods.
For each of the $T$ dimension reduction methods $\mathcal{R}^{(t)}$, the transfer matrix $\A^{(t)} \in \mathbb{R}^{d_e \times d}$ is build using $\mathcal{R}^{(t)}$ where $t \in \{1,\ldots,T\}$.
In this way, we propose to use supervised dimension reduction algorithms, like the PLS~\cite{hellandStructurePartialLeast1988}, to guide the optimization process through highly varying linear subspaces of the objective function.
This procedure allows to tackle the issue of~\citet{binoisChoiceLowdimensionalDomain2020,WangBatchedHighdimensionalBayesian2018} by relying on random linear subspaces defined with random Gaussian transfer matrices.
In their works, the optimization can be performed in a subspace in which the objective function is not varying, meaning the optimum of the objective function could not be discovered.
In the BO framework, the optimization process is usually performed in an hypercube. 
However $\As^{(t)}$ is not an hypercube.
As $\Omega = [-1,1]^d$, it is possible to compute $\mathcal{B}^{(t)} \subset \mathbb{R}^{d_e}$~\cite{binoisChoiceLowdimensionalDomain2020} the smallest hypercube containing all points of $\As^{(t)}$ such as 
\begin{equation}
    \mathcal{B}^{(t)} = \left[-\sum\limits_{i=1}^d\left| A^{(t)}_{1,i} \right|, \sum\limits_{i=1}^d\left| A^{(t)}_{1,i} \right| \right] \times \cdots \times \left[-\sum\limits_{i=1}^d\left| A^{(t)}_{d_e,i} \right|, \sum\limits_{i=1}^d\left| A^{(t)}_{d_e,i} \right| \right].
    \label{eq:borne_sub}
\end{equation}
Performing the optimization process in $\mathcal{B}^{(t)}$ leads to define a backward application from $\mathcal{B}^{(t)}$ to $\Omega$ to compute the objective function on the desired point. 
% Thus, no point $\x \in \Omega$, solution of the problem $$\x=\left[\A^{(t)}\right]^{+}\bm{u}$$ 
% with {$\left[\A^{(t)}\right]^{+} = \left[\A^{(t)}\right]^\top \left[ \A^{(t)} \left[\A^{(t)}\right]^\top \right]^{-1}$} the pseudo-inverse of $\A^{(t)}$, for all $\bm{u} \in \mathbb{R}^{d_e}$, is forgotten in the optimization process.
% In fact, $\As^{(t)} \subset \mathcal{B}^{(t)}$.
% However, $\left[\A^{(t)}\right]^{+}$ implies that some points $\bm{u} \in \mathcal{B}^{(t)}$ can not provide $\x = \left[\A^{(t)}\right]^{+} \bm{u}$ with $\x \in \Omega$.
% This property is actually used to consider objective function values on the closer edge of the $\mathcal{B}^{(t)}$, image by $\left[\A^{(t)}\right]^{+}$ in $\Omega$.
% This transformation, introduced in the following, is realized thanks to a backward application from $\mathcal{B}^{(t)}$ to $\Omega$.

\subsection{The backward application}
\label{ssec:reverse}

We use the backward application introduced by~\cite{binoisChoiceLowdimensionalDomain2020}. Namely, a bijective application $\gamma_B^{(t)} : \As^{(t)} \subset \mathbb{R}^d_e \mapsto \Omega \subset \mathbb{R}^d$ such that
\begin{equation}
    \gamma_B^{(t)}(\bm{u}) = \arg \min\limits_{\x \in \Omega} \left\{ \left\| \x - \left[\A^{(t)}\right]^{+} \bm{u} \right\|^2, \ \text{s.c. } \ \A^{(t)}\x = \bm{u}  \right\},
    \label{eq:quad_prob}
\end{equation}
where {$\left[\A^{(t)}\right]^{+} = \left[\A^{(t)}\right]^\top \left[ \A^{(t)} \left[\A^{(t)}\right]^\top \right]^{-1}$} the pseudo-inverse of $\A^{(t)}$. 
This problem requires to solve a quadratic optimization problem.
As $\As^{(t)} \subset \mathcal{B}^{(t)}$, $\gamma_B^{(t)}$ defines an injection from $\mathcal{B}^{(t)}$ to $\Omega$, meaning some points of $\mathcal{B}^{(t)}$ do not have any image in $\Omega$.
For instance, Figure~\ref{fig:backpropa} displays $\Omega \subset \mathbb{R}^{10}$ projected in a domain $\mathcal{B}^{(t)} \subset \mathbb{R}^2$.
The $\As^{(t)}$ domain is 
%produced 
in white while the points in the black domain do not have any image in $\Omega$ by $\gamma_B^{(t)}$.
\begin{figure}[ht!]
    \centering
    \includegraphics[width=0.5\textwidth]{Figures/emb/Back_proj.pdf}
    \caption{Illustration of $\mathcal{B}^{(t)}$ and $\As^{(t)}$. In black, the points of $\mathcal{B}^{(t)}$ without image in $\Omega$ by $\gamma_B^{(t)}$; in white, the points of $\As^{(t)}$ corresponding to the points of $\mathcal{B}^{(t)}$ with an image in $\Omega$ by $\gamma_B^{(t)}$.}
    \label{fig:backpropa}
\end{figure}

The method developed by~\cite{binoisChoiceLowdimensionalDomain2020} is using the $\gamma_B^{(t)}$ even if such backward application is not defined all over $\mathcal{B}^{(t)}$ by itself.
In fact, the function $f^{(t)}(\bm{u})=f\left(\gamma_B^{(t)}(\bm{u})\right)$ is only defined on $\As^{(t)}$, meaning the optimization problem in the linear subspace is given by:
\begin{equation}
    \min_{\bm{u}\in\As^{(t)}}\left\{f^{(t)}(\bm{u})\right\}.
    \label{eq:rrembo_pb}
\end{equation}
However, standard BO algorithms are only solving optimization problems with an objective function defined on an hypercube. \citet{binoisChoiceLowdimensionalDomain2020} used  $\mathcal{B}^{(t)}$, see~\eqref{eq:borne_sub}, the smallest hypercube containing $\As^{(t)}$.
This way, some points reachable by the BO algorithm might not be evaluated on the objective function as they do not have any image in $\Omega$ using $\gamma_B^{(t)}$.
To fix this issue, an additional modification of the EI acquisition function is thus introduced in \cite{binoisChoiceLowdimensionalDomain2020}. In fact, over $\As^{(t)}$, the acquisition function $\alpha^{(l)}_{EI_{ext}}(\bm{u}) = \alpha^{(l)}_{EI}(\bm{u})$~\eqref{eq:EI} while on $\mathcal{B}^{(t)} \backslash \As^{(t)}$, the acquisition function $\alpha^{(l)}_{EI_{ext}}(\bm{u}) = -\| \bm{u} \|_2$.
As the EI acquisition function is positive (i.e., lower bounded), the optimization of the acquisition function provides a point of $\As^{(t)}$. This quadratic problem~\eqref{eq:quad_prob} must be solved at each acquisition function evaluation to know if $\bm{u}$ belongs to $\As^{(t)}$.
Depending on the number of evaluations of the $\alpha^{(l)}_{EI_{ext}}$ acquisition function, an optimization process performed with the RREMBO algorithm can be expensive in CPU time.
Actually the cost in CPU time grows with the number of design variables $d$.
Figure~\ref{fig:proj_pb:proj} shows the objective function of 10 design variables projected in a linear subspace of 2 dimension.
\begin{figure}[!hbt]
    \centering
    \subfloat[Projected objective function. \label{fig:proj_pb:proj}]{\includegraphics[width=0.5\textwidth]{Figures/emb/REMBO_Iter.png}}
    \subfloat[${\alpha^{(l)}_{EI_{ext}}}$. \label{fig:proj_pb:ei}]{\includegraphics[width=0.5\textwidth]{Figures/emb/REMBO_Iner.png}}
    \caption{Projection of an objective function of 10 design variables into a linear subspace of 2 dimensions and the corresponding ${\alpha^{(l)}_{EI_{ext}}}$ acquisition function. The grey area is the unfeasible domain, the green squares are DoE points of $\As^{(t)}$, the red squares are DoE points of $\mathcal{B}^{(t)} \backslash \As^{(t)}$ and the green star is a solution of the optimization sub-problem.}
    \label{fig:proj_pb}
\end{figure}
The grey area shows that an important part of $\mathcal{B}^{(t)}$ does not have any image in $\Omega$. 
In Figure~\ref{fig:proj_pb:ei}, one can see the grey domain is replaced by negative values increasing towards the center of $\mathcal{B}^{(t)}$ allowing convergence to points having an image by $\gamma_B^{(t)}$ in $\Omega$.

\subsection{The optimization problem}
\label{ssec:sub_prob}

Section~\ref{ssec:reverse} shows that the backward application $\gamma_B^{(t)}$ is bijective from $\As^{(t)}$ to $\Omega$ and injective from $\mathcal{B}^{(t)}$ to $\Omega$. 
Thus some points of $\mathcal{B}^{(t)}$ do not have image by $\gamma_B^{(t)}$ in $\Omega$. 
To avoid points of $\mathcal{B}^{(t)} \backslash \As^{(t)}$, we define an optimization problem with a constraint which is feasible on $\As^{(t)}$ and unfeasible on $\mathcal{B}^{(t)} \backslash \As^{(t)}$.
For the value of the constraint when $\bm{u} \not\in \As^{(t)}$, we chose the opposite of the 2-norm of $\bm{u}$ .
The constraint value is negative and tends to zero when $\bm{u}$ is getting closer to $\As^{(t)}$ in norm.
To define the constraint value in the feasible zone, ones can rely on~\eqref{eq:borne_sub} and on $\Omega = [-1,1]^d$.
The equation shows that points on the edge of $\mathcal{B}^{(t)}$, having an image in $\Omega$ by $\gamma_B^{(t)}$, are in the corners of $\Omega$. 
The corners of a domain are considered to be the points whose components are $-1$ or $1$.
However, the corners of the domain are the farthest points from the center $x_c = \bm{0} \in \mathbb{R}^d$ and have the same norm. 
All the images from points of $\mathcal{B}^{(t)}$ by $\gamma_B^{(t)}$ have a lower norm than the corners of $\Omega$. 
Hence the constraint of the optimization problem is defined as the difference between the norm of the corners $\Omega$ and the norm of $\gamma_{B}^{(t)}(\bm{u})$ when $\bm{u} \in \As^{(t)}$.
Thus, the constraint value tends to zero when $\bm{u}$ is getting closer to $\mathcal{B}^{(t)} \backslash \As^{(t)}$.
Eventually, the constraint function is normalized to provide values in $[-1,1]$.
This normalization on the bounds of the constraint function balances the importance of the feasible and unfeasible domain in the optimization process.

To summarize, the constraint is given by $g^{(t)}(\bm{u}) \geq 0$ where:
\begin{equation}
    g^{(t)}(\bm{u}) = \left\{
    \begin{tabular}{ll}
        $1 - \frac{\left\| \gamma_B^{(t)}\left(\bm{u}\right)\right\|^2_2}{d}$ &  if $\bm{u} \in \mathcal{A}^{(t)}$\\
        $-\left\|\bm{u}^{\A^{(t)}}\right\|^2_2$ &  otherwise
    \end{tabular}
    \right.
    \label{eq:it_cst}
\end{equation}
with $u^{\A^{(t)}}_i = u_i / \sum\limits_{j=1}^d\left| A^{(t)}_{i,j} \right|$.
The terms $u^{\A^{(t)}}_i$ and $u_i$ are respectively the $i^{\text{th}}$ components of the $\bm{u}^{\A^{(t)}}$ and $\bm{u}$ vectors.
One remarks that $f^{(t)}(\bm{u}) = f\left(\gamma_B^{(t)}(\bm{u})\right)$ function is not defined all over $\mathcal{B}^{(t)}$. 
That is why one seeks to give a value to $f^{(t)}$ on $\mathcal{B}^{(t)} \backslash \As^{(t)}$ to obtain a function not only defined on $\mathcal{B}^{(t)}$.
The extension of $f^{(t)}$ on $\mathcal{B}^{(t)} \backslash \As^{(t)}$ is given by $f^{(t)}(\bm{u}) = f\left(\gamma_W^{(t)}(\bm{u})\right)$ with  
\begin{equation}
     \gamma_W^{(t)}(\bm{u}) \in \arg \min\limits_{\x \in \Omega} \| \x - \left[\A^{(t)}\right]^{+} \bm{u} \|^2.
    \label{eq:quad_prob_wang}
\end{equation}
Indeed, the $\gamma_W^{(t)}$ application exists for all points $\bm{u} \in \mathcal{B}^{(t)}$ although it can provide the same $\x \in \Omega$ for different $\bm{u} \in \mathcal{B}^{(t)}$.
These points of $\Omega$ are moreover reachable with points of $\As^{(t)}$ using the $\gamma_B^{(t)}$ backward application.
Thus, one of the objective function minima is necessarily in $\As^{(t)}$.
Eventually, the objective function $f^{(t)}$ is given on $\mathcal{B}^{(t)}$ by 
\begin{equation}
    f^{(t)}(\bm{u}) = \left\{
    \begin{tabular}{ll}
        $f\left(\gamma_B^{(t)}\left(\bm{u}\right)\right)$    &  if $\bm{u} \in \mathcal{A}^{(t)}$\\
        $f\left(\gamma_W^{(t)}\left(\bm{u}\right)\right)$  &  otherwise
    \end{tabular}
    \right.
    \label{eq:it_obj}
\end{equation}
In fact, one solves the following constrained optimization problem in the $\mathcal{B}^{(t)}$ hypercube:
\begin{equation}
    \min\limits_{\bm{u} \in \mathcal{B}^{(t)}} \left\{ f^{(t)}(\bm{u}) \quad \text{s.c.} \quad g^{(t)}(\bm{u}) \geq 0 \right\}.
    \label{eq:it_pb}
\end{equation}
where $f^{(t)}(\bm{u})$ and $g^{(t)}(\bm{u})$ are respectively defined by Eq.~\eqref{eq:it_obj} and Eq.~\eqref{eq:it_cst}.
To solve this problem, a standard CBO algorithm~\cite{frazierTutorialBayesianOptimization2018,priemOptimisationBayesienneSous2020,ShahriariTakingHumanOut2016}, like SEGOMOE~\cite{bartoliAdaptiveModelingStrategy2019,priemUpperTrustBound2020c} can be used.
Figure~\ref{fig:it_pb:pb} shows the optimization problem~\eqref{eq:it_pb} for the $10$ design variable function given in Figure~\ref{fig:proj_pb:proj} and projected in a 2 dimensional linear subspace. 
\begin{figure}[!htb]
    \centering
    \subfloat[Optimization problem \eqref{eq:it_pb}. \label{fig:it_pb:pb}]{\includegraphics[width=0.5\textwidth]{Figures/EGORSE_Iter.png}}
    \subfloat[SEGOMOE optimization sub-problem associated to problem~\eqref{eq:it_pb}. \label{fig:it_pb:spb}]{\includegraphics[width=0.5\textwidth]{Figures/EGORSE_Iner.png}}
    \caption{An optimization problem~\eqref{eq:it_pb} in a 2 dimensional linear subspace and the associated SEGOMOE optimization sub-problem. The grey area is the unfeasible domain, the green squares are DoE points of $\As^{(t)}$, the red squares are DoE points of $\mathcal{B}^{(t)} \backslash \As^{(t)}$ and the green star is a solution of the optimization sub-problem.}
    \label{fig:it_pb}
\end{figure}
One sees that the grey unfeasible area corresponds to the grey area of Figure~\ref{fig:proj_pb:proj}.
Figure~\ref{fig:it_pb:spb} shows the optimization sub-problem of the SEGOMOE algorithm using the EI acquisition function where the green and red squares are points of the initial DoE.
The predicted unfeasible zone, in grey in Figure~\ref{fig:it_pb:spb}, contains the unfeasible points of the initial DoE. 
Recall, with SEGOMOE, only the mean of the GP modeling the constraint is used to defined the feasible zones.
On the contrary of RREMBO no quadratic problem solving is needed to solve the optimization sub-problem. 
In fact, the introduced process avoids the quadratic problem solving in the optimization sub-problem by defining functions existing all over $\mathcal{B}^{(t)}$, which is not the case of RREMBO (see Section~\ref{ssec:reverse}).
The quadratic problem is only solved when the objective function is called at each iteration of the SEGOMOE algorithm.
Thus, the introduced EGORSE algorithm should be faster in CPU time than RREMBO.

\subsection{Adaptive learning of the linear subspace}

To ease the convergence process, we use a linear subspace discovered by supervised dimension reduction methods, i.e. taking points of the considered domain and the associated function values as inputs.
In our case, linear dimension reduction methods are used to provide the transfer matrix $\A^{(t)} \in \mathbb{R}^{d \times d_e}$ from $\Omega$ to $\As^{(t)}$.
 In order to find the minimum of the objective function, we solve the optimization problem~\eqref{eq:it_pb} with a CBO algorithm~\cite{frazierTutorialBayesianOptimization2018,ShahriariTakingHumanOut2016,priemOptimisationBayesienneSous2020} in $\mathcal{B}^{(t)}$ generated by the $\A^{(t)}$ matrix with a maximum of \texttt{max\_nb\_it\_sub} iterations.
However, in larger space, the linear subspace approximation can lack of accuracy if the number of points used by the dimension reduction methods is not sufficient.
To improve the generated subspace, the previous process is iterated using all the evaluated points during the previous iterations.
Note that points outside of the subspace $\As^{(t)}$ are added to provide information which is non biased by the subspace selection.
In that way, points coming from an optimization performed in a subspace defined by other dimension reduction methods can be used.
For instance, a transfer matrix defined by a random Gaussian distribution is chosen. 
The convergence properties, given by~\citet{binoisChoiceLowdimensionalDomain2020}, are thus preserved. 
Finally, we 
%propose to 
generalize this approach by using several dimension reduction methods, which can be unsupervised (like random Gaussian transfer matrices or hash tables~\cite{binoisChoiceLowdimensionalDomain2020,nayebiFrameworkBayesianOptimization2019}), or supervised (like PLS or MGP~\cite{hellandStructurePartialLeast1988,GarnettActiveLearningLinear2014}), to consider their advantages. 
The overall process of EGORSE is finally given by Algorithm~\ref{alg:EGORSE} whereas the flow chart of the method is described by the \textit{eXtended Design Structure Matrix} (XDSM) \cite{lambeExtensionsDesignStructure2012} diagram of Figure~\ref{fig:xdsm}.
\begin{algorithm}[!hbtp]
     \begin{algorithmic}[1]
        \INPUT{: an objective function $f$, an initial {DoE} $\mathcal{D}_f^{(0)}$, a maximum number of iterations \mbox{max\_nb\_it}, a maximum number of iterations by subspace max\_nb\_it\_sub, a number of active directions $d_e$, a list $\mathcal{R} = \left\{\mathcal{R}^{(1)},\ldots,\mathcal{R}^{(T)}\right\}$ of $T \in \mathbb{N}^+$ supervised or unsupervised dimension reduction methods\;}
        \FOR{$i = 0$ \TO \mbox{max\_nb\_it} - 1}
            \FOR{$t = 1$ \TO $T$}
                \STATE {Build $\A^{(t)} \in \mathbb{R}^{d_e \times d}$ with $\mathcal{R}^{(t)}$ using or not $\mathcal{D}_f^{(i)}$}
                \STATE {Define $\mathcal{B}^{(t)}$ (see Eq.~\eqref{eq:borne_sub})}
                \STATE {Build $f^{(t)}$ (see Eq.~\eqref{eq:it_obj})}
                \STATE {Build $g^{(t)}$ (seeEq.~\eqref{eq:it_cst})}
                \STATE {Solve the optimization problem
                $\min\limits_{\bm{u} \in \mathcal{B}^{(t)}} \left\{ f^{(t)}(\bm{u}) \quad \text{s.c.} \quad g^{(t)}(\bm{u}) \geq 0 \right\}$
                with a CBO algorithm using \mbox{max\_nb\_it\_sub} maximum iterations.}
            \ENDFOR
            \STATE {$\mathcal{D}_f^{(i+1)} = \mathcal{D}_f^{(i)} \cup \{\text{Points already evaluated on $f$ at iteration $i$}\}$}
        \ENDFOR
        \OUTPUT{: The best point regarding the value of $f$ in  $\mathcal{D}_f^{(\text{max\_nb\_it})}$\;}
    \end{algorithmic}
    \caption{The EGORSE process.}
    \label{alg:EGORSE}
\end{algorithm}
\begin{figure}[h]
    \centering
    \includegraphics[width=\textwidth]{Figures/EGORSE_XDSM_v1.pdf}
    \caption{An XDSM diagram of the EGORSE framework.}
    \label{fig:xdsm}
\end{figure}
To conclude, the EGORSE algorithm includes three main contributions:
\begin{itemize}
    \item The restriction of the number of quadratic problems solved with a new formulation of the subspace optimization problem.
    \item The use of supervised dimension reduction methods promoting the search in highly varying directions of the objective function in $\Omega$.
    \item The adaptive learning of the linear subspace using all the evaluated points.
\end{itemize}

{\color{black}
\subsection{Efficient supervised embeddings}

Building an efficient supervised embedding (i.e., the transfer matrix $A^{(t)}$) plays a key role in the EGORSE framework. In this work, we consider two well-known supervised embedding techniques: the \textit{Partial Least-Squares (PLS)}~\cite{hellandStructurePartialLeast1988} and the \textit{Marginal Gaussian Process (MGP)}~\cite{GarnettActiveLearningLinear2014} based methods. In this section, we recall briefly the main features of the PLS and MGP methods. To simplify the notations, we will omit the iteration indices $(t)$ in this section; we will refer to the transfer matrix using within EGORSE by $A$ and to the current DoE by $\mathcal{D}_f$. 
These two embeddings PLS and MGP coupled with GP are available in the SMT toolbox~\cite{bouhlelPythonSurrogateModeling2019,SMT2023}\footnote{\href{https://github.com/SMTorg/smt}{https://github.com/SMTorg/smt}}.


\paragraph{On the PLS transfer matrix ( $\A_{\mbox{PLS}}$).}
The PLS~\cite{hellandStructurePartialLeast1988} method searches for the most important $d_e$ orthonormal directions $\bm{a}_{(i)} \in \mathbb{R}^d$, $i \in \{1,\ldots,d_e\}$ of $\Omega$ in regards of the influence of (related to the DoE $\mathcal{D}_f$) the inputs \mbox{$\bm{X}^{(l)}_{(0)} = \left[\left[\x^{(0)}\right]^\top,\ldots,\left[\x^{(l)}\right]^\top \right]^\top \in \mathbb{R}^{d \times l}$} on the outputs $\bm{Y}^{(l)}_{f,(0)} = \bm{Y}^{(l)}_f \in \mathbb{R}^l$.
These directions are recursively given by: 
\begin{gather*}
	\bm{a}'_{(i)} \in \arg \max_{\bm{a}' \in \mathbb{R}^d} \left\{ \bm{a}'^\top \left[\bm{X}^{(l)}_{(i)}\right]^\top \bm{Y}^{(l)}_{f,(i)} \left[\bm{Y}^{(l)}_{s,(i)}\right]^\top \bm{X}^{(l)}_{(i)} \bm{a}' \ , \ \bm{a}'^\top \bm{a}' = 1 \right\}, \label{eq:kpls_opt}\\
	\bm{t}_{(i)} = \bm{X}^{(l)}_{(i)} \bm{a}'_{(i)}, \quad
	\bm{p}_{(i)} = \left[\bm{X}^{(l)}_{(i)}\right]^\top \bm{t}_{(i)} , \quad c_{(i)} = \left[\bm{Y}^{(l)}_{s,(i)}\right]^\top \bm{t}_{(i)} \\ \bm{X}^{(l)}_{(i+1)} = \bm{X}^{(l)}_{(i)} - \bm{t}_{(i)} \bm{p}_{(i)}, \quad
	\bm{Y}^{(l)}_{f,(i+1)} = \bm{Y}^{(l)}_{f,(i)} - c_{(i)} \bm{t}_{(i)}.
 %\\ \A' = [\bm{a}'_{(1)},\ldots,\bm{a}'_{(d_e)}], \quad \bm{P} = [\bm{p}'_{(1)},\ldots,\bm{p}'_{(d_e)}], \quad
	%\A = \A' (\bm{P}^\top \A')^{-1},
\end{gather*}
where $\bm{X}^{(l)}_{(i+1)} \in \mathbb{R}^{d \times l}$ and $\bm{Y}^{(l)}_{f,(i+1)} \in \mathbb{R}^d$ are the residuals of the projection of $\bm{X}^{(l)}_{(i)}$ and $\bm{Y}^{(l)}_{f,(i)}$ on the $\text{i}^{\text{th}}$ principal component $\bm{t}_{(i)} \in \mathbb{R}^d$.
Finally, $\bm{p}_{(i)} \in \mathbb{R}^d$ and $c_{(i)} \in \mathbb{R}$ are respectively the regression coefficients of $\bm{X}^{(l)}_{(i)}$ and $\bm{Y}^{(l)}_{s,(i)}$ on the $\text{i}^{\text{th}}$ principal component $\bm{t}_{(i)}$ for $i \in \{1,\ldots,d_e\}$.
In fact, the square of the covariance between $\bm{a}'_{(i)}$ and $\left[\bm{X}^{(l)}_{(i)}\right]^\top \bm{Y}^{(l)}_{f,(i)}$ is recursively maximized. In this case, the PLS transfer matrix is given by:
$$
\A_{\mbox{PLS}} = \A' (\bm{P}^\top \A')^{-1},
$$
where $\A' = [\bm{a}'_{(1)},\ldots,\bm{a}'_{(d_e)}] $ and $
	\bm{P} = [\bm{p}'_{(1)},\ldots,\bm{p}'_{(d_e)}]$. 


\paragraph{On the MGP transfer matrix ($\A_{\mbox{MGP}}$).}
With the MGP~\cite{GarnettActiveLearningLinear2014}, the matrix is considered as the realization of a Gaussian distribution \mbox{$\mathbb{P}(A) = \mathcal{N}\left(\A_{\mbox{MGP}}, \bm{\Sigma}_{\mbox{MGP}} \right)$} where $\A_{\mbox{MGP}}$ and $\bm{\Sigma}_{\mbox{MGP}}$ are respectively the prior mean and the covariance matrix.
The density function of $\mathbb{P}(A)$ is noted $p(\A)$.  Then, the transfer matrix $\A_{\mbox{MGP}}$ used when referring to the MGP dimension reduction method is given as the maximum of the  posterior probability law of $\A$ with respect to the DoE $\mathcal{D}_f$, i.e.,
$$
\A_{\mbox{MGP}} \in \arg \max_{A}  p\left(\A | \mathcal{D}_f\right):= p(\A) \cdot \mathcal{L}\left(\bm{Y}_f,\A\right),
$$
 where  $\mathbb{P}\left(\A | \mathcal{D}_f\right)$ represents the density function of $\mathbb{P}\left(\A | \mathcal{D}_f\right)$ and $\mathcal{L}$ the likelihood. The covariance matrix $\bm{\Sigma}_{\mbox{MGP}}$ is estimated by the inverse of the logarithm of $p\left(\A | \mathcal{D}_f\right)$ Hessian  matrix evaluated at $\A_{\mbox{MGP}}$
\begin{equation*}
            \hat{\bm{\Sigma}}^{-1}_{\mbox{MGP}}  = - \nabla^2 \log p\left(\A_{\mbox{MGP}} | \mathcal{D}_f\right),
\end{equation*}
where $\nabla^2$ is the Hessian operator with respect to $\A$. The remaining details on the MGP can be found in~\cite{GarnettActiveLearningLinear2014}.
}


\section{A sensitivity analysis over \texttt{EGORSE} hyper-parameters}
\label{sec:sensitivity-anal}
\subsection{Implementation details}

\texttt{EGORSE} method is implemented with Python 3.8. A CBO process (see Section~\ref{ssec:sub_prob}) is performed at each iteration of the \texttt{EGORSE} algorithm
using SEGO~\cite{SasenaExplorationmetamodelingsampling2002}  (from the SEGOMOE toolbox~\cite{bartoliAdaptiveModelingStrategy2019}).
The SEGOMOE toolbox uses the SMT~\cite{bouhlelPythonSurrogateModeling2019}, a Python package used to build the GP model.
In the CBO process, the EI acquisition function is optimized in two steps. 
First, a good starting point is found by solving the sub-optimization problem~\eqref{eq:iner_opt} with the ISRES~\cite{runarsson2005search} from the NLOPT~\cite{johnson2014nlopt} Python toolbox.
ISRES is an evolutionary optimization algorithm able to solve multi-modal optimization problems with equality and inequality constraints.
The algorithm explores the domain to find an optimal area maximizing the acquisition function and respecting the feasibility criteria.
However, such algorithm needs many function evaluations to converge.
To limit this number of evaluations, the solution provided by ISRES is refined with a gradient based optimization algorithm as the analytical derivatives of the acquisition function and the feasibility criteria are easily available from the GP approximations and provided as outputs from SMT~\cite{bouhlelPythonSurrogateModeling2019}.
The gradient based algorithm used is SNOPT~\cite{Gillsnopt2005} from the PyOptSparse~\cite{Perezpyopt2012} Python toolbox whose initial guess is given by ISRES.
To solve the quadratic problem~\eqref{eq:quad_prob}, the CVXOPT~\cite{andersen2013cvxopt} Python toolbox is chosen.
A point $\bm{u}\in\mathcal{B}_B^{(t)}$ is considered to belong to $\As^{(t)}$ if the CVXOPT optimization status '\verb'optimal'' is reached meaning that the quadratic problem has a solution.

\subsection{On the setting of \texttt{EGORSE} hyper-parameters}
\label{sssec:hyperparam}

\texttt{EGORSE} is controlled by a set of hyper-parameters. To select the value of these hyper-parameters, a parametric study is performed on two optimization problems.
The hyper-parameters considered in this study are the following.
\begin{itemize}
    \item \textbf{The number of points in the initial DoE}. It impacts the supervised dimension reduction methods. On the overall \texttt{EGORSE} versions, three sizes of initial DoE are tested: 5 points, $d$ points and $2d$ points where $d$ is the number of design variables.
    \item \textbf{The supervised dimension reduction method.} It changes the behavior of the algorithm favoring different directions of the domain.
    The PLS~\cite{hellandStructurePartialLeast1988} and MGP~\cite{gardnerDiscoveringExploitingAdditive2017} methods are considered in this study. 
    The PLS method is coming from the scikit-learn~\cite{scikit-learn} Python toolbox. 
    The MGP method is implemented within SMT~\cite{bouhlelPythonSurrogateModeling2019}.
    \item \textbf{The unsupervised dimension reduction method} relies on a random Gaussian transfer matrix like in~\citet{binoisChoiceLowdimensionalDomain2020} work or on hash table like in~\citet{nayebiFrameworkBayesianOptimization2019} work.
\end{itemize}
In what comes next, the following 6 possible variants of \texttt{EGORSE} will be tested and compared:
\begin{enumerate}
    \item \texttt{EGORSE Gaussian}: it uses Gaussian random transfer matrix.
    \item  \texttt{EGORSE Hash}: it uses random matrix defined by Hash table.
    \item  \texttt{EGORSE PLS}: it uses transfer matrix defined by the PLS method.
    \item  \texttt{EGORSE PLS+Gaussian}: it uses Gaussian random transfer matrix and transfer matrix defined by the PLS.
    \item  \texttt{EGORSE MGP}: it uses transfer matrix defined by the MGP method.
    \item  \texttt{EGORSE MGP+Gaussian}: it uses Gaussian random transfer matrix and transfer matrix defined by the MGP.
\end{enumerate}
\subsection{Study on two analytical problems}
\subsubsection{Definition of the two problems}

The considered class of problems is an adjustment of the Modified Branin (MB) problem~\cite{ParrInfillsamplingcriteria2012} whose number of design variables is artificially increased.
This problem is commonly used in the literature~\cite{binoisChoiceLowdimensionalDomain2020,WangBayesianoptimizationbillion2016,nayebiFrameworkBayesianOptimization2019} and it is  defined as follows:
\begin{equation}
    \min_{\bm{u} \in \Omega_1}{f_1(\bm{u})},
\end{equation}
where $\Omega_1 = [-5,10] \times [0,15]$ and
\begin{equation}
    f_1(\bm{u}) = \left[ \left( u_2 - \frac{5.1u_1^2}{4\pi^2} + \frac{5u_1}{\pi} - 6 \right)^2 +\left( 10 - \frac{10}{8\pi} \right) \cos{(u_1) + 1} \right] + \frac{5u_1 + 25}{15}.
\end{equation}
The modified version of the Branin problem is selected because it count three local minima including a global one.
The value of the global optimum is about ${\text{MB\_$d$}}_{min} = 1.1$.
Furthermore, the problem is normalized to have $\bm{u} \in [-1,1]^2$.
To artificially increase the number of design variables, a random matrix $\A_d \in \mathbb{R}^{2 \times d}$ is generated such that for all $\x \in [-1,1]^d$, $\A_d \x = \bm{u}$ belongs to $[-1,1]^2$.
An objective function MB\_$d$, where $d$ is the number of design variables, is defined such that $\text{MB\_$d$}(\x) = f_1(\A_d\x)$.
Eventually, we solve the following optimization problem:
\begin{equation}
    \min\limits_{x \in [-1,1]^d} \text{MB\_$d$}(\x) = \min\limits_{x \in [-1,1]^d} f_1(\A_d\x).
\end{equation}
In the following numerical experiments are conducted on two functions of respective dimension 10 and 100. These two test functions are denoted $\text{MB\_$10$}$ and $\text{MB\_$100$}$.

\subsubsection{Convergence plots}
\label{sssec:conv}

To study the \texttt{EGORSE} hyper-parameter impact, convergence plots are obtained for both problems.
Ten independent optimizations are thus performed on each of the problems, for each of the \texttt{EGORSE} versions, using 10 initial DoE.
The number of iterations is imposed to 10 and the number of evaluations by sub-space optimization is set to $20d_e$ and $d_e=2$. 
For  \texttt{EGORSE PLS, \texttt{EGORSE} MGP, \texttt{EGORSE} Hash and \texttt{EGORSE} Gaussian}, the number of iterations is doubled to keep a fixed number of evaluations, i.e. 800 evaluations per run.
Indeed, at each iteration, the problem is evaluated only $20d_e$ for \texttt{EGORSE} composed of a unique dimension reduction method against $2\times20d_e$ for \texttt{EGORSE} composed of two dimension reduction methods.  
All of the \texttt{EGORSE} versions are then compared displaying the evolution of the mean and standard deviation on the $10$ optimizations of the lower value of the objective function evaluated with respect to the number of evaluations. 
For readability, the standard deviation is displayed with a reduction factor of four. 

\subsubsection{Results analysis}
\label{sssec:impact}

In this section, the convergence robustness and speed of the six \texttt{EGORSE} versions are analyzed with convergence plots (see Section~\ref{sssec:conv}). {\textcolor{black}{Reaching the global minimum (value of 1) of the modified Branin optimization problem in high dimension~\cite{ParrInfillsamplingcriteria2012}  is very hard specially with a limited budget of function evaluations, so the the main purpose here is to compare the convergence speed and the quality of the best value found by  each method.}} 
The convergence plots of the considered \texttt{EGORSE} versions for the MB\_10 problems are displayed in Figure~\ref{fig:EGORSE_MB10}.
\begin{figure}[!htbp]
    \centering
    \subfloat[DoE of 5 points.\label{fig:EGORSE_MB10_5}]{\includegraphics[width=0.50\textwidth]{Figures/CV/CV_MB_10_EGORSE_5.pdf}}
    \subfloat[DoE of 10 points.\label{fig:EGORSE_MB10_d}]{\includegraphics[width=0.50\textwidth]{Figures/CV/CV_MB_10_EGORSE_D.pdf}} \\
    \subfloat[DoE of 20 points.\label{fig:EGORSE_MB10_2d}]{\includegraphics[width=0.50\textwidth]{Figures/CV/CV_MB_10_EGORSE_2D.pdf}}
    \subfloat[Best versions for each DoE size. \label{fig:EGORSE_MB10_best}]{\includegraphics[width=0.50\textwidth]{Figures/CV/CV_MB_10_EGORSE_ALL.pdf}}
    \caption{Convergence plots of 6 versions of \texttt{EGORSE} applied on the MB\_10 problem. The grey vertical line shows the size of the initial DoE.}
    \label{fig:EGORSE_MB10}
\end{figure}
In Figure~\ref{fig:EGORSE_MB10_5}, one can see that the \texttt{EGORSE Gaussian} algorithm offers the best performance in term of convergence speed and robustness for an initial DoE of 5 points. 
However, Figures~\ref{fig:EGORSE_MB10_d} and~\ref{fig:EGORSE_MB10_2d} show that \texttt{EGORSE PLS+Gaussian} outperforms \texttt{EGORSE Gaussian} in term of convergence speed and robustness for larger initial DoE. 
Figure~\ref{fig:EGORSE_MB10_best} finally compares these different versions and they are hardly distinguishable for the MB\_10 problem. 
Thus, all the six \texttt{EGORSE} algorithms seem to provide equivalent performance for low dimensional problems.
Figure~\ref{fig:EGORSE_MB100} displays the convergence plots of the six \texttt{EGORSE} versions applied to the MB\_100 problem.
\begin{figure}[!htbp]
    \centering
    \subfloat[DoE of 5 points.\label{fig:EGORSE_MB100_5}]{\includegraphics[width=0.50\textwidth]{Figures/CV/CV_MB_100_EGORSE_5.pdf}}
    \subfloat[DoE of $d$ points.\label{fig:EGORSE_MB100_d}]{\includegraphics[width=0.50\textwidth]{Figures/CV/CV_MB_100_EGORSE_D.pdf}} \\
    \subfloat[DoE of $2d$ points.\label{fig:EGORSE_MB100_2d}]{\includegraphics[width=0.50\textwidth]{Figures/CV/CV_MB_100_EGORSE_2D.pdf}}
    \subfloat[Best versions for each DoE size. \label{fig:EGORSE_MB100_best}]{\includegraphics[width=0.50\textwidth]{Figures/CV/CV_MB_100_EGORSE_ALL.pdf}}
    \caption{Convergence plots of 6 versions of \texttt{EGORSE} applied on the MB\_100 problem. The grey vertical line shows the size of the initial DoE.}
    \label{fig:EGORSE_MB100}
\end{figure}
One can see that the different \texttt{EGORSE} versions are much more distinguishable on Figure~\ref{fig:EGORSE_MB100}. 
In fact, Figures~\ref{fig:EGORSE_MB100_5},~\ref{fig:EGORSE_MB100_d} and~\ref{fig:EGORSE_MB100_2d} show that \texttt{EGORSE PLS+Gaussian} provides the best convergence speed and robustness trade-off for all the initial DoE sizes tested.
To select the best number of initial DoE points, Figure~\ref{fig:EGORSE_MB100_best} displays the convergence plots of  \texttt{EGORSE PLS+Gaussian} for the three tested number of points in the initial DoE.
One can easily see that  \texttt{EGORSE PLS+Gaussian} with an initial DoE of $d$ points  provides the best performance in term of convergence speed and robustness.
in terms of evaluations number, the use of an initial DoE of $d$ points allows the algorithm to explore the domain in an interesting direction more quickly than a $2d$ points initial DoE.  
On the contrary, using an initial DoE of 5 points forces the algorithm to seek for the best direction for a long time.

To conclude, choosing the PLS and Gaussian dimension reduction methods with an initial DoE of $d$ points seems the most suitable to obtain the best performance of \texttt{EGORSE}.
Nevertheless, \texttt{EGORSE} capabilities have to be compared with HDBO algorithms to validate its usefulness as it is proposed in the following.

\section{Comparison with state-of-the-art HDBO methods}
\label{sec:num}
\subsection{BO algorithms and setup details}
\label{ssec:setup_all}

\texttt{EGORSE} is now compared to the following state-of-the-art algorithms:
\begin{itemize}
    \item \texttt{TuRBO}~\cite{erikssonScalableGlobalOptimization2019a}: a HDBO algorithm using trust regions to favor the exploitation of the DoE data.
    Tests are performed with the TuRBO\footnote{https://github.com/uber-research/TuRBO}~\cite{erikssonScalableGlobalOptimization2019a} Python toolbox.
    \item   \texttt{EGO-KPLS}~\cite{BouhlelEfficientglobaloptimization2018}: an HDBO method relying on the reduction of the number of GP hyper-parameters.
    This allows to speed up the GP building. 
    The SEGOMOE~\cite{bartoliAdaptiveModelingStrategy2019} Python toolbox is used.
    All the hyper-parameters of this algorithm are the default ones.
    The number of principal components for the KPLS model is set to two.
    \item  \texttt{RREMBO}~\cite{binoisChoiceLowdimensionalDomain2020}: a HDBO method using the random Gaussian transfer matrix to reduce the number of dimensions of the optimization problem through random embedding.
     \texttt{RREMBO}\footnote{https://github.com/mbinois/RRembo} implementation of this algorithm is used.
    The parameters are also set by default.
    \item   \texttt{HESBO}~\cite{nayebiFrameworkBayesianOptimization2019}: a HDBO algorithm using Hash tables to generate the transfer matrix and construct so called hashing-enhanced subspaces. 
    We use the \texttt{HESBO}\footnote{https://github.com/aminnayebi/HesBO} Python toolbox with the default parameters.
\end{itemize}
For \texttt{EGORSE}, the version showing the best performance in term of convergence speed and robustness in Section~\ref{sssec:impact} is selected, i.e.  \texttt{{\texttt{EGORSE}} PLS + Gaussian} with an initial DoE of $d$ points.
To achieve this comparison, $10$ optimizations for each problem and  for each studied method are completed to analyze the statistical behavior of these BO algorithms.
Because of the different strategies implemented in the previously introduced algorithms, a specific test plan must be adopted for each of them.
\begin{itemize}
    \item \texttt{EGORSE}: The test plan of Section~\ref{sssec:conv} is implemented.
    \item  \texttt{TuRBO}: Five trust regions are used with a maximum of $800$ evaluations of the objective function. Note that it is not possible to provide the same initial DoE used for \texttt{EGORSE} in  \texttt{TuRBO}. The number of points generated at the beginning of the algorithm is thus imposed to $d$. The EI acquisition function is chosen.
     \item  \texttt{EGO-KPLS}: The optimization is performed with a maximum evaluation number of $800$, with the initial DoE used in \texttt{EGORSE} and with the EI acquisition function. 
     \item  \texttt{RREMBO \& HESBO}: $20$ optimizations of $20d_e$ evaluations are performed for each of the initial DoE for  \texttt{RREMBO} and  \texttt{HESBO}. These $20$ optimizations are then concatenated and considered as a unique optimization process. The number of effective directions are imposed to $d_e=2$ and the acquisition function is EI.  \texttt{RREMBO} and  \texttt{HESBO} are equivalent to \texttt{EGORSE} without supervised dimension reduction method.
\end{itemize}
\subsection{Results analysis}
\label{ssec:bo_mb:res}

In this section, a comparison between \texttt{EGORSE} and the four studied algorithms is performed in term of robustness, convergence speed both in CPU time and in number of iterations.
Figure~\ref{fig:ALL} provides the iteration convergence plots, as introduced in Section~\ref{sssec:conv}, and the time convergence plots drawing the evolution of the means of the best discovered function values against the CPU time. 
\begin{figure}[!hbtp]
    \centering
    \subfloat[MB\_10 iteration convergence plot. \label{fig:ALL:mb10_nb}]{\includegraphics[width=0.50\textwidth]{Figures/CV/CV_MB_10_ALL_NBCall.pdf}}
    \subfloat[MB\_100 iteration convergence plot. \label{fig:ALL:mb100_nb}]{\includegraphics[width=0.50\textwidth]{Figures/CV/CV_MB_100_ALL_NBCall.pdf}} \\
    \subfloat[MB\_10 time convergence plot. \label{fig:ALL:mb10_time}]{\includegraphics[width=0.50\textwidth]{Figures/CV/CV_MB_10_ALL_Time.pdf}}
    \subfloat[MB\_100 time convergence plot. \label{fig:ALL:mb100_time}]{\includegraphics[width=0.50\textwidth]{Figures/CV/CV_MB_100_ALL_Time.pdf}}
    \caption{Iteration and time convergence plots for 5 HDBO algorithms on the MB\_10 and MB\_100 problems. The grey vertical line shows the size of the initial DoE.}
    \label{fig:ALL}
\end{figure}

Figure~\ref{fig:ALL:mb10_nb} shows that  \texttt{TuRBO} and  \texttt{EGO-KPLS} are converging the fastest and with a low standard deviation.
Moreover, the convergence plots of  \texttt{EGORSE},  \texttt{RREMBO} and  \texttt{HESBO} are hardly distinguishable. 
Figure~\ref{fig:ALL:mb100_nb} displays that  \texttt{EGO-KPLS} converges the fastest to the lowest values with a low standard deviation.
 \texttt{TuRBO} is also providing good performance even if it converges slower than  \texttt{EGO-KPLS}. 
Regarding the three methods using dimension reduction procedure,  \texttt{EGORSE} is converging to the lowest value with a relatively low standard deviation.
The good performance of  \texttt{EGO-KPLS} and  \texttt{TuRBO} is certainly due to the ability of these algorithms to search all over $\Omega$, which is not the case for other methods.
However, when the dimension of $\Omega$ increases, the ability to search all over $\Omega$ becomes a drawback. In fact, a complete search in $\Omega$ is intractable in time is this case.

Figures~\ref{fig:ALL:mb10_time} and~\ref{fig:ALL:mb100_time} depict the convergence CPU time necessary to obtain the regarded value.
First, the RREMBO, TuRBO and EGO-KPLS complete the optimization procedure in more than 8 hours on the MB\_100 problem against an hour on the MB\_10 problem.
This suggests that  \texttt{RREMBO},  \texttt{TuRBO} and  \texttt{EGO-KPLS} are intractable in time for larger problems.
Then, one can easily see that \texttt{EGORSE} is converging the fastest in CPU time than the other algorithms on the MB\_100 problem.
In fact, the computation time needed to find the enrichment point is much lower than the one for  \texttt{TuRBO},  \texttt{EGO-KPLS} and  \texttt{RREMBO}.
This was sought in the definition of the enrichment sub-problem introduced in Section~\ref{ssec:sub_prob}.
Finally,  \texttt{EGORSE} is converging to a lower value than  \texttt{HESBO} in a similar amount of time. 
Thus,  \texttt{EGORSE} seems more interesting to solve HDBO problems than the studied algorithms.
Note that only  \texttt{HESBO} and  \texttt{EGORSE} are able to perform an optimization procedure on HDBO problems.

To conclude this section, several sets of hyper-parameters of  \texttt{EGORSE} have been tested on two problems of dimension 10 and 100. 
The analysis of the obtained results has shown that  \texttt{EGORSE PLS+Gaussian} with an initial DoE of $d$ points is performing the best. 
A comparison of  \texttt{EGORSE} with HDBO algorithms has also been carried out. 
It has pointed out that  \texttt{TuRBO},  \texttt{EGO-KPLS} and  \texttt{RREMBO} are intractable in time for HDBO problems.
Furthermore,  \texttt{EGORSE} has appeared to be the most suitable to solve HDBO problem efficiently. 

\section{Evaluation of \texttt{EGORSE} on a high dimensional planning optimization}
\label{sec:rover}

In this section, the \texttt{EGORSE} algorithm is evaluated to find an optimal path planning problem using 600 design optimization variables. 
\subsection{Problem definition and implementation details}

The Rover\_600 path planning problems relies on the same idea than MB\_d problems except that the objective function is a adjustment of the Rover\_60~\cite{WangBatchedHighdimensionalBayesian2018} problem.
It consists on a robot routing from a starting point $x_{start}$ to a goal point $x_{goal}$ in a forest. 
The robot trajectory is a spline defined by 30 control points. 
These points, including the starting and the goal ones, are the design variables of the optimization problem that belong to $\Omega=[0,1]^{60}$.
The objective function is minimal when the robot follows the shortest trajectory without meeting a tree.
The minimum of the function is $f_{min} = -5$.
Figure~\ref{fig:rover_dom} gives an example of trajectory.
\begin{figure}[!htp]
    \centering
    \includegraphics[width=0.8\textwidth]{Figures/rover_domain.pdf}
    \caption{Example of a robot trajectory in a forest.}
    \label{fig:rover_dom}
\end{figure}

To increase the number of design variables, the problem is normalized in $\Omega=[-1,1]^{60}$, a random matrix $\A_d \in \mathbb{R}^{60  \times 600}$ is generated such that all $\x \in [-1,1]^d$, $\A_d \x = \bm{u} \in [-1,1]^{60}$.
An objective function Rover\_600, where $d=600$ is the number of design variables, is defined such that $\text{Rover\_600}(\x) = \text{Rover\_60}(\A_d\x)$.
Eventually, we solve the following optimization problem:
\begin{equation}
    \min\limits_{x \in [-1,1]^600} \text{Rover\_600}(\x) = \min\limits_{x \in [-1,1]^{60}} \text{Rover\_60}(\A_d\x).
\end{equation}

%\subsection{BO algorithms and setup details}
%\label{sssec:detail_Rover}
We note that, in our tests, \texttt{TuRBO},  \texttt{EGO-KPLS} and  \texttt{RREMBO} algorithms are not included anymore in this comparison as they are intractable in time for optimization problems with more than a hundred design variables (see Section~\ref{ssec:bo_mb:res}). So, only \texttt{EGORSE PLS+Gaussian} is compared to \texttt{HESBO}. The same test plan as in Section~\ref{ssec:setup_all} is used with $200$ optimizations (instead of $20$) of $20d_e$ evaluations.

\subsection{Result analysis}
Time and iteration convergence plots of  \texttt{EGORSE} and  \texttt{HESBO} are depicted in Figure~\ref{fig:Rover}.
\begin{figure}[!hbt]
    \centering
    \subfloat[Time convergence plot. \label{fig:Rover:Time}]{\includegraphics[width=0.50\textwidth]{Figures/CV/CV_Rover_600_ALL_Time.pdf}}
    \subfloat[Iteration convergence plot. \label{fig:Rover:nbCall}]{\includegraphics[width=0.50\textwidth]{Figures/CV/CV_Rover_600_ALL_NBCall.pdf}}
    \caption{CPU time convergence plots of  \texttt{HESBO, RREMBO, TuRBO, EGO-KPSL} on the Rover\_600 problem with an initial DoE of $d$ points. The grey vertical line shows the size of the initial DoE.}
    \label{fig:Rover}
\end{figure}
Figure~\ref{fig:Rover:nbCall} clearly shows that  \texttt{EGORSE} is converging fast to the lowest objective value with  a very low standard deviation. 
However, the obtained objective value is larger than the known optimal one (i.e. $f_{min}=-5$).
This is due to two main reasons that may be addressed in further research:
\begin{itemize}
    \item The number of effective directions used in  \texttt{EGORSE} (i.e. $d_e=2$) is much lower than the actual number of effective directions (i.e. $d_e=60$). The effective search space is not covering the space in which Rover\_600 is varying.
    \item The dimension reduction method PLS is global. The local variations of the function, in which the global optimum can be located, are thus deleted. Even if this problem is tackled by searching in randomly generated subspace,  \texttt{EGORSE} cannot provide better results. 
\end{itemize}
Figure~\ref{fig:Rover:Time} shows that  \texttt{HESBO} is performing the optimization procedure faster than  \texttt{EGORSE}.
In fact,  \texttt{HESBO} does not solve any quadratic problem at each iteration.
However, one can see that the time difference is not significant.

\section{Conclusion}
In this paper, we propose a novel approach that integrates conditional generative models with DAS for circuit generation. 
Our framework begins with the design of CircuitVQ, a circuit tokenizer trained using a Circuit AutoEncoder. 
Building on this, we develop CircuitAR, a masked autoregressive model that utilizes CircuitVQ as its tokenizer. 
CircuitAR can generate preliminary circuit structures directly from truth tables, which are then refined by DAS to produce functionally equivalent circuits. 
The scalability and capability emergence of CircuitAR highlights the potential of masked autoregressive modeling for circuit generation tasks, akin to the success of large models in language and vision domains. 
Extensive experiments demonstrate the superior performance of our method, underscoring its efficiency and effectiveness. 
This work bridges the gap between probabilistic generative models and precise circuit generation, offering a robust and automated solution for logic synthesis.

%\subsection{Lloyd-Max Algorithm}
\label{subsec:Lloyd-Max}
For a given quantization bitwidth $B$ and an operand $\bm{X}$, the Lloyd-Max algorithm finds $2^B$ quantization levels $\{\hat{x}_i\}_{i=1}^{2^B}$ such that quantizing $\bm{X}$ by rounding each scalar in $\bm{X}$ to the nearest quantization level minimizes the quantization MSE. 

The algorithm starts with an initial guess of quantization levels and then iteratively computes quantization thresholds $\{\tau_i\}_{i=1}^{2^B-1}$ and updates quantization levels $\{\hat{x}_i\}_{i=1}^{2^B}$. Specifically, at iteration $n$, thresholds are set to the midpoints of the previous iteration's levels:
\begin{align*}
    \tau_i^{(n)}=\frac{\hat{x}_i^{(n-1)}+\hat{x}_{i+1}^{(n-1)}}2 \text{ for } i=1\ldots 2^B-1
\end{align*}
Subsequently, the quantization levels are re-computed as conditional means of the data regions defined by the new thresholds:
\begin{align*}
    \hat{x}_i^{(n)}=\mathbb{E}\left[ \bm{X} \big| \bm{X}\in [\tau_{i-1}^{(n)},\tau_i^{(n)}] \right] \text{ for } i=1\ldots 2^B
\end{align*}
where to satisfy boundary conditions we have $\tau_0=-\infty$ and $\tau_{2^B}=\infty$. The algorithm iterates the above steps until convergence.

Figure \ref{fig:lm_quant} compares the quantization levels of a $7$-bit floating point (E3M3) quantizer (left) to a $7$-bit Lloyd-Max quantizer (right) when quantizing a layer of weights from the GPT3-126M model at a per-tensor granularity. As shown, the Lloyd-Max quantizer achieves substantially lower quantization MSE. Further, Table \ref{tab:FP7_vs_LM7} shows the superior perplexity achieved by Lloyd-Max quantizers for bitwidths of $7$, $6$ and $5$. The difference between the quantizers is clear at 5 bits, where per-tensor FP quantization incurs a drastic and unacceptable increase in perplexity, while Lloyd-Max quantization incurs a much smaller increase. Nevertheless, we note that even the optimal Lloyd-Max quantizer incurs a notable ($\sim 1.5$) increase in perplexity due to the coarse granularity of quantization. 

\begin{figure}[h]
  \centering
  \includegraphics[width=0.7\linewidth]{sections/figures/LM7_FP7.pdf}
  \caption{\small Quantization levels and the corresponding quantization MSE of Floating Point (left) vs Lloyd-Max (right) Quantizers for a layer of weights in the GPT3-126M model.}
  \label{fig:lm_quant}
\end{figure}

\begin{table}[h]\scriptsize
\begin{center}
\caption{\label{tab:FP7_vs_LM7} \small Comparing perplexity (lower is better) achieved by floating point quantizers and Lloyd-Max quantizers on a GPT3-126M model for the Wikitext-103 dataset.}
\begin{tabular}{c|cc|c}
\hline
 \multirow{2}{*}{\textbf{Bitwidth}} & \multicolumn{2}{|c|}{\textbf{Floating-Point Quantizer}} & \textbf{Lloyd-Max Quantizer} \\
 & Best Format & Wikitext-103 Perplexity & Wikitext-103 Perplexity \\
\hline
7 & E3M3 & 18.32 & 18.27 \\
6 & E3M2 & 19.07 & 18.51 \\
5 & E4M0 & 43.89 & 19.71 \\
\hline
\end{tabular}
\end{center}
\end{table}

\subsection{Proof of Local Optimality of LO-BCQ}
\label{subsec:lobcq_opt_proof}
For a given block $\bm{b}_j$, the quantization MSE during LO-BCQ can be empirically evaluated as $\frac{1}{L_b}\lVert \bm{b}_j- \bm{\hat{b}}_j\rVert^2_2$ where $\bm{\hat{b}}_j$ is computed from equation (\ref{eq:clustered_quantization_definition}) as $C_{f(\bm{b}_j)}(\bm{b}_j)$. Further, for a given block cluster $\mathcal{B}_i$, we compute the quantization MSE as $\frac{1}{|\mathcal{B}_{i}|}\sum_{\bm{b} \in \mathcal{B}_{i}} \frac{1}{L_b}\lVert \bm{b}- C_i^{(n)}(\bm{b})\rVert^2_2$. Therefore, at the end of iteration $n$, we evaluate the overall quantization MSE $J^{(n)}$ for a given operand $\bm{X}$ composed of $N_c$ block clusters as:
\begin{align*}
    \label{eq:mse_iter_n}
    J^{(n)} = \frac{1}{N_c} \sum_{i=1}^{N_c} \frac{1}{|\mathcal{B}_{i}^{(n)}|}\sum_{\bm{v} \in \mathcal{B}_{i}^{(n)}} \frac{1}{L_b}\lVert \bm{b}- B_i^{(n)}(\bm{b})\rVert^2_2
\end{align*}

At the end of iteration $n$, the codebooks are updated from $\mathcal{C}^{(n-1)}$ to $\mathcal{C}^{(n)}$. However, the mapping of a given vector $\bm{b}_j$ to quantizers $\mathcal{C}^{(n)}$ remains as  $f^{(n)}(\bm{b}_j)$. At the next iteration, during the vector clustering step, $f^{(n+1)}(\bm{b}_j)$ finds new mapping of $\bm{b}_j$ to updated codebooks $\mathcal{C}^{(n)}$ such that the quantization MSE over the candidate codebooks is minimized. Therefore, we obtain the following result for $\bm{b}_j$:
\begin{align*}
\frac{1}{L_b}\lVert \bm{b}_j - C_{f^{(n+1)}(\bm{b}_j)}^{(n)}(\bm{b}_j)\rVert^2_2 \le \frac{1}{L_b}\lVert \bm{b}_j - C_{f^{(n)}(\bm{b}_j)}^{(n)}(\bm{b}_j)\rVert^2_2
\end{align*}

That is, quantizing $\bm{b}_j$ at the end of the block clustering step of iteration $n+1$ results in lower quantization MSE compared to quantizing at the end of iteration $n$. Since this is true for all $\bm{b} \in \bm{X}$, we assert the following:
\begin{equation}
\begin{split}
\label{eq:mse_ineq_1}
    \tilde{J}^{(n+1)} &= \frac{1}{N_c} \sum_{i=1}^{N_c} \frac{1}{|\mathcal{B}_{i}^{(n+1)}|}\sum_{\bm{b} \in \mathcal{B}_{i}^{(n+1)}} \frac{1}{L_b}\lVert \bm{b} - C_i^{(n)}(b)\rVert^2_2 \le J^{(n)}
\end{split}
\end{equation}
where $\tilde{J}^{(n+1)}$ is the the quantization MSE after the vector clustering step at iteration $n+1$.

Next, during the codebook update step (\ref{eq:quantizers_update}) at iteration $n+1$, the per-cluster codebooks $\mathcal{C}^{(n)}$ are updated to $\mathcal{C}^{(n+1)}$ by invoking the Lloyd-Max algorithm \citep{Lloyd}. We know that for any given value distribution, the Lloyd-Max algorithm minimizes the quantization MSE. Therefore, for a given vector cluster $\mathcal{B}_i$ we obtain the following result:

\begin{equation}
    \frac{1}{|\mathcal{B}_{i}^{(n+1)}|}\sum_{\bm{b} \in \mathcal{B}_{i}^{(n+1)}} \frac{1}{L_b}\lVert \bm{b}- C_i^{(n+1)}(\bm{b})\rVert^2_2 \le \frac{1}{|\mathcal{B}_{i}^{(n+1)}|}\sum_{\bm{b} \in \mathcal{B}_{i}^{(n+1)}} \frac{1}{L_b}\lVert \bm{b}- C_i^{(n)}(\bm{b})\rVert^2_2
\end{equation}

The above equation states that quantizing the given block cluster $\mathcal{B}_i$ after updating the associated codebook from $C_i^{(n)}$ to $C_i^{(n+1)}$ results in lower quantization MSE. Since this is true for all the block clusters, we derive the following result: 
\begin{equation}
\begin{split}
\label{eq:mse_ineq_2}
     J^{(n+1)} &= \frac{1}{N_c} \sum_{i=1}^{N_c} \frac{1}{|\mathcal{B}_{i}^{(n+1)}|}\sum_{\bm{b} \in \mathcal{B}_{i}^{(n+1)}} \frac{1}{L_b}\lVert \bm{b}- C_i^{(n+1)}(\bm{b})\rVert^2_2  \le \tilde{J}^{(n+1)}   
\end{split}
\end{equation}

Following (\ref{eq:mse_ineq_1}) and (\ref{eq:mse_ineq_2}), we find that the quantization MSE is non-increasing for each iteration, that is, $J^{(1)} \ge J^{(2)} \ge J^{(3)} \ge \ldots \ge J^{(M)}$ where $M$ is the maximum number of iterations. 
%Therefore, we can say that if the algorithm converges, then it must be that it has converged to a local minimum. 
\hfill $\blacksquare$


\begin{figure}
    \begin{center}
    \includegraphics[width=0.5\textwidth]{sections//figures/mse_vs_iter.pdf}
    \end{center}
    \caption{\small NMSE vs iterations during LO-BCQ compared to other block quantization proposals}
    \label{fig:nmse_vs_iter}
\end{figure}

Figure \ref{fig:nmse_vs_iter} shows the empirical convergence of LO-BCQ across several block lengths and number of codebooks. Also, the MSE achieved by LO-BCQ is compared to baselines such as MXFP and VSQ. As shown, LO-BCQ converges to a lower MSE than the baselines. Further, we achieve better convergence for larger number of codebooks ($N_c$) and for a smaller block length ($L_b$), both of which increase the bitwidth of BCQ (see Eq \ref{eq:bitwidth_bcq}).


\subsection{Additional Accuracy Results}
%Table \ref{tab:lobcq_config} lists the various LOBCQ configurations and their corresponding bitwidths.
\begin{table}
\setlength{\tabcolsep}{4.75pt}
\begin{center}
\caption{\label{tab:lobcq_config} Various LO-BCQ configurations and their bitwidths.}
\begin{tabular}{|c||c|c|c|c||c|c||c|} 
\hline
 & \multicolumn{4}{|c||}{$L_b=8$} & \multicolumn{2}{|c||}{$L_b=4$} & $L_b=2$ \\
 \hline
 \backslashbox{$L_A$\kern-1em}{\kern-1em$N_c$} & 2 & 4 & 8 & 16 & 2 & 4 & 2 \\
 \hline
 64 & 4.25 & 4.375 & 4.5 & 4.625 & 4.375 & 4.625 & 4.625\\
 \hline
 32 & 4.375 & 4.5 & 4.625& 4.75 & 4.5 & 4.75 & 4.75 \\
 \hline
 16 & 4.625 & 4.75& 4.875 & 5 & 4.75 & 5 & 5 \\
 \hline
\end{tabular}
\end{center}
\end{table}

%\subsection{Perplexity achieved by various LO-BCQ configurations on Wikitext-103 dataset}

\begin{table} \centering
\begin{tabular}{|c||c|c|c|c||c|c||c|} 
\hline
 $L_b \rightarrow$& \multicolumn{4}{c||}{8} & \multicolumn{2}{c||}{4} & 2\\
 \hline
 \backslashbox{$L_A$\kern-1em}{\kern-1em$N_c$} & 2 & 4 & 8 & 16 & 2 & 4 & 2  \\
 %$N_c \rightarrow$ & 2 & 4 & 8 & 16 & 2 & 4 & 2 \\
 \hline
 \hline
 \multicolumn{8}{c}{GPT3-1.3B (FP32 PPL = 9.98)} \\ 
 \hline
 \hline
 64 & 10.40 & 10.23 & 10.17 & 10.15 &  10.28 & 10.18 & 10.19 \\
 \hline
 32 & 10.25 & 10.20 & 10.15 & 10.12 &  10.23 & 10.17 & 10.17 \\
 \hline
 16 & 10.22 & 10.16 & 10.10 & 10.09 &  10.21 & 10.14 & 10.16 \\
 \hline
  \hline
 \multicolumn{8}{c}{GPT3-8B (FP32 PPL = 7.38)} \\ 
 \hline
 \hline
 64 & 7.61 & 7.52 & 7.48 &  7.47 &  7.55 &  7.49 & 7.50 \\
 \hline
 32 & 7.52 & 7.50 & 7.46 &  7.45 &  7.52 &  7.48 & 7.48  \\
 \hline
 16 & 7.51 & 7.48 & 7.44 &  7.44 &  7.51 &  7.49 & 7.47  \\
 \hline
\end{tabular}
\caption{\label{tab:ppl_gpt3_abalation} Wikitext-103 perplexity across GPT3-1.3B and 8B models.}
\end{table}

\begin{table} \centering
\begin{tabular}{|c||c|c|c|c||} 
\hline
 $L_b \rightarrow$& \multicolumn{4}{c||}{8}\\
 \hline
 \backslashbox{$L_A$\kern-1em}{\kern-1em$N_c$} & 2 & 4 & 8 & 16 \\
 %$N_c \rightarrow$ & 2 & 4 & 8 & 16 & 2 & 4 & 2 \\
 \hline
 \hline
 \multicolumn{5}{|c|}{Llama2-7B (FP32 PPL = 5.06)} \\ 
 \hline
 \hline
 64 & 5.31 & 5.26 & 5.19 & 5.18  \\
 \hline
 32 & 5.23 & 5.25 & 5.18 & 5.15  \\
 \hline
 16 & 5.23 & 5.19 & 5.16 & 5.14  \\
 \hline
 \multicolumn{5}{|c|}{Nemotron4-15B (FP32 PPL = 5.87)} \\ 
 \hline
 \hline
 64  & 6.3 & 6.20 & 6.13 & 6.08  \\
 \hline
 32  & 6.24 & 6.12 & 6.07 & 6.03  \\
 \hline
 16  & 6.12 & 6.14 & 6.04 & 6.02  \\
 \hline
 \multicolumn{5}{|c|}{Nemotron4-340B (FP32 PPL = 3.48)} \\ 
 \hline
 \hline
 64 & 3.67 & 3.62 & 3.60 & 3.59 \\
 \hline
 32 & 3.63 & 3.61 & 3.59 & 3.56 \\
 \hline
 16 & 3.61 & 3.58 & 3.57 & 3.55 \\
 \hline
\end{tabular}
\caption{\label{tab:ppl_llama7B_nemo15B} Wikitext-103 perplexity compared to FP32 baseline in Llama2-7B and Nemotron4-15B, 340B models}
\end{table}

%\subsection{Perplexity achieved by various LO-BCQ configurations on MMLU dataset}


\begin{table} \centering
\begin{tabular}{|c||c|c|c|c||c|c|c|c|} 
\hline
 $L_b \rightarrow$& \multicolumn{4}{c||}{8} & \multicolumn{4}{c||}{8}\\
 \hline
 \backslashbox{$L_A$\kern-1em}{\kern-1em$N_c$} & 2 & 4 & 8 & 16 & 2 & 4 & 8 & 16  \\
 %$N_c \rightarrow$ & 2 & 4 & 8 & 16 & 2 & 4 & 2 \\
 \hline
 \hline
 \multicolumn{5}{|c|}{Llama2-7B (FP32 Accuracy = 45.8\%)} & \multicolumn{4}{|c|}{Llama2-70B (FP32 Accuracy = 69.12\%)} \\ 
 \hline
 \hline
 64 & 43.9 & 43.4 & 43.9 & 44.9 & 68.07 & 68.27 & 68.17 & 68.75 \\
 \hline
 32 & 44.5 & 43.8 & 44.9 & 44.5 & 68.37 & 68.51 & 68.35 & 68.27  \\
 \hline
 16 & 43.9 & 42.7 & 44.9 & 45 & 68.12 & 68.77 & 68.31 & 68.59  \\
 \hline
 \hline
 \multicolumn{5}{|c|}{GPT3-22B (FP32 Accuracy = 38.75\%)} & \multicolumn{4}{|c|}{Nemotron4-15B (FP32 Accuracy = 64.3\%)} \\ 
 \hline
 \hline
 64 & 36.71 & 38.85 & 38.13 & 38.92 & 63.17 & 62.36 & 63.72 & 64.09 \\
 \hline
 32 & 37.95 & 38.69 & 39.45 & 38.34 & 64.05 & 62.30 & 63.8 & 64.33  \\
 \hline
 16 & 38.88 & 38.80 & 38.31 & 38.92 & 63.22 & 63.51 & 63.93 & 64.43  \\
 \hline
\end{tabular}
\caption{\label{tab:mmlu_abalation} Accuracy on MMLU dataset across GPT3-22B, Llama2-7B, 70B and Nemotron4-15B models.}
\end{table}


%\subsection{Perplexity achieved by various LO-BCQ configurations on LM evaluation harness}

\begin{table} \centering
\begin{tabular}{|c||c|c|c|c||c|c|c|c|} 
\hline
 $L_b \rightarrow$& \multicolumn{4}{c||}{8} & \multicolumn{4}{c||}{8}\\
 \hline
 \backslashbox{$L_A$\kern-1em}{\kern-1em$N_c$} & 2 & 4 & 8 & 16 & 2 & 4 & 8 & 16  \\
 %$N_c \rightarrow$ & 2 & 4 & 8 & 16 & 2 & 4 & 2 \\
 \hline
 \hline
 \multicolumn{5}{|c|}{Race (FP32 Accuracy = 37.51\%)} & \multicolumn{4}{|c|}{Boolq (FP32 Accuracy = 64.62\%)} \\ 
 \hline
 \hline
 64 & 36.94 & 37.13 & 36.27 & 37.13 & 63.73 & 62.26 & 63.49 & 63.36 \\
 \hline
 32 & 37.03 & 36.36 & 36.08 & 37.03 & 62.54 & 63.51 & 63.49 & 63.55  \\
 \hline
 16 & 37.03 & 37.03 & 36.46 & 37.03 & 61.1 & 63.79 & 63.58 & 63.33  \\
 \hline
 \hline
 \multicolumn{5}{|c|}{Winogrande (FP32 Accuracy = 58.01\%)} & \multicolumn{4}{|c|}{Piqa (FP32 Accuracy = 74.21\%)} \\ 
 \hline
 \hline
 64 & 58.17 & 57.22 & 57.85 & 58.33 & 73.01 & 73.07 & 73.07 & 72.80 \\
 \hline
 32 & 59.12 & 58.09 & 57.85 & 58.41 & 73.01 & 73.94 & 72.74 & 73.18  \\
 \hline
 16 & 57.93 & 58.88 & 57.93 & 58.56 & 73.94 & 72.80 & 73.01 & 73.94  \\
 \hline
\end{tabular}
\caption{\label{tab:mmlu_abalation} Accuracy on LM evaluation harness tasks on GPT3-1.3B model.}
\end{table}

\begin{table} \centering
\begin{tabular}{|c||c|c|c|c||c|c|c|c|} 
\hline
 $L_b \rightarrow$& \multicolumn{4}{c||}{8} & \multicolumn{4}{c||}{8}\\
 \hline
 \backslashbox{$L_A$\kern-1em}{\kern-1em$N_c$} & 2 & 4 & 8 & 16 & 2 & 4 & 8 & 16  \\
 %$N_c \rightarrow$ & 2 & 4 & 8 & 16 & 2 & 4 & 2 \\
 \hline
 \hline
 \multicolumn{5}{|c|}{Race (FP32 Accuracy = 41.34\%)} & \multicolumn{4}{|c|}{Boolq (FP32 Accuracy = 68.32\%)} \\ 
 \hline
 \hline
 64 & 40.48 & 40.10 & 39.43 & 39.90 & 69.20 & 68.41 & 69.45 & 68.56 \\
 \hline
 32 & 39.52 & 39.52 & 40.77 & 39.62 & 68.32 & 67.43 & 68.17 & 69.30  \\
 \hline
 16 & 39.81 & 39.71 & 39.90 & 40.38 & 68.10 & 66.33 & 69.51 & 69.42  \\
 \hline
 \hline
 \multicolumn{5}{|c|}{Winogrande (FP32 Accuracy = 67.88\%)} & \multicolumn{4}{|c|}{Piqa (FP32 Accuracy = 78.78\%)} \\ 
 \hline
 \hline
 64 & 66.85 & 66.61 & 67.72 & 67.88 & 77.31 & 77.42 & 77.75 & 77.64 \\
 \hline
 32 & 67.25 & 67.72 & 67.72 & 67.00 & 77.31 & 77.04 & 77.80 & 77.37  \\
 \hline
 16 & 68.11 & 68.90 & 67.88 & 67.48 & 77.37 & 78.13 & 78.13 & 77.69  \\
 \hline
\end{tabular}
\caption{\label{tab:mmlu_abalation} Accuracy on LM evaluation harness tasks on GPT3-8B model.}
\end{table}

\begin{table} \centering
\begin{tabular}{|c||c|c|c|c||c|c|c|c|} 
\hline
 $L_b \rightarrow$& \multicolumn{4}{c||}{8} & \multicolumn{4}{c||}{8}\\
 \hline
 \backslashbox{$L_A$\kern-1em}{\kern-1em$N_c$} & 2 & 4 & 8 & 16 & 2 & 4 & 8 & 16  \\
 %$N_c \rightarrow$ & 2 & 4 & 8 & 16 & 2 & 4 & 2 \\
 \hline
 \hline
 \multicolumn{5}{|c|}{Race (FP32 Accuracy = 40.67\%)} & \multicolumn{4}{|c|}{Boolq (FP32 Accuracy = 76.54\%)} \\ 
 \hline
 \hline
 64 & 40.48 & 40.10 & 39.43 & 39.90 & 75.41 & 75.11 & 77.09 & 75.66 \\
 \hline
 32 & 39.52 & 39.52 & 40.77 & 39.62 & 76.02 & 76.02 & 75.96 & 75.35  \\
 \hline
 16 & 39.81 & 39.71 & 39.90 & 40.38 & 75.05 & 73.82 & 75.72 & 76.09  \\
 \hline
 \hline
 \multicolumn{5}{|c|}{Winogrande (FP32 Accuracy = 70.64\%)} & \multicolumn{4}{|c|}{Piqa (FP32 Accuracy = 79.16\%)} \\ 
 \hline
 \hline
 64 & 69.14 & 70.17 & 70.17 & 70.56 & 78.24 & 79.00 & 78.62 & 78.73 \\
 \hline
 32 & 70.96 & 69.69 & 71.27 & 69.30 & 78.56 & 79.49 & 79.16 & 78.89  \\
 \hline
 16 & 71.03 & 69.53 & 69.69 & 70.40 & 78.13 & 79.16 & 79.00 & 79.00  \\
 \hline
\end{tabular}
\caption{\label{tab:mmlu_abalation} Accuracy on LM evaluation harness tasks on GPT3-22B model.}
\end{table}

\begin{table} \centering
\begin{tabular}{|c||c|c|c|c||c|c|c|c|} 
\hline
 $L_b \rightarrow$& \multicolumn{4}{c||}{8} & \multicolumn{4}{c||}{8}\\
 \hline
 \backslashbox{$L_A$\kern-1em}{\kern-1em$N_c$} & 2 & 4 & 8 & 16 & 2 & 4 & 8 & 16  \\
 %$N_c \rightarrow$ & 2 & 4 & 8 & 16 & 2 & 4 & 2 \\
 \hline
 \hline
 \multicolumn{5}{|c|}{Race (FP32 Accuracy = 44.4\%)} & \multicolumn{4}{|c|}{Boolq (FP32 Accuracy = 79.29\%)} \\ 
 \hline
 \hline
 64 & 42.49 & 42.51 & 42.58 & 43.45 & 77.58 & 77.37 & 77.43 & 78.1 \\
 \hline
 32 & 43.35 & 42.49 & 43.64 & 43.73 & 77.86 & 75.32 & 77.28 & 77.86  \\
 \hline
 16 & 44.21 & 44.21 & 43.64 & 42.97 & 78.65 & 77 & 76.94 & 77.98  \\
 \hline
 \hline
 \multicolumn{5}{|c|}{Winogrande (FP32 Accuracy = 69.38\%)} & \multicolumn{4}{|c|}{Piqa (FP32 Accuracy = 78.07\%)} \\ 
 \hline
 \hline
 64 & 68.9 & 68.43 & 69.77 & 68.19 & 77.09 & 76.82 & 77.09 & 77.86 \\
 \hline
 32 & 69.38 & 68.51 & 68.82 & 68.90 & 78.07 & 76.71 & 78.07 & 77.86  \\
 \hline
 16 & 69.53 & 67.09 & 69.38 & 68.90 & 77.37 & 77.8 & 77.91 & 77.69  \\
 \hline
\end{tabular}
\caption{\label{tab:mmlu_abalation} Accuracy on LM evaluation harness tasks on Llama2-7B model.}
\end{table}

\begin{table} \centering
\begin{tabular}{|c||c|c|c|c||c|c|c|c|} 
\hline
 $L_b \rightarrow$& \multicolumn{4}{c||}{8} & \multicolumn{4}{c||}{8}\\
 \hline
 \backslashbox{$L_A$\kern-1em}{\kern-1em$N_c$} & 2 & 4 & 8 & 16 & 2 & 4 & 8 & 16  \\
 %$N_c \rightarrow$ & 2 & 4 & 8 & 16 & 2 & 4 & 2 \\
 \hline
 \hline
 \multicolumn{5}{|c|}{Race (FP32 Accuracy = 48.8\%)} & \multicolumn{4}{|c|}{Boolq (FP32 Accuracy = 85.23\%)} \\ 
 \hline
 \hline
 64 & 49.00 & 49.00 & 49.28 & 48.71 & 82.82 & 84.28 & 84.03 & 84.25 \\
 \hline
 32 & 49.57 & 48.52 & 48.33 & 49.28 & 83.85 & 84.46 & 84.31 & 84.93  \\
 \hline
 16 & 49.85 & 49.09 & 49.28 & 48.99 & 85.11 & 84.46 & 84.61 & 83.94  \\
 \hline
 \hline
 \multicolumn{5}{|c|}{Winogrande (FP32 Accuracy = 79.95\%)} & \multicolumn{4}{|c|}{Piqa (FP32 Accuracy = 81.56\%)} \\ 
 \hline
 \hline
 64 & 78.77 & 78.45 & 78.37 & 79.16 & 81.45 & 80.69 & 81.45 & 81.5 \\
 \hline
 32 & 78.45 & 79.01 & 78.69 & 80.66 & 81.56 & 80.58 & 81.18 & 81.34  \\
 \hline
 16 & 79.95 & 79.56 & 79.79 & 79.72 & 81.28 & 81.66 & 81.28 & 80.96  \\
 \hline
\end{tabular}
\caption{\label{tab:mmlu_abalation} Accuracy on LM evaluation harness tasks on Llama2-70B model.}
\end{table}

%\section{MSE Studies}
%\textcolor{red}{TODO}


\subsection{Number Formats and Quantization Method}
\label{subsec:numFormats_quantMethod}
\subsubsection{Integer Format}
An $n$-bit signed integer (INT) is typically represented with a 2s-complement format \citep{yao2022zeroquant,xiao2023smoothquant,dai2021vsq}, where the most significant bit denotes the sign.

\subsubsection{Floating Point Format}
An $n$-bit signed floating point (FP) number $x$ comprises of a 1-bit sign ($x_{\mathrm{sign}}$), $B_m$-bit mantissa ($x_{\mathrm{mant}}$) and $B_e$-bit exponent ($x_{\mathrm{exp}}$) such that $B_m+B_e=n-1$. The associated constant exponent bias ($E_{\mathrm{bias}}$) is computed as $(2^{{B_e}-1}-1)$. We denote this format as $E_{B_e}M_{B_m}$.  

\subsubsection{Quantization Scheme}
\label{subsec:quant_method}
A quantization scheme dictates how a given unquantized tensor is converted to its quantized representation. We consider FP formats for the purpose of illustration. Given an unquantized tensor $\bm{X}$ and an FP format $E_{B_e}M_{B_m}$, we first, we compute the quantization scale factor $s_X$ that maps the maximum absolute value of $\bm{X}$ to the maximum quantization level of the $E_{B_e}M_{B_m}$ format as follows:
\begin{align}
\label{eq:sf}
    s_X = \frac{\mathrm{max}(|\bm{X}|)}{\mathrm{max}(E_{B_e}M_{B_m})}
\end{align}
In the above equation, $|\cdot|$ denotes the absolute value function.

Next, we scale $\bm{X}$ by $s_X$ and quantize it to $\hat{\bm{X}}$ by rounding it to the nearest quantization level of $E_{B_e}M_{B_m}$ as:

\begin{align}
\label{eq:tensor_quant}
    \hat{\bm{X}} = \text{round-to-nearest}\left(\frac{\bm{X}}{s_X}, E_{B_e}M_{B_m}\right)
\end{align}

We perform dynamic max-scaled quantization \citep{wu2020integer}, where the scale factor $s$ for activations is dynamically computed during runtime.

\subsection{Vector Scaled Quantization}
\begin{wrapfigure}{r}{0.35\linewidth}
  \centering
  \includegraphics[width=\linewidth]{sections/figures/vsquant.jpg}
  \caption{\small Vectorwise decomposition for per-vector scaled quantization (VSQ \citep{dai2021vsq}).}
  \label{fig:vsquant}
\end{wrapfigure}
During VSQ \citep{dai2021vsq}, the operand tensors are decomposed into 1D vectors in a hardware friendly manner as shown in Figure \ref{fig:vsquant}. Since the decomposed tensors are used as operands in matrix multiplications during inference, it is beneficial to perform this decomposition along the reduction dimension of the multiplication. The vectorwise quantization is performed similar to tensorwise quantization described in Equations \ref{eq:sf} and \ref{eq:tensor_quant}, where a scale factor $s_v$ is required for each vector $\bm{v}$ that maps the maximum absolute value of that vector to the maximum quantization level. While smaller vector lengths can lead to larger accuracy gains, the associated memory and computational overheads due to the per-vector scale factors increases. To alleviate these overheads, VSQ \citep{dai2021vsq} proposed a second level quantization of the per-vector scale factors to unsigned integers, while MX \citep{rouhani2023shared} quantizes them to integer powers of 2 (denoted as $2^{INT}$).

\subsubsection{MX Format}
The MX format proposed in \citep{rouhani2023microscaling} introduces the concept of sub-block shifting. For every two scalar elements of $b$-bits each, there is a shared exponent bit. The value of this exponent bit is determined through an empirical analysis that targets minimizing quantization MSE. We note that the FP format $E_{1}M_{b}$ is strictly better than MX from an accuracy perspective since it allocates a dedicated exponent bit to each scalar as opposed to sharing it across two scalars. Therefore, we conservatively bound the accuracy of a $b+2$-bit signed MX format with that of a $E_{1}M_{b}$ format in our comparisons. For instance, we use E1M2 format as a proxy for MX4.

\begin{figure}
    \centering
    \includegraphics[width=1\linewidth]{sections//figures/BlockFormats.pdf}
    \caption{\small Comparing LO-BCQ to MX format.}
    \label{fig:block_formats}
\end{figure}

Figure \ref{fig:block_formats} compares our $4$-bit LO-BCQ block format to MX \citep{rouhani2023microscaling}. As shown, both LO-BCQ and MX decompose a given operand tensor into block arrays and each block array into blocks. Similar to MX, we find that per-block quantization ($L_b < L_A$) leads to better accuracy due to increased flexibility. While MX achieves this through per-block $1$-bit micro-scales, we associate a dedicated codebook to each block through a per-block codebook selector. Further, MX quantizes the per-block array scale-factor to E8M0 format without per-tensor scaling. In contrast during LO-BCQ, we find that per-tensor scaling combined with quantization of per-block array scale-factor to E4M3 format results in superior inference accuracy across models. 



\section*{Acknowledgments}
This work is part of the activities of ONERA - ISAE - ENAC joint research group.
The authors would like to express their special thanks to Nina Moello for her test on the implementation of the log-likelihood gradient and Hessian in the opensource Python toolbox SMT\footnote{https://github.com/SMTorg/smt} and for her implementation of the HESBO algorithm in our computer environment.

\bibliography{main}

\end{document}
