\documentclass{article}
\usepackage{fullpage}

%\usepackage{UFrecherche}
\newcommand{\dataset}{{\cal D}}
\newcommand{\fracpartial}[2]{\frac{\partial #1}{\partial  #2}}

%\usepackage[export]{adjustbox}
\usepackage{xcolor}
\usepackage{tikz}
\usepackage{subcaption}
\usepackage{mathtools}

\usepackage{graphicx,amssymb,amsmath,amsthm}
\usepackage{lineno} 

\newtheorem{theorem}{Theorem}
\newtheorem{lemma}[theorem]{Lemma}
\newtheorem{corollary}[theorem]{Corollary}
\newtheorem{statement}[theorem]{Statement}
\newtheorem{question}[theorem]{Question}
\newtheorem{problem}[theorem]{Problem}
\newtheorem{definition}{Definition}
\newtheorem{property}{Property}
\newtheorem{proposition}[theorem]{Proposition}

\newtheorem{observation}{Observation}
\newtheorem{remark}{Remark}

\newcommand{\pablo}[1]{{\color{blue}{\bf Pablo:} #1}}

\newcommand{\miguel}[1]{{\color{olive}{\bf MA:} #1}}

\newcommand{\ruy}[1]{{\color{red}{\bf Ruy:} #1}}

\newcommand{\nes}[1]{{\color{violet}{\bf Nestaly:} #1}}

\newcommand{\higes}[1]{{\color{orange}{\bf Higes:} #1}}

%\linenumbers
% Heading arguments are {Semester}{year}{Objective}{Date}{}{authors}
%\ufheading{Math PhD}{2022-2023}{Problem Pool}{30.Nov.2022}{Research - 22/23}{M.A. Pérez-Cutiño}

% Short headings should be running head and authors last names

%\ShortHeadings{Pool of problems}{M.A. Pérez-Cutiño, J. Valverde and J.M. Díaz-Bañez}
%\firstpageno{1}

\begin{document}

\title{The Euclidean $3$-Matching Problem is NP-hard}

\author{ José-Miguel Díaz-Báñez\thanks{Department of Applied Mathematics, University of Seville, SPAIN. \texttt{dbanez@us.es}} \and 
Ruy Fabila-Monroy\thanks{Departamento de Matem\'aticas, Cinvestav, MEXICO. \texttt{ruyfabila@math.cinvestav.edu.mx}}. \and 
José-Manuel Higes-López\footnotemark[1] \thanks{ \texttt{jhiges@us.es}} \and 
Nestaly Marín\thanks{Centro de Investigaci\'on Cient\'ifica y de Educaci\'on Superior de Ensenada, Baja California, MEXICO. {\tt nestaly@cicese.mx}} \and 
Miguel-Angel Peréz-Cutiño\footnotemark[1] \thanks{\texttt{migpercut@alum.us.es}} \and
Pablo Pérez-Lantero\thanks{Universidad de Santiago de Chile (USACH), Facultad de Ciencia, Departamento de Matemática y Ciencia de la Computación, Chile. \texttt{pablo.perez.l@usach.cl.}}
}



\maketitle

\begin{abstract}
A $3$-matching of a graph is a partition of its vertices into $3$-element subsets; this generalizes the idea of matchings in graphs. Let $G$ be
a complete edge-weighted graph; to each $3$-set of vertices of $G$ assign the sum of its minimum spanning tree as its weight. 
The $3$-Matching Problem is the combinatorial optimization problem of finding a minimum weight $3$-matching  of $G$, where the weight of a $3$-matching is the sum of the weights of its $3$-sets.
While the $3$-Matching Problem is known to be NP-hard,
the complexity of its Euclidean variant remains open. In the Euclidean $3$-Matching Problem, the vertices of $G$ are a set of points in the plane,
and the weight of an edge is given by the distance between its endpoints.
In this paper we show that the Euclidean $3$-Matching Problem is NP-hard.
\end{abstract}


\section{Introduction}



The \emph{Euclidean Minimum Weight Perfect Matching Problem} (E2-MP)~\cite{hougardy2024fast,vaidya1988geometry} is the algorithmic problem
of finding a minimum weight perfect matching of a set of points in the plane, where the weight of a matching is the sum of the lengths
of its edges. The \emph{Euclidean 3-Matching Problem} (E3-MP) is a generalization of E2-MP, where we now partition the point set into $3$-sets (sets of three elements).
This partition is called a $3$-matching. The weight of a $3$-set, $T$, denoted by $w(T)$, is the weight of its 
Euclidean minimum spanning tree(the sum of the two shortest distances among its three pairs of points),
the weight of a $3$-matching is the sum of the weights of its $3$-sets.

%%

E$3$-MP is the geometric version of the more general \emph{$3$-Matching Problem} ($3$-MP). 
In the $3$-MP, a complete edge-weighted graph of $3\ell$ vertices is given and the objective is to partition its vertices into $\ell$ sets of three vertices each, so that the sum of the weights of the minimum spanning tree of each set is minimized. Although $3$-MP has been proved to be NP-hard in general, even for unit distance graphs~\cite{johnsson1998euclidean}, the computational complexity of E$3$-MP remains an open question.

The E$3$-MP is phrased as a decision problem as follows:


\vspace{.5cm}

\textbf{The Euclidean 3-Matching Problem (E3-MP) }

\emph{Given a set $S$ of $3n$ points in the plane and a parameter $W >0$. Does there exists a partition of $S$ into disjoint $3$-sets $\{T_1,T_2,\dots,T_{n}\}$ such that \[\sum_{i=1}^{n} w(T_i) \le W?\] }

\vspace{.5cm}

Applications of E3-MP and 3-MP include task assignment (assigning groups of three workers or machines to optimize efficiency), clustering and data analysis (grouping data points into sets of three based on similarity), network design (creating efficient tripartite structures in networks), and logistics and routing (assigning delivery routes in clusters of three locations). Descriptions of such real-world applications can be found in the literature~\cite{crama2012production,johnsson1996determining}.
%

A proof of the NP-hardness of the Euclidean version would provide a strong justification for the use of heuristics and approximation algorithms.
Johnsson et al.~\cite{johnsson1998euclidean} provided six lower bounds for the solution of the E$3$-MP and presented experimental results on several heuristics. The same authors later presented 
an exact solution for this problem with empirical performance tests~\cite{magyar1999exact}. Castro Campos et al.~\cite{castro2017integer} introduced three new integer programming models and presented specialized heuristics for the problem, comparing their solutions and execution times against exact models.

%%

In this paper, we prove that the E$3$-MP is NP-hard using a reduction from a specific version of the 3-SAT problem.
Note that proving NP-completeness would require showing that verifying a solution can be
done in polynomial time. Since we are dealing with Euclidean distances, this is related to the \emph{Sum of Square Roots Problem} (SSRP)\footnote{The SSRP asks whether, given positive integers $a_1,\dots,a_n$ and an integer $t$, the inequality $\sum_{i=1}^n \sqrt{a_i} \le t$ holds. The difficulty arises in that the square roots might have to be evaluated to a very large precision.},
the complexity of which is an open problem.
We now provide an explanation of the version of 3-SAT used in our reduction.



\section{Cubic Planar Monotone 1-in-3 SAT}

Let $\psi$ be an instance of $3$-SAT.
If  every variable of $\psi$ appears in exactly $3$ clauses, we say that $\psi$ is \emph{cubic}. Let $G_\psi$ be the bipartite graph where on one side of the partition we have a vertex for each variable of $\psi$ and on the other side we have a vertex for each clause; two vertices are adjacent in $G_\psi$ if and only if the variable vertex appears in the clause vertex. We say that $\psi$ is \emph{planar} if $G_\psi$ is planar. We say that $\psi$ is \emph{monotone} if in each clause, either all literals are positive or all literals are negative (we call these \emph{positive clauses} and \emph{negative clauses}, respectively). Finally, $1$-in-$3$-SAT is the problem of finding an assignment of truth values to the variables of $\psi$ so that in every clause exactly one of its three literals is satisfied. 
Moore and Robson showed that cubic planar monotone $1$-in-$3$-SAT is NP-complete~\cite{hard_tiling}. In what follows, assume that $\psi$ is an instance of cubic planar monotone $1$-in-$3$-SAT.






\begin{figure}
 	\centering
 	\includegraphics[width=0.5\textwidth]{figures/grid_embedding_2.pdf}
 	\caption{The graph $G_\psi$, and a grid embedding $D_\psi$, with respect to the formula
 	$\psi:=\protect \overbrace{(x_1 \lor x_2 \lor x_4)}^{C_1} \land (  \protect \overbrace{\overline{x_1} \lor \overline{x_2} \lor \overline{x_3})}^{C_2} \land
 	\protect \overbrace{(x_2 \lor x_3 \lor x_4)}^{C_3} \land \protect \overbrace{(\overline{x_1} \lor \overline{x_3} \lor \overline{x_4})}^{C_4}$ }\label{fig:grid_embedding}
 \end{figure}

\subsection*{Grid Embeddings}

A \emph{planar embedding} of a graph $G$ is a drawing of the graph in the plane where: vertices are mapped to
distinct points;  edges are drawn as simple curves joining their respective endpoints; edges only pass
through their endpoints; and  two edges do not intersect in their interior. A planar embedding of $G$
defines a cyclic order on the neighbours of every one of its vertices, by considering the clockwise order of the edges
as they appear around each vertex. These orders of the neighbours of each vertex is known as a \emph{rotation system}; it is used to represent 
the embedding since it is invariant under homeomorphisms of the plane.  An alternative way to represent planar embeddings is to give, for each face of the embedding, the circular sequence of edges encountered when walking clockwise around the boundary of that face. This is called a \emph{planar representation}. It is straightforward to obtain a planar representation from a rotation system, and vice versa.
Chiba, Nishizeki and Abe~\cite{linear_embedding} gave a linear-time algorithm that given a planar graph produces the rotation system of a planar embedding of the graph. 


A \emph{grid embedding} of $G$ is a
planar embedding such that the vertices are mapped to points with integer coordinates and the edges
are drawn as paths following the horizontal and vertical grid lines with integer coordinates. Given a planar graph $G$ on $n$ vertices and a planar representation $P$, Tamassia~\cite{bends} gave an $O(n^2\log n)$-time algorithm that computes a grid embedding of $G$ that realizes $P$ --- that is, the faces in the grid embedding correspond to the face descriptions in $P$. His algorithm in addition minimizes the total number of bends
on the edges of the grid embedding. In what follows, we assume that we have obtained in polynomial time
a grid embedding, $D_\psi$, of $G_\psi$. See Figure~\ref{fig:grid_embedding} for an example of a formula $\psi$, its graph $G_\psi$, and a grid embedding $D_\psi$ of $G_\psi$.




\section{Reduction to the Euclidean 3-matching problem}

We construct in polynomial time a set $S_\psi$ of $3m$ grid points, such that $S_\psi$ has
a Euclidean $3$-matching of weight equal to $2m$ if and only if $\psi$ is satisfiable.
A key observation is that
since $S_\psi$ is a set of grid points, the distance between any two of its points is at
least one. Therefore, every $3$-set of $S_\psi$ has weight at least two. Thus, in a $3$-matching
of $S_\psi$ of weight equal to $2m$, every $3$-set must have weight equal to exactly two.
We now describe the gadgets used in our construction.
Due to the complexity of the gadgets, we constantly refer to the figures. 
\begin{figure}[ht]
 	\centering
 	\includegraphics[width=.4\textwidth]{figures/var_gadget_2.pdf}
 	\caption{The variable gadget}\label{fig:var_gadget}
 \end{figure}

The variable gadget, which encodes the TRUE/FALSE state of each boolean variable in our reduction, is shown in Figure~\ref{fig:var_gadget} in a $16 \times 16$ grid.
The points of the gadget are painted black.
The grey points lie outside the gadget. 
The blue edges are forced in the sense that their endpoints must
belong to the same $3$-set. 
The $3$-sets of points joined by paths of length $2$ are the $3$-sets obtained when setting the variable to TRUE; the $3$-sets enclosed by trominoes are the $3$-sets obtained when setting the variable to FALSE. 
A simple case analysis shows that any other $3$-matching different from these two TRUE/FALSE configurations contains a $3$-set of weight strictly greater than two.
We refer to the points by their coordinates. Consider the point at $(8,8)$. In a $3$-matching in which every $3$-set has weight equal to two, this point must be matched either with $\{(9,8),(10,8)\}$, 
$\{(8,7),(8,9)\}$, $\{(8,7),(9,8)\}$, $\{(8,9),(9,8)\}$, $\{(8,7),(8,6)\}$ or $\{(8,9),(8,10)\}$.
If $(8,8)$ is matched to $\{(9,8),(10,8)\}$, then the rest of the $3$-sets of the variable gadget are forced and this corresponds to the FALSE configuration shown in Figure~\ref{fig:var_gadget}.
If $(8,8)$ is matched to $\{(8,7),(8,9)\}$, then the rest of the $3$-sets of the variable gadget are forced and this corresponds to the TRUE configuration shown in Figure~\ref{fig:var_gadget}.
If $(8,8)$ is matched to $\{(8,7),(9,8)\}$, then $(10,8)$, $(11,8)$, and $(12,8)$ must be in the same $3$-set; this implies that $(12,9)$
is in a $3$-set of weight greater than two.
By the same argument, $(8,8)$ cannot be matched to $\{(8,9),(9,8)\}$. If $(8,8)$ is matched to $\{(8,7),(8,6)\}$, then $(8,5)$, $(8,4)$ and $(8,3)$ must be in the same $3$-set;
this implies that $(7,3)$ is in a $3$-set of weight greater than two. By a similar argument $(8,8)$ cannot be matched to $\{(8,9),(8,10)\}$.
% 

%%
The variable gadgets propagate their value via \emph{wire gadgets}; these are sequences of rectangular trominoes $t_1,\dots,t_{s}$ enclosing $3$-sets of $S$ such that $t_1$ contains a point adjacent to the variable gadget (i.e., one of the grey points shown in Figure~\ref{fig:var_gadget}).
When the corresponding variable is set to TRUE, this adjacent point is matched to two points inside the variable gadget, triggering a propagation where, for all $i = 1,\dots,s-1$, two points inside $t_i$ must be matched with a point of $t_{i+1}$, effectively transmitting the TRUE value along the wire.
When the variable is FALSE, the $3$-matching simply maintains the $3$-sets enclosed in each tromino of the wire gadget, preserving the FALSE value

%%






  \begin{figure}[h]
 	\centering
 	\includegraphics[width=1.0 \textwidth]{figures/clauses.pdf}
 	\caption{Clause gadgets}\label{fig:clause_gadget}
 \end{figure}

 

The clause gadgets, which encode the SAT clauses in our reduction, are shown in Figure~\ref{fig:clause_gadget}: to the left we have the \emph{positive clause gadget}
and to the right the \emph{negative clause gadget}. Each one fits in an $18\times 18$ grid and enforces the clause constraints through careful arrangement of points and trominoes.
\begin{itemize}
    \item In the positive clause gadget there is exactly one point that is not enclosed in a tromino. In order to match this point
     in a $3$-set of weight equal to $2$, one of the wires must carry a TRUE value. Moreover, if other wire is also carrying a TRUE value, then
     the last tromino of the wire gadget must have two points that cannot be matched to a third point to produce a $3$-set of weight equal to $2$.
     Thus, if there is a $3$-matching of $S$ of weight $2m$, then exactly one of the three wires carries a TRUE value. In Figure~\ref{fig:clause_gadget} (left)
     the red edges depict the wire carrying the TRUE value.
    

    \item In the negative clause gadget there are exactly five points that are not enclosed in a tromino. In order to have a matching of $S$ of weight 
    equal to $2m$, two of these points must be matched with two wires, each one carrying a TRUE value. The three remaining points can be matched together
    in a $3$-set.  The remaining wire must carry a FALSE value; otherwise
    there would be exactly two points left, which cannot be matched in a $3$-set of weight equal to $2$.
    Thus, if there is a $3$-matching of $S$ of weight $2m$, then exactly one of the three wires carries a FALSE value.
    In Figure~\ref{fig:clause_gadget} (right)
     the red edges represent the two wires carrying a TRUE value.
\end{itemize}
These gadget properties ensure that the clause constraints of the original SAT formula are satisfied: positive clauses require exactly one TRUE literal, while negative clauses require exactly one FALSE literal.
%

To complete our reduction, we use the grid embedding $D_\psi$ as a template. First, we refine the grid by dividing each unit square into an $18 \times 18$ grid of smaller squares, transforming each original edge into a path of $18$ consecutive edges. We replace each variable vertex $x_i$ in $D_\psi$, with a $16 \times 16$ square region,  removing internal edge portions while preserving external connections. Within each such region, we place a variable gadget, rotated to align its three interface points with the preserved edge portions from $D_\psi$.
We similarly replace each clause vertex $C_j$ with an $18 \times 18$ square region containing the appropriate clause gadget. The remaining edge segments become wire gadgets through a systematic partitioning into consecutive triads of points, each enclosed in a rectangular tromino.
Figure~\ref{fig:cons} shows this construction for our sample formula and its grid embedding from Figure~\ref{fig:grid_embedding}. By our gadget properties, a solution to $\psi$ exists if and only if $S_\psi$ has a $3$-matching of weight $2m$, proving the NP-hardness of the Euclidean $3$-Matching Problem.

%






\begin{figure}
 	\centering
 	\includegraphics[width=1.0\textwidth]{figures/cons.pdf}
 	\caption{The grid embedding $D_\psi$ associated to the formula
 	$\psi:=\protect \overbrace{(x_1 \lor x_2 \lor x_4)}^{C_1} \land (  \protect \overbrace{\overline{x_1} \lor \overline{x_2} \lor \overline{x_3})}^{C_2} \land
 	\protect \overbrace{(x_2 \lor x_3 \lor x_4)}^{C_3} \land \protect \overbrace{(\overline{x_1} \lor \overline{x_3} \lor \overline{x_4})}^{C_4}$ }\label{fig:cons}
 \end{figure}







\section{Conclusions}
%%

We have proved that the 3-Matching Problem remains NP-hard in its Euclidean version, resolving a long-standing open question in the literature. Our reduction technique from Cubic Monotone 1-in-3 SAT appears extensible to the $k$-Matching Problem for every $k>3$. While such constructions would become more intricate, they can build upon the core ideas we developed for $k=3$.

The development of approximation algorithms remains a promising research direction. Despite the application of various heuristic techniques to the general 3-Matching Problem—including branch-and-bound, greedy algorithms, local search, simulated annealing, and genetic algorithms—no approximation algorithms with provable guarantees are currently known. We believe that exploiting the geometric properties inherent in the Euclidean version could yield crucial insights for developing such algorithms.



\section*{Acknowledgments}
This work began during the 4th Reunion of Optimization, Mathematics, and Algorithms Workshop (\emph{ROMA}). We thank the participants for fruitful scientific discussions.
José-Miguel Díaz-Báñez, José-Manuel Higes and Miguel-Angel Pérez-Cutiño were partially supported by grants PID2020-114154RB-I00 and TED2021-129182B-19
I00 funded by MCIN/AEI/10.13039/501100011033 and the European Union20
NextGenerationEU/PRTR. Ruy Fabila-Monroy was partially supported by 
CONACYT FORDECYT-PRONACES/39570/2020.
Nestaly~Mar\'in was partially supported by UNAM Postdoctoral Program (POSDOC).

\bibliography{references}
\bibliographystyle{abbrv}

\end{document}
