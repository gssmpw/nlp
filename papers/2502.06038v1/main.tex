\documentclass{article}
\pdfoutput=1
\usepackage{Custom}
%\usepackage{Custom2}
\usepackage{icml2025/fancyhdr}
\usepackage{icml2025/algorithm}
\usepackage{icml2025/algorithmic}
\usepackage[accepted]{icml2025/icml2025}
\usepackage{listings}
\usepackage{multicol}
\usepackage{authblk}
\usepackage{multirow}
\bibliographystyle{icml2025/icml2025}


%\date{today}
\title{\bfseries\Large
    Provably Overwhelming Transformer Models with Designed Inputs
}

\renewcommand\Affilfont{\fontsize{9}{10.8}\itshape}

%\author[1, 2]{Lev Stambler{\href{mailto:levstamb@umd.edu}{levstamb@umd.edu}}}
%\author[1, 2]{Seyed Sajjad Nezhadi}
%\author[1, 2, 3]{Matthew Coudron}
%
%\affil[1]{Joint Center for Quantum Information and Computer Science, University of Maryland}
%\affil[2]{Department of Computer Science, University of Maryland}
%% \affil[3]{Neon Tetra LLC}
%\affil[3]{National Institute of Standards and Technology}
% If you read this, I hope that you are having a nice day!

\newcommand{\mnote}[1]{{\highlightname{Matt C.}{#1}{blue}}}
\newcommand{\snote}[1]{{\highlightname{Sajjad}{#1}{red}}}
\newcommand{\mkd}[1]{{\highlightname{Matt K}{#1}{orange}}}
\newcommand{\lev}[1]{{\highlightname{Lev}{#1}{purple}}}
\newcommand{\claude}[1]{{\highlightname{Claude}{#1}{green}}}

 %\renewcommand{\mnote}[1]{}
 %\renewcommand{\snote}[1]{}
 %\renewcommand{\mkd}[1]{}
 %\renewcommand{\lev}[1]{}
 %\renewcommand{\claude}[1]{}



% Worst-Case Over-Squashing Proofs for Transformer Models
% Provably Overwhelming Transformer Models with Designed Inputs

\icmltitlerunning{
    Provably Overwhelming Transformer Models with Designed Inputs
}
\begin{document}
\sloppy


\twocolumn[
\icmltitle{
    Provably Overwhelming Transformer Models with Designed Inputs
}
\begin{icmlauthorlist}
	\icmlauthor{Lev Stambler}{Quics,UMD,NT}
	\icmlauthor{Seyed Sajjad Nezhadi}{Quics,UMD,Iluvatar}
	\icmlauthor{Matthew Coudron}{Quics,UMD,NIST}
\end{icmlauthorlist}
\icmlaffiliation{Quics}{Joint Center for Quantum Information and Computer Science, University of Maryland}
\icmlaffiliation{UMD}{Department of Computer Science, University of Maryland}
\icmlaffiliation{NIST}{National Institute of Standards and Technology}
\icmlaffiliation{NT}{Neon Tetra LLC}
\icmlaffiliation{Iluvatar}{iluvatar Technologies}
\icmlcorrespondingauthor{Lev Stambler}{levstamb@umd.edu}


\icmlkeywords{ML Theory, Formal Guarantees, Transformers, Interpretability, Machine Learning, ICML}

\vskip 0.3in
]
\numberwithin{theorem}{section}  % This links theorem numbers with section numbers

\theoremstyle{plain}     % For theorems, lemmas, etc.


% \maketitle

%%% REVIEW
\newcommand{\tocite}{{\color{red}CITE} }
\newcommand{\toref}{{\color{red}REF} }

%%% LOGO
\newcommand{\usc}{\raisebox{-1pt}{\includegraphics[height=0.8em]{figures/usc_logo.png}}}
\newcommand{\vuam}{\raisebox{-1pt}{\includegraphics[height=0.8em]{figures/vu_logo.png}}}

%%% SIGNS and SYMBOLS
\newcommand{\grad}{\texttt{grad-CROP}}
\newcommand{\att}{\texttt{att-CROP}}
\newcommand{\seg}{\texttt{seg}}
\newcommand{\clip}{\texttt{clip-CROP}}
\newcommand{\sam}{\texttt{sam-CROP}}
\newcommand{\yolo}{\texttt{yolo-CROP}}
\newcommand{\hc}{\texttt{human-CROP}}
\newcommand{\zsvqa}{\texttt{ZSVQA}}
\newcommand{\vic}{\textbf{ViCrop}}
\newcommand{\xmark}{\text{\ding{55}}}
\newcommand{\cmark}{\text{\ding{51}}}
\newcommand{\success}{\texttt{\color{green} \cmark}}
\newcommand{\failure}{\texttt{\color{red} \xmark}}
\newcommand{\rel}{\texttt{rel-att}}
\newcommand{\gra}{\texttt{grad-att}}
\newcommand{\pgra}{\texttt{pure-grad}}
\newcommand{\relh}{\texttt{rel-att$^h$}}
\newcommand{\grah}{\texttt{grad-att$^h$}}
\newcommand{\pgrah}{\texttt{pure-grad$^h$}}


%%% Text Abb.
\makeatletter
\DeclareRobustCommand\onedot{\futurelet\@let@token\@onedot}
\def\@onedot{\ifx\@let@token.\else.\null\fi\xspace}

\def\aka{\emph{a.k.a}\onedot} \def\Eg{\emph{E.g}\onedot}
\def\eg{\emph{e.g}\onedot} \def\Eg{\emph{E.g}\onedot}
\def\ie{\emph{i.e}\onedot} \def\Ie{\emph{I.e}\onedot}
\def\cf{\emph{c.f}\onedot} \def\Cf{\emph{C.f}\onedot}
\def\etc{\emph{etc}\onedot} \def\vs{\emph{vs}\onedot}
\def\wrt{w.r.t\onedot} \def\dof{d.o.f\onedot}
\def\etal{\emph{et al}\onedot}
\makeatletter



\definecolor{myred}{HTML}{FF8577}
\definecolor{mygreen}{HTML}{0FA958}
\definecolor{myblue}{HTML}{1982C4}
\definecolor{codegreen}{rgb}{0,0.5,0}
\definecolor{codegray}{rgb}{0.5,0.5,0.5}
\definecolor{codepurple}{rgb}{0.07,0,0.53}
\definecolor{codered}{RGB}{189,41,0}
\definecolor{codecomment}{RGB}{153,153,153}
\definecolor{backcolour}{rgb}{0.96,0.96,0.96}
\definecolor{royalblue}{rgb}{0.0, 0.14, 0.4}
\definecolor{egyptianblue}{rgb}{0.06, 0.2, 0.65}
\definecolor{royalazure}{rgb}{0.0, 0.22, 0.66}
\definecolor{portlandorange}{rgb}{1.0, 0.35, 0.21}
\definecolor{sienna}{RGB}{183,105,68}
\definecolor{saddlebrown}{RGB}{139,69,19}
\definecolor{mediumbrown}{RGB}{83,41,11}
\definecolor{darkbrown}{RGB}{58,28,7}
\hypersetup{
    colorlinks=true,
    linkcolor=sienna,
    urlcolor=royalblue,
    citecolor=royalblue,
}

\printAffiliationsAndNotice{} % otherwise use the standard text.

\iffalse
\begin{abstract}
We develop an algorithm which, given a trained single-layer transformer model $\mathcal{M}$ as input, as well as a string of tokens $s$ of length $n_{fix}$ and an integer $n_{free}$, can generate a mathematical proof that $\mathcal{M}$ is ``overwhelmed'' by $s$.
We say that $\mathcal{M}$ is ``overwhelmed'' by $s$ when the output of the model evaluated on this string plus any additional string $t$, $\mathcal{M}(s + t)$, is completely insensitive to the value of the string $t$ whenever $\text{length}(t) \leq n_{free}$.
   
An exhaustive approach to verifying such a worst-case statement would require time exponential in $n_{free}$, but our approach uses a carefully designed analysis of Lipschitz continuity, combined with convex relaxations to produce a worst-case proof in time and space $\widetilde{O}(n_{fix}^2 + n_{free}^3)$. 
Along the way, we prove a particularly strong worst-case form of ``over-squashing'' \cite{alon2021bottleneckgraphneuralnetworks, barbero2024transformers}, which we use to bound the model's behavior.

Our technique uses computer-aided proofs to establish this type of operationally relevant guarantee about transformer models.
We empirically test our algorithm on a single layer transformer complete with an attention head, layer-norm, MLP/ReLU layers, and RoPE positional encoding.
For this model, our algorithm produces proofs of natural overwhelming strings when the ``free string'' is restricted to be an element of a permutation set.  

We believe that this work is a stepping stone towards the difficult task of obtaining useful guarantees for trained transformer models.
For example, ``overwhelming'' strings can be used to prove no-go results for prompt engineering: no ``prompt'' can equip a model $\mathcal{M}$ to properly execute specific tasks like correcting errors in code.
\end{abstract}
\fi

\begin{abstract}
We develop an algorithm which, given a trained transformer model $\mathcal{M}$ as input, as well as a string of tokens $s$ of length $n_{fix}$ and an integer $n_{free}$, can generate a mathematical proof that $\mathcal{M}$ is ``overwhelmed'' by $s$, in time and space $\widetilde{O}(n_{fix}^2 + n_{free}^3)$.
We say that $\mathcal{M}$ is ``overwhelmed'' by $s$ when the output of the model evaluated on this string plus any additional string $t$, $\mathcal{M}(s + t)$, is completely insensitive to the value of the string $t$ whenever length($t$) $\leq n_{free}$.
Along the way, we prove a particularly strong worst-case form of ``over-squashing'' \cite{alon2021bottleneckgraphneuralnetworks, barbero2024transformers}, which we use to bound the model's behavior.
Our technique uses computer-aided proofs to establish this type of operationally relevant guarantee about transformer models.
We empirically test our algorithm on a single layer transformer complete with an attention head, layer-norm, MLP/ReLU layers, and RoPE positional encoding.
We believe that this work is a stepping stone towards the difficult task of obtaining useful guarantees for trained transformer models.
\end{abstract}

%%%%%% general TODOs
% \pagebreak

%%%%%%%%%%%%%%%%%%%%%%%%%%%%%%%%%%%%%%%%%%%%%%%%%%
\section{Introduction}
\label{sec:intro}

\begin{figure*}[tb]
    \centering
    \includegraphics[width=0.848\linewidth]{figs/circuitnn.pdf} 
    \caption{Illustration of differentiable CircuitNN. CircuitNN is designed based on differentiable NAND gates. After DAS is guided by PI and PO pairs of the truth table, CircuitNN can get the precise circuit architecture logic equivalent to the truth table.}
    \label{fig:circuitnn}
\end{figure*}

% 1. Describe the importance of logic synthesis
% 2. Existing Problems
% (a) Neural Architecture Search: Unstable, Predefined Setting, etc.
% (b) Circuit Generation: Probabilistic Model, Logic Equivalence

With the rapid advancement of technology, the scale of integrated circuits (ICs) has expanded exponentially. 
This expansion has introduced significant challenges in chip manufacturing, particularly concerning power and area metrics.
A primary objective in IC design is achieving the same circuit function with fewer transistors, thereby reducing power usage and area occupancy.

Logic synthesis~\cite{hachtel2005logicsynth}, a critical step in electronic design automation (EDA), transforms behavioral-level circuit designs into optimized gate-level circuits, ultimately yielding the final IC layout. 
The primary goal of logic synthesis is to identify the physical implementation with the fewest gates for a given circuit function. 
This task constitutes a challenging NP-hard combinatorial optimization problem. 
Current logic synthesis tools~\cite{brayton2010abc, wolf2013yosys} rely on human-designed heuristics, often leading to sub-optimal outcomes.

Differentiable architecture search (DAS) techniques~\cite{liu2018darts, chu2020darts} offer novel perspectives on addressing challenges in this problem.
Circuit functions can be represented through truth tables, which map binary inputs to their corresponding outputs. 
Truth tables provide a precise representation of input-output relationships, ensuring the design of functionally equivalent circuits.
Inspired by this, researchers~\cite{deepmind2024ai4sys, wang2024tnet} have begun exploring the application of DAS to synthesize circuits directly from truth tables.
Specifically, \citet{deepmind2024ai4sys} proposed CircuitNN, a framework that learns differentiable connection structures with logic gates, enabling the automatic generation of logic circuits from truth tables.
This approach significantly reduces the complexity of traditional circuit generation. 
Building on this, \citet{wang2024tnet} introduced T-Net, a triangle-shaped variant of CircuitNN, incorporating regularization techniques to enhance the efficiency of DAS.

Despite these advancements, several challenges remain. 
The computational complexity of DAS grows quadratically with the number of gates, posing scalability issues.
Although triangle-shaped architecture~\cite{wang2024tnet} partially mitigates this problem, redundancy persists. 
%Additionally, DAS is susceptible to converging to local optima, limiting the ability to search architectures that satisfy the given truth tables~\cite{liu2018darts}. 
%Furthermore, hyperparameters (network depth and layer width) require extensive searches, introducing complexity and prolonging the synthesis process. 
Additionally, DAS is susceptible to converging to local optima~\cite{liu2018darts} and hyperparameters (network depth and layer width) require extensive searches. 
The challenges arise from the vast search space in DAS. 
% Even with predefined settings for CircuitNN, finding a configuration that meets the truth table requires extensive trial and error during the DAS process. 
Intuitively, limiting the search space through predefined parameters (network depth, gates per layer, and connection probabilities) can significantly reduce the complexity.

Recent advances~\cite{openai2023gpt4, abramson2024alphafold3, esser2024sd3, li2024mar} in conditional generative models have demonstrated remarkable performance across language, vision, and graph generation tasks. 
Motivated by these developments, we propose a novel approach to circuit generation that generates preliminary circuit structures to guide DAS in generating refined circuits matching specified truth tables. 
Firstly, we introduce CircuitVQ, a tokenizer with a discrete codebook for circuit tokenization. 
Built upon our Circuit AutoEncoder framework~\cite{hou2022graphmae,li2023maskgae,wu2025mgvga}, CircuitVQ is trained through a circuit reconstruction task. 
Specifically, the CircuitVQ encoder encodes input circuits into discrete tokens using a learnable codebook, while the decoder reconstructs the circuit adjacency matrix based on these tokens.
Subsequently, the CircuitVQ encoder serves as a circuit tokenizer for CircuitAR pretraining, which employs a masked autoregressive modeling paradigm~\cite{chang2022maskgit, li2023mage}. 
In this process, the discrete codes function as supervision signals. 
After training, CircuitAR can generate discrete tokens progressively, which can be decoded into initial circuit structures by the decoder of the CircuitVQ. 
These prior insights can guide DAS in producing refined circuits that match the target truth tables precisely.

Our key contributions can be summarized as follows:
\begin{itemize}
\item We introduce CircuitVQ, a circuit tokenizer that facilitates graph autoregressive modeling for circuit generation, based on our Circuit AutoEncoder framework;
\item Develop CircuitAR, a model trained using masked autoregressive modeling, which generates initial circuit structures conditioned on given truth tables;
\item Propose a refinement framework that integrates differentiable architecture search to produce functionally equivalent circuits guided by target truth tables;
\item Comprehensive experiments demonstrating the scalability and capability emergence of our CircuitAR and the superior performance of the proposed circuit generation approach.
\end{itemize}

% Motivation
% (a) Diffusion (Vision, Graph), Autoregressive (Language, Vision)
% (b) Circuit Generation for Predefined Setting
% (c) Neural Architecture Search for Strict Logic Equivalence

% Contribution
% (a) Circuit Tokenizer (new transformer arch, training strategy)
% (b) CircuitAR (train and gen strategies, post-ar strategy)
% (c) Extensive Evaluation including BitD (Bit Distance) for Scalability

\newcommand{\ours}{$\text{Q}$LASS}
To illustrate equilibria and dynamics of performative prediction games, we focus on a scenario in which a \emph{duopoly} of mortgage companies, i.e. banks, compete to sell loans to customers.

\paragraph{Customer Model:} In our game, each bank is trying to attract customers from a given population $\mathcal{P}$. We model this population as comprised of individuals with a single-dimensional type: we denote individual $j$'s type as $y_j \in [0,1]$. For simplicity, we assume that \(y\) represents the customer’s probability of repaying the loan\footnote{In practice, a customer's (normalized) credit score can be interpreted as a noisy observation of $y_j$. This also corresponds to credit scores being \emph{calibrated}.}, i.e., $y_j := \P[Y_j = 1]$, where $Y_j$ is a random variable such that $Y_j = 0$ means that $j$ defaults on their loan, and $Y_j = 1$ means they repay their loan. Customer types in the population are drawn from a known distribution $D_y$ supported on $[0,1]$. 

\paragraph{Game between Banks:} Each Bank \(i \in \{1, 2\}\) selects two parameters \( (\tau_i, \gamma_i) := \theta_i\), where:
\begin{itemize}
    \item \(\tau_i \in \{\tau_l,\tau_h\}\) is the credit score threshold for approving a customer\footnote{We restrict the bank to only pick between two thresholds, $\tau_l$ and $\tau_h$. However, we highlight how our results are affected when we expand the strategy space to $n > 2$ actions in our experiments of Appendix \ref{app:3gamma}.}. Specifically, a customer $j$ with credit score \(y_j\) is approved by Bank $i$ if and only if \(y_j \geq \tau_i\);
    \item \(\gamma_i \in \{\gamma_l, \gamma_h\}\) is the interest rate offered to approved customers.
\end{itemize}
We denote as shorthand the space of allowable thresholds by $\Gamma := [0,1]$ and allowable interests rates by $\Lambda := [0,1]$. %The latter is set without loss of generality---we simply normalize the rates to be at most $1$. 
% {\color{red} Vidya: just thinking about this but is it natural to restrict interest rate to $1$? I don't think it would affect the equilibrium structure of the game but theoretically I think the interest rate could be anything in $[0,\infty)$.} {\color{green} Guanghui: Could we say something like this is without loss of generality} \gua{changed.}\juba{I think we repeated this twice, the next sentence already had this}
The loan amount is normalized to $1$ in the entire paper, without loss of generality; in this case, if a customer chooses Bank $i$, and the customer is approved by the bank at an interest rate of $\gamma_i$, the expected utility for the bank is equal to
\[
(1+\gamma_i)\cdot \P[Y_i = 1]-\P[Y_i = 0] = (1+\gamma_i)y_i-(1-y_i).
\]


%In practice, the credit score \(y\) serves as a noisy observation of the true likelihood of the customer's repayment. 

\paragraph{Banks' Utilities:} For given parameter choices \(\theta_1 = (\tau_1, \gamma_1)\) by Bank 1 and \(\theta_2 = (\tau_2, \gamma_2)\) by Bank 2, a \emph{rational} customer with credit score $y$ acts as follows:

\begin{enumerate}
    \item \textbf{Qualified for a single bank}: 
        \begin{itemize}
        \item If \(\tau_1 \leq y < \tau_2\), the customer goes to Bank 1, as the score qualifies for Bank 1 but not Bank 2. Conversely, if \(\tau_2 \leq y < \tau_1\), the customer chooses Bank 2.
    \end{itemize}
    \item \textbf{Qualified for both banks}:
     \begin{itemize}
        \item If \(\tau_1, \tau_2 \leq y\) and \(\gamma_1 < \gamma_2\), the customer selects Bank 1 for its lower interest rate. Conversely, if \(\gamma_1 > \gamma_2\), the customer chooses Bank 2.
        \item If \(\gamma_1 = \gamma_2\), the customer picks each bank with probability $1/2$. 
    \end{itemize}
    \item \textbf{Unqualified for both banks}:
    \begin{itemize}
        \item If \(y < \tau_1\) and \(y < \tau_2\), the customer is rejected by both banks.
    \end{itemize}
\end{enumerate}

The expected reward for Bank 1, denoted as \(u_1(\theta_1, \theta_2)\), can then be expressed as:
\begin{align}\label{eq:utility}
    u_1(\theta_1, \theta_2) 
    &=  \mathbb{E}_{y \sim D_y} \left[ \mathbb{I}\{\underbrace{\tau_1 \leq y < \tau_2 \ \cup \ (\tau_1, \tau_2 \leq y \ \cap \ \gamma_1 < \gamma_2)}_{\text{accepted by Bank 1}}\} \cdot \big((1+\gamma_1)y - (1-y)\big) \right] \nonumber\\
    & + \frac{1}{2} \mathbb{E}_{y \sim D_y} \left[ \mathbb{I}\{\underbrace{\tau_1, \tau_2 \leq y \ \cap \ \gamma_1 = \gamma_2}_{\text{accepted by both Banks}}\} \cdot \big((1+\gamma_1)y - (1-y)\big) \right].
\end{align}
Note that the problem is \emph{symmetric}, i.e., the utility function for Bank 2 can be derived by swapping the roles of \(\theta_1\) and \(\theta_2\). I.e., $u_2(\theta_1, \theta_2) = u_1(\theta_2, \theta_1)$. 

% If a bank only attracts customers between thresholds $\tau_a$ and $\tau_b$, for $\tau_a<\tau_b$, we call $[\tau_a,\tau_b]$ the \emph{threshold} range for that bank. For example, if Bank $1$ sets a threshold of $\tau_1$, Bank $2$ a threshold of $\tau_2 > \tau_1$, and $\gamma_1 > \gamma_2$, then Bank 1 has a threshold range of $[\tau_1,\tau_2]$, while bank $2$ has a threshold range of $[\tau_2,1]$.
% Note that the parameters set by \emph{both} banks, i.e. $(\theta_1,\theta_2)$ both influence the threshold range for each of Bank 1 and 2.  If $\tau_1>\tau_2$, $\gamma_1>\gamma_2$, then $\tau_a>\tau_b$, and the bank does not attract any customers. 
% {\color{red} is it possible for $\tau_a > \tau_b$, leading to the bank never attracting customers?} \gua{if $\gamma_1>\gamma_2$, $\tau_1>\tau_2$, then it gets no customer. I think it also makes sense.}\juba{I think we said we wanted to delete the discussion of the threshold range, no?}

% \noindent \textbf{Discrete Model}   
% We now present the discrete version of our model, where the interest rates and thresholds are selected from finite sets \(\Gamma\) and \(\Lambda\), respectively, with $\tau\in[0,1], \gamma\in[0,1]$,  for all $\tau\in\Lambda$ and $\gamma\in\Gamma$, \(|\Gamma| = n\) and \(|\Lambda| = m\). Let \(p_1, p_2 \in \Delta(\Gamma \times \Lambda)\) represent the mixed strategies of the two banks, where \(\Delta(\Gamma \times \Lambda)\) denotes the set of probability distributions over the discrete decision space \(\Gamma \times \Lambda\).


% \begin{Remark}
%    Note that our proposed problem can be reformulated as a standard multi-player performative prediction problem \citep{narang2023multiplayer}. However, in our problem, the data distribution faced by each learner breaks the Lipschitzness assumption of previous work~\citep{hardt2023performative,narang2023multiplayer}. A small modification in one of the learner's thresholds can completely change how demand is allocated across both learners, as is often the case in Bertrand-style games. 
% \end{Remark} 

% \gua{I made some changes to Remark 1, please have a look}
\begin{Remark}
   Previous works in multi-learner performative prediction~\citep{narang2023multiplayer} resort to an insensitivity assumption, i.e., the data distribution faced by each player can only changes slightly when the parameters also change slightly; formally, the data distribution faced by each player is Lipschitz in their decisions. This is immediately not true in our setting: the bank slightly changing its parameters can completely changes the demand distribution of customers it faces. Intuitively, this is because of Bertrand-competition-style effects, where if two banks have similar rates, one bank that lowers their rate by a small amount suddenly captures the entire customer demand that is eligible for that rate.%\juba{made further light edits adding intuition}
   
   In Appendix \ref{Appendix:refumulation}, we discuss this problem more carefully by reformulating our problem in the standard multi-learner performative prediction form given by~\citep{narang2023multiplayer}. We show the distribution is not Lipschitz with respect to the parameters, and thus does not satisfy the insensitivity assumption. 
%Prior work~\citep{hardt2023performative,narang2023multiplayer} showed that, for a general multi-agent performative prediction framework to work, insensitivity assumptions are needed: in the \textbf{worst case}, they can construct settings where the insensitivity assumption does not hold and simple dynamics do not converge anymore. We add nuance to this picture. We will show that our dynamics often converge, even absent insensitivity assumptions, highlighting that while the impossibility results of previous work hold in the worst case, they may not hold in the ``average case'' and especially not in problems motivated by applications. In particular, we will show convergence to a variety of equilibria of our game, and often to symmetric Nash equilibria where insensitivity is immediately violated.
     
\end{Remark}



% \paragraph{Relationship to Performative Prediction} A central point of our work is to highlight that \textcolor{red}{needs writing from intro}. We highlight how our work specifically ties to ``Performative Prediction'' below:


%\textcolor{red}{needs a definition environment}



%Here, \(\E_{\theta_1, \theta_2}\) represents the expected utility of the banks over their respective strategies \((\theta_1, \theta_2)\). These inequalities ensure that neither bank can unilaterally improve its expected utility by deviating from its mixed strategy in the equilibrium.



%and  for all $\tau\in\Gamma$, we have $\tau\in\Lambda$, $(\tau,\gamma)\in[0,1]^2$. Let $\Gamma\times\Lambda$
%In this paper, we focus on the most fundamental case, where there are two choices for each parameter: $0\leq\tau_{\ell}<\tau_{h}\leq 1$, and $0\leq \gamma_{\ell}< \gamma_{h}\leq 1$. In this case, the utility for each pair of decisions forms a $4\times4$ matrix (given in Table \ref{tab:my-table}). We consider the canonical case where $\tau_{\ell}=\frac{1}{2+\gamma_{h}}$, and $\tau_{h}=\frac{1}{2+\gamma_{\ell}}.$ Note that these are natural choices for the thresholds, in the sense that, if there is only one bank and the interest rate is set to be $\gamma$, then $\frac{1}{2+\gamma}$ is the optimal threshold corresponding to the fixed $\gamma$.


%and the thresholds are chosen in $\Lambda=\{\tau^{(1)},\dots,\tau^{(m)}\}$. Here, we only assume that, for each $\gamma\in\Gamma$, there at least exist one $\tau\in\Lambda$ such that $f(\gamma,\tau,1)>0$. Note that this is a very minor assumption, in the sense that, if for a $\gamma$ such that $f(\gamma,\tau,1)<0$ for all $\tau\in\Lambda$, then adopting this decision will lead to negative utility regardless of the opponent's decision, and thus is not an interesting case. 

%\textcolor{red}{The model section is missing the dynamic version of the game. We should clearly define the one-shot and the dynamic game}
% we only considered one-shot case in our paper



% TODO more details in methodology and data processing
% merge methodology and 
\section{Framework for Analyzing Emotion}
In this section, we present our framework for analyzing emotion. We first establish a basic understanding of emotion polarity by determining the sentiment valence of each root tweet and comment. We then use multi-label emotion detection to predict the emotion categories associated with each post. Based on this data, we explore the interactive nature of emotions, by identifying common patterns in emotion transition pairs between temporally-adjacent posts. Finally we investigate the emotional trajectory within threads to understand how emotional intensity and type shift over time, by aggregating the predicted labels for posts at each time stamp in a given thread. As part of this, we contrast rumour with non-rumour threads, to gain a holistic understanding of emotional expression in rumours and non-rumours on Twitter.

% elaborate a bit on why we choose EmoLLM, compared with other automatic emotion detection methods
\paragraph{Affective Computing: Automatic Emotion Detection}
Manually annotating emotions is both costly and time-consuming, so we use an LLM-based emotion detection model, EmoLLM~\citep{liu2024emollms}, which is specifically designed for sentiment analysis and emotion detection. The model was instruction-tuned on SemEval 2018 Task1 using a comprehensive emotion labeling scheme grounded in established theoretical frameworks. We prompt the model to perform Valence Ordinal Classification (V-oc), Emotion Classification (E-c), and Emotion Intensity regression (E-i). Detailed prompts are shown in \Cref{tab:emollm_ins}.

\paragraph{Categorical Emotion Labeling Scheme} \label{para:emotion_label}
Numerous emotion label sets  have been proposed~\citep{Ekman1992AnAF, Plutchik1980AGP, Russell1980ACM}. According to \citet{Ekman1992AnAF, Plutchik1980AGP}, certain emotions, such as joy, fear, and sadness, are considered more fundamental than others, both physiologically and cognitively. The Valence-Arousal-Dominance (VAD) model \citep{Russell1980ACM} categorizes emotions within a three-dimensional space of valence (positivity-negativity), arousal (active-passive), and dominance (dominant-submissive). Inspired by \citet{mohammad-etal-2018-semeval}, we incorporate elements from both basic emotion theories and the VAD model, and further ground EmoLLM emotion predictions to develop the following emotion label schemes: (1) \textit{neutral or no emotion}; (2) \textit{negative emotions}: anger (also includes annoyance and rage),  disgust (also includes disinterest, dislike, and loathing), fear (also includes apprehension, anxiety, and terror), pessimism (also includes cynicism, and no confidence), sadness (also includes pensiveness and grief); 3) \textit{positive emotions}: joy (also includes serenity and ecstasy), love (also includes affection), optimism (also includes hopefulness and confidence), anticipation (also includes interest and vigilance), surprise (also includes distraction and amazement) and trust (also includes acceptance, liking, and admiration). 


\paragraph{Emotion Polarity: Sentiment Valence} 
To understand the basic emotion polarity expressed in rumour and non-rumour content, we begin with sentiment valence analysis. Sentiment valence aims to capture the overall emotional tone conveyed by a post, in terms of how positive or negative it is~\citep{liu2024emosurvey}. We frame the sentiment valence task as ordinal regression~\citep{mohammad-etal-2018-semeval}. As shown in \Cref{tab:emollm_ins}, for a given tweet post, we classify it into one of seven ordinal levels of sentiment intensity, spanning varying degrees of positive and negative valence, that best represents the tweeter's mental state. The tweet posts within a thread can be divided into two categories: root tweets, which are posted by the publisher, and follow posts, which include all subsequent replies under the root post. We begin by conducting sentiment valence analysis on each post within the thread conversation. 
% TJB: confused by how comments can include all subsequent replies; we seem to be overloading the terminology, for comments to be both individual posts and series of posts
% RX: yes, I am unifiying all terms.
For each category, we compute the mean sentiment valence to enable further investigation into the specific emotions associated with different sentiment valences over a thread.
% TJB: clarify for comments whether the classification is done over the combined meta-document (i.e. the root + all comments to that point) or individually over the separate documents and then combined ... or over individual documents, in which case the statement about "all replies" needs clarification
% RX: we separate root and comments for each tweet conversation, the former is the root tweet posted by the publisher while the rest are comments. "all replies" mean all comments under root tweet, we aggregate them by computing the mean sentiment, and then average over each part.

\paragraph{Emotion Distribution} 
Following sentiment valence analysis, we then examine specific emotions and their distribution in rumour and non-rumour tweet posts.
Motivated by the fact that a certain tweet might exhibit more than one emotion, we frame the task as multi-label emotion detection problem. As shown as V-oc in \Cref{tab:emollm_ins}, given a tweet, we classify it into one of seven ordinal classes, corresponding to various levels of positive and negative sentiment intensity. To reduce noise from automatic emotion detectors, we take the top-three predicted emotions for each tweet. We then aggregate and plot the emotion distribution to provide an overview of dominant emotional trends across the rumour and non-rumour posts. Given that the follow posts make up the majority of the data compared to the root posts, we will focus on using follow posts in our next analysis.
% TJB: what is the basis of saying that the signal is richer? simply that there are more reply posts than root posts? clarify
% RX: yes, and we are more interested in interaction in comments.

\paragraph{Emotion Transitions} 
Emotions are contagious and highly interactive~\citep{Ferrara_2015}. When publishers write tweets that convey their emotions, readers are likely to respond with emotional reactions of their own~\citep{Ferrara_2015,emotion_dynamics}. In this part, we model this interactive nature of emotions in the form of emotion transition pairs, which are built from two chronologically-adjacent tweets. In each pair, the first element represents the emotion inferred from the initial content published at a given time, and the second element represents the emotion inferred from the reply content published immediately after. For example, if the first tweet exhibits \textit{joy} \textit{trust} and \textit{anticipation}, and the second tweet shows \textit{anger}, \textit{disgust} and \textit{surprise}, we form the pairs (\textit{joy}, \textit{anger}), (\textit{joy}, \textit{disgust}), (\textit{joy}, \textit{surprise}), (\textit{trust}, \textit{surprise}), (\textit{trust}, \textit{surprise}), (\textit{trust}, \textit{disgust}), (\textit{anticipation}, \textit{anger}), (\textit{anticipation}, \textit{surprise}) and (\textit{anticipation}, \textit{disgust}). We create transitions for all combinations of emotion pairs and explore the likelihood of emotion transition pairs occurring in rumour and non-rumour content. Exploring emotion transitions allows us to understand the emotional flow in social media conversations and uncover typical patterns of rumour and non-rumour content, and any differences between the two.

\paragraph{Emotion Trajectories} 
We explore the cumulative trajectory of emotion over time to observe how emotions evolve during the conversational thread. We collect all detected emotion labels for each tweet from both rumour and non-rumour content, then track cumulative emotion counts at each chronological step. Finally, we visualize these trends and apply regression models to analyze the growth of emotions over time. This temporal analysis reveals how emotions accumulate or intensify across time, offering insight into the trajectory of emotions in rumour and non-rumour content.

\begin{table*}[!h]
    \centering
    \small
    \begin{tabular}{cccccccccccc}
        \toprule
        \textbf{Setting} & \textbf{Ru} & \textbf{Non} & \textbf{p} & \textbf{\#Ru/Non} & \textbf{T} & \textbf{F} & \textbf{U} & \textbf{$p$ (U vs T)} & \textbf{$p$ (U vs F)} & \textbf{\#T/\#F/\#U} \\
        \midrule
        \textbf{PHEME root} & \textbf{$-$0.25} & $-$0.17 & 0.00 & 2602/2602 & $-$0.21 & $-$0.11 & \textbf{$-$0.39} & 7.75e-11 & 4.41e-11 & 629/629/629 \\
        \textbf{PHEME follow} & \textbf{$-$0.33} & $-$0.26 & 6.47e-09 & & $-$0.35 & $-$0.20 & \textbf{$-$0.39} & 0.03 & 8.38e-15 & \\
        \textbf{Twitter15 root} & \textbf{$-$0.26} & $-$0.01 & 3.51e-05 & 372/372 & $-$0.21 & $-$0.20 & \textbf{$-$0.34} & 0.01 & 0.01 & 359/359/359 \\
        \textbf{Twitter15 follow} & \textbf{$-$0.27} & $-$0.06 & 1.65e-09 & & $-$0.24 & $-$0.25 & \textbf{$-$0.30} & 0.16 & 0.21 & \\
        \textbf{Twitter16 root} & \textbf{$-$0.18} & \z0.07 & 0.00 & 205/205 & \z0.11 & $-$0.22 & \textbf{$-$0.30} & 1.35e-06 & 0.18 & 63/63/63 \\
        \textbf{Twitter16 follow} & \textbf{$-$0.31} & $-$0.12 & 9.19e-06 & & $-$0.30 & \textbf{$-$0.36} & $-$0.27 & 0.67 & 0.90 & \\
        % \textbf{CoAID root} & \textbf{$-$0.34} & $-$0.16 & 0.01 & 167/167 & - & - & - & - & - & - \\
        % \textbf{CoAID follow} & \textbf{$-$0.24} & $-$0.13 & 0.01 & & - & - & - & - & - & \\
        \bottomrule
    \end{tabular}
    \caption{Valence Ordinal Regression results for all datasets. root = root posts, follow = follow posts, Ru = rumour, Non = Non-rumour, T = True rumour, F = False rumour, U = Unverified rumour; $p$ values indicates significance of a one-tailed t-test.}
\label{tab:voc_results}
\end{table*}

\begin{algorithm}[ht!] 
\caption{PC Algorithm}
\label{pc}
\begin{algorithmic}[1] 
\State \textbf{Input:} Data $\mathbf{X}$, significance level $\alpha$
\State \textbf{Output:} Completed Partially Directed Acyclic Graph (CPDAG)

\State Initialize a complete undirected graph $G$ with all variables as nodes.

\State \textbf{Step 1: Skeleton Identification}
\For{each pair of variables $(X, Y)$ in $G$}
    \State Find the subset $S \subseteq \text{Adj}(X, G) \setminus \{Y\}$ such that 
    $X \indep Y \mid S$ with significance $\alpha$.
    \If{such a subset $S$ exists}
        \State Remove the edge $X - Y$ from $G$.
    \EndIf
\EndFor

\State \textbf{Step 2: Edge Orientation}
\For{each triple of variables $(X, Y, Z)$ in $G$ where $X - Z - Y$ and $X, Y$ are not adjacent}
    \If{$Z \notin S$ for all separating sets $S$ for $X$ and $Y$}
        \State Orient as $X \to Z \leftarrow Y$ (identify a collider).
    \EndIf
\EndFor

\While{possible}
    \For{each edge $(X - Y)$ in $G$}
        \If{there exists a directed path $X \to \dots \to Z$ such that $Z - Y$}
            \State Orient as $X \to Y$ (acyclicity rule).
        \ElsIf{orienting $X - Y$ as $X \to Y$ creates a new v-structure}
            \State Orient as $X \to Y$ (v-structure rule).
        \EndIf
    \EndFor
\EndWhile

\State \textbf{return} the CPDAG representing the equivalence class of causal graphs.

% how we frame the task, compute the emotion intensity, how to aggregate on conversation level

\end{algorithmic}
\end{algorithm}


\paragraph{Causal Relationship of Emotions in Rumour \& Non-Rumour Threads}
To gain a deeper insight into the relationship between rumours and the emotions underlying them, we extend our analysis beyond statistical correlation by conducting a causal analysis. Specifically, we apply the Peter-Clark (PC) algorithm \cite{Spirtes2000}, a classical constraint-based causal discovery algorithm on the three merged datasets. 

Uncovering causal relations between variables of interest is never an easy problem. Under the fundamental assumption of \textit{causal Markov condition} that a variable is conditionally independent of all its non-effects given its direct cause, \textit{faithfulness} ensures that the casual graph exactly encodes the independence and conditional independence relations among variables. These two assumptions allow us to infer causal relationships from observed statistical independencies, forming the cornerstone of constraint-based causal discovery methods. 

The PC algorithm identifies causal relationships among the variables of interest, represented as a directed acyclic graph (DAG), by numerating the independence and conditional independence relationships. The algorithm consists of two main steps: 
\begin{enumerate}
    \item \textbf{Skeleton Identification}: Starting with a complete undirected graph where all variables are connected, edges are iteratively removed based on conditional independence and independence relationships among variables, inferred by a conditional independence test. This step returns an undirected graph, which we call a skeleton. 
    \item \textbf{Edge Orientation}: After constructing the skeleton, edges are oriented by a set of predefined rules (Meek's Rule \cite{meek1997graphical}) to avoid cycles and orient collider structures.
\end{enumerate}

The complete PC algorithm is provided in algorithm \ref{pc}. It returns a  completed partially directed acyclic graph (CPDAG), which represents an equivalence class of causal graphs that are consistent with the observed data’s independence and conditional independence relations. In our implementation, we adopt the  Fisher-z test \cite{fisher_probable_1921} to infer the conditional independence relations.

%%% Local Variables:
%%% mode: latex
%%% TeX-master: "../main_anonymous"
%%% End:

\newcommand{\ModelFinal}{\Model^{final}}
\newcommand{\peakToPeak}{\mathrm{PTP}}
\newcommand{\RepIndxs}{\mathcal{R}}
\newcommand{\FreeIndxs}{\mathcal{F}}
\newcommand{\softSumMaxF}{\alpha_{free}^{\max}}
\newcommand{\softSumMinR}{\alpha_{fix}^{\min}}
\newcommand{\softSumMaxR}{\alpha_{fix}^{\max}}
\newcommand{\softSumMinF}{\alpha_{free}^{\min}}

\newcommand{\ssMaxFA}{{\alpha}_{free}^{\max}}
\newcommand{\ssMinRA}{{\alpha}_{fix}^{\min}}
\newcommand{\ssMaxRA}{{\alpha}_{fix}^{\max}}
\newcommand{\ssMinFA}{{\alpha}_{free}^{\min}}


\newcommand{\ssMaxFB}{{\beta}_{free}^{\max}}
\newcommand{\ssMinRB}{{\beta}_{fix}^{\min}}
\newcommand{\ssMaxRB}{{\beta}_{fix}^{\max}}
\newcommand{\ssMinFB}{{\beta}_{free}^{\min}}

\newcommand{\logRMin}{L_{fix}^{\min}}
\newcommand{\logRMax}{L_{fix}^{\max}}
\newcommand{\logFMin}{L_{free}^{\min}}
\newcommand{\logFMax}{L_{free}^{\max}}
\newcommand{\Samp}{\textsf{Samp}}
\newcommand{\varAt}{{\Var_j^{(k)}}}
\newcommand{\varAtMin}{{\Var_{j, \min}^{(k)}}}
\newcommand{\varAtMax}{{\Var_{j, \max}^{(k)}}}
\newcommand{\preSMMax}{\ell_{i, \max}^{(k)}}
\newcommand{\preSMMin}{\ell_{i, \min}^{(k)}}

\section{Input Restrictions and Proving Overwhelming}
\label{sec:meta_framework}

In this section, we will provide an algorithm to decide ``overwhelming."
Along the way, we develop a generalizable method of upper-bounding the worst-case deviation of a single layer of a transformer model under input restriction.

\subsubsection*{Input Restrictions and Designed Space}
We use the notation from \textit{Analysis of Boolean Functions} to denote restrictions on inputs to a function \cite{o2014analysis}.
The restriction will fix certain tokens to be a specific value and leave the rest of the tokens free.

\begin{definition}[Input Restriction, \cite{o2014analysis} definition 3.18]
	\label{def:input_restriction}
	Let $f : \calX^n \rightarrow \calY$ be some function and $J \subset [n]$ and $\notJ = [n] \setminus J$.
	Let $z \in \calX^\notJ$.
	Then, we write $f_{J \mid z} : \calX^J \rightarrow \calY$ (``the restriction of $f$ to $J$ given $z$'') as the subfunction of $f$ that is obtained by fixing the coordinates of $\notJ$ to the values in $z$.
	Given $y \in \calX^J$ and $z \in \calX^\notJ$, we write $f_{J \mid z}(y)$ as $f(y, z)$ even though $y$ and $z$ are not literally concatenated.
\end{definition}

Throughout this paper, we will consider a specific input restriction where the first $\nfix$ tokens are restricted to a string $\desSet$ and the last token is fixed to token $\query$.
We will denote said restriction as $\desF{f}$ where $(\nfix, \nctx)$ denotes the set $\{ \nfix + 1, \ldots, \nctx - 1\}$.

Next, it will be useful to define the set of all possible inputs under an input restriction.

\begin{definition}[Designed Space]
	\label{def:InpSpace}
	Recall, from \cref{def:one_hot_space}, that $\OneHotSpace$ is the set of all one-hot vectors of size $\dVocab$.
	We denote by $\OneHotSpace^n$ the set of matrices where each row is a one-hot vector of size $\dVocab$.
	Then, let $\InpSpace \subset \OneHotSpace^\nctx$ designate the set of all possible inputs under a \textbf{specific} input restriction.
	That is, 
	\[
		\InpSpace = \left\{ X \in \R^{\nctx \times \dVocab} \mid X = 
		\begin{bmatrix}
			\vece_{\desSet_1} \\
			\vece_{\desSet_2} \\
			\vdots \\
			\vece_{\desSet_s} \\
			Y \\
			\vecQuery
		\end{bmatrix}, Y \in \freeSpace
		\right\}.
	\]
	where $\freeSpace$ is the space of free tokens.
\end{definition}
\subsection{Algorithm for deciding Overwhelming}
Here we consider zero temperature sampling setting where we can define our model with sampling as
\[
	\ModelFinal(X) = \arg\max_{i \in [\dVocab]} \Model(X) \cdot \vec{e}_i^T
\]
where $\Model$ is defined in \cref{def:one_layer_transformer}.
The model simply selects the token with the highest logit weight rather than sampling from the output distribution. 
% \footnote{This is equivalent to setting the sampling temperature to zero.}

We define the ``peak-to-peak difference'' to be the difference between the logit for the most likely token and the logit for the second most likely token for sample $X$.

\begin{definition}[Peak-to-peak difference]
    Let $X \in \InpSpace$ be any element from the restriction and $k = \ModelFinal(X)$.
    Then, we let 
    \[
    	\peakToPeak(\Model, X) = \min_{j \in [\dVocab], j \neq k} \Model(X) \cdot \left(\vec{e}_{k}^T - \vec{e}_j^T\right).
    \]
\end{definition}

\iffalse
\begin{algorithm}[H] \label{alg:overwhelmCheck}
	\SetKwInOut{Input}{Input}\SetKwInOut{Output}{Output}
	\Input{A fixed single-layer transformer model $\Model$, a string of tokens $s$ denoting a fixed part of the input to $\Model$, integer $\nfree$ denoting the number of free tokens in the input to $\Model$, and a final query token $q$
	}
	\Output{Either the string ``Overwhelmed'' indicating that the output of the model $\Model$ evaluated on $s$ concatenated with $\nfree$ free tokens and query token $q$ is proven to be invariant under the choice of the free tokens, OR the string ``Inconclusive'' if no such proof is obtained.} 

	Calculate a bound $W$ such that $\WD(\vec{e}_\nctx \cdot \desF{\Model}, X) \leq W$ \\
	\If{
		$
		W < \peakToPeak(\desF{\Model}, \InpSpace) / 2
	$ }{\Return ``Overwhelmed''  }

	\Else{\Return ``Inconclusive''}
	\caption{Overwhelmed Verifier - Algorithm Sketch }
\end{algorithm} 
\fi

To prove a model is overwhelmed by a fixed input we bound the worst-case deviation by the peak-to-peak difference.
The following theorem summarizes the bound our algorithm is tasked with verifying. 
\begin{theorem} \label{thm:metathm}
        If
        \[
		\WD(\Model; \InpSpace)_\infty < \peakToPeak(\Model, X) / 2
	\]
        for some $X \in \InpSpace$,
	then the output of $\ModelFinal$ is ``overwhelmed'' under the restriction.
\end{theorem}
\begin{proof}
	As $\ModelFinal$ always selects the token with the maximum logit value, if the coordinate-wise deviation of the restricted model's output never differs by more than $\peakToPeak(\desF{\Model}, X) / 2$ for any $X \in \InpSpace$, the token with the second highest logit value will never exceed that of the token with the highest value.
\end{proof}

Ultimately, we will want to make use of the above theorem alongside \cref{alg:overwhelmCheckDet} to prove the following theorem:
\begin{theorem}[Input Restriction] \label{thm:InpRes}
	%The model $\desF{\ModelFinal}$ over domain $[X]$ if 
	If
    \begin{align*}
	    &\WD(\desF{\Model}; \InpSpace)_\infty
        \\&< \peakToPeak(\desF{\Model}, X) / 2,
    \end{align*}
	then the output of model $\desF{\Model}$ is fixed for all inputs in $\InpSpace$.
	Moreover, we can use \cref{alg:overwhelmCheckDet} to produce an upper bound $W$ for $\WD(\desF{\Model}; \InpSpace)_\infty$.
\end{theorem}

To prove the above theorem, we will:
\begin{itemize}[nosep]
    \item break down the model into its components.
    \item bound the worst-case deviation of each component as a function of the blowup and shift from layer-normalization.
    \item use the triangle inequality of worst-case deviation to bound the worst-case deviation of the model prior to unembedding.
    \item use the Lipschitz constant of the unembedding matrix to bound the worst-case deviation of the model.
\end{itemize}
The formal proof can also be found in \cref{subsec:InpResProof}.

\begin{algorithm}[tb] 
	\caption{Algorithm for deciding Overwhelming}
	\label{alg:overwhelmCheckDet}
	\begin{algorithmic}
		\STATE {\bfseries Input:}  Model $\Model$, fixed string $s$, contenxt length $\nctx$, query token $q$.
		\STATE {\bfseries Output:} ``Overwhelmed'' or ``Inconclusive''.
		\STATE
		\STATE Calculate $B^{\min}, B^{\max}, S^{\min}, S^{\max}$ as in \cref{def:blowup_shift}.
            \STATE Calculate pre-softmax extremal logit values $\ell^{\min}$ and $\ell^{\max}$ via \cref{alg:lminlmax}.
		\STATE Using the above, calculate $\ssMaxRB$ and $\ssMinRB$ as in \cref{def:soft-extrem-values} and \cref{lem:min_max_softmax}.
		\STATE Calculate upper-bound $W^{\attn}$ as in \cref{lem:att_bound}.
		\STATE Calculate upper-bound $W^{\fenc}$ as in \cref{lem:WD_fenc}.
		\STATE Calculate \begin{align*}
			&W = \Lip(\Unembed) \\
                &\cdot \left(W^{\attn} + \LipMLP_\infty \cdot W^{\fenc} + W^{\fenc}\right)
		\end{align*}
		as per \cref{lem:mlpbound} and 
		\STATE Sample $X \gets \InpSpace$
		\IF{$W < \peakToPeak(\desF{\Model}, X) / 2$ }
		\STATE Return ``Overwhelmed''
		\ELSE
		\STATE Return ``Inconclusive''
		\ENDIF
	\end{algorithmic}
\end{algorithm}


%
%\begin{algorithm}[H] \label{alg:overwhelmCheckDet}
%	\SetKwInOut{Input}{Input}\SetKwInOut{Output}{Output}
%	\Input{A fixed single-layer transformer model $\Model$, a string of tokens $s$ denoting a fixed part of the input to $\Model$, integer $\nctx$ denoting the number of total tokens in the input to $\Model$, and a final query token $q$.}
%	\Output{Either the string ``Overwhelmed'' indicating that the output of the model $\Model$ evaluated on $s$ concatenated with $\nfree$ free tokens and query token $q$ is proven to be invariant under the choice of the free tokens, OR the string ``Inconclusive'' if no such proof is obtained.} 
%	Calculate $B^{\min}, B^{\max}, S^{\min}, S^{\max}$ as in \cref{def:blowup_shift}.
%	\\
%	Calculate $\ssMaxRB$ and $\ssMinRB$ as in \cref{def:soft-extrem-values} and \cref{lem:min_max_softmax}.
%	\\
%	Calculate \begin{align*}
%		W^{\attn} &= 2 \cdot (\ssMaxRB - \ssMinRB) \cdot \left(\max_{B \in [B^{\min}, B^{\max}]} \|E[\desSet \cup \{q\}, j] \cdot \diag(B)\|  + \max(\norm{S^{\min}}_\infty, \norm{S^{\max}}_\infty \right) \\
%			  &+ 2 \cdot \ssMaxFB \cdot \max_{B, S} \norm{(E B + S) \cdot V}_{\frobInf}
%	\end{align*}
%	as in \cref{lem:att_bound}.
%	\\
%	Calculate  $W^{\fenc} = \max_{t \in [\dVocab]} 
%	\max_{j \in [\dEmb]}
%	\max_{B \in [\vec{0}, B^{\max} - B^{\min}]}
%	\left|X[t] \cdot \Embed \cdot \diag(B) \cdot \vec{e}_j\right| + S_j^{\max} - S_j^{\min}$ as per \cref{lem:worst_case_deviation}.
%	\\
%	Calculate $W = \Lip(\Unembed) \cdot \left(W^{\attn} + \LipMLP_\infty \cdot W^{\fenc} + W^{\fenc}\right)$ 
%	as per \cref{lem:mlpbound} and 
%	\\
%	Calculate $\peakToPeak(\desF{\Model}, \InpSpace)$ by evaluating $\desF{\Model}$ on $X \gets \InpSpace$ \\
%	\If{
%	$W < \peakToPeak(\desF{\Model}, \InpSpace) / 2$ }{\Return ``Overwhelmed''
%}
%\Else{\Return ``Inconclusive''}
%
%\caption{Algorithm for deciding Overwhelming}
%\end{algorithm} 



\subsection{Proof Overview of \cref{thm:InpRes} (Algorithm Correctness)} \label{sec:algoCorrectness}
The proofs for all the lemmas and statements in this section are deferred to \cref{sec:proofs_framework}.

\subsubsection*{Breaking Down the Model}
Discussing normalization as a function of expectation and variance is a bit cumbersome.
So, we will first define an analagous ``blowup'' and ``shift'' for normalization.

\begin{definition}[Blowup and Shift Sets]
	\label{def:blowup_shift}
	For some $X \in \InpSpace$, Let $B_j(X)$ (for blowup) denote
	\[
		\frac{1}{\sqrt{\Var[X \Embed \cdot e_j^T] + \eps}} \cdot \gamma
	\]
	and let $S_j(X)$ (for shift) denote
	\[
		\frac{-\E[X \Embed \cdot e_j^T]}{\sqrt{\Var[X \Embed \cdot e_j^T] + \eps}} \cdot \gamma + \beta.
	\]
	We define $\blowupSet_j = \{B_j(X) | X \in  \InpSpace \}$, and  $\shiftSet_j  = \{S_j(X) | X \in  \InpSpace \}$.
	Moreover, for ease of notation, we will define
	\[
		\blowupSet = \{ (B_1(X), \dots, B_\dEmb(X) \mid X \in \InpSpace\}
	\]
	\[
		\shiftSet = \{ (S_1(X), \dots, S_\dEmb(X)) \mid X \in \InpSpace \}
	\]
	and the set $\blowupSet\shiftSet$ to equal
	\[
		\left\{ \bigg((B_1(X), \dots, B_\dEmb(X)), (S_1(X), \dots, S_\dEmb(X))\bigg)  \right\}
	\]
	where $X \in \InpSpace$.
	Finally, let $B^{\min}_j = \min(\blowupSet_j) $, $B^{\max}_j = \max(\blowupSet_j)$, $S^{\min}_j = \min(\shiftSet_j)$ and $S^{\max}_j = \max(\shiftSet_j)$.
\end{definition}

We will now break down the model in a way such that it is easier to analyze.
For component $\component \in \{\attn, \MLP, \Iden\}$, we will use $f^{\component} : \OneHotSpace^\nctx \times \blowupSet\shiftSet \to \R^{\nctx \times \dVocab}$ to denote the following function:
\[
	f^{\component}(X, (B, S)) = \component \circ \fenc(X, (B, S)) 
\]
where
\[
	\fenc(X, (B, S)) = (X \cdot \Embed \cdot \diag(B) + S).
\]
Note that $\fenc(X, (B(X), S(X)))$ is exactly equal to $\LayerNorm(X \cdot \Embed)$ for layernorm function $\LayerNorm$.

We can rewrite our model as
\[
    \Model = \Unembed \circ (f^{\attn} + f^{\MLP} + f^{\Iden}) 
\]
and so, by the monotonicity of lifting for worst-case deviation,
\begin{align*}    
    &\WD(\desF{\Model}; \InpSpace) \\
    \leq& \sum_{\component} \WD(\Unembed \cdot \vec{e}_\nctx \cdot \desF{f^{\component}}; \InpSpace \times \blowupSet \shiftSet)
\end{align*}
for $\component \in \{\attnH, \MLP, \Iden\}$.


% To bound the output behavior of a single-layer transformer model, we will need to bound the components of the model to get a valid upper bound $W$ on worst-case deviation where  $W$ is the calculated upper-bound in \cref{alg:overwhelmCheckDet}.

% We can then use Lipschitz constant for the unembedding matrix to get
% \begin{align*}
% 	&\WD(\desF{\Model}; \InpSpace)
%     \\ &\leq \sum_{\component} \WD(\Unembed \cdot \vec{e}_\nctx \cdot \desF{f^{\component}} ; \InpSpace \times \blowupSet \shiftSet)_\infty \\
% 			  &\leq
% 			  \Lip(\Unembed)_\infty \sum_{\component} \WD(\vec{e}_\nctx \cdot \desF{f^{\component}} ; \InpSpace \times \blowupSet \shiftSet)_\infty \\
% 			  &\leq \frac{1}{2} \cdot \peakToPeak(\desF{\Model}, X).
% \end{align*}
% for a large enough $\ndes$.

\subsubsection{Bounding Blowup and Shift}
We can get rather straightforward bounds on the blowup and shift of the model.
\begin{lemma}[Blowup and Shift Bounds]
	\label{lem:boundsBS}
	We bound blowup and shift: for every $B_j \in \blowupSet_j$ and $S_j \in \shiftSet_j$
	\[
		\frac{\gamma}{\sqrt{\Var_j^{\max} + \eps}} \leq B_j \leq \frac{\gamma}{\sqrt{\Var_j^{\min} + \eps}}
	\]
	and
	\begin{align*}
	&\min\left(B_j^{\min} \mu_j^{\min}, B_j^{\min} \mu_j^{\max}, B_j^{\max}, \mu_j^{\min}, B_j^{\max} \mu_j^{\max}\right)
	\leq
	S_j\\
	&\leq
	\max\left(B_j^{\min} \mu_j^{\min}, B_j^{\min} \mu_j^{\max}, B_j^{\max}, \mu_j^{\min}, B_j^{\max} \mu_j^{\max}\right)
	\end{align*}
	where $\mu^{\min}_j, \mu^{\max}_j$ are the lower and upper bounds on the expectation of the $j$-th column of $X \cdot \Embed$. $\Var_j^{\min}, \Var_j^{\max}$ are similarly defined for the variance.
    Both are bounded in \cref{lem:boundsVar} within \cref{sec:proofs_framework}.
\end{lemma}
Moreover, we let $B^{\min}$ be a vector of the minimum values of the blowup and $B^{\max}$ be a vector of the maximum values of the blowup.
Define $S^{\min}$ and $S^{\max}$ similarly for the shift.

\subsubsection{Bounding Attention}
\label{subsec:attention_bounds}
For a multi-headed attention mechanism, $\attn(X) = [\attnH_1(X); \ldots; \attnH_H(X)]$: i.e.\ the output is a concatenation of the individual attention head outputs.
So, the infinity norm worst-case deviation is simply bounded by the maximum worst-case deviation over the individual attention heads.
We thus have the following lemma:
\begin{lemma}
	\label{lem:att_bound_by_heads}
	\begin{align*}
		&\WD(\vec{e}_\nctx \cdot \desF{f^{\attn}} ; \InpSpace \times \blowupSet \shiftSet)_\infty 
	     \\ &\leq 
	     \max_h \WD(\vec{e}_\nctx \cdot \desF{f^{\attnH_h}} ; \InpSpace \times \blowupSet \shiftSet)_\infty.
	\end{align*}
\end{lemma}
The rest of this section will focus on bounding the worst-case deviation of a single attention head which will be the most challenging part of the proof.

We will show how to bound the contribution of each token to the attention weights by:
(1) establishing bounds for the attention weight coming from fixed tokens to the query token, (2) establishing an upper bound for the attention weight coming from free tokens to the query token.
We can then demonstrate that the attention weights are heavily biased towards the fixed tokens.

We implicitly consider rotary-type positional encodings in the attention mechanism, where the positional encodings are absorbed into the calculation of the logits.
As previously mentioned, we use $\ell$ to denote the logits on the query token.  It will be useful, in the proof of our results, to define and compute worst-case upper and lower bounds for $\ell$, which we define formally in \cref{def:lminlmax}.
% Denote the minimum and maximum values of the logits as $\ell^{\min}$ and $\ell^{\max}$ respectively.
To obtain bounds $\ell^{\min}$ and $\ell^{\max}$, we provide a simple algorithm in \cref{alg:lminlmax} to compute a bound on the logits:
\begin{lemma}[Bounds on Logits]
	\label{lem:lminlmax}
	Algorithm~\ref{alg:lminlmax} computes the minimum and maximum values of the logits $\ell$ given point-wise bounds $B^{\min}$, $B^{\max}$, $S^{\min}$, and $S^{\max}$ as specified in \cref{lem:boundsBS}.
\end{lemma}
% \vspace{-0.2cm}


% In the following lemmas, we represent the positions of restricted tokens (including the query token) as elements of the set $\{1, 2, \dots, \nfix \} \cup \{ \nctx \}$ (denoted by $[\nfix] \cup \{\nctx\}$), and the free tokens (including the query token for simplicity) as elements of the set $\{\nfix + 1, \ndes+2, \dots, \nctx - 1\}$ (denoted by $(\nfix, \nctx)$).
% \footnote{Technically, the query token is part of the restriction.
% However, to simplify the notation and the proof, we consider the query token as a free token without loss of generality.}

We now define the most consequential value to bound the worst-case deviation of the attention mechanism.
\begin{definition}[Softmax Extremal Values]\label{def:soft-extrem-values}
	Let $\softSumMinF = \sum_{j \in (\nfix, \nctx)} e^{\ell_j^{\min}}$ and $\softSumMaxF = \sum_{j \in (\nfix, \nctx)} e^{\ell_j^{\max}}$.
	Similarly, let $\softSumMinR = \sum_{i \in [\nfix] \cup \{\nctx\}} e^{\ell_i^{\min}}$ and $\softSumMaxR = \sum_{i \in [\nfix] \cup \{\nctx\}} e^{\ell_i^{\max}}$.
	Then, 
	\[
		\ssMinFB = \frac{\softSumMinF}{\softSumMaxR + \softSumMinF} \quad \text{and} \quad \ssMaxFB = \frac{\softSumMaxF}{\softSumMinR + \softSumMaxF}
	\]
	and
	\[
		\ssMinRB = \frac{\softSumMinR}{\softSumMinR + \softSumMaxF} \quad \text{and} \quad \ssMaxRB = \frac{\softSumMaxR}{\softSumMaxR + \softSumMinF}.
	\]
	%Let the lower-bound (resp. upper-bound) for softmax in \cref{eq:softmax_upper} be $\ssMinFB$ ($\ssMaxRB$) and
	%the lower-bound (resp. upper-bound) of softmax in \cref{eq:softmax_lower} be $\ssMinRB$ ($\ssMaxRB$).
\end{definition}

These extremal values can be used to get the bounds:

\begin{lemma}[Minimum and Maximum after Softmax]
	\label{lem:min_max_softmax}
	\begin{equation}
		\label{eq:softmax_upper}
		\ssMinFB \leq
		\sum_{j \in (\nfix, \nctx)} \softmax(\ell_j) \leq
		\ssMaxFB
	\end{equation}
	aswell as,
	\begin{align}
		\label{eq:softmax_lower}
		\ssMinRB \leq
		\sum_{j \in [\nfix] \cup \{\nctx\}} \softmax(\ell_j) \leq
		\ssMaxRB.
	\end{align}
\end{lemma}


In the following, $E[[\nfix] \cup \{\nctx\}, :] \in \R^{\nfix \times \dEmb}$ denotes the matrix with the rows in $[\nfix] \cup \{\nctx\}$ selected. 

\begin{lemma}[Worst-case Deviation of Attention]
	\label{lem:att_bound}
	Worst-case deviation of attention is bounded as follows,
	\begin{align*}
		&\WD(\vecNctx \cdot f^{\attnH}_{\ndes \mid r, q} ; \InpSpace \times \blowupSet \shiftSet)_\infty
	     \\ &\leq
	     2 \cdot (\ssMaxRB - \ssMinRB) \cdot \\
         &\max_{B, S}\norm{\Embed[[\nfix] \cup \{\nctx\}, :] \cdot \diag(B) + S}_{\frobInf} \\
		&+ 2 \cdot \ssMaxFB \cdot \max_{B, S} \norm{(E \cdot \diag(B) + S) \cdot V}_{\frobInf}\\
		&+ \norm{V}_\infty \cdot \ssMaxFB \cdot \max_{t \in s \cup \{q\}} \WD(\fenc; \{\vece_t \} \times \blowupSet \shiftSet\}))_\infty
	\end{align*}
        where $V$ is the value matrix in the attention head (see \cref{sec:appendix_model}).
\end{lemma}

Building on the above, we obtain a worst-case deviation bound using point-wise upper and lower bounds on $B \in \blowupSet$ and $S \in \shiftSet$.

%\begin{definition}[$B^{\min} $, $B^{\max}$, $S^{\min}$ ,$S^{\max}$]

%\end{definition}
\begin{corollary} \label{cor:Bmax} 
	Let $B^{\min} $, $B^{\max}$, $S^{\min}$, and $S^{\max}$ be as defined at the end of Definition \ref{def:blowup_shift}.
	Then, we have
	\begin{align*}
		\WD&(\vecNctx \cdot f^{\attnH}_{\ndes \mid r, q} ; \InpSpace \times \blowupSet \shiftSet)_\infty \leq  \\
		   & 2 \cdot (\ssMaxRB - \ssMinRB) \cdot \bigg(\max_{B \in [B^{\min}, B^{\max}]} \|E[[\nfix] \cup \{\nctx\}, :] \\
		   &\cdot \diag(B)\|_{\frobInf}  
	   + \max\big(\|{S^{\min}}\|_\infty, \|{S^{\max}}\|_\infty \big)\bigg) \\
		   &+ 2 \cdot \ssMaxFB \cdot \max_{B, S} \norm{(E B + S) \cdot V}_{\frobInf} \\
		   &+ \norm{V}_\infty \cdot \ssMaxFB \cdot \max_{t \in s \cup \{q\}} \WD(\fenc; \{\vece_t\} \times \blowupSet \shiftSet))_\infty
	\end{align*}
	where the maximization can be calculated with a simple linear program
    \footnote{
        We use the notation $B \in [B^{\min}, B^{\max}]$ to denote the set of vectors, $B$, where $B_j^{\min} \leq B_j \leq B_j^{\max}$.
    }.
\end{corollary}


\subsubsection{Bounding MLP and Identity}
Because the MLP and identity components are simple feed-forward networks without any ``cross-talk'' between tokens, we can use simple Lipschitz bounds.

\begin{lemma}[Worst-case Deviation of MLP and Identity]
	\label{lem:mlpbound}
	\begin{align*}
		\WD(\vec{e}_\nctx \cdot \desF{f^{\MLP}} ; \InpSpace \times \blowupSet \shiftSet)_\infty 
		\\ \leq
		\LipMLP_\infty \cdot \WD(\vec{e}_\nctx \cdot  \fenc; \InpSpace \times \blowupSet \shiftSet)_\infty
	\end{align*}
	and
	\begin{align*}
		\WD(\vec{e}_\nctx \cdot \desF{f^{\Iden}}  ; \InpSpace \times \blowupSet \shiftSet)_\infty \\ = \WD(\vec{e}_\nctx \cdot \fenc; \InpSpace \times \blowupSet \shiftSet)_\infty.
	\end{align*}
\end{lemma}


\newcommand{\ModelFinalC}{{\ModelFinal}'}
\newcommand{\eqclassAlgWD}{\texttt{BoundSoftmax}\WD}
\newcommand{\findAlpha}{\texttt{Find}\_\beta}
\newcommand{\findAlphaMin}{\text{Algorithm to find }\ssMinFA}
\newcommand{\findAlphaMax}{\text{Algorithm to find }\ssMaxFA}
\newcommand{\findAlphaMinR}{\text{Algorithm to find }\ssMinRA}
\newcommand{\findAlphaMaxR}{\text{Algorithm to find }\ssMaxRA}
\newcommand{\obj}{\text{objective}}
\newcommand{\freeToks}{\texttt{FreeToks}}

\newcommand{\minBilin}{\texttt{MinBilinear}}
\newcommand{\maxBilin}{\texttt{MaxBilinear}}
\newcommand{\lowerBounds}{\vec{r}^{\min}}
\newcommand{\upperBounds}{\vec{r}^{\max}}

\subsubsection{Bounding the Encoding Function}
\label{sec:concrcase}
We upper bound the worst-case deviation for $\fenc$ using
bounds on the ``blowup and shift'' in \cref{lem:boundsBS}:
\begin{lemma}[Worst-case deviation of $\fenc$]
	\label{lem:WD_fenc}
	\begin{align*}
		\WD&(\vec{e}_\nctx \cdot \fenc; \InpSpace \times \blowupSet \shiftSet)_\infty \leq 
		\\ \max_{t \in [\dVocab]} &
		\max_{j \in [\dEmb]}
		\max_{B \in [\vec{0}, B^{\max} - B^{\min}]} \\
		&\left|X[t] \cdot \Embed \cdot \diag(B) \cdot \vec{e}_j\right| + S_j^{\max} - S_j^{\min}
	\end{align*}
	where the inner maximum term can be computed via a simple linear program.
\end{lemma}

%Finally, we provide an algorithm which bounds the extremal values of the softmax, $\ssMinFB, \ssMaxFB, \ssMinRB, \ssMaxRB$ in \lev{TODO: appendix or not>}.
%Our algorithm for computing the extremal values of the softmax requires upper and lower bounding a bilinear form: we provide a simple, though non-optimal, algorithm for this task in \cref{fig:bilinOpt} in \cref{sec:appConcrcase}.
%			Now, we have the bound for $\fenc$, we simply need to bound the extremal values of softmax, $\softSumMinF, \softSumMaxF, \softSumMinR, \softSumMaxR$, to get a bound on the worst-case deviation.
%			\lev{TODO: alg reference here? I.e. point to lemma/ algorithm}


\iffalse
\begin{figure}[H]
	\begin{mdframed}
		$\findAlphaMinR:$%(\desF{\Model})$
		\begin{itemize}
			\item Let $\fenc(\vec{e}) = \vec{e} \cdot \Embed \cdot \diag(B) + S$
			\item For $j \in [s]$, let 
				\begin{align*}
					\vec{\ell}[j] = \frac{1}{\sqrt{\dEmb}} 
					\minBilin\bigg(Q \PosRot_{j, \nctx} K^T, \Embed[\desSet_j] \cdot \diag(B^{\min}) + S^{\min}, \\
					\Embed[\desSet_j] \cdot \diag(B^{\max}) + S^{\max} \bigg)
				\end{align*}
			\item Return $\sum_j \exp\left(\vec{\ell}_j\right)$
		\end{itemize}
		$\findAlphaMaxR:$%(\desF{\Model})$
		\begin{itemize}
			\item Let $\fenc(\vec{e}) = \vec{e} \cdot \Embed \cdot \diag(B) + S$
			\item For $j \in [s]$, let 
				\begin{align*}
					\vec{\ell}[j] = \frac{1}{\sqrt{\dEmb}} 
					\maxBilin\bigg(Q \PosRot_{j, \nctx} K^T, \Embed[\desSet_j] \cdot \diag(B^{\min}) + S^{\min}, \\
					\Embed[\desSet_j] \cdot \diag(B^{\max}) + S^{\max} \bigg)
				\end{align*}
			\item Return $\sum_j \exp\left(\vec{\ell}_j\right)$
		\end{itemize}
		$\findAlphaMax:$%(\desF{\Model})$
		\begin{itemize}
			\item Let $\fenc(\vec{e}) = \vec{e} \cdot \Embed \cdot \diag(B) + S$
			\item For $j \in (\ndes, \nctx]$, let 
				\begin{align*}
					\vec{\ell}[j] = \frac{1}{\sqrt{\dEmb}} \max_{t \in [\dVocab]} 
					\maxBilin\bigg(Q \PosRot_{j, \nctx} K^T, \Embed[t] \cdot \diag(B^{\min}) + S^{\min}, \\
					\Embed[t] \cdot \diag(B^{\max}) + S^{\max} \bigg)
				\end{align*}
			\item Return $\sum_j \exp\left(\vec{\ell}_j\right)$
		\end{itemize}

	\end{mdframed}
	\caption{The algorithm $\findAlpha$ which computes the pre-softmax logits for a given model $\Model$.
	}
	\label{fig:find_ell_alg}
\end{figure}

\fi

%	\begin{algorithm}[H] \label{alg:overwhelmCheckDetLP}
%		\SetKwInOut{Input}{Input}\SetKwInOut{Output}{Output}
%		\Input{$\ssMinRB, \ssMaxRB, \ssMaxFB, \softSumMaxR$}
%		\Output{\lev{TODO}} 
%		Calculate \begin{align*}
%			W^{\attn} &= 2 \cdot (\ssMaxRB - \ssMinRB) \cdot \left(\max_{B \in [B^{\min}, B^{\max}]} |E[\desSet, j] \cdot \diag(B)|  + \max(\norm{S^{\min}}_\infty, \norm{S^{\max}}_\infty \right) \\
%				  &+ 2 \cdot \ssMaxFB \cdot \max_{B, S} \norm{(E B + S) \cdot V}_{\frobInf}
%		\end{align*}
%		as in \cref{lem:att_bound}.
%		\\
%		Calculate $W = \Lip(\Unembed) \cdot \left(W^{\attn} + \LipMLP_\infty \cdot W^{\fenc} + W^{\fenc}\right)$ 
%		as per \cref{lem:mlpbound} and 
%		\\
%		Calculate $\peakToPeak(\desF{\Model}, \InpSpace)$ by evaluating $\desF{\Model}$ on $X \gets \InpSpace$ \\
%		\If{
%		$W < \peakToPeak(\desF{\Model}, \InpSpace) / 2$ }{\Return ``Squashed''
%	}
%	\Else{\Return ``Inconclusive''}
%	\caption{Meta Algorithm for Calculating bound on $\WD(\desF{\Model}, [X])$}
%\end{algorithm} 


%\newcommand{\findAlpha}{\texttt{Find}\_\beta}
%\newcommand{\findAlphaMin}{\texttt{Find}\_\ssMinFB}
%\newcommand{\findAlphaMax}{\texttt{Find}\_\ssMaxFB}
%\newcommand{\findAlphaMinR}{\texttt{Find}\_\ssMinRB}
%\newcommand{\findAlphaMaxR}{\texttt{Find}\_\ssMaxRB}

\section{Overwhelming Under Permutation Invariance}
\label{sec:perm_invar}
In \cref{sec:meta_framework}, we provided a concrete algorithm that decides overwhelming for a fixed context size $\nctx$.
In this section, we provide a more refined algorithm which can provide a tighter bound on the worst-case deviation in the restricted setting where the free tokens are allowed to be any permutation of a fixed string.
Up until this point, we have been ``lifting'' the domain of $\InpSpace$ into $\InpSpace \times \blowupSet \shiftSet$ to prove worst-case deviation bounds: i.e.\ we have been separating out the effect of layer normalization when proving a fixed string to be overwhelming.

In this section, we nullify the effect of layer normalization by defining an equivalence relation on $\InpSpace$ where the blowup and shift sets are the same for all elements in the equivalence class.
Then, we can derive a simpler upper-bound on the worst-case deviation of the model over the equivalence class as we can treat the layer normalization as a constant factor!

\begin{definition}[Permutation Equivalence relation]
	We will define the equivalence relation \(\sim\) on \(\InpSpace\).
	Let $X = (\desSet || X_f || \vecQuery) \in \InpSpace$ where $X$ are the free tokens, $\desSet$ are the fixed tokens, and $\vecQuery$ is the query token.
	Then, we define \(\sim\) such that for $X = (\desSet || X_f || \vecQuery), Y = (\desSet || Y_f || \vecQuery) \in \InpSpace$,
	\[
		X \sim Y \iff X_f = \pi(Y_f)
	\]
	where $\pi$ is a permutation of the free tokens.
	Note that if $X \sim Y$, then $\E[X \cdot \Embed] = \E[Y \cdot \Embed]$ and $\Var[X \cdot \Embed] = \Var[Y \cdot \Embed]$.
    Moreover, let $\freeToks$ be the multi-set of potential free tokens for $[X]$.
    I.e. $\freeToks = \{X_f[1], \dots, X_f[\nfree]\}$.
\end{definition}
% We abuse notation to define $B(X)$ to be the vectors for $X$'s blowup and $S(X)$ to be the vectors for $X$'s shift.

To bound the worst-case deviation over an equivalence class we use a similar algorithm (\cref{alg:overwhelmCheck2}) to \cref{alg:overwhelmCheckDet}.
We still bound the worst-case deviation of attention, but now the blowup and shift sets are singletons, leading to much tighter bounds.
Additionally, since blowup and shift are constant, the worst-case deviations of $f^{\MLP}, f^{\Iden}$ are $0$, and, we can use a linear program to find the extremal softmax contributions for the free tokens.

%\snote{TODO: Write preamble}
%Up until this section, we have been ``lifting'' the domain of individual components of the model to prove worst-case deviation bounds.
%The lifting, done by seperating out the effect of layer normalization, suggests that we should be able to ``factor out'' the effect of layer normalization in the model.
%Then, this equivalence relation partitions \(\InpSpace\) into disjoint equivalence classes, denoted by \([X] = \{Y \in \InpSpace \mid Y \sim X\}\), where each class contains all elements with the same variance and expectation.

\begin{theorem}[Input Restriction and Permutation Invariance] \label{thm:InpResPermInv}
	%The model $\desF{\ModelFinal}$ over domain $[X]$ if 
	If
	\[
		\WD(\desF{\Model}; [X])_\infty < \peakToPeak(\desF{\Model}, X) / 2
	\]
	% where $\blowupSet = \{B(X)\}$ and $\shiftSet = \{S(X)\}$ (i.e.\ blowup and shift are just singelton sets),
	then the output of model $\desF{\Model}$ is fixed for all inputs in $[X]$.
	Moreover, \cref{alg:overwhelmCheck2} produces an upper bound $W$ for $\WD(\desF{\Model}; \InpSpace)_\infty$.
	%\footnote{There is a notion here of ``sample-complexity'' which we will not go into and leave for future work
	%	Specifically, if we treat $X_1, \dots, X_D$ as samples from $\InpSpace$ and take the peak-to-peak difference to be $\max_{i \in [D]} \peakToPeak(\desF{\Model}, X_i) / 2$, then we have a \emph{larger bound} and thus can get away with larger worst-case deviation while still proving domain-collapse.
	%	In theory, the larger $D$ is, the more likely we are to have a larger peak-to-peak difference.
	%}
\end{theorem}
A full proof of \cref{thm:InpResPermInv} is provided in \cref{subsec:proofInpResPermInv}.
% Moreover, we prove the correctness of \cref{alg:overwhelmCheck2} in \cref{sec:appendix_perm}.
%\begin{algorithm}[H] \label{alg:overwhelmCheck2}
%	\SetKwInOut{Input}{Input}\SetKwInOut{Output}{Output}
%	\Output{Either the string ``Overwhelmed'' indicating that the output of the model $\Model$ evaluated on $s$ concatenated with $\nfree$ free tokens and query token $q$ is proven to be invariant under a permutation class, OR the string ``Inconclusive'' if no such proof is obtained.} 
%
%	Calculate $
%	B(X) = \frac{\gamma}{\sqrt{\Var[X \cdot E] + \eps}}
%	$
%	and $
%	S(X) = - \E[X \cdot E] B(X) + \beta
%	$
%	Set $B^{\min}, B^{\max} = B$ and $S^{\min}, S^{\max} = S$
%	\\
%	Set $\fenc(\vec{e}) = \vece \cdot E \cdot \diag(B) + S$ and
%	calculate bounds for $\ssMinFA, \ssMaxFA$ using either the naive algorithm in \cref{alg:naiveAlpha} or the linear program in \cref{alg:LPAlpha}.\\
%	Let
%	\[
%		\alpha_{fix} = 
%		\exp\left(
%			\fenc(\vec{e}_q) \cdot Q \Theta_{\nctx, \nctx} K^T \fenc(\vec{e}_{q})
%		\right)    
%		+  \sum_{j \in [\nfix]} \exp\left(
%			\fenc(\vec{e}_q) \cdot Q \Theta_{i, \nctx} K^T \fenc(\vec{e}_{\desSet_j})
%		\right)
%	\]
%	\\
%	Calculate $\ssMaxFB = \frac{\ssMaxFA}{\alpha_{fix} + \ssMaxFA}$ and $\ssMaxFB = \frac{\ssMinFA}{\alpha_{fix} + \ssMinFA}$ \\
%	Set $W^{\fenc} = 0$ to bound the worst-case deviation of $\fenc$ to $0$ \\
%	Let $\ssMaxFB = \frac{\ssMaxFA}{\ssMaxFA + \alpha_{fix}}$,
%	$\ssMinRB = \frac{\alpha_{fix}}{\alpha_{fix} + \ssMinFA}$,
%	and $\ssMinRB = \frac{\alpha_{fix}}{\alpha_{fix} + \ssMaxFA}$
%	\\
%	Calculate \begin{align*}
%		W^{\attn} &= 2 \cdot (\ssMaxRB - \ssMinRB) \cdot \|E[\desSet \cup \{s\}, :] \cdot \diag(B) + S\|_{\frobInf} \\
%			  &+ 2 \cdot \ssMaxFB \norm{(E[\freeToks, :] B + S) \cdot V}_{\frobInf}
%	\end{align*}
%	as in \cref{lem:att_bound} except that we have $\norm{(E[\freeToks, :] B + S) \cdot V}_{\frobInf}$ 
%	instead of $\norm{(E B + S) \cdot V}_{\frobInf}$ \footnote{
%		Recall that $E[\freeToks, :]$ connotes selecting the rows of $E$ corresponding to the tokens in $\freeToks$.
%	}\\
%
%	Calculate $W = \Lip(\Unembed) \cdot W^{\attn} $ 
%	Calculate $\peakToPeak(\desF{\Model}, \InpSpace)$ by evaluating $\desF{\Model}$ on $Y \gets [X]$ \\
%	\If{
%	$W < \peakToPeak(\desF{\Model}, \InpSpace) / 2$ }{\Return ``Overwhelmed''
%}
%\Else{\Return ``Inconclusive''}
%
%\caption{Overwhelming Verifier for Permutation Invariance}
%\end{algorithm} 

\begin{algorithm}[h] 
	\caption{Algorithm for deciding Overwhelming}
	\label{alg:overwhelmCheck2}
	\begin{algorithmic}
		\STATE \textbf{Input:} $\Model, \desSet, \query$
		\STATE \textbf{Output:} Return ``Overwhelemed'' or ``Inconclusive''
		\STATE
		\STATE Set $B = \frac{\gamma}{\sqrt{\Var[X \cdot E] + \eps}}
		$
		\STATE Set $
		S = - \E[X \cdot E] B + \beta
		$
		\STATE Set $\fenc(\vec{e}) = \vece \cdot E \cdot \diag(B) + S$
		\STATE Calculate $\ssMinFA, \ssMaxFA$ using the naive algorithm (\cref{alg:naiveAlpha}) or the linear program (\cref{alg:LPAlpha})
		\STATE Let
		\begin{align*}
			\alpha_{fix} = 
			\exp\left(
				\fenc(\vec{e}_q) \cdot Q \Theta_{\nctx, \nctx} K^T \fenc(\vec{e}_{q})
			\right)    \\
			+  \sum_{j \in [\nfix]} \exp\left(
				\fenc(\vec{e}_q) \cdot Q \Theta_{i, \nctx} K^T \fenc(\vec{e}_{\desSet_j})
			\right)
		\end{align*}
		\STATE Calculate $\ssMaxFB$, $\ssMinRB$, $\ssMinRB$
		 \STATE 
		 Calculate $W^{\attn}$ as in \cref{lem:attn_perm}
		\STATE Calculate $W = \Lip(\Unembed) \cdot W^{\attn} $ 
		% \STATE Sample $Y \gets [X]$ 
		\IF{
		$W <  \peakToPeak(\desF{\Model}, X) / 2$ }
		\STATE Return ``Overwhelmed''
		\ELSE 
		\STATE Return ``Inconclusive''
		\ENDIF
	\end{algorithmic}
\end{algorithm}



\subsection{Bounding Attention}
% Similar to the previous section, we need to bound the worst-case deviation of attention.
% Unlike the previous section though, because the blowup and shift do not vary, we can find \emph{tighter} bounds on attention's worst-case deviation.
To bound the worst-case deviation of attention, we provide two algorithms to bound $\ssMinFA, \ssMaxFA$ as defined in \cref{lem:min_max_softmax}.
The first, in \cref{alg:naiveAlpha}, takes a ``naive'' approach by iterating through the position of the free tokens and finding an extremal value for each position.
The second, in \cref{alg:LPAlpha}, uses a linear program to get a tighter bound.
To see why \cref{alg:LPAlpha} provides a valid bound, consider a similar program except where an \emph{integer} linear program is used.
Then, the integer version of \cref{alg:LPAlpha} finds the permutation of free tokens which maximizes (resp. minimizes) the pre-softmax logits.
Relaxing to a linear program, we have an upper-bound (resp. lower-bound) on the pre-softmax logits.
We can formally state the correctness of \cref{alg:naiveAlpha} and \cref{alg:LPAlpha} in the following lemma:
\begin{lemma}[Correctness of \cref{alg:LPAlpha} and \cref{alg:naiveAlpha}]
	\label{lem:corrAlpha}
	 Both \cref{alg:LPAlpha} and \cref{alg:naiveAlpha} provide valid bounds on $\ssMinFA$ and $\ssMaxFA$.
\end{lemma}
\vspace{-0.2cm}

\begin{algorithm}[tb]
	\caption{
		Linear Program to find $\ssMinFA$.\\
		To find $\ssMaxFA$, switch the $\min$ to a $\max$ in the objective.
	}
	\label{alg:LPAlpha}
	\begin{algorithmic}
	\STATE \textbf{Input:} {$\fenc$ and model $\Model$}
	\STATE
	\STATE Return the optimal value to the following linear program:
	\begin{align*}
		\min \quad & \sum_{j \in (\ndes, \nctx)} \exp\bigg(\sum_{t \in \freeToks} \\
			\fenc(\vecQuery) \cdot &Q \cdot \PosRot_{j, \nctx} K^T \cdot \fenc(\vec{e}_t)^T
	\bigg) \cdot x_{j, t} \\
		\text{subject to} \quad
			   & \forall j, \sum_{t \in \freeToks} x_{j, t} = 1 \\
			   & \forall t, \sum_{j \in (\ndes, \nctx)} x_{j, t} = 1, \\
			   &  \forall j, t, x_{j, t} \geq 0.
	\end{align*}
	\end{algorithmic}
\end{algorithm} 

Then, with the bounds on the softmax contributions, we can state a specialized version of the bound on attention's worst-case deviation for the permutation invariant case.

Then, we can state the following lemma:
\begin{lemma}[Worst-case Deviation of Attention]
	\label{lem:attn_perm}
	Worst-case deviation of attention is bounded as follows for fixed blowup $B$ and shift $S$ when considering the permutation class $[X]$ with free tokens $\freeToks$:
	\begin{align*}
		&\WD(\vecNctx \cdot f^{\attnH}_{\ndes \mid r, q} ; \InpSpace \times \blowupSet \shiftSet)_\infty
	     \\ &\leq
		2 \cdot (\ssMaxRB - \ssMinRB) \cdot \norm{\Embed[[\nfix] \cup \{\nctx\}, :] \cdot B + S}_{\frobInf} \\
		&+ 2 \cdot \ssMaxFB \cdot \norm{(E[(\nfix,\nctx),: ] B + S) \cdot V}_{\frobInf}.
	\end{align*}
\end{lemma}
The proof follows analogously to that of \cref{lem:att_bound} except for two main differences:
(1) we have a further restriction on the free tokens and thus only need to consider $E[(\nfix,\nctx), :]$ instead of $E$
and (2) becuase the blowup and shift are fixed, $\WD(\fenc; \{\vece_t\} \times \blowupSet \shiftSet)_\infty = 0$ and so wecan drop the last term in the original bound.


%%%%%%%%%%%%%%%%%%%%%%

%\begin{algorithm}[H] \label{alg:naiveAlpha}
%	\SetKwInOut{Input}{Input}\SetKwInOut{Output}{Output}
%	\Input{$\fenc$ and model $\Model$}
%	% \Output{\lev{TODO}}
%	Let $\fenc(\vec{e}) = \vec{e} \cdot \Embed \cdot \diag(B) + S$ \\
%	\For{$k \in (\ndes, \nctx]$}{
%		Let \begin{equation*}
%			\vec{\ell}^{\min}_k = \frac{1}{\sqrt{\dEmb}} \min_{t \in \freeToks}  \fenc(\vecQuery) \cdot Q \PosRot_{k, \nctx} K^T \cdot \fenc(\vec{e}_t)^T
%		\end{equation*}
%		and 
%		\begin{equation*}
%			\vec{\ell}^{\max}_k = \frac{1}{\sqrt{\dEmb}} \max_{t \in \freeToks}  \fenc(\vecQuery) \cdot Q \PosRot_{k, \nctx} K^T \cdot \fenc(\vec{e}_t)^T
%		\end{equation*}
%		where $\PosRot_{k, \nctx}$ is the matrix representing the effect of RoPE encodings
%	}
%	Return $\ssMinFA = \sum_k \exp\left(\vec{\ell}^{\min}_k\right), \ssMaxFA = \sum_k \exp\left(\vec{\ell}^{\max}_k\right)$
%	\caption{Naive Algorithm to find $\ssMinFA$ and $\ssMaxFA$}
%\end{algorithm} 
%
%\begin{algorithm}[H] \label{alg:LPAlpha}
%	\SetKwInOut{Input}{Input}\SetKwInOut{Output}{Output}
%	\Input{$\fenc$ and model $\Model$}
%	% \Output{\lev{TODO}}
%
%	Return the optimal value to the following linear program:
%	\begin{align*}
%		\min \quad & \sum_{j \in (\ndes, \nfree]} \exp\left(\sum_{t \in \freeToks} \fenc(\vecQuery) \cdot Q \cdot \PosRot_{j, \nctx} K^T \cdot \fenc(\vec{e}_t)^T\right) \cdot x_{j, t} \\
%		\text{subject to} \quad
%			   & \forall j, \sum_{t \in \freeToks} x_{j, t} = 1 \\
%			   & \forall t, \sum_{j \in (\ndes, \nfree]} x_{j, t} = 1 \\
%			   & \forall j, t, x_{j, t} \geq 0.
%	\end{align*}
%	\caption{Linear Program to find $\ssMinFA$.
%		To find $\ssMaxFA$, we have the same linear program except that we switch the $\min$ to a $\max$ in the objective. 
%	}
%\end{algorithm} 
%

% \begin{algorithm}[H] \label{alg:freezecheckerLP}
% 	\SetKwInOut{Input}{Input}\SetKwInOut{Output}{Output}
% 	\Input{A fixed transformer model M, a string of tokens $s$ denoting a fixed part of the input to M, and integer $k$ denoting the number of free tokens in the input to M }
% 	\Output{Either the string "Frozen" indicating that the output of the model M evaluated on $s$ concatenated with $k$ free tokens is proven to be invariant under the choice of the free tokens, OR the string "Inconclusive" if no such proof is obtained.} 

% \mnote{Populate this area with the precise pseudocode for the computation of the LP upper bound to $\WD(\desF{\Model}; [X] \times \blowupSet \shiftSet)_\infty$ using the terminology set in the rest of the paper.}



% For each $k \in [\tau]$ define $LP^+_k$ to be the value of the following linear program and let $\vec{\alpha}^k_+$ be the vector minimizing the value of the following linear program:

% \begin{align}
%     &\textbf{ Minimize:} \nonumber \\
%     &&\vec{\alpha} \cdot \grad E(\mathcal{F})  \\
%      &\textbf{Over:} \nonumber \\
%      && \vec{\alpha} \in \mathbb{R}^{\tau} \\
%     &\textbf{Subject To:} \nonumber \\
%     &&\vec{\alpha} \cdot \grad D_{ij}(\mathcal{F}) \geq 0 \text{    } \forall (i,j) \in \text{Saturated}(\mathcal{F}) \\
%     &&\text{ and } \nonumber\\
%     &&\alpha_k \geq \delta   \\
%     && \| \vec{\alpha} \|_{\infty} \leq \epsilon
% \end{align}


% For each $k \in [\tau]$ define $LP^-_k$ to be the value of the following linear program and let $\vec{\alpha}^k_-$ be the vector minimizing the value of the following linear program:
% \begin{align}
%     &\textbf{ Minimize:} \nonumber \\
%     &&\vec{\alpha} \cdot \grad E(\mathcal{F})  \\
%     &\textbf{Over:} \nonumber \\
%      && \vec{\alpha} \in \mathbb{R}^{\tau} \\
%     &\textbf{Subject To:} \nonumber \\
%     &&\vec{\alpha} \cdot \grad D_{ij}(\mathcal{F}) \geq 0 \text{    } \forall (i,j) \in \text{Saturated}(\mathcal{F}) \\
%     &&\text{ and } \nonumber\\
%     &&\alpha_k \leq -\delta   \\
%     && \| \vec{\alpha} \|_{\infty} \leq \epsilon 
% \end{align}



% \If{
% 	$
% 		\WD(\desF{\Model}; [X] \times \blowupSet \shiftSet)_\infty < \peakToPeak(\desF{\Model}, X) / 2
% 	$  \mnote{this needs to be expanded to show the actual if decision we compute, which is an upper bound on the worst case deviation rather than the exact wcd}}{\Return ``Frozen''  }

% \Else{\Return ``Inconclusive''}



% 	\caption{Freeze Verifier - LP Bound  \mnote{Under Construction}}
% \end{algorithm} 


After which the proof of \cref{thm:InpResPermInv} follows analogously to that of \cref{thm:InpRes}.

% \begin{proof}[Proof sketch of \cref{thm:InpResPermInv}]
%     First note that because the blowup and shift are constant for $[X]$, then the worst-case deviation of $\fenc$ is $0$ as the encoding function is fixed to a linear function for the domain $[X]$.
%     Then, noting that \cref{alg:overwhelmCheck2} uses the same framework as \cref{alg:overwhelmCheckDet}, we now need to prove that $\ssMinFA, \ssMaxFA, \ssMinRA, \ssMaxRA$ are valid extremal softmax bounds,
% \end{proof}


% and the algorithm in \cref{alg:overwhelmCheckDet}, we provide a simple proof of the above theorem.


% \lev{TODO: remove this ``proof'' structure and replace with filling in missing pieces}


% \begin{proof}
% 	First note that
% 	\[
% 		\WD(\vecNctx \cdot \desF{f^{\component}}; [X] \times \blowupSet \shiftSet ) = 0
% 	\]
% 	for $\component \in \{\MLP, \Iden\}$ as fixing the query token \emph{and the blowup/ shift} fixes the output of the MLP and identitiy for the last token.

% 	Now, we have to find the extremal softmax contributions for the fixed and free tokens.
% 	For the fixed tokens, because variance and expectation are fixed, we can note that there logits always equal
% 	\[
% 		\vec{\ell}_i = \frac{1}{\sqrt{\nctx}} \fenc(\vecDes) \cdot Q \PosRot_{i, \nctx} K^T \cdot \fenc(\vecQuery)
% 	\]
% 	for $i \in [\ndes]$ where $\PosRot_{i, \nctx}$ is the rotation matrix for RoPE position $i$.

% 	So then, we need to bound $\softSumMinF$ and $\softSumMaxF$.
% 	A naive apporach to computing $\softSumMaxF$ (resp. $\softSumMinF$) is outlined in \cref{fig:find_ell_alg_naive}.
% 	The algorithm simply finds the maximum (resp. minimum) of the pre-softmax logits for the free tokens at each position.
% 	Alternatively, we can formalize finding $\softSumMaxF$ (resp. $\softSumMinF$) as a mixed integer linear program (MILP) as outlined in \cref{fig:find_ell_alg_LP}.
% 	The key idea is that we can think of the class of permutation matrices $X \in \{0, 1\}^{\nfree \times \nfree}$ where $X[j, t] = x_{j, t}$ as being specified by a set of linear-integer constraints.
% 	Then, if we find the $X$ which maximizes (resp. minimizes) the sum of the pre-softmax logits, we can get a bound on the softmax sum.
% 	Of course, integer-linear programming is NP-hard,
% 	so we can relax the constraints to a linear program (LP) and get a bound on the softmax sum.
% 	%Then, we can use the results of \cref{lem:min_max_softmax} to get bounds on the softmax sum.

% 	Next, we can use the bounds on $\ssMaxFB, \ssMinRB, \ssMaxRB$ alongside \cref{lem:min_max_softmax} to get bounds on the worst-case deviation of the model from softmax.  \mnote{I think we should specify precisely how we do this in pseudocode inside Algoriths \ref{alg:overwhelmCheck} and \ref{alg:freezecheckerLP}.}

% 	Finally, we can use the lifted worst-case deviation to get a bound on the worst-case deviation of the model conditioning on a fixed $B$ and $S$.
% \end{proof}


\section{Evaluating on a Single Layer Model}
\label{sec:model_eval}

In this section, we empirically demonstrate our algorithm on a single layer transformer model trained for next token prediction on a standard text corpus.
Specifically, we run \cref{alg:overwhelmCheck2} to calculate a bound on the worst-case deviation and peak-to-peak difference for the model where the domain has an input restriction with free tokens drawn from a single permutation class as in \cref{sec:perm_invar}. When \cref{alg:overwhelmCheck2} outputs $\OverwQ$ it follows from \cref{thm:InpResPermInv} (informally, \cref{thm:informal_overwhelm_perm} in the Introduction) that the output of the model is the same for all permutations.

\subsection{Experimental Setup}
We train a single-layer transformer model with an embedding dimension of 512 and the BERT tokenizer \cite{devlin-etal-2019-bert}.
The model's architecture is outlined in \cref{sec:model}.
We train on the AG News dataset \footnote{see \href{https://pytorch.org/text/stable/_modules/torchtext/datasets/ag_news.html}{PyTorch's documentation}}, using the training split with a batch size of 8, learning rate of 5e-5 using the Adam optimizer, and $20$ epochs.

We examine a few different types of input restrictions
\footnote{
    The first token is always fixed to the ``BOS'' token which connotes the beginning of a string in the BERT tokenizer. For simplicity, we think of the BOS token as part of the fixed tokens.
}:
\begin{itemize}[nosep]
	\item \textbf{Random String:} Randomly sample alpha-numeric tokens (including space and punctuation) to form $\desSet$
	\item \textbf{Random Sentences:} Sample (and concatenate) sentences from the AG News testing set, to form $\desSet$
	\item \textbf{Repeating Tokens:} Form $\desSet$ by repeating the string ``what is it'' (which is $3$ tokens in the BERT tokenizer).
\end{itemize}

In each case, we have a \emph{tunable} number of tokens in $\desSet$\footnote{Larger input restrictions are produced by concatenation.}.
For the permutation class, we consider the following two cases:

\begin{itemize}[nosep]
	\item \textbf{Hamlet}: We use the famous quote from Shakespeare's \emph{Hamlet}:
    % \begin{quote}
    ``To be or not to be, that is the question. Whether it is a nobler in the mind to suffer the slings and arrows''
        % \end{quote}

	\item \textbf{The Old Man and the Sea}: We fix the free tokens to be a snippet from the opening chapter of Hemingway's \emph{The Old Man and the Sea}:
		% \begin{quote}
		 ``The blotches ran well down the sides of his face and his hands had the deep-creased scars from handling heavy fish on the cords.''	
		% \end{quote}
\end{itemize}

Finally, the question mark token ``?'' is used as the query token.

In \cref{fig:smallex}, we show our algorithm in action for the ``Hamlet'' permutation class and repeated fixed string.
In \cref{sec:appOut}, we plot the worst-case deviation in \cref{fig:worst-case} and the peak-to-peak difference in \cref{fig:ptp} for each of the input restrictions and permutation classes.

\begin{figure}
    \centering
\includegraphics[width=\linewidth,keepaspectratio,clip]{figs/plots/w_analysis_single.pdf}
    \caption{
        Worst-case deviation for the ``Hamlet'' permutation string and fixed repeated string.
        The $x$-axis is the number of total tokens and the $y$-axis is the log-scale worst-case deviation.
        We mark cases of provable overwhelming with a red ``x.''
    }
    \label{fig:smallex}
\end{figure}

\subsection*{Observations}
The experimental results for worst-case deviation bounds in \cref{fig:worst-case} and peak-to-peak difference in \cref{fig:ptp} provide interesting insights and questions.
The upper-bound on worst-case deviation trend downwards as the fixed input token size is increased.
But only the ``repeated-string'' setting is monotonically decreasing.
Moreover, all plots exhibit a ``phase transition:'' at around 180 tokens and again around 350 tokens, the upper-bound on worst-case deviation has a sharp drop.
We note that the linear program seems to be the most useful for the ``AG News'' dataset when paired with the \emph{The Old Man and the Sea} free string.
Finally, we note that in both plots, ``not much'' seems to happen prior to 180 tokens.
Specifically, both the peak-to-peak difference and bound on worst-case deviation remain flat for the smaller fixed string.

\subsection{Overwhelming Continues Through Generation}
When a transformer model is \overwQ the generated token immediately after the text is fixed for any free string. However, this does not apriori imply that subsequent tokens when the model is used to repeatedly generate tokens will be fixed. 

To test this we run examples of text generation where we begin with an overwhelming string $s$ and use the model to generate text. We then check if the model remains \overwQ as the generated text is included as part of the fixed string. The results of these tests are plotted in \cref{fig:contOverwhelm} within \cref{sec:appOut}. One would hope the model remains overwhelmed as the fixed string is increased by the addition of the newly generated tokens. This occurs for the Hamlet free string with the repeated and random fixed strings, where as expected the worst-case deviation remains mostly constant throughout token generation. For the longer free string and the AG News fixed strings the worst-case deviation peaks after a few tokens and the model stops to be provably \overwQ.

\section*{Conclusion}
This paper aims to enhance our understanding of the computational complexity of computing various Shapley value variants. We found that for various ML models --- including decision trees, regression tree ensembles, weighted automata, and linear regression --- both local and global interventional and baseline SHAP can be computed in polynomial time under HMM modeled distributions. This extends popular algorithms, such as TreeSHAP, beyond their empirical distributional scope. We also establish strict complexity gaps between the various SHAP variants (baseline, interventional, and conditional) and prove the intractability of computing SHAP for tree ensembles and neural networks in simplified scenarios. Overall, we present SHAP as a versatile framework whose complexity depends on four key factors: \begin{inparaenum}[(i)] \item model type, \item SHAP variant, \item distribution modeling approach, \item and local vs. global explanations\end{inparaenum}. We believe this perspective provides deeper insight into the computational complexity of SHAP, paving the way for future work.




%We believe that our framework provides a more intricate understanding of SHAP computation complexity across different models, distributions, and variants, paving the way for further research.

Our work opens promising directions for future research. First, expanding our computational analysis to other SHAP-related metrics, such as asymmetric SHAP~\citep{frye20} and SAGE~\citep{covert2020understanding}, would be valuable. Additionally, we aim to explore more expressive distribution classes and relaxed assumptions beyond those in Section \ref{sec:tractable} while maintaining tractable SHAP computation. Finally, when exact computation is intractable (Section \ref{sec:intractable}), investigating the approximability of SHAP metrics through approximation and parameterized complexity theory~\citep{downey2012parameterized} is an important direction.

%Our work opens several promising avenues for future research on the computational properties of explainable AI methods, with a particular focus on SHAP. First, it would be interesting to broaden the computational analysis conducted in this work to include other popular SHAP-related metrics in the literature, such as asymmetric SHAP \cite{frye20} and SAGE \cite{covert2020understanding}. Also, in the future, we aim to explore more expressive distribution classes and relaxed distributional assumptions—extending beyond those examined in Section \ref{sec:tractable} —that still yield tractable SHAP computation. Finally, when exact computation proves intractable (Section \ref{sec:intractable}), it is worthwhile to theoretically investigate the question of the approximability of computing the SHAP metrics across various configurations, through the lens of approximation and parametrized complexity theory \cite{arora2009computational}.

%This paper aims to deepen our understanding of the computational complexity involved in obtaining different Shapley value variants. We found that for a variety of ML models, including decision trees, tree ensembles for regression, weighted automata, and linear regression models — computing both local and global interventional and baseline SHAP can be done in polynomial time when distributions are modeled by HMMs. This extends the distributional scope of popular algorithms like TreeSHAP, which is limited to empirical distributions. Additionally, we demonstrate a strict complexity gap between SHAP variants, showing that interventional and baseline SHAP can be strictly easier to compute than conditional SHAP. Despite these positive results, we uncovered intractability for various SHAP variants in neural networks and tree ensembles. Finally, we provided generalized complexity relations across SHAP variants. We believe that our framework offers a deeper understanding of the complexity involved in computing SHAP across various variants, models, distributions, as well as in both local and global computations, laying the groundwork for future research.

%\section{Algorithm for verifying single attention head}

\begin{algorithm}[H]
	\caption{$\softmaxCheck:$ Check the softmax}
	\KwData{$k, \nfree, \aExt, \aDic^{min}, \aDic^{max}, \vExt, \vDic^{min}, \vquery$}
	\KwResult{Verification of $\bot$ or $\top$}
	\lev{We need a \emph{upper bound} on $\aDic$ as well}
	Set
	\[
		w = \frac{\sum_{i\in [k]} \exp(\aDic^{min})}{\sum_{i \in [k]} \exp(\aDic^{max}) +
		\sum_{i \in [\nfree]} \exp(\aExt)}.\;
	\]
	%\lev{Hrmm this kind of implies $w > \half$ so if we get $w$ so large it does not really matter if we have the right associated $v$. Maybe instead of thinking about $\tThresh$ we can phrase it as $\calD_{next}$}\;
 \lev{We need to bound \emph{the subtraction} potential from $\vDic$}
	If $w \cdot \vDic^{min} \geq (1 - w) \cdot \vExt$ return $\top$ otherwise return $\bot$ \;
\end{algorithm}

\begin{algorithm}[H]
	\caption{$\findKRep$: find number of times to repeat token $\tau$ at the end of a free ranging sequence of $\nfree$ tokens to ensure a dictator restriction \lev{We can probably have better bounds vis a vis position by writing an expectation} \lev{re-write in terms of infinity norms}}
	\KwData{$\xQuery, \xDic, \nfree, \klen$}
	\KwResult{Result in $\{\top, \bot\}$}
	Fix $\tQuery = \OneHot(\xQuery), \tDic = \OneHot(\xDic)$	\;
	Let $\pQuery = \OneHot(\nfree + k)$\;
	Let $\aExt = \frac{1}{\sqrt{d}} \max_{i \in [\dVocab]} \left[\tQuery EQK^T E^T \OneHot(i)^T + \norm{\pQuery QK^TP^T}_\infty \right]$ \;
	Let $\aDic^{min} = \frac{1}{\sqrt{d}} \left(\tQuery E_q QK^T E^T \cdot \tDic^T - \norm{\pQuery QK^TP^T}_\infty \right)$ \;
	Let $\aDic^{max} = \frac{1}{\sqrt{d}} \left(\tQuery E_q QK^T E^T \cdot \tDic^T + \norm{\pQuery QK^TP^T}_\infty \right)$ \;
	$\tThresh = \OneHot\left(\arg \max_{i \in [\dVocab]} \tDic \cdot EVOU \cdot \OneHot(i)^T - \norm{PVOU \cdot \OneHot(i)^T}_\infty + \tQuery E_q U \cdot \OneHot(i)^T \right)$ \;
	\lev{TODO: monotonicity theorem necessary}\;
	$\vDic^{min} = \tDic \cdot EVOU \cdot \tThresh^T - \norm{PVOU \cdot \tThresh^T}_\infty + \tQuery E_qU \cdot \tThresh^T$\;
	$\vExt = \max_i \max_j \left(\OneHot(i) \cdot EVOU \cdot \OneHot(j)^T + \norm{PVOU \cdot \OneHot(j)^T}_\infty\right)$\;
	\If{$\vDic^{min} \leq 0$}{
		\Return $\bot$\;
	}
	\Return $\softmaxCheck(k, \nfree, \aExt, \aDic^{min}, \aDic^{max}, \vExt, \vDic^{min})$\;
\end{algorithm}

\subsection{Soundness}

Sketch for proof is using the lemmas below.

\begin{lemma}[Infinity Norm for Positional Encoding Bound]
    Fill in lemma
\end{lemma}
\begin{proof}
    Should be easy \lev{TODO}
\end{proof}

\begin{lemma}[Passing $\softmaxCheck$ line $2$ implies fixing (assuming extremal next-token)]
	Assume that the dictator tokens, $\xDic$, are in location $L \subset [\nfree + k]$ where $|L| = k$ and that for some one hot token, $\vec{t}$, $\tDic \cdot EVOU \cdot \vec{t} + \frac{1}{k} \sum_{\ell \in L} \OneHot(\ell) \cdot PVOU \cdot \vec{t} \geq \vDic^{min}$.
	Also, assume that the token $x$ corresponding to $\vec{t}$ is the maximum of such possible sequences.
	Then if $\softmaxCheck$ return $\top$, we have a dictator sequence where the next token is $x$
\end{lemma}
\begin{proof}
	We begin by noting that	$\vExt$ is the maximum possible value of $(\mathbf{x}E + P_{enc}) VOU \OneHot(j)$ for all possible tokens $j$.
	As such, if $(1 - w) \cdot \vExt < z$ for any $z$, then $(1 - w) \cdot \frac{1}{N} \sum_{a \in N} (T_a \cdot E + P_a \cdot P) VOU \cdot {T'_a}^T < z$ for all settings of tokens $T, T' \subseteq {\{\OneHot(1), \dots \OneHot(\dVocab)\}}^N$ and positions $P \subseteq \{\OneHot(0), \dots \OneHot(\nctx)\}^\nctx$.
	So, if for a token $\vec{t}$, we have that $w \cdot \left[\sum_{i \in L} (\tDic E + \OneHot(L_i)) VOU \cdot \vec{t}^T + \tQuery E_q U \cdot \vec{t} \right] > (1 - w) \cdot \vExt$, then a greedy sampling strategy for the next token will select $\vec{t}$ unless there is a token $\vec{t'}$ where 
	\[
		 \sum_{i \in L} (\tDic E + \OneHot(L_i)) VOU \cdot \vec{t}^T + \tQuery E_q U \cdot \vec{t} > \sum_i (\tDic E + \OneHot(L_i)) VOU \cdot \vec{t}^T + \tQuery E_q U \cdot \vec{t}^T.
	\]
	But then, $\vec{t'}$ is selected irrespective of the non-dictator tokens as well.
	We can also see that 
	\[
		\frac{1}{k} \sum_{i \in L} (\tDic E + \OneHot(L_i)) VOU \cdot \vec{t}^T + \tQuery E_q U \cdot \vec{t}^T \geq \vDic.
	\]
	And so, we have a dictator token.

\end{proof}

\begin{lemma}[Monotonicity of extremal token and $\softmaxCheck$]
	If there is another token which is more likely to be outputted, then the $k$ value should pass even if its smaller
\end{lemma}
\begin{proof}
	Application of the above using $L$...
\end{proof}

\begin{lemma}[Non-extremal next token okay]
	Show how if there is a token with more weight (due to say position reasons) then that's okay)
	Use above lemma
\end{lemma}


\begin{lemma}[Passing $\softmaxCheck$ implies fixing]
	Part of this is assuming all the inputs represent proper bounds \emph{for the predicted token} are correct (does not need to be fixed to the predicted token as monotonicity there)
\end{lemma}

\begin{theorem}[Soundness for $k$]
	Just need to prove that the inputs to $\softmaxCheck$ are indeed extremal
\end{theorem}

\section{Verifying Attention with Parallel Linear Layer}
We will use the following model

\begin{align}
	\mathcal{M}(\textbf{t}) =& \sigma^* 
	       \left( (\tQuery E + \pQuery  P) Q K^T  (P^T + E^T \textbf{x}^T) /\sqrt{d} \right) \cdot (\textbf{x}E+P)VOU + (\tQuery E + \pQuery P)U\\
        &+ \pQuery \cdot \relu((\textbf{x}E + P) A_{enc} + b) A_{dec}
\end{align}

Then,
\begin{align}\label{eq:linlayer}
    \pQuery \cdot \relu((\textbf{x}E + P) A_{enc} + b) A_{dec} = 
    \relu((\tQuery E + \pQuery P) A_{enc} + b) A_{dec}
\end{align}

\cref{eq:linlayer} is fixed by $\pQuery$ and $\tQuery$ being one hot vectors.
We can then use the same technique as before

% TODO: Remove P from inifinty norm (its just VOU)

\lev{TODO: Ask people in AI if sequential or parallel are more popular}

\section{Adding in the Layer Norm}
\lev{Write somewhere that $ \tQuery \cdot \relu(\mathcal{L} \cdot A_{enc} + b) \cdot A_{dec} =  \cdot \relu(\tQuery \cdot \mathcal{L} \cdot A_{enc} + b) \cdot A_{dec}$}
\begin{align*}
	\mathcal{M}(\textbf{t}) =& \sigma^* 
	       \left(\tQuery \mathcal{L} \cdot  Q K^T  \mathcal{L}^T /\sqrt{d} \right) \cdot \mathcal{L} VOU + \tQuery\mathcal{L} U\\
        &+ \tQuery \cdot \relu(\mathcal{L} \cdot A_{enc} + b) \cdot A_{dec}
\end{align*}
where
\[
    \mathcal{L} = LN(\mathbf{x} E + P)
\]
\newcommand{\tild}[1]{\widetilde{#1}}

\lev{Specify normalized normalized all one. Specifify, that everything is entrywise!!...
Maybe fix a column and then write out the operation!}
\newcommand{\LNInp}{\tild{\mathbf{x}}}

We define $LN(\tild{\mathbf{x}})$ as 
\[
	LN(\LNInp) = \frac{\LNInp - \OneMat \cdot \LNInp}{\sqrt{\OneMat \cdot (\LNInp\circ\LNInp) - (\OneMat \cdot \LNInp) \circ (\OneMat \cdot \LNInp) + \eps \cdot \OneMat'}} \cdot \gamma + \beta \cdot \OneMat'
\]

Where $\OneMat \in \mathbb{R}^{\nctx \times \nctx}$ and $\OneMat' \in \mathbb{R}^{\nctx \times \dEmb}$ are the matrices with all ones and $\eps$ is a small constant.

\lev{positional encoding in layer norm make things tricky!}
\lev{Think of layernorm as a left multiplication by a fixed vector/ matrix and fixed offset.}

\lev{I think that bounds here need to bounded in a smarter way. Maybe consider your sequence length of repeated tokens, and then this gives us a much tighter average bound and upper bound for the denominator (simply done by maximizing variance?).}

\subsection{Bounding the Layer Norm Scalars}
Given that layernorm acts \emph{coordinate}-wise (column-wise) in the columns of the input, we can provide bounds per coordinate $c \in [\dEmb]$.

\begin{lemma}[Upper bound on expectation]
	\label{lemma:upperExpBound}
	For a token repeated $\klen$ times with $\nfree$ tokens, the expectation in the layer norm is upper bounded by
	\[
		\frac{\klen \left(\tDic \cdot E \cdot \vec{t}_c^T\right)}{\klen + \nfree} +
			\frac{\nfree \left(\norm{E \cdot \vec{t}_c^T}_\infty\right)}{\klen + \nfree} + \norm{P \cdot \vec{t}_c^T}_\infty.
	\]	
\end{lemma}
\begin{proof}
	The proof follows from a simple expansion expectation and noting that the dictator token is repeated $\klen$ times.
\end{proof}

\begin{lemma}[Lower bound on expectation]
	\label{lemma:lowerExpBound}
	\[
		\frac{\klen \left(\tDic \cdot E \cdot \vec{t}_c^T\right)}{\klen + \nfree} -
			\frac{\nfree \left(\norm{E \cdot \vec{t}_c^T}_\infty\right)}{\klen + \nfree} - \norm{P \cdot \vec{t}_c^T}_\infty.
	\]
\end{lemma}

\begin{lemma}[Upper bound on variance]
	\label{lemma:upperVarBound}
	(Pretty easy by combining bounds on expectation)

	For a token repeated $\klen$ times with $\nfree$ tokens, the variance in the layer norm is upper bounded by
	\[
		AHAHA
	\]
\end{lemma}

\lev{Easier to state lemma below when Layer norm is properly defined column wise}
\lev{add in the epsilon part, epsilon. }
\begin{lemma}[Upper bound on denominator]
    For a token repeated $\klen$ times with $\nfree$ tokens, the denominator in the layer norm is upper bounded by, ignoring the $\eps$,
    % https://stats.stackexchange.com/questions/142651/does-the-uniform-distribution-have-the-greatest-variance-among-all-concave-distr#:~:text=The%20uniform%20distribution%20on%20a,two%20points%20of%20the%20graph).
    \[
	    \frac{\left(\norm{E \cdot \vec{t_c}^T}_\infty + \norm{P \cdot \vec{t_c}^T}_\infty \right)}{\sqrt{3}}.
    \]
\end{lemma}
\begin{proof}
	We first (\lev{TODO: cite}) note that for any random variable bounded in $[-a + b, a + b]$, the standard deviation is at most $a/\sqrt{3}$.
	Then, we can note that the difference between the maximum and minimum of the embedding summed with positional encoding is at most $2(\norm{E \cdot \vec{t_c}^T}_\infty + \norm{P \cdot \vec{t_c}^T}_\infty)$.
 \lev{elaborate a bit maybe?}
\end{proof}

\begin{lemma}[Lower bound on denominator]
	The denominator in the layer norm is lower bounded by, ignoring the $\eps$,
	\[
		AHAHAH %\frac{1}{\sqrt{\nctx}} \sqrt{\sum_{p \in [\nctx]} \left(\min(\vec{t_p} \cdot P \cdot \vec{t_c}^T)\right)^2}
	\]
\end{lemma}
\begin{proof}
	For $E_c = E \cdot \vec{t}_c, P_c = P \cdot \vec{t}_c$, we can rewrite the denominator as the squareroot of
	\begin{align*}
		\OneMat \cdot (xE_c + P_c) \circ (xE_c + P_c) - (\OneMat \cdot (xE_c + P_c)) \circ (\OneMat \cdot (xE_c + P_c)) =& \OneMat \cdot (xE_c \circ xE_c) + \OneMat \cdot (P_c \circ P_c) + 2 \cdot \OneMat \cdot (xE_c \circ P_c)	\\
														 &- (\OneMat \cdot (xE_c))^2 - (\OneMat \cdot P_c)^2 - 2 \cdot (\OneMat xE_c \circ \OneMat P_c).
	\end{align*}
	Then, $\OneMat \cdot(xE_c \circ xE_c) - (\OneMat \cdot xE_c)^2 \geq 0$ and $\OneMat \cdot (P_c \circ P_c) - (\OneMat \cdot P_c)^2$ is a fixed value.

	So, we now must find
	\[
		\min_x (\OneMat \cdot (xE \circ P) - \OneMat xE \circ \OneMat P).
	\]
    \lev{Given series of fixed shifts, can we move the fixed shift back to 0? Look at pairwise differences in $E$ and in $P$ ($O(\dVocab^2$). Write out pairwise matrix, build a bound. Easy to show something about average sequence.
    Worth looking at, is $\beta$ positive? Because otherwise repeated entries get subtracted off}
    \lev{Maybe there is a way to remove the bounds}
\end{proof}

\pagebreak


%\section{List of Ideas}

\begin{itemize}
    \item No go result on how memory is stored to show $.... 2 + 5 =$ and $.... 3 + 4 =$ will fail because the previous will always overwhelm. Generalizes to attn heads in parrallel (pay attention to only a few tokens). Maybe in asymptotic for many heads in serial. Something you can prove w/o making the model very contrived.
    \item Formalize model as dynamical system up to some error
    \item Same model, but if you have long context, then will it still select max (or smthng) at the end? Ask to prove model will produce max on any (or up to a bound). Show something with a state machine type model (like RNNs/ Mamba, etc) or Neural Turing Machines or Reinforcement Learning. For example for max, you can always have the model store some information and that gets passed back into the input.
\end{itemize}
%% propose BF GDA/SGDA
% introduce adaptive regret notion and non-degenerative populations
%%% theoretical results
% convergence of GDA/SGDA in our setting for exponential policies
% hypotethis that it should work still by clipping the policy side ( then experiments), then argue that it can help for convergence of NN policies (talk about L0-L1 smoothness and past results on that matter, theoretical results left for future work on GDA/SGDA)
% 
%%% experiments
% repeated prisonner's dilemma as a first example to show results on GDA, (with specific or full set of deterministic policies??)
% Show clipping effect (works on higher lr)
% boxplots, learned distributions, learning curve on worst-case regret ?

% more advanced experiments on random POMDPS (provide figure for the env ?)
% show empirically that it still works. clipping as well ?

% final set of experiments on complex environments with NN policies.
% > leduc poker, melting pot
% > cooperation on mujoco tasks (say the "human" controls a part of the robot, and the agent assists)

\bibliography{cubebib}

\appendix
\onecolumn
\section{Experimental Outcomes}
\label{sec:appOut}
In this appendix we provide the plots for \cref{sec:model_eval}.
Firstly, \cref{fig:worst-case} contains the plots of worst-case deviation with and without the use of the linear program, and  \cref{fig:ptp} plots the peak-to-peak difference. In \cref{fig:ptp} whenever our algorithm proves ``overwhelming'', the point is colored red and marked with a `x.'
\vspace{0pt}\nopagebreak
\begin{figure}[H]
    \centering
	\includegraphics[width=\textwidth, keepaspectratio, trim={0cm 0cm 0 1.5cm},clip]{figs/plots/w_analysis_lp_log_independent.pdf}
	\caption{The worst-case deviation for the model for the three different input restrictions and two different permutation classes.
    The $x$-axis is the number of tokens and the $y$-axis is the base-10 logarithm of the worst-case deviation.
    }
	\label{fig:worst-case}
\end{figure}
\pagebreak
\begin{figure}[H]
    \centering
	\includegraphics[width=\textwidth, keepaspectratio, trim={0cm 0cm 0 1.5cm},clip]{figs/plots/w_analysis_ptp_linear_independent.pdf}
	\caption{The peak-to-peak difference for the model for the three different input restrictions and two different permutation classes.
	Red ``x''s indicate that the worst-case deviation is less than half of the peak-to-peak difference and, thus, the model output is provably invariant over the permutation class.
        The $x$-axis is the number of tokens and the $y$-axis is the peak-to-peak deviation.
    }
	\label{fig:ptp}
\end{figure}

\subsection*{Overwhelming Through Generation}
In \cref{fig:contOverwhelm} we plot the bounds on worst-case deviation and the peak-to-peak difference as the model is used to continually generate text. The newly generated tokens are continuously added to the fixed string and fed again to the model to generate the next token. The model is ``overwhelmed'' whenever the worst-case deviation is less than the peak-to-peak difference in the plot.

\vspace{0pt}\nopagebreak
\begin{figure}[H]
    \centering
	\includegraphics[ width=\textwidth, keepaspectratio, trim={0cm 0cm 0 1.5cm},clip]{figs/plots/continued_gen.pdf}
	\caption{
            The peak-to-peak difference and the worst-case deviation when we continue generation through multiple tokens.
            The $x$-axis connotes the total number of tokens ($n_{ctx}$) throughout a generation.
            When the line for $PTP / 2$ is above the worst-case deviation line, then our algorithm provably guarantees that the output is fixed.}
	\label{fig:contOverwhelm}
\end{figure}

\section{Appendix for Model Details}
\label{sec:appendix_model}
We provide a formal definition of each of the components in the model used in this paper.
\begin{itemize}
	\item $\softmax$ is the softmax function
		\[
			\softmax(\vec{\alpha})[i] = \frac{e^{\vec{\alpha[i]}}}{\sum_{i=1}^{\dVocab} e^{\vec{\alpha}[i]}}
		\]
	\item $\relu$ is the rectified linear unit function
		\[
			\relu(x) = \max(0, x)
		\]
	\item $\LayerNorm$ is the layer normalization function which for matrix $X$ performs the following column wise function for columns of $X$ (i.e.\ $X[:, j]$):
		\[
			\LayerNorm(X[i, j]) = \frac{X[i, j] - \Expec[X[:, j]]}{\sqrt{\Var[X[:, j]] + \eps} } \cdot \gamma + \beta
		\]
		where $\gamma$ and $\beta$ are learned parameters and addition and division are element-wise.
	\item $\MLP$ is a one-layer feed-forward network with ReLU activation. The $\MLP$ is row wise of $X$:
		\[
			\MLP(X[i]) = \relu(X[i, :] \cdot A_{enc} + b_{enc}) \cdot A_{dec}
		\]
		Depending on the model there may be additional bias terms added after the ReLU activation.
		For simplicitily, we will only consider one bias term prior to the ReLU activation though the results of this paper can be easily be extended to two bias terms.
	\item $\RoPE$ \cite{su2024roformer} is the rotary position encoding which applies a rotation to key and query vectors. For input $X$ at row $i$, the rotation is applied as follows: 
		\[
			\RoPE(X)[i, 2j] = \cos(i\theta_j)X[i, {2j}] - \sin(i\theta_j)X[i, 2j + 1]
		\]
		and
		\[
			\RoPE(X)[i, 2j + 1] = \sin(i\theta_j)X[i, {2j}] + \cos(i\theta_j)X[i, 2j + 1]
		\]
		where $\theta_j = 10000^{-2j/\dEmb}$ is the frequency for dimension $j$.
		%\item $\attn$ is the attention mechanism: for matrix $X \in \R^{\nctx \times \dEmb}$, the attention mechanism with RoPE is
		%	\[
		%		\attn(X) = \softmax\left( \frac{(\RoPE(XQ, i))(\RoPE(XK, i))^T}{\sqrt{d}} \right) \cdot XV
		%	\]
		%	where $i$ is the position index for each row of $X$, and the softmax is applied column-wise.
	\item $\attnH$ is an attention head for matrix $X \in \R^{\nctx \times \dEmb}$, an attention head does the following:
		\[
			\attnH(X) = \softmax\left( \frac{\RoPE(X Q) \cdot \RoPE(K^T X^T)}{\sqrt{\dEmb / H}} \right) \cdot X V
		\]
		where the softmax is applied column-wise.
        As in Ref.~\cite{black2022gptneox20bopensourceautoregressivelanguage}, we can re-write the effect of RoPE via a rotation matrix $\Theta_{i, j}$ such that
        \[
            \left(\RoPE(X Q) \cdot \RoPE(K^T X^T)\right)[i, j] = \vece_i \cdot X Q \cdot \Theta_{i, j} \cdot  K^T  X^T \cdot \vece_j^T
        \]
	\item Often, we have a multi-head attention mechanism which is the concatenation of $H$ attention mechanisms along the last dimension:
		\[
			\attn(X) = [\attnH_1(X); \ldots; \attnH_H(X)]
		\]
		where $\attnH_h$ is the $h$-th attention mechanism as outlined
	\item $\Embed \in \R^{\dVocab \times \dEmb}$ is the embedding function which maps a one hot vector in $\R^\dVocab$ to a vector in $\R^{\dEmb}$.
	\item $\Unembed \in \R^{\dEmb \times \dVocab}$ is the unembedding function which maps a vector in $\R^{\dEmb}$ to a one hot vector in $\R^\dVocab$.
\end{itemize}



\section{Proofs for Worst-Case Deviation}
\label{sec:proofs_worst_case_deviation}

\begin{lemma}[Properties of the Worst-Case Deviation]
	\label{lemma:worst_case_deviation_properties}
	For any functions $f, g: \calX \to \R^n$, norm $p$ and lift to $\calY \times \calZ$, we have the following properties:
	\begin{itemize}[nosep]
		\item Triangle inequality for (lifted) $\WD$:	
			\[
				\WD(f + g; \calX)_p \leq \WD(f : \calX)_p + \WD(g : \calX)_p.
			\]
		\item Lifting monotonicity:
			\[
				\WD(f; \calX)_p \leq \WD(f; \calY \times \calZ)_p.
			\]
		\item Lipschitz composition:
			For function $g$
               \[
				\WD(g \circ f; \calX)_p \leq \Lip(g)_p \cdot \WD(f; \calX)_p.
			\]
		As a corollary, we have that for linear operators $A$,
		\[
			\WD(A f; \calX)_p \leq \norm{A}_p \cdot \WD(f; \calX)_p.
		\]
		as $\Lip(A)_p = \norm{A}_p$.

		\item $p$-norm bounds for $p, q \geq 1$ and $q > p$:
			\[
				\WD(f; \calX)_q \leq \WD(f; \calX)_p
			\]
	\end{itemize}
\end{lemma}


\begin{proof}[Proof of worst-case deviation properties, \cref{lemma:worst_case_deviation_properties}]
	\label{proof:worst_case_deviation_properties}
	We will prove each of the properties in turn.
	\begin{itemize}
		\item Triangle inequality:
			We can view the maximization over $\calX$ as occuring disjointly for $f$ and $g$:
			I.e. \begin{align*}
				\WD(f + g; \calX)_p 
			&\leq
            \sup_{X_1, X_2 \in \mathcal{X}} \norm{
			f(X_1) + g(X_1) - (f(X_2) + g(X_2))} \\
			&\leq \sup_{X_1, X_2, X_1', X_2' \in \mathcal{X}}  \big[\norm{f(X_1) - f(X_2)} 
            + \norm{g(X_1') - g(X_2')}\big] 
			\tag{by triangle inequality of norms}\\
			&\leq \WD(f ; \calX)_p + \WD(g ; \calX)_p
			\end{align*}
			as desired.
		\item Lifting monotonicity:
			The proof follows from the definition of the lifted worst-case deviation:
			\begin{align*}
				\WD(f ; \calX)_p &= \sup_{(Y_1, Z_1), (Y_2, Z_2) \in \calX \subseteq \calY \times \calZ} \norm{f(Y_1, Z_1) - f(Y_2, Z_2)}_p \\
						 &\leq \sup_{Y_1, Y_2 \in \calY, \; Z_1, Z_2 \in \calZ} \norm{f(Y_1, Z_1) - f(Y_2, Z_2)}_p \\
						 &= \WD(f ; \calY \times \calZ)_p.
			\end{align*}
		\item Lipschitz composition:
			We simply have that
			\begin{align*}
				&\WD(A f; \calX)_p = \sup_{X_1, X_2 \in \calX} \norm{A f(X_1) - A f(X_2)}_p \\
						  &= \sup_{X_1, X_2 \in \calX} \norm{A (f(X_1) - f(X_2))}_p \\
						  &\leq \sup_{X_1', X_2'} \frac{\norm{A(X_1') - A(X_2')}_p}{\norm{X_1' - X_2'}_p} \cdot \sup_{X_1, X_2 \in \calX} \norm{f(X_1) - f(X_2)}_p \\
						  &=\Lip(A)_p \cdot \WD(f; \calX)_p.
			\end{align*}
		\item $p$-norm bounds for $p, q \geq 1$ and $q > p$:
			Because we restrict $f$ to be a function which outputs vectors and $\norm{\vec{x}}_q \leq \norm{\vec{x}}_p$ for $q > p$, we have that for all $X_1, X_2 \in \calX$, $\norm{f(X_1) - f(X_2)}_q \leq \norm{f(X_1) - f(X_2)}_p$.
			So, if there exists $X_1, X_2 \in \calX$ such that $\norm{f(X_1) - f(X_2)}_q = \alpha$, then there must exist $X_1, X_2 \in \calX$ such that $\norm{f(X_1) - f(X_2)}_p \geq \alpha$.
			Thus, the supremum over $\calX$ for $q$ is less than or equal to the supremum over $\calX$ for $p$.
		\end{itemize}
	\end{proof}


\section{Proofs for Generic Framework}
\label{sec:proofs_framework}

\subsection{Proofs for Blowup and Shift Bounds}
We first need to bound the variance and expectation of the columns of $\Embed \cdot X$:
\begin{lemma}[Bounds on Expectation and Variance]
	\label{lem:boundsVar}
	Let $X \in \InpSpace$ be an input sequence.
	% Write $X = (s | X_{\text{free}} | q)$$
 %    \left(\bigtimes_{i = 1}^{\ndes} \{ \desSet_i\} \right) \times X_{\text{free}} \times \{\vecQuery\}$, where
	% $\desSet_i$ are the desired token sets,
	% $X_{\text{free}}$ consists of the possible free tokens, and
	% $\vecQuery$ is the query token.
	Then for any column $j$, the expectation $\mu_j$ of the $j$-th column of $\Embed \cdot X$ satisfies:
	\begin{align*}
		\frac{1}{\nctx} &\left(
			\sum_{i=1}^{\ndes} \Embed[\desSet_i] \vec{e}_j^T + 
			\nfree \min_{x \in [\dVocab]} (\Embed[x] \vec{e}_j^T) + 
			\Embed[q]\vec{e}_j^T
		\right)
		\leq \mu_j  \\
				&\leq
				\frac{1}{\nctx}\left(
					\sum_{i=1}^{\ndes} \Embed[\desSet_i]\vec{e}_j^T + 
					\nfree \max_{x \in [\dVocab]} (\Embed[x]\vec{e}_j^T) + 
					\Embed[q]\vec{e}_j^T
				\right)
		\end{align*}
		Let $\mu_j^{\min}$ and $\mu_j^{\max}$ denote the lower and upper bounds for the $j$-th column respectively.
		Then, we can bound the variance of the $j$-th column of $\vec{x} \Embed$ by
		\begin{align*}
			\frac{1}{\nctx} &\bigg[
				\min_{\mu' \in [\mu_j^{\min}, \mu_j^{\max}]}  \sum_{i = 1}^s \left(\Embed[\desSet_i] \vec{e}_j^T -  \mu' \right)^2 + \nfree \cdot \min_{x \in [\dVocab]} \left( \Embed[x] \vec{e}_j^T -  \mu' \right)^2 + \left( \Embed[q] \vec{e}_j^T -  \mu' \right)^2
			\bigg] \\
					&\leq \Var_j[(\Embed \cdot X)[:, j]] \leq \\
				\frac{1}{\nctx} &\bigg[
					\sum_{i = 1}^s \max_{\mu' \in [\mu_j^{\min}, \mu_j^{\max}]} \left(\Embed[\desSet_i] \vec{e}_j^T -  \mu' \right)^2 + \nfree \cdot \max_{x \in [\dVocab]} \left( \Embed[x] \vec{e}_j^T -  \mu' \right)^2 + \left( \Embed[q] \vec{e}_j^T -  \mu' \right)^2
				\bigg].
				\end{align*}
				We define $\Var_j^{\min}, \Var_j^{\max}$ to be the lower and upper bounds respectively.
			\end{lemma}
			\begin{proof}
				The bounds on $\mu_j$ follow as we minimize and maximize the contribution of each free row.
				The proof of for variance bounds follows similarly.
			\end{proof}

We now have a simple proof for \cref{lem:boundsBS} restated below:
\begin{lemma}[Blowup and Shift Bounds]
	We bound blowup and shift: for every $B_j \in \blowupSet_j$ and $S_j \in \shiftSet_j$
	\[
		\frac{\gamma}{\sqrt{\Var_j^{\max} + \eps}} \leq B_j \leq \frac{\gamma}{\sqrt{\Var_j^{\min} + \eps}}
	\]
	and
	\begin{align*}
	&\min\left(B_j^{\min} \mu_j^{\min}, B_j^{\min} \mu_j^{\max}, B_j^{\max}, \mu_j^{\min}, B_j^{\max} \mu_j^{\max}\right)
	\leq
	S_j\\
	&\leq
	\max\left(B_j^{\min} \mu_j^{\min}, B_j^{\min} \mu_j^{\max}, B_j^{\max}, \mu_j^{\min}, B_j^{\max} \mu_j^{\max}\right)
	\end{align*}
\end{lemma}
\begin{proof}
	Given the bounds on expectation and variance, $B^{\min}, B^{\max}$ follow trivially from the definition of blowup and shift sets.
	$S^{\min}$ (resp. $S^{\max}$) follow from a minimization (maximization) over the possible shifts given the blowup bounds.
\end{proof}

\subsection{Appendix for Bounding Worst-case Deviation of $\fenc$}
\label{sec:appConcrcase}

Recall the definition of blowup and shift sets from \cref{def:blowup_shift}.
			Let $B^{\min}, B^{\max}$ be the vectors $(B_1^{\min}, \dots, B_\dEmb^{\min})$ and $(B_1^{\max}, \dots, B_\dEmb^{\max})$ respectively.
			Define $S^{\min}, S^{\max}$ analogously.
			We restate \cref{lem:WD_fenc} for convenience:
			\begin{lemma}[Worst-case deviation of $\fenc$, \cref{lem:WD_fenc}]
				\label{lem:worst_case_deviation}
				Let $\vec{x} \in \InpSpace$.
				Then,
				\[
					\WD(\vece_{\nctx} \cdot \fenc; \InpSpace \times \blowupSet \shiftSet)_\infty \leq 
					\max_{t \in [\dVocab]} 
					\max_{j \in [\dEmb]}
					\max_{B \in [\vec{0}, B^{\max} - B^{\min}]}
					\left|X[t] \cdot \Embed \cdot \diag(B) \cdot \vec{e}_j\right| + \left|S_j^{\max} - S_j^{\min}  \right|
                    \]	
				where the inner maximum term can be computed via a simple linear program.
			\end{lemma}
			\begin{proof}
                    First, note that for $j \in [\dEmb]$,
				\begin{align*}
					\WD(\vece_{j}^T \cdot (\fenc)^T; \InpSpace \times \blowupSet \shiftSet)_\infty 
						&= \max_{t, B, S, B', S'} \left\|
						X[t] \cdot E \cdot \diag(B) + S - X[t] \cdot E \cdot \diag(B') - S'
						\right\|_\infty \\
						&\leq max_{t, B, S, B', S'}
						\left|\left(X[t] \cdot E \cdot (\diag(B) - \diag(B')) + S - S'\right) \cdot \vec{e}_j\right| \\
						&\leq max_{t, B, B'}  \left|\left(X[t] \cdot E \cdot (\diag(B) - \diag(B')) \right) \cdot \vec{e}_j\right| + |S_j^{\max} - S_j^{\min}|.
				\end{align*}		
				Note then that $\diag(B)_j - \diag(B')_j$ is constrained by $B^{\max}_j - B^{\min}_j$.
                We can then maximize over $j$ to get
                \begin{align*}
                    \WD(\vece_\nctx \cdot \fenc; \InpSpace \times \blowupSet \shiftSet)_\infty 
                    &\leq 
                    \WD(\fenc; \InpSpace \times \blowupSet \shiftSet)_\frobInf  \\
                    &\leq \max_{t, B, B'} \max_j  \left|\left(X[t] \cdot E \cdot (\diag(B) - \diag(B')) \right) \cdot \vec{e}_j\right| + |S_j^{\max} - S_j^{\min}|
				\end{align*}
                as desired.
			\end{proof}



\subsection{Proofs and Subalgorithms for Attention Bounds}
\begin{definition}[$\ell^{min}$, $\ell^{max}$] \label{def:lminlmax}
	We use $\ell^{\min}_i$ to denote 
	a worst-case lower bound on the smallest possible logit at the $i$-th position of the input to the softmax in model $\Model$. 
	Similarly, we use $\ell^{\max}_i$ to denote a worst-case upper bound on the largest possible logit at the $i$-th position of the input to the softmax in model $\Model$.
\end{definition}

We provide the algorithm to find the extremal values $\vec{\ell}^{\min}$ and $\vec{\ell}^{\max}$ in \cref{alg:lminlmax}.
At a high level, the algorithm computes upper and lower bounds for each position in the input sequence prior to the softmax.
The bounds make use of the upper and lower bounds on the blowup and shift sets, $\blowupSet \shiftSet$.


\begin{algorithm}[h] 
	\caption{
        Algorithm to find $\vec{\ell}^{\min}$ and $ \vec{\ell}^{\max}$.
        To find the minimums and maximums over $\blowupSet \shiftSet$, we use the point-wise upper and lower bounds on the blowup and shift sets alongside the bounds obtained from \cref{alg:bilin}.}
	\label{alg:lminlmax}
	\begin{algorithmic}
		\STATE {\bfseries Input:}  $\blowupSet\shiftSet, \Model$
		\STATE
		\FOR{$k \in [\nfix]$}
		\STATE Let \begin{align*}
		\vec{\ell}^{\min}_k = \frac{1}{\sqrt{\dEmb}} \min_{B, S  \in \blowupSet\shiftSet} (\vece_{q} E \cdot \diag(B) + S) \cdot Q \PosRot_{k, \nctx} 
		K^T \cdot  (\vece_{\desSet_k} E \cdot  \diag(B) + S)^T
		\end{align*}
		and 
		\[
		\vec{\ell}^{\max}_k = \frac{1}{\sqrt{\dEmb}}  \max_{B, S \in \blowupSet\shiftSet} (\vece_{q} E \cdot \diag(B) + S) \cdot Q \PosRot_{k, \nctx} K^T \cdot  (\vece_{\desSet_k} E \cdot  \diag(B) + S)^T
		\]
		\ENDFOR
		\STATE
		\FOR{$k \in (\nfix, \nctx)$}
		\STATE Let \[
		\vec{\ell}^{\min}_k = \frac{1}{\sqrt{\dEmb}} \min_{t \in [\dVocab]} \min_{B, S \in \blowupSet\shiftSet} (\vece_q E \cdot \diag(B) + S) \cdot Q \PosRot_{k, \nctx} K^T \cdot  (\vece_t E \cdot  \diag(B) + S)^T
		\]
		and 
		\[
		\vec{\ell}^{\max}_k = \frac{1}{\sqrt{\dEmb}} \max_{t \in [\dVocab]} \max_{B, S \in \blowupSet\shiftSet} (\vece_q E \cdot \diag(B) + S) \cdot Q \PosRot_{k, \nctx} K^T \cdot  (\vece_t E \cdot  \diag(B) + S)^T
		\]
		\ENDFOR
		\STATE
		\STATE Set the extremal values of the query token:
		\[
		\vec{\ell}^{\min}_\nctx = \frac{1}{\sqrt{\dEmb}} \min_{B, S \in \blowupSet\shiftSet} (\vece_{q} E \cdot \diag(B) + S) \cdot Q \PosRot_{k, \nctx} K^T \cdot  (\vece_{q} E \cdot  \diag(B) + S)^T
		\]
		and
		\[
		\vec{\ell}^{\max}_\nctx = \frac{1}{\sqrt{\dEmb}}  \max_{B, S \in \blowupSet\shiftSet} (\vece_{q} E \cdot \diag(B) + S) \cdot Q \PosRot_{k, \nctx} K^T \cdot  (\vece_{q} E \cdot  \diag(B) + S)^T
		\]
		\STATE
		\STATE Return $\vec{\ell}^{\min}, \vec{\ell}^{\max}$
	\end{algorithmic}
\end{algorithm}


\begin{algorithm}[h] 
	\caption{Simple algorithm to bound the minimum of $\vec{x} A \vec{y}^T$ for $X^{\min}_i \leq \vec{x}_i \leq X^{\max}_i$ and $Y^{\min}_i \leq \vec{y}_i \leq Y^{\max}_i$.
        To find the maximum, switch the minimum to maximum and vice versa.
    }
	\label{alg:bilin}
	\begin{algorithmic}
            \STATE \textbf{Input:} $A, X^{\min}, X^{\max}, Y^{\min}, Y^{\max}$
            \STATE
            \STATE $m \leftarrow 0$
            \FOR{$i = 1$ to $n$}
            \FOR{$j = 1$ to $m$}
            \IF{$A_{ij} > 0$}
            \STATE $m \leftarrow m + A_{ij} \cdot \max\left(X^{\min}_i Y^{\min}_j, X^{\max}_i Y^{\min}_j, X^{\min}_i Y^{\max}_j, X^{\max}_i Y^{\max}_j \right)$
            \ELSE
            \STATE $m \leftarrow m + A_{ij} \cdot  \min\left(X^{\min}_i Y^{\min}_j, X^{\max}_i Y^{\min}_j, X^{\min}_i Y^{\max}_j, X^{\max}_i Y^{\max}_j \right)$
            \ENDIF
            \ENDFOR
            \ENDFOR
            \STATE Return $m$
        \end{algorithmic}
\end{algorithm}

\begin{proof}[Proof sketch for \cref{lem:lminlmax} (correctness of \cref{alg:lminlmax})]
	In \cref{alg:lminlmax}, for each position $k$, we compute the minimum and maximum logit for the $k$-th position by maximizing over possible blowup and shift sets and input tokens for the free tokens.
	Moreover, \cref{alg:bilin} computes an upper and lower bound for the restricted bilinear form which \cref{alg:lminlmax} uses to compute the extremal values.

    The correctness of the bilinear bound in \cref{alg:bilin} follows from a series of relaxations when writing out the explicit formula for the bilinear multiplication.
    To prove an upper-bound, we have that 
    \begin{align*}
        m &= \sum_{i} x_i A_{i, j} y_j \\
        &\leq \sum_i \max (A_{i, j} X_i^{\min} Y_i^{\min},  A_{i, j} X_i^{\min} Y_i^{\max}, A_{i, j} X_i^{\max} Y_i^{\min}, A_{i, j} X_i^{\max} Y_i^{\max}).
    \end{align*}
    Then, if $A_{i, j}$ is negative, we can factor the coefficient out of the maximization and flip to a minimization. Otherwise, we can simply factor $A_{i, j}$ out of the maximization.
    Thus, \cref{alg:bilin} gives a valid upper-bound. The lower-bound follows analogously.
\end{proof}

We now prove \cref{lem:min_max_softmax}, restated here:

\begin{lemma}[Minimum and Maximum after Softmax]
	\begin{equation}
		\label{eq:softmax_upper}
		\ssMinFB \leq
		\sum_{j \in (\nfix, \nctx)} \softmax(\ell_j) \leq
		\ssMaxFB
	\end{equation}
	aswell as,
	\begin{align}
		\label{eq:softmax_lower}
		\ssMinRB \leq
		\sum_{j \in [\nfix] \cup \{\nctx\}} \softmax(\ell_j) \leq
		\ssMaxRB.
	\end{align}
\end{lemma}
\begin{proof}[Proof of \cref{lem:min_max_softmax}]
	We will prove the right hand side of \cref{eq:softmax_upper} and the left hand side of \cref{eq:softmax_lower} as the other follow by the same proof.
	Note the second inequality follows from the first inequality and the fact that the sum of the attention weights is $1$.

	Now, we will show the first inequality.
	Clearly, smaller values of $\vec{\ell}_i$ results in larger value of $\softmax(\vec{\ell}_j)$.
	Then, let $\vec{\eta}$ be a vector where $\sum_{j \in (\ndes, \nctx]} \softmax(\vec{\eta}_j) > \sum_{j \in (\ndes, \nctx]} \softmax(\vec{\ell}_j)$.
	Then, we have that
	\begin{align*}
		0 &\leq \left[\left(\sum_j e^{\vec{\eta}_j}\right) - \left(\sum_j e^{\vec{\ell}_j}\right)\right] \left(\sum_i e^{\vec{\ell}_i}\right)  \\
		\Rightarrow &\left(\sum_j e^{\vec{\eta}_j} \right)\left(\sum_i e^{\vec{\ell}_i} + \sum_j e^{\vec{\ell}_j} \right) \leq
        \left(\sum_j e^{\vec{\ell}_j}\right) \left(\sum_i e^{\vec{\eta}_i} + \sum_j e^{\vec{\eta}_j}\right) \\
		\Rightarrow & \frac{\sum_j e^{\vec{\eta}_j}}{\sum_i e^{\vec{\ell}_i} + \sum_j e^{\vec{\eta}_j}} \leq 
        \frac{\sum_j e^{\vec{\ell}_j}}{\sum_i e^{\vec{\ell}_i} + \sum_j e^{\vec{\ell}_j}}.
	\end{align*}
\end{proof}

\newcommand{\free}{\text{free}}
\newcommand{\fix}{\text{fix}}
Finally, we prove the bound on worst-case deviation of attention (\cref{lem:att_bound}):
\begin{lemma}
	Worst-case deviation of attention is bounded as follows,
	\begin{align*}
		\WD(\vecNctx \cdot f^{\attnH}_{\ndes \mid r, q} ; \InpSpace \times \blowupSet \shiftSet)_\infty
	     &\leq
	     2 \cdot (\ssMaxRB - \ssMinRB) \cdot \max_{B, S}\norm{\Embed[[\nfix] \cup \{\nctx\}, :] \cdot \diag(B) + S}_{\frobInf} \\
		&+ 2 \cdot \ssMaxFB \cdot \max_{B, S} \norm{(E \cdot \diag(B) + S) \cdot V}_{\frobInf}
        + \ssMaxFB  \norm{V}_\infty \cdot \max_{t \in s \cup \{q\}} \WD(\fenc; \{\vece_t\} \times \blowupSet \shiftSet\}))_\infty
	\end{align*}
        where $V$ is the value matrix in the attention head (see \cref{sec:appendix_model}).
\end{lemma}
\begin{proof}[Proof of \cref{lem:att_bound}]
	\label{proof:att_bound}
	Define $\vec{p}(X) \in \R^{\nctx}$ as the probability vector for the query token post-softmax.
	I.e.
	\[
		\vec{p}_j = \softmax(\vec{\ell}_j) \text{ for } j \in [\nctx]
	\]
	where \[
		\vec{\ell} = \frac{\vecNctx \cdot(Y \cdot \RoPE(Q) \RoPE(K^T) Y^T}{\sqrt{\dEmb}}
	\]
	for $Y = X \Embed \cdot \diag(B(X)) + S(X)$.
	Moreover, let $\vec{p}(X)_{\free} = \vec{p}(X)[(\nfix, \nctx)]$ (i.e.\ the values corresponding to the free tokens) and $\vec{p}(X)_{\fix} = \vec{p}(X)[[\nfix] \cup \{\nctx\}]$ (the values corresponding to the fixed and query token).
	Now, we re-write the worst-case deviation of the attention head: for $X, X' \in \InpSpace$, $(B, S), (B', S') \in \blowupSet \shiftSet$,
	\begin{align*}
		\WD&(\vecNctx \cdot f^{\attnH}_{\ndes \mid r, q} ; \InpSpace \times \blowupSet \shiftSet)_\infty
			\leq  \max_{X, X', (B, S), (B', S')} 
			\norm{\vec{p}(X) \cdot (X \Embed \cdot \diag(B) + S) V - \vec{p}(X') \cdot (X \Embed \cdot \diag(B') + S') V}_\infty \tag{by definition of an attention head} \\
		   &\leq \max_{X, X', (B, S), (B', S')} \bigg\|\vec{p}(X)_\fix \cdot (X[[\nfix] \cup \{\nctx\}:, ] \Embed \cdot \diag(B) + S) V + \vec{p}(X)_\free \cdot (X[(\nfix, \nctx):, ] \Embed \cdot \diag(B) \\ &+ S) V - \vec{p}(X')_\fix \cdot (X'[[\nfix] \cup \{\nctx\}:, ] \Embed \cdot \diag(B') + S') V - \vec{p}(X')_\free \cdot (X'[(\nfix, \nctx):, ] \Embed \cdot \diag(B) + S) V \bigg\| \\
		   &\leq \max_{X, X', (B, S), (B', S')} \bigg\|\vec{p}(X)_\fix \cdot (X[[\nfix] \cup \{\nctx\}:, ] \Embed \cdot \diag(B) + S) V - \vec{p}(X')_\fix \cdot (X'[[\nfix] \cup \{\nctx\}:, ] \Embed \cdot \diag(B') \\ &+ S') V\bigg\| + \bigg\|\vec{p}(X)_\free \cdot (X[(\nfix, \nctx):, ] \Embed \cdot \diag(B) + S) V - \vec{p}(X')_\free \cdot (X'[(\nfix, \nctx):, ] \Embed \cdot \diag(B) + S) V \bigg\|.
		   \tag{by triangle inequality} \\
	\end{align*}
	Now, we will bound the two norms in the above equation separately.
	First, note that $\ssMinFB \leq \sum \vec{p}(X)_\free \leq \ssMaxFB$ and $\ssMinRB \leq \sum \vec{p}(X)_\fix \leq \ssMaxRB$ by definition of the extremal values (\cref{def:soft-extrem-values}).
	For the case of the free tokens,
	\begin{align*}
		&\max_{X, X', B, B', S, S'} \bigg\|\vec{p}(X)_\free \cdot (X[(\nfix, \nctx):, ] \Embed \cdot \diag(B) + S) V - \vec{p}(X')_\free \cdot (X'[(\nfix, \nctx):, ] \Embed \cdot \diag(B) + S) V \bigg\|_\infty
		\\
		&\leq 2 \cdot \max_{X, B, S} \bigg\|\vec{p}(X)_\free \cdot (X[(\nfix, \nctx):, ] \Embed \cdot \diag(B) + S) V \bigg\|_\infty \tag{by triangle inequality} \\
		&\leq 2 \cdot \max_{B, S} \sum_{i \in (\nfix, \nctx)} \vec{p}(X)_\free[i]  \cdot \left\| (X[i, :] \Embed \cdot \diag(B) + S) V \right\|_\infty \tag{by  triangle inequality} \\
		&\leq 2 \cdot \max_{B, S} \max_i \ssMaxFB \cdot \left\|(X[i, :] \Embed \cdot \diag(B) + S) V \right\|_\infty \tag{by definition of $\vec{p}(X)_\free$} 
	\end{align*}
	where the last inequality follows from the fact that $\vec{p}(X)_\free$ is non-negative and the sum is at most $\ssMaxFB$.
	Finally, note that $\max_i \ssMaxFB \| X[i, :] \Embed \cdot \diag(B) + S) V\| \leq \ssMaxFB\max_{t \in [\dVocab]} \|\Embed[t, :] \cdot (\diag(B) + S) V\|$ as the maximum free token contribution is at most the maximum contribution from any possible token.
	Finally, for the free tokens, we get a bound
	\[
		2 \cdot \ssMaxFB \cdot \max_{B, S, t \in [\dVocab]} \norm{(\Embed[t] \cdot \diag(B) + S) V}_\infty 
		= 2 \cdot \ssMaxFB \cdot \max_{B, S} \norm{(\Embed \cdot \diag(B) + S) V}_\frobInf.
	\]
	We now bound the fixed tokens contribution:
	\[
		\max_{X, X', (B, S), (B', S')} \bigg\|\vec{p}(X)_\fix \cdot (X[[\nfix] \cup \{\nctx\}:, ] \Embed \cdot \diag(B) + S) V - \vec{p}(X')_\fix \cdot (X'[[\nfix] \cup \{\nctx\}:, ] \Embed \cdot \diag(B') + S') V\bigg\|
	\]
	We start by noting that the fixed tokens must be an element of $\desSet \cup \{q\}$ and are, by definition, fixed for all $X$
	And so,
	\begin{align*}
		&\max_{X, X', (B, S), (B', S')} \bigg\|\vec{p}(X)_\fix \cdot (X[[\nfix] \cup \{\nctx\}:, ] \Embed \cdot \diag(B) + S) V - \vec{p}(X')_\fix \cdot (X'[[\nfix] \cup \{\nctx\}:, ] \Embed \cdot \diag(B') \\ &+ S') V\bigg\| = \max_{X, X', (B, S), (B', S')} \bigg\|(\vec{p}(X)_\fix \cdot (\Embed[s \cup \{q\}, :] \cdot \diag(B) + S) V - \vec{p}(X')_\fix \cdot (\Embed[s \cup \{q\}, :] \cdot \diag(B') + S') V\bigg\| \\
		&\leq \max_{X, X', (B, S), (B', S')} \sum_i \bigg\|\vec{p}(X)_\fix[i] \cdot (\Embed[t_i, :] \cdot \diag(B) + S) V - \vec{p}(X')_\fix[i] \cdot (\Embed[t_i, :] \cdot \diag(B') + S') V\bigg\| \tag{by the triangle inequality}
	\end{align*}
	where $t_i$ is the $i$-th fixed token.
	Now, we want to re-write the last line to factor out the differing blowup and shift values.
	\begin{align*}
		&\max_{X, X', (B, S), (B', S')} \sum_i \bigg\|\vec{p}(X)_\fix[i] \cdot (\Embed[t_i, :] \cdot \diag(B) + S) V - \vec{p}(X')_\fix[i] \cdot (\Embed[t_i, :] \cdot \diag(B') + S') V\bigg\| \\
		&= \max_{X, X', (B, S), (B', S')} \sum_i \bigg\|\vec{p}(X)_\fix[i] \cdot (\Embed[t_i, :] \cdot \diag(B) + S) V \\
		-& \vec{p}(X')_\fix[i] \cdot (\Embed[t_i, :] \cdot (\diag(B) + S - (\diag(B) + S - \diag(B') - S')) V\bigg\| \\
		 &\leq \max_{X, X', (B, S)} \sum_i \bigg\|\vec{p}(X)_\fix[i] \cdot (\Embed[t_i, :] \cdot \diag(B) + S) V - \vec{p}(X')_\fix[i] \cdot (\Embed[t_i, :] \cdot \diag(B) + S) V\bigg\| \\
		+& \max_{X, (B, S), (B', S')} \sum_i \vec{p}(X)_\fix[i] \cdot \bigg\| (\Embed[t_i, :] \cdot \diag(B) + S) V - (\Embed[t_i, :] \cdot \diag(B') + S') V\bigg\| \tag{By the triangle inequality}
	\end{align*}
	We now bound the two terms separately.
	For the second term, we have that
	\begin{align*}
		&\max_{X', (B, S), (B', S')} \sum_i \vec{p}(X')_\fix[i] \cdot \bigg\| (\Embed[t_i, :] \cdot \diag(B) + S) V - (\Embed[t_i, :] \cdot \diag(B') + S') V\bigg\| \\
		&\leq \max_{(B, S), (B', S')} \max_i \ssMaxRB \cdot \bigg\| (\Embed[t_i, :] \cdot \diag(B) + S) - (\Embed[t_i, :] \cdot \diag(B') + S')\bigg\| \cdot \norm{V}_\infty \tag{By definition of $\vec{p}(X')_\fix$} \\
		&\leq \norm{V}_\infty \cdot \ssMaxRB \max_{t \in s \cup \{q\}} \WD(\fenc; \{\vece_{t}\} \times \blowupSet \shiftSet).
	\end{align*}

	For the first term, we get
	\begin{align*}
		&\max_{(B, S), X, X'} \sum_i \bigg\|\vec{p}(X)_\fix[i] \cdot (\Embed[t_i, :] \cdot \diag(B) + S) V - \vec{p}(X')_\fix[i] \cdot (\Embed[t_i, :] \cdot \diag(B) + S) V\bigg\| \\	
		&= \max_{(B, S), X,X'} \sum_i \bigg\|\left(\vec{p}(X)_\fix[i] - \vec{p}(X')_\fix[i]\right) \cdot (\Embed[t_i, :] \cdot \diag(B) + S) V \bigg\|\\
		&= \max_{(B, S), X, X'} \sum_i \left(\vec{p}(X)_{\fix}[i] - \vec{p}(X')_{\fix}[i]\right) \cdot \bigg\|(\Embed[t_i, :] \cdot \diag(B) + S) V \bigg\|\\
		&\leq \max_{(B, S), i} (\ssMaxFB - \ssMinFB) \cdot \bigg\|(\Embed[t_i, :] \cdot \diag(B) + S) V \bigg\|
	\end{align*}
	where the last inequality follows from the fact that $\vec{p}(X)_\fix[i] - \vec{p}(X')_\fix[i]$ is at most $\ssMaxFB - \ssMinFB$.
	Then, by definition of the $\frobInf$ norm, we have that
	\[
		\max_{(B, S), i} (\ssMaxFB - \ssMinFB) \cdot \bigg\|(\Embed[t_i, :] \cdot \diag(B) + S) V \bigg\| = (\ssMaxFB - \ssMinFB) \cdot \max_{B, S} \norm{(\Embed[s \cup \{q\}, :] \cdot \diag(B) + S) V}_\frobInf.
	\]

	Putting the above together, we get the desired bound:
	\begin{align*}
		\WD(\vecNctx \cdot f^{\attnH}_{\ndes \mid r, q} ; \InpSpace \times \blowupSet \shiftSet)_\infty
	     &\leq
	     (\ssMaxRB - \ssMinRB) \cdot \max_{B, S}\norm{\Embed[\desSet \cup \{q\}, :] \cdot \diag(B) + S}_{\frobInf} \\
	     &+ 2 \ssMaxFB \cdot \max_{B, S} \norm{(E \cdot \diag(B) + S) \cdot V}_{\frobInf}.
	     + \norm{V}_\infty \ssMaxRB \max_{t \in s \cup \{q\}} \WD(\fenc; \{\vece_{t}\} \times \blowupSet \shiftSet).
	\end{align*}

\end{proof}


\subsection{Proofs for MLP Bounds}
\begin{proof}[Proof of \cref{lem:mlpbound}]
	\label{proof:mlp_bound}
	The second statement follows from $\vec{e}_\nctx \cdot f^{\Iden}_{\ndes \mid r, q} = \fenc$ directly.
	Then, the first statement follows from the Lipschitz constant of the MLP:
	\begin{align*}
		\norm{\MLP(\vec{e}_\nctx \Embed B + S) - \MLP(\vec{e}_\nctx \Embed B' + S')}_\infty
			&\leq \LipMLP_\infty \cdot \norm{\vec{e}_\nctx \Embed B + S - \vec{e}_\nctx \Embed B' + S'}_\infty \\ &\leq \LipMLP_\infty \cdot \WD(\vec{e}_\nctx \cdot  \fenc; \InpSpace \times \blowupSet \shiftSet)_\infty.
	\end{align*}
\end{proof}

\subsection{Proof of \cref{thm:InpRes}}
\label{subsec:InpResProof}
First we will restate \cref{thm:InpRes} for convenience:
\begin{theorem}
	%The model $\desF{\ModelFinal}$ over domain $[X]$ if 
	If
	\begin{align*}
	    &\WD(\desF{\Model}; \InpSpace)_\infty
	  \\&< \peakToPeak(\desF{\Model}, X) / 2,
	\end{align*}
	then the output of model $\desF{\Model}$ is fixed for all inputs in $\InpSpace$.
	Moreover, we can use \cref{alg:overwhelmCheckDet} to produce an upper bound $W$ for $\WD(\desF{\Model}; \InpSpace)_\infty$.
\end{theorem}
\begin{proof}
	We first make use of \cref{thm:metathm} to prove the first statement of the above theorem.
	Now, we just need to prove that the bound, $W$ in \cref{alg:overwhelmCheckDet}, is a valid bound on the lift $\WD(\desF{\Model}; \InpSpace \times \blowupSet \shiftSet)_\infty$.
	By the monotonicity of lifting, we then have that $\WD(\desF{\Model}; \InpSpace)_\infty \leq \WD(\desF{\Model}; \InpSpace \times \blowupSet \shiftSet)_\infty.$

	First note that, by \cref{lemma:worst_case_deviation_properties}, we have that
	\begin{align*}
		\WD(\desF{\Model}; \InpSpace \times \blowupSet \shiftSet)_\infty &\leq \Lip(\Unembed)_\infty \cdot (\WD(f^{\attnH}; \InpSpace \times \blowupSet \shiftSet)_\infty \\ &+ \WD(f^{\MLP}; \InpSpace \times \blowupSet \shiftSet)_\infty + \WD(f^{\Iden}; \InpSpace \times \blowupSet \shiftSet)_\infty) 
	\end{align*}
	where
	\[
		f^{\component}(X, (B, S)) = \component \circ \fenc(X, (B, S)) 
	\]
	and
	\[
		\fenc(X, (B, S)) = (X \cdot \Embed \cdot \diag(B) + S).
	\]
	Then, we can use \cref{lem:att_bound} to get that $W^\attn$ is a valid bound on the worst-case deviation of $f^{\attnH}$.
	Then, we use \cref{lem:mlpbound} to get that $\Lip(\MLP)_\infty \cdot \WD(\fenc; \InpSpace \times \blowupSet\shiftSet)$ is a valid bound on the worst-case deviation of $f^{\MLP}$.
	Finally, we use \cref{lem:WD_fenc} to get the bound on $\fenc$.
\end{proof}
%\subsection{Lipschitz Constant of the Later Layers}
%\label{subsec:later_layer_lip}
%
%\subsubsection{Lipschitz Constant of Query and Key}
%\label{subsubsec:query_key_lip}
%
%\lev{TODO check, and cite for quad form
%also use picture on this medium post %https://medium.com/@ngiengkianyew/understanding-rotary-positional-encoding-40635a4d078e
%for rotary encodings explanation...
%}
%\begin{lemma}
%	\label{lem:quad_form_lip}
%	Let $f(x,y) = x^T A y$ where $A = Q \cdot \diag(R(\theta_i)) \cdot K^T$ and $R(\theta_i)$ are 2×2 rotation matrices.
%	For all $p$-norms with $p \geq 2$, the Lipschitz constant of the query and key matrices is bounded by \lev{TODO fix}:
%	$$\|f(x_1,y_1) - f(x_2,y_2)\| \leq \|A\|(\|x_1-x_2\|\|y_1\| + \|x_2\|\|y_1-y_2\|)$$
%	with $\|A\| \leq \|Q\|\|K\|$.
%\end{lemma}
%\begin{proof}
%	Consider the difference of bilinear forms $f(x_1,y_1) - f(x_2,y_2) = x_1^TAy_1 - x_2^TAy_2$.
%	This can be rewritten as $(x_1^T-x_2^T)Ay_1 + x_2^TA(y_1-y_2)$ by adding and subtracting $x_2^TAy_1$.
%	Applying the triangle inequality and submultiplicativity of norms yields \[
%		\|f(x_1,y_1) - f(x_2,y_2)\| \leq \|A\|(\|x_1-x_2\|\|y_1\| + \|x_2\|\|y_1-y_2\|).
%	\]
%	The operator norm of $A$ can be bounded as \[
%		\|A\|_p = \|Q \cdot \diag(R(\theta_i)) \cdot K^T\| \leq \|Q\|\|\diag(R(\theta_i))\|\|K\| \leq \|Q\|\|K\|
%	\]
%	where the last inequality follows from the fact that rotation matrices have norm at most 1 for all $p \geq 2$.
%	\lev{TODO: specify $p \geq 2$}
%\end{proof}
%
%%\begin{lemma}
%%	\lev{TODO: check rotation}
%%	Let $f(x) = x^T A x$ where $A = Q \cdot \text{diag}(R(\theta_i))  \cdot K^T$ and $R(\theta_i)$ are 2×2 rotation matrices. Then the Lipschitz constant $\Lip(f)_p$ for norm $1 \leq p \leq \infty$ of $f$ is bounded by:
%%	\[
%%		\Lip(f)_p \leq 2\norm{Q}_p \norm{K^T}_p
%%	\]
%%	where $\|\cdot\|$ denotes the operator norm.
%%\end{lemma}
%%
%%\begin{proof}
%%	First, recall that for any differentiable function $f$, the Lipschitz constant $L$ is given by:
%%	$$L = \sup_{x \neq y} \frac{\|\nabla f(x) - \nabla f(y)\|_p}{\|x-y\|_p}$$
%%	For quadratic forms, this equals: \lev{TODO: citation here}
%%	$$L = \sup_{\|x\|=1} \|\nabla f(x)\|$$
%%	So, we can then calculate the gradient:
%%	\[
%%		\nabla f(x) = (A + A^T)x = (Q \cdot \text{diag}(R(\theta_i))\cdot K^T + K\cdot \text{diag}(R(\theta_i)^T)\cdot Q^T)x
%%	\]
%%	For any rotation matrix $R(\theta)$:
%%	\begin{align*}
%%		R(\theta) &= \begin{bmatrix} \cos(\theta) & -\sin(\theta) \\ \sin(\theta) & \cos(\theta) \end{bmatrix} \\
%%	\end{align*}
%%	which implies that $\|R(\theta)\|_p \leq 1$.
%%	Also, by the submultiplicativity of operator norms:
%%	$$\|A\| = \|Q \cdot \text{diag}(R(\theta_i)) \cdot K^T\| \leq \|Q\| \|\text{diag}(R(\theta_i))\| \|K^T\|$$
%%	Finally, since $\|\text{diag}(R(\theta_i))\| = \max_i \|R(\theta_i)\| \leq 1$,
%%	we get that $\|A\| \leq \|Q\| \|K\|$.
%%	Therefore,
%%	\[
%%		L = \|\nabla f\| = \|A + A^T\| \leq \|A\| + \|A^T\| = 2\|A\| \leq 2\|Q\| \|K\|.
%%	\]
%%\end{proof}
%%

\section{Appendix for Permutation Invariance}
\label{sec:appendix_perm}

\begin{algorithm}[tb] 
	\caption{Naive Algorithm to find $\ssMinFA$.\\
		To find $\ssMaxFA$, switch the $\min$ to a $\max$ in the objective.
	}
	\label{alg:naiveAlpha}
	\begin{algorithmic}
	\STATE \textbf{Input:} {$\fenc$ and model $\Model$}
	% \Output{\lev{TODO}}
	Let $\fenc(\vec{e}) = \vec{e} \cdot \Embed \cdot \diag(B) + S$ \\
	\FOR{$k \in (\ndes, \nctx]$}
	\STATE {
		Set \begin{align*}
			\vec{\ell}^{\min}_k = \frac{1}{\sqrt{\dEmb}} \min_{t \in \freeToks} 
			\fenc(\vecQuery) \cdot Q \PosRot_{k, \nctx} K^T \cdot \fenc(\vec{e}_t)^T
		\end{align*}
	}
	\ENDFOR
	\STATE Return $\ssMinFA = \sum_k \exp\left(\vec{\ell}^{\min}_k\right)$
	\end{algorithmic}
\end{algorithm}

\subsection{Proof of Attention Worst-Case Deviation}
In this subsection, we provide a proof of \cref{lem:corrAlpha} and \cref{lem:attn_perm}.
First, we prove the correcntess of the algorithms to find $\ssMinFA$ and $\ssMaxFA$.
\begin{lemma}[Restatement of \cref{lem:corrAlpha}]
	 Both \cref{alg:LPAlpha} and \cref{alg:naiveAlpha} provide valid bounds on $\ssMinFA$ and $\ssMaxFA$.
\end{lemma}
\begin{proof}
	We provide a proof for $\ssMinFA$; the proof for $\ssMaxFA$ is analogous.

	First note that the blowup and shift are fixed and thus $\fenc$ is only a function of the its input token: i.e.\ we can rewrite $\fenc(\vec{e}_t, BS)$ as $\fenc(\vec{e}) = \vece \cdot E \cdot \diag(B) + S$ for fixed $B$ and $S$.
	Then, for the naive algorithm, \cref{alg:naiveAlpha}, the correctness follows from the fact that the minimum possible logit value from each position can be calculated by enumerating over all the free tokens.

	For the LP-based algorithm, \cref{alg:LPAlpha}, we first consider the LP as an integer linear program.
	Consider the permutation matrix $X$ with entries $x_{j, t}$, where $x_{j, t} = 1$ if token $t$ is selected for position $j$ and $x_{j, t} = 0$ otherwise.
	Then, the LP can be seen as finding the permutation matrix which minizes the objective function, $\ssMinFA$.
	So, the relaxed LP is a lower bound on the integer LP which is a lower bound on the true value of $\ssMinFA$.
\end{proof}

We now restate the lemma for the worst-case deviation of attention.
\begin{lemma}
	Worst-case deviation of attention is bounded as follows for fixed blowup $B$ and shift $S$ when considering the permutation class $[X]$ with free tokens $\freeToks$:
	\begin{align*}
		&\WD(\vecNctx \cdot f^{\attnH}_{\ndes \mid r, q} ; \InpSpace \times \blowupSet \shiftSet)_\infty
	     \\ &\leq
		2 \cdot (\ssMaxRB - \ssMinRB) \cdot \norm{\Embed[[\nfix] \cup \{\nctx\}, :] \cdot B + S}_{\frobInf} \\
		&+ 2 \cdot \ssMaxFB \cdot \norm{(E[(\nfix,\nctx),: ] B + S) \cdot V}_{\frobInf}.
	\end{align*}
\end{lemma}
\begin{proof}
	The proof follows analogously to the proof of \cref{lem:att_bound} (bound on worst-case deviation of attention in the generic case).
	The main difference is that we have a singleton set for the blowup and shift and that we restrict the free tokens.
	First, this implies that the restriction of $\blowupSet, \shiftSet$ to singelton sets means that no optimization over blowup and shift are needed in the bounds as the blowup and shift become constants.
	Second, the restriction of the free tokens means that the worst-case contribution from the value function, $(E \diag(B) + S \cdot V)$, is restricted to the embeddings selected by the free tokens.
	So, we replace $\norm{E \diag(B) + S \cdot V}_{\frobInf}$ with $\norm{(E[(\nfix,\nctx),: ] \diag(B) + S) \cdot V}_{\frobInf}$.
	Moreover, $\WD(\fenc; \{\vece_t\} \times \blowupSet \shiftSet)_\infty = 0$ as the blowup and shift are fixed and so we can drop the last term in the bound in \cref{lem:att_bound}.
\end{proof}

\subsection{Proof of \cref{thm:InpResPermInv}}
\label{subsec:proofInpResPermInv}
We first restate the theorem for convenience.
\begin{theorem}
	If
	\[
		\WD(\desF{\Model}; [X])_\infty < \peakToPeak(\desF{\Model}, X) / 2
	\]
	then the output of model $\desF{\Model}$ is fixed for all inputs in $[X]$.
	Moreover, we \cref{alg:overwhelmCheck2} in \cref{sec:appendix_perm} produces an upper bound $W$ for $\WD(\desF{\Model}; \InpSpace \times \blowupSet \shiftSet)_\infty$.
\end{theorem}
\begin{proof}
	The proof follows analogously to the proof of \cref{thm:InpRes} in \cref{subsec:InpResProof}.
	The key difference is that, due to the fixing of the blowup and shift, the bound on $W$ simplifies.
	First, the worst-case deviation of $\vece_{\nctx} \cdot \fenc(X)$ is $0$ as the blowup and shift are fixed.
	So, the worst-case deviation of the MLP and identity function are $0$ as well.
	Then, algorithm uses either \cref{alg:naiveAlpha} or \cref{alg:LPAlpha} to find bounds on the extremal values of attention with provable correctness as per \cref{lem:corrAlpha}.
	Next, the algorithm simplified worst-case deviation of attention (as per \cref{lem:attn_perm}) to bound the worst-case deviation of attention.
	Finally, the algorithm uses the Lipschitz constant of the unembedding function, invoking \cref{lemma:worst_case_deviation_properties}, to bound the worst-case deviation of the model.
\end{proof}

\section{Overwhelming for $\nctx \to \infty$}
\label{sec:convergence}

In this section, we will consider a specific set $\desSet = \bigtimes_{\nfix} \vecRep$ and $\query = \vecRep$ for some fixed $r \in [\dVocab]$.
In words, we will consider the fixed input to be one repeated token.
We term this a ``repetition restriction.''

Moreover, it will be useful to define the set of all possible inputs under a repetition restriction.
\begin{definition}[Repetition Space]
	\label{def:RepSpace}
	Let $\RepSpace \subset \OneHotSpace^\nctx$ be the set of all possible inputs under a repetition restriction.
	That is, 
	\[
		\RepSpace = \left\{ X \in \R^{\nctx \times \dVocab} \mid X = 
		\begin{bmatrix}
			\vecRep^T \\
			\vecRep^T \\
			\vdots \\
			\vecRep^T \\
			Y \\
			\vecQuery
		\end{bmatrix}, Y \in \OneHotSpace^{\nfree}
		\right\}.
	\]
\end{definition}

For simplicity, we will also not consider positional encodings in this section. 
I.e.\ we remove the use of RoPE in the attention mechanism.
Though as Ref.~\cite{barbero2024transformers} pointed out, rotary positional encodings \cite{su2024roformer} converge to providing zero information as $\nctx \to \infty$.

\begin{theorem}[Asymptotic Convergence to a Fixed Model]
	\label{thm:convergence}
	If $\frac{\nfree}{\nctx} \in o(1)$ as a function of $\nctx$, then the repetition restriction converges to a fixed model if $\peakToPeak(\Model, X)$ is positive for $X = \vece_r$.
\end{theorem}

To prove this theorem, we will need to (a) find a way to compute a $\peakToPeak$ function of the model and input \emph{independent of} $\nctx$ and (b) use the framework of \cref{sec:meta_framework} to individually bound the worst-case deviation for each component of the model.

\subsection*{Computing Peak-to-Peak Difference}
Given that $\peakToPeak$ is computed by evaluating the model on a single input, finding a $\peakToPeak$ value is normally a simple task.
But, in the asymptotic case, we need to ``shortcut'' the computation of $\peakToPeak$ for an $X \in \RepSpace$.
We can simply do this by setting all the free tokens to the repetition token: $X = {\vecRep}^{\nfree}$.

\begin{lemma}[Shortcut for $\peakToPeak$]
	\label{lem:gap_shortcut}
	We can compute $\peakToPeak(\Model, X)$ in time $O(1)$ for $X=\vece_\rep^{\nctx}$ even as $\nctx \rightarrow \infty$.
\end{lemma}
\begin{proof}
	We can note that $\E[X] = \vecDes \cdot \Embed$ and $\Var[X] = \Var[\vecDes \cdot \Embed] = 0$.
	So, we can easily compute blowup, $B$, and shift, $S$.
        Note, for attention head $\vece_\nctx \cdot \desF{f^\attnH}$, the output equals 
        \[
            \vec{p}(X) \cdot (XE \cdot \diag(B) + S) \cdot V
        \]
        for some probability vector $\vec{p}(X)$.
        Because $X = \vecRep^\nctx$, $(XE \cdot \diag(B) + S) \cdot V= \vec{a}^\nctx$ for vector $\vec{a} = (\vecRep \cdot E \cdot \diag(B) + S) V$.
        Thus,
        $$
            \vec{p}(X) \cdot (XE \cdot \diag(B) + S) \cdot V = \vec{a}.
        $$
	We therefore get that the output of $\vece_\nctx \cdot \desF{f^\attnH}$ is fixed to $\vecRep (\Embed \cdot \diag(B) + S) \cdot V$ for all context windows $\nctx$.
	Finally, we can simply compute the value of $\vece_\nctx \cdot \desF{f^\MLP}$ because we know $B$ and $S$ and so just need to compute $\MLP(\vec{e}_r (E \cdot \diag(B) + S))$.
\end{proof}

\subsection*{Bounds on Expectation and Variance}
\begin{lemma}[Bounds on Expectation and Variance]
	\label{lem:bounds}
	Let $X' \in \RepSpace$ and $X = \left(\bigtimes_{i = 1}^{\ndes} \vecDes \right) \times X' \times \vecQuery$. I.e.\ $X$ is an expansion of the input with the first tokens being the repetition token and the last being the query token.
	Then, we can bound the column wise expectation for column $j$, $\mu_j$, and variance, $\Var_j$ of $X$ by
	\begin{align*}
		\frac{1}{\nctx}\left(
			\ndes \cdot \vecDes \Embed \vec{e}_j^T + \nfree \min_i (\vec{e}_i \Embed \vec{e}_j^T) + \vecQuery \Embed \vec{e}_j^T
		\right)
		\leq \mu_j \leq  \\
		\frac{1}{\nctx}\left(
			\ndes \cdot \vecDes \Embed \vec{e}_j^T + \nfree \max_i (\vec{e}_i \Embed \vec{e}_j^T) + \vecQuery \Embed \vec{e}_j^T
		\right).
	\end{align*}
	Let $\mu_j^{\min}$ and $\mu_j^{\max}$ denote the lower and upper bounds.
	Then, we can bound the variance of the $j$-th column of $\vec{x} \Embed$ by
	\begin{align*}
            \Var_j 
			\leq &\frac{1}{\nctx} \cdot \max_{\mu' \in [\mu_j^{\min}, \mu_j^{\max}]} \bigg[
			 (\ndes +1) \cdot  \left( \vecDes \Embed \vec{e}_j^T -  \mu' \right)^2 + 
			 \nfree \cdot \max_{i \in \dVocab} \left( \vec{e}_i \Embed \vec{e}_j^T -  \mu' \right)^2
			\bigg].
	\end{align*}
	We define $\Var_j^{\min}, \Var_j^{\max}$ to be the lower and upper bounds respectively.
\end{lemma}
\begin{proof}
	The bounds on $\mu_j$ follow as we minimize and maximize the contribution of each free row to the expectation of the column of $j$.
	The proof of for variance bounds follows similarly.
	For the lower bound, we note that the variance contribution from each row is at least $0$.
	For the upper bound, we simply maximize the contribution of each row to the variance. 
\end{proof}

Recall the definition of blowup and shift sets from \cref{def:blowup_shift}.
\begin{lemma}[Blowup and Shift Set Bounds]
	\label{lem:blowup_shift}
	Recall that $\gamma$ and $\beta$ are the learned constants for layer normalization.
	Let $\vec{x} \in \InpSpace$.
	Then, we can bound the blowup and shift sets for the $j$-th column of $\vec{x} \Embed$.
	For all $S_j \in \shiftSet_j$ and $B_j \in \blowupSet_j$, we have
	\begin{align*}
		\frac{\gamma}{\sqrt{\Var_j^{\max} + \eps}} \leq B_j \leq \frac{\gamma}{\sqrt{\eps}}
	\end{align*}
	Let $B_j^{\min}$ and $B_j^{\max}$ denote the lower and upper bounds respectively.
	Then,
	\begin{align*}
		B_j^{\min} \mu_j^{\min} + \beta \leq S_j \leq B_j^{\max} \mu_j^{\max} + \beta.
	\end{align*}
\end{lemma}

Now as a corollary of \cref{lem:bounds} and \cref{lem:blowup_shift}, we can show that the minimum and maximums of the blowup and shift sets can be made arbitrarily close together.
\begin{corollary}[Convergence of Blowup and Shift Sets]
	If $\frac{\nfree}{\nctx} \in o(1)$, then for every $\delta_1, \delta_2 \in (0, 1)$, there exists some setting of $\nctx'$ such that for all $\nctx > \nctx'$,
	\[
		B_j^{\max} - B_j^{\min} \leq \delta_1 \quad \text{and} \quad S_j^{\max} - S_j^{\min} \leq \delta_2.
	\]
\end{corollary}
\begin{proof}[Proof]
    The corollary follows from the fact that $\mu_j^{\min}, \mu_j^{\max}$ converge to $\vec{e}_r E \vec{e}_j^T$ as $\nctx \rightarrow \infty$.
    So then, $(\vec{e}_r E \vec{e}_j^T - \mu')^2$ for all $\mu' \in [\mu_j^{\min}, \mu_j^{\max}]$ converges to $0$ and thus the upper bound on $\Var_j$ converges to $0$ as long as $\frac{\nfree}{\nctx}$ converges to $0$.
\end{proof}

Therefore the blowup and shift sets converge, let $B_j'$ and $S_j'$ denote the converged values for $B_j$ and $S_j$ respectively.

Now, we need to show a bound on the difference between the maximum logit for the free tokens and the minimum logit for the repeated tokens in the attention head.

\subsubsection*{Bound on Attention Head Difference}
To bound attention, we will take advantage of the repeated structure as well as the ideas in \cref{lem:min_max_softmax} (bounds on the attention weights).

\begin{corollary}[Sum of Attention Weights]
	\label{cor:sum_attention}
	Let $\logRMin \leq \min_{i \in [\ndes] \cup \{n_ctx\}} \ell_i$ and $\logFMin \leq \min_{j \in (\nfix, \nctx)} \ell_j$.
	Also, let $\logRMax \geq \max_{i \in [\ndes] \cup \{n_ctx\}} \ell_i$ and $\logFMax \geq \max_{i \in (\nfix, \nctx)} \ell_i$. Then,
	\begin{align*}
	\frac{\ndes}{\ndes + 1 + (\nfree) \cdot e^{\logFMax - \logRMin}} 
\leq 
		\sum_{i \in [\ndes] \cup \{\nctx\}} \softmax(\vec{\ell}_i)
	\leq
		\frac{\ndes}{\ndes + 1 + (\nfree) \cdot e^{\logFMin - \logRMax}}
	\end{align*}
	and 
	\begin{align*}
		\frac{\nfree}{\ndes + 1 + (\nfree) \cdot e^{\logFMin - \logRMax}} 
        \leq \sum_{j \in (\nfix, \nctx)} \softmax(\vec{\ell}_j)
        \leq \frac{\nfree}{\ndes + 1 + (\nfree) \cdot e^{\logFMax - \logRMin}}.
	\end{align*}
\end{corollary}

\begin{lemma}[Bound on Difference]
	\label{lem:bound_diff}
	We can bound the difference between the maximum logit for the free tokens and the minimum logit for $\ndes$ repeated tokens
	\begin{align}
		\logFMax - \logRMin &\leq \theta
	\end{align}
	and
	\begin{align}
		\logFMin - \logRMax &\leq \theta
	\end{align}
	where $\theta$ is a small constant.
\end{lemma}
\begin{proof}
	We will prove the first inequality as the second follows analogously.
	Assume that $i \in [\dVocab]$ is the token which maximizes the difference.
	We will relly on the convergence of the blowup and shift sets to show that the difference between the maximum logit for the free tokens and the minimum logit for the repeated tokens in the attention head is bounded.
	
	Connote $\blowupSet - B'$ as the set of blowup values shifted by the converged value and $\shiftSet - S'$ as the set of shift values shifted by the converged value.
	Then $\blowupSet - B'$ and $\shiftSet - S'$ are bounded by $\delta_1, \delta_2$ respectively, we can write the following bound on the difference.
	\begin{align*}
		\logFMax - \logRMin &\leq \max_{\eps \in \blowupSet - B', \; \eps_S \in \shiftSet - S'} (\vec{e}_i - \vecDes) E \cdot \diag(B' + \eps) \\ & \cdot Q K^T (\diag(B' + \eps) \cdot E^T \vecQuery^T + S^T + \eps_S) \\
					   &= \max_{\eps \in \blowupSet - B', \eps_S \in \shiftSet - S'} \\& \sum_{d_1 \in \{\eps, B'\}} \sum_{d_2 \in \{\eps, B'\}} \sum_{d_3 \in \{\eps, S'\}} (\vec{e}_i - \vecDes) E \cdot \diag(d_1) \cdot Q K^T (\diag(d_2) \cdot E^T \vecQuery^T + d_3) \\
					   &\leq \max_{\eps \in \blowupSet - B', \eps_S \in \shiftSet - S'} (\vec{e}_i - \vecDes) E \cdot \diag(B') \cdot Q K^T (\diag(B') \cdot E^T \vecQuery^T + S^T) + \poly(\eps, \eps_S)\\
					   &\leq (\vec{e}_i - \vecDes) E \cdot \diag(B')  \cdot Q K^T (\diag(B') \cdot E^T \vecQuery^T + S^T) + \poly(\delta_1, \delta_2, B', S')
                       \tag{By the bounds on the blowup and shift set}
                       \\
					   &\leq (\vec{e}_i - \vecDes) E \cdot \diag(B')  \cdot Q K^T (\diag(B') \cdot E^T \vecQuery^T + S^T) + \poly(\delta_1, \delta_2)
	\end{align*}
	where the last inequality follows from considering $B'$ and $S'$ as a constant.
	Then, because $\delta_1, \delta_2$ decrease as a function of $\nctx$ and \cref{cor:sum_attention},
	we can bound
	\begin{align*}
		&(\vec{e}_i - \vecDes) E \cdot \diag(B') 
        \cdot 
        Q K^T (\diag(B') \cdot E^T \vecQuery^T + S^T) + \poly(\delta_1, \delta_2) \leq \theta
	\end{align*}
	for a large enough $\nctx$.
\end{proof}


\begin{lemma}[Attention Bound, \cref{lem:att_bound}]
	\label{lem:convgattn}
	For large enough $\nctx$, we get
	\[
		\WD(\desF{f^\attnH}; \RepSpace)_\infty \leq o(1).
	\]
\end{lemma}
\begin{proof}
	\label{proof:convgattn}
	Recall that \cref{lem:att_bound} gives us
	\begin{align*}
		&\WD(\vecNctx \cdot f^{\attnH}_{\ndes \mid r, q} ; \InpSpace \times \blowupSet \shiftSet)_\infty  
						 \leq					   (\ssMaxRB - \ssMinRB) \cdot \max_{B, S}\norm{\vecDes \Embed B + S} + 
											   2 \cdot \ssMaxFB \cdot \max_{B, S}\norm{(\vecDes \Embed B + S) \cdot V}_\infty
	\end{align*}
	Because $\logFMax - \logRMin \leq \theta$ and $\logFMin - \logRMax \leq \theta$,
	we can use \cref{cor:sum_attention} to upper-bound $\ssMaxRB - \ssMinRB$ and $\ssMaxFB$ by
	\begin{align*}
		1 - \frac{\ndes}{\ndes + (\nfree + 1) \cdot \theta}
		= \frac{(\nfree + 1) \cdot \theta}{\ndes + (\nfree + 1) \cdot \theta} 
		\leq \frac{\nfree \cdot \theta}{\ndes} 
		\leq \theta \cdot o(1).
	\end{align*}
	So then
	\[
		\WD(\desF{f^{\attnH}} ; \InpSpace \times \blowupSet \shiftSet)_\infty \leq 
		o(1) 
	\]
	as $\theta \in o(1)$.
\end{proof}

\subsubsection*{Bound on $\fenc$}
Finally, as per \cref{sec:meta_framework}, we need to bound the worst-case deviation of the $\fenc$ function.
Given that $\fenc$ is a linear operation as a function of blowup and shift sets, we can use the convergence of the blowup and shift sets to show that the worst-case deviation of $\fenc$ is bounded.

\begin{lemma}[Bound on $\fenc$]
	\label{lem:conv_fenc_bound}
	For large enough $\nctx$, we have
	\[
		\WD(\vec{e}_\nctx \cdot \desF{\fenc}; \RepSpace)_\infty \leq \poly(\delta_1, \delta_2).
	\],
    where $\delta_1, \delta_2$ are the bounds on the blowup and shift differences.
\end{lemma}
%\begin{proof}
%	The proof follows trivially from the fact that $\vec{e}_{\nctx} \cdot \desF{\fenc} = \vecQuery \cdot E B + S$.
%	Then, we can use the $\delta_1, \delta_2$ bounds on the blowup and shift sets to show that the worst-case deviation of $\vec{e}_\nctx \cdot \desF{\fenc}$ is bounded a polynomial function of $\delta_1, \delta_2$.
%	So, as long as $\peakToPeak(\desF{\Model}, \vecDes^{\nfree})$ is positive, we can always set $\nctx$ to be large enough such that $\WD(\vec{e}_\nctx \cdot \desF{\fenc}; \InpSpace \times \blowupSet \shiftSet)$ is arbitarily small and thus $\WD(\vec{e}_\nctx \cdot \desF{\fenc}; \InpSpace)_2$ is arbitraily small by the lifting monotonicity of worst-case deviation (\cref{lemma:liftMon}).
%\end{proof}

We are now ready to prove \cref{thm:convergence}.
\begin{proof}[Proof of \cref{thm:convergence}]
	Note that we reduced $\WD(\vec{e}_\nctx \cdot \desF{\Model}; \RepSpace)_\infty$ to be upper-bounded by 
	$
	O(1) \cdot \WD(\vec{e}_\nctx \cdot \desF{\fenc}; \RepSpace) + o(1)
	$
	through \cref{lem:mlpbound} and \cref{lem:convgattn}.
	Then, by \cref{lem:conv_fenc_bound}, we have that \cref{thm:convergence} holds as $\WD$ goes to $0$ as $\nctx \rightarrow \infty$.
    So, as long as $\peakToPeak(\desF{\Model}, X)$ is positive for \emph{some} $X \in \RepSpace$, then we converge to ``overwhelming'' by \cref{thm:metathm}.
    Note that $\vece_\rep^\nctx \in \RepSpace$ and by \cref{lem:gap_shortcut}, we can compute a sample of peak-to-peak deviation for all $\nctx$.
\end{proof}







\end{document}
