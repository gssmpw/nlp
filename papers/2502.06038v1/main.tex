\documentclass{article}
\pdfoutput=1
\usepackage{Custom}
%\usepackage{Custom2}
\usepackage{icml2025/fancyhdr}
\usepackage{icml2025/algorithm}
\usepackage{icml2025/algorithmic}
\usepackage[accepted]{icml2025/icml2025}
\usepackage{listings}
\usepackage{multicol}
\usepackage{authblk}
\usepackage{multirow}
\bibliographystyle{icml2025/icml2025}


%\date{today}
\title{\bfseries\Large
    Provably Overwhelming Transformer Models with Designed Inputs
}

\renewcommand\Affilfont{\fontsize{9}{10.8}\itshape}

%\author[1, 2]{Lev Stambler{\href{mailto:levstamb@umd.edu}{levstamb@umd.edu}}}
%\author[1, 2]{Seyed Sajjad Nezhadi}
%\author[1, 2, 3]{Matthew Coudron}
%
%\affil[1]{Joint Center for Quantum Information and Computer Science, University of Maryland}
%\affil[2]{Department of Computer Science, University of Maryland}
%% \affil[3]{Neon Tetra LLC}
%\affil[3]{National Institute of Standards and Technology}
% If you read this, I hope that you are having a nice day!

\newcommand{\mnote}[1]{{\highlightname{Matt C.}{#1}{blue}}}
\newcommand{\snote}[1]{{\highlightname{Sajjad}{#1}{red}}}
\newcommand{\mkd}[1]{{\highlightname{Matt K}{#1}{orange}}}
\newcommand{\lev}[1]{{\highlightname{Lev}{#1}{purple}}}
\newcommand{\claude}[1]{{\highlightname{Claude}{#1}{green}}}

 %\renewcommand{\mnote}[1]{}
 %\renewcommand{\snote}[1]{}
 %\renewcommand{\mkd}[1]{}
 %\renewcommand{\lev}[1]{}
 %\renewcommand{\claude}[1]{}



% Worst-Case Over-Squashing Proofs for Transformer Models
% Provably Overwhelming Transformer Models with Designed Inputs

\icmltitlerunning{
    Provably Overwhelming Transformer Models with Designed Inputs
}
\begin{document}
\sloppy


\twocolumn[
\icmltitle{
    Provably Overwhelming Transformer Models with Designed Inputs
}
\begin{icmlauthorlist}
	\icmlauthor{Lev Stambler}{Quics,UMD,NT}
	\icmlauthor{Seyed Sajjad Nezhadi}{Quics,UMD,Iluvatar}
	\icmlauthor{Matthew Coudron}{Quics,UMD,NIST}
\end{icmlauthorlist}
\icmlaffiliation{Quics}{Joint Center for Quantum Information and Computer Science, University of Maryland}
\icmlaffiliation{UMD}{Department of Computer Science, University of Maryland}
\icmlaffiliation{NIST}{National Institute of Standards and Technology}
\icmlaffiliation{NT}{Neon Tetra LLC}
\icmlaffiliation{Iluvatar}{iluvatar Technologies}
\icmlcorrespondingauthor{Lev Stambler}{levstamb@umd.edu}


\icmlkeywords{ML Theory, Formal Guarantees, Transformers, Interpretability, Machine Learning, ICML}

\vskip 0.3in
]
\numberwithin{theorem}{section}  % This links theorem numbers with section numbers

\theoremstyle{plain}     % For theorems, lemmas, etc.


% \maketitle

%%%%%%%%%%%---SETME-----%%%%%%%%%%%%%
%replace @@ with the submission number submission site.
\newcommand{\thiswork}{INF$^2$\xspace}
%%%%%%%%%%%%%%%%%%%%%%%%%%%%%%%%%%%%


%\newcommand{\rev}[1]{{\color{olivegreen}#1}}
\newcommand{\rev}[1]{{#1}}


\newcommand{\JL}[1]{{\color{cyan}[\textbf{\sc JLee}: \textit{#1}]}}
\newcommand{\JW}[1]{{\color{orange}[\textbf{\sc JJung}: \textit{#1}]}}
\newcommand{\JY}[1]{{\color{blue(ncs)}[\textbf{\sc JSong}: \textit{#1}]}}
\newcommand{\HS}[1]{{\color{magenta}[\textbf{\sc HJang}: \textit{#1}]}}
\newcommand{\CS}[1]{{\color{navy}[\textbf{\sc CShin}: \textit{#1}]}}
\newcommand{\SN}[1]{{\color{olive}[\textbf{\sc SNoh}: \textit{#1}]}}

%\def\final{}   % uncomment this for the submission version
\ifdefined\final
\renewcommand{\JL}[1]{}
\renewcommand{\JW}[1]{}
\renewcommand{\JY}[1]{}
\renewcommand{\HS}[1]{}
\renewcommand{\CS}[1]{}
\renewcommand{\SN}[1]{}
\fi

%%% Notion for baseline approaches %%% 
\newcommand{\baseline}{offloading-based batched inference\xspace}
\newcommand{\Baseline}{Offloading-based batched inference\xspace}


\newcommand{\ans}{attention-near storage\xspace}
\newcommand{\Ans}{Attention-near storage\xspace}
\newcommand{\ANS}{Attention-Near Storage\xspace}

\newcommand{\wb}{delayed KV cache writeback\xspace}
\newcommand{\Wb}{Delayed KV cache writeback\xspace}
\newcommand{\WB}{Delayed KV Cache Writeback\xspace}

\newcommand{\xcache}{X-cache\xspace}
\newcommand{\XCACHE}{X-Cache\xspace}


%%% Notions for our methods %%%
\newcommand{\schemea}{\textbf{Expanding supported maximum sequence length with optimized performance}\xspace}
\newcommand{\Schemea}{\textbf{Expanding supported maximum sequence length with optimized performance}\xspace}

\newcommand{\schemeb}{\textbf{Optimizing the storage device performance}\xspace}
\newcommand{\Schemeb}{\textbf{Optimizing the storage device performance}\xspace}

\newcommand{\schemec}{\textbf{Orthogonally supporting Compression Techniques}\xspace}
\newcommand{\Schemec}{\textbf{Orthogonally supporting Compression Techniques}\xspace}



% Circular numbers
\usepackage{tikz}
\newcommand*\circled[1]{\tikz[baseline=(char.base)]{
            \node[shape=circle,draw,inner sep=0.4pt] (char) {#1};}}

\newcommand*\bcircled[1]{\tikz[baseline=(char.base)]{
            \node[shape=circle,draw,inner sep=0.4pt, fill=black, text=white] (char) {#1};}}

\printAffiliationsAndNotice{} % otherwise use the standard text.

\iffalse
\begin{abstract}
We develop an algorithm which, given a trained single-layer transformer model $\mathcal{M}$ as input, as well as a string of tokens $s$ of length $n_{fix}$ and an integer $n_{free}$, can generate a mathematical proof that $\mathcal{M}$ is ``overwhelmed'' by $s$.
We say that $\mathcal{M}$ is ``overwhelmed'' by $s$ when the output of the model evaluated on this string plus any additional string $t$, $\mathcal{M}(s + t)$, is completely insensitive to the value of the string $t$ whenever $\text{length}(t) \leq n_{free}$.
   
An exhaustive approach to verifying such a worst-case statement would require time exponential in $n_{free}$, but our approach uses a carefully designed analysis of Lipschitz continuity, combined with convex relaxations to produce a worst-case proof in time and space $\widetilde{O}(n_{fix}^2 + n_{free}^3)$. 
Along the way, we prove a particularly strong worst-case form of ``over-squashing'' \cite{alon2021bottleneckgraphneuralnetworks, barbero2024transformers}, which we use to bound the model's behavior.

Our technique uses computer-aided proofs to establish this type of operationally relevant guarantee about transformer models.
We empirically test our algorithm on a single layer transformer complete with an attention head, layer-norm, MLP/ReLU layers, and RoPE positional encoding.
For this model, our algorithm produces proofs of natural overwhelming strings when the ``free string'' is restricted to be an element of a permutation set.  

We believe that this work is a stepping stone towards the difficult task of obtaining useful guarantees for trained transformer models.
For example, ``overwhelming'' strings can be used to prove no-go results for prompt engineering: no ``prompt'' can equip a model $\mathcal{M}$ to properly execute specific tasks like correcting errors in code.
\end{abstract}
\fi

\begin{abstract}
We develop an algorithm which, given a trained transformer model $\mathcal{M}$ as input, as well as a string of tokens $s$ of length $n_{fix}$ and an integer $n_{free}$, can generate a mathematical proof that $\mathcal{M}$ is ``overwhelmed'' by $s$, in time and space $\widetilde{O}(n_{fix}^2 + n_{free}^3)$.
We say that $\mathcal{M}$ is ``overwhelmed'' by $s$ when the output of the model evaluated on this string plus any additional string $t$, $\mathcal{M}(s + t)$, is completely insensitive to the value of the string $t$ whenever length($t$) $\leq n_{free}$.
Along the way, we prove a particularly strong worst-case form of ``over-squashing'' \cite{alon2021bottleneckgraphneuralnetworks, barbero2024transformers}, which we use to bound the model's behavior.
Our technique uses computer-aided proofs to establish this type of operationally relevant guarantee about transformer models.
We empirically test our algorithm on a single layer transformer complete with an attention head, layer-norm, MLP/ReLU layers, and RoPE positional encoding.
We believe that this work is a stepping stone towards the difficult task of obtaining useful guarantees for trained transformer models.
\end{abstract}

%%%%%% general TODOs
% \pagebreak

%%%%%%%%%%%%%%%%%%%%%%%%%%%%%%%%%%%%%%%%%%%%%%%%%%
\section{Introduction}


\begin{figure}[t]
\centering
\includegraphics[width=0.6\columnwidth]{figures/evaluation_desiderata_V5.pdf}
\vspace{-0.5cm}
\caption{\systemName is a platform for conducting realistic evaluations of code LLMs, collecting human preferences of coding models with real users, real tasks, and in realistic environments, aimed at addressing the limitations of existing evaluations.
}
\label{fig:motivation}
\end{figure}

\begin{figure*}[t]
\centering
\includegraphics[width=\textwidth]{figures/system_design_v2.png}
\caption{We introduce \systemName, a VSCode extension to collect human preferences of code directly in a developer's IDE. \systemName enables developers to use code completions from various models. The system comprises a) the interface in the user's IDE which presents paired completions to users (left), b) a sampling strategy that picks model pairs to reduce latency (right, top), and c) a prompting scheme that allows diverse LLMs to perform code completions with high fidelity.
Users can select between the top completion (green box) using \texttt{tab} or the bottom completion (blue box) using \texttt{shift+tab}.}
\label{fig:overview}
\end{figure*}

As model capabilities improve, large language models (LLMs) are increasingly integrated into user environments and workflows.
For example, software developers code with AI in integrated developer environments (IDEs)~\citep{peng2023impact}, doctors rely on notes generated through ambient listening~\citep{oberst2024science}, and lawyers consider case evidence identified by electronic discovery systems~\citep{yang2024beyond}.
Increasing deployment of models in productivity tools demands evaluation that more closely reflects real-world circumstances~\citep{hutchinson2022evaluation, saxon2024benchmarks, kapoor2024ai}.
While newer benchmarks and live platforms incorporate human feedback to capture real-world usage, they almost exclusively focus on evaluating LLMs in chat conversations~\citep{zheng2023judging,dubois2023alpacafarm,chiang2024chatbot, kirk2024the}.
Model evaluation must move beyond chat-based interactions and into specialized user environments.



 

In this work, we focus on evaluating LLM-based coding assistants. 
Despite the popularity of these tools---millions of developers use Github Copilot~\citep{Copilot}---existing
evaluations of the coding capabilities of new models exhibit multiple limitations (Figure~\ref{fig:motivation}, bottom).
Traditional ML benchmarks evaluate LLM capabilities by measuring how well a model can complete static, interview-style coding tasks~\citep{chen2021evaluating,austin2021program,jain2024livecodebench, white2024livebench} and lack \emph{real users}. 
User studies recruit real users to evaluate the effectiveness of LLMs as coding assistants, but are often limited to simple programming tasks as opposed to \emph{real tasks}~\citep{vaithilingam2022expectation,ross2023programmer, mozannar2024realhumaneval}.
Recent efforts to collect human feedback such as Chatbot Arena~\citep{chiang2024chatbot} are still removed from a \emph{realistic environment}, resulting in users and data that deviate from typical software development processes.
We introduce \systemName to address these limitations (Figure~\ref{fig:motivation}, top), and we describe our three main contributions below.


\textbf{We deploy \systemName in-the-wild to collect human preferences on code.} 
\systemName is a Visual Studio Code extension, collecting preferences directly in a developer's IDE within their actual workflow (Figure~\ref{fig:overview}).
\systemName provides developers with code completions, akin to the type of support provided by Github Copilot~\citep{Copilot}. 
Over the past 3 months, \systemName has served over~\completions suggestions from 10 state-of-the-art LLMs, 
gathering \sampleCount~votes from \userCount~users.
To collect user preferences,
\systemName presents a novel interface that shows users paired code completions from two different LLMs, which are determined based on a sampling strategy that aims to 
mitigate latency while preserving coverage across model comparisons.
Additionally, we devise a prompting scheme that allows a diverse set of models to perform code completions with high fidelity.
See Section~\ref{sec:system} and Section~\ref{sec:deployment} for details about system design and deployment respectively.



\textbf{We construct a leaderboard of user preferences and find notable differences from existing static benchmarks and human preference leaderboards.}
In general, we observe that smaller models seem to overperform in static benchmarks compared to our leaderboard, while performance among larger models is mixed (Section~\ref{sec:leaderboard_calculation}).
We attribute these differences to the fact that \systemName is exposed to users and tasks that differ drastically from code evaluations in the past. 
Our data spans 103 programming languages and 24 natural languages as well as a variety of real-world applications and code structures, while static benchmarks tend to focus on a specific programming and natural language and task (e.g. coding competition problems).
Additionally, while all of \systemName interactions contain code contexts and the majority involve infilling tasks, a much smaller fraction of Chatbot Arena's coding tasks contain code context, with infilling tasks appearing even more rarely. 
We analyze our data in depth in Section~\ref{subsec:comparison}.



\textbf{We derive new insights into user preferences of code by analyzing \systemName's diverse and distinct data distribution.}
We compare user preferences across different stratifications of input data (e.g., common versus rare languages) and observe which affect observed preferences most (Section~\ref{sec:analysis}).
For example, while user preferences stay relatively consistent across various programming languages, they differ drastically between different task categories (e.g. frontend/backend versus algorithm design).
We also observe variations in user preference due to different features related to code structure 
(e.g., context length and completion patterns).
We open-source \systemName and release a curated subset of code contexts.
Altogether, our results highlight the necessity of model evaluation in realistic and domain-specific settings.





% !TeX root = main.tex 


\newcommand{\lnote}{\textcolor[rgb]{1,0,0}{Lydia: }\textcolor[rgb]{0,0,1}}
\newcommand{\todo}{\textcolor[rgb]{1,0,0.5}{To do: }\textcolor[rgb]{0.5,0,1}}


\newcommand{\state}{S}
\newcommand{\meas}{M}
\newcommand{\out}{\mathrm{out}}
\newcommand{\piv}{\mathrm{piv}}
\newcommand{\pivotal}{\mathrm{pivotal}}
\newcommand{\isnot}{\mathrm{not}}
\newcommand{\pred}{^\mathrm{predict}}
\newcommand{\act}{^\mathrm{act}}
\newcommand{\pre}{^\mathrm{pre}}
\newcommand{\post}{^\mathrm{post}}
\newcommand{\calM}{\mathcal{M}}

\newcommand{\game}{\mathbf{V}}
\newcommand{\strategyspace}{S}
\newcommand{\payoff}[1]{V^{#1}}
\newcommand{\eff}[1]{E^{#1}}
\newcommand{\p}{\vect{p}}
\newcommand{\simplex}[1]{\Delta^{#1}}

\newcommand{\recdec}[1]{\bar{D}(\hat{Y}_{#1})}





\newcommand{\sphereone}{\calS^1}
\newcommand{\samplen}{S^n}
\newcommand{\wA}{w}%{w_{\mathfrak{a}}}
\newcommand{\Awa}{A_{\wA}}
\newcommand{\Ytil}{\widetilde{Y}}
\newcommand{\Xtil}{\widetilde{X}}
\newcommand{\wst}{w_*}
\newcommand{\wls}{\widehat{w}_{\mathrm{LS}}}
\newcommand{\dec}{^\mathrm{dec}}
\newcommand{\sub}{^\mathrm{sub}}

\newcommand{\calP}{\mathcal{P}}
\newcommand{\totspace}{\calZ}
\newcommand{\clspace}{\calX}
\newcommand{\attspace}{\calA}

\newcommand{\Ftil}{\widetilde{\calF}}

\newcommand{\totx}{Z}
\newcommand{\classx}{X}
\newcommand{\attx}{A}
\newcommand{\calL}{\mathcal{L}}



\newcommand{\defeq}{\mathrel{\mathop:}=}
\newcommand{\vect}[1]{\ensuremath{\mathbf{#1}}}
\newcommand{\mat}[1]{\ensuremath{\mathbf{#1}}}
\newcommand{\dd}{\mathrm{d}}
\newcommand{\grad}{\nabla}
\newcommand{\hess}{\nabla^2}
\newcommand{\argmin}{\mathop{\rm argmin}}
\newcommand{\argmax}{\mathop{\rm argmax}}
\newcommand{\Ind}[1]{\mathbf{1}\{#1\}}

\newcommand{\norm}[1]{\left\|{#1}\right\|}
\newcommand{\fnorm}[1]{\|{#1}\|_{\text{F}}}
\newcommand{\spnorm}[2]{\left\| {#1} \right\|_{\text{S}({#2})}}
\newcommand{\sigmin}{\sigma_{\min}}
\newcommand{\tr}{\text{tr}}
\renewcommand{\det}{\text{det}}
\newcommand{\rank}{\text{rank}}
\newcommand{\logdet}{\text{logdet}}
\newcommand{\trans}{^{\top}}
\newcommand{\poly}{\text{poly}}
\newcommand{\polylog}{\text{polylog}}
\newcommand{\st}{\text{s.t.~}}
\newcommand{\proj}{\mathcal{P}}
\newcommand{\projII}{\mathcal{P}_{\parallel}}
\newcommand{\projT}{\mathcal{P}_{\perp}}
\newcommand{\projX}{\mathcal{P}_{\mathcal{X}^\star}}
\newcommand{\inner}[1]{\langle #1 \rangle}

\renewcommand{\Pr}{\mathbb{P}}
\newcommand{\Z}{\mathbb{Z}}
\newcommand{\N}{\mathbb{N}}
\newcommand{\R}{\mathbb{R}}
\newcommand{\E}{\mathbb{E}}
\newcommand{\F}{\mathcal{F}}
\newcommand{\var}{\mathrm{var}}
\newcommand{\cov}{\mathrm{cov}}


\newcommand{\calN}{\mathcal{N}}

\newcommand{\jccomment}{\textcolor[rgb]{1,0,0}{C: }\textcolor[rgb]{1,0,1}}
\newcommand{\fracpar}[2]{\frac{\partial #1}{\partial  #2}}

\newcommand{\A}{\mathcal{A}}
\newcommand{\B}{\mat{B}}
%\newcommand{\C}{\mat{C}}

\newcommand{\I}{\mat{I}}
\newcommand{\M}{\mat{M}}
\newcommand{\D}{\mat{D}}
%\newcommand{\U}{\mat{U}}
\newcommand{\V}{\mat{V}}
\newcommand{\W}{\mat{W}}
\newcommand{\X}{\mat{X}}
\newcommand{\Y}{\mat{Y}}
\newcommand{\mSigma}{\mat{\Sigma}}
\newcommand{\mLambda}{\mat{\Lambda}}
\newcommand{\e}{\vect{e}}
\newcommand{\g}{\vect{g}}
\renewcommand{\u}{\vect{u}}
\newcommand{\w}{\vect{w}}
\newcommand{\x}{\vect{x}}
\newcommand{\y}{\vect{y}}
\newcommand{\z}{\vect{z}}
\newcommand{\fI}{\mathfrak{I}}
\newcommand{\fS}{\mathfrak{S}}
\newcommand{\fE}{\mathfrak{E}}
\newcommand{\fF}{\mathfrak{F}}

\newcommand{\Risk}{\mathcal{R}}

\renewcommand{\L}{\mathcal{L}}
\renewcommand{\H}{\mathcal{H}}

\newcommand{\cn}{\kappa}
\newcommand{\nn}{\nonumber}


\newcommand{\Hess}{\nabla^2}
\newcommand{\tlO}{\tilde{O}}
\newcommand{\tlOmega}{\tilde{\Omega}}

\newcommand{\calF}{\mathcal{F}}
\newcommand{\fhat}{\widehat{f}}
\newcommand{\calS}{\mathcal{S}}

\newcommand{\calX}{\mathcal{X}}
\newcommand{\calY}{\mathcal{Y}}
\newcommand{\calD}{\mathcal{D}}
\newcommand{\calZ}{\mathcal{Z}}
\newcommand{\calA}{\mathcal{A}}
\newcommand{\fbayes}{f^B}
\newcommand{\func}{f^U}


\newcommand{\bayscore}{\text{calibrated Bayes score}}
\newcommand{\bayrisk}{\text{calibrated Bayes risk}}

\newtheorem{example}{Example}[section]
\newtheorem{exc}{Exercise}[section]
%\newtheorem{rem}{Remark}[section]

\newtheorem{theorem}{Theorem}[section]
\newtheorem{definition}{Definition}
\newtheorem{proposition}[theorem]{Proposition}
\newtheorem{corollary}[theorem]{Corollary}

\newtheorem{remark}{Remark}[section]
\newtheorem{lemma}[theorem]{Lemma}
\newtheorem{claim}[theorem]{Claim}
\newtheorem{fact}[theorem]{Fact}
\newtheorem{assumption}{Assumption}

\newcommand{\iidsim}{\overset{\mathrm{i.i.d.}}{\sim}}
\newcommand{\unifsim}{\overset{\mathrm{unif}}{\sim}}
\newcommand{\sign}{\mathrm{sign}}
\newcommand{\wbar}{\overline{w}}
\newcommand{\what}{\widehat{w}}
\newcommand{\KL}{\mathrm{KL}}
\newcommand{\Bern}{\mathrm{Bernoulli}}
\newcommand{\ihat}{\widehat{i}}
\newcommand{\Dwst}{\calD^{w_*}}
\newcommand{\fls}{\widehat{f}_{n}}


\newcommand{\brpi}{\pi^{br}}
\newcommand{\brtheta}{\theta^{br}}

% \newcommand{\M}{\mat{M}}
% \newcommand\Mmh{\mat{M}^{-1/2}}
% \newcommand{\A}{\mat{A}}
% \newcommand{\B}{\mat{B}}
% \newcommand{\C}{\mat{C}}
% \newcommand{\Et}[1][t]{\mat{E_{#1}}}
% \newcommand{\Etp}{\Et[t+1]}
% \newcommand{\Errt}[1][t]{\mat{\bigtriangleup_{#1}}}
% \newcommand\cnM{\kappa}
% \newcommand{\cn}[1]{\kappa\left(#1\right)}
% \newcommand\X{\mat{X}}
% \newcommand\fstar{f_*}
% \newcommand\Xt[1][t]{\mat{X_{#1}}}
% \newcommand\ut[1][t]{{u_{#1}}}
% \newcommand\Xtinv{\inv{\Xt}}
% \newcommand\Xtp{\mat{X_{t+1}}}
% \newcommand\Xtpinv{\inv{\left(\mat{X_{t+1}}\right)}}
% \newcommand\U{\mat{U}}
% \newcommand\UTr{\trans{\mat{U}}}
% \newcommand{\Ut}[1][t]{\mat{U_{#1}}}
% \newcommand{\Utinv}{\inv{\Ut}}
% \newcommand{\UtTr}[1][t]{\trans{\mat{U_{#1}}}}
% \newcommand\Utp{\mat{U_{t+1}}}
% \newcommand\UtpTr{\trans{\mat{U}_{t+1}}}
% \newcommand\Utptild{\mat{\widetilde{U}_{t+1}}}
% \newcommand\Us{\mat{U^*}}
% \newcommand\UsTr{\trans{\mat{U^*}}}
% \newcommand{\Sigs}{\mat{\Sigma}}
% \newcommand{\Sigsmh}{\Sigs^{-1/2}}
% \newcommand{\eye}{\mat{I}}
% \newcommand{\twonormbound}{\left(4+\DPhi{\M}{\Xt[0]}\right)\twonorm{\M}}
% \newcommand{\lamj}{\lambda_j}

% \renewcommand\u{\vect{u}}
% \newcommand\uTr{\trans{\vect{u}}}
% \renewcommand\v{\vect{v}}
% \newcommand\vTr{\trans{\vect{v}}}
% \newcommand\w{\vect{w}}
% \newcommand\wTr{\trans{\vect{w}}}
% \newcommand\wperp{\vect{w}_{\perp}}
% \newcommand\wperpTr{\trans{\vect{w}_{\perp}}}
% \newcommand\wj{\vect{w_j}}
% \newcommand\vj{\vect{v_j}}
% \newcommand\wjTr{\trans{\vect{w_j}}}
% \newcommand\vjTr{\trans{\vect{v_j}}}

% \newcommand{\DPhi}[2]{\ensuremath{D_{\Phi}\left(#1,#2\right)}}
% \newcommand\matmult{{\omega}}

\section{Model}
\label{sec:model}
Let $[N] = \{1, 2, \dots, N \}$ be a set of $N$ agents.
We examine an environment in which a system interacts with the agents over $T$ rounds.
Every round $t\leq T$ comprises $N$ \emph{sessions}, each session represents an encounter of the system with exactly one agent, and each agent interacts exactly once with the system every round.
I.e., in each round $t$ the agents arrive sequentially. 


\paragraph{Arrival order} The \emph{arrival order} of round $t$, denoted as $\ordv_t=(\ord_t(1),\dots, \ord_t(N))$, is an element from set of all permutations of $[N]$. Each entry $q$ in $\ordv_t$ is the index of the agent that arrives in the $q^{\text{th}}$ session of round $t$.
For example, if $\ord_t(1) = 2$, then agent $2$ arrives in the first session of round $t$.
Correspondingly, $\ord_t^{-1}(i)=q$ implies that agent $i$ arrives in the $q^{\text{th}}$ session of round $t$. 

As we demonstrate later, the arrival order has an immediate impact on agent rewards. We call the mechanism by which the arrival order is set \emph{arrival function} and denote it by $\ordname$. Throughout the paper, we consider several arrival functions such as the \emph{uniform arrival} function, denoted by $\uniord$, and the \emph{nudged arrival} $\sugord$; we introduce those formally in Sections~\ref{sec:uniform} and~\ref{sec:nudge}, respectively.

%We elaborate more on this concept in Section~\ref{sec: arrival}.


\paragraph{Arms} We consider a set of $K \geq 2$ arms, $A = \{a_1, \ldots, a_K\}$. The reward of arm $a_i$ in round $t$ is a random variable $X_i^t \sim \mathcal{D}^t_i$, where the rewards $(X_i^t)_{i,t}$ are mutually independent and bounded within the interval $[0,1]$. The reward distribution $\mathcal{D}^t_i$ of arm $a_i$, $i\in [K]$ at round $t\in T$ is assumed to be non-stationary but independent across arms and rounds. We denote the realized reward of arm $a_i$ in round $t$ by $x_i^t$. We assume \emph{reward consistency}, meaning that rewards may vary between rounds but remain constant within the sessions of a single round. Specifically, if an arm $a_i$ is selected multiple times during round~$t$, each selection yields the same reward $x_i^t$, where the superscript $t$ indicates its dependence on the round rather than the session. This consistency enables the system to leverage information obtained from earlier sessions to make more informed decisions in later sessions within the same round. We provide further details on this principle in Subsection~\ref{subsec:information}.


\paragraph{Algorithms} An algorithm is a mapping from histories to actions. We typically expect algorithms to maximize some aggregated agent metric like social welfare. Let $\mathcal H^{t,q}$ denote the information observed during all sessions of rounds $1$ to $t-1$ and sessions $1$ to $q-1$ in round $t$.  The history $\mathcal H^{t,q}$ is an element from $(A \times [0,1])^{(t-1) \cdot N +q-1}$, consisting of pairs of the form (pulled arm, realized reward). Notice that we restrict our attention to \emph{anonymous} algorithms, i.e., algorithms that do not distinguish between agents based on their identities. Instead, they only respond to the history of arms pulled and rewards observed, without conditioning on which specific agent performed each action.
%In the most general case, algorithms make decisions at session $q$ of round $t$  based on the entire history $\mathcal H^{t,q}$ and the index of the arriving agent $\ord_t(q)$. %Furthermore, we sometimes assume that algorithms have Bayesian information, i.e., algorithms are aware of the distributions $(\mathcal D_i)^K_{i=1}$. 
Furthermore, we sometimes assume that algorithms have Bayesian information, meaning they are aware of the reward distributions $(\mathcal{D}^t_i)_{i,t}$. If such an assumption is required to derive a result, we make it explicit. %Otherwise, we do not assume any additional knowledge about the algorithm’s information. %This distinction allows us to analyze both general algorithms without prior distributional knowledge and specialized algorithms that leverage Bayesian information.


\paragraph{Rewards} Let $\rt{i}$ denote the reward received by agent $i \in [N]$ at round $t$, and let $\Rt{i}$ denote her cumulative reward at the end of round $t$, i.e., $\Rt{i} = \sum_{\tau=1}^{t}{r^{\tau}_{i}}$. We further denote the \emph{social welfare} as the sum of the rewards all agents receive after $T$ rounds. Formally, $\sw=\sum^{N}_{i=1}{R^T_i}$. We emphasize that social welfare is independent of the arrival order. 


\paragraph{Envy}
We denote by $\adift{i}{j}$ the reward discrepancy of agents $i$ and $j$ in round $t$; namely, $\adift{i}{j}= \rt{i} - \rt{j}$. %We call this term \omer{name??} reward discrepancy in round $t$. 
The (cumulative) \emph{envy} between two agents at round $t$ is the difference in their cumulative rewards. Formally, $\env_{i,j}^t= \Rt{i} - \Rt{j}$ is the envy after $t$ rounds between agent $i$ and $j$. We can also formulate envy as the sum of reward discrepancies, $\env_{i,j}^t= \sum^{t}_{\tau=1}{\adif{i}{j}^\tau}$. Notice that envy is a signed quantity and can be either positive or negative. Specifically, if $\env_{i,j}^t < 0$, we say that agent $i$ envies agent $j$, and if $\env_{i,j}^t > 0$, agent $j$ envies agent $i$. The main goal of this paper is to investigate the behavior of the \emph{maximal envy}, defined as
\[
\env^t = \max_{i,j \in [N]} \env^t_{i,j}.
\]
For clarity, the term \emph{envy} will refer to the maximal envy.\footnote{ We address alternative definitions of envy in Section~\ref{sec:discussion}.} % Envy can also be defined in alternative ways, such as by averaging pairwise envy across all agents. We address average envy in Section~\ref{sec:avg_envy}.}
Note that $\env_{i,j}^t$ are random variables that depend on the decision-making algorithm, realized rewards, and the arrival order, and therefore, so is $\env^t$. If a result we obtain regarding envy depends on the arrival order $\ordname$, we write $\env^t(\ordname)$. Similarly, to ease notation, if $\ordname$ can be understood from the context, it is omitted.



\paragraph{Further Notation} We use the subscript $(q)$ to address elements of the $q^{\text{th}}$ session, for $q\in [N]$.
That is, we use the notation $\rt{(q)}$ to denote the reward granted to the agent that arrives in the $q^{\text{th}}$ session of round $t$ and $\Rt{(q)}$ to denote her cumulative reward. %Additionally, we introduce the notation $\at{(q)}$ to denote the arm pulled in that session.
Correspondingly, $\sdift{q}{w} = \rt{(q)} - \rt{(w)}$ is the reward discrepancy of the agents arriving in the $q^{\text{th}}$ and $w^{\text{th}}$ sessions of round $t$, respectively. 
To distinguish agents, arms, sessions and rounds, we use the letters $i,j$ to mark agents and arms, $q,w$ for sessions, and $t,\tau$ for rounds.


\subsection{Example}
\label{sec: example}
To illustrate the proposed setting and notation, we present the following example, which serves as a running example throughout the paper.

\begin{table}[t]
\centering
\begin{tabular}{|c|c|c|c|}
\hline
$t$ (round) & $\ordv_t$ (arrival order) & $x_1^t$ & $x_2^t$ \\ \hline
1           & 2, 1                     & 0.6     & 0.92    \\ \hline
2           & 1, 2                     & 0.48    & 0.1     \\ \hline
3           & 2, 1                     & 0.15    & 0.8     \\ \hline
\end{tabular}
\caption{
    Data for Example~\ref{example 1}.
}
\label{tbl: example}
\end{table}

\begin{algorithm}[t]
\caption{Algorithm for Example~\ref{example 1}}
\label{alguni}
\DontPrintSemicolon 
\For{round $t = 1$ to $T$}{
    pull $a_{1}$ in the first session\label{alguniexample: first}\\
    \lIf{$x^t_1 \geq \frac{1}{2}$}{pull $a_{1}$ again in second session \label{alguniexample: pulling a again}}
    \lElse{pull $a_{2}$ in second session \label{alguniexample: sopt else}}
}
\end{algorithm}


\begin{example}\label{example 1}
We consider $K=2$ uniform arms, $X_1,X_2 \sim \uni{0,1}$, and $N=2$ for some $T\geq 3$. We shall assume arm decision are made by Algorithm~\ref{alguni}: In the first session, the algorithm pulls $a_1$; if it yields a reward greater than $\nicefrac{1}{2}$, the algorithm pulls it again in the second session (the ``if'' clause). Otherwise, it pulls $a_2$.



We further assume that the arrival orders and rewards are as specified in Table~\ref{tbl: example}. Specifically, agent 2 arrives in the first session of round $t=1$, and pulling arm $a_2$ in this round would yield a reward of $x^1_2 = 0.92$. Importantly, \emph{this information is not available to the decision-making algorithm in advance} and is only revealed when or if the corresponding arms are pulled.




In the first round, $\boldsymbol{\eta}^1 = \left(2,1\right)$; thus, agent 2 arrives in the first session.
The algorithm pulls arm $a_1$, which means, $a^1_{(1)} = a_1$, and the agent receives $r_{2}^1=r_{(1)}^1=x_1^1=0.6$.
Later that round, in the second session, agent 1 arrives, and the algorithm pulls the same arm again since $x^1_1 = 0.6 \geq \nicefrac{1}{2}$ due to the ``if'' clause.
I.e., $a^1_{(2)} = a_1$ and $r_{1}^1 = r_{(2)}^1 = x_1^1 = 0.6$.
Even though the realized reward of arm $a_2$ in that round is higher ($0.92$), the algorithm is not aware of that value.
At the end of the first round, $R^1_1 = R^1_{(2)} = R^1_2 = R^1_{(1)} = 0.6$. The reward discrepancy is thus $\adif{1}{2}^1 = \adif{2}{1}^1= \sdif{2}{1}^1 = 0.6 - 0.6 =0$. 



In the second round, agent 1 arrives first, followed by agent 2.
Firstly, the algorithm pulls arm $a_1$ and agent 1 receives a reward of $r_{1}^2 = r_{(1)}^2 = x_1^2 = 0.48$.
Because the reward is lower than $\nicefrac{1}{2}$, in the second session the algorithm pulls the other arm ($a^2_{(2)} = a_2$), granting agent 2 a reward of $r_{2}^2 = r_{(2)}^2 = x_2^2 = 0.1$.
At the end of the second round, $R^2_1 = R^2_{(1)} = 0.6 + 0.48 = 1.08$ and $R^2_2 = R^2_{(2)} = 0.6 + 0.1 = 0.7$. Furthermore, $\sdif{2}{1}^2 = \adif{2}{1}^2 = r^2_{2} - r^2_{1} = 0.1 - 0.48 = -0.38$.

In the third and final round, agent 2 arrives first again, and receives a reward  of $0.15$ from $a_1$. When agent 1 arrives in the second session, the algorithm pulls arm $a_2$, and she receives a reward of $0.8$. As for the reward discrepancy, $\sdif{2}{1}^3 = \adif{2}{1}^3 = r^3_{2} - r^3_{1} = 0.15 - 0.8 = -0.75$. 

Finally, agent 1 has a cumulative reward of $R^3_1 = R^3_{(2)} = 0.6 + 0.48 + 0.8 = 1.88$, whereas agent~2 has a cumulative reward of $R^3_2 = R^3_{(1)} = 0.6 + 0.1 + 0.15 = 0.85$. In terms of envy, $\env^1_{1,2}= \adif{1}{2}^1 =0$, $\env^2_{1,2}=\adif{1}{2}^1+\adif{1}{2}^2= 0.38$, and $\env^3_{1,2} = -\env^3_{2,1} = R^3_1-R^3_2 = 1.88-0.85 = 1.03$, and consequently the envy in round 3 is $\env^3 = 1.03$.
\end{example}


\subsection{Information Exploitation}
\label{subsec:information}

In this subsection, we explain how algorithms can exploit intra-round information.
Since rewards are consistent in the sessions of each round, acquiring information in each session can be used to increase the reward of the following sessions.
In other words, the earlier sessions can be used for exploration, and we generally expect agents arriving in later sessions to receive higher rewards.
Taken to the extreme, an agent that arrives after all arms have been pulled could potentially obtain the highest reward of that round, depending on how the algorithm operates.

To further demonstrate the advantage of late arrival, we reconsider Example~\ref{example 1} and Algorithm~\ref{alguni}. 
The expected reward for the agent in the first session of round $t$ is $\E{\rt{(1)}}=\mu_1=\frac{1}{2}$, yet the expected reward of the agent in the second session is
\begin{align*}
\E{\rt{(2)}}=\E{\rt{(2)}\mid X^t_1 \geq \frac{1}{2} }\prb{X^t_1 \geq \frac{1}{2}} + \E{\rt{(2)}\mid X^t_1 < \frac{1}{2} }\prb{X^t_1 < \frac{1}{2}};
\end{align*}
thus, $\E{\rt{(2)}} =\E{X^t_1\mid X^t_1 \geq \frac{1}{2} }\cdot \frac{1}{2} + \mu_2\cdot\frac{1}{2} = \frac{5}{8}$.
Consequently, the expected welfare per round is $\E{\rt{(1)}+\rt{(2)}}=1+\frac{1}{8}$, and the benefit of arriving in the second session of any round $t$ is $\E{\rt{(2)} - \rt{(1)}} = \frac{1}{8}$. This gap creates envy over time, which we aim to measure and understand.
%This discrepancy generates envy over time, and our paper aims to better understand it.
\subsection{Socially Optimal Algorithms}
\label{sec: sw}
Since our model is novel, particularly in its focus on the reward consistency element, studying social welfare maximizing algorithms represents an important extension of our work. While the primary focus of this paper is to analyze envy under minimal assumptions about algorithmic operations, we also make progress in the direction of social welfare optimization. See more details in Section~\ref{sec:discussion}.%Due to space limitations, we defer the discussion on socially optimal algorithms to  \ifnum\Includeappendix=0{the appendix}\else{Section~\ref{appendix:sociallyopt}}\fi.




% Since our model is novel and specifically the reward consistency element, it might be interesting to study social welfare optimization. While the main focus of our paper is to study envy under minimal assumptions on how the algorithm operates, we take steps toward this direction as well. Due to space limitations, we defer the discussion on socially optimal algorithms to  \ifnum\Includeappendix=0{the appendix}\else{Section~\ref{appendix:sociallyopt}}\fi.  We devise a socially optimal algorithm for the two-agent case, offer efficient and optimal algorithms for special cases of $N>2$ agents, and an inefficient and approximately optimal algorithm for any instance with $N>2$. Moreover, we address the welfare-envy tradeoff in Section~\ref{sec:extensions}.


% Social welfare, unlike envy, is entirely independent of the arrival order. While the main focus of our paper is to study envy under minimal assumptions on how the algorithm operates, socially optimal algorithms might also be of interest. Due to space limitations, we defer the discussion on socially optimal algorithms to  \ifnum\Includeappendix=0{the appendix}\else{Section~\ref{appendix:sociallyopt}}\fi. We devise a socially optimal algorithm for the two-agent case, offer efficient and optimal algorithms for special cases of $N>2$ agents, and an inefficient and approximately optimal algorithm for any instance with $N>2$. %Furthermore, we treat the welfare-envy tradeoff of the special case of Example~\ref{example 1}.



\begin{figure*}[ht]
    \centering
    \includegraphics[width=\textwidth, trim=79 280 93 123, clip]{figures/framework_img.pdf}
    \caption{The pipeline of the \ENDow{} framework 
    %where each component is specified in a given configuration. 
    which yields a downstream task score and a WER score of the transcript set input to the task. The pipeline is executed for several severeties of noising and types of cleaning techniques. %Acoustic noising is applied at $k$ intensities, providing $k+1$ audio versions (including the non-noised version), eventually producing $k+2$ transcript versions (including the source transcript). Applying transcript cleaning reveals the effect of \textit{types} of noise. 
    Resulting scores are plotted on a graph for the analyses, as in, e.g., \autoref{fig_cleaning_graphs}.}
    %The pipeline is executed on $k+1$ intensities of acoustic noising (including the non-noised version), producing $k+2$ scores for the downstream task (including execution on the source transcripts). This process eventually describes the effect of the \textit{intensity} of transcript noise on the downstream task. The process is repeated for $m$ cleaning techniques ($m+1$ when including no cleaning), to analyze the benefit of a cleaning approach and the effect of the \textit{types} of transcript noise.}
    \label{fig_framework}
\end{figure*}
\newcommand{\ModelFinal}{\Model^{final}}
\newcommand{\peakToPeak}{\mathrm{PTP}}
\newcommand{\RepIndxs}{\mathcal{R}}
\newcommand{\FreeIndxs}{\mathcal{F}}
\newcommand{\softSumMaxF}{\alpha_{free}^{\max}}
\newcommand{\softSumMinR}{\alpha_{fix}^{\min}}
\newcommand{\softSumMaxR}{\alpha_{fix}^{\max}}
\newcommand{\softSumMinF}{\alpha_{free}^{\min}}

\newcommand{\ssMaxFA}{{\alpha}_{free}^{\max}}
\newcommand{\ssMinRA}{{\alpha}_{fix}^{\min}}
\newcommand{\ssMaxRA}{{\alpha}_{fix}^{\max}}
\newcommand{\ssMinFA}{{\alpha}_{free}^{\min}}


\newcommand{\ssMaxFB}{{\beta}_{free}^{\max}}
\newcommand{\ssMinRB}{{\beta}_{fix}^{\min}}
\newcommand{\ssMaxRB}{{\beta}_{fix}^{\max}}
\newcommand{\ssMinFB}{{\beta}_{free}^{\min}}

\newcommand{\logRMin}{L_{fix}^{\min}}
\newcommand{\logRMax}{L_{fix}^{\max}}
\newcommand{\logFMin}{L_{free}^{\min}}
\newcommand{\logFMax}{L_{free}^{\max}}
\newcommand{\Samp}{\textsf{Samp}}
\newcommand{\varAt}{{\Var_j^{(k)}}}
\newcommand{\varAtMin}{{\Var_{j, \min}^{(k)}}}
\newcommand{\varAtMax}{{\Var_{j, \max}^{(k)}}}
\newcommand{\preSMMax}{\ell_{i, \max}^{(k)}}
\newcommand{\preSMMin}{\ell_{i, \min}^{(k)}}

\section{Input Restrictions and Proving Overwhelming}
\label{sec:meta_framework}

In this section, we will provide an algorithm to decide ``overwhelming."
Along the way, we develop a generalizable method of upper-bounding the worst-case deviation of a single layer of a transformer model under input restriction.

\subsubsection*{Input Restrictions and Designed Space}
We use the notation from \textit{Analysis of Boolean Functions} to denote restrictions on inputs to a function \cite{o2014analysis}.
The restriction will fix certain tokens to be a specific value and leave the rest of the tokens free.

\begin{definition}[Input Restriction, \cite{o2014analysis} definition 3.18]
	\label{def:input_restriction}
	Let $f : \calX^n \rightarrow \calY$ be some function and $J \subset [n]$ and $\notJ = [n] \setminus J$.
	Let $z \in \calX^\notJ$.
	Then, we write $f_{J \mid z} : \calX^J \rightarrow \calY$ (``the restriction of $f$ to $J$ given $z$'') as the subfunction of $f$ that is obtained by fixing the coordinates of $\notJ$ to the values in $z$.
	Given $y \in \calX^J$ and $z \in \calX^\notJ$, we write $f_{J \mid z}(y)$ as $f(y, z)$ even though $y$ and $z$ are not literally concatenated.
\end{definition}

Throughout this paper, we will consider a specific input restriction where the first $\nfix$ tokens are restricted to a string $\desSet$ and the last token is fixed to token $\query$.
We will denote said restriction as $\desF{f}$ where $(\nfix, \nctx)$ denotes the set $\{ \nfix + 1, \ldots, \nctx - 1\}$.

Next, it will be useful to define the set of all possible inputs under an input restriction.

\begin{definition}[Designed Space]
	\label{def:InpSpace}
	Recall, from \cref{def:one_hot_space}, that $\OneHotSpace$ is the set of all one-hot vectors of size $\dVocab$.
	We denote by $\OneHotSpace^n$ the set of matrices where each row is a one-hot vector of size $\dVocab$.
	Then, let $\InpSpace \subset \OneHotSpace^\nctx$ designate the set of all possible inputs under a \textbf{specific} input restriction.
	That is, 
	\[
		\InpSpace = \left\{ X \in \R^{\nctx \times \dVocab} \mid X = 
		\begin{bmatrix}
			\vece_{\desSet_1} \\
			\vece_{\desSet_2} \\
			\vdots \\
			\vece_{\desSet_s} \\
			Y \\
			\vecQuery
		\end{bmatrix}, Y \in \freeSpace
		\right\}.
	\]
	where $\freeSpace$ is the space of free tokens.
\end{definition}
\subsection{Algorithm for deciding Overwhelming}
Here we consider zero temperature sampling setting where we can define our model with sampling as
\[
	\ModelFinal(X) = \arg\max_{i \in [\dVocab]} \Model(X) \cdot \vec{e}_i^T
\]
where $\Model$ is defined in \cref{def:one_layer_transformer}.
The model simply selects the token with the highest logit weight rather than sampling from the output distribution. 
% \footnote{This is equivalent to setting the sampling temperature to zero.}

We define the ``peak-to-peak difference'' to be the difference between the logit for the most likely token and the logit for the second most likely token for sample $X$.

\begin{definition}[Peak-to-peak difference]
    Let $X \in \InpSpace$ be any element from the restriction and $k = \ModelFinal(X)$.
    Then, we let 
    \[
    	\peakToPeak(\Model, X) = \min_{j \in [\dVocab], j \neq k} \Model(X) \cdot \left(\vec{e}_{k}^T - \vec{e}_j^T\right).
    \]
\end{definition}

\iffalse
\begin{algorithm}[H] \label{alg:overwhelmCheck}
	\SetKwInOut{Input}{Input}\SetKwInOut{Output}{Output}
	\Input{A fixed single-layer transformer model $\Model$, a string of tokens $s$ denoting a fixed part of the input to $\Model$, integer $\nfree$ denoting the number of free tokens in the input to $\Model$, and a final query token $q$
	}
	\Output{Either the string ``Overwhelmed'' indicating that the output of the model $\Model$ evaluated on $s$ concatenated with $\nfree$ free tokens and query token $q$ is proven to be invariant under the choice of the free tokens, OR the string ``Inconclusive'' if no such proof is obtained.} 

	Calculate a bound $W$ such that $\WD(\vec{e}_\nctx \cdot \desF{\Model}, X) \leq W$ \\
	\If{
		$
		W < \peakToPeak(\desF{\Model}, \InpSpace) / 2
	$ }{\Return ``Overwhelmed''  }

	\Else{\Return ``Inconclusive''}
	\caption{Overwhelmed Verifier - Algorithm Sketch }
\end{algorithm} 
\fi

To prove a model is overwhelmed by a fixed input we bound the worst-case deviation by the peak-to-peak difference.
The following theorem summarizes the bound our algorithm is tasked with verifying. 
\begin{theorem} \label{thm:metathm}
        If
        \[
		\WD(\Model; \InpSpace)_\infty < \peakToPeak(\Model, X) / 2
	\]
        for some $X \in \InpSpace$,
	then the output of $\ModelFinal$ is ``overwhelmed'' under the restriction.
\end{theorem}
\begin{proof}
	As $\ModelFinal$ always selects the token with the maximum logit value, if the coordinate-wise deviation of the restricted model's output never differs by more than $\peakToPeak(\desF{\Model}, X) / 2$ for any $X \in \InpSpace$, the token with the second highest logit value will never exceed that of the token with the highest value.
\end{proof}

Ultimately, we will want to make use of the above theorem alongside \cref{alg:overwhelmCheckDet} to prove the following theorem:
\begin{theorem}[Input Restriction] \label{thm:InpRes}
	%The model $\desF{\ModelFinal}$ over domain $[X]$ if 
	If
    \begin{align*}
	    &\WD(\desF{\Model}; \InpSpace)_\infty
        \\&< \peakToPeak(\desF{\Model}, X) / 2,
    \end{align*}
	then the output of model $\desF{\Model}$ is fixed for all inputs in $\InpSpace$.
	Moreover, we can use \cref{alg:overwhelmCheckDet} to produce an upper bound $W$ for $\WD(\desF{\Model}; \InpSpace)_\infty$.
\end{theorem}

To prove the above theorem, we will:
\begin{itemize}[nosep]
    \item break down the model into its components.
    \item bound the worst-case deviation of each component as a function of the blowup and shift from layer-normalization.
    \item use the triangle inequality of worst-case deviation to bound the worst-case deviation of the model prior to unembedding.
    \item use the Lipschitz constant of the unembedding matrix to bound the worst-case deviation of the model.
\end{itemize}
The formal proof can also be found in \cref{subsec:InpResProof}.

\begin{algorithm}[tb] 
	\caption{Algorithm for deciding Overwhelming}
	\label{alg:overwhelmCheckDet}
	\begin{algorithmic}
		\STATE {\bfseries Input:}  Model $\Model$, fixed string $s$, contenxt length $\nctx$, query token $q$.
		\STATE {\bfseries Output:} ``Overwhelmed'' or ``Inconclusive''.
		\STATE
		\STATE Calculate $B^{\min}, B^{\max}, S^{\min}, S^{\max}$ as in \cref{def:blowup_shift}.
            \STATE Calculate pre-softmax extremal logit values $\ell^{\min}$ and $\ell^{\max}$ via \cref{alg:lminlmax}.
		\STATE Using the above, calculate $\ssMaxRB$ and $\ssMinRB$ as in \cref{def:soft-extrem-values} and \cref{lem:min_max_softmax}.
		\STATE Calculate upper-bound $W^{\attn}$ as in \cref{lem:att_bound}.
		\STATE Calculate upper-bound $W^{\fenc}$ as in \cref{lem:WD_fenc}.
		\STATE Calculate \begin{align*}
			&W = \Lip(\Unembed) \\
                &\cdot \left(W^{\attn} + \LipMLP_\infty \cdot W^{\fenc} + W^{\fenc}\right)
		\end{align*}
		as per \cref{lem:mlpbound} and 
		\STATE Sample $X \gets \InpSpace$
		\IF{$W < \peakToPeak(\desF{\Model}, X) / 2$ }
		\STATE Return ``Overwhelmed''
		\ELSE
		\STATE Return ``Inconclusive''
		\ENDIF
	\end{algorithmic}
\end{algorithm}


%
%\begin{algorithm}[H] \label{alg:overwhelmCheckDet}
%	\SetKwInOut{Input}{Input}\SetKwInOut{Output}{Output}
%	\Input{A fixed single-layer transformer model $\Model$, a string of tokens $s$ denoting a fixed part of the input to $\Model$, integer $\nctx$ denoting the number of total tokens in the input to $\Model$, and a final query token $q$.}
%	\Output{Either the string ``Overwhelmed'' indicating that the output of the model $\Model$ evaluated on $s$ concatenated with $\nfree$ free tokens and query token $q$ is proven to be invariant under the choice of the free tokens, OR the string ``Inconclusive'' if no such proof is obtained.} 
%	Calculate $B^{\min}, B^{\max}, S^{\min}, S^{\max}$ as in \cref{def:blowup_shift}.
%	\\
%	Calculate $\ssMaxRB$ and $\ssMinRB$ as in \cref{def:soft-extrem-values} and \cref{lem:min_max_softmax}.
%	\\
%	Calculate \begin{align*}
%		W^{\attn} &= 2 \cdot (\ssMaxRB - \ssMinRB) \cdot \left(\max_{B \in [B^{\min}, B^{\max}]} \|E[\desSet \cup \{q\}, j] \cdot \diag(B)\|  + \max(\norm{S^{\min}}_\infty, \norm{S^{\max}}_\infty \right) \\
%			  &+ 2 \cdot \ssMaxFB \cdot \max_{B, S} \norm{(E B + S) \cdot V}_{\frobInf}
%	\end{align*}
%	as in \cref{lem:att_bound}.
%	\\
%	Calculate  $W^{\fenc} = \max_{t \in [\dVocab]} 
%	\max_{j \in [\dEmb]}
%	\max_{B \in [\vec{0}, B^{\max} - B^{\min}]}
%	\left|X[t] \cdot \Embed \cdot \diag(B) \cdot \vec{e}_j\right| + S_j^{\max} - S_j^{\min}$ as per \cref{lem:worst_case_deviation}.
%	\\
%	Calculate $W = \Lip(\Unembed) \cdot \left(W^{\attn} + \LipMLP_\infty \cdot W^{\fenc} + W^{\fenc}\right)$ 
%	as per \cref{lem:mlpbound} and 
%	\\
%	Calculate $\peakToPeak(\desF{\Model}, \InpSpace)$ by evaluating $\desF{\Model}$ on $X \gets \InpSpace$ \\
%	\If{
%	$W < \peakToPeak(\desF{\Model}, \InpSpace) / 2$ }{\Return ``Overwhelmed''
%}
%\Else{\Return ``Inconclusive''}
%
%\caption{Algorithm for deciding Overwhelming}
%\end{algorithm} 



\subsection{Proof Overview of \cref{thm:InpRes} (Algorithm Correctness)} \label{sec:algoCorrectness}
The proofs for all the lemmas and statements in this section are deferred to \cref{sec:proofs_framework}.

\subsubsection*{Breaking Down the Model}
Discussing normalization as a function of expectation and variance is a bit cumbersome.
So, we will first define an analagous ``blowup'' and ``shift'' for normalization.

\begin{definition}[Blowup and Shift Sets]
	\label{def:blowup_shift}
	For some $X \in \InpSpace$, Let $B_j(X)$ (for blowup) denote
	\[
		\frac{1}{\sqrt{\Var[X \Embed \cdot e_j^T] + \eps}} \cdot \gamma
	\]
	and let $S_j(X)$ (for shift) denote
	\[
		\frac{-\E[X \Embed \cdot e_j^T]}{\sqrt{\Var[X \Embed \cdot e_j^T] + \eps}} \cdot \gamma + \beta.
	\]
	We define $\blowupSet_j = \{B_j(X) | X \in  \InpSpace \}$, and  $\shiftSet_j  = \{S_j(X) | X \in  \InpSpace \}$.
	Moreover, for ease of notation, we will define
	\[
		\blowupSet = \{ (B_1(X), \dots, B_\dEmb(X) \mid X \in \InpSpace\}
	\]
	\[
		\shiftSet = \{ (S_1(X), \dots, S_\dEmb(X)) \mid X \in \InpSpace \}
	\]
	and the set $\blowupSet\shiftSet$ to equal
	\[
		\left\{ \bigg((B_1(X), \dots, B_\dEmb(X)), (S_1(X), \dots, S_\dEmb(X))\bigg)  \right\}
	\]
	where $X \in \InpSpace$.
	Finally, let $B^{\min}_j = \min(\blowupSet_j) $, $B^{\max}_j = \max(\blowupSet_j)$, $S^{\min}_j = \min(\shiftSet_j)$ and $S^{\max}_j = \max(\shiftSet_j)$.
\end{definition}

We will now break down the model in a way such that it is easier to analyze.
For component $\component \in \{\attn, \MLP, \Iden\}$, we will use $f^{\component} : \OneHotSpace^\nctx \times \blowupSet\shiftSet \to \R^{\nctx \times \dVocab}$ to denote the following function:
\[
	f^{\component}(X, (B, S)) = \component \circ \fenc(X, (B, S)) 
\]
where
\[
	\fenc(X, (B, S)) = (X \cdot \Embed \cdot \diag(B) + S).
\]
Note that $\fenc(X, (B(X), S(X)))$ is exactly equal to $\LayerNorm(X \cdot \Embed)$ for layernorm function $\LayerNorm$.

We can rewrite our model as
\[
    \Model = \Unembed \circ (f^{\attn} + f^{\MLP} + f^{\Iden}) 
\]
and so, by the monotonicity of lifting for worst-case deviation,
\begin{align*}    
    &\WD(\desF{\Model}; \InpSpace) \\
    \leq& \sum_{\component} \WD(\Unembed \cdot \vec{e}_\nctx \cdot \desF{f^{\component}}; \InpSpace \times \blowupSet \shiftSet)
\end{align*}
for $\component \in \{\attnH, \MLP, \Iden\}$.


% To bound the output behavior of a single-layer transformer model, we will need to bound the components of the model to get a valid upper bound $W$ on worst-case deviation where  $W$ is the calculated upper-bound in \cref{alg:overwhelmCheckDet}.

% We can then use Lipschitz constant for the unembedding matrix to get
% \begin{align*}
% 	&\WD(\desF{\Model}; \InpSpace)
%     \\ &\leq \sum_{\component} \WD(\Unembed \cdot \vec{e}_\nctx \cdot \desF{f^{\component}} ; \InpSpace \times \blowupSet \shiftSet)_\infty \\
% 			  &\leq
% 			  \Lip(\Unembed)_\infty \sum_{\component} \WD(\vec{e}_\nctx \cdot \desF{f^{\component}} ; \InpSpace \times \blowupSet \shiftSet)_\infty \\
% 			  &\leq \frac{1}{2} \cdot \peakToPeak(\desF{\Model}, X).
% \end{align*}
% for a large enough $\ndes$.

\subsubsection{Bounding Blowup and Shift}
We can get rather straightforward bounds on the blowup and shift of the model.
\begin{lemma}[Blowup and Shift Bounds]
	\label{lem:boundsBS}
	We bound blowup and shift: for every $B_j \in \blowupSet_j$ and $S_j \in \shiftSet_j$
	\[
		\frac{\gamma}{\sqrt{\Var_j^{\max} + \eps}} \leq B_j \leq \frac{\gamma}{\sqrt{\Var_j^{\min} + \eps}}
	\]
	and
	\begin{align*}
	&\min\left(B_j^{\min} \mu_j^{\min}, B_j^{\min} \mu_j^{\max}, B_j^{\max}, \mu_j^{\min}, B_j^{\max} \mu_j^{\max}\right)
	\leq
	S_j\\
	&\leq
	\max\left(B_j^{\min} \mu_j^{\min}, B_j^{\min} \mu_j^{\max}, B_j^{\max}, \mu_j^{\min}, B_j^{\max} \mu_j^{\max}\right)
	\end{align*}
	where $\mu^{\min}_j, \mu^{\max}_j$ are the lower and upper bounds on the expectation of the $j$-th column of $X \cdot \Embed$. $\Var_j^{\min}, \Var_j^{\max}$ are similarly defined for the variance.
    Both are bounded in \cref{lem:boundsVar} within \cref{sec:proofs_framework}.
\end{lemma}
Moreover, we let $B^{\min}$ be a vector of the minimum values of the blowup and $B^{\max}$ be a vector of the maximum values of the blowup.
Define $S^{\min}$ and $S^{\max}$ similarly for the shift.

\subsubsection{Bounding Attention}
\label{subsec:attention_bounds}
For a multi-headed attention mechanism, $\attn(X) = [\attnH_1(X); \ldots; \attnH_H(X)]$: i.e.\ the output is a concatenation of the individual attention head outputs.
So, the infinity norm worst-case deviation is simply bounded by the maximum worst-case deviation over the individual attention heads.
We thus have the following lemma:
\begin{lemma}
	\label{lem:att_bound_by_heads}
	\begin{align*}
		&\WD(\vec{e}_\nctx \cdot \desF{f^{\attn}} ; \InpSpace \times \blowupSet \shiftSet)_\infty 
	     \\ &\leq 
	     \max_h \WD(\vec{e}_\nctx \cdot \desF{f^{\attnH_h}} ; \InpSpace \times \blowupSet \shiftSet)_\infty.
	\end{align*}
\end{lemma}
The rest of this section will focus on bounding the worst-case deviation of a single attention head which will be the most challenging part of the proof.

We will show how to bound the contribution of each token to the attention weights by:
(1) establishing bounds for the attention weight coming from fixed tokens to the query token, (2) establishing an upper bound for the attention weight coming from free tokens to the query token.
We can then demonstrate that the attention weights are heavily biased towards the fixed tokens.

We implicitly consider rotary-type positional encodings in the attention mechanism, where the positional encodings are absorbed into the calculation of the logits.
As previously mentioned, we use $\ell$ to denote the logits on the query token.  It will be useful, in the proof of our results, to define and compute worst-case upper and lower bounds for $\ell$, which we define formally in \cref{def:lminlmax}.
% Denote the minimum and maximum values of the logits as $\ell^{\min}$ and $\ell^{\max}$ respectively.
To obtain bounds $\ell^{\min}$ and $\ell^{\max}$, we provide a simple algorithm in \cref{alg:lminlmax} to compute a bound on the logits:
\begin{lemma}[Bounds on Logits]
	\label{lem:lminlmax}
	Algorithm~\ref{alg:lminlmax} computes the minimum and maximum values of the logits $\ell$ given point-wise bounds $B^{\min}$, $B^{\max}$, $S^{\min}$, and $S^{\max}$ as specified in \cref{lem:boundsBS}.
\end{lemma}
% \vspace{-0.2cm}


% In the following lemmas, we represent the positions of restricted tokens (including the query token) as elements of the set $\{1, 2, \dots, \nfix \} \cup \{ \nctx \}$ (denoted by $[\nfix] \cup \{\nctx\}$), and the free tokens (including the query token for simplicity) as elements of the set $\{\nfix + 1, \ndes+2, \dots, \nctx - 1\}$ (denoted by $(\nfix, \nctx)$).
% \footnote{Technically, the query token is part of the restriction.
% However, to simplify the notation and the proof, we consider the query token as a free token without loss of generality.}

We now define the most consequential value to bound the worst-case deviation of the attention mechanism.
\begin{definition}[Softmax Extremal Values]\label{def:soft-extrem-values}
	Let $\softSumMinF = \sum_{j \in (\nfix, \nctx)} e^{\ell_j^{\min}}$ and $\softSumMaxF = \sum_{j \in (\nfix, \nctx)} e^{\ell_j^{\max}}$.
	Similarly, let $\softSumMinR = \sum_{i \in [\nfix] \cup \{\nctx\}} e^{\ell_i^{\min}}$ and $\softSumMaxR = \sum_{i \in [\nfix] \cup \{\nctx\}} e^{\ell_i^{\max}}$.
	Then, 
	\[
		\ssMinFB = \frac{\softSumMinF}{\softSumMaxR + \softSumMinF} \quad \text{and} \quad \ssMaxFB = \frac{\softSumMaxF}{\softSumMinR + \softSumMaxF}
	\]
	and
	\[
		\ssMinRB = \frac{\softSumMinR}{\softSumMinR + \softSumMaxF} \quad \text{and} \quad \ssMaxRB = \frac{\softSumMaxR}{\softSumMaxR + \softSumMinF}.
	\]
	%Let the lower-bound (resp. upper-bound) for softmax in \cref{eq:softmax_upper} be $\ssMinFB$ ($\ssMaxRB$) and
	%the lower-bound (resp. upper-bound) of softmax in \cref{eq:softmax_lower} be $\ssMinRB$ ($\ssMaxRB$).
\end{definition}

These extremal values can be used to get the bounds:

\begin{lemma}[Minimum and Maximum after Softmax]
	\label{lem:min_max_softmax}
	\begin{equation}
		\label{eq:softmax_upper}
		\ssMinFB \leq
		\sum_{j \in (\nfix, \nctx)} \softmax(\ell_j) \leq
		\ssMaxFB
	\end{equation}
	aswell as,
	\begin{align}
		\label{eq:softmax_lower}
		\ssMinRB \leq
		\sum_{j \in [\nfix] \cup \{\nctx\}} \softmax(\ell_j) \leq
		\ssMaxRB.
	\end{align}
\end{lemma}


In the following, $E[[\nfix] \cup \{\nctx\}, :] \in \R^{\nfix \times \dEmb}$ denotes the matrix with the rows in $[\nfix] \cup \{\nctx\}$ selected. 

\begin{lemma}[Worst-case Deviation of Attention]
	\label{lem:att_bound}
	Worst-case deviation of attention is bounded as follows,
	\begin{align*}
		&\WD(\vecNctx \cdot f^{\attnH}_{\ndes \mid r, q} ; \InpSpace \times \blowupSet \shiftSet)_\infty
	     \\ &\leq
	     2 \cdot (\ssMaxRB - \ssMinRB) \cdot \\
         &\max_{B, S}\norm{\Embed[[\nfix] \cup \{\nctx\}, :] \cdot \diag(B) + S}_{\frobInf} \\
		&+ 2 \cdot \ssMaxFB \cdot \max_{B, S} \norm{(E \cdot \diag(B) + S) \cdot V}_{\frobInf}\\
		&+ \norm{V}_\infty \cdot \ssMaxFB \cdot \max_{t \in s \cup \{q\}} \WD(\fenc; \{\vece_t \} \times \blowupSet \shiftSet\}))_\infty
	\end{align*}
        where $V$ is the value matrix in the attention head (see \cref{sec:appendix_model}).
\end{lemma}

Building on the above, we obtain a worst-case deviation bound using point-wise upper and lower bounds on $B \in \blowupSet$ and $S \in \shiftSet$.

%\begin{definition}[$B^{\min} $, $B^{\max}$, $S^{\min}$ ,$S^{\max}$]

%\end{definition}
\begin{corollary} \label{cor:Bmax} 
	Let $B^{\min} $, $B^{\max}$, $S^{\min}$, and $S^{\max}$ be as defined at the end of Definition \ref{def:blowup_shift}.
	Then, we have
	\begin{align*}
		\WD&(\vecNctx \cdot f^{\attnH}_{\ndes \mid r, q} ; \InpSpace \times \blowupSet \shiftSet)_\infty \leq  \\
		   & 2 \cdot (\ssMaxRB - \ssMinRB) \cdot \bigg(\max_{B \in [B^{\min}, B^{\max}]} \|E[[\nfix] \cup \{\nctx\}, :] \\
		   &\cdot \diag(B)\|_{\frobInf}  
	   + \max\big(\|{S^{\min}}\|_\infty, \|{S^{\max}}\|_\infty \big)\bigg) \\
		   &+ 2 \cdot \ssMaxFB \cdot \max_{B, S} \norm{(E B + S) \cdot V}_{\frobInf} \\
		   &+ \norm{V}_\infty \cdot \ssMaxFB \cdot \max_{t \in s \cup \{q\}} \WD(\fenc; \{\vece_t\} \times \blowupSet \shiftSet))_\infty
	\end{align*}
	where the maximization can be calculated with a simple linear program
    \footnote{
        We use the notation $B \in [B^{\min}, B^{\max}]$ to denote the set of vectors, $B$, where $B_j^{\min} \leq B_j \leq B_j^{\max}$.
    }.
\end{corollary}


\subsubsection{Bounding MLP and Identity}
Because the MLP and identity components are simple feed-forward networks without any ``cross-talk'' between tokens, we can use simple Lipschitz bounds.

\begin{lemma}[Worst-case Deviation of MLP and Identity]
	\label{lem:mlpbound}
	\begin{align*}
		\WD(\vec{e}_\nctx \cdot \desF{f^{\MLP}} ; \InpSpace \times \blowupSet \shiftSet)_\infty 
		\\ \leq
		\LipMLP_\infty \cdot \WD(\vec{e}_\nctx \cdot  \fenc; \InpSpace \times \blowupSet \shiftSet)_\infty
	\end{align*}
	and
	\begin{align*}
		\WD(\vec{e}_\nctx \cdot \desF{f^{\Iden}}  ; \InpSpace \times \blowupSet \shiftSet)_\infty \\ = \WD(\vec{e}_\nctx \cdot \fenc; \InpSpace \times \blowupSet \shiftSet)_\infty.
	\end{align*}
\end{lemma}


\newcommand{\ModelFinalC}{{\ModelFinal}'}
\newcommand{\eqclassAlgWD}{\texttt{BoundSoftmax}\WD}
\newcommand{\findAlpha}{\texttt{Find}\_\beta}
\newcommand{\findAlphaMin}{\text{Algorithm to find }\ssMinFA}
\newcommand{\findAlphaMax}{\text{Algorithm to find }\ssMaxFA}
\newcommand{\findAlphaMinR}{\text{Algorithm to find }\ssMinRA}
\newcommand{\findAlphaMaxR}{\text{Algorithm to find }\ssMaxRA}
\newcommand{\obj}{\text{objective}}
\newcommand{\freeToks}{\texttt{FreeToks}}

\newcommand{\minBilin}{\texttt{MinBilinear}}
\newcommand{\maxBilin}{\texttt{MaxBilinear}}
\newcommand{\lowerBounds}{\vec{r}^{\min}}
\newcommand{\upperBounds}{\vec{r}^{\max}}

\subsubsection{Bounding the Encoding Function}
\label{sec:concrcase}
We upper bound the worst-case deviation for $\fenc$ using
bounds on the ``blowup and shift'' in \cref{lem:boundsBS}:
\begin{lemma}[Worst-case deviation of $\fenc$]
	\label{lem:WD_fenc}
	\begin{align*}
		\WD&(\vec{e}_\nctx \cdot \fenc; \InpSpace \times \blowupSet \shiftSet)_\infty \leq 
		\\ \max_{t \in [\dVocab]} &
		\max_{j \in [\dEmb]}
		\max_{B \in [\vec{0}, B^{\max} - B^{\min}]} \\
		&\left|X[t] \cdot \Embed \cdot \diag(B) \cdot \vec{e}_j\right| + S_j^{\max} - S_j^{\min}
	\end{align*}
	where the inner maximum term can be computed via a simple linear program.
\end{lemma}

%Finally, we provide an algorithm which bounds the extremal values of the softmax, $\ssMinFB, \ssMaxFB, \ssMinRB, \ssMaxRB$ in \lev{TODO: appendix or not>}.
%Our algorithm for computing the extremal values of the softmax requires upper and lower bounding a bilinear form: we provide a simple, though non-optimal, algorithm for this task in \cref{fig:bilinOpt} in \cref{sec:appConcrcase}.
%			Now, we have the bound for $\fenc$, we simply need to bound the extremal values of softmax, $\softSumMinF, \softSumMaxF, \softSumMinR, \softSumMaxR$, to get a bound on the worst-case deviation.
%			\lev{TODO: alg reference here? I.e. point to lemma/ algorithm}


\iffalse
\begin{figure}[H]
	\begin{mdframed}
		$\findAlphaMinR:$%(\desF{\Model})$
		\begin{itemize}
			\item Let $\fenc(\vec{e}) = \vec{e} \cdot \Embed \cdot \diag(B) + S$
			\item For $j \in [s]$, let 
				\begin{align*}
					\vec{\ell}[j] = \frac{1}{\sqrt{\dEmb}} 
					\minBilin\bigg(Q \PosRot_{j, \nctx} K^T, \Embed[\desSet_j] \cdot \diag(B^{\min}) + S^{\min}, \\
					\Embed[\desSet_j] \cdot \diag(B^{\max}) + S^{\max} \bigg)
				\end{align*}
			\item Return $\sum_j \exp\left(\vec{\ell}_j\right)$
		\end{itemize}
		$\findAlphaMaxR:$%(\desF{\Model})$
		\begin{itemize}
			\item Let $\fenc(\vec{e}) = \vec{e} \cdot \Embed \cdot \diag(B) + S$
			\item For $j \in [s]$, let 
				\begin{align*}
					\vec{\ell}[j] = \frac{1}{\sqrt{\dEmb}} 
					\maxBilin\bigg(Q \PosRot_{j, \nctx} K^T, \Embed[\desSet_j] \cdot \diag(B^{\min}) + S^{\min}, \\
					\Embed[\desSet_j] \cdot \diag(B^{\max}) + S^{\max} \bigg)
				\end{align*}
			\item Return $\sum_j \exp\left(\vec{\ell}_j\right)$
		\end{itemize}
		$\findAlphaMax:$%(\desF{\Model})$
		\begin{itemize}
			\item Let $\fenc(\vec{e}) = \vec{e} \cdot \Embed \cdot \diag(B) + S$
			\item For $j \in (\ndes, \nctx]$, let 
				\begin{align*}
					\vec{\ell}[j] = \frac{1}{\sqrt{\dEmb}} \max_{t \in [\dVocab]} 
					\maxBilin\bigg(Q \PosRot_{j, \nctx} K^T, \Embed[t] \cdot \diag(B^{\min}) + S^{\min}, \\
					\Embed[t] \cdot \diag(B^{\max}) + S^{\max} \bigg)
				\end{align*}
			\item Return $\sum_j \exp\left(\vec{\ell}_j\right)$
		\end{itemize}

	\end{mdframed}
	\caption{The algorithm $\findAlpha$ which computes the pre-softmax logits for a given model $\Model$.
	}
	\label{fig:find_ell_alg}
\end{figure}

\fi

%	\begin{algorithm}[H] \label{alg:overwhelmCheckDetLP}
%		\SetKwInOut{Input}{Input}\SetKwInOut{Output}{Output}
%		\Input{$\ssMinRB, \ssMaxRB, \ssMaxFB, \softSumMaxR$}
%		\Output{\lev{TODO}} 
%		Calculate \begin{align*}
%			W^{\attn} &= 2 \cdot (\ssMaxRB - \ssMinRB) \cdot \left(\max_{B \in [B^{\min}, B^{\max}]} |E[\desSet, j] \cdot \diag(B)|  + \max(\norm{S^{\min}}_\infty, \norm{S^{\max}}_\infty \right) \\
%				  &+ 2 \cdot \ssMaxFB \cdot \max_{B, S} \norm{(E B + S) \cdot V}_{\frobInf}
%		\end{align*}
%		as in \cref{lem:att_bound}.
%		\\
%		Calculate $W = \Lip(\Unembed) \cdot \left(W^{\attn} + \LipMLP_\infty \cdot W^{\fenc} + W^{\fenc}\right)$ 
%		as per \cref{lem:mlpbound} and 
%		\\
%		Calculate $\peakToPeak(\desF{\Model}, \InpSpace)$ by evaluating $\desF{\Model}$ on $X \gets \InpSpace$ \\
%		\If{
%		$W < \peakToPeak(\desF{\Model}, \InpSpace) / 2$ }{\Return ``Squashed''
%	}
%	\Else{\Return ``Inconclusive''}
%	\caption{Meta Algorithm for Calculating bound on $\WD(\desF{\Model}, [X])$}
%\end{algorithm} 


%\newcommand{\findAlpha}{\texttt{Find}\_\beta}
%\newcommand{\findAlphaMin}{\texttt{Find}\_\ssMinFB}
%\newcommand{\findAlphaMax}{\texttt{Find}\_\ssMaxFB}
%\newcommand{\findAlphaMinR}{\texttt{Find}\_\ssMinRB}
%\newcommand{\findAlphaMaxR}{\texttt{Find}\_\ssMaxRB}

\section{Overwhelming Under Permutation Invariance}
\label{sec:perm_invar}
In \cref{sec:meta_framework}, we provided a concrete algorithm that decides overwhelming for a fixed context size $\nctx$.
In this section, we provide a more refined algorithm which can provide a tighter bound on the worst-case deviation in the restricted setting where the free tokens are allowed to be any permutation of a fixed string.
Up until this point, we have been ``lifting'' the domain of $\InpSpace$ into $\InpSpace \times \blowupSet \shiftSet$ to prove worst-case deviation bounds: i.e.\ we have been separating out the effect of layer normalization when proving a fixed string to be overwhelming.

In this section, we nullify the effect of layer normalization by defining an equivalence relation on $\InpSpace$ where the blowup and shift sets are the same for all elements in the equivalence class.
Then, we can derive a simpler upper-bound on the worst-case deviation of the model over the equivalence class as we can treat the layer normalization as a constant factor!

\begin{definition}[Permutation Equivalence relation]
	We will define the equivalence relation \(\sim\) on \(\InpSpace\).
	Let $X = (\desSet || X_f || \vecQuery) \in \InpSpace$ where $X$ are the free tokens, $\desSet$ are the fixed tokens, and $\vecQuery$ is the query token.
	Then, we define \(\sim\) such that for $X = (\desSet || X_f || \vecQuery), Y = (\desSet || Y_f || \vecQuery) \in \InpSpace$,
	\[
		X \sim Y \iff X_f = \pi(Y_f)
	\]
	where $\pi$ is a permutation of the free tokens.
	Note that if $X \sim Y$, then $\E[X \cdot \Embed] = \E[Y \cdot \Embed]$ and $\Var[X \cdot \Embed] = \Var[Y \cdot \Embed]$.
    Moreover, let $\freeToks$ be the multi-set of potential free tokens for $[X]$.
    I.e. $\freeToks = \{X_f[1], \dots, X_f[\nfree]\}$.
\end{definition}
% We abuse notation to define $B(X)$ to be the vectors for $X$'s blowup and $S(X)$ to be the vectors for $X$'s shift.

To bound the worst-case deviation over an equivalence class we use a similar algorithm (\cref{alg:overwhelmCheck2}) to \cref{alg:overwhelmCheckDet}.
We still bound the worst-case deviation of attention, but now the blowup and shift sets are singletons, leading to much tighter bounds.
Additionally, since blowup and shift are constant, the worst-case deviations of $f^{\MLP}, f^{\Iden}$ are $0$, and, we can use a linear program to find the extremal softmax contributions for the free tokens.

%\snote{TODO: Write preamble}
%Up until this section, we have been ``lifting'' the domain of individual components of the model to prove worst-case deviation bounds.
%The lifting, done by seperating out the effect of layer normalization, suggests that we should be able to ``factor out'' the effect of layer normalization in the model.
%Then, this equivalence relation partitions \(\InpSpace\) into disjoint equivalence classes, denoted by \([X] = \{Y \in \InpSpace \mid Y \sim X\}\), where each class contains all elements with the same variance and expectation.

\begin{theorem}[Input Restriction and Permutation Invariance] \label{thm:InpResPermInv}
	%The model $\desF{\ModelFinal}$ over domain $[X]$ if 
	If
	\[
		\WD(\desF{\Model}; [X])_\infty < \peakToPeak(\desF{\Model}, X) / 2
	\]
	% where $\blowupSet = \{B(X)\}$ and $\shiftSet = \{S(X)\}$ (i.e.\ blowup and shift are just singelton sets),
	then the output of model $\desF{\Model}$ is fixed for all inputs in $[X]$.
	Moreover, \cref{alg:overwhelmCheck2} produces an upper bound $W$ for $\WD(\desF{\Model}; \InpSpace)_\infty$.
	%\footnote{There is a notion here of ``sample-complexity'' which we will not go into and leave for future work
	%	Specifically, if we treat $X_1, \dots, X_D$ as samples from $\InpSpace$ and take the peak-to-peak difference to be $\max_{i \in [D]} \peakToPeak(\desF{\Model}, X_i) / 2$, then we have a \emph{larger bound} and thus can get away with larger worst-case deviation while still proving domain-collapse.
	%	In theory, the larger $D$ is, the more likely we are to have a larger peak-to-peak difference.
	%}
\end{theorem}
A full proof of \cref{thm:InpResPermInv} is provided in \cref{subsec:proofInpResPermInv}.
% Moreover, we prove the correctness of \cref{alg:overwhelmCheck2} in \cref{sec:appendix_perm}.
%\begin{algorithm}[H] \label{alg:overwhelmCheck2}
%	\SetKwInOut{Input}{Input}\SetKwInOut{Output}{Output}
%	\Output{Either the string ``Overwhelmed'' indicating that the output of the model $\Model$ evaluated on $s$ concatenated with $\nfree$ free tokens and query token $q$ is proven to be invariant under a permutation class, OR the string ``Inconclusive'' if no such proof is obtained.} 
%
%	Calculate $
%	B(X) = \frac{\gamma}{\sqrt{\Var[X \cdot E] + \eps}}
%	$
%	and $
%	S(X) = - \E[X \cdot E] B(X) + \beta
%	$
%	Set $B^{\min}, B^{\max} = B$ and $S^{\min}, S^{\max} = S$
%	\\
%	Set $\fenc(\vec{e}) = \vece \cdot E \cdot \diag(B) + S$ and
%	calculate bounds for $\ssMinFA, \ssMaxFA$ using either the naive algorithm in \cref{alg:naiveAlpha} or the linear program in \cref{alg:LPAlpha}.\\
%	Let
%	\[
%		\alpha_{fix} = 
%		\exp\left(
%			\fenc(\vec{e}_q) \cdot Q \Theta_{\nctx, \nctx} K^T \fenc(\vec{e}_{q})
%		\right)    
%		+  \sum_{j \in [\nfix]} \exp\left(
%			\fenc(\vec{e}_q) \cdot Q \Theta_{i, \nctx} K^T \fenc(\vec{e}_{\desSet_j})
%		\right)
%	\]
%	\\
%	Calculate $\ssMaxFB = \frac{\ssMaxFA}{\alpha_{fix} + \ssMaxFA}$ and $\ssMaxFB = \frac{\ssMinFA}{\alpha_{fix} + \ssMinFA}$ \\
%	Set $W^{\fenc} = 0$ to bound the worst-case deviation of $\fenc$ to $0$ \\
%	Let $\ssMaxFB = \frac{\ssMaxFA}{\ssMaxFA + \alpha_{fix}}$,
%	$\ssMinRB = \frac{\alpha_{fix}}{\alpha_{fix} + \ssMinFA}$,
%	and $\ssMinRB = \frac{\alpha_{fix}}{\alpha_{fix} + \ssMaxFA}$
%	\\
%	Calculate \begin{align*}
%		W^{\attn} &= 2 \cdot (\ssMaxRB - \ssMinRB) \cdot \|E[\desSet \cup \{s\}, :] \cdot \diag(B) + S\|_{\frobInf} \\
%			  &+ 2 \cdot \ssMaxFB \norm{(E[\freeToks, :] B + S) \cdot V}_{\frobInf}
%	\end{align*}
%	as in \cref{lem:att_bound} except that we have $\norm{(E[\freeToks, :] B + S) \cdot V}_{\frobInf}$ 
%	instead of $\norm{(E B + S) \cdot V}_{\frobInf}$ \footnote{
%		Recall that $E[\freeToks, :]$ connotes selecting the rows of $E$ corresponding to the tokens in $\freeToks$.
%	}\\
%
%	Calculate $W = \Lip(\Unembed) \cdot W^{\attn} $ 
%	Calculate $\peakToPeak(\desF{\Model}, \InpSpace)$ by evaluating $\desF{\Model}$ on $Y \gets [X]$ \\
%	\If{
%	$W < \peakToPeak(\desF{\Model}, \InpSpace) / 2$ }{\Return ``Overwhelmed''
%}
%\Else{\Return ``Inconclusive''}
%
%\caption{Overwhelming Verifier for Permutation Invariance}
%\end{algorithm} 

\begin{algorithm}[h] 
	\caption{Algorithm for deciding Overwhelming}
	\label{alg:overwhelmCheck2}
	\begin{algorithmic}
		\STATE \textbf{Input:} $\Model, \desSet, \query$
		\STATE \textbf{Output:} Return ``Overwhelemed'' or ``Inconclusive''
		\STATE
		\STATE Set $B = \frac{\gamma}{\sqrt{\Var[X \cdot E] + \eps}}
		$
		\STATE Set $
		S = - \E[X \cdot E] B + \beta
		$
		\STATE Set $\fenc(\vec{e}) = \vece \cdot E \cdot \diag(B) + S$
		\STATE Calculate $\ssMinFA, \ssMaxFA$ using the naive algorithm (\cref{alg:naiveAlpha}) or the linear program (\cref{alg:LPAlpha})
		\STATE Let
		\begin{align*}
			\alpha_{fix} = 
			\exp\left(
				\fenc(\vec{e}_q) \cdot Q \Theta_{\nctx, \nctx} K^T \fenc(\vec{e}_{q})
			\right)    \\
			+  \sum_{j \in [\nfix]} \exp\left(
				\fenc(\vec{e}_q) \cdot Q \Theta_{i, \nctx} K^T \fenc(\vec{e}_{\desSet_j})
			\right)
		\end{align*}
		\STATE Calculate $\ssMaxFB$, $\ssMinRB$, $\ssMinRB$
		 \STATE 
		 Calculate $W^{\attn}$ as in \cref{lem:attn_perm}
		\STATE Calculate $W = \Lip(\Unembed) \cdot W^{\attn} $ 
		% \STATE Sample $Y \gets [X]$ 
		\IF{
		$W <  \peakToPeak(\desF{\Model}, X) / 2$ }
		\STATE Return ``Overwhelmed''
		\ELSE 
		\STATE Return ``Inconclusive''
		\ENDIF
	\end{algorithmic}
\end{algorithm}



\subsection{Bounding Attention}
% Similar to the previous section, we need to bound the worst-case deviation of attention.
% Unlike the previous section though, because the blowup and shift do not vary, we can find \emph{tighter} bounds on attention's worst-case deviation.
To bound the worst-case deviation of attention, we provide two algorithms to bound $\ssMinFA, \ssMaxFA$ as defined in \cref{lem:min_max_softmax}.
The first, in \cref{alg:naiveAlpha}, takes a ``naive'' approach by iterating through the position of the free tokens and finding an extremal value for each position.
The second, in \cref{alg:LPAlpha}, uses a linear program to get a tighter bound.
To see why \cref{alg:LPAlpha} provides a valid bound, consider a similar program except where an \emph{integer} linear program is used.
Then, the integer version of \cref{alg:LPAlpha} finds the permutation of free tokens which maximizes (resp. minimizes) the pre-softmax logits.
Relaxing to a linear program, we have an upper-bound (resp. lower-bound) on the pre-softmax logits.
We can formally state the correctness of \cref{alg:naiveAlpha} and \cref{alg:LPAlpha} in the following lemma:
\begin{lemma}[Correctness of \cref{alg:LPAlpha} and \cref{alg:naiveAlpha}]
	\label{lem:corrAlpha}
	 Both \cref{alg:LPAlpha} and \cref{alg:naiveAlpha} provide valid bounds on $\ssMinFA$ and $\ssMaxFA$.
\end{lemma}
\vspace{-0.2cm}

\begin{algorithm}[tb]
	\caption{
		Linear Program to find $\ssMinFA$.\\
		To find $\ssMaxFA$, switch the $\min$ to a $\max$ in the objective.
	}
	\label{alg:LPAlpha}
	\begin{algorithmic}
	\STATE \textbf{Input:} {$\fenc$ and model $\Model$}
	\STATE
	\STATE Return the optimal value to the following linear program:
	\begin{align*}
		\min \quad & \sum_{j \in (\ndes, \nctx)} \exp\bigg(\sum_{t \in \freeToks} \\
			\fenc(\vecQuery) \cdot &Q \cdot \PosRot_{j, \nctx} K^T \cdot \fenc(\vec{e}_t)^T
	\bigg) \cdot x_{j, t} \\
		\text{subject to} \quad
			   & \forall j, \sum_{t \in \freeToks} x_{j, t} = 1 \\
			   & \forall t, \sum_{j \in (\ndes, \nctx)} x_{j, t} = 1, \\
			   &  \forall j, t, x_{j, t} \geq 0.
	\end{align*}
	\end{algorithmic}
\end{algorithm} 

Then, with the bounds on the softmax contributions, we can state a specialized version of the bound on attention's worst-case deviation for the permutation invariant case.

Then, we can state the following lemma:
\begin{lemma}[Worst-case Deviation of Attention]
	\label{lem:attn_perm}
	Worst-case deviation of attention is bounded as follows for fixed blowup $B$ and shift $S$ when considering the permutation class $[X]$ with free tokens $\freeToks$:
	\begin{align*}
		&\WD(\vecNctx \cdot f^{\attnH}_{\ndes \mid r, q} ; \InpSpace \times \blowupSet \shiftSet)_\infty
	     \\ &\leq
		2 \cdot (\ssMaxRB - \ssMinRB) \cdot \norm{\Embed[[\nfix] \cup \{\nctx\}, :] \cdot B + S}_{\frobInf} \\
		&+ 2 \cdot \ssMaxFB \cdot \norm{(E[(\nfix,\nctx),: ] B + S) \cdot V}_{\frobInf}.
	\end{align*}
\end{lemma}
The proof follows analogously to that of \cref{lem:att_bound} except for two main differences:
(1) we have a further restriction on the free tokens and thus only need to consider $E[(\nfix,\nctx), :]$ instead of $E$
and (2) becuase the blowup and shift are fixed, $\WD(\fenc; \{\vece_t\} \times \blowupSet \shiftSet)_\infty = 0$ and so wecan drop the last term in the original bound.


%%%%%%%%%%%%%%%%%%%%%%

%\begin{algorithm}[H] \label{alg:naiveAlpha}
%	\SetKwInOut{Input}{Input}\SetKwInOut{Output}{Output}
%	\Input{$\fenc$ and model $\Model$}
%	% \Output{\lev{TODO}}
%	Let $\fenc(\vec{e}) = \vec{e} \cdot \Embed \cdot \diag(B) + S$ \\
%	\For{$k \in (\ndes, \nctx]$}{
%		Let \begin{equation*}
%			\vec{\ell}^{\min}_k = \frac{1}{\sqrt{\dEmb}} \min_{t \in \freeToks}  \fenc(\vecQuery) \cdot Q \PosRot_{k, \nctx} K^T \cdot \fenc(\vec{e}_t)^T
%		\end{equation*}
%		and 
%		\begin{equation*}
%			\vec{\ell}^{\max}_k = \frac{1}{\sqrt{\dEmb}} \max_{t \in \freeToks}  \fenc(\vecQuery) \cdot Q \PosRot_{k, \nctx} K^T \cdot \fenc(\vec{e}_t)^T
%		\end{equation*}
%		where $\PosRot_{k, \nctx}$ is the matrix representing the effect of RoPE encodings
%	}
%	Return $\ssMinFA = \sum_k \exp\left(\vec{\ell}^{\min}_k\right), \ssMaxFA = \sum_k \exp\left(\vec{\ell}^{\max}_k\right)$
%	\caption{Naive Algorithm to find $\ssMinFA$ and $\ssMaxFA$}
%\end{algorithm} 
%
%\begin{algorithm}[H] \label{alg:LPAlpha}
%	\SetKwInOut{Input}{Input}\SetKwInOut{Output}{Output}
%	\Input{$\fenc$ and model $\Model$}
%	% \Output{\lev{TODO}}
%
%	Return the optimal value to the following linear program:
%	\begin{align*}
%		\min \quad & \sum_{j \in (\ndes, \nfree]} \exp\left(\sum_{t \in \freeToks} \fenc(\vecQuery) \cdot Q \cdot \PosRot_{j, \nctx} K^T \cdot \fenc(\vec{e}_t)^T\right) \cdot x_{j, t} \\
%		\text{subject to} \quad
%			   & \forall j, \sum_{t \in \freeToks} x_{j, t} = 1 \\
%			   & \forall t, \sum_{j \in (\ndes, \nfree]} x_{j, t} = 1 \\
%			   & \forall j, t, x_{j, t} \geq 0.
%	\end{align*}
%	\caption{Linear Program to find $\ssMinFA$.
%		To find $\ssMaxFA$, we have the same linear program except that we switch the $\min$ to a $\max$ in the objective. 
%	}
%\end{algorithm} 
%

% \begin{algorithm}[H] \label{alg:freezecheckerLP}
% 	\SetKwInOut{Input}{Input}\SetKwInOut{Output}{Output}
% 	\Input{A fixed transformer model M, a string of tokens $s$ denoting a fixed part of the input to M, and integer $k$ denoting the number of free tokens in the input to M }
% 	\Output{Either the string "Frozen" indicating that the output of the model M evaluated on $s$ concatenated with $k$ free tokens is proven to be invariant under the choice of the free tokens, OR the string "Inconclusive" if no such proof is obtained.} 

% \mnote{Populate this area with the precise pseudocode for the computation of the LP upper bound to $\WD(\desF{\Model}; [X] \times \blowupSet \shiftSet)_\infty$ using the terminology set in the rest of the paper.}



% For each $k \in [\tau]$ define $LP^+_k$ to be the value of the following linear program and let $\vec{\alpha}^k_+$ be the vector minimizing the value of the following linear program:

% \begin{align}
%     &\textbf{ Minimize:} \nonumber \\
%     &&\vec{\alpha} \cdot \grad E(\mathcal{F})  \\
%      &\textbf{Over:} \nonumber \\
%      && \vec{\alpha} \in \mathbb{R}^{\tau} \\
%     &\textbf{Subject To:} \nonumber \\
%     &&\vec{\alpha} \cdot \grad D_{ij}(\mathcal{F}) \geq 0 \text{    } \forall (i,j) \in \text{Saturated}(\mathcal{F}) \\
%     &&\text{ and } \nonumber\\
%     &&\alpha_k \geq \delta   \\
%     && \| \vec{\alpha} \|_{\infty} \leq \epsilon
% \end{align}


% For each $k \in [\tau]$ define $LP^-_k$ to be the value of the following linear program and let $\vec{\alpha}^k_-$ be the vector minimizing the value of the following linear program:
% \begin{align}
%     &\textbf{ Minimize:} \nonumber \\
%     &&\vec{\alpha} \cdot \grad E(\mathcal{F})  \\
%     &\textbf{Over:} \nonumber \\
%      && \vec{\alpha} \in \mathbb{R}^{\tau} \\
%     &\textbf{Subject To:} \nonumber \\
%     &&\vec{\alpha} \cdot \grad D_{ij}(\mathcal{F}) \geq 0 \text{    } \forall (i,j) \in \text{Saturated}(\mathcal{F}) \\
%     &&\text{ and } \nonumber\\
%     &&\alpha_k \leq -\delta   \\
%     && \| \vec{\alpha} \|_{\infty} \leq \epsilon 
% \end{align}



% \If{
% 	$
% 		\WD(\desF{\Model}; [X] \times \blowupSet \shiftSet)_\infty < \peakToPeak(\desF{\Model}, X) / 2
% 	$  \mnote{this needs to be expanded to show the actual if decision we compute, which is an upper bound on the worst case deviation rather than the exact wcd}}{\Return ``Frozen''  }

% \Else{\Return ``Inconclusive''}



% 	\caption{Freeze Verifier - LP Bound  \mnote{Under Construction}}
% \end{algorithm} 


After which the proof of \cref{thm:InpResPermInv} follows analogously to that of \cref{thm:InpRes}.

% \begin{proof}[Proof sketch of \cref{thm:InpResPermInv}]
%     First note that because the blowup and shift are constant for $[X]$, then the worst-case deviation of $\fenc$ is $0$ as the encoding function is fixed to a linear function for the domain $[X]$.
%     Then, noting that \cref{alg:overwhelmCheck2} uses the same framework as \cref{alg:overwhelmCheckDet}, we now need to prove that $\ssMinFA, \ssMaxFA, \ssMinRA, \ssMaxRA$ are valid extremal softmax bounds,
% \end{proof}


% and the algorithm in \cref{alg:overwhelmCheckDet}, we provide a simple proof of the above theorem.


% \lev{TODO: remove this ``proof'' structure and replace with filling in missing pieces}


% \begin{proof}
% 	First note that
% 	\[
% 		\WD(\vecNctx \cdot \desF{f^{\component}}; [X] \times \blowupSet \shiftSet ) = 0
% 	\]
% 	for $\component \in \{\MLP, \Iden\}$ as fixing the query token \emph{and the blowup/ shift} fixes the output of the MLP and identitiy for the last token.

% 	Now, we have to find the extremal softmax contributions for the fixed and free tokens.
% 	For the fixed tokens, because variance and expectation are fixed, we can note that there logits always equal
% 	\[
% 		\vec{\ell}_i = \frac{1}{\sqrt{\nctx}} \fenc(\vecDes) \cdot Q \PosRot_{i, \nctx} K^T \cdot \fenc(\vecQuery)
% 	\]
% 	for $i \in [\ndes]$ where $\PosRot_{i, \nctx}$ is the rotation matrix for RoPE position $i$.

% 	So then, we need to bound $\softSumMinF$ and $\softSumMaxF$.
% 	A naive apporach to computing $\softSumMaxF$ (resp. $\softSumMinF$) is outlined in \cref{fig:find_ell_alg_naive}.
% 	The algorithm simply finds the maximum (resp. minimum) of the pre-softmax logits for the free tokens at each position.
% 	Alternatively, we can formalize finding $\softSumMaxF$ (resp. $\softSumMinF$) as a mixed integer linear program (MILP) as outlined in \cref{fig:find_ell_alg_LP}.
% 	The key idea is that we can think of the class of permutation matrices $X \in \{0, 1\}^{\nfree \times \nfree}$ where $X[j, t] = x_{j, t}$ as being specified by a set of linear-integer constraints.
% 	Then, if we find the $X$ which maximizes (resp. minimizes) the sum of the pre-softmax logits, we can get a bound on the softmax sum.
% 	Of course, integer-linear programming is NP-hard,
% 	so we can relax the constraints to a linear program (LP) and get a bound on the softmax sum.
% 	%Then, we can use the results of \cref{lem:min_max_softmax} to get bounds on the softmax sum.

% 	Next, we can use the bounds on $\ssMaxFB, \ssMinRB, \ssMaxRB$ alongside \cref{lem:min_max_softmax} to get bounds on the worst-case deviation of the model from softmax.  \mnote{I think we should specify precisely how we do this in pseudocode inside Algoriths \ref{alg:overwhelmCheck} and \ref{alg:freezecheckerLP}.}

% 	Finally, we can use the lifted worst-case deviation to get a bound on the worst-case deviation of the model conditioning on a fixed $B$ and $S$.
% \end{proof}


\section{Evaluating on a Single Layer Model}
\label{sec:model_eval}

In this section, we empirically demonstrate our algorithm on a single layer transformer model trained for next token prediction on a standard text corpus.
Specifically, we run \cref{alg:overwhelmCheck2} to calculate a bound on the worst-case deviation and peak-to-peak difference for the model where the domain has an input restriction with free tokens drawn from a single permutation class as in \cref{sec:perm_invar}. When \cref{alg:overwhelmCheck2} outputs $\OverwQ$ it follows from \cref{thm:InpResPermInv} (informally, \cref{thm:informal_overwhelm_perm} in the Introduction) that the output of the model is the same for all permutations.

\subsection{Experimental Setup}
We train a single-layer transformer model with an embedding dimension of 512 and the BERT tokenizer \cite{devlin-etal-2019-bert}.
The model's architecture is outlined in \cref{sec:model}.
We train on the AG News dataset \footnote{see \href{https://pytorch.org/text/stable/_modules/torchtext/datasets/ag_news.html}{PyTorch's documentation}}, using the training split with a batch size of 8, learning rate of 5e-5 using the Adam optimizer, and $20$ epochs.

We examine a few different types of input restrictions
\footnote{
    The first token is always fixed to the ``BOS'' token which connotes the beginning of a string in the BERT tokenizer. For simplicity, we think of the BOS token as part of the fixed tokens.
}:
\begin{itemize}[nosep]
	\item \textbf{Random String:} Randomly sample alpha-numeric tokens (including space and punctuation) to form $\desSet$
	\item \textbf{Random Sentences:} Sample (and concatenate) sentences from the AG News testing set, to form $\desSet$
	\item \textbf{Repeating Tokens:} Form $\desSet$ by repeating the string ``what is it'' (which is $3$ tokens in the BERT tokenizer).
\end{itemize}

In each case, we have a \emph{tunable} number of tokens in $\desSet$\footnote{Larger input restrictions are produced by concatenation.}.
For the permutation class, we consider the following two cases:

\begin{itemize}[nosep]
	\item \textbf{Hamlet}: We use the famous quote from Shakespeare's \emph{Hamlet}:
    % \begin{quote}
    ``To be or not to be, that is the question. Whether it is a nobler in the mind to suffer the slings and arrows''
        % \end{quote}

	\item \textbf{The Old Man and the Sea}: We fix the free tokens to be a snippet from the opening chapter of Hemingway's \emph{The Old Man and the Sea}:
		% \begin{quote}
		 ``The blotches ran well down the sides of his face and his hands had the deep-creased scars from handling heavy fish on the cords.''	
		% \end{quote}
\end{itemize}

Finally, the question mark token ``?'' is used as the query token.

In \cref{fig:smallex}, we show our algorithm in action for the ``Hamlet'' permutation class and repeated fixed string.
In \cref{sec:appOut}, we plot the worst-case deviation in \cref{fig:worst-case} and the peak-to-peak difference in \cref{fig:ptp} for each of the input restrictions and permutation classes.

\begin{figure}
    \centering
\includegraphics[width=\linewidth,keepaspectratio,clip]{figs/plots/w_analysis_single.pdf}
    \caption{
        Worst-case deviation for the ``Hamlet'' permutation string and fixed repeated string.
        The $x$-axis is the number of total tokens and the $y$-axis is the log-scale worst-case deviation.
        We mark cases of provable overwhelming with a red ``x.''
    }
    \label{fig:smallex}
\end{figure}

\subsection*{Observations}
The experimental results for worst-case deviation bounds in \cref{fig:worst-case} and peak-to-peak difference in \cref{fig:ptp} provide interesting insights and questions.
The upper-bound on worst-case deviation trend downwards as the fixed input token size is increased.
But only the ``repeated-string'' setting is monotonically decreasing.
Moreover, all plots exhibit a ``phase transition:'' at around 180 tokens and again around 350 tokens, the upper-bound on worst-case deviation has a sharp drop.
We note that the linear program seems to be the most useful for the ``AG News'' dataset when paired with the \emph{The Old Man and the Sea} free string.
Finally, we note that in both plots, ``not much'' seems to happen prior to 180 tokens.
Specifically, both the peak-to-peak difference and bound on worst-case deviation remain flat for the smaller fixed string.

\subsection{Overwhelming Continues Through Generation}
When a transformer model is \overwQ the generated token immediately after the text is fixed for any free string. However, this does not apriori imply that subsequent tokens when the model is used to repeatedly generate tokens will be fixed. 

To test this we run examples of text generation where we begin with an overwhelming string $s$ and use the model to generate text. We then check if the model remains \overwQ as the generated text is included as part of the fixed string. The results of these tests are plotted in \cref{fig:contOverwhelm} within \cref{sec:appOut}. One would hope the model remains overwhelmed as the fixed string is increased by the addition of the newly generated tokens. This occurs for the Hamlet free string with the repeated and random fixed strings, where as expected the worst-case deviation remains mostly constant throughout token generation. For the longer free string and the AG News fixed strings the worst-case deviation peaks after a few tokens and the model stops to be provably \overwQ.

\section{Conclusion}
In this work, we propose a simple yet effective approach, called SMILE, for graph few-shot learning with fewer tasks. Specifically, we introduce a novel dual-level mixup strategy, including within-task and across-task mixup, for enriching the diversity of nodes within each task and the diversity of tasks. Also, we incorporate the degree-based prior information to learn expressive node embeddings. Theoretically, we prove that SMILE effectively enhances the model's generalization performance. Empirically, we conduct extensive experiments on multiple benchmarks and the results suggest that SMILE significantly outperforms other baselines, including both in-domain and cross-domain few-shot settings.

%\section{Algorithm for verifying single attention head}

\begin{algorithm}[H]
	\caption{$\softmaxCheck:$ Check the softmax}
	\KwData{$k, \nfree, \aExt, \aDic^{min}, \aDic^{max}, \vExt, \vDic^{min}, \vquery$}
	\KwResult{Verification of $\bot$ or $\top$}
	\lev{We need a \emph{upper bound} on $\aDic$ as well}
	Set
	\[
		w = \frac{\sum_{i\in [k]} \exp(\aDic^{min})}{\sum_{i \in [k]} \exp(\aDic^{max}) +
		\sum_{i \in [\nfree]} \exp(\aExt)}.\;
	\]
	%\lev{Hrmm this kind of implies $w > \half$ so if we get $w$ so large it does not really matter if we have the right associated $v$. Maybe instead of thinking about $\tThresh$ we can phrase it as $\calD_{next}$}\;
 \lev{We need to bound \emph{the subtraction} potential from $\vDic$}
	If $w \cdot \vDic^{min} \geq (1 - w) \cdot \vExt$ return $\top$ otherwise return $\bot$ \;
\end{algorithm}

\begin{algorithm}[H]
	\caption{$\findKRep$: find number of times to repeat token $\tau$ at the end of a free ranging sequence of $\nfree$ tokens to ensure a dictator restriction \lev{We can probably have better bounds vis a vis position by writing an expectation} \lev{re-write in terms of infinity norms}}
	\KwData{$\xQuery, \xDic, \nfree, \klen$}
	\KwResult{Result in $\{\top, \bot\}$}
	Fix $\tQuery = \OneHot(\xQuery), \tDic = \OneHot(\xDic)$	\;
	Let $\pQuery = \OneHot(\nfree + k)$\;
	Let $\aExt = \frac{1}{\sqrt{d}} \max_{i \in [\dVocab]} \left[\tQuery EQK^T E^T \OneHot(i)^T + \norm{\pQuery QK^TP^T}_\infty \right]$ \;
	Let $\aDic^{min} = \frac{1}{\sqrt{d}} \left(\tQuery E_q QK^T E^T \cdot \tDic^T - \norm{\pQuery QK^TP^T}_\infty \right)$ \;
	Let $\aDic^{max} = \frac{1}{\sqrt{d}} \left(\tQuery E_q QK^T E^T \cdot \tDic^T + \norm{\pQuery QK^TP^T}_\infty \right)$ \;
	$\tThresh = \OneHot\left(\arg \max_{i \in [\dVocab]} \tDic \cdot EVOU \cdot \OneHot(i)^T - \norm{PVOU \cdot \OneHot(i)^T}_\infty + \tQuery E_q U \cdot \OneHot(i)^T \right)$ \;
	\lev{TODO: monotonicity theorem necessary}\;
	$\vDic^{min} = \tDic \cdot EVOU \cdot \tThresh^T - \norm{PVOU \cdot \tThresh^T}_\infty + \tQuery E_qU \cdot \tThresh^T$\;
	$\vExt = \max_i \max_j \left(\OneHot(i) \cdot EVOU \cdot \OneHot(j)^T + \norm{PVOU \cdot \OneHot(j)^T}_\infty\right)$\;
	\If{$\vDic^{min} \leq 0$}{
		\Return $\bot$\;
	}
	\Return $\softmaxCheck(k, \nfree, \aExt, \aDic^{min}, \aDic^{max}, \vExt, \vDic^{min})$\;
\end{algorithm}

\subsection{Soundness}

Sketch for proof is using the lemmas below.

\begin{lemma}[Infinity Norm for Positional Encoding Bound]
    Fill in lemma
\end{lemma}
\begin{proof}
    Should be easy \lev{TODO}
\end{proof}

\begin{lemma}[Passing $\softmaxCheck$ line $2$ implies fixing (assuming extremal next-token)]
	Assume that the dictator tokens, $\xDic$, are in location $L \subset [\nfree + k]$ where $|L| = k$ and that for some one hot token, $\vec{t}$, $\tDic \cdot EVOU \cdot \vec{t} + \frac{1}{k} \sum_{\ell \in L} \OneHot(\ell) \cdot PVOU \cdot \vec{t} \geq \vDic^{min}$.
	Also, assume that the token $x$ corresponding to $\vec{t}$ is the maximum of such possible sequences.
	Then if $\softmaxCheck$ return $\top$, we have a dictator sequence where the next token is $x$
\end{lemma}
\begin{proof}
	We begin by noting that	$\vExt$ is the maximum possible value of $(\mathbf{x}E + P_{enc}) VOU \OneHot(j)$ for all possible tokens $j$.
	As such, if $(1 - w) \cdot \vExt < z$ for any $z$, then $(1 - w) \cdot \frac{1}{N} \sum_{a \in N} (T_a \cdot E + P_a \cdot P) VOU \cdot {T'_a}^T < z$ for all settings of tokens $T, T' \subseteq {\{\OneHot(1), \dots \OneHot(\dVocab)\}}^N$ and positions $P \subseteq \{\OneHot(0), \dots \OneHot(\nctx)\}^\nctx$.
	So, if for a token $\vec{t}$, we have that $w \cdot \left[\sum_{i \in L} (\tDic E + \OneHot(L_i)) VOU \cdot \vec{t}^T + \tQuery E_q U \cdot \vec{t} \right] > (1 - w) \cdot \vExt$, then a greedy sampling strategy for the next token will select $\vec{t}$ unless there is a token $\vec{t'}$ where 
	\[
		 \sum_{i \in L} (\tDic E + \OneHot(L_i)) VOU \cdot \vec{t}^T + \tQuery E_q U \cdot \vec{t} > \sum_i (\tDic E + \OneHot(L_i)) VOU \cdot \vec{t}^T + \tQuery E_q U \cdot \vec{t}^T.
	\]
	But then, $\vec{t'}$ is selected irrespective of the non-dictator tokens as well.
	We can also see that 
	\[
		\frac{1}{k} \sum_{i \in L} (\tDic E + \OneHot(L_i)) VOU \cdot \vec{t}^T + \tQuery E_q U \cdot \vec{t}^T \geq \vDic.
	\]
	And so, we have a dictator token.

\end{proof}

\begin{lemma}[Monotonicity of extremal token and $\softmaxCheck$]
	If there is another token which is more likely to be outputted, then the $k$ value should pass even if its smaller
\end{lemma}
\begin{proof}
	Application of the above using $L$...
\end{proof}

\begin{lemma}[Non-extremal next token okay]
	Show how if there is a token with more weight (due to say position reasons) then that's okay)
	Use above lemma
\end{lemma}


\begin{lemma}[Passing $\softmaxCheck$ implies fixing]
	Part of this is assuming all the inputs represent proper bounds \emph{for the predicted token} are correct (does not need to be fixed to the predicted token as monotonicity there)
\end{lemma}

\begin{theorem}[Soundness for $k$]
	Just need to prove that the inputs to $\softmaxCheck$ are indeed extremal
\end{theorem}

\section{Verifying Attention with Parallel Linear Layer}
We will use the following model

\begin{align}
	\mathcal{M}(\textbf{t}) =& \sigma^* 
	       \left( (\tQuery E + \pQuery  P) Q K^T  (P^T + E^T \textbf{x}^T) /\sqrt{d} \right) \cdot (\textbf{x}E+P)VOU + (\tQuery E + \pQuery P)U\\
        &+ \pQuery \cdot \relu((\textbf{x}E + P) A_{enc} + b) A_{dec}
\end{align}

Then,
\begin{align}\label{eq:linlayer}
    \pQuery \cdot \relu((\textbf{x}E + P) A_{enc} + b) A_{dec} = 
    \relu((\tQuery E + \pQuery P) A_{enc} + b) A_{dec}
\end{align}

\cref{eq:linlayer} is fixed by $\pQuery$ and $\tQuery$ being one hot vectors.
We can then use the same technique as before

% TODO: Remove P from inifinty norm (its just VOU)

\lev{TODO: Ask people in AI if sequential or parallel are more popular}

\section{Adding in the Layer Norm}
\lev{Write somewhere that $ \tQuery \cdot \relu(\mathcal{L} \cdot A_{enc} + b) \cdot A_{dec} =  \cdot \relu(\tQuery \cdot \mathcal{L} \cdot A_{enc} + b) \cdot A_{dec}$}
\begin{align*}
	\mathcal{M}(\textbf{t}) =& \sigma^* 
	       \left(\tQuery \mathcal{L} \cdot  Q K^T  \mathcal{L}^T /\sqrt{d} \right) \cdot \mathcal{L} VOU + \tQuery\mathcal{L} U\\
        &+ \tQuery \cdot \relu(\mathcal{L} \cdot A_{enc} + b) \cdot A_{dec}
\end{align*}
where
\[
    \mathcal{L} = LN(\mathbf{x} E + P)
\]
\newcommand{\tild}[1]{\widetilde{#1}}

\lev{Specify normalized normalized all one. Specifify, that everything is entrywise!!...
Maybe fix a column and then write out the operation!}
\newcommand{\LNInp}{\tild{\mathbf{x}}}

We define $LN(\tild{\mathbf{x}})$ as 
\[
	LN(\LNInp) = \frac{\LNInp - \OneMat \cdot \LNInp}{\sqrt{\OneMat \cdot (\LNInp\circ\LNInp) - (\OneMat \cdot \LNInp) \circ (\OneMat \cdot \LNInp) + \eps \cdot \OneMat'}} \cdot \gamma + \beta \cdot \OneMat'
\]

Where $\OneMat \in \mathbb{R}^{\nctx \times \nctx}$ and $\OneMat' \in \mathbb{R}^{\nctx \times \dEmb}$ are the matrices with all ones and $\eps$ is a small constant.

\lev{positional encoding in layer norm make things tricky!}
\lev{Think of layernorm as a left multiplication by a fixed vector/ matrix and fixed offset.}

\lev{I think that bounds here need to bounded in a smarter way. Maybe consider your sequence length of repeated tokens, and then this gives us a much tighter average bound and upper bound for the denominator (simply done by maximizing variance?).}

\subsection{Bounding the Layer Norm Scalars}
Given that layernorm acts \emph{coordinate}-wise (column-wise) in the columns of the input, we can provide bounds per coordinate $c \in [\dEmb]$.

\begin{lemma}[Upper bound on expectation]
	\label{lemma:upperExpBound}
	For a token repeated $\klen$ times with $\nfree$ tokens, the expectation in the layer norm is upper bounded by
	\[
		\frac{\klen \left(\tDic \cdot E \cdot \vec{t}_c^T\right)}{\klen + \nfree} +
			\frac{\nfree \left(\norm{E \cdot \vec{t}_c^T}_\infty\right)}{\klen + \nfree} + \norm{P \cdot \vec{t}_c^T}_\infty.
	\]	
\end{lemma}
\begin{proof}
	The proof follows from a simple expansion expectation and noting that the dictator token is repeated $\klen$ times.
\end{proof}

\begin{lemma}[Lower bound on expectation]
	\label{lemma:lowerExpBound}
	\[
		\frac{\klen \left(\tDic \cdot E \cdot \vec{t}_c^T\right)}{\klen + \nfree} -
			\frac{\nfree \left(\norm{E \cdot \vec{t}_c^T}_\infty\right)}{\klen + \nfree} - \norm{P \cdot \vec{t}_c^T}_\infty.
	\]
\end{lemma}

\begin{lemma}[Upper bound on variance]
	\label{lemma:upperVarBound}
	(Pretty easy by combining bounds on expectation)

	For a token repeated $\klen$ times with $\nfree$ tokens, the variance in the layer norm is upper bounded by
	\[
		AHAHA
	\]
\end{lemma}

\lev{Easier to state lemma below when Layer norm is properly defined column wise}
\lev{add in the epsilon part, epsilon. }
\begin{lemma}[Upper bound on denominator]
    For a token repeated $\klen$ times with $\nfree$ tokens, the denominator in the layer norm is upper bounded by, ignoring the $\eps$,
    % https://stats.stackexchange.com/questions/142651/does-the-uniform-distribution-have-the-greatest-variance-among-all-concave-distr#:~:text=The%20uniform%20distribution%20on%20a,two%20points%20of%20the%20graph).
    \[
	    \frac{\left(\norm{E \cdot \vec{t_c}^T}_\infty + \norm{P \cdot \vec{t_c}^T}_\infty \right)}{\sqrt{3}}.
    \]
\end{lemma}
\begin{proof}
	We first (\lev{TODO: cite}) note that for any random variable bounded in $[-a + b, a + b]$, the standard deviation is at most $a/\sqrt{3}$.
	Then, we can note that the difference between the maximum and minimum of the embedding summed with positional encoding is at most $2(\norm{E \cdot \vec{t_c}^T}_\infty + \norm{P \cdot \vec{t_c}^T}_\infty)$.
 \lev{elaborate a bit maybe?}
\end{proof}

\begin{lemma}[Lower bound on denominator]
	The denominator in the layer norm is lower bounded by, ignoring the $\eps$,
	\[
		AHAHAH %\frac{1}{\sqrt{\nctx}} \sqrt{\sum_{p \in [\nctx]} \left(\min(\vec{t_p} \cdot P \cdot \vec{t_c}^T)\right)^2}
	\]
\end{lemma}
\begin{proof}
	For $E_c = E \cdot \vec{t}_c, P_c = P \cdot \vec{t}_c$, we can rewrite the denominator as the squareroot of
	\begin{align*}
		\OneMat \cdot (xE_c + P_c) \circ (xE_c + P_c) - (\OneMat \cdot (xE_c + P_c)) \circ (\OneMat \cdot (xE_c + P_c)) =& \OneMat \cdot (xE_c \circ xE_c) + \OneMat \cdot (P_c \circ P_c) + 2 \cdot \OneMat \cdot (xE_c \circ P_c)	\\
														 &- (\OneMat \cdot (xE_c))^2 - (\OneMat \cdot P_c)^2 - 2 \cdot (\OneMat xE_c \circ \OneMat P_c).
	\end{align*}
	Then, $\OneMat \cdot(xE_c \circ xE_c) - (\OneMat \cdot xE_c)^2 \geq 0$ and $\OneMat \cdot (P_c \circ P_c) - (\OneMat \cdot P_c)^2$ is a fixed value.

	So, we now must find
	\[
		\min_x (\OneMat \cdot (xE \circ P) - \OneMat xE \circ \OneMat P).
	\]
    \lev{Given series of fixed shifts, can we move the fixed shift back to 0? Look at pairwise differences in $E$ and in $P$ ($O(\dVocab^2$). Write out pairwise matrix, build a bound. Easy to show something about average sequence.
    Worth looking at, is $\beta$ positive? Because otherwise repeated entries get subtracted off}
    \lev{Maybe there is a way to remove the bounds}
\end{proof}

\pagebreak


%\section{List of Ideas}

\begin{itemize}
    \item No go result on how memory is stored to show $.... 2 + 5 =$ and $.... 3 + 4 =$ will fail because the previous will always overwhelm. Generalizes to attn heads in parrallel (pay attention to only a few tokens). Maybe in asymptotic for many heads in serial. Something you can prove w/o making the model very contrived.
    \item Formalize model as dynamical system up to some error
    \item Same model, but if you have long context, then will it still select max (or smthng) at the end? Ask to prove model will produce max on any (or up to a bound). Show something with a state machine type model (like RNNs/ Mamba, etc) or Neural Turing Machines or Reinforcement Learning. For example for max, you can always have the model store some information and that gets passed back into the input.
\end{itemize}
%Logging:


UserID
ScenarioID
controlMode
requestID (Nummer der Request)
elapsed Time
distanceTravelledSinceLastLog
distanceToEndOfInstructedPath (Luftlinie zum Ende des gefolgten Pfades)
lengthOfCurrentInstructedPath (Gesamtlänge noch zu folgendem gezeichneter Pfad + generierte Verbindungsstrecke Auto <-> Pfad)
lengthOfCurrentInstructedInputPath (Gesamtlänge noch zu folgendem gezeichneter Pfad)
distanceToEnd (Auto <-> Ende bis wo hin operiert werden muss)
vehiclePosition (Verlauf der unity positions vektoren -> nun auch mit offset, also vergleichbar, wenn man damit was anfangen will)
vehicleSpeed (in kmh)
constructionSiteEntered (ob erstmalig unmittelbar vor der ConstructionSite angekommen)
endReached ("Request geschafft")
closest Lane (nummer lane 0-indiziert von links bis rechts)
currentLaneDeviation (Spurabweichung zur closest Lane Mitte  auch in metern)
timeOfCollisionAvoidanceTraffic (Zeit summe wenn Verkehr in der CollisionAvoidance Range ist (egal ob hinten oder vorne, noch ob es das auto stört))
timeOfCollisionAvoidanceObstacle (wie oben für Obstacles <- Baustellenfahrzeuge, Baustellenmarker, Metallgrenzen links und rechts von Straße)
timeOfCollisionAvoidancePedestrian (wie oben für Fußgänger)
amountOfAdditionInput (Summe wie oft der Pfad erweitert wurde +1 pro Aktion)
amountOfAdditionMarkers (Summe um wie viele "Stellen" der Pfad erweitert wurde +x pro Aktion <- relevant für Anteil an Snap to Mid in Trajectory, für PathPlanning und Waypoint immer +1)
amountOfSnapToMiddleInput (+1 wenn Addition Input und mindestens einmal Snap To Mid verwendet wurde, für path planning immer 1, weil im Grunde immer Mittig)
amountOfSnapToMiddleMarkers (wie amount Of AdditionMarkers, nur dass nur hochgezählt wird wenn die "Stelle" gesnappt wurde)
amountOfReadjustmentInput (wie obiges nur statt Addition: Readjustment, heißt: anfang und ende sind in alter Trajektorie oder nah parallel dazu, Waypoint wird verschoben, PathPlanning backwards, immer +1)
timeSinceLastInput (Zeit seit letztem input)
currentlyNeglectedTime (aktuelle Zeit wie lang vehikel schon still steht (colision avoidance) oder wenn Ende vom angegebenen Pfad)
blindTimeSum (wenn vehikel nicht neglected und schon eingaben gemacht wurden -> aufsummieren Zeit wenn nicht in Main oder secondary view)
isMainRequest
isSecondaryRequest
totalRequestAmount (insgesamte Anzahl an Requests in dem Szenario, hat nicht direkt was mit dieser request instanz zu tun, aber generelle info)
sideOfConstructionSite (ob construction site links oder rechts bei der Straße




Für ein Szenario/ case jeweils::
OverviewLog:
UserID
ScenarioID
controlMode
elapsedTime
activeRequests (wie viele grade angezeigt werden)
succeededRequests (wie viele bisher geschafft sind)
mousePositionX
mousePositionY
usingMultiView (also wenn zwei Requests gleichzeitig angezeigt = true sonst = false)
usingSingleViewMain (auch nur = true wenn nur ein vehikel in der Hauptanzeige ist)
EyeGazeArea (das was du noch meintest -> was aktuell angeschaut wird: string wert aus: RequestList, MainView, SecondaryView, Instructions, None)
mouseClickLeft (summe)
mouseClickRight (summe)
pressLeftCtrl (summe)
pressShift (summe)
TimestampLog:
UserID
ScenarioID
controlMode
elapsed time
timeStampEvent (String wert aus RequestStarted, RequestFinished, RequestOpenedMain, RequestOpenedSecondary, RequestRemovedMain, RequestRemovedSecondary)
additionalInfo (eigentlich immer nur die Request Nummer des vehikels)
-> Es kommen immer Logs rein sobald eines der genannten Events auftritt. Bei den Request Opened/removed events muss man aufpassen, da jede variablen änderung geloggt wird. Demnach sind für den selben elapsedTimeTimestamp nur die Anfangs und Endzustände der jeweiligen Slots relevant.



UnitEyeLog (-> Standard Implementation) + 
 4 areas of interest definierst: links das panel mit den requests, das video mit der aktuellen Szene (ego view),  die szene top down view, und noch das Panel mit den Gründen unten
ist im OverviewLog

\bibliography{cubebib}

\appendix
\onecolumn
\section{Experimental Outcomes}
\label{sec:appOut}
In this appendix we provide the plots for \cref{sec:model_eval}.
Firstly, \cref{fig:worst-case} contains the plots of worst-case deviation with and without the use of the linear program, and  \cref{fig:ptp} plots the peak-to-peak difference. In \cref{fig:ptp} whenever our algorithm proves ``overwhelming'', the point is colored red and marked with a `x.'
\vspace{0pt}\nopagebreak
\begin{figure}[H]
    \centering
	\includegraphics[width=\textwidth, keepaspectratio, trim={0cm 0cm 0 1.5cm},clip]{figs/plots/w_analysis_lp_log_independent.pdf}
	\caption{The worst-case deviation for the model for the three different input restrictions and two different permutation classes.
    The $x$-axis is the number of tokens and the $y$-axis is the base-10 logarithm of the worst-case deviation.
    }
	\label{fig:worst-case}
\end{figure}
\pagebreak
\begin{figure}[H]
    \centering
	\includegraphics[width=\textwidth, keepaspectratio, trim={0cm 0cm 0 1.5cm},clip]{figs/plots/w_analysis_ptp_linear_independent.pdf}
	\caption{The peak-to-peak difference for the model for the three different input restrictions and two different permutation classes.
	Red ``x''s indicate that the worst-case deviation is less than half of the peak-to-peak difference and, thus, the model output is provably invariant over the permutation class.
        The $x$-axis is the number of tokens and the $y$-axis is the peak-to-peak deviation.
    }
	\label{fig:ptp}
\end{figure}

\subsection*{Overwhelming Through Generation}
In \cref{fig:contOverwhelm} we plot the bounds on worst-case deviation and the peak-to-peak difference as the model is used to continually generate text. The newly generated tokens are continuously added to the fixed string and fed again to the model to generate the next token. The model is ``overwhelmed'' whenever the worst-case deviation is less than the peak-to-peak difference in the plot.

\vspace{0pt}\nopagebreak
\begin{figure}[H]
    \centering
	\includegraphics[ width=\textwidth, keepaspectratio, trim={0cm 0cm 0 1.5cm},clip]{figs/plots/continued_gen.pdf}
	\caption{
            The peak-to-peak difference and the worst-case deviation when we continue generation through multiple tokens.
            The $x$-axis connotes the total number of tokens ($n_{ctx}$) throughout a generation.
            When the line for $PTP / 2$ is above the worst-case deviation line, then our algorithm provably guarantees that the output is fixed.}
	\label{fig:contOverwhelm}
\end{figure}

\section{Appendix for Model Details}
\label{sec:appendix_model}
We provide a formal definition of each of the components in the model used in this paper.
\begin{itemize}
	\item $\softmax$ is the softmax function
		\[
			\softmax(\vec{\alpha})[i] = \frac{e^{\vec{\alpha[i]}}}{\sum_{i=1}^{\dVocab} e^{\vec{\alpha}[i]}}
		\]
	\item $\relu$ is the rectified linear unit function
		\[
			\relu(x) = \max(0, x)
		\]
	\item $\LayerNorm$ is the layer normalization function which for matrix $X$ performs the following column wise function for columns of $X$ (i.e.\ $X[:, j]$):
		\[
			\LayerNorm(X[i, j]) = \frac{X[i, j] - \Expec[X[:, j]]}{\sqrt{\Var[X[:, j]] + \eps} } \cdot \gamma + \beta
		\]
		where $\gamma$ and $\beta$ are learned parameters and addition and division are element-wise.
	\item $\MLP$ is a one-layer feed-forward network with ReLU activation. The $\MLP$ is row wise of $X$:
		\[
			\MLP(X[i]) = \relu(X[i, :] \cdot A_{enc} + b_{enc}) \cdot A_{dec}
		\]
		Depending on the model there may be additional bias terms added after the ReLU activation.
		For simplicitily, we will only consider one bias term prior to the ReLU activation though the results of this paper can be easily be extended to two bias terms.
	\item $\RoPE$ \cite{su2024roformer} is the rotary position encoding which applies a rotation to key and query vectors. For input $X$ at row $i$, the rotation is applied as follows: 
		\[
			\RoPE(X)[i, 2j] = \cos(i\theta_j)X[i, {2j}] - \sin(i\theta_j)X[i, 2j + 1]
		\]
		and
		\[
			\RoPE(X)[i, 2j + 1] = \sin(i\theta_j)X[i, {2j}] + \cos(i\theta_j)X[i, 2j + 1]
		\]
		where $\theta_j = 10000^{-2j/\dEmb}$ is the frequency for dimension $j$.
		%\item $\attn$ is the attention mechanism: for matrix $X \in \R^{\nctx \times \dEmb}$, the attention mechanism with RoPE is
		%	\[
		%		\attn(X) = \softmax\left( \frac{(\RoPE(XQ, i))(\RoPE(XK, i))^T}{\sqrt{d}} \right) \cdot XV
		%	\]
		%	where $i$ is the position index for each row of $X$, and the softmax is applied column-wise.
	\item $\attnH$ is an attention head for matrix $X \in \R^{\nctx \times \dEmb}$, an attention head does the following:
		\[
			\attnH(X) = \softmax\left( \frac{\RoPE(X Q) \cdot \RoPE(K^T X^T)}{\sqrt{\dEmb / H}} \right) \cdot X V
		\]
		where the softmax is applied column-wise.
        As in Ref.~\cite{black2022gptneox20bopensourceautoregressivelanguage}, we can re-write the effect of RoPE via a rotation matrix $\Theta_{i, j}$ such that
        \[
            \left(\RoPE(X Q) \cdot \RoPE(K^T X^T)\right)[i, j] = \vece_i \cdot X Q \cdot \Theta_{i, j} \cdot  K^T  X^T \cdot \vece_j^T
        \]
	\item Often, we have a multi-head attention mechanism which is the concatenation of $H$ attention mechanisms along the last dimension:
		\[
			\attn(X) = [\attnH_1(X); \ldots; \attnH_H(X)]
		\]
		where $\attnH_h$ is the $h$-th attention mechanism as outlined
	\item $\Embed \in \R^{\dVocab \times \dEmb}$ is the embedding function which maps a one hot vector in $\R^\dVocab$ to a vector in $\R^{\dEmb}$.
	\item $\Unembed \in \R^{\dEmb \times \dVocab}$ is the unembedding function which maps a vector in $\R^{\dEmb}$ to a one hot vector in $\R^\dVocab$.
\end{itemize}



\section{Proofs for Worst-Case Deviation}
\label{sec:proofs_worst_case_deviation}

\begin{lemma}[Properties of the Worst-Case Deviation]
	\label{lemma:worst_case_deviation_properties}
	For any functions $f, g: \calX \to \R^n$, norm $p$ and lift to $\calY \times \calZ$, we have the following properties:
	\begin{itemize}[nosep]
		\item Triangle inequality for (lifted) $\WD$:	
			\[
				\WD(f + g; \calX)_p \leq \WD(f : \calX)_p + \WD(g : \calX)_p.
			\]
		\item Lifting monotonicity:
			\[
				\WD(f; \calX)_p \leq \WD(f; \calY \times \calZ)_p.
			\]
		\item Lipschitz composition:
			For function $g$
               \[
				\WD(g \circ f; \calX)_p \leq \Lip(g)_p \cdot \WD(f; \calX)_p.
			\]
		As a corollary, we have that for linear operators $A$,
		\[
			\WD(A f; \calX)_p \leq \norm{A}_p \cdot \WD(f; \calX)_p.
		\]
		as $\Lip(A)_p = \norm{A}_p$.

		\item $p$-norm bounds for $p, q \geq 1$ and $q > p$:
			\[
				\WD(f; \calX)_q \leq \WD(f; \calX)_p
			\]
	\end{itemize}
\end{lemma}


\begin{proof}[Proof of worst-case deviation properties, \cref{lemma:worst_case_deviation_properties}]
	\label{proof:worst_case_deviation_properties}
	We will prove each of the properties in turn.
	\begin{itemize}
		\item Triangle inequality:
			We can view the maximization over $\calX$ as occuring disjointly for $f$ and $g$:
			I.e. \begin{align*}
				\WD(f + g; \calX)_p 
			&\leq
            \sup_{X_1, X_2 \in \mathcal{X}} \norm{
			f(X_1) + g(X_1) - (f(X_2) + g(X_2))} \\
			&\leq \sup_{X_1, X_2, X_1', X_2' \in \mathcal{X}}  \big[\norm{f(X_1) - f(X_2)} 
            + \norm{g(X_1') - g(X_2')}\big] 
			\tag{by triangle inequality of norms}\\
			&\leq \WD(f ; \calX)_p + \WD(g ; \calX)_p
			\end{align*}
			as desired.
		\item Lifting monotonicity:
			The proof follows from the definition of the lifted worst-case deviation:
			\begin{align*}
				\WD(f ; \calX)_p &= \sup_{(Y_1, Z_1), (Y_2, Z_2) \in \calX \subseteq \calY \times \calZ} \norm{f(Y_1, Z_1) - f(Y_2, Z_2)}_p \\
						 &\leq \sup_{Y_1, Y_2 \in \calY, \; Z_1, Z_2 \in \calZ} \norm{f(Y_1, Z_1) - f(Y_2, Z_2)}_p \\
						 &= \WD(f ; \calY \times \calZ)_p.
			\end{align*}
		\item Lipschitz composition:
			We simply have that
			\begin{align*}
				&\WD(A f; \calX)_p = \sup_{X_1, X_2 \in \calX} \norm{A f(X_1) - A f(X_2)}_p \\
						  &= \sup_{X_1, X_2 \in \calX} \norm{A (f(X_1) - f(X_2))}_p \\
						  &\leq \sup_{X_1', X_2'} \frac{\norm{A(X_1') - A(X_2')}_p}{\norm{X_1' - X_2'}_p} \cdot \sup_{X_1, X_2 \in \calX} \norm{f(X_1) - f(X_2)}_p \\
						  &=\Lip(A)_p \cdot \WD(f; \calX)_p.
			\end{align*}
		\item $p$-norm bounds for $p, q \geq 1$ and $q > p$:
			Because we restrict $f$ to be a function which outputs vectors and $\norm{\vec{x}}_q \leq \norm{\vec{x}}_p$ for $q > p$, we have that for all $X_1, X_2 \in \calX$, $\norm{f(X_1) - f(X_2)}_q \leq \norm{f(X_1) - f(X_2)}_p$.
			So, if there exists $X_1, X_2 \in \calX$ such that $\norm{f(X_1) - f(X_2)}_q = \alpha$, then there must exist $X_1, X_2 \in \calX$ such that $\norm{f(X_1) - f(X_2)}_p \geq \alpha$.
			Thus, the supremum over $\calX$ for $q$ is less than or equal to the supremum over $\calX$ for $p$.
		\end{itemize}
	\end{proof}


\section{Proofs for Generic Framework}
\label{sec:proofs_framework}

\subsection{Proofs for Blowup and Shift Bounds}
We first need to bound the variance and expectation of the columns of $\Embed \cdot X$:
\begin{lemma}[Bounds on Expectation and Variance]
	\label{lem:boundsVar}
	Let $X \in \InpSpace$ be an input sequence.
	% Write $X = (s | X_{\text{free}} | q)$$
 %    \left(\bigtimes_{i = 1}^{\ndes} \{ \desSet_i\} \right) \times X_{\text{free}} \times \{\vecQuery\}$, where
	% $\desSet_i$ are the desired token sets,
	% $X_{\text{free}}$ consists of the possible free tokens, and
	% $\vecQuery$ is the query token.
	Then for any column $j$, the expectation $\mu_j$ of the $j$-th column of $\Embed \cdot X$ satisfies:
	\begin{align*}
		\frac{1}{\nctx} &\left(
			\sum_{i=1}^{\ndes} \Embed[\desSet_i] \vec{e}_j^T + 
			\nfree \min_{x \in [\dVocab]} (\Embed[x] \vec{e}_j^T) + 
			\Embed[q]\vec{e}_j^T
		\right)
		\leq \mu_j  \\
				&\leq
				\frac{1}{\nctx}\left(
					\sum_{i=1}^{\ndes} \Embed[\desSet_i]\vec{e}_j^T + 
					\nfree \max_{x \in [\dVocab]} (\Embed[x]\vec{e}_j^T) + 
					\Embed[q]\vec{e}_j^T
				\right)
		\end{align*}
		Let $\mu_j^{\min}$ and $\mu_j^{\max}$ denote the lower and upper bounds for the $j$-th column respectively.
		Then, we can bound the variance of the $j$-th column of $\vec{x} \Embed$ by
		\begin{align*}
			\frac{1}{\nctx} &\bigg[
				\min_{\mu' \in [\mu_j^{\min}, \mu_j^{\max}]}  \sum_{i = 1}^s \left(\Embed[\desSet_i] \vec{e}_j^T -  \mu' \right)^2 + \nfree \cdot \min_{x \in [\dVocab]} \left( \Embed[x] \vec{e}_j^T -  \mu' \right)^2 + \left( \Embed[q] \vec{e}_j^T -  \mu' \right)^2
			\bigg] \\
					&\leq \Var_j[(\Embed \cdot X)[:, j]] \leq \\
				\frac{1}{\nctx} &\bigg[
					\sum_{i = 1}^s \max_{\mu' \in [\mu_j^{\min}, \mu_j^{\max}]} \left(\Embed[\desSet_i] \vec{e}_j^T -  \mu' \right)^2 + \nfree \cdot \max_{x \in [\dVocab]} \left( \Embed[x] \vec{e}_j^T -  \mu' \right)^2 + \left( \Embed[q] \vec{e}_j^T -  \mu' \right)^2
				\bigg].
				\end{align*}
				We define $\Var_j^{\min}, \Var_j^{\max}$ to be the lower and upper bounds respectively.
			\end{lemma}
			\begin{proof}
				The bounds on $\mu_j$ follow as we minimize and maximize the contribution of each free row.
				The proof of for variance bounds follows similarly.
			\end{proof}

We now have a simple proof for \cref{lem:boundsBS} restated below:
\begin{lemma}[Blowup and Shift Bounds]
	We bound blowup and shift: for every $B_j \in \blowupSet_j$ and $S_j \in \shiftSet_j$
	\[
		\frac{\gamma}{\sqrt{\Var_j^{\max} + \eps}} \leq B_j \leq \frac{\gamma}{\sqrt{\Var_j^{\min} + \eps}}
	\]
	and
	\begin{align*}
	&\min\left(B_j^{\min} \mu_j^{\min}, B_j^{\min} \mu_j^{\max}, B_j^{\max}, \mu_j^{\min}, B_j^{\max} \mu_j^{\max}\right)
	\leq
	S_j\\
	&\leq
	\max\left(B_j^{\min} \mu_j^{\min}, B_j^{\min} \mu_j^{\max}, B_j^{\max}, \mu_j^{\min}, B_j^{\max} \mu_j^{\max}\right)
	\end{align*}
\end{lemma}
\begin{proof}
	Given the bounds on expectation and variance, $B^{\min}, B^{\max}$ follow trivially from the definition of blowup and shift sets.
	$S^{\min}$ (resp. $S^{\max}$) follow from a minimization (maximization) over the possible shifts given the blowup bounds.
\end{proof}

\subsection{Appendix for Bounding Worst-case Deviation of $\fenc$}
\label{sec:appConcrcase}

Recall the definition of blowup and shift sets from \cref{def:blowup_shift}.
			Let $B^{\min}, B^{\max}$ be the vectors $(B_1^{\min}, \dots, B_\dEmb^{\min})$ and $(B_1^{\max}, \dots, B_\dEmb^{\max})$ respectively.
			Define $S^{\min}, S^{\max}$ analogously.
			We restate \cref{lem:WD_fenc} for convenience:
			\begin{lemma}[Worst-case deviation of $\fenc$, \cref{lem:WD_fenc}]
				\label{lem:worst_case_deviation}
				Let $\vec{x} \in \InpSpace$.
				Then,
				\[
					\WD(\vece_{\nctx} \cdot \fenc; \InpSpace \times \blowupSet \shiftSet)_\infty \leq 
					\max_{t \in [\dVocab]} 
					\max_{j \in [\dEmb]}
					\max_{B \in [\vec{0}, B^{\max} - B^{\min}]}
					\left|X[t] \cdot \Embed \cdot \diag(B) \cdot \vec{e}_j\right| + \left|S_j^{\max} - S_j^{\min}  \right|
                    \]	
				where the inner maximum term can be computed via a simple linear program.
			\end{lemma}
			\begin{proof}
                    First, note that for $j \in [\dEmb]$,
				\begin{align*}
					\WD(\vece_{j}^T \cdot (\fenc)^T; \InpSpace \times \blowupSet \shiftSet)_\infty 
						&= \max_{t, B, S, B', S'} \left\|
						X[t] \cdot E \cdot \diag(B) + S - X[t] \cdot E \cdot \diag(B') - S'
						\right\|_\infty \\
						&\leq max_{t, B, S, B', S'}
						\left|\left(X[t] \cdot E \cdot (\diag(B) - \diag(B')) + S - S'\right) \cdot \vec{e}_j\right| \\
						&\leq max_{t, B, B'}  \left|\left(X[t] \cdot E \cdot (\diag(B) - \diag(B')) \right) \cdot \vec{e}_j\right| + |S_j^{\max} - S_j^{\min}|.
				\end{align*}		
				Note then that $\diag(B)_j - \diag(B')_j$ is constrained by $B^{\max}_j - B^{\min}_j$.
                We can then maximize over $j$ to get
                \begin{align*}
                    \WD(\vece_\nctx \cdot \fenc; \InpSpace \times \blowupSet \shiftSet)_\infty 
                    &\leq 
                    \WD(\fenc; \InpSpace \times \blowupSet \shiftSet)_\frobInf  \\
                    &\leq \max_{t, B, B'} \max_j  \left|\left(X[t] \cdot E \cdot (\diag(B) - \diag(B')) \right) \cdot \vec{e}_j\right| + |S_j^{\max} - S_j^{\min}|
				\end{align*}
                as desired.
			\end{proof}



\subsection{Proofs and Subalgorithms for Attention Bounds}
\begin{definition}[$\ell^{min}$, $\ell^{max}$] \label{def:lminlmax}
	We use $\ell^{\min}_i$ to denote 
	a worst-case lower bound on the smallest possible logit at the $i$-th position of the input to the softmax in model $\Model$. 
	Similarly, we use $\ell^{\max}_i$ to denote a worst-case upper bound on the largest possible logit at the $i$-th position of the input to the softmax in model $\Model$.
\end{definition}

We provide the algorithm to find the extremal values $\vec{\ell}^{\min}$ and $\vec{\ell}^{\max}$ in \cref{alg:lminlmax}.
At a high level, the algorithm computes upper and lower bounds for each position in the input sequence prior to the softmax.
The bounds make use of the upper and lower bounds on the blowup and shift sets, $\blowupSet \shiftSet$.


\begin{algorithm}[h] 
	\caption{
        Algorithm to find $\vec{\ell}^{\min}$ and $ \vec{\ell}^{\max}$.
        To find the minimums and maximums over $\blowupSet \shiftSet$, we use the point-wise upper and lower bounds on the blowup and shift sets alongside the bounds obtained from \cref{alg:bilin}.}
	\label{alg:lminlmax}
	\begin{algorithmic}
		\STATE {\bfseries Input:}  $\blowupSet\shiftSet, \Model$
		\STATE
		\FOR{$k \in [\nfix]$}
		\STATE Let \begin{align*}
		\vec{\ell}^{\min}_k = \frac{1}{\sqrt{\dEmb}} \min_{B, S  \in \blowupSet\shiftSet} (\vece_{q} E \cdot \diag(B) + S) \cdot Q \PosRot_{k, \nctx} 
		K^T \cdot  (\vece_{\desSet_k} E \cdot  \diag(B) + S)^T
		\end{align*}
		and 
		\[
		\vec{\ell}^{\max}_k = \frac{1}{\sqrt{\dEmb}}  \max_{B, S \in \blowupSet\shiftSet} (\vece_{q} E \cdot \diag(B) + S) \cdot Q \PosRot_{k, \nctx} K^T \cdot  (\vece_{\desSet_k} E \cdot  \diag(B) + S)^T
		\]
		\ENDFOR
		\STATE
		\FOR{$k \in (\nfix, \nctx)$}
		\STATE Let \[
		\vec{\ell}^{\min}_k = \frac{1}{\sqrt{\dEmb}} \min_{t \in [\dVocab]} \min_{B, S \in \blowupSet\shiftSet} (\vece_q E \cdot \diag(B) + S) \cdot Q \PosRot_{k, \nctx} K^T \cdot  (\vece_t E \cdot  \diag(B) + S)^T
		\]
		and 
		\[
		\vec{\ell}^{\max}_k = \frac{1}{\sqrt{\dEmb}} \max_{t \in [\dVocab]} \max_{B, S \in \blowupSet\shiftSet} (\vece_q E \cdot \diag(B) + S) \cdot Q \PosRot_{k, \nctx} K^T \cdot  (\vece_t E \cdot  \diag(B) + S)^T
		\]
		\ENDFOR
		\STATE
		\STATE Set the extremal values of the query token:
		\[
		\vec{\ell}^{\min}_\nctx = \frac{1}{\sqrt{\dEmb}} \min_{B, S \in \blowupSet\shiftSet} (\vece_{q} E \cdot \diag(B) + S) \cdot Q \PosRot_{k, \nctx} K^T \cdot  (\vece_{q} E \cdot  \diag(B) + S)^T
		\]
		and
		\[
		\vec{\ell}^{\max}_\nctx = \frac{1}{\sqrt{\dEmb}}  \max_{B, S \in \blowupSet\shiftSet} (\vece_{q} E \cdot \diag(B) + S) \cdot Q \PosRot_{k, \nctx} K^T \cdot  (\vece_{q} E \cdot  \diag(B) + S)^T
		\]
		\STATE
		\STATE Return $\vec{\ell}^{\min}, \vec{\ell}^{\max}$
	\end{algorithmic}
\end{algorithm}


\begin{algorithm}[h] 
	\caption{Simple algorithm to bound the minimum of $\vec{x} A \vec{y}^T$ for $X^{\min}_i \leq \vec{x}_i \leq X^{\max}_i$ and $Y^{\min}_i \leq \vec{y}_i \leq Y^{\max}_i$.
        To find the maximum, switch the minimum to maximum and vice versa.
    }
	\label{alg:bilin}
	\begin{algorithmic}
            \STATE \textbf{Input:} $A, X^{\min}, X^{\max}, Y^{\min}, Y^{\max}$
            \STATE
            \STATE $m \leftarrow 0$
            \FOR{$i = 1$ to $n$}
            \FOR{$j = 1$ to $m$}
            \IF{$A_{ij} > 0$}
            \STATE $m \leftarrow m + A_{ij} \cdot \max\left(X^{\min}_i Y^{\min}_j, X^{\max}_i Y^{\min}_j, X^{\min}_i Y^{\max}_j, X^{\max}_i Y^{\max}_j \right)$
            \ELSE
            \STATE $m \leftarrow m + A_{ij} \cdot  \min\left(X^{\min}_i Y^{\min}_j, X^{\max}_i Y^{\min}_j, X^{\min}_i Y^{\max}_j, X^{\max}_i Y^{\max}_j \right)$
            \ENDIF
            \ENDFOR
            \ENDFOR
            \STATE Return $m$
        \end{algorithmic}
\end{algorithm}

\begin{proof}[Proof sketch for \cref{lem:lminlmax} (correctness of \cref{alg:lminlmax})]
	In \cref{alg:lminlmax}, for each position $k$, we compute the minimum and maximum logit for the $k$-th position by maximizing over possible blowup and shift sets and input tokens for the free tokens.
	Moreover, \cref{alg:bilin} computes an upper and lower bound for the restricted bilinear form which \cref{alg:lminlmax} uses to compute the extremal values.

    The correctness of the bilinear bound in \cref{alg:bilin} follows from a series of relaxations when writing out the explicit formula for the bilinear multiplication.
    To prove an upper-bound, we have that 
    \begin{align*}
        m &= \sum_{i} x_i A_{i, j} y_j \\
        &\leq \sum_i \max (A_{i, j} X_i^{\min} Y_i^{\min},  A_{i, j} X_i^{\min} Y_i^{\max}, A_{i, j} X_i^{\max} Y_i^{\min}, A_{i, j} X_i^{\max} Y_i^{\max}).
    \end{align*}
    Then, if $A_{i, j}$ is negative, we can factor the coefficient out of the maximization and flip to a minimization. Otherwise, we can simply factor $A_{i, j}$ out of the maximization.
    Thus, \cref{alg:bilin} gives a valid upper-bound. The lower-bound follows analogously.
\end{proof}

We now prove \cref{lem:min_max_softmax}, restated here:

\begin{lemma}[Minimum and Maximum after Softmax]
	\begin{equation}
		\label{eq:softmax_upper}
		\ssMinFB \leq
		\sum_{j \in (\nfix, \nctx)} \softmax(\ell_j) \leq
		\ssMaxFB
	\end{equation}
	aswell as,
	\begin{align}
		\label{eq:softmax_lower}
		\ssMinRB \leq
		\sum_{j \in [\nfix] \cup \{\nctx\}} \softmax(\ell_j) \leq
		\ssMaxRB.
	\end{align}
\end{lemma}
\begin{proof}[Proof of \cref{lem:min_max_softmax}]
	We will prove the right hand side of \cref{eq:softmax_upper} and the left hand side of \cref{eq:softmax_lower} as the other follow by the same proof.
	Note the second inequality follows from the first inequality and the fact that the sum of the attention weights is $1$.

	Now, we will show the first inequality.
	Clearly, smaller values of $\vec{\ell}_i$ results in larger value of $\softmax(\vec{\ell}_j)$.
	Then, let $\vec{\eta}$ be a vector where $\sum_{j \in (\ndes, \nctx]} \softmax(\vec{\eta}_j) > \sum_{j \in (\ndes, \nctx]} \softmax(\vec{\ell}_j)$.
	Then, we have that
	\begin{align*}
		0 &\leq \left[\left(\sum_j e^{\vec{\eta}_j}\right) - \left(\sum_j e^{\vec{\ell}_j}\right)\right] \left(\sum_i e^{\vec{\ell}_i}\right)  \\
		\Rightarrow &\left(\sum_j e^{\vec{\eta}_j} \right)\left(\sum_i e^{\vec{\ell}_i} + \sum_j e^{\vec{\ell}_j} \right) \leq
        \left(\sum_j e^{\vec{\ell}_j}\right) \left(\sum_i e^{\vec{\eta}_i} + \sum_j e^{\vec{\eta}_j}\right) \\
		\Rightarrow & \frac{\sum_j e^{\vec{\eta}_j}}{\sum_i e^{\vec{\ell}_i} + \sum_j e^{\vec{\eta}_j}} \leq 
        \frac{\sum_j e^{\vec{\ell}_j}}{\sum_i e^{\vec{\ell}_i} + \sum_j e^{\vec{\ell}_j}}.
	\end{align*}
\end{proof}

\newcommand{\free}{\text{free}}
\newcommand{\fix}{\text{fix}}
Finally, we prove the bound on worst-case deviation of attention (\cref{lem:att_bound}):
\begin{lemma}
	Worst-case deviation of attention is bounded as follows,
	\begin{align*}
		\WD(\vecNctx \cdot f^{\attnH}_{\ndes \mid r, q} ; \InpSpace \times \blowupSet \shiftSet)_\infty
	     &\leq
	     2 \cdot (\ssMaxRB - \ssMinRB) \cdot \max_{B, S}\norm{\Embed[[\nfix] \cup \{\nctx\}, :] \cdot \diag(B) + S}_{\frobInf} \\
		&+ 2 \cdot \ssMaxFB \cdot \max_{B, S} \norm{(E \cdot \diag(B) + S) \cdot V}_{\frobInf}
        + \ssMaxFB  \norm{V}_\infty \cdot \max_{t \in s \cup \{q\}} \WD(\fenc; \{\vece_t\} \times \blowupSet \shiftSet\}))_\infty
	\end{align*}
        where $V$ is the value matrix in the attention head (see \cref{sec:appendix_model}).
\end{lemma}
\begin{proof}[Proof of \cref{lem:att_bound}]
	\label{proof:att_bound}
	Define $\vec{p}(X) \in \R^{\nctx}$ as the probability vector for the query token post-softmax.
	I.e.
	\[
		\vec{p}_j = \softmax(\vec{\ell}_j) \text{ for } j \in [\nctx]
	\]
	where \[
		\vec{\ell} = \frac{\vecNctx \cdot(Y \cdot \RoPE(Q) \RoPE(K^T) Y^T}{\sqrt{\dEmb}}
	\]
	for $Y = X \Embed \cdot \diag(B(X)) + S(X)$.
	Moreover, let $\vec{p}(X)_{\free} = \vec{p}(X)[(\nfix, \nctx)]$ (i.e.\ the values corresponding to the free tokens) and $\vec{p}(X)_{\fix} = \vec{p}(X)[[\nfix] \cup \{\nctx\}]$ (the values corresponding to the fixed and query token).
	Now, we re-write the worst-case deviation of the attention head: for $X, X' \in \InpSpace$, $(B, S), (B', S') \in \blowupSet \shiftSet$,
	\begin{align*}
		\WD&(\vecNctx \cdot f^{\attnH}_{\ndes \mid r, q} ; \InpSpace \times \blowupSet \shiftSet)_\infty
			\leq  \max_{X, X', (B, S), (B', S')} 
			\norm{\vec{p}(X) \cdot (X \Embed \cdot \diag(B) + S) V - \vec{p}(X') \cdot (X \Embed \cdot \diag(B') + S') V}_\infty \tag{by definition of an attention head} \\
		   &\leq \max_{X, X', (B, S), (B', S')} \bigg\|\vec{p}(X)_\fix \cdot (X[[\nfix] \cup \{\nctx\}:, ] \Embed \cdot \diag(B) + S) V + \vec{p}(X)_\free \cdot (X[(\nfix, \nctx):, ] \Embed \cdot \diag(B) \\ &+ S) V - \vec{p}(X')_\fix \cdot (X'[[\nfix] \cup \{\nctx\}:, ] \Embed \cdot \diag(B') + S') V - \vec{p}(X')_\free \cdot (X'[(\nfix, \nctx):, ] \Embed \cdot \diag(B) + S) V \bigg\| \\
		   &\leq \max_{X, X', (B, S), (B', S')} \bigg\|\vec{p}(X)_\fix \cdot (X[[\nfix] \cup \{\nctx\}:, ] \Embed \cdot \diag(B) + S) V - \vec{p}(X')_\fix \cdot (X'[[\nfix] \cup \{\nctx\}:, ] \Embed \cdot \diag(B') \\ &+ S') V\bigg\| + \bigg\|\vec{p}(X)_\free \cdot (X[(\nfix, \nctx):, ] \Embed \cdot \diag(B) + S) V - \vec{p}(X')_\free \cdot (X'[(\nfix, \nctx):, ] \Embed \cdot \diag(B) + S) V \bigg\|.
		   \tag{by triangle inequality} \\
	\end{align*}
	Now, we will bound the two norms in the above equation separately.
	First, note that $\ssMinFB \leq \sum \vec{p}(X)_\free \leq \ssMaxFB$ and $\ssMinRB \leq \sum \vec{p}(X)_\fix \leq \ssMaxRB$ by definition of the extremal values (\cref{def:soft-extrem-values}).
	For the case of the free tokens,
	\begin{align*}
		&\max_{X, X', B, B', S, S'} \bigg\|\vec{p}(X)_\free \cdot (X[(\nfix, \nctx):, ] \Embed \cdot \diag(B) + S) V - \vec{p}(X')_\free \cdot (X'[(\nfix, \nctx):, ] \Embed \cdot \diag(B) + S) V \bigg\|_\infty
		\\
		&\leq 2 \cdot \max_{X, B, S} \bigg\|\vec{p}(X)_\free \cdot (X[(\nfix, \nctx):, ] \Embed \cdot \diag(B) + S) V \bigg\|_\infty \tag{by triangle inequality} \\
		&\leq 2 \cdot \max_{B, S} \sum_{i \in (\nfix, \nctx)} \vec{p}(X)_\free[i]  \cdot \left\| (X[i, :] \Embed \cdot \diag(B) + S) V \right\|_\infty \tag{by  triangle inequality} \\
		&\leq 2 \cdot \max_{B, S} \max_i \ssMaxFB \cdot \left\|(X[i, :] \Embed \cdot \diag(B) + S) V \right\|_\infty \tag{by definition of $\vec{p}(X)_\free$} 
	\end{align*}
	where the last inequality follows from the fact that $\vec{p}(X)_\free$ is non-negative and the sum is at most $\ssMaxFB$.
	Finally, note that $\max_i \ssMaxFB \| X[i, :] \Embed \cdot \diag(B) + S) V\| \leq \ssMaxFB\max_{t \in [\dVocab]} \|\Embed[t, :] \cdot (\diag(B) + S) V\|$ as the maximum free token contribution is at most the maximum contribution from any possible token.
	Finally, for the free tokens, we get a bound
	\[
		2 \cdot \ssMaxFB \cdot \max_{B, S, t \in [\dVocab]} \norm{(\Embed[t] \cdot \diag(B) + S) V}_\infty 
		= 2 \cdot \ssMaxFB \cdot \max_{B, S} \norm{(\Embed \cdot \diag(B) + S) V}_\frobInf.
	\]
	We now bound the fixed tokens contribution:
	\[
		\max_{X, X', (B, S), (B', S')} \bigg\|\vec{p}(X)_\fix \cdot (X[[\nfix] \cup \{\nctx\}:, ] \Embed \cdot \diag(B) + S) V - \vec{p}(X')_\fix \cdot (X'[[\nfix] \cup \{\nctx\}:, ] \Embed \cdot \diag(B') + S') V\bigg\|
	\]
	We start by noting that the fixed tokens must be an element of $\desSet \cup \{q\}$ and are, by definition, fixed for all $X$
	And so,
	\begin{align*}
		&\max_{X, X', (B, S), (B', S')} \bigg\|\vec{p}(X)_\fix \cdot (X[[\nfix] \cup \{\nctx\}:, ] \Embed \cdot \diag(B) + S) V - \vec{p}(X')_\fix \cdot (X'[[\nfix] \cup \{\nctx\}:, ] \Embed \cdot \diag(B') \\ &+ S') V\bigg\| = \max_{X, X', (B, S), (B', S')} \bigg\|(\vec{p}(X)_\fix \cdot (\Embed[s \cup \{q\}, :] \cdot \diag(B) + S) V - \vec{p}(X')_\fix \cdot (\Embed[s \cup \{q\}, :] \cdot \diag(B') + S') V\bigg\| \\
		&\leq \max_{X, X', (B, S), (B', S')} \sum_i \bigg\|\vec{p}(X)_\fix[i] \cdot (\Embed[t_i, :] \cdot \diag(B) + S) V - \vec{p}(X')_\fix[i] \cdot (\Embed[t_i, :] \cdot \diag(B') + S') V\bigg\| \tag{by the triangle inequality}
	\end{align*}
	where $t_i$ is the $i$-th fixed token.
	Now, we want to re-write the last line to factor out the differing blowup and shift values.
	\begin{align*}
		&\max_{X, X', (B, S), (B', S')} \sum_i \bigg\|\vec{p}(X)_\fix[i] \cdot (\Embed[t_i, :] \cdot \diag(B) + S) V - \vec{p}(X')_\fix[i] \cdot (\Embed[t_i, :] \cdot \diag(B') + S') V\bigg\| \\
		&= \max_{X, X', (B, S), (B', S')} \sum_i \bigg\|\vec{p}(X)_\fix[i] \cdot (\Embed[t_i, :] \cdot \diag(B) + S) V \\
		-& \vec{p}(X')_\fix[i] \cdot (\Embed[t_i, :] \cdot (\diag(B) + S - (\diag(B) + S - \diag(B') - S')) V\bigg\| \\
		 &\leq \max_{X, X', (B, S)} \sum_i \bigg\|\vec{p}(X)_\fix[i] \cdot (\Embed[t_i, :] \cdot \diag(B) + S) V - \vec{p}(X')_\fix[i] \cdot (\Embed[t_i, :] \cdot \diag(B) + S) V\bigg\| \\
		+& \max_{X, (B, S), (B', S')} \sum_i \vec{p}(X)_\fix[i] \cdot \bigg\| (\Embed[t_i, :] \cdot \diag(B) + S) V - (\Embed[t_i, :] \cdot \diag(B') + S') V\bigg\| \tag{By the triangle inequality}
	\end{align*}
	We now bound the two terms separately.
	For the second term, we have that
	\begin{align*}
		&\max_{X', (B, S), (B', S')} \sum_i \vec{p}(X')_\fix[i] \cdot \bigg\| (\Embed[t_i, :] \cdot \diag(B) + S) V - (\Embed[t_i, :] \cdot \diag(B') + S') V\bigg\| \\
		&\leq \max_{(B, S), (B', S')} \max_i \ssMaxRB \cdot \bigg\| (\Embed[t_i, :] \cdot \diag(B) + S) - (\Embed[t_i, :] \cdot \diag(B') + S')\bigg\| \cdot \norm{V}_\infty \tag{By definition of $\vec{p}(X')_\fix$} \\
		&\leq \norm{V}_\infty \cdot \ssMaxRB \max_{t \in s \cup \{q\}} \WD(\fenc; \{\vece_{t}\} \times \blowupSet \shiftSet).
	\end{align*}

	For the first term, we get
	\begin{align*}
		&\max_{(B, S), X, X'} \sum_i \bigg\|\vec{p}(X)_\fix[i] \cdot (\Embed[t_i, :] \cdot \diag(B) + S) V - \vec{p}(X')_\fix[i] \cdot (\Embed[t_i, :] \cdot \diag(B) + S) V\bigg\| \\	
		&= \max_{(B, S), X,X'} \sum_i \bigg\|\left(\vec{p}(X)_\fix[i] - \vec{p}(X')_\fix[i]\right) \cdot (\Embed[t_i, :] \cdot \diag(B) + S) V \bigg\|\\
		&= \max_{(B, S), X, X'} \sum_i \left(\vec{p}(X)_{\fix}[i] - \vec{p}(X')_{\fix}[i]\right) \cdot \bigg\|(\Embed[t_i, :] \cdot \diag(B) + S) V \bigg\|\\
		&\leq \max_{(B, S), i} (\ssMaxFB - \ssMinFB) \cdot \bigg\|(\Embed[t_i, :] \cdot \diag(B) + S) V \bigg\|
	\end{align*}
	where the last inequality follows from the fact that $\vec{p}(X)_\fix[i] - \vec{p}(X')_\fix[i]$ is at most $\ssMaxFB - \ssMinFB$.
	Then, by definition of the $\frobInf$ norm, we have that
	\[
		\max_{(B, S), i} (\ssMaxFB - \ssMinFB) \cdot \bigg\|(\Embed[t_i, :] \cdot \diag(B) + S) V \bigg\| = (\ssMaxFB - \ssMinFB) \cdot \max_{B, S} \norm{(\Embed[s \cup \{q\}, :] \cdot \diag(B) + S) V}_\frobInf.
	\]

	Putting the above together, we get the desired bound:
	\begin{align*}
		\WD(\vecNctx \cdot f^{\attnH}_{\ndes \mid r, q} ; \InpSpace \times \blowupSet \shiftSet)_\infty
	     &\leq
	     (\ssMaxRB - \ssMinRB) \cdot \max_{B, S}\norm{\Embed[\desSet \cup \{q\}, :] \cdot \diag(B) + S}_{\frobInf} \\
	     &+ 2 \ssMaxFB \cdot \max_{B, S} \norm{(E \cdot \diag(B) + S) \cdot V}_{\frobInf}.
	     + \norm{V}_\infty \ssMaxRB \max_{t \in s \cup \{q\}} \WD(\fenc; \{\vece_{t}\} \times \blowupSet \shiftSet).
	\end{align*}

\end{proof}


\subsection{Proofs for MLP Bounds}
\begin{proof}[Proof of \cref{lem:mlpbound}]
	\label{proof:mlp_bound}
	The second statement follows from $\vec{e}_\nctx \cdot f^{\Iden}_{\ndes \mid r, q} = \fenc$ directly.
	Then, the first statement follows from the Lipschitz constant of the MLP:
	\begin{align*}
		\norm{\MLP(\vec{e}_\nctx \Embed B + S) - \MLP(\vec{e}_\nctx \Embed B' + S')}_\infty
			&\leq \LipMLP_\infty \cdot \norm{\vec{e}_\nctx \Embed B + S - \vec{e}_\nctx \Embed B' + S'}_\infty \\ &\leq \LipMLP_\infty \cdot \WD(\vec{e}_\nctx \cdot  \fenc; \InpSpace \times \blowupSet \shiftSet)_\infty.
	\end{align*}
\end{proof}

\subsection{Proof of \cref{thm:InpRes}}
\label{subsec:InpResProof}
First we will restate \cref{thm:InpRes} for convenience:
\begin{theorem}
	%The model $\desF{\ModelFinal}$ over domain $[X]$ if 
	If
	\begin{align*}
	    &\WD(\desF{\Model}; \InpSpace)_\infty
	  \\&< \peakToPeak(\desF{\Model}, X) / 2,
	\end{align*}
	then the output of model $\desF{\Model}$ is fixed for all inputs in $\InpSpace$.
	Moreover, we can use \cref{alg:overwhelmCheckDet} to produce an upper bound $W$ for $\WD(\desF{\Model}; \InpSpace)_\infty$.
\end{theorem}
\begin{proof}
	We first make use of \cref{thm:metathm} to prove the first statement of the above theorem.
	Now, we just need to prove that the bound, $W$ in \cref{alg:overwhelmCheckDet}, is a valid bound on the lift $\WD(\desF{\Model}; \InpSpace \times \blowupSet \shiftSet)_\infty$.
	By the monotonicity of lifting, we then have that $\WD(\desF{\Model}; \InpSpace)_\infty \leq \WD(\desF{\Model}; \InpSpace \times \blowupSet \shiftSet)_\infty.$

	First note that, by \cref{lemma:worst_case_deviation_properties}, we have that
	\begin{align*}
		\WD(\desF{\Model}; \InpSpace \times \blowupSet \shiftSet)_\infty &\leq \Lip(\Unembed)_\infty \cdot (\WD(f^{\attnH}; \InpSpace \times \blowupSet \shiftSet)_\infty \\ &+ \WD(f^{\MLP}; \InpSpace \times \blowupSet \shiftSet)_\infty + \WD(f^{\Iden}; \InpSpace \times \blowupSet \shiftSet)_\infty) 
	\end{align*}
	where
	\[
		f^{\component}(X, (B, S)) = \component \circ \fenc(X, (B, S)) 
	\]
	and
	\[
		\fenc(X, (B, S)) = (X \cdot \Embed \cdot \diag(B) + S).
	\]
	Then, we can use \cref{lem:att_bound} to get that $W^\attn$ is a valid bound on the worst-case deviation of $f^{\attnH}$.
	Then, we use \cref{lem:mlpbound} to get that $\Lip(\MLP)_\infty \cdot \WD(\fenc; \InpSpace \times \blowupSet\shiftSet)$ is a valid bound on the worst-case deviation of $f^{\MLP}$.
	Finally, we use \cref{lem:WD_fenc} to get the bound on $\fenc$.
\end{proof}
%\subsection{Lipschitz Constant of the Later Layers}
%\label{subsec:later_layer_lip}
%
%\subsubsection{Lipschitz Constant of Query and Key}
%\label{subsubsec:query_key_lip}
%
%\lev{TODO check, and cite for quad form
%also use picture on this medium post %https://medium.com/@ngiengkianyew/understanding-rotary-positional-encoding-40635a4d078e
%for rotary encodings explanation...
%}
%\begin{lemma}
%	\label{lem:quad_form_lip}
%	Let $f(x,y) = x^T A y$ where $A = Q \cdot \diag(R(\theta_i)) \cdot K^T$ and $R(\theta_i)$ are 2×2 rotation matrices.
%	For all $p$-norms with $p \geq 2$, the Lipschitz constant of the query and key matrices is bounded by \lev{TODO fix}:
%	$$\|f(x_1,y_1) - f(x_2,y_2)\| \leq \|A\|(\|x_1-x_2\|\|y_1\| + \|x_2\|\|y_1-y_2\|)$$
%	with $\|A\| \leq \|Q\|\|K\|$.
%\end{lemma}
%\begin{proof}
%	Consider the difference of bilinear forms $f(x_1,y_1) - f(x_2,y_2) = x_1^TAy_1 - x_2^TAy_2$.
%	This can be rewritten as $(x_1^T-x_2^T)Ay_1 + x_2^TA(y_1-y_2)$ by adding and subtracting $x_2^TAy_1$.
%	Applying the triangle inequality and submultiplicativity of norms yields \[
%		\|f(x_1,y_1) - f(x_2,y_2)\| \leq \|A\|(\|x_1-x_2\|\|y_1\| + \|x_2\|\|y_1-y_2\|).
%	\]
%	The operator norm of $A$ can be bounded as \[
%		\|A\|_p = \|Q \cdot \diag(R(\theta_i)) \cdot K^T\| \leq \|Q\|\|\diag(R(\theta_i))\|\|K\| \leq \|Q\|\|K\|
%	\]
%	where the last inequality follows from the fact that rotation matrices have norm at most 1 for all $p \geq 2$.
%	\lev{TODO: specify $p \geq 2$}
%\end{proof}
%
%%\begin{lemma}
%%	\lev{TODO: check rotation}
%%	Let $f(x) = x^T A x$ where $A = Q \cdot \text{diag}(R(\theta_i))  \cdot K^T$ and $R(\theta_i)$ are 2×2 rotation matrices. Then the Lipschitz constant $\Lip(f)_p$ for norm $1 \leq p \leq \infty$ of $f$ is bounded by:
%%	\[
%%		\Lip(f)_p \leq 2\norm{Q}_p \norm{K^T}_p
%%	\]
%%	where $\|\cdot\|$ denotes the operator norm.
%%\end{lemma}
%%
%%\begin{proof}
%%	First, recall that for any differentiable function $f$, the Lipschitz constant $L$ is given by:
%%	$$L = \sup_{x \neq y} \frac{\|\nabla f(x) - \nabla f(y)\|_p}{\|x-y\|_p}$$
%%	For quadratic forms, this equals: \lev{TODO: citation here}
%%	$$L = \sup_{\|x\|=1} \|\nabla f(x)\|$$
%%	So, we can then calculate the gradient:
%%	\[
%%		\nabla f(x) = (A + A^T)x = (Q \cdot \text{diag}(R(\theta_i))\cdot K^T + K\cdot \text{diag}(R(\theta_i)^T)\cdot Q^T)x
%%	\]
%%	For any rotation matrix $R(\theta)$:
%%	\begin{align*}
%%		R(\theta) &= \begin{bmatrix} \cos(\theta) & -\sin(\theta) \\ \sin(\theta) & \cos(\theta) \end{bmatrix} \\
%%	\end{align*}
%%	which implies that $\|R(\theta)\|_p \leq 1$.
%%	Also, by the submultiplicativity of operator norms:
%%	$$\|A\| = \|Q \cdot \text{diag}(R(\theta_i)) \cdot K^T\| \leq \|Q\| \|\text{diag}(R(\theta_i))\| \|K^T\|$$
%%	Finally, since $\|\text{diag}(R(\theta_i))\| = \max_i \|R(\theta_i)\| \leq 1$,
%%	we get that $\|A\| \leq \|Q\| \|K\|$.
%%	Therefore,
%%	\[
%%		L = \|\nabla f\| = \|A + A^T\| \leq \|A\| + \|A^T\| = 2\|A\| \leq 2\|Q\| \|K\|.
%%	\]
%%\end{proof}
%%

\section{Appendix for Permutation Invariance}
\label{sec:appendix_perm}

\begin{algorithm}[tb] 
	\caption{Naive Algorithm to find $\ssMinFA$.\\
		To find $\ssMaxFA$, switch the $\min$ to a $\max$ in the objective.
	}
	\label{alg:naiveAlpha}
	\begin{algorithmic}
	\STATE \textbf{Input:} {$\fenc$ and model $\Model$}
	% \Output{\lev{TODO}}
	Let $\fenc(\vec{e}) = \vec{e} \cdot \Embed \cdot \diag(B) + S$ \\
	\FOR{$k \in (\ndes, \nctx]$}
	\STATE {
		Set \begin{align*}
			\vec{\ell}^{\min}_k = \frac{1}{\sqrt{\dEmb}} \min_{t \in \freeToks} 
			\fenc(\vecQuery) \cdot Q \PosRot_{k, \nctx} K^T \cdot \fenc(\vec{e}_t)^T
		\end{align*}
	}
	\ENDFOR
	\STATE Return $\ssMinFA = \sum_k \exp\left(\vec{\ell}^{\min}_k\right)$
	\end{algorithmic}
\end{algorithm}

\subsection{Proof of Attention Worst-Case Deviation}
In this subsection, we provide a proof of \cref{lem:corrAlpha} and \cref{lem:attn_perm}.
First, we prove the correcntess of the algorithms to find $\ssMinFA$ and $\ssMaxFA$.
\begin{lemma}[Restatement of \cref{lem:corrAlpha}]
	 Both \cref{alg:LPAlpha} and \cref{alg:naiveAlpha} provide valid bounds on $\ssMinFA$ and $\ssMaxFA$.
\end{lemma}
\begin{proof}
	We provide a proof for $\ssMinFA$; the proof for $\ssMaxFA$ is analogous.

	First note that the blowup and shift are fixed and thus $\fenc$ is only a function of the its input token: i.e.\ we can rewrite $\fenc(\vec{e}_t, BS)$ as $\fenc(\vec{e}) = \vece \cdot E \cdot \diag(B) + S$ for fixed $B$ and $S$.
	Then, for the naive algorithm, \cref{alg:naiveAlpha}, the correctness follows from the fact that the minimum possible logit value from each position can be calculated by enumerating over all the free tokens.

	For the LP-based algorithm, \cref{alg:LPAlpha}, we first consider the LP as an integer linear program.
	Consider the permutation matrix $X$ with entries $x_{j, t}$, where $x_{j, t} = 1$ if token $t$ is selected for position $j$ and $x_{j, t} = 0$ otherwise.
	Then, the LP can be seen as finding the permutation matrix which minizes the objective function, $\ssMinFA$.
	So, the relaxed LP is a lower bound on the integer LP which is a lower bound on the true value of $\ssMinFA$.
\end{proof}

We now restate the lemma for the worst-case deviation of attention.
\begin{lemma}
	Worst-case deviation of attention is bounded as follows for fixed blowup $B$ and shift $S$ when considering the permutation class $[X]$ with free tokens $\freeToks$:
	\begin{align*}
		&\WD(\vecNctx \cdot f^{\attnH}_{\ndes \mid r, q} ; \InpSpace \times \blowupSet \shiftSet)_\infty
	     \\ &\leq
		2 \cdot (\ssMaxRB - \ssMinRB) \cdot \norm{\Embed[[\nfix] \cup \{\nctx\}, :] \cdot B + S}_{\frobInf} \\
		&+ 2 \cdot \ssMaxFB \cdot \norm{(E[(\nfix,\nctx),: ] B + S) \cdot V}_{\frobInf}.
	\end{align*}
\end{lemma}
\begin{proof}
	The proof follows analogously to the proof of \cref{lem:att_bound} (bound on worst-case deviation of attention in the generic case).
	The main difference is that we have a singleton set for the blowup and shift and that we restrict the free tokens.
	First, this implies that the restriction of $\blowupSet, \shiftSet$ to singelton sets means that no optimization over blowup and shift are needed in the bounds as the blowup and shift become constants.
	Second, the restriction of the free tokens means that the worst-case contribution from the value function, $(E \diag(B) + S \cdot V)$, is restricted to the embeddings selected by the free tokens.
	So, we replace $\norm{E \diag(B) + S \cdot V}_{\frobInf}$ with $\norm{(E[(\nfix,\nctx),: ] \diag(B) + S) \cdot V}_{\frobInf}$.
	Moreover, $\WD(\fenc; \{\vece_t\} \times \blowupSet \shiftSet)_\infty = 0$ as the blowup and shift are fixed and so we can drop the last term in the bound in \cref{lem:att_bound}.
\end{proof}

\subsection{Proof of \cref{thm:InpResPermInv}}
\label{subsec:proofInpResPermInv}
We first restate the theorem for convenience.
\begin{theorem}
	If
	\[
		\WD(\desF{\Model}; [X])_\infty < \peakToPeak(\desF{\Model}, X) / 2
	\]
	then the output of model $\desF{\Model}$ is fixed for all inputs in $[X]$.
	Moreover, we \cref{alg:overwhelmCheck2} in \cref{sec:appendix_perm} produces an upper bound $W$ for $\WD(\desF{\Model}; \InpSpace \times \blowupSet \shiftSet)_\infty$.
\end{theorem}
\begin{proof}
	The proof follows analogously to the proof of \cref{thm:InpRes} in \cref{subsec:InpResProof}.
	The key difference is that, due to the fixing of the blowup and shift, the bound on $W$ simplifies.
	First, the worst-case deviation of $\vece_{\nctx} \cdot \fenc(X)$ is $0$ as the blowup and shift are fixed.
	So, the worst-case deviation of the MLP and identity function are $0$ as well.
	Then, algorithm uses either \cref{alg:naiveAlpha} or \cref{alg:LPAlpha} to find bounds on the extremal values of attention with provable correctness as per \cref{lem:corrAlpha}.
	Next, the algorithm simplified worst-case deviation of attention (as per \cref{lem:attn_perm}) to bound the worst-case deviation of attention.
	Finally, the algorithm uses the Lipschitz constant of the unembedding function, invoking \cref{lemma:worst_case_deviation_properties}, to bound the worst-case deviation of the model.
\end{proof}

\section{Overwhelming for $\nctx \to \infty$}
\label{sec:convergence}

In this section, we will consider a specific set $\desSet = \bigtimes_{\nfix} \vecRep$ and $\query = \vecRep$ for some fixed $r \in [\dVocab]$.
In words, we will consider the fixed input to be one repeated token.
We term this a ``repetition restriction.''

Moreover, it will be useful to define the set of all possible inputs under a repetition restriction.
\begin{definition}[Repetition Space]
	\label{def:RepSpace}
	Let $\RepSpace \subset \OneHotSpace^\nctx$ be the set of all possible inputs under a repetition restriction.
	That is, 
	\[
		\RepSpace = \left\{ X \in \R^{\nctx \times \dVocab} \mid X = 
		\begin{bmatrix}
			\vecRep^T \\
			\vecRep^T \\
			\vdots \\
			\vecRep^T \\
			Y \\
			\vecQuery
		\end{bmatrix}, Y \in \OneHotSpace^{\nfree}
		\right\}.
	\]
\end{definition}

For simplicity, we will also not consider positional encodings in this section. 
I.e.\ we remove the use of RoPE in the attention mechanism.
Though as Ref.~\cite{barbero2024transformers} pointed out, rotary positional encodings \cite{su2024roformer} converge to providing zero information as $\nctx \to \infty$.

\begin{theorem}[Asymptotic Convergence to a Fixed Model]
	\label{thm:convergence}
	If $\frac{\nfree}{\nctx} \in o(1)$ as a function of $\nctx$, then the repetition restriction converges to a fixed model if $\peakToPeak(\Model, X)$ is positive for $X = \vece_r$.
\end{theorem}

To prove this theorem, we will need to (a) find a way to compute a $\peakToPeak$ function of the model and input \emph{independent of} $\nctx$ and (b) use the framework of \cref{sec:meta_framework} to individually bound the worst-case deviation for each component of the model.

\subsection*{Computing Peak-to-Peak Difference}
Given that $\peakToPeak$ is computed by evaluating the model on a single input, finding a $\peakToPeak$ value is normally a simple task.
But, in the asymptotic case, we need to ``shortcut'' the computation of $\peakToPeak$ for an $X \in \RepSpace$.
We can simply do this by setting all the free tokens to the repetition token: $X = {\vecRep}^{\nfree}$.

\begin{lemma}[Shortcut for $\peakToPeak$]
	\label{lem:gap_shortcut}
	We can compute $\peakToPeak(\Model, X)$ in time $O(1)$ for $X=\vece_\rep^{\nctx}$ even as $\nctx \rightarrow \infty$.
\end{lemma}
\begin{proof}
	We can note that $\E[X] = \vecDes \cdot \Embed$ and $\Var[X] = \Var[\vecDes \cdot \Embed] = 0$.
	So, we can easily compute blowup, $B$, and shift, $S$.
        Note, for attention head $\vece_\nctx \cdot \desF{f^\attnH}$, the output equals 
        \[
            \vec{p}(X) \cdot (XE \cdot \diag(B) + S) \cdot V
        \]
        for some probability vector $\vec{p}(X)$.
        Because $X = \vecRep^\nctx$, $(XE \cdot \diag(B) + S) \cdot V= \vec{a}^\nctx$ for vector $\vec{a} = (\vecRep \cdot E \cdot \diag(B) + S) V$.
        Thus,
        $$
            \vec{p}(X) \cdot (XE \cdot \diag(B) + S) \cdot V = \vec{a}.
        $$
	We therefore get that the output of $\vece_\nctx \cdot \desF{f^\attnH}$ is fixed to $\vecRep (\Embed \cdot \diag(B) + S) \cdot V$ for all context windows $\nctx$.
	Finally, we can simply compute the value of $\vece_\nctx \cdot \desF{f^\MLP}$ because we know $B$ and $S$ and so just need to compute $\MLP(\vec{e}_r (E \cdot \diag(B) + S))$.
\end{proof}

\subsection*{Bounds on Expectation and Variance}
\begin{lemma}[Bounds on Expectation and Variance]
	\label{lem:bounds}
	Let $X' \in \RepSpace$ and $X = \left(\bigtimes_{i = 1}^{\ndes} \vecDes \right) \times X' \times \vecQuery$. I.e.\ $X$ is an expansion of the input with the first tokens being the repetition token and the last being the query token.
	Then, we can bound the column wise expectation for column $j$, $\mu_j$, and variance, $\Var_j$ of $X$ by
	\begin{align*}
		\frac{1}{\nctx}\left(
			\ndes \cdot \vecDes \Embed \vec{e}_j^T + \nfree \min_i (\vec{e}_i \Embed \vec{e}_j^T) + \vecQuery \Embed \vec{e}_j^T
		\right)
		\leq \mu_j \leq  \\
		\frac{1}{\nctx}\left(
			\ndes \cdot \vecDes \Embed \vec{e}_j^T + \nfree \max_i (\vec{e}_i \Embed \vec{e}_j^T) + \vecQuery \Embed \vec{e}_j^T
		\right).
	\end{align*}
	Let $\mu_j^{\min}$ and $\mu_j^{\max}$ denote the lower and upper bounds.
	Then, we can bound the variance of the $j$-th column of $\vec{x} \Embed$ by
	\begin{align*}
            \Var_j 
			\leq &\frac{1}{\nctx} \cdot \max_{\mu' \in [\mu_j^{\min}, \mu_j^{\max}]} \bigg[
			 (\ndes +1) \cdot  \left( \vecDes \Embed \vec{e}_j^T -  \mu' \right)^2 + 
			 \nfree \cdot \max_{i \in \dVocab} \left( \vec{e}_i \Embed \vec{e}_j^T -  \mu' \right)^2
			\bigg].
	\end{align*}
	We define $\Var_j^{\min}, \Var_j^{\max}$ to be the lower and upper bounds respectively.
\end{lemma}
\begin{proof}
	The bounds on $\mu_j$ follow as we minimize and maximize the contribution of each free row to the expectation of the column of $j$.
	The proof of for variance bounds follows similarly.
	For the lower bound, we note that the variance contribution from each row is at least $0$.
	For the upper bound, we simply maximize the contribution of each row to the variance. 
\end{proof}

Recall the definition of blowup and shift sets from \cref{def:blowup_shift}.
\begin{lemma}[Blowup and Shift Set Bounds]
	\label{lem:blowup_shift}
	Recall that $\gamma$ and $\beta$ are the learned constants for layer normalization.
	Let $\vec{x} \in \InpSpace$.
	Then, we can bound the blowup and shift sets for the $j$-th column of $\vec{x} \Embed$.
	For all $S_j \in \shiftSet_j$ and $B_j \in \blowupSet_j$, we have
	\begin{align*}
		\frac{\gamma}{\sqrt{\Var_j^{\max} + \eps}} \leq B_j \leq \frac{\gamma}{\sqrt{\eps}}
	\end{align*}
	Let $B_j^{\min}$ and $B_j^{\max}$ denote the lower and upper bounds respectively.
	Then,
	\begin{align*}
		B_j^{\min} \mu_j^{\min} + \beta \leq S_j \leq B_j^{\max} \mu_j^{\max} + \beta.
	\end{align*}
\end{lemma}

Now as a corollary of \cref{lem:bounds} and \cref{lem:blowup_shift}, we can show that the minimum and maximums of the blowup and shift sets can be made arbitrarily close together.
\begin{corollary}[Convergence of Blowup and Shift Sets]
	If $\frac{\nfree}{\nctx} \in o(1)$, then for every $\delta_1, \delta_2 \in (0, 1)$, there exists some setting of $\nctx'$ such that for all $\nctx > \nctx'$,
	\[
		B_j^{\max} - B_j^{\min} \leq \delta_1 \quad \text{and} \quad S_j^{\max} - S_j^{\min} \leq \delta_2.
	\]
\end{corollary}
\begin{proof}[Proof]
    The corollary follows from the fact that $\mu_j^{\min}, \mu_j^{\max}$ converge to $\vec{e}_r E \vec{e}_j^T$ as $\nctx \rightarrow \infty$.
    So then, $(\vec{e}_r E \vec{e}_j^T - \mu')^2$ for all $\mu' \in [\mu_j^{\min}, \mu_j^{\max}]$ converges to $0$ and thus the upper bound on $\Var_j$ converges to $0$ as long as $\frac{\nfree}{\nctx}$ converges to $0$.
\end{proof}

Therefore the blowup and shift sets converge, let $B_j'$ and $S_j'$ denote the converged values for $B_j$ and $S_j$ respectively.

Now, we need to show a bound on the difference between the maximum logit for the free tokens and the minimum logit for the repeated tokens in the attention head.

\subsubsection*{Bound on Attention Head Difference}
To bound attention, we will take advantage of the repeated structure as well as the ideas in \cref{lem:min_max_softmax} (bounds on the attention weights).

\begin{corollary}[Sum of Attention Weights]
	\label{cor:sum_attention}
	Let $\logRMin \leq \min_{i \in [\ndes] \cup \{n_ctx\}} \ell_i$ and $\logFMin \leq \min_{j \in (\nfix, \nctx)} \ell_j$.
	Also, let $\logRMax \geq \max_{i \in [\ndes] \cup \{n_ctx\}} \ell_i$ and $\logFMax \geq \max_{i \in (\nfix, \nctx)} \ell_i$. Then,
	\begin{align*}
	\frac{\ndes}{\ndes + 1 + (\nfree) \cdot e^{\logFMax - \logRMin}} 
\leq 
		\sum_{i \in [\ndes] \cup \{\nctx\}} \softmax(\vec{\ell}_i)
	\leq
		\frac{\ndes}{\ndes + 1 + (\nfree) \cdot e^{\logFMin - \logRMax}}
	\end{align*}
	and 
	\begin{align*}
		\frac{\nfree}{\ndes + 1 + (\nfree) \cdot e^{\logFMin - \logRMax}} 
        \leq \sum_{j \in (\nfix, \nctx)} \softmax(\vec{\ell}_j)
        \leq \frac{\nfree}{\ndes + 1 + (\nfree) \cdot e^{\logFMax - \logRMin}}.
	\end{align*}
\end{corollary}

\begin{lemma}[Bound on Difference]
	\label{lem:bound_diff}
	We can bound the difference between the maximum logit for the free tokens and the minimum logit for $\ndes$ repeated tokens
	\begin{align}
		\logFMax - \logRMin &\leq \theta
	\end{align}
	and
	\begin{align}
		\logFMin - \logRMax &\leq \theta
	\end{align}
	where $\theta$ is a small constant.
\end{lemma}
\begin{proof}
	We will prove the first inequality as the second follows analogously.
	Assume that $i \in [\dVocab]$ is the token which maximizes the difference.
	We will relly on the convergence of the blowup and shift sets to show that the difference between the maximum logit for the free tokens and the minimum logit for the repeated tokens in the attention head is bounded.
	
	Connote $\blowupSet - B'$ as the set of blowup values shifted by the converged value and $\shiftSet - S'$ as the set of shift values shifted by the converged value.
	Then $\blowupSet - B'$ and $\shiftSet - S'$ are bounded by $\delta_1, \delta_2$ respectively, we can write the following bound on the difference.
	\begin{align*}
		\logFMax - \logRMin &\leq \max_{\eps \in \blowupSet - B', \; \eps_S \in \shiftSet - S'} (\vec{e}_i - \vecDes) E \cdot \diag(B' + \eps) \\ & \cdot Q K^T (\diag(B' + \eps) \cdot E^T \vecQuery^T + S^T + \eps_S) \\
					   &= \max_{\eps \in \blowupSet - B', \eps_S \in \shiftSet - S'} \\& \sum_{d_1 \in \{\eps, B'\}} \sum_{d_2 \in \{\eps, B'\}} \sum_{d_3 \in \{\eps, S'\}} (\vec{e}_i - \vecDes) E \cdot \diag(d_1) \cdot Q K^T (\diag(d_2) \cdot E^T \vecQuery^T + d_3) \\
					   &\leq \max_{\eps \in \blowupSet - B', \eps_S \in \shiftSet - S'} (\vec{e}_i - \vecDes) E \cdot \diag(B') \cdot Q K^T (\diag(B') \cdot E^T \vecQuery^T + S^T) + \poly(\eps, \eps_S)\\
					   &\leq (\vec{e}_i - \vecDes) E \cdot \diag(B')  \cdot Q K^T (\diag(B') \cdot E^T \vecQuery^T + S^T) + \poly(\delta_1, \delta_2, B', S')
                       \tag{By the bounds on the blowup and shift set}
                       \\
					   &\leq (\vec{e}_i - \vecDes) E \cdot \diag(B')  \cdot Q K^T (\diag(B') \cdot E^T \vecQuery^T + S^T) + \poly(\delta_1, \delta_2)
	\end{align*}
	where the last inequality follows from considering $B'$ and $S'$ as a constant.
	Then, because $\delta_1, \delta_2$ decrease as a function of $\nctx$ and \cref{cor:sum_attention},
	we can bound
	\begin{align*}
		&(\vec{e}_i - \vecDes) E \cdot \diag(B') 
        \cdot 
        Q K^T (\diag(B') \cdot E^T \vecQuery^T + S^T) + \poly(\delta_1, \delta_2) \leq \theta
	\end{align*}
	for a large enough $\nctx$.
\end{proof}


\begin{lemma}[Attention Bound, \cref{lem:att_bound}]
	\label{lem:convgattn}
	For large enough $\nctx$, we get
	\[
		\WD(\desF{f^\attnH}; \RepSpace)_\infty \leq o(1).
	\]
\end{lemma}
\begin{proof}
	\label{proof:convgattn}
	Recall that \cref{lem:att_bound} gives us
	\begin{align*}
		&\WD(\vecNctx \cdot f^{\attnH}_{\ndes \mid r, q} ; \InpSpace \times \blowupSet \shiftSet)_\infty  
						 \leq					   (\ssMaxRB - \ssMinRB) \cdot \max_{B, S}\norm{\vecDes \Embed B + S} + 
											   2 \cdot \ssMaxFB \cdot \max_{B, S}\norm{(\vecDes \Embed B + S) \cdot V}_\infty
	\end{align*}
	Because $\logFMax - \logRMin \leq \theta$ and $\logFMin - \logRMax \leq \theta$,
	we can use \cref{cor:sum_attention} to upper-bound $\ssMaxRB - \ssMinRB$ and $\ssMaxFB$ by
	\begin{align*}
		1 - \frac{\ndes}{\ndes + (\nfree + 1) \cdot \theta}
		= \frac{(\nfree + 1) \cdot \theta}{\ndes + (\nfree + 1) \cdot \theta} 
		\leq \frac{\nfree \cdot \theta}{\ndes} 
		\leq \theta \cdot o(1).
	\end{align*}
	So then
	\[
		\WD(\desF{f^{\attnH}} ; \InpSpace \times \blowupSet \shiftSet)_\infty \leq 
		o(1) 
	\]
	as $\theta \in o(1)$.
\end{proof}

\subsubsection*{Bound on $\fenc$}
Finally, as per \cref{sec:meta_framework}, we need to bound the worst-case deviation of the $\fenc$ function.
Given that $\fenc$ is a linear operation as a function of blowup and shift sets, we can use the convergence of the blowup and shift sets to show that the worst-case deviation of $\fenc$ is bounded.

\begin{lemma}[Bound on $\fenc$]
	\label{lem:conv_fenc_bound}
	For large enough $\nctx$, we have
	\[
		\WD(\vec{e}_\nctx \cdot \desF{\fenc}; \RepSpace)_\infty \leq \poly(\delta_1, \delta_2).
	\],
    where $\delta_1, \delta_2$ are the bounds on the blowup and shift differences.
\end{lemma}
%\begin{proof}
%	The proof follows trivially from the fact that $\vec{e}_{\nctx} \cdot \desF{\fenc} = \vecQuery \cdot E B + S$.
%	Then, we can use the $\delta_1, \delta_2$ bounds on the blowup and shift sets to show that the worst-case deviation of $\vec{e}_\nctx \cdot \desF{\fenc}$ is bounded a polynomial function of $\delta_1, \delta_2$.
%	So, as long as $\peakToPeak(\desF{\Model}, \vecDes^{\nfree})$ is positive, we can always set $\nctx$ to be large enough such that $\WD(\vec{e}_\nctx \cdot \desF{\fenc}; \InpSpace \times \blowupSet \shiftSet)$ is arbitarily small and thus $\WD(\vec{e}_\nctx \cdot \desF{\fenc}; \InpSpace)_2$ is arbitraily small by the lifting monotonicity of worst-case deviation (\cref{lemma:liftMon}).
%\end{proof}

We are now ready to prove \cref{thm:convergence}.
\begin{proof}[Proof of \cref{thm:convergence}]
	Note that we reduced $\WD(\vec{e}_\nctx \cdot \desF{\Model}; \RepSpace)_\infty$ to be upper-bounded by 
	$
	O(1) \cdot \WD(\vec{e}_\nctx \cdot \desF{\fenc}; \RepSpace) + o(1)
	$
	through \cref{lem:mlpbound} and \cref{lem:convgattn}.
	Then, by \cref{lem:conv_fenc_bound}, we have that \cref{thm:convergence} holds as $\WD$ goes to $0$ as $\nctx \rightarrow \infty$.
    So, as long as $\peakToPeak(\desF{\Model}, X)$ is positive for \emph{some} $X \in \RepSpace$, then we converge to ``overwhelming'' by \cref{thm:metathm}.
    Note that $\vece_\rep^\nctx \in \RepSpace$ and by \cref{lem:gap_shortcut}, we can compute a sample of peak-to-peak deviation for all $\nctx$.
\end{proof}







\end{document}
