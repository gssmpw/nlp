\section{Experimental Outcomes}
\label{sec:appOut}
In this appendix we provide the plots for \cref{sec:model_eval}.
Firstly, \cref{fig:worst-case} contains the plots of worst-case deviation with and without the use of the linear program, and  \cref{fig:ptp} plots the peak-to-peak difference. In \cref{fig:ptp} whenever our algorithm proves ``overwhelming'', the point is colored red and marked with a `x.'
\vspace{0pt}\nopagebreak
\begin{figure}[H]
    \centering
	\includegraphics[width=\textwidth, keepaspectratio, trim={0cm 0cm 0 1.5cm},clip]{figs/plots/w_analysis_lp_log_independent.pdf}
	\caption{The worst-case deviation for the model for the three different input restrictions and two different permutation classes.
    The $x$-axis is the number of tokens and the $y$-axis is the base-10 logarithm of the worst-case deviation.
    }
	\label{fig:worst-case}
\end{figure}
\pagebreak
\begin{figure}[H]
    \centering
	\includegraphics[width=\textwidth, keepaspectratio, trim={0cm 0cm 0 1.5cm},clip]{figs/plots/w_analysis_ptp_linear_independent.pdf}
	\caption{The peak-to-peak difference for the model for the three different input restrictions and two different permutation classes.
	Red ``x''s indicate that the worst-case deviation is less than half of the peak-to-peak difference and, thus, the model output is provably invariant over the permutation class.
        The $x$-axis is the number of tokens and the $y$-axis is the peak-to-peak deviation.
    }
	\label{fig:ptp}
\end{figure}

\subsection*{Overwhelming Through Generation}
In \cref{fig:contOverwhelm} we plot the bounds on worst-case deviation and the peak-to-peak difference as the model is used to continually generate text. The newly generated tokens are continuously added to the fixed string and fed again to the model to generate the next token. The model is ``overwhelmed'' whenever the worst-case deviation is less than the peak-to-peak difference in the plot.

\vspace{0pt}\nopagebreak
\begin{figure}[H]
    \centering
	\includegraphics[ width=\textwidth, keepaspectratio, trim={0cm 0cm 0 1.5cm},clip]{figs/plots/continued_gen.pdf}
	\caption{
            The peak-to-peak difference and the worst-case deviation when we continue generation through multiple tokens.
            The $x$-axis connotes the total number of tokens ($n_{ctx}$) throughout a generation.
            When the line for $PTP / 2$ is above the worst-case deviation line, then our algorithm provably guarantees that the output is fixed.}
	\label{fig:contOverwhelm}
\end{figure}
