\section{Proofs for Worst-Case Deviation}
\label{sec:proofs_worst_case_deviation}

\begin{lemma}[Properties of the Worst-Case Deviation]
	\label{lemma:worst_case_deviation_properties}
	For any functions $f, g: \calX \to \R^n$, norm $p$ and lift to $\calY \times \calZ$, we have the following properties:
	\begin{itemize}[nosep]
		\item Triangle inequality for (lifted) $\WD$:	
			\[
				\WD(f + g; \calX)_p \leq \WD(f : \calX)_p + \WD(g : \calX)_p.
			\]
		\item Lifting monotonicity:
			\[
				\WD(f; \calX)_p \leq \WD(f; \calY \times \calZ)_p.
			\]
		\item Lipschitz composition:
			For function $g$
               \[
				\WD(g \circ f; \calX)_p \leq \Lip(g)_p \cdot \WD(f; \calX)_p.
			\]
		As a corollary, we have that for linear operators $A$,
		\[
			\WD(A f; \calX)_p \leq \norm{A}_p \cdot \WD(f; \calX)_p.
		\]
		as $\Lip(A)_p = \norm{A}_p$.

		\item $p$-norm bounds for $p, q \geq 1$ and $q > p$:
			\[
				\WD(f; \calX)_q \leq \WD(f; \calX)_p
			\]
	\end{itemize}
\end{lemma}


\begin{proof}[Proof of worst-case deviation properties, \cref{lemma:worst_case_deviation_properties}]
	\label{proof:worst_case_deviation_properties}
	We will prove each of the properties in turn.
	\begin{itemize}
		\item Triangle inequality:
			We can view the maximization over $\calX$ as occuring disjointly for $f$ and $g$:
			I.e. \begin{align*}
				\WD(f + g; \calX)_p 
			&\leq
            \sup_{X_1, X_2 \in \mathcal{X}} \norm{
			f(X_1) + g(X_1) - (f(X_2) + g(X_2))} \\
			&\leq \sup_{X_1, X_2, X_1', X_2' \in \mathcal{X}}  \big[\norm{f(X_1) - f(X_2)} 
            + \norm{g(X_1') - g(X_2')}\big] 
			\tag{by triangle inequality of norms}\\
			&\leq \WD(f ; \calX)_p + \WD(g ; \calX)_p
			\end{align*}
			as desired.
		\item Lifting monotonicity:
			The proof follows from the definition of the lifted worst-case deviation:
			\begin{align*}
				\WD(f ; \calX)_p &= \sup_{(Y_1, Z_1), (Y_2, Z_2) \in \calX \subseteq \calY \times \calZ} \norm{f(Y_1, Z_1) - f(Y_2, Z_2)}_p \\
						 &\leq \sup_{Y_1, Y_2 \in \calY, \; Z_1, Z_2 \in \calZ} \norm{f(Y_1, Z_1) - f(Y_2, Z_2)}_p \\
						 &= \WD(f ; \calY \times \calZ)_p.
			\end{align*}
		\item Lipschitz composition:
			We simply have that
			\begin{align*}
				&\WD(A f; \calX)_p = \sup_{X_1, X_2 \in \calX} \norm{A f(X_1) - A f(X_2)}_p \\
						  &= \sup_{X_1, X_2 \in \calX} \norm{A (f(X_1) - f(X_2))}_p \\
						  &\leq \sup_{X_1', X_2'} \frac{\norm{A(X_1') - A(X_2')}_p}{\norm{X_1' - X_2'}_p} \cdot \sup_{X_1, X_2 \in \calX} \norm{f(X_1) - f(X_2)}_p \\
						  &=\Lip(A)_p \cdot \WD(f; \calX)_p.
			\end{align*}
		\item $p$-norm bounds for $p, q \geq 1$ and $q > p$:
			Because we restrict $f$ to be a function which outputs vectors and $\norm{\vec{x}}_q \leq \norm{\vec{x}}_p$ for $q > p$, we have that for all $X_1, X_2 \in \calX$, $\norm{f(X_1) - f(X_2)}_q \leq \norm{f(X_1) - f(X_2)}_p$.
			So, if there exists $X_1, X_2 \in \calX$ such that $\norm{f(X_1) - f(X_2)}_q = \alpha$, then there must exist $X_1, X_2 \in \calX$ such that $\norm{f(X_1) - f(X_2)}_p \geq \alpha$.
			Thus, the supremum over $\calX$ for $q$ is less than or equal to the supremum over $\calX$ for $p$.
		\end{itemize}
	\end{proof}

