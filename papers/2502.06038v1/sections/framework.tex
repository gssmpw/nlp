\section{Worst-Case Deviation and a Custom Norm}
\label{sec:worst_case_framework}
When proving statements about our model, we will find it useful to talk about a variation on the Lipschitz constant and maximum absolute deviation that we term the ``worst-case deviation.''
Simply put, the worst-case deviation of a function $f: \calX \to \R^n$ is the maximum distance between the outputs of $f$ on any two points in $\calX$.

Before defining the worst-case deviation more formally, we will first define the Lipschitz constant of a function.
\begin{definition}[Lipschitz Constant]
	\label{def:lipschitz_constant}
	For a function $f: \calX \to \R^n$ and norm $p$, we define the Lipschitz constant of $f$ as
	\[
		\Lip(f)_p := \sup_{X_1, X_2 \in \calX} \frac{\norm{f(X_1) - f(X_2)}_p}{\norm{X_1 - X_2}_p}.
	\]
\end{definition}

\begin{definition}[Worst-Case Deviation]
	\label{def:worst_case_deviation}
	For a function $f: \mathcal{X} \to \R^n$, we define the worst-case deviation of $f$ as
	\[
		\WD(f; \calX)_p := \sup_{X_1, X_2 \in \calX} \norm{f(X_1) - f(X_2)}_p
	\]
\end{definition}

It will be convenient to ``lift'' the input space into two parts and then find bounds on the worst-case deviation by optimizing over the two parts separately.
\begin{definition}[Lifted Worst-Case Deviation]
	\label{def:lifted_worst_case_deviation}
	For a function $f: \mathcal{X} \to \R^n$ where $\calX \subseteq \calY \times \calZ$ and $f$ is well defined on $\calY$ and $\calZ$, we define the lifted worst-case deviation of $f$ as
	\[
		\WD(f; \calY \times \calZ)_p := \sup_{Y_1, Y_2 \in \calY} \sup_{Z_1, Z_2 \in \calZ} \norm{f(Y_1, Z_1) - f(Y_2, Z_2)}_p
	\]
\end{definition}

In \cref{sec:proofs_worst_case_deviation}, we state and prove some useful properties of the worst-case deviation.
Mainly, we note that $\WD$ behaves mostly like a norm but has the nice property that ``lifting'' the domain is monotonic.
I.e.\ for lifted domain $\calX \subseteq \calY \times \calZ$, $\WD(f; \calX) \leq \WD(f; \calY \times \calZ)$.

When discussing the worst-case deviation in this paper, we will use the $\infty$-norm for vectors.
For convenience, we will define an analogue of the Frobenius norm for the $\infty$-norm.

\begin{definition}[$\frobInf$-norm]
	\label{def:frobenius_infinity_norm}
	For a matrix $M \in \R^{n \times m}$, we define the Frobenius-$\infty$ norm ($\frobInf$-norm) as
	\[
		\norm{M}_{\frobInf} := \max_{i \in [n]} \max_{j \in [m]} \abs{M_{ij}}.
	\]
\end{definition}

