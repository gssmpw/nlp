\documentclass{llncs}  

\usepackage{graphicx} 
\usepackage[T1]{fontenc}
\usepackage{amsmath}
\usepackage[english]{babel}
\usepackage{array,tabularx}
\usepackage{booktabs} 
\usepackage{makeidx}  % allows for index generation
\usepackage{xparse}
\usepackage{moredefs}
\usepackage{lips}
\usepackage{epstopdf} 
% \usepackage{minted}
\usepackage{color}
\usepackage{csquotes}
\usepackage{xcolor}
\usepackage{times}
\usepackage[hidelinks]{hyperref}
\usepackage[nolist,nohyperlinks]{acronym}
\usepackage{comment}
\usepackage{paralist}
\usepackage{cleveref}
\usepackage{wrapfig}
\usepackage{multirow}
\usepackage{microtype}
\usepackage{listings}
\usepackage[super]{nth}
\usepackage{natbib}
\usepackage{siunitx}


% ------------------
% | WI Adjustments |
% ------------------

% \setlength{\parindent}{0pt} % disable indent globally

\usepackage{float}
% Make table captions above the table
\floatstyle{plaintop}
\restylefloat{table}

%fix captions of figures
\usepackage[style=base,labelfont=bf,labelsep=period,tableposition=top,font=small]{caption}
\captionsetup[figure]{name=Figure}
\setlength{\parskip}{0pt} % no parskip, only parindent
\setlength{\textfloatsep}{18pt plus 0pt minus 18pt} %abstand von text und figure/caption
\setlength{\floatsep}{12pt plus 0pt minus 0pt} %abstand zwischen floats
\setlength{\intextsep}{18pt plus 0pt minus 18pt} %abstand zwischen text und anderen arten von figures

\makeatletter
\renewcommand\@biblabel[1]{#1.}
\makeatother
\addto\captionsenglish{\renewcommand{\bibname}{\leftline{References}}}

% \let\oldbibliography\bibliography% Store \bibliography in \oldbibliography
% \renewcommand{\bibliography}[1]{{%
%  \let\chapter\section% Copy \section over \chapter
%  \oldbibliography{#1}}}% Old \bibliography

% WI footer
\usepackage{fancyhdr}
\fancypagestyle{WI_footer}{\fancyhf{}\renewcommand{\headrulewidth}{0pt}\fancyfoot[L]{\color{black}\scriptsize 19th International Conference on Wirtschaftsinformatik\\September 2024, Würzburg, Germany}}

% -----------
% | Custom  |
% -----------
\newcommand{\ks}[1]{\hangindent=0.5cm \textbf{#1}}
\crefformat{footnote}{#2\footnotemark[#1]#3}

\begin{document}
%\frontmatter          % for the preliminaries
\pagestyle{headings}  % switches on printing of running heads

\mainmatter     

\appendix
\label{appendix}

\section{Treatments}
\label{appendix:treatments}

\textbf{Human}

\begin{figure}[h!]
    \centering
    \includegraphics[width=0.67\textwidth]{figures/Human_Treatment.pdf}
    \caption{Visual and textual treatments for the Human condition}
    \label{fig:human_treatment}
\end{figure}

\noindent\textbf{White-box AI}

\begin{figure}[h!]
    \centering
    \includegraphics[width=0.67\textwidth]{figures/White-box_Treatment.pdf}
    \caption{Visual and textual treatments for the White-box AI condition}
    \label{fig:white-box_treatment}
\end{figure}

\newpage
\noindent\textbf{Black-box AI}

\begin{figure}[h!]
    \centering
    \includegraphics[width=0.67\textwidth]{figures/Black-box Treatment.pdf}
    \caption{Visual and textual treatments for the Black-box AI condition}
    \label{fig:black-box_treatment}
\end{figure}

\section{Willingness to Share Data}
\label{app:willingness}

We assessed participants' data-sharing willingness for the SleepOptima app mock-up using a 14-item, 7-point Likert scale developed specifically for this study. The scale ranged from 1 ("Strongly Unwilling") to 7 ("Strongly Willing") and covered seven data categories, including demographics, physical and mental health, and sexual activities. Each category contained two items. We calculated the average of these 14 items to obtain a continuous measure of willingness to share sensitive health-related data, with mean scores and standard deviations categorized by treatment condition detailed in Table \ref{tab:willingness_comparison}.

Prior to the main study, we conducted a pre-test with $N = 30$ participants from Prolific to ensure that the questions were appropriately ordered based on their sensitivity level. The pre-test participants rated their willingness to share different types of information with health and well-being apps on the same 7-point Likert scale ("Strongly Unwilling" to "Strongly Willing"). The pre-test encompassed three items for each of the seven categories. Results from the pre-test showed that seven items had no clear differences in sensitivity and were thus removed from the survey. The remaining 14 items (two per category) were ordered according to the sensitivity classification derived from the pre-test results and included in our main survey (see Table \ref{tab:willingness_comparison}).

To assess the scale's reliability, we calculated Cronbach's alpha. The analysis revealed a raw alpha of 0.93 and a standardized alpha of 0.94, indicating high internal consistency reliability. However, it is important to note that a Cronbach's alpha greater than 0.9 may also suggest potential redundancy among the scale items \citep{streiner2003starting}. Further research is needed to examine the scale's dimensionality and determine whether some items could be measuring the same underlying construct.

The item-total correlations ranged from 0.61 to 0.86, suggesting that each item contributes positively to the overall scale reliability. However, given the high Cronbach's alpha, future studies should investigate the scale's factor structure to ensure that the items are not overly redundant.

% We acknowledge that we have not provided evidence of the scale's validity, which is a limitation of our study. Establishing the scale's validity would involve assessing its content validity, convergent validity, and discriminant validity. Future research should aim to establish the validity of our scale or utilize existing validated measures.

% Willingness to Share Data Table
\begin{table}[H]
    \centering
        \caption{Mean (M) and standard deviation (SD) of willingness to share personal data for the 14 requested items, differentiated by our three treatment conditions: Human expert, White-box AI system, and Black-box AI system.}
        \label{tab:willingness_comparison}
        
        \sisetup{
           separate-uncertainty,
           table-number-alignment=center,
           print-zero-integer=true
        }
        \begin{tabular*}{\textwidth}{@{\extracolsep{\fill}} l
            S[table-format=2.2(2)]
            S[table-format=2.2(2)]
            S[table-format=2.2(2)]
        }
        \toprule
        \textit{Category} & {Human} & {White-box AI} & {Black-box AI} \\
        \cmidrule{2-2} \cmidrule{3-3} \cmidrule{4-4}
        {Item} & {$M \pm SD$} & {$M \pm SD$} & {$M \pm SD$} \\
        \midrule
        \textit{Demographics} & & & \\
        Gender & 6.58(0.67) & 6.45(1.03) & 6.57(0.77)  \\
        Age & 6.57(0.69) & 6.48(0.90) & 6.61(0.67) \\[0.1cm]
        \textit{Activity Level} & & & \\
        Average Exercise Days per Week & 6.44(0.82) & 6.31(1.08) & 6.39(0.91) \\
        Activity Outside of Work & 6.42(0.85) & 6.31(1.09) & 6.39(0.78) \\[0.1cm]
        \textit{Sleep History} & & & \\
        Sleep Quality & 6.57(0.63) & 6.40(0.95) & 6.51(0.74) \\
        Sleep Medication & 6.20(1.04) & 6.27(1.03) & 6.34(1.14) \\[0.1cm]
        \textit{Physical Health Status} & & & \\
        Chronic Illness & 5.91(1.27) & 5.75(1.32) & 6.13(1.19) \\
        Chronic Medication & 5.85(1.32) & 5.69(1.37) & 5.97(1.30) \\[0.1cm]
        \textit{Mental Health Status} & & & \\
        Stress Level & 6.14(0.92) & 5.94(1.19) & 6.13(1.00) \\
        Mental Health Disorder & 5.53(1.57) & 5.60(1.45) & 5.74(1.27) \\[0.1cm]
        \textit{Substance Use} & & & \\
        Influence Substances Sleep & 5.48(1.72) & 5.72(1.51) & 5.86(1.35) \\
        Usage of Certain Substances & 5.14(1.89) & 5.63(1.71) & 5.72(1.49) \\[0.1cm]
        \textit{Sexual Activities} & & & \\
        Satisfaction Sex Life & 4.24(1.88) & 4.23(1.92) & 4.82(1.83) \\
        Sexual Medication & 4.80(2.05) & 4.75(1.93) & 5.08(1.79) \\
    \bottomrule
        \end{tabular*}
\end{table}

\section{Additional Constructs}
\label{appendix:additional_constructs}

\noindent \textbf{Trust in AI}\\

\noindent To operationalize the construct of Trust in AI, we adapted the trust scale for the XAI context from \cite{hoffman2023measures} to align with our investigation's focus on comparing the willingness to share data between human and AI data-processing entities. Recognizing the need for consistency across our measures and to increase the sensitivity of our analysis, we expanded the original 5-point Likert scale to a 7-point scale ranging from 1 ("Strongly Disagree") to 7 ("Strongly Agree"). In addition, we generalized the trust statements to ensure the applicability of the scale across all participant groups, including those without direct AI experience. This approach allowed us to capture a broad spectrum of Trust in AI, which is essential for analyzing its influence on user data-sharing behavior.

The adapted scale consists of seven items. By averaging the scores of these items, we obtain a continuous measure of Trust in AI.

\begin{enumerate}
    \item I generally feel confident in Artificial Intelligence systems. They tend to work well.
    \item The outputs from Artificial Intelligence systems are usually predictable.
    \item In general, Artificial Intelligence systems are reliable. I can expect them to be accurate most of the time.
    \item I feel safe relying on Artificial Intelligence systems for accurate information or solutions.
    \item Artificial Intelligence systems usually operate efficiently and quickly.
    \item In general, Artificial Intelligence systems can perform tasks better than novice human users.
    \item I like using Artificial Intelligence systems for decision-making.
\end{enumerate}

\noindent \textbf{Trust in People}\\

\noindent In our examination of Trust in People, we adapted the General Trust Scale from \cite{yamagishi1986provision}, originally a 6-item questionnaire designed to measure general trustworthiness and honesty among people. To align with the consistent measurement scales used throughout our study and to increase the depth of our analysis, we expanded the scale from its original 5-point Likert format to a 7-point scale ranging from 1 ("Strongly Disagree") to 7 ("Strongly Agree").

This adaptation not only facilitates consistent data collection across different constructs, but also increases the granularity with which we capture participants' levels of Trust in People. Through this adapted scale, our study aims to shed light on the comparative dynamics of Trust in People versus Trust in AI, and how these different forms of trust may differentially affect users' willingness to share data.

\begin{enumerate}
    \item Most people are basically honest.
    \item Most people are trustworthy.
    \item Most people are basically good and kind.
    %\item Please select ‘Strongly Agree’ to show that you are paying attention to this question.
    \item Most people are trustful of others.
    \item I am trustful.
    \item Most people will respond in kind when they are trusted by others.
\end{enumerate}

\noindent \textbf{Privacy Concerns}\\

\noindent In assessing the construct of Privacy Concerns, our choice favored the Internet Users' Information Privacy Concerns (IUIPC) scale developed by \cite{malhotra2004internet} over the Concern for Information Privacy (CFIP) scale developed by \cite{smith1996information}. This decision was based on several key considerations. First and foremost, the IUIPC scale provides a more nuanced exploration of online Privacy Concerns that is closely aligned with the digital context of our study. Its emphasis on the perspective of Internet users as consumers directly aligns with our investigation of data sharing behaviors in online environments. While the CFIP scale provides valuable insights into individuals' concerns about organizational privacy practices and organizational responsibilities, the IUIPC scale shifts its focus to users' perceptions of fairness and justice in privacy matters, particularly when dealing with online companies \citep{gross2021validity}.

Furthermore, although integrating both the IUIPC and CFIP scales could theoretically provide a more comprehensive understanding of Privacy Concerns, such an approach was deemed impractical for our study. In particular, we were mindful of the potential for increased survey length to induce participant fatigue, which could affect response quality and engagement. Given these considerations, the decision to use only the IUIPC scale was deemed the most appropriate to capture the specific nuances of Privacy Concerns relevant to our research focus, while ensuring survey brevity and participant responsiveness.

Following \cite{malhotra2004internet} we measured each item on a 7-point Likert scale ranging from 1 ("Strongly Disagree") to 7 ("Strongly Agree"). The scores from each item are averaged to create a continuous measure of Privacy Concerns in our context, allowing for a detailed analysis of the potential effects of Privacy Concerns on data-sharing behavior.

\begin{enumerate}
    \item It usually bothers me when online companies ask me for personal information.
    \item When online companies ask me for personal information, I sometimes think twice before providing it.
    \item It bothers me to give personal information to so many online companies.
    \item I’m concerned that online companies are collecting too much personal information about me.
    \item Consumer online privacy is really a matter of consumers’ right to exercise control and autonomy over decisions about how their information is collected, used, and shared.
    \item  Consumer control of personal information lies at the heart of consumer privacy.
    %\item Please select ‘Strongly Disagree’ to show that you are paying attention to this question.
    \item I believe that online privacy is invaded when control is lost or unwillingly reduced as a result of a marketing transaction.
    \item Companies seeking information online should disclose the way the data are collected, processed and used.
    \item A good consumer online privacy policy should have a clear and conspicious disclosure.
    \item It is very important to me that I am aware and knowledgeable about how my personal information will be used.
\end{enumerate}

\noindent \textbf{Validation of Scales}\\

\noindent To assess the convergent and discriminant validity of our measures, we conducted a multitrait-multimethod (MTMM) analysis \citep{campbell1959convergent}. The results of the MTMM analysis showed high correlations between items within each scale (Trust in People: 0.60-0.85; Trust in AI: 0.41-0.82; Privacy Concerns: 0.57-0.78), supporting convergent validity. The correlations between items across scales were low (Trust in People and Trust in AI: 0.08-0.26; Trust in People and Privacy Concerns: -0.08-0.10; Trust in AI and Privacy Concerns: -0.12-0.09), indicating good discriminant validity.

\bibliographystyle{agsm}
\bibliography{literature}

\end{document}