\section{Related Work}
\if0
Our problem can be formulated as a fair division problem with assignment valuations, which generalize the commonly assumed additive valuations.
Here, each class is represented by a ``meta'' agent whose valuation for a set of items is determined by the maximum total weight of an optimal matching between class members and items.
An assignment valuation, also known as an OXS valuation, form an important subclass of gross substitutes valuations and submodular valuations____.
%
For binary assignment valuations, an allocation satisfying both EF1 (a relaxed notion of envy-freeness) and non-wastefulness always exist and can be computed in polynomial time____. 
However, the compatibility between EF1 and non-wastefulness for general assignment valuations remains an open question.
\fi

\paragraph{Asymptotic fair division.}
Our work is closely related to the growing literature on asymptotic fair division ____. 
%
Dickerson et al.____ initiated the study of asymptotic fair division. They assumed that valuations are additive and drawn from a distribution with positive variances. 
Although the non-existence of an envy-free allocation also holds in this setting, Dickerson et al.____ demonstrated that a welfare-maximizing algorithm for additive agents produces an envy-free allocation with a probability that approaches $1$ as $m$ goes to infinity when $m=\mathrm{\Omega}(n\log n)$. 
%Moreover, Manurangsi and Suksompong____ showed that the round-robin algorithm for additive agents, allowing each agent to choose their favorite item in order, computes an envy-free allocation with a probability that approaches $1$ as $m\to \infty$ when $m=\mathrm{\Omega}(n\log n/\log \log n)$.  Apart from requiring fewer items for establishing asymptotic envy-freeness, the round-robin algorithm has another advantage over the welfare-maximizing algorithm; it achieves an approximation of envy-freeness, specifically envy-freeness up to one item (EF1), for additive agents____.
%

Following ____, Manurangsi and Suksompong____ proved that, assuming that utilities are drawn from a polynomial-bounded distribution when $m$ is divisible by $n$, an envy-free allocation exists for agents with additive valuations with a probability that approaches $1$ as $m\to\infty$.
%
Moreover, Manurangsi and Suksompong____ showed that under the assumption that utilities are drawn from a PDF-bounded distribution, agents have additive valuations and $m=\mathrm{\Omega}(n\log n/\log \log n)$, the round-robin algorithm returns an envy-free allocation with a probability that approaches $1$ as $n\to\infty$. Apart from requiring fewer items for establishing asymptotic envy-freeness, the round-robin algorithm has another advantage over the welfare-maximizing algorithm; it achieves an approximation of envy-freeness, specifically envy-freeness up to one item (EF1), for additive agents____.

Bai and G\"{o}lz____ extend these results to the case where agents have asymmetric distributions when distributions are PDF-bounded.
Furthermore, Benad\`{e} et al.____ demonstrated that the round-robin algorithm produces an SD envy-free allocation with a probability that approaches $1$ as $m\to\infty$ when agents have order-consistent valuation functions, items are renamed by a uniformly random permutation, $m$ is divisible by $n$, and $m=\omega(n^2)$.
%Here, a PDF-bounded distribution is polynomial-bounded, and a polynomial-bounded distribution with a positive variance.
Several papers have studied the asymptotic existence of allocations that satisfy other fairness notions, such as proportionality____ or a maximin share guarantee____. 



\paragraph{Asymptotic house allocation.}
In the one-to-one house allocation problem, Gan et al.____ studied the asymptotic existence of an envy-free allocation, which requires that there is no envy between every pair of individual agents. 
They demonstrated that when the agents' preferences are drawn uniformly at random and $m=\mathrm{\Omega}(n\log n)$, the probability that an envy-free allocation exists converges to $1$ as $n$ goes to infinity.
Furthermore, Manurangsi and Suksompong____ showed that, if $m/n\geq\mathrm{e}+\varepsilon$, where $\mathrm{e}$ is Napier's constant and $\varepsilon>0$ is any constant, an envy-free allocation can be found with a probability that converges to $1$ as $n\to \infty$. Conversely, if $m/n\leq\mathrm{e}-\varepsilon$, there is no envy-free allocation with a probability that converges to $1$ as $n\to \infty$. 
In the house assignment problem, the cardinal model is equivalent to the ordinal setting because agents must compare only individual items. However, in our setting, classes must compare bundles of items. 
As mentioned in Section~\ref{sec:mainresults}, the asymptotic existence of a class envy-free matching where each agent is matched to exactly one item readily follows from the result in the house allocation problem____, though such a matching may be wasteful. 



\if0
Several group fairness notions have been explored in various settings____. 
The majority of these studies concentrate on scenarios where each group possesses an additive valuation.
%
Manurangsi and Suksompong____ investigated the asymptotic existence of allocations satisfying specific group fairness criteria. Their model differs fundamentally from ours in its treatment of intra-group item allocation. 
In their framework, agents within a group collectively share the items allocated to that group, with an individual agent's utility derived from the aggregate utility of all items assigned to their group. See Appendix~\ref{sec:relatedworks} for a more extensive discussion about the further related work. Omitted proofs can be found in Appendix \ref{app:nesting_lemma} -- \ref{appendix:finalproof}. 
\fi



\paragraph{Assignment valuations.}
Our problem can be formulated as a fair division problem with \emph{assignment valuations}, which may violate the additive assumption prevalent in fair division literature. Specifically, each class is represented by a ``meta'' agent, and the meta agent's valuation for each set of items is determined by the maximum total weight of an optimal matching between the class members and the items in a given bipartite graph.  
%An assignment valuation is also known as an OXS valuation which is an important subclass of gross substitutes valuations____ and submodular valuations. A \emph{binary} assignment valuation, the underlying bipartite graph of which has binary edge weights, is a matroid rank function of a transversal matroid. 
An assignment valuation is a generalization of additive valuations. That is, given an arbitrary additive valuation $v$ for $m$ items, an equivalent assignment valuation can be created by representing each agent $i$ with a class $i$ with $m$ copies of agents, with every copy having the same edge weight $v_i(j)$ towards each item $j$. 
However, the same construction cannot be used to represent the distributional model for additive valuations using assignment valuations.
%It is known that when agents have binary assignment valuations and more generally submodular valuations with dichotomous marginals, an allocation satisfying the approximate fairness notion of envy-freeness, known as EF1, and non-wastefulness exists and can be computed in polynomial time____. In fact, EF1 is compatible with a stronger efficiency notion of utilitarian-optimality in the binary assignment setting. However, it remains an open question whether the same compatibility between EF1 and non-wastefulness holds for general assignment valuations. For further discussion, see Section $5$ of ____. 
%


\paragraph{Random assignment.}
We briefly review the literature on the theory of random assignments. 
%The theory of random assignments deals with a bipartite graph with random edge weights, and the main focus has been on the analyzing the expected minimum total weight of a matching ____. 
%More specifically, 
Let $C_{n,m,r}$ be the minimum total weight of the matching with $r$ edges in a bipartite graph with $n$ and $m$ vertices on each side when edge weights are independently assigned from the exponential distribution with rate $1$. Here, we assume that $n\leq m$. 
When $n=m=r$, Walkup____ showed that the expected value is bounded above when $n$ goes to infinity. 
Following numerous papers on experimental results and improved bounds~(see the introduction in____ for details), Karp____ improved the upper bound and showed that the expected value is smaller than $2$ for any $n$. Aldous____ showed that $\mathbb{E}[C_{n,n,n}]$ converges to $\frac{\pi^2}{6}$ as $n$ goes to infinity. 
For a more general combination of $n,m$, and $r$, Linusson and W\"{a}stlund____ and Nair, Prabhakar and Sharma____
obtained a concrete formula for the expected minimum total weight of a matching
given by $\mathbb{E}[C_{n,m,r}]  = \sum_{i=1}^r \frac{1}{n} \sum_{j=0}^{i-1} \frac{1}{m-j}$.
W\"{a}stlund____ provided a concise and elegant proof for this result, by analyzing the expected difference between the minimum weight of matching with $r$ edges and that with $r-1$ edges and showing that $\mathbb{E}[C_{n,m,r}] - \mathbb{E}[C_{n,m,r-1}] = \frac{1}{n}\sum_{j=0}^{r-1}\frac{1}{m-j}$.
Frieze and Johansson____ and Frieze____ explored a similar approach for random bipartite graphs and non-bipartite graphs, and W\"{a}stlund____ and Larsson____ extended the result in____ to more general distributions in some pseudo-dimension.
Recently, the random assignment problem where edge weights are drawn independently from a standard Gaussian distribution is investigated____.

%%%%%%%%%%%%%%%%%%%%%%%%%%%%%%%%%%%%%%%%%%%%%%%%%%%%%%%%%%%%%%%%%%%%%%%%