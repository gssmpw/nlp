\section{Related Work}
Representation theory applied to neural networks has been studied both theoretically ____ and applied to a variety of groups, architectures and data type: CNNs ____, Graph Neural Networks ____, Transformers,  ____, point clouds ____, 
chemistry ____.
On the topic of disentanglement of group action and symmetry learning, ____ factorize a Lie group from the orbits in data space,
while ____ learn through an autoencoder architecture invariant and equivariant representations of any group acting on the data. 
____ learns disentangled representations solely from data pairs.
____ propose an architecture based on Lie algebras that can automatically discover symmetries from data.
____ predict molecular conformations from molecular graphs
in an roto-translation invariant fashion with equivariant Markov kernels.

In the context of interpreting the latent space of diffusion models, ____ explores the local structure of the latent space (trajectory) of diffusion models using Riemannian geometry. Specifically, the authors assign a basis to each point in the tangent space through singular value decomposition (SVD), and show that these directions correspond to semantically meaningful features for image-based models. 
Similarly, ____ propose a method to uncover semantically meaningful directions in the semantic latent space ($h$-space) ____ of denoising diffusion models (DDMs) by PCA. ____ propose a method to  learn disentangled and interpretable latent representations of diffusion models in an unsupervised way. 
We note that the aforementioned works aim to extract meaningful latent factors ____ in traditional DDMs, often restricting to human-interpretable semantic features and focusing on image generation.

Related to our study is the field of diffusion on Riemannian manifolds. ____ propose diffusion in a product space -- a condition which is not a necessary in our framework -- defined by the flow coordinates in the respective Riemannian sub-manifolds. When the Riemannian manifold is a Lie group, 
their method yields dynamics similar to ours, as illustrated in an example in Section \ref{ss:torsion}. 
In fact, our formalism could be combined with their approach to obtain Lie algebra-induced dynamics on Riemannian manifolds, yielding a unified framework for modeling diffusion processes on a broader class of manifolds.
In fact, our formalism could be integrated with their approach to create a unified framework for diffusion processes on the broader class of Riemannian manifolds admitting a Lie group action. 
These techniques has been applied in a variety of use cases ____ for protein docking, ligand and protein generation.
The works ____
leverage trivialized momentum to perform diffusion on the Lie algebra (isomorphic to \(\mathbb{R}^n\)) instead of the Lie group, thereby eliminating curvature terms, although their approach is to date only feasible for Abelian groups. 
An interesting connection with our work is the work of ____: the authors propose a bijection to map a non-linear problem to a linear one, to approximate a bridge between two non-trivial distributions. Our case can be seen as a bijection between the (curved) Lie group manifold and the (flat) Euclidean data space.