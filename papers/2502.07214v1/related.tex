
% In the realm of decision-making, the integration of recourse mechanisms represents a significant advancement, particularly influencing how decisions impact individuals in critical sectors such as finance, healthcare, and education. Originating from the need to enhance the interpretability~\cite{gunning2019xai,arrieta2020explainable,guidotti2018survey} and accountability of machine learning and artificial intelligence models, recourse mechanisms provide individuals a clear pathway to understand and potentially modify the decisions that affect their lives directly. In the financial sector, recourse helps loan applicants understand necessary improvements for a higher likelihood of loan approval~\cite{barocas2017fairness, mukerjee2002multi}. In healthcare, it offers patients explanations and potential lifestyle changes aligned with their treatment plans~\cite{bastani2017interpretability,bertossi2020asp,liang2014deep}. Similarly, in education, recourse aids in analyzing student assessments and admissions, providing insights into academic outcomes, and suggesting avenues for improvement~\cite{fiok2022explainable,khosravi2022explainable}.

% Most algorithms developed for recourse in decision-making are based on gradient methods, which rely on the differentiability of cost functions associated with decision models~\cite{shao2022gradient}. These algorithms typically provide one-hop solutions, offering direct, feasible changes that an individual can make to alter the decision outcome in their favor. This approach, however, assumes that all aspects of the decision-making model are differentiable, which is not always the case.

% Initially developed to make machine learning and AI models more interpretable and accountable, recourse mechanisms face significant challenges, particularly when navigating the high-dimensional space of user attributes. Each attribute acts as a distinct dimension, complicating the identification of optimal recourse paths in real-world scenarios. Traditional recourse algorithms, mainly gradient-based, offer one-hop solutions but often struggle with discrete decisions or thresholds, limiting their applicability and effectiveness. Moreover, these methods lack comprehensive theoretical analysis and rely on approximations, which can compromise their efficacy and transparency. To address these issues, our work aims to push the boundaries of recourse methodologies by proposing innovative solutions for non-differentiable cost functions. This expansion allows for a wider application of recourse strategies, enhancing their ability to handle a variety of decision models that are not amenable to traditional approaches, ultimately providing a more principled foundation for analyzing and implementing recourse in decision-making processes.

% Transitioning from these challenges, contemporary methods such as FACE and GDPR's counterfactual explanations serve as practical implementations that address the issues of interpretability and operational feasibility in high-dimensional spaces.


% Building on these innovations, methods like FACE (Poyiadzi et al. 2020)\cite{poyiadzi2020face} which employs shortest path distances with density-weighted metrics to generate feasible and actionable counterfactual explanations, demonstrate how contemporary recourse mechanisms can integrate more intuitive and natural frameworks for machine learning applications. FACE constructs a graph over the data points using approaches such as KDE, k-NN, or e-graph. By approximating the f-distance using a finite graph over the dataset and incorporating additional feasibility and classifier confidence constraints, FACE provides a natural and intuitive framework for generating meaningful explanations in machine learning applications. Similarly, Automated Decisions and the GDPR (Wachter et al. 2017)\cite{wachter2017counterfactual} discusses counterfactual explanations in the context of GDPR and introduces the foundational ideas that could be extended using shortest path methods to generate actionable insights. This approach aims to generate alternative scenarios related to specific decisions by computing the shortest path, thereby providing an understanding of the logic behind automated decisions while meeting the transparency and interpretability requirements of the GDPR. 

% In the evolving landscape of algorithmic recourse, Feasible Recourse Plan via Diverse Interpolation(Nguyen et al. 2023)~\cite{nguyen2023feasible} proposes a novel approach to generate diverse recourse plans, integrating multiple criteria such as cost, validity, diversity, and adherence to the data manifold, aiming to align with the different preferences of users and the data structure. It emphasizes the construction of diverse prototypes using training data, contributing a comprehensive method to produce insightful recourse plans and data structure-aligned. Additionally, it presents state-of-the-art techniques to generate counterfactual points along the data manifold, enhancing the diversity and distribution understanding of the data. 
% In the context of algorithmic recourse, our approach uniquely emphasizes dynamic pathways for altering decision outcomes of models, focusing on mapping out feasible paths that go beyond mere adjustments of data points. Unlike the methods proposed by FACE and Automated Decisions and the GDPR, which largely center on generating counterfactual explanations using shortest path distances within a graph-based framework or under GDPR compliance, our method delves into the continuous and practical transformations required to achieve favorable outcomes. By emphasizing the entire journey of changes rather than isolated adjustments, our methodology not only integrates the user's actual constraints and possibilities but also aligns with diverse user preferences and data structures as highlighted by Feasible Recourse Plan via Diverse Interpolation. This path-oriented approach offers a holistic and action-oriented solution, providing a broader perspective on recourse actions and ensuring a deeper integration with the practical realities faced by users.

% Our approach is particularly effective in complex scenarios where multiple changes are feasible, and each change impacts the decision in different ways. Our algorithm stands out by handling these multi-criteria cases effectively, ensuring that the recourse provided is not only theoretically optimal but also practically feasible and fair.

% -

In decision-making systems, the integration of recourse mechanisms significantly influences how decisions impact individuals in critical sectors such as finance, healthcare, and education. These mechanisms, designed to enhance the interpretability and accountability of machine learning and AI models, provide individuals a clear pathway to understand and potentially modify decisions affecting their lives ~\cite{gunning2019xai,arrieta2020explainable,guidotti2018survey}. In finance, recourse helps loan applicants understand necessary improvements for higher loan approval chances ~\cite{barocas2017fairness}. In healthcare, it offers patients explanations and potential lifestyle changes aligned with their treatment plans ~\cite{bastani2017interpretability,bertossi2020asp,liang2014deep}. Similarly, in education, recourse aids in analyzing student assessments and admissions, providing insights and improvement suggestions ~\cite{fiok2022explainable,khosravi2022explainable}.

Most algorithms for recourse in decision-making rely on gradient methods, assuming differentiable cost functions associated with decision models ~\cite{shao2022gradient}. These algorithms typically offer one-hop solutions, which are direct changes an individual can make to alter the decision outcome. However, these methods struggle with discrete decisions or thresholds and lack comprehensive theoretical analysis, relying on approximations that compromise efficacy and transparency.

To address these issues, our work proposes innovative solutions for non-differentiable cost functions, expanding the application of recourse strategies. This approach enhances the ability to handle various decision models not amenable to traditional approaches, providing a more principled foundation for analyzing and implementing recourse.

Contemporary methods like FACE (Poyiadzi et al. 2020)\cite{poyiadzi2020face} and GDPR (Wachter et al. 2017)\cite{wachter2017counterfactual}’s counterfactual explanations serve as practical implementations addressing interpretability and operational feasibility in high-dimensional spaces. FACE employs shortest path distances with density-weighted metrics to generate feasible and actionable counterfactual explanations \cite{poyiadzi2020face}. Automated Decisions and the GDPR discuss counterfactual explanations in the GDPR context, introducing foundational ideas extended using shortest path methods for actionable insights (Wachter et al. 2017)\cite{wachter2017counterfactual}.

Building on these innovations, our approach, Feasible Recourse Plan via Diverse Interpolation, proposes a novel method to generate diverse recourse plans, integrating multiple criteria such as cost, validity, diversity, and adherence to the data manifold, Feasible Recourse Plan via Diverse Interpolation(Nguyen et al. 2023)~\cite{nguyen2023feasible}. This method emphasizes constructing diverse prototypes using training data, enhancing the generation of insightful recourse plans and understanding the data structure. Our approach focuses on mapping out feasible paths for altering decision outcomes, integrating user constraints and preferences, and ensuring practical and fair solutions.