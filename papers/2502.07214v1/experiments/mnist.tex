\subsection{MNIST}

In this experiment, we explore the influence of multiple criteria on the MNIST dataset.

The MNIST dataset is a large collection of handwritten digit images, widely utilized for training image processing systems and benchmarking machine learning algorithms.

We commence by extracting a subset from the original MNIST dataset. Subsequently, we construct a K-Nearest Neighbors (KNN) graph using the L2 metric for distance calculation. Leveraging this graph, we introduce two cost criteria: 1) the difference between the labels of the images, and 2) the L2 distance between the images themselves. Furthermore, we employ the max function to update the former cost.

In essence, the aim is to identify the shortest path transition from the source to the target, while adhering to a specific threshold of label difference.

The following table and Figure 2 present the results of our experiment.
\begin{table}[ht]
\centering


\begin{tabular}{|c|l|c|c|}
    \hline
    Image & Path& Cost$_1$& Cost$_2$\\
    \hline
    1 & 2 $\rightarrow$ 4 $\rightarrow$ 4 $\rightarrow$ 6$\rightarrow$ 8 & 2 & 31 \\
    2 & 2 $\rightarrow$3$\rightarrow$ 8 & 5 & 15 \\
    3 & 2 $\rightarrow$ 8 & 6 & 8.7 \\
    \hline
\end{tabular}
\caption{Paths from 2 to 8}
\label{tab:mnist_path}
\end{table}
% We compare our approach with the Fast Approximate Counterfactual Explanations (FACE) method. In FACE, we use Kernel Density Estimation (KDE) as the cost for each edge and apply Dijkstra's algorithm. The resulting images are presented in the figure 3. From this figure, it is evident that the FACE algorithm can only generate a single path.

% By comparing the results of our multi-criteria approach with FACE, we aim to gain insights into the effectiveness of different cost functions in generating counterfactual explanations for the MNIST dataset.

% Our experiments clearly illustrated that while the FACE method is constrained to producing a single transformation path, our multi-cost approach facilitates exploring a wider array of potential transformations. This capacity not only enriches the set of counterfactual explanations available but also enhances the interpretability and adaptability of the model by providing diverse alternatives that meet different criteria.