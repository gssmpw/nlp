\subsection{Adult}


    In this study, we introduce a novel method based on multi-cost shortest path algorithms and validate its effectiveness on the Adult dataset. The Adult dataset is Here are the specific steps and methodologies we employed in our experiment.


We started by standardizing the "Adult" dataset to ensure uniformity across various features, allowing for more accurate comparisons. Using a random forest model with 100 trees, then we sorted this standardized dataset and also maintained a non-standardized version for diverse analytical needs. Based on a set of predefined actionable criteria, we established an actionable dataset. During this process, some criteria were set as immutable to reflect real-world conditions. For instance, in the Adult dataset, gender was designated as immutable. From this actionable dataset, we randomly selected 256 instances that met predefined actionable criteria for further analysis. These instances were assessed using a second random forest model to identify the data points with the lowest success probability, indicating areas with the highest challenges.

For data analysis, we created a KNN-graph (k=4) from the standardized dataset and applied our multi-cost shortest path algorithm to explore optimal changes in three key dimensions: education, hours per week, and age. We used the non-standardized data to calculate the shortest paths based on these criteria, aiming to find the most optimized change strategies within these three dimensions. To ensure a realistic application, we designed different cost functions for each criterion. Taking "age" for example, when designing the cost function, it is notable that age can not decrease during the whole recourse process. %As mentioned in Algorithm~\ref{algo:backtracking}, we use Pareto optimal 2 to prune the paths, retaining only a representative set of costs.
    
Figure~\ref{fig:enter-label} visualized the results of the experiment that getting multiple flipped counterfactuals. These four paths, originating from the same starting point where the success probability is the worst, illustrate the diverse impacts caused by Pareto optimal solutions, and all four paths ultimately reach an endpoint where the success probability exceeds 0.75. The introduction of Kernel Density Estimation (KDE) into the algorithm demonstrates a better trade-off with other costs, enhancing the algorithm's ability to navigate through the solution space effectively. To better visualize and understand the patterns in the data, we applied Principal Component Analysis (PCA) for dimension reduction.

%In conclusion, the application of our algorithm on the Adult dataset has demonstrated significant capabilities in addressing complex decision-making scenarios where multiple factors such as age and hours worked per week impact income predictions. Our approach effectively identifies optimal paths for improving individual income predictions by leveraging a multi-criteria framework. This enables a tailored intervention strategy that considers multiple aspects of an individual’s profile.