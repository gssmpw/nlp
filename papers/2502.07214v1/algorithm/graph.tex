Our approach involves handling the minimization of cost for a single edge, which can be analogized to the concept of finding the shortest path. We manage this by maintaining and updating a table to capture the trade-offs among various types of costs. This table is instrumental in recording and managing the interplay of different cost criteria for a given source node $s$ and destination node $v$.

\begin{algorithm}
\caption{Generate Edges and Targets}
\begin{algorithmic}[1]
    \State \textbf{Input} $V$
    \State $E \gets \{\}$
    \For{vertex $u$ in $V$}
        \For{vertex $v$ in $V$}
            \If{is-connected(vertex $u$, vertex $v$)}
                \State $E$.add(make-edge(vertex $u$, vertex $v$))
            \EndIf
        \EndFor
    \EndFor
    \State targets $\gets \{\}$
    \For{vertex $u$ in $V$}
        \If{is-target(vertex $u$)}
            \State targets.add(vertex $u$)
        \EndIf
    \EndFor
    \State \%make vertex-dummy
    \For{vertex $t$ in targets}
        \State $E$.add(make-edge(vertex $t$, vertex-dummy))
        \State \% make the cost of edge zero
    \EndFor
\end{algorithmic}
\end{algorithm}

Firstly, we extract data point from high dimensional data space, and then we build a graph for those data points through k-NN. Initially, an empty set of edges, denoted as $E$, is created, and then iterates through each pair of vertices in the input set $V$, checking if there is a connection between them. If a connection is found, an edge is created between the two vertices and added to the set $E$. Subsequently, the algorithm identifies target vertices by iterating through each vertex in $V$ and checking if it is a target vertex. Target vertices are added to a set called `targets`. In order to address this problem. A dummy vertex is created and we introduce it into to the graph structure. For each target vertex identified, an edge with zero cost is added between the target vertex and the dummy vertex. 

Secondly, user then decides on the properties of target instance - counterfactual. Under what circumstances is it considered a successful flip? Before that, here are some rule your customized cost function need to follow:
Assume s :[0,1]x[0,1] → [0,1], make $\mu_{A \cup B}(X)$ = s[$\mu_A(X) \cup \mu_B(X)$]

\begin{itemize}
    \item [(1)]
    Boundary conditions:\\
    s(1,1) = 1, and s$(\mu_A(X),0)$ = s(0,$\mu_A(X)$) = $\mu_A(X)$
    \item [(2)]
    Monotonicity:\\
    if $\mu_A(X)$<$\mu_c(X)$ and $\mu_B(X)$<$\mu_D(X)$\\
    then $s(\mu_A(X), \mu_B(X)) \leq s(\mu_C(X), \mu_D(X))$
    \item [(3)]
    commutativity:\\
    s($\mu_A(X), \mu_B(X)$) = s($\mu_B(X), \mu_A(X)$)
    \item [(4)]
    Associativity:\\
    s($\mu_A(X), s(\mu_B(X), \mu_C(X)$)) = s(s($\mu_A(X), \mu_B(X)), \mu_C(X)$)
\end{itemize}
For example, taking the maximum value and addition. After all preparatory work is completed, we can start our recourse algorithm.



