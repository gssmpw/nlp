
\textbf{time complexity}

When we run the dynamic programming equation (1) in multi-shortest path algorithm. The time complexity analysis involves two main steps: concatenate and prune, which are repeated for at most $V-1$ iterations based on the Bellman-Ford algorithm for all edges. In the concatenate step, for each vertex v, we combine the distance tables $D_{u \in in(v)}$ from all incoming neighbors u $\in$ in(v) with the respective edge costs $d(u, v)$. This step has a time complexity of $O(n)$, where n is the size of the distance table $D$.

The subsequent pruning step aims to retain only the optimal solutions in the concatenated distance table. The time complexity of this pruning operation depends on the number of criteria considered. When dealing with two criteria, the pruning step leverages the theory of maximal point sets, resulting in a time complexity of O(n). However, when the number of criteria exceeds two, the time complexity increases to $O(n\log^{n-3}\log\log n)$.

Considering both the concatenate and prune steps, the overall time complexity of the multi-shortest path algorithm is $O(n|V||E|)$. Here, $|V|$ and$ |E|$ represent the number of vertices and edges in the graph respectively. The factor n accounts for the size of the distance table $D$, which may vary depending on the specific problem instance and the number of criteria considered.

 \textbf{correctness}
% we need to prove two things:\\

% \begin{itemize}
   
%      \item [(1)]
%     All optimal solution are in $D$
%      \item [(2)]
%     All solution in $D$ is optimal (when without constraint for $D$ size)
% \end{itemize}
% once we can prove (1) and (2) are correct, which means (1) equals to (2),then we can prove our algorithm is correct. Here is the prove for item 1, since the dp equation has already consider all edge that link in to $v$, we can promised that we find all possible temporary solution, which containing all optimal solutions.
% As for item 2, Assume $f(v,\ell)$ is Pareto optimal and $\exists$ $d_u \in f(u, \ell-1)$ is not Pareto optimal solution. Since the feature of Monotonicity of cost function. After $d_u \in f(u, \ell-1)$ connect with weight cost$ \overline{uv} $ still is not Pareto optimal solution. After step of prune, those worse solution would be dominate, which means $f(v,\ell)$ will not be the Pareto optimal solution. $\to \gets .$ Q.E.D. that, $\forall d_v \in f(v, \ell)$ is Pareto optimal, then $\forall _{u \in \text{in}(v)} \left(f(u, \ell - 1) + d(u, v)\right) and \forall d_v \in f(v, \ell - 1)$ must be Pareto optimal solution. To be more specific, only when last iteration are Pareto optimal, can next iteration still contain all Pareto optimal solution. the only exception which we can't explain is the source vertex  $f(v, 0)$, which doesn't have any coming edge. However, This first step is established, so it is proved.

To prove the correctness of the multi-shortest path algorithm, we need to establish two key points:


\begin{enumerate}
    \item 1. All optimal solutions are present in the distance table $D$.
    \item 2. All solutions in the distance table $D$ are optimal (when there is no constraint on the size of $D$).
\end{enumerate}

If we can prove that (1) and (2) are correct, implying that they are equivalent, then we can establish the correctness of our algorithm.

Regarding item (1), since the dynamic programming equation considers all edges that link to vertex $v$, we can guarantee that we find all possible temporary solutions, which contain all optimal solutions.

As for item (2), assume that $f(v, \ell)$ is Pareto optimal, and there exists $d_u \in f(u, \ell-1)$ that is not a Pareto optimal solution. Due to the monotonicity feature of the cost function, after $d_u \in f(u, \ell-1)$ is connected with the weight cost $d(u, v)$, it still will not be a Pareto optimal solution. After the pruning step, these worse solutions would be dominated, which means $f(v, \ell)$ will not be the Pareto optimal solution. This leads to the conclusion that for all $d_v \in f(v, \ell)$ to be Pareto optimal, all $f(u, \ell-1)$ for $u \in \text{in}(v)$ and $f(v, \ell-1)$ must be Pareto optimal solutions. In other words, only when the previous iteration contains Pareto optimal solutions can the next iteration still contain all Pareto optimal solutions. The only exception we cannot explain is the source vertex $f(v, 0)$, which does not have any incoming edges. However, this initial step is established, so it is proved.\\

\textbf{Convergence property}\\
If $s \rightsquigarrow u \rightarrow v$ is shortest path in $G$ for some $u,v \in V$ ,and if $u.d = \delta(s,u)$ at any time prior to relaxing $edge(u,v)$, then $v.d = \delta(s,v)$ at all times afterward.\\


