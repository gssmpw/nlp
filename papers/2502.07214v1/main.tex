%\documentclass{ecai} 
\documentclass[runningheads]{llncs}
%
\usepackage[T1]{fontenc}
% T1 fonts will be used to generate the final print and online PDFs,
% so please use T1 fonts in your manuscript whenever possible.
% Other font encondings may result in incorrect characters.
%
\usepackage{comment}

\usepackage{graphicx}
% Used for displaying a sample figure. If possible, figure files should
% be included in EPS format.
%
% If you use the hyperref package, please uncomment the following two lines
% to display URLs in blue roman font according to Springer's eBook style:
%\usepackage{color}
%\renewcommand\UrlFont{\color{blue}\rmfamily}
%\urlstyle{rm}
%
%%%%%%%%%%%%%%%%%%%%%%%%%%%%%%%%%%%%%%%%%%%%%%%%%%%%%%%%%%%%%%%%%%%%%%%%

%%% Load any packages you require here. 

\usepackage{latexsym}
\usepackage{amssymb}
\usepackage{amsmath}
\let\proof\relax \let\endproof\relax \usepackage{amsthm}
\usepackage{booktabs}
\usepackage{enumitem}
\usepackage{graphicx}
\usepackage{color}
\usepackage{algorithm}
\usepackage{algpseudocode}


\usepackage{caption}
\usepackage{subcaption}

\algnewcommand\algorithmicforeach{\textbf{for each}}
\algdef{S}[FOR]{ForEach}[1]{\algorithmicforeach\ #1\ \algorithmicdo}


%%%%%%%%%%%%%%%%%%%%%%%%%%%%%%%%%%%%%%%%%%%%%%%%%%%%%%%%%%%%%%%%%%%%%%%%

%%% Define any theorem-like environments you require here.

\newtheorem{fact}[theorem]{Fact}

%%%%%%%%%%%%%%%%%%%%%%%%%%%%%%%%%%%%%%%%%%%%%%%%%%%%%%%%%%%%%%%%%%%%%%%%

%%% Define any new commands you require here.

\newcommand{\BibTeX}{B\kern-.05em{\sc i\kern-.025em b}\kern-.08em\TeX}
\def\R{\mathbb{R}}
\newcommand{\rey}[1]{{\color{blue}Rey: #1 }}
\newcommand{\todo}[1]{{\color{red}TO DO: #1 }}

%%
%% \BibTeX command to typeset BibTeX logo in the docs
\AtBeginDocument{%
  \providecommand\BibTeX{{%
    Bib\TeX}}}

%% Rights management information.  This information is sent to you
%% when you complete the rights form.  These commands have SAMPLE
%% values in them; it is your responsibility as an author to replace
%% the commands and values with those provided to you when you
%% complete the rights form.
\begin{document}

\begin{frontmatter}

%%% Use this command to specify your submission number.
%%% In doubleblind mode, it will be printed on the first page.


%%% Use this command to specify the title of your paper.
%\title{Algorithmic Recourse in (non-differentiable) Multi-cost Function}
\title{Pareto Optimal Algorithmic Recourse in Multi-cost Function}

%%% Use this combinations of commands to specify all authors of your 
%%% paper. Use \fnms{} and \snm{} to indicate everyone's first names 
%%% and surname. This will help the publisher with indexing the 
%%% proceedings. Please use a reasonable approximation in case your 
%%% name does not neatly split into "first names" and "surname".
%%% Specifying your ORCID digital identifier is optional. 
%%% Use the \thanks{} command to indicate one or more corresponding 
%%% authors and their email address(es). If so desired, you can specify
%%% author contributions using the \footnote{} command.

\author{Wen-Ling Chen \and Hong-Chang Huang \and
Kai-Hung Lin \and Shang-Wei Hwang \and Hao-Tsung Yang}
%

\authorrunning{W. Chen et al.}
% First names are abbreviated in the running head.
% If there are more than two authors, 'et al.' is used.
%
\institute{National Central University, Taiwan}
%
\maketitle              % typeset the header of the contribution
%
%%% Use this environment to include an abstract of your paper.
\begin{abstract}
%In real-world machine learning applications for decision-making, such as determining loan approvals or credit scores, significant efforts have been made recently to find diverse recommendations that explain the algorithmic decisions. These diverse recommendations provide actionable feedback to users, catering to a wide range of user preferences. 

In decision-making systems, algorithmic recourse aims to identify minimal-cost actions to alter an individual's features, thereby obtaining a desired outcome. This empowers individuals to understand, question, or alter decisions that negatively affect them. However, due to the variety and sensitivity of system environments and individual personalities, quantifying the cost of a single function is nearly impossible while considering multiple criteria situations. Most current recourse mechanisms use gradient-based methods that assume cost functions are differentiable, often not applicable in real-world scenarios, resulting in sub-optimal solutions that compromise various criteria. These solutions are typically intractable and lack rigorous theoretical foundations, raising concerns regarding interpretability, reliability, and transparency from the explainable AI (XAI) perspective. To address these issues, this work proposes an algorithmic recourse framework that handles non-differentiable and discrete multi-cost functions. By formulating recourse as a multi-objective optimization problem and assigning weights to different criteria based on their importance, our method identifies Pareto optimal recourse recommendations. To demonstrate scalability, we incorporate the concept of $\epsilon$-net, proving the ability to find approximated Pareto optimal actions. Experiments show the trade-off between different criteria and the method's scalability in large graphs. Compared to current heuristic practices, our approach provides a stronger theoretical foundation and better aligns recourse suggestions with real-world requirements.

\keywords{multi-objective optimization \and recourse \and shortest path \and Pareto optimality.}

%The multi-criteria decision-making paradigm holds promise for propelling recourse analysis toward more effective, generalizable, and transparent solutions, even in high-dimensional decision spaces characterizing numerous real-world applications.
\end{abstract}

\end{frontmatter}


%%
%% The "title" command has an optional parameter,
%% allowing the author to define a "short title" to be used in page headers.


%%
%% The "author" command and its associated commands are used to define
%% the authors and their affiliations.
%% Of note is the shared affiliation of the first two authors and the
%% "authornote" and "authornotemark" commands

%% This command processes the author affiliation and title
%% information and builds the first part of the formatted document.

\section{Introduction}
\section{Introduction}


\begin{figure}[t]
\centering
\includegraphics[width=0.6\columnwidth]{figures/evaluation_desiderata_V5.pdf}
\vspace{-0.5cm}
\caption{\systemName is a platform for conducting realistic evaluations of code LLMs, collecting human preferences of coding models with real users, real tasks, and in realistic environments, aimed at addressing the limitations of existing evaluations.
}
\label{fig:motivation}
\end{figure}

\begin{figure*}[t]
\centering
\includegraphics[width=\textwidth]{figures/system_design_v2.png}
\caption{We introduce \systemName, a VSCode extension to collect human preferences of code directly in a developer's IDE. \systemName enables developers to use code completions from various models. The system comprises a) the interface in the user's IDE which presents paired completions to users (left), b) a sampling strategy that picks model pairs to reduce latency (right, top), and c) a prompting scheme that allows diverse LLMs to perform code completions with high fidelity.
Users can select between the top completion (green box) using \texttt{tab} or the bottom completion (blue box) using \texttt{shift+tab}.}
\label{fig:overview}
\end{figure*}

As model capabilities improve, large language models (LLMs) are increasingly integrated into user environments and workflows.
For example, software developers code with AI in integrated developer environments (IDEs)~\citep{peng2023impact}, doctors rely on notes generated through ambient listening~\citep{oberst2024science}, and lawyers consider case evidence identified by electronic discovery systems~\citep{yang2024beyond}.
Increasing deployment of models in productivity tools demands evaluation that more closely reflects real-world circumstances~\citep{hutchinson2022evaluation, saxon2024benchmarks, kapoor2024ai}.
While newer benchmarks and live platforms incorporate human feedback to capture real-world usage, they almost exclusively focus on evaluating LLMs in chat conversations~\citep{zheng2023judging,dubois2023alpacafarm,chiang2024chatbot, kirk2024the}.
Model evaluation must move beyond chat-based interactions and into specialized user environments.



 

In this work, we focus on evaluating LLM-based coding assistants. 
Despite the popularity of these tools---millions of developers use Github Copilot~\citep{Copilot}---existing
evaluations of the coding capabilities of new models exhibit multiple limitations (Figure~\ref{fig:motivation}, bottom).
Traditional ML benchmarks evaluate LLM capabilities by measuring how well a model can complete static, interview-style coding tasks~\citep{chen2021evaluating,austin2021program,jain2024livecodebench, white2024livebench} and lack \emph{real users}. 
User studies recruit real users to evaluate the effectiveness of LLMs as coding assistants, but are often limited to simple programming tasks as opposed to \emph{real tasks}~\citep{vaithilingam2022expectation,ross2023programmer, mozannar2024realhumaneval}.
Recent efforts to collect human feedback such as Chatbot Arena~\citep{chiang2024chatbot} are still removed from a \emph{realistic environment}, resulting in users and data that deviate from typical software development processes.
We introduce \systemName to address these limitations (Figure~\ref{fig:motivation}, top), and we describe our three main contributions below.


\textbf{We deploy \systemName in-the-wild to collect human preferences on code.} 
\systemName is a Visual Studio Code extension, collecting preferences directly in a developer's IDE within their actual workflow (Figure~\ref{fig:overview}).
\systemName provides developers with code completions, akin to the type of support provided by Github Copilot~\citep{Copilot}. 
Over the past 3 months, \systemName has served over~\completions suggestions from 10 state-of-the-art LLMs, 
gathering \sampleCount~votes from \userCount~users.
To collect user preferences,
\systemName presents a novel interface that shows users paired code completions from two different LLMs, which are determined based on a sampling strategy that aims to 
mitigate latency while preserving coverage across model comparisons.
Additionally, we devise a prompting scheme that allows a diverse set of models to perform code completions with high fidelity.
See Section~\ref{sec:system} and Section~\ref{sec:deployment} for details about system design and deployment respectively.



\textbf{We construct a leaderboard of user preferences and find notable differences from existing static benchmarks and human preference leaderboards.}
In general, we observe that smaller models seem to overperform in static benchmarks compared to our leaderboard, while performance among larger models is mixed (Section~\ref{sec:leaderboard_calculation}).
We attribute these differences to the fact that \systemName is exposed to users and tasks that differ drastically from code evaluations in the past. 
Our data spans 103 programming languages and 24 natural languages as well as a variety of real-world applications and code structures, while static benchmarks tend to focus on a specific programming and natural language and task (e.g. coding competition problems).
Additionally, while all of \systemName interactions contain code contexts and the majority involve infilling tasks, a much smaller fraction of Chatbot Arena's coding tasks contain code context, with infilling tasks appearing even more rarely. 
We analyze our data in depth in Section~\ref{subsec:comparison}.



\textbf{We derive new insights into user preferences of code by analyzing \systemName's diverse and distinct data distribution.}
We compare user preferences across different stratifications of input data (e.g., common versus rare languages) and observe which affect observed preferences most (Section~\ref{sec:analysis}).
For example, while user preferences stay relatively consistent across various programming languages, they differ drastically between different task categories (e.g. frontend/backend versus algorithm design).
We also observe variations in user preference due to different features related to code structure 
(e.g., context length and completion patterns).
We open-source \systemName and release a curated subset of code contexts.
Altogether, our results highlight the necessity of model evaluation in realistic and domain-specific settings.






\begin{comment}
\section{RELATED WORK}
\putsec{related}{Related Work}

\noindent \textbf{Efficient Radiance Field Rendering.}
%
The introduction of Neural Radiance Fields (NeRF)~\cite{mil:sri20} has
generated significant interest in efficient 3D scene representation and
rendering for radiance fields.
%
Over the past years, there has been a large amount of research aimed at
accelerating NeRFs through algorithmic or software
optimizations~\cite{mul:eva22,fri:yu22,che:fun23,sun:sun22}, and the
development of hardware
accelerators~\cite{lee:cho23,li:li23,son:wen23,mub:kan23,fen:liu24}.
%
The state-of-the-art method, 3D Gaussian splatting~\cite{ker:kop23}, has
further fueled interest in accelerating radiance field
rendering~\cite{rad:ste24,lee:lee24,nie:stu24,lee:rho24,ham:mel24} as it
employs rasterization primitives that can be rendered much faster than NeRFs.
%
However, previous research focused on software graphics rendering on
programmable cores or building dedicated hardware accelerators. In contrast,
\name{} investigates the potential of efficient radiance field rendering while
utilizing fixed-function units in graphics hardware.
%
To our knowledge, this is the first work that assesses the performance
implications of rendering Gaussian-based radiance fields on the hardware
graphics pipeline with software and hardware optimizations.

%%%%%%%%%%%%%%%%%%%%%%%%%%%%%%%%%%%%%%%%%%%%%%%%%%%%%%%%%%%%%%%%%%%%%%%%%%
\myparagraph{Enhancing Graphics Rendering Hardware.}
%
The performance advantage of executing graphics rendering on either
programmable shader cores or fixed-function units varies depending on the
rendering methods and hardware designs.
%
Previous studies have explored the performance implication of graphics hardware
design by developing simulation infrastructures for graphics
workloads~\cite{bar:gon06,gub:aam19,tin:sax23,arn:par13}.
%
Additionally, several studies have aimed to improve the performance of
special-purpose hardware such as ray tracing units in graphics
hardware~\cite{cho:now23,liu:cha21} and proposed hardware accelerators for
graphics applications~\cite{lu:hua17,ram:gri09}.
%
In contrast to these works, which primarily evaluate traditional graphics
workloads, our work focuses on improving the performance of volume rendering
workloads, such as Gaussian splatting, which require blending a huge number of
fragments per pixel.

%%%%%%%%%%%%%%%%%%%%%%%%%%%%%%%%%%%%%%%%%%%%%%%%%%%%%%%%%%%%%%%%%%%%%%%%%%
%
In the context of multi-sample anti-aliasing, prior work proposed reducing the
amount of redundant shading by merging fragments from adjacent triangles in a
mesh at the quad granularity~\cite{fat:bou10}.
%
While both our work and quad-fragment merging (QFM)~\cite{fat:bou10} aim to
reduce operations by merging quads, our proposed technique differs from QFM in
many aspects.
%
Our method aims to blend \emph{overlapping primitives} along the depth
direction and applies to quads from any primitive. In contrast, QFM merges quad
fragments from small (e.g., pixel-sized) triangles that \emph{share} an edge
(i.e., \emph{connected}, \emph{non-overlapping} triangles).
%
As such, QFM is not applicable to the scenes consisting of a number of
unconnected transparent triangles, such as those in 3D Gaussian splatting.
%
In addition, our method computes the \emph{exact} color for each pixel by
offloading blending operations from ROPs to shader units, whereas QFM
\emph{approximates} pixel colors by using the color from one triangle when
multiple triangles are merged into a single quad.


\end{comment}

%\todo{multi-criteria shortest path}

\section{PROBLEM DEFINITION}
\section{Problem definition\label{sec:probdef}}
%\com{I just wrote an initial problem definition as a starting point. Probably overcomplicated. I did not explicitly state that this problem is describing a global embedding, but I think it should be specified. Also, for simplicity, I formulated the problem of embedding the system for the complete $\mathcal{X} \times \mathcal{U}$ space, instead of for a region of it.}

Consider a nonlinear dynamical system given by the \emph{state-space} (SS) representation 
\begin{subequations}
	\label{eq:nl_dyn}
\begin{align}
	%\begin{aligned}
		\diff x(t) & = f(x(t), u(t)), \\
		y(t)       & = h(x(t), u(t)),
	%\end{aligned}
\end{align}
\end{subequations}
where $t \in \mathbb{T}$ is time, $\diff$ is $\diff x(t)= \frac{d}{dt}x(t)$ in the continuous-time case with $\mathbb{T}=\mathbb{R}$ and $\diff x(t)= x(t+1)$ in the discrete-time case with $\mathbb{T}=\mathbb{Z}$, $x(t) \in \mathcal{X} \subseteq \R^{\nx}$, $u(t) \in \mathcal{U} \subseteq \R^{\dnu}$ and $y(t) \in \mathcal{Y} \subseteq \R^{\ny}$ with $\nx, \dnu, \ny \in \mathbb{N}$ are the state, input, and output signals associated with the system, respectively. The functions $f: \R^{\nx} \times \R^{\dnu} \to \R^{\nx}$ and $h: \R^{\nx} \to \R^{\ny}$ are continuously differentiable once, i.e., $f, h \in \mathcal{C}_1$, and the solutions of \eqref{eq:nl_dyn} have left compact support, they are forward complete and unique, i.e., for any initial condition $x(t_0)$ and any input trajectory $u(t)$, the solutions of \eqref{eq:nl_dyn} are uniquely determined for all $t \geq t_0$. Moreover, $(0, 0) \in \mathcal{X} \times \mathcal{U}$. Finally we denote the behavior of \eqref{eq:nl_dyn}, i.e., the set of all possible trajectories by
%
\begin{equation}
	\label{eq:nl_behavior}
	\mathcal{B} \coloneq \left\{ (x, u, y) \in \left(\mathcal{X}, \mathcal{U}, \mathcal{Y}\right)^{\mathbb{T}} \mid (x, u, y) ~ \text{satisfy}~\eqref{eq:nl_dyn} \right\},
\end{equation}
%
where $\mathcal{X}^{\mathbb{T}}$ denotes the set of all signals $\mathbb{T} \rightarrow \mathcal{X}$ with left compact support.

In this paper, our objective is to automatically convert the system description \eqref{eq:nl_dyn} into a \emph{linear parameter-varying} (LPV) representation of the form
%
\begin{subequations}
	\label{eq:lpv_dyn}
	\begin{align}
		\diff x(t) & = A(p(t))x(t) + B(p(t))u(t); \\
		y(t)       & = C(p(t))x(t) + D(p(t))u(t),
	\end{align}
\end{subequations}
%
where $p \in \Pset \subseteq \R^{\np}$ with $\np \in \mathbb{N}$ is the scheduling variable, $A : \Pset \rightarrow \R^{\nx \times \nx}$, $B : \Pset \rightarrow \R^{\nx \times \dnu}$, $C : \Pset \rightarrow \R^{\ny \times \nx}$, and $D : \Pset \rightarrow \R^{\ny \times \dnu}$ are smooth real-valued matrix functions, and the solutions of \eqref{eq:lpv_dyn} with left compact support are forward complete and unique. For a given scheduling trajectory $p(t)\in\mathcal{P}$, the behavior of \eqref{eq:lpv_dyn} is defined as
%
\begin{equation}
	\label{eq:p_behavior}
	\mathcal{B}_p \coloneq \left\{ (x, u, y) \in \left(\mathcal{X}, \mathcal{U}, \mathcal{Y}\right)^{\mathbb{T}} \mid (x, u, y, p) ~ \text{satisfy}~\eqref{eq:lpv_dyn} \right\},
\end{equation}
%
and the behavior of \eqref{eq:lpv_dyn} for all possible scheduling trajectories is defined as
%
\begin{equation}
	\label{eq:lpv_behavior}
	\mathcal{B}_{\text{LPV}} \coloneq \bigcup_{p \in \Pset^{\mathbb{T}}} \mathcal{B}_p(p).
\end{equation}


We consider the LPV representation \eqref{eq:lpv_dyn} to be a so-called global LPV embedding of \eqref{eq:nl_dyn} if in addition, we can construct a so-called \emph{scheduling map} $\eta : \mathcal{X} \times \mathcal{U} \rightarrow \Pset$,
%
\begin{equation}
	\label{eq:scheduling_map}
	p(t) = \eta(x(t), u(t)),
\end{equation}
such that
%
\begin{subequations}
\label{eq:LPV_realization}
	\begin{align}
		f(x, u) & = A(\eta(x, u)) x + B(\eta(x, u)) u, \\
		h(x, u) & = C(\eta(x, u)) x + D(\eta(x, u)) u, 
	\end{align}
\end{subequations}
%
for all $(x, u) \in \mathcal{X} \times \mathcal{U}$. This gives that $\eta(\mathcal{X},\mathcal{U})\subseteq \mathcal{P} \subseteq \mathbb{R}^{n_\mathrm{p}}$, where $\mathcal{P}$ is often chosen be a compact convex set, if the model is further utilized for analysis or control synthesis.
Consequently, the behavior of the nonlinear system is included (embedded) in the behavior of the LPV system, i.e., $\mathcal{B} \subseteq \mathcal{B}_{\text{LPV}}$. In the next section, we will discuss the proposed method to achieve this objective.


% , and the scheduling variable $p(t)$ can be expressed as a function of the states and inputs of the system via the so-called \emph{scheduling map} $\eta$ as $p(t) = \eta(x(t), u(t))$, \com{which is often expected to belong to a certain function class, i.e. affine, polynomial, etc.} In the next section, we will discuss the proposed method to achieve this objective

\section{MULTI-CRITERIA RECOURSE}
\label{subsec:algo}
%In this context, our objective is to provide conceivable actionable and feasible paths for transforming a specific data point (akin to the input for recourse) into its plausible counterfactual.
\begin{algorithm}[h!]
\caption{Gait-Net-augmented Sequential CMPC}
\label{alg:gaitMPC}
\begin{algorithmic}[1]
\Require $\mathbf q, \: \dot{\mathbf q}, \: \mathbf q^\text{cmd}, \: \dot{\mathbf q}^\text{cmd}$
\State \textbf{intialize} $\bm x_0 = f_\text{j2m}(\mathbf q, \: \dot{\mathbf q}), \: \bm u^0 =\bm u_\text{IG}, \: dt^0 = 0.05$ 
\State $\{ \mathbf q^\text{ref},\:\dot{\mathbf q}^\text{ref},\:\bm p_f^\text{ref}\} = f_\text{ref} \big(\mathbf q, \: \dot{\mathbf q}, \: \mathbf q^\text{cmd}, \: \dot{\mathbf q}^\text{cmd} \big)$
\State $\bm x^\text{ref} = f_\text{j2m}(\mathbf q^\text{ref},\:\dot{\mathbf q}^\text{ref},\:\bm p_f^\text{ref})$
\State $ j = 0$ 
\While{$j \leq j_\text{max} \:\text{and}\: \bm \eta \leq \delta \bm u  $} 
\State $\delta \bm u^{j} = \texttt{cmpc}(\bm x^\text{ref},\:\bm p_f^\text{ref},\:\bm p_c^\text{ref},\: \bm x_0,\: dt^j, \: \bm u^j)$
\State $\bm u^{j+1} = \bm u^j + \delta \bm u^j$ 
\State $dt^{j+1} = \Pi_\text{GN}(\mathbf q, \: \dot{\mathbf q},\: \bm p_f^{j})$
\State $\{ \bm x^\text{ref},\:\bm p_f^\text{ref}\}= f_\text{IK}(\bm p_f^{j},\:\bm p_c^{j},\: dt^{j+1})$
\State $j=j+1$
\EndWhile \\
\Return $\bm u^{j+1} $
\end{algorithmic}
\end{algorithm}
%\begin{tikzpicture}[scale=0.6]
\pgfdeclarelayer{bg}
\pgfsetlayers{bg,main}
\node[vertex, fill=white, fill opacity=0.8, text opacity=1] (0) at (3.56, 4.1) {\scriptsize 0};
\node[vertex, fill=white, fill opacity=0.8, text opacity=1] (1) at (1.5, 3.91) {\scriptsize 1};
\node[vertex, fill=white, fill opacity=0.8, text opacity=1] (2) at (0.5, 3.37) {\scriptsize 2};
\node[vertex, fill=white, fill opacity=0.8, text opacity=1] (4) at (2.8, 1.21) {\scriptsize 4};
\node[vertex, fill=white, fill opacity=0.8, text opacity=1] (6) at (3.56, 2.29) {\scriptsize 6};
\node[vertex, fill=white, fill opacity=0.8, text opacity=1] (7) at (4.5, 0.2) {\scriptsize 7};
\node[vertex, fill=white, fill opacity=0.8, text opacity=1] (9) at (0.67, 0.2) {\scriptsize 9};
\node[vertex, fill=white, fill opacity=0.8, text opacity=1] (10) at (2.6, 3.5) {\scriptsize 10};
\node[vertex, fill=white, fill opacity=0.8, text opacity=1] (11) at (2.2, 2.6) {\scriptsize 11};
\node[vertex, fill=white, fill opacity=0.8, text opacity=1] (12) at (1.7, 1.3) {\scriptsize 12};
\node[vertex, fill=white, fill opacity=0.8, text opacity=1] (13) at (4.04, 2.83) {\scriptsize 13};
\node[vertex, fill=white, fill opacity=0.8, text opacity=1] (3) at (4.04, 1.75) {\scriptsize 3};
\node[vertex, fill=white, fill opacity=0.8, text opacity=1] (5) at (1.34, 0.67) {\scriptsize 5};
\node[vertex, fill=white, fill opacity=0.8, text opacity=1] (8) at (5.00, 1.21) {\scriptsize 8};
\begin{pgfonlayer}{bg}
\draw[-latex] (0) edge[bend left=10] (1);
\draw[-latex] (0) edge[bend left=10] (2);
\draw[-latex] (0) edge[bend left=10] (4);
\draw[-latex] (0) edge[bend left=10] (6);
\draw[-latex] (0) edge[bend left=10] (7);
\draw[-latex] (0) edge[bend left=10] (9);
\draw[-latex] (0) edge[bend left=10] (10);
\draw[-latex] (0) edge[bend left=10] (11);
\draw[-latex] (0) edge[bend left=10] (12);
\draw[-latex] (1) edge[bend left=10] (2);
\draw[-latex] (1) edge[bend left=10] (4);
\draw[-latex] (1) edge[bend left=10] (13);
\draw[-latex] (2) edge[bend left=10] (0);
\draw[-latex] (2) edge[bend left=10] (1);
\draw[-latex] (2) edge[bend left=10] (3);
\draw[-latex] (2) edge[bend left=10] (4);
\draw[-latex] (2) edge[bend left=10] (5);
\draw[-latex] (2) edge[bend left=10] (6);
\draw[-latex] (2) edge[bend left=10] (7);
\draw[-latex] (2) edge[bend left=10] (8);
\draw[-latex] (2) edge[bend left=10] (9);
\draw[-latex] (4) edge[bend left=10] (0);
\draw[-latex] (4) edge[bend left=10] (1);
\draw[-latex] (4) edge[bend left=10] (2);
\draw[-latex] (4) edge[bend left=10] (5);
\draw[-latex] (4) edge[bend left=10] (6);
\draw[-latex] (4) edge[bend left=10] (8);
\draw[-latex] (4) edge[bend left=10] (9);
\draw[-latex] (4) edge[bend left=10] (10);
\draw[-latex] (4) edge[bend left=10] (11);
\draw[-latex] (6) edge[bend left=10] (2);
\draw[-latex] (6) edge[bend left=10] (4);
\draw[-latex] (6) edge[bend left=10] (7);
\draw[-latex] (7) edge[bend left=10] (2);
\draw[-latex] (7) edge[bend left=10] (12);
\draw[-latex] (9) edge[bend left=10] (0);
\draw[-latex] (9) edge[bend left=10] (2);
\draw[-latex] (9) edge[bend left=10] (4);
\draw[-latex] (9) edge[bend left=10] (8);
\draw[-latex] (10) edge[bend left=10] (0);
\draw[-latex] (10) edge[bend left=10] (4);
\draw[-latex] (10) edge[bend left=10] (11);
\draw[-latex] (11) edge[bend left=10] (10);
\draw[-latex] (3) edge[bend left=10] (2);
\draw[-latex] (5) edge[bend left=10] (2);
\draw[-latex] (5) edge[bend left=10] (4);
\draw[-latex] (5) edge[bend left=10] (7);
\draw[-latex] (8) edge[bend right=10] (0);
\draw[-latex] (8) edge[bend left=10] (2);
\draw[-latex] (8) edge[bend left=10] (4);
\draw[-latex] (8) edge[bend left=10] (9);
\end{pgfonlayer}
\end{tikzpicture}

%\paragraph{Algorithm}
we adapt and modify the Bellman-Ford algorithm which use a dynamic programming approach. The main goal is to handle the minima cost within all considered criteria.

\begin{algorithm}
\caption{Multi-shortest-path}
\begin{algorithmic}[1]
    \Function{Multi-shortest-path}{$G = \{E,V\}, source$}
        \State {\# Initialization}
        \For{each $V$ in $G$}
            \State$D_v \gets \phi$
        \EndFor
        \State$D_{source} \gets \{0\}$
        %\State$t_s \gets$add a terminal point
        
        \State {\# Bellman-Ford algorithm.}
        \For{$i \gets 1$ to $v-1$}
            \For{each $E$}
                (x, y belong to E)
                \State update($D_y, (D_x, T_{xy})$)
            \EndFor
            \State$D_y \gets$dominate($D_y$)
        \EndFor
        \State \Return$D$
    \EndFunction
\end{algorithmic}
\end{algorithm}

\begin{figure}[ht!]
\centering
\includegraphics[width=60mm]{structure.png}
\caption{Schematic diagram for using Use $u$ to relax the distance of $\overline{sv}$ \label{structure}}
\end{figure}

Considering multiple cost criteria, algorithm2 computes the shortest paths from a given source vertex to all other vertices in a graph. The distance list $D_v$ represents the set of target points for reaching vertex $v$ from the source vertex. Initially, all item in $D_v$ are empty except for the source vertex. We set the distance to itself $D_s$ as zero for all cost dimensions. 

Secondly, it employs a modified version of the Bellman-Ford algorithm. During each iteration, it iterates over the vertices excluding the source and conducts a tailored merging process, for each edge $(x, y)$ in the graph, performs an update operation. The update operation extends the $D_y$for the destination vertex $y$ by combining the distance from the source vertex $x$ with the cost of the current edge $(x, y)$ and pruning step. Let $f^\prime (u, \ell - 1) = update(f(u, \ell - 1), d(u, v))$, which means update $D_u$ by connect with edge $\overline{uv}$
\todo{use "update" in pseudo code might be confused with the dp equation}
\begin{equation}
\begin{aligned}
f(v, \ell) = prune_{u \in \text{in}(v)}(\text{concatenate}\left(f^\prime (u, \ell - 1)) ,f(v, \ell - 1)\right)
\end{aligned}
\end{equation}

\textbf {concatenate} 

The concatenate operation is responsible for updating the distance $D$ estimate f(v, $\ell$) for a target vertex v at iteration $\ell$ by combining the distance estimates f(u, $\ell$ - 1) from the source to all vertices u that have an incoming edge to v (denoted by u $\in$ in(v)), with the respective costs d(u, v) of the edges from u to v. The f(u, $\ell$ - 1) represents the distance estimate from the source to the vertex u at the previous iteration $\ell$ - 1. Based on the Bellman-Ford algorithm, the concatenate operation uses the distance estimate f(u, $\ell$ - 1) from the source to u to update or "relax" the cost or distance estimate f(v, $\ell$) for the target vertex v. It considers all possible paths from adjacent vertices to v through u, accumulating the distances along the way.\\

\textbf {prune} 
 
After obtaining the distances through the concatenate operation. The prune operation is employed to refine the distances estimate f(v, $\ell$) for a vertex v at iteration $\ell$ by comparing the trade-offs between the current estimate f(v, $\ell$ - 1) and the newly merged distances from the adjacent vertices u $\in$ in(v). It then iterates through the sorted list, ensures that the distance list remains compact and efficient by retaining only the dominant points that represent the most significant trade-offs in the multi-dimensional space.  


\todo {whether we need to add table size control}
Additionally, if the size of the pruned distance list exceeds a specified limit, the prune function further reduces the list by sampling a fixed number of points at equal intervals, preserving the overall shape of the cost curve while limiting its size. This sampling process helps manage the size of the distance lists and ensures efficiency in subsequent calculations.
% Once all edges have been processed, the algorithm returns the list of distance $D$. These distance capture the trade-offs between different cost criteria and can be used for efficient decision-making in various scenarios, such as transportation or communication networks, where multiple cost factors influence path selection.

\begin{algorithm}
\caption{Backtracking Algorithm}
\begin{algorithmic}[1]
    \Function{backtracking}{$S, T, D$}
        \State paths$\gets$[]
        \For{$u$ belongs to$D_T$}
            \State path$\gets$[]
            \State$v \gets T$
            \While{$v \neq S$}
                \State path.add($u$)
                \State cost$\gets$get edge cost($v, u$)
                \For{each$i$belongs to$D_u$}
                    \If{cost$v ==$cost$u +$cost}
                        \State find parent if cost$v ==$cost$u +$cost
                        \State$v \gets u$
                        \State$u \gets i$
                    \EndIf
                \EndFor
            \EndWhile
            \State paths.add(path)
        \EndFor
        \State \Return paths
    \EndFunction
\end{algorithmic}
\end{algorithm}

The Backtracking Algorithm is used to find paths in a graph from a target vertex back to a source vertex, given certain constraints. The algorithm operates by exploring possible paths starting from each vertex in a set of destination vertices $D_T$, backtracking to the source vertex $S$. For each destination vertex $u$ in the set $D_T$, the algorithm initiates a search for a path by traversing edges backward from $u$ to $S$. It does so by iterative examining edges and their associated costs, identifying the parent vertex that minimizes the total cost to reach $u$ from $S$. This process continues until the source vertex $S$ is reached. The algorithm then records the discovered path and proceeds to explore paths from other destination vertices. Once all possible paths have been explored, the algorithm returns the collected paths. This approach is useful for various applications, including finding optimal routes in transportation networks or identifying dependencies in computational tasks.
%
\textbf{time complexity}

When we run the dynamic programming equation (1) in multi-shortest path algorithm. The time complexity analysis involves two main steps: concatenate and prune, which are repeated for at most $V-1$ iterations based on the Bellman-Ford algorithm for all edges. In the concatenate step, for each vertex v, we combine the distance tables $D_{u \in in(v)}$ from all incoming neighbors u $\in$ in(v) with the respective edge costs $d(u, v)$. This step has a time complexity of $O(n)$, where n is the size of the distance table $D$.

The subsequent pruning step aims to retain only the optimal solutions in the concatenated distance table. The time complexity of this pruning operation depends on the number of criteria considered. When dealing with two criteria, the pruning step leverages the theory of maximal point sets, resulting in a time complexity of O(n). However, when the number of criteria exceeds two, the time complexity increases to $O(n\log^{n-3}\log\log n)$.

Considering both the concatenate and prune steps, the overall time complexity of the multi-shortest path algorithm is $O(n|V||E|)$. Here, $|V|$ and$ |E|$ represent the number of vertices and edges in the graph respectively. The factor n accounts for the size of the distance table $D$, which may vary depending on the specific problem instance and the number of criteria considered.

 \textbf{correctness}
% we need to prove two things:\\

% \begin{itemize}
   
%      \item [(1)]
%     All optimal solution are in $D$
%      \item [(2)]
%     All solution in $D$ is optimal (when without constraint for $D$ size)
% \end{itemize}
% once we can prove (1) and (2) are correct, which means (1) equals to (2),then we can prove our algorithm is correct. Here is the prove for item 1, since the dp equation has already consider all edge that link in to $v$, we can promised that we find all possible temporary solution, which containing all optimal solutions.
% As for item 2, Assume $f(v,\ell)$ is Pareto optimal and $\exists$ $d_u \in f(u, \ell-1)$ is not Pareto optimal solution. Since the feature of Monotonicity of cost function. After $d_u \in f(u, \ell-1)$ connect with weight cost$ \overline{uv} $ still is not Pareto optimal solution. After step of prune, those worse solution would be dominate, which means $f(v,\ell)$ will not be the Pareto optimal solution. $\to \gets .$ Q.E.D. that, $\forall d_v \in f(v, \ell)$ is Pareto optimal, then $\forall _{u \in \text{in}(v)} \left(f(u, \ell - 1) + d(u, v)\right) and \forall d_v \in f(v, \ell - 1)$ must be Pareto optimal solution. To be more specific, only when last iteration are Pareto optimal, can next iteration still contain all Pareto optimal solution. the only exception which we can't explain is the source vertex  $f(v, 0)$, which doesn't have any coming edge. However, This first step is established, so it is proved.

To prove the correctness of the multi-shortest path algorithm, we need to establish two key points:


\begin{enumerate}
    \item 1. All optimal solutions are present in the distance table $D$.
    \item 2. All solutions in the distance table $D$ are optimal (when there is no constraint on the size of $D$).
\end{enumerate}

If we can prove that (1) and (2) are correct, implying that they are equivalent, then we can establish the correctness of our algorithm.

Regarding item (1), since the dynamic programming equation considers all edges that link to vertex $v$, we can guarantee that we find all possible temporary solutions, which contain all optimal solutions.

As for item (2), assume that $f(v, \ell)$ is Pareto optimal, and there exists $d_u \in f(u, \ell-1)$ that is not a Pareto optimal solution. Due to the monotonicity feature of the cost function, after $d_u \in f(u, \ell-1)$ is connected with the weight cost $d(u, v)$, it still will not be a Pareto optimal solution. After the pruning step, these worse solutions would be dominated, which means $f(v, \ell)$ will not be the Pareto optimal solution. This leads to the conclusion that for all $d_v \in f(v, \ell)$ to be Pareto optimal, all $f(u, \ell-1)$ for $u \in \text{in}(v)$ and $f(v, \ell-1)$ must be Pareto optimal solutions. In other words, only when the previous iteration contains Pareto optimal solutions can the next iteration still contain all Pareto optimal solutions. The only exception we cannot explain is the source vertex $f(v, 0)$, which does not have any incoming edges. However, this initial step is established, so it is proved.\\

\textbf{Convergence property}\\
If $s \rightsquigarrow u \rightarrow v$ is shortest path in $G$ for some $u,v \in V$ ,and if $u.d = \delta(s,u)$ at any time prior to relaxing $edge(u,v)$, then $v.d = \delta(s,v)$ at all times afterward.\\




\section{SCALABILITY ENHANCEMENT}
\label{sec:scalability}
In Section~\ref{subsec:algo}, we show that finding all Pareto optimal on the actionability graph is crucial to the size of the Pareto tables $\tau$, edge size $|E|$, and number of iterations $\eta$. Since $\eta$ corresponds to the number of hops of the output paths, which is usually as a constant (considering in reality that a long feasible path is redundant and non-interpretable), the bottleneck of the running time is mainly on $\tau$ and $|E|$. Additionally, $\tau$ is bounded by the number of different paths from each pair of nodes, which highly depends on the size of the vertices and the connectivity. To decrease the graph size and simplify the connectivity structure, one idea is to shrink the vertices of the graph such that there are only a small number of ``representative'' nodes, and the shortest path in this shrunk graph still preserves or approximates the distance of the original graph. This is the idea of core-set from the computational geometry perspective (see survey in~\cite{agarwal2005geometric}). The challenge here is that the cost functions are not specific but highly general. Additionally,  we also need to incorporate all the $k$ cost functions to get the Pareto optimal. Fortunately, the computation of the cost functions usually contains some structures rather than arbitrary values for any pair of points. This inspires us to utilize the idea of $\epsilon$-net~\cite{haussler1986epsilon} to shrink the size of the graph and also ensure the quality. 

To explicitly explain our idea, we first define the notation of shrinkable.''

\begin{definition}
\label{def: shrinkable}
    Given $G=(V,E)$ and a cost function $c$, we say a vertex $i$ is $\kappa$-shrinkable to vertex $j$ if and only if $\forall (p,i) \in E$
    $$ 
    (p,j) \in E \text{ and } c(p,j) \leq \kappa c(p,i)
    $$
\end{definition}

Given the approximation factor $\kappa$, one can iteratively shrink all the shrinkable vertices in the graph until there is no shrinkable vertex anymore. We call this induced subgraph a shrunk graph $G_S$ and the one with the smallest cardinality is $G_S^*$. Obviously, any $G_S$  preserves $\kappa$-approximation factor. The shortest path between any pair of the nodes in $G$ has another path in $G_S$ which is at most $\kappa l$ times, where $l$ is the number of the hops of the path. However, finding the $G_S^*$ is highly non-trivial. It depends on the order of vertices in the shrinking procedure. A toy example is in the following. Consider a graph with vertices $\{p,i,j,r\}$ and edges $\{(p,i),(p,j),(p,r)\}$ where $c(p,i)=\kappa c(p,j)= \kappa^2 c(p,r)$, if $j$ shrinks to $i$, then $i,r$ are not shrinkable. On the other hand, if $i$ shrinks to $j$, then $r$ can shrink to $j$ too. Thus, we want to have another subgraph that catches most properties of $G_S$ and can be generated efficiently. This is the place where the $\epsilon$-net joins into our work. In the following, we will first introduce the formal definition of $\epsilon$-net and then show how to utilize it under our context.

\begin{definition}
\label{def: epsilon-net}
Given a range space $(\mathcal{X}$,$\mathcal{R})$, let $\mathcal{A} \subset \mathcal{X}$ be a finite subset, and $0<\epsilon<1$. Then a subset $\mathcal{N} \subset \mathcal{A}$ is called an $\epsilon$-net of $\mathcal{A}$ w.r.t to $\mathcal{R}$ if 

$$
\forall r \in \mathcal{R}, |r \cap \mathcal{A}|>\epsilon |\mathcal{A}| \rightarrow r\cap \mathcal{N} \neq \emptyset
$$
\end{definition}

Now, we define the $\epsilon$-net under our context. 

\begin{definition}
\label{def: our epsilon-net}
We say $G_\epsilon$ is an $\epsilon$-net of $G$ if for any vertex $v$ in some shrunk graph $G_S$ that is shrunk by more than $\epsilon n$ vertices, then either $v \in G_\epsilon$ or $u \in G_\epsilon$, where $v$ is shrinkable to $u$.    
\end{definition}

To see Definition~\ref{def: epsilon-net} and Definition~\ref{def: our epsilon-net} are equivalent, one can see the element $r$ of a range $r \in \mathcal{R}$ is a subset of $V$ which is an instance of the shrinking procedure. That is all the elements in $r$ shrink into a point in some graph $G_S$. Thus, an $\epsilon$-net should include one of the points in $r$, which leads to Definition~\ref{def: our epsilon-net}. The interpretation of $G_\epsilon$ is that it includes the majority of vertices (i.e., which is shrunk from $\epsilon n$ vertices) among all the $G_S$.

Notice that with an arbitrary cost function, one cannot have an $G_\epsilon$ or a $G_S$ with a small size. However, if the cost function has some structure, one can analyze the VC-dimension of the range space and utilize the theorem proved by Haussler-Welzl~\cite{haussler1986epsilon}, which states that any random sample set $S$ of $G$ with size 
\begin{equation}
\label{eq:Haussler-Welzl}
O(\frac{|VC|}{\epsilon} \log \frac{1}{\epsilon} + \frac{1}{\epsilon} \log \frac{1}{\delta})  
\end{equation}
is an $\epsilon$-net, with probaility more than $1-\delta$. In addition, we can use one sample set $S$ to fit all the cost functions. Thus, assume $\delta$ is a constant, the size of $S$ for the multi-cost function is $O(|VC|^*/\epsilon \log 1/\epsilon)$, where $|VC|^*$ is the largest VC-dimension among all the cost functions, which is highly scalable in respect to the graph size $n$ and the number of cost functions $k$.

\paragraph{Demonstration}
We demonstrate how to analyze the $VC$-dimension for a cost function $c$ with certain properties to generate the $\epsilon$-net. Assume function $c$ has a property that for every $(i,j) \in E$,  $\Delta(i,j) \leq c(i,j) \leq \kappa \Delta(i,j)$, where $\Delta(i,j)$ is the distance metric of the data space. Notice that this property is the same as discrete Lipschitz continuity~\cite{jiang2011free} except it also has the lower bound on $c(i,j)$. This property is commonly true when the cost function has a similar structure in the metric of the data space, L-norm class for instance. In real analysis, it can be interpreted that the cost of two points cannot be too high concerning the distance of the data space (e.g., the effort of increasing the income from 10,000 to 12,000 can not be too large) but also not too low if the distance of the data space is far (e.g., there should be non-negligible effort of increasing the income from 8,000 to 15,000). 

Now, observe that this property implies that the two points are $\kappa$-shrinkable if and only if they are within the distance of 1 in the data space. Thus, an instance of the shrinking procedure is the same as putting a ball with a radius $1/2$ in the data space and all the points in this ball can shrink into a point. We can see the range $\mathcal{R}$ is the balls in the data space and The $\epsilon$-net here is asking what is the smallest sample set so that any ball containing at least $\epsilon n$ points also contains one of the sampled points. Fortunately, the $VC$-dimension of shattering the points via balls is at most shattering the points via the hyperplanes, which has the VC-dimension as $d+1$. Via Haussler-Welzl's theorem in Equation~\ref{eq:Haussler-Welzl}, a random sampled set with size $O( (d+1)^*/\epsilon \log 1/\epsilon)$ is an $\epsilon$-net for function $c$.  

%\section{FRAMEWORK EXTENSION}
%\subsection{Diverse Recourse}
%\subsection{Load-balancing in Multiple Recourse Actions}

\section{EXPERIMENT}


%%%%%%%%%%%%%%%%%%%%%%%%%%%%%%%%%%%%%%%%%%%%%%%%%%%%%%%%%%%%%%%%%%%%%%%%%%%%%%%%%%%%%%%%%%%%%%%%%%%%%%

%%%%%%%%%%%%%%%%%%%%%%%%%%%%%%%%%%%%%%%%%%%%%%
\begin{table*}[t]
\setlength{\tabcolsep}{3pt}
\centering
\renewcommand{\arraystretch}{1.1}
\tabcolsep=0.2cm
\begin{adjustbox}{max width=\textwidth}  % Set the maximum width to text width
\begin{tabular}{c| cccc ||  c| cc cc}
\toprule
General & \multicolumn{3}{c}{Preference} & Accuracy & Supervised & \multicolumn{3}{c}{Preference} & Accuracy \\ 
LLMs & PrefHit & PrefRecall & Reward & BLEU & Alignment & PrefHit & PrefRecall & Reward & BLEU \\ 
\midrule
GPT-J & 0.2572 & 0.6268 & 0.2410 & 0.0923 & Llama2-7B & 0.2029 & 0.803 & 0.0933 & 0.0947 \\
Pythia-2.8B & 0.3370 & 0.6449 & 0.1716 & 0.1355 & SFT & 0.2428 & 0.8125 & 0.1738 & 0.1364 \\
Qwen2-7B & 0.2790 & 0.8179 & 0.1593 & 0.2530 & Slic & 0.2464 & 0.6171 & 0.1700 & 0.1400 \\
Qwen2-57B & 0.3086 & 0.6481 & 0.6854 & 0.2568 & RRHF & 0.3297 & 0.8234 & 0.2263 & 0.1504 \\
Qwen2-72B & 0.3212 & 0.5555 & 0.6901 & 0.2286 & DPO-BT & 0.2500 & 0.8125 & 0.1728 & 0.1363 \\ 
StarCoder2-15B & 0.2464 & 0.6292 & 0.2962 & 0.1159 & DPO-PT & 0.2572 & 0.8067 & 0.1700 & 0.1348 \\
ChatGLM4-9B & 0.2246 & 0.6099 & 0.1686 & 0.1529 & PRO & 0.3025 & 0.6605 & 0.1802 & 0.1197 \\ 
Llama3-8B & 0.2826 & 0.6425 & 0.2458 & 0.1723 & \textbf{\shortname}* & \textbf{0.3659} & \textbf{0.8279} & \textbf{0.2301} & \textbf{0.1412} \\ 
\bottomrule
\end{tabular}
\end{adjustbox}
\caption{Main results on the StaCoCoQA. The left shows the performance of general LLMs, while the right presents the performance of the fine-tuned LLaMA2-7B across various strong benchmarks for preference alignment. Our method SeAdpra is highlighted in \textbf{bold}.}
\label{main}
\vspace{-0.2cm}
\end{table*}
%%%%%%%%%%%%%%%%%%%%%%%%%%%%%%%%%%%%%%%%%%%%%%%%%%%%%%%%%%%%%%%%%%%%%%%%%%%%%%%%%%%%%%%%%%%%%%%%%%%%
\begin{table}[h]
\centering
\renewcommand{\arraystretch}{1.02}
% \tabcolsep=0.1cm
\begin{adjustbox}{width=0.48\textwidth} % Adjust table width
\begin{tabularx}{0.495\textwidth}{p{1.2cm} p{0.7cm} p{0.95cm}p{0.95cm}p{0.7cm}p{0.7cm}}
     \toprule
    \multirow{2}{*}{\small \textbf{Dataset}} & \multirow{2}{*}{\small Model} & \multicolumn{2}{c}{\small Preference} & \multicolumn{2}{c}{\small Acc } \\ 
    & & \small \textit{PrefHit} & \small \textit{PrefRec} & \small \textit{Reward} & \small \textit{Rouge} \\ 
    \midrule
    \multirow{2}{*}{\small \textbf{Academia}}   & \small PRO & 33.78 & 59.56 & 69.94 & 9.84 \\ 
                                & \small \textbf{Ours} & 36.44 & 60.89 & 70.17 & 10.69 \\ 
    \midrule
    \multirow{2}{*}{\small \textbf{Chemistry}}  & \small PRO & 36.31 & 63.39 & 69.15 & 11.16 \\ 
                                & \small \textbf{Ours} & 38.69 & 64.68 & 69.31 & 12.27 \\ 
    \midrule
    \multirow{2}{*}{\small \textbf{Cooking}}    & \small PRO & 35.29 & 58.32 & 69.87 & 12.13 \\ 
                                & \small \textbf{Ours} & 38.50 & 60.01 & 69.93 & 13.73 \\ 
    \midrule
    \multirow{2}{*}{\small \textbf{Math}}       & \small PRO & 30.00 & 56.50 & 69.06 & 13.50 \\ 
                                & \small \textbf{Ours} & 32.00 & 58.54 & 69.21 & 14.45 \\ 
    \midrule
    \multirow{2}{*}{\small \textbf{Music}}      & \small PRO & 34.33 & 60.22 & 70.29 & 13.05 \\ 
                                & \small \textbf{Ours} & 37.00 & 60.61 & 70.84 & 13.82 \\ 
    \midrule
    \multirow{2}{*}{\small \textbf{Politics}}   & \small PRO & 41.77 & 66.10 & 69.52 & 9.31 \\ 
                                & \small \textbf{Ours} & 42.19 & 66.03 & 69.74 & 9.38 \\ 
    \midrule
    \multirow{2}{*}{\small \textbf{Code}} & \small PRO & 26.00 & 51.13 & 69.17 & 12.44 \\ 
                                & \small \textbf{Ours} & 27.00 & 51.77 & 69.46 & 13.33 \\ 
    \midrule
    \multirow{2}{*}{\small \textbf{Security}}   & \small PRO & 23.62 & 49.23 & 70.13 & 10.63 \\ 
                                & \small \textbf{Ours} & 25.20 & 49.24 & 70.92 & 10.98 \\ 
    \midrule
    \multirow{2}{*}{\small \textbf{Mean}}       & \small PRO & 32.64 & 58.05 & 69.64 & 11.51 \\ 
                                & \small \textbf{Ours} & \textbf{34.25} & \textbf{58.98} & \textbf{69.88} & \textbf{12.33} \\ 
    \bottomrule
\end{tabularx}
\end{adjustbox}
\caption{Main results (\%) on eight publicly available and popular CoQA datasets, comparing the strong list-wise benchmark PRO and \textbf{ours with bold}.}
\label{public}
\end{table}



%%%%%%%%%%%%%%%%%%%%%%%%%%%%%%%%%%%%%%%%%%%%%%%%%%%%%
\begin{table}[h]
\centering
\renewcommand{\arraystretch}{1.02}
\begin{tabularx}{0.48\textwidth}{p{1.45cm} p{0.56cm} p{0.6cm} p{0.6cm} p{0.50cm} p{0.45cm} X}
\toprule
\multirow{2}{*}{Method} & \multicolumn{3}{c}{Preference \((\uparrow)\)} & \multicolumn{3}{c}{Accuracy \((\uparrow)\)} \\ \cmidrule{2-4} \cmidrule{5-7}
& \small PrefHit & \small PrefRec & \small Reward & \small CoSim & \small BLEU & \small Rouge \\ \midrule
\small{SeAdpra} & \textbf{34.8} & \textbf{82.5} & \textbf{22.3} & \textbf{69.1} & \textbf{17.4} & \textbf{21.8} \\ 
\small{-w/o PerAl} & \underline{30.4} & 83.0 & 18.7 & 68.8 & \underline{12.6} & 21.0 \\
\small{-w/o PerCo} & 32.6 & 82.3 & \underline{24.2} & 69.3 & 16.4 & 21.0 \\
\small{-w/o \(\Delta_{Se}\)} & 31.2 & 82.8 & 18.6 & 68.3 & \underline{12.4} & 20.9 \\
\small{-w/o \(\Delta_{Po}\)} & \underline{29.4} & 82.2 & 22.1 & 69.0 & 16.6 & 21.4 \\
\small{\(PerCo_{Se}\)} & 30.9 & 83.5 & 15.6 & 67.6 & \underline{9.9} & 19.6 \\
\small{\(PerCo_{Po}\)} & \underline{30.3} & 82.7 & 20.5 & 68.9 & 14.4 & 20.1 \\ 
\bottomrule
\end{tabularx}
\caption{Ablation Results (\%). \(PerCo_{Se}\) or \(PerCo_{Po}\) only employs Single-APDF in Perceptual Comparison, replacing \(\Delta_{M}\) with \(\Delta_{Se}\) or \(\Delta_{Po}\). The bold represents the overall effect. The underlining highlights the most significant metric for each component's impact.}
\label{ablation}
% \vspace{-0.2cm}
\end{table}

\subsection{Dataset}

% These CoQA datasets contain questions and answers from the Stack Overflow data dump\footnote{https://archive.org/details/stackexchange}, intended for training preference models. 

Due to the additional challenges that programming QA presents for LLMs and the lack of high-quality, authentic multi-answer code preference datasets, we turned to StackExchange \footnote{https://archive.org/details/stackexchange}, a platform with forums that are accompanied by rich question-answering metadata. Based on this, we constructed a large-scale programming QA dataset in real-time (as of May 2024), called StaCoCoQA. It contains over 60,738 programming directories, as shown in Table~\ref{tab:stacocoqa_tags}, and 9,978,474 entries, with partial data statistics displayed in Figure~\ref{fig:dataset}. The data format of StaCoCoQA is presented in Table~\ref{fig::stacocoqa}.

The initial dataset \(D_I\) contains 24,101,803 entries, and is processed by the following steps:
(1) Select entries with "Questioner-picked answer" pairs to represent the preferences of the questioners, resulting in 12,260,106 entries in the \(D_Q\).
(2) Select data where the question includes at least one code block to focus on specific-domain programming QA, resulting in 9,978,474 entries in the dataset \(D_C\).
(3) All HTML tags were cleaned using BeautifulSoup \footnote{https://beautiful-soup-4.readthedocs.io/en/latest/} to ensure that the model is not affected by overly complex and meaningless content.
(4) Control the quality of the dataset by considering factors such as the time the question was posted, the size of the response pool, the difference between the highest and lowest votes within a pool, the votes for each response, the token-level length of the question and the answers, which yields varying sizes: 3K, 8K, 18K, 29K, and 64K. 
The controlled creation time variable and the data details after each processing step are shown in Table~\ref{tab:statistics}.

To further validate the effectiveness of SeAdpra, we also select eight popular topic CoQA datasets\footnote{https://huggingface.co/datasets/HuggingFaceH4/stack-exchange-preferences}, which have been filtered to meet specific criteria for preference models \cite{askell2021general}. Their detailed data information is provided in Table~\ref{domain}.
% Examples of some control variables are shown in Table~\ref{tab:statistics}.
% \noindent\textbf{Baselines}. 
% Following the DPO \cite{rafailov2024direct}, we evaluated several existing approaches aligned with human preference, including GPT-J \cite{gpt-j} and Pythia-2.8B \cite{biderman2023pythia}.  
% Next, we assessed StarCoder2 \cite{lozhkov2024starcoder}, which has demonstrated strong performance in code generation, alongside several general-purpose LLMs: Qwen2 \cite{qwen2}, ChatGLM4 \cite{wang2023cogvlm, glm2024chatglm} and LLaMA serials \cite{touvron2023llama,llama3modelcard}.
% Finally, we fine-tuned LLaMA2-7B on the StaCoCoQA and compared its performance with other strong baselines for supervised learning in preference alignment, including SFT, RRHF \cite{yuan2024rrhf}, Silc \cite{zhao2023slic}, DPO, and PRO \cite{song2024preference}.
%%%%%%%%%%%%%%%%%%%%%%%%%%%%%%%%%%%%%%%%%%%%%%%%%%%%%%%%%%%%%%%%%%%%%%%%%%%%%%%%%%%%%%%%%%%%%%%%%%%%%%%%%%%%%%%%%%%%%%%%%%%%%%%%%%

% For preference evaluation, traditional win-rate assessments are costly and not scalable. For instance, when an existing model \(M_A\) is evaluated against comparison methods \((M_B, M_C, M_D)\) in terms of win rates, upgrading model \(M_A\) would necessitate a reevaluation of its win rates against other models. Furthermore, if a new comparison method \(M_E\) is introduced, the win rates of model \(M_A\) against \(M_E\) would also need to be reassessed. Whether AI or humans are employed as evaluation mediators, binary preference between preferred and non-preferred choices or to score the inference results of the modified model, the costs of this process are substantial. 
% Therefore, from an economic perspective, we propose a novel list preference evaluation method. We utilize manually ranking results as the gold standard for assessing human preferences, to calculate the Hit and Recall, referred to as PrefHit and PrefRecall, respectively. Regardless of whether improving model \(M_A\) or expanding comparison method \(M_E\), only the calculation of PrefHit and PrefRecall for the modified model is required, eliminating the need for human evaluation. 
% We also employ a professional reward model\footnote{https://huggingface.co/OpenAssistant/reward-model-deberta-v3-large}
% for evaluation, denoted as the Reward metric.

% \subsection{Baseline} 
% Following the DPO \cite{rafailov2024direct}, we evaluated several existing approaches aligned with human preference, including GPT-J \cite{gpt-j} and Pythia-2.8B \cite{biderman2023pythia}.  
% Next, we assessed StarCoder2 \cite{lozhkov2024starcoder}, which has demonstrated strong performance in code generation, alongside several general-purpose LLMs: Qwen2 \cite{qwen2}, ChatGLM4 \cite{wang2023cogvlm, glm2024chatglm} and LLaMA serials \cite{touvron2023llama,llama3modelcard}.
% Finally, we fine-tuned LLaMA2-7B on the StaCoCoQA and compared its performance with other strong baselines for supervised learning in preference alignment, including SFT, RRHF \cite{yuan2024rrhf}, Silc \cite{zhao2023slic}, DPO, and PRO \cite{song2024preference}.
\subsection{Evaluation Metrics}
\label{sec: metric}
For preference evaluation, we design PrefHit and PrefRecall, adhering to the "CSTC" criterion outlined in Appendix \ref{sec::cstc}, which overcome the limitations of existing evaluation methods, as detailed in Appendix \ref{metric::mot}.
In addition, we demonstrate the effectiveness of thees new evaluation from two main aspects: 1) consistency with traditional metrics, and 2) applicability in different application scenarios in Appendix \ref{metric::ana}.
Following the previous \cite{song2024preference}, we also employ a professional reward.
% Following the previous \cite{song2024preference}, we also employ a professional reward model\footnote{https://huggingface.co/OpenAssistant/reward-model-deberta-v3-large} \cite{song2024preference}, denoted as the Reward.

For accuracy evaluation, we alternately employ BLEU \cite{papineni2002bleu}, RougeL \cite{lin2004rouge}, and CoSim. Similar to codebertscore \cite{zhou2023codebertscore}, CoSim not only focuses on the semantics of the code but also considers structural matching.
Additionally, the implementation details of SeAdpra are described in detail in the Appendix \ref{sec::imp}.
\subsection{Main Results}
We compared the performance of \shortname with general LLMs and strong preference alignment benchmarks on the StaCoCoQA dataset, as shown in Table~\ref{main}. Additionally, we compared SeAdpra with the strongly supervised alignment model PRO \cite{song2024preference} on eight publicly available CoQA datasets, as presented in Table~\ref{public} and Figure~\ref{fig::public}.

\textbf{Larger Model Parameters, Higher Preference.}
Firstly, the Qwen2 series has adopted DPO \cite{rafailov2024direct} in post-training, resulting in a significant enhancement in Reward.
In a horizontal comparison, the performance of Qwen2-7B and LLaMA2-7B in terms of PrefHit is comparable.
Gradually increasing the parameter size of Qwen2 \cite{qwen2} and LLaMA leads to higher PrefHit and Reward.
Additionally, general LLMs continue to demonstrate strong capabilities of programming understanding and generation preference datasets, contributing to high BLEU scores.
These findings indicate that increasing parameter size can significantly improve alignment.

\textbf{List-wise Ranking Outperforms Pair-wise Comparison.}
Intuitively, list-wise DPO-PT surpasses pair-wise DPO-{BT} on PrefHit. Other list-wise methods, such as RRHF, PRO, and our \shortname, also undoubtedly surpass the pair-wise Slic.

\textbf{Both Parameter Size and Alignment Strategies are Effective.}
Compared to other models, Pythia-2.8B achieved impressive results with significantly fewer parameters .
Effective alignment strategies can balance the performance differences brought by parameter size. For example, LLaMA2-7B with PRO achieves results close to Qwen2-57B in PrefHit. Moreover, LLaMA2-7B combined with our method SeAdpra has already far exceeded the PrefHit of Qwen2-57B.

\textbf{Rather not Higher Reward, Higher PrefHit.}
It is evident that Reward and PrefHit are not always positively correlated, indicating that models do not always accurately learn human preferences and cannot fully replace real human evaluation. Therefore, relying solely on a single public reward model is not sufficiently comprehensive when assessing preference alignment.

% In conclusion, during ensuring precise alignment, SeAdpra will focuse on PrefHit@1, even though the trade-off between PrefHit and PrefRecall is a common issue and increasing recall may sometimes lead to a decrease in hit rate. The positive correlation between Reward and BLEU, indicates that improving the quality of the generated text typically enhances the Reward. 
% Most importantly, evaluating preferences solely based on reward is clearly insufficient, as a high reward does not necessarily correspond to a high PrefHit or PrefRecall.
%%%%%%%%%%%%%%%%%%%%%%%%%%%%%%%%%%%%%%%%%%%
%%%%%%%%%%%%
\begin{figure}
  \centering
  \begin{subfigure}{0.49\linewidth}
    \includegraphics[width=\linewidth]{latex/pic/hit.png}
    \caption{The PrefHit}
    \label{scale:hit}
  \end{subfigure}
  \begin{subfigure}{0.49\linewidth}
    \includegraphics[width=\linewidth]{latex/pic/Recall.png}
    \caption{The PrefRecall}
    \label{scale:recall}
  \end{subfigure}
  \medskip
  \begin{subfigure}{0.48\linewidth}
    \includegraphics[width=\linewidth]{latex/pic/reward.png}
    \caption{The Reward}
    \label{scale:reward}
  \end{subfigure}
  \begin{subfigure}{0.48\linewidth}
    \includegraphics[width=\linewidth]{latex/pic/bleu.png}
    \caption{The BLEU}
    \label{scale:bleu}
  \end{subfigure}
  \caption{The performance with Confidence Interval (CI) of our SeAdpra and PRO at different data scales.}
  \label{fig:scale}
  % \vspace{-0.2cm}
\end{figure}
%%%%%%%%%%%%%%%%%%%%%%%%%%%%%%%%%%%%%%%%%%%%%%%%%%%%%%%%%%%%%%%%%%%%%%%%%%%%%%%%%%%%%%%%%%%%%%%%%%%%%%%%%%%%%%%%

\subsection{Ablation Study}

In this section, we discuss the effectiveness of each component of SeAdpra and its impact on various metrics. The results are presented in Table \ref{ablation}.

\textbf{Perceptual Comparison} aims to prevent the model from relying solely on linguistic probability ordering while neglecting the significance of APDF. Removing this Reward will significantly increase the margin, but PrefHit will decrease, which may hinder the model's ability to compare and learn the preference differences between responses.

\textbf{Perceptual Alignment} seeks to align with the optimal responses; removing it will lead to a significant decrease in PrefHit, while the Reward and accuracy metrics like CoSim will significantly increase, as it tends to favor preference over accuracy.

\textbf{Semantic Perceptual Distance} plays a crucial role in maintaining semantic accuracy in alignment learning. Removing it leads to a significant decrease in BLEU and Rouge. Since sacrificing accuracy recalls more possibilities, PrefHit decreases while PrefRecall increases. Moreover, eliminating both Semantic Perceptual Distance and Perceptual Alignment in \(PerCo_{Po}\) further increases PrefRecall, while the other metrics decline again, consistent with previous observations.


\textbf{Popularity Perceptual Distance} is most closely associated with PrefHit. Eliminating it causes PrefHit to drop to its lowest value, indicating that the popularity attribute is an extremely important factor in code communities.

% In summary, each module has a varying impact on preference and accuracy, but all outperform their respective foundation models and other baselines, as shown in Table \ref{main}, proving their effectiveness.


\subsection{Analysis and Discussion}

\textbf{SeAdpra adept at high-quality data rather than large-scale data.}
In StaCoCoQA, we tested PRO and SeAdpra across different data scales, and the results are shown in Figure~\ref{fig:scale}.
Since we rely on the popularity and clarity of questions and answers to filter data, a larger data scale often results in more pronounced deterioration in data quality. In Figure~\ref{scale:hit}, SeAdpra is highly sensitive to data quality in PrefHit, whereas PRO demonstrates improved performance with larger-scale data. Their performance on Prefrecall is consistent. In the native reward model of PRO, as depicted in Figure~\ref{scale:reward}, the reward fluctuations are minimal, while SeAdpra shows remarkable improvement.

\textbf{SeAdpra is relatively insensitive to ranking length.} 
We assessed SeAdpra's performance on different ranking lengths, as shown in Figure 6a. Unlike PRO, which varied with increasing ranking length, SeAdpra shows no significant differences across different lengths. There is a slight increase in performance on PrefHit and PrefRecall. Additionally, SeAdpra performs better at odd lengths compared to even lengths, which is an interesting phenomenon warranting further investigation.


\textbf{Balance Preference and Accuracy.} 
We analyzed the effect of control weights for Perceptual Comparisons in the optimization objective on preference and accuracy, with the findings presented in Figure~\ref{para:weight}.
When \( \alpha \) is greater than 0.05, the trends in PrefHit and BLEU are consistent, indicating that preference and accuracy can be optimized in tandem. However, when \( \alpha \) is 0.01, PrefHit is highest, but BLEU drops sharply.
Additionally, as \( \alpha \) changes, the variations in PrefHit and Reward, which are related to preference, are consistent with each other, reflecting their unified relationship in the optimization. Similarly, the variations in Recall and BLEU, which are related to accuracy, are also consistent, indicating a strong correlation between generation quality and comprehensiveness. 

%%%%%%%%%%%%%%%%%%%%%%%%%%%%%%%%%%%%%%%%%%%%%%%%%%%%%%%%%%%%%%%%%%%%%%%%%%%%%%%%%
\begin{figure}
  \centering
  \begin{subfigure}{0.475\linewidth}
    \includegraphics[width=\linewidth]{latex/pic/Rank1.png}
    \caption{Ranking length}
    \label{para:rank}
  \end{subfigure}
  \begin{subfigure}{0.475\linewidth}
    \includegraphics[width=\linewidth]{latex/pic/weights1.png}
    \caption{The \(\alpha\) in \(Loss\)}
    \label{para:weight}
  \end{subfigure}
  \caption{Parameters Analysis. Results of experiments on different ranking lengths and the weight \(\alpha\) in \(Loss\).}
  \label{fig:para}
  % \vspace{-0.2cm}
\end{figure}
%%%%%%%%%%%%%%%%%%%%%%%%%%%%%%%%%%%%%%%%%%%%
\begin{figure*}
  \centering
  \begin{subfigure}{0.305\linewidth}
    \includegraphics[width=\linewidth]{latex/pic/se2.pdf}
    \caption{The \(\Delta_{Se}\)}
    \label{visual:se}
  \end{subfigure}
  \begin{subfigure}{0.305\linewidth}
    \includegraphics[width=\linewidth]{latex/pic/po2.pdf}
    \caption{The \(\Delta_{Po}\)}
    \label{visual:po}
  \end{subfigure}
  \begin{subfigure}{0.305\linewidth}
    \includegraphics[width=\linewidth]{latex/pic/sv2.pdf}
    \caption{The \(\Delta_{M}\)}
    \label{visual:sv}
  \end{subfigure}
  \caption{The Visualization of Attribute-Perceptual Distance Factors (APDF) matrix of five responses. The blue represents the response with the highest APDF, and SeAdpra aligns with the fifth response corresponding to the maximum Multi-APDF in (c). The green represents the second response that is next best to the red one.}
  \label{visual}
  % \vspace{-0.2cm}
\end{figure*}
%%%%%%%%%%%%%%%%%%%%%%%%%%%%%%%%%%%%%%%%%
\textbf{Single-APDF Matrix Cannot Predict the Optimal Response.} We randomly selected a pair with a golden label and visualized its specific iteration in Figure~\ref{visual}.
It can be observed that the optimal response in a Single-APDF matrix is not necessarily the same as that in the Multi-APDF matrix.
Specifically, the optimal response in the Semantic Perceptual Factor matrix \(\Delta_{Se}\) is the fifth response in Figure~\ref{visual:se}, while in the Popularity Perceptual Factor matrix \(\Delta_{Po}\) (Figure~\ref{visual:po}), it is the third response. Ultimately, in the Multiple Perceptual Distance Factor matrix \(\Delta_{M}\), the third response is slightly inferior to the fifth response (0.037 vs. 0.038) in Figure~\ref{visual:sv}, and this result aligns with the golden label.
More key findings regarding the ADPF are described in Figure \ref{fig::hot1} and Figure \ref{fig::hot2}.
%\begin{figure*}[t]
    \subcapraggedrighttrue
    \subcaphangtrue
        \centering
        \subfigure{\includegraphics[width=0.3\textwidth]{figures/drawing_region/vision/MNIST/LinearNN.pdf}} \hfill
        \subfigure{\includegraphics[width=0.3\textwidth]{figures/drawing_region/vision/MNIST/MLPNN.pdf}} \hfill
        \subfigure{\includegraphics[width=0.3\textwidth]{figures/drawing_region/vision/MNIST/CNN.pdf}}
        \caption{Experimental Results of \textit{Survival Game} in handwritten digit recognition task (MNIST). The red line is a power law curve drawn based on the distribution of the model's data points. Results suggest that a system reaches a higher-level of intelligence if it better models the task.}
        \label{fig:mnist_result}
    \end{figure*}

%\subsection{Adult}


    In this study, we introduce a novel method based on multi-cost shortest path algorithms and validate its effectiveness on the Adult dataset. The Adult dataset is Here are the specific steps and methodologies we employed in our experiment.


We started by standardizing the "Adult" dataset to ensure uniformity across various features, allowing for more accurate comparisons. Using a random forest model with 100 trees, then we sorted this standardized dataset and also maintained a non-standardized version for diverse analytical needs. Based on a set of predefined actionable criteria, we established an actionable dataset. During this process, some criteria were set as immutable to reflect real-world conditions. For instance, in the Adult dataset, gender was designated as immutable. From this actionable dataset, we randomly selected 256 instances that met predefined actionable criteria for further analysis. These instances were assessed using a second random forest model to identify the data points with the lowest success probability, indicating areas with the highest challenges.

For data analysis, we created a KNN-graph (k=4) from the standardized dataset and applied our multi-cost shortest path algorithm to explore optimal changes in three key dimensions: education, hours per week, and age. We used the non-standardized data to calculate the shortest paths based on these criteria, aiming to find the most optimized change strategies within these three dimensions. To ensure a realistic application, we designed different cost functions for each criterion. Taking "age" for example, when designing the cost function, it is notable that age can not decrease during the whole recourse process. %As mentioned in Algorithm~\ref{algo:backtracking}, we use Pareto optimal 2 to prune the paths, retaining only a representative set of costs.
    
Figure~\ref{fig:enter-label} visualized the results of the experiment that getting multiple flipped counterfactuals. These four paths, originating from the same starting point where the success probability is the worst, illustrate the diverse impacts caused by Pareto optimal solutions, and all four paths ultimately reach an endpoint where the success probability exceeds 0.75. The introduction of Kernel Density Estimation (KDE) into the algorithm demonstrates a better trade-off with other costs, enhancing the algorithm's ability to navigate through the solution space effectively. To better visualize and understand the patterns in the data, we applied Principal Component Analysis (PCA) for dimension reduction.

%In conclusion, the application of our algorithm on the Adult dataset has demonstrated significant capabilities in addressing complex decision-making scenarios where multiple factors such as age and hours worked per week impact income predictions. Our approach effectively identifies optimal paths for improving individual income predictions by leveraging a multi-criteria framework. This enables a tailored intervention strategy that considers multiple aspects of an individual’s profile.
%\begin{figure}
    \centering
    \includegraphics[width=0.5\textwidth]{experiments/scale.pdf}
    \caption{show scalability}
    \label{fig:scalability}
\end{figure}


%\section{Additional Figures}

In this section, we present some additional figures that demonstrate the negative effects of random edge-dropping, particularly focusing on providing empirical evidence for scenarios not covered by the theory in \autoref{sec:theory}. Additionally, we provide visual evidence of the negative effects of DropNode, despite the fact that it preserves sensitivity between nodes (in expectation).

\subsection{Symmetrically Normalized Propagation Matrix}
\label{sec:fig-sym-norm}

\begin{figure}
    \centering
    \includegraphics[width=\linewidth]{assets/linear-gcn_symmetric_Proteins.png}
    % \includegraphics[width=0.48\linewidth]{assets/linear-gcn_symmetric_MUTAG.png}
    \caption{Entries of $\ddot{\transition}^6$, averaged after binning node-pairs by their shortest distance.}
    \label{fig:linear-gcn_symmetric}
\end{figure}

The results in \autoref{sec:linear} correspond to the use of $\hat{\adjacency} = \propagation^\asym$ for aggregating messages -- in each message passing step, only the in-degree of node $i$ is used to compute the aggregation weights of the incoming messages. In practice, however, it is more common to use the symmetrically normalized propagation matrix, $\propagation = \propagation^\sym$, which ensures that nodes with high degree do not dominate the information flow in the graph \cite{kipf2017gcn}. As in \autoref{thm:sensitivity-l-layer}, we are looking for 
$$
    \expectation_{\mask\layer{1}, \ldots, \mask\layer{L}} \sb{\prod_{l=1}^L \propagation\layer{\ell}} = \ddot{\transition}^L
$$
where $\ddot{\transition} \coloneqq \expectation_{\mathsf{DE}} [\propagation^{\sym}]$. While $\ddot{\transition}$ is analytically intractable, we can approximate it using Monte-Carlo sampling. Accordingly, we use 20 samples of $\mask$ to compute an approximation of $\ddot{\transition}$, and plot out the entries of $\ddot{\transition}^L$, as we did for $\dot{\transition}^L$ in \autoref{fig:linear-gcn_asymmetric}. The results are presented in \autoref{fig:linear-gcn_symmetric}, which shows that while the sensitivity between nearby nodes is affected to a lesser extent compared to those observed in \autoref{fig:linear-gcn_asymmetric}, that between far-off nodes is significantly reduced, same as earlier.

\subsection{Upper Bound on Expected Sensitivity}

\begin{figure}
    \centering
    \includegraphics[width=\linewidth]{assets/linear-gcn_black-extension_Proteins.png}
    % \includegraphics[width=0.48\linewidth]{assets/linear-gcn_black-extension_MUTAG.png}
    \caption{Entries of $\sum_{l=0}^6 \dot{\transition}^l$, averaged after binning node-pairs by their shortest distance.}
    \label{fig:black_extension}
\end{figure}

\citet{black2023resistance} showed that the sensitivity between any two nodes in a graph can be bounded using the sum of the powers of the propagation matrix. In \autoref{sec:nonlinear}, we extended this bound to random edge-dropping methods with independent edge masks smapled in each layer:

\begin{align*}
    \expectation_{\mask\layer{1}, \ldots, \mask\layer{L}} \sb{\norm{\frac{\partial \representation\layer{L}_i}{\partial \feature_j}}_1}
    =
    \zeta_3^{\rb{L}} \rb{\sum_{\ell=0}^L \expectation \sb{\propagation}^{\ell}}_{ij}
\end{align*}

Although this bound does not have a closed form, we can use real-world graphs to study its entries. We randomly sample 100 molecular graphs from the Proteins dataset \cite{dobson2003proteins} and plot the entries of $\sum_{l=0}^6 \dot{\transition}^\ell$ (corresponding to \inline{DropEdge}) against the shortest distance between node-pairs. The results are presented in \autoref{fig:black_extension}. We observe a polynomial decline in sensitivity as the \inline{DropEdge} probability increases, suggesting that it is unsuitable for capturing LRIs.

\subsection{Test Accuracy versus DropNode Probability}

\begin{figure}[t]
    \centering
    \subcaptionbox{Homophilic datasets}{\includegraphics[width=0.7\linewidth]{assets/DropNode_homophilic.png}} \\
    \vspace{2mm}
    \subcaptionbox{Heterophilic datasets}{\includegraphics[width=\linewidth]{assets/DropNode_heterophilic.png}}
    \caption{Dropping probability versus test accuracy of DropNode-GCN.}
    \label{fig:dropnode}
\end{figure}

In \autoref{eqn:dropnode}, we noted that the expectation of sensitivity remains unchanged when using DropNode. However, these results were only in expectation. In practice, a high DropNode probability will result in poor communication between distant nodes, preventing the model from learning to effectively model LRIs. This is supported by the results in \autoref{tab:correlation}, where we observed a negative correlation between the test accuracy and DropNode probability. Moreover, DropNode was the only algorithm which recorded negative correlations on homophilic datasets. In \autoref{fig:dropnode}, we visualize these relationships, noting the stark contrast with \autoref{fig:acc-trends}, particularly in the trends with homophilic datasets.

\section{CONCLUSION}
In conclusion, our algorithm proposes a novel idea that can be generalized to non-differentiable or discrete cost functions in the field of recourse. The algorithm is applied to multi-cost scenarios where users can combine their background knowledge in professional fields, set the most suitable cost function, and then finally get a more practical counterfactual. We show that our algorithm can find all the Pareto recourse plans optimally and can be scaled to a large graph with the utilization of $\epsilon$-net. We conduct experiments on two data sets separately. For the MNIST data set, we compared the performance of Pareto optimal paths under different sampling sizes and showed the result that transfers from one handwriting number into another and offers the whole process. For Adults, we show the diversity in each cost function, and those diverse cost functions finally lead to multi-Pareto optimal solutions.


\emph{Acknowledgement:} This work is supported by NSTC:113-2221-E-008-086.
%\section{Appendices}

%If your work needs an appendix, add it before the
%``\verb|\end{document}|'' command at the conclusion of your source
%document.

%Start the appendix with the ``\verb|appendix|'' command:
%\begin{verbatim}
%  \appendix
%\end{verbatim}
%and note that in the appendix, sections are lettered, not
%numbered. This document has two appendices, demonstrating the section
%and subsection identification method.



%%
%% The next two lines define the bibliography style to be used, and
%% the bibliography file.
\bibliographystyle{splncs04}
\bibliography{ref.bib}



\end{document}
\endinput
%%
%% End of file `sample-manuscript.tex'.
