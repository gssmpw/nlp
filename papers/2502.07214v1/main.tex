%\documentclass{ecai} 
\documentclass[runningheads]{llncs}
%
\usepackage[T1]{fontenc}
% T1 fonts will be used to generate the final print and online PDFs,
% so please use T1 fonts in your manuscript whenever possible.
% Other font encondings may result in incorrect characters.
%
\usepackage{comment}

\usepackage{graphicx}
% Used for displaying a sample figure. If possible, figure files should
% be included in EPS format.
%
% If you use the hyperref package, please uncomment the following two lines
% to display URLs in blue roman font according to Springer's eBook style:
%\usepackage{color}
%\renewcommand\UrlFont{\color{blue}\rmfamily}
%\urlstyle{rm}
%
%%%%%%%%%%%%%%%%%%%%%%%%%%%%%%%%%%%%%%%%%%%%%%%%%%%%%%%%%%%%%%%%%%%%%%%%

%%% Load any packages you require here. 

\usepackage{latexsym}
\usepackage{amssymb}
\usepackage{amsmath}
\let\proof\relax \let\endproof\relax \usepackage{amsthm}
\usepackage{booktabs}
\usepackage{enumitem}
\usepackage{graphicx}
\usepackage{color}
\usepackage{algorithm}
\usepackage{algpseudocode}


\usepackage{caption}
\usepackage{subcaption}

\algnewcommand\algorithmicforeach{\textbf{for each}}
\algdef{S}[FOR]{ForEach}[1]{\algorithmicforeach\ #1\ \algorithmicdo}


%%%%%%%%%%%%%%%%%%%%%%%%%%%%%%%%%%%%%%%%%%%%%%%%%%%%%%%%%%%%%%%%%%%%%%%%

%%% Define any theorem-like environments you require here.

\newtheorem{fact}[theorem]{Fact}

%%%%%%%%%%%%%%%%%%%%%%%%%%%%%%%%%%%%%%%%%%%%%%%%%%%%%%%%%%%%%%%%%%%%%%%%

%%% Define any new commands you require here.

\newcommand{\BibTeX}{B\kern-.05em{\sc i\kern-.025em b}\kern-.08em\TeX}
\def\R{\mathbb{R}}
\newcommand{\rey}[1]{{\color{blue}Rey: #1 }}
\newcommand{\todo}[1]{{\color{red}TO DO: #1 }}

%%
%% \BibTeX command to typeset BibTeX logo in the docs
\AtBeginDocument{%
  \providecommand\BibTeX{{%
    Bib\TeX}}}

%% Rights management information.  This information is sent to you
%% when you complete the rights form.  These commands have SAMPLE
%% values in them; it is your responsibility as an author to replace
%% the commands and values with those provided to you when you
%% complete the rights form.
\begin{document}

\begin{frontmatter}

%%% Use this command to specify your submission number.
%%% In doubleblind mode, it will be printed on the first page.


%%% Use this command to specify the title of your paper.
%\title{Algorithmic Recourse in (non-differentiable) Multi-cost Function}
\title{Pareto Optimal Algorithmic Recourse in Multi-cost Function}

%%% Use this combinations of commands to specify all authors of your 
%%% paper. Use \fnms{} and \snm{} to indicate everyone's first names 
%%% and surname. This will help the publisher with indexing the 
%%% proceedings. Please use a reasonable approximation in case your 
%%% name does not neatly split into "first names" and "surname".
%%% Specifying your ORCID digital identifier is optional. 
%%% Use the \thanks{} command to indicate one or more corresponding 
%%% authors and their email address(es). If so desired, you can specify
%%% author contributions using the \footnote{} command.

\author{Wen-Ling Chen \and Hong-Chang Huang \and
Kai-Hung Lin \and Shang-Wei Hwang \and Hao-Tsung Yang}
%

\authorrunning{W. Chen et al.}
% First names are abbreviated in the running head.
% If there are more than two authors, 'et al.' is used.
%
\institute{National Central University, Taiwan}
%
\maketitle              % typeset the header of the contribution
%
%%% Use this environment to include an abstract of your paper.
\begin{abstract}
%In real-world machine learning applications for decision-making, such as determining loan approvals or credit scores, significant efforts have been made recently to find diverse recommendations that explain the algorithmic decisions. These diverse recommendations provide actionable feedback to users, catering to a wide range of user preferences. 

In decision-making systems, algorithmic recourse aims to identify minimal-cost actions to alter an individual's features, thereby obtaining a desired outcome. This empowers individuals to understand, question, or alter decisions that negatively affect them. However, due to the variety and sensitivity of system environments and individual personalities, quantifying the cost of a single function is nearly impossible while considering multiple criteria situations. Most current recourse mechanisms use gradient-based methods that assume cost functions are differentiable, often not applicable in real-world scenarios, resulting in sub-optimal solutions that compromise various criteria. These solutions are typically intractable and lack rigorous theoretical foundations, raising concerns regarding interpretability, reliability, and transparency from the explainable AI (XAI) perspective. To address these issues, this work proposes an algorithmic recourse framework that handles non-differentiable and discrete multi-cost functions. By formulating recourse as a multi-objective optimization problem and assigning weights to different criteria based on their importance, our method identifies Pareto optimal recourse recommendations. To demonstrate scalability, we incorporate the concept of $\epsilon$-net, proving the ability to find approximated Pareto optimal actions. Experiments show the trade-off between different criteria and the method's scalability in large graphs. Compared to current heuristic practices, our approach provides a stronger theoretical foundation and better aligns recourse suggestions with real-world requirements.

\keywords{multi-objective optimization \and recourse \and shortest path \and Pareto optimality.}

%The multi-criteria decision-making paradigm holds promise for propelling recourse analysis toward more effective, generalizable, and transparent solutions, even in high-dimensional decision spaces characterizing numerous real-world applications.
\end{abstract}

\end{frontmatter}


%%
%% The "title" command has an optional parameter,
%% allowing the author to define a "short title" to be used in page headers.


%%
%% The "author" command and its associated commands are used to define
%% the authors and their affiliations.
%% Of note is the shared affiliation of the first two authors and the
%% "authornote" and "authornotemark" commands

%% This command processes the author affiliation and title
%% information and builds the first part of the formatted document.

\section{Introduction}
\section{Introduction}
\label{sec:intro}

\begin{figure*}[tb]
    \centering
    \includegraphics[width=0.848\linewidth]{figs/circuitnn.pdf} 
    \caption{Illustration of differentiable CircuitNN. CircuitNN is designed based on differentiable NAND gates. After DAS is guided by PI and PO pairs of the truth table, CircuitNN can get the precise circuit architecture logic equivalent to the truth table.}
    \label{fig:circuitnn}
\end{figure*}

% 1. Describe the importance of logic synthesis
% 2. Existing Problems
% (a) Neural Architecture Search: Unstable, Predefined Setting, etc.
% (b) Circuit Generation: Probabilistic Model, Logic Equivalence

With the rapid advancement of technology, the scale of integrated circuits (ICs) has expanded exponentially. 
This expansion has introduced significant challenges in chip manufacturing, particularly concerning power and area metrics.
A primary objective in IC design is achieving the same circuit function with fewer transistors, thereby reducing power usage and area occupancy.

Logic synthesis~\cite{hachtel2005logicsynth}, a critical step in electronic design automation (EDA), transforms behavioral-level circuit designs into optimized gate-level circuits, ultimately yielding the final IC layout. 
The primary goal of logic synthesis is to identify the physical implementation with the fewest gates for a given circuit function. 
This task constitutes a challenging NP-hard combinatorial optimization problem. 
Current logic synthesis tools~\cite{brayton2010abc, wolf2013yosys} rely on human-designed heuristics, often leading to sub-optimal outcomes.

Differentiable architecture search (DAS) techniques~\cite{liu2018darts, chu2020darts} offer novel perspectives on addressing challenges in this problem.
Circuit functions can be represented through truth tables, which map binary inputs to their corresponding outputs. 
Truth tables provide a precise representation of input-output relationships, ensuring the design of functionally equivalent circuits.
Inspired by this, researchers~\cite{deepmind2024ai4sys, wang2024tnet} have begun exploring the application of DAS to synthesize circuits directly from truth tables.
Specifically, \citet{deepmind2024ai4sys} proposed CircuitNN, a framework that learns differentiable connection structures with logic gates, enabling the automatic generation of logic circuits from truth tables.
This approach significantly reduces the complexity of traditional circuit generation. 
Building on this, \citet{wang2024tnet} introduced T-Net, a triangle-shaped variant of CircuitNN, incorporating regularization techniques to enhance the efficiency of DAS.

Despite these advancements, several challenges remain. 
The computational complexity of DAS grows quadratically with the number of gates, posing scalability issues.
Although triangle-shaped architecture~\cite{wang2024tnet} partially mitigates this problem, redundancy persists. 
%Additionally, DAS is susceptible to converging to local optima, limiting the ability to search architectures that satisfy the given truth tables~\cite{liu2018darts}. 
%Furthermore, hyperparameters (network depth and layer width) require extensive searches, introducing complexity and prolonging the synthesis process. 
Additionally, DAS is susceptible to converging to local optima~\cite{liu2018darts} and hyperparameters (network depth and layer width) require extensive searches. 
The challenges arise from the vast search space in DAS. 
% Even with predefined settings for CircuitNN, finding a configuration that meets the truth table requires extensive trial and error during the DAS process. 
Intuitively, limiting the search space through predefined parameters (network depth, gates per layer, and connection probabilities) can significantly reduce the complexity.

Recent advances~\cite{openai2023gpt4, abramson2024alphafold3, esser2024sd3, li2024mar} in conditional generative models have demonstrated remarkable performance across language, vision, and graph generation tasks. 
Motivated by these developments, we propose a novel approach to circuit generation that generates preliminary circuit structures to guide DAS in generating refined circuits matching specified truth tables. 
Firstly, we introduce CircuitVQ, a tokenizer with a discrete codebook for circuit tokenization. 
Built upon our Circuit AutoEncoder framework~\cite{hou2022graphmae,li2023maskgae,wu2025mgvga}, CircuitVQ is trained through a circuit reconstruction task. 
Specifically, the CircuitVQ encoder encodes input circuits into discrete tokens using a learnable codebook, while the decoder reconstructs the circuit adjacency matrix based on these tokens.
Subsequently, the CircuitVQ encoder serves as a circuit tokenizer for CircuitAR pretraining, which employs a masked autoregressive modeling paradigm~\cite{chang2022maskgit, li2023mage}. 
In this process, the discrete codes function as supervision signals. 
After training, CircuitAR can generate discrete tokens progressively, which can be decoded into initial circuit structures by the decoder of the CircuitVQ. 
These prior insights can guide DAS in producing refined circuits that match the target truth tables precisely.

Our key contributions can be summarized as follows:
\begin{itemize}
\item We introduce CircuitVQ, a circuit tokenizer that facilitates graph autoregressive modeling for circuit generation, based on our Circuit AutoEncoder framework;
\item Develop CircuitAR, a model trained using masked autoregressive modeling, which generates initial circuit structures conditioned on given truth tables;
\item Propose a refinement framework that integrates differentiable architecture search to produce functionally equivalent circuits guided by target truth tables;
\item Comprehensive experiments demonstrating the scalability and capability emergence of our CircuitAR and the superior performance of the proposed circuit generation approach.
\end{itemize}

% Motivation
% (a) Diffusion (Vision, Graph), Autoregressive (Language, Vision)
% (b) Circuit Generation for Predefined Setting
% (c) Neural Architecture Search for Strict Logic Equivalence

% Contribution
% (a) Circuit Tokenizer (new transformer arch, training strategy)
% (b) CircuitAR (train and gen strategies, post-ar strategy)
% (c) Extensive Evaluation including BitD (Bit Distance) for Scalability


\begin{comment}
\section{RELATED WORK}

\section{Related Work} \label{sec:related}

% \textbf{Adversarial Attack}
\textbf{Attacks on SLAM.} 
%With the rise of machine learning, 
The robustness of computer vision systems is being actively investigated. With the emergence of adversarial images in the digital domain by adding optimized noise directly to images~\cite{szegedy2013intriguing,carlini2017towards}, researchers find that such attacks also exist physically in the real world \cite{eykholt2018robust,song2018physical,zhao2019seeing}. To fill the gap between attacks in the digital and physical worlds, recent studies have demonstrated that attacks on real-world computer vision systems are practical \cite{eykholt2018robust,li2019adversarial,man2020ghostimage,sharif2016accessorize,zhao2019seeing,zhou2018invisible}. However, attacks on traditional computer vision methods such as SLAM are relatively less explored. \cite{yoshida2022adversarial} proposes an attack against the scan matching algorithm in LiDAR-based SLAM, while most SLAMs in AR/VR devices rely on different sensors like RGB/depth cameras and IMUs. \cite{ikram2022perceptual} and \cite{chen2024adversary} mislead visual SLAM by poisoning the images with special patterns, and \cite{wang2021can} causes the camera to fail using infrared light. In our work, we demonstrate attacks on Visual-Inertial SLAM (VI-SLAM) by perturbing the IMU readings, rather than cameras, and showing its impact on XR user experience. 

\textbf{Acoustic Injection Attacks.} Among various physical attacks, acoustic injection attacks are attractive due to their low cost. Son~\etal~\cite{son2015rocking} were the first to introduce acoustic attacks on MEMS gyroscopes, demonstrating how these attacks could lead to sensor denial-of-service and result in drone crashes. WALNUT~\cite{trippel2017walnut} expanded on this by developing output biasing and control attacks that enable precise manipulation of MEMS accelerometer outputs using modulated sound waves. Wang et al.~\cite{wang2017sonic} demonstrated a sonic gun, showcasing the vulnerability of various smart devices (\eg drones and self-balancing vehicles) to acoustic attacks. Tu et al. \cite{tu2018injected} designed side-swing and switching attacks to alter the outputs of MEMS gyroscopes and accelerometers. Furthermore, Ji et al. \cite{ji2021poltergeist} fool the object detectors by applying acoustic attack to the image stabilizers commonly used in modern cameras. However, none of the existing works study the relationship between the acoustic injections and SLAM outputs on recent XR devices. 

% \zijian{Do we need one session about security in AR/VR?}
% \yicheng{TODO}
%\jiasi{cite the AIVR paper (UMass Amherst?) paper is we have not already. They add IMU perturbation but w/o SLAM, iirc} \yicheng{Cited}

\textbf{XR Security and Privacy.} 
%Security and privacy concerns in XR systems have gained significant attention. 
For single-user XR systems, researchers have demonstrated various side-channel attacks to extract sensitive information (\eg keystrokes) through video feeds~\cite{ling2019know}, head movements~\cite{nair2023unique, slocum2023going}, architectural hints~\cite{zhang2023its,shang2020arspy}, power usage~\cite{li2024dangers}, and EM side-channel leakages~\cite{al2021vr}. In multi-user XR systems, Su et al.~\cite{su2024remote} use avatar motion data to infer keystrokes in shared VR environments. Slocum et al.~\cite{slocum2024doesn} reveal vulnerabilities in the shared state frameworks of multi-user AR. Similarly, Lebeck et al.~\cite{lebeck2017securing} highlight risks like deceptive virtual objects and emphasize access control for managing shared physical and virtual spaces. Ruth et al.~\cite{ruth2019secure} further propose a secure multi-user AR framework focusing on content sharing and permissions.
Chandio et al.~\cite{chandio2024stealthy} %introduced a multi-modal spatiotemporal attack that 
simultaneously manipulated visual and inertial sensors to disrupt XR pose estimation. However, their study evaluated the attack using offline datasets and assumed the attacker's capability to manipulate IMU data streams through acoustic means, without real experiments. Ours is the first to demonstrate acoustic injection attacks on recent XR devices, like the Hololens 2, in the real world.
 


\end{comment}

%\todo{multi-criteria shortest path}

\section{PROBLEM DEFINITION}
We consider a fixed predictive model $h: \mathcal{X} \rightarrow Y$, with $\mathcal{X}= \mathcal{X}^1 \times \mathcal{X}^2 \cdots \mathcal{X}^d$. Each attribute $\mathcal{X}_i$ can be either a continuous or discrete value (e.g., categories). The output can be either binary class $\mathcal{Y}=\{0,1\}$ or stochastic $\hat{\mathcal{Y}}=[0,1]$, which is the probability of the user classified into $1$. The 0 and 1 represent the negative outcome and positive outcome, respectively. A corresponding application example can be a loan approval system where 0 means the ``loan denied'' and 1 means ``loan approved.''

Given a set of $n$ accessible data points $\{x_1,x_2, \cdots x_n \}, x_i \in \mathcal{X}$ and a set of $k$ cost functions $C=\{c_1,c_2, \cdots c_k\}$, one can construct a directed graph which we called actionability graph, where each node is an accessible data point (such as training samples). There is a directed edge from $u$ to $v$ if a feasible action exists from $u$ to $v$. Each cost function $c_i: \mathcal{X} \times \mathcal{X} \rightarrow \mathcal{R}^+$ is a metric that represents the ``distance'' of two data points. Notice that, the cost functions can be non-differentiable, an attribute of the education level is discrete and non-differentiable. Now, consider a specific point $s$ where $h(s)=0$, a feasible path $P$ in the actionability graph is the one that every edge in this path is a feasible action. By abusing the notation of cost function $c$, we denote the total cost of path $P$ is $c(P)$ for the function $c$, which can be other aggregation functions as long as they are metrics, such as summation or any maximum function. 

We define path $P$ as dominated by path $Q$ if and only if for any cost function $c_i$, $c_i(P) \geq c_i(Q)$. Our objective is to find a set of feasible paths in the graph such that the end of each path is a recourse point $t$, where $h(t)=1$ and each path is Pareto optimal concerning all the cost functions. That is, taking summation as the example, path $P$ is Pareto optimal if and only if there does not exist a path $Q$ such that for any cost function $c$, $\sum_{(x,y) \in P} c(x,y) > \sum_{(x,y) \in Q} c(x,y)$.

\section{MULTI-CRITERIA RECOURSE}
\label{subsec:algo}
%In this context, our objective is to provide conceivable actionable and feasible paths for transforming a specific data point (akin to the input for recourse) into its plausible counterfactual.

\section{\thename}
\subsection{End-to-End Driving Policy}
The overall framework of \thename{} is depicted in Fig.~\ref{fig:framework}. 
\thename{} takes multi-view image sequences as input, transforms the sensor data into scene token embeddings, outputs the probabilistic distribution of actions, and samples an action to control the vehicle. 

\boldparagraph{BEV Encoder.} 
We first employ a BEV encoder~\cite{li2022bevformer} to transform multi-view image features from the perspective view to the Bird's Eye View (BEV), obtaining a feature map in the BEV space. This feature map is then used to learn instance-level map features and agent features.

\boldparagraph{Map Head.} 
Then we utilize a group of map tokens~\cite{maptrv2, liao2022maptr, lanegap} to learn the vectorized map elements of the driving scene from the BEV feature map, including lane centerlines, lane dividers, road boundaries, arrows, traffic signals, \etc.

\boldparagraph{Agent Head.} 
Besides, a group of agent tokens~\cite{jiang2022pip} is adopted to predict the motion information of other traffic participants, including location, orientation, size, speed, and multi-mode future trajectories.

\boldparagraph{Image Encoder.} 
Apart from the above instance-level map and agent tokens, we also use an individual image encoder~\cite{vit,he2016resnet} to transform the original images into image tokens. These image tokens provide dense and rich scene information for planning, complementary to the instance-level tokens.

\begin{figure}[t]
\centering
\includegraphics[width=0.98\linewidth]{fig/post-training-2.pdf} 
\caption{\textbf{Post-training.}  $N$  workers parallelly run. The generated rollout data $(s_t,a_t, r_{t+1},s_{t+1},...)$ are recorded in a rollout buffer. Rollout data and human driving demonstrations are used in RL- and IL-training steps to fine-tune the AD policy synergistically.
}
\label{fig:post-training}
\end{figure}

\boldparagraph{Action Space.} 
To accelerate the convergence of RL training, we design a decoupled discrete action representation. 
We divide the action into two independent components: lateral action and longitudinal action. 
The action space is constructed over a short $0.5$-second time horizon, during which the vehicle's motion is approximated by assuming constant linear and angular velocities. 
Under this assumption, the lateral action $a^x$ and longitudinal action $a^y$ can be directly computed based on the current linear and angular velocities.
By combining decoupling with a limited temporal scope and simplified motion model, our approach effectively reduces the dimensionality of the action space, accelerating training convergence.


\boldparagraph{Planning Head.} 
We use $E_\text{scene}$ to denote the scene representation, which consists of map tokens, agent tokens, and image tokens. We initialize a planning embedding denoted as $E_\text{plan}$. A cascaded Transformer decoder $\phi$ takes the planning embedding $E_\text{plan}$ as the query and the scene representation $E_\text{scene}$ as both key and value.

The output of the decoder $\phi$ is then combined with navigation information $E_\text{navi}$ and ego state $E_\text{state}$ to output the probabilistic distributions of the lateral action $a^x$ and the longitudinal action $a^y$:
\begin{equation}
\begin{aligned}
     \pi(a^x\mid s) = & \text{softmax}(\text{MLP}(\phi(E_\text{plan}, E_\text{scene}) \\
    & + E_\text{navi} + E_\text{state})), \\
     \pi(a^y\mid s) = & \text{softmax}(\text{MLP}(\phi(E_\text{plan}, E_\text{scene}) \\
     & + E_\text{navi} + E_\text{state})),
\label{eq:action distribution}
\end{aligned}
\end{equation}
where $E_\text{plan}$, $E_\text{navi}$, $E_\text{state}$, and the output of $\text{MLP}$ are all of the same dimension ($1 \times D$).

The planning head also outputs the value functions $V_x(s)$ and $V_y(s)$, which estimate the expected cumulative rewards for the lateral and longitudinal actions, respectively: 
\begin{equation}
\begin{aligned}
    & V_x(s) = \text{MLP}(\phi(E_\text{plan}, E_\text{scene}) + E_\text{navi} + E_\text{state}), \\
    & V_y(s) = \text{MLP}(\phi(E_\text{plan}, E_\text{scene}) + E_\text{navi} + E_\text{state}).
\end{aligned}
\end{equation}
The value functions are used in RL training (Sec.~\ref{sec:optimization}).

\subsection{Training Paradigm}
We adopt a three-stage training paradigm: perception pre-training, planning pre-training, and reinforced post-training, as shown in Fig.~\ref{fig:framework}.

\boldparagraph{Perception Pre-Training.} 
Information in the image is sparse and low-level. In the first stage,  
the map head and the agent head explicitly output map elements and agent motion information, which are supervised with ground-truth labels. Consequently,  
map tokens and agent tokens implicitly encode the corresponding high-level information.  
In this stage, we only update the parameters of the BEV encoder, the map head, and the agent head.



\boldparagraph{Planning Pre-Training.} 
In the second stage, to prevent the unstable cold start of RL training, IL is first performed to initialize the probabilistic distribution of actions based on large-scale real-world driving demonstrations from expert drivers. In this stage, we only update the parameters of the image encoder and the planning head, while the parameters of the BEV encoder, map head, and agent head are frozen. The optimization objectives of perception tasks and planning tasks may conflict with each other. However, with the training stage and parameters decoupled, such conflicts are mostly avoided.

\boldparagraph{Reinforced Post-Training.} 
In the reinforced post-training, RL and IL synergistically fine-tune the distribution. RL aims to guide the policy to be sensitive to critical risky events and adaptive to out-of-distribution situations. IL serves as the regularization term to keep the policy's behavior similar to that of humans.

We select a large amount of risky dense-traffic clips from collected driving demonstrations. For each clip, we train an independent 3DGS model that reconstructs the clip and serves as a digital driving environment.  
As shown in Fig.~\ref{fig:post-training}, we set $N$ parallel workers.  
Each worker randomly samples a 3DGS environment and begins rollout, i.e., the AD policy controls the ego vehicle to move and iteratively interacts with the 3DGS environment. After the rollout process of this 3DGS environment ends, the generated rollout data $(s_t,a_t, r_{t+1},s_{t+1},...)$ are recorded in a rollout buffer, and the worker will sample a new 3DGS environment for another round of rollout.

As for policy optimization, we iteratively perform RL-training steps and IL-training steps. For RL-training steps, we sample data from the rollout buffer and follow the Proximal Policy Optimization (PPO) framework~\cite{PPO} to update the AD policy. For IL-training steps, we use real-world driving demonstrations to update the policy. After a fixed number of training steps, the updated AD policy is sent to every worker to replace the old one, to avoid a distribution shift between data collection and optimization.
We only update the parameters of the image encoder and the planning head. The parameters of the BEV encoder, the map head, and the agent head are frozen.  
The detailed RL design is presented below.

\subsection{Interaction Mechanism between AD Policy and 3DGS Environment}
In the 3DGS environment, the ego vehicle acts according to the AD policy. Other traffic participants act according to real-world data in a log-replay manner.  
A simplified kinematic bicycle model is employed to iteratively update the ego vehicle's pose at every $\Delta t$ seconds as follows:  
\begin{equation}
\begin{aligned}
x_{t+1}^{w} & = x_{t}^w + v_t \cos \left(\psi_{t}^w\right) \Delta t, \\
y_{t+1}^{w} & = y_{t}^w + v_t \sin \left(\psi_{t}^w\right) \Delta t, \\
\psi_{t+1}^{w} & = \psi_{t}^w + \frac{v_t}{L} \tan \left(\delta_t\right) \Delta t,
\label{equation:kinematic_model}
\end{aligned}
\end{equation}  
where $x_t^{w}$ and $y_t^{w}$ denote the position of the ego vehicle relative to the world coordinate; $\psi_t^w$ is the heading angle that defines the vehicle's orientation with respect to the world $x$-coordinate; $v_t$ is the linear velocity of the ego vehicle; $\delta_t$ is the steering angle of the front wheels; and $L$ is the wheelbase, i.e., the distance between the front and rear axles.

During the rollout process, the AD policy outputs actions $(a_t^x, a_t^y)$ for a $0.5$-second time horizon at time step $t$. We derive the linear velocity $v_t$ and steering angle $\delta_t$ based on $(a_t^x, a_t^y)$.  
Based on the kinematic model in Eq.~\ref{equation:kinematic_model},  
the pose of the ego vehicle in the world coordinate system is updated from ${p}_t = (x_{t}^w, y_{t}^w, \psi_{t}^w)$ to ${p}_{t+1} = (x_{t+1}^{w}, y_{t+1}^{w}, \psi_{t+1}^{w})$.  

Based on the updated ${p}_{t+1}$, the 3DGS environment computes the new ego vehicle's state $s_{t+1}$. The updated pose ${p}_{t+1}$ and state $s_{t+1}$ serve as the input for the next iteration of the inference process.

The 3DGS environment also generates rewards $\mathcal{R}$ (Sec.~\ref{sec:reward}) according to multi-source information (including trajectories of other agents, map information, the expert trajectory of the ego vehicle, and the parameters of Gaussians), which are used to optimize the AD policy (Sec.~\ref{sec:optimization}).

\begin{figure}[t]
\centering
\includegraphics[width=1.0\linewidth]{fig/reward.pdf} 
\caption{\textbf{Example diagram of four types of reward sources.}  (1): Collision with a dynamic obstacle ahead triggers a reward $r_{\text{dc}}$. (2): Hitting a static roadside obstacle incurs a reward $r_{\text{sc}}$. (3): Moving onto the curb exceeds the positional deviation threshold $d_{\text{max}}$, triggering a reward $r_{\text{pd}}$. (4): Drifting toward the adjacent lane exceeds the heading deviation threshold $\psi_{\text{max}}$, triggering a reward $r_{\text{hd}}$.
}
\label{fig: reward source}
\end{figure}
\subsection{Reward Modeling}
\label{sec:reward}
The reward is the source of the training signal, which determines the optimization direction of RL. The reward function is designed to guide the ego vehicle's behavior by penalizing unsafe actions and encouraging alignment with the expert trajectory. It is composed of four reward components: (1) collision with dynamic obstacles, (2) collision with static obstacles, (3) positional deviation from the expert trajectory, and (4) heading deviation from the expert trajectory:
\begin{equation}
\begin{aligned}
\mathcal{R} = \{r_{\text{dc}}, r_{\text{sc}}, r_{\text{pd}}, r_{\text{hd}}  \}. 
\end{aligned}
\end{equation}

As illustrated in Fig.~\ref{fig: reward source}, these reward components are triggered under specific conditions.  
In the 3DGS environment, dynamic collision is detected if the ego vehicle's bounding box overlaps with the annotated bounding boxes of dynamic obstacles, triggering a negative reward $r_{\text{dc}}$. Similarly, static collision is identified when the ego vehicle's bounding box overlaps with the Gaussians of static obstacles, resulting in a negative reward $r_{\text{sc}}$.  
Positional deviation is measured as the Euclidean distance between the ego vehicle's current position and the closest point on the expert trajectory. A deviation beyond a predefined threshold $d_{\text{max}}$ incurs a negative reward $r_{\text{pd}}$.  
Heading deviation is calculated as the angular difference between the ego vehicle's current heading angle $ \psi_t $ and the expert trajectory's matched heading angle $\psi_{\text{expert}}$. A deviation beyond a threshold $ \psi_{\text{max}}$ results in a negative reward $r_{\text{hd}}$.

Any of these events, including dynamic collision, static collision, excessive positional deviation, or excessive heading deviation, triggers immediate episode termination. Because after such events occur, the 3DGS environment typically generates noisy sensor data, which is detrimental to RL training.

\subsection{Policy Optimization}
\label{sec:optimization}
In the closed-loop environment, the error in each single step accumulates over time. The aforementioned rewards are not only caused by the current action but also by the actions of the preceding steps.  
The rewards are propagated forward with Generalized Advantage Estimation (GAE)~\cite{gae} to optimize the action distribution of the preceding steps.

Specifically, for each time step $t$, we store the current state $s_t$, action $a_t$, reward $r_t$, and the estimate of the value $V(s_t)$.  
Based on the decoupled action space, and considering that different rewards have different correlations to lateral and longitudinal actions, the reward $r_t$ is divided into lateral reward $r_t^x$ and longitudinal reward $r_t^y$:
\begin{equation}
\begin{aligned}
r_t^x &= r_t^{\text{sc}} + r_t^{\text{pd}} + r_t^{\text{hd}}, \\
r_t^y &= r_t^{\text{dc}}.
\label{eq:reward-decouple}
\end{aligned}
\end{equation}
Similarly, the value function $V(s_t)$ is decoupled into two components: $V_x(s_t)$ for the lateral dimension and $V_y(s_t)$ for the longitudinal dimension. These value functions estimate the expected cumulative rewards for the lateral and longitudinal actions, respectively. The advantage estimates $\hat{A}_t^x$ and $\hat{A}_t^y$ are then computed as follows:
\begin{equation}
\begin{aligned}
\delta_t^x &= r_t^x + \gamma V_x(s_{t+1}) - V_x(s_t), \\
\delta_t^y &= r_t^y + \gamma V_y(s_{t+1}) - V_y(s_t), \\
\hat{A}_t^x &= \sum_{l=0}^{\infty}(\gamma \lambda)^l \delta_{t+l}^x, \\
\hat{A}_t^y &= \sum_{l=0}^{\infty}(\gamma \lambda)^l \delta_{t+l}^y,
\label{eq:advantage}
\end{aligned}
\end{equation}
where $\delta_t^x$ and $\delta_t^y$ are the temporal difference errors for the lateral and longitudinal dimensions, $\gamma$ is the discount factor, and $\lambda$ is the GAE parameter that controls the trade-off between bias and variance.

To further clarify the relationship between the advantage estimates and the reward components, we decompose $\hat{A}_t^x$ and $\hat{A}_t^y$ based on the reward decomposition in Eq.~\ref{eq:reward-decouple} and the advantage estimation in Eq.~\ref{eq:advantage}. Specifically, we derive the following decomposition:
\begin{equation}
\begin{aligned}
\hat{A}_t^x &= \hat{A}_t^{\text{sc}} + \hat{A}_t^{\text{pd}} + \hat{A}_t^{\text{hd}}, \\
\hat{A}_t^y &= \hat{A}_t^{\text{dc}},
\end{aligned}
\end{equation}
where $\hat{A}_t^{\text{sc}}$ is the advantage estimate for avoiding static collisions, $\hat{A}_t^{\text{pd}}$ is the advantage estimate for minimizing positional deviations, $\hat{A}_t^{\text{hd}}$ is the advantage estimate for minimizing heading deviations, and $\hat{A}_t^{\text{dc}}$ is the advantage estimate for avoiding dynamic collisions.

These advantage estimates are used to guide the update of the AD policy $\pi_{\theta}$, following the PPO framework~\cite{PPO}. By leveraging the decomposed advantage estimates $\hat{A}_t^x$ and $\hat{A}_t^y$, we can independently optimize the lateral and longitudinal dimensions of the policy. This is achieved by defining separate objective functions $\mathcal{L}_x^{\text{CLIP}}(\theta)$ and $\mathcal{L}_y^{\text{CLIP}}(\theta)$ for each dimension,  as follows:
\begin{equation}
\begin{aligned}
\mathcal{L}_x^{\text{PPO}}(\theta) &= \mathbb{E}_t \left[ \min \left( \rho_t^x \hat{A}_t^x, \ \text{clip}(\rho_t^x, 1-\epsilon_x, 1+\epsilon_x) \hat{A}_t^x \right) \right], \\
\mathcal{L}_y^{\text{PPO}}(\theta) &= \mathbb{E}_t \left[ \min \left( \rho_t^y \hat{A}_t^y, \ \text{clip}(\rho_t^y, 1-\epsilon_y, 1+\epsilon_y) \hat{A}_t^y \right) \right], \\
\mathcal{L}^{\text{PPO}}(\theta) &= \mathcal{L}_x^{\text{PPO}}(\theta) + \mathcal{L}_y^{\text{PPO}}(\theta),
\end{aligned}
\end{equation}
where $\rho_t^x = \frac{\pi_{\theta}(a_t^x \mid s_t)}{\pi_{\theta_{\text{old}}}(a_t^x \mid s_t)}$ is the importance sampling ratio for the lateral dimension, $\rho_t^y = \frac{\pi_{\theta}(a_t^y \mid s_t)}{\pi_{\theta_{\text{old}}}(a_t^y \mid s_t)}$ is the importance sampling ratio for the longitudinal dimension, $\epsilon_x$ and $\epsilon_y$ are small constants that control the clipping range for the lateral and longitudinal dimensions, ensuring stable policy updates.

The clipped objective function $\mathcal{L}^{\text{PPO}}(\theta)$ prevents excessively large updates to the policy parameters $\theta$, thereby maintaining training stability.

\begin{table*}[ht]
    \centering
{
\begin{tabular}{lccccccccc}
    \toprule
    RL:IL & CR$\downarrow$ & DCR$\downarrow$ & SCR$\downarrow$ & DR$\downarrow$ & PDR$\downarrow$ & HDR$\downarrow$ &ADD$\downarrow$ & Long. Jerk$\downarrow$ & Lat. Jerk$\downarrow$ \\
    \midrule
     0:1  & 0.229 & 0.211 & 0.018 & 0.066 & 0.039 & 0.027  & 0.238 & 3.928 & 0.103\\
     1:0  & 0.143 & 0.128 & 0.015 &0.080 &0.065 &0.015 &0.345 &4.204 &0.085\\
     2:1 & 0.137 & 0.125 & 0.012 & 0.059 & 0.050 & 0.009  & 0.274 & 4.538 & 0.092\\
     4:1 & 0.089 & 0.080 & 0.009 & 0.063 & 0.042 & 0.021  & 0.257 & 4.495 & 0.082 \\
     8:1 & 0.125 & 0.116 & 0.009 & 0.084 & 0.045 & 0.039  & 0.323 & 5.285 & 0.115\\
    \bottomrule
\end{tabular}
}
    \caption{\textbf{Ablation on RL-to-IL step mixing ratios in the reinforced post-training stage.}}
    \label{tab:ratio}
\end{table*}

\subsection{Auxiliary Objective}
RL usually faces the challenge of sparse rewards, which makes the convergence process unstable and slow. To speed up convergence, we introduce auxiliary objectives that provide dense guidance to the entire action distribution.

The auxiliary objectives are designed to penalize undesirable behaviors by incorporating specific reward sources, including dynamic collisions, static collisions, positional deviations, and heading deviations. These objectives are computed based on the actions \( a_t^{x, \text{old}} \) and \( a_t^{y, \text{old}} \) selected by the old AD policy \( \pi_{\theta_{\text{old}}} \) at time step \( t \). To facilitate the evaluation of these actions, we separate the probability distribution of the action into four parts:
\begin{equation}
\begin{aligned}
\Delta \pi_y^{\text{dec}} &= \sum_{a_t^y < a_t^{y, \text{old}}} \pi_\theta(a_t^y \mid s_t), \\
\Delta \pi_y^{\text{acc}} &= \sum_{a_t^y > a_t^{y, \text{old}}} \pi_\theta(a_t^y \mid s_t), \\
\Delta \pi_x^{\text{left}} &= \sum_{a_t^x < a_t^{x, \text{old}}} \pi_\theta(a_t^x \mid s_t), \\
\Delta \pi_x^{\text{right}} &= \sum_{a_t^x > a_t^{x, \text{old}}} \pi_\theta(a_t^x \mid s_t).
\end{aligned}
\end{equation}
Here, \( \Delta \pi_y^{\text{dec}} \) represents the total probability of deceleration actions, \( \Delta \pi_y^{\text{acc}} \) represents the total probability of acceleration actions, \( \Delta \pi_x^{\text{left}} \) represents the total probability of leftward steering actions, and \( \Delta \pi_x^{\text{right}} \) represents the total probability of rightward steering actions.

\boldparagraph{Dynamic Collision Auxiliary Objective.}  
The dynamic collision auxiliary objective adjusts the longitudinal control action \(a_t^y\) based on the location of potential collisions relative to the ego vehicle. If a collision is detected ahead, the policy prioritizes deceleration actions (\(a_t^y < a_t^{y, \text{old}}\)); if a collision is detected behind, it encourages acceleration actions (\(a_t^y > a_t^{y, \text{old}}\)). To formalize this behavior, we define a directional factor \(f_\text{dc}\):
\begin{equation}
\begin{aligned}
f_\text{dc} = \begin{cases} 
1 & \text{if the collision is ahead}, \\
-1 & \text{if the collision is behind}.
\end{cases} 
\end{aligned}
\end{equation}

The auxiliary objective for dynamic collision avoidance is defined as:
\begin{equation}
\begin{aligned}
\mathcal{L}_\text{dc}(\theta_y) = \mathbb{E}_t \left[ 
    \hat{A}_t^\text{dc} \cdot f_\text{dc} \cdot (\Delta \pi_y^{\text{dec}} - \Delta \pi_y^{\text{acc}})
\right],
\end{aligned}
\end{equation}
where \(\hat{A}_t^\text{dc}\) is the advantage estimate for dynamic collision avoidance.

\boldparagraph{Static Collision Auxiliary Objective.}  
The static collision auxiliary objective adjusts the steering control action $a_t^x$ based on the proximity to static obstacles. If the static obstacle is detected on the left side, the policy promotes rightward steering actions ($a_t^x > a_t^{x,\text{old}}$); if the static obstacle is detected on the right side, it promotes leftward steering actions ($a_t^x < a_t^{x,\text{old}}$). To formalize this behavior, we define a directional factor $f_\text{sc}$:  
\begin{equation}
\begin{aligned}
f_\text{sc} = \begin{cases} 
1 & \text{if static obstacle is on the left}, \\
-1 & \text{if static obstacle is on the right}.
\end{cases} 
\end{aligned}
\end{equation}

The auxiliary objective for static collision avoidance is defined as:  
\begin{equation}
\begin{aligned}
\mathcal{L}_\text{sc}(\theta_x) = \mathbb{E}_t \left[ 
    \hat{A}_t^\text{sc} \cdot f_\text{sc} \cdot (\Delta \pi_x^{\text{right}} - \Delta \pi_x^{\text{left}})
\right],
\end{aligned}
\end{equation}  
where $\hat{A}_t^\text{sc}$ is the advantage estimate for static collision avoidance.  

\boldparagraph{Positional Deviation Auxiliary Objective.}  
The positional deviation auxiliary objective adjusts the steering control action $a_t^x$ based on the ego vehicle's lateral deviation from the expert trajectory. If the ego vehicle deviates leftward, the policy promotes rightward corrections ($a_t^x > a_t^{x,\text{old}}$); if it deviates rightward, it promotes leftward corrections ($a_t^x < a_t^{x,\text{old}}$). We formalize this with a directional factor $f_\text{pd}$:  
\begin{equation}
\begin{aligned}
f_\text{pd} = \begin{cases} 
1 & \text{if ego vehicle deviates leftward}, \\
-1 & \text{if ego vehicle deviates rightward}.
\end{cases} 
\end{aligned}
\end{equation}

The auxiliary objective for positional deviation correction is:
\begin{equation}
\begin{aligned}
\mathcal{L}_\text{pd}(\theta_x) = \mathbb{E}_t \left[ 
    \hat{A}_t^\text{pd} \cdot f_\text{pd} \cdot (\Delta \pi_x^{\text{right}} - \Delta \pi_x^{\text{left}})
\right],
\end{aligned}
\end{equation}  
where $\hat{A}_t^\text{pd}$ estimates the advantage of trajectory alignment.

\boldparagraph{Heading Deviation Auxiliary Objective.}  
The heading deviation auxiliary objective adjusts the steering control action $a_t^x$ based on the angular difference between the ego vehicle’s current heading and the expert’s reference heading. If the ego vehicle deviates counterclockwise, the policy promotes clockwise corrections ($a_t^x > a_t^{x,\text{old}}$); if it deviates clockwise, it promotes counterclockwise corrections ($a_t^x < a_t^{x,\text{old}}$). To formalize this behavior, we define a directional factor $f_\text{hd}$:  
\begin{equation}
\begin{aligned}
f_\text{hd} = \begin{cases} 
1 & \text{if ego vehicle deviates clockwise}, \\
-1 & \text{if ego vehicle deviates counterclockwise}.
\end{cases} 
\end{aligned}
\end{equation}

The auxiliary objective for heading deviation correction is then defined as:  
\begin{equation}
\begin{aligned}
\mathcal{L}_\text{hd}(\theta_x) = \mathbb{E}_t \left[ 
    \hat{A}_t^\text{hd} \cdot f_\text{hd} \cdot (\Delta \pi_x^{\text{right}} - \Delta \pi_x^{\text{left}})
\right],
\end{aligned}
\end{equation}  
where $\hat{A}_t^\text{hd}$ is the advantage estimate for heading alignment.  

\begin{table*}[ht]
\begin{center}
\centering
\resizebox{0.98\textwidth}{!}{
\begin{tabular}{cccccccccccccc}
\toprule
\multirow{2}{*}{ID} & Dynamic & Static & Position & Heading & \multirow{2}{*}{CR$\downarrow$} &\multirow{2}{*}{DCR$\downarrow$} &\multirow{2}{*}{SCR$\downarrow$} &\multirow{2}{*}{DR$\downarrow$} &\multirow{2}{*}{PDR$\downarrow$} &\multirow{2}{*}{HDR$\downarrow$} &\multirow{2}{*}{ADD$\downarrow$} &\multirow{2}{*}{Long. Jerk$\downarrow$} &\multirow{2}{*}{Lat. Jerk$\downarrow$}\\
& Collision & Collision & Deviation & Deviation & & & & & & & & & \\
\midrule
1 & \cmark  &  &  &  & 0.172 & 0.154 & 0.018 & 0.092 & 0.033 & 0.059  & 0.259 & 4.211 & 0.095 \\
2 &  & \cmark & \cmark & \cmark & 0.238 & 0.217 & 0.021 & 0.090 & 0.045 & 0.045  & 0.241 & 3.937 & 0.098 \\
3 & \cmark &  & \cmark & \cmark & 0.146 & 0.128 & 0.018 & 0.060 & 0.030 & 0.030  & 0.263 & 3.729 & 0.083\\
4 & \cmark & \cmark &  & \cmark & 0.151 & 0.142 & 0.009 & 0.069 & 0.042 & 0.027 & 0.303 & 3.938 & 0.079\\
5 & \cmark & \cmark & \cmark &  & 0.166 & 0.157 & 0.009 & 0.048 & 0.036 & 0.012 & 0.243 & 3.334 & 0.067\\
6 & \cmark & \cmark & \cmark & \cmark & 0.089 & 0.080 & 0.009 & 0.063 & 0.042 & 0.021 & 0.257 & 4.495 & 0.082 \\
\bottomrule
\end{tabular}
}
\end{center}
\vspace{-2mm}
\caption{\textbf{Ablation on reward sources.} The table shows the impact of different reward components on performance.}
\label{tab:reward_ablation}
\end{table*}

\begin{table*}[ht]
\begin{center}
\centering
\resizebox{0.98\textwidth}{!}{
\begin{tabular}{ccccccccccccccc}
\toprule
\multirow{2}{*}{ID} & \multirow{2}{*}{PPO Obj.}  & Dynamic Col. & Static Col. & Position Dev. & Heading Dev. & \multirow{2}{*}{CR$\downarrow$} & \multirow{2}{*}{DCR$\downarrow$}  & \multirow{2}{*}{SCR$\downarrow$} & \multirow{2}{*}{DR$\downarrow$} & \multirow{2}{*}{PDR$\downarrow$} & \multirow{2}{*}{HDR$\downarrow$} & \multirow{2}{*}{ADD$\downarrow$} & \multirow{2}{*}{Long. Jerk$\downarrow$} & \multirow{2}{*}{Lat. Jerk$\downarrow$} \\
& & Auxiliary Obj. & Auxiliary Obj. & Auxiliary Obj. & Auxiliary Obj. & & & & & & & & & \\
\midrule
1 &\cmark&  &  &  &  & 0.249 & 0.223 & 0.026 & 0.077 & 0.047 & 0.030  & 0.266 & 4.209 & 0.104 \\
2 &\cmark& \cmark &  &  &  & 0.178 & 0.163 & 0.015 & 0.151 & 0.101 & 0.050 & 0.301 & 3.906 & 0.085 \\
3 &\cmark&  & \cmark & \cmark & \cmark & 0.137 & 0.125 & 0.012 & 0.157 & 0.145 & 0.012 & 0.296 & 3.419 & 0.071 \\
4 &\cmark& \cmark &  & \cmark & \cmark & 0.169 & 0.151 & 0.018 & 0.075 & 0.042 & 0.033 & 0.254 & 4.450 & 0.098 \\
5 &\cmark& \cmark & \cmark &  & \cmark & 0.149 & 0.134 & 0.015 & 0.063 & 0.057 & 0.006 & 0.324 & 3.980 & 0.086 \\
6 &\cmark& \cmark & \cmark & \cmark & & 0.128 & 0.119  & 0.009 & 0.066 & 0.030 & 0.036  & 0.254 & 4.102 & 0.092 \\
7 &&\cmark  &\cmark  &\cmark  &\cmark  & 0.187 &0.175  &0.012 &0.077 &0.056  &0.021  &0.309  &5.014  &0.112  \\
8 &\cmark& \cmark & \cmark & \cmark & \cmark & 0.089 & 0.080 & 0.009 & 0.063 & 0.042 & 0.021  & 0.257 & 4.495 & 0.082 \\
\bottomrule
\end{tabular}
}
\end{center}
\vspace{-2mm}
\caption{\textbf{Ablation on auxiliary objectives.} The table shows the impact of different auxiliary objectives on performance.}
\label{tab:auxiliary_ablation}
\end{table*}

\boldparagraph{Overall Auxiliary Objectives.}  
The overall auxiliary objectives are a weighted sum of the individual objectives:
\begin{equation}
\begin{aligned}
\mathcal{L}_\text{aux}(\theta) = &\lambda_1 \mathcal{L}_\text{dc}(\theta_y) + \lambda_2 \mathcal{L}_\text{sc}(\theta_x)  + \\ 
&\lambda_3 \mathcal{L}_\text{pd}(\theta_x) +\lambda_4 \mathcal{L}_\text{hd}(\theta_x),
\end{aligned}
\end{equation}
where $\lambda_1$, $\lambda_2$, $\lambda_3$, and $\lambda_4$ are weighting coefficients that balance the contributions of each auxiliary objective.

\boldparagraph{Optimization Objective.}  
The final optimization objective combines the clipped PPO objective with the auxiliary objective:
\begin{equation}
\mathcal{L}(\theta) = \mathcal{L}^{\text{PPO}}(\theta) + \mathcal{L}_\text{aux}(\theta).
\end{equation}

%\begin{tikzpicture}[scale=0.6]
\pgfdeclarelayer{bg}
\pgfsetlayers{bg,main}
\node[vertex, fill=white, fill opacity=0.8, text opacity=1] (0) at (2.5, 4.8) {\scriptsize 0};
\node[vertex, fill=white, fill opacity=0.8, text opacity=1] (1) at (4, 4.5) {\scriptsize 1};
\node[vertex, fill=white, fill opacity=0.8, text opacity=1] (3) at (0.99, 0.90) {\scriptsize 3};
\node[vertex, fill=white, fill opacity=0.8, text opacity=1] (5) at (5.00, 2) {\scriptsize 5};
\node[vertex, fill=white, fill opacity=0.8, text opacity=1] (6) at (5.00, 0.90) {\scriptsize 6};
\node[vertex, fill=white, fill opacity=0.8, text opacity=1] (7) at (1, 4.4) {\scriptsize 7};
\node[vertex, fill=white, fill opacity=0.8, text opacity=1] (8) at (2.6, 2.6) {\scriptsize 8};
\node[vertex, fill=white, fill opacity=0.8, text opacity=1] (11) at (4.9, 3.5) {\scriptsize 11};
\node[vertex, fill=white, fill opacity=0.8, text opacity=1] (2) at (2.45, 1.1) {\scriptsize 2};
\node[vertex, fill=white, fill opacity=0.8, text opacity=1] (10) at (3.9, 1.1) {\scriptsize 10};
\node[vertex, fill=white, fill opacity=0.8, text opacity=1] (9) at (4.08, 1.9) {\scriptsize 9};
\node[vertex, fill=white, fill opacity=0.8, text opacity=1] (4) at (1.1, 3.31) {\scriptsize 4};
\node[vertex, fill=white, fill opacity=0.8, text opacity=1] (12) at (3.5, 3.3) {\scriptsize 12};
\begin{pgfonlayer}{bg}
\draw[-latex] (0) edge[bend left=10] (1);
\draw[-latex] (0) edge[bend left=10] (3);
\draw[-latex] (0) edge[bend left=10] (5);
\draw[-latex] (0) edge[bend left=10] (6);
\draw[-latex] (0) edge[bend left=10] (7);
\draw[-latex] (0) edge[bend left=10] (8);
\draw[-latex] (0) edge[bend left=10] (11);
\draw[-latex] (1) edge[bend left=10] (0);
\draw[-latex] (1) edge[bend left=10] (3);
\draw[-latex] (1) edge[bend left=10] (5);
\draw[-latex] (1) edge[bend left=10] (11);
\draw[-latex] (3) edge[bend left=10] (0);
\draw[-latex] (3) edge[bend left=10] (1);
\draw[-latex] (3) edge[bend left=10] (5);
\draw[-latex] (3) edge[bend left=10] (6);
\draw[-latex] (3) edge[bend left=10] (7);
\draw[-latex] (3) edge[bend left=10] (8);
\draw[-latex] (3) edge[bend left=10] (9);
\draw[-latex] (3) edge[bend left=10] (11);
\draw[-latex] (5) edge[bend left=10] (0);
\draw[-latex] (5) edge[bend left=10] (1);
\draw[-latex] (5) edge[bend left=10] (3);
\draw[-latex] (5) edge[bend left=10] (6);
\draw[-latex] (5) edge[bend left=10] (11);
\draw[-latex] (6) edge[bend left=10] (0);
\draw[-latex] (6) edge[bend left=10] (3);
\draw[-latex] (6) edge[bend left=10] (4);
\draw[-latex] (6) edge[bend left=10] (5);
\draw[-latex] (6) edge[bend left=10] (11);
\draw[-latex] (7) edge[bend left=10] (0);
\draw[-latex] (7) edge[bend left=10] (8);
\draw[-latex] (7) edge[bend left=10] (9);
\draw[-latex] (8) edge[bend left=10] (0);
\draw[-latex] (8) edge[bend left=10] (7);
\draw[-latex] (8) edge[bend left=10] (9);
\draw[-latex] (11) edge[bend left=10] (0);
\draw[-latex] (11) edge[bend left=10] (1);
\draw[-latex] (11) edge[bend left=10] (3);
\draw[-latex] (11) edge[bend left=10] (5);
\draw[-latex] (11) edge[bend left=10] (6);
\draw[-latex] (2) edge[bend left=10] (0);
\draw[-latex] (2) edge[bend left=10] (1);
\draw[-latex] (2) edge[bend left=10] (3);
\draw[-latex] (2) edge[bend left=10] (5);
\draw[-latex] (2) edge[bend left=10] (6);
\draw[-latex] (2) edge[bend left=10] (10);
\draw[-latex] (2) edge[bend left=10] (11);
\draw[-latex] (10) edge[bend left=10] (1);
\draw[-latex] (10) edge[bend left=10] (3);
\draw[-latex] (10) edge[bend left=10] (5);
\draw[-latex] (10) edge[bend left=10] (6);
\draw[-latex] (10) edge[bend left=10] (7);
\draw[-latex] (10) edge[bend left=10] (8);
\draw[-latex] (10) edge[bend left=10] (9);
\draw[-latex] (10) edge[bend left=10] (11);
\draw[-latex] (9) edge[bend left=10] (7);
\draw[-latex] (9) edge[bend left=10] (8);
\draw[-latex] (4) edge[bend left=10] (0);
\draw[-latex] (4) edge[bend left=10] (1);
\draw[-latex] (4) edge[bend left=10] (3);
\draw[-latex] (4) edge[bend left=10] (5);
\draw[-latex] (4) edge[bend left=10] (6);
\draw[-latex] (4) edge[bend left=10] (7);
\draw[-latex] (4) edge[bend left=10] (8);
\draw[-latex] (4) edge[bend left=10] (9);
\draw[-latex] (4) edge[bend left=10] (10);
\draw[-latex] (4) edge[bend left=10] (11);
\draw[-latex] (4) edge[bend left=10] (12);
\draw[-latex] (12) edge[bend left=10] (1);
\draw[-latex] (12) edge[bend left=10] (2);
\draw[-latex] (12) edge[bend left=10] (3);
\draw[-latex] (12) edge[bend left=10] (5);
\draw[-latex] (12) edge[bend left=10] (7);
\draw[-latex] (12) edge[bend left=10] (8);
\draw[-latex] (12) edge[bend left=10] (9);
\draw[-latex] (12) edge[bend left=10] (10);
\draw[-latex] (12) edge[bend left=10] (11);
\end{pgfonlayer}
\end{tikzpicture}

%\paragraph{Algorithm}
we adapt and modify the Bellman-Ford algorithm which use a dynamic programming approach. The main goal is to handle the minima cost within all considered criteria.

\begin{algorithm}
\caption{Multi-shortest-path}
\begin{algorithmic}[1]
    \Function{Multi-shortest-path}{$G = \{E,V\}, source$}
        \State {\# Initialization}
        \For{each $V$ in $G$}
            \State$D_v \gets \phi$
        \EndFor
        \State$D_{source} \gets \{0\}$
        %\State$t_s \gets$add a terminal point
        
        \State {\# Bellman-Ford algorithm.}
        \For{$i \gets 1$ to $v-1$}
            \For{each $E$}
                (x, y belong to E)
                \State update($D_y, (D_x, T_{xy})$)
            \EndFor
            \State$D_y \gets$dominate($D_y$)
        \EndFor
        \State \Return$D$
    \EndFunction
\end{algorithmic}
\end{algorithm}

\begin{figure}[ht!]
\centering
\includegraphics[width=60mm]{structure.png}
\caption{Schematic diagram for using Use $u$ to relax the distance of $\overline{sv}$ \label{structure}}
\end{figure}

Considering multiple cost criteria, algorithm2 computes the shortest paths from a given source vertex to all other vertices in a graph. The distance list $D_v$ represents the set of target points for reaching vertex $v$ from the source vertex. Initially, all item in $D_v$ are empty except for the source vertex. We set the distance to itself $D_s$ as zero for all cost dimensions. 

Secondly, it employs a modified version of the Bellman-Ford algorithm. During each iteration, it iterates over the vertices excluding the source and conducts a tailored merging process, for each edge $(x, y)$ in the graph, performs an update operation. The update operation extends the $D_y$for the destination vertex $y$ by combining the distance from the source vertex $x$ with the cost of the current edge $(x, y)$ and pruning step. Let $f^\prime (u, \ell - 1) = update(f(u, \ell - 1), d(u, v))$, which means update $D_u$ by connect with edge $\overline{uv}$
\todo{use "update" in pseudo code might be confused with the dp equation}
\begin{equation}
\begin{aligned}
f(v, \ell) = prune_{u \in \text{in}(v)}(\text{concatenate}\left(f^\prime (u, \ell - 1)) ,f(v, \ell - 1)\right)
\end{aligned}
\end{equation}

\textbf {concatenate} 

The concatenate operation is responsible for updating the distance $D$ estimate f(v, $\ell$) for a target vertex v at iteration $\ell$ by combining the distance estimates f(u, $\ell$ - 1) from the source to all vertices u that have an incoming edge to v (denoted by u $\in$ in(v)), with the respective costs d(u, v) of the edges from u to v. The f(u, $\ell$ - 1) represents the distance estimate from the source to the vertex u at the previous iteration $\ell$ - 1. Based on the Bellman-Ford algorithm, the concatenate operation uses the distance estimate f(u, $\ell$ - 1) from the source to u to update or "relax" the cost or distance estimate f(v, $\ell$) for the target vertex v. It considers all possible paths from adjacent vertices to v through u, accumulating the distances along the way.\\

\textbf {prune} 
 
After obtaining the distances through the concatenate operation. The prune operation is employed to refine the distances estimate f(v, $\ell$) for a vertex v at iteration $\ell$ by comparing the trade-offs between the current estimate f(v, $\ell$ - 1) and the newly merged distances from the adjacent vertices u $\in$ in(v). It then iterates through the sorted list, ensures that the distance list remains compact and efficient by retaining only the dominant points that represent the most significant trade-offs in the multi-dimensional space.  


\todo {whether we need to add table size control}
Additionally, if the size of the pruned distance list exceeds a specified limit, the prune function further reduces the list by sampling a fixed number of points at equal intervals, preserving the overall shape of the cost curve while limiting its size. This sampling process helps manage the size of the distance lists and ensures efficiency in subsequent calculations.
% Once all edges have been processed, the algorithm returns the list of distance $D$. These distance capture the trade-offs between different cost criteria and can be used for efficient decision-making in various scenarios, such as transportation or communication networks, where multiple cost factors influence path selection.

\begin{algorithm}
\caption{Backtracking Algorithm}
\begin{algorithmic}[1]
    \Function{backtracking}{$S, T, D$}
        \State paths$\gets$[]
        \For{$u$ belongs to$D_T$}
            \State path$\gets$[]
            \State$v \gets T$
            \While{$v \neq S$}
                \State path.add($u$)
                \State cost$\gets$get edge cost($v, u$)
                \For{each$i$belongs to$D_u$}
                    \If{cost$v ==$cost$u +$cost}
                        \State find parent if cost$v ==$cost$u +$cost
                        \State$v \gets u$
                        \State$u \gets i$
                    \EndIf
                \EndFor
            \EndWhile
            \State paths.add(path)
        \EndFor
        \State \Return paths
    \EndFunction
\end{algorithmic}
\end{algorithm}

The Backtracking Algorithm is used to find paths in a graph from a target vertex back to a source vertex, given certain constraints. The algorithm operates by exploring possible paths starting from each vertex in a set of destination vertices $D_T$, backtracking to the source vertex $S$. For each destination vertex $u$ in the set $D_T$, the algorithm initiates a search for a path by traversing edges backward from $u$ to $S$. It does so by iterative examining edges and their associated costs, identifying the parent vertex that minimizes the total cost to reach $u$ from $S$. This process continues until the source vertex $S$ is reached. The algorithm then records the discovered path and proceeds to explore paths from other destination vertices. Once all possible paths have been explored, the algorithm returns the collected paths. This approach is useful for various applications, including finding optimal routes in transportation networks or identifying dependencies in computational tasks.
%
\textbf{time complexity}

When we run the dynamic programming equation (1) in multi-shortest path algorithm. The time complexity analysis involves two main steps: concatenate and prune, which are repeated for at most $V-1$ iterations based on the Bellman-Ford algorithm for all edges. In the concatenate step, for each vertex v, we combine the distance tables $D_{u \in in(v)}$ from all incoming neighbors u $\in$ in(v) with the respective edge costs $d(u, v)$. This step has a time complexity of $O(n)$, where n is the size of the distance table $D$.

The subsequent pruning step aims to retain only the optimal solutions in the concatenated distance table. The time complexity of this pruning operation depends on the number of criteria considered. When dealing with two criteria, the pruning step leverages the theory of maximal point sets, resulting in a time complexity of O(n). However, when the number of criteria exceeds two, the time complexity increases to $O(n\log^{n-3}\log\log n)$.

Considering both the concatenate and prune steps, the overall time complexity of the multi-shortest path algorithm is $O(n|V||E|)$. Here, $|V|$ and$ |E|$ represent the number of vertices and edges in the graph respectively. The factor n accounts for the size of the distance table $D$, which may vary depending on the specific problem instance and the number of criteria considered.

 \textbf{correctness}
% we need to prove two things:\\

% \begin{itemize}
   
%      \item [(1)]
%     All optimal solution are in $D$
%      \item [(2)]
%     All solution in $D$ is optimal (when without constraint for $D$ size)
% \end{itemize}
% once we can prove (1) and (2) are correct, which means (1) equals to (2),then we can prove our algorithm is correct. Here is the prove for item 1, since the dp equation has already consider all edge that link in to $v$, we can promised that we find all possible temporary solution, which containing all optimal solutions.
% As for item 2, Assume $f(v,\ell)$ is Pareto optimal and $\exists$ $d_u \in f(u, \ell-1)$ is not Pareto optimal solution. Since the feature of Monotonicity of cost function. After $d_u \in f(u, \ell-1)$ connect with weight cost$ \overline{uv} $ still is not Pareto optimal solution. After step of prune, those worse solution would be dominate, which means $f(v,\ell)$ will not be the Pareto optimal solution. $\to \gets .$ Q.E.D. that, $\forall d_v \in f(v, \ell)$ is Pareto optimal, then $\forall _{u \in \text{in}(v)} \left(f(u, \ell - 1) + d(u, v)\right) and \forall d_v \in f(v, \ell - 1)$ must be Pareto optimal solution. To be more specific, only when last iteration are Pareto optimal, can next iteration still contain all Pareto optimal solution. the only exception which we can't explain is the source vertex  $f(v, 0)$, which doesn't have any coming edge. However, This first step is established, so it is proved.

To prove the correctness of the multi-shortest path algorithm, we need to establish two key points:


\begin{enumerate}
    \item 1. All optimal solutions are present in the distance table $D$.
    \item 2. All solutions in the distance table $D$ are optimal (when there is no constraint on the size of $D$).
\end{enumerate}

If we can prove that (1) and (2) are correct, implying that they are equivalent, then we can establish the correctness of our algorithm.

Regarding item (1), since the dynamic programming equation considers all edges that link to vertex $v$, we can guarantee that we find all possible temporary solutions, which contain all optimal solutions.

As for item (2), assume that $f(v, \ell)$ is Pareto optimal, and there exists $d_u \in f(u, \ell-1)$ that is not a Pareto optimal solution. Due to the monotonicity feature of the cost function, after $d_u \in f(u, \ell-1)$ is connected with the weight cost $d(u, v)$, it still will not be a Pareto optimal solution. After the pruning step, these worse solutions would be dominated, which means $f(v, \ell)$ will not be the Pareto optimal solution. This leads to the conclusion that for all $d_v \in f(v, \ell)$ to be Pareto optimal, all $f(u, \ell-1)$ for $u \in \text{in}(v)$ and $f(v, \ell-1)$ must be Pareto optimal solutions. In other words, only when the previous iteration contains Pareto optimal solutions can the next iteration still contain all Pareto optimal solutions. The only exception we cannot explain is the source vertex $f(v, 0)$, which does not have any incoming edges. However, this initial step is established, so it is proved.\\

\textbf{Convergence property}\\
If $s \rightsquigarrow u \rightarrow v$ is shortest path in $G$ for some $u,v \in V$ ,and if $u.d = \delta(s,u)$ at any time prior to relaxing $edge(u,v)$, then $v.d = \delta(s,v)$ at all times afterward.\\




\section{SCALABILITY ENHANCEMENT}
\label{sec:scalability}
In Section~\ref{subsec:algo}, we show that finding all Pareto optimal on the actionability graph is crucial to the size of the Pareto tables $\tau$, edge size $|E|$, and number of iterations $\eta$. Since $\eta$ corresponds to the number of hops of the output paths, which is usually as a constant (considering in reality that a long feasible path is redundant and non-interpretable), the bottleneck of the running time is mainly on $\tau$ and $|E|$. Additionally, $\tau$ is bounded by the number of different paths from each pair of nodes, which highly depends on the size of the vertices and the connectivity. To decrease the graph size and simplify the connectivity structure, one idea is to shrink the vertices of the graph such that there are only a small number of ``representative'' nodes, and the shortest path in this shrunk graph still preserves or approximates the distance of the original graph. This is the idea of core-set from the computational geometry perspective (see survey in~\cite{agarwal2005geometric}). The challenge here is that the cost functions are not specific but highly general. Additionally,  we also need to incorporate all the $k$ cost functions to get the Pareto optimal. Fortunately, the computation of the cost functions usually contains some structures rather than arbitrary values for any pair of points. This inspires us to utilize the idea of $\epsilon$-net~\cite{haussler1986epsilon} to shrink the size of the graph and also ensure the quality. 

To explicitly explain our idea, we first define the notation of shrinkable.''

\begin{definition}
\label{def: shrinkable}
    Given $G=(V,E)$ and a cost function $c$, we say a vertex $i$ is $\kappa$-shrinkable to vertex $j$ if and only if $\forall (p,i) \in E$
    $$ 
    (p,j) \in E \text{ and } c(p,j) \leq \kappa c(p,i)
    $$
\end{definition}

Given the approximation factor $\kappa$, one can iteratively shrink all the shrinkable vertices in the graph until there is no shrinkable vertex anymore. We call this induced subgraph a shrunk graph $G_S$ and the one with the smallest cardinality is $G_S^*$. Obviously, any $G_S$  preserves $\kappa$-approximation factor. The shortest path between any pair of the nodes in $G$ has another path in $G_S$ which is at most $\kappa l$ times, where $l$ is the number of the hops of the path. However, finding the $G_S^*$ is highly non-trivial. It depends on the order of vertices in the shrinking procedure. A toy example is in the following. Consider a graph with vertices $\{p,i,j,r\}$ and edges $\{(p,i),(p,j),(p,r)\}$ where $c(p,i)=\kappa c(p,j)= \kappa^2 c(p,r)$, if $j$ shrinks to $i$, then $i,r$ are not shrinkable. On the other hand, if $i$ shrinks to $j$, then $r$ can shrink to $j$ too. Thus, we want to have another subgraph that catches most properties of $G_S$ and can be generated efficiently. This is the place where the $\epsilon$-net joins into our work. In the following, we will first introduce the formal definition of $\epsilon$-net and then show how to utilize it under our context.

\begin{definition}
\label{def: epsilon-net}
Given a range space $(\mathcal{X}$,$\mathcal{R})$, let $\mathcal{A} \subset \mathcal{X}$ be a finite subset, and $0<\epsilon<1$. Then a subset $\mathcal{N} \subset \mathcal{A}$ is called an $\epsilon$-net of $\mathcal{A}$ w.r.t to $\mathcal{R}$ if 

$$
\forall r \in \mathcal{R}, |r \cap \mathcal{A}|>\epsilon |\mathcal{A}| \rightarrow r\cap \mathcal{N} \neq \emptyset
$$
\end{definition}

Now, we define the $\epsilon$-net under our context. 

\begin{definition}
\label{def: our epsilon-net}
We say $G_\epsilon$ is an $\epsilon$-net of $G$ if for any vertex $v$ in some shrunk graph $G_S$ that is shrunk by more than $\epsilon n$ vertices, then either $v \in G_\epsilon$ or $u \in G_\epsilon$, where $v$ is shrinkable to $u$.    
\end{definition}

To see Definition~\ref{def: epsilon-net} and Definition~\ref{def: our epsilon-net} are equivalent, one can see the element $r$ of a range $r \in \mathcal{R}$ is a subset of $V$ which is an instance of the shrinking procedure. That is all the elements in $r$ shrink into a point in some graph $G_S$. Thus, an $\epsilon$-net should include one of the points in $r$, which leads to Definition~\ref{def: our epsilon-net}. The interpretation of $G_\epsilon$ is that it includes the majority of vertices (i.e., which is shrunk from $\epsilon n$ vertices) among all the $G_S$.

Notice that with an arbitrary cost function, one cannot have an $G_\epsilon$ or a $G_S$ with a small size. However, if the cost function has some structure, one can analyze the VC-dimension of the range space and utilize the theorem proved by Haussler-Welzl~\cite{haussler1986epsilon}, which states that any random sample set $S$ of $G$ with size 
\begin{equation}
\label{eq:Haussler-Welzl}
O(\frac{|VC|}{\epsilon} \log \frac{1}{\epsilon} + \frac{1}{\epsilon} \log \frac{1}{\delta})  
\end{equation}
is an $\epsilon$-net, with probaility more than $1-\delta$. In addition, we can use one sample set $S$ to fit all the cost functions. Thus, assume $\delta$ is a constant, the size of $S$ for the multi-cost function is $O(|VC|^*/\epsilon \log 1/\epsilon)$, where $|VC|^*$ is the largest VC-dimension among all the cost functions, which is highly scalable in respect to the graph size $n$ and the number of cost functions $k$.

\paragraph{Demonstration}
We demonstrate how to analyze the $VC$-dimension for a cost function $c$ with certain properties to generate the $\epsilon$-net. Assume function $c$ has a property that for every $(i,j) \in E$,  $\Delta(i,j) \leq c(i,j) \leq \kappa \Delta(i,j)$, where $\Delta(i,j)$ is the distance metric of the data space. Notice that this property is the same as discrete Lipschitz continuity~\cite{jiang2011free} except it also has the lower bound on $c(i,j)$. This property is commonly true when the cost function has a similar structure in the metric of the data space, L-norm class for instance. In real analysis, it can be interpreted that the cost of two points cannot be too high concerning the distance of the data space (e.g., the effort of increasing the income from 10,000 to 12,000 can not be too large) but also not too low if the distance of the data space is far (e.g., there should be non-negligible effort of increasing the income from 8,000 to 15,000). 

Now, observe that this property implies that the two points are $\kappa$-shrinkable if and only if they are within the distance of 1 in the data space. Thus, an instance of the shrinking procedure is the same as putting a ball with a radius $1/2$ in the data space and all the points in this ball can shrink into a point. We can see the range $\mathcal{R}$ is the balls in the data space and The $\epsilon$-net here is asking what is the smallest sample set so that any ball containing at least $\epsilon n$ points also contains one of the sampled points. Fortunately, the $VC$-dimension of shattering the points via balls is at most shattering the points via the hyperplanes, which has the VC-dimension as $d+1$. Via Haussler-Welzl's theorem in Equation~\ref{eq:Haussler-Welzl}, a random sampled set with size $O( (d+1)^*/\epsilon \log 1/\epsilon)$ is an $\epsilon$-net for function $c$.  

%\section{FRAMEWORK EXTENSION}
%\subsection{Diverse Recourse}
%\subsection{Load-balancing in Multiple Recourse Actions}

\section{EXPERIMENT}

\begin{figure*}[!h]
    \centering
    \begin{subfigure}[b]{0.8\linewidth}
        \centering
        \includegraphics[width=0.45\linewidth]{images/residual/text/CIReVL_Recall5.png}
        \hfil
        \includegraphics[width=0.45\linewidth]{images/residual/text/pic2word_recall5.png}
        \caption{\textbf{PDV-T}: Impact of $\alpha$ scaling on composed text embeddings}
        \label{fig:residual_text_sub}
    \end{subfigure}
    
    \begin{subfigure}[b]{0.8\linewidth}
        \centering
        \includegraphics[width=0.45\linewidth]{images/residual/image/CIReVL_Recall5.png}
        \hfil
        \includegraphics[width=0.45\linewidth]{images/residual/image/pic2word_recall5.png}
        \caption{\textbf{PDV-I}: Impact of $\alpha$ scaling on composed image embeddings}
        \label{fig:residual_image_sub}
    \end{subfigure}
    
    \begin{subfigure}[b]{0.8\linewidth}
        \centering
        \includegraphics[width=0.45\linewidth]{images/residual/fusion/CIReVL_Recall5.png}
        \hfil
        \includegraphics[width=0.45\linewidth]{images/residual/fusion/pic2word_recall5.png}
        \caption{\textbf{PDV-F}: Impact of varying $\beta$ with on composed fused embeddings}
        \label{fig:residual_fusion_sub}
    \end{subfigure}
    \caption{Impact of changing $\alpha$/$\beta$ on Recall@5 performance across different PDV applications. For each row, results are shown for the CIReVL (left) and Pic2Word (right) baseline methods.}
    \label{fig:residual_all}
\end{figure*}

\section{Experiments} 
\label{sec:exp}
\noindent\textbf{Implementation Details.} We utilize the official implementations of four ZS-CIR baseline methods: CIReVL\footnote{https://github.com/ExplainableML/Vision\_by\_Language} and LDRE \footnote{https://github.com/yzy-bupt/LDRE} as representative caption-based feature extraction approaches and Pic2Word\footnote{https://github.com/google-research/composed\_image\_retrieval} and SEARLE\footnote{https://github.com/miccunifi/SEARLE} as representative pseudo tokenization-based methods. All feature extraction processes follow the original implementations provided by these baseline methods. However, to calculate $\Delta_{PDV}$, we need text embeddings without prompts, which are not provided in the original implementations. For CIReVL and LDRE, we obtain these embeddings by passing the generated image captions directly to CLIP. For Pic2Word and SEARL, we construct the base text embedding by passing the phrase ``a photo of $\langle$token$\rangle$" to CLIP, where $\langle$token$\rangle$ represents the extracted image token obtained via text inversion.

\noindent\textbf{Datasets and Base Vision-Language Models.} Following previous work, we evaluated our method on a suite of datasets including Fashion-IQ \cite{wu2021fashion}, CIRR \cite{liu2021image} and CIRCO \cite{baldrati2023zero}. Our proposed method is a plug-and-play approach requiring no additional training, leveraging only pre-trained models. For feature extraction, we use three CLIP variants: ViT-B/32, ViT-L/14, and ViT-G/14, and used the same pre-trained weights as used by the baseline methods. For image tokenization, we employ the pre-trained Pic2Word model. 

\subsection{Effect of Using the PDV}
We now explore the impact of the three proposed uses of the PDV: Using the PDV to augment text queries (PDV-T, see Sec. \ref{sec:exp1}), using the PDV to augment image queries (PDV-I, see Sec. \ref{sec:exp2}), and using the PDV in queries that fuse image and text data (PDV-F, see Sec. \ref{sec:exp3}).

\begin{table*}
	\footnotesize
	\centering
	\begin{tabular}{l|l|c|c|c|cccccccc}
		\hline
		\textbf{Fashion-IQ} & & & & & \multicolumn{2}{c}{\textbf{Shirt}} & \multicolumn{2}{c}{\textbf{Dress}} & \multicolumn{2}{c}{\textbf{Toptee}} & \multicolumn{2}{c}{\textbf{Average}} \\ \hline
		Backbone & Method& $\beta$ & $\alpha_{I}$& $\alpha_{T}$ & R@10 & R@50 & R@10 & R@50 & R@10 & R@50 & R@10 & R@50 \\
		\hline
		\multirow{6}{*}{ViT-B/32} %
		& SEARLE & - & - & - & 24.14 & 41.81 & 18.39 & 38.08 & 25.91 & 47.02 & 22.81 & 42.30 \\
		& SEARLE + \textbf{PDV-F} & 0.9 & 1.1 & 0.9 & \hli{24.83} & 41.71 & \hli{20.13} & \hli{41.40} & \hli{25.96} & \hli{47.17}  & \hli{23.64} & \hli{43.43} \\
		& CIReVL \textdagger &- & -& -& 28.36 & 47.84 & 25.29 & 46.36 & 31.21 & 53.85 & 28.29 & 49.35 \\
		& CIReVL + \textbf{PDV-F} & 0.75 & 1.4 & 1.4 & \hlb{32.88} & \hlb{52.80} & \hlb{32.67} & \hlb{54.49} & \hlb{38.91} & \hlb{61.81} & \hlb{34.82} & \hlb{56.37} \\
		& LDRE \textdagger & - & - & - & 27.38 & 46.27 & 19.97 & 41.84 & 27.07 & 48.78 & 24.81 & 45.63 \\
		& SEIZE \textdagger & - & - & - & \underline{29.38} & \underline{47.97} & \underline{25.37} & \underline{46.84} & \underline{32.07} & \underline{54.78} & \underline{28.94} & \underline{49.86} \\
		\hline
		\multirow{8}{*}{ViT-L/14} & Pic2Word & & & & 25.96 & 43.52 & 19.63 & 40.90 & 27.28 & 47.83 & 24.29 & 44.08 \\
		& Pic2Word + \textbf{PV-F} & 0.8 & 1.0 & 1.0 & \hli{28.21} & \hli{44.55} & \hli{20.92} & \hli{42.24} & \hli{29.02} & \hli{48.90}& \hli{26.05} & \hli{45.23}\\
		& SEARLE & - & - & - & 26.84 & 45.19 & 20.08 & 42.19 & 28.40 & 49.62 & 25.11 & 45.67 \\
		& SEARLE +\textbf{PDV-F} & 0.8 & 1.2 & 1.0 & \hli{28.66} & \hli{46.76} & \hli{23.60} & \hli{46.41} & \hli{31.00} & \hli{52.32} & \hli{27.75} & \hli{48.50} \\
		& CIReVL \textdagger & & & & 29.49 & 47.40 & 24.79 & 44.76 & 31.36 & 53.65 & 28.55 & 48.57 \\
		
		& CIReVL + \textbf{PDV-F} & 0.55 & 1 & 1.3 & \hlb{37.78} & \hlb{54.22} & \hlb{33.61} & \hlb{56.07} & \hlb{41.61} & \hlb{62.16} & \hlb{37.67} & \hlb{57.48} \\
		& LinCIR & - & - & - & 29.10 & 46.81 & 20.92 & 42.44 & 28.81 & 50.18 & 26.82 & 46.49 \\
        & SEIZE & -& -& -& \underline{33.04} & \underline{53.22} & \underline{30.93} & \underline{50.76} & \underline{35.57} & \underline{58.64} & \underline{33.18} & \underline{54.21} \\
		\hline
        \multirow{6}{*}{ViT-G/14} & Pic2Word  & - & - & - & 33.17 & 50.39 & 25.43 & 47.65 & 35.24 & 57.62 & 31.28 & 51.89\\
         & SEARLE  & - & - & - & 36.46 & 55.35 & 28.16 & 50.32 & 39.83 & 61.45 & 34.81 & 55.71\\
		  & CIReVL \textdagger & -& -& -& 33.71 & 51.42 & 27.07 & 49.53 & 35.80 & 56.14 & 32.19 & 52.36 \\
		& CIReVL + \textbf{PV-F} & 0.6 & 1.4 & 1.4 & \hli{41.90} & \hli{58.19} & \hlb{40.70} & \hlb{62.82} & \underline{\hli{48.09}}& \hli{67.77}& \underline{\hli{43.56}}& \hli{62.93}\\
        & LinCIR & - & - & - & \textbf{46.76} & \underline{65.11} & 38.08& 60.88& \textbf{50.48}& \underline{71.09}& \textbf{45.11} & \underline{65.69}\\
        & SEIZE & - & - & - & \underline{43.60} & \textbf{65.42}& \underline{39.61} & \underline{61.02} & 45.94& \textbf{71.12}& 43.05& \textbf{65.85}\\
		\hline
	\end{tabular}
	\caption{Average recall for different methods on Fashion-IQ validation dataset. \textdagger~denotes that numbers are taken from the original paper.}
	\label{tab:fashion_iq_results}
\end{table*}


\begin{table*}
	\centering
	\footnotesize
	\setlength{\tabcolsep}{4pt}
	\begin{tabular}{ll|c|c|c|cccc|cccc|ccc}
		\hline
		\multicolumn{2}{c|}{\textbf{Dataset}} & & & &  \multicolumn{4}{c|}{\textbf{CIRCO}} & \multicolumn{7}{c}{\textbf{CIRR}} \\
		\hline
		\multicolumn{2}{c|}{Metric} & & & & \multicolumn{4}{c|}{mAP@k} & \multicolumn{4}{c|}{Recall@k} &\multicolumn{3}{c}{$R_s$@k} \\
		\cline{3-16}
		Arch & Method & $\beta$ & $\alpha_I$ & $\alpha_T$ & k=5 & k=10 & k=25 & k=50 & k=1 & k=5 & k=10 & k=50 & k=1 & k=2 & k=3 \\
		\hline
		\multirow{8}{*}{ViT-B/32} 
		& PALAVRA\cite{cohen2022my} \textdagger & -& -& -& 4.61 & 5.32 & 6.33 & 6.80 & 16.62 & 43.49 & 58.51 & 83.95 & 41.61 & 65.30 & 80.94 \\
		& SEARLE \textdagger & -& -&- & 9.35 & 9.94 & 11.13 & 11.84 & 24.00 & 53.42 & 66.82 
		& 89.78 & 54.89 & 76.60 & 88.19 \\
		& SEARLE + \textbf{PDV-F} & 0.9 & 1.4 & 1.2 & \hli{9.99} & \hli{10.50}  & \hli{11.70} & \hli{12.40} & \hli{24.53} & \hli{53.71} & \hli{67.33} & \hli{89.81} & \hli{56.94} & \hli{78.05} & \hli{88.99} \\
		&CIReVL \textdagger & - & - & -& 14.94 & 15.42 & 17.00 & 17.82 & 23.94 & 52.51 & 66.00 & 86.95 & 60.17 & 80.05 & 90.19 \\
		& CIReVL + \textbf{PDV-F} & 0.75 & 1.4 & 1.2 & \hlb{19.90} & \hlb{20.61} & \hlb{22.64} & \hlb{23.52} & \hlb{33.25} & \hlb{64.15} & \hlb{75.23} & \hlb{92.43} & \hlb{65.81} &\underline{\hli{83.76}} &\underline{\hli{92.10}} \\
		& LDRE & -& -& -& 17.81 & 18.04 & 19.73 & 20.67 & 25.69 & 55.52 & 68.77 & 89.86 & 60.10 & 80.58 & 91.04 \\
		& LDRE + \textbf{PDV-F} & 0.75 & 1.4 & 1.4 & \hli{17.80} & \hli{18.78} & \hli{20.61} & \hli{21.56} & \underline{\hli{29.30}} & \underline{\hli{60.39}} & \underline{\hli{72.51}} & \underline{\hli{91.42}} & \hli{63.06} & \hli{82.36} & \hli{91.54} \\
        & SEIZE & -&- &- & \underline{19.04} & \underline{19.64} & \underline{21.55}& \underline{22.49}& 27.47 & 57.42& 70.17 & - & \underline{65.59} & \textbf{84.48}& \textbf{92.77} \\
 		\hline
		\multirow{10}{*}{ViT-L/14}
		& Pic2Word & -& -& -& 6.81 & 7.49 & 8.51 & 9.07 & 23.69 & 51.32 & 63.66 & 86.21 & 53.61 & 74.34 & 87.28 \\
		& Pic2Word + \textbf{PDV-F} & 0.85 & 1.2 & 1.0 & \hli{7.74} &  \hli{8.67} & \hli{9.77} & \hli{10.37} & \hli{23.90} & \hli{51.95} & \hli{64.63} & \hli{87.04} & \hli{53.16}  & \hli{74.07} & \hli{87.08}\\
		& SEARLE \textdagger & - & - & - & 11.68 & 12.73 & 14.33 & 15.12 & 24.24 & 52.48 & 66.29 & 88.84 & 53.76 & 75.01 & 88.19 \\
		& SEARLE + \textbf{PDV-F} & 0.85 & 1.4 & 1.2 & \hli{12.58} & \hli{13.57} & \hli{15.30} & \hli{16.07} & \hli{25.64} & \hli{53.61} & \hli{66.58} & \hli{88.55} & \hli{55.83} & \hli{76.48} & \hli{88.53} \\
		& CIReVL \textdagger & -& -& -& 18.57 & 19.01 & 20.89 & 21.80 & 24.55 & 52.31 & 64.92 & 86.34 & 59.54 & 79.88 & 89.69 \\
		& CIReVL + \textbf{PDV-F} & 0.75 & 1.4 & 1.2 & \hlb{25.67} & \hlb{26.61} & \underline{\hli{28.81}} & \hlb{29.95} & \hlb{36.24} & \hlb{66.17} & \hlb{76.96} & \hlb{92.29} & \hlb{68.07} & \hlb{85.35} & \hlb{93.47} \\
		& LDRE & -& -& -& 22.32 & 23.75 & 25.97 & 27.03 & 26.68 &55.45  & 67.49 & 88.65 & 60.39 & 80.53 & 90.15 \\
		& LDRE + \textbf{PDV-F} & 0.75 & 1.4 & 1.4 & \hli{25.23} & \hli{26.52} & \hlb{28.94} & \hlb{29.95} & \underline{\hli{30.16}} & \underline{\hli{59.98}} & \underline{\hli{71.90}} & \underline{\hli{90.87}} & \hli{63.66} & \hli{82.87} & \hli{91.57} \\

        & LinCIR & - & - & - &12.59 &13.58 &15.00 &15.85 &25.04 &53.25 &66.68 & - &57.11 &77.37 &88.89\\
        & SEIZE & -& -& -& 24.98 & 25.82 &28.24 &\underline{29.35}& 28.65 &57.16& 69.23& - &\underline{66.22} &\underline{84.05} &\underline{92.34} \\
        

        
		\hline
		\multirow{7}{*}{ViT-G/14} & CIReVL \textdagger & -& -& -& 26.77 & 27.59 & 29.96 & 31.03 & 34.65 & 64.29 & 75.06 & 91.66 & 67.95 & 84.87 & 93.21 \\

		& CIReVL + \textbf{PDV-F} & 0.75 & 1.4 & 1.2 & \hli{30.02} & \hli{31.46} & \hli{34.01} & \hli{35.08} & \hli{38.15} &\hli{67.93} & \hli{77.90} & \hli{92.77} & \hli{69.37} & \hli{85.37} & \hli{93.45}  \\
		
		& LDRE & -& -& -& \underline{33.30} & \underline{34.32} & \underline{37.17} & \underline{38.27} & 37.40 & 66.96 & 78.17 & 93.66 & 68.84 & 85.64 & 93.90 \\
		& LDRE + \textbf{PDV-F} & 0.75 & 1.4 & 1.4 & \hlb{34.88} & \hlb{36.41} & \hlb{39.12} & \hlb{40.23} & \hlb{42.51} & \hlb{72.22} & \hlb{81.71} & \hlb{94.94} & \underline{\hli{72.39}} & \underline{\hli{88.34}} & \underline{\hli{94.80}} \\
        & SEARLE & - & - & - & 13.20 &13.85 &15.32 &16.04 & 34.80 & 64.07 & 75.11 &-&68.72 &84.70 &93.23 \\
        & LinCIR & - & - & - & 19.71 &21.01 &23.13 &24.18 &35.25 &64.72 &76.05 & - &63.35 &82.22 &91.98 \\
        & SEIZE & -& -& -& 32.46 & 33.77 &36.46 &37.55 &\underline{38.87} & \underline{69.42} & \underline{79.42} & -&\textbf{74.15} & \textbf{89.23} & \textbf{95.71} \\
		\hline
	\end{tabular}
	\caption{Performance comparison on CIRCO and CIRR test datasets. As in previous works, for CIRCO, mAP@k is reported, while for CIRR both Recall@k and $R_s$@k metrics are used. \textdagger~denotes that numbers are taken from the original paper.}
	\label{tab:circo_cirr_results}
\end{table*}

\noindent{\textbf{Analysing the PDV for Text (PDV-T)}}
\label{sec:exp1}
To investigate how scaling the prompt vector, $\Delta_{PDV}$, affects retrieval performance with composed text embeddings, we conducted experiments using two zero-shot approaches (CIReVL and Pic2Word) with different backbone networks across three datasets. We evaluated the performance by varying the scaling parameter, $\alpha$ (Eq. \ref{eqn:text_embedding}), from -0.5 to 3 by an interval of 0.1.

The results are presented in Figure \ref{fig:residual_text_sub}. To account for scale variations across different experiments, we report relative recall values, where a baseline of zero is established at $\alpha=1$. As shown in Figure \ref{fig:residual_text_sub}, varying $\alpha$ leads to significant changes in relative recall performance\footnote{See supplementary material for Recall@10 and Recall@50 figures}. Our analysis reveals method-specific patterns across datasets. With CIReVL, increasing $\alpha$ improves relative recall on both FashionIQ and CIRCO datasets. In contrast, Pic2Word shows no significant improvement on FashionIQ and CIRR when varying $\alpha$, while CIRCO's performance improves when $\alpha$ is reduced to 0.8-1.0. This divergent behavior is fundamentally linked to each method's ability to generate an accurate $\Delta_{PDV}$. As demonstrated in Tables \ref{tab:fashion_iq_results} and \ref{tab:circo_cirr_results}, CIReVL consistently outperforms Pic2Word across various benchmarks, indicating its superior ability to generate a more accuraute composed query, and thus a more accurate $\Delta_{PDV}$. Consequently, increasing $\alpha$ yields greater benefits for CIReVL compared to Pic2Word.

We visualize the top-5 retrieval results using CIReVL with a ViT-B-32 backbone across three datasets (one reference image from each) under varying $\alpha$ values, as shown in Figure \ref{fig:residual_qual}\red{a}. As $\alpha$ increases, the retrieved results show stronger alignment with the prompt. Conversely, when $\alpha$ exceeds 1, the results include semantically related but unseen variations, while $\alpha$ values below 0.5 yields results opposite to the prompt's intent. For instance, ``brighter blue and sleeveless" retrieves ``dark blue with sleeves," ``plain background" yields ``natural/dark background," and ``young boy" returns ``adult" images.





\noindent{\textbf{Analysing the PDV for Image (PDV-I)}}
\label{sec:exp2}
To evaluate whether $\Delta_{PDV}$ enhances the retrieval performance of image embeddings, we conducted experiments following the protocol described in Section~\ref{sec:exp1}. We modified image embeddings by adding $\Delta_{PDV}$ scaled with $\alpha$ values ranging from -0.5 to 2.0, where $\alpha=0$ represents the original image-only embeddings. As shown in Figure \ref{fig:residual_image_sub}, Recall@K exhibits a positive correlation with $\alpha$ for values below 1. This upward trend continues until $\alpha=2.0$ for CIReVL, while Pic2Word's performance peaks when $\alpha$ reaches 1.4.  The performance of PDV-I was evaluated on the CIRR and CIRCO datasets by comparing it with other visual embedding-based methods, as detailed in Table \ref{tab:circo_cirr_results_pdv-I}. The results reveal that PDV-I achieved marginal improvements over existing approaches.

Following the methodology in Section~\ref{sec:exp1}, we conduct similar visualizations, with results shown in Figure \ref{fig:residual_qual}\red{b}. As with PDV-T, increasing $\alpha$ leads to stronger alignment between retrieved results and the prompt. When $\alpha$ exceeds 0.5, the results exhibit semantic relationships to the query, while $\alpha$ values below 0.5 yield results opposing the prompt's intent.
Notably, PDV-I's top retrievals demonstrate higher visual similarity to reference images compared to PDV-F, as evidenced by the preserved design elements in the clothing item (left) and laptop (middle). This characteristic is particularly valuable for applications include fashion search \cite{wu2021fashion} and logo retrieval \cite{tursun2019component}, where visual similarity plays a crucial role.



\begin{figure*}[!tbh]
	\centering
	\includegraphics[width=0.825\linewidth]{images/qualitative/PV_qual_all_mini.pdf}
	\caption{Visualisation of the impact of $\alpha$/$\beta$ scaling on top-5 retrieval results. CIReVL with ViT-B-32 Clip model is the baseline method used. Representative examples with prompts from three datasets: FashionIQ (left), CIRR (middle), and CIRCO (right) are shown at the top. \textbf{\textcolor{boxgreen}{Green}} and \textbf{\textcolor{boxblue}{blue}} bounding boxes indicate true positives and near-true positives, respectively.}
	\label{fig:residual_qual}
	
\end{figure*}

\noindent{\textbf{Analysing PDV Fusion (PDV-F)}}
\label{sec:exp3}
Finally, we evaluate the effectiveness of fusing image and text-composed embeddings by varying the fusion parameter, $\beta$, from 0 to 1 while maintaining $\alpha=1$
for both PDV-I and PDV-F. At $\beta=0$, the model relies solely on composed image embeddings, while at $\beta=1$, it uses only composed text embeddings. As shown in Figure \ref{fig:residual_fusion_sub}, the fusion of both embeddings consistently outperforms using either embedding type alone. Optimal retrieval performance is typically achieved when $\beta$ is between 0.4 and 0.8.

We similarly visualize the top-5 retrieved results across different $\beta$ values. As shown in Figure \ref{fig:residual_qual}\red{c}, when $\beta$ is small, the retrieved results maintain high visual similarity to the reference image. Conversely, as $\beta$ exceeds 0.5, the results demonstrate stronger semantic alignment with the prompt.



\subsection{ZS-CIR Benchmark Comparison}






\begin{table*}
	\centering
	\footnotesize
	\setlength{\tabcolsep}{4pt}
	\begin{tabular}{l|l|c|cccc|cccc|ccc}
		\hline
		\multicolumn{2}{c|}{\textbf{Dataset}} & & \multicolumn{4}{c|}{\textbf{CIRCO}} & \multicolumn{7}{c}{\textbf{CIRR}} \\
		\hline
		& Metric & & \multicolumn{4}{c|}{mAP@k} & \multicolumn{4}{c|}{Recall@k} & \multicolumn{3}{c}{$R_s$@k} \\
		\cline{2-14}
		Arch & Method & $\alpha_I$ & k=5 & k=10 & k=25 & k=50 & k=1 & k=5 & k=10 & k=50 & k=1 & k=2 & k=3 \\
		\hline
		\multirow{6}{*}{ViT-B/32} 
		& Image-only \textdagger & - & 1.34 & 1.60 & 2.12 & 2.41 & 6.89 & 22.99 & 33.68 & 59.23 & 21.04 & 41.04 & 60.31 \\
		& Text-only \textdagger & - & 2.56 & 2.67 & 2.98 & 3.18 & 21.81 & 45.22 & 57.42 & 81.01 & 62.24 & 81.13 & 90.70 \\
		& Image + Text \textdagger & - & 2.65 & 3.25 & 4.14 & 4.54 & 11.71 & 35.06 & 48.94 & 77.49 & 32.77 & 56.89 & 74.96 \\
		& SEARLE + \textbf{PDV-I} & 1.5 & 4.77 & 5.23  & 6.31 & 6.82 & 16.65 & 42.53 & 55.16 & 81.42 & 44.68 & 67.78 & 82.94\\
		& CIReVL + \textbf{PDV-I} & 2.0 & \textbf{10.29 }& \textbf{10.80} & \textbf{12.23} & \textbf{12.93} & \textbf{27.18} & \textbf{56.53} & \textbf{67.76} & \textbf{87.64} & \textbf{59.81} & \textbf{79.59} & \textbf{90.15}\\
		& LDRE + \textbf{PDV-I} & 2.0 & 8.00 & 8.88 & 10.06 & 10.72 & 23.37 & 51.21 & 63.69 & 85.57 & 55.57 & 76.63 & 88.15\\
		\hline
	\end{tabular}
	\caption{PDV-I performance on CIRCO and CIRR test datasets. Note that the image-only approach utilizes the visual embedding of the reference image, whereas the text-only approach employs the text embedding of the prompt.}
	\label{tab:circo_cirr_results_pdv-I}
\end{table*}

We evaluated PDV-F alongside four baseline approaches (CIReVL, LDRE, Pic2Word, and SEARLE) across three benchmarks. Notably, CIReVL was tested with three different backbones on three datasets, as its models and intermediate results are publicly available. However, for the remaining methods, we conducted partial evaluations due to limited open-source availability or restricted support.

The numerical results are presented in Tables \ref{tab:fashion_iq_results} and \ref{tab:circo_cirr_results}.
On the FashionIQ benchmark, PDV-F yields substantial improvements for all baseline approaches, with CIReVL showing particularly strong gains that scale with backbone size. Similarly, all methods demonstrate significant performance improvements on CIRCO and CIRR datasets. Notably, CIReVL achieves larger improvements compared to other methods, with the most substantial gains observed when using small and medium backbone architectures. Our PDV-F implementation within the CIReVL framework consistently outperformed other state-of-the-art methods, including LinCIR and SEIZE, across most evaluation metrics. Similar to SEIZE, PDV-F offers the advantage of being entirely training-free; however, unlike SEIZE, it does not significantly increase feature extraction computational costs. While LinCIR demonstrates exceptional inference speed, it lacks the training-free nature of our approach, requiring dedicated model training before deployment.  





%\begin{figure*}[!t]
\centering
\begin{minipage}{0.32\textwidth}
  \centering
\includegraphics[width=0.8\textwidth]{fig/5_mnist_images/image_sample/000000000.png}
\subcaption{Real (private) images}\label{fig:mnist_real}
\end{minipage}%
\begin{minipage}{0.32\textwidth}
  \centering
\includegraphics[width=0.8\textwidth]{fig/5_mnist_images/eps10/000000000.png}
\subcaption{Simulator-generated images}\label{fig:mnist_simulator}
\end{minipage}%
\begin{minipage}{0.32\textwidth}
  \centering
\includegraphics[width=0.8\textwidth]{fig/5_mnist_images/eps10/000000004.png}
\subcaption{\simpe{}-generated images ($\epsilon=10$)}\label{fig:mnist_simpe}
\end{minipage}
\vspace{-0.2cm}
\caption{The real and generated images on \mnist{} under the ``\classunavail{}'' setting. Each row corresponds to one class.
The simulator generates images that are very different from the real ones and are from the incorrect classes. Starting from these bad images, \simpe{} can effectively guide the generation of the simulator towards high-quality images with correct classes.} \label{fig:mnist}
\end{figure*}
%\subsection{Adult}


    In this study, we introduce a novel method based on multi-cost shortest path algorithms and validate its effectiveness on the Adult dataset. The Adult dataset is Here are the specific steps and methodologies we employed in our experiment.


We started by standardizing the "Adult" dataset to ensure uniformity across various features, allowing for more accurate comparisons. Using a random forest model with 100 trees, then we sorted this standardized dataset and also maintained a non-standardized version for diverse analytical needs. Based on a set of predefined actionable criteria, we established an actionable dataset. During this process, some criteria were set as immutable to reflect real-world conditions. For instance, in the Adult dataset, gender was designated as immutable. From this actionable dataset, we randomly selected 256 instances that met predefined actionable criteria for further analysis. These instances were assessed using a second random forest model to identify the data points with the lowest success probability, indicating areas with the highest challenges.

For data analysis, we created a KNN-graph (k=4) from the standardized dataset and applied our multi-cost shortest path algorithm to explore optimal changes in three key dimensions: education, hours per week, and age. We used the non-standardized data to calculate the shortest paths based on these criteria, aiming to find the most optimized change strategies within these three dimensions. To ensure a realistic application, we designed different cost functions for each criterion. Taking "age" for example, when designing the cost function, it is notable that age can not decrease during the whole recourse process. %As mentioned in Algorithm~\ref{algo:backtracking}, we use Pareto optimal 2 to prune the paths, retaining only a representative set of costs.
    
Figure~\ref{fig:enter-label} visualized the results of the experiment that getting multiple flipped counterfactuals. These four paths, originating from the same starting point where the success probability is the worst, illustrate the diverse impacts caused by Pareto optimal solutions, and all four paths ultimately reach an endpoint where the success probability exceeds 0.75. The introduction of Kernel Density Estimation (KDE) into the algorithm demonstrates a better trade-off with other costs, enhancing the algorithm's ability to navigate through the solution space effectively. To better visualize and understand the patterns in the data, we applied Principal Component Analysis (PCA) for dimension reduction.

%In conclusion, the application of our algorithm on the Adult dataset has demonstrated significant capabilities in addressing complex decision-making scenarios where multiple factors such as age and hours worked per week impact income predictions. Our approach effectively identifies optimal paths for improving individual income predictions by leveraging a multi-criteria framework. This enables a tailored intervention strategy that considers multiple aspects of an individual’s profile.
%\begin{figure}
    \centering
    \includegraphics[width=0.5\textwidth]{experiments/scale.pdf}
    \caption{show scalability}
    \label{fig:scalability}
\end{figure}


%% \begin{figure}[ht]
%      \centering
%      \begin{subfigure}{0.9\linewidth}
%      \centering
%             \begin{tikzpicture}
%             \tikzstyle{vertex}=[circle,fill=none,draw=black,minimum size=17pt,inner sep=0pt]
% \node[vertex] (S) at (0,0) {$S$};
% \node[vertex] (A) at (2,0) {$A$};
% \node[vertex] (D) at (1,1) {$D$};
% \path (S) edge (D);
% \path (D) edge (A);
% \path[red] (S) edge (A);
%             \end{tikzpicture}
%         \caption{Causal graph for $\model \in \modelsunconfedge$ illustrating all possible functional dependencies.}
%         \label{fig:no-cf-edge}
%         \end{subfigure}    \hfill
% %              \begin{subfigure}{0.45\linewidth}
% %              \centering
% %             \begin{tikzpicture}
% %             \tikzstyle{vertex}=[circle,fill=none,draw=black,minimum size=17pt,inner sep=0pt]
% % \node[vertex] (S) at (0,0) {$S$};
% % \node[vertex] (A) at (2,0) {$A$};
% % \node[vertex] (D) at (1,1) {$D$};
% % \path (S) edge (D);
% % \path (D) edge (A);
% % %\path[red] (S) edge (A);
% %             \end{tikzpicture}
% %         \caption{Causal graph for $\model \in \nullgraphunconf$ illustrating all possible functional dependencies.}
% %         \label{fig:no-cf-no-edge}
% %         \end{subfigure}
% \end{figure}

% \begin{figure}[h]
%      \centering
%             \begin{tikzpicture}
%             \tikzstyle{vertex}=[circle,fill=none,draw=black,minimum size=17pt,inner sep=0pt]
% \node[vertex] (S) at (0,0) {$S$};
% \node[vertex] (A) at (2,0) {$A$};
% \node[vertex] (D) at (1,1) {$D$};
% \path (S) edge (D);
% \path (D) edge (A);
% \path[bidirected] (D) edge[bend left=60] (A);
% \path[red] (S) edge (A);
% % \draw[->, line width=0.3mm]  (S)--(D);
% % \draw[->, line width=0.3mm]  (D)--(A);
% % \draw[->, line width=0.3mm]  (S)--(A);
% % \draw[<->, line width=0.3mm]  (D)--(A);
%             \end{tikzpicture}
%         \caption{Causal graph for $\model \in \modelsunconfedge$ illustrating all possible functional dependencies.}
%         \label{fig:cf-no-edge}
% \end{figure}

% \begin{figure}[h]
%      \centering
%             \begin{tikzpicture}
%             \tikzstyle{vertex}=[circle,fill=none,draw=black,minimum size=17pt,inner sep=0pt]
% \node[vertex] (S) at (0,0) {$S$};
% \node[vertex] (A) at (3,-0.5) {$A$};
% \node[vertex] (D) at (1,1) {$D$};
% \node[vertex] (S') at (1,-0.5) {$S'$};
% \path (S) edge (D);
% \path (D) edge (A);
% \path[bidirected] (D) edge[bend left=60] (A);
% \path[red] (S') edge (A);
% %\path (S) edge (S'); 
%  \path (S) edge node[near start, below] {=} (S');
% % \draw[->, line width=0.3mm]  (S)--(D);
% % \draw[->, line width=0.3mm]  (D)--(A);
% % \draw[->, line width=0.3mm]  (S)--(A);
% % \draw[<->, line width=0.3mm]  (D)--(A);
%             \end{tikzpicture}
%         \caption{Causal graph for $\model \in \modelsedge$ illustrating all possible functional dependencies.} 
%         \label{fig:cf-edge}
% \end{figure}


%              \begin{subfigure}{0.45\linewidth}
%              \centering
%             \begin{tikzpicture}
%             \tikzstyle{vertex}=[circle,fill=none,draw=black,minimum size=17pt,inner sep=0pt]
% \node[vertex] (S) at (0,0) {$S$};
% \node[vertex] (A) at (2,0) {$A$};
% \node[vertex] (D) at (1,1) {$D$};
% \path (S) edge (D);
% \path (D) edge (A);
% %\path[red] (S) edge (A);
%             \end{tikzpicture}
%         \caption{Causal graph for $\model \in \nullgraphunconf$ illustrating all possible functional dependencies.}
%         \label{fig:no-cf-no-edge}
%         \end{subfigure}
%\end{figure}

\begin{figure*}[t]
     \centering
     \begin{subfigure}{0.32\linewidth}
     \centering
            \begin{tikzpicture}
            \tikzstyle{vertex}=[circle,fill=none,draw=black,minimum size=17pt,inner sep=0pt]
\node[vertex] (S) at (0,0) {$S$};
\node[vertex] (A) at (2,0) {$A$};
\node[vertex] (D) at (1,1) {$D$};
\path (S) edge (D);
\path (D) edge (A);
\path (S) edge (A);
            \end{tikzpicture}
        \caption{$\model \in \modelsunconfedge$}
        \label{fig:no-cf-edge}
\end{subfigure}
     \begin{subfigure}{0.32\linewidth}
     \centering
            \begin{tikzpicture}
            \tikzstyle{vertex}=[circle,fill=none,draw=black,minimum size=17pt,inner sep=0pt]
\node[vertex] (S) at (0,0) {$S$};
\node[vertex] (A) at (2,0) {$A$};
\node[vertex] (D) at (1,1) {$D$};
%\node[vertex] (S') at (1,-0.5) {$S'$};
\path (S) edge (D);
\path (D) edge (A);
\path[bidirected] (D) edge[bend left=60] (A);
\path (S) edge (A);
%\path (S) edge (S'); 
% \path (S) edge node[near start, below] {=} (S');
% \draw[->, line width=0.3mm]  (S)--(D);
% \draw[->, line width=0.3mm]  (D)--(A);
% \draw[->, line width=0.3mm]  (S)--(A);
% \draw[<->, line width=0.3mm]  (D)--(A);
            \end{tikzpicture}
        \caption{$\model \in \modelsedgerelax$} 
        \label{fig:cf-edge}
        \end{subfigure}
         \begin{subfigure}{0.32\linewidth}
     \centering
            \begin{tikzpicture}
            \tikzstyle{vertex}=[circle,fill=none,draw=black,minimum size=17pt,inner sep=0pt]
\node[vertex] (S) at (0,0) {$S$};
\node[vertex] (A) at (2,0) {$A$};
\node[vertex] (D) at (1,1) {$D$};
%\node[vertex] (S') at (1,-0.5) {$S'$};
\path (S) edge (D);
\path (D) edge (A);
\path[bidirected] (D) edge[bend left=60] (A);
%\path (S) edge (A);
%\path (S) edge (S'); 
% \path (S) edge node[near start, below] {=} (S');
% \draw[->, line width=0.3mm]  (S)--(D);
% \draw[->, line width=0.3mm]  (D)--(A);
% \draw[->, line width=0.3mm]  (S)--(A);
% \draw[<->, line width=0.3mm]  (D)--(A);
            \end{tikzpicture}
        \caption{$\model \in \nullgraph$ and $\model \in \modeliv$} 
        \label{fig:cf-edge-iv}
        \end{subfigure}
        \caption{Causal graphs, $\cg{\model}$, assumed in various model classes.}
\end{figure*}

% \begin{figure}
%      \centering
%             \begin{tikzpicture}
%             \tikzstyle{vertex}=[circle,fill=none,draw=black,minimum size=17pt,inner sep=0pt]
% \node[vertex] (Z) at (0,0) {$Z$};
% \node[vertex] (Y) at (3,0) {$Y$};
% \node[vertex] (X) at (1.5,0) {$X$};
% %\node[vertex] (S') at (1,-0.5) {$S'$};
% \path (Z) edge (X);
% \path (X) edge (Y);
% \path[bidirected] (X) edge[bend left=60] (Y);
% %\path[red] (S') edge (A);
% %\path (S) edge (S'); 
%  %\path (S) edge node[near start, below] {=} (S');
% % \draw[->, line width=0.3mm]  (S)--(D);
% % \draw[->, line width=0.3mm]  (D)--(A);
% % \draw[->, line width=0.3mm]  (S)--(A);
% % \draw[<->, line width=0.3mm]  (D)--(A);
%             \end{tikzpicture}
%         \caption{Causal graph of $M \in \modeliv$} 
%         \label{fig:iv}
%         \end{figure}

\section{CONCLUSION}
In conclusion, our algorithm proposes a novel idea that can be generalized to non-differentiable or discrete cost functions in the field of recourse. The algorithm is applied to multi-cost scenarios where users can combine their background knowledge in professional fields, set the most suitable cost function, and then finally get a more practical counterfactual. We show that our algorithm can find all the Pareto recourse plans optimally and can be scaled to a large graph with the utilization of $\epsilon$-net. We conduct experiments on two data sets separately. For the MNIST data set, we compared the performance of Pareto optimal paths under different sampling sizes and showed the result that transfers from one handwriting number into another and offers the whole process. For Adults, we show the diversity in each cost function, and those diverse cost functions finally lead to multi-Pareto optimal solutions.


\emph{Acknowledgement:} This work is supported by NSTC:113-2221-E-008-086.
%\section{Appendices}

%If your work needs an appendix, add it before the
%``\verb|\end{document}|'' command at the conclusion of your source
%document.

%Start the appendix with the ``\verb|appendix|'' command:
%\begin{verbatim}
%  \appendix
%\end{verbatim}
%and note that in the appendix, sections are lettered, not
%numbered. This document has two appendices, demonstrating the section
%and subsection identification method.



%%
%% The next two lines define the bibliography style to be used, and
%% the bibliography file.
\bibliographystyle{splncs04}
\bibliography{ref.bib}



\end{document}
\endinput
%%
%% End of file `sample-manuscript.tex'.
