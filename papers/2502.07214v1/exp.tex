To demonstrate our algorithm, we conducted experiments using two datasets. MNIST dataset~\cite{lecun2010mnist} and the Adult dataset~\cite{misc_adult_2}. The experiments have two parts. The first part is to show the ability of our approaches to find all the Pareto optimal recourse paths with multi-criteria in different scenarios. The second part shows the quality of the Pareto recourse paths under different numbers of random samples. This supports the argument that one can use random samples as the $\epsilon$-net with a certain number of samples.

\subsection{Scenarios}

\subsubsection{MNIST:} Our first scenario is to find a recourse path from one digit to another digit with the higher number where the number of each picture can never go lower(e.g. $3 \rightarrow 5 \rightarrow 7$ is allowed, but $3 \rightarrow 6 \rightarrow 5 \rightarrow 7 $ is not). We commence by extracting a subset from the original MNIST dataset~\cite{lecun2010mnist}. we define $cost_1$ as the absolute difference between the images. An actionable edge exists if and only if the number of the first image is lower than the second image, aiming for a recourse path where the number incrementally increases. $cost_2$ is the typical L2 distance between the images. The overall cost of $cost_1$ is the maximum value of $cost_1$ among all edges in the path, while the overall cost of $cost_2$ is the sum of $cost_2$ values among all edges.


\subsubsection{Adult:} The Adult dataset is widely employed in predictive modeling to determine whether an individual's income exceeds \$50,000 per year based on 14 features, such as age, education, occupation, hours worked per week, and others. The following outlines the specific steps and methodologies we employed in our experiment, our goal is to find the recourse path with an income of more than \$50,000.

We established an actionable dataset based on a set of predefined actionable criteria. During this process, some criteria were designated as immutable to reflect real-world conditions. For instance, in the Adult dataset, gender was considered immutable. From this actionable dataset, we randomly selected 256 instances that met the predefined actionable criteria for further analysis. A point with the lowest predicted probability by a simple one-layered MLP model was selected as the starting point.

To create the actionability graph, we utilized a KNN-graph (k=4) from the standardized dataset and applied our multi-cost shortest path algorithm to explore optimal changes in three key dimensions: age, education, and hours per week. We employed the non-standardized data to calculate the shortest paths based on these criteria, aiming to identify the most optimized change strategies within these three dimensions. To ensure a realistic application, we designed different cost functions for each criterion. The first cost is the percentage of the negative value of the Kernel Density Estimation (KDE) score, which is introduced in FACE~\cite{poyiadzi2020face}. This cost is also known as the negative log-likelihood (NLL), where a lower value signifies a higher density. The remaining costs are the L1 distances between the three key dimensions. Taking "age" as an example, when designing the cost function, it is notable that age cannot decrease during the entire recourse process.
%As mentioned in Algorithm~\ref{algo: backtracking}, we use Pareto optimal 2 to prune the paths, retaining only a representative set of costs.


%First, we applied our algorithm to the MNIST dataset\cite{lecun2010mnist}, comparing the different paths generated based on varying requirements. Second, we applied our algorithm to the Adult dataset\cite{misc_adult_2} to achieve the goal of flipping the income to earning over \$50000 per year.

\subsection{Multi-criteria paths}

\begin{figure*}[ht!]
     \centering
     \begin{subfigure}[b]{0.9\textwidth}
         \centering
         \includegraphics[width=0.6\linewidth]{experiments/0.pdf}
         \caption{$Path$ with $cost_1$ \textbf{2}
         with $cost_2$ \textbf{31}.
         }
         \label{fig:y equals x}
     \end{subfigure}
     \hfill
     \begin{subfigure}[b]{0.9\textwidth}
         \centering
         \includegraphics[width=0.35\textwidth]{experiments/1.pdf}
         \caption{$Path$ with $cost_1$ \textbf{5}
         with $cost_2$ \textbf{15}.
         }
         \label{fig:three sin x}
    \end{subfigure}
     \hfill
     \begin{subfigure}[b]{0.9\textwidth}
         \centering
         \includegraphics[width=0.25\textwidth]{experiments/2.pdf}
         \caption{$Path$ with $cost_1$ \textbf{6}
         with $cost_2$ \textbf{8.7}.
         }
         \label{fig:five over x}
     \end{subfigure}
        \caption{The paths in MNIST under different criteria of $cost_1$ and $cost_2$.}
        \label{fig:three graphs}
        \vspace{0.3cm}
\end{figure*}

Figure~\ref{fig:three graphs} present all the Pareto optimal paths we found in MNIST with two cost functions $cost_1,cost_2$. The starting point is an arbitrary image labeled as 2 and we only sampled 256 images to construct the actionability graph. This is to simplify the graph for clear visualization. $cost_1$ represents the maximum number of digit transformations in the path. For example, in Path $(2,31)$, the maximum number of changes between images is at most 2. $cost_2$ is the sum of the L2-norm and the values have been standardized with the mean and variance among all the samples in the MNIST dataset. In this figure, one can see that by relaxing $cost_1$, the algorithm did find another path with lower $cost_2$, which demonstrates that Algorithm~\ref{algo:pareto-shortest} finds paths with different criteria.

\begin{comment}
\begin{table}[ht]
\centering
\begin{tabular}{|c|l|c|c|}
    \hline
    Path& Cost$_1$& Cost$_2$\\
    \hline
     2 $\rightarrow$ 4 $\rightarrow$ 4 $\rightarrow$ 6$\rightarrow$ 8 & 2 & 31 \\
     2 $\rightarrow$3$\rightarrow$ 8 & 5 & 15 \\
     2 $\rightarrow$ 8 & 6 & 8.7 \\
    \hline
\end{tabular}
\caption{Paths from 2 to 8}
\label{tab:my_table}
\end{table}
\end{comment}

\begin{figure}[ht]
    \centering
    \includegraphics[width=0.9\textwidth]{experiments/adult-3-cost-with-kde-paths.pdf}
    \caption{Multiple different paths from a starting data point to an end data point. It shows that the algorithm finds multi-criteria paths and can have a diverse path. The four paths show the trade-off between different criteria, which are the KDE, age, education-num, and hours-per-week.}
    \vspace{0.4cm}
    \label{fig:enter-label}
\end{figure}

Figure~\ref{fig:enter-label} visualized the results of the experiment in the adult dataset. The model is visualized with the probability output. Regions with more blue color are more likely to reject the object and regions with more red color are more likely to accept the object. However, since the original plot is in high-dimension we applied Principal Component Analysis (PCA) to reduce the dimension into two dimensions. The colors are just for indication and cannot be very accurate. The blue points represent data points with a success probability of less than 0.5, while the red points represent data points with a success probability greater than 0.75, which are also the endpoints. The numbers in the parentheses of the path represent age, education-num, hours-per-week, and KDE. The color bar and background colors indicate the estimated success probability, with redder hues signifying higher probabilities and bluer hues indicating lower probabilities.

These four paths, originating from the same starting point where the success probability is the worst, illustrate the diverse impacts caused by Pareto optimal solutions, and all four paths ultimately reach an endpoint where the success probability exceeds 0.75. The introduction of Kernel Density Estimation (KDE) into the algorithm demonstrates a better trade-off with other costs, enhancing the algorithm's ability to navigate through the solution space effectively. 



\begin{figure}[ht!]
    \centering
    \includegraphics[width=0.8\textwidth]{scale.pdf}
    \caption{The trend among different sampling sizes shows that the quality of the Pareto paths becomes better and more stable when the sampling size gets larger. However, the improvement is not that significant when the size is large enough ($\geq 256$).}
    \label{fig:scalability}
    \vspace{0.2cm}
\end{figure}

\subsection{Scalability}
This experiment is designed to see the quality of the recourse paths when the graph is too large and we have to use $\epsilon$-net to run our algorithm. We use the MNIST dataset and randomly sample points with sizes $128,256,512,1024$ and run 32 trials. We then compare the Pareto optimal paths under different sampling sizes. Figure~\ref{fig:scalability} is the reported result. The x-axis is the criterion of $cost_1$ and the y-axis is the box-plot of $cost_2$ value among 32 trials. Generally, one can see that when the number of samples increased, the quality of the Pareto path was improved too. This is reasonable since the algorithm is more likely to find a better path for lower $cost_2$. However, the improvement is not that clear when the number of samples is large. For example, in the sampling size of 512 (blue box), the average value of $cost_2$ when $cost_1=7$ and $cost_1=9$ is smaller than the ones with a sampling size of 1024 (purple box). We believe that is because the influence of the sampling size on the quality is not that significant and the reported $cost_2$ value is fickle to the random sampling process. Responding to the discussion in Section~\ref{sec:scalability}, this supports the argument of the $\epsilon$-net which shows a random sampling can bound the quality when the size is large enough.