We consider a fixed predictive model $h: \mathcal{X} \rightarrow Y$, with $\mathcal{X}= \mathcal{X}^1 \times \mathcal{X}^2 \cdots \mathcal{X}^d$. Each attribute $\mathcal{X}_i$ can be either a continuous or discrete value (e.g., categories). The output can be either binary class $\mathcal{Y}=\{0,1\}$ or stochastic $\hat{\mathcal{Y}}=[0,1]$, which is the probability of the user classified into $1$. The 0 and 1 represent the negative outcome and positive outcome, respectively. A corresponding application example can be a loan approval system where 0 means the ``loan denied'' and 1 means ``loan approved.''

Given a set of $n$ accessible data points $\{x_1,x_2, \cdots x_n \}, x_i \in \mathcal{X}$ and a set of $k$ cost functions $C=\{c_1,c_2, \cdots c_k\}$, one can construct a directed graph which we called actionability graph, where each node is an accessible data point (such as training samples). There is a directed edge from $u$ to $v$ if a feasible action exists from $u$ to $v$. Each cost function $c_i: \mathcal{X} \times \mathcal{X} \rightarrow \mathcal{R}^+$ is a metric that represents the ``distance'' of two data points. Notice that, the cost functions can be non-differentiable, an attribute of the education level is discrete and non-differentiable. Now, consider a specific point $s$ where $h(s)=0$, a feasible path $P$ in the actionability graph is the one that every edge in this path is a feasible action. By abusing the notation of cost function $c$, we denote the total cost of path $P$ is $c(P)$ for the function $c$, which can be other aggregation functions as long as they are metrics, such as summation or any maximum function. 

We define path $P$ as dominated by path $Q$ if and only if for any cost function $c_i$, $c_i(P) \geq c_i(Q)$. Our objective is to find a set of feasible paths in the graph such that the end of each path is a recourse point $t$, where $h(t)=1$ and each path is Pareto optimal concerning all the cost functions. That is, taking summation as the example, path $P$ is Pareto optimal if and only if there does not exist a path $Q$ such that for any cost function $c$, $\sum_{(x,y) \in P} c(x,y) > \sum_{(x,y) \in Q} c(x,y)$.