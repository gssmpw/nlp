% Artificial intelligence (AI) plays a crucial role in processing vast amounts of data and generating solutions.  Recourse provides a mechanism for individuals to understand how decisions are made and offers a way to contest or change outcomes that may be unfavorable.
\begin{figure}[ht]
\centering
\includegraphics[width=90mm]{toy_eg.pdf}
\caption{In this example, our approach provides three different paths which are all pareto optimal for the recourse plans. Each path contains different criteria and the user can choose one that fits him the most.}
\label{fig:toy eg}
\vspace{0.2cm}
\end{figure}



%\todo{AI is used in decision-making systems or recommendation systems. Intuition of using recourse.}

Deep learning has been the most popular technique in Artificial Intelligence (AI)~\cite{pouyanfar2018survey},
~\cite{zhao2019object}, 
~\cite{wang2021deep}, 
~\cite{otter2020survey}, 
~\cite{danilevsky2020survey}, 
~\cite{turay2022toward}, ~\cite{tang2022perception}, and people increasingly rely on AI for generating decisions ~\cite{zhang2021artificial},~\cite{peres2020industrial}. However, the opacity of deep learning models complicates tracing and understanding decision rationales ~\cite{daston2021objectivity},~\cite{balasubramaniam2022transparency},~\cite{srivastava2022xai}. This issue becomes critical in systems involving safety or human activity, such as the 2018 Uber autonomous vehicle accident ~\cite{lawless2022toward}, racial bias in image restoration ~\cite{reconstructpictureofBarackObama}, or inappropriate advice from medical chatbots ~\cite{chatbottoldafakepatienttokillthemselves}. These incidents raise fundamental trust concerns in AI systems.

The use of recourse in decision-making and recommendation systems is driven by the need for transparency and accountability, allowing individuals to understand, question, or alter decisions that negatively impact them, such as in loan approvals, medical diagnostics, or education. By integrating recourse, AI systems align more with user needs, enabling participation and influence in decision-making, thereby promoting trust and fairness in AI applications ~\cite{chen2020simple}. Current algorithmic recourse methods are often gradient-based, utilizing computational efficiency to navigate high-dimensional solution spaces and identify necessary modifications ~\cite{fokkema2023attribution},~\cite{karimi2021algorithmic},~\cite{NEURIPS2020_8ee7730e},~\cite{upadhyay2021towards}. These methods incorporate constraints to ensure feasible and applicable changes, adhering to ethical and legal standards ~\cite{gardner2022responsibility},~\cite{venkatasubramanian2020philosophical}.

However, the application of recourse faces significant challenges~\cite{karimi2020survey}, particularly in navigating the high-dimensional space of user features to pinpoint the optimal recourse. In real-world scenarios, determining the most effective recourse is complicated by the multifaceted nature of user attributes and the limitations of existing methods, which often yield suboptimal solutions that compromise various goals~\cite{ustun2019actionable}. These methods lack thorough theoretical analysis and often rely on heuristic methods such as gradient descent, raising concerns about their interpretability and the transparency they are meant to enhance. In the following, we address those challenges in three directions.

\begin{enumerate}
\item \textbf{Multi-Cost Scenarios}: Dealing with multi-cost scenarios in algorithmic recourse methods is complex. Real-world decisions are influenced by multiple types of costs, such as money, time, and effort. Current methods often optimize a single cost, neglecting trade-offs between different costs. Gradient-based approaches handle bi-criteria scenarios by merging loss functions, but deriving suitable weights for multiple costs is unrealistic. Multi-cost scenarios require a nuanced approach to capture the interplay between different costs and provide balanced recommendations.

\item \textbf{Intractable recourse}. As mentioned above, many algorithmic recourse methods are gradient-based or heuristic-based, which lack robust theoretical underpinnings~\cite{karimi2020algorithmic}. This can lead to solutions that are not only suboptimal but also unreliable and unpredictable in their effectiveness. Recent work from Slack et al.~\cite{slack2021counterfactual} emphasizes the vulnerabilities of those recourse applications that are highly sensitive to small changes in the input and can be manipulated under a slight perturbation from an adversary. Hence, providing a solid theoretical framework and tractable solution is needed to assess the reliability, increase the interpretability, and directly influence the trustworthiness of the recourse.

\item \textbf{Simplistic Cost Functions}: Another challenge lies in the overly simplistic and unrealistic cost functions employed by many recourse methods~\cite{chereda2021explaining,dwork2012fairness,verma2020counterfactual}. These functions often assume smoothness and continuity, which may not accurately reflect the discrete and non-differentiable nature of real-world cost functions. For instance, in a loan application scenario, the cost of improving one's credit score cannot be modeled as a continuous function due to the discrete steps involved in credit score changes. The oversimplification of cost functions can lead to recommendations that are impractical or unattainable for individuals, thereby diminishing the utility of the recourse.
\end{enumerate}


\paragraph{Our contribution:} %\todo{In this work, we propose...multi-cost and non-differentiable function}
In this work, we propose a novel approach to algorithmic recourse to navigate the challenges of multi-cost scenarios and can be generalized to non-differentiable or discrete cost functions. Our method first constructs the actionability graph where each edge represents a possible action and the multi-cost is recorded on the edges. Then, we modified the shortest path algorithm to fit the multi-cost function and found all Pareto optimal paths, where each path represents a feasible action list under certain multi-criteria for the user. The analysis shows that our algorithm is optimal and the time complexity is polynomial on the graph size. We then provide a highly non-trivial method to scale our approach into a large graph via a shrinking procedure, which preserves the approximation factor of the Pareto optimal paths and utilizes $\epsilon$-net, which covers most of the points among all the shrunk graphs. 


A toy example in Figure ~\ref{fig:toy eg} illustrates a model predicting whether a person's income exceeds $50K/yr$. For a specific user wanting to increase their income, our algorithm provides three Pareto optimal paths under two cost functions: years of action and change in education levels. Each path represents feasible actions balancing these costs.


There are a few works that use a path-oriented approach for the recourse action~\cite{poyiadzi2020face,nguyen2023feasible} but only focus on specific criteria such as diversity or the density of the path. Other work that uses mixed-integer programming to deal with multi-criteria~\cite{russell2019efficient,mothilal2020explaining} but only allows a limited number of cost functions to fit in. Dandl et al. propose a model-agnostic method that utilizes multi-objective evolutionary algorithm~\cite{dandl2020multi} which is close to our objective but again focuses on specific criteria only. To the best of our knowledge, we are the first work to discuss the multi-cost recourse plan where the cost functions are free to choose for the users as long as they are metric functions. Additionally, we propose a solution to the scalability problem which has not been discussed in any other path-oriented recourse approaches. Lastly, our proposed solution is simple, tractable, and interpretable bringing the trustworthiness of the system, which we emphasize as an important feature in the XAI community. Our experiments show that our algorithm can find different paths under different criteria and is scalable for large graphs.



