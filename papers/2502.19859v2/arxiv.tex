%%%%%%%% ICML 2025 EXAMPLE LATEX SUBMISSION FILE %%%%%%%%%%%%%%%%%

\documentclass{article}
\usepackage[margin=1in]{geometry}
% Recommended, but optional, packages for figures and better typesetting:
\PassOptionsToPackage{numbers, compress}{natbib}
%\usepackage[final]{neurips_2022}
% before loading neurips_2022

\usepackage{natbib}
\usepackage{microtype}
\usepackage{graphicx}
\usepackage{booktabs} % for professional tables

% hyperref makes hyperlinks in the resulting PDF.
% If your build breaks (sometimes temporarily if a hyperlink spans a page)
% please comment out the following usepackage line and replace
% \usepackage{icml2025} with \usepackage[nohyperref]{icml2025} above.
\usepackage{hyperref}

%\usepackage{selectp}
%\outputonly{1-13}
% Attempt to make hyperref and algorithmic work together better:
\newcommand{\theHalgorithm}{\arabic{algorithm}}

% Use the following line for the initial blind version submitted for review:
% \usepackage{icml2025}

% If accepted, instead use the following line for the camera-ready submission:
%\usepackage[accepted]{icml2025}
\usepackage{thmtools,thm-restate}
% For theorems and such
\usepackage{amsmath}
\usepackage{amssymb}
\usepackage{mathtools}
\usepackage{amsthm}
\newcommand{\floor}[1]{\left\lfloor #1 \right\rfloor}
% if you use cleveref..
\usepackage[capitalize,noabbrev]{cleveref}

%%%%%%%%%%%%%%%%%%%%%%%%%%%%%%%%
% THEOREMS
%%%%%%%%%%%%%%%%%%%%%%%%%%%%%%%%
\theoremstyle{plain}
\newtheorem{theorem}{Theorem}[section]
\newtheorem{proposition}[theorem]{Proposition}
\newtheorem{lemma}[theorem]{Lemma}
\newtheorem{corollary}[theorem]{Corollary}
\theoremstyle{definition}
\newtheorem{definition}[theorem]{Definition}
\newtheorem{assumption}[theorem]{Assumption}
\theoremstyle{remark}
\newtheorem{remark}[theorem]{Remark}
%%%%LUCA STUFF%%%%%%
\newcommand{\State}{\STATE}
\newcommand{\For}{\FOR}
\newcommand{\EndFor}{\ENDFOR}
\newcommand{\If}{\IF}
\newcommand{\EndIf}{\ENDIF}
\makeatletter
\newtheorem*{rep@theorem}{\rep@title}
\newcommand{\newreptheorem}[2]{%
	\newenvironment{rep#1}[1]{%
		\def\rep@title{#2 \ref{##1}}%
		\begin{rep@theorem}}%
		{\end{rep@theorem}}}
\newcommand{\mcf}{\mathcal}
\newcommand{\mbf}{\mathbf}
\newcommand{\mbb}{\mathbb}
% Useful packages
\usepackage{amsmath}
\usepackage{graphicx}
\usepackage{algorithm, algorithmic}
%\usepackage{algpseudocode}
\usepackage{amsmath}
\usepackage{amssymb}
\usepackage{subcaption}
\usepackage{xcolor}
\usepackage{amsthm}
%\usepackage[colorlinks=true, allcolors=blue,unicode,psdextra]{hyperref}

\usepackage[capitalize,noabbrev]{cleveref}
\newcommand{\normm}[1]{\lVert#1\rVert}               %norm
\newcommand{\normbig}[1]{\big\lVert#1\big\rVert}
\newcommand{\half}{\tfrac{1}{2}}
\newcommand{\innerprod}[2]{\left\langle{#1},{#2}\right\rangle}
\DeclareMathOperator*{\argmin}{arg\,min}
\DeclareMathOperator*{\argmax}{arg\,max}
\newcommand{\expert}{{\pi_{\textup{E}}}}
\newcommand{\apprentice}{\pi_{\textup{A}}}
\newcommand{\mbs}{\boldsymbol}
\newcommand{\cost}{c}
\newcommand{\true}{c_{\textup{true}}}
\newcommand{\weight}{\mbf{w}}
\newcommand{\uv}{\mbf{u}}
\newcommand{\mv}{\mbs{\mu}}
\newcommand{\op}{\mbf{T}_{\gamma}}
\newcommand{\initial}{\mbs{\nu}_0}
\usepackage{dsfont}
\newcommand{\val}{\mbf{V}}
\newcommand{\thv}{\mbs{\theta}}
\newcommand{\lv}{\mbs{\lambda}}
\newcommand{\lag}{\mcf{L}}
\newcommand{\rlag}{{\mcf{L}_r}}
\newcommand{\mlag}{\bar{\mcf{L}}_r}






\newcommand{\phim}{\mbs{\Phi}}
\newcommand{\psim}{\mbs{\Psi}}
\newcommand{\cma}{\mbf{C}}
\newcommand{\psiv}{\mbs{\psi}}
\newcommand{\phiv}{\mbs{\phi}}
\newcommand{\rhov}{\mbs{\rho}}
\newcommand{\fev}{\mbs{\rho}_{\cma}(\widehat{\expert})}
\newcommand{\efev}{\mbs{\rho}_{\cma}(\widehat{\widehat{\expert}})}
\newcommand{\fevphi}{\mbs{\rho}_{\phim}(\widehat{\expert})}
\newcommand{\efevphi}{\mbs{\rho}_{\phim}(\widehat{\widehat{\expert}})}
\newcommand{\mth}{\mv_{\thv}}
\newcommand{\ul}{\uv_{\lv}}
\newcommand{\cw}{\cost_{\weight}}
\newcommand{\cbg}{C_{\beta,\gamma}}
\newcommand{\mexp}{\mv_{\widehat{\expert}}}
\newcommand{\xv}{\mbf{x}}
\newcommand{\yv}{\mbf{y}}
\newcommand{\zv}{\mbf{z}}
\newcommand{\pth}{\pi_{\thv}}
\newcommand{\pem}{\pi_{\mv}}
\newcommand{\tha}{\thv_{\textup{A}}}
\newcommand{\la}{\lv_{\textup{A}}}
\newcommand{\wa}{\weight_{\textup{A}}}
\newcommand{\tht}{\tilde{\thv}}
\newcommand{\ut}{\tilde{\uv}}
\newcommand{\wt}{\tilde{\weight}}
\newcommand{\mt}{\tilde{\mv}}
\newcommand{\lt}{\tilde{\lv}}
\newcommand{\ma}{\mv_{\textup{A}}}
\newcommand{\ua}{\uv_{\textup{A}}}
\newcommand{\mmat}{\mbf{M}}
\newcommand{\bmat}{\mbf{B}}
\newcommand{\pmat}{\mbf{P}}
\newcommand{\eith}{\mbf{e}_{i_{\thv}}}
\newcommand{\cith}{\cost_{i_{\thv}}}
\newcommand{\ith}{i_{\thv}}
\newcommand{\mpa}{\mv_{\apprentice}}
\newcommand{\weirdv}{\val_{\cith}^{\pth}}
\newcommand{\weirdrho}{\rho_{\cith}(\pth)}

\newcommand{\sspace}{\mcf{S}}
\newcommand{\aspace}{\mcf{A}}
 \newcommand{\thN}{\widehat{\thv}_N}
 \newcommand{\lN}{\widehat{\lv}_N}
 \newcommand{\wN}{\widehat{\weight}_N}
 \newcommand{\pN}{\widehat{\pi}_N}
 \newcommand{\zN}{\widehat{\zv}_N}
 \newcommand{\norm}[1]{\left\| {#1} \right\|}
 \newcommand{\fn}{\mathcal{F}_{n-1}}
 
 \newcommand{\hi}{\hat{\iota}}
 \newcommand{\hx}{\hat{x}}
 \newcommand{\ha}{\hat{a}}
 \newcommand{\hy}{\hat{y}}
 \newcommand{\ti}{\tilde{\iota}}
 \newcommand{\tx}{\tilde{x}}
 \newcommand{\ta}{\tilde{a}}
 \newcommand{\ty}{\tilde{y}}
 \newcommand{\txp}{\tilde{x}'}
 \newcommand{\bi}{\bar{\iota}}
 \newcommand{\bx}{\bar{x}}
 \newcommand{\ba}{\bar{a}}
 \newcommand{\gv}{\mbf{g}}
 \newcommand{\gth}{\mbf{g}_{\thv}}
 \newcommand{\gl}{\mbf{g}_{\lv}}
 \newcommand{\gw}{\mbf{g}_{\weight}}
 \newcommand{\sadav}{\epsilon_{\textup{sad}}(\zN)}
 
 \newcommand{\mpm}{\mv_{\pem}}
 \newcommand{\sad}{\epsilon_{\textup{sad}}(\boldsymbol{\theta},\boldsymbol{\lambda},\weight)}
 \newcommand{\lvz}{\lv_0}
 \newcommand{\ulz}{\uv_{\lvz}}
 \newcommand{\wtrue}{\weight_{\textup{true}}}
 % Parentheses
\newcommand{\bc}[1]{\left\{{#1}\right\}}
\newcommand{\br}[1]{\left({#1}\right)}
\newcommand{\bs}[1]{\left[{#1}\right]}
\newcommand{\abs}[1]{\left| {#1} \right|}
%%%%%%%%%%%%%%%%
% Todonotes is useful during development; simply uncomment the next line
%    and comment out the line below the next line to turn off comments
%\usepackage[disable,textsize=tiny]{todonotes}
\usepackage[textsize=tiny]{todonotes}
\usepackage{subcaption}
\newcommand\blfootnote[1]{%
  \begingroup
  \renewcommand\thefootnote{}\footnote{#1}%
  \addtocounter{footnote}{-1}%
  \endgroup
}
% The \icmltitle you define below is probably too long as a header.
% Therefore, a short form for the running title is supplied here:
%\icmltitlerunning{IL-SOAR : Imitation Learning with Soft Optimistic  Actor cRitic}
\title{ IL-SOAR : Imitation Learning with Soft Optimistic  Actor cRitic}


% The \author macro works with any number of authors. There are two commands
% used to separate the names and addresses of multiple authors: \And and \AND.
%
% Using \And between authors leaves it to LaTeX to determine where to break the
% lines. Using \AND forces a line break at that point. So, if LaTeX puts 3 of 4
% authors names on the first line, and the last on the second line, try using
% \AND instead of \And before the third author name.


\author{
\and
\textbf{Stefano Viel*\blfootnote{Equal contribution.}} \\
\texttt{stefano.viel@epfl.ch} \\
EPFL 
\and
\textbf{Luca Viano*} \\
\texttt{luca.viano@epfl.ch} \\
LIONS \\
EPFL
\and
\textbf{Volkan Cevher} \\
\texttt{volkan.cevher@epfl.ch} \\
LIONS \\
EPFL
}

\begin{document}

\maketitle
% this must go after the closing bracket ] following \twocolumn[ ...

% This command actually creates the footnote in the first column
% listing the affiliations and the copyright notice.
% The command takes one argument, which is text to display at the start of the footnote.
% The \icmlEqualContribution command is standard text for equal contribution.
% Remove it (just {}) if you do not need this facility.

%\printAffiliationsAndNotice{}  % leave blank if no need to mention equal contribution
% otherwise use the standard text.

\begin{abstract}
This paper introduces the SOAR framework for imitation learning. SOAR is an algorithmic template 
that learns a policy from expert demonstrations with a primal dual style algorithm that alternates cost and policy updates. Within the policy updates, the SOAR framework uses an actor critic method with multiple critics to estimate the critic uncertainty and build an optimistic critic fundamental to drive exploration.


When instantiated in the tabular setting, we get a provable algorithm with guarantees that matches the best known results in $\epsilon$.

Practically, the SOAR template is shown to boost consistently the performance of imitation learning algorithms based on Soft Actor Critic such as $f$-IRL, ML-IRL and CSIL in several MuJoCo environments.
Overall, thanks to SOAR, the required number of episodes to achieve the same performance is reduced by half.%\footnote{Project code available at \url{https://github.com/stefanoviel/SOAR-IL/tree/master}}
\end{abstract}

\section{Introduction}
Several recent state of the art imitation learning (IL) algorithms \cite{ni2021f,zeng2022maximum,Garg:2021,watson2023coherent,viano2022proximal} are built on Soft Actor Critic (SAC) \cite{Haarnoja:2018} to perform the policy updates.
SAC uses \emph{entropy} regularized policy updates to maintain a strictly positive probability of taking each action. However, this is known to be an inefficient exploration strategy if deployed alone \cite{cesa2017boltzmann}.

Indeed, several recent theoretical imitation learning achieve performance guarantees by adding exploration bonuses on top of the regularized policy updates, which encourage the learner to visit state-action pairs that have not been visited previously. Unfortunately, such works are only available in the tabular setting \cite{Shani:2021,xu2023provably} and in the linear setting \cite{viano2024imitation}. The design of the exploration bonuses in these works is strictly tight to the tabular or linear structure of the transition dynamics, therefore, these analyses offer little insight on how to design an efficient exploration mechanism using neural network function approximation.

There is, therefore, a lack of a technique that satisfies the following two requirements.
\begin{itemize}
\vspace{-2mm}
\item It is statistically and computationally efficient in the tabular setting.
\vspace{-2mm}
\item It can be implemented easily in continuous states and actions problems requiring neural networks function approximation.
\end{itemize}
In this paper, we present a general template, dubbed Soft Optimistic Actor cRitic Imitation Learning (SOAR-IL) satisfying these requirements. 

The main idea is to act according to an \emph{optimistic} critic within the SAC block on which many IL algorithms rely.
Here, optimism means appropriately underestimating the expected cumulative cost incurred by playing a policy in the environment. This principle known as \emph{optimism in the face of uncertainty} has led to several successful algorithms in the bandits community.
\begin{figure*}[t]
    \centering
\includegraphics[width=\textwidth]{final_figs/method_averages_comparison.png}
\vspace{-4mm}
    \caption{\small{Summary of experimental results. Each plot compares the average normalized return across $4$ MuJoCo environments with $16$ expert trajectories for a base algorithm and its SOAR-enhanced version. SOAR replaces the single critic in SAC-based methods with multiple critics to compute an optimistic estimate. Across all algorithms, incorporating SOAR consistently improves performance. ML-IRL (SA) stands for ML-IRL \cite{zeng2022maximum} from expert state-action demonstrations.}}
    \label{fig:average}
\end{figure*}


While optimism is often achieved using the structure of the problem (tabular, linear, etc.), in this work, we build optimistic estimators using an ensemble technique. That is, multiple estimators for the same quantity are maintained and aggregated to obtain an optimistic estimator.
This technique scales well with deep imitation learning.
To summarize, we have the following contributions.
% 
\paragraph{Theoretical contribution}
We show that there exists a computationally efficient algorithm that uses an ensemble based exploration technique that gives access to $\mathcal{O}(\epsilon^{-2})$ expert trajectories and $\mathcal{O}(\epsilon^{-2})$ interactions in a tabular MDP outputs a policy such that its cumulative expected cost is at most $\epsilon$ higher than the expert cumulative expected cost with high probability.
% 
\paragraph{Practical Contribution} We apply an ensemble-based exploration technique, SOAR, to boost the performance of deep imitation learning algorithms built on SAC, demonstrating its effectiveness on MuJoCo environments. Specifically, we show that incorporating SOAR consistently boosts the performance of base methods such as Coherent Soft Imitation Learning (CSIL)\cite{watson2023coherent}, Maximum Likelihood IRL (ML-IRL)\cite{zeng2022maximum} and RKL \cite{ni2021f}. As shown in Figure~\ref{fig:average}, our approach consistently outperforms the base algorithms across all MuJoCo environments. Notably, SOAR achieves the best performance of the baselines requiring only approximately half the number of learning episodes.

\section{Preliminaries and Notation}
The environment is abstracted as Markov Decision Process (MDP) \cite{Puterman:1994} 
which consists of a tuple $(\sspace, \aspace, P, c, \initial, \gamma)$ where $\sspace$ is the state space, $\aspace$ is the action space, $P: \sspace\times\aspace \rightarrow \Delta_{\sspace}$ is the transition kernel, that is, 
$P(s'|s,a)$ denotes the probability of landing in state $s'$ after choosing action $a$ in state $s$. Moreover, $\initial$ is a distribution over states from which the initial state is sampled. Finally, $c: \sspace\times \aspace \rightarrow [0,1]$ is the cost function, and $\gamma \in [0,1)$ is called the discount factor.% (see Protocol~\ref{prot:interaction} in \Cref{sec:protocol}).

\textbf{Value functions and occupancy measures} We define the state value function at state $s \in \sspace$ for the policy $\pi$ under the cost function $c$ as $V_c^{\pi}(s) \triangleq \mathbb{E}\bs{\sum^{\infty}_{h=0} \gamma^{h} c(s_h,a_h) | s_1 = s}$.
The
expectation over both the randomness of the transition dynamics and the one of the learner's policy.
Another convenient quantity is the occupancy measure of a policy $\pi$ denoted as $d^{\pi}\in \Delta_{\sspace\times\aspace}$ and defined as follows
$d^{\pi}(s,a) \triangleq (1 - \gamma)\sum^{\infty}_{h=0} \gamma^{h}\mathbb{P}\bs{s,a \text{ is visited after $h$ steps acting with $\pi$}}$. We can also define the state occupancy measure
as $d^{\pi}(s) \triangleq (1 - \gamma)\sum^{\infty}_{h=0} \gamma^{h}\mathbb{P}\bs{s\text{ is visited after $h$ steps acting with $\pi$}}$.



\textbf{Imitation Learning} In imitation learning, the learner is given a dataset $\mathcal{D}_{\expert}$ of expert trajectories collected by an unknown expert policy $\expert$.\footnote{In order, to accommodate state-only and state-action with a unified analysis we overload the notation for the expert dataset. $\mathcal{D}_\expert$ denotes a collection of samples from the expert state occupancy measure in the former case and a collection of state-actions sampled from the state-action occupancy measure in the latter case.}
By trajectory $\boldsymbol{\tau}^k$, we mean the sequence of states and actions sampled rolling out the policy $\pi_k$ for a number of steps sampled from the distribution $\mathrm{Geometric}(1 - \gamma)$.
 Given $\mathcal{D}_{\expert}$, the learner adopts an algorithm $\mathcal{A}$ to learn a policy $\pi^{\mathrm{out}}$ such that $\innerprod{\initial}{V^{\widehat{\pi}^k}_{c_{\mathrm{true}}}  - V^{\expert}_{c_{\mathrm{true}}} } \leq \epsilon$ with high probability.
 
We use the notation $\mathbf{e}_s$ to denote a vector in $\mathbb{R}^{\abs{\sspace}}$ zero everywhere but in the coordinate corresponding to the state $s$ ( for an arbitrary ordering of the states). Analogously, we use  $\mathbf{e}_{s,a}$ to denote a vector in $\mathbb{R}^{\abs{\sspace}\abs{\aspace}}$ zero everywhere but in the $(s,a)^{th}$ entry which equals one.
\section{The Algorithm}
\begin{algorithm}
\caption{SOAR-Imitation Learning \label{alg:meta}}
\begin{algorithmic}[1]
\REQUIRE Reward step size $\alpha$, Expert dataset $\mathcal{D}_\expert$, Discount factor $\gamma$, Policy step size $\eta$.
\State Initialize $\pi^1$ as uniform distribution over $\mathcal{A}$.%, learner dataset $\mathcal{D}_\pi$.
%\State Initialize neural networks: $r_\theta$ for reward, $\{Q_1, \ldots, Q_K\}$ for q-value estimation, $\pi$ for policy
\State Initialize empty replay buffer,i.e. $\mathcal{D}^0 = \bc{}$
\For{$k = 1$ to $K$}
    \State $\tau^k \gets \textsc{CollectTrajectory}(\pi^{k})$
    \State Add $\tau^k$ to replay buffer,i.e. $\mathcal{D}^{k} = \mathcal{D}^{k - 1} \cup \tau^k $.
    \State $ c^k \gets \textsc{UpdateCost}(c^{k-1}, \mathcal{D}_\expert, \mathcal{D}^k, \alpha)$
    \For{$\ell = 1$ to $L$}
        \State Compute estimator $Q^k_\ell$.
    \EndFor
    \State $Q^k = \textsc{OptimisticQ}( \bc{Q^k_\ell}^L_{\ell=1} )$.
    %\State $\pi^k(a|s) \propto \exp\br{- \eta \sum^k_{\tau =1} Q^k(s,a)}$
    \State $\pi^k(a|s) = \textsc{PolicyUpdate}(\eta, \bc{Q^\tau(s,a)}^K_{\tau=1})$ %\propto \exp\br{- \eta \sum^k_{\tau =1} Q^k(s,a)}$
\EndFor
\end{algorithmic}
\end{algorithm}
In this Section, we describe \Cref{alg:meta}. A meta-algorithm that encompasses several existing imitation learning algorithms.
Inside each iteration of the main for loop, the learner collects a new trajectory sampling actions from the policy $\pi^k$ (Line 4 in \Cref{alg:meta}) and then performs the following steps.
\begin{itemize}
    \item \textbf{The Cost update.} At Line 6 of \Cref{alg:meta}, the learner updates an estimate of the true unknown cost function with the algorithm-dependent routine $\textsc{UpdateCost}$. %Below, we provide examples of how $\textsc{UpdateCost}$ is implemented in popular imitation learning algorithms.
    For instance, Generative Adversarial Imitation Learning (GAIL), Adversarial Inverse Reinforcement Learning (AIRL), and Discriminator Actor Critic (DAC) \cite{Ho:2016b,Fu:2018,Kostrikov:2019} use a reward derived from a discriminator neural network trained to distinguish state-action pairs visited by the expert from those visited by the learner. 
    
    Using a fixed cost function obtained from a behavioral cloning warm up is, instead, the approach taken in CSIL \cite{watson2023coherent}. Moreover, updating the reward to minimize an information theoretic divergence between expert and learner state occupancy measure is the approach taken in RKL \cite{ni2021f}. Finally, \cite{zeng2022maximum} updates the cost using online gradient descent (OGD) \cite{Zinkevich2003}.
    %
    \item \textbf{The state-action value function update.} In the \emph{for} loop at Lines 7-9 of \Cref{alg:meta}, the learner updates $L$ different critics trained on different subsets of the data sampled from the replay buffer $\mathcal{D}^k$ denoted as $\bc{\mathcal{D}^k_\ell}^L_{\ell=1}$. For a fixed state-action pair each dataset contains independent samples from $P(\cdot|s,a)$. This allows creating $L$ jointly independent random variables $\bc{Q^k_\ell}^L_{\ell=1}$ that estimate the ideal value iteration update (i.e. $c^k + \gamma P V^k$) which cannot be implemented exactly due to the lack of knowledge on $P$. 
    
    In \Cref{sec:tabular}, we provide an explicit way to compute $L$ slightly optimistic estimates for the tabular setting. Moreover, in the deep imitation learning experiments, we train $L$ different critics via temporal difference, as it is commonly done in Soft Actor Critic implementation ( see \cite{Haarnoja:2018} ). 
    
    Finally, in Line 10 of \Cref{alg:meta}, the $L$ critics are aggregated to generate an estimate $Q^{k+1}$ which is, with high probability, optimistic i.e. $Q^{k+1} \leq c^k + \gamma P V^k$. In other words, it underestimates the update that could have been performed by value iteration if the transition matrix $P$ was known to the learner.
    We provide aggregation routines that satisfy this requirement if $L$ is large enough.
    %
    \item \textbf{The policy update} As the last step of each inner loop, the learner updates the policy using the optimistic state-action value function estimate.
    In the tabular case, we will instantiate the update using an online mirror descent (OMD) step \cite{Beck:2003,nemirovskij1983problem} (also known as the multiplicative weights update \cite{warmuth1997continuous,auer1995gambling}).
    As it will be evident from \Cref{sec:analysis}, this update we can ensure that the KL divergence between consecutive policies is upper bounded in terms of the policy step size $\eta$.
    For the continuous state-action experiments, the online mirror descent is approximated via a gradient descent step on the SAC loss.
\end{itemize} 
\begin{remark}
Notice that only one critic is used in the implementation of SAC ($L =1$) that serves as base RL algorithm for several commonly used IL algorithms ( GAIL \cite{Ho:2016b}, AIRL \cite{Fu:2018}, IQ-Learn \cite{Garg:2021}, PPIL \cite{viano2022proximal}, RKL \cite{ni2021f} and ML-IRL \cite{zeng2022maximum}). As proven in \Cref{cor:optimism}, $L=1$ is not enough to ensure optimism, not even in the tabular case.
In our experiments, we show that a value of $L$ larger than $1$ is beneficial in all the MuJoCo environments we tested on.
\end{remark}

\subsection{Algorithm with guarantees in the tabular case}
\label{sec:tabular}


We consider an instance of \Cref{alg:meta} in the tabular case for which we will prove theoretical sample efficiency guarantees. We present the pseudocode in \Cref{alg:theory_version}.

For what concerns the analysis, the first step is to extract the policy achieving the sample complexity guarantees above via an online-to-batch conversion. 
That is, the output policy is sampled uniformly from a collection of $K$ policies $\bc{\pi^k}^K_{k=1}$. 
The sample complexity result follows from proving that the policies $\bc{\pi^k}^K_{k=1}$ produced by \Cref{alg:theory_version} is a sequence with sublinear regret in high probability. More formally, we define the regret as follows.
\begin{definition}\label{def:regret}
\textbf{Regret}
The regret is defined as follows
\begin{equation*}
\mathrm{Regret}(K)\triangleq\frac{1}{1 - \gamma}\sum^K_{k=1} \innerprod{\true}{d^{\pi^k} - d^{\expert}}
\end{equation*}
\end{definition}
\begin{remark}
Notice that the regret defined in this way satisfies $\mathrm{Regret}(K) = \sum^K_{k=1} \innerprod{\initial}{V^{\pi^k}_{c_{\mathrm{true}}} - V^{\expert}_{c_{\mathrm{true}}} }$. For this reason, we require the factor $(1-\gamma)^{-1}$ in the definition.
\end{remark}
Omitting dependencies on the horizon and the state action spaces cardinality, we will guarantee that
$$
\mathrm{Regret}(K) \leq \mathcal{O}(K^{1/2} + K \abs{\mathcal{D}_\expert}^{-1/2}),
$$
with high probability.
Notice that this bound is sublinear in $K$, for  $\abs{\mathcal{D}_\expert} = \mathcal{O}(K)$.
To obtain such bound, we adopt the following decomposition for $(1-\gamma)\mathrm{Regret}(K)$ adapted from \cite{Shani:2021} to accommodate the infinite horizon setting.
\begin{equation}
    \underbrace{\sum^K_{k=1} \innerprod{c^k}{d^{\pi^k} - d^{\expert}}}_{:= (1-\gamma)\mathrm{Regret}_{\pi}(K,\expert)} + \underbrace{\sum^K_{k=1} \innerprod{c_{\mathrm{true}} - c^k}{d^{\pi^k} - d^{\expert}}}_{:= (1-\gamma)\mathrm{Regret}_c(K, c_{\mathrm{true}})} \label{eq:dec}
\end{equation}
\begin{algorithm}
\caption{Tabular SOAR-IL \label{alg:theory_version}}
\begin{algorithmic}[1]
\REQUIRE Step size $\eta$, Expert dataset $\mathcal{D}_\expert$, Discount factor $\gamma$, Reward step size $\alpha$, $N^0(s,a)=0$ for all $s,a$, number of estimators $L= 36 \log \br{\abs{\sspace}\abs{\aspace} K /\delta}$.
\State Initialize $\pi^1$ as uniform distribution over $\mathcal{A}$

\For{$k = 1$ to $K$}
    \State Sample trajectory length $L^k \sim \mathrm{Geometric}(1 - \gamma)$.
    \State $\tau^k= \bc{(s^k_t, a^k_t)}^{L^k}_{t=1}$ rolling out $\pi^k$ for $L^k$ steps.
    \State Update counts for all $s^k_t, a^k_t \in \tau_k$:
    $$N^k(s^k_t, a^k_t) = N^{k-1}(s^k_t, a^k_t) + 1.$$
    %\begin{equation*}
    %N^k(s^k_t, a^k_t, s^k_{t+1}) = N^{k-1}(s^k_t, a^k_t,s^k_{t+1}) + 1
    %\end{equation*}
    \State Add $s^k_t, a^k_t, s^k_{t+1}$ to the datasets with index $\ell = N^k(s^k_t, a^k_t)  \mod L$, $$\mathcal{D}_\ell^k = \mathcal{D}_\ell^{k-1} \cup \bc{s^k_t, a^k_t},~~~~ \mathcal{R}_\ell^k = \mathcal{R}_\ell^{k-1} \cup \bc{s^k_t, a^k_t,s^k_{t+1}}. $$ %~~~~\text{for}~~~~ \ell = N^k(s^k_t, a^k_t) \mod L .$$ 
    \State $ c^k = \textsc{CostUpdateTabular}(c^{k-1}, \tau^k, \mathcal{D}_\expert).$
    \For{$\ell = 1$ to $L$}
        \State $N^k_\ell(s,a,s') = \sum_{\bar{s},\bar{a},\bar{s}'\in \mathcal{R}^k_\ell} \mathds{1}_{\bc{\bar{s},\bar{a},\bar{s}' = s,a,s'}}.$
        \State $N^k_\ell(s,a) = \sum_{\bar{s},\bar{a}\in \mathcal{D}^k_\ell} \mathds{1}_{\bc{\bar{s},\bar{a} = s,a}}.$
        \State $\widehat{P}^k_\ell(\cdot|s,a) = \frac{N^k_\ell(s,a,\cdot)}{N^k_\ell(s,a) + 2}$
    \EndFor
    \State $Q^{k+1} = \textsc{OptimisticQTabular}(V^k, \bc{\widehat{P}^k_\ell}^L_{\ell=1}, c^k)$.
    \State $\pi^{k+1}(a|s) \propto \pi^{k}(a|s) \exp\br{-\eta Q^{k+1}(s,a)}$
    \State $ V^{k+1}(s) = \innerprod{\pi^{k+1}(\cdot|s)}{Q^{k+1}(s,\cdot)}$
\EndFor
\State \textbf{Return} The mixture policy $\widehat{\pi}^K$.
\end{algorithmic}
\end{algorithm}

The algorithmic design for the tabular setting aims at updating the cost variable so that the term $\mathrm{Regret}_c$ grows sublinearly (see Line 7 in \Cref{alg:theory_version}). 

We consider both cases of imitation from state-action expert data (Lines 4-6 of $\textsc{CostUpdateTabular}$ ) and state-only expert data (Lines 2-3 of $\textsc{CostUpdateTabular}$). These cases differ only in the stochastic loss for the cost update. Notice that we overload the notation to address both state-only and state-action imitation learning with a unified analysis.
In particular, $\widehat{d^{\pi^k}}$ is an unbiased estimate of the learner occupancy measure. For state-only imitation learning we use $ \widehat{d^{\pi^k}} = \mathbf{e}_{s^k_{L^k}}$ and estimated expert occupancy measure equals to $\widehat{d^\expert} = \abs{\mathcal{D}_\expert}^{-1}\sum_{s\in \mathcal{D}_\expert} \mathbf{e}_s$ while for state-action imitation learning $ \widehat{d^{\pi^k}} = \mathbf{e}_{s^k_{L^k}, a^k_{L^k}} $ and $\widehat{d^\expert} = \abs{\mathcal{D}_\expert}^{-1}\sum_{s,a\in \mathcal{D}_\expert} \mathbf{e}_{s,a}$.
The formal bound on $\mathrm{Regret}_c$ is given in \Cref{thm:reward_regret_bound}.
\begin{algorithm}
\caption{\textsc{CostUpdateTabular}}
\begin{algorithmic}[1]
\REQUIRE Current cost vector $c^{k-1}$, trajectory $\tau^k$, expert dataset $\mathcal{D}_\expert$.
\IF {$\textsc{State-Only} = \textsc{TRUE}$}
    \State $\widehat{d^{\pi^k}} = \mathbf{e}_{s^k_{L^k}}$.
    
    \State $\widehat{d^\expert} = \abs{\mathcal{D}_\expert}^{-1}\sum_{s\in \mathcal{D}_\expert} \mathbf{e}_s$
    \ELSE
    \State $\widehat{d^{\pi^k}} = \mathbf{e}_{s^k_{L^k},a^k_{L^k}}$.
    
    \State $\widehat{d^\expert} = \abs{\mathcal{D}_\expert}^{-1}\sum_{s,a \in \mathcal{D}_\expert} \mathbf{e}_{s,a}$
    \ENDIF
    \State \textbf{Return:} $ c^{k} \gets \Pi_{\mathcal{C}}\bs{c^{k-1} - \alpha (\widehat{d^{\expert}} - \widehat{d^{\pi^k}} )} $
\end{algorithmic}
\end{algorithm}
The rest of the algorithm aims to provide a sublinear bound on $\mathrm{Regret}_\pi$.
In particular, the updates for the estimated transition kernels $\bc{\widehat{P}^k_\ell}^L_{\ell=1}$ in Lines 8-12 of \Cref{alg:theory_version} serves to build $L$ slightly optimistic \footnote{The optimism is achieved by adding $2$ in the denominator of the estimated transition kernels.} estimate of the ideal value function update. 

In the routine $\textsc{OptimisticQTabular}$, we propose two aggregation rules to generate the optimistic $Q$ value estimate to be used in the policy update step. The first one, takes the minimum of the $L$ estimators as in \Cref{eq:update1}, while the second option \Cref{eq:update2} considers the mean of the $L$ estimators minus a factor proportional to the empirical standard deviation. By Samuelson's inequality \cite{samuelson1968deviant}, we prove that the second option is more optimistic.
\begin{algorithm}
\caption{\textsc{OptimisticQTabular}}
\begin{algorithmic}[1]
\REQUIRE current state value function estimate $V^k$, ensemble of estimated transitions $\bc{\widehat{P}^k_\ell}^L_{\ell=1}$, cost $c^k$.
    \State \textcolor{blue}{// Option 1} \begin{equation} \text{\textbf{return}}~~~Q^{k+1} = c^k + \gamma \min_{\ell \in [L]} \widehat{P}^k_\ell V^k \tag{Min}\label{eq:update1}\end{equation}
    \State \textcolor{blue}{// Option 2}
    \begin{equation}
        \text{\textbf{return}}~~~ Q^{k+1} = c^k + \gamma \max\bs{\frac{1}{L}\sum^L_{\ell =1} \widehat{P}^k_\ell V^k - \sigma^k   , 0}\tag{Mean-Std}\label{eq:update2} 
    \end{equation}
    \State with $\sigma^k = \sqrt{\sum^L_{\ell=1} \br{
    \widehat{P}^k_\ell V^k
    - \frac{1}{L} \sum^L_{\ell'=1} \widehat{P}^k_{\ell'} V^k}^2}.$
\end{algorithmic}
\end{algorithm}

Finally, an iteration of the tabular case algorithm is concluded by the policy update implemented via OMD.
%The sublinear bound on the regret for the player that updates $\pi$ ($\mathrm{Regret}_\pi$) is proven in \Cref{thm:policy_regret}.

Having described our main techniques we are in the position of stating our main theoretical results hereafter.
\begin{theorem}
\label{thm:main_result}\textbf{Main Result}
For any MDP, let us consider either the update \Cref{eq:update1} or \Cref{eq:update2}, it holds that with probability $1-5\delta$ that $\frac{\mathrm{Regret}(K)}{K}$ of Tabular SOAR-IL (\Cref{alg:theory_version}) is upper bounded by
\begin{equation*}
\widetilde{\mathcal{O}}\br{\sqrt{\frac{\abs{\sspace}^4 \abs{\aspace}  \log (1 / \delta)}{(1-\gamma)^5 K}}} + \sqrt{ \frac{\abs{\sspace}^2\abs{\aspace} \log \br{\abs{\sspace}\abs{\aspace}/\delta} (\log(\abs{\sspace}) + 2)^2}{(1-\gamma)^2\abs{\mathcal{D}_\expert}}}.
\end{equation*}
Therefore, choosing $K = \widetilde{\mathcal{O}}\br{\frac{\abs{\sspace}^4 \abs{\aspace}  \log (1 / \delta)}{(1-\gamma)^5 \epsilon^2}}$ and $\abs{\mathcal{D}_\expert} = \frac{\abs{\sspace}^2\abs{\aspace}  \log \br{\abs{\sspace}\abs{\aspace} /\delta} (\log(\abs{\sspace}) + 2)^2}{ \epsilon^2 (1-\gamma)^2}$ it holds that the mixture policy $\widehat{\pi}_K$ satisfies $\innerprod{\initial}{V^{\widehat{\pi}^k}_{c_{\mathrm{true}}}  - V^{\expert}_{c_{\mathrm{true}}} } \leq \epsilon$ with probability at least $1-5 \delta$.
%Using the update given in \Cref{eq:update2} gives $K = \widetilde{\mathcal{O}}\br{\frac{\abs{\sspace}^2 \abs{\aspace}  \log (1 / \delta)}{(1-\gamma)^5 \epsilon^2}}$.
\end{theorem}
\begin{remark}
The bound on $\abs{\mathcal{D}_\expert}$ is the bound on the number of either state-only or state-action expert trajectories depending on the setting considered.
\end{remark}
\begin{remark}
The gurantees are stated for the mixture policy $\widehat{\pi}^K$, i.e. the policy which has an occupancy measure equal to the average occupancy measure of the policies in the no-regret sequence. That is, it holds that $d^{\widehat{\pi}^K} = K^{-1} \sum^K_{k=1} d^{\pi^k}$. The policy $\widehat{\pi}^K$ cannot be computed without knowledge of $P$ but sampling a trajectory from it can be done by choosing an index $k \sim \mathrm{Unif}([K])$ at the beginning of each new episode and continuing rolling out the policy $\pi^k$ for a number of steps sampled from $\mathrm{Geom}(1-\gamma)$.
\end{remark}
\begin{remark}
In the case of state-only expert dataset the provided upper bounds for $K$ and $\abs{\mathcal{D}_\expert}$ are optimal up to log factors in the precision parameters $\epsilon$. Indeed, these upper bounds match the lower bounds in \cite{moulin2025optimistically}.
\end{remark}
\section{Theoretical analysis}
\label{sec:analysis}
We need to start with an important remark on the structure of the MDP considered in the proof.
\begin{remark}
\label{remark:MDP_comment}
For technical reasons, in particular for the proof of \Cref{cor:optimism}, we consider as intermediate step in the proof MDPs where from each state action pairs is possible to observe a transition to only two other possible states.
While this restriction on the dynamics appears to be limiting any MDP can be cast into this form at the cost of a quadratic blow up in the number of states, from $\abs{\sspace}$ to $\abs{\sspace}^2$. To see this, for a general MDP where from a given state action pair a transition to all possible $\abs{\sspace}$ states can be observed 
 is equivalent to a binarized MDP where this \emph{one layer} transition is represented with a tree of depth at most $\log_2 \br{\abs{\sspace}}$ with binary transitions only. Moreover, the discount factor in the binarized MDP should be set to $\gamma_{\mathrm{bin}} = \gamma^{-\log_2\abs{\sspace}}$ to maintain the return unchanged.
 We consider in this section a binarized MDP with $\abs{\sspace}$ states in this section and we squared the number of states in stating \Cref{thm:main_result} which holds for general MDPs. Moreover, in stating the result for general MDP we also inflated the effective horizon by a factor $\log_2\abs{\sspace}$ as shown in \Cref{lemma:eff_horizon}.
 \end{remark}
As mentioned, the proof is decomposed into two main parts: (i) bounding the policy regret $\mathrm{Regret}_\pi$ and (ii) bounding the cost updates regret  $\mathrm{Regret}_c$.
In particular, we can prove the two following results.
\begin{restatable}{theorem}{thmpolicyregret}\label{thm:policy_regret}\textbf{Policy Regret}
In a binarized MDP with $\abs{\sspace}$ states and discount factor $\gamma$, it holds that with probability $1 - 3\delta$, for any policy $\pi^\star$, $\mathrm{Regret}_{\pi}(K,\pi^\star) $ is upper bounded by
\begin{equation*}
\frac{\log \abs{\aspace}}{\eta (1-\gamma)} + \frac{\eta K}{(1-\gamma)^4} + \widetilde{\mathcal{O}}\br{\frac{ \sqrt{ K \abs{\sspace}^2 \abs{\aspace} \log (1/\delta)}}{(1-\gamma)^2}}
\end{equation*}
and for $\eta = \sqrt{ \frac{\log \abs{\aspace} (1-\gamma)^3}{K}}$ it holds that using the update in \eqref{eq:update1} or in \eqref{eq:update2} it holds that $\mathrm{Regret}(K,\pi^\star)$ is upper bounded by $\widetilde{\mathcal{O}}\br{ \sqrt{ \frac{K \abs{\sspace}^2 \abs{\aspace} \log (1/\delta)}{(1-\gamma)^5}}}$.
%\begin{align*}
% &\sqrt{\frac{\log \abs{\aspace} K }{ (1-\gamma)^5}} + \widetilde{\mathcal{O}}\br{\frac{ \sqrt{ K \abs{\sspace}^2 \abs{\aspace} \log (1/\delta)}}{1-\gamma}} \\&\leq \widetilde{\mathcal{O}}\br{ \sqrt{ \frac{K \abs{\sspace}^2 \abs{\aspace} \log (1/\delta)}{(1-\gamma)^5}}}.
%\end{align*}
\end{restatable}
\begin{restatable}{theorem}{thmcostregret}
\label{thm:reward_regret_bound} \textbf{Cost Regret}
In a binarized MDP with $\abs{\sspace}$ states and discount factor $\gamma$, it holds that with probability $1-2\delta$, $(1-\gamma)\mathrm{Regret}_c(K; c_{\mathrm{true}})$ is upper bounded by
\begin{equation*}
    4 \sqrt{K \log (1/\delta)} + K \sqrt{ \frac{\abs{\sspace} \abs{\aspace}\log \br{\abs{\sspace}\abs{\aspace}/\delta}}{2 \abs{\mathcal{D}_\expert}}}
\end{equation*}
\end{restatable}
\begin{remark}
Once \Cref{thm:policy_regret,thm:reward_regret_bound} are proven the bound on \Cref{thm:main_result} follows trivially by a union bound and bounding $\mathrm{Regret}_\pi$ and $\mathrm{Regret}_c$ with \Cref{thm:policy_regret} and \Cref{thm:reward_regret_bound} respectively and dividing everything by $K$ (because in \Cref{thm:main_result} we consider the quantity $\mathrm{Regret}(K)/K$). Finally, we also divide by $1-\gamma$, to match the definition of $\mathrm{Regret}(K)$ in \Cref{def:regret}.
\end{remark}
\subsection{Proof Sketch of \Cref{thm:policy_regret}}
The regret decomposition towards the proof of \Cref{thm:policy_regret} leverages the following Lemma.
\begin{restatable}{lemma}{lemmapdl}
\label{lemma:infinite_extendend_pdl}
Consider the MDP $M = (\sspace, \aspace, \gamma, P, \cost)$ and two policies $\pi, \pi^\prime: \sspace \rightarrow \Delta_{\aspace}$. Then consider for any $\widehat{Q} \in \mathbb{R}^{\abs{\sspace}\abs{\aspace}}$ and $\widehat{V}^{\pi}(s) = \innerprod{\pi(\cdot|s)}{\widehat{Q}(s,\cdot)}$ and $Q^{\pi^\prime}, V^{\pi^\prime}$ be respectively the state-action and state value function of the policy $\pi$ in MDP $M$. Then, it holds that $(1-\gamma)\innerprod{\initial}{\widehat{V}^{\pi} - V^{\pi^\prime}}$ equals
\begin{equation*}
      \innerprod{d^{\pi^\prime}}{\widehat{Q} - \cost - \gamma P \widehat{V}^{\pi}} + 
    \mathbb{E}_{s\sim d^{\pi^\prime}}\bs{\innerprod{\widehat{Q}(s,\cdot}{\pi(\cdot|s) -\pi^\prime(\cdot|s)}}.
\end{equation*}
\end{restatable}
\begin{remark}
This Lemma is a generalization of the well-known performance difference Lemma \cite{Kakade:2002} to the case of inexact value functions. Indeed, notice that if $\widehat{Q} = Q^\pi$, then the first term in the decomposition equals zero and the result boils down to the standard performance difference Lemma. For arbitrary $\widehat{Q}$, the first term is a temporal difference error averaged by the occupancy measure $d^{\pi^\prime}$.
\end{remark}
We can apply two times \Cref{lemma:infinite_extendend_pdl} on each of the summands of the sum from $k=1$ to $K$, to obtain a convenient decomposition of $\mathrm{Regret}_\pi$ . Denoting $\delta^{k}(s,a) \triangleq \cost^k(s,a) + \gamma P V^k(s,a) - Q^{k+1}(s,a)$ and $g^k(s,a) \triangleq  Q^{k+1}(s,a) - Q^{k}(s,a)$, we have that
\begingroup
\allowdisplaybreaks
\begin{align}
&(1 - \gamma)  \mathrm{Regret}_\pi(K;\pi^\star) = \nonumber \\ & (1-\gamma) \sum^K_{k=1} \innerprod{\initial}{V^{\pi^k}_{c^k} - V^k + V^k - V^{\pi^\star}_{c^k}} = \nonumber \\ &\sum^K_{k=1} \mathbb{E}_{s\sim d^{\pi^\star}}\bs{
    \innerprod{Q^k(s,\cdot)}{\pi^k(s) - \pi^\star(s)}} \quad \tag{BTRL} \label{eq:main_OMD}\\&\phantom{\leq}+ \sum^K_{k=1}  \sum_{s,a} \bs{d^{\pi^k}(s,a) - d^{\pi^\star}(s,a)}\cdot\bs{\delta^k(s,a)} \tag{Optimism} \label{eq:main_optimism}
    \\&\phantom{\leq}+ \sum^K_{k=1} \mathbb{E}_{s,a\sim d^{\pi^k}}\bs{g^k(s,a)} - \sum^K_{k=1} \mathbb{E}_{s,a\sim d^{\pi^\star}}\bs{g^k(s,a)} \quad \label{eq:main_shift} \tag{Shift}
\end{align}
Next, we bound each of these terms individually.
Starting from the first term, the next Lemma shows that our policy update (Line 14 of \Cref{alg:theory_version}) can be seen as an instance of Be the regularized leader (BTRL) ( see e.g. \cite{orabona2023modern} ). Therefore, it guarantees that for any sequence $\bc{Q^k}^K_{k=1}$, the term \eqref{eq:main_OMD} is bounded as follows.
\begin{restatable}{lemma}{localregret}\label{lemma:local_regret}

Let us consider the sequence of policies $\bc{\pi^k}^K_{k=1}$ generated by \Cref{alg:theory_version}  for all $\eta > 0$ then it holds that $ \eqref{eq:main_OMD} \leq \frac{\log \abs{\aspace}}{\eta}.$
\end{restatable}
Next, we show that thanks to the multiplicative weights update for the policy the KL divergence between consecutive policies is upper bounded by the policy step size $\eta$, i.e. $D_{KL}(\pi^{k+1}(\cdot|s), \pi^k(\cdot|s)) \leq \mathcal{O}(\eta)$ for all $s \in \sspace$. Thanks to this \emph{slow changing} property, we can prove the following bound on \eqref{eq:main_shift}.
\begin{restatable}{lemma}{shiftlemma}
        \label{lemma:shift}
For the sequence of policies $\bc{\pi^k}^K_{k=1}$ generated by \Cref{alg:theory_version},  for all $\eta > 0$, it holds that $\eqref{eq:main_shift} \leq \frac{\eta K }{(1-\gamma)^3}.$
\end{restatable}
\begin{remark}
The step size choice for $\eta$ in \Cref{thm:policy_regret} is made to trade off optimally the bounds in \Cref{lemma:local_regret,lemma:shift}.
\end{remark}
Finally, the most technical part of the proof aims at bounding the term \eqref{eq:main_optimism}.
\begin{restatable}{lemma}{lemmaoptimism} \label{lemma:main_optimism}
Let us consider an MDP where $\max_{s,a\in\sspace\times\aspace} \mathrm{supp}(P(\cdot|s,a)) = 2$. For each $k \in [K]$, if the $Q^{k+1}$ in \Cref{alg:theory_version}, are updated according to \eqref{eq:update1} or \eqref{eq:update2},  the iterates produced by \Cref{alg:theory_version} satisfy with probability $1-3\delta$ that
\begin{equation*}
    \eqref{eq:main_optimism} \leq \widetilde{\mathcal{O}}\br{\frac{  \sqrt{K \abs{\sspace}^2 \abs{\aspace} \log (1/\delta)}}{1-\gamma}}.
\end{equation*}

%Moreover, if the update for $Q^{k+1}$ according to \eqref{eq:update2}
%then it holds that
%\begin{equation*}
%    \eqref{eq:main_optimism} \leq \widetilde{\mathcal{O}}\br{\frac{  \sqrt{K \abs{\sspace}^4 \abs{\aspace} \log (1/\delta)}}{1-\gamma}}.
%\end{equation*}
\end{restatable}
\textbf{Proof sketch of \Cref{lemma:main_optimism}}
The proof of this Lemma, leverages that the temporal difference errors $\delta^k(s,a)$ produced by \Cref{alg:theory_version} are positive with high probability %\footnote{In fact with probability exponentially increasing with the number of critics $L$.} 
as shown by the next result\footnote{In the main text, we present the proof for the update in \eqref{eq:update1}. The case of update as in \eqref{eq:update2} is deferred to the Appendix.}. 
\begin{restatable}{corollary}{coroptimism}
    \label{cor:optimism}
    Consider an MDP where $\max_{s,a\in\sspace\times\aspace} \mathrm{supp}(P(\cdot|s,a)) = 2$, then for  $L = 36\log \br{\frac{ \abs{\sspace}\abs{\aspace} K}{\delta}}$ it holds that with probability at least $1-\delta$
    \begin{equation*}
    \min_{\ell\in[L]} \widehat{P}^k_\ell V^k(s,a) \leq P V^k(s,a)  ~~~ \forall ~~~s,a\in \sspace\times\aspace,~~~\forall ~~k \in [K].
    \end{equation*} 
\end{restatable}

\Cref{cor:optimism} implies that $-\innerprod{d^{\pi^\star}}{\delta^k} \leq 0$ for all $k \in [K]$ and therefore that $\eqref{eq:main_optimism} \leq \sum^K_{k=1}\innerprod{d^{\pi^k}}{\delta^k}$.
\begin{remark}
The above inequality, it is crucial for obtaining the result. Indeed, it upper bounds $\eqref{eq:main_optimism}$ with the \emph{on-policy} temporal difference errors \footnote{That is the temporal difference errors $\delta^k$ averaged by the learner occupancy measures $d^{\pi^k}$} which are small enough to ensure sublinear regret. To see this (informally) consider two cases. First, let us assume that $d^{\pi^k}$ is relatively large for some action pair. Then, that action pair is expected to be visited often in the rollouts and therefore $\delta^k$ is expected to be small. Vice versa, if $\delta^k$ for a certain state-action pair is large, this means that for that state-action pair $d^{\pi^k}$ is relatively small. Overall, we always expect the product $\innerprod{d^{\pi^k}}{\delta^k}$ to be a small quantity.
Notice that the same arguments could not have been carried out replacing $d^{\pi^k}$ with $d^{\pi^\star}$ because the rollouts used in \Cref{alg:theory_version} are not sampled with $\pi^\star$.
\end{remark}
To formalize the above intuition, we upper bound the temporal difference errors with the inverse of the number of times each state-action pair is visited.
\begin{lemma} (\textbf{Simplified Version of \Cref{lemma:bounded_optimism_ensemble}} )
Let us consider a binarized MDP with $\abs{\sspace}$ states and discount factor $\gamma$. With probability $1-\delta$, it holds that for all $s,a\in \sspace\times \aspace$ and for all $k \in [K]$,
\begin{align*}
\delta^k(s,a) &\leq \widetilde{\mathcal{O}}\br{\sqrt{\frac{L\abs{\sspace} \log (1/\delta)}{(N^k(s,a) + 1) (1-\gamma)^2}}}.
\end{align*}
\end{lemma}
Therefore, by concentration inequalities and noticing that $s^k_{L^k}, a^k_{L^k} \sim d^{\pi^k}$, it holds that with high probability
\begin{align*}
\sum^K_{k=1}&\innerprod{d^{\pi^k}}{\delta^k} = \widetilde{\mathcal{O}}\br{\sum^K_{k=1} \delta^k(s^k_{L^k}, a^k_{L^k})} \\
&\leq \widetilde{\mathcal{O}}\br{\sum^K_{k=1}\sqrt{\frac{L\abs{\sspace} \log (1/\delta)}{(N^k(s^k_{L^k},a^k_{L^k}) + 1) (1-\gamma)^2}}} \\
&\leq \widetilde{\mathcal{O}}\br{\sqrt{K\sum^K_{k=1}\frac{L\abs{\sspace} \log (1/\delta)}{(N^k(s^k_{L^k},a^k_{L^k}) + 1) (1-\gamma)^2}}}.
\end{align*}
At this point, the proof is concluded by bounding the last sum over $K$ with a standard numerical sequences argument (see \Cref{lemma:count_based}).
\subsection{Proof Sketch of \Cref{thm:reward_regret_bound}}
The proof of this term is considerably easier than the bound of the regret for the policy player because we have exact knowledge of the decision variables domain \footnote{$\mathcal{C}$ is taken to be the $\ell_{\infty}$-ball of radius $1$}.
The first step in the proof is to decompose $(1-\gamma)\mathrm{Regret}_c$ as follows
\begin{align*}
&\sum^K_{k=1} \innerprod{c_{\mathrm{true}} - c^k}{\widehat{d^{\pi^k}} - \widehat{d^\expert}}
%\\&\phantom{=}
+\sum^K_{k=1} \innerprod{c_{\mathrm{true}} - c^k}{ d^{\pi^k}- \widehat{d^{\pi^k}}} \\
&\phantom{=}+\sum^K_{k=1} \innerprod{c_{\mathrm{true}} - c^k}{ \widehat{d^\expert} - d^\expert}.
\end{align*}
The first term in the decomposition is upper bounded by $\mathcal{O}(\sqrt{K})$ via a standard online gradient descent analysis \cite{Zinkevich2003}.
Since $\widehat{d^{\pi^k}}$ is an unbiased estimate of the learner occupancy measure, the second term in the decomposition is the sum of a martingale difference sequence. Therefore, an application of the Azuma-Hoeffding inequality ensures that this term grows as $\widetilde{\mathcal{O}}\br{\log(1/\delta)\sqrt{K}}$ with probability at least $1-\delta$.

Finally, the last term is bounded as $\widetilde{\mathcal{O}}\br{K \log(1/\delta)\abs{\mathcal{D}_\expert}^{-1/2}}$ with probability at least $1-\delta$. This is done, proving that for the empirical average estimators for the expert occupancy measure it holds that $\norm{d^\expert-\widehat{d^\expert}}_1 \leq \abs{\mathcal{D}_\expert}^{-1/2} \log(1/\delta)$ with probability at least $1-\delta$. A union bound concludes the proof of \Cref{thm:reward_regret_bound}. The formal proof is deferred to the Appendix.
%
\section{SOAR for continuous state and actions problems.}
In this section, we explain how \Cref{alg:meta} is instantiated in imitation learning problems with continuous states and action spaces, which therefore requires neural networks to approximate the value function and policy updates.
Since in our analysis for the tabular case, we need to use multiplicative weights/softmax updates, we decided to use SAC, which is an approximation of such updates in the continuous state-action setting.

However, the standard SAC keeps only one network, often called the critic network, to estimate the $Q$ values.
On the other hand, we use a pair of them to avoid the excessive overestimation noticed in Double DQN \cite{vanhasselt2015deepreinforcementlearningdouble}.
Since it uses only one pair of critics, SAC cannot achieve optimism reliably with high probability. 

To fix this issue, we consider multiple critics and we used as an optimistic estimate the mean minus the standard deviation of the ensemble as explained in \Cref{alg:optQnn}. In addition, the standard deviation needs to be truncated at a threshold, as was done in the tabular analysis, to avoid the value function estimators growing out of the attainable range.
For any state $s$, the attainable range for the state value function is $\bs{0, (1-\gamma)^{-1} }$.
\begin{algorithm}
\caption{\textsc{OptimisticQ-NN} \label{alg:optQnn}}
\begin{algorithmic}[1]
\REQUIRE Replay buffer $\mathcal{D}$, 
Estimators $\bc{Q_{\ell}}^L_{\ell=1}$, %current policy $\pi$, 
 maximum standard deviation $\sigma$.
    \State $\bc{s_i}^N_{i=1} \gets$ sample observations from $\mathcal{D}$
    \State $a_i \gets \pi(s_i)$
    % \State $q_{\min} \gets$ empty list
    %\For{$i = 1$ to $L$}
    %    \State $q_{\min}[i] \gets \min \left( Q^{(1)}[i](o, a), Q^{(2)}[i](o, a) \right)$
    %\EndFor
    \State $\bar{Q}(s_i,a_i) = \frac{1}{L} \sum_{\ell=1}^L Q_\ell (s_i,a_i)  $
    \State $\text{std-Q}(s_\ell,a_\ell) = \sqrt{\frac{1}{L} \sum_{\ell=1}^L \left( Q_\ell (s_i,a_i) - \bar{Q}(s_i,a_i) \right)^2}$
    \State $\overline{\text{std-Q}}(s_i,a_i) \gets \text{Clip}(\text{std-Q}(s_i,a_i), 0, \sigma)$.
    \State $Q(s_i,a_i) = \bar{Q}(s_i,a_i) - \overline{\text{std-Q}}(s_i,a_i)$
    \State \textbf{Return:}  $Q(s_i,a_i)$ for all $i=1, \dots, N$.
    %\State Update policy weights using Adam \cite{Kingma:2015} on the loss $\mathcal{L}_\pi$.
\end{algorithmic}
\end{algorithm}
Each of the estimators (critics) $\bc{Q_\ell}^L_{\ell=1}$ is trained in the same way (minimizing the squared Bellman error as in standard SAC ) on a different dataset collected by the same actor. That is, on independent identically distributed datasets. For completeness, the SAC critic training is included in \Cref{alg:updateCritics} in \Cref{app:pseudo}.
In the continuous setting, it is clearly not possible to compute the optimistic state-action value at every state-action pair. Thankfully, it suffices to compute the optimistic state action  value function $Q$, invoking the routine $\textsc{OptimisticQ-NN}$, only for the state-actions in a minibatch $\mathcal{D} = \bc{s_i,a_i}^N_{i=1}$.
Indeed, the policy network weights does not require perfect knowledge of $Q$ over $\sspace\times\aspace$ but only an Adam \cite{Kingma:2015} update step on the loss
$\mathcal{L}_\pi = \frac{1}{N} \sum_{i=1}^N \left( -\eta \log \pi(a_i|s_i) + Q(s_i,a_i)  \right).
$

%Finally, to complete the description of an imitation learning algorithm we need to specify an instance for $\textsc{UpdateCost}$ in \Cref{alg:meta}.
%For that part, we do not introduce any novelty. Instead, we will show in the experiment section 
In the next section, we show that for multiple choices of $\textsc{UpdateCost}$ (ML-IRL, CSIL and RKL) replacing the standard SAC critic update routine with $\textsc{OptimisticQ-NN}$ leads to improved performance.

\begin{figure*}[t]
    \centering
\includegraphics[width=\textwidth]{final_figs/state_only16.png}
    \caption{%Comparison of state-only imitation learning methods across OpenAI Gym environments, with dashed lines showing baseline algorithms and solid lines showing SOAR-enhanced versions. 
    \small{\textbf{Experiments from State-Only Expert Trajectories}. 16 expert trajectories, average over 5 seeds, $L=4$ 
    Clipping values $\sigma$ - ML-IRL: [Ant: 10.0, Hopper: 50.0, Walker2d: 0.5, Humanoid: 5.0], rkl: [Ant: 0.8, Hopper: 50.0, Walker2d: 30.0, Humanoid: 100.0]}}
    \label{fig:state_only}
\end{figure*}

\begin{figure*}[t]
    \centering
\includegraphics[width=\textwidth]{final_figs/state_actions16.png}
    \caption{%Performance of state-action imitation learning methods across environments, with dashed lines showing baseline algorithms and solid lines showing SOAR-enhanced versions. 
    \small{\textbf{Experiments from State-Action Expert Trajectories}. 16 expert trajectories, average over 5 seeds, $L=4$.
    Clipping values $\sigma$ - CSIL: [Ant: 10.0, Hopper: 5.0, Walker2d: 0.5, Humanoid: 0.1], ML-IRL(SA): [Ant: 5.0, Hopper: 10.0, Walker2d: 0.5, Humanoid: 50.0]}}
\label{fig:state_actions}
\end{figure*}
\section{Experiments}
%We show how the improvements in the exploration mechanism of Soft Actor Critic, that is using multiple critic and an optimistic aggregation mechanism improves significantly the performance of all the aforementioned IL algorithms building on SAC for the policy update. 
We perform experiments for both state only and state action IL on the following MuJoCo \cite{Todorov:2012} environments: Ant, Hopper, Walker2d, and Humanoid.

For the state-only IL setting, we showcase the improvement on RKL \cite{ni2021f} and ML-IRL (State-Only) \cite{zeng2022maximum}. In both cases, we found that using $L=4$ critic networks and an appropriately chosen value for the standard deviation clipping threshold $\sigma$ consistently improves upon the baseline.
In the Appendix \ref{sec:training_proc}, we conduct an ablation study for $L$ and $\sigma$.

We denote our derived algorithms as RKL+SOAR and ML-IRL+SOAR. In addition to observing an improvement over standard RKL and ML-IRL, we outperform the state-only version of the recently introduced OPT-AIL algorithm \cite{xu2024provably} (see Figure~\ref{fig:state_only}) which incorporates an alternative, more complicated, deep exploration technique. 

For the state-action experiments, we plug in the SOAR template on CSIL and the state-action version of ML-IRL. We coined the derived versions CSIL+SOAR and ML-IRL+SOAR (see Appendix~\ref{app:pseudo} for detailed pseudocodes of these algorithms). We also compare with GAIL \cite{Ho:2016b}, SQIL \cite{sqil}, and OPT-AIL. We observe that the exploration mechanism injected by the SOAR principle allows us to achieve reliably superior results (see Figure~\ref{fig:state_actions}).

Further details about the hyperparameters are provided in the Appendix \ref{sec:training_proc}. Moreover, we notice that for all the algorithms in the higher-dimensional and thus more challenging environments (Ant-v5 and Humanoid-v5), the advantage of the SOAR exploration technique becomes more evident. 

The experts trajectory are obtained from policy networks trained via SAC. The expert returns are reported in Table \ref{tab:performance}.
\begin{table}
    \centering
    \setlength{\tabcolsep}{4pt}
    \caption{\small{Expert returns}}\resizebox{0.45\textwidth}{!}{\begin{tabular}{c|c|c|c|c}
        Method & Ant-v5 & Hopper-v5 & Humanoid-v5 & Walker2d-v5 \\
        \hline
        Expert & 4061.41 & 3500.87 & 5237.48 & 5580.39 \\
        return & $\pm$ 730.58 & $\pm$ 4.33 & $\pm$ 414.69 & $\pm$ 20.30 \\
    \end{tabular}}
    \label{tab:performance}
\end{table}

\section{Conclusions and Open Questions}
While there has been interest in developing heuristically effective exploration techniques in deep RL, the same is not true for deep IL.
For example, even in the detailed study \emph{What matters in Adversarial Imitation Learning ?} \cite{orsini2021matters} the effectiveness of deep exploration techniques is not investigated.
Prior to our work, only few studied the benefits of exploration in imitation learning, mostly in the state-only regime \cite{kidambi2021mobile}. However, their theoretical algorithm uses bonuses that cannot be implemented with neural networks.
Similarly, the recent work \cite{xu2024provably} uses exploration technique in Deep IL but requires solving a complicated non-concave maximization problem.
Our approach is remarkably easier to implement. It achieves convincing empirical results results and enjoys theoretical guarantees.
%In addition, notice that it is likely that many future Deep IL algorithms will be based on SAC for the policy updates. It is reasonable to expect an improvement if for those algorithms SAC is modified using multiple critics as proposed in this paper
Moreover, our framework can be expected to be beneficial for any existing or future deep IL algorithm using SAC for policy updates.


\paragraph{Open Questions}
%There are several open questions to investigate in future work.
On the theoretical side, we plan to analyze the ensemble exploration technique in the linear MDP case.
From the practical one, we will investigate if the exploration enhanced versions of DQN \cite{osband2016deep,osband2018randomized} can speed up imitation learning from visual input.
Finally, the same idea might find application in the LLM finetuning given the recently highlighted potential of IL for this task \cite{wulfmeier2024imitating,foster2024behavior}.
\newpage
%\section*{Impact Statement}
%The current submission is expected to have an impact in the imitation learning community particularly because it highlights the benefits of exploration, not only theoretically but also in simulated robotics experiments.
%Impact can be expected also in the broader machine learning community and in related disciplines such as robotics and control theory.
%Beyond that, we do not expect direct impact on the society (i.e. outside the machine learning community and IT industry ).
\section*{Acknowledgments}
This work is funded (in part) through a PhD fellowship of the Swiss Data Science Center, a joint venture between EPFL and ETH Zurich.
This work was supported by Hasler Foundation Program: Hasler Responsible AI (project number 21043). Research was sponsored by the Army Research Office and was accomplished under Grant Number W911NF-24-1-0048. This work was supported by the Swiss National Science Foundation (SNSF) under grant number 200021\_205011.
\bibliography{sample, example_paper}
\bibliographystyle{icml2025}


%%%%%%%%%%%%%%%%%%%%%%%%%%%%%%%%%%%%%%%%%%%%%%%%%%%%%%%%%%%%%%%%%%%%%%%%%%%%%%%
%%%%%%%%%%%%%%%%%%%%%%%%%%%%%%%%%%%%%%%%%%%%%%%%%%%%%%%%%%%%%%%%%%%%%%%%%%%%%%%
% APPENDIX
%%%%%%%%%%%%%%%%%%%%%%%%%%%%%%%%%%%%%%%%%%%%%%%%%%%%%%%%%%%%%%%%%%%%%%%%%%%%%%%
%%%%%%%%%%%%%%%%%%%%%%%%%%%%%%%%%%%%%%%%%%%%%%%%%%%%%%%%%%%%%%%%%%%%%%%%%%%%%%%
\newpage
\appendix
\onecolumn
\section{Related Works}
\putsec{related}{Related Work}

\noindent \textbf{Efficient Radiance Field Rendering.}
%
The introduction of Neural Radiance Fields (NeRF)~\cite{mil:sri20} has
generated significant interest in efficient 3D scene representation and
rendering for radiance fields.
%
Over the past years, there has been a large amount of research aimed at
accelerating NeRFs through algorithmic or software
optimizations~\cite{mul:eva22,fri:yu22,che:fun23,sun:sun22}, and the
development of hardware
accelerators~\cite{lee:cho23,li:li23,son:wen23,mub:kan23,fen:liu24}.
%
The state-of-the-art method, 3D Gaussian splatting~\cite{ker:kop23}, has
further fueled interest in accelerating radiance field
rendering~\cite{rad:ste24,lee:lee24,nie:stu24,lee:rho24,ham:mel24} as it
employs rasterization primitives that can be rendered much faster than NeRFs.
%
However, previous research focused on software graphics rendering on
programmable cores or building dedicated hardware accelerators. In contrast,
\name{} investigates the potential of efficient radiance field rendering while
utilizing fixed-function units in graphics hardware.
%
To our knowledge, this is the first work that assesses the performance
implications of rendering Gaussian-based radiance fields on the hardware
graphics pipeline with software and hardware optimizations.

%%%%%%%%%%%%%%%%%%%%%%%%%%%%%%%%%%%%%%%%%%%%%%%%%%%%%%%%%%%%%%%%%%%%%%%%%%
\myparagraph{Enhancing Graphics Rendering Hardware.}
%
The performance advantage of executing graphics rendering on either
programmable shader cores or fixed-function units varies depending on the
rendering methods and hardware designs.
%
Previous studies have explored the performance implication of graphics hardware
design by developing simulation infrastructures for graphics
workloads~\cite{bar:gon06,gub:aam19,tin:sax23,arn:par13}.
%
Additionally, several studies have aimed to improve the performance of
special-purpose hardware such as ray tracing units in graphics
hardware~\cite{cho:now23,liu:cha21} and proposed hardware accelerators for
graphics applications~\cite{lu:hua17,ram:gri09}.
%
In contrast to these works, which primarily evaluate traditional graphics
workloads, our work focuses on improving the performance of volume rendering
workloads, such as Gaussian splatting, which require blending a huge number of
fragments per pixel.

%%%%%%%%%%%%%%%%%%%%%%%%%%%%%%%%%%%%%%%%%%%%%%%%%%%%%%%%%%%%%%%%%%%%%%%%%%
%
In the context of multi-sample anti-aliasing, prior work proposed reducing the
amount of redundant shading by merging fragments from adjacent triangles in a
mesh at the quad granularity~\cite{fat:bou10}.
%
While both our work and quad-fragment merging (QFM)~\cite{fat:bou10} aim to
reduce operations by merging quads, our proposed technique differs from QFM in
many aspects.
%
Our method aims to blend \emph{overlapping primitives} along the depth
direction and applies to quads from any primitive. In contrast, QFM merges quad
fragments from small (e.g., pixel-sized) triangles that \emph{share} an edge
(i.e., \emph{connected}, \emph{non-overlapping} triangles).
%
As such, QFM is not applicable to the scenes consisting of a number of
unconnected transparent triangles, such as those in 3D Gaussian splatting.
%
In addition, our method computes the \emph{exact} color for each pixel by
offloading blending operations from ROPs to shader units, whereas QFM
\emph{approximates} pixel colors by using the color from one triangle when
multiple triangles are merged into a single quad.


\if 0
\section{Background and Notation}

In imitation learning, the environment is abstracted as a Markov Decision Process (MDP) which consists of a tuple $(S, \mathcal{A}, P, c, \nu_0)$ where $S$ is the state space, $\mathcal{A}$ is the action space, $P : S \times \mathcal{A} \to \Delta_S$ is the transition kernel, that is, $P(s'|s, a)$ denotes the probability of landing in state $s'$ after choosing action $a$ in state $s$. Moreover, $\nu_0$ is a distribution over states from which the initial state is sampled. Finally, $c : S \times \mathcal{A} \to [-1, 1]$ is the cost function. In the infinite horizon setting, we endow the MDP tuple with an additional element called the discount factor $\gamma \in [0, 1]$. 
The agent plays action in the environment sampled from a policy $\pi : \mathcal{S} \to \Delta_\mathcal{A}$. The learner is allowed to adopt an algorithm to update the policy across episodes given the previously observed history.

\section*{Value functions and occupancy measures}
We define the state value function at state $s \in \mathcal{S}$ for the policy $\pi$ under the reward function $r$ as $V^\pi(s;c) \triangleq \mathbb{E} \left[\sum_{h=0}^{\infty} \gamma^h c(s_h,a_h)|s_1 = s\right]$.  The expectation over both the randomness of the transition dynamics and the one of the learner's policy.

Another convenient quantity is the occupancy measure of a policy $\pi$ denoted as $d^\pi \in \Delta_{\mathcal{S}\times\mathcal{A}}$ and defined as follows 

$$d^\pi(s,a) \triangleq (1-\gamma)\sum_{h=0}^{\infty} \gamma^h\mathbb{P}[s,a \text{ is visited after } h \text{ steps acting with } \pi]$$

We can also define the state occupancy measure as 

$$d^\pi(s) \triangleq (1-\gamma)\sum_{h=0}^{\infty} \gamma^h\mathbb{P}[s \text{ is visited after } h \text{ steps acting with } \pi]$$. 

\section*{Imitation Learning}
In imitation learning, the learner is given a dataset $\mathcal{D}_{\tau_E} \triangleq \{\tau^k\}_{k=1}^{\tau_E}$ containing $\tau_E$ trajectories collected in the MDP by an expert policy $\pi_E$ according to Algorithm 1. By trajectory $\tau^k$, we mean the sequence of states and actions sampled at the $k^{th}$ iteration of Algorithm 1, that is $\tau^k = \{(s_h^k,a_h^k)\}_{h=1}^H$ for finite horizon case. For the infinite horizon case, the trajectories have random lenght sampled from the distribution Geometric$(1-\gamma)$. Given $\mathcal{D}_{\tau_E}$, the learner adopts an algorithm $\mathcal{A}$ to learn a policy $\pi^{\text{out}}$ such that is $\epsilon$-suboptimal according to the next definition.



\begin{definition}
An algorithm $\mathcal{A}$ is said $\epsilon$-suboptimal if it outputs a policy $\pi$ whose value function with respect to the unknown true cost $c_{\text{true}}$ satisfies $\mathbb{E}_\mathcal{A}\mathbb{E}_{s_1\sim\nu_0}[V^\pi(s_1;c_{\text{true}}) - V^{\pi_E}(s_1;c_{\text{true}})] \leq \epsilon$ where the first expectation is on the randomness of the algorithm $\mathcal{A}$.
\end{definition}



The justification for the update of Q in Algorithm \ref{alg:estimateQ} is that we want our $Q^k(s,a)$ to be a lower bound of $Q(s,a)$. Thus we need to find a function $b(s,a)$ such that:
\[
- b(s,a) \leq PV^{k}(s,a) - z^k(s,a) \leq b(s,a)
\]
Where $z(s,a)$ is our estimate of $PV^{k}(s,a)$. So, $b(s,a)$ bounds the error of our estimate of $PV^{k}(s,a)$. If we subtract it from our estimate of $Q(s,a)$ we get the following:

\[
Q^{k}(s,a) = r(s,a) + z^k(s,a) + b(s,a)
\]
And by considering the definition of $Q(s,a)$:
\[
Q(s,a) = r(s,a) + PV^{k}(s,a)
\]
We can show that $Q^{k}(s,a)$ is a upper bound for $Q(s,a)$.
\[
Q(s,a) - Q^{k}(s,a) = PV^{k}(s,a) - z^k(s,a) - b(s,a) \leq b(s,a) - b(s,a) = 0
\]

In order to implement it with Neural network we need a way to estimate $b(s,a)$. In the Linear implementation of the algorithm $b(s, a)$ was estimated using the feature vector which we can't use in this case. The idea was to have multiple neural network and estimate z by taking the average of this NNs and b as their std. 
\fi



% \begin{algorithm}[H]
% \caption{UpdateZNetwork Function}
% \begin{algorithmic}[1]
% \Function{UpdateZNetwork}{$z_k, \tau, r, \gamma, \eta$}
%     \State $s, a \gets$ states and actions from $\tau$
%     \State $R \gets r(s, a)$ \Comment{Estimate rewards using reward network}
%     \State Update $z_k$ parameters $\theta_Z^k$ according to:
%     \State $\theta_Z^{k+1} = \theta_Z^k - \eta \nabla \frac{1}{N} \sum_{i=1}^N \left( z_k(s_i, a_i) -  \gamma \sum_{j=i+1}^N \gamma^{j-i-1} R(s_j, a_j) \right)^2$
%     \State \Return $z_k$
% \EndFunction
% \end{algorithmic}
% \end{algorithm}

% \begin{algorithm}[H]
% \caption{EstimateQ Function with Clipped Standard Deviation}
% \label{alg:estimateQ_clipped}
% \begin{algorithmic}[1]
% \Function{EstimateQ}{$r, \{z_1, \ldots, z_K\}, \tau, \text{clip\_min}, \text{clip\_max}$}
%     \State $s, a \gets$ states and actions from $\tau$
%     \State $\bar{z}(s,a) \gets \frac{1}{K}\sum_{k=1}^K z_k(s,a)$
%     \State $\sigma_z(s,a) \gets \sqrt{\frac{1}{K}\sum_{k=1}^K (z_k(s,a) - \bar{z})^2}$
%     \State $\sigma_z(s,a) \gets \min(\text{clip\_max}, \max(\text{clip\_min}, \sigma_z(s,a)))$ \Comment{Clipping $\sigma_z$}
%     \State \Return $r(s,a) + \bar{z}(s,a) + \sigma_z(s,a)$  \Comment{$\bar{z}$ serves as the 'b' term}
% \EndFunction
% \end{algorithmic}
% \end{algorithm}


% \begin{algorithm}[H]
% \caption{Update Policy Function}
% \begin{algorithmic}[1]
% \Function{UpdatePolicy}{$\pi, Q, \eta$}
%     \State $\nabla_\theta J(\theta) = \mathbb{E}_{s \sim \mathcal{D}} [\nabla_\theta \langle \pi_\theta(\cdot|s), \eta Q(s,\cdot) + \log \frac{\pi_\theta(\cdot|s)}{\pi(\cdot|s)} \rangle]$
%     \State Update $\theta \gets \theta + \nabla_\theta J(\theta)$
%     \State \Return $\pi_\theta$
% \EndFunction
% \end{algorithmic}
% \end{algorithm}

%In the actual implementation of the algorithm there are a few changes from what we see above. Specifically, we used the f-irl implementation. The main differences are in the update of the Q function where we used memory replay, a target network and a double q network the reduce the overestimation of Q. The only other difference from f-irl is in the update of the reward network where the f-divergence is used. TODO. 

\if 0
\subsection{Plots of baseline and best clipping values}

It works, plots from different Mujoco envs performing grid search on the clipping value. The following plots only have the baseline (one neural network without adding the standard deviation) and the clipping value which performs the best. For all the following plots we only use 4 neural networks. 

\section{f-irl}

\begin{figure}[H]
    \centering
    \includegraphics[width=\linewidth]{figs/grid_search_clip/Hopper-v5_average_real_det_return_1_4_without_conf_int_10.0.png}
    \caption{}
    \label{fig:enter-label}
\end{figure}

\begin{figure}[H]
    \centering
    \includegraphics[width=\linewidth]{figs/grid_search_clip/Walker2d-v5_average_real_det_return_1_4_without_conf_int_10.0.png}
    \caption{}
    \label{fig:enter-label}
\end{figure}

\begin{figure}[H]
    \centering
    \includegraphics[width=\linewidth]{figs/grid_search_clip/Ant-v5_average_real_det_return_1_4_without_conf_int_0.5.png}
    \caption{In other you can see the difference much more}
    \label{fig:enter-label}
\end{figure}

\begin{figure}[H]
    \centering
    \includegraphics[width=\linewidth]{figs/grid_search_clip/Humanoid-v5_average_real_det_return_1_4_without_conf_int_100.0.png}
    \caption{In other you can see the difference much more}
    \label{fig:enter-label}
\end{figure}

\begin{figure}[H]
    \centering
    \includegraphics[width=\linewidth]{figs/grid_search_clip/HalfCheetah-v5_average_real_det_return_1_4_without_conf_int_500.0.png}
    \caption{In this case we aren't able to replicate the results of the paper. Probably the right hyperparameter wasn't the one used in the final experiments. So we can see that the improvement by adding the right clipping value is much larger.}
    \label{fig:enter-label}
\end{figure}



\section{Maximum-Likelihood Inverse Reinforcement Learning}

\begin{figure}[H]
    \centering
    \includegraphics[width=\linewidth]{figs/grid_search_maxentirl/Hopper-v5_average_real_det_return_1_4_without_conf_int_50.0.png}
    \caption{}
    \label{fig:enter-label}
\end{figure}

\begin{figure}[H]
    \centering
    \includegraphics[width=\linewidth]{figs/grid_search_maxentirl/Walker2d-v5_average_real_det_return_1_4_without_conf_int_10.0.png}
    \caption{}
    \label{fig:enter-label}
\end{figure}

\begin{figure}[H]
    \centering
    \includegraphics[width=\linewidth]{figs/grid_search_maxentirl/Ant-v5_average_real_det_return_1_4_without_conf_int_10.0.png}
    \caption{}
    \label{fig:enter-label}
\end{figure}

\begin{figure}[H]
    \centering
    \includegraphics[width=\linewidth]{figs/grid_search_maxentirl/Humanoid-v5_average_real_det_return_1_4_without_conf_int_5.0.png}
    \caption{}
    \label{fig:enter-label}
\end{figure}

\begin{figure}[H]
    \centering
    \includegraphics[width=\linewidth]{figs/grid_search_maxentirl/average_real_det_return_4.0_500.0.png}
    \caption{}
    \label{fig:enter-label}
\end{figure}


\section{CSIL}


\begin{figure}[H]
    \centering
    \includegraphics[width=\linewidth]{figs/grid_search_CSIL/ant_average_real_det_return_4.0_without_conf_int_10.0.png}
    \caption{}
    \label{fig:enter-label}
\end{figure}

\begin{figure}[H]
    \centering
    \includegraphics[width=\linewidth]{figs/grid_search_CSIL/hopper_average_real_det_return_4.0_without_conf_int_10.0.png}
    \caption{}
    \label{fig:enter-label}
\end{figure}

\begin{figure}[H]
    \centering
    \includegraphics[width=\linewidth]{figs/grid_search_CSIL/humanoid_ average_real_det_return_4.0_0.1.png}
    \caption{}
    \label{fig:enter-label}
\end{figure}

\begin{figure}[H]
    \centering
    \includegraphics[width=\linewidth]{figs/grid_search_CSIL/walker_average_real_det_return_4.0_0.5.png}
    \caption{}
    \label{fig:enter-label}
\end{figure}

\subsection{maxentirl SA}

\begin{figure}[H]
    \centering
    \includegraphics[width=\linewidth]{figs/grid_search_maxentirl_sa/Ant-v5_average_real_det_return_4.0_5.0.png}
    \caption{}
    \label{fig:enter-label}
\end{figure}

\begin{figure}[H]
    \centering
    \includegraphics[width=\linewidth]{figs/grid_search_maxentirl_sa/HalfCheetah-v5_average_real_det_return_4.0_without_conf_int_50.0.png}
    \caption{}
    \label{fig:enter-label}
\end{figure}


\begin{figure}[H]
    \centering
    \includegraphics[width=\linewidth]{figs/grid_search_maxentirl_sa/Hopper-v5_average_real_det_return_4.0_without_conf_int_10.0.png}
    \caption{}
    \label{fig:enter-label}
\end{figure}

\begin{figure}[H]
    \centering
    \includegraphics[width=\linewidth]{figs/grid_search_maxentirl_sa/Humanoid-v5_average_real_det_return_4.0_without_conf_int_50.0.png}
    \caption{}
    \label{fig:enter-label}
\end{figure}

\begin{figure}[H]
    \centering
    \includegraphics[width=\linewidth]{figs/grid_search_maxentirl_sa/Walker2d-v5_average_real_det_return_4.0_without_conf_int_0.5.png}
    \caption{}
    \label{fig:enter-label}
\end{figure}


\subsection{maxentirl with dynamic clipping (first attempt)}

\begin{figure}[H]
    \centering
    \includegraphics[width=\linewidth]{figs/dynamic_clipping_first/Ant-v5_average_real_det_return_4.0_without_conf_int_0.1_0.5_1_5_10_50_100_500_1000.png}
    \caption{}
    \label{fig:enter-label}
\end{figure}

\begin{figure}[H]
    \centering
    \includegraphics[width=\linewidth]{figs/dynamic_clipping_first/HalfCheetah-v5_average_real_det_return_4.0_without_conf_int_0.1_0.5_1_5_10_50_100_500_1000.png}
    \caption{}
    \label{fig:enter-label}
\end{figure}

\begin{figure}[H]
    \centering
    \includegraphics[width=\linewidth]{figs/dynamic_clipping_first/Hopper-v5_average_real_det_return_4.0_without_conf_int_0.1_0.5_1_5_10_50_100_500_1000.png}
    \caption{}
    \label{fig:enter-label}
\end{figure}

\begin{figure}[H]
    \centering
    \includegraphics[width=\linewidth]{figs/dynamic_clipping_first/Humanoid-v5_average_real_det_return_4.0_without_conf_int_50.0.png}
    \caption{}
    \label{fig:enter-label}
\end{figure}

\begin{figure}[H]
    \centering
    \includegraphics[width=\linewidth]{figs/dynamic_clipping_first/Walker2d-v5_average_real_det_return_4.0_without_conf_int_0.1_0.5_1_5_10_50_100_500_1000.png}
    \caption{}
    \label{fig:enter-label}
\end{figure}
\fi
\if 0
\subsection{Why is clipping necessary?}

So we tested it on Mujoco, comparing different number of NNs to see if having a higher number would lead to better results. 

\begin{figure}[H]
    \centering
    \includegraphics[width=\linewidth]{figs/hopper_add_std_failed.jpeg}
    \caption{Mujoco Hopper with added std as described in Algorithm \ref{alg:estimateQ_clipped}, but without the clipping, with one neural network it does well but adding more doesn't give good performance. For the agent we're using the SAC agent. }
    \label{fig:enter-label}
\end{figure}



\noindent The above formulation doesn't work; in the following proof, we justify why the clipping is needed.

\noindent As we showed above, we have that:
\[
Q(s, a) \leq Q^k(s, a)
\]

We need the error of our estimation to be bounded:
\[
\| P V^k(s, a) - z_n(s, a) \|_\infty \leq b_n(s, a)
\]
which implies:
\[
-b(s, a) \leq P V^k(s, a) - z_n(s, a)
\]
\[
\implies z_n(s, a) \leq P V^k(s, a) + b_n(s, a).
\]

\noindent The update of the $q$-value is:
\[
Q^{k+1} = V^k + \gamma z_n + \gamma b_n
\]
\[
\leq V^k + \gamma P V^k + 2 \gamma b_n.
\]

\noindent We know that:
\[
\| Q^{k+1} \|_\infty \leq \frac{\| r \|_\infty}{1 - \gamma}  \tag{1}
\]
which is acceptable as long as $\gamma < 1$. If that is not the case, the $Q$ values might diverge or be bounded by a negative value, which doesn't make sense. 

\noindent In our case:
\begin{align*}
\| Q^{n+1} \|_\infty &\leq \| r_n \|_\infty + \gamma \| P V_n \|_\infty + 2 \gamma \| b_n \|_\infty \\
&\leq \| r_n \|_\infty + \gamma \| V_n \|_\infty + 2 \gamma \| V_n \|_\infty \\
&\leq \| r_n \|_\infty + 3 \gamma \| Q_n \|_\infty
\end{align*}

\noindent This implies:
\[
\| Q^n \|_\infty \leq \frac{\| r_n \|_\infty}{1 - 3 \gamma}
\]

\noindent Since:
\[
V_m(s) = \sum_a Q(s, a) \pi(s, a),
\]
we have:
\[
V_m(s) \leq \max_a Q(s, a).
\]

\noindent In this case, $\gamma < \frac{1}{3}$ in order for the above inequality to hold, but if that's the case the $Q$ would only consider a few steps into the future. If, on the other hand, $\gamma > \frac{1}{3}$ Q could diverge. 

\noindent To overcome the problem, we clip $b_n$:
\[
\| b_n \|_\infty \leq B_{\text{max}}.
\]

\noindent Thus:
\[
\| Q^{n+1} \|_\infty \leq \| r_n \|_\infty + 2 \delta B_{\text{max}} + \delta \| Q^{k} \|_\infty.
\]

\noindent We can write:
\[
\| r_n \|_\infty + 2 \delta B_{\text{max}} = \| r_n' \|_\infty,
\]
thus returning to (1) where $\gamma < 1$.


This shows why the clipping of $b$ is necessary. 
\fi






%\subsection{Final results with all the clipping values we tried}

% \begin{figure}[H]
%     \centering
%     \includegraphics[width=\linewidth]{figs/grid_search_clip/Ant-v5_average_real_det_return_1_4_without_conf_int_0.1_0.5_0.8_1.0_2.0_3.0_5.0_10.0_20.0_50.0_100.0.png}
%     \caption{}
%     \label{fig:enter-label}
% \end{figure}

% \begin{figure}[H]
%     \centering
%     \includegraphics[width=\linewidth]{figs/grid_search_clip/HalfCheetah-v5_average_real_det_return_1_4_without_conf_int_0.1_0.5_0.8_1.0_2.0_3.0_5.0_10.0_20.0_50.0_100.0_200.0_300.0_500.0_1000.0.png}
%     \caption{}
%     \label{fig:enter-label}
% \end{figure}

% \begin{figure}[H]
%     \centering
%     \includegraphics[width=\linewidth]{figs/grid_search_clip/Hopper-v5_average_real_det_return_1_4_without_conf_int_1.0_2.0_3.0_5.0_10.0_20.0_50.0.png}
%     \caption{}
%     \label{fig:enter-label}
% \end{figure}


% \begin{figure}[H]
%     \centering
%     \includegraphics[width=\linewidth]{figs/grid_search_clip/Walker2d-v5_average_real_det_return_1_4_without_conf_int_0.1_0.5_0.8_1.0_2.0_3.0_5.0_10.0_20.0_50.0_100.0.png}
%     \caption{}
%     \label{fig:enter-label}
% \end{figure}

%\section{Additional material}


% \subsection{10 seed, memory replay}

% \begin{figure}[H]
%     \centering
%     \includegraphics[width=1\linewidth]{figs/10_seeds_memory_basic/Average true reward over time for different NN_10_seeds.png}
    
%     \caption{10 seeds, average mean True reward across run with the same number of NNs at the same episode}
%     \label{fig:enter-label}
% \end{figure}

% \begin{figure}[H]
%     \centering
%     \includegraphics[width=\linewidth]{figs/10_seeds_memory_basic/avg_max_true_reward_vs_num_of_NNs_normal_10_seeds.png}
%     \caption{10 seeds, for each run (with a certain seed and number of NN) we take the episode with the maximum mean true reward. We then average this maximum across the runs with the same number of NNs}
%     \label{fig:enter-label}
% \end{figure}

% \begin{figure}[H]
%     \centering
%     \includegraphics[width=\linewidth]{figs/10_seeds_memory_basic/avg_mean_true_reward_vs_num_of_NNs_normal_10_seeds.png}
%     \caption{10 seeds, for each run (with a certain seed and number of NN) we take the average true reward of all the episodes. We then average this averages across the runs with the same number of NNs. }
%     \label{fig:enter-label}
% \end{figure}

% In the two plots above it seems that with one NN the performance is worse, but the confidence intervals are quite big. The run above are all done without recomputing the reward every time (with the update reward network. 

% \subsection{20 seed, memory replay}

% \begin{figure}[H]
%     \centering
%     \includegraphics[width=1\linewidth]{figs/20_seeds_runs_memory_replay/Average true reward over time for different NN counts CI.png}
%     \caption{20 seeds, average mean True reward across run with the same number of NNs at the same episode}
%     \label{fig:enter-label}
% \end{figure}

% \begin{figure}[H]
%     \centering
%     \includegraphics[width=\linewidth]{figs/20_seeds_runs_memory_replay/avg_max_true_reward_vs_num_of_NNs.png}
%     \caption{20 seeds, for each run (with a certain seed and number of NN) we take the episode with the maximum mean true reward. We then average this maximum across the runs with the same number of NNs}
%     \label{fig:enter-label}
% \end{figure}

% \begin{figure}[H]
%     \centering
%     \includegraphics[width=\linewidth]{figs/20_seeds_runs_memory_replay/avg_mean_true_reward_vs_num_of_NNs.png}
%     \caption{20 seeds, for each run (with a certain seed and number of NN) we take the average true reward of all the episodes. We then average this averages across the runs with the same number of NNs. }
%     \label{fig:enter-label}
% \end{figure}


% \subsection{z std multiplied by 10}

% \begin{figure}[H]
%     \centering
%     \includegraphics[width=1\linewidth]{figs/runs_memory_replay_zmul/Average true reward over time for different NN counts CI.png}
%     \caption{10 seeds, average mean True reward across run with the same number of NNs at the same episode, z std is multiplied by 10}
% \end{figure}


% \begin{figure}[H]
%     \centering
%     \includegraphics[width=\linewidth]{figs/runs_memory_replay_zmul/avg_max_true_reward_vs_num_of_NNs.png}
%     \caption{10 seeds, for each run (with a certain seed and number of NN) we take the episode with the maximum mean true reward. We then average this maximum across the runs with the same number of NNs. z std is multiplied by 10}
%     \label{fig:enter-label}
% \end{figure}

% \begin{figure}[H]
%     \centering
%     \includegraphics[width=\linewidth]{figs/runs_memory_replay_zmul/avg_mean_true_reward_vs_num_of_NNs.png}
%     \caption{10 seeds, for each run (with a certain seed and number of NN) we take the average true reward of all the episodes. We then average this averages across the runs with the same number of NNs. z std is multiplied by 10}
%     \label{fig:enter-label}
% \end{figure}

% \subsection{Recomputing reward every time we fetch from replay buffer}

% \begin{figure}[H]
%     \centering
%     \includegraphics[width=1\linewidth]{figs/runs_memory_replay_recompute/Average true reward over time for different NN counts CI.png}
%     \caption{20 seeds, average mean True reward across run with the same number of NNs at the same episode - the reward is recomputed on the state-action pair every time we retrieve it from the buffer}
% \end{figure}


% \begin{figure}[H]
%     \centering
%     \includegraphics[width=\linewidth]{figs/runs_memory_replay_recompute/avg_max_true_reward_vs_num_of_NNs.png}
%     \caption{20 seeds, for each run (with a certain seed and number of NN) we take the episode with the maximum mean true reward - the reward is recomputed on the state-action pair every time we retrieve it from the buffer}
%     \label{fig:enter-label}
% \end{figure}

% \begin{figure}[H]
%     \centering
%     \includegraphics[width=\linewidth]{figs/runs_memory_replay_recompute/avg_mean_true_reward_vs_num_of_NNs.png}
%     \caption{20 seeds, for each run (with a certain seed and number of NN) we take the average true reward of all the episodes. We then average this averages across the runs with the same number of NNs. - the reward is recomputed on the state-action pair every time we retrieve it from the buffer}
%     \label{fig:enter-label}
% \end{figure}





% We implemented another way to estimate q. Instead of trying to overestimate the q by adding the std we take the max over all the z network and use that as our overestimation. This removes the need to tune the hyperparameter of beta which multiplied the std. 

% \begin{algorithm}
% \caption{EstimateQ Function with Minimum Pair Q-values and Maximum Selection}
% \label{alg:estimateQ}
% \begin{algorithmic}[1]
% \Function{EstimateQ}{$r, \{(z_1, z_2), \ldots, (z_{2K-1}, z_{2K})\}, \tau, \text{uncertainty\_coef}$}
%     \State $s, a \gets$ states and actions from $\tau$
%     \State $q_{\text{min\_pairs}}(s,a) \gets \{\min(z_i(s,a), z_{i+1}(s,a)) \,|\, i \in \{1, 3, \ldots, 2K-1\}\}$
%     \State $q_{\text{max}}(s,a) \gets \max(q_{\text{min\_pairs}}(s,a))$
%     \State \Return $r(s,a) + q_{\text{max}}(s,a)$
% \EndFunction
% \end{algorithmic}
% \end{algorithm}


% \begin{figure}
%     \centering
%     \includegraphics[width=\linewidth]{figs/hopper_add_max_min.png}
%     \caption{Here we used \ref{alg:estimateQ} to estimate the bonus. }
%     \label{fig:enter-label}
% \end{figure}


% One of the main problem with the above configuration was that in the case of multiple NNs the std would sometime diverge and be the most important factor in the computation of the quality function. To overcome this we added a clipping of the std when using it to compute the loss of pi, below the run where the std was clipped at 1. The implementation of the estimation of the q value with clipping is reported in Algorithm \ref{alg:estimateQ_clipping}, the first results are reported in Figure \ref{fig:clipping_1}

% \begin{algorithm}
% \caption{EstimateQ Function with Mean and Exploration Bonus from Minimum Q-values}
% \label{alg:estimateQ_clipping}
% \begin{algorithmic}[1]
% \Function{EstimateQ}{$r, \{(z_1, z_2), \ldots, (z_{2K-1}, z_{2K})\}, \tau, \text{uncertainty\_coef}, \text{std\_clip}$}
%     \State $s, a \gets$ states and actions from $\tau$
%     \State $q_{\text{min\_pairs}}(s,a) \gets \{\min(z_i(s,a), z_{i+1}(s,a)) \,|\, i \in \{1, 3, \ldots, 2K-1\}\}$
%     \State $q_{\text{mean}}(s,a) \gets \frac{1}{|\text{q\_min\_pairs}|} \sum_{q \in q_{\text{min\_pairs}}} q$
    
%     \If{$|\text{q\_min\_pairs}| > 1$}
%         \State $\sigma_q(s,a) \gets \sqrt{\frac{1}{|\text{q\_min\_pairs}|} \sum_{q \in q_{\text{min\_pairs}}} (q - q_{\text{mean}}(s,a))^2}$
%         \State $\sigma_q(s,a) \gets \min(\sigma_q(s,a), \text{std\_clip})$ \Comment{Clamp std to max value of std\_clip}
%         \State $\text{exploration\_bonus} \gets \text{uncertainty\_coef} \cdot \sigma_q(s,a)$
%     \Else
%         \State $\text{exploration\_bonus} \gets 0$
%     \EndIf
    
%     \State \Return $r(s,a) + q_{\text{mean}}(s,a) + \text{exploration\_bonus}$
% \EndFunction
% \end{algorithmic}
% \end{algorithm}


% \begin{figure}
%     \centering
%     \includegraphics[width=\linewidth]{figs/hopper_clipping_std_1.png}
%     \caption{Result clipping of std at 1 and different NNs. }
%     \label{fig:clipping_1}
% \end{figure} 


% Next, we did a more comprehensive grid search for different number of NNs and different clipping values. The results are reported below. 

\section{Theoretical Analysis}

\subsection{Upper bounding the policy regret}
\if 0
\begin{lemma}
Let $\delta \in [0,1)$ and $L \geq 1$, then for any $k \geq 0$ and for any fixed $V \in \mathcal{V}$, it holds for a fixed state-action pair $s,a \in \sspace\times \aspace$
\begin{equation*}
\mathbb{P}\bs{ \min_{\ell\in[L]}\widehat{P}_\ell \bar{V}(s,a) \leq P \bar{V}(s,a)} \geq 1 - e^{-2 L/7}.
\end{equation*}
\end{lemma}
\begin{proof}
Without loss of generality we assume that for each state-action pair $s,a$ the distribution $P(\cdot|s,a)$ has support on exactly two states. This is without loss of generality because any MDP can be recast in an MDP satisfying the required property at the cost of incresing the number of states from $\abs{\sspace}$ to $\abs{\sspace} \log \abs{\aspace}$.
Moreover, let us consider a fixed element of the state value function class $V \in \mathcal{V}:=\bc{f \in \mathbb{R}^{\abs{S}}~~|~~ \norm{f}_{\infty} \leq \frac{1}{1-\gamma}, f(s) \geq 0 ~~\forall s \in \sspace}$. We can now denote by $V_{\min}(s,a) = \min_{s' \in \mathrm{supp}(P(\cdot|s,a))}  V(s')$, $V_{\max}(s,a) = \max_{s' \in \mathrm{supp}(P(\cdot|s,a))}  V(s')$ and similarly we define the states $s_{\min}$ and $s_{\max}$ as follows
$s_{\min}(s,a) = \argmin_{s' \in \mathrm{supp}(P(\cdot|s,a))}  V(s')$ and $s_{\max}(s,a) = \argmax_{s' \in \mathrm{supp}(P(\cdot|s,a))}  V(s')$.
Given this definitions, we can see that for $s' \sim P(\cdot|s,a)$
\begin{equation*}
     \bar{V}(s') = \frac{V(s') - V_{\min}(s,a)}{V_{\max}(s,a) - V_{\min}(s,a)}
\end{equation*}
is a Bernoulli random variable and for any $k\in[K]$
$$
\widehat{P}^k_\ell \bar{V}(s,a) = (N^k_\ell(s,a)+2)^{-1}\sum_{\bar{s},\bar{a},s' \in \mathcal{R}^k_\ell} \bar{V}(s') \mathds{1}_{\bc{\bar{s},\bar{a}=s,a}}
$$
is a multinomial distribution. 
Therefore, applying \cite[Lemma 3]{cassel2024batch}, we conclude the  proof.
\end{proof}
\begin{corollary}
Choosing $L = \frac{7}{2}\log \br{\frac{\abs{\sspace}\abs{\aspace}}{\delta}}$ it holds that for any fixed $V \in \mathcal{V}$
\begin{equation*}
\mathbb{P}\bs{ \min_{\ell\in[L]} \widehat{P}^k_\ell \bar{V}(s,a) \leq P \bar{V}(s,a) ~~~ \forall ~~~s,a\in \sspace\times\aspace} \geq 1 - \delta.
\end{equation*}
\end{corollary}

\begin{corollary}
\label{cor:optimism}
   Choosing $L = \frac{7 \abs{\sspace}}{2}\log \br{\frac{\abs{\sspace}\abs{\aspace}}{\delta (1-\gamma) \epsilon_{\mathrm{cov}}}}$ it holds that
\begin{equation*}
\mathbb{P}\bs{ \min_{\ell\in[L]} \widehat{P}^k_\ell V^k(s,a) \leq P V^k(s,a) + 2 \epsilon_{\mathrm{cov}} ~~~ \forall ~~~s,a\in \sspace\times\aspace} \geq 1 - \delta.
\end{equation*} 
\end{corollary}
\begin{proof}
    
Let us consider the state value function $V^k \in \mathcal{V}$ and let us introduce a $\epsilon_{\mathrm{cov}}$-covering set $\mathcal{C}_{\epsilon_{\mathrm{cov}}}(\mathcal{V})$ such that for any $V \in \mathcal{V}$ there exists $\tilde{V}\in \mathcal{C}_{\epsilon_{\mathrm{cov}}}(\mathcal{V})$ such that $\norm{\tilde{V} - V}_{\infty} \leq \epsilon_{\mathrm{cov}}$.
Therefore let us denote by $\tilde{V}^k$ the element of $\mathcal{C}_{\epsilon_{\mathrm{cov}}}(\mathcal{V})$ such that $\norm{V^k - \tilde{V}^k }_{\infty} \leq \epsilon_{\mathrm{cov}} $.
    Let us also define 
    $$
    \bar{V}^k(s') = \frac{\tilde{V}^k(s') - \tilde{V}^k_{\min}(s,a)}{\tilde{V}^k_{\max}(s,a) - \tilde{V}^k_{\min}(s,a)}
    $$
    which implies
    $$
    P \bar{V}^k(s,a) = \frac{P \tilde{V}^k(s,a) - \tilde{V}^k_{\min}(s,a)}{\tilde{V}^k_{\max}(s,a) - \tilde{V}^k_{\min}(s,a)}
    $$
    Moreover, these definitions imply that
\begin{align*}
\widehat{P}^k_\ell \bar{V}^k(s,a) &= (N^k_\ell(s,a)+2)^{-1}\sum_{\bar{s},\bar{a},s' \in \mathcal{R}_\ell^k} \bar{V}^k(s') \mathds{1}_{\bc{\bar{s},\bar{a}=s,a}} \\
&= (N^k_\ell(s,a)+2)^{-1}\sum_{\bar{s},\bar{a},s' \in \mathcal{R}_\ell^k}  \frac{\tilde{V}^k(s') - \tilde{V}^k_{\min}(s,a)}{\tilde{V}^k_{\max}(s,a) - \tilde{V}^k_{\min}(s,a)} \mathds{1}_{\bc{\bar{s},\bar{a}=s,a}} \\
&=  \frac{\widehat{P}^k_\ell \tilde{V}^k(s,a) - (N^k_\ell(s,a)+2)^{-1}N^k_\ell \tilde{V}^k_{\min}(s,a)}{\tilde{V}^k_{\max}(s,a) - \tilde{V}^k_{\min}(s,a)} \\
&> \frac{\widehat{P}^k_\ell \tilde{V}^k(s,a) - \tilde{V}^k_{\min}(s,a)}{\tilde{V}^k_{\max}(s,a) - \tilde{V}^k_{\min}(s,a)}.
\end{align*}
    Therefore for any $\ell \in [L]$
    \begin{align*}
\widehat{P}^k_\ell \bar{V}^k(s,a) \leq P \bar{V}^k(s,a) &\implies \widehat{P}^k_\ell \bar{V}^k(s,a) \leq \frac{P \tilde{V}^k(s,a) - \tilde{V}^k_{\min}(s,a)}{\tilde{V}^k_{\max}(s,a) - \tilde{V}^k_{\min}(s,a)} \\
& \implies \frac{\widehat{P}^k_\ell \tilde{V}^k(s,a) - \tilde{V}^k_{\min}(s,a)}{\tilde{V}^k_{\max}(s,a) - \tilde{V}^k_{\min}(s,a)}  \leq \frac{P \tilde{V}^k(s,a) - \tilde{V}^k_{\min}(s,a)}{\tilde{V}^k_{\max}(s,a) - \tilde{V}^k_{\min}(s,a)}  \\
& \implies \widehat{P}^k_\ell \tilde{V}^k(s,a)  \leq P \tilde{V}^k(s,a) .
    \end{align*}
Therefore, by the above chain of implications and by Corollary~\ref{cor:uniform_bound}, choosing $L =\frac{7}{2}\log \br{\frac{\abs{\sspace}\abs{\aspace} \abs{\mathcal{C}_{\epsilon_{\mathrm{cov}}}(\mathcal{V})}}{\delta}}$, we have that with probability $1-\delta$
\[\widehat{P}^k_\ell \tilde{V}^k(s,a)  \leq P \tilde{V}^k(s,a)\]

Then by the properties of the covering set, we have that
\begin{align*}
\min_{\ell \in [L]} \widehat{P}^k_{\ell} V^k(s,a) \leq \min_{\ell \in [L]} \widehat{P}^k_{\ell} \tilde{V}^k(s,a) + \epsilon_{\mathrm{cov}},
\end{align*}
and that
\begin{align*}
   P V^k(s,a) \geq P\tilde{V}^k(s,a) - \epsilon_{\mathrm{cov}}. 
\end{align*}
Therefore, we have that
$$
\min_{\ell \in [L]} \widehat{P}^k_{\ell} \tilde{V}^k(s,a) \leq P\tilde{V}^k(s,a) \implies \min_{\ell \in [L]} \widehat{P}^k_{\ell} V^k(s,a) \leq PV^k(s,a) + 2 \epsilon_{\mathrm{cov}}.
$$
Finally, we conclude computing the size of the covering set. Since it is a infinite norm $\abs{\sspace}$ dimensional ball of radius $\frac{1}{1-\gamma}$. We have that $\abs{\mathcal{C}_{\epsilon_{\mathrm{cov}}}(\mathcal{V})} \leq \br{\frac{1}{(1-\gamma)\epsilon_{\mathrm{cov}}}}^{\abs{\sspace}}$. Plugging in this fact in the choice of $L$ concludes the proof.
\end{proof}
\textcolor{red}{Alternative proof (maybe)}
\fi
\coroptimism*
\begin{proof}
Let us fix a state-action-next state triplet $s,a,s'$, a batch index $\ell \in [L]$ and an iteration index $k \in [K]$.
Then, we consider the following stochastic estimator for the probability of transitioning to $s'$ from state $s$ taking action $a$.
\begin{align*}
\widehat{P}^k_\ell(s'|s,a) &=\frac{N^k_\ell(s,a,s')}{N^k_\ell(s,a) + 2} = \frac{\sum_{\bar{s},\bar{a},\bar{s}'\in \mathcal{R}^k_\ell} \mathds{1}_{\bc{\bar{s},\bar{a},\bar{s}' = s,a,s'}}}{N^k_\ell(s,a) + 2} \\
&= \frac{\sum_{\bar{s},\bar{a},\bar{s}'\in \mathcal{R}^k_\ell : \bar{s},\bar{a} = s,a } \mathds{1}_{\bc{\bar{s}' = s'}}}{N^k_\ell(s,a) + 2}.
\end{align*}
Notice that in above estimators the denominator corresponds to the number of visits of the pair $s,a$ up to the time $k \in [K]$ within the batch $\ell\in[L]$, i.e. $N^k_\ell(s,a)$,  increased by $2$ for technical reasons. At the numerator instead we have the sum of $N^k_\ell(s,a)$ indicators functions which equals one when the state following the state action pair $s,a$ is equal to $s'$. Each of this indicator is a random variable distributed 
according to a Bernoulli random variable with mean $P(s'|s,a)$.
At this point, we can use the technique introduced in \cite{cassel2024batch}. In particular, we will show that
\begin{equation*}
\mathbb{P}\bs{ \widehat{P}^k_\ell(s'|s,a) \leq P(s'|s,a)} \geq \frac{1}{4} ~~~ \forall s,a,s' \in \sspace\times\aspace\times\mathrm{children}(s), \forall k \in [K], \forall \ell\in[L].
\end{equation*}
We distinguish $3$ cases: $N^k_\ell(s,a) = 0$, $N^k_\ell(s,a) = 1$, $N^k_\ell(s,a) \geq 2$.
If $N^k_\ell(s,a) = 0$, then, we have that
\begin{align*}
    \widehat{P}^k_\ell(s'|s,a) = 0 \leq P(s'|s,a).
\end{align*}
If $N^k_\ell(s,a) = 1$, we distinguish two cases. If $P(s'|s,a) \geq \frac{1}{3}$, then
\begin{align*}
\widehat{P}^k_\ell(s'|s,a) \leq \frac{1}{3} \leq P(s'|s,a).
\end{align*}
Otherwise, if $N^k_\ell(s,a) = 1$, and $P(s'|s,a) \leq \frac{1}{3}$ then
\begin{align*}
\mathbb{P}\bs{
\frac{\mathds{1}_{\bc{\bar{s}' = s'}}}{3} \leq P(s'|s,a) 
} = \mathbb{P}\bs{
\mathds{1}_{\bc{\bar{s}' = s'}} = 0} = 1 - P(s'|s,a) \geq \frac{2}{3} \geq \frac{1}{4}.
\end{align*}
Finally, for $N^k_\ell(s,a) \geq 2$,we have that for $P(s'|s,a) \geq 1 - \frac{1}{N^k_\ell(s,a)}$ it holds that
\[
\widehat{P}^k_\ell(s'|s,a) \leq \frac{N^k_\ell(s,a)}{N^k_\ell(s,a)+2} = 1 - \frac{2}{N^k_\ell(s,a) + 2} \leq 1 - \frac{1}{N^k_\ell(s,a)} \leq P(s'|s,a),
\]
where the last inequality holds for $N^k_\ell \geq 2$.
Otherwise, for $P(s'|s,a) \leq 1 - \frac{1}{N^k_\ell(s,a)}$ we can apply \citep[Lemma 2]{cassel2024batch} (adapted from \citep[Corollary 1]{wiklund2023another} ) to obtain 
\begin{equation*}
\mathbb{P}\bs{\widehat{P}^k_\ell(s'|s,a) \leq P(s'|s,a)} \geq 
\mathbb{P}\bs{ \sum_{\bar{s},\bar{a},\bar{s}'\in \mathcal{R}^k_\ell : \bar{s},\bar{a} = s,a } \mathds{1}_{\bc{\bar{s}' = s'}} \leq N^k_\ell(s,a) P(s'|s,a)} \geq \frac{1}{4}.
\end{equation*}
At this point notice that for any positive vector $V \in [0, (1-\gamma)^{-1}]^{\abs{\sspace}}$, it holds that
\begin{align*}
\mathbb{P}\bs{\widehat{P}^k_\ell(s'|s,a) \leq P(s'|s,a)~~\forall s'\in \mathrm{children}(s)} &= \mathbb{P}\bs{\widehat{P}^k_\ell(s'|s,a)  V(s') \leq P(s'|s,a) V(s')~~\forall s'\in \mathrm{children}(s)} \\ & \leq \mathbb{P}\bs{\sum_{s'\in\sspace}\widehat{P}^k_\ell(s'|s,a)  V(s') \leq \sum_{s'\in\sspace} P(s'|s,a) V(s')}
\end{align*}
where the inequality holds because of the following implication between events
\begin{equation*}
\widehat{P}^k_\ell(s'|s,a)  V(s') \leq  P(s'|s,a) V(s') ~~\forall s'\in \mathrm{children}(s)
 \implies \sum_{s'\in \mathrm{children}(s)}\widehat{P}^k_\ell(s'|s,a)  V(s') \leq \sum_{s'\in \mathrm{children}(s)} P(s'|s,a) V(s').
\end{equation*}
Moreover, since the estimation at each state $s'$ is independent we have that
\begin{align*}
\mathbb{P}\bs{\widehat{P}^k_\ell(s'|s,a) \leq P(s'|s,a)~~\forall s'\in \mathrm{children}(s)} = \prod_{s'\in \mathrm{children}(s)} \mathbb{P}\bs{\widehat{P}^k_\ell(s'|s,a) \leq P(s'|s,a)} \geq \br{\frac{1}{4}}^{\abs{\mathrm{children}(s)}}
\end{align*}
Then, for any $s,a \in \sspace \times \aspace$ is concluded by the following chain of inequalities
\begin{align*}
\mathbb{P}&\bs{\min_{\ell\in[L]}\sum_{s'\in \mathrm{children}(s)}\widehat{P}^k_\ell(s'|s,a)  V(s') \geq \sum_{s'\in \mathrm{children}(s)} P(s'|s,a) V(s')} \\&= \mathbb{P}\bs{\sum_{s'\in\mathrm{children}(s)}\widehat{P}^k_\ell(s'|s,a)  V(s') \geq \sum_{s'\in \mathrm{children}(s)} P(s'|s,a) V(s')~~ \forall \ell \in [L]} \\
&= \prod_{\ell \in [L]}\br{1 - \mathbb{P}\bs{\sum_{s'\in \mathrm{children}(s)}\widehat{P}^k_\ell(s'|s,a)  V(s') \leq \sum_{s'\in \mathrm{children}(s)} P(s'|s,a) V(s')}} 
\\
&\leq \prod_{\ell\in[L]}\br{ 1-\mathbb{P}\bs{\widehat{P}^k_\ell(s'|s,a)  \leq P(s'|s,a)~~\forall s'\in \mathrm{children}(s)} } \\
&\leq \prod_{\ell\in[L]}\br{ 1- \frac{1}{4}^{\abs{ \mathrm{children}(s)}} } \leq e^{-L \log \br{\frac{4^{\abs{ \mathrm{children}(s)}}}{4^{\abs{\mathrm{children}(s)}} - 1}}}.
\end{align*}
Therefore,
choosing $L \geq \frac{\log \br{1/\delta}}{\log \br{\frac{4^{\abs{ \mathrm{children}(s)}}}{4^{\abs{\mathrm{children}(s)}} - 1}}}$, we ensure that
\begin{align*}
\mathbb{P}\bs{\min_{\ell\in[L]} \widehat{P}^k_\ell V^k(s,a) \geq P V^k(s,a)} \leq \delta.
\end{align*}
For $\abs{\mathrm{children(s)}} = 2$, we have that $\br{\log \br{\frac{4^{\abs{ \mathrm{children}(s)}}}{4^{\abs{\mathrm{children}(s)}} - 1}}}^{-1}\leq 36$, therefore with a union bound over the sets $\sspace,\aspace$ and $[K]$ we conclude that
for $L \geq 36 \log \br{\frac{\abs{\sspace}\abs{\aspace} K}{\delta}}$, it holds that
\begin{align*}
\mathbb{P}\bs{\min_{\ell\in[L]} \widehat{P}^k_\ell V^k(s,a) \leq P V^k(s,a) ~~\forall ~~s,a,k \in \sspace\times\aspace\times [K]} \geq 1-\delta.
\end{align*}
\end{proof}

\subsection{Policy Regret Decomposition}

\thmpolicyregret*
\begin{proof}
The theorem is proven  with the following regret decomposition
in virtue of \Cref{lemma:infinite_extendend_pdl} already presented in the main text. Denoting $\delta^{k}(s,a) \triangleq \cost^k(s,a) + \gamma P V^k(s,a) - Q^{k+1}(s,a)$ and $g^k(s,a) \triangleq  Q^{k+1}(s,a) - Q^{k}(s,a)$
\begin{align}
(1 - \gamma) & \mathrm{Regret}_\pi(K;\pi^\star) = \sum^K_{k=1} \mathbb{E}_{s\sim d^{\pi^\star}}\bs{
    \innerprod{Q^k(s,\cdot)}{\pi^k(s) - \pi^\star(s)}} \quad \tag{BTRL} \label{eq:OMD}\\&\phantom{\leq}+ \sum^K_{k=1}  \sum_{s,a} \bs{d^{\pi^k}(s,a) - d^{\pi^\star}(s,a)}\cdot\bs{\delta^k(s,a)} \tag{Optimism} \label{eq:optimism}
    \\&\phantom{\leq}+ \sum^K_{k=1} \mathbb{E}_{s,a\sim d^{\pi^k}}\bs{g^k(s,a)} - \sum^K_{k=1} \mathbb{E}_{s,a\sim d^{\pi^\star}}\bs{g^k(s,a)} \quad \label{eq:shift} \tag{Shift}
\end{align}
At this point, we bound each term with the following Lemmas, we obtain
\begin{align*}
&\sqrt{\frac{\log \abs{\aspace} K }{ (1-\gamma)^5}} + \widetilde{\mathcal{O}}\br{\frac{ \sqrt{ K \abs{\sspace}^2 \abs{\aspace} \log (1/\delta)}}{(1-\gamma)^2}} \\&\leq \widetilde{\mathcal{O}}\br{ \sqrt{ \frac{K \abs{\sspace}^2 \abs{\aspace} \log (1/\delta)}{(1-\gamma)^5}}}.
\end{align*}
\end{proof}
\textbf{Bound on \eqref{eq:main_OMD}}
Then, we continue bounding the first term invoking the following lemma.
\localregret*
\begin{proof}

    \begin{align}
\sum^K_{k=1} \mathbb{E}_{s\sim d^{\pi^\star}_{\rho_{e_k}}}\bs{
    \innerprod{Q^k(s,\cdot)}{\pi^k(s) - \pi^\star(s)}} &= \sum_{k \in K} \mathbb{E}_{s\sim d^{\pi^\star}}\bs{
    \innerprod{Q^k(s,\cdot)}{\pi^k(s) - \pi^\star(s)}} \nonumber \\
    &=\mathbb{E}_{s\sim d^{\pi^\star}}\bs{ \sum_{k \in K} 
    \innerprod{Q^k(s,\cdot)}{\pi^k(s) - \pi^\star(s)}} \label{eq:first}
    \end{align}
Then,  applying a standard regret bound for Be the Regularized Leader (BTRL) it holds that for all $s \in \sspace$
\begin{align*}
    \sum_{k \in K} 
    \innerprod{Q^k(s,\cdot)}{\pi^k(s) - \pi^\star(s)} &= \frac{\log \abs{\aspace}}{\eta}
\end{align*}
Then, plugging into \eqref{eq:first} we conclude that
\begin{equation*}
\eqref{eq:main_OMD} \leq \frac{\log \abs{\aspace}}{\eta}.
\end{equation*}
\end{proof}
\noindent\textbf{Bound on \eqref{eq:main_shift}}
We follow the idea from \cite{moulin2023optimistic} of controlling this term proving that two consecutive policies will have occupancy measures within a $\mathcal{O}(\eta)$ total variation distance. 
\shiftlemma*
\begin{proof}
Let us denote $Q_{\max} = (1-\gamma)^{-1}$
\begin{align*}
    \sum^K_{k=1} \innerprod{d^{\pi^k}}{Q^{k} - Q^{k+1}} &=  \br{ \innerprod{d^{\pi^1}}{Q^1} + \sum^{K-1}_2 \innerprod{Q^k}{d^{\pi^k} - d^{\pi^{k-1}} } - \innerprod{d^{\pi^K}}{Q^K}} \\
&\leq  Q_{\max} + \frac{\eta Q^2_{\max} (K-2) }{(1-\gamma)} \leq \frac{\eta Q^2_{\max} (K-1) }{(1-\gamma)} .
\end{align*}
In the first inequality, we used Lemma~\ref{lemma:slow}.
Finally, noticing that
\begin{equation*}
- \sum^K_{k=1} \innerprod{d^{\pi^\star}}{Q^{k} - Q^{k+1}} = \innerprod{d^{\pi^\star}}{Q^K - Q^1} \leq Q_{\max}
\end{equation*}
and that
\begin{equation*}
    \eqref{eq:main_shift} = \sum^K_{k=1} \innerprod{d^{\pi^k}}{Q^{k} - Q^{k+1}} - \sum^K_{k=1} \innerprod{d^{\pi^\star}}{Q^{k} - Q^{k+1}}
\end{equation*}
allows us to conclude the proof summing the two bounds.
    \end{proof}
\textbf{Bound on \eqref{eq:main_optimism}}
\lemmaoptimism*
\begin{proof}
For the optimism term we can observe that using the update of the $Q$ values, we have that
$$
\delta^k(s,a) = \gamma (P V^k(s,a) - \min_{\ell \in [L]} \widehat{P}^k_{\ell} V^k(s,a))
$$
Therefore, in virtue of \Cref{cor:optimism} with probability $1-\delta$ it holds that
\begin{equation*}
\delta^k(s,a) \geq 0 ~~~\forall ~~~s,a \in \sspace\times\aspace.
\end{equation*}
Therefore, with probability $1-\delta$, we have that
\begin{align*}
\sum^K_{k=1}  \sum_{s,a} \bs{d^{\pi^k}(s,a) - d^{\pi^\star}(s,a)}\cdot\bs{\delta^k(s,a)} &\leq \sum^K_{k=1}  \sum_{s,a} d^{\pi^k}(s,a) \delta_k(s,a) \\
&=\gamma\sum^K_{k=1} \mathbb{E}_{s,a\sim d^{\pi^k}} \bs{P V^k(s,a) - \min_{\ell \in [L]} \widehat{P}^k_{\ell} V^k(s,a)}
\end{align*}
At this point, using \Cref{lemma:bounded_optimism_ensemble} and  a union bound we have that with probability $1-2\delta$, it holds that% and $\epsilon_{\mathrm{cov}}=K^{-1}$, it holds that
\begin{align*}
\sum^K_{k=1}  \sum_{s,a} \bs{d^{\pi^k}(s,a) - d^{\pi^\star}(s,a)}\cdot\bs{\delta^k(s,a)} &\leq 
\gamma\sum^K_{k=1} \mathbb{E}_{s,a\sim d^{\pi^k}} \bs{\sqrt{\frac{\abs{\sspace}}{(N^k(s,a)/L + 1) (1-\gamma)^2}\log \br{\frac{\abs{\sspace}\abs{\aspace}  L K }{(1-\gamma)\delta}}}} \\
&\phantom{=}+  \gamma\sum^K_{k=1} \mathbb{E}_{s,a\sim d^{\pi^k}} \bs{\frac{2}{(N^k(s,a)/L + 1)(1-\gamma)}} + 4 \\
&\leq \sqrt{K \sum^K_{k=1} \mathbb{E}_{s,a\sim d^{\pi^k}} \bs{\frac{\abs{\sspace}}{(N^k(s,a)/L + 1) (1-\gamma)^2}\log \br{\frac{\abs{\sspace}\abs{\aspace}  L K }{(1-\gamma)\delta}}}} \\
&\phantom{=}+  \gamma\sum^K_{k=1} \mathbb{E}_{s,a\sim d^{\pi^k}} \bs{\frac{2}{(N^k(s,a)/L + 1)(1-\gamma)}} + 4 \\
\end{align*}
At this point, since $L^k$ is geometrically distributed for every $k \in [K]$ it holds that $L^k \leq L_{\max} := \frac{\log (K/\delta)}{(1-\gamma)}$ for all $k \in [K]$ with probability $1-\delta$.
Therefore, we can invoke \Cref{lemma:count_based}  to bound $\sum^K_{k=1} \mathbb{E}_{s,a\sim d^{\pi^k}} \bs{\frac{2}{(N^k(s,a)/L + 1)}}$. Another union bound ensures that with probability $1 - 3\delta$, we have that
\begin{align*}
    \sum^K_{k=1}  \sum_{s,a} \bs{d^{\pi^k}(s,a) - d^{\pi^\star}(s,a)}\cdot\bs{\delta^k(s,a)} &\leq \frac{1}{1-\gamma}\sqrt{K \abs{\sspace} \log \br{\frac{\abs{\sspace} \abs{\aspace} L K}{(1-\gamma)\delta}} (2 L \abs{\sspace} \abs{\aspace} \log \br{K L_{\max}} + 4 \log \br{2K/\delta} ) } \\
    &\phantom{=}+\frac{2}{1-\gamma} (2 L \abs{\sspace} \abs{\aspace} \log \br{K L_{\max}} + 4 \log \br{2K/\delta}) \\
&=\widetilde{\mathcal{O}}\br{\frac{  \sqrt{K \abs{\sspace}^2 \abs{\aspace} \log (1/\delta)}}{1-\gamma}}.
\end{align*}
where the $\widetilde{\mathcal{O}}$ notation hides logarithmic factors in $K, 1-\gamma, \abs{\sspace}$ and $\abs{\aspace}$.

Now, we prove the part of the Theorem that considers the update for $Q^{k+1}$ given in \eqref{eq:update2}. Under this update, we have that
$$ \delta^k(s,a) = \gamma \br{PV^k(s,a)  -  \max\bs{ \frac{1}{L} \sum^L_{\ell=1} \widehat{P}^k_\ell V^k(s,a) - \sqrt{\sum^L_{\ell=1} \br{
    \widehat{P}^k_\ell V^k(s,a)
    - \frac{1}{L} \sum^L_{\ell=1} \widehat{P}^k_\ell V^k(s,a)}^2},0}}$$
Then, applying Samuelson's inequality \Cref{lemma:samuelson}, we have that
$$
\min_{\ell\in[L]} \widehat{P}^k_\ell V^k(s,a) \geq \frac{1}{L} \sum^L_{\ell=1} \widehat{P}^k_\ell V^k(s,a) - \sqrt{\sum^L_{\ell=1} \br{
    \widehat{P}^k_\ell V^k(s,a)
    - \frac{1}{L} \sum^L_{\ell=1} \widehat{P}^k_\ell V^k(s,a)}^2}
$$
moreover, it holds that
$$
\min_{\ell\in[L]} \widehat{P}^k_\ell V^k(s,a) \geq 0
$$
Therefore,
\[
\min_{\ell\in[L]} \widehat{P}^k_\ell V^k(s,a) \geq \max \bs{\frac{1}{L} \sum^L_{\ell=1} \widehat{P}^k_\ell V^k(s,a) - \sqrt{\sum^L_{\ell=1} \br{
    \widehat{P}^k_\ell V^k(s,a)
    - \frac{1}{L} \sum^L_{\ell=1} \widehat{P}^k_\ell V^k(s,a)}^2}, 0}
\]
which implies 
$$ \delta^k(s,a) \geq \gamma \br{PV^k(s,a)  -  \min_{\ell\in[L]} \widehat{P}^k_\ell V^k(s,a)} \geq 0
$$
where the last inequality holds with probability $1-\delta$ thanks to \Cref{cor:optimism}.
Therefore, with probability $1-\delta$, we have that
\begin{align*}
\sum^K_{k=1}  &\sum_{s,a} \bs{d^{\pi^k}(s,a) - d^{\pi^\star}(s,a)}\cdot\bs{\delta^k(s,a)} \leq \sum^K_{k=1}  \sum_{s,a} d^{\pi^k}(s,a) \delta_k(s,a) \\
&\leq\gamma\sum^K_{k=1} \mathbb{E}_{s,a\sim d^{\pi^k}} \bs{\frac{1}{L}\sum^L_{\ell=1} \br{P V^k(s,a) -  \widehat{P}^k_{\ell} V^k(s,a)} + \sqrt{\sum^L_{\ell=1} \br{\widehat{P}^k_\ell V^k (s,a) - \frac{1}{L}\sum^L_{\ell=1} \widehat{P}^k_{\ell} V^k(s,a)}^2}} ,
\end{align*}
where the last inequality holds removing the maximum.
For the first term, we can use \Cref{lemma:bounded_optimism_ensemble} and continue as in the previous proof to show that with probability $1-\delta$
\begin{equation*}
    \gamma\sum^K_{k=1} \mathbb{E}_{s,a\sim d^{\pi^k}} \bs{\frac{1}{L}\sum^L_{\ell=1} \br{P V^k(s,a) -  \widehat{P}^k_{\ell} V^k(s,a)}} \leq \widetilde{\mathcal{O}}\br{\frac{  \sqrt{K \abs{\sspace}^2 \abs{\aspace} \log (1/\delta)}}{1-\gamma}}.
\end{equation*}
So, we are left with bounding the term 
\begin{equation*}
\gamma\sum^K_{k=1} \mathbb{E}_{s,a\sim d^{\pi^k}} \bs{\sqrt{\sum^L_{\ell=1} \br{\widehat{P}^k_\ell V^k (s,a) - \frac{1}{L}\sum^L_{\ell'=1} \widehat{P}^k_{\ell'} V^k(s,a)}^2}}
\end{equation*}
To this end, by Jensen's inequality we have that
\begin{align*}
\gamma\sum^K_{k=1} & \mathbb{E}_{s,a\sim d^{\pi^k}} \bs{\sqrt{\sum^L_{\ell=1} \br{\widehat{P}^k_\ell V^k (s,a) - \frac{1}{L}\sum^L_{\ell'=1} \widehat{P}^k_{\ell'} V^k(s,a)}^2}} \\&\leq \gamma\sum^K_{k=1} \mathbb{E}_{s,a\sim d^{\pi^k}} \bs{\sqrt{\frac{1}{L}\sum^L_{\ell'=1}\sum^L_{\ell=1} \br{\widehat{P}^k_\ell V^k (s,a) -  \widehat{P}^k_{\ell'} V^k(s,a)}^2}} \\
\\&\leq \gamma\sum^K_{k=1} \mathbb{E}_{s,a\sim d^{\pi^k}} \bs{\sqrt{\frac{2}{L}\sum^L_{\ell'=1}\sum^L_{\ell=1} \br{\widehat{P}^k_\ell V^k (s,a) -  P V^k(s,a)}^2}} \\
&\leq \gamma\sum^K_{k=1} \mathbb{E}_{s,a\sim d^{\pi^k}} \bs{\sqrt{2\sum^L_{\ell=1} \br{\widehat{P}^k_\ell V^k (s,a) -  P V^k(s,a)}^2}} \\
\end{align*}
At this point, notice that invoking \Cref{lemma:bounded_optimism_ensemble}, we have that for all $\ell \in [L]$ with probability $1-\delta$
\begin{align*}
&\br{\widehat{P}^k_\ell V^k (s,a)  -  P V^k(s,a)}^2 \\ &\leq \br{\sqrt{\frac{\abs{\sspace}}{(N^k(s,a)/L + 1) (1-\gamma)^2}\log \br{\frac{\abs{\sspace}\abs{\aspace}  L K }{(1-\gamma)\delta}}} + \frac{2}{K} +  \frac{2}{(N^k(s,a)/L + 1)(1-\gamma)}}^2 \\
&\leq \frac{3\abs{\sspace}}{(N^k(s,a)/L + 1) (1-\gamma)^2}\log \br{\frac{\abs{\sspace}\abs{\aspace}  L K }{(1-\gamma)\delta}} + \frac{12}{K^2} + \frac{12}{(N^k(s,a)/L + 1)^2(1-\gamma)^2} \\
&= \widetilde{\mathcal{O}}\br{\frac{3\abs{\sspace}\log \br{\frac{1}{\delta}}}{(N^k(s,a)/L + 1) (1-\gamma)^2}}.
\end{align*}
Therefore, plugging into the previous display we obtain
\begin{align*}
\gamma\sum^K_{k=1} & \mathbb{E}_{s,a\sim d^{\pi^k}} \bs{\sqrt{\sum^L_{\ell=1} \br{\widehat{P}^k_\ell V^k (s,a) - \frac{1}{L}\sum^L_{\ell'=1} \widehat{P}^k_{\ell'} V^k(s,a)}^2}} \\&\leq  \sum^K_{k=1} \mathbb{E}_{s,a\sim d^{\pi^k}} \bs{\sqrt{\sum^L_{\ell=1} \widetilde{\mathcal{O}}\br{\frac{3\abs{\sspace}\log \br{\frac{1}{\delta}}}{(N^k(s,a)/L + 1) (1-\gamma)^2}} }} \\
&\leq  \sqrt{ K \sum^K_{k=1} \mathbb{E}_{s,a\sim d^{\pi^k}}\bs{\sum^L_{\ell=1} \widetilde{\mathcal{O}}\br{\frac{3\abs{\sspace}\log \br{\frac{1}{\delta}}}{(N^k(s,a)/L + 1) (1-\gamma)^2}} }} \\
&\leq  \widetilde{\mathcal{O}}\br{\sqrt{ K \sum^K_{k=1} \mathbb{E}_{s,a\sim d^{\pi^k}}\bs{\frac{3\abs{\sspace} L^2 \log \br{\frac{1}{\delta}}}{(N^k(s,a) + 1) (1-\gamma)^2}} }} 
\\
&\leq  \widetilde{\mathcal{O}}\br{\sqrt{ \frac{\abs{\sspace}^2 \log \br{\frac{1}{\delta}} K}{(1-\gamma)^2} \sum^K_{k=1} \mathbb{E}_{s,a\sim d^{\pi^k}}\bs{\frac{1}{N^k(s,a) + 1}} }} 
\end{align*}
Finally, we bound $\mathbb{E}_{s,a\sim d^{\pi^k}}\bs{\frac{1}{N^k(s,a) + 1}}$ using \Cref{lemma:count_based} under the event $L^k \leq L_{\max}:=\frac{\log(K/\delta)}{(1-\gamma)}$ which holds with probability $1-\delta$. Thanks to an union bound we have that with probability $1-2\delta$,
\begin{align*}
\gamma\sum^K_{k=1} \mathbb{E}_{s,a\sim d^{\pi^k}} &\bs{\sqrt{\sum^L_{\ell=1} \br{\widehat{P}^k_\ell V^k (s,a) - \frac{1}{L}\sum^L_{\ell'=1} \widehat{P}^k_{\ell'} V^k(s,a)}^2}} \\&\leq 
\widetilde{\mathcal{O}}\br{\sqrt{ \frac{\abs{\sspace}^2 \log \br{\frac{1}{\delta}} K}{(1-\gamma)^2} \abs{\sspace} \abs{\aspace} \log \br{K L_{\max}} + 4 \log \br{2KL_{\max}/\delta}  }} \\
&\leq \widetilde{\mathcal{O}}\br{\frac{  \sqrt{K \abs{\sspace}^2 \abs{\aspace} \log (1/\delta)}}{1-\gamma}}.
\end{align*}
Therefore, the proof is concluded also for the case of $Q$ value being updated as in \eqref{eq:update2}.
\end{proof}
\begin{lemma} \label{lemma:bounded_optimism_ensemble}
With probability $1-\delta$, it holds that for all $s,a\in \sspace\times \aspace$,
\begin{align*}
P V^k(s,a) - \widehat{P}^k_{\ell} V^k(s,a) &\leq \sqrt{\frac{\abs{\sspace}}{(N^k(s,a)/L + 1) (1-\gamma)^2}\log \br{\frac{\abs{\sspace}\abs{\aspace}  L K }{(1-\gamma)\delta}}} + \frac{2}{K} \\&\phantom{=}+  \frac{2}{(N^k(s,a)/L + 1)(1-\gamma)} ~~~~~ \forall \ell \in [L]
\end{align*}
In particular, the above statement implies that
\begin{align*}
P V^k(s,a) - \min_{\ell\in[L]}\widehat{P}^k_{\ell} V^k(s,a) &\leq \sqrt{\frac{\abs{\sspace}}{(N^k(s,a)/L + 1) (1-\gamma)^2}\log \br{\frac{\abs{\sspace}\abs{\aspace}  L K }{(1-\gamma)\delta}}} + \frac{2}{K} \\&\phantom{=}+  \frac{2}{(N^k(s,a)/L + 1)(1-\gamma)} 
\end{align*}
\end{lemma}
\begin{proof}
    Let us introduce the value class of the possible value functions generated by \Cref{alg:theory_version}.
    i.e. $\mathcal{V} = \bc{f \in \mathbb{R}^{\abs{S}}~~|~~ \norm{f}_{\infty} \leq \frac{1}{1-\gamma}, f(s) \geq 0 ~~\forall s \in \sspace}$.
    Let us introduce a $\epsilon_{\mathrm{cov}}$-covering set $\mathcal{C}_{\epsilon_{\mathrm{cov}}}(\mathcal{V})$ such that for any $V \in \mathcal{V}$ there exists $\tilde{V}\in \mathcal{C}_{\epsilon_{\mathrm{cov}}}(\mathcal{V})$ such that $\norm{\tilde{V} - V}_{\infty} \leq \epsilon_{\mathrm{cov}}$.
Therefore let us denote by $\tilde{V}^k$ the element of $\mathcal{C}_{\epsilon_{\mathrm{cov}}}(\mathcal{V})$ such that $\norm{V^k - \tilde{V}^k }_{\infty} \leq \epsilon_{\mathrm{cov}} $.
Then, let us consider a generic $\tilde{V} \in \mathcal{C}_{\epsilon_{\mathrm{cov}}}(\mathcal{V})$,
\begin{align*}
P \tilde{V}(s,a) - \frac{1}{N^k_{\ell}(s,a)}\sum_{\bar{s},\bar{a},s' \in \mathcal{R}^k_\ell}\tilde{V}(s') \mathds{1}_{\bc{s,a=\bar{s},\bar{a}}} &= \frac{1}{N^k_{\ell}(s,a)} \sum_{\bar{s},\bar{a},s' \in \mathcal{R}^k_\ell} P \tilde{V} (s,a) \mathds{1}_{\bc{s,a=\bar{s},\bar{a}}} \\&\phantom{=}- \frac{1}{N^k_{\ell}(s,a)}\sum_{\bar{s},\bar{a},s' \in \mathcal{R}^k_\ell}\tilde{V}(s') \mathds{1}_{\bc{s,a=\bar{s},\bar{a}}} \\
&= \frac{1}{N^k_{\ell}(s,a)} \sum_{\bar{s},\bar{a},s' \in \mathcal{R}^k_\ell} \br{P \tilde{V} (s,a) - \tilde{V}(s')}\mathds{1}_{\bc{s,a=\bar{s},\bar{a}}} \\
%&\phantom{=}- \frac{1}{N^k_{\ell}(s,a)(N^k_{\ell}(s,a) + 2)}\sum_{\bar{s},\bar{a},s' \in \mathcal{R}^k_\ell}\tilde{V}(s') \mathds{1}_{\bc{s,a=\bar{s},\bar{a}}} 
\end{align*}
Then, by Azuma-Hoeffding inequality it holds that with probability $1-\delta$
\begin{equation*}
\sum_{\bar{s},\bar{a},s' \in \mathcal{R}^k_\ell} \br{P \tilde{V} (s,a) - \tilde{V}(s')}\mathds{1}_{\bc{s,a=\bar{s},\bar{a}}} \leq \sqrt{\frac{N^k_{\ell}(s,a) \log (1/\delta)}{(1-\gamma)^2 2}}.
\end{equation*}
%Moreover the second term is negative since for all $\tilde{V} \in \mathcal{V}$, it holds that $\tilde{V} \geq 0$ elementwise.
Therefore, we can conclude that with probability at least $1-\delta$
\begin{equation*}
P \tilde{V}(s,a) - \frac{1}{N^k_{\ell}(s,a)}\sum_{\bar{s},\bar{a},s' \in \mathcal{R}^k_\ell}\tilde{V}(s') \mathds{1}_{\bc{s,a=\bar{s},\bar{a}}} \leq \sqrt{\frac{\log (1/\delta)}{N^k_{\ell}(s,a) (1-\gamma)^2 2}}.
\end{equation*}
Now, by a union bound over $\mathcal{C}_{\epsilon_{\mathrm{cov}}}(\mathcal{V})$ and $[L]$ and $\sspace\times\aspace$ and denoting $$\bar{P}^k_{\ell} \tilde{V}(s,a) := \frac{1}{N^k_{\ell}(s,a)}\sum_{\bar{s},\bar{a},s' \in \mathcal{R}^k_\ell}\tilde{V}(s') \mathds{1}_{\bc{s,a=\bar{s},\bar{a}}},$$  it holds that 
\begin{align*}
\mathbb{P}\bs{P \tilde{V}(s,a) - \bar{P}^k_{\ell} \tilde{V}(s,a) \leq \sqrt{\frac{\log (\abs{\sspace}\abs{\aspace}\abs{\mathcal{C}_{\epsilon_{\mathrm{cov}}}(\mathcal{V})} L/\delta)}{N^k_{\ell}(s,a) (1-\gamma)^2}} ~~\forall s,a,\ell\in\sspace\times\aspace\times [L],~~\tilde{V}\in\mathcal{V}} \geq 1-\delta
\end{align*}
Therefore, now let us consider the element $\tilde{V}^k \in \mathcal{C}_{\epsilon_{\mathrm{cov}}}(\mathcal{V})$ such that $\norm{V^k - \tilde{V}^k}_{\infty} \leq \epsilon_{\mathrm{cov}}$.
Then, we have that for all $\ell \in [L]$
\begin{align*}
P V^k(s,a) - \bar{P}^k_{\ell} V^k(s,a) &= P \tilde{V}^k(s,a) - \bar{P}^k_{\ell} \tilde{V}^k(s,a) + (P - \bar{P}^k_\ell) (\tilde{V}^k - V^k) \\
&= P \tilde{V}^k(s,a) - \bar{P}^k_{\ell} \tilde{V}^k(s,a) + 2 \epsilon_{\mathrm{cov}} \\
&\leq \sqrt{\frac{\log (\abs{\sspace}\abs{\aspace}\abs{\mathcal{C}_{\epsilon_{\mathrm{cov}}}(\mathcal{V})} L/\delta)}{N^k_{\ell}(s,a) (1-\gamma)^2}} + 2 \epsilon_{\mathrm{cov}} \\
&\leq \sqrt{\frac{\abs{\sspace}}{N^k_{\ell}(s,a) (1-\gamma)^2}\log \br{\frac{\abs{\sspace}\abs{\aspace}  L}{(1-\gamma)\epsilon_{\mathrm{cov}}\delta}}} + 2 \epsilon_{\mathrm{cov}} \\
\end{align*}
With $\epsilon_{\mathrm{cov}}=K^{-1}$, we get
\begin{align*}
P V^k(s,a) - \bar{P}^k_{\ell} V^k(s,a) \leq \sqrt{\frac{\abs{\sspace}}{N^k_{\ell}(s,a) (1-\gamma)^2}\log \br{\frac{\abs{\sspace}\abs{\aspace}  L K }{(1-\gamma)\delta}}} + \frac{2}{K} 
\end{align*}
Then, we can continue as follows 
\begin{align*}
\widehat{P}^k_\ell V^k(s,a) &= \frac{N^k_\ell(s,a)}{N^k_\ell(s,a) + 2} \bar{P}^k_\ell V^k(s,a) \\
&\geq \frac{N^k_\ell(s,a)}{N^k_\ell(s,a) + 2} 
\bs{P V^k(s,a) - \sqrt{\frac{\abs{\sspace}}{N^k_{\ell}(s,a) (1-\gamma)^2}\log \br{\frac{\abs{\sspace}\abs{\aspace}  L K }{(1-\gamma)\delta}}} - \frac{2}{K} } \\
&=
P V^k(s,a) - \frac{2}{N^k_\ell(s,a) + 2} 
P V^k(s,a) - \sqrt{\frac{\abs{\sspace}}{(N^k_{\ell}(s,a) + 2) (1-\gamma)^2}\log \br{\frac{\abs{\sspace}\abs{\aspace}  L K }{(1-\gamma)\delta}}} - \frac{2}{K}  \\
&\geq 
P V^k(s,a) - \frac{2}{(N^k_\ell(s,a) + 2)(1-\gamma)} 
 - \sqrt{\frac{\abs{\sspace}}{(N^k_{\ell}(s,a) + 2) (1-\gamma)^2}\log \br{\frac{\abs{\sspace}\abs{\aspace}  L K }{(1-\gamma)\delta}}} - \frac{2}{K}  
\end{align*}
Finally, rearranging and using that $N^k_\ell (s,a) = \floor{\frac{N^k(s,a)}{L}} \geq \frac{N^k(s,a)}{L} - 1, $
we obtain that with probability $1-\delta$, it holds that for all $\ell\in[L]$
\begin{align*}
P V^k(s,a) - \widehat{P}^k_{\ell} V^k(s,a) &\leq \sqrt{\frac{\abs{\sspace}}{(N^k(s,a)/L + 1) (1-\gamma)^2}\log \br{\frac{\abs{\sspace}\abs{\aspace}  L K }{(1-\gamma)\delta}}} + \frac{2}{K} \\&\phantom{=}+  \frac{2}{(N^k(s,a)/L + 1)(1-\gamma)} 
\end{align*}
\end{proof}
\subsection{Upper bound the regret of the reward player}
\thmcostregret*
\begin{proof}
We decompose the regret as follows 
\begin{align*}
\sum^K_{k=1} \innerprod{c_{\mathrm{true}} - c^k}{d^{\pi^k} - d^\expert} &= \sum^K_{k=1} \innerprod{c_{\mathrm{true}} - c^k}{\mathbf{e}_{s^k_{L^k}} - \widehat{d^\expert}} \\
&\phantom{=}+\sum^K_{k=1} \innerprod{c_{\mathrm{true}} - c^k}{ d^{\pi^k}-\mathbf{e}_{s^k_{L^k}}} \\
&\phantom{=}+\sum^K_{k=1} \innerprod{c_{\mathrm{true}} - c^k}{ \widehat{d^\expert} - d^\expert}
\end{align*}
For the first term, we can invoke a standard online gradient descent bound
and get
\begin{align*}
    \sum^K_{k=1} \innerprod{c_{\mathrm{true}} - c^k}{\mathbf{e}_{s^k_{L^k}} - \widehat{d^\expert}} &\leq \frac{2}{\eta} + \eta K \norm{\widehat{d^\expert}}_2/2 \\
    &\leq \frac{2}{\eta} + \eta K \norm{\widehat{d^\expert}}_1/2 \\
    &\leq \frac{2}{\eta} + \frac{\eta K}{2}\\
\end{align*}
Therefore choosing $\eta = \sqrt{\frac{4}{K}}$, we get
\begin{align*}
    \sum^K_{k=1} \innerprod{c_{\mathrm{true}} - c^k}{\mathbf{e}_{s^k_{L^k}} - \widehat{d^\expert}} &\leq 2\sqrt{K}.
\end{align*}
Then, we can handle the remaining two terms. In particular $\sum^K_{k=1} \innerprod{c_{\mathrm{true}} - c^k}{ d^{\pi^k}-\mathbf{e}_{s^k_{L^k}}}$ is the sum of a martingale difference sequence.
Therefore, applying the Azuma-Hoeffding inequality it holds that with probability $1-\delta$
\begin{equation*}
\sum^K_{k=1} \innerprod{c_{\mathrm{true}} - c^k}{ d^{\pi^k}-\mathbf{e}_{s^k_{L^k}}} \leq \sqrt{2 K \log (1/\delta)}
\end{equation*}
where we used that $\abs{\innerprod{c_{\mathrm{true}} - c^k}{ d^{\pi^k}-\mathbf{e}_{s^k_{L^k}}}} \leq 2$ for all $k \in [K]$.
Finally for the expert concentration term, we have that
\begin{align*}
    \sum^K_{k=1} \innerprod{c_{\mathrm{true}} - c^k}{ \widehat{d^\expert} - d^\expert} \leq K \sqrt{\abs{\sspace}\abs{\aspace}} \norm{d^{\expert} - d^\expert}_{\infty}
\end{align*}
Then, for any fixed state action pair $s,a$ with probability $1- \delta/(\abs{\sspace}\abs{\aspace})$ by Azuma-Hoeffding inequality it holds that
\begin{align*}
d^{\expert}(s) - d^\expert(s) = \frac{1}{\abs{\mathcal{D}_\expert}} \sum_{s' \in \mathcal{D}_\expert } \mathds{1}_{\bc{s'=s}} - d^\expert (s) \leq \sqrt{\frac{\log (\abs{\sspace}\abs{\aspace}\delta^{-1})}{2 \abs{\mathcal{D}_\expert}}}
\end{align*}
Therefore, by a union bound it holds that with probability $ 1-\delta$,
\begin{equation*}
\norm{d^{\expert} - d^\expert}_{\infty} \leq \sqrt{\frac{\log (\abs{\sspace} \abs{\aspace}\delta^{-1})}{2 \abs{\mathcal{D}_\expert}}}.
\end{equation*}
Putting together, the bounds on the three terms allow to conclude the proof.
\end{proof}
\section{Technical Lemmas}
\lemmapdl*
\begin{proof}
    \begin{equation*}
        \innerprod{d^{\pi'}}{\hat{Q}} = \innerprod{d^{\pi'}}{\hat{Q} - \cost - \gamma P \hat{V}^\pi} + \innerprod{d^{\pi'}}{\cost + \gamma P \hat{V}^\pi}
    \end{equation*}
Then, using the property of occupancy measure we have that $\innerprod{d^{\pi'}}{\cost} = (1 - \gamma)\innerprod{\initial}{V^{\pi'}}$ where $V^{\pi'}$ is the value function of the policy $\pi'$ in the MDP.
Then, it holds that
\begin{align*}
        \innerprod{d^{\pi'}}{\hat{Q}} &= \innerprod{d^{\pi'}}{\hat{Q} - \cost - \gamma P \hat{V}^\pi} + (1 - \gamma)\innerprod{\initial}{V^{\pi'}}+\innerprod{d^{\pi'}}{\gamma P \hat{V}^\pi} \\
        & = \innerprod{d^{\pi'}}{\hat{Q} - \cost - \gamma P \hat{V}^\pi} + (1 - \gamma)\innerprod{\initial}{V^{\pi'}}+\innerprod{\gamma P^T d^{\pi'}}{\hat{V}^\pi}
        \\ & = \innerprod{d^{\pi'}}{\hat{Q} - \cost - \gamma P \hat{V}^\pi} + (1 - \gamma)\innerprod{\initial}{V^{\pi'}}+\innerprod{E^T d^{\pi'} -(1 - \gamma) \initial}{\hat{V}^\pi} \\
        & = \innerprod{d^{\pi'}}{\hat{Q} - \cost - \gamma P \hat{V}^\pi} + (1 - \gamma)\innerprod{\initial}{V^{\pi'} - \hat{V}^\pi}+\innerprod{E^T d^{\pi'}}{\hat{V}^\pi}.
    \end{align*}
    Rearranging and using the definition of $\widehat{V}^\pi$ yields the conclusion.
\end{proof}
\begin{lemma}
\label{lemma:count_based}
Let us assume that $L^k \leq L_{\max}$ for all $k \in [K]$. Then, it holds that with probability $1-\delta$
\begin{equation*}
\sum^K_{k=1}\mathbb{E}_{s,a\sim d^{\pi^k}} \bs{\frac{1}{N^k(s,a)/L + 1}}  \leq 
2 L \abs{\sspace} \abs{\aspace} \log \br{K L_{\max}} + 4 \log \br{2KL_{\max}/\delta} 
\end{equation*}
\end{lemma}
\begin{proof}
Let us assume that $L^k \leq L_{\max}$ for all $k \in [K]$. It holds that with probability $1-\delta$
\begin{align*}
\sum^K_{k=1}\mathbb{E}_{s,a\sim d^{\pi^k}} \bs{\frac{1}{N^k(s,a)/L + 1}} &\leq 2\sum^K_{k=1}\sum^{L^k}_{t=1}\frac{1}{N^k(s^k_t,a^k_t)/L + 1} + 4 \log \br{2KL_{\max}/\delta} \\
& \leq 2\sum_{s,a\in\sspace\times\aspace}\sum^K_{k=1}\sum^{L^k}_{t=1}\frac{\mathds{1}_{\bc{s,a=s^k_t,a^k_t}}}{N^k(s,a)/L + 1} + 4 \log \br{2KL_{\max}/\delta} \\
& \leq 2 L \sum_{s,a\in\sspace\times\aspace}\sum^K_{k=1}\sum^{L^k}_{t=1}\frac{\mathds{1}_{\bc{s,a=s^k_t,a^k_t}}}{N^k(s,a)+ 1} + 4 \log \br{2KL_{\max}/\delta} \\
& \leq 2 L \sum_{s,a\in\sspace\times\aspace}\sum^K_{k=1}\sum^{L^k}_{t=1}\frac{\mathds{1}_{\bc{s,a=s^k_t,a^k_t}}}{\sum^k_{\tau=1}\sum^{L_\tau}_{t=1} \mathds{1}_{\bc{s,a=s^\tau_t,a^\tau_t}} + 1} + 4 \log \br{2KL_{\max}/\delta} \\
& \leq 2 L \sum_{s,a\in\sspace\times\aspace} \log \br{\sum^K_{k=1}\sum^{L^k}_{t=1}\mathds{1}_{\bc{s,a=s^k_t,a^k_t}}} + 4 \log \br{2KL_{\max}/\delta} \\
& \leq 2 L \abs{\sspace} \abs{\aspace} \log \br{K L_{\max}} + 4 \log \br{2KL_{\max}/\delta} \\
\end{align*}
where we used \Cref{lemma:numerical_sequence} for $f(x) = x^{-1}$.
\end{proof}
\begin{lemma}
\label{lemma:numerical_sequence}
Let $a_0 \geq 0$ and $f [0, \infty ) \rightarrow [0, \infty ) $ be a non increasing function , then
\begin{equation*}
\sum^T_{t=1} \alpha_t f(a_0 + \sum^T_{t=1} \alpha_t) \leq \int^{\sum^T_{t=1} a_t}_{a_0} f(x) dx
\end{equation*}
\end{lemma}
\begin{proof}
See \cite{orabona2023modern} Lemma 4.13.
\end{proof}
\begin{lemma}
    \label{lemma:slow} The sequence of policies $\bc{\pi^k}^K_{k=1}$ generated by \Cref{alg:theory_version}
    and let $d^\pi$ denote the occupancy measure for the policy $\pi$. Then it holds that
    \begin{equation*}
    \forall k ~~\in [K]~~~\norm{d^{\pi^k} - d^{\pi^{k+1}}}_1 \leq \frac{\eta Q_{\max}}{(1-\gamma)}
    \end{equation*}
    \end{lemma}
    \begin{proof}
        By Lemma A.1 in \cite{sun2019dual} it holds that
        \begin{equation*}
    \norm{d^{\pi^k} - d^{\pi^{k+1}}}_1 \leq \frac{1}{1-\gamma} \mathbb{E}_{x \sim d^{\pi^k}}\bs{\norm{\pi^k(\cdot|x) - \pi^{k+1}(\cdot|x)}_1}
        \end{equation*}
    Then, we notice that by $1$-strong convexity of the KL divergence it holds that
        \begin{align*}
     \frac{1}{2}\mathbb{E}_{x \sim d^{\pi^k}}&\bs{\norm{\pi^k(\cdot|x) - \pi^{k+1}(\cdot|x)}^2_1} \leq  \frac{1}{2}\mathbb{E}_{x \sim d^{\pi^k}}\bs{D_{KL}(\pi^{k+1}(\cdot|x), \pi^k(\cdot|x))} \\
     &\leq  \frac{1}{2}\mathbb{E}_{x \sim d^{\pi^k}}\sum_{a \in \aspace} \pi^{k+1}(a|x) \br{- \eta Q_k(x,a) - \log\br{\sum_{a \in \aspace} \pi^{k}(a|x) \exp(- \eta Q_k(x,a)) }} \\
     &= -\frac{\eta}{2} \mathbb{E}_{x \sim d^{\pi^k}}\sum_{a \in \aspace} \pi^{k+1}(a|x) Q_k(x,a) - \frac{1}{2}\mathbb{E}_{x \sim d^{\pi^k}} \log\br{\sum_{a \in \aspace} \pi^{k}(a|x) \exp(- \eta Q_k(x,a)) } \\
     & \leq -\frac{\eta}{2} \mathbb{E}_{x \sim d^{\pi^k}}\sum_{a \in \aspace} \pi^{k+1}(a|x) Q_k(x,a)  + \frac{\eta }{2}\mathbb{E}_{x \sim d^{\pi^k}}\sum_{a \in \aspace} \pi^{k}(a|x) Q_k(x,a)
        \end{align*}
        where the last inequality follows by Jensen's inequality and convexity of $- \log$.
        Hence, we continue the upper bound as follows
        \begin{align*}
    \frac{1}{2}\mathbb{E}_{x \sim d^{\pi^k}}\bs{\norm{\pi^k(\cdot|x) - \pi^{k+1}(\cdot|x)}^2_1} &= \frac{\eta}{2}\mathbb{E}_{x \sim d^{\pi^k}}\sum_{a \in \aspace} Q_k(x,a) \cdot (\pi^k(\cdot|x) - \pi^{k+1}(\cdot|x)) \\
    & \leq \frac{\eta Q_{\max}}{2}\cdot \mathbb{E}_{x \sim d^{\pi^k}}\bs{\norm{\pi^k(\cdot|x) - \pi^{k+1}(\cdot|x)}_1}
        \end{align*}
        Which implies, by Jensen's inequality and diving both sides by $\frac{1}{2}\mathbb{E}_{x \sim d^{\pi^k}}\bs{\norm{\pi^k(\cdot|x) - \pi^{k+1}(\cdot|x)}_1}$ that
        \begin{equation*}
            \mathbb{E}_{x \sim d^{\pi^k}}\bs{\norm{\pi^k(\cdot|x) - \pi^{k+1}(\cdot|x)}_1} \leq \eta Q_{\max}.
        \end{equation*}
    \end{proof}
\begin{lemma}\textbf{Samuelson's inequality}
Let us consider $L$ scalars $\bc{X_{\ell}}^L_{\ell=1}$ and denote the sample mean as $\bar{X} = L^{-1}\sum^L_{\ell=1} X_\ell$
and the empirical standard deviation as $\hat{\sigma} = \sqrt{\frac{\sum^L_{\ell=1}(X_\ell - \bar{X})^2}{L-1}}$, then it holds that
\begin{equation*}
\bar{X} - \sqrt{L-1} \hat{\sigma} \leq X_{\ell} \leq \bar{X} + \sqrt{L-1} \hat{\sigma} ~~~ \forall ~~~\ell\in[L]
\end{equation*}
\label{lemma:samuelson}
\end{lemma}
\begin{proof}
Let us consider an arbitrary vector $v \in \mathbb{R}^L$. Then, we have that $\norm{v}_\infty \leq \norm{v}_2$. At this point let us consider $v = [X_1 - \bar{X}, \dots, X_L - \bar{X}]^T$. Moreover,
let us define as $\ell^\star$ the index such that $\norm{v}_{\infty} = \abs{X_{\ell^\star} - \bar{X}}$.
Then, we have that for all $\ell \in [L]$,
\begin{equation*}
\abs{X_{\ell} - \bar{X}} \leq \abs{X_{\ell^\star} - \bar{X}}
\leq \sqrt{\sum^L_{\ell=1} (X_\ell - \bar{X})^2} = \sqrt{L-1}\hat{\sigma}.
\end{equation*}
Therefore, rewriting the absolute value it holds that
\begin{equation*}
\bar{X} - \sqrt{L-1}\hat{\sigma} \leq X_\ell \leq \bar{X} + \sqrt{L-1}\hat{\sigma}
\end{equation*}
\end{proof}
The next lemma says that the effective horizon in the original MDP and the binarized MDP is equal up to a $\log_2\br{\abs{\sspace}}$ factor.
\begin{lemma} \label{lemma:eff_horizon}
It holds that \begin{equation*}
\frac{1}{1 - \gamma^{1/\log_2 \abs{\sspace}}} \leq \frac{\log_2{\abs{\sspace}} + 2}{1-\gamma}.
\end{equation*}
\end{lemma}
\begin{proof}
\begin{align*}
\frac{1}{1-\gamma^{1/\log_2 \abs{\sspace}}} &= \frac{1}{1-\gamma} \frac{1-\gamma}{1 -\gamma^{1/\log_2 \abs{\sspace}}} \\
&= \frac{1}{1-\gamma} \frac{1-\gamma^{\log_2\abs{\sspace}}_{\mathrm{bin}}}{1 -\gamma_{\mathrm{bin}}} \\
&= \frac{1}{1-\gamma} \sum^{\log_2\abs{\sspace}+1}_{t=0} \gamma^t_{\mathrm{bin}} \\
&\leq \frac{\log_2 \abs{\sspace}+2}{1-\gamma}.
\end{align*}
\end{proof}
\section{Implementation details}
\label{sec:training_proc}

\textbf{Environment:} We use the Hopper-v5, Ant-v5, HalfCheetah-v5, and Walker2d-v5 environments from OpenAI Gym.

\textbf{Expert Samples:} The expert policy is trained using SAC. The training configuration uses 3000 epochs. The agent explores randomly for the first 10 episodes before starting policy learning. A replay buffer of 1 million experiences is used, with a batch size of 100 and a learning rate of 1e-3. The temperature parameter ($\alpha$) is set to 0.2. The policy updates occur every 50 steps, with 1 update per interval. After training 64 experts trajectories are collected to be used later for the agent training. 

\textbf{IL algorithms implementation:} Our starting code base is taken from the repository of \texttt{f-IRL}\footnote{\url{https://github.com/twni2016/f-IRL/tree/main}} \cite{ni2021f}, and the implementation of the other algorithms are based on this one. For more details about the implementation please refer to our repository. %\footnote{\url{https://github.com/stefanoviel/SOAR-IL/tree/master}} 
The most important hyperparameters are reported in Table \ref{tab:hyperparameters}

\begin{itemize}
   \item \textbf{ML-IRL, f-IRL and rkl:} These algorithms were already implemented in the \texttt{f-IRL} repository. The method leverages SAC as the underlying reinforcement learning algorithm and different type of objectives for the cost update. The multi-Q-network exploration bonus is implemented inside the SAC update, there we keep track of multiple Q-networks and use their mean and standard deviation to update the policy. The clipping is applied on the standard deviation which serves as the exploration bonus. 
   \item \textbf{CSIL:} We started from the \texttt{f-IRL} implementation, maintaining the same hyperparameters for a fair comparison. The key modification was removing reward model training from the RL loop, instead training it only once before entering the loop using behavioral cloning and $L_2$ normalization, after which the reward model remained fixed throughout the training.
   \item \textbf{OPT-AIL (state-only and state-action):} We started from the implementation of \texttt{ML-IRL} and added the OPT-AIL exploration bonus, incorporating optimism-regularized Bellman error minimization for Q-value functions as described in the original article \cite{xu2024provably}.
    
    The updated Q-loss can be formulated as:
    
    \begin{equation}
    \mathcal{L}_Q = \mathbb{E}\left[\left(Q_\theta(s,a) - \left(r + \gamma(1-d)(Q_{\bar{\theta}}(s',a') - \alpha \log \pi(a'|s'))\right)\right)^2\right] - \lambda \mathbb{E}[Q_\theta(s,a)]
    \end{equation}
    
    Where:
    - $Q_\theta$ is the current Q-network
    - $Q_{\bar{\theta}}$ is the target Q-network
    - $r$ is the learned reward
    - $\gamma$ is the discount factor
    - $d$ is the done flag
    - $\alpha$ is the entropy coefficient
    - $\lambda$ is the optimism regularization parameter
    - The expectation is taken across the data distribution $\mathcal{D}$ sampled from the replay buffer, which includes state-action-reward-next state-done transitions and individual state-action pairs.

   \item \textbf{SQIL} \cite{sqil}: was implemented by initializing a replay buffer with expert trajectories and assigning them a reward of 1, while collecting additional on-policy trajectories from the agent's current policy with a reward of 0. During training, the \texttt{SAC} agent learns from both expert and agent-generated transitions, effectively learning to imitate expert behavior through the asymmetric reward structure. The agent updates its policy by sampling from this mixed replay buffer, where the expert transitions provide a high-reward signal to guide the learning process.
   
   \item \textbf{GAILs:} we used the implementation available from \texttt{Stable-Baselines3} \cite{stable-baselines3}. This is the only method not based on \texttt{SAC}.
\end{itemize}



\begin{table}[]
\centering
\caption{Core Hyperparameters Across Environments}
\begin{tabular}{lcccc}
\toprule
Parameter & Walker2d & Humanoid & Hopper & Ant \\
\midrule
\text{Number of Iterations} & 1.5\,\text{M} & 1\,\text{M} & 1\,\text{M} & 1.2\,\text{M} \\
Reward Network size & [64, 64] & [64, 64] & [64, 64] & [128, 128] \\
Policy Network size & [256, 256] & [256, 256] & [256, 256] & [256, 256] \\
Reward Learning Rate & 1e-4 & 1e-4 & 1e-4 & 1e-4 \\
SAC Learning Rate & 1e-3 & 1e-3 & 1e-3 & 1e-3 \\
% Expert Trajectories & 16 & 16 & 16 & 16 \\
\bottomrule
 \label{tab:hyperparameters}
\end{tabular}
\end{table}

\subsection{Hyperparameters tuning}

To determine how different numbers of neural networks changed the performance, we conducted an ablation study on both the clipping value and the number of neural networks. We performed this analysis for the ML-IRL algorithm on the Ant-V5 environment. We chose this environment since previous experiments showed that its higher complexity led to higher variance in the performance of different algorithms. It was necessary to perform a grid search on the number of neural networks because we noticed that different numbers of neural networks preferred different clipping values.

The results are reported in \Cref{fig:best_clip_nn}. As we observed, in all cases, adding more neural networks leads to better performances. However, this improvement does not increase proportionally with the number of neural networks; in fact, the run with 10 neural networks is outperformed by the one with 4. This led us to select 4 as the fixed value of neural networks, also justified by the much slower training time of the 10-network case.

\begin{figure}[]
\centering
\includegraphics[width=0.6\linewidth]{final_figs/best_clipping_returns.png}
\label{fig:best_clip_nn}
\caption{Mean return of ML-IRL Ant-v5 with a different number of neural networks. The grid search for the clipping values was performed over the following values [0.1 0.5 1 5 10 50]. Results are averaged over 3 seeds.}
\end{figure}


For every environment and algorithm, we performed a grid search over different clipping values. The range of clipping values varied across algorithms. Figure \ref{fig:comparison_plots} shows the different values used in the search and their impact on performance. The difference in performance across clipping values is small in simpler environments (e.g., Hopper or Walker2d) while it becomes more evident in more complex environments with larger state-action spaces. These plots also show the necessity of the clipping for the exploration bonus. In most environments and algorithms, when a large clipping value is applied, it leads to performance degradation. 

Empirically, the Q-network's standard deviation diverges due to unclipped Q-values. Without value clipping, high Q-values for specific state-action pairs increase the probability of being visited, causing more of these pairs to accumulate in the replay buffer across different rollouts and potentially amplifying the standard deviation across the q-network for the next update. This justifies the necessity of a clipping value on the exploration bonus. 



\begin{figure}[]
    \centering
    \begin{subfigure}[b]{0.48\textwidth}
        \includegraphics[width=\textwidth]{clipping_gridsearch/cisl_comparison_without_ci.png}
        \caption{CSIL Comparison}
    \end{subfigure}
    \hfill
    \begin{subfigure}[b]{0.48\textwidth}
        \includegraphics[width=\textwidth]{clipping_gridsearch/maxentirl_comparison_without_ci.png}
        \caption{ML-IRL Comparison}
    \end{subfigure}

    \vspace{1em}

    \begin{subfigure}[b]{0.48\textwidth}
        \includegraphics[width=\textwidth]{clipping_gridsearch/maxentirl_sa_comparison_without_ci.png}
        \caption{ML-IRL-SA Comparison}
    \end{subfigure}
    \hfill
    \begin{subfigure}[b]{0.48\textwidth}
        \includegraphics[width=\textwidth]{clipping_gridsearch/rkl_comparison_without_ci.png}
        \caption{RKL Comparison}
    \end{subfigure}
    
    \caption{Comparison of clipping values across different environments, showing the effect on average return for each environment.}
    \label{fig:comparison_plots}
\end{figure}

\newpage
\section{Experiments with single expert trajectory}

Here we report the we report the result of the experiments using a single trajectory. Our findings indicate that the performance remained consistent regardless of the number of trajectories used and the performance are comparable to the ones with 16 trajectories. Notable differences in performance improvement were observed in Humanoid-v5 and ant state environments in the state only settings, where a more pronounced gap was evident.

\begin{figure*}[h]
    \centering
\includegraphics[width=\textwidth]{final_figs/state_only1.png}
    \caption{%Comparison of state-only imitation learning methods across OpenAI Gym environments, with dashed lines showing baseline algorithms and solid lines showing SOAR-enhanced versions. 
    \small{\textbf{Experiments from State-Only Expert Trajectories}. 1 expert trajectories, average over 3 seeds, $L=4$ 
     Clipping values $\sigma$ - ML-IRL: [Ant: 0.1, Hopper: 0.1, Walker2d: 50.0, Humanoid: 0.5],  
    rkl: [Ant: 0.1, Hopper: 0.5, Walker2d: 1.0, Humanoid: 50.0]}}
    \label{fig:state_only_1traj}
\end{figure*}

\begin{figure*}[h]
    \centering
\includegraphics[width=\textwidth]{final_figs/state_actions1.png}
    \caption{%Performance of state-action imitation learning methods across environments, with dashed lines showing baseline algorithms and solid lines showing SOAR-enhanced versions. 
    \small{\textbf{Experiments from State-Action Expert Trajectories}. 1 expert trajectories, average over 3 seeds, $L=4$.
    Clipping values $\sigma$ - CSIL: [Ant: 0.5, Hopper: 50.0, Walker2d: 0.1, Humanoid: 0.1],  
ML-IRL(SA): [Ant: 0.1, Hopper: 0.5, Walker2d: 0.1, Humanoid: 1.0]}}
\label{fig:state_actions_1traj}
\end{figure*}

\section{Omitted Pseudocodes}
\label{app:pseudo}
This section introduces the omitted pseudocodes to clarify the implementation of the algorithms based on SOAR. We first give a pseudocode (see \Cref{alg:deepmeta}) that mirrors \Cref{alg:meta} in the setting where deep neural network approximation is needed due to the continuous structure of the state-action space.
The critic training is the same as in the standard SAC \cite{Haarnoja:2018} but we report it in \Cref{alg:updateCritics} for safe completeness. 
Notice that we adopt the double critic training originally proposed in \cite{vanhasselt2015deepreinforcementlearningdouble} to avoid an excessive underestimation of the critics value\footnote{Notice that \cite{vanhasselt2015deepreinforcementlearningdouble} talks about excessive overestimation of the prediction target in the critic training rather than underestimation. This difference is due to the fact that their paper casts RL as reward maximization while we adopt a cost minimization perspective. For the same reason we take the maximum between the two critics rather than the minimum as done in \cite{vanhasselt2015deepreinforcementlearningdouble}.}.
\begin{algorithm}[H]
\caption{Base Method + SOAR pseudocode \label{alg:deepmeta}}
\begin{algorithmic}[1]
\REQUIRE Policy step size $\eta$, cost step size $\alpha$, expert dataset $\mathcal{D}_{\tau_E}$, discount factor $\gamma$, maximum standard deviation parameter $\sigma$, %critics network $\{Q^1_\ell, Q^1_\ell^\text{targ}\}_{\ell=1}^L$.
\State Initialize actor network $\pi_{\psi^1}$ randomly.
\State Initialize the cost network $c_{w^1}$ randomly.
\STATE Initialize the $L$ critics $\{Q_{\theta^1_1}, \ldots, Q_{\theta^1_L}\}$ randomly.
\STATE Initialize the $L$ target critics $\{Q_{\theta^{1,\text{targ}}_1}, \ldots, Q_{\theta^{1,\text{targ}}_L}\}$ randomly.
\State Initialize $L$ empty replay buffers $\bc{\mathcal{D}^k_\ell}^L_{\ell=1}$. (One for each critic)
\For{$k = 1$ to $K$}
    \State $\tau_\ell^k \gets \textsc{CollectTrajectory}(\pi)$ for each $\ell \in [L]$.
    \State Add $\tau_\ell^k$ to replay buffer $\mathcal{D}_\ell^k \gets \mathcal{D}_\ell^{k-1} \cup \tau_\ell^k$.
    \State Let $\mathcal{D}^k = \cup^L_{\ell=1} \mathcal{D}^k_\ell$
    \State $ c_{w^{k}} \gets \textsc{UpdateCost}(c_{w^{k-1}}, \mathcal{D}_{\expert}, \mathcal{D}^k, \alpha)$ using the Base Method (such as CSIL, $f$-IRL or ML-IRL ). 
    \For{$\ell = 1$ to $L$}
        \State $Q_{\theta^{k+1}_\ell}, Q_{\theta_\ell^{k+1,\text{targ}}} = \textsc{UpdateCritics}(\mathcal{D}^k_\ell, \pi_{\psi^k}, \eta, \gamma, c_{\theta^k})$
    \EndFor
    \State $\bc{Q^{k+1}(s,a)}_{s,a \in \mathcal{D}^k} = \textsc{OptimisticQ-NN}(\mathcal{D}^k, 
\bc{Q_{\theta^{k+1}_\ell}}^L_{\ell=1}, \sigma)~~~~~~$ (see \Cref{alg:optQnn})
    \State Define the loss
    $\mathcal{L}^k_\pi = \frac{1}{\abs{\mathcal{D}^k}} \sum_{s,a \in \mathcal{D}^k}\left( -\eta \log \pi_{\psi^k}(a|s) +  Q^{k+1}(s,a) \right).
$
    %$\pi_{\psi^{k+1}} \gets %\textsc{UpdatePi}(\cup^L_{\ell=1} \mathcal{D}^k_\ell, \{Q_{\theta^k_\ell}\}_{\ell=1}^L, \pi_{\psi^k}, \eta, \sigma)$
    \State Update policy weights to $\psi^{k+1}$ using Adam \cite{Kingma:2015} on the loss $\mathcal{L}^k_\pi$.
\EndFor
\State \textbf{Return} $\pi$
\end{algorithmic}
\end{algorithm}



\begin{algorithm}[H]
\caption{\textsc{UpdateCritics} \label{alg:updateCritics}}
\begin{algorithmic}[1]
\REQUIRE $\mathcal{D}^k_\ell, \pi_{\psi^k}, \alpha, \gamma,  c_w$
    \State Let $\mathcal{B} = \bc{s_i, a_i, r, s'_i, \text{done}_i}^N_{i=1}$ be a minibatch sampled from $\mathcal{D}$
    %\State $q_k^{(1)}(s, a) \gets Q_k^{(1)}(s, a)$
    %\State $q_k^{(2)}(s, a) \gets Q_k^{(2)}(s, a)$
    \State $a'_i \gets \pi(s'_i)$ for all $i \in [N]$.
    %\State $q_\text{targ}^{(1)} \gets Q_k^{\text{targ},(1)}(s', a')$
    %\State $q_\text{targ}^{(2)} \gets Q_k^{\text{targ},(2)}(s', a')$
    \State Define $Q_{\theta_{\ell}^{k,\text{targ}}}(s_i,a_i) \gets \max \left( Q_{\theta_{\ell}^{k,\text{targ},(1)}}(s_i,a_i), Q_{\theta_{\ell}^{k,\text{targ},(2)}}(s_i,a_i) \right)$ for all $s_i,a_i \in \mathcal{B}$. 
    \State $\text{Backup}_i \gets c_w(s_i,a_i) + \gamma (1 - \text{done}_i) \left( Q_{\theta_{\ell}^{k,\text{targ}}}(s_i,a_i) + \alpha \log \pi_{\psi^k}(a'_i|s'_i) \right)$
    \State $\mathcal{L}_{\theta^{k,(1)}_\ell} = \frac{1}{N} \sum_{i=1}^N \left( Q_{\theta^{k,(1)}_\ell}(s_i, a_i) - \text{Backup}_i \right)^2$
    \State $\mathcal{L}_{\theta^{k,(2)}_\ell} = \frac{1}{N} \sum_{i=1}^N \left( Q_{\theta^{k,(2)}_\ell}(s_i, a_i) - \text{Backup}_i \right)^2$
    \State $\theta^{k+1,(1)}_\ell \gets \theta^{k,(1)}_\ell - \eta_Q \nabla \mathcal{L}_{\theta^{k,(1)}_\ell}$
    \State $\theta^{k+1,(2)}_\ell \gets \theta^{k,(2)}_\ell - \eta_Q \nabla \mathcal{L}_{\theta^{k,(2)}_\ell}$.
    \State $\theta^{k+1, \text{targ},(1)} \gets (1 - \tau_{\text{targ}})\theta^{k, \text{targ},(1)} + \tau_{\text{targ}} \theta^{k,(1)} $.
    \State $\theta^{k+1, \text{targ},(2)} \gets (1 - \tau_{\text{targ}})\theta^{k, \text{targ},(2)} + \tau_{\text{targ}} \theta^{k,(2)} $.
    \State $Q_{\theta^{k+1}_\ell}(s,a) = \max \br{ Q_{\theta^{k+1,(1)}_\ell}(s,a), Q_{\theta^{k+1,(2)}_\ell}(s,a)}$ for all $s,a \in \mathcal{D}^k$.
    \State $Q_{\theta^{k+1, \text{targ}}_\ell}(s,a) = \max \br{ Q_{\theta^{k+1, \text{targ},(1)}_\ell}(s,a), Q_{\theta^{k+1, \text{targ},(2)}_\ell}(s,a)}$ for all $s,a \in \mathcal{D}^k$.
    \State \textbf{return} $Q_{\theta^{k+1}_\ell}, Q_{\theta^{k+1, \text{targ}}_\ell} $.
\end{algorithmic}
\end{algorithm}
\subsection{Instantiating the cost update}
We show after how the algorithmic template in \Cref{alg:meta} captures different imitation learning algorithms just changing the cost update.
For example, $f$-IRL with the reversed KL divergence (RKL) can be seen as \Cref{alg:meta} with the cost update described in \Cref{alg:fIRL}.
Moreover, our SOAR+RKL is obtained plugging in the cost update in \Cref{alg:fIRL} in \Cref{alg:deepmeta}.
\begin{algorithm}[H]
\caption{\textsc{UpdateCost} for RKL ($f$-IRL for reversed KL divergence) \cite{ni2021f} \label{alg:fIRL}}
\begin{algorithmic}[1]
\REQUIRE $c, \mathcal{D}_{\expert}, \mathcal{D}_{\pi^k}, \alpha$, divergence generating function $f(x) = - \log (x)$ for the reversed KL divergence, prior distribution over trajectories $p(\tau)$.
    \State $\rho_w(\tau) = \frac{1}{Z} p(\tau) e^{-c_w(\tau)}$
    \State $\chi^\star \gets \argmax_\omega \mathbb{E}_{s \sim \mathcal{D}_{\expert}} [\log D_\chi(s)] + \mathbb{E}_{s \sim \mathcal{D}^k} [\log (1 - D_\chi(s))]$
    % set is as a sum over the states
    \vspace{0.3em}
    \State Estimate the density ratio:
    \State $\frac{\rho_E(s)}{\rho_w(s)} = \frac{D_{\chi^\star}(s)}{1 - D_{\chi^\star}(s)}$
    \vspace{0.3em}
    % \State $\mathcal{L}_f(\theta) = D_f\left(\rho_E(s) \parallel \rho_\theta(s)\right)$
    \State Compute the stochastic gradient $$\widehat{\nabla_w} = \frac{1}{ T} \mathbb{E}_{\tau \sim \rho_w} 
    \bs{
    \sum_{t=1}^T h_f \left( \frac{\rho_E(s_t)}{\rho_w(s_t)}  \right) \cdot
    \left( - \sum_{t=1}^T \nabla_\theta c_w(s_t) \right)
    } - \frac{1}{ T} \mathbb{E}_{\tau \sim \rho_w} 
    \bs{
    \sum_{t=1}^T h_f \left( \frac{\rho_E(s_t)}{\rho_\theta(s_t)}  \right)} \cdot
    \mathbb{E}_{\tau \sim \rho_\theta} 
    \bs{\left( - \sum_{t=1}^T \nabla_\theta c_w(s_t) \right)
    }$$
    %\State $\widehat{\nabla_\theta L_f(\theta)} = \frac{1}{\alpha T} \, \text{cov}_{\tau \sim \rho_\theta(\tau)} 
    %\left( 
    %\left( \sum_{t=1}^T h_f \left( \frac{\rho_E(s_t)}{\rho_\theta(s_t)} \right) \right),
    %\left( \sum_{t=1}^T \nabla_\theta r_\theta(s_t) \right)
    %\right)$
    \State $w \gets w - \alpha \widehat{\nabla_w}$
    \State \textbf{Return} $c_w$
\end{algorithmic}
\end{algorithm}
Next, we present the cost update for the algorithm ML-IRL \cite{zeng2022maximum}. We present it for the state-action version. The state-only version is obtained simply omitting the action dependence everywhere.
\begin{algorithm}[H]
\caption{\textsc{UpdateCost} for ML-IRL (State-Action version) \cite{zeng2022maximum} \label{alg:MLIRL}}
\begin{algorithmic}[1]
\REQUIRE $c_{w}, \mathcal{D}_{\expert}, \tau^k = \bc{s^k_t,a^k_t}^{L^k}_{t=1}, \alpha$.
    \State Sample a state-action trajectory $\tau_E = \bc{s_t^E, a_t^E}^{L_E}_{t=1}$ from the expert dataset $\mathcal{D}_\expert$ where $L_E$ is a geometric random variable with parameter $(1-\gamma)^{-1}$.
    \State Compute the stochastic loss 
    $$ \widehat{\mathcal{L}}_{w}  =\sum^{L_E}_{t=0} \gamma^t c_w(s^E_t,a^E_t) -  \sum^{L^k}_{t=0} \gamma^t c_w(s^k_t,a^k_t)
    $$
    
    
    \State $w \gets w - \alpha \nabla_w \widehat{ \mathcal{L}_w}$
    \State \textbf{Return} $c_w$
\end{algorithmic}
\end{algorithm}

To conclude, we present the cost update for CSIL. Notice that since the cost used by CSIL does not leverage the information of the policy at iteration $k$ we can move the cost update before the main loop and keep a constant cost function fixed during the training of the policy.
Notice that the CSIL the reward is simply given by the log probabilities learned by the behavioural cloning policy. Therefore the reward parameters in CSIL coincides with the parameters of the behavioral cloning policy network.
We point out that using a reward of this form is similarly done in \cite{vieillard2020munchausen}. We plan to explore further the connection between Munchausen RL and CSIL in future work.

\begin{algorithm}[H]
\caption{\textsc{UpdateCost} for CSIL \cite{watson2023coherent}\label{alg:CSIL}}
\begin{algorithmic}[1]
\REQUIRE $\mathcal{D}_{\expert}$.
    \State Compute the behavioral cloning policy finding an approximate solution to the following problem.
    $$
    w^\star = \argmax_{w} \sum_{s,a \in \mathcal{D}_\expert}\log \pi_{w}(a|s)
    $$
    \State \textbf{Return} $c_{w^\star}(s,a) = - \log \pi_{w^\star}(a|s)$
\end{algorithmic}
\end{algorithm}

%%%%%%%%%%%%%%%%%%%%%%%%%%%%%%%%%%%%%%%%%%%%%%%%%%%%%%%%%%%%%%%%%%%%%%%%%%%%%%%
%%%%%%%%%%%%%%%%%%%%%%%%%%%%%%%%%%%%%%%%%%%%%%%%%%%%%%%%%%%%%%%%%%%%%%%%%%%%%%%


\end{document}


% This document was modified from the file originally made available by
% Pat Langley and Andrea Danyluk for ICML-2K. This version was created
% by Iain Murray in 2018, and modified by Alexandre Bouchard in
% 2019 and 2021 and by Csaba Szepesvari, Gang Niu and Sivan Sabato in 2022.
% Modified again in 2023 and 2024 by Sivan Sabato and Jonathan Scarlett.
% Previous contributors include Dan Roy, Lise Getoor and Tobias
% Scheffer, which was slightly modified from the 2010 version by
% Thorsten Joachims & Johannes Fuernkranz, slightly modified from the
% 2009 version by Kiri Wagstaff and Sam Roweis's 2008 version, which is
% slightly modified from Prasad Tadepalli's 2007 version which is a
% lightly changed version of the previous year's version by Andrew
% Moore, which was in turn edited from those of Kristian Kersting and
% Codrina Lauth. Alex Smola contributed to the algorithmic style files.
