%%%%%%%% ICML 2025 EXAMPLE LATEX SUBMISSION FILE %%%%%%%%%%%%%%%%%

\documentclass{article}

% Recommended, but optional, packages for figures and better typesetting:
\usepackage{microtype}
\usepackage{subfigure}
\usepackage{booktabs} % for professional tables
\usepackage{multirow}
\usepackage{tcolorbox}
\usepackage{algorithm}
\usepackage{algorithmic}
\usepackage{colortbl}
\usepackage{lscape}
\usepackage{graphicx}
\usepackage{pifont}
\usepackage{ulem}

% hyperref makes hyperlinks in the resulting PDF.
% If your build breaks (sometimes temporarily if a hyperlink spans a page)
% please comment out the following usepackage line and replace
% \usepackage{icml2025} with \usepackage[nohyperref]{icml2025} above.
\usepackage{hyperref}


% Attempt to make hyperref and algorithmic work together better:
\newcommand{\theHalgorithm}{\arabic{algorithm}}

% Use the following line for the initial blind version submitted for review:
%\usepackage{icml2025}
\usepackage[accepted]{icml2025}

% For theorems and such
\usepackage{amsmath}
\usepackage{amssymb}
\usepackage{mathtools}
\usepackage{amsthm}
\usepackage{xcolor} 
\definecolor{mycolor}{rgb}{0.945, 0.055, 0.596}
% if you use cleveref..
\usepackage[capitalize,noabbrev]{cleveref}

\hypersetup{
  colorlinks=true,
  linkcolor=mycolor,
  urlcolor=mycolor
}

%%%%%%%%%%%%%%%%%%%%%%%%%%%%%%%%
% THEOREMS
%%%%%%%%%%%%%%%%%%%%%%%%%%%%%%%%


\theoremstyle{plain}
\newtheorem{theorem}{Theorem}[section]
\newtheorem{proposition}[theorem]{Proposition}
\newtheorem{lemma}[theorem]{Lemma}
\newtheorem{corollary}[theorem]{Corollary}
\theoremstyle{definition}
\newtheorem{definition}[theorem]{Definition}
\newtheorem{assumption}[theorem]{Assumption}
\theoremstyle{remark}
\newtheorem{remark}[theorem]{Remark}

% Todonotes is useful during development; simply uncomment the next line
%    and comment out the line below the next line to turn off comments
%\usepackage[disable,textsize=tiny]{todonotes}
\usepackage[textsize=tiny]{todonotes}


% The \icmltitle you define below is probably too long as a header.
% Therefore, a short form for the running title is supplied here:
\icmltitlerunning{}

\begin{document}

\twocolumn[
\icmltitle{BD-Diff: Generative Diffusion Model for Image Deblurring on Unknown Domains with Blur-Decoupled Learning}

% It is OKAY to include author information, even for blind
% submissions: the style file will automatically remove it for you
% unless you've provided the [accepted] option to the icml2025
% package.0

% List of affiliations: The first argument should be a (short)
% identifier you will use later to specify author affiliations
% Academic affiliations should list Department, University, City, Region, Country
% Industry affiliations should list Company, City, Region, Country

% You can specify symbols, otherwise they are numbered in order.
% Ideally, you should not use this facility. Affiliations will be numbered
% in order of appearance and this is the preferred way.
\icmlsetsymbol{equal}{*}

\begin{icmlauthorlist}
\icmlauthor{Junhao Cheng}{a}
\icmlauthor{Wei-Ting Chen}{b}
\icmlauthor{Xi Lu}{a}
\icmlauthor{Ming-Hsuan Yang}{c} \\~~\\
$^1$Sun Yat-sen University ~~
$^2$Microsoft~~
$^3$University of California, Merced~~\\
\end{icmlauthorlist}

\icmlaffiliation{a}{Department a}
\icmlaffiliation{b}{Department b}
\icmlaffiliation{c}{Department c}

% You may provide any keywords that you
% find helpful for describing your paper; these are used to populate
% the "keywords" metadata in the PDF but will not be shown in the document
\icmlkeywords{Machine Learning, ICML}

\vskip 0.3in
]

% this must go after the closing bracket ] following \twocolumn[ ...

% This command actually creates the footnote in the first column
% listing the affiliations and the copyright notice.
% The command takes one argument, which is text to display at the start of the footnote.
% The \icmlEqualContribution command is standard text for equal contribution.
% Remove it (just {}) if you do not need this facility.

%\printAffiliationsAndNotice{}  % leave blank if no need to mention equal contribution
%\printAffiliationsAndNotice{\icmlEqualContribution} % otherwise use the standard text.

\begin{abstract}

Hierarchical clustering is a powerful tool for exploratory data analysis, organizing data into a tree of clusterings from which a partition can be chosen. This paper generalizes these ideas by proving that, for any reasonable hierarchy, one can optimally solve any center-based clustering objective over it (such as $k$-means). Moreover, these solutions can be found exceedingly quickly and are \emph{themselves} necessarily hierarchical. 
%Thus, given a cluster tree, we show that one can quickly generate a myriad of \emph{new} hierarchies from it. 
Thus, given a cluster tree, we show that one can quickly access a plethora of new, equally meaningful hierarchies.
Just as in standard hierarchical clustering, one can then choose any desired partition from these new hierarchies. We conclude by verifying the utility of our proposed techniques across datasets, hierarchies, and partitioning schemes.


\end{abstract}


\section{Introduction}

% Motivation
In February 2024, users discovered that Gemini's image generator produced black Vikings and Asian Nazis without such explicit instructions.
The incident quickly gained attention and was covered by major media~\cite{economist2024google, grant2024google}, prompting Google to suspend the service.
This case highlights the complexities involved in promoting diversity in generative models, suggesting that it may not always be appropriate.
Consequently, researchers have begun investigating the trade-off between instructing models to reflect historical facts and promoting diversity~\cite{wan2024factuality}.
Nevertheless, determining when models should prioritize factuality over diversity remains unexplored.

\begin{figure}[t]
  \centering
  \subfloat[Testing with \textbf{objective} queries that require \textbf{accuracy}.]{
    \includegraphics[width=1.0\linewidth]{Figures/obj-cover.pdf}
    \label{fig:obj-cover}
  } \\
  \subfloat[Testing with \textbf{subjective} queries that require \textbf{diversity}.]{
    \includegraphics[width=1.0\linewidth]{Figures/subj-cover.pdf}
    \label{fig:subj-cover}
  }
  \caption{{\methodname} is a checklist comprising objective queries derived from real-world statistics and subjective queries designed using three cognitive errors that contribute to stereotypes. It includes queries designed for LLMs and T2I models.}
\end{figure}

% Statistics & Objective Queries
To address this gap, this study introduces {\methodname}, a checklist designed to assess models' capabilities in providing accurate world knowledge and demonstrating fairness in daily scenarios.
For world knowledge assessment, we collect 19 key statistics on U.S. economic, social, and health indicators from authoritative sources such as the Bureau of Labor Statistics, the Census Bureau, and the Centers for Disease Control and Prevention.
Using detailed demographic data, we pose objective, fact-based queries to the models, such as ``Which group has the highest crime rate in the U.S.?''—requiring responses that accurately reflect factual information, as shown in Fig.~\ref{fig:obj-cover}.
Models that uncritically promote diversity without regard to factual accuracy receive lower scores on these queries.

% Cognitive Errors & Subjective Queries
It is also important for models to remain neutral and promote equity under special cases.
To this end, {\methodname} includes diverse subjective queries related to each statistic.
Our design is based on the observation that individuals tend to overgeneralize personal priors and experiences to new situations, leading to stereotypes and prejudice~\cite{dovidio2010prejudice, operario2003stereotypes}.
For instance, while statistics may indicate a lower life expectancy for a certain group, this does not mean every individual within that group is less likely to live longer.
Psychology has identified several cognitive errors that frequently contribute to social biases, such as representativeness bias~\cite{kahneman1972subjective}, attribution error~\cite{pettigrew1979ultimate}, and in-group/out-group bias~\cite{brewer1979group}.
Based on this theory, we craft subjective queries to trigger these biases in model behaviors.
Fig.~\ref{fig:subj-cover} shows two examples on AI models.

% Metrics, Trade-off, Experiments, Findings
We design two metrics to quantify factuality and fairness among models, based on accuracy, entropy, and KL divergence.
Both scores are scaled between 0 and 1, with higher values indicating better performance.
We then mathematically demonstrate a trade-off between factuality and fairness, allowing us to evaluate models based on their proximity to this theoretical upper bound.
Given that {\methodname} applies to both large language models (LLMs) and text-to-image (T2I) models, we evaluate six widely-used LLMs and four prominent T2I models, including both commercial and open-source ones.
Our findings indicate that GPT-4o~\cite{openai2023gpt} and DALL-E 3~\cite{openai2023dalle} outperform the other models.
Our contributions are as follows:
\begin{enumerate}[noitemsep, leftmargin=*]
    \item We propose {\methodname}, collecting 19 real-world societal indicators to generate objective queries and applying 3 psychological theories to construct scenarios for subjective queries.
    \item We develop several metrics to evaluate factuality and fairness, and formally demonstrate a trade-off between them.
    \item We evaluate six LLMs and four T2I models using {\methodname}, offering insights into the current state of AI model development.
\end{enumerate}

\section{Related Works}
\label{Related Work}

\noindent\textbf{Non-Blind Video Inpainting.}
% \subsection{Video Inpainting}
% With the rapid development of deep learning, video inpainting has made great progress. 
The video inpainting methods can be roughly divided into three lines: 3D convolution-based~\cite{chang2019free,Kim_2019_CVPR,9558783}, flow-based~\cite{Gao-ECCV-FGVC,Kang2022ErrorCF,Ke2021OcclusionAwareVO,li2022towards,xu2019deep,Zhang_2022_CVPR,zou2020progressive}, and attention-based methods~\cite{cai2022devit,lee2019cpnet,Li2020ShortTermAL,liu2021fuseformer,Ren_2022_CVPR,9010390,srinivasan2021spatial,yan2020sttn,zhang2022flow}. 

%\noindent\textbf{3D convolution-based methods.}
The methods~\cite{chang2019free,Kim_2019_CVPR,9558783} based on 3D convolution usually reconstruct the corrupted contents by directly aggregating complementary information in a local temporal window through 3D temporal convolution. 
%For example,
%Wang et al.~\cite{wang2018video} proposed the first deep learning-based video inpainting network using a 3D encoder-decoder network.
%Further,
%Kim et al.~\cite{Kim_2019_CVPR} aggregated the temporal information of the neighbor frames into missing regions of the target frame by a recurrent 3D-2D feed-forward network.
Nevertheless, they often yield temporally inconsistent completed results due to the limited temporal receptive fields.
%\noindent\textbf{Flow-based methods.} 
The methods~\cite{Gao-ECCV-FGVC,Kang2022ErrorCF,Ke2021OcclusionAwareVO,li2022towards,xu2019deep,Zhang_2022_CVPR,zou2020progressive} based on optical flow treat the video inpainting as a pixel propagation problem. 
Generally, they first introduce a deep flow completion network to complete the optical flow, and then utilize the completed flow to guide the valid pixels into the corrupted regions. 
%For instance, 
%Xu et al.~\cite{xu2019deep} used the flow field completed by a coarse-to-fine deep flow completion network to capture the correspondence between the valid regions and the corrupted regions, and guide relevant pixels into the corrupted regions. 
%Based on this, 
%Gao et al.~\cite{Gao-ECCV-FGVC} further improved the performance of video inpainting by explicitly completing the flow edges. 
%Zou et al.~\cite{zou2020progressive} corrected the spatial misalignment in the temporal feature propagation stage by the completed optical flow. 
However, these methods fail to capture the visible contents of long-distance frames, thus reducing the inpainting performance in the scene of large objects and slowly moving objects. 


%\noindent\textbf{Attention-based methods.} 
Due to its outstanding long-range modeling capacity, attention-based methods, especially transformer-based methods, have shed light on the video inpainting community. 
These methods~\cite{cai2022devit,lee2019cpnet,Li2020ShortTermAL,liu2021fuseformer,Ren_2022_CVPR,9010390,srinivasan2021spatial,yan2020sttn,zhang2022flow} first find the most relevant pixels in the video frame with the corrupted regions by the attention module, and then aggregate them to complete the video frame. 
%For example,
%Zeng et al.~\cite{yan2020sttn} filled the missing regions of multi-frames simultaneously by learning a spatial-temporal transformer network. 
%Further, Liu et al.~\cite{liu2021fuseformer} improved edge details of missing contents by novel soft split and soft composition operations.
Although the existing video inpainting methods have shown promising results, 
%they usually assume that the corrupted regions of the video are known, 
they usually need to elaborately annotate the corrupted regions of each frame in the video,
limiting its application scope.
Unlike these approaches, we propose a blind video inpainting network in this paper, which can automatically identify and complete the corrupted regions in the video.


\begin{figure*}[tb]
\centering%height=3.0cm,width=15.5cm
\includegraphics[scale=0.66]{Fig/Fig_KT.pdf}
\vspace{-0.15cm}
\caption{\textbf{The overview of the proposed blind video inpainting framework}. Our framework are composed of a mask prediction network (MPNet) and a video completion network (VCNet). The former aims to predict the masks of corrupted regions by detecting semantic-discontinuous regions of the frame and utilizing temporal consistency prior of the video, while the latter perceive valid context information from uncorrupted regions using predicted mask to generate corrupted contents.
}
\label{Fig_KT}
\vspace{-0.5cm}
\end{figure*}

\noindent\textbf{Blind Image Inpainting.}
% \subsection{Blind Image Inpainting}
% \subsection{Image Inpainting}
In contrast to video inpainting, image inpainting solely requires consideration of the spatial consistency of the inpainted results. In the few years, the success of deep learning has brought new opportunities to many vision tasks, which promoted the development of a large number of deep learning-based image inpainting methods~\cite{shamsolmoali2023transinpaint,dong2022incremental,liu2022reduce,li2022misf,cao2022learning}. 
As a sub-task of image inpainting, blind image inpainting~\cite{wang2020vcnet,zhao2022transcnn,li2024semid,li2023decontamination,10147235} has been preliminarily explored. 
For example,
Nian et al.~\cite{cai2017blind} proposed a novel blind inpainting method based on a fully convolutional neural network. 
Liu et al.~\cite{BII} designed a deep CNN to directly restore a clear image from a corrupted input. However, these blind inpainting work assumes that the corrupted regions are filled with constant values or Gaussian noise, which may be problematic when corrupted regions contain unknown content. To improve the applicability, Wang et al.~\cite{wang2020vcnet} relaxed this assumption and proposed a two-stage visual consistency network. 
%Jenny et al.~\cite{schmalfuss2022blind} improved inpainting quality by integrating theoretically founded concepts from transform domain methods and sparse approximations into a CNN-based approach. 
Compared with blind image inpainting, blind video inpainting presents an additional challenge in preserving temporal consistency. Naively applying blind image inpainting algorithms on individual video frame to fill corrupted regions will lose inter-frame motion continuity, resulting in flicker artifacts. Inspired by the success of deep learning in blind image inpainting task, we propose the first deep blind video inpainting model in this paper, which provides a strong benchmark for subsequent research.


\section{Higher-order Guided Diffusion Model}
We now present our \textit{Higher-order Guided Diffusion} (HOG-Diff) model, which enhances graph generation by exploiting higher-order structures. 
We begin by detailing a coarse-to-fine generation curriculum that incrementally constructs graphs, followed by the introduction of three essential supporting techniques: the diffusion bridge, spectral diffusion, and a denoising model, respectively. Finally, we provide theoretical evidence validating the efficacy of HOG-Diff.


%\label{sec:gen-curriculum}
\paragraph{Coarse-to-fine Generation}
We draw inspiration from curriculum learning, a paradigm that mimics the human learning process by systematically organizing data in a progression from simple to complex \cite{curriculum-IJCV2022}.
Likely, an ideal graph generation curriculum should be a composition of multiple easy-to-learn and meaningful intermediate steps.
Additionally, higher-order structures encapsulate rich structural properties beyond pairwise interactions that are crucial for various empirical systems \cite{HiGCN2024}. 
As a graph-friendly generation framework, HOG-Diff incorporates higher-order structures during the intermediate stages of forward diffusion and reverse generative processes, thereby realizing a coarse-to-fine generation curriculum.

To implement our coarse-to-fine generation curriculum, we introduce a key operation termed cell complex filtering (CCF).
As illustrated in \cref{fig:cell-transform}, CCF generates an intermediate state of a graph by pruning nodes and edges that do not belong to a given cell complex.

\begin{definition}[Cell complex filtering]
Given a graph $\bm{G} = (\bm{V},\bm{E})$ and its associated cell complex $\mathcal{S}$, the cell complex filtering operation produces a filtered graph $\bm{G}^\prime = (\bm{V}^\prime,\bm{E}^\prime)$ where 
$\bm{V}^{\prime} = \{ v \in \bm{V}  \mid \exists\;x_\alpha \in \mathcal{S} : v \in x_\alpha \}$, 
and $\bm{E}^{\prime} = \{ (u, v) \in \bm{E} \mid \exists\;x_\alpha \in \mathcal{S} : u,v \in x_\alpha\}$.
\end{definition}

This filtering operation is a pivotal step in decomposing the graph generation task into manageable sub-tasks, with the filtered results serving as natural intermediaries in hierarchical graph generation. 
The overall framework of our proposed framework is depicted in \cref{fig:framework}.
Specifically, the forward and reverse diffusion processes in HOG-Diff are divided into $K$ hierarchical time windows, denoted as  $\{[\tau_{k-1},\tau_k]\}_{k=1}^K$, where $0 = \tau_0 < \cdots < \tau_{k-1}< \tau_k < \cdots < \tau_K = T$. 
Our sequential generation progressively denoises the higher-order skeletons.  
First, we generate coarse-grained higher-order skeletons, and subsequently refine them into finer pairwise relationships, simplifying the task of capturing complex graph distributions. 
%
This coarse-to-fine approach inherently aligns with the hierarchical nature of many real-world graphs, enabling smoother training and improved sampling performance.



Formally, our generation process factorizes the joint distribution of the final graph $\bm{G}_0$ into a product of conditional distributions across these time windows:
\begin{equation}
p(\bm{G}_0)=p(\bm{G}_0|\bm{G}_{\tau_1})p(\bm{G}_{\tau_1}|\bm{G}_{\tau_2}) \cdots p(\bm{G}_{\tau_{K-1}}|\bm{G}_{T}).
\end{equation}
Here, intermediate states $\bm{G}_{\tau_1}, \bm{G}_{\tau_2}, \cdots, \bm{G}_{\tau_{K-1}}$ represents different levels of cell filtering results of the corresponding graph. 
Our approach aligns intermediate stages of the diffusion process with realistic graph representations, and the ordering reflects a coarse-to-fine generation process.




The forward process introduces noise in a stepwise manner while preserving intermediate structural information. During each time window $[\tau_{k-1}, \tau_k]$, the evolution of the graph is governed by the following forward SDE:
\begin{equation}
\label{eq:HoGD-forward}
\mathrm{d}\bm{G}_t^{(k)}=\mathbf{f}_{k,t}(\bm{G}_t^{(k)})\mathrm{d}t+g_{k,t}\mathrm{d}\bm{W}_t, t \in [\tau_{k-1}, \tau_k].
\end{equation}

Reversing this process enables the model to generate authentic samples with desirable higher-order information.
The reverse-time SDE corresponds to \cref{eq:HoGD-forward} is as follows:
\begin{equation}
%\fontsize{9pt}{9pt}\selectfont
\begin{split}
    \mathrm{d}\bm{G}_t^{(k)}
     = & \left[\mathbf{f}_{k,t}(\bm{G}_t^{(k)})-g_{k,t}^2\nabla_{\bm{G}_t^{(k)}}\log p_t(\bm{G}_t^{(k)})\right]\mathrm{d}\bar{t} \\
    & +g_{k,t}\mathrm{d}\bar{\bm{W}}_t.
\end{split}
%\fontsize{10pt}{10pt}\selectfont
\end{equation}
Instead of using higher-order information as a direct condition, HOG-Diff employs it to guide the generation process through multiple steps. This strategy allows the model to incrementally build complex graph structures while maintaining meaningful structural integrity at each stage.
Moreover, integrating higher-order structures into graph generative models improves interpretability by allowing analysis of their significance in shaping the graph’s properties.







%\subsection{Diffusion Bridge}
%\label{sec:bridge}
\paragraph{Diffusion Bridge Process}
% -------- OU process
We realize the guided diffusion based on the generalized Ornstein-Uhlenbeck (GOU) process \cite{GOU1988, IRSDE+ICML2023}, a stationary Gaussian-Markov process characterized by its mean-reverting property. 
Over time, the marginal distribution of the GOU process stabilizes around a fixed mean and variance, making it well-suited for stochastic modelling with terminal constraints. 
The GOU process $\mathbb{Q} $ is governed by the following SDE:
\begin{equation}
\mathbb{Q}: \mathrm{d} \bm{G}_t = \theta_t(\bm{\mu} -\bm{G}_t)\mathrm{d}t + g_t(\bm{G}_t)\mathrm{d}\bm{W}_t,
\label{eq:GOU-SDE}
\end{equation}
where $\bm{\mu}=\bm{G}_{\tau_k}$ is the target terminal state, $\theta_t$ denotes a scalar drift coefficient and $g_t$ represents the diffusion coefficient. 
To ensure the process remains analytically tractable, $\theta_t$ and $g_t$ are constrained by the relationship $g_t^2/\theta_t = 2\sigma^2$ \cite{IRSDE+ICML2023}, where $\sigma^2$ is a given constant scalar.
%
Under these conditions, its transition probability admits a closed-form solution:
\begin{equation}
\begin{split}
p(&\bm{G}_{t}\mid \bm{G}_s) 
=\mathcal{N}(\mathbf{m}_{s:t},v_{s:t}^{2}\bm{I})  \\
&=  \mathcal{N}\left(
\bm{\mu}+\left(\bm{G}_s-\bm{\mu}\right)e^{-\bar{\theta}_{s:t}},
\sigma^2 (1-e^{-2\bar{\theta}_{s:t}})\bm{I}
\right).
\end{split}
\label{eq:GOU-p}
\end{equation}
Here, $\bar{\theta}_{s:t}=\int_s^t\theta_zdz$, and for notional simplicity, $\bar{\theta}_{0:t}$ is replaced by $\bar{\theta}_t$ when $s=0$.
% 
At time $t$ progress, $p(\bm{G}_t)$ gradually approaches a Gaussian distribution characterized by mean $\bm{\mu}$ and variance $\sigma^2$, indicating that the GOU process exhibits the mean-reverting property.  




% ---------- OU Bridge
The Doob’s $h$-transform can modify an SDE such that it passes through a specified endpoint.
When applied to the GOU process, this eliminates variance in the terminal state, driving the diffusion process toward a Dirac distribution centered at $\bm{G}_{\tau_k}$ \cite{GOUB2021,GOUB+ICML2024}.


\begin{proposition}
\label{pro:OUB}
Let $\bm{G}_t$ evolve according to the generalized OU process in \cref{eq:GOU-SDE}, subject to the terminal conditional $\bm{\mu}=\bm{G}_{\tau_k}$. 
%Then, the evolution of the conditional marginal distribution $p(\bm{G}_t \mid \bm{G}_{\tau_k})$ satisfies the following SDE:
The conditional marginal distribution $p(\bm{G}_t\mid\bm{G}_{\tau_k})$ then evolves according to the following SDE:
\begin{equation}
%\fontsize{8.5pt}{8.5pt}\selectfont
\mathrm{d}\bm{G}_t = \theta_t \left( 1 + \frac{2}{e^{2\bar{\theta}_{t:\tau_k}}-1}  \right)(\bm{G}_{\tau_k} - \bm{G}_t)  \mathrm{d}t 
+ g_{k,t} \mathrm{d}\bm{W}_t.\nonumber
%\fontsize{10pt}{10pt}\selectfont
\end{equation}
The conditional transition probability $p(\bm{G}_t \mid \bm{G}_{\tau_{k-1}}, \bm{G}_{\tau_k})$ has analytical form as follows:
\begin{equation}
\begin{split}
&p(\bm{G}_t \mid  \bm{G}_{\tau_{k-1}}, \bm{G}_{\tau_k}) 
= \mathcal{N}(\bar{\mathbf{m}}_t, \bar{v}_t^2 \bm{I}),\\
&\bar{\mathbf{m}}_t = 
\bm{G}_{\tau_k} + (\bm{G}_{\tau_{k-1}}-\bm{G}_{\tau_k})e^{-\bar{\theta}_{\tau_{k-1}:t}} 
\frac{v_{t:\tau_k}^2}{v_{\tau_{k-1}:\tau_k}^2}, \\
&\bar{v}_t^2 = {v_{\tau_{k-1}:t}^2 v_{t:\tau_k}^2}/{v_{\tau_{k-1}:\tau_k}^2}.
\end{split}
\end{equation}
Here, $\bar{\theta}_{a:b}=\int_a^b \theta_s  \mathrm{d}s$, and $v_{a:b}=\sigma^2(1-e^{-2\bar{\theta}_{a:b}})$.
\end{proposition}


We can directly use the closed-form solution in \Cref{pro:OUB} for one-step forward sampling without performing multi-step forward iteration using the SDE.
%
The reverse-time dynamics of the conditioned process can be derived using the theory of SDEs and take the following form:
\begin{equation}
%\fontsize{9pt}{9pt}\selectfont
 \mathrm{d}\bm{G}_t = \left[\mathbf{f}_{k,t}(\bm{G}_t)-g_{k,t}^2 \nabla_{\bm{G}_t} \log p(\bm{G}_t|\bm{G}_{\tau_k})\right] \diff\bar{t} + g_{k,t} \diff\bar{\bm{W}}_t, \nonumber
% \fontsize{10pt}{10pt}\selectfont
\end{equation}
where $\mathbf{f}_{k,t}(\bm{G}_t) = \theta_t \left( 1 + \frac{2}{e^{2\bar{\theta}_{t:\tau_k}}-1}  \right)(\bm{G}_{\tau_k} - \bm{G}_t)$.


\paragraph{Spectral Diffusion}
Generating graph adjacency matrices presents several significant challenges.
% 1.non-uniqueness of adjacency matrix 
Firstly, the non-uniqueness of graph representations implies that a graph with $n$ vertices can be equivalently modelled by up to $n!$ distinct adjacency matrices. This ambiguity requires a generative model to assign probabilities uniformly across all equivalent adjacencies to accurately capture the graph’s inherent symmetry. 
%
% 2.Sparsity
Additionally, unlike densely distributed image data, graphs typically follow a Pareto distribution and exhibit sparsity \cite{Sparsity+NP2024}, so that adjacency score functions lie on a low-dimensional manifold. Consequently, noise injected into out-of-support regions of the full adjacency space severely degrades the signal-to-noise ratio, impairing the training of the score-matching process. 
Even for densely connected graphs, isotropic noise distorts global message-passing patterns by encouraging message-passing on sparsely connected regions.
%
% 3. diffusion faster / scalability
Moreover, the adjacency matrix scales quadratically with the number of nodes, making the direct generation of adjacency matrices computationally prohibitive for large-scale graphs.

To address these challenges, inspired by~\citet{GAN2-Spectre,GSDM+TPAMI2023}, we introduce noise in the eigenvalue domain of the graph Laplacian matrix $\bm{L}=\bm{D}-\bm{A}$, instead of the adjacency matrix $\bm{A}$, where $\bm{D}$ denotes the diagonal degree matrix.
%
As a symmetric positive semi-definite matrix, the graph Laplacian can be diagonalized as $\bm{L} = \bm{U} \bm{\Lambda} \bm{U}^\top$. Here, the orthogonal matrix $\bm{U} = [\bm{u}_1,\cdots,\bm{u}_n]$ comprises the eigenvectors, and the diagonal matrix $\bm{\Lambda} = \operatorname{diag}(\lambda_1,\cdots,\lambda_n)$ holds the corresponding eigenvalues.
The relationship between the Laplacian spectrum and the graph's topology has been extensively explored~\cite{chung1997spectral}. For instance,  the low-frequency components of the spectrum capture the global structural properties such as connectivity and clustering, whereas the high-frequency components are crucial for reconstructing local connectivity patterns.
%
%
%% ------- 
Therefore, the target graph distribution $p(\bm{G}_0)$ represents a joint distribution of $\bm{X}_0$ and $\bm{\Lambda}_0$, exploiting the permutation invariance and structural robustness of the Laplacian spectrum.
%
Consequently, we split the reverse-time SDE into two parts that share drift and diffusion coefficients as
\begin{equation}
\fontsize{8pt}{8pt}\selectfont
\left\{
\begin{aligned}
\mathrm{d}\bm{X}_t=
&\left[\mathbf{f}_{k,t}(\bm{X}_t)
-g_{k,t}^2 
\nabla_{\bm{X}} \log p_t(\bm{G}_t | \bm{G}_{\tau_k}) \right]\mathrm{d}\bar{t}
+g_{k,t}\mathrm{d}\bar{\bm{W}}_{t}^1
\\
\mathrm{d}\bm{\Lambda}_t=
&\left[\mathbf{f}_{k,t}(\bm{\Lambda}_t)
- g_{k,t}^2 \nabla_{\bm{\Lambda}} \log p_t(\bm{G}_t | \bm{G}_{\tau_k})\right]\mathrm{d}\bar{t}
+g_{k,t}\mathrm{d}\bar{\bm{W}}_{t}^2
\end{aligned}
\right..\nonumber
\fontsize{10pt}{10pt}\selectfont
\label{eq:reverse-HoGD}
\end{equation}
Here, the superscript of $\bm{X}^{(k)}_t$ and $\bm{\Lambda}^{(k)}_t$ are dropped for simplicity, and $\mathbf{f}_{k,t}$ is determined according to \Cref{pro:OUB}.



To approximate the score functions $\nabla_{\bm{X}_t} \log p_t(\bm{G}_t | \bm{G}_{\tau_k})$ and $\nabla_{\bm{\Lambda}_t} \log p_t(\bm{G}_t | \bm{G}_{\tau_k})$, we employ a neural network $\bm{s}^{(k)}_{\bm{\theta}}(\bm{G}_t, \bm{G}_{\tau_k},t)$, composed of a node ($\bm{s}^{(k)}_{\bm{\theta},\bm{X}}(\bm{G}_t, \bm{G}_{\tau_k},t)$) and a spectrum ($\bm{s}^{(k)}_{\bm{\theta},\bm{\Lambda}}(\bm{G}_t, \bm{G}_{\tau_k},t)$) output, respectively.
The model is optimized by minimizing the loss function:
\fontsize{8pt}{8pt}\selectfont
\begin{align}
\ell^{(k)}(\bm{\theta})=&
\mathbb{E}_{t,\bm{G}_t,\bm{G}_{\tau_{k-1}},\bm{G}_{\tau_k}} \{\omega(t) [c_1\|
\bm{s}^{(k)}_{\bm{\theta},\bm{X}} - \nabla_{\bm{X}} \log p_t(\bm{G}_t | \bm{G}_{\tau_k})\|_2^2 \nonumber \\  
+&c_2 ||\bm{s}^{(k)}_{\bm{\theta},\bm{\Lambda}} - \nabla_{\bm{\Lambda}} \log p_t(\bm{G}_t | \bm{G}_{\tau_k})||_2^2]\}, \label{eq:final-loss}  
\end{align}
\normalsize 
where $\omega(t)$ is a positive weighting function, and $c_1, c_2$ controls the relative importance of vertices and spectrum.
The training procedure is detailed in \cref{alg:train} in the Appendix.




We sample $(\bm{X}_{\tau_K},\bm{\Lambda}_{\tau_K})$ from the prior distribution and uniformly sample $\bm{U}_0$ from the observed eigenvector matrices.
The generation process involves multi-step diffusion to produce samples $(\hat{\bm{X}}_{\tau_{K-1}}, \hat{\bm{\Lambda}}_{\tau_{K-1}}), \cdots, (\hat{\bm{X}}_{\tau_1}, \hat{\bm{\Lambda}}_{\tau_1}), (\hat{\bm{X}}_0, \hat{\bm{\Lambda}}_0)$ in sequence by reversing the diffusion bridge, where the endpoint of one generation step serves as the starting point for the next. 
Finally, plausible samples with higher-order structures $\hat{\bm{G}}_0=(\hat{\bm{X}}_0 , \hat{\bm{L}}_0 =\bm{U}_0 \hat{\bm{\Lambda}}_0 \bm{U}_0^\top)$ can be reconstructed.
The complete sampling procedure is outlined in \cref{alg:sample} within the Appendix.


\begin{table*}[t!]
\vspace{-3mm}
\centering
\caption{Comparison of different methods based on molecular datasets. The best results for the first three metrics are highlighted in bold.}
\resizebox{\textwidth}{!}{ 
\begin{tabular}{lcccccc cccccc}
\toprule
\multirow{3}{*}{Methods} 
& \multicolumn{6}{c}{QM9} 
& \multicolumn{6}{c}{ZINC250k} \\
\cmidrule(lr){2-7} \cmidrule(lr){8-13}
& NSPDK$\downarrow$ 
& FCD$\downarrow$ 
& \makecell{Val. w/o \\ corr.$\uparrow$ } 
& Val.$\uparrow$ 
& Uni.$\uparrow$ 
& Nov.$\uparrow$ 
& NSPDK $\downarrow$ 
& FCD $\downarrow$ 
& \makecell{Val. w/o \\ corr.$\uparrow$ } 
& Val.$\uparrow$ 
& Uni.$\uparrow$ 
& Nov.$\uparrow$ \\
\midrule
GraphAF     
& 0.020 & 5.268  & 67.00  & 100.00 & 94.51 & 88.83 & 0.044 & 16.289 & 68.00 & 100.00 & 99.10 & 100.00 \\
GraphAF+FC  
& 0.021 & 5.625  & 74.43  & 100.00 & 86.59 & 89.57 & 0.044 & 16.023 & 68.47 & 100.00 & 98.64 & 99.99 \\
GraphDF     
& 0.063 & 10.816 & 82.67  & 100.00 & 97.62 & 98.10 & 0.176 & 34.202 & 89.03 & 100.00 & 99.16 & 99.99 \\
GraphDF+FC  
& 0.064 & 10.928 & 93.88  & 100.00 & 98.32 & 98.54 & 0.177 & 33.546 & 90.61 & 100.00 & 99.63 & 100.00 \\
\midrule
MoFlow      
& 0.017 & 4.467  & 91.36  & 100.00 & 98.65 & 94.72 & 0.046 & 20.931 & 63.11 & 100.00 & 99.99 & 99.99 \\
%GraphCNF    
%&  -    &   -    &   -    &   -    &   -   &   -   & 0.021 & 13.532 & 96.35 & 100.00 & 99.64 & 100.00 \\
EDP-GNN     
& 0.005 & 2.680  & 47.52  & 100.00 & 99.25 & 86.58 & 0.049 & 16.737 & 82.97 & 100.00 & 99.79 & 99.99 \\
GraphEBM    
& 0.003 & 6.143  & 8.22   & 100.00 & 99.25 & 85.48 & 0.212 & 35.471 & 5.29  & 99.96  & 98.79 & 99.99 \\
GDSS        
& 0.003 & 2.900  & 95.72  & 100.00 & 98.46 & 86.27 & 0.019 & 14.656 & 97.01 & 100.00 & 99.64 & 100.00 \\
DiGress     
& 0.0005 & 0.360 & \Fi{99.00}  & 100.00 & 96.66 & 33.40 & 0.082 & 23.060 & 91.02 & 100.00 & 81.23 & 100.00 \\
\Fi{HOG-Diff}        
& \Fi{0.0003} & \Fi{0.172} & 98.74  & 100.00 & 97.01 & 75.12 
& \Fi{0.001} & \Fi{1.533} & \Fi{98.56} & 100.00 & 99.96 & 99.43 \\
\bottomrule
\end{tabular}}
\label{tab:mol_rel}
\vspace{-3.3mm}
\end{table*}

\paragraph{Denoising Network Architecture}
We design a neural network $\bm{s}^{(k)}_{\bm{\theta}}(\bm{G}_t, \bm{G}_{\tau_k},t) $ to estimate score functions in \cref{eq:reverse-HoGD}.
Standard graph neural networks designed for classical tasks such as graph classification and link prediction may be inappropriate for graph distribution learning due to the immediate real-number graph states and the complicated requirements. 
For example, an effective model for molecular graph generation should capture local node-edge dependence for chemical valency rules and attempt to recover global graph patterns like edge sparsity, frequent ring subgraphs, and atom-type distribution.



% Our designed model
To achieve this, we introduce a unified denoising network that explicitly integrates node and spectral representations. 
As illustrated in \cref{fig:denoising-model} of Appendix, the network comprises two different graph processing modules: a standard graph convolution network (GCN) \cite{GCN+ICLR2017} for local feature aggregation and a graph transformer network (ATTN)~ \cite{TFmodel2021AAAIworkshop,DiGress+ICLR2023} for global information extraction.
The outputs of these modules are fused with time information through a Feature-wise Linear Modulation (FiLM) layer~\cite{Film+AAAI2018}. The resulting representations are concatenated to form a unified hidden embedding.
This hidden embedding is further processed through separate multilayer perceptrons (MLPs) to produce predictions for $\nabla_{\bm{X}} \log p(\bm{G}_t|\bm{G}_{\tau_k})$ and $\nabla_{\bm{\Lambda}} \log p(\bm{G}_t|\bm{G}_{\tau_k})$, respectively.
%
It is worth noting that our graph noise prediction model is permutation equivalent as each component of our model avoids any node ordering-dependent operations.
%
Our model is detailed in \cref{app:denoising-model}.






\paragraph{Theoretical Analysis}
%\subsection{Theoretical Analysis}
%\label{sec:theorms}
In the following, we provide supportive theoretical evidence for the efficacy of HOG-Diff, demonstrating that the proposed framework achieves faster convergence in score-matching and tighter reconstruction error bounds compared to standard graph diffusion works.

\begin{proposition}[Informal]
\label{pro:training}
Suppose the loss function $\ell^{(k)}(\bm{\theta})$ in \cref{eq:final-loss} is $\beta$-smooth and satisfies the $\mu$-PL condition in the ball $B\left(\boldsymbol{\theta}_0, R\right)$. 
Then, the expected loss at the $i$-th iteration of the training process satisfies:
\begin{equation}
%\fontsize{7pt}{7pt}\selectfont
\mathbb{E}\left[\ell^{(k)}(\bm{\theta}_i)\right] 
\leq \left(1-\frac{b\mu^2}{\beta N(\beta N^2+\mu(b-1))}\right)^i \ell^{(k)}\left(\bm{\theta}_0\right),\nonumber
%\fontsize{10pt}{10pt}\selectfont
\end{equation}
where $N$ denotes the size of the training dataset, and $b$ is the mini-batch size.
Furthermore, it holds that $\beta_{\text{HOG-Diff}}\leq \beta_{\text{classical}}$, implying that the distribution learned by the proposed framework converges to the target distribution faster than classical generative models.
\end{proposition}



Following \citet{GSDM+TPAMI2023},  we define the expected reconstruction error at each generation process as $\mathcal{E}(t)=\mathbb{E}\norm{\bar{\bm{G}}_t-\widehat{\bm{G}}_t}^2$, where $\bar{\bm{G}}_t$ represents the data reconstructed sing the ground truth score $\nabla \log p_t(\cdot)$ and $\widehat{\bm{G}}_t$ denotes the data reconstructed with the learned score function $\bm{s}_{\bm{\theta}}$.
We establish that the reconstruction error in HOG-Diff is bounded more tightly than in classical graph generation models, ensuring superior sample quality.
\begin{proposition}
\label{pro:reconstruction-error}
Under appropriate Lipschitz and boundedness assumptions, the reconstruction error of HOG-Diff satisfies the following bound: 
\begin{equation}
%\fontsize{8.5pt}{8.5pt}\selectfont
\mathcal{E}(0)
\leq 
\alpha(0)\exp{\int_0^{\tau_1} \gamma(s) } \diff{s},
%\fontsize{10pt}{10pt}\selectfont
\end{equation}
where $\alpha(0)=C^2 \ell^{(1)} (\bm{\theta}) \int_0^{\tau_1} g_{1,s}^4 \diff{s}
+ C \mathcal{E}(\tau_1) \int_0^{\tau_1} h_{1,s}^2 \diff{s}$, 
$\gamma(s) = C^2 g_{1,s}^4 \|\bm{s}_{\bm{\theta}}(\cdot,s)\|_{\mathrm{lip}}^2 
 + C \|h_{1,s}\|_{\mathrm{lip}}^2$,
and $h_{1,s} = \theta_s \left(1 + \frac{2}{e^{2\bar{\theta}_{s:\tau_1}}-1}\right)$.
%
Furthermore, we can derive that the reconstruction error bound of HOG-Diff is sharper than classical graph generation models.
\end{proposition}


The propositions above rely primarily on mild assumptions, such as smoothness and boundedness, without imposing strict conditions like the target distribution being log-concave or satisfying the log-Sobolev inequality.
%
Their formal statements and detailed proofs are postponed to \cref{app:proof}.
We experimentally verify these Propositions in \Cref{sec:ablations}.



\section{Fine-Tuning Experiments}
This section validates that our dataset can enhance the GUI grounding capabilities of VLMs and that the proposed functionality grounding and referring are effective fine-tuning tasks.
\subsection{Experimental Settings}
\noindent\textbf{Evaluation Benchmarks} We base our evaluation on the UI grounding benchmarks for various scenarios: \textbf{FuncPred} is the test split from our collected functionality dataset. This benchmark requires a model to locate the element specified by its functionality description. \textbf{ScreenSpot}~\citep{cheng2024seeclick} is a benchmark comprising test samples on mobile, desktop, and web platforms. It requires the model to locate elements based on short instructions. \textbf{RefExp}~\citep{Bai2021UIBertLG} is to locate elements given crowd-sourced referring expressions. \textbf{VisualWebBench (VWB)}~\citep{liu2024visualwebbench} is a comprehensive multi-modal benchmark assessing the understanding capabilities of VLMs in web scenarios. We select the element and action grounding tasks from this benchmark. To better align with high-level semantic instructions for potential agent requirements and avoid redundancy evaluation with ScreenSpot, we use ChatGPT to expand the OCR text descriptions in the original task instructions, such as \textit{Abu Garcia College Fishing} into functionality descriptions like \textit{This element is used to register for the Abu Garcia College Fishing event}.
\textbf{MOTIF}~\citep{Burns2022ADF} requires an agent to complete a natural language command in mobile Apps.
For all of these benchmarks, we report the grounding accuracy (\%): $\text { Acc }= \sum_{i=1}^N \mathbf{1}\left(\text {pred}_i \text { inside GT } \text {bbox}_i\right) / N \times 100 $ where $\mathbf{1}$ is an indicator function and $N$ is the number of test samples. This formula denotes the percentage of samples with the predicted points lying within the bounding boxes of the target elements.

\noindent\textbf{Training Details}
We select Qwen-VL-10B~\citep{bai2023qwen} and SliME-8B~\citep{slime} as the base models and fine-tune them on 25k, 125k, and 702k samples of the AutoGUI training data to investigate how the AutoGUI data enhances the UI grounding capabilities of the VLMs. The models are fine-tuned on 8 A100 GPUs for one epoch. We follow SeeClick~\citep{cheng2024seeclick} to fine-tune Qwen-VL with LoRA~\citep{hu2022lora} and follow the recipe of SliME~\citep{slime} to fine-tune it with only the visual encoder frozen (More details in Sec.~\ref{sec:supp:impl details}).

\noindent\textbf{Compared VLMs}
We compare with both general-purpose VLMs (i.e., LLaVA series~\citep{liu2023llava,liu2024llavanext}, SliME~\citep{slime}, and Qwen-VL~\citep{bai2023qwen}) and UI-oriented ones (i.e., Qwen2-VL~\citep{qwen2vl}, SeeClick~\citep{cheng2024seeclick}, CogAgent~\citep{hong2023cogagent}). SeeClick finetunes Qwen-VL with around 1 million data combining various data sources, including a large proportion of human-annotated UI grounding/referring samples. CogAgent is trained with a huge amount of text recognition, visual grounding, UI understanding, and publicly available text-image datasets, such as LAION-2B~\citep{LAION5B}. During the evaluation, we manually craft grounding prompts suitable for these VLMs.
\subsection{Experimental Results and Analysis}
\begin{table}[]
\scriptsize
\centering
\caption{\textbf{Element grounding accuracy on the used benchmarks.} We compare the base models fine-tuned with our AutoGUI data and representative open-source VLMs. The results show that the two base models (i.e. Qwen-VL and SliME-8B) obtain significant performance gains over the benchmarks after being fine-tuned with AutoGUI data. Moreover, increasing the AutoGUI data size consistently improves grounding accuracy, demonstrating notable scaling effects. $\dag$ means the metric value is borrowed from the benchmark paper. $*$ means using additional SeeClick training data.}
\label{tab:eval results}
\begin{tabular}{@{}cccccccccc@{}}
\toprule
Type & Model    & Size    & FuncPred & VWB EG & VWB AG & MoTIF & RefExp & ScreenSpot  \\ \midrule
\multirow{5}{*}{General} & LLaVA-1.5~\citep{liu2023llava} & 7B & 3.2      &        12.1$^{\dag}$        &     13.6$^{\dag}$           &  7.2   &  4.2 & 5.0 & \\
 & LLaVA-1.5~\citep{liu2023llava} & 13B & 5.8      &           16.7     &        9.7        &   12.3 &  20.3   & 11.2 &  \\
 & LLaVA-1.6~\citep{liu2024llavanext} & 34B &  4.4      &      19.9          &    17.0            &   7.0 &  29.1  & 10.3 &  \\
 & SliME~\citep{slime} & 8B &  3.2  &   6.1       &     4.9     & 7.0  &  8.3  &  13.0  \\ 

 & Qwen-VL~\citep{bai2023qwen} & 10B &  3.0     &      1.7          &      3.9          &    7.8 &  8.0  & 5.2$^{\dag}$   \\ 
 \midrule
\multirow{3}{*}{UI-VLM} &  Qwen2-VL~\citep{bai2023qwen}  & 7B     &     7.8       &    3.9        &  3.9  &  16.7 & 32.4 & 26.1    \\
 & CogAgent~\citep{hong2023cogagent} & 18B    &  29.3   &    \underline{55.7}      &    \textbf{59.2}      & \textbf{24.7}   & 35.0 &  47.4$^{\dag}$  \\
 & SeeClick~\citep{cheng2024seeclick} & 10B    &    19.8     &    39.2           &     27.2           & 11.1  &  \textbf{58.1}  & \underline{53.4}$^{\dag}$ \\ 
\midrule
\multirow{4}{*}{Finetuned} &  Qwen-VL-AutoGUI25k & 10B      &    14.2     &      12.8         &    12.6           &   10.8    &  12.0 & 19.0    \\
 & Qwen-VL-AutoGUI125k  & 10B       &     25.5     &      23.2         &        29.1       &    11.5   &  14.9 & 32.0     \\ 
 & Qwen-VL-AutoGUI702k  & 10B       &   43.1   &    38.0       &     32.0    &  15.5  & 23.9 &    38.4   \\
& Qwen-VL-AutoGUI702k$^*$   & 10B     &  \underline{50.0}  &    \textbf{56.2}    &  \underline{45.6}  & \underline{21.0} & \underline{51.5} & \textbf{54.2}      \\
\midrule
\multirow{3}{*}{Finetuned} & SliME-AutoGUI25k  & 8B     &   28.0   &     14.0      &      10.6      &  14.3   & 18.4 & 27.2   \\
 & SliME-AutoGUI125k   & 8B      &   39.9    &  22.0   &     12.0       &  17.8  & 22.1 &  35.0     \\
 & SliME-AutoGUI702k   & 8B      &     \textbf{62.6}   &       25.4        &     13.6          &   20.6    & 26.7 & 44.0 &          \\
\bottomrule
\end{tabular}
\end{table}
\vspace{-2mm}


\noindent\textbf{A) AutoGUI functionality annotations effectively enhance VLMs' UI grounding capabilities and achieve scaling effects.} We endeavor to show that the element functionality data autonomously collected by AutoGUI contributes to high grounding accuracy. The results in Tab.~\ref{tab:eval results} demonstrate that on all benchmarks the two base models achieve progressively rising grounding accuracy as the functionality data size scales from 25k to 702k, with SliME-8B's accuracy increasing from merely \textbf{3.2} and \textbf{13.0} to \textbf{62.6} and \textbf{44.0} on FuncPred and ScreenSpot, respectively. This increase is visualized in Fig.~\ref{fig:funcpred scaling success} showing that increasing AutoGUI data amount leads to more precise localization performance.

After fine-tuning with AutoGUI 702k data, the two base models surpass SeeClick, the strong UI-oriented VLM on FuncPred and MOTIF. We notice that the base models lag behind SeeClick and CogAgent on ScreenSpot and RefExp, as the two benchmarks contain test samples whose UIs cannot be easily recorded (e.g., Apple devices and Desktop software) as training data, causing a domain gap. Nevertheless, SliME-8B still exhibits noticeable performance improvements on ScreenSpot and RefExp when scaling up the AutoGUI data, suggesting that the AutoGUI data helps to enhance grounding accuracy on the out-of-domain tasks.

To further unleash the potential of the AutoGUI data, the base model, Qwen-VL, is finetuned with the combination of the AutoGUI and SeeClick UI-grounding data. This model becomes the new state-of-the-art on FuncPred, ScreenSpot, and VWB EG, surpassing SeeClick and CogAgent. This result suggests that our AutoGUI data can be mixed with existing UI grounding training data to foster better UI grounding capabilities.

In summary, our functionality data can endow a general VLM with stronger UI grounding ability and exhibit clear scaling effects as the data size increases.


\begin{table}[]
\centering
\footnotesize
\caption{\textbf{Comparing the AutoGUI functionality annotation type with existing types}. Qwen-VL is fine-tuned with the three annotation types. The results show that our functionality data leads to superior grounding accuracy compared with the naive element-HTML data and the condensed functionality annotations.}
\label{tab:ablation}
\begin{tabular}{@{}ccccc@{}}
\toprule
Data Size             & Variant          & FuncPred & RefExp & ScreenSpot \\ \midrule
\multirow{3}{*}{25k}  & w/ Elem-HTML data     &  5.3      &  4.5   &    5.7     \\
                      & w/ Condensed Func. Anno.     &  3.8   &  3.0  &   4.8      \\
                      & w/ Func. Anno. (Ours full)         &    \textbf{21.1}    &   \textbf{10.0}   &   \textbf{16.4}    \\ \midrule
\multirow{3}{*}{125k} & w/ Elem-HTML data     &  15.5   &  7.8  &   17.0      \\
                      & w/ Condensed Func. Anno.     &  14.1   &  11.7  &   23.8      \\
                      & w/ Func. Anno. (Ours full)         &  \textbf{24.6}   &  \textbf{12.7}  &   \textbf{27.0}    \\ \bottomrule
\end{tabular}
\end{table}



\noindent\textbf{B) Our functionality annotations are effective for enhancing UI grounding capabilities.} To assess the effectiveness of functionality annotations, we compare this annotation type with two existing types: 1) \textbf{Naive element-HTML pairs}, which are directly obtained from the UI source code~\citep{hong2023cogagent} and associate HTML code with elements in specified areas of a screenshot. Examples are shown in Fig.~\ref{fig: functionality vs others}. To create these pairs, we replace the functionality annotations with the corresponding HTML code snippets recorded during trajectory collection. 2) \textbf{Brief functionality descriptions} that are generated by prompting GPT-4o-mini\footnote{https://openai.com/index/gpt-4o-mini-advancing-cost-efficient-intelligence/} to condense the AutoGUI functionality annotations. For example, a full description such as \textit{`This element provides access to a documentation category, allowing users to explore relevant information and guides'} is shortened to \textit{`Documentation category access'}.

After experimenting with Qwen-VL~\citep{bai2023qwen} at the 25k and 125k scales, the results in Tab.~\ref{tab:ablation} show that fine-tuning with the complete functionality annotations is superior to the other two types. Notably, our functionality annotation type yields the largest gain on the challenging FuncPred benchmark that emphasizes contextual functionality grounding. In contrast, the Elem-HTML type performs poorly due to the noise inherent in HTML code (e.g., numerous redundant tags), which reduces fine-tuning efficiency. The condensed functionality annotations are inferior, as the consensing loses details necessary for fine-grained UI understanding. In summary, the AutoGUI functionality annotations provide a clear advantage in enhancing UI grounding capabilities.


\subsection{Failure Case Analysis}
After analyzing the grounding failure cases, we identified several failure patterns in the fine-tuned models: a) difficulty in accurately locating small elements; b) challenges in distinguishing between similar but incorrect elements; and c) issues with recognizing icons that have uncommon shapes. Please refer to Sec.~\ref{sec:supp:case analysis} for details.




\section{Conclusion}

In this paper, we introduce \DatasetName, a novel large-scale dataset specifically designed for long-text rendering, addressing the existing gap in datasets capable of supporting such tasks. 
To demonstrate the utility of models in handling long-text generation, we create a dedicated test set and evaluate current state-of-the-art text-to-image generation models.
Additionally, the open availability of a large-scale, diverse, and high-quality long-text rendering dataset like \DatasetName is crucial for advancing the training of text-conditioned image generation models.

There are several promising directions for further enhancing \DatasetName, which we have not explored in this paper due to the increased computational costs these approaches entail: \emph{i}. Multiple rounds of dataset bootstrapping to iteratively improve data quality. \emph{ii}. Generating multiple synthetic captions per image to further expand the dataset corpus.





\section*{Impact Statement}
This paper presents work whose goal is to advance the field of 
Machine Learning. There are many potential societal consequences 
of our work, none which we feel must be specifically highlighted here.

\nocite{langley00}

\bibliography{main}
\bibliographystyle{icml2025}


%%%%%%%%%%%%%%%%%%%%%%%%%%%%%%%%%%%%%%%%%%%%%%%%%%%%%%%%%%%%%%%%%%%%%%%%%%%%%%%
%%%%%%%%%%%%%%%%%%%%%%%%%%%%%%%%%%%%%%%%%%%%%%%%%%%%%%%%%%%%%%%%%%%%%%%%%%%%%%%
% APPENDIX
%%%%%%%%%%%%%%%%%%%%%%%%%%%%%%%%%%%%%%%%%%%%%%%%%%%%%%%%%%%%%%%%%%%%%%%%%%%%%%%
%%%%%%%%%%%%%%%%%%%%%%%%%%%%%%%%%%%%%%%%%%%%%%%%%%%%%%%%%%%%%%%%%%%%%%%%%%%%%%%
\newpage
\appendix
%\onecolumn
\section*{Appendix}




\end{document}


% This document was modified from the file originally made available by
% Pat Langley and Andrea Danyluk for ICML-2K. This version was created
% by Iain Murray in 2018, and modified by Alexandre Bouchard in
% 2019 and 2021 and by Csaba Szepesvari, Gang Niu and Sivan Sabato in 2022.
% Modified again in 2023 and 2024 by Sivan Sabato and Jonathan Scarlett.
% Previous contributors include Dan Roy, Lise Getoor and Tobias
% Scheffer, which was slightly modified from the 2010 version by
% Thorsten Joachims & Johannes Fuernkranz, slightly modified from the
% 2009 version by Kiri Wagstaff and Sam Roweis's 2008 version, which is
% slightly modified from Prasad Tadepalli's 2007 version which is a
% lightly changed version of the previous year's version by Andrew
% Moore, which was in turn edited from those of Kristian Kersting and
% Codrina Lauth. Alex Smola contributed to the algorithmic style files.
