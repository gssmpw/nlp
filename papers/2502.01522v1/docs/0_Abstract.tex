\begin{abstract}
Generative diffusion models trained on large-scale datasets have achieved remarkable progress in image synthesis. In favor of their ability to supplement missing details and generate aesthetically pleasing contents, recent works have applied them to image deblurring tasks via training an adapter on blurry-sharp image pairs to provide structural conditions for restoration. However, acquiring substantial amounts of realistic paired data is challenging and costly in real-world scenarios. On the other hand, relying solely on synthetic data often results in overfitting, leading to unsatisfactory performance when confronted with unseen blur patterns. To tackle this issue, we propose BD-Diff, a generative-diffusion-based model designed to enhance deblurring performance on unknown domains by decoupling structural features and blur patterns through joint training on three specially designed tasks. We employ two Q-Formers as structural representations and blur patterns extractors separately. The features extracted by them will be used for the supervised deblurring task on synthetic data and the unsupervised blur-transfer task by leveraging unpaired blurred images from the target domain simultaneously.
Furthermore, we introduce a reconstruction task to make the structural features and blur patterns complementary. This blur-decoupled learning process enhances the generalization capabilities of BD-Diff when encountering unknown domain blur patterns. Experiments on real-world datasets demonstrate that BD-Diff outperforms existing state-of-the-art methods in blur removal and structural preservation in various challenging scenarios. The codes will be released in: \textit{\url{https://github.com/donahowe/BD-Diff}}.
\vspace{-4mm}
\end{abstract}