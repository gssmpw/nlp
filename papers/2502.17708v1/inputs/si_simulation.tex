\section{Simulation Results}
\label{sec:simulation}

\subsection{MCMC Plots of Key Parameters}
% tau mcmc
\begin{figure}[!ht]
     \centering
         \includegraphics[width=\textwidth]{figs/tau_mcmc.pdf}
         \caption{MCMC convergence of $\pmb\tau$ posterior samples in simulation. Horizontal red line indicates the true values of $\pmb\tau$.}
	 \label{sim_tau_mcmc}
\end{figure}

% theta mcmc
\begin{figure}[!ht]
     \centering
         \includegraphics[width=\textwidth]{figs/theta_mcmc_doc1.pdf}
         \caption{MCMC convergence of $\pmb\theta$ parameters for the first document. $\pmb\theta$ values are obtained by transforming the posterior samples of $\pmb\eta$ of the corresponding document. Horizontal red line indicates the true values of $\pmb\theta$ for the first document for each topic. We do not display the MCMC convergence for other documents, but all documents show similar level of convergence to the true value of $\pmb\theta$.}
	 \label{sim_theta_mcmc}
\end{figure}


\subsection{Recovery of the True Latent Variables}
% BK version
We generate 100 simulation datasets with similar sizes as our application datasets. Specifically, we set the simulation datasets to have about equal number of documents, paragraphs, unique words and words.\footnote{106 documents, an average of 44 paragraphs per document, 5838 unique words, and an average of 51 words per paragraph.} Citations are generated based on the hyperparameters we input, and we set them so that the number of citations will be similar to those in our application data. This exercise gives us some evidence on the validity of our results on the application datasets.

We show that the PCTM can recover the true parameters from random initialization using our Gibbs sampler. 
We fit the PCTM on one of the simulation datasets while the initial parameters of the paragraph topic, $\mathbf{Z}$, and the distribution of topics, $\pmb\eta$, are randomly initialized.  
Then, we compare the estimated paragraph topics and the distribution of topics with the true values of those parameters.

Figure \ref{simulation_Z} plots the posterior samples of paragraph topics against the true paragraph topics. Numbers on the x-axis and y-axis denote topic labels. The darkness of cell colors is proportional to the number of paragraphs in those cells. The cell in the second row and the third column, for example, denotes the number of paragraphs that are assigned topic 2 in posterior samples when the true topic is 3. Darker colors on the diagonal lines suggest that the model recovers true topics correctly, which we see on the right panel of Figure \ref{simulation_Z}. In comparison, the left panel of Figure \ref{simulation_Z} illustrates that the Gibbs sampler was initiated with randomly generated values of paragraph topics. 

We conduct a similar exercise with the document-level topic mixture $\pmb\eta$. To make the comparison more rooted in conventional topic models, we convert $\pmb\eta$ to $\pmb\theta$ using softmax in this exercise. In Figure \ref{simulation_theta_modes}, we plot the mode of posterior samples of $\pmb\theta$ against the mode of the true topic mixture. The darker colors indicate a higher number of documents in the corresponding cell. Similar to Figure \ref{simulation_Z}, we observe evenly spread colors on the left panel as opposed to the concentrated dark colors on the diagonal entries on the right panel. This shows that the PCTM recovers 

These two results verify that the PCTM can recover true topics from random initialization when applied to simulation data.
This adds to the credibility of the topic estimations in our application since our simulation data resembles our application data. 

\begin{figure}[ht!]
     \centering
     \begin{subfigure}[b]{0.4\textwidth}
         \centering
         \includegraphics[width=\textwidth]{figs/simulation_Z_crosstab_init_true.pdf}
         \label{Z_mode_init}
     \end{subfigure}
     \hfill
     \begin{subfigure}[b]{0.4\textwidth}
         \centering
         \includegraphics[width=\textwidth]{figs/simulation_Z_crosstab_est_true.pdf}
	 \label{Z_mode_est}
     \end{subfigure}
        \caption{The comparison of the estimated and the true topics of paragraphs.
        On the right panel, the ($k, l$) cell shows the number of paragraphs whose estimated topic is $l$ while the true topic is $k$.
        We estimate topics using the paragraph topic parameter, \textbf{Z}, using the last draw from our Gibbs sampler.
        The cells with darker colors indicate a higher number of paragraphs.
        The concentration on the diagonal elements means that the topics are estimated correctly.
        As a comparison, the left panel plots randomly initialized paragraph topics against true paragraph topics.  
        They show that the PCTM can recover the true topics even when the topics are randomly provided at the initialization of our Gibbs sampler.}
        \label{simulation_Z}
\end{figure}

\begin{figure}[ht!]
     \centering
     \begin{subfigure}[b]{0.4\textwidth}
         \centering
         \includegraphics[width=\textwidth]{figs/simulation_theta_crosstab_init_true.pdf}
         \label{theta_mode_init}
     \end{subfigure}
     \hfill
     \begin{subfigure}[b]{0.4\textwidth}
         \centering
         \includegraphics[width=\textwidth]{figs/simulation_theta_crosstab_est_true.pdf}
	 \label{theta_mode_est}
     \end{subfigure}
        \caption{The comparison of the estimated and the true topic distribution of documents.
        On the right panel, the ($k, l$) cell shows the number of documents whose mode of the estimated topic distribution, $\btheta$, across $K$ topics is $l$
        while the mode of the true topic distribution is $k$.
        We obtain $\btheta$ by applying the softmax transformation on each draw of $\pmb\eta$ in our Gibbs sampler, and then obtain the estimated $\btheta$ by their posterior mean. 
        The cells with darker colors mean a higher number of documents are in the cell.
        The concentration on the diagonal elements means that the modes of the topic distributions are estimated correctly.
        As a comparison, the left panel plots the mode of randomly initialized $\btheta$ against true mode of $\btheta$.  
        It shows that the PCTM can recover the true mode of the topic distribution even when the topics are randomly provided at the initialization of our Gibbs sampler.}
        \label{simulation_theta_modes}
\end{figure}

