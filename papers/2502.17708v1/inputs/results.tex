\section{Empirical Results}
\label{sec:result}

This section presents the results of applying the PCTM to the SCOTUS dataset, focusing on the privacy issue area.\footnote{We also present additional results with a dataset on voting rights issue area in Supplementary Information, Section~\ref{sec:voting}.}
We present three main results.
First, we fit the PCTM and existing alternatives, LDA and RTM, to the SCOTUS opinions on the privacy issue area and discuss the advantages of the PCTM over the existing models.\footnote{The convergence diagnostics of the PCTM and discussions about the parameters not discussed in this section are provided in Supplementary Information, Section~\ref{sec:si_privacy}}
We find that the main advantage of the PCTM is its ability to use paragraph-level topics to extract informative topic-specific subset of the citation network.
Second, we utilize these topic-specific subnetworks to measure the importance of cases within each topic, following the methodological framework of \cite{fowler2007network}.
Our analysis reveals that case importance varies substantially across topic domains.
Third, we conduct the predictive analysis of the topic structure of \textit{Dobbs v. Jackson Women's Health Organization}, the recent case that overruled \textit{Roe v. Wade}, based on words and citations in its paragraphs.
We find that the predicted topics of \textit{Dobbs v. Jackson Women's Health Organization} address abortion in markedly different ways from post-\textit{Roe v. Wade} cases, but in ways reminiscent of pre-\textit{Roe v. Wade} cases. 
Together, these results demonstrate the advantage of the PCTM in uncovering valuable insights from the text and citation data of the SCOTUS opinions. 

\subsection{Topic Composition of SCOTUS Opinions on Privacy}
Table \ref{word_mat_privacy} displays the top 10 most frequent words for each topic estimated in the PCTM. 
The Supreme Court Database assigns 4 issue codes to opinions of the privacy issue area, but we identify 7 distinct topics in the PCTM.\footnote{The four issue codes identified by the Supreme Court Database are privacy, abortion, right to die and Freedom of Information Act. 
To determine the optimal number of topics for our analysis, we implemented an iterative approach, beginning with a 4-topics specification and systematically increasing the number of topics up to 15. We ultimately selected a 7-topics model as it provided the most coherent representation of legally salient themes within the privacy issue area, based on our substantive knowledge of constitutional law and privacy jurisprudence.}  
The labels in the table are provided by the authors.
% word matrix
\begin{table}[t!]
\centering
\begin{tabular}{c| l l l l l l l}
  \hline
Topic & \textcolor{SkyBlue}{Regulation} & \textcolor{Blue}{Procedural} & \textcolor{Aquamarine}{Const.} & \textcolor{Green}{Speech} & \textcolor{ForestGreen}{Damage} & \textcolor{Brown}{Privacy} & \textcolor{Red}{Public}\\ 
Label & \textcolor{SkyBlue}{of} & \textcolor{Blue}{Posture} & \textcolor{Aquamarine}{Rights} & \textcolor{Green}{\&} & \textcolor{ForestGreen}{to} & \textcolor{Brown}{vs} & \textcolor{Red}{Disclosure}\\ 
 & \textcolor{SkyBlue}{Abortion} & \textcolor{Blue}{} & \textcolor{Aquamarine}{to} & \textcolor{Green}{Protest} & \textcolor{ForestGreen}{Privacy} & \textcolor{Brown}{Govnt.} & \textcolor{Red}{of Private}\\ 
 & \textcolor{SkyBlue}{Procedure} & \textcolor{Blue}{} & \textcolor{Aquamarine}{Abortion} & \textcolor{Green}{} & \textcolor{ForestGreen}{} & \textcolor{Brown}{Interest} & \textcolor{Red}{Information}\\ 
  \hline
1 & \textcolor{SkyBlue}{abort} & \textcolor{Blue}{appeal} & \textcolor{Aquamarine}{right} & \textcolor{Green}{clinic} & \textcolor{ForestGreen}{damag} & \textcolor{Brown}{drug} & \textcolor{Red}{inform}\\ 
2   & \textcolor{SkyBlue}{parent} & \textcolor{Blue}{district} & \textcolor{Aquamarine}{abort} & \textcolor{Green}{injunct} & \textcolor{ForestGreen}{act} & \textcolor{Brown}{act} & \textcolor{Red}{agenc}\\ 
3   & \textcolor{SkyBlue}{minor} & \textcolor{Blue}{board} & \textcolor{Aquamarine}{constitu} & \textcolor{Green}{right} & \textcolor{ForestGreen}{actual} & \textcolor{Brown}{test} & \textcolor{Red}{exmpt}\\ 
4   & \textcolor{SkyBlue}{physician} & \textcolor{Blue}{ani} & \textcolor{Aquamarine}{protect} & \textcolor{Green}{public} & \textcolor{ForestGreen}{congress} & \textcolor{Brown}{student} & \textcolor{Red}{disclosur}\\ 
5   & \textcolor{SkyBlue}{perform} & \textcolor{Blue}{order} & \textcolor{Aquamarine}{medic} & \textcolor{Green}{speech} & \textcolor{ForestGreen}{person} & \textcolor{Brown}{school} & \textcolor{Red}{record}\\ 
6   & \textcolor{SkyBlue}{woman} & \textcolor{Blue}{agency} & \textcolor{Aquamarine}{amend} & \textcolor{Green}{petition} & \textcolor{ForestGreen}{privaci} & \textcolor{Brown}{respond} & \textcolor{Red}{public}\\ 
7   & \textcolor{SkyBlue}{medic} & \textcolor{Blue}{document} & \textcolor{Aquamarine}{decis} & \textcolor{Green}{protest} & \textcolor{ForestGreen}{right} & \textcolor{Brown}{use} & \textcolor{Red}{govern}\\ 
8   & \textcolor{SkyBlue}{interest} & \textcolor{Blue}{rule} & \textcolor{Aquamarine}{person} & \textcolor{Green}{zone} & \textcolor{ForestGreen}{ani} & \textcolor{Brown}{ani}  & \textcolor{Red}{act}\\ 
9   & \textcolor{SkyBlue}{health} & \textcolor{Blue}{unit} & \textcolor{Aquamarine}{interest} & \textcolor{Green}{interest} & \textcolor{ForestGreen}{general} & \textcolor{Brown}{district}  & \textcolor{Red}{congress}\\ 
10   & \textcolor{SkyBlue}{consent} & \textcolor{Blue}{act} & \textcolor{Aquamarine}{life} & \textcolor{Green}{person} & \textcolor{ForestGreen}{doe} & \textcolor{Brown}{petition}  & \textcolor{Red}{foia}\\ 
   \hline
\end{tabular}
\caption{Top 10 words of highest probability for each topic from the PCTM.}
\label{word_mat_privacy}
\end{table}

The first and the third topics both address abortion as the substantive case in point but differ in the context in how abortion is addressed. 
Paragraphs of the first topic illuminate abortion as a woman's right and discuss the conditions in which the decision can be restricted or unrestricted, such as a woman's health, being a minor, or being ill-informed by her physician, etc. The third topic addresses it in a broader context of a person's right to life and death (e.g., is the right to birth control limited to married couples). The second topic addresses the processes involving lower and higher courts, which we believe to be a byproduct of having paragraphs as the unit for topic assignments. Almost all majority opinions in the SCOTUS have at least one paragraph discussing how the case was appealed from the lower court to higher courts. Since the set of vocabulary and citations in those paragraphs are generally distinct from other paragraphs, the PCTM tends to assign a topic for this category. Paragraphs of the fourth topic mostly concern public protests and speeches surrounding (anti-) abortion decisions in courts. The fifth topic addresses what constitutes damage to privacy under the Privacy Act of 1974. The sixth and seventh topics both concern the public disclosure of private information. The sixth topic, which we label as \texttt{Privacy vs Government Interest}, mainly addresses access to private information, such as the history of drug abuse that might disrupt the operations of government agencies. The seventh topic, on the other hand, concerns whether the way private information is recorded constitutes a violation of Privacy Act of 1974.

% LDA vs RTM vs PCTM
\begin{figure}[t!]
     \centering
     \begin{subfigure}[b]{0.32\textwidth}
         \centering
         \includegraphics[width=\textwidth, trim={2cm, 3cm, 2cm, 2cm}, clip]{figs/cit_network_LDA.pdf}
         \caption{LDA}
         \label{LDA_net}
     \end{subfigure}
     \hfill
     \begin{subfigure}[b]{0.32\textwidth}
         \centering
         \includegraphics[width=\textwidth, trim={2cm, 3cm, 2cm, 2cm}, clip]{figs/cit_network_RTM_circled.pdf}
         \caption{RTM}
	 \label{RTM_net}
     \end{subfigure}
     \hfill
     \begin{subfigure}[b]{0.32\textwidth}
         \centering
         \includegraphics[width=\textwidth, trim={2cm, 3cm, 2cm, 2cm}, clip]{figs/cit_network_PCTM_circled.pdf}
         \caption{PCTM}
	 \label{PCTM_net}
     \end{subfigure}
        \caption{The result of three topic models, LDA, RTM, and PCTM from (a) to (c), on the US Supreme Court opinions of the privacy issue area.
        A node represents an opinion, and an edge represents a citation between opinions.
        The color composition of a node follows the topic proportion of words (LDA, RTM) or paragraphs (PCTM) in the given opinion.
        The color of an edge is based on the estimated topic of the paragraph where the citation is made.
        Note that the topic spaces of the three models are not exactly the same.
        Same colors are assigned to topics that share the top 5 most frequent words between the three models.
        (a) LDA estimates topic structure of documents without reference to the citation network.
        (b) RTM takes into account the linkage between documents for the estimation of topics, but assumes that edges are undirected and remains agnostic about the topics of citations.
        (c) PCTM recognizes the directions of edges and estimates the topic structure of both documents and citations.
        PCTM offers a semantic context over how documents are connected by identifying the topic of the paragraph in which a citation is made.}
        \label{network_figures}
\end{figure}
Next, we compare the results of LDA, RTM, and PCTM on the privacy issue area of the SCOTUS opinions.
Figure~\ref{network_figures} displays the results of LDA, RTM, and the PCTM on the entire SCOTUS opinions on the privacy issue area. 
LDA assigns topics based on words without reference to how documents are connected.
RTM incorporates the networked structure of documents but assumes that connections between documents are undirected and binary.
Moreover, RTM remains agnostic to the semantic context of citations since it does not consider their location within documents, which is reflected in the uniformly gray edges shown in Figure~\ref{RTM_net}.

By contrast, in Figure~\ref{PCTM_net}, the PCTM assigns topics to paragraphs, which allows citations within the same document to have different topics. 
For example, focus on the case, \textit{NASA vs. Nelson}, represented by the node at the center of the network highlighted by a black circle in Figure~\ref{RTM_net} and Figure~\ref{PCTM_net}.
In Figure~\ref{PCTM_net}, it has six out-going edges colored differently according to the PCTM, which implies that the citations are made in the paragraphs addressing different topics.
By contrast, the same case in Figure~\ref{RTM_net} has six edges colored gray, which means that RTM does not differentiate the topics of the citations. 
This showcases the advantage in the PCTM can provide a richer insight into the topic structure of the citations by identifying the topic of the paragraph in which a citation is made. 

%%%% Two column version
\begin{table}[t!]
     \begin{center}
     \begin{tabular}{| p{5cm} | p{5cm} | }
     \hline
     \textcolor{Brown}{Privacy vs Govnt. Interest} & \textcolor{Red}{Public Disclosure of Private Information} \\ \hline \hline
     \includegraphics[width=0.3\textwidth, height=12mm]{figs/para_topic6_example.png}
     &
     \includegraphics[width=0.3\textwidth, height=12mm]{figs/para_topic7_example.png}
      \\ \hline
\tiny{
\textcolor{Brown}{With these interests in view, we conclude that the challenged portions of both SF-85 and Form 42 consist of reasonable, employment-related inquiries that further the Government’s interests in managing its internal operations. See Engquist, 553 U. S., at 598-599; \textbf{Whalen v. Roe, 429 U. S.}, at 597-598. As to SF-85, the only part of the form challenged here is its request for information about “any treatment or counseling received” for illegal-drug use within the previous year. ... The Government has good reason to ask employees about their recent illegal-drug use. Like any employer, the Government is entitled to have its projects staffed by reliable, law-abiding persons who will “ ‘efficiently and effectively’” discharge their duties.}
}
      &
\tiny{
\textcolor{Red}{... Here, the former interest, ``in avoiding disclosure of personal matters,'' is implicated. Because events summarized in a rap sheet have been previously disclosed to the public, respondents contend that Medico’s privacy interest in avoiding disclosure of a federal compilation of these events approaches zero. We reject respondents’ cramped notion of personal privacy ... We have also recognized the privacy interest in keeping personal facts away from the public eye. In \textbf{Whalen v. Roe, 429 U. S. 589 (1977)}, we held that ``the State of New York may record, in a centralized computer file, the names and addresses of all persons who have obtained, pursuant to a doctor’s prescription, certain drugs for which there is both a lawful and an unlawful market.'' Id., at 591. In holding only that the Federal Constitution does not prohibit such a compilation, we recognized that such a centralized computer file posed a ``threat to privacy'':}
}
      \\ \hline
      \end{tabular}
      \caption{Paragraphs containing the same citations but assigned with different topics, \texttt{Privacy vs Government Interest} and \texttt{Public Disclosure of Information}.
       The top row displays a pair of opinions and a citation between the two color-coded by topics, and the left node is the citing opinion and the right node is the cited opinion.
       The second row for each topic contains the text of the paragraph where the citation is made in the two citing opinions in the first row.}
      \label{para_table2}
      \end{center}
\end{table}

To highlight the advantage of the PCTM in finding heterogeneous semantic context around citations, we provide example paragraphs containing citations to the same case but with different topics in Table~\ref{para_table2}. 
Since Supreme Court cases typically address multiple legal domains, subsequent citations to these cases often engage with distinct aspects of their jurisprudence.
For instance, \textit{NASA v. Nelson} and \textit{US v. RCFP} in Table \ref{para_table2} both cite \textit{Whalen v. Roe}, but in distinct substantive contexts. 
For \textit{NASA v. Nelson}, the focus was on whether the employer (NASA) should have access to private information (history of drug abuse) of its employees whereas for \textit{US v. RCFP}, citing \textit{Whalen v. Roe} was mainly about the form of record-keeping of private information (in rap sheet in \textit{US v. RCFP} and in computer files in \textit{Whalen v. Roe}) and the consequent public disclosure of that information. 
This demonstrates the semantic heterogeneity of citations even when they refer to the same document, the nuance that the PCTM can capture with paragraph-level topic assignment.

The PCTM also allows us to visualize the evolution of topics over time by extracting topic-specific subnetworks.
We show how the topics on abortion (\texttt{Regulation of Abortion Procedure} and \texttt{Constitutional Rights to Abortion}) have changed over time.
To emphasize this aspect, we extract from our citation network 11 selected opinions on Reproductive rights in Figure \ref{reprod}.\footnote{The 11 opinions on reproductive rights are selected based on Figure 4 of \cite{clark2012genealogy}.}
\begin{figure}[t!]
     \centering
     \begin{subfigure}[b]{0.45\textwidth}
         \centering
         \includegraphics[width=\textwidth]{figs/roe_v_wade_network_tree_gray_scale1.pdf}
         \caption{Constitutional Rights to Abortion}
         \label{reprod_3}
     \end{subfigure}
     \hfill
     \begin{subfigure}[b]{0.45\textwidth}
         \centering
         \includegraphics[width=\textwidth]{figs/roe_v_wade_network_tree_gray_scale2.pdf}
         \caption{Regulation of Abortion Procedures}
	 \label{reprod_1}
     \end{subfigure}
     \caption{The citation network of 11 selected opinions on reproductive rights. The opinions are part of the SCOTUS subset on the privacy issue area. The left panel highlights the paragraphs and citations of \texttt{Constitutional Rights to Abortion} topic. The right panel colors the paragraphs and citations of \texttt{Regulation of Abortion Procedures} topic. The y-axis represents chronological order such that opinions placed lower indicate older in time and opinions placed in the upper part of the figure are more recent documents.}
        \label{reprod}
\end{figure}

Figure \ref{reprod} displays the topic structure of the 11 selected opinions on reproductive rights. We observe that the topic structure of the subnetwork is governed mostly by two topics -- \texttt{Regulation of Abortion Procedures} or \texttt{Constitutional Rights to Abortion}. 
Earlier opinions predominantly focus on the \texttt{Constitutional Rights to Abortion} topic, establishing the constitutional foundations through cases like \textit{Griswold v. Connecticut} (1965), which centered on privacy rights and reproductive autonomy. 
Later cases shifted toward the \texttt{Regulation of Abortion Procedures}, addressing specific implementation questions such as viability standards and the undue burden test. 
This evolution is exemplified in \textit{Planned Parenthood v. Casey} (1992), which both reaffirmed constitutional protections and established new regulatory frameworks, stating that ``The ability of women to participate equally in the economic and social life of the Nation has been facilitated by their ability to control their reproductive lives.''

While the discussion so far has focused on the substantive implications the PCTM can provide, we also provide discussion about the advantage of the PCTM in predicting new words and citations compared to existing models in Supplementary Information, Section~\ref{sec:predict_prob}.

\subsection{Document-importance in Topic-specific Citation Networks}
The PCTM's ability to assign topics to citations enables extraction of topic-specific subnetworks. 
We construct these subnetworks by including opinions that either send or receive citations of topic $k$. Figure~\ref{subnetwork_privacy_figures} displays the resulting subnetworks for three topics: \texttt{Regulation of Abortion Procedures}, \texttt{Constitutional Rights to Abortion}, and \texttt{Public Disclosure of Private Information}.

% subnetwork analysis, reference to Fowler score
\begin{figure}[t!]
     \centering
     \begin{subfigure}[b]{0.3\textwidth}
         \centering
         \includegraphics[width=\textwidth]{figs/topic_subgraph1.pdf}
         \caption{Regulation of Abortion}
         \label{privacy_topic1}
     \end{subfigure}
     \hfill
     \begin{subfigure}[b]{0.3\textwidth}
         \centering
         \includegraphics[width=\textwidth]{figs/topic_subgraph3.pdf}
         \caption{Const. Right Abortion}
	 \label{privacy_topic2}
     \end{subfigure}
     \hfill
     \begin{subfigure}[b]{0.3\textwidth}
         \centering
         \includegraphics[width=\textwidth]{figs/topic_subgraph7.pdf}
         \caption{Disclosure \\ of Private Info}
	 \label{privacy_topic3}
     \end{subfigure}
        \caption{Subnetworks specific to each topic. The subnetworks are created by extracting opinions that either send or receive citations of the given topic. The topic-specific subnetworks can be useful in revealing whether and the extent to which topological features of the network varies by topic. For each subnetwork, paragraphs of other topics are all colored in gray for better visualization.}
        \label{subnetwork_privacy_figures}
\end{figure}

Topic-specific subnetworks represent citation patterns within distinct semantic domains, enabling the application of established network analysis methods to semantically coherent subsets of citations. These methods include the ``family tree of law'' approach developed by \cite{clark2012genealogy} and the importance score proposed in \cite{fowler2007network}. Here, we focus on Fowler et al.'s importance scores, which measure an opinion's precedential significance and predict its likelihood of future citations. 
Recognizing semantic differences, however, is critical when computing importance scores because the absence of a citation to a precedent could have two different meanings: that the given precedent does not carry much legal weight or that the given precedent addresses a completely distinct legal issue. 
To demonstrate the significance of semantic context in citation analysis, we compare importance scores computed on the complete network with those derived from topic-specific subnetworks.

The importance score has two parts based on their citation directions. The outward relevance score is based on the number of citations an opinion makes, evaluating the opinion's weight in referencing pertinent legal questions. An opinion with high outward relevance score cites many other opinions that are also deemed important and legally relevant. The inward relevance score is based on the number of citations an opinion receives from other opinions, gauging the extent to which it serves as the integral part of the law as a precedent. An opinion with high inward relevance score is cited by many other important and influential opinions. Since these scores are computed using eigenvectors, they are invariant to scales. In this light, \cite{fowler2007network} suggests using ranks of inward and outward relevance scores as the measure of importance for opinions as precedents. 

\begin{table}[t!]
\centering
\scriptsize
\begin{tabular}{llll}
 & Top 1 Inward-relevant & Top 2 Inward-relevant & Top 3 Inward-relevant \\ 
  \hline
  \hline
  \textit{All Topics} & Planned Parenthood v. Danforth & Roe v. Wade & Griswold v. Connecticut  \\   
  \hline
  \textit{Reg. Abortion} & Planned Parenthood v. Danforth & Colautti v. Franklin & Bellotti v. Baird \\ 
  \textit{Proc. Posture} & Renegotiation Board v. Bannercraft & Hickman v. Taylor & EPA v. Mink \\ 
  \textit{Const. Abortion} & Griswold v. Connecticut & Roe v. Wade & Eisenstadt v. Baird \\ 
  \textit{Speech \& Protest} & Schenck v. Pro-choice Network & Madsen v. Women's Health Center & Roe v. Wade \\ 
  \textit{Damage to Privacy} & Doe v. Chao & US ex rel. Touhy v. Ragen & US v. Reynolds \\ 
  \textit{Privacy v. Govnt.} & Vernonia v. Wayne & Chandler v. Miller & Whalen v. Roe \\ 
  \textit{Pub. Disclosure} & EPA v. Mink & Air Force v. Rose & NLRB v. Sears \\ 
   \hline
\end{tabular}
\caption{Top 3 most inward-relevant cases by topics. The inward relevance scores are computed following \cite{fowler2007network}.}
\label{privacy_inward}
\end{table}

In Table \ref{privacy_inward}, none of the topic-specific top 3 inward-relevant cases exactly match those that are from the entire citation network of the privacy cases. The top 3 inward-relevant for all topics (row 1) seem to be drawing information from two topics -- \texttt{Regulation of Abortion} and \texttt{Constitutional Rights to Abortion}. 
If one is interested in \texttt{Speech \& Protest}, for example, \textit{Schenck v. Pro-choice Network} is the most inward-relevant. \textit{Schenck v. Pro-choice Network} is an influential case that draws the line between public safety and free speech. In \textit{Schenck v. Pro-choice Network}, the SCOTUS concluded that the fifteen feet buffer zone between anti-abortion protestors and abortion clinics was constitutional, but deemed unconstitutional fifteen feet buffer zone between protestors and people seeking entrance to clinics. 
For \texttt{Public Disclosure of Information} topic, \textit{EPA v. Mink} is the most inward-relevant. The case addresses the disclosure of secret documents prepared for a scheduled underground nuclear test, gauging the balance between the Freedom of Information Act (1966) and national security matters. 
Both examples show that one can draw very different conclusions on which case is most inward-relevant, depending on the legal context and area.

\begin{table}[t!]
\centering
\scriptsize
\begin{tabular}{llll}
 & Top 1 Outward-relevant & Top 2 Outward-relevant  & Top 3 Outward-relevant  \\ 
  \hline
  \hline
  \textit{All Topics} & Hodgson v. Minnesota & Akron v. Akron Center & Webster v. Reproductive Health\\  
  \hline
  \textit{Reg. Abortion} & Akron v. Akron Center & Hodgson v. Minnesota & Webster v. Reproductive Health \\ 
  \textit{Proc. Posture}& NLRB v. Sears  & US v. Weber & DOI v. KWUPA \\ 
  \textit{Const. Abortion}  & Carey v. Population Services Int. & Planned Parenthood v. Casey & Hodgson v. Minnesota \\ 
  \textit{Speech \& Protest}  & Hill v. Colorado & Schenck v. Pro-choice Network & Roe v. Wade \\ 
  \textit{Damage to Privacy} & Federal Aviation Admin. v. Cooper & NASA v. Nelson & US ex rel. Touhy v. Ragen \\ 
  \textit{Privacy v. Govnt.}  & Board of Education v. Earls & Chandler v. Miller & Whalen v. Roe \\ 
  \textit{Pub. Disclosure} & DOJ v. Reporters Comm. & FBI v. Abramson & DOJ v. Tax Analysts \\ 
   \hline
\end{tabular}
\caption{Top 3 most outward-relevant cases by topics. The outward relevance scores are computed following \cite{fowler2007network}.}
\label{privacy_outward}
\end{table}

Table~\ref{privacy_outward} shows that while the top three outward-relevant cases in the complete citation network primarily reflect rankings from the \texttt{Regulation of Abortion} and \texttt{Constitutional Rights to Abortion} topics, different patterns emerge when examining specific topics. For instance, in the \texttt{Public Disclosure of Information} topic, \textit{Department of Justice v. Reporters Committee for the Freedom of Press} is the most outward-relevant case. The given case addresses whether the FBI should disclose criminal records to media outlets in the interest of public knowledge and safety. Together with Table~\ref{privacy_inward}, Table~\ref{privacy_outward} shows that legal context can be heterogeneous within the privacy issue area, and such semantic heterogeneity can lead to varying conclusions on the precedential importance of cases.

%\subsection{Application on a New Case in Abortion}
\subsection{Topic Prediction for a New Abortion Case}

This section presents additional results on a new controversial case regarding abortion. 
On June 24 2022, the Supreme Court made a landmark decision on abortion that invoked a nationwide controversy.
In the case, \textit{Dobbs v. Jackson Women's Health Organization}, the SCOTUS held that abortion is not a part of constitutional rights, and it conferred individual states the right to ban abortion. 
This case overturned both \textit{Roe v. Wade} and \textit{Planned Parenthood v. Casey}, the landmark precedents that have served as the legal basis for the constitutional rights to abortion.  
While qualitative reading of \textit{Dobbs v. Jackson Women's Health Organization} suggests that this case is a clear deviation from the recent trends in abortion rulings in many ways, it is difficult to demonstrate the deviations in a quantitative way. 

Using the PCTM, we examine how the topic structure of \textit{Dobbs v. Jackson Women's Health Organization} differs from the recent rulings on abortion in our corpus.
To do so, we computed the predicted probability of topics of the paragraphs in \textit{Dobbs v. Jackson Women's Health Organization} using the model fitted on our abortion corpus.
We first train the PCTM on the abortion corpus used in the above analysis and then computed the posterior predictive distribution of topics. 
The exact formula to obtain the posterior predictive probability is in Supplementary Information, Section~\ref{sec:predictive}.

To validate that the meaning of the topics is consistent in the new case, \textit{Dobbs v. Jackson Women's Health Organization}, we provide a qualitative analysis of the estimated topics by focusing on the paragraphs that cite the same precedent.
Table~\ref{dobbspara} presents two paragraphs that cite the same precedent, \textit{Planned Parenthood v. Casey} (505 U.S., 878), but with different estimated topics.
The left paragraph has the estimated topic \texttt{Constitutional Rights to Abortion} while the right paragraph has the topic \texttt{Regulation of Abortion Procedure}.
The left paragraph is an introductory paragraph of the judges' criticism of Casey's argument that abortion is a part of the liberty protected by the Fourteenth Amendment.
This is clearly related to whether abortion is a part of constitutional rights or not. 
By contrast, the right paragraph criticizes the ``undue burden'' test that Casey decides.
The undue burden test offers criteria about what kind of state regulations on abortion is prohibited.
Therefore, we can infer that this paragraph discusses a more specific issue about how states regulate abortions. 
By reading these paragraphs, we can verify that the interpretation of the topics in this new case match our interpretations of the topics in the abortion corpus.

\begin{table}[t!]
  \centering
  \begin{tabular}{| m{0.45\textwidth} | m{0.45\textwidth} | }
  \hline
  \textcolor{Aquamarine}{Constitutional Rights to Abortion} & \textcolor{SkyBlue}{Regulation of Abortion Procedures}\\
  \hline
  \hline
  We turn to Casey's bold assertion that the abortion right is an aspect of the ``liberty'' protected by the Due Process Clause of the Fourteenth Amendment. \textbf{505 U.S., at 846}
  &
  The Casey plurality tried to put meaning into the ``undue burden'' test by setting out three subsidiary rules [...] The first rule is that ``a provision of law is invalid, if its purpose or effect is to place a substantial obstacle in the path of a woman seeking an abortion before the fetus attains viability.'' \textbf{505 U.S., at 878}\\
  \hline
\end{tabular}
  \caption{Comparison of Paragraphs in \textit{Dobbs v. Jackson} with Different Estimated Topics on Abortion.\\
  Both paragraphs cite the same precedent, \textit{Planned Parenthood v. Casey} (505 U.S., 878), but with different estimated topics. 
  }
  \label{dobbspara}
\end{table}

How do the topic structure in \textit{Dobbs v. Jackson Women's Health Organization} differ from the recent landmark cases in our corpus? 
For comparison, we also computed the predicted probability of the topics for the two recent precedents about abortion in our corpus: \textit{Gonzales v. Carhard} and \textit{Stenberg v. Carhard}, two recent landmark cases in abortion in our corpus.
Figure~\ref{abortion_topic_bar} shows the predicted probability of topics for each paragraph for the three cases on abortion, \textit{Gonzales v. Carhard}, \textit{Stenberg v. Carhard}, and \textit{Dobbs v. Jackson Women's Health Organization}, from top to bottom.
Each vertical bar represents a paragraph, and each bar is colored according to the predicted probability of topics.
Since we want to focus on the difference in the legal discourse regarding abortion, we focus our analysis on the two topics relevant to abortion: \texttt{Constitutional Rights to Abortion} or \texttt{Regulation of Abortion Procedure}. 
While more than 90\% of the paragraphs of both Gonzales and Stenberg are assigned with \texttt{Regulation of Abortion Procedure} topic, only 28\% of the paragraphs in \textit{Dobbs v. Jackson} are assigned with the \texttt{Regulation of Abortion Procedure} topic and 67\% of the paragraphs are assigned with \texttt{Constitutional Rights to Abortion}. 
This accurately reflects the fact that \textit{Dobbs v. Jackson Women's Health Organization} is distinct from the current trend in the abortion rulings in our corpus. 

\begin{figure}[t!]
  \includegraphics[width=\textwidth]{figs/abortion_topic_bar.pdf}
  \caption{Predicted Probability of Topics for the Paragraphs of Dobbs v. Jackson.\\
    Each vertical bar represents a paragraph. 
    Each paragraph is colored according to the predicted probability of topics.
    We focus on two topics related to abortion: \texttt{Constitutional Rights to Abortion} and \texttt{Regulation of Abortion Procedure}.
    The case are \textit{Gonzales v. Cargard}, \textit{Stenberg v. Carhard}, and \textit{Dobbs v. Jackson Women's Health Organization}, from top to bottom.
    \textit{Dobbs v. Jackson Women's Health Organization} case have more paragraphs with \texttt{Constitutional rights to abortion} topic rather than \texttt{Regulation of abortion procedure} topic while the two recent precedents in our corpus, \textit{Gonzales v. Carhard} and \textit{Stenberg v. Carhard}, are the opposite. 
    This shows that \textit{Dobbs v. Jackson Women's Health Organization} goes against the recent trend in the abortion cases in our corpus, where the stronger emphasis is placed on how abortion can be regulated by the states instead of whether abortion is a part of the constitutional rights, as shown in \textit{Gonzales v. Carhard} and \textit{Stenberg v. Carhard}.
  }
  \label{abortion_topic_bar} 
\end{figure}

