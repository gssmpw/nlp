\section{The United States Supreme Court Opinions}
\label{sec:applicationBK}

The SCOTUS as the highest judicial authority in the United States holds a pivotal role in social science studies with implications for social norms, public policy, and individual rights. 
At the center of the SCOTUS ruling is the principle of \textit{stare decisis} in which a decision must ``stand by things decided.''
The \textit{stare decisis} establishes that precedents take crucial importance in the SCOTUS as they exert varying levels of influence on future rulings.
In particular, landmark cases such as \textit{Roe v. Wade} are a crucial subject of study in social sciences. 
With the \textit{stare decisis}, a SCOTUS ruling is not just about the case at hand, but also about how to interpret the relevant precedents that together shape the boundary of social norms and behavior. 

Due to the significance of precedents in the SCOTUS, many social science studies have been dedicated to exploring various aspects of precedents.
Many scholars have focused on the political processes involved in the choice and the representation of precedents in the SCOTUS majority opinions \citep{hansford2006politics, bailey2008does,clark2010locating}.
How precedents are treated by future cases and eventually fade away was another focal point of research \citep{black2013citation, broughman2017after}.

Mapping the SCOTUS cases and citations into a network, past studies employed network analysis to measure the structural properties of precedents in the citation network.
\cite{clark2012genealogy}, for instance, fits the latent tree model to the SCOTUS citation network and uncovers the hierarchy of precedents as an estimation of the evolution of legal doctrine.
Another strand of research highlights the positions precedents take in the citation network \citep{fowler2005authority, fowler2007network}.
\cite{fowler2007network} and \cite{fowler2008authority} propose a variation of the eigenvector centrality score to gauge how legally ``central'' a case is for the SCOTUS at a given point in time.

While recognizing the usefulness of the network analysis for the SCOTUS citation network, we find that existing approaches commonly overlook the topic heterogeneity of the citation network.
Network analysis of the SCOTUS citation network treats presence and absence of citations as informative signals.
In \cite{fowler2007network} and \cite{fowler2008authority}, for example, precedents that attract many citations are likely to be structurally central and precedents without many citations are considered to be peripheral.
While the presence of citations can be an informative signal for the importance of a case, the absence of citations may simply be due to topic inconsistency rather than its importance.
That is, we do not expect a case to cite a precedent if the given precedent addresses completely distinct legal topics. 
When the network analytic methods are applied to the universe of cases without special attention to the topic differences between them, one may mistakenly interpret the topic differences as indicators of importance.

Another point we highlight is that the topic space of an opinion is multidimensional. 
For example, \textit{Roe v. Wade} is mostly known for the right to privacy in abortion, but it also addresses other legal topics such as substantive due process, end-of-life decisions, and legislative restraints.
A citation to \textit{Roe v. Wade} can be concerning the right to privacy, but it could also be about other topics such as legislative restraints. 
This suggests that subsetting down to a broad legal category of cases for network analysis, such as seen in \cite{clark2010locating} where authors limit their scope to search and seizure and the freedom of religion opinions, may not be sufficient to capture nuanced legal topics that a case touches upon. 

To address the above key challenge, we propose to incorporate the text of the SCOTUS majority opinions with the citation network. 
In the following sections, we propose a model that incorporates both the text and the citation network of the SCOTUS majority opinions. 
Our model can uncover the topic structure of the majority opinions with topic model while utilizing the network linkage in the citation network.
We apply our model to all privacy opinions in the SCOTUS and demonstrate that the resulting topic-homogenous citation subnetwork can be used for further network analysis.

For the application of our model, we obtain the universe of the SCOTUS majority opinions on the privacy issue area from the Caselaw Access Project\footnote{\url{https://case.law}}.
Subsetting follows the issue area categorization provided by Supreme Court Database \citep{scdb}. 
The privacy issue area is chosen for our application because existing literature on citation networks of the SCOTUS cases often focuses on this issue \citep{fowler2007network, clark2012genealogy}.
It is also an important application given the recent controversial decision that overruled the landmark case on constitutional rights to abortion. 
The Privacy opinions subset consists of 106 documents with 4,669 paragraphs, 5,838 unique words, and 452 citations.
More details of data pre-processing for each subset are available in Supplementary Information, Section~\ref{sec:data}.

