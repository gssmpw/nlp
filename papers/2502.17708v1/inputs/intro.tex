\section{Introduction}

Social scientists often use citation network data to study the influence of documents, such as academic articles, books, laws, and court opinions. 
Research in judicial politics has analyzed the citation networks of the United States Supreme Court (SCOTUS) opinions, revealing how some cases exert significant influence on future rulings \citep{clark2012genealogy, fowler2007network}. 
Similarly, in international relations, scholars explore how citations shape power dynamics in areas like trade \citep{pelc2014politics}, human rights \citep{lupu2012precedent}, and jurisdictional conflicts \citep{larsson2017speaking}.

Conventional approaches seek to summarize document attributes within a network, but often overlook the diverse semantic contexts in which citations occur.
Since the semantic content of documents influences citation network structures \citep{bai2018neural, chang2010hrtm, zhang2022dynamic}, accounting for semantic heterogeneity in document networks can reveal information that might otherwise remain hidden.
The measures of precedential importance for various courts of law, for instance, implicitly treat the absence of citations as a reflection of limited precedential value rather than a potential semantic disconnect between documents \citep{fowler2007network, lupu2012precedent, pelc2014politics}. 
However, a high volume of citations to a court case may indicate its status as a landmark case, the popularity of legal topics it addresses, or both.

Recognizing the importance of semantic heterogeneity in document networks, previous studies have used human-coded topics to ensure semantic coherence, restricting their analyses to documents within discrete semantic domains such as criminal justice  \citep{olsen2017finding} or reproductive rights \citep{clark2012genealogy}.
However, human coding often captures broad categories, leaving significant semantic variation within these groups unaddressed. 
Also, researchers may wish to automatically detect semantic heterogeneity at the granularity that fits their research purpose, or the semantic context itself can be of research interest rather than an object to control for. 

This paper develops a Bayesian topic model that systematically integrates citation network and document text. 
Our proposed model, the paragraph-citation topic model (PCTM), extends conventional topic models by assigning a topic to each paragraph of the citing document, allowing citations to share topics with text of the paragraphs that they are in. 
This marks a departure from other topic models for document networks (i.e. Relational Topic Models) by allowing citations in one document to have heterogeneous topics. 
Our empirical analysis demonstrates that citations within individual documents frequently span multiple substantive areas. 
Moreover, our findings reveal considerable topical diversity in citations to individual documents, illustrating how a single opinion can intersect multiple domains of legal discourse (i.e., \textit{Roe v. Wade} engages with various legal issue areas, including civil procedure, constitutional law, healthcare policy, privacy rights, and beyond).

