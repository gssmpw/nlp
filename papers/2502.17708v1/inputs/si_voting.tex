\section{Results on the SCOTUS cases on Voting Rights}
\label{sec:voting}

The SCOTUS documents and citations on voting rights proliferated exponentially since the enactment of Voting Rights Act (VRA) in 1965. A number of sections in VRA were challenged over the course of modern American political history, and the majority of those challenges made their way to the Supreme Court. The Supreme Court database assigns 3 issue codes for opinions related to voting.\footnote{The three issue codes on voting are voting, Voting Rights Act of 1965, Ballot Access.} After examining a subset of documents with these issue codes, we decided to set the number of topics to 4 for PCTM.

Table \ref{word_mat_voting} presents the 10 words that appear most frequently for each topic. 
\begin{table}[ht!]
\centering
\begin{tabular}{c| l l l l}
  \hline
Topic & \textcolor{SkyBlue}{Voter} & \textcolor{Blue}{Ballot} & \textcolor{Aquamarine}{Preclearance} & \textcolor{Green}{Voter}  \\ 
Label & \textcolor{SkyBlue}{Eligibility} & \textcolor{Blue}{Access} & \textcolor{Aquamarine}{Requirement} & \textcolor{Green}{Dilution} \\ 
  \hline
1 & \textcolor{SkyBlue}{counti} & \textcolor{Blue}{ballot} & \textcolor{Aquamarine}{chang} & \textcolor{Green}{plan}  \\ 
2   & \textcolor{SkyBlue}{resid} & \textcolor{Blue}{primari} & \textcolor{Aquamarine}{attorney} & \textcolor{Green}{minor} \\ 
3   & \textcolor{SkyBlue}{appel} & \textcolor{Blue}{polit} & \textcolor{Aquamarine}{preclear} & \textcolor{Green}{black} \\ 
4   & \textcolor{SkyBlue}{school} & \textcolor{Blue}{offic} & \textcolor{Aquamarine}{counti} & \textcolor{Green}{major} \\ 
5   & \textcolor{SkyBlue}{properti} & \textcolor{Blue}{counti} & \textcolor{Aquamarine}{practic} & \textcolor{Green}{polit} \\ 
6   & \textcolor{SkyBlue}{citi} & \textcolor{Blue}{file} & \textcolor{Aquamarine}{procedur} & \textcolor{Green}{popul} \\ 
7   & \textcolor{SkyBlue}{tax} & \textcolor{Blue}{interest} & \textcolor{Aquamarine}{cover} & \textcolor{Green}{racial} \\ 
8   & \textcolor{SkyBlue}{board} & \textcolor{Blue}{independ} & \textcolor{Aquamarine}{plan} & \textcolor{Green}{member} \\ 
9   & \textcolor{SkyBlue}{citizen} & \textcolor{Blue}{nomin} & \textcolor{Aquamarine}{section} & \textcolor{Green}{dilut} \\ 
10   & \textcolor{SkyBlue}{test} & \textcolor{Blue}{burden} & \textcolor{Aquamarine}{object} & \textcolor{Green}{white} \\ 
   \hline
\end{tabular}
\caption{Top 10 words of highest probability for each topic from PCTM.}
\label{word_mat_voting}
\end{table}
The first topic \texttt{Voter Eligibility} includes paragraphs that address conditions under which a voter is eligible to register for certain elections. For example, \texttt{Allen et al. v. State Board of Elections et al.} (1969) contains a paragraph of the first topic that discusses whether a 31-year-old man was eligible to cast his vote in a local school district election based on his tax records and property ownership in the neighborhood. The second topic \texttt{Ballot Access} concerns the issue of candidates' access to ballots. A paragraph of this topic in \texttt{Carrington v. Rash et al.} (1965) states that ``... the Texas system creates barriers to candidate access to the primary ballot, thereby tending to limit the field of candidates from which voters might choose.'' Preclearance requirement in Voting Rights Act of 1965 section 5. is the primary issue in the third topic. \texttt{Cipriano v. City of Houma et al.} (1969) contains a paragraph of this topic that stipulates ``... and unless and until the court enters such judgment no person shall be denied the right to vote for failure to comply with such qualification, prerequisite, standard, practice, or procedure: Provided, That such qualification, prerequisite, standard, practice, or procedure may be enforced without such proceeding if the' qualification, prerequisite, standard, practice, or procedure has been submitted by the chief legal officer or other appropriate official ...'' The fourth topic, on the other hand, addresses Voting Rights Act of 1965, section 2 that prohibits voting practices that leads to dilution of voting strength of minority groups. For example, \texttt{Mcdonald et al. v. Board of Election Commissioners of Chicago et al.} (1969) contains multiple paragraphs of this topic one of which states that ``... the Court upheld a constitutional challenge by Negroes and Mexican-Americans to parts of a legislative reapportionment plan adopted by the State of Texas ... .''

The 4 topics that PCTM identified have varying presence in American political history over time. Figure \ref{voting_topic_growth} shows the cumulative count of paragraphs of each topic. 
% topic growth over time        
\begin{figure}[ht!]
     \centering
         \includegraphics[width=\textwidth]{figs/topic_over_time_cum.pdf}
         \caption{Cumulative number of topics in Voting Rights subset over time.}
	 \label{voting_topic_growth}
\end{figure}
The growth of \texttt{Voter Eligibility} topic (in light blue) is most evident until the 1980s and the topics on \texttt{Preclearance Requirement} (in light green) or \texttt{Voter Dilution} (in dark green) become more prevalent in relatively recent periods. This is consistent with \cite{ansolabehere2008end} that describes that discourses on malapportionment was more common in earlier periods, and the topics on equal representation and access to vote, especially with respect to race and minority groups, are becoming more prominent issues in modern American politics. 


% topic-specific subgraphs
\begin{figure}[t!]
     \centering
     \begin{subfigure}[b]{0.23\textwidth}
         \centering
         \includegraphics[width=\textwidth]{figs/topic_subgraph1_voting.pdf}
         \caption{Voter Eligibility}
         \label{voting_topic1}
     \end{subfigure}
     \hfill
     \begin{subfigure}[b]{0.23\textwidth}
         \centering
         \includegraphics[width=\textwidth]{figs/topic_subgraph2_voting.pdf}
         \caption{Ballot Access}
	 \label{voting_topic2}
     \end{subfigure}
     \hfill
     \begin{subfigure}[b]{0.23\textwidth}
         \centering
         \includegraphics[width=\textwidth]{figs/topic_subgraph3_voting.pdf}
         \caption{Preclearance \\ Requirement}
	 \label{voting_topic3}
     \end{subfigure}
     \hfill
     \begin{subfigure}[b]{0.23\textwidth}
         \centering
         \includegraphics[width=\textwidth]{figs/topic_subgraph4_voting.pdf}
         \caption{Voter Dilution}
	 \label{voting_topic4}
     \end{subfigure}
        \caption{The subnetwork specific to each topic. The subnetworks are created by extracting opinions that either send or receive citations of the given topic. The topic-specific subnetworks can be useful in revealing whether and the extent to which topological features of the network varies by topic. For each subnetwork, paragraphs of other topics are all colored in gray for better visualization.}
        \label{subnetwork_voting_figures}
\end{figure}

Figure \ref{subnetwork_voting_figures} shows groups of cases that make citations of the given topic. The location of cases on each network is based on their connection patterns such that cases that cite other cases jointly are placed closer to each other. The majority of cases in the third and the fourth panel are located very close to each other, indicating that those cases heavily cite each other. On the other hand, the citation subnetwork in the first panel (\texttt{Voter Eligibility}) is more spread out in comparison. This reflects the fact that opinions on \texttt{Preclearance Requirement} and \texttt{Voter Dilution} have proliferated in a shorter period of time, closely building up on past cases of the same topic whereas opinions on \texttt{Voter Eligibility} have expanded more independently and incrementally over a longer period of time.

The coefficients in the latent citation propensity for Voting subset also have expected signs, with posterior samples of $\tau_1$ and $\tau_2$ both staying above 0. That is, for the citation decisions of opinions for Voting, the authority as well as the topic similarity of precedents have positive impacts. Moreover, the distribution of all $\pmb\tau$ entries stays very similar between the Privacy and the Voting subset, indicating that the citation dynamics do not vary much between different issue areas within the SCOTUS 

\begin{figure}[ht!]
     \centering
     \begin{subfigure}[b]{0.45\textwidth}
         \centering
         \includegraphics[width=\textwidth]{figs/oddsrat_box_voting.pdf}
         \caption{Change in Log Odds Ratio by Additional Citations to Precedents}
         \label{reg_interpret_indegree_voting}
     \end{subfigure}
     \hfill
     \begin{subfigure}[b]{0.45\textwidth}
         \centering
         \includegraphics[width=\textwidth]{figs/oddsrat_topic_voting.pdf}
         \caption{Change in Log Odds Ratio by Increases in $\eta_{j,z_{ip}}$}
         \label{reg_interpret_topic_voting}
     \end{subfigure}
     \caption{Changes in the log odds ratio of citation between a paragraph and a precedent as we increment the authority and the topic similarity of the given precedent. Same exercise used in Figure \ref{reg_interpret_topic_privacy} is employed to create this figure.}
        \label{reg_interpret_voting}
\end{figure}

Similar to the exercise to create Figure \ref{reg_interpret_privacy}, 10,000 randomly drawn pairs of paragraphs and precedents for the Voting subset were used to generate Figure \ref{reg_interpret_voting}. The left panel of Figure \ref{reg_interpret_voting} presents the improvements in the log odds ratio as we increment the authority of the given precedent. For example, if the given precedent had 3 more citations, the odds of the given paragraph citing the given precedent increases by about 25\%. The right panel shows changes in log odds ratio as the topic similarity between the given precedent and the given paragraph increases. 

