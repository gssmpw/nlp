\section{Concluding Remarks}
\label{sec:conclusion}

Social scientists often use citation networks to study how documents influence following documents in various domains, such as political science, international relations, and legal studies.
However, conventional approaches to analyzing citation networks often overlook the semantic context in which citations occur.
While existing studies use document-level labels to find the context of citations, this approach assumes that all citations within a document are made under the same context, which may lead to misunderstanding of how citations reflect the influence of documents.
To address this challenge, this paper proposes a novel joint model of text and citations, the paragraph-citation topic model.
The key innovation of the PCTM is to assign topics to paragraphs, which allows citations in different paragraphs to be associated with different topics.
After deriving a collapsed Gibbs sampler for inference, we applied the PCTM to the SCOTUS opinions on privacy issues to highlight the diversity of topics of citations within each document.
Also, the model uncovered informative subnetworks of the judicial opinions that shared citations with the same topic.

The applications of the PCTM need not be limited to citation networks of legal documents. 
The model will help address a number of important research questions in the analysis of document networks. 
For example, a researcher can use the latent citation propensity in the PCTM to understand the role of authors' gender in citation making in academic journals. 
Since academic articles address diverse scholarly subjects, capturing semantic contexts in the analysis of citation formation is critical, and can be properly addressed in our model.
Moreover, as the PCTM estimates topic-specific subnetworks of citations using information from both text and networks in a unified framework, it can be used together with established measures of networks, such as legal importance scores in \cite{fowler2007network}, to produce better academic insights. 
