\section{The Proposed Model}
\label{sec:model}
Our proposed model is built on a topic model, a popular model to discover latent clusters or topics of documents \citep{blei2003lda, blei2007correlated}. 
A topic model that analyzes documents with citation networks must address the following questions: 
By what process do authors decide to cite another document? 
How does the topic structure enter into citation decisions, and conversely, how do citations help determine the topic structure of citing and cited documents? 

To address these questions, we augment a topic model by latent citation propensity to model authors' decisions to make citations in relation to the topic structure. 
The latent citation propensity is shaped by a regression model that reflects the known factors of strategic citation behavior such as the authority (or popularity) of the cited document \citep{larsson2017speaking, lupu2012precedent, lupu2013strategic, pelc2014politics} as well as the similarity of topics between citing and cited documents.

Additionally, we propose to use paragraphs as the unit for the topic assignment. 
We view citations as the directed reference from a paragraph to another document. 
The advantage of this perspective is that it reflects a more realistic data-generating process. 
A paragraph is often the vehicle of one coherent topic, and citations within that paragraph are likely to refer to documents of very similar, if not the same, topic prevalence. 
For example, an opinion in the SCOTUS typically identifies multiple legal doctrines that apply to a given case and addresses them in different paragraphs. 
Therefore, citations within one paragraph are likely to point to a collection of opinions that address the same legal doctrine. 
In other words, citations in paragraphs of different topics are likely to be references to different legal contexts, even if they are from the same document. 
We believe such characteristics are not limited to legal documents of the SCOTUS, but a general feature of any document network, and they should be reflected in the process of uncovering topic structure. 
Below, we delineate our modeling strategy that addresses the above questions in detail. 

\subsection{Paragraph-citation Topic Model}
First, we introduce the notation.
Let $N$, $G$, $V$, and $K$ be the total number of documents, total number of paragraphs, and total number of unique words, and the number of topics, respectively.
We use $N_{ip}$ to denote the number of words in paragraph $p$ of document $i$.
Our data consist of words, $\textbf{W}$, and citations, $\textbf{D}$.
$\textbf{W}$ is a matrix of size $G \times V$ where each row is $\textbf{w}_{ip}$, a vector of length $V$ that represents the number of times each unique word appears in a paragraph $p$ of document $i$.
$\textbf{D}$ is a matrix of size $G \times N$ where each element, $D_{ipj}$ is a binary variable that indicates the existence of a citation from $p$th paragraph in $i$th document towards $j$th document.
$\textbf{D}^*$ is a matrix of size $G \times N$ and its element, $D_{ipj}^*$, is a latent variable that represents the latent citation propensity of $p$th paragraph in $i$th document to cite $j$th document.
We have another latent variable $\textbf{Z}$, a vector of length $G$ where each element is $z_{ip}$, a scalar that takes a value from $\{1,\ldots, K\}$, and it represents the topic assignment of $p$th paragraph in $i$th document.
We have three main parameters to estimate: $\pmb\eta$, $\pmb\Psi$, and $\pmb\tau$.
$\pmb\eta$ is a matrix of size $N \times K$ where each row is $\pmb\eta_i$, a vector of length $K$ that represents the topic proportion of document $i$, generated from a multivariate normal distribution with mean $\pmb\mu$ and covariance $\pmb\Sigma$.
$\pmb\mu$ is further generated from a normal distribution with mean $\pmb\mu_0$ and covariance $\pmb\Sigma_0$.
$\pmb\Psi$ is a matrix of size $K \times V$ where each row is $\pmb\Psi_k$, a vector of length $V$ that represents the word distribution of topic $k$. 
$\pmb\Psi_k$ is generated from a Dirichlet distribution with parameter $\pmb\beta$.
$\pmb\tau$ is a vector that represents the coefficients of the regression model that shapes the latent citation propensity, generated from a multivariate normal distribution with mean $\pmb\mu_{\tau}$ and covariance $\pmb\Sigma_{\tau}$.

The data-generating process is modeled as follows.

\begin{align}
  \begin{split}
	D_{ipj} &= \begin{cases}
		1 \text{ if } D_{ipj}^* \geq 0 \\
		0 \text { if } D_{ipj}^* < 0
	\end{cases}\\
    D_{ipj}^* &\sim \mathcal{N}(\pmb\tau^T\textbf{x}_{ipj},1)\quad \text{where}\ \textbf{x}_{ipj} = [1, \kappa_{j}^{(i)}, \eta_{j,z_{ip}}]\\
	\textbf{w}_{ip} &\sim \text{Multinomial}(N_{ip},\pmb\Psi_{z_{ip}})  \\
	z_{ip} &\sim \text{Multinomial}(1,\text{softmax}(\pmb\eta_i))  \\
	\pmb\Psi_k &\sim \text{Dirichlet}(\pmb\beta)  \\
	\pmb\eta_i &\sim \mathcal{N}(\pmb\mu,\pmb\Sigma)  \\
	\pmb\mu &\sim \mathcal{N}(\pmb\mu_0, \pmb\Sigma_0)  \\
	\pmb\tau &\sim \mathcal{N}(\pmb\mu_{\tau},\pmb\Sigma_{\tau})
  \end{split}
\end{align}

\noindent where $\mathbf{x}_{ipj}$ is a vector of covariates that shape the latent citation propensity for $p$th paragraph in document $i$ to cite document $j$. $\mathbf{x}_{ipj}$ consists of 3 terms -- the intercept, indegree, and $\eta_{j,z_{ip}}$, and $\pmb\tau = [\tau_0, \tau_1, \tau_2]$ is a vector of coefficients.
The intercept in $\textbf{x}_{ipj}$ is to capture the overall sparsity of the citation network. Since networks in the real world are generally very sparse, we expect the intercept $\tau_0$ to be negative. 
The indegree of a precedent is included to capture the authority. 
This follows existing studies of strategic citation that commonly point to the importance of the authority of a precedent as one of the major attracting factors of citations  \citep{hansford2006politics,lupu2012precedent, lupu2013strategic}. 
This is also consistent with a well-known dynamic in social networks called ``rich-get-richer'' or, more technically, ``preferential attachment'' where popular individuals become more popular \citep{newman2001clustering,wang2008measuring}. 
The indegree term is denoted $\kappa_j^{(i)}$, with superscript $(i)$ to indicate the authority of the $j$th document at the time of $i$'s writing. We expect its coefficient $\tau_1$ to be positive. 
Finally, $\eta_{j,z_{ip}}$ is added to capture the topic similarity between the citing paragraph $ip$ and document $j$. 
Since we expect that citations are more likely to occur between documents of similar topics, we expect its coefficient $\tau_2$ to be positive. 
 
While we currently include 3 document-level covariates in $\textbf{x}$, researchers can add other covariates that fit their research purposes. 
For instance, the political ideology of judges in a precedent and a citing case can be an important factor in citation decisions \citep{lupu2013strategic}. 
Then researchers can include a binary copartisanship indicator in $\textbf{x}_{ipj}$ that takes 1 if the author of opinion $i$ and the author of opinion $j$ are appointed by presidents of the same party and 0 otherwise. 

Given words and citations, $\textbf{W}$ and $\textbf{D}$, our posterior probability is 
{\small
\begin{align}
	p(\pmb\eta,\pmb\Psi,\textbf{Z},\pmb\tau|\textbf{W},\textbf{D}) \propto p(\pmb\mu|\pmb\mu_0,\pmb\Sigma_0)p(\pmb\tau|\pmb\mu_{\tau},\pmb\Sigma_{\tau})p(\pmb\eta|\pmb\mu,\pmb\Sigma)p(\pmb\Psi|\pmb\beta)p(\textbf{Z}|\pmb\eta)p(\textbf{W}|\pmb\Psi,\textbf{Z})p(\textbf{D}|\textbf{D}^*)p(\textbf{D}^*|\pmb\tau,\pmb\eta,\textbf{Z},\textbf{D})
\end{align}
}

