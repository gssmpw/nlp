\section{More Results on the SCOTUS cases on Privacy} 
\label{sec:si_privacy}

\subsection{Influence of the In-degree and Topic Similarity on the Probability of Citation}
% privacy
\begin{figure}[ht!]
     \centering
     \begin{subfigure}[b]{0.45\textwidth}
         \centering
         \includegraphics[width=\textwidth]{figs/oddsrat_box_privacy.pdf}
         \caption{Change in Log Odds Ratio by Additional Citations to Precedents}
         \label{reg_interpret_indegree_privacy}
     \end{subfigure}
     \hfill
     \begin{subfigure}[b]{0.45\textwidth}
         \centering
         \includegraphics[width=\textwidth]{figs/oddsrat_topic_privacy.pdf}
         \caption{Change in Log Odds Ratio by Increases in $\eta_{j,z_{ip}}$}
         \label{reg_interpret_topic_privacy}
     \end{subfigure}
     \caption{Changes in the log odds ratio of citation between a paragraph and a precedent as we increment the authority and the topic similarity of the given precedent. 10,000 random pairs of paragraphs and precedents were drawn from the data to create this figure. The left panel displays the distribution of improvements in log odds ratio if the given precedent had given additional citations. Each point is one of the 10,000 randomly drawn paragraph-precedent pairs. The right panel shows the improvements in log odds ratio if the given precedent were more topically similar to the given paragraph. The black points represent the average improvements in log odds ratio, and gray lines indicate the 2.5\% and 97.5\% quantile of log odds improvements respectively.}
     \label{reg_interpret_privacy}
\end{figure}

The $\pmb\tau$ coefficients in the latent citation propensity have expected signs. The average value of posterior samples for $\tau_1$ is 0.7 and the 95\% credible interval does not include 0, which suggests that the authority of documents has a positive impact on citation likelihood given topics. Similarly, posterior samples for $\tau_2$ stays above 0, suggesting that topic similarity between precedents and the citing paragraphs has a positive impact on citation decisions.

In Figure \ref{reg_interpret_privacy} we offer one way to interpret coefficients $\pmb\tau$ in latent citation propensity.\footnote{For more detailed information on the posterior samples of $\pmb\tau$, see Supplementary Information E.}
Since the latent citation propensity follows the structure of probit regression, one can employ the conventional approach to interpreting the coefficients where we calculate improvements in predicted probability as we increment one predictor while fixing other predictors at their means. This approach, however, presents two potential challenges. First, citation networks are usually sparse. Under our modeling framework, the sparse feature of citation networks is more emphasized as paragraphs are the unit that makes citations. The citation network for the Privacy subset contains only 452 citations when the fully connected network would have 243,685 citations. Partly due to such sparsity, improvements in predicted probability can be highly marginal. Second, the authority of a precedent, or the indegree, is known to follow the power-law distribution which is highly skewed to the right \citep{eom2011characterizing}. When a distribution is highly skewed, the mean is less likely to be the representative value of the distribution.

To address the above two challenges, we examine improvements in log odds ratio rather than predicted probability. Additionally, when incrementing one predictor we follow \cite{hanmer2013behind} and use observed values of other predictors rather than their means. To create Figure \ref{reg_interpret_privacy} we randomly sampled 10,000 paragraph-precedent pairs from the subset data and computed the extent of improvements in log odds ratio as we increased the authority and topic similarity of the given precedent. The left panel presents the improvements in log odds ratio when the authority of the given precedent is incremented. For example, if the given precedent had 3 more citations, the odds of the given paragraph citing the given precedent increases by about 20\%. Similarly, the right panel displays improvements in log odds ratio as the given precedent becomes more topically similar ($\eta_{j,z_{ip}}$) to the given paragraph. 


\subsection{MCMC Convergence Diagnostics}
% tau mcmc
\begin{figure}[t!]
     \centering
         \includegraphics[width=\textwidth]{figs/tau_mcmc_mar_privacy.pdf}
         \caption{MCMC convergence of $\pmb\tau$ posterior samples for the SCOTUS application on Privacy issue area. Horizontal red line indicates the true values of $\pmb\tau$.}
	 \label{privacy_tau_mcmc}
\end{figure}

% theta mcmc
\begin{figure}[t!]
     \centering
         \includegraphics[width=\textwidth]{figs/privacy_theta_mcmc_doc18.pdf}
         \caption{MCMC convergence of $\pmb\theta$ parameters for the 18th document in the subset of Privacy issue area. $\pmb\theta$ values are obtained by transforming the posterior samples of $\pmb\eta$ of the corresponding document. Horizontal red line indicates the true values of $\pmb\theta$ for the 18th document for each topic. We do not display the MCMC convergence for other documents, but all documents show similar level of convergence to the true value of $\pmb\theta$.}
	 \label{privacy_theta_mcmc}
\end{figure}

% tau mcmc
\begin{figure}[t!]
     \centering
         \includegraphics[width=\textwidth]{figs/tau_mcmc_mar_voting.pdf}
         \caption{MCMC convergence of $\pmb\tau$ posterior samples for the SCOTUS application on Voting Rights issue area. Horizontal red line indicates the true values of $\pmb\tau$.}
	 \label{voting_tau_mcmc}
\end{figure}

% theta mcmc
\begin{figure}[t!]
     \centering
         \includegraphics[width=\textwidth]{figs/voting_theta_mcmc_doc105.pdf}
         \caption{MCMC convergence of $\pmb\theta$ parameters for the 105th document in the subset of Voting Rights issue area. $\pmb\theta$ values are obtained by transforming the posterior samples of $\pmb\eta$ of the corresponding document. Horizontal red line indicates the true values of $\pmb\theta$ for the 105th document for each topic. We do not display the MCMC convergence for other documents, but all documents show similar level of convergence to the true value of $\pmb\theta$.}
	 \label{voting_theta_mcmc}
\end{figure}


