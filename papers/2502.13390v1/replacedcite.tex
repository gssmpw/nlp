\section{Prior Work}
\subsubsection{Massive Grant-Free Transmission in Cell-Free Wireless Communication Systems}
Recent results have focused on AUD, channel estimation (CE), and data detection (DD) for massive grant-free transmission in cell-free wireless communication systems ____.
%
Reference____ proposes two different AUD algorithms based on dominant APs and clustering, respectively. On this basis, a parallel AUD algorithm is developed to reduce complexity.
%
Reference____ proposes a covariance-based cooperative AUD method in which APs exchange their low-dimensional local information with neighbors.
%
Reference____ introduces a correlation-based AUD algorithm, accompanied by simulation results and empirical analysis, demonstrating that the cell-free system outperforms the collocated system in AUD performance.
%
Reference____ proposes centralized and distributed AUD algorithms for asynchronous transmission caused by low-cost oscillators.
%
For near-real-time transmission, reference____ introduces a deep-learning-based AUD algorithm, in which distributed computing units employ convolutional neural networks for preliminary AUD, and the CPU subsequently refines them through transfer learning.
%
Capitalizing on the a-priori distribution of channel coefficients, reference____ introduces a modified expectation-maximization approximate message passing (AMP) algorithm for CE, followed by AUD through the posterior support probabilities.
%
Reference____ proposes a two-stage AUD and CE method, in which AUD is first conducted via adjacent APs utilizing vector AMP, and CE is performed through a linear estimator.
%
Reference____ performs joint AUD and CE through a single-measurement-vector-based minimum mean square error (MMSE) estimation approach at each AP independently.
%
Considering both centralized and edge computing paradigms, reference____ presents an AMP-based approach for the joint AUD and CE while addressing quantization accuracy.
%
In millimeter-wave systems, reference____ introduces two distinct algorithms for joint AUD and CE, leveraging the inherent characteristic that each UE's channel predominantly comprises a few significant propagation paths.
%
Reference____ presents a joint AUD and DD (JAD) algorithm, employing an adaptive AP selection method based on local log-likelihood ratios.
%
Reference____ performs AUD, MMSE-based CE, and successive interference cancellation (SIC)-based data decoding under a probabilistic $K$-repitition scheme.
%
Reference____ first presents a joint AUD and CE approach for grant-free transmission using orthogonal time frequency space (OTFS) modulation in low Earth orbit (LEO) satellite communication systems. 
Subsequently, it introduces a least squares-based parallel time domain signal detection method.
%
Reference____ presents a Gaussian approximation-based Bayesian message passing algorithm for JACD, combined with an advanced low-coherence pilot design. 
%
Our previous work in____ introduced a DU-based JAD algorithm, in which all algorithm hyper-parameters are optimized using machine learning. 
%
In addition, our study in____ presents a box-constrained FBS algorithm designed for JACD.
%
Unlike previous methods, this paper tackles the task of JACD for massive grant-free transmission in cell-free systems. 
%
To improve JACD performance, we capture the sporadic UE activity more accurately by representing both the channel matrix and the data matrix as sparse matrices.


\subsubsection{Joint Active User Detection, Channel Estimation, and Data Detection for Single-Cell Massive Grant-Free Transmission}

JACD for single-cell massive grant-free transmission has been extensively investigated in____.
%
Considering low-precision data converters, reference____ utilizes bilinear generalized AMP (Bi-GAMP) with belief propagation algorithms for JACD in single-cell mMTC systems.
%
Reference____ proposes a bilinear message-scheduling generalized AMP for JACD, in which the channel decoder beliefs are used to refine AUD and DD.
%
Reference____ develops a Bi-GAMP algorithm for JACD,  capturing the row-sparse channel matrix structure stemming from channel correlations.
%
Reference____ divides the JACD scheme into slot-wise AUD and joint signal and channel estimation, which are addressed using message passing.
%
Reference____ combines AMP and belief propagation (BP) to perform JACD for asynchronous mMTC systems, in which the UEs transmit different lengths of data packets.
%
Reference____ introduces a turbo-structured receiver for JACD and data decoding, utilizing the channel decoder's information to improve CE and DD performance.
%
Reference____ introduces a JACD algorithm based on message passing and Markov random fields for LEO satellite-enabled mMTC scenarios, which employs OTFS modulation to capitalize on the sparsity in the delay-Doppler-angle domain.
%
In contrast to these message-passing-based JACD methods that have been designed for single-cell systems that primarily focus on UE activity sparsity, we consider the JACD problem in cell-free systems by taking into account two distinct sources of sparsity in the channel matrix: (i) column sparsity, which stems from the sporadic UE activity in mMTC scenarios, and (ii) block sparsity within each non-zero column, which stems from the vast discrepancies in large-scale channel fading between UEs and distributed APs in cell-free systems____. 
%
Furthermore, we propose our JACD methods within the FBS framework, incorporating efficient strategies to improve the JACD performance and reduce computational complexity.


\subsubsection{Deep-Unfolding for  Massive Grant-Free Transmission and Cell-Free Systems}
%
DU techniques____ have increasingly found application in the domain of massive grant-free transmission and cell-free systems____, which adeptly utilize backpropagation and stochastic gradient descent to automatically learn algorithm hyper-parameters. 
%
Reference____ employs DU to a linearized alternating direction method of multipliers and vector AMP, improving the performance of joint AUD and CE.
%
Reference____ applies DU to AMP-BP to improve JACD performance, fully exploiting the three-level sparsity inherent in UE activity, transmission delay, and data length diversity.
%
Reference____ introduces a model-driven sparse recovery algorithm to estimate the sparse channel matrix in mMTC scenarios, effectively utilizing the a-priori knowledge of partially known supports.
%
Reference____ unfolds AMP, tailored for JAD in mMTC under single-phase non-coherent schemes, wherein the embedded parameters are trained to mitigate the performance degradation caused by the non-ideal i.i.d. model.
%
Reference____ uses DU together with an iterative shrinkage thresholding algorithm for joint AUD and CE, wherein multiple computable matrices are treated as trainable parameters, thereby providing improving optimization flexibility.
%
Reference____ proposes a DU-based multi-user beamformer for cell-free systems, improving robustness to imperfect channel state information (CSI), where APs are equipped with fully digital or hybrid analog-digital arrays.
%
Reference____ unfolds a zero-forcing algorithm to achieve multi-user precoding, reducing complexity and improving the robustness under imperfect CSI.
%
In contrast to these results, we deploy DU to train all hyper-parameters in our proposed algorithms to improve the JACD performance and employ approximations for high-complexity steps to decrease computational complexity.
%
Furthermore, we train the hyper-parameters of our soft-output AUD module that generates information on UE activity probability.