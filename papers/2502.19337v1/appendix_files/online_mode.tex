
\section{PREDICTING POSTERIOR FOR UNSEEN DATA}
\label{sec:online_mode}

During inference, given an input set of $N$ data points, the forward transition $P_{F}[c_n|c_{1:n-1}, x_{1:N}]$ (as formulated in~\autoref{eq:gen1}) predicts the assignment for $x_n$ based on the previously predicted assignments $c_{1:n-1}$ and the complete set of data points $x_{1:N}$.
%
While this is the primary formulation, GFNCP has also an "online mode", in which it is capable of predicting assignments for previously unseen data points, enabling it to perform the forward transition $P_{F}[c_n|c_{1:n-1}, x_{1:n-1}]$. This is achieved by setting $U=0$ (see~\autoref{Q}), thus eliminating the encoding of yet unlabeled points.\\
%

To illustrate this functional modality, we performed several experiments where we trained and evaluated GFNCP in online mode (i.e., with $U=0$) on Mixture of Gaussians (MoG), MNIST, and ImageNet-50/100/200 (IN50/IN100/IN200) datasets. In all of these experiments, we followed the exact same training procedure as described in the paper. In~\autoref{tab:rebuttal_R2} we show the performance gap between the online mode and the primary mode (reproduced from ~\autoref{tab:clustering_stats_IN_50_100_200} in the paper). 
We note that the degradation in performance is negligible.

\subimport{../tables/}{rebuttal_using_U_0_R2}