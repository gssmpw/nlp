\section{Appendix}\label{sec:appendix}
\subsection{Bound for Precision and Recall in Pilot Sample} \label{subsec:s_p_in_appendix}
When the deviations between the proxy and oracle distances for different data points are i.i.d. random variables\cite{DujianPQA}, each data point selected as a nearest neighbor by the proxy model has an independent and approximately uniform probability of being a true nearest neighbor according to the oracle. Hence, we let each data \( x_i \in \text{SPRinT}_S \cap S_{p} \) be a Bernoulli random variable \(X_i\), defined by: 
\begin{equation}
X_i = 
\begin{cases} 
1 & \text{if } x_i \in NN, \\
0 & \text{otherwise.}
\end{cases}
\end{equation}
The precision in the pilot sample \( S_{p} \) can then be expressed as:
\begin{equation}
\text{P}_{S_{p}} = \frac{1}{|\text{SPRinT}_S \cap S_{p}|} \sum_{i=1}^{|\text{SPRinT}_S \cap S_{p}|} X_i.
\end{equation}
Similarly, the precision in the entire sample \( S \) is given by:
\begin{equation}
\text{P}_S = \frac{1}{|\text{SPRinT}_S|} \sum_{i=1}^{|\text{SPRinT}_S|} X_i.
\end{equation}

By Hoeffding's inequality, the probability that the precision in \( S_{p} \) deviates from that in \( S \) by more than \( \omega_{s_p} \) is bounded as follows:
\begin{equation}
\Pr\left[ |\text{P}_{S_{p}} - \text{P}_S| \geq \omega_{s_p} \right] \leq 2 \exp\left( \frac{-2 s_{p} \omega_{s_p}^2}{(b-a)^2} \right) = 2 \exp\left( -2 s_{p} \omega_{s_p}^2 \right),
\label{eq:Hoef}
\end{equation}
where \([a, b] = [0, 1]\) is the range of precision. With confidence level \((1-\alpha)\), the minimal \(s_p\) is given by
\begin{equation}
s_p \geq \frac{\ln(2/\alpha)}{2\omega_{s_p}}.
\end{equation}
The precision in \( S_{p} \) serves as a reliable approximation of the precision in \( S \), with the probability of large deviations decreasing exponentially as \( s_{p} \) grows. We can derive a similar bound for recall.
 
% let each data \( y_i \in NN \cap S_{p} \) be a Bernoulli random variable \(Y_i\), where 
% \[
% Y_i = 
% \begin{cases} 
% 1 & \text{if the data is in } NN_{\text{SPRinT}}, \\
% 0 & \text{otherwise.}
% \end{cases}
% \]
% Then we can express the recall in \( S_{p} \) as:
% \[
% \text{Recall}_{S_{p}} = \frac{1}{|NN \cap S_{p}|} \sum_{i=1}^{|NN \cap S_{p}|} Y_i
% \]
% Similarly, the recall in \( S \) is:
% \[
% \text{Recall}_S = \frac{1}{|NN|} \sum_{i=1}^{|NN|} Y_i
% \]


If the deviations between the proxy and oracle distances for different data points are not i.i.d. random variables, the Bernoulli random variable assumption for each data point in \(\text{SPRinT}_S\cap S_p\) no longer holds. Instead, we use Chebyshev's inequality \cite{ibe2013markov}, which only requires a finite variance, to provide a looser bound on the probability of deviation between precision in \(S_p\) and \(S\), as follows:
\begin{equation}
\Pr\left(|\text{P}_S - \text{P}_{S_p}|\geq \epsilon \right) \leq \frac{\text{Var}(\text{P}_{S_p})}{\epsilon^2},
\label{eq:Cheb}
\end{equation}
where \(\text{Var}(\text{P}_{S_p})\) is the empirical variance of precision in \(S_p\).

Although Chebyshev’s inequality provides a less tight bound than Hoeffding’s inequality, it allows us to control the deviations in precision and recall as \( s_p \) grows. By increasing \( s_p \), the pilot sample better captures the distributional deviations in \( S \), making precision and recall in \( S_p \) reliable estimates for those in \( S \). 