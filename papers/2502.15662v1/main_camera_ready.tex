%%%%%%%%%%%%%%%%%%%%%%%%%%%%%%%%%%%%%%%%%%%%%%%%%%%%%%%%%%%%%%%%%%%%%%%%

%%% LaTeX Template for AAMAS-2025 (based on sample-sigconf.tex)
%%% Prepared by the AAMAS-2025 Program Chairs based on the version from AAMAS-2025. 

%%%%%%%%%%%%%%%%%%%%%%%%%%%%%%%%%%%%%%%%%%%%%%%%%%%%%%%%%%%%%%%%%%%%%%%%

%%% Start your document with the \documentclass command.


%%% == IMPORTANT ==
%%% Use the first variant below for the final paper (including auithor information).
%%% Use the second variant below to anonymize your submission (no authoir information shown).
%%% For further information on anonymity and double-blind reviewing, 
%%% please consult the call for paper information
%%% https://aamas2025.org/index.php/conference/calls/submission-instructions-main-technical-track/

%%%% For anonymized submission, use this
%\documentclass[sigconf,anonymous]{aamas} 

%%%% For camera-ready, use this
\documentclass[sigconf]{aamas} 


%%% Load required packages here (note that many are included already).

\usepackage{balance} % for balancing columns on the final page
\usepackage[ruled,noend]{algorithm2e}

\newcommand{\R}{{\mathbb R}}
%\newcommand{\C}{{\mathbb C}}
\newcommand{\Z}{{\mathbb Z}}
\newcommand{\N}{{\mathbb N}}

\usepackage{fancyhdr}
\pagestyle{fancy}
\lhead{}
\rhead{}
\chead{}
\lfoot{}
\rfoot{}
\cfoot{\thepage\\
 \begin{small} \textbf{DISTRIBUTION STATEMENT A.} Approved for public release; distribution is unlimited. \end{small}}
\renewcommand{\headrulewidth}{0pt}
\renewcommand{\footrulewidth}{0pt}



%
\setlength\unitlength{1mm}
\newcommand{\twodots}{\mathinner {\ldotp \ldotp}}
% bb font symbols
\newcommand{\Rho}{\mathrm{P}}
\newcommand{\Tau}{\mathrm{T}}

\newfont{\bbb}{msbm10 scaled 700}
\newcommand{\CCC}{\mbox{\bbb C}}

\newfont{\bb}{msbm10 scaled 1100}
\newcommand{\CC}{\mbox{\bb C}}
\newcommand{\PP}{\mbox{\bb P}}
\newcommand{\RR}{\mbox{\bb R}}
\newcommand{\QQ}{\mbox{\bb Q}}
\newcommand{\ZZ}{\mbox{\bb Z}}
\newcommand{\FF}{\mbox{\bb F}}
\newcommand{\GG}{\mbox{\bb G}}
\newcommand{\EE}{\mbox{\bb E}}
\newcommand{\NN}{\mbox{\bb N}}
\newcommand{\KK}{\mbox{\bb K}}
\newcommand{\HH}{\mbox{\bb H}}
\newcommand{\SSS}{\mbox{\bb S}}
\newcommand{\UU}{\mbox{\bb U}}
\newcommand{\VV}{\mbox{\bb V}}


\newcommand{\yy}{\mathbbm{y}}
\newcommand{\xx}{\mathbbm{x}}
\newcommand{\zz}{\mathbbm{z}}
\newcommand{\sss}{\mathbbm{s}}
\newcommand{\rr}{\mathbbm{r}}
\newcommand{\pp}{\mathbbm{p}}
\newcommand{\qq}{\mathbbm{q}}
\newcommand{\ww}{\mathbbm{w}}
\newcommand{\hh}{\mathbbm{h}}
\newcommand{\vvv}{\mathbbm{v}}

% Vectors

\newcommand{\av}{{\bf a}}
\newcommand{\bv}{{\bf b}}
\newcommand{\cv}{{\bf c}}
\newcommand{\dv}{{\bf d}}
\newcommand{\ev}{{\bf e}}
\newcommand{\fv}{{\bf f}}
\newcommand{\gv}{{\bf g}}
\newcommand{\hv}{{\bf h}}
\newcommand{\iv}{{\bf i}}
\newcommand{\jv}{{\bf j}}
\newcommand{\kv}{{\bf k}}
\newcommand{\lv}{{\bf l}}
\newcommand{\mv}{{\bf m}}
\newcommand{\nv}{{\bf n}}
\newcommand{\ov}{{\bf o}}
\newcommand{\pv}{{\bf p}}
\newcommand{\qv}{{\bf q}}
\newcommand{\rv}{{\bf r}}
\newcommand{\sv}{{\bf s}}
\newcommand{\tv}{{\bf t}}
\newcommand{\uv}{{\bf u}}
\newcommand{\wv}{{\bf w}}
\newcommand{\vv}{{\bf v}}
\newcommand{\xv}{{\bf x}}
\newcommand{\yv}{{\bf y}}
\newcommand{\zv}{{\bf z}}
\newcommand{\zerov}{{\bf 0}}
\newcommand{\onev}{{\bf 1}}

% Matrices

\newcommand{\Am}{{\bf A}}
\newcommand{\Bm}{{\bf B}}
\newcommand{\Cm}{{\bf C}}
\newcommand{\Dm}{{\bf D}}
\newcommand{\Em}{{\bf E}}
\newcommand{\Fm}{{\bf F}}
\newcommand{\Gm}{{\bf G}}
\newcommand{\Hm}{{\bf H}}
\newcommand{\Id}{{\bf I}}
\newcommand{\Jm}{{\bf J}}
\newcommand{\Km}{{\bf K}}
\newcommand{\Lm}{{\bf L}}
\newcommand{\Mm}{{\bf M}}
\newcommand{\Nm}{{\bf N}}
\newcommand{\Om}{{\bf O}}
\newcommand{\Pm}{{\bf P}}
\newcommand{\Qm}{{\bf Q}}
\newcommand{\Rm}{{\bf R}}
\newcommand{\Sm}{{\bf S}}
\newcommand{\Tm}{{\bf T}}
\newcommand{\Um}{{\bf U}}
\newcommand{\Wm}{{\bf W}}
\newcommand{\Vm}{{\bf V}}
\newcommand{\Xm}{{\bf X}}
\newcommand{\Ym}{{\bf Y}}
\newcommand{\Zm}{{\bf Z}}

% Calligraphic

\newcommand{\Ac}{{\cal A}}
\newcommand{\Bc}{{\cal B}}
\newcommand{\Cc}{{\cal C}}
\newcommand{\Dc}{{\cal D}}
\newcommand{\Ec}{{\cal E}}
\newcommand{\Fc}{{\cal F}}
\newcommand{\Gc}{{\cal G}}
\newcommand{\Hc}{{\cal H}}
\newcommand{\Ic}{{\cal I}}
\newcommand{\Jc}{{\cal J}}
\newcommand{\Kc}{{\cal K}}
\newcommand{\Lc}{{\cal L}}
\newcommand{\Mc}{{\cal M}}
\newcommand{\Nc}{{\cal N}}
\newcommand{\nc}{{\cal n}}
\newcommand{\Oc}{{\cal O}}
\newcommand{\Pc}{{\cal P}}
\newcommand{\Qc}{{\cal Q}}
\newcommand{\Rc}{{\cal R}}
\newcommand{\Sc}{{\cal S}}
\newcommand{\Tc}{{\cal T}}
\newcommand{\Uc}{{\cal U}}
\newcommand{\Wc}{{\cal W}}
\newcommand{\Vc}{{\cal V}}
\newcommand{\Xc}{{\cal X}}
\newcommand{\Yc}{{\cal Y}}
\newcommand{\Zc}{{\cal Z}}

% Bold greek letters

\newcommand{\alphav}{\hbox{\boldmath$\alpha$}}
\newcommand{\betav}{\hbox{\boldmath$\beta$}}
\newcommand{\gammav}{\hbox{\boldmath$\gamma$}}
\newcommand{\deltav}{\hbox{\boldmath$\delta$}}
\newcommand{\etav}{\hbox{\boldmath$\eta$}}
\newcommand{\lambdav}{\hbox{\boldmath$\lambda$}}
\newcommand{\epsilonv}{\hbox{\boldmath$\epsilon$}}
\newcommand{\nuv}{\hbox{\boldmath$\nu$}}
\newcommand{\muv}{\hbox{\boldmath$\mu$}}
\newcommand{\zetav}{\hbox{\boldmath$\zeta$}}
\newcommand{\phiv}{\hbox{\boldmath$\phi$}}
\newcommand{\psiv}{\hbox{\boldmath$\psi$}}
\newcommand{\thetav}{\hbox{\boldmath$\theta$}}
\newcommand{\tauv}{\hbox{\boldmath$\tau$}}
\newcommand{\omegav}{\hbox{\boldmath$\omega$}}
\newcommand{\xiv}{\hbox{\boldmath$\xi$}}
\newcommand{\sigmav}{\hbox{\boldmath$\sigma$}}
\newcommand{\piv}{\hbox{\boldmath$\pi$}}
\newcommand{\rhov}{\hbox{\boldmath$\rho$}}
\newcommand{\upsilonv}{\hbox{\boldmath$\upsilon$}}

\newcommand{\Gammam}{\hbox{\boldmath$\Gamma$}}
\newcommand{\Lambdam}{\hbox{\boldmath$\Lambda$}}
\newcommand{\Deltam}{\hbox{\boldmath$\Delta$}}
\newcommand{\Sigmam}{\hbox{\boldmath$\Sigma$}}
\newcommand{\Phim}{\hbox{\boldmath$\Phi$}}
\newcommand{\Pim}{\hbox{\boldmath$\Pi$}}
\newcommand{\Psim}{\hbox{\boldmath$\Psi$}}
\newcommand{\Thetam}{\hbox{\boldmath$\Theta$}}
\newcommand{\Omegam}{\hbox{\boldmath$\Omega$}}
\newcommand{\Xim}{\hbox{\boldmath$\Xi$}}


% Sans Serif small case

\newcommand{\Gsf}{{\sf G}}

\newcommand{\asf}{{\sf a}}
\newcommand{\bsf}{{\sf b}}
\newcommand{\csf}{{\sf c}}
\newcommand{\dsf}{{\sf d}}
\newcommand{\esf}{{\sf e}}
\newcommand{\fsf}{{\sf f}}
\newcommand{\gsf}{{\sf g}}
\newcommand{\hsf}{{\sf h}}
\newcommand{\isf}{{\sf i}}
\newcommand{\jsf}{{\sf j}}
\newcommand{\ksf}{{\sf k}}
\newcommand{\lsf}{{\sf l}}
\newcommand{\msf}{{\sf m}}
\newcommand{\nsf}{{\sf n}}
\newcommand{\osf}{{\sf o}}
\newcommand{\psf}{{\sf p}}
\newcommand{\qsf}{{\sf q}}
\newcommand{\rsf}{{\sf r}}
\newcommand{\ssf}{{\sf s}}
\newcommand{\tsf}{{\sf t}}
\newcommand{\usf}{{\sf u}}
\newcommand{\wsf}{{\sf w}}
\newcommand{\vsf}{{\sf v}}
\newcommand{\xsf}{{\sf x}}
\newcommand{\ysf}{{\sf y}}
\newcommand{\zsf}{{\sf z}}


% mixed symbols

\newcommand{\sinc}{{\hbox{sinc}}}
\newcommand{\diag}{{\hbox{diag}}}
\renewcommand{\det}{{\hbox{det}}}
\newcommand{\trace}{{\hbox{tr}}}
\newcommand{\sign}{{\hbox{sign}}}
\renewcommand{\arg}{{\hbox{arg}}}
\newcommand{\var}{{\hbox{var}}}
\newcommand{\cov}{{\hbox{cov}}}
\newcommand{\Ei}{{\rm E}_{\rm i}}
\renewcommand{\Re}{{\rm Re}}
\renewcommand{\Im}{{\rm Im}}
\newcommand{\eqdef}{\stackrel{\Delta}{=}}
\newcommand{\defines}{{\,\,\stackrel{\scriptscriptstyle \bigtriangleup}{=}\,\,}}
\newcommand{\<}{\left\langle}
\renewcommand{\>}{\right\rangle}
\newcommand{\herm}{{\sf H}}
\newcommand{\trasp}{{\sf T}}
\newcommand{\transp}{{\sf T}}
\renewcommand{\vec}{{\rm vec}}
\newcommand{\Psf}{{\sf P}}
\newcommand{\SINR}{{\sf SINR}}
\newcommand{\SNR}{{\sf SNR}}
\newcommand{\MMSE}{{\sf MMSE}}
\newcommand{\REF}{{\RED [REF]}}

% Markov chain
\usepackage{stmaryrd} % for \mkv 
\newcommand{\mkv}{-\!\!\!\!\minuso\!\!\!\!-}

% Colors

\newcommand{\RED}{\color[rgb]{1.00,0.10,0.10}}
\newcommand{\BLUE}{\color[rgb]{0,0,0.90}}
\newcommand{\GREEN}{\color[rgb]{0,0.80,0.20}}

%%%%%%%%%%%%%%%%%%%%%%%%%%%%%%%%%%%%%%%%%%
\usepackage{hyperref}
\hypersetup{
    bookmarks=true,         % show bookmarks bar?
    unicode=false,          % non-Latin characters in AcrobatÕs bookmarks
    pdftoolbar=true,        % show AcrobatÕs toolbar?
    pdfmenubar=true,        % show AcrobatÕs menu?
    pdffitwindow=false,     % window fit to page when opened
    pdfstartview={FitH},    % fits the width of the page to the window
%    pdftitle={My title},    % title
%    pdfauthor={Author},     % author
%    pdfsubject={Subject},   % subject of the document
%    pdfcreator={Creator},   % creator of the document
%    pdfproducer={Producer}, % producer of the document
%    pdfkeywords={keyword1} {key2} {key3}, % list of keywords
    pdfnewwindow=true,      % links in new window
    colorlinks=true,       % false: boxed links; true: colored links
    linkcolor=red,          % color of internal links (change box color with linkbordercolor)
    citecolor=green,        % color of links to bibliography
    filecolor=blue,      % color of file links
    urlcolor=blue           % color of external links
}
%%%%%%%%%%%%%%%%%%%%%%%%%%%%%%%%%%%%%%%%%%%


\usepackage{enumitem}
%%%%%%%%%%%%%%%%%%%%%%%%%%%%%%%%%%%%%%%%%%%%%%%%%%%%%%%%%%%%%%%%%%%%%%%%

%%% AAMAS-2025 copyright block (do not change!)

%%% AAMAS-2025 copyright block (do not change!)

\makeatletter
\gdef\@copyrightpermission{
  \begin{minipage}{0.2\columnwidth}
   \href{https://creativecommons.org/licenses/by/4.0/}{\includegraphics[width=0.90\textwidth]{by}}
  \end{minipage}\hfill
  \begin{minipage}{0.8\columnwidth}
   \href{https://creativecommons.org/licenses/by/4.0/}{This work is licensed under a Creative Commons Attribution International 4.0 License.}
  \end{minipage}
  \vspace{5pt}
}
\makeatother

\setcopyright{ifaamas}
\acmConference[AAMAS '25]{Proc.\@ of the 24th International Conference
on Autonomous Agents and Multiagent Systems (AAMAS 2025)}{May 19 -- 23, 2025}
{Detroit, Michigan, USA}{Y.~Vorobeychik, S.~Das, A.~Nowé  (eds.)}
\copyrightyear{2025}
\acmYear{2025}
\acmDOI{}
\acmPrice{}
\acmISBN{}


%%%%%%%%%%%%%%%%%%%%%%%%%%%%%%%%%%%%%%%%%%%%%%%%%%%%%%%%%%%%%%%%%%%%%%%%

%%% == IMPORTANT ==
%%% Use this command to specify your EasyChair submission number.
%%% In anonymous mode, it will be printed on the first page.

\acmSubmissionID{950}

%%% Use this command to specify the title of your paper.

\title[AAMAS-2025 Formatting Instructions]{
%A Bayesian Curriculum for Online Task Transfer \\
%Automated Curriculum Generation for Reinforcement Learning \\
%Automated Bayesian Curriculum Generation\\
%A Bayesian Approach for Automated Curricula\\
%A Self-Guided Bayesian Curriculum for Reinforcement Learning\\
%An Automated Bayesian Curriculum for Reinforcement Learning\\
Automating Curriculum Learning for Reinforcement Learning using a Skill-Based Bayesian Network
% \\ commented-out to fix a compilation error -- Dana Nau
}
%https://aamas2025.org/index.php/conference/calls/submission-instructions-main-technical-track/

%%% Provide names, affiliations, and email addresses for all authors.

\author{Vincent Hsiao}
\affiliation{
  \institution{NRC Postdoctoral Fellow, \\Naval Research Laboratory}
  \city{Washington DC}
  \country{United States}}
\email{vincent.hsiao.ctr@us.navy.mil}

\author{Mark Roberts}
\affiliation{
  \institution{Naval Research Laboratory}
  \city{Washington DC}
  \country{United States}}
\email{mark.c.roberts20.civ@us.navy.mil}

\author{Laura M. Hiatt}
\affiliation{
  \institution{Naval Research Laboratory}
  \city{Washington DC}
  \country{United States}}
\email{laura.m.hiatt.civ@us.navy.mil}

\author{George Konidaris}
\affiliation{
    \institution{Brown University}
    \city{Providence RI}
    \country{United States}}
\email{gdk@brown.edu}

\author{Dana Nau}
\affiliation{
    \institution{University of Maryland}
    \city{College Park, MD}
    \country{United States}}
\email{nau@umd.edu}



%%% Use this environment to specify a short abstract for your paper.
    
\begin{abstract}
%typically focuses on improving the performance on some target task by training on a sequence of similar tasks.
A major challenge for reinforcement learning is automatically generating curricula to reduce training time or improve performance in some target task.
We introduce SEBNs (Skill-Environment Bayesian Networks)
which model a probabilistic relationship between a set of skills, a set of goals that relate to the reward structure, and a set of environment features to predict policy performance on (possibly unseen) tasks.  
We develop an algorithm that uses the inferred estimates of agent success from SEBN to weigh the possible next tasks by expected improvement.
We evaluate the benefit of the resulting curriculum on three environments: a discrete gridworld, continuous control, and simulated robotics.
The results show that curricula constructed using SEBN frequently outperform other baselines.
\end{abstract}

%%% The code below was generated by the tool at http://dl.acm.org/ccs.cfm.
%%% Please replace this example with code appropriate for your own paper.


%%% Use this command to specify a few keywords describing your work.
%%% Keywords should be separated by commas.

\keywords{Bayesian Networks, 
Automated Curriculum Generation,
Reinforcement Learning,
Transfer Learning}

%%%%%%%%%%%%%%%%%%%%%%%%%%%%%%%%%%%%%%%%%%%%%%%%%%%%%%%%%%%%%%%%%%%%%%%%

%%% Include any author-defined commands here.
         
\newcommand{\BibTeX}{\rm B\kern-.05em{\sc i\kern-.025em b}\kern-.08em\TeX}

%%%%%%%%%%%%%%%%%%%%%%%%%%%%%%%%%%%%%%%%%%%%%%%%%%%%%%%%%%%%%%%%%%%%%%%%

\begin{document}

%%% The following commands remove the headers in your paper. For final 
%%% papers, these will be inserted during the pagination process.

\pagestyle{fancy}
\fancyhead{}


%%% The next command prints the information defined in the preamble.

\maketitle 
%%%%%%%%%%%%%%%%%%%%%%%%%%%%%%%%%%%%%%%%%%%%%%%%%%%%%%%%%%%%%%%%%%%%%%%%

%\frommak{ {\color{red} \begin{small} 8-pages + references + supplemental;     Abstract:10/9 AOE  Paper:10/16 AOE \end{small} }}

\fancypagestyle{firststyle}
{
\cfoot{ \begin{small} \textbf{DISTRIBUTION STATEMENT A.} Approved for public release; distribution is unlimited. \end{small}}
}
\thispagestyle{firststyle}


%=====================================================
%=====================================================
%=====================================================
%=====================================================
\section{Introduction}
Adapting skills to new or unseen tasks is a major challenge in Reinforcement Learning (RL). Curriculum Learning \cite{bengio2009curriculum, narvekarEtAl20.jmlr.clForRL}, an approach for training agents using a sequence of increasingly difficult environments, often promotes the effective development of policies with more robust capabilities. However, customizing a curriculum to a particular student often requires substantial human insight and oversight. This is especially challenging for robotics, where the environment or tasks that need to be performed can change frequently. An ideal solution to this problem would be an automated curriculum that enables the robot to discern for itself when it needs to adapt, how long should train, and in what environments. 

Past work on automated curriculum generation such as \cite{stout2010competence, kumar2024practice} has primarily focused on choosing what skills to train while holding the environment itself static. More recent approaches that build a curriculum over different environments such as \cite{parker2022evolving} do not consider agent skill competencies. Furthermore, these environment-based approaches require explicit evaluation on an environment before being able to calculate an estimate of agent success or regret to add those environments to the curriculum. 

We address the aforementioned issues by introducing Skill Environment Bayesian Networks (SEBNs) as a potential method for estimating agent competency level and selecting the most appropriate environments for training. 
%SEBNs provide a natural way to incorporate options. \fromvincent{still thinking of a way to write this, maybe this doesn't need to be mentioned in the introduction...}
SEBNs model a probabilistic relationship between these goals, (latent) competencies, and environment features using data from past rollouts. Using this model, we can estimate agent success rates on new (possibly unseen) environments. We use these estimates to select the next set of training tasks within a curriculum in what we call an SEBN-guided automated curriculum. Importantly, SEBN does not require explicit evaluation on each possible environment to estimate agent success. 


The contributions of the paper include:
    (1) Introducing and formalizing the SEBN for skills, task features, and reward structure;
    (2) Providing an algorithm for constructing curricula using SEBNs; 
    (3) Introducing  Megagrid, a gridworld environment that simplifies generating partially-specified environments for transfer learning; 
    (4) Assessing SEBN-based curricula on three distinct environments: a discrete gridworld (DoorKey), continuous control (BipedalWalker), and a difficult simulated robotics domain (robosuite); and
    (5) Demonstrating via experiments that SEBN curricula produce more robust policies that reach success more quickly than other curricula in the continuous control and robotics environments, and performed comparably in the gridworld environment. 


%=====================================================
%=====================================================
%=====================================================
%=====================================================
\section{Background and Preliminaries}

%The SEBN is a data structure that estimates competency level in an environment and uses those estimates to predict expected future reward.  
Bayesian statistics rely on some sort of informed prior, provided or learned, to estimate the future values.  
In an SEBN, part of this prior is provided in the form of the network and in the strength of relationships, and part of the prior is learned through the collection of samples from the environment.
We next provide our motivation for this Bayesian approach (\secref{sec:ecd})
with some background on Bayesian Networks (\secref{sec:bns}).  
We then formalize the curriculum learning problem (\secref{sec:curriculum-learning}), how we use task features to construct tasks (\secref{sec:taskdescriptors}), and an extension to the options framework (\secref{sec:options}).



%-------------------------------------------------------
%-------------------------------------------------------
\subsection{Evidence Centered Design}
\label{sec:ecd}

Our motivation for using a Bayesian Network to estimate learning proficiency comes from the method of Evidence Centered Design (ECD) \cite{mislevy2003brief}, a technique used in human educational assessments. 
In Evidence-Centered Design, statistical models, such as Bayesian networks, are used to measure the proficiency levels of a given student. These proficiency measurements are then used to inform task and assessment creation. For example, ECD could be used to help model and analyze the performance of a tennis player. The Bayesian network in this domain can include nodes that represent latent competencies (e.g., mobility, footwork, dynamic vision, etc.) and nodes that represent observable metrics (e.g., number of successful serves, return rate, game score). The performance of a tennis player on the observable metrics is used to infer their latent capabilities. New training goals can then be set using these estimated capabilities. This technique is effective in human educational contexts, and we hypothesize that a similar approach could be applied to assist in designing a curriculum to improve learning in robotic agents. 

%-------------------------------------------------------
%-------------------------------------------------------
\subsection{Bayesian Networks (BNs)}
\label{sec:bns}

Bayesian Networks (BNs) \citep{pearl} are a type of graphical model that provide an efficient way to represent and reason about probabilistic relationships among a set of random variables. A Bayesian Network $(X,D,\parentfunctions)$ is defined by a set of variables $X$, their corresponding domains $D$, and a set of parent functions \parentfunctions that specify the conditional probability distributions of each variable given its parents. 
When $D$ is discrete, these parent functions are typically specified in a tabular format known as Conditional Probability Tables (CPTs). %Bayesian Networks model complex probabilistic distributions through compact representation of conditional independence relationships among the variables. 

It is common to use BNs to model relationships between latent and observable variables. 
Once constructed, the network can be used to infer latent values from observed values. 
New data can be entered in the form of evidence values for observed variables in a BN. 
The probabilities over other variables in the network are calculated by conditioning on this observed evidence, i.e., as a conditional probability: $P(X_1|X_2,...,X_N)$. 
We will employ a standard bucket elimination algorithm (aka variable elimination) \cite{darwiche2009modeling, dechter2013reasoning} to perform inference.
In this paper, the observable variables of the BN relate to the environment of an agent and a target performance it is attempting to achieve; both are modeled as a task in an MDP problem and defined in \secref{sec:curriculum-learning}.  The unobservable variables will relate to a set of latent competencies that we define in \secref{sec:latent-skills}. 

%-------------------------------------------------------
%-------------------------------------------------------
\subsection{Curriculum Learning}
\label{sec:curriculum-learning}

We adapt the notation of 
\citeauthor{narvekarEtAl20.jmlr.clForRL}~\cite{narvekarEtAl20.jmlr.clForRL} to describe a \emph{task} as the interaction of an agent with its environment to meet some objective.  
A task, formalized as an episodic Markov Decision Process (MDP), is a tuple 
$\mdpmaster = (\mdpstates, \mdpactions, \mdptransition, \mdpreward{})$, where 
    \mdpstates{} 
    is the set of states, 
    \mdpactions~is the set of actions, 
    $\mdptransition(\mdpstate'|\mdpstate, \mdpaction)$ gives the probability of being in state $\mdpstate'$ after taking action \mdpaction in state \mdpstate, and 
    $\mdpreward{}(\mdpstate, \mdpaction, \mdpstate')$ is the reward function after taking action \mdpaction in state \mdpstate and transitioning to state $\mdpstate'$.
A solution to \mdpmaster is a policy $\pi$ that maximizes the cumulative sum of rewards for an episode of length T, i.e., $\sum_{t=1}^{T} R_{t}$.

Let \tasks be a set of all tasks an agent could complete in \mdpmaster, where a task $\taski \in \tasks$ is a task-specific MDP $\taski = (\mdpstatesi, \mdpactionsi, \mdptransitioni, \mdprewardi)$.  For all tasks in \tasks, let \transitiondomain be the set of all possible transition samples from \tasks (see \citeauthor{narvekarEtAl20.jmlr.clForRL} \cite{narvekarEtAl20.jmlr.clForRL} for a complete definition).  
In their formalism, a Curriculum $\curriculum = (\currvertices, \curredges, \taskdescriptor, \tasks)$ 
is a directed acyclic graph, where \currvertices is the set of vertices, $\curredges \subseteq \{(x,y) | (x,y) \in \currvertices \times \currvertices \land x \neq y\}$ is a set of directed edges,  $\taskdescriptor: \currvertices \rightarrow \mathcal{P}(\transitiondomain)$ is a function that associates samples within each vertex, and $\mathcal{P}(\transitiondomain)$ is the powerset of \transitiondomain.

In this paper, we develop what Narvekar~\etal~\cite{narvekarEtAl20.jmlr.clForRL} call a task-level curriculum, where each vertex $v \in \currvertices$ is associated with samples from a single task in \tasks.  
That is, the mapping function for task \taski is 
$\taskdescriptor: \currvertices \rightarrow \{ \transitiondomain_i | \taski \in \tasks \}$.  
For convenience, we will refer to a task's available samples at vertex $v$ as $\taski^{\taskdescriptor}$.
In other words, a task descriptor $\taskdescriptor_i$ is used to construct task \taski, and a curriculum is a sequence of tasks \task{1}, \task{2}, ..., \task{target} up to some target task.


Before we describe how we construct this function using task features, we point out some deviations from the model just described.  
The curriculum being a DAG is a very strong assumption and is not true of the SEBN-guided curriculum.  
While the episodic MDP model of \citeauthor{narvekarEtAl20.jmlr.clForRL} provides a more comprehensible model of curriculum learning, the RL algorithms of this paper actually learn with a discount factor \discount, and one could argue that the Partially Observable MDP might be more appropriate.  Both changes would be extensions to the simplified MDP model presented here. 
%\fromlaura{clarify that this theta is the same as our task descriptor? And/or relate it to section 2.4?}
%\fromlaura{I got a little twisted around here by the significance on a discount factor (which seems standard) and the mention of POMDPs. I edited but with tracking changes on feel free to revert.} 
%\frommak{Narvekar's model uses episodic MDPs, which terminate rather than continue indefinitely.  So the discount factor isn't needed for these.  Similarly, we are really learning over POMDPs for the robot.  I was just trying to clarify that we recognize the limitations of this abstract model for getting the main points across.}

%\fromlaura{perhaps my earlier edits weren't appropriate -- this paragraph seems to be where it says what we do differently than Narvekar. I suggest moving the footnote here, too -- to emphasize that we remove the strong assumption of the DAG.}

%-------------------------------------------------------
%-------------------------------------------------------
\subsection{Task Descriptors (Env't+Target Features)}
\label{sec:taskdescriptors}
We will use \emph{task features} to define \taskdescriptor for a task $\taski^{\taskdescriptor}$.
This is a common approach to quantify potential transfer between two tasks (e.g., \cite{rostamiEtAl20.jair.usingTaskDescriptions,isele2016task,narvekar2016source,konidaris2012transfer}).
The notion is that two tasks that share similar features will exhibit better transfer.
We adopt a task descriptor similar to \citeauthor{rostamiEtAl20.jair.usingTaskDescriptions}~\citep{rostamiEtAl20.jair.usingTaskDescriptions} and \citeauthor{narvekar2016source}~\citep{narvekar2016source}.

Specifically, we parameterize \taskdescriptor with a vector that 
consists of set of environment-specific features $E$ and a set of one or more performance targets $\Target$.
Thus, $\taskdescriptor((\mathbb{Z}_0)^{|E|}(\mathbb{Z}_0)^{|K|})$ will indicate the specific task $\taski^{\taskdescriptor}$.
We will often omit the task descriptor for clarity and just reference \taski.  Note that the task descriptor is underspecified with respect to the environment, so one configuration of \taskdescriptor represents a class of different environments an agent can encounter. 


\paragraph{Example Task Descriptors using Bipedal Walker.} 
The Bipedal Walker (BPW) benchmark \cite{towers_gymnasium_2023} involves two-legged agent moving through terrain in a 2D environment.  Figure~\ref{fig:bipedalwalkerenv} shows some example terrains.
The top portion Figure~\ref{fig:sebn-bipedal-walker} shows $E$ and \Target for the bipedal walker. 
Here, $E$ consists of five features that control the difficulty of the environment, and there is a single target of moving to the right by at least 30 steps (The dashed latent competencies are defined in \secref{sec:latent-skills}). 
The task descriptor for this BPW is $\taskdescriptor($\mbox{\scriptsize{pit-gap, stump-height, stair-width, stair-steps, roughness, moved>30}}$)$. A task $\taski^{\taskdescriptor}$ for BPW involves a particular setting of the parameters for \taskdescriptor. \exampledone


%-------------------------------------------------------
%-------------------------------------------------------
\subsection{Extending Targets to Include Options}
\label{sec:options}

\noindent
One could imagine a richer set of targets in $K$, even ones that are hierarchically organized with a natural decomposition of subtasks.
The options framework \cite{suttonEtAl99.aij.betweenMDPsAndSemiMDPs}
is a common model for such situations.
Briefly, a subtask $j$ for task \taski is an option
$o_j = ( \mdpstates{j}, \pi_j, \oterm_j )$, 
where $\mdpstates{j} \subseteq \mdpstatesi$ are the starting states of the option, $\pi_j$ is used to take action while the option is enabled, and $\oterm_j: \mdpstatesi \rightarrow \zeroone$ is a function that indicates the option has terminated.

For a specific task \taski, a subtask  $\task{i}^j = (\mdpstates{j}, \mdpactionsi, \mdptransitioni, \mdpreward{j})$ indicates that an option has a specific context: it works over a set of states \mdpstates{j} that are a subset of the tasks states \mdpstatesi{}, it uses a specific reward independent of the task reward, and it has the same actions and transitions as the original task \taski.
Each option $j$ is enabled as part of the feature parameters for \taskdescriptor (i.e., $(\mathbb{Z}_0)^{|K|})$). 

%Two of the environments in this paper have a single ``option" (BipedalWalker and Robotsuite Door), which means the system is learning a single policy for the entire problem.  For this simplified situation, \task{} describes the overall task with reward \mdpreward{} and $\task{i}^{\taskdescriptor}$ describes the $i$th task within a curriculum, which is constructed using task descriptor \taskdescriptor.  

\begin{figure}
\begin{scriptsize}
\begin{tabular}{ll}
    %trim(left bottom right top)
    \includegraphics[trim={0 16cm 0 0}, clip, width=0.20\textwidth]{figures/bipedalwalkerenv.png} &     
    \includegraphics[trim={0 4.5cm 0 11cm}, clip, width=0.20\textwidth]{figures/bipedalwalkerenv.png}  \\
    (P3 S0 N0 W0 R0) & 
    (P0 S0 N3 W3 R0)\\
    \includegraphics[trim={0 10cm 0 6.3cm}, clip, width=0.20\textwidth]{figures/bipedalwalkerenv.png} & 
    \includegraphics[trim={0 0 0 16cm}, clip, width=0.20\textwidth]{figures/bipedalwalkerenv.png}\\
    (P0 S1 N0 W0 R0) & 
    (P0 S0 N0 W0 R4)\\
\end{tabular}
\end{scriptsize}
    \caption{Challenge environments for BipedalWalker  with corresponding descriptors (P:pit gap, S:stump height, W:stair width, N:stair steps, and R:ground roughness).  }

    \label{fig:bipedalwalkerenv}
\end{figure}

\begin{figure}
    \includegraphics[width=0.4\textwidth]{figures/SEBN-bipedal-walker.png}
    \caption{The SEBN for the Bipedal Walker environment.  }
    \label{fig:sebn-bipedal-walker}
\end{figure}



\paragraph{Example Task Descriptors using DoorKey.}
Suppose we want an agent to learn to navigate in a gridworld environment to a goal while opening locked doors.
Fig.~\ref{fig:doorkeyenv} shows several possible environments for this agent and their corresponding environment feature vector. In the easiest environment (top left), the agent (the white arrow) starts very near the goal (``A'') in an empty grid. Adding additional obstacles such as a wall, shown as chess rooks, or a locked door, shown as a lock, with a key to unlock it, adjusts the environmental features accordingly. The first three components of the task descriptor  \taskdescriptor indicate whether the distance (D) of agent starts near (within 2 squares) of any point of interest (key or goal), the presence of a wall (W), and the presence of a locked door (L).

DoorKey also enables the use of options.  
The top right part of Figure~\ref{fig:doorkeysebn} shows a network of targets $K$ for this problem, corresponding to ordered subtasks.
This problem has three options (at(goal), opened(door), and has(key)), each with its own reward. 
A distinct policy is learned for each of these options.
The last three components of \taskdescriptor indicate which of these three subgoals are enabled for a task: getting a key (K), opening a door (O), or being at (A) a cell. \exampledone




%=====================================================
%=====================================================
%=====================================================
%=====================================================
\section{Bayesian Curriculum Learning}

The key idea in this section is to use a BN to estimate performance on $K$ over the environmental features from $E$ plus a set of estimated proficiency \Proficiency on latent competencies, which are hidden or unobserved.
Before we formally define the SEBN, we describe extend the DoorKey example to discuss this process.

\paragraph{Example of Latent Competencies}
Suppose we have collected past data of the agent's performance on different environments in $E$. For example, say we ran our agent on the empty-grid (D0 W0 L0), wall-only (D1 W1 L0), and door-only (D1 W0 L1) environments and recorded that the agent was successful on the empty-grid and door-only environments but not the wall-only environment. 
This failure might be due to the  agent not yet knowing how to navigate around walls. We can think of this ability as a latent "avoid wall" capability that an agent needs to have mastered to solve tasks that require it. Furthermore, using the notion that there is this latent capability, we can easily predict that the agent will fail on the wall-and-door (D0 W1 L1) environment without having any data of the agent's performance on that specific type of environment. \exampledone

The bottom row of Fig. \ref{fig:doorkeysebn} shows a set of latent competencies or capabilities \Proficiency. In this example, we chose four latent competencies: (move, pick up, avoid wall, open) that we expect the agent to need to master to successfully solve different tasks.  These latent variables are provided by an expert, similar to \citeauthor{abelEtAl15.icaps.goalBasedActionPriors} \cite{abelEtAl15.icaps.goalBasedActionPriors}.


We can use the SEBN to predict two important quantities. 
First, when faced with a new (possibly unseen) task $\taski \in \tasks$, we need to estimate the proficiency of each competency in \Proficiency.  This is important because competencies will advance at different rates and some tasks will require more proficiency than others for specific competencies.

Second, when faced with a new (possibly unseen) task, we need to be able to predict performance on $k_j \in K$ given the current estimates of competency level (from the first step). This is important because it can be used to select from a set of candidate tasks for training in the next iteration of a curriculum.


\begin{figure}
\begin{scriptsize}
    
\begin{tabular}{p{1.8cm}p{1.8cm}p{1.8cm}p{1.8cm}}
\includegraphics[width=0.09\textwidth]{figures/doorkey1} & 
\includegraphics[width=0.09\textwidth]{figures/doorkey2} &
\includegraphics[width=0.09\textwidth]{figures/doorkey3} & 
\includegraphics[width=0.09\textwidth]{figures/doorkey4} \\
(D0 W0 L0 K0 O0 A1) & (D1 W1 L0 K0 O0 A1) &
(D1 W0 L1 K0 O0 A1) & (D0 W1 L1 K1 O1 A1) \\

\end{tabular}
\end{scriptsize}
    \caption{Example environments for DoorKey with  corresponding environment features (D:distance, W:wall, L:locked door) and target features (K:key, O:opened, A:at).}
    \label{fig:doorkeyenv}
\end{figure}
\begin{figure}
    %\includegraphics[width=0.45\textwidth]{figures/SEBN-door-key.png}
    \includegraphics[width=0.35\textwidth]{figures/SEBN-colored.png}
    \caption{The SEBN for the DoorKey environment.  }
    \label{fig:doorkeysebn}
\end{figure}



%\fromlaura{this seems super important and it might be nice to highlight it more loudly. the fact that the proficiency of the latent skills, if we do it right, can directly feed into an estimate of proficiency of the task is really cool. I suggest we ALSO highlight how it can be used for curricula learning, because this doesn't address that it reminds me too that I misread "skills" in the abstract to refer to policies when I first read it -- how we usually use them -- and not latent skills. or maybe not -- maybe they are the same thing? now I'm second-guessing myself :) } 
%\frommak{I am unsure how to fix this or where things are confusing.  Was this comment written before you read Section 3.2, where a clearer definition is provided.  I can't tell if you are suggesting a major restructuring of the content or you have some other solution in mind, or maybe this is just an out loud comment? }
%\fromlaura{definitely not a major restructuring. but I am a bit confused with how (if at all) this connects with RL "skills" (aka policies) as we currently use the term. maybe a reviewer wouldn't have that issue because they aren't familiar with our group's normal terminology. is there a correspondence? or are policies completely orthogonal to all of this?}

Returning to Fig.~\ref{fig:doorkeysebn}, suppose a new task is defined over E and \Target.  The proficiency estimate(s) can be calculated using the links to the bottom row. Once these estimates are provided back to the network, the expected reward can be calculated in the target layer \Target. 


%\fromlaura{do we need to say what happens if we don't have a latent skill that meaningfully connects to the new task? such as if we hadn't ever seen a door before and now need to pen a door for the new task? Would we even know that that requires a different skill? It might not be relevant here, for sure. but readers might wonder ahead of the actual technical explanation.}
%\frommak{Latent skills are  provided externally to the SEBN and static. We don't add new skills during learning.    This is defined in Section~\ref{sec:variable-distributions}, but is there something more that needs to be said here?}\fromlaura{maybe just a quick acknowledgement of that here -- since there is no guarantee that a new task will connect meaningfully to the existing skills.}
%\fromlaura{I would have appreciated an example of the task descriptor}



%-------------------------------------------------------
%-------------------------------------------------------
\subsection{Competencies (Latent or Explicit)} 
\label{sec:latent-skills}
As with the "avoid wall" latent competency for DoorKey, we propose that there is some shared latent set of competencies of which mastery over can predict an agent's success rate on different metrics in different environments. More concretely, let $\kappa_i \in K$ be a set of observable metrics, which can be any measurable target (e.g., a standard reward function, a shaped or partitioned reward function, or the completion an option). For example, we can define a binary variable that is 1 if an agent received a reward of more than a threshold value, and 0 otherwise. 

Drawing inspiration from ECD, we propose that there exists a set of latent competencies $\Psi$ that are not directly observable but can be inferred from the observed metrics. These competencies are such that the probability of success on a given observable metric $\kappa_i$ of an agent on a specific task $m_i$ is dependent on sufficient mastery of the corresponding latent competencies. This relationship allows us estimate the impact of competencies on unseen tasks. A BN allows us to model this relationship, estimate competency levels from data, and subsequently estimate success rates on different environments. 

Explicit competencies can be derived from techniques that decompose a task into subtasks.  
For example, in hierarchical planning, a method decomposes an abstract task.  
A set of such methods could be used to construct the competencies.
For example, recent work has used hierarchical goal networks to decompose tasks and train RL policies. \cite{patra2022hierarchical,patra2023relating}.
They call the resulting policies goal skills.
For the SEBN defined in Fig. \ref{fig:doorkeysebn}, all four of the competencies in $\Psi$ could be defined as an explicit goal skills.
In the case of this SEBN, the dependencies in the network are exactly the same as a corresponding hierarchical goal skill network. Consequently, we could take any problem with a heirarchical goal skill network, define environment-conditioned dependencies, and turn it into a SEBN. 

The flexibility of letting competencies be latent or explicit allows us to model environments where intermediate decompositions may not be well-defined.
In the absence of an easy way to check if an agent satisfies a goal for a given goal-skill, then it can be set to be a latent competency in the SEBN.







%-------------------------------------------------------
%-------------------------------------------------------
\subsection{Skill Environment Bayesian Network}
We can now define a Skill Environment Bayesian Network (SEBN) for estimating the competency level of an agent and modeling the relationship between agent competency levels, env features, and observable goals/metrics. The SEBN is a tuple $\{X, D, \parentfunctions\}$, where the variable set $X = E \cup \Target \cup \Proficiency$ is split into three sets of variables:
%\begin{itemize}
    %\item 

    
\textbf{Environment Variables.} $E$ is a set of variables that represent the features of an environment descriptor. Each variable in this set corresponds to a specific feature of the environment (and thus the domain of a given variable is the set of possible values the corresponding environment feature can take). In the gridworld example, $E$ consists of features for wall, door, and distance. 

    %\item 
    \textbf{Target Variables.} \Target is set of variables that directly correspond one or more targets. If there is a single policy, then \Target will have a single node, as in Figure~\ref{fig:sebn-bipedal-walker}. But if there are options available to the agent, then each $\target{j} \in K$ corresponds to the option for task $\taski^j$ (cf. \secref{sec:options}). These variables then provide estimates of the value of executing that option in the current environment. For the purposes of this paper, that estimate is thresholded such that each variable returns a boolean value corresponding to whether its estimate meets a performance threshold (roughly corresponding to an estimate of whether the option will succeed or fail). 

    %\item
    \textbf{Competency Variables.}  \Proficiency contains the set of variables that represent the competency levels of an agent. Each variable in this set denotes the level of proficiency an agent has in a particular competency and takes values in a range from $\{0,1,...,N\}$ where $N$ is the highest proficiency level for a given competency. For this paper, we will define competency with two or three levels of proficiency. Competency levels are roughly ordered, as provided in a set of requirement specifications, by a human expert. The rationale for writing these specifications is to convey whether a given environment requires sufficient proficiency in several competencies. Specifications follow (roughly) ordered values of competency (e.g. a higher “move” competency should enable harder tasks). The competencies can be latent (e.g., capturing whether an agent avoids obstacles while moving) or explicitly learnable (cf. \secref{sec:latent-skills}). 
%\end{itemize}

The parent functions \parentfunctions provide the distribution of possible values of each variable conditioned on their parent variables. To construct these parent functions, we specify a list of competency requirements that is procedurally used to construct the corresponding CPTs. We provide specific detail about this process in \secref{sec:variable-distributions}. 
%\fromlaura{I would have appreciated examples of this for the gridworld example, which didn't mention parent functions.}  \frommak{they are provided in the next section}


Once constructed, we can use the SEBN to estimate an agent's competency levels and determine what environments or competencies the agent should focus on learning next. To do this, we must estimate two quantities for a task $\taski \in \tasks$: 
%\begin{itemize}
%\item 
\textbf{Competency Level:} $P(\Proficiency = \proficiency| \target{i} = r, \tasks = \taski)$ - the probability that an agent has a competency level of \proficiency, given its prior performance of at least reward $r$ on target \target{i} for task \taski.  
%\item 
\textbf{Expected task reward:} $P(\target{i} = r| \tasks = \taski, \Proficiency = \proficiency)$ - the probability that an agent can achieve a reward of at least $r$ for target \target{i} conditioned on task \taski with a given competency level \proficiency.
%\end{itemize}

We will estimate the competency levels 
%\taski ($P(\Proficiency= \proficiency| \mdprewardi = r, E = \taski)$) 
using past rollouts.
We will then apply the estimates of competency levels to estimate the expected task reward 
%($P(\mdprewardi = r| E = \taski, \Proficiency = \proficiency)$,
over a collection of candidate environments $\taski \in \tasks$. 
These estimated probabilities will be used to determine which environments the agent should train on next.

%Our aim is to estimate the first probability, $P(S = s| T = t, E = t_m)$ using past rollouts, and then use the estimated skill proficiencies to estimate the second probability, $P(T = t| E = t_m, S = s)$, on a collection of candidate environments $t_m$. These estimated probabilities will be used to determine which environments the agent should train on next.

%-------------------------------------------------------
%-------------------------------------------------------
\subsection{Defining variable distributions} 
\label{sec:variable-distributions}
To complete our network definition, we need to define the distributions of each variable in the network. The distribution of variables in $\Psi$ and \Target are defined using a hierarchical structure. We start by defining the leaves of this structure which are located in \Proficiency. 

\paragraph{Defining \Proficiency}
Consider the "move" competency $move \in \Psi$ in the gridworld navigation environment. In Fig. \ref{fig:doorkeysebn}, we define $move$ as having three levels of proficiency $\{0,1,2\}$, associated with a corresponding parameter set $\phi_{move} = \{0.8, 0.2, 0.0\}$ such that the probability of the "move" at competency level $j$ is given by:
%\begin{align}
$P(move = j) = \phi_{move}(j)$.
%\end{align}
In this case, the parameters denote that currently the agent has a $move$ competency of 80\% probability for no mastery and a 20\% probability of having level one mastery.

More generally, \Proficiency can have a hierarchical structure and there can be latent competencies within $\Psi$ that depend on other latent competencies.  %(Though not used in this paper, the distributions for these hierarchical competencies would follow the same convention as variables in \Target without being conditioned on environment features.)  
To allow for these hierarchies in \Proficiency, we define $B$ to be a subset of $\Psi$ which represents a base set of competencies (the leaves). The distribution of these base competencies is determined by a set of associated parameters $\phi_{B}$. 

\paragraph{Defining \Target}
The variables in \Target represent the success of an agent on a given target. In an SEBN, we seek to model the relationship where the success probability of an agent on a variable $\target{i} \in \Target$ is dependent on three types of parent variables:
%\begin{enumerate}
%\item 
(1) the environment features $e_i \in E$ relevant to $\target{i}$
%\item
(2) the agent's current competency levels $\psi_i \in \Psi$ 
%\item 
(3) other targets $\target{i}' \in K$
%\end{enumerate}
This means that variables in \Target depend on other variables in $\Psi$ and \Target as well as a set of variables in $E$. The success of a task depends on the agent's mastery of the competencies required for a given environment configuration and an independent failure rate $\lambda$. A task succeeds with a rate of $(1-\lambda)$, if all sub-requirements in \Target and $\Psi$ are satisfied. 

More concretely, for each relevant environment configuration $e_i$ in relation to a goal variable $\target{i} \in K$, we define a set of competency level requirements $R_{\target{i}}$ that an agent must master to successfully complete the task of $k_i$ on the environment $e_i$. For example, in Fig. \ref{fig:doorkeysebn}, \textit{haskey} depends on the \textit{distance} environment feature and the latent $move$ and $pickup$ competencies. Suppose we define the following competency level requirements for the \textit{haskey} node:
\begin{small}
%\begin{align}
\begin{itemize}[noitemsep,topsep=0pt,parsep=0pt,partopsep=0pt]
    \item[] haskey: (distance=0 ~|~ move=1,pick up = 1) 
    \item[] haskey: (distance=1 ~|~ move=2, pick up = 1)
\end{itemize}
%\end{align}
\end{small}
These state that if the key is close (distance=0), the agent needs a level of proficiency of 1 in the move and pick up competencies to successfully get the key. However, if the key is far (distance=1), the agent needs a higher level of proficiency in the move competency (move=2). The agent should have a high chance of success if it meets all necessary requirements and a high chance of failure otherwise. 

We directly translate these requirements into entries in the corresponding CPTs in the following way.
%\begin{itemize}
    %\item
    For the first competency level (distance=0), we have that:
    \begin{small}
    \begin{itemize}[noitemsep,topsep=0pt,parsep=0pt,partopsep=0pt]
    %\begin{align}
        \item[] $P(haskey = 1| distance = 0, move = 0, pick up = 0) = \lambda $
        \item[] $P(haskey = 1| distance = 0, move = 0, pick up = 1) = \lambda$ 
        \item[] $P(haskey = 1| distance = 0, move = 1, pick up = 0) = \lambda$ 
        \item[] $P(haskey = 1| distance = 0, move >= 1, pick up >= 1) = (1 - \lambda)$
    %\end{align}
    \end{itemize}
    \end{small}
    %\item 
    For the next competency level (distance=1), we have that:
    \begin{small}
    \begin{itemize}[noitemsep,topsep=0pt,parsep=0pt,partopsep=0pt]
    %\begin{align}
        \item[] $P(haskey = 1| distance = 1, move = 0, pick up = 0) = \lambda$
        \item[] $P(haskey = 1| distance = 1, move = 0, pick up = 1) = \lambda$
        \item[] $P(haskey = 1| distance = 1, move = 1, pick up = 0) = \lambda$
        \item[] $P(haskey = 1| distance = 1, move = 1, pick up = 1) = \lambda$
        \item[] $P(haskey = 1| distance = 1, move >= 2, pick up >= 1) = (1 - \lambda)$
    %\end{align}
    \end{itemize}
    \end{small}
%\end{itemize}
These state that the agent has a success rate of $(1-\lambda)$ for a given goal for an environment setting if it satisfies all necessary requirements and a success rate of $\lambda$ if there is one or more requirement missing.

For the environment features set $E$, the distribution of variables in this set is fully controlled for the purpose of curriculum generation. Therefore, the data (samples obtained from rollouts) can be used as the distribution for variables in $E$.

%-------------------------------------------------------
%-------------------------------------------------------
\subsection{Curriculum through Inference}
The general algorithm to generate an SEBN-guided automated curriculum is as follows. First we define a generation of the algorithm as $L$ rollouts. Let $\tasks$ be the set of possible tasks and $P_\tasks(m_i)_t$ be the probability that task $m_i = (e_i, \target{i})$ is chosen in the current generation $t$. We first initialize $P_\tasks(m_i)_0$ to some initial task distribution. This initial weighting can be biased towards easier tasks or can be set to a uniform distribution for better initial competency estimation. Let $\Phi_{B} = \{...,\phi_{B_i},...\}$ be the set of parameters associated with the base competencies $B$ in the SEBN. For each generation, we take the following steps:
\begin{enumerate}[noitemsep,topsep=0pt,parsep=0pt,partopsep=0pt]
    \item Sample $L$ rollouts $(m_i = (e_i, \target{i}) \sim P_\tasks, o_i)$. For each rollout, we record a set of observable metrics $\target{i}$. 
    \item Solve the MLE problem:
    \begin{align}
       \Phi_{B}^* = \text{arg}\max_{\Phi_{B}} \prod_{L} P(E = e_i, K = \target{i}) 
       \label{eq:mle}
    \end{align}
    for $\Phi_{B}^*$, the values of the parameters for the base competencies, which is an estimate of agent proficiency level in those competencies.
    \item Update the task distribution for the next generation: $P_\tasks(m_i) = F(P_{\Phi_{B}^*}(\target{i}|m_i))$ where $F$ is some function that maps the estimated success rate of an environment $P_{\Phi_{B}^*}(\target{i}|m_i)$ to a probability distribution.
\end{enumerate}
For our work, we define $F$ using the following fitness function:
\begin{align}
    F(m_i)_t = (P_{\Phi_{B}^*}(\target{i}|m_i)_t - P_{\Phi_{B}^*}(\target{i}|m_i)_{t-1})^2 \nonumber\\
    P_\tasks(m_i)_{t+1} = 0.5 \cdot \frac{F(m_i)}{\sum_{m_i} F(m_i)} + 0.5 \cdot P_\tasks(m_i)_{t}.
    \label{eq:fitness}
\end{align}
In \cite{stout2010competence}, it was proposed that curricula should focus on problems where the agent improves the most or has the most expected \textit{improvement in competence}. To simulate this in our approach, we choose a fitness function where the fitness of the environment is a function of the difference between the current estimated success rate and the estimated success rate on the last generation's SEBN. We also add a smoothing factor to improve learning stability.

Note that our curriculum is agnostic to the choice of learning algorithm and policy which can be assumed to be black boxes. The curriculum only requires observations of rollouts and not the internal reward structure of a given policy. 

\begin{algorithm}
\SetAlgoLined
\begin{small}
\While{ not converged}{
    Sample L environments $m_i = (e_i,\target{i}) \sim P_{\tasks_0}$ \\
    \For{$i = 1 \rightarrow L$}{
        Collect rollout data $(m_i, \target{i})$ while training policy $\pi$
    }
    Solve the MLE problem (Equation \ref{eq:mle}) for estimated competency level $\Phi_{B}^*$ on SEBN $(X,D,\Phi)$ \\
    For candidate environments, estimate agent success rate $P_{\Phi_{B}^*}(\target{i}|m_i)$  using SEBN with $\Phi$ updated with $\Phi_{B}^*$ \\
    For each candidate environment $m_i$, update $P_\tasks(m_i)_{t+1}$ using expected improvement weighting of $P_{\Phi_{B}^*}(\target{i}|m_i)$ (Equation~\ref{eq:fitness})
}
 \caption{SEBN-guided Automated Curriculum\newline
 \textbf{Input:} Initial tasks $P_{\tasks_0}$, Generation size $L$, SEBN $(X,D,\Phi)$ \newline
 \textbf{Initialize:} Initialize policy $\pi$ }
\end{small}
 \label{alg:SEBN}
\end{algorithm}


%-------------------------------------------------------
%-------------------------------------------------------
\subsection{Candidate Selection}
\label{sec:approximate-inference}

On a domain such as Megagrid, we can evaluate $P_{\Phi_{B}^*}(\target{i}|m_i)$ for every combination of environmental features. However, calculating $P_{\Phi_{B}^*}(\target{i}|m_i)$ for all possible environments in BipedalWalker took too much time at the end of each rollout generation. In general, for domains with a large amount of environmental features, it is impractical to evaluate $P_{\Phi_{B}^*}(\target{i}|m_i)$ for every single task $m_i \in \tasks$.  

One way to solve this computational problem is to search in the space of possible environment configurations and only update the distribution of the most promising environments. For our search procedure, we adapt a greedy sample-search procedure based on KL-Search from \cite{hsiao2024surrogate}. This algorithm employs a heuristic search along variables in a Bayesian network to minimize a KL distance heuristic between two networks. By modifying algorithm to instead search for nodes with maximum differences between the current generation's SEBN and the next generation's SEBN updated with $\Phi_{B}^*$, we can find environments where the estimated success rate changes the most. This modification produces a tree search algorithm that selects nodes in the OR-tree corresponding to a given SEBN where the difference $(P_{\Phi_{s_i}^*}(\target{i}|m_i)_t - P_{\Phi_{s_i}^*}(\target{i}|m_i)_{t-1})$ is greatest. 

Given a partial configuration $X$, we use the following heuristic:
\begin{align}
    h(X_1 {=} 0) = |\log (P_{t-1}(X)) - \log (P_t(X))| \cdot P_t(X)
    \label{eq:heuristic}
\end{align}
where $P_{t}(X_1 {=} 0) = P_{\Phi_{s_i}^*}(\target{i}|X1)_t$. % This ensures that we pick nodes where the difference in estimated success rates between the previous and current SEBNs are the greatest. 
We use Weighted Mini-bucket Elimination \cite{liu2011bounding} with an ibound of 20 to estimate SEBN probabilities when exact inference is too computationally expensive. Once we select $N = 20$ nodes on the OR-tree using this method, we perform the same calculations from Eq. \ref{eq:fitness} over the 20 selected environment configurations to define our curriculum for the next generation.


%=====================================================
%=====================================================
%=====================================================
%=====================================================
\section{Experimental Evaluation}
To demonstrate the effect of our proposed curriculum learning approach, we evaluate the SEBN curriculum on three environments.
%\begin{enumerate}
%\item 
\textbf{DoorKey:} A MiniGrid \cite{MinigridMiniworld23} inspired domain with explicit intermediate goal skills. In this domain, the agent must learn to navigate through a grid-based environment to reach a goal location, while also learning to achieve intermediate goals along the way.
%\item 
\textbf{BipedalWalkerHardcore:} a simulated bipedal robot must learn to walk forward as quickly as possible while maintaining balance. The bipedal robot can encounter a variety of obstacles such as rough terrain, stumps, pits and stairs that need robust policies.
%\item 
\textbf{Robosuite:} a robotic arm (Kinova Gen 3) must learn to open a door with different weight and latch settings.
%\end{enumerate}

In each of these domains, we compare the performance of reinforcement learning agents trained with and without our proposed SEBN-guided curriculum as well as two additional controls:
%\begin{itemize}
    %\item 
    \textbf{Uniform curriculum (or Domain Randomization \cite{tobin2017domain}):} all environments have an equal probability of being selected.
    %\item 
    \textbf{Anti-curriculum:} the probability difference in Eq. \ref{eq:fitness} is replaced with (1 - difference) and we use a min priority queue for candidate selection during the sample-search algorithm.
%\end{itemize} 

We evaluate the agents on their ability to learn effective policies that can achieve high rewards in each domain, as well as their ability to generalize to new tasks and environments. Our results show that the SEBN-guided curriculum consistently improves the performance of reinforcement learning agents across all three domains. All runs are performed on an AMD EPYC 7H12 64 core CPU with networks being handled on an A100 GPU.

\begin{figure}
    \includegraphics[width=0.36\textwidth]{figures/csv_plot_new}
    \caption{Result of employing a SEBN-guided automated curriculum on the DoorKey environment. %\frommak{Further space can be saved by making the three results plots be a figure* environment with a single caption and labelled subplots using subfigure (so they can be referenced properly).  this would require moving some details to prose, but that's fine}
    }
    \label{fig:doorkeyres}
\end{figure}

\paragraph{DoorKey}
We start with a gridworld environment named Megagrid based on MiniGrid \cite{MinigridMiniworld23}; we reimplemented this standard gridworld environment to enable easier generation of environments using the task descriptor\footnote{Please send an email to the authors to request Megagrid code}. We evaluate on the simple DoorKey environment to evaluate the effectiveness of our curriculum learning approach when combined with explicit goal-skills. We use the SEBN shown in Fig. \ref{fig:doorkeysebn} which has the following variables:
\begin{itemize}[noitemsep,topsep=0pt,parsep=0pt,partopsep=0pt]
    \item[$E$:] includes a wall feature \zeroone, door feature \zeroone, and a distance feature \zeroone. A selection of the different environments that can be experience by an agent can seen in Fig. \ref{fig:doorkeyenv}. 
    \item[$\Psi$:] includes "move to" \zot, "pick up" \zeroone, "avoid wall" \zeroone, "drop" \zeroone, and "open door" \zeroone.
    \item[\Target:] includes three options: "at(goal)" \zeroone, "opened(door)" \zeroone, and "has(key)" \zeroone~ each trained with a PPO policy.
\end{itemize}
Observations of the environment are provided as partially observed cardinal direction data, following the sensor convention for the Lightworld domain in \cite{konidaris2012transfer}. We assume that the agent has four cardinal sensors for each item. For example, in the rightmost env of Fig. \ref{fig:doorkeyenv}, the agent would receive the observation that there is a key one step above it, a wall one step below it, and the goal 4 steps below it (0.875 key - up, 0.875 wall - down, 0.5 goal - down). An agent gets a reward of 1 if it reaches the goal square or completes any intermediate goals required (e.g. picking up a key or opening a door). There is no step penalty but the episode is automatically terminated if no progress has been made in 50 steps.

We train the policies using an Actor-Critic architecture \cite{sutton2018reinforcement} trained using Proximal Policy Optimization (PPO) \cite{schulman2017proximal}. The policy and value networks each have four hidden layers which are used to calculate their corresponding outputs.

The evaluation curve of our policies can be seen in Fig. \ref{fig:doorkeyres}. This evaluation is performed on the hardest environment (rightmost environment in Fig. \ref{fig:doorkeyenv}). Because this environment is relatively simple, it is easy for policies learned without a curriculum to achieve a high success rate. However, there is a noticeable jumpstart where the SEBN-guided curriculum provides a gain in learning efficiency.

We performed a generalization experiment where the learned policies are transferred to an evaluation on an \emph{unseen} and much larger 32 x 32 grid environment. The policies learned using the SEBN-guided curriculum are more robust to this change in grid size succeeded at an average rate of 93 percent compared to only 82 percent without a curriculum. %(see appendix for full plot). 

%The results can be seen in Fig. \ref{fig:doorkeylarge}.



%-------------------------------------------------------
%-------------------------------------------------------
%\subsection{Continuous navigation on varied terrain}
\paragraph{BipedalWalker Hardcore}
For the next domain, we evaluate our SEBN-guided curriculum on BipedalWalker (BPW), a continuous control environment. We employ a modified version of the BipedalWalkerHardcore environment from \cite{parker2022evolving} to suit a limited computational budget. We define the following SEBN:
\begin{itemize}[noitemsep,topsep=0pt,parsep=0pt,partopsep=0pt]
    \item[$E$:] includes five design parameters: ground roughness $\{0 - 7\}$, pit gap $\{0 - 3\}$, stump height $\{0 - 2\}$, stair width $\{0 - 3\}$, and stair steps $\{0 - 3\}$. Since there are too many different environments to perform exact inference, it is necessary to use a sample-search procedure to select candidate environments.
    \item[$\Psi$:] includes  "move", "climb", "jump", "balance", "descend". All with proficiency levels $\{0,1,2\}$. 
    \item[\Target:] includes one observable metric \zeroone: whether the agent has traveled a distance of 30 units ($\approx$1/3rd of the level).
\end{itemize}
The observation of the agent consists of internal sensor measurements such as (hull angle speed, angular velocity, horizontal velocity, etc..) an a set of 10 lidar rangefinder measurements. On this environment, the robotic walker gets a dense positive reward for traveling forward on the terrain, a small negative reward for using its motors, and a negative reward of -100 if it falls down. On environments that are too challenging for the agent at its current capabilities, this reward structure can promote a locally optimal behavior of simply staying still and preventing itself from falling. 

To learn our policies, we use the TD3-Fork algorithm from Honghao et al. \cite{wei2020fork}, which was shown to train much faster than standard PPO on BPW. We train agents for 3.5 million timesteps. During training we evaluate against four specific challenges (Stairs, PitGap, Stump, and Roughness in Fig. \ref{fig:bipedalwalkerenv}) as well as a combined environment (Evaluation) that contains all challenges together. We compare against a policy trained on only the combined environment.

\begin{figure}
    %\includegraphics[width=0.23\textwidth]{figures/csv_plot_bpw0}
    %\includegraphics[width=0.23\textwidth]{figures/csv_plot_bpw1}
    %\includegraphics[width=0.23\textwidth]{figures/csv_plot_bpw2}
    %\includegraphics[width=0.23\textwidth]{figures/csv_plot_bpw3}
    \includegraphics[width=0.36\textwidth]{figures/csv_plot_bpw4}
    \caption{SEBN-guided automated curriculum on BipedalWalker. Evaluation environments are randomly generated within a given environment feature set.
    }
    \label{fig:bipedalwalkerres}
\end{figure}
The results of the evaluation on the combined environment can be seen in Fig. \ref{fig:bipedalwalkerres}. Since we do not train for a large number of environment timesteps, we can observe that without a curriculum, TD3-Fork does not manage to learn to the point of a positive reward on any of the test environments.

There is a significant reward divergence in the results starting at 1 million timesteps. The uniform and anti curriculum perform better than having no curriculum with the uniform curriculum perform marginally better than the anti-curriculum. It can be seen on each graph that the policies trained using the SEBN-guided curriculum manages to get to a point where the agent starts receiving a positive reward for each environment, getting past the initial hurdle of the locally optimal staying still behavior.

It is interesting to observe that the learning curve for the SEBN-guided curriculum has much higher variance than the learning curves for other methods. Due to the size of the environment design space, it is necessary to use approximate inference techniques to search for candidate environments. Because we use a sample-search procedure, the search process is not guaranteed to find environments with the largest heuristic difference (see Eq. \ref{eq:heuristic}). This means that sub-optimal environments can be introduced into the curriculum. This introduces a large factor into learning curve variance beyond standard noise from reinforcement learning algorithms.


%-------------------------------------------------------
%-------------------------------------------------------
%\subsection{Simulated robotics}

\begin{figure}
    \includegraphics[width=0.36\textwidth]{figures/csv_plot_robo}
    \caption{SEBN-guided automated curriculum on the Robosuite Door environment. 15 policies are learned for each line and evaluated on the hardest (mass=6, latch=1) environment.}
    \label{fig:robosuiteres}
\end{figure}

\paragraph{Robosuite - Open Door}
In our final domain, we evaluate the SEBN-guided curriculum in a simulated robotics domain using the robosuite simulation environment. In particular we choose the Door task in which an agent needs to manipulate a robot to open a door. For our environment design space, we include two main parameters: the weight of the door (with 6 different settings) and whether the door has a latch or not. The specific robot that we choose to simulate is a Kinova3 arm. %An example of the environment can be seen in Fig. \ref{fig:robosuiteenv}. 
We define the following SEBN:
\begin{itemize}[noitemsep,topsep=0pt,parsep=0pt,partopsep=0pt]
    \item[$E$:] includes design parameters mass $\{0 -6 \}$ and latch \zeroone.
    \item[$\Psi$:] includes "move arm" \zeroone, "unlock" \zeroone, "door" \zot. 
    \item[\Target:] includes one observable metric \zeroone: whether the agent has successfully opened the door.
\end{itemize}

To learn our policies, we use PPO on a neural network with two hidden layers. We use the inbuilt observation setup and reward shaping in the robosuite environment to accelerate the learning process and we train our agents for 2 million timesteps. We evaluate our learned policies every 100,000 timesteps and the results of the experiment can be seen in Fig. \ref{fig:robosuiteres}. 

It is interesting to observe that the default training method stagnates after reaching a reward plateau. When viewing the actual behavior of the learned policy, this reward plateau is indicative of learning a policy of moving the arm towards the door handle but not actually moving the door. It is possible that either the weight of the door or the presence of a latch in harder environments prevents the agent from attempting the difficult action of applying force to the door to get an increased reward. Since the SEBN-guided curriculum will have started with at least some of its distribution in the easy case of a light door with no latch, the agent will have learned that opening the door can give a positive reward and transfer this behavior to more difficult environments. 

We also performed a generalization study, where we test our learned policies on an \emph{unseen} heavy door env not included during our learning process. We observed that with the policy learned by the SEBN-guided curriculum is easily transferred to more heavy doors (obtaining a reward of 240 on the heavy door env compared to 150 without a curriculum), in contrast to the other methods. % (see appendix for the full graph).


%=====================================================
%=====================================================
%=====================================================
%=====================================================
\section{Related Work} 
\paragraph{Automated Curriculum Generation}  Several topics relate to ordering tasks to improve learning performance. A few approaches have considered the problem of estimating agent skill competencies. In the context of education, in addition to ECD \cite{mislevy2003brief}, another approach close to ours is that of \citeauthor{greenEtal2011.iaai.learningASkillTeachingCurriculum} \cite{{greenEtal2011.iaai.learningASkillTeachingCurriculum}}, who used a BN to determine the next task for the human student.  
This approach is similar to \cite{kumar2024practice}, which considered an active learning problem in a robotics domain of choosing which skills to practice to maximize future task success, which involves estimating the competence of each skill and situating it in the task distribution through competence-aware planning. In contrast to our approach, they employ a simplified Bayesian time series model that does not relate environmental features with goal and skill competencies. This limits the applicability of their approach towards only choosing what skill to train and not the agent's environment. Similar to our selection process, \citeauthor{BalleraEtal2014.icetc.personalizingElearning} \cite{BalleraEtal2014.icetc.personalizingElearning} used a roulette wheel selection of tasks. 

A related area is the literature on Unsupervised Environment Design (UED) \cite{dennis2020emergent} and other developed mechanisms for curating environments based on a regret heuristic \cite{jiang2021replay}. In prior UED approaches such as PAIRED \cite{dennis2020emergent}, the agent's curriculum is generated using a regret-based heuristic. The heuristic is typically an estimate of the true regret, since the optimal policy is unknown. In PAIRED, this heuristic is calculated by learning an antagonistic policy and evaluating the difference between it and the protagonist policy. In contrast, like ACCEL \cite{parker2022evolving}, our method does not need to learn a second antagonistic policy and instead uses rollouts from a single agent to compute the next part of the curriculum. In contrast to ACCEL, %however, which relies on local changes to the task descriptor (evolutionary mutations) and subsequent regret pruning, 
our curriculum does not rely on local changes and can incorporate larger jumps in environment selection. Furthermore, while it is necessary to use rollouts on each environment for ACCEL to obtain a regret estimate, we can estimate success rates on unseen environments by leveraging the relationships encoded between competencies and environmental features in our SEBN.


\paragraph{Task Descriptors}
As mentioned in \secref{sec:taskdescriptors}, grouping tasks using features, i.e., task descriptors, are a well understood technique for task creation (\cite{rostamiEtAl20.jair.usingTaskDescriptions,isele2016task,narvekar2016source,konidaris2012transfer}).  The key idea in these works is to facilitate learning transfer by creating similar tasks that share common features.  These features can leave certain variables free during task construction that enable a family of similar tasks. 
Our work supplements prior work by adding the target of the task to the task descriptor, allowing the curriculum to emphasize subtasks.

To our knowledge, there has been limited previous work in integrating curriculum learning on both skills and environment features. However, it can be said that our research expands on the concept of using task descriptors in the creation of automated curricula. In \cite{patra2023relating}, a task-graph curricula is used to generate a curriculum over tasks and environmental features. However, they employ a simple greedy best-first search on the task-graph to choose an order for their curriculum. This is different from our approach that updates a distribution over the task-graph and dynamically adjusts this distribution based on data from rollouts.

\paragraph{Hierarchical Goal Networks}

The structure of our Skill Environment Bayesian network shares similarity to both goal skill networks and fault diagnosis networks. In fault diagnosis networks \cite{cai2017bayesian}, BNs are used to model the relationship between a set of sensors and a set of faults. In our case, the sensors are analogous to an SEBN's observable goal metrics, and the faults are analogous to an SEBN's skills. An SEBN can then be seen as a fault diagnosis network where different roll-outs are independent tests that determine what latent competencies may have not been mastered.




\paragraph{Expert guidance for RL training}
One of the limitations of the SEBN is its reliance on the expert-provided competencies. 
As noted, these could be derived from hierarchical approaches. 
But providing domain knowledge is common in many hierarchical RL settings.
Similar to our work,  \citeauthor{patra2022hierarchical} \cite{patra2022hierarchical}, provided a hierarchical learning structure.
This kind of expert knowledge is common in  imitation learning (e.g., \cite{zhang19leveraging} \cite{hussein2017imitation} \cite{le2018hierarchical}), where an expert human guides a learning agent.
Providing expert guidance is also common in Hierarchical RL approaches (e.g., \cite{ahmadiTaylorStone07.aamas.IFSA}) and in standard RL approaches (e.g., \cite{andreas2017modular}).
Finally, expert guidance was shown to be helpful for a sparse-reward task in an Object Oriented MDP setting \cite{abelEtAl15.icaps.goalBasedActionPriors}.


%\paragraph{Option discovery}

%\paragraph{Goal-Conditioned RL}
%goal-conditioned and OO RL (Abel et al. ICAPS 2015); 

%\paragraph{Unsupervised Environment Design}
%Unsupervised Experiment Design; sampling-based guidance (HER,PER); 




%=====================================================
%=====================================================
%=====================================================
%=====================================================
\section{Conclusion and Future Work}
We presented a novel method for generating curriculum over environmental features using a Skill-Environment Bayesian Network. This network is used to estimate agent competency level based off of past rollouts and can be used to infer estimated agent success rates on unseen environments. We demonstrate the effectiveness of this approach on a variety of domains.

For this work, we relied on a pre-defined set of skills and environmental features. In future work, we would like to extend the model to handle more open ended environments where we can add new environmental features or agent skills dynamically to the SEBN. It might be interesting to see if we can apply techniques such as GO-MTL \cite{kumar2012learning} for learning a latent space over tasks and approaches for detecting critical regions \cite{molina2020learn} to learn new skills.

There has also been a great deal of interest in the use of Large Language Models (LLMs) in planning domains for the purpose of automatically generating planning models. As SEBNs can be built from a heirarchical goal network, it might stand to reason that LLMs could also be used to automatically generate SEBNs from domain documents. Since there is much more leeway in the skills required (since our method supports latent competency nodes), it may be easier to generate these SEBNs than equivalent goal networks.

\begin{acks}
We thank the Basic Research Office of OUSD and NRL for funding this research.
\end{acks}

%%%%%%%%%%%%%%%%%%%%%%%%%%%%%%%%%%%%%%%%%%%%%%%%%%%%%%%%%%%%%%%%%%%%%%%%

%%% The next two lines define, first, the bibliography style to be 
%%% applied, and, second, the bibliography file to be used.

\newpage
\balance
\bibliographystyle{ACM-Reference-Format} 
\bibliography{sample}

%%%%%%%%%%%%%%%%%%%%%%%%%%%%%%%%%%%%%%%%%%%%%%%%%%%%%%%%%%%%%%%%%%%%%%%%
\newpage

\onecolumn

\appendix

\section{Supplementary Material}

This is extra material for the submission titled
"\system: Pessimistic Cardinality Estimation Using
  $\ell_p$-Norms of Degree Sequences."
This material is organized as follows.
Section~\ref{subsec:proof:th:lpbase:eq:lptdb} gives the proof for Theorem~\ref{th:lpbase:eq:lptdb}.
Section~\ref{subsec:lptd} describes  a third optimized algorithm for estimating arbitrary conjunctive queries, which uses hypertree decompositions of the queries. This is not yet implemented in \system.
Section~\ref{subsec:lpflow:details} gives more details on the \lpflow optimization and the proof for  Theorem~\ref{th:lpbase:eq:lpflow}.
\nop{Section~\ref{app:predicates-examples} gives examples on how \system accommodates selection predicates.
Section~\ref{app:further-experiments} details further experiments.
}

%%%%%%%%%%%%%%%%%%%%%%%%%%%%%%%%%%%%%%%%%%%%%%%%%%%%%%%%%%%%%%%%%%%%%%%%%%%%%%%%%%%%%%%%%%%%
\subsection{Proof of Theorem~\ref{th:lpbase:eq:lptdb}}
\label{subsec:proof:th:lpbase:eq:lptdb}
For simplicity of presentation, we assume here that the
query $Q$ is connected.
Denote by $b_{\text{base}}$ and $b_{\text{Berge}}$ the values of
\lpbase and \lptdb.  The inequality
$b_{\text{base}}\leq b_{\text{Berge}}$ follows from two observations:
\begin{itemize}
\item For any acyclic query $Q$ and any polymatroid $h$, the
  inequality $E_Q \geq h(X_1\cdots X_n)$ is a Shannon inequality.
  This is a well known
  inequality~\cite{DBLP:journals/tse/Lee87,DBLP:conf/sigmod/KenigMPSS20}
  (which we review in Lemma~\ref{lemma:tony:lee} below).
\item Any feasible solution to \lpbase can be converted to a feasible
  solution of \lptdb by simply ``forgetting'' the terms $h(U)$ that do
  not occur in \lptdb.
\end{itemize}

For the converse, $b_{\text{Berge}}\leq b_{\text{base}}$, we will
prove that every feasible solution $h$ to \lptdb can be extended to a
feasible solution to \lpbase (by defining $h(U)$ for all terms $h(U)$
that did not appear in \lptdb), such that $E_Q = h(X_1\cdots X_n)$.
We will actually prove something stronger: that $h$ can be extended to
a \emph{normal polymatroid}.

\begin{definition} \label{def:normal}
  A set function $h : 2^{\set{X_1,\ldots,X_n}} \rightarrow \R$ is a \emph{normal
    polymatroid} if $h(\emptyset)=0$ and it satisfies:
%
  \begin{align}
    \forall U \subseteq \set{X_1,\ldots,X_n}: \sum_{W\subseteq U}(-1)^{|W|+1}h(W) \geq &0 \label{eq:normal}
  \end{align}
\end{definition}
% 
It is known that every normal polymatroid is an entropic vector, and
every entropic vector is a polymatroid, but none of the converse
holds.

The inequality $b_{\text{Berge}}\leq b_{\text{base}}$ follows from two lemmas:

\begin{lemma} \label{lemma:normal:extension:of:one:bag} Let
  $V = \set{X_1, \ldots, X_n}$, let $a_1, \ldots, a_n, A$ be $n+1$
  non-negative numbers such that:
%
  \begin{align}
    a_1 + \cdots + a_n \geq & A \ \ \ \ \text{and}\ \ \ \ \ a_i \leq A, \forall i=1,n \label{eq:a:a}
  \end{align}
%
  Then there exists a normal polymatroid $h : 2^V \rightarrow \R$ such
  that $h(X_i)=a_i$ for all $i=1,n$ and $h(V) = A$.
\end{lemma}

\begin{lemma}[Stitching Lemma] \label{lemma:stich} Let
  $V_1, V_2$ be two sets of variables, $Z \defeq V_1 \cap V_2$. Let
  $h_1:2^{V_1} \rightarrow \R$, $h_2: 2^{V_2} \rightarrow \R$ be two
  normal polymatroids that agree on their common variables $Z$: in
  other words there exists $h : 2^Z \rightarrow \R$ such that
  $\forall U \subseteq Z$, $h_1(U)=h(U)=h_2(U)$.  Define the following
  function $h' : 2^{V_1\cup V_2} \rightarrow \R$:
%
  \begin{align}
    h'(U) \defeq & h_1(U \cap V_1|U \cap Z) + h_2(U \cap V_2|U \cap Z) + h(U \cap Z) \label{eq:stich}
  \end{align}
%
  Then $h'$ is a normal polymatroid that agrees with $h_1$ on $V_1$ and
  with $h_2$ on $V_2$, and, furthermore, satisfies:
%
  \begin{align}
    h'(V_1 \cup V_2) = & h'(V_1) + h'(V_2) - h'(V_1\cap V_2) \label{eq:independent}
  \end{align}
  %
  In essence, this says that $V_1, V_2$ are independent conditioned on
  $V_1 \cap V_2$.
\end{lemma}


Notice that $h'$ can be written equivalently as
$h'(U) = h_1(U\cap V_1)+h_2(U\cap V_2) - h(U\cap Z)$.  While each term
is a normal polymatroid, it is not obvious that $h'$ is too, because
of the difference operation.  In fact, if $h_1, h_2, h$ are
polymatroids, then $h'$ is not a polymatroid in general.


The two lemmas prove Theorem~\ref{th:lpbase:eq:lptdb}, by showing that
$b_{\text{Berge}}\leq b_{\text{base}}$, as follows.  Consider any
feasible solution $h$ to \lptdb.  Consider first a single atom
$R_j(V_j)$ of $Q$: $h$ is only defined on all its variables and on the
entire set $V_j$.  By Lemma~\ref{lemma:normal:extension:of:one:bag},
we can extend $h$ to a normal polymatroid
$h: 2^{V_j} \rightarrow \Rp$.  We do this separately for each
$j=1,m$.  Next, we stitch these polymatroids together in order to
construct a polymatroid on all variables,
$h:2^{\set{X_1,\ldots, X_n}}\rightarrow \Rp$, and for this purpose we
use the Stitching Lemma~\ref{lemma:stich}.  Notice that the Lemma is
stronger than what we need, since in our case the intersection
$V_1 \cap V_2$ always has size 1 (since $Q$ is Berge-acyclic): we need
the stronger version for our third algorithm described in Sec.~\ref{subsec:lptd}.  By
using the conditional independence equality~\eqref{eq:independent}, we
can prove that $E_Q = h(X_1\cdots X_n)$, which completes the proof of
Theorem~\ref{th:lpbase:eq:lptdb}.

It remains to prove the two lemmas.

\begin{proof}[Proof of Lemma~\ref{lemma:normal:extension:of:one:bag}]
  We briefly review an alternative definition of normal polymatroids
  from~\cite{DBLP:conf/lics/Suciu23}.  For any $U \subseteq V$, the
  step function at $U$ is $h^U$ defined as:
%
\begin{align}
\forall X \subseteq V: &&  h^U(X) \defeq &
                  \begin{cases}
                    1 & \mbox{if $U\cap X\neq \emptyset$}\\
                    0 & \mbox{otherwise}
                  \end{cases} \label{eq:step:function}
\end{align}
%
When $U=\emptyset$, then $h^U \equiv 0$, so we will assume
w.l.o.g. that $U \neq \emptyset$.  A function $h : 2^V \rightarrow \R$
is a normal polymatroid iff it is a non-negative linear combination of
step functions:
%
\begin{align}
  h = & \sum_{U \subseteq V, U \neq \emptyset} c_U h^U
\end{align}
%
where $c_U \geq 0$ for all $U$.

We prove the lemma by induction on $n$, the number of variables in
$V$.  If $n=1$ then the lemma holds trivially because we define
$h(X_1) \defeq a_1$, so assume $n \geq 2$.  Rename variables such that
$a_1 \geq a_2 \geq \cdots \geq a_n$, and let $k \leq n$ be the
smallest number such that $a_1 + \cdots + a_k \geq A$: such $k$ must
exist by assumption of the lemma.  We prove the lemma in two cases.

{\bf Case 1}: $k=n$.  Let $\delta \defeq \sum_{i=1,n}a_i - A$ be the
excess of the inequality~\eqref{eq:a:a}: notice that $a_1 \geq \delta$
and $a_n \geq \delta$.  We define $h$ as follows:
  %
\begin{align*}
  h = & (a_1-\delta)h^{X_1}+\delta h^{X_1,X_n}+\sum_{i=2,n-1}a_i h^{X_i} + (a_n-\delta)h^{X_n}
\end{align*}
  %
By construction, $h$ is a normal polymatroid, and one can check by
direct calculation that $h(X_i)=a_i$ for all $i$ and $h(V)=A$.

{\bf Case 2}: $k < n$.  We prove by induction on $m=k,k+1,\ldots,n$
that there exists a normal polymatroid
$h : 2^{\set{X_1, \ldots, X_m}} \rightarrow \R$ s.t. $h(X_i)=a_i$ for
$i=1,m$ and $h(X_1\ldots X_m)=A$.  The claim holds for $m=k$ by Case
1.  Assuming it holds for $m-1$, let
$h' : 2^{\set{X_1,\ldots,X_{m-1}}}\rightarrow \R$ be such that
$h'(X_i)=a_i$ for $i=1,m-1$ and $h'(X_1\cdots X_{m-1})=A$.  We show
that we can extend it to $X_m$.  For that we first represent $h'$ over
the basis of step functions:
  %
\begin{align*}
  h' = & \sum_{U \subseteq \set{X_1,\ldots,X_{m-1}}, U\neq \emptyset}c_U  h^U
\end{align*}
  %
for some coefficients $c_U \geq 0$, and note that
$\sum_U c_U = h'(X_1\ldots X_{m-1})=A$.  Define $h$ as follows:
  %
  \begin{align*}
    h = & \sum_{U \subseteq \set{X_1,\ldots,X_{m-1}}, U\neq \emptyset}c_U\left(1-\frac{a_m}{A}\right) h^U+ \sum_{U \subseteq \set{X_1,\ldots,X_{m-1}}, U\neq \emptyset}c_U\frac{a_m}{A} h^{U\cup\set{X_m}}
  \end{align*}
  %
  By assumption of the lemma $a_m \leq A$, which implies that all
  coefficients above are $\geq 0$, hence $h$ is a normal
  polymatroid. Furthermore, by direct calculations we check that, for
  $i<m$, $h(X_i) = h'(X_i) = a_i$ (because
  $h^{U\cup \set{X_m}}(X_i)=h^U(X_i)$), and, for $i<m$,
  $h(X_m) = a_m$, because $h^U(X_m) = 0$ and
  $h^{U\cup\set{X_m}}(X_m)=1$, and the claim follows from
  $\sum_U c_U = h'(X_1\cdots X_{m-1})=A$.  Finally, we also have
  $h(X_1\ldots X_m) = \sum_U c_U = A$, as required.
\end{proof}

Finally, we prove the Stitching Lemma~\ref{lemma:stich}.  For that we
need two propositions.

\begin{proposition} \label{prop:technical:1} Let $h : 2^V \rightarrow \R$
  be a normal polymatroid, and $V_0 \supseteq V$ a superset of
  variables.  Define $h': 2^{V_0}\rightarrow \R$ by
  $h'(U) \defeq h(U \cap V)$ for all $U \subseteq V_0$.  Then $h'$ is
  a normal polymatroid.  In other words, $h'$ extends $h$ to $V_0$ by
  simply ignoring the additional variables.
\end{proposition}

\begin{proof}
  We verify condition~\eqref{eq:normal} directly.  When
  $U \subseteq V$, then $h'(W)=h(W)$ for all $W \subseteq U$ and the
  condition holds because $h$ is a normal polymatroid.  When
  $U\not\subseteq V$, then we claim that
  $\sum_{W\subseteq U}(-1)^{|W|+1}h(W \cap V)=0$.  Indeed, fix a variable
  $X_i \in U$, $X_i \not\in V$, and pair every subset $W \subseteq U$
  that does not contain $X_i$ with $W' \defeq W \cup \set{X_i}$.  Then
  $h(W\cap V)=h(W'\cap V)$ and the two terms corresponding to $W$ and
  $W'$ in~\eqref{eq:normal} cancel out, proving that the
  expression~\eqref{eq:normal} is $=0$.
\end{proof}

\begin{proposition} \label{prop:technical:2} Let $h: 2^V \rightarrow \R$ be
  a normal polymatroid, and $Z\subseteq V$ a subset of variables.
  Define the following set functions $h', h'' : 2^V\rightarrow \R$:
%
  \begin{align}
\forall U \subseteq V: &&    h'(U) \defeq & h(U\cap Z) &  h''(U) \defeq & h(U|U\cap Z) \label{eq:hprime:normal}
  \end{align}
%
  (Recall that $h(B|A) = h(AB)-h(A)$.)  Then both $h', h''$ are normal
  polymatroids.
\end{proposition}

\begin{proof}
  We consider two cases as above.  When $U\subseteq Z$, then for all
  $W \subseteq U$ we have $h'(W)=h(W)$, and $h''(W)=0$:
  condition~\eqref{eq:normal} holds for $h'$ because it holds for $h$,
  and it holds for $h''$ trivially since it is $=0$.  No consider
  $U\not\subseteq Z$.  Then we claim that the
  expression~\eqref{eq:normal} for $h'$ is $0$:
%
  \begin{align*}
    \sum_{W \subseteq U} (-1)^{|W|+1}h'(W)=& \sum_{W \subseteq U} (-1)^{|W|+1}h(W\cap Z)=0
  \end{align*}
%
  We use the same argument as in the previous lemma: pick a variable
  $X_i$ s.t. $X_i \in U$ and $X_i \not\in Z$ and pair each set
  $W \subseteq U$ that does not contain $X_i$ with the set
  $W' \defeq W \cup \set{X_i}$.  Then $h(W\cap Z)=h(W'\cap Z)$ and two
  terms for $W$ and $W'$ cancel out.  Finally,
  condition~\eqref{eq:normal} for $h''$ follows similarly:
%
  \begin{align}
    \sum_{W\subseteq U}(-1)^{|W|+1}h''(W) = & \sum_{W\subseteq U}(-1)^{|W|+1}h(W|W\cap Z) =\underbrace{\sum_{W\subseteq U}(-1)^{|W|+1}h(W)}_{\geq 0}-\underbrace{\sum_{W\subseteq U}(-1)^{|W|+1}h(W\cap Z)}_{=0}
  \end{align}
%
  The first term is $\geq 0$ because $h$ is a normal polymatroid, and
  the second term is $=0$, as we have seen.
\end{proof}

Finally, we prove the Stitching Lemma~\ref{lemma:stich}.

\begin{proof}[Proof of Lemma~\ref{lemma:stich}] We first use the two
  propositions to show that $h'$ from Eq.~\eqref{eq:stich} is a normal polymatroid.  Define two
  helper functions $h'_1 : 2^{V_1} \rightarrow \R$ and
  $h'_2 : 2^{V_2} \rightarrow \R$:
%
  \begin{align*}
\forall U \subseteq V_1:\   h'_1(U) \defeq & h_1(U|U \cap Z) & 
\forall U \subseteq V_2:\   h'_2(U) \defeq & h_2(U|U \cap Z)
  \end{align*}
%
  By Lemma~\ref{prop:technical:2}, both $h'_1, h'_2$ are normal
  polymatroids.  Next, we extend $h'_1, h'_2, h$ to the entire set
  $V_1 \cup V_2$ by defining:
  \begin{align*}
  \forall U \subseteq V_1 \cup V_2:&& h''_1(U) \defeq &h'_1(U\cap V_1)  & h''_2(U) \defeq &h'_2(U\cap V_2)  & h''(U) \defeq & h(U\cap Z)
  \end{align*}
%
  By Lemma~\ref{prop:technical:1} each of them is a normal
  polymatroid.  Since $h'$ in the corollary is their sum, it is also a
  normal polymatroid.

  We check that it agrees with $h_1$ on $V_1$.  For any $U \subseteq
  V_1$, we have $h_2(U\cap V_2|U \cap Z) = 0$ therefore:
%
  \begin{align*}
    h'(U) = & h_1(U \cap V_1|U \cap Z) + h(U \cap Z)= h_1(U|U\cap Z)+h_1(U\cap Z) = h_1(U)
  \end{align*}
%
  Similarly, $h'$ agrees with $h_2$ on $V_2$.  Finally,
  condition~\eqref{eq:independent} follows by setting
  $U:= V_1 \cup V_2$ in~\eqref{eq:stich} and applying the definition
  of conditional: $h(B|A)=h(B)-h(A)$ when $A \subseteq B$.
\end{proof}


%%%%%%%%%%%%%%%%%%%%%%%%%%%%%%%%%%%%%%%%%%%%%%%%%%%%%%%%%%%%%%%%%%%%%%%%%%%%%%%%%%%%%%%%%%%%
\subsection{\lptd: Using Hypertree Decomposition}
\label{subsec:lptd}

The \lptdb algorithm is strictly limited by two requirements: $Q$
needs to be Berge-acyclic, and all statistics need to be full.  We
describe here a generalization of \lptdb, called \lptd, which drops
these two limitations.  When the restrictions of \lptdb are met, then
\lptd is slightly less efficient, however, its advantage is that it
can work on any query and constraints, without any restrictions.


A \emph{Hypertree Decomposition} of a full conjunctive query $Q$
 is a pair $(T,\chi)$, where $T$ is a tree and $\chi:
 \nodes(T)\rightarrow 2^V$ satisfying the following:
%
 \begin{itemize}
 \item For every variable $X_i$, the set
   $\setof{t \in \nodes(T)}{X_i \in \chi(t)}$ is connected.  This is
   called the \emph{running intersection property}.
 \item For every atom $R_j(V_j)$ of $Q$, $\exists t \in \nodes(T)$
   s.t. $V_j \subseteq \chi(t)$.
 \end{itemize}
%
 Each set $\chi(t) \subseteq V$ is called a \emph{bag}.  The
 \emph{width} of the tree $T$ is defined as
 $w(T) \defeq \max_{t \in \nodes(T)}|\chi(t)|$. We review a lemma by
 Lee~\cite{DBLP:journals/tse/Lee87}:

 \begin{lemma} \label{lemma:tony:lee} ~\cite{DBLP:journals/tse/Lee87}
   Let $(T,\chi:\nodes(T)\rightarrow 2^V)$ have the running
   intersection property and let $h : 2^V \rightarrow \R$ be a set
   function.  Define:
%
  \begin{align}
    E_{T,h} \defeq & \sum_{t \in \nodes(T)}h(\chi(t))-\sum_{(t_1,t_2)\in\edges(T)}h(\chi(t_1)\cap\chi(t_2)) \label{eq:et}
  \end{align}
%
  (1) If $h$ is a polymatroid (i.e. it satisfies the basic Shannon
  inequalities), then $E_{T,h} \geq h(V)$. (2) Suppose $h$ is the
  entropic vector defined by a uniform probability distribution on a
  relation $R(X_1, \ldots, X_n)$.  Then, $E_{T,h}=h(V)$ if for every
  $(t_1,t_2)\in \edges(T)$, the join dependency
  $R = \Pi_{V_1}(R) \Join \Pi_{V_2}(R)$ holds, where
  $V_1, V_2\subseteq V$ are the variables occurring on the two
  connected components of $T$ obtained by removing the edge
  $(t_1,t_2)$.
\end{lemma}

For a simple illustration, consider the 3-way join $J_3$
(Eq.~\eqref{eq:j3}).  Its tree decomposition $T$ has 3 bags
$XY, YZ, ZU$, and $E_{T,h} = h(XY)+h(YZ)+h(ZU)-h(Y)-h(Z)$; one can
check that $E_{T,h} \geq h(XYZU)$ using two applications of
submodularity.


Our new linear program, called \lptd, is constructed from a
hypertree decomposition $(T,\chi)$ of the query as follows:

\smallskip

\noindent {\bf The Real-valued Variables} are all expressions of the
form $h(U)$ for $U \subseteq \chi(t)$, $t \in \nodes(T)$.  The total
number of real-valued variable is $\sum_{t \in \nodes(T)}
2^{|\chi(t)|}$, i.e. it is exponential in the tree-width of the query.

\smallskip

\noindent {\bf The Objective} is to maximize $E_{T,h}$
(Eq.~\eqref{eq:et}), subject to the following constraints.

\smallskip

\noindent {\bf Statistics Constraints:}
All statistics constraints Eq.~\eqref{eq:h:p} of \lpbase.
Since we don't have numerical variable $h(U)$ for all $U$, we must
check that~\eqref{eq:h:p} uses only available numerical
variables.  This holds, because each statistics refers to some atom
$R_j(V_j)$, and there exists of some bag such that
$V_j \subseteq \chi(t)$, therefore we have numerical variables $h(U)$
for all $U \subseteq V_j$.

\smallskip

\noindent {\bf Normality Constraints:}
For each bag $\chi(t)$, \lptd contains all constraints of the
form~\eqref{eq:normal}.  In other words, the restriction of $h$ to
$\chi(t)$ is normal.


We prove:

\begin{theorem} \label{th:lptd} \lpbase and \lptd compute the same
  value.
\end{theorem}

The theorem holds only when all degree constraints used in the
statistics are \emph{simple}, as we assumed throughout this paper.
For a simple illustration, the \lptd for $J_3$ consists of 7 numerical
variables \\
$h(X),h(Y),h(Z),h(U),h(XY),h(YZ),h(ZU)$ (we omit
$h(\emptyset)=0$) and the following Normality Constraints:
%
\begin{align*}
  h(X)+h(Y)-h(XY)\geq & 0 & h(Y)+h(Z)-h(YZ)\geq & 0 \\
  h(Z)+h(U)-h(ZU) \geq & 0
\end{align*}



To compare \lptd and \lptdb, assume that the query $Q$ is Berge-acyclic
and all statistics are on simple and full degree constraints.  The
difference is that, for each atom $R_j(V_j)$, \lptdb has only $1+|V_j|$
real-valued variables and only $1+|V_j|$ additivity constraints, while
\lptdb has $2^{|V_j|}$ variables and normality constraints.

In the remainder of this section we prove Theorem~\ref{th:lptd}.

Denote by $b_{\text{base}}$ and $b_{\text{td}}$ the optimal solutions
of \lpbase and \lptd respectively.  We will prove that
$b_{\text{base}}=b_{\text{td}}$.

First, we claim that $b_{\text{base}}\leq b_{\text{td}}$.  It is known
from~\cite{DBLP:journals/pacmmod/KhamisNOS24} that \lpbase has an
optimal solution $h^*$ that is a normal polymatroid; thus
$b_{\text{base}}=h^*(V)$ (recall that $V$ is the set of all
variables), where $h^*$ is normal.  Then $h^*$ is also a feasible
solution to \lptd, therefore its optimal value is at least as large as
the value given by $h^*$, in other words
$b_{\text{td}}\geq E_{T,h^*}$.  By Lemma~\ref{lemma:tony:lee}, we have
$E_{T,h^*}\geq h^*(V) = b_{\text{base}}$, which proves the claim.


Second, we prove that $b_{\text{td}}\leq b_{\text{base}}$ by using the Stitching Lemma~\ref{lemma:stich}.
Let $h^*$ be an optimal solution to \lptd, thus
$b_{\text{td}}=E_{T,h^*}$.  The function $h^*$ is defined only on
subsets of the bags $\chi(t)$, $t \in \nodes(T)$, and on each such
subset, it is a normal polymatroid.  We extend it to a normal
polymatroid defined on all variables
$V = \bigcup_{t \in \nodes(T)}\chi(t)$ by repeatedly applying the
Stitching Lemma~\ref{lemma:stich}. Condition~\eqref{eq:independent} of
the corollary implies that this extension satisfies
$h^*(V)=E_{T,h^*}$.  Thus, $h^*$ is a normal polymatroid, and, hence,
a feasible solution to \lpbase.  It follows that the optimal value of
\lpbase is at least as large as that given by $h^*$, in other words
$b_{\text{base}} \geq h^*(V)$.  This completes the proof of the claim.

%%%%%%%%%%%%%%%%%%%%%%%%%%%%%%%%%%%%%%%%%%%%%%%%%%%%%%%%%%%%%%%%%%%%%%%%%%%%%%%%%%%%%%%%%%%%

\subsection{\lpflow: Missing Details from Section~\ref{subsec:lpflow}}
\label{subsec:lpflow:details}

Section~\ref{subsec:lpflow} gives the high-level idea of the \lpflow algorithm
using an example. We give here a more formal description of the algorithm
and prove Theorem~\ref{th:lpbase:eq:lpflow}.
The input to \lpflow is an arbitrary conjunctive query $Q$ of the form Eq.~\eqref{eq:cq}
(not necessarily a full query) and a set of statistics on the input database consisting of $\ell_p$-norms
of {\em simple} degree sequences, i.e. statistics of the form $\lp{\degree_{R_j}(V|U)}_p$ where $|U|\leq 1$.
For the purpose of describing \lpflow,
we construct a flow network $G=(\nodes,\edges)$ that is defined as follows:
(Recall that $\vars(Q) =\{X_1, \ldots, X_n\}$ is the set of variables of the query.)
\begin{itemize}
    \item The set of nodes $\nodes\subseteq 2^{\vars(Q)}$ consists of the following nodes:
    \begin{itemize}
        \item The node $\emptyset$, which is the source node of the flow network.
        \item A node $\{X_i\}$ for every variable $X_i \in \vars(Q)$.
        \item A node $UV$ for every statistics $\lp{\degree_{R_j}(V|U)}_p$. 
    \end{itemize}
    \item The set of edges $\edges$ consists of two types of edges:
    \begin{itemize}
        \item {\bf Forward edges:} These are edges of the form $(a, b)$
        where $a, b \in \nodes$ and $a \subset b$. Each such edge $(a, b)$ has a finite capacity $c_{a, b}$. In particular, for every statistics
        $\lp{\degree_{R_j}(V|U)}_p$, we have two forward edges:
        One edge $(\emptyset, U)$ and another $(U, UV)$. (Recall that $|U|\leq 1$.)
        \item {\bf Backward edges:} These are edges of the form $(a, b)$
        where $a, b \in \nodes$ and $b \subset a$. Each such edge $(a, b)$ has an infinite
        capacity $\infty$.
        In particular, for every statistics $\lp{\degree_{R_j}(V|U)}_p$ and
        every variable $X_i \in UV$, we have a backward edge $(UV, \{X_i\})$.
    \end{itemize}
\end{itemize}
We are now ready to describe the linear program for \lpflow. Recall that $V_0$ is the set of \groupby variables in the query $Q$ from Eq.~\eqref{eq:cq}.

\smallskip

\noindent {\bf The Real-valued Variables} are of two types:
\begin{itemize}
    \item Every statistics $\lp{\degree_{R_j}(V|U)}_p$ has an associated {\em non-negative}
    variable $w_{U,V,p}$.
    \item For every \groupby variable $X_i \in V_0$, we have a flow variable $f_{a, b; X_i}$
    for every edge $(a, b) \in \edges$.
\end{itemize}

\smallskip

\noindent {\bf The Objective} is to minimize the following sum over all available statistics
$\lp{\degree_{R_j}(V|U)}_p$:
\begin{align}
    \sum w_{U,V,p} \cdot \log \lp{\degree_{R_j}(V|U)}_p
    \label{eq:lpflow:objective}
\end{align}

\smallskip

\noindent {\bf The Constraints} are of two types:
\begin{itemize}
    \item {\em Flow conservation constraints:} For every \groupby variable $X_i \in V_0$,
    the variables $f_{a, b;X_i}$ must define a valid flow from the source node
    $\emptyset$ to the sink node $\{X_i\}$ that has a capacity $\geq 1$.
    This means that for every node $c \in \nodes - \{\emptyset\}$, we must have:
    \begin{align}
        \sum_a f_{a, c;X_i} -\sum_b f_{c, b; X_i} \geq 1, & \quad\quad\text{ if $c =\{X_i\}$}\label{eq:flow:conservation:1}\\
        \sum_a f_{a, c;X_i} -\sum_b f_{c, b; X_i} \geq 0, & \quad\quad\text{ otherwise}
        \label{eq:flow:conservation:2}
    \end{align}
    \item {\em Flow capacity constraints:} For every \groupby variable $X_i \in V_0$
    and every {\em forward} edge $(a, b)$, the flow variable $f_{a, b;X_i}$ must satisfy:
    \begin{align}
        f_{a, b;X_i} \leq c_{a, b} \label{eq:flow:capacity}
    \end{align}
    where $c_{a, b}$ is the {\em capacity} of the forward edge $(a, b)$.
    (Recall that backward edges have infinite capacity.)
    The capacity variables $c_{a, b}$ are determined by the statistics constraints.
    In particular, every statistics $\lp{\degree_{R_j}(V|U)}_p$ contributes a capacity
    of $w_{U,V,p}$ to $c_{U,UV}$ and a capacity of $\frac{w_{U,V,p}}{p}$ to $c_{\emptyset,U}$.
    Formally,
    \begin{align}
        c_{\emptyset,U} &\defeq
        \sum_{p} w_{\emptyset,U,p}
        +\sum_{V,p} \frac{w_{U,V,p}}{p}\label{eq:lpflow:c}\\
        c_{U, UV} &\defeq
        \sum_{p} w_{U,V,p} &\text{ if $U \neq \emptyset$}\nonumber
    \end{align}
\end{itemize}
\smallskip



We are now ready to prove Theorem~\ref{th:lpbase:eq:lpflow}, which says that the linear programs
for \lpbase and \lpflow have the same optimal value.
To that end, we first write the dual LP for \lpbase.
For every statistics constraint of the form Eq.~\eqref{eq:h:p}, we introduce a dual variable $w_{U,V,p}$. The dual of \lpbase is equivalent to:
\begin{align}
    \min\quad &\sum w_{U,V,p} \cdot \log \lp{\degree_R(V|U)}_p\label{eq:lpbase:dual}\\
    \text{s.t.}\quad& \text{The following is a valid Shannon inequality:}\nonumber\\
    &h(V_0) \leq \sum w_{U,V,p} \left(\frac{1}{p}h(U)+h(V|U)\right)\label{eq:lpflow:shannon}\\
    & w_{U,V,p} \geq 0\nonumber
\end{align}
Inequality~\eqref{eq:lpflow:shannon} satisfies the property that for every $h(V|U)$
on the RHS, we have $|U| \leq 1$. In order to check that such an inequality is a valid
Shannon inequality, we rely on a key result from~\cite{DBLP:journals/corr/abs-2211-08381}.
In particular,~\cite{DBLP:journals/corr/abs-2211-08381} is concerned
with Shannon inequalities of the following form.
Let $\calX=\{X_1, \ldots, X_n\}$ be a set of variables, and $\calC$
be a set of distinct pairs $(U, V)$ where $U, V\subseteq \calX$, $U \cap V = \emptyset$
and $|U| \leq 1$.
For every pair $(U,V)\in\calC$, let $c_{U,UV}$ be a non-negative constant.
Moreover, let $V_0$ be a subset of $\calX$.
Consider the following inequality:
\begin{align}
    h(V_0) \leq \sum_{(U,V)\in\calC} c_{U, UV} h(V|U)
    \label{eq:lpflow:shannon:general}
\end{align}
\cite{DBLP:journals/corr/abs-2211-08381} describes a reduction from
the problem of checking whether Eq.~\eqref{eq:lpflow:shannon:general} is a valid Shannon inequality to a collection of $|V_0|$ network flow problems.
These flow problems are over the same network $G=(\nodes, \edges)$,
which is similar to the flow network described above for \lpflow. In particular,
\begin{itemize}
    \item The nodes are $\emptyset$, $\{X_i\}$ for each variable $X_i \in \calX$, and $UV$ for each $(U,V)\in\calC$.
    \item The edges have two types:
    \begin{itemize}
        \item Forward edges: For each $(U,V)\in\calC$, we have a forward edge $(U,UV)$ with capacity $c_{U,UV}$.
        \item Backward edges: For each $(U,V)\in\calC$ and each variable $X_i \in UV$,
        we have a backward edge $(UV, \{X_i\})$ with infinite capacity.
    \end{itemize}
\end{itemize}
\begin{lemma}[\cite{DBLP:journals/corr/abs-2211-08381}]
    Inequality~\eqref{eq:lpflow:shannon:general} is a valid Shannon inequality if and only if
    for each variable $X_i \in V_0$, there exists a flow $\left(f_{a, b;X_i}\right)_{(a, b)\in\edges}$ from the source node $\emptyset$
    to the sink node $\{X_i\}$ with capacity at least $1$.
    In particular, the flow variables $\left(f_{a, b;X_i}\right)_{(a, b)\in\edges}$ must satisfy the flow conservation constraints~\eqref{eq:flow:conservation:1} and~\eqref{eq:flow:conservation:2} and the flow capacity constraints~\eqref{eq:flow:capacity}.
\end{lemma}
Using the above lemma, we can prove Theorem~\ref{th:lpbase:eq:lpflow} as follows.
Take inequality~\eqref{eq:lpflow:shannon} and group together identical conditionals
on the RHS in order to convert it into the form of Eq.~\eqref{eq:lpflow:shannon:general}.
The coefficients $c_{U,UV}$ of the resulting Eq.~\eqref{eq:lpflow:shannon:general}
will be identical to those defined by Eq.~\eqref{eq:lpflow:c}.
Then, we can apply the lemma to check the validity of Eq.~\eqref{eq:lpflow:shannon}.
But now, the dual LP~\eqref{eq:lpbase:dual} for \lpbase is equivalent to the linear program for \lpflow.

%%%%%%%%%%%%%%%%%%%%%%%%%%%%%%%%%%%%%%%%%%%%%%%%%%%%%%%%%%%%%%%%%%%%%%%%%%%%%%%%%%%%%%%%%%%%
\nop{
\subsection{Examples: Handling predicates in \system}
\label{app:predicates-examples}

We show how \system handles predicates in the following examples.
Figure~\ref{fig:predicate_example} (left) shows a simple example of a relation $R(X,A,B)$,
where $X$ is a join attribute and $A$ and $B$ are predicate attributes.
The degree sequence for the relation $R$ with respect to the join attribute $X$ is $\deg_{R}(* | X) = (3,2)$. The $\ell_1$ and $\ell_{\infty}$-norm of the degree sequence $(3,2)$ are $3+2 = 5$ and $\max(3,2) = 3$, respectively.

For queries with predicates on attributes $A$ and $B$, using the $\ell_p$-norms of the degree sequence for the entire relation can lead to overestimation, since only a subset of tuples satisfy the predicates.
We show that how to compute tighter bounds of the $\ell_p$-norms for queries with predicates following our discussion in Section~\ref{sec:histograms}.




\begin{figure}[h]
  \centering
  \begin{minipage}[b]{0.50\textwidth}
      \centering
      $R$ \\[0.2em]
      \begin{tabular}{|c|c|c|}
          \hline
          $X$ & $A$ & $B$\\
          \hline
          1 & 1 & 0\\
          1 & 1 & 8\\
          1 & 2 & 13\\
          2 & 3 & 22\\
          2 & 4 & 40\\
          \hline
      \end{tabular}
  \end{minipage}
  % \hfill
  % ex 2
  \begin{minipage}[b]{0.16\textwidth}
      \centering
      $T$ \\[0.2em]
      \begin{tabular}{|c|c|}
          \hline
          $TID$ & $SID$ \\
          \hline
          1 & 1\\
          2 & 2\\
          3 & 2\\
          4 & 3\\
          5 & 3\\
          \hline
      \end{tabular}
  \end{minipage}
  \begin{minipage}[b]{0.16\textwidth}
    \centering
    $S$ \\[0.2em]
    \begin{tabular}{|c|c|}
        \hline
        $SID$ & $A$ \\
        \hline
        1 & 1\\
        2 & 1\\
        3 & 2\\
        \hline
    \end{tabular}
  \end{minipage}
  \begin{minipage}[b]{0.16\textwidth}
    \centering
    $TS$ \\[0.2em]
    \begin{tabular}{|c|c|c|}
        \hline
        $TID$ & $SID$ & $A$ \\
        \hline
        1 & 1 & 1\\
        2 & 2 & 1\\
        3 & 2 & 1\\
        4 & 3 & 2\\
        5 & 3 & 2\\
        \hline
    \end{tabular}
  \end{minipage}
  \caption{Left: relation $R(X,A,B)$, where $X$ is the join attribute and $A$ and $B$ are predicate attributes. Right: Right: relations $T(TID, SID)$ and $S(SID, A)$ where $SID$ is a foreign key in $T$ and the primary key in $S$, and $TS$ is the join of $T$ and $S$.}
  \label{fig:predicate_example}
\end{figure}

\paragraph{Equality Predicate.} 
Consider an equality predicate on $A$.
The $A$-values in $R$, sorted by frequency in descending order, are $(1,2,3,4)$. 
For this example, we consider the most common value $A=1$ as the only MCV for $A$.
We fetch the tuples satisfying $A=1$, which are the first two tuples, and get the degree sequence $\deg_{R}(* | X, A=1) = (2)$.
The $\ell_p$-norm of the degree sequence is $2$ for any $p\geq 1$.

We also compute one degree sequence for all non-MCVs of $A$, i.e., $A\in\{2,3,4\}$.
We fetch the last three tuples in $R$ and compute the degree sequence $\deg_{R}(* | X, A\in \{2, 3, 4\}) = (2, 1)$.
For an arbitrary non-MCV, there is at most one distinct $X$-value in $R$ associated with it,
so we take the top value of $\deg_{R}(* | X, A\in \{2, 3, 4\})$ as the degree sequence, i.e., $(2)$, for all non-MCVs of $A$.
The $\ell_p$-norm of the degree sequence is $2$ for any $p\geq 1$.

An alternative and more accurate, yet more expensive approach is to compute the degree sequence for each non-MCV of $A$: $\deg_{R}(* | X, A=2) = (1)$, $\deg_{R}(* | X, A=3) = (1)$, and $\deg_{R}(* | X, A=4) = (1)$, and then compute the maximum of their $\ell_p$-norms.
The $\ell_p$-norms of these degree sequences are $1$, so the maximum of the $\ell_p$-norms is $1$, for any $p\geq 1$.

To estimate for a query with the predicate $A=1$, we use the $\ell_p$-norms for the MCV of $A$, i.e., $\ell_p = 2$. For a query with the predicate $A=2$, where $2$ is a non-MCV of $A$, we use the $\ell_p$-norms for non-MCVs of $A$, i.e., $\ell_p = 2$ or, if we use the more accurate alternative, $\ell_p = 1$.

\paragraph{Range Predicate.}
Consider a range predicate on $B$
The domain of $B$ is $[0, 40]$. We create a hierarchy of histograms with $2^k, 2^{k-1}, \ldots, 2^0$ buckets.
For this example, we use $k=2$, which means creating $4$ buckets for the bottom layer, $2$ buckets for the next layer, and one bucket for the entire domain.
The buckets for the layers are $\{[0, 10), [10, 20), [20, 30), [30, 40]\}$,  $\{[0, 20), [20, 40]\}$, and $\{[0,40]\}$.

We first construct the $\ell_p$-norms within each bucket $[s, e)$.
For this, we fetch the tuples where $B$ is in the bucket
and compute the degree sequence $\deg_{R}(* | X, B \in [s, e))$ and several $\ell_p$-norms on this degree sequence.
The degree sequences for the buckets in the layers are $((2), (1), (1), (1))$, then $((3), (2))$, and finally $(5)$. The $\ell_p$-norms within each bucket 


thus their $\ell_1$ and $\ell_{\infty}$-norms are 
$\{(\ell_1=\ell_{\infty}=2), (\ell_1=\ell_{\infty}=1), (\ell_1=\ell_{\infty}=1), (\ell_1=\ell_{\infty}=1)\}$ and $\{(\ell_1=\ell_{\infty}=3), (\ell_1=\ell_{\infty}=2)\}$, respectively.

To estimate for the range predicate $5 \leq B \leq 18$, we first find the smallest bucket that covers the range, which is the bucket $[0, 20)$, and use the corresponding $\ell_p$-norms: $\ell_1 = 2$ and $\ell_{\infty} = 2$.


\paragraph{Conjunction and Disjunction of Predicates.}
We show how \system handles the conjunction and disjunction of predicates on $A$ and $B$.
Consider the predicates $A = 0$ and $5 \leq B \leq 18$.
We fetch the $\ell_p$-norms for the two predicates as discussed in the previous examples: $\ell_1 = 1$ and $\ell_{\infty} = 1$ for $A = 1$, and $\ell_1 = 2$ and $\ell_{\infty} = 2$ for $5 \leq B \leq 18$.

For the conjunction of the predicates, we take the minimum of these $\ell_p$-norms to estimate the query. The result is $\ell_1 = \min(1, 2) = 1$ and $\ell_{\infty} = \min(1, 2) = 1$.

For the disjunction of the predicates, we take the sum of the $\ell_1$-norms and the maximum of the $\ell_{\infty}$-norms, which results in $\ell_1 = 1+2 = 3$ and $\ell_{\infty} = \max(1, 2) = 2$.



\paragraph{Optimization 1: Predicate Propagation via FK-PK Joins.}
Consider two relations $T(TID, SID)$ and $S(SID, A)$ in Figure~\ref{fig:predicate_example} (right),
 where $SID$ is a foreign key in $T$ and a primary key in $S$, and $A$ is an equality predicate attribute.
We compute the $\ell_p$-norms for predicates on $A$ in $S$ as discussed in the previous examples: $\ell_1 = 2$ and $\ell_{\infty} = 1$ for the MCV $A=1$, and $\ell_1 = 1$ and $\ell_{\infty} = 1$ for non-MCVs of $A$.

For relation $R$, we apply the optimization to propagate the predicate on $A$ from $S$ to $T$:
We precompute the join results for the FK-PK join $TS(TID, SID, A) = T(TID, SID) \wedge S(SID, A)$ (Figure~\ref{fig:predicate_example} (right)). The size of the join results is bounded by the size of the FK relation $T$.
For this example, we consider only one MCV of $A$, so the only MCV of $A$ is $A=1$.
We fetch the tuples satisfying $A=1$ in $TS$, which are the first two tuples in $TS$, and compute the degree sequence $\deg_{TS}(* | TID, A=1) = (1,1,1)$.
The $\ell_p$-norms of the degree sequence are $\ell_1 = 1+1+1 = 3$ and $\ell_{\infty} = 1$.
Regarding the non-MCVs of $A$, there is only one non-MCV of $A$, which is $A=2$. We compute the degree sequence for $A=2$ in $TS$, i.e., $\deg_{TS}(* | TID, A=2) = (1,1)$ and the $\ell_p$-norms for the degree sequence are $\ell_1 = 1+1 = 2$ and $\ell_{\infty} = 1$.

Consider a query with predicate $A=1$.
For relation $S$, we use the $\ell_p$-norms for the MCV $A=1$ in $S$, i.e., $\ell_1 = 2$ and $\ell_{\infty} = 1$.
For relation $T$, we use the $\ell_p$-norms for the MCV $A=1$ in the join result $TS$, i.e., $\ell_1 = 3$ and $\ell_{\infty} = 1$.

\paragraph{Optimization 2: Compute $\ell_p$-norms for Prefixes of the Degree Sequence.}
Consider two relations $R(X,A)$ and $S(X,B)$ where $X$ is a join attribute, and their degree sequences are $(100, 99, \ldots, 2, 1)$ and $(2, 1)$, respectively.
This means that there are $100$ distinct $X$-values in $R$ and $2$ distinct $X$-values in $S$, which is significantly mis-calibrated.
When the two relations are joined, at most two $X$-values appear in the join results.
If we use the $\ell_p$-norms of the whole degree sequence for $R$, which are $\ell_1 = 5050$ and $\ell_{\infty} = 100$, the estimation can be significantly overestimated.

We reduce overestimation by computing the $\ell_p$-norms for prefixes of the degree sequence for $R$.
We compute the $\ell_p$-norms for the top-$2^i$ values of the degree sequence for $R$ for $i>0$. 
For example, for $i=1$, we compute the $\ell_p$-norms for the top-$2$ values of the degree sequence, which are $(100, 99)$: $\ell_1 = 100+99 = 199$ and $\ell_{\infty} = 100$.
For the join of $R$ and $S$, since there are at most two $X$-values in the join results, we can use these $\ell_p$-norms for the estimation.
}

%%%%%%%%%%%%%%%%%%%%%%%%%%%%%%%%%%%%%%%%%%%%%%%%%%%%%%%%%%%%%%%%%%%%%%%%%%%%%%%%%%%%%%%%%%%%



% moved to the main body
\nop{
%%%%%%%%%%%%%%%%%%%%%%%%%%%%%%
\begin{figure*}[t]
    \centering
    \begin{minipage}[b]{0.48\textwidth}
        \centering
        \includegraphics[width=\textwidth]{experiments/overall_runtime.pdf}
    \end{minipage}
    \hfill
    \begin{minipage}[b]{0.48\textwidth}
        \centering
        \includegraphics[width=\textwidth]{experiments/relative_runtime.pdf}
    \end{minipage}
    \caption{Left: Overall evaluation time of all queries in a benchmark for \psql when using estimates for all subqueries from \system, \safebound, \dbx, \psql and true cardinalities. Right: Relative evaluation times compared to the baseline evaluation time obtained when using true cardinalities.}
    \label{fig:runtime}
\end{figure*}
%%%%%%%%%%%%%%%%%%%%%%%%%%%%%%


\subsection{Further Experiments}
\label{app:further-experiments}

We complement the experiments in Section~\ref{sec:experiments} with further experiments that cannot be accommodated in the main body due to lack of space.

\subsubsection{Estimation Errors}

Fig.~\ref{fig:estimates-STATS} shows that the accuracy of the estimators decreases with the number of relations per query (shown for STATS, a similar trend also holds for JOBlight and JOBrange): The traditional estimators underestimate more, whereas the pessimistic estimators overestimate more. \neurocard starts with a large overestimation for a join of two relations and decreases its estimation as we increase the number of relations; the other ML-based estimators follow this trend but at a smaller scale.


\subsubsection{Optimization Improvements}
Fig.~\ref{fig:improvements-optimizations} shows the improvements to the estimation accuracy brought by each of the two optimizations discussed in Sec.~\ref{sec:histograms}, when taken in isolation.

The left figure shows that,  when propagating predicates from the primary-key relation to the foreign-key relations, the estimation error can improve by over an order of magnitude in the worst case (corresponding to the upper dots in the plot) and by roughly 5x in the median case (corresponding to the red line in the boxplots). 

The right figure shows that,  when using prefix degree sequences for the degree sequences of relations without predicates, the estimation error can improve by up to 50\% for JOBlight queries, up to 65\% for JOBrange queries and up to 10\% for STATS queries. The improvement is measured as the division of (i) the difference between the estimation error without this optimization and the estimation error with this optimization and (2) the the estimation error without this optimization.




\subsubsection{Evaluation Times}




Fig.~\ref{fig:runtime} (left) shows the aggregated \psql evaluation time of all queries in JOBlight, JOBrange, and STATS when using estimates for all sub-queries from \system, \safebound, \dbx, \psql, and true cardinalities (left).
Fig.~\ref{fig:runtime} (right)  shows the relative evaluation times compared to the baseline evaluation time obtained when using true cardinalities. 
We have two observations. First, overestimation can be beneficial for performance of expensive queries, which has been discussed in Section~\ref{sec:experiments}. Second, overestimation can be detrimental for performance of less expensive queries in some cases.

The first observation is reflected in the overall evaluation times, which are dominated by the most expensive queries in the benchmark (some of which are listed in Fig.~\ref{fig:most-expensive-queries}). 
Traditional approaches lead to higher evaluation times for the expensive queries, and therefore to higher overall evaluation times, while the pessimistic approaches lead to lower evaluation times for those expensive queries. Overall, the  evaluation times for the pessimistic approaches are about the same (JOBlight and STATS) or lower (JOBrange) than the baseline evaluation times.
The second observation is reflected in the relative evaluation times for the JOB benchmarks. 
The boxplots for the traditional approaches are lower than those for the pessimistic approaches, indicating that the traditional approaches perform better for the less expensive queries in the benchmarks.

\factorjoin has both high overall evaluation time and high relative evaluation time.
It estimates very accurately for the queries in STATS, thus has similar evaluation time to the baseline evaluation time. For the queries in JOBlight and JOBrange, it mostly overestimates, which leads to lower evaluation times for the expensive queries. However, the overestimations are significant, which makes it perform worse than the pessimistic approaches for the less expensive queries, as shown in the right plot of Fig.~\ref{fig:runtime}. This leads to the high overall evaluation time of \factorjoin.

}

\nop{
\subsection{About Traditional Estimators}

% PostgreSQL
\subsubsection{\psql}
\psql uses four types of statistics to estimate cardinalities:
\begin{itemize}
    \item cardinality of each relation (row count), and the number of pages per relation
    \item number of distinct values for each attribute
    \item MCVs for each attribute and their relative frequencies, i.e., the estimated fraction of rows that have the MCV as the respective attribute value
    \item histogram if the values in each attribute and relative frequency of each histogram bucket
\end{itemize}

The first statistic, the cardinality of each relation, is very accurate. The other statistics, however, are estimated based on a sample of the data. The default sampling size is 30,000 rows. We found that those statistics are often times inaccurate. For example, for some join attributes of the IMDB dataset, the domain size was underestimated by 70\%.

While the statistics mentioned above are computed by default, \psql can be instructed to compute multivariate statistics capturing correlations between attributes of the same relation via the \texttt{CREATE STATISTICS} command.

\paragraph{Selectivity of Equality Predicates and Join Conditions}

\psql uses the concept of {\em selectivity} of a (filter or join) condition to estimate the cardinality of a query output. If the predicate value is a MCV, then \psql considers the relative frequency of this value as its selectivity. If the value is not an MCV, then \psql either use histograms to estimate the frequency of the value in the relation, or it falls back to a default estimate, such as assuming a uniform distribution. In the latter case, the selectivity is assumed to be the estimated domain size of the attribute divided by cardinality of relation. Correlations of attributes across relations are not considered. For join conditions, \psql assumes that the join attributes are independent.

Consider the following join of the two relation $R(A,C)$ and $S(B,C)$ with two equality predicates on the non-join attributes.
\begin{verbatim}
    SELECT * FROM R, S WHERE R.A = 5 AND S.B = 10 AND R.C = S.C;
\end{verbatim}
The estimated cardinality of this query is:
\begin{align*}
    |R| * |S| * \sel{R.A = 5} * \sel{S.B = 10} * \sel{R.C = S.C}.
\end{align*}

\paragraph{Range Conditions}

\subsubsection{\duckdb}
% DuckDB
}


\end{document}

%%%%%%%%%%%%%%%%%%%%%%%%%%%%%%%%%%%%%%%%%%%%%%%%%%%%%%%%%%%%%%%%%%%%%%%%

