%%%%%%%%%%%%%%%%%%%%%%%%%%%%%%%%%%%%%%%%%%%%%%%%%%%%%%%%%%%%%%%%%%%%%%%%

%%% LaTeX Template for AAMAS-2025 (based on sample-sigconf.tex)
%%% Prepared by the AAMAS-2025 Program Chairs based on the version from AAMAS-2025. 

%%%%%%%%%%%%%%%%%%%%%%%%%%%%%%%%%%%%%%%%%%%%%%%%%%%%%%%%%%%%%%%%%%%%%%%%

%%% Start your document with the \documentclass command.


%%% == IMPORTANT ==
%%% Use the first variant below for the final paper (including auithor information).
%%% Use the second variant below to anonymize your submission (no authoir information shown).
%%% For further information on anonymity and double-blind reviewing, 
%%% please consult the call for paper information
%%% https://aamas2025.org/index.php/conference/calls/submission-instructions-main-technical-track/

%%%% For anonymized submission, use this
%\documentclass[sigconf,anonymous]{aamas} 

%%%% For camera-ready, use this
\documentclass[sigconf]{aamas} 


%%% Load required packages here (note that many are included already).

\usepackage{balance} % for balancing columns on the final page
\usepackage[ruled,noend]{algorithm2e}

\newcommand{\R}{{\mathbb R}}
%\newcommand{\C}{{\mathbb C}}
\newcommand{\Z}{{\mathbb Z}}
\newcommand{\N}{{\mathbb N}}

\usepackage{fancyhdr}
\pagestyle{fancy}
\lhead{}
\rhead{}
\chead{}
\lfoot{}
\rfoot{}
\cfoot{\thepage\\
 \begin{small} \textbf{DISTRIBUTION STATEMENT A.} Approved for public release; distribution is unlimited. \end{small}}
\renewcommand{\headrulewidth}{0pt}
\renewcommand{\footrulewidth}{0pt}


\newcommand{\thought}[1]{{\color[rgb]{0.2,0.39,0.66}(#1)}}
\newcommand{\todo}[1]{{\color[rgb]{1.0,0.0,0.0}(#1)}}
\newcommand{\hsh}[1]{{\color{green!50!black} Henrik: #1}}
\newcommand{\st}[1]{{\color{red!50!black} Sebastian: #1}}

\newcommand{\ulm}[1]{_{\scaleto{\mathrm{#1}}{3pt}}}
\newcommand\at[2]{\left.#1\right|_{#2}}











\newtheorem{assumption}{Assumption}

\DeclareMathOperator*{\argmax}{arg\,max}
\DeclareMathOperator*{\argmin}{arg\,min}

\newcommand{\swname}[1]{\texttt{#1}}
\newcommand{\ie}{i\/.\/e\/.,\/~}
\newcommand{\eg}{e\/.\/g\/.,\/~}
\newcommand{\cf}{cf\/.\/~}

\newcommand{\fig}{Fig\/.\/~}
\newcommand{\defn}{Def\/.\/~}
\newcommand{\sect}{Sec\/.\/~}
\newcommand{\tabl}{Tab\/.\/~}
\newcommand{\algo}{Algorithm~}
\newcommand{\theo}{Theorem~}

\newcommand{\bnnl}{3 hidden layers}
\newcommand{\bnnn}{50 neurons}
\newcommand{\bnna}{tanh activations}

\newcommand{\capt}[1]{\mdseries{\emph{#1}}}

\newcommand{\videolink}{at \url{https://youtu.be/_d7AqTRjz6g}}
\newcommand{\codelink}{\url{https://github.com/wheelbot/mini-wheelbot}}

\newcommand{\fakepar}[1]{\vspace{0mm}\noindent\textbf{#1.}}

\newcommand{\needref}{\textcolor{red}{[REF]}}

\newcommand{\plotfontsize}{9pt}

\usepackage{enumitem}
%%%%%%%%%%%%%%%%%%%%%%%%%%%%%%%%%%%%%%%%%%%%%%%%%%%%%%%%%%%%%%%%%%%%%%%%

%%% AAMAS-2025 copyright block (do not change!)

%%% AAMAS-2025 copyright block (do not change!)

\makeatletter
\gdef\@copyrightpermission{
  \begin{minipage}{0.2\columnwidth}
   \href{https://creativecommons.org/licenses/by/4.0/}{\includegraphics[width=0.90\textwidth]{by}}
  \end{minipage}\hfill
  \begin{minipage}{0.8\columnwidth}
   \href{https://creativecommons.org/licenses/by/4.0/}{This work is licensed under a Creative Commons Attribution International 4.0 License.}
  \end{minipage}
  \vspace{5pt}
}
\makeatother

\setcopyright{ifaamas}
\acmConference[AAMAS '25]{Proc.\@ of the 24th International Conference
on Autonomous Agents and Multiagent Systems (AAMAS 2025)}{May 19 -- 23, 2025}
{Detroit, Michigan, USA}{Y.~Vorobeychik, S.~Das, A.~Nowé  (eds.)}
\copyrightyear{2025}
\acmYear{2025}
\acmDOI{}
\acmPrice{}
\acmISBN{}


%%%%%%%%%%%%%%%%%%%%%%%%%%%%%%%%%%%%%%%%%%%%%%%%%%%%%%%%%%%%%%%%%%%%%%%%

%%% == IMPORTANT ==
%%% Use this command to specify your EasyChair submission number.
%%% In anonymous mode, it will be printed on the first page.

\acmSubmissionID{950}

%%% Use this command to specify the title of your paper.

\title[AAMAS-2025 Formatting Instructions]{
%A Bayesian Curriculum for Online Task Transfer \\
%Automated Curriculum Generation for Reinforcement Learning \\
%Automated Bayesian Curriculum Generation\\
%A Bayesian Approach for Automated Curricula\\
%A Self-Guided Bayesian Curriculum for Reinforcement Learning\\
%An Automated Bayesian Curriculum for Reinforcement Learning\\
Automating Curriculum Learning for Reinforcement Learning using a Skill-Based Bayesian Network
% \\ commented-out to fix a compilation error -- Dana Nau
}
%https://aamas2025.org/index.php/conference/calls/submission-instructions-main-technical-track/

%%% Provide names, affiliations, and email addresses for all authors.

\author{Vincent Hsiao}
\affiliation{
  \institution{NRC Postdoctoral Fellow, \\Naval Research Laboratory}
  \city{Washington DC}
  \country{United States}}
\email{vincent.hsiao.ctr@us.navy.mil}

\author{Mark Roberts}
\affiliation{
  \institution{Naval Research Laboratory}
  \city{Washington DC}
  \country{United States}}
\email{mark.c.roberts20.civ@us.navy.mil}

\author{Laura M. Hiatt}
\affiliation{
  \institution{Naval Research Laboratory}
  \city{Washington DC}
  \country{United States}}
\email{laura.m.hiatt.civ@us.navy.mil}

\author{George Konidaris}
\affiliation{
    \institution{Brown University}
    \city{Providence RI}
    \country{United States}}
\email{gdk@brown.edu}

\author{Dana Nau}
\affiliation{
    \institution{University of Maryland}
    \city{College Park, MD}
    \country{United States}}
\email{nau@umd.edu}



%%% Use this environment to specify a short abstract for your paper.
    
\begin{abstract}
%typically focuses on improving the performance on some target task by training on a sequence of similar tasks.
A major challenge for reinforcement learning is automatically generating curricula to reduce training time or improve performance in some target task.
We introduce SEBNs (Skill-Environment Bayesian Networks)
which model a probabilistic relationship between a set of skills, a set of goals that relate to the reward structure, and a set of environment features to predict policy performance on (possibly unseen) tasks.  
We develop an algorithm that uses the inferred estimates of agent success from SEBN to weigh the possible next tasks by expected improvement.
We evaluate the benefit of the resulting curriculum on three environments: a discrete gridworld, continuous control, and simulated robotics.
The results show that curricula constructed using SEBN frequently outperform other baselines.
\end{abstract}

%%% The code below was generated by the tool at http://dl.acm.org/ccs.cfm.
%%% Please replace this example with code appropriate for your own paper.


%%% Use this command to specify a few keywords describing your work.
%%% Keywords should be separated by commas.

\keywords{Bayesian Networks, 
Automated Curriculum Generation,
Reinforcement Learning,
Transfer Learning}

%%%%%%%%%%%%%%%%%%%%%%%%%%%%%%%%%%%%%%%%%%%%%%%%%%%%%%%%%%%%%%%%%%%%%%%%

%%% Include any author-defined commands here.
         
\newcommand{\BibTeX}{\rm B\kern-.05em{\sc i\kern-.025em b}\kern-.08em\TeX}

%%%%%%%%%%%%%%%%%%%%%%%%%%%%%%%%%%%%%%%%%%%%%%%%%%%%%%%%%%%%%%%%%%%%%%%%

\begin{document}

%%% The following commands remove the headers in your paper. For final 
%%% papers, these will be inserted during the pagination process.

\pagestyle{fancy}
\fancyhead{}


%%% The next command prints the information defined in the preamble.

\maketitle 
%%%%%%%%%%%%%%%%%%%%%%%%%%%%%%%%%%%%%%%%%%%%%%%%%%%%%%%%%%%%%%%%%%%%%%%%

%\frommak{ {\color{red} \begin{small} 8-pages + references + supplemental;     Abstract:10/9 AOE  Paper:10/16 AOE \end{small} }}

\fancypagestyle{firststyle}
{
\cfoot{ \begin{small} \textbf{DISTRIBUTION STATEMENT A.} Approved for public release; distribution is unlimited. \end{small}}
}
\thispagestyle{firststyle}


%=====================================================
%=====================================================
%=====================================================
%=====================================================
\section{Introduction}
Adapting skills to new or unseen tasks is a major challenge in Reinforcement Learning (RL). Curriculum Learning \cite{bengio2009curriculum, narvekarEtAl20.jmlr.clForRL}, an approach for training agents using a sequence of increasingly difficult environments, often promotes the effective development of policies with more robust capabilities. However, customizing a curriculum to a particular student often requires substantial human insight and oversight. This is especially challenging for robotics, where the environment or tasks that need to be performed can change frequently. An ideal solution to this problem would be an automated curriculum that enables the robot to discern for itself when it needs to adapt, how long should train, and in what environments. 

Past work on automated curriculum generation such as \cite{stout2010competence, kumar2024practice} has primarily focused on choosing what skills to train while holding the environment itself static. More recent approaches that build a curriculum over different environments such as \cite{parker2022evolving} do not consider agent skill competencies. Furthermore, these environment-based approaches require explicit evaluation on an environment before being able to calculate an estimate of agent success or regret to add those environments to the curriculum. 

We address the aforementioned issues by introducing Skill Environment Bayesian Networks (SEBNs) as a potential method for estimating agent competency level and selecting the most appropriate environments for training. 
%SEBNs provide a natural way to incorporate options. \fromvincent{still thinking of a way to write this, maybe this doesn't need to be mentioned in the introduction...}
SEBNs model a probabilistic relationship between these goals, (latent) competencies, and environment features using data from past rollouts. Using this model, we can estimate agent success rates on new (possibly unseen) environments. We use these estimates to select the next set of training tasks within a curriculum in what we call an SEBN-guided automated curriculum. Importantly, SEBN does not require explicit evaluation on each possible environment to estimate agent success. 


The contributions of the paper include:
    (1) Introducing and formalizing the SEBN for skills, task features, and reward structure;
    (2) Providing an algorithm for constructing curricula using SEBNs; 
    (3) Introducing  Megagrid, a gridworld environment that simplifies generating partially-specified environments for transfer learning; 
    (4) Assessing SEBN-based curricula on three distinct environments: a discrete gridworld (DoorKey), continuous control (BipedalWalker), and a difficult simulated robotics domain (robosuite); and
    (5) Demonstrating via experiments that SEBN curricula produce more robust policies that reach success more quickly than other curricula in the continuous control and robotics environments, and performed comparably in the gridworld environment. 


%=====================================================
%=====================================================
%=====================================================
%=====================================================
\section{Background and Preliminaries}

%The SEBN is a data structure that estimates competency level in an environment and uses those estimates to predict expected future reward.  
Bayesian statistics rely on some sort of informed prior, provided or learned, to estimate the future values.  
In an SEBN, part of this prior is provided in the form of the network and in the strength of relationships, and part of the prior is learned through the collection of samples from the environment.
We next provide our motivation for this Bayesian approach (\secref{sec:ecd})
with some background on Bayesian Networks (\secref{sec:bns}).  
We then formalize the curriculum learning problem (\secref{sec:curriculum-learning}), how we use task features to construct tasks (\secref{sec:taskdescriptors}), and an extension to the options framework (\secref{sec:options}).



%-------------------------------------------------------
%-------------------------------------------------------
\subsection{Evidence Centered Design}
\label{sec:ecd}

Our motivation for using a Bayesian Network to estimate learning proficiency comes from the method of Evidence Centered Design (ECD) \cite{mislevy2003brief}, a technique used in human educational assessments. 
In Evidence-Centered Design, statistical models, such as Bayesian networks, are used to measure the proficiency levels of a given student. These proficiency measurements are then used to inform task and assessment creation. For example, ECD could be used to help model and analyze the performance of a tennis player. The Bayesian network in this domain can include nodes that represent latent competencies (e.g., mobility, footwork, dynamic vision, etc.) and nodes that represent observable metrics (e.g., number of successful serves, return rate, game score). The performance of a tennis player on the observable metrics is used to infer their latent capabilities. New training goals can then be set using these estimated capabilities. This technique is effective in human educational contexts, and we hypothesize that a similar approach could be applied to assist in designing a curriculum to improve learning in robotic agents. 

%-------------------------------------------------------
%-------------------------------------------------------
\subsection{Bayesian Networks (BNs)}
\label{sec:bns}

Bayesian Networks (BNs) \citep{pearl} are a type of graphical model that provide an efficient way to represent and reason about probabilistic relationships among a set of random variables. A Bayesian Network $(X,D,\parentfunctions)$ is defined by a set of variables $X$, their corresponding domains $D$, and a set of parent functions \parentfunctions that specify the conditional probability distributions of each variable given its parents. 
When $D$ is discrete, these parent functions are typically specified in a tabular format known as Conditional Probability Tables (CPTs). %Bayesian Networks model complex probabilistic distributions through compact representation of conditional independence relationships among the variables. 

It is common to use BNs to model relationships between latent and observable variables. 
Once constructed, the network can be used to infer latent values from observed values. 
New data can be entered in the form of evidence values for observed variables in a BN. 
The probabilities over other variables in the network are calculated by conditioning on this observed evidence, i.e., as a conditional probability: $P(X_1|X_2,...,X_N)$. 
We will employ a standard bucket elimination algorithm (aka variable elimination) \cite{darwiche2009modeling, dechter2013reasoning} to perform inference.
In this paper, the observable variables of the BN relate to the environment of an agent and a target performance it is attempting to achieve; both are modeled as a task in an MDP problem and defined in \secref{sec:curriculum-learning}.  The unobservable variables will relate to a set of latent competencies that we define in \secref{sec:latent-skills}. 

%-------------------------------------------------------
%-------------------------------------------------------
\subsection{Curriculum Learning}
\label{sec:curriculum-learning}

We adapt the notation of 
\citeauthor{narvekarEtAl20.jmlr.clForRL}~\cite{narvekarEtAl20.jmlr.clForRL} to describe a \emph{task} as the interaction of an agent with its environment to meet some objective.  
A task, formalized as an episodic Markov Decision Process (MDP), is a tuple 
$\mdpmaster = (\mdpstates, \mdpactions, \mdptransition, \mdpreward{})$, where 
    \mdpstates{} 
    is the set of states, 
    \mdpactions~is the set of actions, 
    $\mdptransition(\mdpstate'|\mdpstate, \mdpaction)$ gives the probability of being in state $\mdpstate'$ after taking action \mdpaction in state \mdpstate, and 
    $\mdpreward{}(\mdpstate, \mdpaction, \mdpstate')$ is the reward function after taking action \mdpaction in state \mdpstate and transitioning to state $\mdpstate'$.
A solution to \mdpmaster is a policy $\pi$ that maximizes the cumulative sum of rewards for an episode of length T, i.e., $\sum_{t=1}^{T} R_{t}$.

Let \tasks be a set of all tasks an agent could complete in \mdpmaster, where a task $\taski \in \tasks$ is a task-specific MDP $\taski = (\mdpstatesi, \mdpactionsi, \mdptransitioni, \mdprewardi)$.  For all tasks in \tasks, let \transitiondomain be the set of all possible transition samples from \tasks (see \citeauthor{narvekarEtAl20.jmlr.clForRL} \cite{narvekarEtAl20.jmlr.clForRL} for a complete definition).  
In their formalism, a Curriculum $\curriculum = (\currvertices, \curredges, \taskdescriptor, \tasks)$ 
is a directed acyclic graph, where \currvertices is the set of vertices, $\curredges \subseteq \{(x,y) | (x,y) \in \currvertices \times \currvertices \land x \neq y\}$ is a set of directed edges,  $\taskdescriptor: \currvertices \rightarrow \mathcal{P}(\transitiondomain)$ is a function that associates samples within each vertex, and $\mathcal{P}(\transitiondomain)$ is the powerset of \transitiondomain.

In this paper, we develop what Narvekar~\etal~\cite{narvekarEtAl20.jmlr.clForRL} call a task-level curriculum, where each vertex $v \in \currvertices$ is associated with samples from a single task in \tasks.  
That is, the mapping function for task \taski is 
$\taskdescriptor: \currvertices \rightarrow \{ \transitiondomain_i | \taski \in \tasks \}$.  
For convenience, we will refer to a task's available samples at vertex $v$ as $\taski^{\taskdescriptor}$.
In other words, a task descriptor $\taskdescriptor_i$ is used to construct task \taski, and a curriculum is a sequence of tasks \task{1}, \task{2}, ..., \task{target} up to some target task.


Before we describe how we construct this function using task features, we point out some deviations from the model just described.  
The curriculum being a DAG is a very strong assumption and is not true of the SEBN-guided curriculum.  
While the episodic MDP model of \citeauthor{narvekarEtAl20.jmlr.clForRL} provides a more comprehensible model of curriculum learning, the RL algorithms of this paper actually learn with a discount factor \discount, and one could argue that the Partially Observable MDP might be more appropriate.  Both changes would be extensions to the simplified MDP model presented here. 
%\fromlaura{clarify that this theta is the same as our task descriptor? And/or relate it to section 2.4?}
%\fromlaura{I got a little twisted around here by the significance on a discount factor (which seems standard) and the mention of POMDPs. I edited but with tracking changes on feel free to revert.} 
%\frommak{Narvekar's model uses episodic MDPs, which terminate rather than continue indefinitely.  So the discount factor isn't needed for these.  Similarly, we are really learning over POMDPs for the robot.  I was just trying to clarify that we recognize the limitations of this abstract model for getting the main points across.}

%\fromlaura{perhaps my earlier edits weren't appropriate -- this paragraph seems to be where it says what we do differently than Narvekar. I suggest moving the footnote here, too -- to emphasize that we remove the strong assumption of the DAG.}

%-------------------------------------------------------
%-------------------------------------------------------
\subsection{Task Descriptors (Env't+Target Features)}
\label{sec:taskdescriptors}
We will use \emph{task features} to define \taskdescriptor for a task $\taski^{\taskdescriptor}$.
This is a common approach to quantify potential transfer between two tasks (e.g., \cite{rostamiEtAl20.jair.usingTaskDescriptions,isele2016task,narvekar2016source,konidaris2012transfer}).
The notion is that two tasks that share similar features will exhibit better transfer.
We adopt a task descriptor similar to \citeauthor{rostamiEtAl20.jair.usingTaskDescriptions}~\citep{rostamiEtAl20.jair.usingTaskDescriptions} and \citeauthor{narvekar2016source}~\citep{narvekar2016source}.

Specifically, we parameterize \taskdescriptor with a vector that 
consists of set of environment-specific features $E$ and a set of one or more performance targets $\Target$.
Thus, $\taskdescriptor((\mathbb{Z}_0)^{|E|}(\mathbb{Z}_0)^{|K|})$ will indicate the specific task $\taski^{\taskdescriptor}$.
We will often omit the task descriptor for clarity and just reference \taski.  Note that the task descriptor is underspecified with respect to the environment, so one configuration of \taskdescriptor represents a class of different environments an agent can encounter. 


\paragraph{Example Task Descriptors using Bipedal Walker.} 
The Bipedal Walker (BPW) benchmark \cite{towers_gymnasium_2023} involves two-legged agent moving through terrain in a 2D environment.  Figure~\ref{fig:bipedalwalkerenv} shows some example terrains.
The top portion Figure~\ref{fig:sebn-bipedal-walker} shows $E$ and \Target for the bipedal walker. 
Here, $E$ consists of five features that control the difficulty of the environment, and there is a single target of moving to the right by at least 30 steps (The dashed latent competencies are defined in \secref{sec:latent-skills}). 
The task descriptor for this BPW is $\taskdescriptor($\mbox{\scriptsize{pit-gap, stump-height, stair-width, stair-steps, roughness, moved>30}}$)$. A task $\taski^{\taskdescriptor}$ for BPW involves a particular setting of the parameters for \taskdescriptor. \exampledone


%-------------------------------------------------------
%-------------------------------------------------------
\subsection{Extending Targets to Include Options}
\label{sec:options}

\noindent
One could imagine a richer set of targets in $K$, even ones that are hierarchically organized with a natural decomposition of subtasks.
The options framework \cite{suttonEtAl99.aij.betweenMDPsAndSemiMDPs}
is a common model for such situations.
Briefly, a subtask $j$ for task \taski is an option
$o_j = ( \mdpstates{j}, \pi_j, \oterm_j )$, 
where $\mdpstates{j} \subseteq \mdpstatesi$ are the starting states of the option, $\pi_j$ is used to take action while the option is enabled, and $\oterm_j: \mdpstatesi \rightarrow \zeroone$ is a function that indicates the option has terminated.

For a specific task \taski, a subtask  $\task{i}^j = (\mdpstates{j}, \mdpactionsi, \mdptransitioni, \mdpreward{j})$ indicates that an option has a specific context: it works over a set of states \mdpstates{j} that are a subset of the tasks states \mdpstatesi{}, it uses a specific reward independent of the task reward, and it has the same actions and transitions as the original task \taski.
Each option $j$ is enabled as part of the feature parameters for \taskdescriptor (i.e., $(\mathbb{Z}_0)^{|K|})$). 

%Two of the environments in this paper have a single ``option" (BipedalWalker and Robotsuite Door), which means the system is learning a single policy for the entire problem.  For this simplified situation, \task{} describes the overall task with reward \mdpreward{} and $\task{i}^{\taskdescriptor}$ describes the $i$th task within a curriculum, which is constructed using task descriptor \taskdescriptor.  

\begin{figure}
\begin{scriptsize}
\begin{tabular}{ll}
    %trim(left bottom right top)
    \includegraphics[trim={0 16cm 0 0}, clip, width=0.20\textwidth]{figures/bipedalwalkerenv.png} &     
    \includegraphics[trim={0 4.5cm 0 11cm}, clip, width=0.20\textwidth]{figures/bipedalwalkerenv.png}  \\
    (P3 S0 N0 W0 R0) & 
    (P0 S0 N3 W3 R0)\\
    \includegraphics[trim={0 10cm 0 6.3cm}, clip, width=0.20\textwidth]{figures/bipedalwalkerenv.png} & 
    \includegraphics[trim={0 0 0 16cm}, clip, width=0.20\textwidth]{figures/bipedalwalkerenv.png}\\
    (P0 S1 N0 W0 R0) & 
    (P0 S0 N0 W0 R4)\\
\end{tabular}
\end{scriptsize}
    \caption{Challenge environments for BipedalWalker  with corresponding descriptors (P:pit gap, S:stump height, W:stair width, N:stair steps, and R:ground roughness).  }

    \label{fig:bipedalwalkerenv}
\end{figure}

\begin{figure}
    \includegraphics[width=0.4\textwidth]{figures/SEBN-bipedal-walker.png}
    \caption{The SEBN for the Bipedal Walker environment.  }
    \label{fig:sebn-bipedal-walker}
\end{figure}



\paragraph{Example Task Descriptors using DoorKey.}
Suppose we want an agent to learn to navigate in a gridworld environment to a goal while opening locked doors.
Fig.~\ref{fig:doorkeyenv} shows several possible environments for this agent and their corresponding environment feature vector. In the easiest environment (top left), the agent (the white arrow) starts very near the goal (``A'') in an empty grid. Adding additional obstacles such as a wall, shown as chess rooks, or a locked door, shown as a lock, with a key to unlock it, adjusts the environmental features accordingly. The first three components of the task descriptor  \taskdescriptor indicate whether the distance (D) of agent starts near (within 2 squares) of any point of interest (key or goal), the presence of a wall (W), and the presence of a locked door (L).

DoorKey also enables the use of options.  
The top right part of Figure~\ref{fig:doorkeysebn} shows a network of targets $K$ for this problem, corresponding to ordered subtasks.
This problem has three options (at(goal), opened(door), and has(key)), each with its own reward. 
A distinct policy is learned for each of these options.
The last three components of \taskdescriptor indicate which of these three subgoals are enabled for a task: getting a key (K), opening a door (O), or being at (A) a cell. \exampledone




%=====================================================
%=====================================================
%=====================================================
%=====================================================
\section{Bayesian Curriculum Learning}

The key idea in this section is to use a BN to estimate performance on $K$ over the environmental features from $E$ plus a set of estimated proficiency \Proficiency on latent competencies, which are hidden or unobserved.
Before we formally define the SEBN, we describe extend the DoorKey example to discuss this process.

\paragraph{Example of Latent Competencies}
Suppose we have collected past data of the agent's performance on different environments in $E$. For example, say we ran our agent on the empty-grid (D0 W0 L0), wall-only (D1 W1 L0), and door-only (D1 W0 L1) environments and recorded that the agent was successful on the empty-grid and door-only environments but not the wall-only environment. 
This failure might be due to the  agent not yet knowing how to navigate around walls. We can think of this ability as a latent "avoid wall" capability that an agent needs to have mastered to solve tasks that require it. Furthermore, using the notion that there is this latent capability, we can easily predict that the agent will fail on the wall-and-door (D0 W1 L1) environment without having any data of the agent's performance on that specific type of environment. \exampledone

The bottom row of Fig. \ref{fig:doorkeysebn} shows a set of latent competencies or capabilities \Proficiency. In this example, we chose four latent competencies: (move, pick up, avoid wall, open) that we expect the agent to need to master to successfully solve different tasks.  These latent variables are provided by an expert, similar to \citeauthor{abelEtAl15.icaps.goalBasedActionPriors} \cite{abelEtAl15.icaps.goalBasedActionPriors}.


We can use the SEBN to predict two important quantities. 
First, when faced with a new (possibly unseen) task $\taski \in \tasks$, we need to estimate the proficiency of each competency in \Proficiency.  This is important because competencies will advance at different rates and some tasks will require more proficiency than others for specific competencies.

Second, when faced with a new (possibly unseen) task, we need to be able to predict performance on $k_j \in K$ given the current estimates of competency level (from the first step). This is important because it can be used to select from a set of candidate tasks for training in the next iteration of a curriculum.


\begin{figure}
\begin{scriptsize}
    
\begin{tabular}{p{1.8cm}p{1.8cm}p{1.8cm}p{1.8cm}}
\includegraphics[width=0.09\textwidth]{figures/doorkey1} & 
\includegraphics[width=0.09\textwidth]{figures/doorkey2} &
\includegraphics[width=0.09\textwidth]{figures/doorkey3} & 
\includegraphics[width=0.09\textwidth]{figures/doorkey4} \\
(D0 W0 L0 K0 O0 A1) & (D1 W1 L0 K0 O0 A1) &
(D1 W0 L1 K0 O0 A1) & (D0 W1 L1 K1 O1 A1) \\

\end{tabular}
\end{scriptsize}
    \caption{Example environments for DoorKey with  corresponding environment features (D:distance, W:wall, L:locked door) and target features (K:key, O:opened, A:at).}
    \label{fig:doorkeyenv}
\end{figure}
\begin{figure}
    %\includegraphics[width=0.45\textwidth]{figures/SEBN-door-key.png}
    \includegraphics[width=0.35\textwidth]{figures/SEBN-colored.png}
    \caption{The SEBN for the DoorKey environment.  }
    \label{fig:doorkeysebn}
\end{figure}



%\fromlaura{this seems super important and it might be nice to highlight it more loudly. the fact that the proficiency of the latent skills, if we do it right, can directly feed into an estimate of proficiency of the task is really cool. I suggest we ALSO highlight how it can be used for curricula learning, because this doesn't address that it reminds me too that I misread "skills" in the abstract to refer to policies when I first read it -- how we usually use them -- and not latent skills. or maybe not -- maybe they are the same thing? now I'm second-guessing myself :) } 
%\frommak{I am unsure how to fix this or where things are confusing.  Was this comment written before you read Section 3.2, where a clearer definition is provided.  I can't tell if you are suggesting a major restructuring of the content or you have some other solution in mind, or maybe this is just an out loud comment? }
%\fromlaura{definitely not a major restructuring. but I am a bit confused with how (if at all) this connects with RL "skills" (aka policies) as we currently use the term. maybe a reviewer wouldn't have that issue because they aren't familiar with our group's normal terminology. is there a correspondence? or are policies completely orthogonal to all of this?}

Returning to Fig.~\ref{fig:doorkeysebn}, suppose a new task is defined over E and \Target.  The proficiency estimate(s) can be calculated using the links to the bottom row. Once these estimates are provided back to the network, the expected reward can be calculated in the target layer \Target. 


%\fromlaura{do we need to say what happens if we don't have a latent skill that meaningfully connects to the new task? such as if we hadn't ever seen a door before and now need to pen a door for the new task? Would we even know that that requires a different skill? It might not be relevant here, for sure. but readers might wonder ahead of the actual technical explanation.}
%\frommak{Latent skills are  provided externally to the SEBN and static. We don't add new skills during learning.    This is defined in Section~\ref{sec:variable-distributions}, but is there something more that needs to be said here?}\fromlaura{maybe just a quick acknowledgement of that here -- since there is no guarantee that a new task will connect meaningfully to the existing skills.}
%\fromlaura{I would have appreciated an example of the task descriptor}



%-------------------------------------------------------
%-------------------------------------------------------
\subsection{Competencies (Latent or Explicit)} 
\label{sec:latent-skills}
As with the "avoid wall" latent competency for DoorKey, we propose that there is some shared latent set of competencies of which mastery over can predict an agent's success rate on different metrics in different environments. More concretely, let $\kappa_i \in K$ be a set of observable metrics, which can be any measurable target (e.g., a standard reward function, a shaped or partitioned reward function, or the completion an option). For example, we can define a binary variable that is 1 if an agent received a reward of more than a threshold value, and 0 otherwise. 

Drawing inspiration from ECD, we propose that there exists a set of latent competencies $\Psi$ that are not directly observable but can be inferred from the observed metrics. These competencies are such that the probability of success on a given observable metric $\kappa_i$ of an agent on a specific task $m_i$ is dependent on sufficient mastery of the corresponding latent competencies. This relationship allows us estimate the impact of competencies on unseen tasks. A BN allows us to model this relationship, estimate competency levels from data, and subsequently estimate success rates on different environments. 

Explicit competencies can be derived from techniques that decompose a task into subtasks.  
For example, in hierarchical planning, a method decomposes an abstract task.  
A set of such methods could be used to construct the competencies.
For example, recent work has used hierarchical goal networks to decompose tasks and train RL policies. \cite{patra2022hierarchical,patra2023relating}.
They call the resulting policies goal skills.
For the SEBN defined in Fig. \ref{fig:doorkeysebn}, all four of the competencies in $\Psi$ could be defined as an explicit goal skills.
In the case of this SEBN, the dependencies in the network are exactly the same as a corresponding hierarchical goal skill network. Consequently, we could take any problem with a heirarchical goal skill network, define environment-conditioned dependencies, and turn it into a SEBN. 

The flexibility of letting competencies be latent or explicit allows us to model environments where intermediate decompositions may not be well-defined.
In the absence of an easy way to check if an agent satisfies a goal for a given goal-skill, then it can be set to be a latent competency in the SEBN.







%-------------------------------------------------------
%-------------------------------------------------------
\subsection{Skill Environment Bayesian Network}
We can now define a Skill Environment Bayesian Network (SEBN) for estimating the competency level of an agent and modeling the relationship between agent competency levels, env features, and observable goals/metrics. The SEBN is a tuple $\{X, D, \parentfunctions\}$, where the variable set $X = E \cup \Target \cup \Proficiency$ is split into three sets of variables:
%\begin{itemize}
    %\item 

    
\textbf{Environment Variables.} $E$ is a set of variables that represent the features of an environment descriptor. Each variable in this set corresponds to a specific feature of the environment (and thus the domain of a given variable is the set of possible values the corresponding environment feature can take). In the gridworld example, $E$ consists of features for wall, door, and distance. 

    %\item 
    \textbf{Target Variables.} \Target is set of variables that directly correspond one or more targets. If there is a single policy, then \Target will have a single node, as in Figure~\ref{fig:sebn-bipedal-walker}. But if there are options available to the agent, then each $\target{j} \in K$ corresponds to the option for task $\taski^j$ (cf. \secref{sec:options}). These variables then provide estimates of the value of executing that option in the current environment. For the purposes of this paper, that estimate is thresholded such that each variable returns a boolean value corresponding to whether its estimate meets a performance threshold (roughly corresponding to an estimate of whether the option will succeed or fail). 

    %\item
    \textbf{Competency Variables.}  \Proficiency contains the set of variables that represent the competency levels of an agent. Each variable in this set denotes the level of proficiency an agent has in a particular competency and takes values in a range from $\{0,1,...,N\}$ where $N$ is the highest proficiency level for a given competency. For this paper, we will define competency with two or three levels of proficiency. Competency levels are roughly ordered, as provided in a set of requirement specifications, by a human expert. The rationale for writing these specifications is to convey whether a given environment requires sufficient proficiency in several competencies. Specifications follow (roughly) ordered values of competency (e.g. a higher “move” competency should enable harder tasks). The competencies can be latent (e.g., capturing whether an agent avoids obstacles while moving) or explicitly learnable (cf. \secref{sec:latent-skills}). 
%\end{itemize}

The parent functions \parentfunctions provide the distribution of possible values of each variable conditioned on their parent variables. To construct these parent functions, we specify a list of competency requirements that is procedurally used to construct the corresponding CPTs. We provide specific detail about this process in \secref{sec:variable-distributions}. 
%\fromlaura{I would have appreciated examples of this for the gridworld example, which didn't mention parent functions.}  \frommak{they are provided in the next section}


Once constructed, we can use the SEBN to estimate an agent's competency levels and determine what environments or competencies the agent should focus on learning next. To do this, we must estimate two quantities for a task $\taski \in \tasks$: 
%\begin{itemize}
%\item 
\textbf{Competency Level:} $P(\Proficiency = \proficiency| \target{i} = r, \tasks = \taski)$ - the probability that an agent has a competency level of \proficiency, given its prior performance of at least reward $r$ on target \target{i} for task \taski.  
%\item 
\textbf{Expected task reward:} $P(\target{i} = r| \tasks = \taski, \Proficiency = \proficiency)$ - the probability that an agent can achieve a reward of at least $r$ for target \target{i} conditioned on task \taski with a given competency level \proficiency.
%\end{itemize}

We will estimate the competency levels 
%\taski ($P(\Proficiency= \proficiency| \mdprewardi = r, E = \taski)$) 
using past rollouts.
We will then apply the estimates of competency levels to estimate the expected task reward 
%($P(\mdprewardi = r| E = \taski, \Proficiency = \proficiency)$,
over a collection of candidate environments $\taski \in \tasks$. 
These estimated probabilities will be used to determine which environments the agent should train on next.

%Our aim is to estimate the first probability, $P(S = s| T = t, E = t_m)$ using past rollouts, and then use the estimated skill proficiencies to estimate the second probability, $P(T = t| E = t_m, S = s)$, on a collection of candidate environments $t_m$. These estimated probabilities will be used to determine which environments the agent should train on next.

%-------------------------------------------------------
%-------------------------------------------------------
\subsection{Defining variable distributions} 
\label{sec:variable-distributions}
To complete our network definition, we need to define the distributions of each variable in the network. The distribution of variables in $\Psi$ and \Target are defined using a hierarchical structure. We start by defining the leaves of this structure which are located in \Proficiency. 

\paragraph{Defining \Proficiency}
Consider the "move" competency $move \in \Psi$ in the gridworld navigation environment. In Fig. \ref{fig:doorkeysebn}, we define $move$ as having three levels of proficiency $\{0,1,2\}$, associated with a corresponding parameter set $\phi_{move} = \{0.8, 0.2, 0.0\}$ such that the probability of the "move" at competency level $j$ is given by:
%\begin{align}
$P(move = j) = \phi_{move}(j)$.
%\end{align}
In this case, the parameters denote that currently the agent has a $move$ competency of 80\% probability for no mastery and a 20\% probability of having level one mastery.

More generally, \Proficiency can have a hierarchical structure and there can be latent competencies within $\Psi$ that depend on other latent competencies.  %(Though not used in this paper, the distributions for these hierarchical competencies would follow the same convention as variables in \Target without being conditioned on environment features.)  
To allow for these hierarchies in \Proficiency, we define $B$ to be a subset of $\Psi$ which represents a base set of competencies (the leaves). The distribution of these base competencies is determined by a set of associated parameters $\phi_{B}$. 

\paragraph{Defining \Target}
The variables in \Target represent the success of an agent on a given target. In an SEBN, we seek to model the relationship where the success probability of an agent on a variable $\target{i} \in \Target$ is dependent on three types of parent variables:
%\begin{enumerate}
%\item 
(1) the environment features $e_i \in E$ relevant to $\target{i}$
%\item
(2) the agent's current competency levels $\psi_i \in \Psi$ 
%\item 
(3) other targets $\target{i}' \in K$
%\end{enumerate}
This means that variables in \Target depend on other variables in $\Psi$ and \Target as well as a set of variables in $E$. The success of a task depends on the agent's mastery of the competencies required for a given environment configuration and an independent failure rate $\lambda$. A task succeeds with a rate of $(1-\lambda)$, if all sub-requirements in \Target and $\Psi$ are satisfied. 

More concretely, for each relevant environment configuration $e_i$ in relation to a goal variable $\target{i} \in K$, we define a set of competency level requirements $R_{\target{i}}$ that an agent must master to successfully complete the task of $k_i$ on the environment $e_i$. For example, in Fig. \ref{fig:doorkeysebn}, \textit{haskey} depends on the \textit{distance} environment feature and the latent $move$ and $pickup$ competencies. Suppose we define the following competency level requirements for the \textit{haskey} node:
\begin{small}
%\begin{align}
\begin{itemize}[noitemsep,topsep=0pt,parsep=0pt,partopsep=0pt]
    \item[] haskey: (distance=0 ~|~ move=1,pick up = 1) 
    \item[] haskey: (distance=1 ~|~ move=2, pick up = 1)
\end{itemize}
%\end{align}
\end{small}
These state that if the key is close (distance=0), the agent needs a level of proficiency of 1 in the move and pick up competencies to successfully get the key. However, if the key is far (distance=1), the agent needs a higher level of proficiency in the move competency (move=2). The agent should have a high chance of success if it meets all necessary requirements and a high chance of failure otherwise. 

We directly translate these requirements into entries in the corresponding CPTs in the following way.
%\begin{itemize}
    %\item
    For the first competency level (distance=0), we have that:
    \begin{small}
    \begin{itemize}[noitemsep,topsep=0pt,parsep=0pt,partopsep=0pt]
    %\begin{align}
        \item[] $P(haskey = 1| distance = 0, move = 0, pick up = 0) = \lambda $
        \item[] $P(haskey = 1| distance = 0, move = 0, pick up = 1) = \lambda$ 
        \item[] $P(haskey = 1| distance = 0, move = 1, pick up = 0) = \lambda$ 
        \item[] $P(haskey = 1| distance = 0, move >= 1, pick up >= 1) = (1 - \lambda)$
    %\end{align}
    \end{itemize}
    \end{small}
    %\item 
    For the next competency level (distance=1), we have that:
    \begin{small}
    \begin{itemize}[noitemsep,topsep=0pt,parsep=0pt,partopsep=0pt]
    %\begin{align}
        \item[] $P(haskey = 1| distance = 1, move = 0, pick up = 0) = \lambda$
        \item[] $P(haskey = 1| distance = 1, move = 0, pick up = 1) = \lambda$
        \item[] $P(haskey = 1| distance = 1, move = 1, pick up = 0) = \lambda$
        \item[] $P(haskey = 1| distance = 1, move = 1, pick up = 1) = \lambda$
        \item[] $P(haskey = 1| distance = 1, move >= 2, pick up >= 1) = (1 - \lambda)$
    %\end{align}
    \end{itemize}
    \end{small}
%\end{itemize}
These state that the agent has a success rate of $(1-\lambda)$ for a given goal for an environment setting if it satisfies all necessary requirements and a success rate of $\lambda$ if there is one or more requirement missing.

For the environment features set $E$, the distribution of variables in this set is fully controlled for the purpose of curriculum generation. Therefore, the data (samples obtained from rollouts) can be used as the distribution for variables in $E$.

%-------------------------------------------------------
%-------------------------------------------------------
\subsection{Curriculum through Inference}
The general algorithm to generate an SEBN-guided automated curriculum is as follows. First we define a generation of the algorithm as $L$ rollouts. Let $\tasks$ be the set of possible tasks and $P_\tasks(m_i)_t$ be the probability that task $m_i = (e_i, \target{i})$ is chosen in the current generation $t$. We first initialize $P_\tasks(m_i)_0$ to some initial task distribution. This initial weighting can be biased towards easier tasks or can be set to a uniform distribution for better initial competency estimation. Let $\Phi_{B} = \{...,\phi_{B_i},...\}$ be the set of parameters associated with the base competencies $B$ in the SEBN. For each generation, we take the following steps:
\begin{enumerate}[noitemsep,topsep=0pt,parsep=0pt,partopsep=0pt]
    \item Sample $L$ rollouts $(m_i = (e_i, \target{i}) \sim P_\tasks, o_i)$. For each rollout, we record a set of observable metrics $\target{i}$. 
    \item Solve the MLE problem:
    \begin{align}
       \Phi_{B}^* = \text{arg}\max_{\Phi_{B}} \prod_{L} P(E = e_i, K = \target{i}) 
       \label{eq:mle}
    \end{align}
    for $\Phi_{B}^*$, the values of the parameters for the base competencies, which is an estimate of agent proficiency level in those competencies.
    \item Update the task distribution for the next generation: $P_\tasks(m_i) = F(P_{\Phi_{B}^*}(\target{i}|m_i))$ where $F$ is some function that maps the estimated success rate of an environment $P_{\Phi_{B}^*}(\target{i}|m_i)$ to a probability distribution.
\end{enumerate}
For our work, we define $F$ using the following fitness function:
\begin{align}
    F(m_i)_t = (P_{\Phi_{B}^*}(\target{i}|m_i)_t - P_{\Phi_{B}^*}(\target{i}|m_i)_{t-1})^2 \nonumber\\
    P_\tasks(m_i)_{t+1} = 0.5 \cdot \frac{F(m_i)}{\sum_{m_i} F(m_i)} + 0.5 \cdot P_\tasks(m_i)_{t}.
    \label{eq:fitness}
\end{align}
In \cite{stout2010competence}, it was proposed that curricula should focus on problems where the agent improves the most or has the most expected \textit{improvement in competence}. To simulate this in our approach, we choose a fitness function where the fitness of the environment is a function of the difference between the current estimated success rate and the estimated success rate on the last generation's SEBN. We also add a smoothing factor to improve learning stability.

Note that our curriculum is agnostic to the choice of learning algorithm and policy which can be assumed to be black boxes. The curriculum only requires observations of rollouts and not the internal reward structure of a given policy. 

\begin{algorithm}
\SetAlgoLined
\begin{small}
\While{ not converged}{
    Sample L environments $m_i = (e_i,\target{i}) \sim P_{\tasks_0}$ \\
    \For{$i = 1 \rightarrow L$}{
        Collect rollout data $(m_i, \target{i})$ while training policy $\pi$
    }
    Solve the MLE problem (Equation \ref{eq:mle}) for estimated competency level $\Phi_{B}^*$ on SEBN $(X,D,\Phi)$ \\
    For candidate environments, estimate agent success rate $P_{\Phi_{B}^*}(\target{i}|m_i)$  using SEBN with $\Phi$ updated with $\Phi_{B}^*$ \\
    For each candidate environment $m_i$, update $P_\tasks(m_i)_{t+1}$ using expected improvement weighting of $P_{\Phi_{B}^*}(\target{i}|m_i)$ (Equation~\ref{eq:fitness})
}
 \caption{SEBN-guided Automated Curriculum\newline
 \textbf{Input:} Initial tasks $P_{\tasks_0}$, Generation size $L$, SEBN $(X,D,\Phi)$ \newline
 \textbf{Initialize:} Initialize policy $\pi$ }
\end{small}
 \label{alg:SEBN}
\end{algorithm}


%-------------------------------------------------------
%-------------------------------------------------------
\subsection{Candidate Selection}
\label{sec:approximate-inference}

On a domain such as Megagrid, we can evaluate $P_{\Phi_{B}^*}(\target{i}|m_i)$ for every combination of environmental features. However, calculating $P_{\Phi_{B}^*}(\target{i}|m_i)$ for all possible environments in BipedalWalker took too much time at the end of each rollout generation. In general, for domains with a large amount of environmental features, it is impractical to evaluate $P_{\Phi_{B}^*}(\target{i}|m_i)$ for every single task $m_i \in \tasks$.  

One way to solve this computational problem is to search in the space of possible environment configurations and only update the distribution of the most promising environments. For our search procedure, we adapt a greedy sample-search procedure based on KL-Search from \cite{hsiao2024surrogate}. This algorithm employs a heuristic search along variables in a Bayesian network to minimize a KL distance heuristic between two networks. By modifying algorithm to instead search for nodes with maximum differences between the current generation's SEBN and the next generation's SEBN updated with $\Phi_{B}^*$, we can find environments where the estimated success rate changes the most. This modification produces a tree search algorithm that selects nodes in the OR-tree corresponding to a given SEBN where the difference $(P_{\Phi_{s_i}^*}(\target{i}|m_i)_t - P_{\Phi_{s_i}^*}(\target{i}|m_i)_{t-1})$ is greatest. 

Given a partial configuration $X$, we use the following heuristic:
\begin{align}
    h(X_1 {=} 0) = |\log (P_{t-1}(X)) - \log (P_t(X))| \cdot P_t(X)
    \label{eq:heuristic}
\end{align}
where $P_{t}(X_1 {=} 0) = P_{\Phi_{s_i}^*}(\target{i}|X1)_t$. % This ensures that we pick nodes where the difference in estimated success rates between the previous and current SEBNs are the greatest. 
We use Weighted Mini-bucket Elimination \cite{liu2011bounding} with an ibound of 20 to estimate SEBN probabilities when exact inference is too computationally expensive. Once we select $N = 20$ nodes on the OR-tree using this method, we perform the same calculations from Eq. \ref{eq:fitness} over the 20 selected environment configurations to define our curriculum for the next generation.


%=====================================================
%=====================================================
%=====================================================
%=====================================================
\section{Experimental Evaluation}
To demonstrate the effect of our proposed curriculum learning approach, we evaluate the SEBN curriculum on three environments.
%\begin{enumerate}
%\item 
\textbf{DoorKey:} A MiniGrid \cite{MinigridMiniworld23} inspired domain with explicit intermediate goal skills. In this domain, the agent must learn to navigate through a grid-based environment to reach a goal location, while also learning to achieve intermediate goals along the way.
%\item 
\textbf{BipedalWalkerHardcore:} a simulated bipedal robot must learn to walk forward as quickly as possible while maintaining balance. The bipedal robot can encounter a variety of obstacles such as rough terrain, stumps, pits and stairs that need robust policies.
%\item 
\textbf{Robosuite:} a robotic arm (Kinova Gen 3) must learn to open a door with different weight and latch settings.
%\end{enumerate}

In each of these domains, we compare the performance of reinforcement learning agents trained with and without our proposed SEBN-guided curriculum as well as two additional controls:
%\begin{itemize}
    %\item 
    \textbf{Uniform curriculum (or Domain Randomization \cite{tobin2017domain}):} all environments have an equal probability of being selected.
    %\item 
    \textbf{Anti-curriculum:} the probability difference in Eq. \ref{eq:fitness} is replaced with (1 - difference) and we use a min priority queue for candidate selection during the sample-search algorithm.
%\end{itemize} 

We evaluate the agents on their ability to learn effective policies that can achieve high rewards in each domain, as well as their ability to generalize to new tasks and environments. Our results show that the SEBN-guided curriculum consistently improves the performance of reinforcement learning agents across all three domains. All runs are performed on an AMD EPYC 7H12 64 core CPU with networks being handled on an A100 GPU.

\begin{figure}
    \includegraphics[width=0.36\textwidth]{figures/csv_plot_new}
    \caption{Result of employing a SEBN-guided automated curriculum on the DoorKey environment. %\frommak{Further space can be saved by making the three results plots be a figure* environment with a single caption and labelled subplots using subfigure (so they can be referenced properly).  this would require moving some details to prose, but that's fine}
    }
    \label{fig:doorkeyres}
\end{figure}

\paragraph{DoorKey}
We start with a gridworld environment named Megagrid based on MiniGrid \cite{MinigridMiniworld23}; we reimplemented this standard gridworld environment to enable easier generation of environments using the task descriptor\footnote{Please send an email to the authors to request Megagrid code}. We evaluate on the simple DoorKey environment to evaluate the effectiveness of our curriculum learning approach when combined with explicit goal-skills. We use the SEBN shown in Fig. \ref{fig:doorkeysebn} which has the following variables:
\begin{itemize}[noitemsep,topsep=0pt,parsep=0pt,partopsep=0pt]
    \item[$E$:] includes a wall feature \zeroone, door feature \zeroone, and a distance feature \zeroone. A selection of the different environments that can be experience by an agent can seen in Fig. \ref{fig:doorkeyenv}. 
    \item[$\Psi$:] includes "move to" \zot, "pick up" \zeroone, "avoid wall" \zeroone, "drop" \zeroone, and "open door" \zeroone.
    \item[\Target:] includes three options: "at(goal)" \zeroone, "opened(door)" \zeroone, and "has(key)" \zeroone~ each trained with a PPO policy.
\end{itemize}
Observations of the environment are provided as partially observed cardinal direction data, following the sensor convention for the Lightworld domain in \cite{konidaris2012transfer}. We assume that the agent has four cardinal sensors for each item. For example, in the rightmost env of Fig. \ref{fig:doorkeyenv}, the agent would receive the observation that there is a key one step above it, a wall one step below it, and the goal 4 steps below it (0.875 key - up, 0.875 wall - down, 0.5 goal - down). An agent gets a reward of 1 if it reaches the goal square or completes any intermediate goals required (e.g. picking up a key or opening a door). There is no step penalty but the episode is automatically terminated if no progress has been made in 50 steps.

We train the policies using an Actor-Critic architecture \cite{sutton2018reinforcement} trained using Proximal Policy Optimization (PPO) \cite{schulman2017proximal}. The policy and value networks each have four hidden layers which are used to calculate their corresponding outputs.

The evaluation curve of our policies can be seen in Fig. \ref{fig:doorkeyres}. This evaluation is performed on the hardest environment (rightmost environment in Fig. \ref{fig:doorkeyenv}). Because this environment is relatively simple, it is easy for policies learned without a curriculum to achieve a high success rate. However, there is a noticeable jumpstart where the SEBN-guided curriculum provides a gain in learning efficiency.

We performed a generalization experiment where the learned policies are transferred to an evaluation on an \emph{unseen} and much larger 32 x 32 grid environment. The policies learned using the SEBN-guided curriculum are more robust to this change in grid size succeeded at an average rate of 93 percent compared to only 82 percent without a curriculum. %(see appendix for full plot). 

%The results can be seen in Fig. \ref{fig:doorkeylarge}.



%-------------------------------------------------------
%-------------------------------------------------------
%\subsection{Continuous navigation on varied terrain}
\paragraph{BipedalWalker Hardcore}
For the next domain, we evaluate our SEBN-guided curriculum on BipedalWalker (BPW), a continuous control environment. We employ a modified version of the BipedalWalkerHardcore environment from \cite{parker2022evolving} to suit a limited computational budget. We define the following SEBN:
\begin{itemize}[noitemsep,topsep=0pt,parsep=0pt,partopsep=0pt]
    \item[$E$:] includes five design parameters: ground roughness $\{0 - 7\}$, pit gap $\{0 - 3\}$, stump height $\{0 - 2\}$, stair width $\{0 - 3\}$, and stair steps $\{0 - 3\}$. Since there are too many different environments to perform exact inference, it is necessary to use a sample-search procedure to select candidate environments.
    \item[$\Psi$:] includes  "move", "climb", "jump", "balance", "descend". All with proficiency levels $\{0,1,2\}$. 
    \item[\Target:] includes one observable metric \zeroone: whether the agent has traveled a distance of 30 units ($\approx$1/3rd of the level).
\end{itemize}
The observation of the agent consists of internal sensor measurements such as (hull angle speed, angular velocity, horizontal velocity, etc..) an a set of 10 lidar rangefinder measurements. On this environment, the robotic walker gets a dense positive reward for traveling forward on the terrain, a small negative reward for using its motors, and a negative reward of -100 if it falls down. On environments that are too challenging for the agent at its current capabilities, this reward structure can promote a locally optimal behavior of simply staying still and preventing itself from falling. 

To learn our policies, we use the TD3-Fork algorithm from Honghao et al. \cite{wei2020fork}, which was shown to train much faster than standard PPO on BPW. We train agents for 3.5 million timesteps. During training we evaluate against four specific challenges (Stairs, PitGap, Stump, and Roughness in Fig. \ref{fig:bipedalwalkerenv}) as well as a combined environment (Evaluation) that contains all challenges together. We compare against a policy trained on only the combined environment.

\begin{figure}
    %\includegraphics[width=0.23\textwidth]{figures/csv_plot_bpw0}
    %\includegraphics[width=0.23\textwidth]{figures/csv_plot_bpw1}
    %\includegraphics[width=0.23\textwidth]{figures/csv_plot_bpw2}
    %\includegraphics[width=0.23\textwidth]{figures/csv_plot_bpw3}
    \includegraphics[width=0.36\textwidth]{figures/csv_plot_bpw4}
    \caption{SEBN-guided automated curriculum on BipedalWalker. Evaluation environments are randomly generated within a given environment feature set.
    }
    \label{fig:bipedalwalkerres}
\end{figure}
The results of the evaluation on the combined environment can be seen in Fig. \ref{fig:bipedalwalkerres}. Since we do not train for a large number of environment timesteps, we can observe that without a curriculum, TD3-Fork does not manage to learn to the point of a positive reward on any of the test environments.

There is a significant reward divergence in the results starting at 1 million timesteps. The uniform and anti curriculum perform better than having no curriculum with the uniform curriculum perform marginally better than the anti-curriculum. It can be seen on each graph that the policies trained using the SEBN-guided curriculum manages to get to a point where the agent starts receiving a positive reward for each environment, getting past the initial hurdle of the locally optimal staying still behavior.

It is interesting to observe that the learning curve for the SEBN-guided curriculum has much higher variance than the learning curves for other methods. Due to the size of the environment design space, it is necessary to use approximate inference techniques to search for candidate environments. Because we use a sample-search procedure, the search process is not guaranteed to find environments with the largest heuristic difference (see Eq. \ref{eq:heuristic}). This means that sub-optimal environments can be introduced into the curriculum. This introduces a large factor into learning curve variance beyond standard noise from reinforcement learning algorithms.


%-------------------------------------------------------
%-------------------------------------------------------
%\subsection{Simulated robotics}

\begin{figure}
    \includegraphics[width=0.36\textwidth]{figures/csv_plot_robo}
    \caption{SEBN-guided automated curriculum on the Robosuite Door environment. 15 policies are learned for each line and evaluated on the hardest (mass=6, latch=1) environment.}
    \label{fig:robosuiteres}
\end{figure}

\paragraph{Robosuite - Open Door}
In our final domain, we evaluate the SEBN-guided curriculum in a simulated robotics domain using the robosuite simulation environment. In particular we choose the Door task in which an agent needs to manipulate a robot to open a door. For our environment design space, we include two main parameters: the weight of the door (with 6 different settings) and whether the door has a latch or not. The specific robot that we choose to simulate is a Kinova3 arm. %An example of the environment can be seen in Fig. \ref{fig:robosuiteenv}. 
We define the following SEBN:
\begin{itemize}[noitemsep,topsep=0pt,parsep=0pt,partopsep=0pt]
    \item[$E$:] includes design parameters mass $\{0 -6 \}$ and latch \zeroone.
    \item[$\Psi$:] includes "move arm" \zeroone, "unlock" \zeroone, "door" \zot. 
    \item[\Target:] includes one observable metric \zeroone: whether the agent has successfully opened the door.
\end{itemize}

To learn our policies, we use PPO on a neural network with two hidden layers. We use the inbuilt observation setup and reward shaping in the robosuite environment to accelerate the learning process and we train our agents for 2 million timesteps. We evaluate our learned policies every 100,000 timesteps and the results of the experiment can be seen in Fig. \ref{fig:robosuiteres}. 

It is interesting to observe that the default training method stagnates after reaching a reward plateau. When viewing the actual behavior of the learned policy, this reward plateau is indicative of learning a policy of moving the arm towards the door handle but not actually moving the door. It is possible that either the weight of the door or the presence of a latch in harder environments prevents the agent from attempting the difficult action of applying force to the door to get an increased reward. Since the SEBN-guided curriculum will have started with at least some of its distribution in the easy case of a light door with no latch, the agent will have learned that opening the door can give a positive reward and transfer this behavior to more difficult environments. 

We also performed a generalization study, where we test our learned policies on an \emph{unseen} heavy door env not included during our learning process. We observed that with the policy learned by the SEBN-guided curriculum is easily transferred to more heavy doors (obtaining a reward of 240 on the heavy door env compared to 150 without a curriculum), in contrast to the other methods. % (see appendix for the full graph).


%=====================================================
%=====================================================
%=====================================================
%=====================================================
\section{Related Work} 
\paragraph{Automated Curriculum Generation}  Several topics relate to ordering tasks to improve learning performance. A few approaches have considered the problem of estimating agent skill competencies. In the context of education, in addition to ECD \cite{mislevy2003brief}, another approach close to ours is that of \citeauthor{greenEtal2011.iaai.learningASkillTeachingCurriculum} \cite{{greenEtal2011.iaai.learningASkillTeachingCurriculum}}, who used a BN to determine the next task for the human student.  
This approach is similar to \cite{kumar2024practice}, which considered an active learning problem in a robotics domain of choosing which skills to practice to maximize future task success, which involves estimating the competence of each skill and situating it in the task distribution through competence-aware planning. In contrast to our approach, they employ a simplified Bayesian time series model that does not relate environmental features with goal and skill competencies. This limits the applicability of their approach towards only choosing what skill to train and not the agent's environment. Similar to our selection process, \citeauthor{BalleraEtal2014.icetc.personalizingElearning} \cite{BalleraEtal2014.icetc.personalizingElearning} used a roulette wheel selection of tasks. 

A related area is the literature on Unsupervised Environment Design (UED) \cite{dennis2020emergent} and other developed mechanisms for curating environments based on a regret heuristic \cite{jiang2021replay}. In prior UED approaches such as PAIRED \cite{dennis2020emergent}, the agent's curriculum is generated using a regret-based heuristic. The heuristic is typically an estimate of the true regret, since the optimal policy is unknown. In PAIRED, this heuristic is calculated by learning an antagonistic policy and evaluating the difference between it and the protagonist policy. In contrast, like ACCEL \cite{parker2022evolving}, our method does not need to learn a second antagonistic policy and instead uses rollouts from a single agent to compute the next part of the curriculum. In contrast to ACCEL, %however, which relies on local changes to the task descriptor (evolutionary mutations) and subsequent regret pruning, 
our curriculum does not rely on local changes and can incorporate larger jumps in environment selection. Furthermore, while it is necessary to use rollouts on each environment for ACCEL to obtain a regret estimate, we can estimate success rates on unseen environments by leveraging the relationships encoded between competencies and environmental features in our SEBN.


\paragraph{Task Descriptors}
As mentioned in \secref{sec:taskdescriptors}, grouping tasks using features, i.e., task descriptors, are a well understood technique for task creation (\cite{rostamiEtAl20.jair.usingTaskDescriptions,isele2016task,narvekar2016source,konidaris2012transfer}).  The key idea in these works is to facilitate learning transfer by creating similar tasks that share common features.  These features can leave certain variables free during task construction that enable a family of similar tasks. 
Our work supplements prior work by adding the target of the task to the task descriptor, allowing the curriculum to emphasize subtasks.

To our knowledge, there has been limited previous work in integrating curriculum learning on both skills and environment features. However, it can be said that our research expands on the concept of using task descriptors in the creation of automated curricula. In \cite{patra2023relating}, a task-graph curricula is used to generate a curriculum over tasks and environmental features. However, they employ a simple greedy best-first search on the task-graph to choose an order for their curriculum. This is different from our approach that updates a distribution over the task-graph and dynamically adjusts this distribution based on data from rollouts.

\paragraph{Hierarchical Goal Networks}

The structure of our Skill Environment Bayesian network shares similarity to both goal skill networks and fault diagnosis networks. In fault diagnosis networks \cite{cai2017bayesian}, BNs are used to model the relationship between a set of sensors and a set of faults. In our case, the sensors are analogous to an SEBN's observable goal metrics, and the faults are analogous to an SEBN's skills. An SEBN can then be seen as a fault diagnosis network where different roll-outs are independent tests that determine what latent competencies may have not been mastered.




\paragraph{Expert guidance for RL training}
One of the limitations of the SEBN is its reliance on the expert-provided competencies. 
As noted, these could be derived from hierarchical approaches. 
But providing domain knowledge is common in many hierarchical RL settings.
Similar to our work,  \citeauthor{patra2022hierarchical} \cite{patra2022hierarchical}, provided a hierarchical learning structure.
This kind of expert knowledge is common in  imitation learning (e.g., \cite{zhang19leveraging} \cite{hussein2017imitation} \cite{le2018hierarchical}), where an expert human guides a learning agent.
Providing expert guidance is also common in Hierarchical RL approaches (e.g., \cite{ahmadiTaylorStone07.aamas.IFSA}) and in standard RL approaches (e.g., \cite{andreas2017modular}).
Finally, expert guidance was shown to be helpful for a sparse-reward task in an Object Oriented MDP setting \cite{abelEtAl15.icaps.goalBasedActionPriors}.


%\paragraph{Option discovery}

%\paragraph{Goal-Conditioned RL}
%goal-conditioned and OO RL (Abel et al. ICAPS 2015); 

%\paragraph{Unsupervised Environment Design}
%Unsupervised Experiment Design; sampling-based guidance (HER,PER); 




%=====================================================
%=====================================================
%=====================================================
%=====================================================
\section{Conclusion and Future Work}
We presented a novel method for generating curriculum over environmental features using a Skill-Environment Bayesian Network. This network is used to estimate agent competency level based off of past rollouts and can be used to infer estimated agent success rates on unseen environments. We demonstrate the effectiveness of this approach on a variety of domains.

For this work, we relied on a pre-defined set of skills and environmental features. In future work, we would like to extend the model to handle more open ended environments where we can add new environmental features or agent skills dynamically to the SEBN. It might be interesting to see if we can apply techniques such as GO-MTL \cite{kumar2012learning} for learning a latent space over tasks and approaches for detecting critical regions \cite{molina2020learn} to learn new skills.

There has also been a great deal of interest in the use of Large Language Models (LLMs) in planning domains for the purpose of automatically generating planning models. As SEBNs can be built from a heirarchical goal network, it might stand to reason that LLMs could also be used to automatically generate SEBNs from domain documents. Since there is much more leeway in the skills required (since our method supports latent competency nodes), it may be easier to generate these SEBNs than equivalent goal networks.

\begin{acks}
We thank the Basic Research Office of OUSD and NRL for funding this research.
\end{acks}

%%%%%%%%%%%%%%%%%%%%%%%%%%%%%%%%%%%%%%%%%%%%%%%%%%%%%%%%%%%%%%%%%%%%%%%%

%%% The next two lines define, first, the bibliography style to be 
%%% applied, and, second, the bibliography file to be used.

\newpage
\balance
\bibliographystyle{ACM-Reference-Format} 
\bibliography{sample}

%%%%%%%%%%%%%%%%%%%%%%%%%%%%%%%%%%%%%%%%%%%%%%%%%%%%%%%%%%%%%%%%%%%%%%%%
\clearpage
\setcounter{page}{1}
%\maketitlesupplementary

\appendix

\begin{center}
    {\Large{\textbf{Appendix}}}
\end{center}

%\section{Rationale}
%\label{sec:rationale}
% 
%Having the supplementary compiled together with the main paper means that:
% 
%\begin{itemize}
%\item The supplementary can back-reference sections of the main paper, for example, we can refer to \cref{sec:intro};
%\item The main paper can forward reference sub-sections within the supplementary explicitly (e.g. referring to a particular experiment); 
%\item When submitted to arXiv, the supplementary will already included at the end of the paper.
%\end{itemize}
% 
%To split the supplementary pages from the main paper, you can use \href{https://support.apple.com/en-ca/guide/preview/prvw11793/mac#:~:text=Delete%20a%20page%20from%20a,or%20choose%20Edit%20%3E%20Delete).}{Preview (on macOS)}, \href{https://www.adobe.com/acrobat/how-to/delete-pages-from-pdf.html#:~:text=Choose%20%E2%80%9CTools%E2%80%9D%20%3E%20%E2%80%9COrganize,or%20pages%20from%20the%20file.}{Adobe Acrobat} (on all OSs), as well as \href{https://superuser.com/questions/517986/is-it-possible-to-delete-some-pages-of-a-pdf-document}{command line tools}.




We organize the Appendix as follows: 
\begin{itemize}
    \item Appendix~\ref{sec:implementation} describes the implementation details. It describes the DNN architectures (VGG, ResNet, and ViT), feature extraction for linear probing, training, and evaluation details of both pre-training and linear probing in various experiments.
    
    \item Appendix~\ref{sec:datasets} provides details on the datasets used in this paper. In total, we use 9 datasets.

    \item Appendix~\ref{sec:nc_metrics} describes four neural collapse metrics ($\mathcal{NC}1 - \mathcal{NC}4$) used in this paper.

    \item Appendix~\ref{sec:mse_ce_supp} presents a comprehensive comparison between MSE and CE.

    \item Appendix~\ref{sec:details_proposition} contains proof on the implication of NC on entropy.

    \item Appendix~\ref{sec:comprehensive_results} provides a comprehensive comparison between the encoder and projector across different architectures. %It includes comprehensive results on 8 OOD datasets for different DNNs including VGG17, ResNet18, ResNet34, ViT-T, and ViT-S.

    \item Appendix~\ref{sec:analysis_entropy_reg} provides detailed analyses on entropy regularization and neural collapse.

    \item Additional experiments and analyses are summarized in Appendix~\ref{sec:additional_exp_supp}. The mechanisms of controlling NC have been examined.
    
    \item Appendix~\ref{sec:imagenet_100_classes} includes the list of 100 classes in the imageNet-100 dataset. %and confirms there is no overlap between ID and OOD datasets.
\end{itemize}




\section{Implementation Details}
\label{sec:implementation}

In this paper, we use several acronyms such as
\textbf{NC} : Neural Collapse, 
\textbf{ETF} : Equiangular Tight Frame,
\textbf{ID} : In-Distribution, 
\textbf{OOD} : Out-of-Distribution,
\textbf{LR} : Learning Rate, 
\textbf{WD} : Weight Decay,
\textbf{GAP} : Global Average Pooling,
\textbf{GN} : Group Normalization, 
\textbf{BN} : Batch Normalization, 
\textbf{WS} : Weight Standardization,  
\textbf{CE} : Cross Entropy, 
\textbf{MSE} : Mean Squared Error,
\textbf{FPR} : False Positive Rate.

We use the terms OOD generalization and OOD transfer interchangeably.


%We implemented our code in Python using PyTorch.


\subsection{Architectures}
\label{sec:arch_details}

\textbf{VGG.}
We modified the VGG-19 architecture to create our VGG-17 encoder. Additionally, we removed two fully connected (FC) layers before the final classifier head. And, we added an adaptive average pooling layer (nn.AdaptiveAvgPool2d), which allows the network to accept any input size while keeping the output dimensions the same. After VGG-17 encoder, we attached a projector consisting of two MLP layers ($512 \rightarrow 2048 \rightarrow 512$) and finally added a classifier head. 
We use ReLU activation between projector layers.
We replace BN with GN+WS in all layers. For GN, we use 32 groups in all layers.

\noindent
\textbf{ResNet.}
We used the entire ResNet-18 or ResNet-34 as the encoder and attached a projector ($512 \rightarrow 2048 \rightarrow 512$) similar to the VGG networks mentioned above. 
We replace BN with GN+WS in all layers. For GN, we use 32 groups in all layers.

\noindent
\textbf{ViT.}
We consider ViT-Tiny/Small (5.73M/21.85M parameters) as the encoder for our experiments. %We use embedding dimension of 192, depth of 18 (i.e., 18 ViT blocks) and 3 heads. 
The projector comprising two MLP layers configured as fixed ETF Simplex and added after the encoder.
%($192 \rightarrow 768 \rightarrow 192$) is added after the encoder.
Following~\cite{beyer2022better}, we omit the learnable position embeddings and instead use the fixed 2D sin-cos position embeddings. Other details adhere to the original ViT paper~\cite{dosovitskiy2020image}. 
\begin{enumerate} %[noitemsep, nolistsep, leftmargin=*]
    \item \textbf{ViT-Tiny Configuration:} patch size=16, embedding dimension=192, \# heads=3, depth=12. Projector has
    output dimension=192 and hidden dimension=768,
    ($192 \rightarrow 768 \rightarrow 192$). We use ReLU activation between projector layers.
    The number of parameters in ViT-Tiny $+$ projector is 6.02M.
    \item \textbf{ViT-Small Configuration:} patch size=16, embedding dimension=384, \# heads=6, depth=12. Projector has
    output dimension=384 and hidden dimension=1536,
    ($384 \rightarrow 1536 \rightarrow 384$). We use ReLU activation between projector layers.
    The number of parameters in ViT-Small $+$ projector is 23.03M.
\end{enumerate}


\subsection{Feature Extraction For Linear Probing}
\label{sec:feat_extract_details}
In experiments with CNNs, at each layer $l$, for each sample, we extract features of dimension $H_{l}\times W_{l}\times C_{l}$, where $H_{l}$, $W_{l}$, and $C_{l}$ denote the height, width and channel dimensions respectively. Next, following~\cite{sarfi2023simulated}, we apply $2\times 2$ adaptive average pooling on each spatial tensor ($H_{l}\times W_{l}$). After average pooling, features of dimension $2\times 2 \times C_{l}$ are flattened and converted into a vector of dimension $4C_{l}$. Finally, a linear probe is trained on the flattened vectors. 
In experiments with ViTs, 
following~\cite{raghu2021vision}, we apply global average-pooling (GAP) to aggregate image tokens excluding the class token and train a linear probe on top of GAP tokens.
We report the best error ($\%$) on the test dataset for linear probing at each layer.


\subsection{VGG Experiments}

\textbf{VGG ID Training:} For training VGG on ImageNet-100, we employ the AdamW optimizer with a LR of $6\times10^{-3}$ and WD of $5\times10^{-2}$ for batch size 512. The model is trained for 100 epochs using the Cosine Annealing LR scheduler with a linear warmup of 5 epochs. 
In all experiments, we use CE and entropy regularization ($\alpha=0.05$) losses. 
However, in some particular experiments comparing CE and MSE, we use MSE loss ($\kappa$=15, M=60) and entropy regularization loss ($\alpha=0.05$). 

\noindent
\textbf{VGG Linear Probing:} We use the AdamW optimizer with a flat LR of $1\times 10^{-3}$ and WD of $0$ for batch size 128. The linear probes are trained for 30 epochs. We use label smoothing of 0.1 with the cross-entropy loss. 


\subsection{ResNet Experiments}

\textbf{ResNet ID Training:} 
For training ResNet-18/34, we employ the AdamW optimizer with an LR of 0.01 and a WD of 0.05 for batch size 512. The model is trained for 100 epochs using the Cosine Annealing LR scheduler with a linear warmup of 5 epochs. 
We use CE and entropy regularization ($\alpha=0.05$) losses. 
%We use MSE ($\kappa$=15, M=60) and regularization losses. 

\noindent
\textbf{ResNet Linear Probing:} In the linear probing experiment, we use the AdamW optimizer with an LR of $1\times 10^{-3}$ and WD of $0$ for batch size 128. The linear probes are trained for 30 epochs. We use label smoothing of 0.1 with cross-entropy loss.


\subsection{ViT Experiments}

\textbf{ViT ID Training:} 
For training ViT-Tiny, we employ the AdamW optimizer with LR of $8\times10^{-4}$ and WD of $5\times10^{-2}$ for batch size 256. The LR is scaled for $n$ GPUs according to: $LR \times n \times \frac{batch size}{512}$. We use an LR of $4\times10^{-4}$ for ViT-Small when the batch size is 256.
We use the Cosine Annealing LR scheduler with warm-up (5 epochs). 
We train the ViT-Tiny/Small for 100 epochs using CE and entropy regularization ($\alpha=0.05$) losses.
%We train the ViT-Tiny/Small for 100 epochs using CE or MSE ($\kappa$=5, M=10 for MSE) and regularization losses.
Following~\cite{raghu2021vision, beyer2022better}, we omit class token and instead use GAP token by global average-pooling image tokens and feed GAP embeddings to the projector. 


\noindent
\textbf{ViT Linear Probing:} We use the AdamW optimizer with LR of $0.01$ and WD of $1\times 10^{-4}$ for batch size 512. The linear probes are trained for 30 epochs. We use label smoothing of 0.1 with cross-entropy loss.

\noindent
\textbf{Augmentation.}
We use random resized crop and random flip augmentations and $224 \times 224$ images as inputs to the DNNs.

In experiments with CE loss, we use label smoothing of $0.1$.


\subsection{Evaluation Criteria}

\textbf{\textit{FPR95.}}
The OOD detection performance is evaluated by the FPR (False Positive Rate) metric. In particular, we use FPR95 (FPR at 95\% True Positive Rate) that evaluates OOD detection performance by measuring the fraction of OOD samples misclassified as ID where threshold, $\lambda$ is chosen when the true positive rate is 95\%. 
Both OOD detection and OOD generalization tasks are evaluated on the \emph{same} OOD test set.



%$\mathbf{\Delta_{E \rightarrow P}}$.
\textbf{\textit{Percentage Change.}}
To capture percentage increase or decrease when switching from the encoder ($E$) to the projector ($P$), we use 
\[
\Delta_{E \rightarrow P} = \frac{(P - E)} {|E|} \times 100.
\]


\textbf{\textit{Normalization for different OOD datasets.}}
In our correlation analysis between NC and OOD detection/generalization (Fig.~\ref{fig:vis_abstract} and~\ref{fig:nc_resnet}), we use min-max normalization for layer-wise OOD detection errors and OOD generalization errors which enables comparison using different OOD datasets. For a given OOD dataset and a DNN consisting of total $L$ layers, let the OOD detection/ generalization error for a layer $l$ be $E_l$. For $L$ layers we have error vector $\mathbf{E} = [E_1, E_2, \cdots E_L]$ which is then normalized by
\[
\mathbf{E}_N = \frac{\mathbf{E} - \mathrm{min}(\mathbf{E})} {\mathrm{max}(\mathbf{E}) - \mathrm{min}(\mathbf{E})}.
\]


\textbf{\textit{Effective Rank.}}
We use RankMe~\cite{garrido2023rankme} to measure the effective rank of the embeddings.



\section{Datasets}
\label{sec:datasets}

\textbf{ImageNet-100.} 
ImageNet-100~\cite{tian2020contrastive} is a subset of ImageNet-1K~\cite{deng2009imagenet} and contains 100 ImageNet classes. It consists of 126689 training images ($224\times 224$) and 5000 test images.
The object categories present in ImageNet-100 are listed in Appendix~\ref{sec:imagenet_100_classes}.

\textbf{CIFAR-100.} 
CIFAR-100~\cite{krizhevsky2014cifar} is a dataset widely used in computer vision. It contains $60,000$ RGB images and $100$ classes, each containing $600$ images. The
dataset is split into $50,000$ training samples and $10,000$ test samples. The images in CIFAR-100 have a
resolution of $32\times 32$ pixels. 
Unlike CIFAR-10, CIFAR-100 has a higher level of granularity, with
more fine-grained classes such as flowers, insects, household items, and a variety of animals and vehicles.
%CIFAR-100~\cite{krizhevsky2014cifar} dataset is similar with CIFAR-10 but with 100 classes. And each class has 600 images. %The out-of-distribution accuracy will be computed with randomly selected 10 classes from CIFAR-100, which is the same protocol used in~\cite{masarczyk2023tunnel}. Note that the classes in CIFAR-100 are mutually exclusive with those in CIFAR-10. 
For linear probing, all samples from both the training and validation datasets were used.


\textbf{NINCO (No ImageNet Class Objects).} NINCO ~\cite{bitterwolf2023ninco} is a dataset with 64 classes. The dataset is curated to eliminate semantic overlap with ImageNet-1K dataset and is used to evaluate the OOD performance of the models pre-trained on imagenet-1K. The NINCO dataset has 5878 samples, and we split it into 4702 samples for training and 1176 samples for evaluation. We do not have a fixed number of samples per class for training and evaluation datasets.

\textbf{ImageNet-Rendition (ImageNet-R).} ImageNet-R incorporates distribution shifts using different artistic renditions of object classes from the original ImageNet dataset~\citep{hendrycks2021many}.
We use a variant of ImageNet-R dataset from~\cite{wang2022dualprompt}.
ImageNet-R is a challenging benchmark for continual learning, transfer learning, and OOD detection. It consists of classes with different styles and intra-class diversity and thereby poses significant distribution shifts for ImageNet-1K pre-trained models~\citep{wang2022dualprompt}.
It contains 200 classes, 24000 training images, and 6000 test images.

\textbf{CUB-200.} CUB-200 is composed of 200 different bird species~\cite{wah2011caltech}. The CUB-200 dataset comprises a total of 11,788 images, with 5,994 images allocated for training and 5,794 images for testing.

\textbf{Aircrafts-100.} Aircrafts or FGVCAircrafts dataset~\cite{maji2013fine} consists of 100 different aircraft categories and 10000 high-resolution images with 100 images per category. The training and test sets contain 6667 and 3333 images respectively.

\textbf{Oxford Pets-37.} The Oxford Pets dataset includes a total of 37 various pet categories, with an approximately equal number of images for dogs and cats, totaling around 200 images for each category~\cite{parkhi2012cats}.

\textbf{Flowers-102.} The Flowers-102 dataset contains 102 flower categories that can be easily found in the UK. Each category of the dataset contains 40 to 258 images.~\cite{nilsback2008automated}

\textbf{STL-10.} STL-10 has 10 classes with 500 training images and 800 test images per class~\cite{coates2011analysis}.


For all datasets, images are resized to $224 \times 224$ to train and evaluate DNNs.

%%%%%%%%%%%%%%%%%%%%%%%%%%%%%%%%%%%%%%%%%%


\section{Neural Collapse Metrics}
\label{sec:nc_metrics}

Neural Collapse (NC) describes a structured organization of representations in DNNs~\cite{papyan2020prevalence, kothapalli2023neural, zhu2021geometric, rangamani2023feature}.
%\begin{tcolorbox}[boxsep=1pt,left=2pt,right=2pt,top=0pt,bottom=0pt]
The following four criteria characterize Neural Collapse:
\begin{enumerate}
    \item \textbf{Feature Collapse} ($\mathcal{NC}1$): Features within each class concentrate around a single mean, with almost no variability within classes.
    \item \textbf{Simplex ETF Structure} ($\mathcal{NC}2$): Class means, when centered at the global mean, are linearly separable, maximally distant, and form a symmetrical structure on a hypersphere known as a Simplex Equiangular Tight Frame (Simplex ETF).
    \item \textbf{Self-Duality} ($\mathcal{NC}3$): The last-layer classifiers align closely with their corresponding class means, forming a self-dual configuration.
    \item \textbf{Nearest Class Mean Decision} ($\mathcal{NC}4$): The classifier operates similarly to the nearest class-center (NCC) decision rule, assigning classes based on proximity to the class means. 
\end{enumerate}
%\end{tcolorbox}


Here, we describe each NC metric used in our results. Let \( \mu_G \) denote the global mean and \( \mu_c \) the \( c \)-th class mean of the features, \( \{z_{c,i}\} \) at layer \( l \), defined as follows:
\[
\mu_G = \frac{1}{nC} \sum_{c=1}^C \sum_{i=1}^n z_{c,i}, \quad \mu_c = \frac{1}{n} \sum_{i=1}^n z_{c,i} \quad (1 \leq c \leq C).
\]
We drop the layer index \( l \) from notation for simplicity.

\noindent
\textbf{Within-Class Variability Collapse ($\mathcal{NC}1$):}  
It measures the relative size of the within-class covariance \( \Sigma_W \) with respect to the between-class covariance \( \Sigma_B \) of the DNN features:
\[
\Sigma_W = \frac{1}{nC} \sum_{c=1}^C \sum_{i=1}^n \left( z_{c,i} - \mu_c \right) \left( z_{c,i} - \mu_c \right)^\top \in \mathbb{R}^{d \times d},
\]
\[
\Sigma_B = \frac{1}{C} \sum_{c=1}^C \left( \mu_c - \mu_G \right) \left( \mu_c - \mu_G \right)^\top \in \mathbb{R}^{d \times d}.
\]

The $\mathcal{NC}1$ metric is defined as:

\[
\mathcal{NC}1 = \frac{1}{C} \operatorname{trace} \left( \Sigma_W \Sigma_B^{\dagger} \right),
\]
where \( \Sigma_B^{\dagger} \) is the pseudo-inverse of \( \Sigma_B \). Note that $\mathcal{NC}1$ is the most dominant indicator of neural collapse.

\noindent
\textbf{Convergence to Simplex ETF ($\mathcal{NC}2$):}  
It quantifies the \( \ell_2 \) distance between the normalized simplex ETF and the normalized \( WW^\top \), as follows:
\[
\mathcal{NC}2 := \left\| \frac{WW^\top}{\| WW^\top \|_F} - \frac{1}{\sqrt{C-1}} \left( I_C - \frac{1}{C} \mathbf{1}_C \mathbf{1}_C^\top \right) \right\|_F,
\]
where \( W \in \mathbb{R}^{C \times d} \) is the weight matrix of the learned classifier.

\noindent
\textbf{Convergence to Self-Duality ($\mathcal{NC}3$):}  
It measures the \( \ell_2 \) distance between the normalized simplex ETF and the normalized \( WZ \):
\[
\mathcal{NC}3 := \left\| \frac{WZ}{\| WZ \|_F} - \frac{1}{\sqrt{C-1}} \left( I_C - \frac{1}{C} \mathbf{1}_C \mathbf{1}_C^\top \right) \right\|_F,
\]
where \( Z = \left[ z_1 - \mu_G \; \cdots \; z_C - \mu_G \right] \in \mathbb{R}^{d \times C} \) is the centered class-mean matrix.


\textbf{Simplification to NCC ($\mathcal{NC}4$):} It measures the collapse of bias \( b \):
\[
\mathcal{NC}4 := \left\| b + W \mu_G   \right\|_2.
\]






%%%%%%%%%%%%%%%%%%%%%%%%%%%%%%%%%%%%%%%%%%%


\section{Mean Squared Error vs. Cross-Entropy}
\label{sec:mse_ce_supp}

Prior work~\cite{kornblith2021better} finds that MSE rivals CE in ID classification task but underperforms CE in OOD transfer. However, the comparison between CE and MSE in OOD detection task remains unexplored.
In this work, we find that CE significantly outperforms MSE in both OOD transfer and OOD detection tasks.
As shown in Table~\ref{tab:mse_ce_comp}, MSE underperforms CE by 6.74\% (absolute) in OOD detection and by 17.71\% (absolute) in OOD generalization. Our OOD generalization results are consistent with~\citet{kornblith2021better}.
CE also obtains lower ID error than MSE, thereby showing good overall performance.

In terms of inducing neural collapse, both MSE and CE are effective and achieve lower NC values (i.e., stronger NC). However, our results suggest that CE does a better job than MSE in enhancing NC without sacrificing OOD transfer. We find MSE to be sensitive to the hyperparameters.
The comparison on all OOD datasets is shown in Table~\ref{tab:main_results}.





\begin{table}[t]
\centering
  \caption{\textbf{Comparison between MSE and CE.} VGG17 networks are trained on \textbf{ImageNet-100} dataset (ID) and evaluated on 8 OOD datasets. For OOD generalization we report $\boldsymbol{\mathcal{E}}_{\text{GEN}}$ (\%) whereas for OOD detection we report $\boldsymbol{\mathcal{E}}_{\text{DET}}$ (\%), both are averaged over 8 OOD datasets. %$\downarrow$ indicates smaller values are better. 
  \textbf{A lower $\mathcal{NC}$ indicates stronger neural collapse.} $+\Delta_{E \rightarrow P}$ and $-\Delta_{E \rightarrow P}$ indicate \% increase and \% decrease respectively, when changing from the encoder ($E$) to projector ($P$). %\textcolor{brown}{CE outperforms MSE in both OOD transfer and OOD detection evaluations.}
  }
  \label{tab:mse_ce_comp}
  \centering
  \resizebox{\linewidth}{!}{
     \begin{tabular}{c|c|cccc|c|c}
     \hline %\hline
     \multicolumn{1}{c|}{\textbf{Method}} &
     \multicolumn{1}{c|}{$\boldsymbol{\mathcal{E}}_{\text{ID}}$} &
     \multicolumn{4}{c|}{\textbf{Neural Collapse}} &
     \multicolumn{1}{c|}{$\boldsymbol{\mathcal{E}}_{\text{GEN}}$} &
     \multicolumn{1}{c}{$\boldsymbol{\mathcal{E}}_{\text{DET}}$} \\
     & $\downarrow$ & $\mathcal{NC}1$ & $\mathcal{NC}2$ & $\mathcal{NC}3$ & $\mathcal{NC}4$ & Avg. $\downarrow$ & Avg. $\downarrow$ \\
     %\hline
     \toprule
     \rowcolor[gray]{0.9}
     \textbf{CE Loss} \\
     Projector & \textbf{12.62} & 0.393 & 0.490 & 0.468 & 0.316 & 66.36 & \textbf{65.10} \\
     Encoder & 15.52 & 2.175 & 0.603 & 0.616 & 5.364 & \textbf{41.85} & 87.62 \\
     \rowcolor{yellow!50}
     $\Delta_{E \rightarrow P}$ & -18.69 & -81.93 & -18.74 & -24.03 & -94.11 & +58.57 & -25.70 \\
    %\hline \hline
    \toprule
    \rowcolor[gray]{0.9}
    \textbf{MSE Loss} \\
    Projector & \textbf{14.04} & 0.469 & 0.743 & 0.279 & 0.382 & 70.87 & \textbf{71.84} \\
    %\hline
    Encoder & 14.74 & 2.267 & 0.843 & 0.673 & 10.773 & \textbf{59.56} & 88.88 \\
    %\hline
    \rowcolor{yellow!50}
    $\Delta_{E \rightarrow P}$ & -4.75 & -79.31 & -11.86 & -58.54 & -96.45 & +18.99 & -19.17 \\
    %\hline %\hline
    \bottomrule
    \end{tabular}}
\end{table}



%\begin{table*}[t]
\centering
  \caption{\textbf{ETF Fixed Projector Vs. Plastic Projector.} The VGG17 models are trained on \textbf{ImageNet-100} dataset (ID) and evaluated on 8 OOD datasets. The same color highlights the rows to compare.
  For OOD transfer we report $\boldsymbol{\mathcal{E}}_{\text{GEN}}$ (\%) whereas for OOD detection we report $\boldsymbol{\mathcal{E}}_{\text{DET}}$ (\%). %\textcolor{brown}{Fixed ETF projector shows higher transfer error (2.47\% absolute) than plastic projector but outperforms plastic projector in ID error (2.48\% absolute) and OOD detection error (8.9\% absolute).}
  } 
  \label{tab:plastic_proj}
  \centering
  \resizebox{\linewidth}{!}{
     \begin{tabular}{cc|cccc|ccccccccc}
     \hline %\hline
     \multicolumn{1}{c}{\textbf{Projector}} &
     \multicolumn{1}{c|}{$\boldsymbol{\mathcal{E}}_{\text{ID}} \downarrow$} &
     \multicolumn{4}{c|}{\textbf{Neural Collapse}} &
     \multicolumn{9}{c}{\textbf{OOD Datasets}} \\
    & IN & $\mathcal{NC}1$ & $\mathcal{NC}2$ & $\mathcal{NC}3$ & $\mathcal{NC}4$ & IN-R & CIFAR & Flowers & NINCO & CUB & Aircrafts & Pets & STL & Avg. \\
    & 100 &  &  &  &  & 200 & 100 & 102 & 64 & 200 & 100 & 37 & 10 & \\    
    \hline
    %% CE Loss
    \textbf{\textcolor{orange}{Transfer Error $\downarrow$}} \\
    %\rowcolor[gray]{0.9}
    \textcolor{blue}{\textbf{Plastic}} \\
    Projector & 15.10 & 0.498 & 0.515 & 0.428 & 1.422 & 87.52 & 64.83 & 79.71 & 53.32 & 87.00 & 93.46 & 48.76 & 28.04 & 67.83 \\

    \rowcolor{yellow!50}
    Encoder & 23.64 & 13.953 & 0.526 & 0.833 & 6.697 & \textbf{69.43} & \textbf{45.12} & \textbf{20.00} & \textbf{23.55} & \textbf{57.90} & \textbf{60.10} & 25.40 & 13.52 & \textbf{39.38} \\
    \hline %\hline

    %\rowcolor[gray]{0.9}
    \textcolor{blue}{\textbf{Fixed ETF}} \\

    Projector & \textbf{12.62} & 0.393 & 0.490 & 0.468 & 0.316 & 91.38 & 65.72 & 64.51 & 64.97 & 82.22 & 97.42 & 43.17 & 21.51 & 66.36 \\
    
    \rowcolor{yellow!50}
    %% ENCODER
    \textbf{Encoder} & 15.52 & 2.175 & 0.603 & 0.616 & 5.364 & 71.52 & 47.24 & 25.10 & 24.32 & 63.67 & 67.81 & \textbf{21.56} & 13.55 & 41.85 \\ % error

    \hline \hline
    \textbf{\textcolor{orange}{Detection Error $\downarrow$}} \\
    %\rowcolor[gray]{0.9}
    \textcolor{blue}{\textbf{Plastic}} \\
    \rowcolor{green!25}
    Projector & 15.10 & 0.498 & 0.515 & 0.428 & 1.422 & 63.05 & \textbf{47.87} & 62.45 & 70.07 & 80.88 & 98.95 & 89.37 & 79.25 & 74.00 \\
    
    Encoder & 23.64 & 13.953 & 0.526 & 0.833 & 6.697 & 81.27 & 98.82 & 93.33 & 86.48 & 79.98 & 99.40 & 91.25 & 93.88 & 90.55 \\
    \hline
    %\rowcolor[gray]{0.9}
    \textcolor{blue}{\textbf{Fixed ETF}} \\
    %% PROJECTOR
    \rowcolor{green!25}
    \textbf{Projector} & \textbf{12.62} & 0.393 & 0.490 & 0.468 & 0.316 & \textbf{60.85} & 48.23 & \textbf{42.35} & \textbf{67.69} & \textbf{56.51} & 99.04 & \textbf{76.32} & \textbf{69.84} & \textbf{65.10} \\
    %\hline
    
    %% ENCODER
    Encoder & 15.52 & 2.175 & 0.603 & 0.616 & 5.364 & 67.17 & 98.14 & 81.76 & 84.95 & 84.57 & 99.70 & 97.36 & 87.34 & 87.62 \\
    
    \hline \hline
    %\vspace{-2em}
    \end{tabular}}
\end{table*}








\begin{comment}

%%% Following results correspond to MSE Loss with Plastic Projector
\begin{table*}[t]
    

\centering
  \caption{\textbf{ETF Fixed Vs. Plastic Projector.} The VGGm-17 models ($F_{\psi}$($G_{\phi}$($H_{\theta}))$) are trained on \textbf{ImageNet-100} dataset (ID) and evaluated on 8 OOD datasets. In \textbf{encoder} method, the embeddings are extracted from the encoder ($H_{\theta}$) and before projector. And, in \textbf{projector} method, the embeddings are extracted after projector ($G_{\phi}$) and before output layer ($F_{\psi}$). For OOD transfer we report the top-1 error whereas for OOD detection we report the FPR95. $\uparrow$ indicates larger values are better and $\downarrow$ indicates smaller values are better. All values except neural collapse are percentages. \textbf{A lower $\mathcal{NC}$ indicates higher neural collapse. %$+\delta$ and $-\delta$ indicate \% increase and \% decrease respectively, when changing from encoder to projector.
  }
  } 
  \label{tab:plastic_proj}
  \centering
  \resizebox{\linewidth}{!}{
     \begin{tabular}{cc|ccc|ccccccccc}
     \hline %\hline
     \multicolumn{1}{c}{\textbf{Method}} &
     \multicolumn{1}{c|}{\textbf{ID Error}} &
     \multicolumn{3}{c|}{\textbf{Neural Collapse}} &
     \multicolumn{9}{c}{\textbf{OOD Datasets}} \\
    & IN & $\mathcal{NC}1$ &  $\mathcal{NC}2$ &  $\mathcal{NC}3$ & IN-R & CIFAR & Flower & NINCO & CUB & AirCrafts & Pet & STL & Avg. \\
    & 100 &  &  &  & 200 & 100 & 102 & 64 & 200 & 100 & 37 & 10 & \\    
    \hline \hline
    \textbf{Transfer Error $\downarrow$} \\
    Projector & \textbf{16.68} & 2.040 & 0.601 & 0.337 & 89.53 & 75.98 & 89.90 & 57.40 & 90.87 & 97.60 & 56.31 & 31.29 & 73.61 \\
     \hline
    \textbf{Encoder} & 19.42 & 3.969 & 0.552 & 0.705 & \textbf{77.72} & \textbf{56.18} & \textbf{36.08} & \textbf{29.93} & \textbf{65.69} & \textbf{73.27} & \textbf{28.59} & \textbf{16.55} & \textbf{48.00} \\ % error
    
    \hline \hline
    \textbf{Detection FPR $\downarrow$} \\
    \textbf{Projector} & \textbf{16.68} & 2.040 & 0.601 & 0.337  & \textbf{72.95} & \textbf{40.80} & \textbf{76.37} & \textbf{72.05} & \textbf{71.87} & \textbf{98.11} & \textbf{87.90} & \textbf{74.05} & \textbf{74.26} \\
    \hline
    Encoder & 19.42 & 3.969 & 0.552 & 0.705 & 98.25 & 99.95 & 88.63 & 97.06 & 98.24 & 87.55 & 97.33 & 98.90 & 95.74 \\
    \hline \hline
    %\vspace{-2em}
    \end{tabular}}
\end{table*}


\end{comment}


\section{Formal Proposition: Collapsing Implies Entropy $-\infty$}
\label{sec:details_proposition}

\begin{proposition}[Entropy under Class-Conditional Collapse]
Consider a mixture of $K$ class-conditional densities $\{p_{\ell,k}\}_{k=1}^K$ in $\mathbb{R}^{d_\ell}$, with mixture weights $\{\pi_k\}$. Suppose that for each $k$, there exists a family of densities $\{p_{\ell,k}(\epsilon) : \epsilon > 0\}$ such that
\[
\lim_{\epsilon \to 0} p_{\ell,k}(\epsilon) = \delta(z - \mu_{\ell,k})
\]
in the weak topology (i.e., they converge to a Dirac delta). Then
\[
\lim_{\epsilon \to 0} H\left(\sum_{k=1}^K \pi_k \, p_{\ell,k}(\epsilon)\right) = -\infty.
\]
\end{proposition}

\paragraph{Proof (Sketch).}
For each fixed $k$,
\[
\lim_{\epsilon \to 0} H(p_{\ell,k}(\epsilon)) = -\infty,
\]
because each $p_{\ell,k}(\epsilon)$ ``collapses'' its support around $\mu_{\ell,k}$. This is analogous to reducing variance to $0$ for a parametric family (e.g., a Gaussian with covariance $\epsilon I$).

The mixture’s differential entropy can be bounded above as
\[
H\left(\sum_{k} \pi_k \, p_{\ell,k}(\epsilon)\right) \leq \sum_{k} \pi_k H(p_{\ell,k}(\epsilon)) + \text{const},
\]
where the constant term arises from mixture overlap considerations (or the standard inequality $H(\sum_i q_i) \leq \sum_i \alpha_i H(q_i) + \log K$ for simpler forms).

Hence, if each $H(p_{\ell,k}(\epsilon)) \to -\infty$, the sum also diverges to $-\infty$. This demonstrates that if each class distribution collapses around its mean, the overall mixture’s differential entropy approaches $-\infty$.



%%%%%%%%%%%%%%%%%%%

\section{Comprehensive Results (Encoder Vs. Projector)}
\label{sec:comprehensive_results}

    
    
\begin{table*}[t!]
    \centering
    \small
    
    \scalebox{0.90}{
    \setlength{\tabcolsep}{1.0pt}
    \begin{tabular}{l c c c r | c c c c c c |c  c c }
    \toprule
    \multirow{1}{*}{Method} & \multirow{1}{*}{Recipe} & \multirow{1}{*}{Complexity} & \multirow{1}{*}{\# P.} & \multirow{1}{*}{\# T.P.}& MME & MMB &POPE & \multicolumn{1}{c} {SEED} & MMMU & MM-Vet& TQA & SQA-I  & \multicolumn{1}{c}{GQA} \\
    \midrule
    \rowcolor{gray!14}
    \multicolumn{14}{l}{\textbf{\textit{Encoder-based VLMs}}} \\ 
    OpenFlamingo~\cite{openflamingo} & \underline{PT, SFT}& Quadratic & 9B& 96.6\%  & - & 4.6 & - & - & - & - & 33.6 & - & - \\
    MiniGPT-4~\cite{minigpt} & \underline{PT, SFT}& Quadratic & 13B& 94.8\%  & 581.7 & 23.0 & - & - & -& 22.1 & - & - & 32.2  \\
    Qwen-VL~\cite{qwenvl} & \underline{PT, SFT}& Quadratic & 7B& 100.0\%  & - & 38.2 & - & 56.3 & - & - & 63.8 & 67.1 & 59.3\\ 
    LLaVA-Phi~\cite{llavaphi}  & \underline{PT, SFT}& Quadratic & 3B& 90.0\%  & 1335.1 & 59.8 & 85.0 & - & - & 28.9& 48.6 & 68.4 & - \\
    MobileVLM-3B~\cite{mobilevlm} & \underline{PT, SFT}& Quadratic & 3B& 90.0\%  & 1288.9 & 59.6 & 84.9 & - & - & - & 47.5 & 61.0 & 59.0  \\
    VisualRWKV~\cite{visualrwkv} & \underline{PT, SFT}&  \textbf{Linear} & 3B& 90.0\%  & 1369.2 & 59.5 & 83.1 & - & - & - & 48.7 & 65.3 & 59.6 \\
    VL-Mamba~\cite{vlmamba} & \underline{PT, SFT}&  \textbf{Linear} & 3B& 90.0\%  & 1369.6 & 57.0 & 84.4 & - & -& 32.6 & 48.9 & 65.4 & 56.2 \\
    Cobra~\cite{cobra} & \underline{PT, SFT}&  \textbf{Linear} & 3.5B& 82.6\%  & - & - & \textbf{88.4} & - & - & - & 58.2 & - & \textbf{62.3}\\
    \midrule
    \rowcolor{gray!14}
    \multicolumn{14}{l}{\textbf{\textit{Decoder-only VLMs}}} \\
    Fuyu-8B (HD)~\cite{fuyu} & \underline{PT, SFT}& Quadratic & 8B& 100.0\%  & 728.6 & 10.7 & 74.1 & - & - & 21.4 & - & - & -\\
    SOLO~\cite{solo} & \underline{PT, SFT}& Quadratic &  7B& 100.0\%   & 1001.3 & - & - & 64.4 & - & - & - & 73.3 & -   \\    
    Chameleon-7B~\cite{chameleon}  & \underline{PT, SFT}& Quadratic &  7B& 100.0\%   & 170 & 31.1 & - & 30.6 & 25.4 & 8.3 & 4.8 & 47.2 & -\\  
    EVE-7B~\cite{eve}  & \underline{PT, SFT}& Quadratic &  7B& 100.0\%  & 1217.3 & 49.5 & 83.6 & 61.3 & \underline{32.3} & 25.6& 51.9 & 63.0 & 60.8 \\
    Emu3~\cite{emu3} & \underline{PT, SFT}& Quadratic & 8B& 100.0\%  & - & 58.5 & 85.2 & \underline{68.2} & 31.6 & \underline{37.2} & \underline{64.7} & \underline{89.2} & 60.3\\
    HoVLE~\cite{hovle} & DT, PT, SFT & Quadratic & \textbf{2.6B}& 100.0\%  & \textbf{1433.5} & \textbf{71.9} & \underline{87.6} & \textbf{70.7} & \textbf{33.7} & \textbf{44.3} & \textbf{66.0} & \textbf{94.8} & \underline{60.9} \\
    \rowcolor{green!15}
    \name{} & \textbf{DT} & \textbf{Linear} & \underline{2.7B}& \underline{14.7\%}  &1303.5 & 57.2 & 85.2 & 62.9& 30.7  & 31.1 &47.7 & 79.2 & 57.4 \\
    \rowcolor{yellow!15}
    \name{} & \textbf{DT} & \underline{Hybrid} & \underline{2.7B}& \textbf{11.2\%}  & \underline{1371.1} & \underline{63.7} & 86.7 & 66.3 & \underline{32.3} & 36.9 & 55.1 & 86.9 & 59.3  \\
    
    \bottomrule
    \end{tabular}
    }
    \vspace{-1em}
    \caption{\textbf{Comparison with existing VLMs on general VLM benchmarks.} ``Recipe'' denotes the adopted training recipe. ``PT'', ``SFT'', and ``DT'' denote the pre-training, supervised fine-tuning, and distillation training, respectively. ``Complexity'' denotes the model computation complexity with respect to the number of tokens. ``\# P.'' denotes the number of total parameters. ``\# T.P.'' denotes the percentage of trainable parameters ($\frac{\text{trainable paramters}}{\text{total parameters}}$). The best performance is highlighted in \textbf{bold} and the second-best result is \underline{underlined}.}
    \label{tab:results_general}
    \end{table*}
 % VGG


\begin{figure*}[t]
    \centering
    \begin{subfigure}[b]{0.48\textwidth}
        \centering
        \includegraphics[width=\textwidth]{images/umap_enc_proj_10c_IN_ninco_64_vgg17.png}
        \caption{UMAP of Embeddings}
        \label{fig:umap_id_ood}
    \end{subfigure}
    \hfill
    \begin{subfigure}[b]{0.48\textwidth}
        \centering
        \includegraphics[width=\textwidth]{images/energy_ood_ninco.png}
        \caption{Energy Score Distribution}
        \label{fig:eng_id_ood}
    \end{subfigure}
    \caption{\textbf{ID \& OOD Data Visualization.} In \textbf{(a)}, The projector exhibits a greater separation between ID and OOD embeddings than the encoder. For clarity, we show 10 ImageNet classes as ID data and 64 classes from the NINCO dataset as OOD data. 
    In \textbf{(b)}, The projector achieves higher energy scores (and lower FPR95) for ID data.
    For ID and OOD datasets, we show ImageNet-100 and NINCO-64 respectively.
    }
    \label{fig:umap_eng_id_ood}
\end{figure*}






%\begin{figure}[t]
%    \centering
%    \includegraphics[width = 0.99\linewidth]{images/umap_enc_proj_10c_IN_ninco_64_vgg17.png}
%  \caption{\textbf{UMAP of ID and OOD Embeddings.} The projector exhibits a greater separation between ID and OOD embeddings. In contrast, the encoder shows a lot of overlap between ID and OOD embeddings. The embeddings are extracted from ImageNet-100 pre-trained VGG17. For clarity, we show 10 ImageNet classes as ID data and 64 classes from the NINCO dataset as OOD data.} 
%  \label{fig:umap_id_ood}
%\end{figure}


\begin{figure*}[t]
    \centering
    \begin{subfigure}[b]{0.48\textwidth}
        \centering
        \includegraphics[width=\textwidth]{images/energy_ood_flower.png}
        \caption{OOD Dataset: Flowers-102}
        \label{fig:eng_flower}
    \end{subfigure}
    \hfill
    \begin{subfigure}[b]{0.48\textwidth}
        \centering
        \includegraphics[width=\textwidth]{images/energy_ood_stl.png}
        \caption{OOD Dataset: STL-10}
        \label{fig:eng_stl}
    \end{subfigure}
    \caption{\textbf{Energy Score Distribution.}
    The projector creates a greater separation between ID and OOD data and achieves a lower FPR95 than the encoder. For better OOD detection, ID data should obtain higher energy scores than OOD data. For ID and OOD datasets, we show ImageNet-100 and Flowers-102/ STL-10 respectively. The energy scores are calculated based on logits from the VGG17 model pre-trained on ImageNet-100.
    }
    \label{fig:more_eng_id_ood}
\end{figure*}











%\begin{figure}[t]
%    \centering
%    \includegraphics[width = 0.99\linewidth]{images/energy_ood_flower.png}
%  \caption{\textbf{Energy Score Distribution.} The projector creates a greater separation between ID and OOD data and achieves a lower FPR95 than the encoder. For better OOD detection, ID data should obtain higher energy than OOD data.
%  For ID and OOD datasets, we show ImageNet-100 and Flowers-102 respectively. The energy scores are calculated based on logits from the VGG17 model pre-trained on ImageNet-100.} 
%  \label{fig:energy_id_ood}
%\end{figure}


\subsection{VGG Experiments}

The detailed VGG17 results are given in Table~\ref{tab:main_results}. VGG results demonstrate that the encoder effectively mitigates NC for OOD generalization and the projector builds collapsed features and excels at the OOD detection task. The results also confirm that NC properties can be built using both CE and MSE loss functions.

\textbf{Qualitative Comparison.} 
We compare and visualize encoder embeddings and projector embeddings using UMAP. We also visualize the energy score distribution of ID and OOD data. The analysis is based on the VGG17 model pre-trained on the ImageNet-100 (ID) dataset and evaluated on OOD datasets: NINCO-64, Flowers-102, and STL-10. We observe the following:
\begin{itemize}[noitemsep, nolistsep, leftmargin=*]
    \item In Fig.~\ref{fig:umap_id_ood}, the UMAP shows that projector embeddings nicely separate ID and OOD sets whereas encoder embeddings exhibit substantial overlap between ID and OOD sets. This demonstrates that, unlike the encoder, the projector can intensify NC and is adept at OOD detection.

    \item We show the energy distribution of ID and OOD sets in Fig.~\ref{fig:eng_id_ood} and~\ref{fig:more_eng_id_ood}. In all comparisons, we observe that the projector outperforms the encoder in separating ID and OOD sets based on energy scores.
    
\end{itemize}




\subsection{ResNet Experiments}

\begin{table*}[ht]
\centering
  \caption{\textbf{Comprehensive ResNet Results.} ResNet models are trained on \textbf{ImageNet-100} dataset (ID) and evaluated on 8 OOD datasets. %In the \textbf{encoder} method, the embeddings are extracted from the encoder and before the projector. And, in \textbf{projector} method, the embeddings are extracted after the projector and before the classifier head.
  For OOD transfer we report $\boldsymbol{\mathcal{E}}_{\text{GEN}}$ (\%) whereas for OOD detection we report $\boldsymbol{\mathcal{E}}_{\text{DET}}$ (\%). All metrics except NC are reported in percentage.
  \textbf{A lower $\mathcal{NC}$ indicates stronger neural collapse.} %\textcolor{brown}{Across all ResNets, the encoder enhances OOD transfer and the projector improves OOD detection.}
  } 
  \label{tab:resnet_results}
  \centering
  \resizebox{\linewidth}{!}{
     \begin{tabular}{cc|cccc|ccccccccc}
     \hline %\hline
     \multicolumn{1}{c}{\textbf{Model}} &
     \multicolumn{1}{c|}{$\boldsymbol{\mathcal{E}}_{\text{ID}} \downarrow$} &
     \multicolumn{4}{c|}{\textbf{Neural Collapse}} &
     \multicolumn{9}{c}{\textbf{OOD Datasets}} \\
    & IN & $\mathcal{NC}1$ &  $\mathcal{NC}2$ &  $\mathcal{NC}3$ & $\mathcal{NC}4$ & IN-R & CIFAR & Flowers & NINCO & CUB & Aircrafts & Pets & STL & Avg. \\
    & 100 &  &  &  &  & 200 & 100 & 102 & 64 & 200 & 100 & 37 & 10 & \\    
    %\hline \hline
    \toprule
    \textcolor{blue}{\textbf{ResNet18}} \\
    \textcolor{orange}{\textbf{Transfer Error $\downarrow$}} \\
    %% PROJECTOR
    Projector & \textbf{16.14} & 0.341 & 0.456 & 0.306 & 0.540 & 86.65 & 60.33 & 63.92 & 50.09 & 81.79 & 94.36 & 43.15 & 24.32 & 63.08 \\
    
    \rowcolor[gray]{0.9}
    %% ENCODER
    \textbf{Encoder} & 20.14 & 1.762 & 0.552 & 0.555 & 10.695 & \textbf{74.17} & \textbf{53.33} & \textbf{31.37} & \textbf{28.15} & \textbf{68.85} & \textbf{81.61} & \textbf{27.72} & \textbf{16.56} & \textbf{47.72} \\ % error

    \midrule
    \textcolor{orange}{\textbf{Detection Error $\downarrow$}} \\
    \rowcolor[gray]{0.9}
    %% PROJECTOR
    \textbf{Projector} & \textbf{16.14} & 0.341 & 0.456 & 0.306 & 0.540 & \textbf{67.92} & \textbf{61.21} & \textbf{71.18} & \textbf{71.09} & \textbf{23.20} & \textbf{99.28} & \textbf{81.41} & \textbf{82.29} & \textbf{69.70} \\
    %\hline
    
    %% ENCODER
    Encoder & 20.14 & 1.762 & 0.552 & 0.555 & 10.695 & 71.50 & 96.44 & 86.27 & 84.78 & 65.48 & 99.43 & 95.86 & 89.63 & 86.17 \\
    
    \hline \hline
    \textcolor{blue}{\textbf{ResNet34}} \\
    \textcolor{orange}{\textbf{Transfer Error $\downarrow$}} \\
    Projector & \textbf{14.54} & 0.252 & 0.672 & 0.294 & 0.324 & 83.93 & 58.65 & 64.41 & 44.05 & 81.65 & 93.58 & 43.64 & 22.87 & 61.60 \\
    
    \rowcolor[gray]{0.9}
    \textbf{Encoder} & 17.20 & 0.737 & 0.634 & 0.871 & 22.587 & \textbf{76.97} & \textbf{54.45} & \textbf{41.47} & \textbf{33.33} & \textbf{71.25} & \textbf{82.00} & \textbf{29.25} & \textbf{16.45} & \textbf{50.65} \\
    
    \hline
    \textcolor{orange}{\textbf{Detection Error $\downarrow$}} \\
    \rowcolor[gray]{0.9}
    \textbf{Projector} & \textbf{14.54} & 0.252 & 0.672 & 0.294 & 0.324 & \textbf{61.72} & \textbf{60.05} & \textbf{47.94} & \textbf{66.24} & \textbf{67.59} & \textbf{98.35} & \textbf{83.78} & \textbf{78.49} & \textbf{70.52} \\

    Encoder & 17.20 & 0.737 & 0.634 & 0.871 & 22.587 & 69.67 & 93.07 & 70.59 & 76.87 & 83.02 & 99.34 & 97.17 & 90.75 & 85.06 \\
    %\hline
    \bottomrule
    %\vspace{-2em}
    \end{tabular}}
\end{table*} % ResNet

\begin{figure}[t]
    \centering
    \includegraphics[width = 0.99\linewidth]{images/nc_ood_detect_transfer_corr_resnet_updated.png}
  \caption{Lower NC1 values (indicating stronger neural collapse) correlate with lower OOD detection error but higher OOD transfer error, and vice versa. This suggests that stronger neural collapse improves OOD detection, while weaker neural collapse enhances OOD generalization. We analyze various layers of \textbf{ResNet18}, pre-trained on ImageNet-100 (ID), and evaluate them on four OOD datasets. $R$ denotes the Pearson correlation coefficient.
  %NC exhibits a positive correlation with OOD detection error and a negative correlation with OOD transfer error when we analyze different layers of ResNet18 models that are pre-trained on ImageNet-100 (ID) and evaluated on four OOD datasets. $R$ denotes the Pearson correlation coefficient.
  } 
  \label{fig:nc_resnet}
\end{figure}



The detailed ResNet18/34 results are given in Table~\ref{tab:resnet_results}. Our findings validate that NC can be controlled in various ResNet architectures for improving OOD detection and OOD generalization performance.
Additionally, NC shows a strong correlation with OOD detection and OOD generalization as illustrated in Fig.~\ref{fig:nc_resnet}.

\begin{figure}[h]
    \centering
    \includegraphics[width = 0.99\linewidth]{images/umap_enc_proj_10c_ce_20_30_resnet.png}
  \caption{\textbf{Visualization of Embedding (ResNet18).} In this UMAP, projector embeddings exhibit greater neural collapse ($\mathcal{NC}1=0.341$) than the encoder embeddings ($\mathcal{NC}1=1.762$) as indicated by the formation of tight clusters around class-means. For clarity, we highlight 10 ImageNet classes by distinct colors. The embeddings are extracted from ImageNet-100 pre-trained ResNet18.} 
  \label{fig:umap_vis_resnet}
\end{figure}


We also visualize embeddings extracted from the encoder and projector of the ResNet18 model. As depicted in Fig.~\ref{fig:umap_vis_resnet}, projector embeddings exhibit much greater neural collapse than encoder embeddings.



\subsection{ViT Experiments} 

\begin{table*}[ht]
\centering
  \caption{\textbf{Comprehensive ViT Results.} ViT-Tiny (6.02M) and ViT-Small (23.03M) are trained on \textbf{ImageNet-100} dataset (ID) and evaluated on 8 OOD datasets. %In the \textbf{encoder} method, the embeddings are extracted from the encoder and before the projector. And, in \textbf{projector} method, the embeddings are extracted after the projector and before the classifier head. 
  For OOD transfer we report $\boldsymbol{\mathcal{E}}_{\text{GEN}}$ (\%) whereas for OOD detection we report $\boldsymbol{\mathcal{E}}_{\text{DET}}$ (\%). All metrics except NC are reported in percentage. \textbf{A lower $\mathcal{NC}$ indicates stronger neural collapse.} %\textcolor{brown}{Across all ViT models, the encoder enhances OOD transfer whereas the projector improves OOD detection.} 
  } 
  \label{tab:vit_results}
  \centering
  \resizebox{\linewidth}{!}{
     \begin{tabular}{cc|cccc|ccccccccc}
     \hline %\hline
     \multicolumn{1}{c}{\textbf{Model}} &
     \multicolumn{1}{c|}{$\boldsymbol{\mathcal{E}}_{\text{ID}} \downarrow$} &
     \multicolumn{4}{c|}{\textbf{Neural Collapse}} &
     \multicolumn{9}{c}{\textbf{OOD Datasets}} \\
    & IN & $\mathcal{NC}1$ &  $\mathcal{NC}2$ &  $\mathcal{NC}3$ & $\mathcal{NC}4$ & IN-R & CIFAR & Flowers & NINCO & CUB & Aircrafts & Pets & STL & Avg. \\
    & 100 &  &  &  &  & 200 & 100 & 102 & 64 & 200 & 100 & 37 & 10 & \\    
    %\hline \hline
    \toprule
    \textcolor{blue}{\textbf{ViT-Tiny}} \\
    \textcolor{orange}{\textbf{Transfer Error $\downarrow$}} \\
    %% PROJECTOR
    Projector & \textbf{32.04} & 2.748 & 0.609 & 0.798 & 1.144 & 87.37 & 60.71 & 64.61 & 39.71 & 80.00 & 92.00 & 54.27 & 29.55 & 63.53 \\
    
    \rowcolor[gray]{0.9}
    %% ENCODER
    \textbf{Encoder} & 33.94 & 5.769 & 0.748 & 0.847 & 2.332 & \textbf{82.28} & \textbf{52.00} & \textbf{42.94} & \textbf{30.36} & \textbf{63.15} & \textbf{84.31} & \textbf{44.86} & \textbf{21.13} & \textbf{52.63} \\

    \midrule
    \textcolor{orange}{\textbf{Detection Error $\downarrow$}} \\
    \rowcolor[gray]{0.9}
    %% PROJECTOR
    \textbf{Projector} & \textbf{32.04} & 2.748 & 0.609 & 0.798 & 1.144 & \textbf{81.12} & \textbf{60.81} & \textbf{77.55} & \textbf{82.40} & \textbf{79.05} & 99.10 & \textbf{95.15} & 90.06 & \textbf{83.16} \\
    
    %% ENCODER
    Encoder & 33.94 & 5.769 & 0.748 & 0.847 & 2.332 & 83.80 & 96.76 & 87.65 & 93.11 & 82.14 & 99.10 & 95.75 & \textbf{88.79} & 90.89 \\
    \hline \hline
    \textcolor{blue}{\textbf{ViT-Small}} \\
    \textcolor{orange}{\textbf{Transfer Error $\downarrow$}} \\
    
    Projector & \textbf{31.28} & 0.822 & 0.522 & 0.712 & 0.962 & 86.57 & 58.46 & 64.51 & 39.20 & 78.25 & 90.70 & 53.86 & 29.30 & 62.61 \\
    
    \rowcolor[gray]{0.9}
    \textbf{Encoder} & 33.40 & 1.610 & 0.601 & 0.740 & 2.814 & \textbf{80.53} & \textbf{49.68} & \textbf{40.49} & \textbf{29.93} & \textbf{61.08} & \textbf{81.28} & \textbf{44.45} & \textbf{20.98} & \textbf{51.05} \\
    \hline
    
    \textcolor{orange}{\textbf{Detection Error $\downarrow$}} \\
    \rowcolor[gray]{0.9}
    \textbf{Projector} & \textbf{31.28} & 0.822 & 0.522 & 0.712 & 0.962 & \textbf{76.03} & \textbf{58.79} & \textbf{75.20} & \textbf{81.97} & \textbf{82.46} & \textbf{98.50} & 95.42 & \textbf{88.74} & \textbf{82.14} \\
    
    Encoder & 33.40 & 1.610 & 0.601 & 0.740 & 2.814 & 82.47 & 96.84 & 90.39 & 92.60 & 86.00 & 99.25 & \textbf{94.36} & 89.04 & 91.37 \\
    
    \bottomrule
    %\vspace{-2em}
    \end{tabular}}
\end{table*} %% ViT

As shown in Table~\ref{tab:vit_results}, the projector outperforms the encoder in OOD detection by absolute 7.73\% (ViT-Tiny) and 9.23\% (ViT-Small). Whereas the encoder outperforms the projector in OOD transfer by absolute 10.90\% (ViT-Tiny) and 11.56\% (ViT-Small). This demonstrates that controlling NC improves OOD detection and generalization in ViTs. 



\section{Analysis on Entropy Regularization}
\label{sec:analysis_entropy_reg}

\begin{table*}[t]
\centering
  \caption{\textbf{Entropy Regularization Vs. No Entropy Regularization.} VGG17 models are pre-trained on \textbf{ImageNet-100} dataset (ID) and evaluated on 8 OOD datasets. Entropy regularization loss with a coefficient, $\alpha$ is applied in the last encoder layer. The same color highlights the rows to compare. All metrics except NC are reported in \%. The lower the NC value, the stronger the neural collapse. For OOD transfer we report $\boldsymbol{\mathcal{E}}_{\text{GEN}}$ (\%) whereas for OOD detection we report $\boldsymbol{\mathcal{E}}_{\text{DET}}$ (\%).
  %\textcolor{brown}{Using entropy penalty enhances OOD transfer by 2.71\% (absolute), OOD detection by 2.36\% (absolute), and ID performance by 0.84\% (absolute).}
  } 
  \label{tab:reg_vs_no_reg}
  \centering
  \resizebox{\linewidth}{!}{
     \begin{tabular}{cc|cccc|ccccccccc}
     \hline %\hline
     \multicolumn{1}{c}{\textbf{Method}} &
     \multicolumn{1}{c|}{$\boldsymbol{\mathcal{E}}_{\text{ID}} \downarrow$} &
     \multicolumn{4}{c|}{\textbf{Neural Collapse}} &
     \multicolumn{9}{c}{\textbf{OOD Datasets} $\downarrow$} \\
    & IN & $\mathcal{NC}1$ & $\mathcal{NC}2$ & $\mathcal{NC}3$ & $\mathcal{NC}4$ & IN-R & CIFAR & Flowers & NINCO & CUB & Aircrafts & Pets & STL & Avg. \\
    & 100 &  &  &  &  & 200 & 100 & 102 & 64 & 200 & 100 & 37 & 10 & \\    
    \hline
    %% CE Loss
    \textbf{\textcolor{orange}{Transfer Error $\downarrow$}} \\
    %\rowcolor[gray]{0.9}
    \textcolor{blue}{\textbf{No Reg.} ($\mathbf{\alpha=0}$)} \\
    Projector & 13.46 & 0.260 & 0.636 & 0.369 & 0.883 & 84.30 & 60.73 & 65.69 & 45.15 & 82.90 & 93.73 & 40.56 & 22.85 & 61.99 \\
    \rowcolor{yellow!50}
    Encoder & 15.24 & 1.308 & 0.719 & 0.619 & 5.184 & 73.52 & 49.26 & 37.06 & 25.51 & 64.31 & 69.58 & 22.24 & 15.00 & 44.56 \\
    \hline %\hline

    
    \textcolor{blue}{\textbf{Reg.} ($\mathbf{\alpha=0.05}$)} \\

    Projector & \textbf{12.62} & 0.393 & 0.490 & 0.468 & 0.316 & 91.38 & 65.72 & 64.51 & 64.97 & 82.22 & 97.42 & 43.17 & 21.51 & 66.36 \\
    
    \rowcolor{yellow!50}
    %% ENCODER
    \textbf{Encoder} & 15.52 & 2.175 & 0.603 & 0.616 & 5.364 & \textbf{71.52} & \textbf{47.24} & \textbf{25.10} & 24.32 & 63.67 & \textbf{67.81} & \textbf{21.56} & \textbf{13.55} & \textbf{41.85} \\ % error

    \hline
    \textcolor{blue}{\textbf{Reg.} ($\mathbf{\alpha=0.1}$)} \\
    Projector & 13.04 & 0.428 & 0.671 & 0.340 & 0.320 & 93.62 & 66.00 & 55.29 & 79.25 & 81.84 & 97.09 & 46.96 & 23.00 & 67.88 \\

    \rowcolor{yellow!50}
    Encoder & 16.12 & 2.861 & 0.538 & 0.636 & 6.677 & 73.05 & 48.61 & 27.84 & \textbf{22.62} & \textbf{61.91} & 70.21 & 22.87 & 13.83 & 42.62 \\
    
    \hline \hline
    \textbf{\textcolor{orange}{Detection Error $\downarrow$}} \\
    %\rowcolor[gray]{0.9}
    \textcolor{blue}{\textbf{No Reg.} ($\mathbf{\alpha=0}$)} \\
    \rowcolor{green!25}
    Projector & 13.46 & 0.260 & 0.636 & 0.369 & 0.883 & 65.22 & 54.32 & 45.20 & 67.18 & 52.37 & 98.41 & 84.38 & 72.58 & 67.46 \\
    
    Encoder & 15.24 & 1.308 & 0.719 & 0.619 & 5.184 & 74.22 & 99.75 & 85.10 & 88.52 & 92.99 & 98.59 & 95.34 & 92.14 & 90.83 \\
    \hline
    %\rowcolor[gray]{0.9}
    \textcolor{blue}{\textbf{Reg.} ($\mathbf{\alpha=0.05}$)} \\
    %% PROJECTOR
    \rowcolor{green!25}
    \textbf{Projector} & \textbf{12.62} & 0.393 & 0.490 & 0.468 & 0.316 & \textbf{60.85} & \textbf{48.23} & \textbf{42.35} & 67.69 & 56.51 & 99.04 & \textbf{76.32} & \textbf{69.84} & \textbf{65.10} \\
    %\hline
    
    %% ENCODER
    Encoder & 15.52 & 2.175 & 0.603 & 0.616 & 5.364 & 67.17 & 98.14 & 81.76 & 84.95 & 84.57 & 99.70 & 97.36 & 87.34 & 87.62 \\

    \hline
    \textcolor{blue}{\textbf{Reg.} ($\mathbf{\alpha=0.1}$)} \\
    \rowcolor{green!25}
    Projector & 13.04 & 0.428 & 0.671 & 0.340 & 0.320 & 61.13 & 54.69 & 43.14 & \textbf{64.63} & \textbf{50.73} & 98.74 & 82.42 & 71.51 & 65.87 \\

    Encoder & 16.12 & 2.861 & 0.538 & 0.636 & 6.677 & 68.72 & 94.67 & 85.78 & 87.76 & 85.49 & 98.92 & 95.15 & 86.28 & 87.85 \\
    
    %\hline \hline
    \bottomrule
    %\vspace{-2em}
    \end{tabular}}
\end{table*}


Table~\ref{tab:reg_vs_no_reg} presents the detailed comparison between a model using the entropy regularization vs another model omitting it. We observe that using entropy penalty enhances OOD transfer by 2.71\% (absolute), OOD detection by 2.36\% (absolute), and ID performance by 0.84\% (absolute).

Additionally, we analyze the impact of the entropy regularization loss coefficient on the ID and OOD transfer. Table~\ref{tab:entropy_loss_coeff} shows that increasing coefficient increases OOD transfer and rank of embeddings. This suggests that entropy regularization helps encode diverse features and reduce redundant features, encouraging utilization of all dimensions. Although entropy regularization is not sensitive to coefficient, over-regularization may hurt ID performance. Thereby, any non-aggressive coefficient can maintain good performance in both ID and OOD tasks. %Typically, a coefficient of $0.1$ or less works fine.


% \textbf{Training Dynamics.}
We also analyze the impact of entropy regularization on encoder embeddings during the training phase. During each training epoch, we measure the NC1 criterion, entropy, and effective rank of encoder embeddings. These experiments are computationally intensive for large-scale datasets. Therefore, we perform small-scale experiments where we train VGG17 models on the ImageNet-10 (10 ImageNet classes) subset for 100 epochs. We evaluate two cases: one with entropy regularization and another without entropy regularization.


The results are illustrated in Fig.~\ref{fig:nc_dynamics}.
%The impact of entropy regularization on NC1 is exhibited in Fig.~\ref{fig:nc_dynamics}. 
We find that entropy regularization achieves higher NC1 values during training compared to the model without any regularization. Thus, it helps mitigate NC during training, thereby contributing to OOD generalization. These findings align with our theoretical analysis showing entropy as an effective mechanism to prevent NC in the encoder.


Entropy regularization also increases the entropy and effective rank of the encoder embeddings. 
This demonstrates that entropy regularization helps encode diverse features, ensuring the features remain sufficiently ``spread out.''


Without the entropy regularization, the entropy of encoder embeddings does not improve. Also, the effective rank ends up at a low value (as low as the number of ID classes). The low rank is a sign of strong neural collapse and suggests that the encoder uses a few feature dimensions to encode information with huge redundancy in other dimensions. This degeneracy of embeddings impairs OOD transfer.
Entropy regularization counteracts this and improves OOD transfer.




\begin{table*}[t]
\centering
  \caption{\textbf{Entropy Regularization Loss Coefficient.} VGG17 models are pre-trained on the ImageNet-10 (10 ImageNet classes) ID dataset and evaluated on 8 OOD datasets. %via linear probing. %Reported is the top-1 error (\%). %Increasing the entropy regularization loss coefficient increases embedding rank and decreases OOD transfer error. 
  $\alpha$ denotes the entropy regularization loss coefficient. We use a regular VGG17 network without the projector to focus on entropy regularization. Effective rank corresponds to penultimate embeddings. For OOD generalization, we report $\boldsymbol{\mathcal{E}}_{\text{GEN}}$ (\%).
  } 
  \label{tab:entropy_loss_coeff}
  \centering
  \resizebox{\linewidth}{!}{
     \begin{tabular}{ccc|ccccccccc}
     \hline %\hline
     \multicolumn{1}{c}{\textbf{Reg. Coeff.}} &
     \multicolumn{1}{c}{$\boldsymbol{\mathcal{E}}_{\text{ID}} \downarrow$} &
     \multicolumn{1}{c|}{\textbf{Rank} $\uparrow$} &
     \multicolumn{9}{c}{$\boldsymbol{\mathcal{E}}_{\text{GEN}} \downarrow$} \\
    $\mathbf{\alpha}$ & IN & IN & IN-R & CIFAR & Flowers & NINCO & CUB & Aircrafts & Pets & STL & Avg. \\
    % \hline %\hline
    & 10 & 10 & 200 & 100 & 102 & 64 & 200 & 100 & 37 & 10 & \\ 
    \hline
    %0 & \textbf{90.80} & 2211.99 & 5.38 & 16.23 & 27.55 & 34.44 & 13.24 & 10.68 & 19.68 & 50.56 & 22.22 \\
    0 & \textbf{9.20} & 2211.99 & 94.62 & 83.77 & 72.45 & 65.56 & 86.76 & 89.32 & 80.32 & 49.44 & 77.78 \\
    %0.1 & 90.20 & 2964.39 & 9.28 & 24.42 & 42.06 & 49.91 & 20.04 & 15.87 & 27.28 & 60.21 & 31.13 \\
    0.1 & 9.80 & 2964.39 & 90.72 & 75.58 & 57.94 & 50.09 & 79.96 & 84.13 & 72.72 & 39.79 & 68.87 \\
    %0.2 & 89.80 & 3170.92 & 9.75 & 27.23 & 42.16 & 49.15 & 20.52 & 15.84 & 29.57 & \textbf{62.29} & 32.06 \\
    0.2 & 10.20 & 3170.92 & 90.25 & 72.77 & 57.84 & 50.85 & 79.48 & 84.16 & 70.43 & \textbf{37.71} & 67.94 \\
    %%0.3 & 89.20 & -- & 10.23 & 28.70 & 46.08 & 50.51 & 21.56 & 16.74 & 30.34 & 61.61 & 33.22 \\
    %0.6 & 88.00 & 3761.33 & \textbf{11.67} & 31.27 & \textbf{49.71} & 52.55 & \textbf{22.90} & 17.43 & \textbf{32.27} & 61.00 & 34.85 \\
    0.6 & 12.00 & 3761.33 & \textbf{88.33} & 68.73 & \textbf{50.29} & 47.45 & \textbf{77.10} & 82.57 & \textbf{67.73} & 39.00 & 65.15 \\
    %1.0 & 87.20 & \textbf{4815.32} & 11.62 & \textbf{32.19} & \textbf{49.71} & \textbf{52.89} & 22.52 & \textbf{18.36} & 31.73 & 61.26 & \textbf{35.04} \\
    1.0 & 12.80 & \textbf{4815.32} & 88.38 & \textbf{67.81} & \textbf{50.29} & \textbf{47.11} & 77.48 & \textbf{81.64} & 68.27 & 38.74 & \textbf{64.96} \\
    \hline %\hline
    %\vspace{-2em}
    \end{tabular}
    }
\end{table*}



\begin{figure*}[t]
    \centering
    \begin{subfigure}[b]{0.45\textwidth}
        \centering
        \includegraphics[width=\textwidth]{images/train_dynamics.png}
        \caption{Impact of Entropy Regularization on NC1}
        \label{fig:entropy_reg}
    \end{subfigure}
    %\hfill
    \begin{subfigure}[b]{0.45\textwidth}
        \centering
        \includegraphics[width=\textwidth]{images/train_dynamics_entropy.png}
        \caption{Entropy Dynamics}
        \label{fig:entropy-dynamics}
    \end{subfigure}
    %
    \begin{subfigure}[b]{0.45\textwidth}
        \centering
        \includegraphics[width=\textwidth]{images/train_dynamics_rank.png}
        \caption{Effective Rank Dynamics}
        \label{fig:effective-rank}
    \end{subfigure}
    %%
    \begin{subfigure}[b]{0.45\textwidth}
        \centering
        \includegraphics[width=\textwidth]{images/train_dynamics_l2.png}
        \caption{Impact of $L_2$ Normalization on NC1}
        \label{fig:l2_norm}
    \end{subfigure}
    
    \caption{\textbf{Analyzing entropy regularization \& $\mathbf{L_2}$ normalization.} 
    \textcolor{blue}{\textbf{(a)}} Entropy regularization reduces neural collapse (indicated by higher NC1 values) in the encoder. %, promoting OOD generalization. 
    \textcolor{blue}{\textbf{(b)}} Entropy regularization increases the entropy of encoder embeddings otherwise entropy remains unchanged.
    \textcolor{blue}{\textbf{(c)}} Entropy regularization increases the effective rank of encoder embeddings otherwise effective rank remains as low as the number of classes (i.e., 10 ImageNet classes).
    \textcolor{blue}{\textbf{(d)}} $L_2$ normalization increases neural collapse (indicated by lower NC1 values) in the projector. %, promoting OOD detection. 
    For this analysis, we train VGG17 networks on the ImageNet-10 subset (10 ImageNet classes) for 100 epochs.
    }
    \label{fig:nc_dynamics}
\end{figure*}




\section{Additional Experimental Results}
\label{sec:additional_exp_supp}

%%%%%%%%%%%%%%%%%%%%%

\subsection{Fixed ETF Projector Vs. Learnable Projector}

In Table~\ref{tab:plastic_proj}, we observe that the fixed ETF projector shows a higher transfer error (2.47\% absolute) than the plastic projector but outperforms the plastic projector in ID error (2.48\% absolute) and OOD detection error (8.9\% absolute). A fixed ETF projector should intensify NC and hinder OOD transfer and our fixed ETF projector fulfills this goal.


\begin{table*}[t]
\centering
  \caption{\textbf{ETF Fixed Projector Vs. Plastic Projector.} The VGG17 models are trained on \textbf{ImageNet-100} dataset (ID) and evaluated on 8 OOD datasets. The same color highlights the rows to compare.
  For OOD transfer we report $\boldsymbol{\mathcal{E}}_{\text{GEN}}$ (\%) whereas for OOD detection we report $\boldsymbol{\mathcal{E}}_{\text{DET}}$ (\%). %\textcolor{brown}{Fixed ETF projector shows higher transfer error (2.47\% absolute) than plastic projector but outperforms plastic projector in ID error (2.48\% absolute) and OOD detection error (8.9\% absolute).}
  } 
  \label{tab:plastic_proj}
  \centering
  \resizebox{\linewidth}{!}{
     \begin{tabular}{cc|cccc|ccccccccc}
     \hline %\hline
     \multicolumn{1}{c}{\textbf{Projector}} &
     \multicolumn{1}{c|}{$\boldsymbol{\mathcal{E}}_{\text{ID}} \downarrow$} &
     \multicolumn{4}{c|}{\textbf{Neural Collapse}} &
     \multicolumn{9}{c}{\textbf{OOD Datasets}} \\
    & IN & $\mathcal{NC}1$ & $\mathcal{NC}2$ & $\mathcal{NC}3$ & $\mathcal{NC}4$ & IN-R & CIFAR & Flowers & NINCO & CUB & Aircrafts & Pets & STL & Avg. \\
    & 100 &  &  &  &  & 200 & 100 & 102 & 64 & 200 & 100 & 37 & 10 & \\    
    \hline
    %% CE Loss
    \textbf{\textcolor{orange}{Transfer Error $\downarrow$}} \\
    %\rowcolor[gray]{0.9}
    \textcolor{blue}{\textbf{Plastic}} \\
    Projector & 15.10 & 0.498 & 0.515 & 0.428 & 1.422 & 87.52 & 64.83 & 79.71 & 53.32 & 87.00 & 93.46 & 48.76 & 28.04 & 67.83 \\

    \rowcolor{yellow!50}
    Encoder & 23.64 & 13.953 & 0.526 & 0.833 & 6.697 & \textbf{69.43} & \textbf{45.12} & \textbf{20.00} & \textbf{23.55} & \textbf{57.90} & \textbf{60.10} & 25.40 & 13.52 & \textbf{39.38} \\
    \hline %\hline

    %\rowcolor[gray]{0.9}
    \textcolor{blue}{\textbf{Fixed ETF}} \\

    Projector & \textbf{12.62} & 0.393 & 0.490 & 0.468 & 0.316 & 91.38 & 65.72 & 64.51 & 64.97 & 82.22 & 97.42 & 43.17 & 21.51 & 66.36 \\
    
    \rowcolor{yellow!50}
    %% ENCODER
    \textbf{Encoder} & 15.52 & 2.175 & 0.603 & 0.616 & 5.364 & 71.52 & 47.24 & 25.10 & 24.32 & 63.67 & 67.81 & \textbf{21.56} & 13.55 & 41.85 \\ % error

    \hline \hline
    \textbf{\textcolor{orange}{Detection Error $\downarrow$}} \\
    %\rowcolor[gray]{0.9}
    \textcolor{blue}{\textbf{Plastic}} \\
    \rowcolor{green!25}
    Projector & 15.10 & 0.498 & 0.515 & 0.428 & 1.422 & 63.05 & \textbf{47.87} & 62.45 & 70.07 & 80.88 & 98.95 & 89.37 & 79.25 & 74.00 \\
    
    Encoder & 23.64 & 13.953 & 0.526 & 0.833 & 6.697 & 81.27 & 98.82 & 93.33 & 86.48 & 79.98 & 99.40 & 91.25 & 93.88 & 90.55 \\
    \hline
    %\rowcolor[gray]{0.9}
    \textcolor{blue}{\textbf{Fixed ETF}} \\
    %% PROJECTOR
    \rowcolor{green!25}
    \textbf{Projector} & \textbf{12.62} & 0.393 & 0.490 & 0.468 & 0.316 & \textbf{60.85} & 48.23 & \textbf{42.35} & \textbf{67.69} & \textbf{56.51} & 99.04 & \textbf{76.32} & \textbf{69.84} & \textbf{65.10} \\
    %\hline
    
    %% ENCODER
    Encoder & 15.52 & 2.175 & 0.603 & 0.616 & 5.364 & 67.17 & 98.14 & 81.76 & 84.95 & 84.57 & 99.70 & 97.36 & 87.34 & 87.62 \\
    
    \hline \hline
    %\vspace{-2em}
    \end{tabular}}
\end{table*}








\begin{comment}

%%% Following results correspond to MSE Loss with Plastic Projector
\begin{table*}[t]
    

\centering
  \caption{\textbf{ETF Fixed Vs. Plastic Projector.} The VGGm-17 models ($F_{\psi}$($G_{\phi}$($H_{\theta}))$) are trained on \textbf{ImageNet-100} dataset (ID) and evaluated on 8 OOD datasets. In \textbf{encoder} method, the embeddings are extracted from the encoder ($H_{\theta}$) and before projector. And, in \textbf{projector} method, the embeddings are extracted after projector ($G_{\phi}$) and before output layer ($F_{\psi}$). For OOD transfer we report the top-1 error whereas for OOD detection we report the FPR95. $\uparrow$ indicates larger values are better and $\downarrow$ indicates smaller values are better. All values except neural collapse are percentages. \textbf{A lower $\mathcal{NC}$ indicates higher neural collapse. %$+\delta$ and $-\delta$ indicate \% increase and \% decrease respectively, when changing from encoder to projector.
  }
  } 
  \label{tab:plastic_proj}
  \centering
  \resizebox{\linewidth}{!}{
     \begin{tabular}{cc|ccc|ccccccccc}
     \hline %\hline
     \multicolumn{1}{c}{\textbf{Method}} &
     \multicolumn{1}{c|}{\textbf{ID Error}} &
     \multicolumn{3}{c|}{\textbf{Neural Collapse}} &
     \multicolumn{9}{c}{\textbf{OOD Datasets}} \\
    & IN & $\mathcal{NC}1$ &  $\mathcal{NC}2$ &  $\mathcal{NC}3$ & IN-R & CIFAR & Flower & NINCO & CUB & AirCrafts & Pet & STL & Avg. \\
    & 100 &  &  &  & 200 & 100 & 102 & 64 & 200 & 100 & 37 & 10 & \\    
    \hline \hline
    \textbf{Transfer Error $\downarrow$} \\
    Projector & \textbf{16.68} & 2.040 & 0.601 & 0.337 & 89.53 & 75.98 & 89.90 & 57.40 & 90.87 & 97.60 & 56.31 & 31.29 & 73.61 \\
     \hline
    \textbf{Encoder} & 19.42 & 3.969 & 0.552 & 0.705 & \textbf{77.72} & \textbf{56.18} & \textbf{36.08} & \textbf{29.93} & \textbf{65.69} & \textbf{73.27} & \textbf{28.59} & \textbf{16.55} & \textbf{48.00} \\ % error
    
    \hline \hline
    \textbf{Detection FPR $\downarrow$} \\
    \textbf{Projector} & \textbf{16.68} & 2.040 & 0.601 & 0.337  & \textbf{72.95} & \textbf{40.80} & \textbf{76.37} & \textbf{72.05} & \textbf{71.87} & \textbf{98.11} & \textbf{87.90} & \textbf{74.05} & \textbf{74.26} \\
    \hline
    Encoder & 19.42 & 3.969 & 0.552 & 0.705 & 98.25 & 99.95 & 88.63 & 97.06 & 98.24 & 87.55 & 97.33 & 98.90 & 95.74 \\
    \hline \hline
    %\vspace{-2em}
    \end{tabular}}
\end{table*}


\end{comment}


%%%%%%%%%%%%%%%%%%%%%

\subsection{Impact of $\mathbf{L_2}$ Normalization on NC}

\begin{table*}[t]
\centering
  \caption{\textbf{$L_2$ Normalization.} The VGG17 models are trained on \textbf{ImageNet-100} dataset (ID) and evaluated on 8 OOD datasets. The same color highlights the rows to compare. %\textcolor{brown}{$L_2$ normalization increases NC and improves OOD detection by 3.83\% (absolute).}
  For OOD detection, we report $\boldsymbol{\mathcal{E}}_{\text{DET}}$ (\%).
  } 
  \label{tab:l2_norm_nc}
  \centering
  \resizebox{\linewidth}{!}{
     \begin{tabular}{cc|cccc|ccccccccc}
     \hline %\hline
     \multicolumn{1}{c}{\textbf{Method}} &
     \multicolumn{1}{c|}{$\boldsymbol{\mathcal{E}}_{\text{ID}} \downarrow$} &
     \multicolumn{4}{c|}{\textbf{Neural Collapse} $\downarrow$} &
     %\multicolumn{9}{c}{\textbf{OOD Datasets} $\downarrow$} \\
     \multicolumn{9}{c}{$\boldsymbol{\mathcal{E}}_{\text{DET}} \downarrow$} \\
    & IN & $\mathcal{NC}1$ & $\mathcal{NC}2$ & $\mathcal{NC}3$ & $\mathcal{NC}4$ & IN-R & CIFAR & Flowers & NINCO & CUB & Aircrafts & Pets & STL & Avg. \\
    & 100 &  &  &  &  & 200 & 100 & 102 & 64 & 200 & 100 & 37 & 10 & \\    
    \toprule
    %% CE Loss
    %\hline %\hline
    %\textbf{\textcolor{orange}{Detection Error $\downarrow$}} \\
    \textcolor{blue}{\textbf{No $L_2$ Norm}} \\
    \rowcolor{green!25}
    Projector & 12.74 & 0.579 & 0.538 & \textbf{0.349} & 1.339 & \textbf{57.43} & 49.41 & 62.35 & 69.81 & 58.04 & 99.58 & 85.28 & 69.53 & 68.93 \\
    
    Encoder & 14.70 & 1.788 & 0.633 & 0.823 & 10.643 & 77.08 & 96.77 & 91.18 & 92.35 & 89.47 & 99.64 & 89.51 & 85.31 & 90.16 \\
    \hline
    %\rowcolor[gray]{0.9}
    \textcolor{blue}{\textbf{$L_2$ Norm}} \\
    %% PROJECTOR
    \rowcolor{green!25}
    \textbf{Projector} & \textbf{12.62} & \textbf{0.393} & \textbf{0.490} & 0.468 & \textbf{0.316} & 60.85 & \textbf{48.23} & \textbf{42.35} & \textbf{67.69} & \textbf{56.51} & 99.04 & \textbf{76.32} & 69.84 & \textbf{65.10} \\
    %\hline
    
    %% ENCODER
    Encoder & 15.52 & 2.175 & 0.603 & 0.616 & 5.364 & 67.17 & 98.14 & 81.76 & 84.95 & 84.57 & 99.70 & 97.36 & 87.34 & 87.62 \\
    
    %\hline 
    \bottomrule
    %\vspace{-2em}
    \end{tabular}}
\end{table*}


We verify whether $L_2$ normalization effectively induces more neural collapse and improves OOD detection.
We analyze two VGG17 models pre-trained on ImageNet-100 dataset where one model uses $L_2$ normalization and the other omits it.
The results are summarized in Table~\ref{tab:l2_norm_nc}.
We find that $L_2$ normalization induces more NC as evidenced by the lower NC1 value than its counterpart.
Consequently, $L_2$ normalization improves OOD detection by 3.83\% (absolute). Also, it achieves lower ID error than the compared model without $L_2$ normalization.

Next, we analyze how $L_2$ normalization impacts NC during training. We perform small-scale experiments since large-scale experiments are compute-intensive.
We train two VGG17 models on the ImageNet-10 (10 ImageNet classes) subset where one model uses $L_2$ normalization and another does not. During training, we measure the NC1 metric for the encoder embeddings.
The impact of $L_2$ normalization on NC1 is exhibited in Fig.~\ref{fig:l2_norm}. We find that $L_2$ normalization helps intensify NC during training. Consequently, it promotes better OOD detection.


%%%%%%%%%%%%%%%%%%%%%

\subsection{Batch Normalization Vs. Group Normalization}

\begin{table*}[t]
\centering
  \caption{\textbf{Batch Norm Vs. Group Norm.} VGG17 models are trained on \textbf{ImageNet-100} dataset (ID) and evaluated on 8 OOD datasets. The same color highlights the rows to compare.
  Group norm is integrated with weight standardization.
  All metrics except NC are reported in percentage. For OOD transfer we report $\boldsymbol{\mathcal{E}}_{\text{GEN}}$ (\%) whereas for OOD detection we report $\boldsymbol{\mathcal{E}}_{\text{DET}}$ (\%).
  %\textcolor{brown}{GroupNorm+WS outperforms BatchNorm by 10.11\% (absolute) in OOD transfer and by 4.37\% (absolute) in OOD detection.}
  } 
  \label{tab:bn_vs_gn}
  \centering
  \resizebox{\linewidth}{!}{
     \begin{tabular}{cc|cccc|ccccccccc}
     \hline %\hline
     \multicolumn{1}{c}{\textbf{Method}} &
     \multicolumn{1}{c|}{$\boldsymbol{\mathcal{E}}_{\text{ID}} \downarrow$} &
     \multicolumn{4}{c|}{\textbf{Neural Collapse}} &
     \multicolumn{9}{c}{\textbf{OOD Datasets}} \\
    & IN & $\mathcal{NC}1$ & $\mathcal{NC}2$ & $\mathcal{NC}3$ & $\mathcal{NC}4$ & IN-R & CIFAR & Flowers & NINCO & CUB & Aircrafts & Pets & STL & Avg. \\
    & 100 &  &  &  &  & 200 & 100 & 102 & 64 & 200 & 100 & 37 & 10 & \\    
    \hline
    %% CE Loss
    \textbf{\textcolor{orange}{Transfer Error $\downarrow$}} \\
    %\rowcolor[gray]{0.9}
    \textcolor{blue}{\textbf{Batch Norm}} \\
    Projector & 12.52 & 0.372 & 0.669 & 0.263 & 0.536 & 89.43 & 66.00 & 63.14 & 64.46 & 83.00 & 94.57 & 38.65 & 21.30 & 65.07 \\
    %\rowcolor[gray]{0.9}
    \rowcolor{yellow!50}
    Encoder & 14.54 & 1.401 & 0.605 & 0.590 & 25.611 & 78.02 & 53.34 & 49.51 & 33.25 & 74.08 & 85.27 & 25.46 & 16.75 & 51.96 \\
    
    \hline %\hline

    %\rowcolor[gray]{0.9}
    \textcolor{blue}{\textbf{Group Norm}} \\

    Projector & 12.62 & 0.393 & 0.490 & 0.468 & 0.316 & 91.38 & 65.72 & 64.51 & 64.97 & 82.22 & 97.42 & 43.17 & 21.51 & 66.36 \\
    
    \rowcolor{yellow!50}
    %% ENCODER
    \textbf{Encoder} & 15.52 & 2.175 & 0.603 & 0.616 & 5.364 & \textbf{71.52} & \textbf{47.24} & \textbf{25.10} & \textbf{24.32} & \textbf{63.67} & \textbf{67.81} & \textbf{21.56} & \textbf{13.55} & \textbf{41.85} \\ % err

    \hline \hline
    \textbf{\textcolor{orange}{Detection Error $\downarrow$}} \\
    %\rowcolor[gray]{0.9}
    \textcolor{blue}{\textbf{Batch Norm}} \\
    \rowcolor{green!25}
    Projector & 12.52 & 0.372 & 0.669 & 0.263 & 0.536 & \textbf{57.30} & 74.62 & 44.12 & \textbf{66.33} & 65.14 & 99.19 & 75.93 & 73.13 & 69.47 \\
    
    Encoder & 14.54 & 1.401 & 0.605 & 0.590 & 25.611 & 92.17 & 99.77 & 91.08 & 91.41 & 99.48 & 98.62 & 85.39 & 93.26 & 93.90 \\
    \hline
    %\rowcolor[gray]{0.9}
    \textcolor{blue}{\textbf{Group Norm}} \\
    %% PROJECTOR
    \rowcolor{green!25}
    \textbf{Projector} & 12.62 & 0.393 & 0.490 & 0.468 & 0.316 & 60.85 & \textbf{48.23} & \textbf{42.35} & 67.69 & \textbf{56.51} & 99.04 & 76.32 & \textbf{69.84} & \textbf{65.10} \\
    %\hline
    
    %% ENCODER
    Encoder & 15.52 & 2.175 & 0.603 & 0.616 & 5.364 & 67.17 & 98.14 & 81.76 & 84.95 & 84.57 & 99.70 & 97.36 & 87.34 & 87.62 \\
    
    %\hline \hline
    %\vspace{-2em}
    \bottomrule
    \end{tabular}}
\end{table*}

We find that group normalization (combined with weight standardization) outperforms batch normalization by 10.11\% (absolute) in OOD transfer and by 4.37\% (absolute) in OOD detection (see Table~\ref{tab:bn_vs_gn}).
Group normalization achieves a higher $\mathcal{NC}1$ value than batch normalization, thereby mitigating NC and enhancing OOD generalization. Group normalization also achieves ID performance similar to that of batch normalization.
Our results demonstrate that group normalization achieves competitive performance and plays a crucial role in OOD generalization.


%%%%%%%%%%%%%%%%%%%%%

\subsection{Comparison with Baseline}

\begin{table*}[t]
\centering
  \caption{\textbf{Comprehensive Comparison with Baseline.} Various DNNs are trained on \textbf{ImageNet-100} dataset (ID) and evaluated on 8 OOD datasets. The regular models e.g., VGG17, ResNet18, and ViT-T do not use mechanisms e.g., regularization loss or a projector to control NC. For OOD transfer we report $\boldsymbol{\mathcal{E}}_{\text{GEN}}$ (\%) whereas for OOD detection we report $\boldsymbol{\mathcal{E}}_{\text{DET}}$ (\%).%\textcolor{brown}{Our model significantly outperforms the baseline in OOD transfer and OOD detection across all DNN architectures.}
  } 
  \label{tab:base_model}
  \centering
  \resizebox{\linewidth}{!}{
     \begin{tabular}{cc|cccc|ccccccccc}
     \hline %\hline
     \multicolumn{1}{c}{\textbf{Model}} &
     \multicolumn{1}{c|}{$\boldsymbol{\mathcal{E}}_{\text{ID}} \downarrow$} &
     \multicolumn{4}{c|}{\textbf{Neural Collapse}} &
     \multicolumn{9}{c}{\textbf{OOD Datasets}} \\
    & IN & $\mathcal{NC}1$ & $\mathcal{NC}2$ & $\mathcal{NC}3$ & $\mathcal{NC}4$ & IN-R & CIFAR & Flowers & NINCO & CUB & Aircrafts & Pets & STL & Avg. \\
    & 100 &  &  &  &  & 200 & 100 & 102 & 64 & 200 & 100 & 37 & 10 & \\    
    %\hline
    \toprule
    %% CE Loss
    %\textbf{\textcolor{blue}{VGG17}} \\
    \textbf{\textcolor{orange}{Transfer Error $\downarrow$}} \\
    VGG17 & 12.18 & 0.766 & 0.705 & 0.486 & 37.491 & 75.60 & 50.11 & 42.75 & 29.17 & 71.35 & 84.13 & 27.58 & 15.65 & 49.54 \\
    
    \rowcolor{yellow!50}
    %% ENCODER
    \textbf{VGG17+Ours} & 12.62 & 0.393 & 0.490 & 0.468 & 0.316 & \textbf{71.52} & \textbf{47.24} & \textbf{25.10} & \textbf{24.32} & \textbf{63.67} & \textbf{67.81} & \textbf{21.56} & \textbf{13.55} & \textbf{41.85} \\ % err

    \hline %\hline
    \textbf{\textcolor{orange}{Detection Error $\downarrow$}} \\
    VGG17 & 12.18 & 0.766 & 0.705 & 0.486 & 37.491 & 96.02 & 97.16 & 97.94 & 93.11 & 95.19 & 98.59 & 87.33 & 94.05 & 94.92 \\

    \rowcolor{yellow!50}
    %% Projector
    \textbf{VGG17+Ours} & 12.62 & 0.393 & 0.490 & 0.468 & 0.316 & \textbf{60.85} & \textbf{48.23} & \textbf{42.35} & \textbf{67.69} & \textbf{56.51} & 99.04 & \textbf{76.32} & \textbf{69.84} & \textbf{65.10} \\
    
    \hline \hline
    %\midrule
    %% RESNET18
    %\textbf{\textcolor{blue}{ResNet18}} \\
    \textbf{\textcolor{orange}{Transfer Error $\downarrow$}} \\
    ResNet18 & 15.38 & 1.11 & 0.658 & 0.590 & 31.446 & 75.75 & \textbf{49.48} & 41.37 & 30.02 & 69.80 & 82.75 & 29.63 & 16.53 & 49.42 \\

    %% ENCODER
    \rowcolor{yellow!50}
    \textbf{ResNet18+Ours} & 16.14 & 0.341 & 0.456 & 0.306 & 0.540 & \textbf{74.17} & 53.33 & \textbf{31.37} & \textbf{28.15} & \textbf{68.85} & \textbf{81.61} & \textbf{27.72} & 16.56 & \textbf{47.72} \\

    \hline %\hline
    \textbf{\textcolor{orange}{Detection Error $\downarrow$}} \\
    ResNet18 & 15.38 & 1.11 & 0.658 & 0.590 & 31.446 & 98.40 & 98.85 & 98.33 & 96.68 & 96.60 & 99.67 & 92.40 & 98.25 & 97.40 \\

    %% PROJECTOR
    \rowcolor{yellow!50}
    \textbf{ResNet18+Ours} & 16.14 & 0.341 & 0.456 & 0.306 & 0.540 & \textbf{67.92} & \textbf{61.21} & \textbf{71.18} & \textbf{71.09} & \textbf{23.20} & \textbf{99.28} & \textbf{81.41} & \textbf{82.29} & \textbf{69.70} \\
    \hline \hline

    
    %% VIT-Tiny
    %\textbf{\textcolor{blue}{ViT-T}} \\
    \textbf{\textcolor{orange}{Transfer Error $\downarrow$}} \\
    ViT-T & 31.78 & 2.467 & 0.657 & 0.601 & 1.015 & 82.18 & 52.64 & 41.67 & 32.74 & 63.48 & 81.61 & 45.11 & 22.00 & 52.68 \\

    %% ENCODER
    \rowcolor{yellow!50}
    \textbf{ViT-T+Ours} & 32.04 & 2.748 & 0.609 & 0.798 & 1.144 & 82.28 & \textbf{52.00} & \textbf{42.94} & \textbf{30.36} & \textbf{63.15} & 84.31 & \textbf{44.86} & \textbf{21.13} & \textbf{52.63} \\

    \hline %\hline
    \textbf{\textcolor{orange}{Detection Error $\downarrow$}} \\
    ViT-T & 31.78 & 2.467 & 0.657 & 0.601 & 1.015 & 85.18 & 91.70 & 87.06 & 89.54 & 87.78 & \textbf{98.35} & \textbf{91.77} & 89.99 & 90.17 \\

    %% PROJECTOR
    \rowcolor{yellow!50}
    \textbf{ViT-T+Ours} & 32.04 & 2.748 & 0.609 & 0.798 & 1.144 & \textbf{81.12} & \textbf{60.81} & \textbf{77.55} & \textbf{82.40} & \textbf{79.05} & 99.10 & \textbf{95.15} & 90.06 & \textbf{83.16} \\
    
    \bottomrule
    %\vspace{-2em}
    \end{tabular}}
\end{table*}

Our experimental results show that our method significantly improves OOD detection and OOD transfer performance across all DNN architectures. We summarize the results in Table~\ref{tab:base_model}. We evaluate VGG17, ResNet18, and ViT-T baselines on 8 OOD datasets and compare them with our models.
The absolute improvements over VGG17 baseline are 7.69\% (OOD generalization) and 
29.82\% (OOD detection). Similarly, our method outperforms other DNNs in all criteria.
Our results corroborate our argument that \emph{controlling NC enables good OOD detection and OOD generalization performance}. It is also evident that a single feature space cannot simultaneously achieve both OOD detection and OOD generalization abilities.



%%%%%%%%%%%%%%%%%%%%%

\subsection{Projector Design Criteria}

Here we study the design choices of the projector network. We want to know how depth and width impact the performance. For this, we examine projectors consisting of a single layer (\textit{depth=1}, $512d$), two layers (\textit{depth=2}, $512d \rightarrow 2048d \rightarrow 512d$), three layers (\textit{depth=3}, $512d \rightarrow 2048d \rightarrow 2048d \rightarrow 512d$),
and a wider variant (\textit{width=2}, $512d \rightarrow 4096d \rightarrow 512d$). All of these variants are trained in identical settings and only the projector is changed. We train VGG17 networks on ImageNet-100 dataset (ID) and evaluate OOD detection/generalization on 8 OOD datasets.
The results are shown in Table~\ref{tab:proj_design}. The projector with depth 2 outperforms other variants across all evaluations.


\begin{table*}[t]
\centering
  \caption{\textbf{Projector Design Criteria.} The VGG17 models are trained on \textbf{ImageNet-100} dataset (ID) and evaluated on 8 OOD datasets. The same color highlights the rows to compare. All compared projectors are configured as fixed simplex ETFs. Our final model has depth 2 and performs better than other variants. All metrics except NC are in percentage. For OOD transfer we report $\boldsymbol{\mathcal{E}}_{\text{GEN}}$ (\%) whereas for OOD detection we report $\boldsymbol{\mathcal{E}}_{\text{DET}}$ (\%).
  } 
  \label{tab:proj_design}
  \centering
  \resizebox{\linewidth}{!}{
     \begin{tabular}{cc|cccc|ccccccccc}
     \hline %\hline
     \multicolumn{1}{c}{\textbf{Criteria}} &
     \multicolumn{1}{c|}{$\boldsymbol{\mathcal{E}}_{\text{ID}} \downarrow$} &
     \multicolumn{4}{c|}{\textbf{Neural Collapse}} &
     \multicolumn{9}{c}{\textbf{OOD Datasets}} \\
    & IN & $\mathcal{NC}1$ & $\mathcal{NC}2$ & $\mathcal{NC}3$ & $\mathcal{NC}4$ & IN-R & CIFAR & Flowers & NINCO & CUB & Aircrafts & Pets & STL & Avg. \\
    & 100 &  &  &  &  & 200 & 100 & 102 & 64 & 200 & 100 & 37 & 10 & \\    
    \hline
    %% CE Loss
    \textbf{\textcolor{orange}{Transfer Error $\downarrow$}} \\
    %\rowcolor[gray]{0.9}
    \textcolor{blue}{\textbf{Depth=1}} \\
    %Projector & -- & -- & -- & -- & -- & -- & -- & -- & -- & -- & -- & -- & -- & -- \\

    Projector & 12.86 & 0.375 & 0.649 & 0.500 & 1.157 & 90.27 & 64.61 & 60.88 & 55.02 & 81.12 & 96.34 & 44.34 & 23.04 & 64.45 \\

    \rowcolor{yellow!50}
    Encoder & 16.34 & 1.673 & 0.667 & 0.589 & 7.936 & 74.08 & 50.61 & 30.00 & 28.06 & 66.74 & 71.95 & 25.73 & 15.75 & 45.37 \\
    \hline %\hline

    %\rowcolor[gray]{0.9}
    \textcolor{blue}{\textbf{Depth=2 (Ours)}} \\

    Projector & \textbf{12.62} & 0.393 & 0.490 & 0.468 & 0.316 & 91.38 & 65.72 & 64.51 & 64.97 & 82.22 & 97.42 & 43.17 & 21.51 & 66.36 \\
    
    \rowcolor{yellow!50}
    %% ENCODER
    \textbf{Encoder} & 15.52 & 2.175 & 0.603 & 0.616 & 5.364 & \textbf{71.52} & \textbf{47.24} & \textbf{25.10} & 24.32 & 63.67 & 67.81 & \textbf{21.56} & \textbf{13.55} & \textbf{41.85} \\ % error

    %\hline
    %textcolor{blue}{\textbf{Depth=3}} \\
    %Projector & 12.88 & 0.323 & 0.667 & 0.560 & 1.138 & 90.83 & 66.01 & 63.53 & 59.10 & 81.95 & 96.28 & 43.50 & 23.05 & 65.53 \\

    %\rowcolor{yellow!50}
    %Encoder & 16.00 & 2.623 & 0.609 & 0.638 & 5.290 & 72.47 & 49.03 & 26.67 & \textbf{23.64} & \textbf{61.34} & \textbf{67.03} & 23.74 & 14.96 & 42.36 \\

    \hline
    \textcolor{blue}{\textbf{Width=2}} \\
    Projector & 13.48 & 0.320 & 0.667 & 0.376 & 0.493 & 89.88 & 66.46 & 64.51 & 53.40 & 82.50 & 95.77 & 41.76 & 23.90 & 64.77 \\

    \rowcolor{yellow!50}
    Encoder & 16.46 & 2.341 & 0.607 & 0.646 & 5.899 & 73.05 & 50.61 & 27.25 & 25.60 & 64.84 & 67.87 & 22.35 & 15.10 & 43.33 \\

    \hline \hline
    \textbf{\textcolor{orange}{Detection Error $\downarrow$}} \\
    %\rowcolor[gray]{0.9}
    \textcolor{blue}{\textbf{Depth=1}} \\
    \rowcolor{green!25}
    
    Projector & 12.86 & 0.375 & 0.649 & 0.500 & 1.157 & 80.15 & 95.98 & 81.68 & 84.18 & 92.75 & \textbf{98.38} & \textbf{73.62} & 92.24 & 87.37 \\
    
    Encoder & 16.34 & 1.673 & 0.667 & 0.589 & 7.936 & 62.72 & 95.04 & 84.65 & 84.95 & 92.22 & 99.43 & 89.75 & 83.66 & 86.55 \\
    
    \hline
    %\rowcolor[gray]{0.9}
    \textcolor{blue}{\textbf{Depth=2 (Ours)}} \\
    %% PROJECTOR
    \rowcolor{green!25}
    \textbf{Projector} & \textbf{12.62} & 0.393 & 0.490 & 0.468 & 0.316 & \textbf{60.85} & \textbf{48.23} & \textbf{42.35} & \textbf{67.69} & \textbf{56.51} & 99.04 & 76.32 & \textbf{69.84} & \textbf{65.10} \\
    %\hline
    
    %% ENCODER
    Encoder & 15.52 & 2.175 & 0.603 & 0.616 & 5.364 & 67.17 & 98.14 & 81.76 & 84.95 & 84.57 & 99.70 & 97.36 & 87.34 & 87.62 \\

    %\hline
    %\textcolor{blue}{\textbf{Depth=3}} \\
    %\rowcolor{green!25}
    %Projector & 12.88 & 0.323 & 0.667 & 0.560 & 1.138 & 74.30 & 92.23 & 68.63 & 81.21 & 84.00 & 98.59 & 69.61 & 90.59 & 82.40 \\

    %Encoder & 16.00 & 2.623 & 0.609 & 0.638 & 5.290 & 77.18 & 99.86 & 94.12 & 86.57 & 83.41 & 99.52 & 94.79 & 89.90 & 90.67 \\

    \hline
    \textcolor{blue}{\textbf{Width=2}} \\
    \rowcolor{green!25}
    Projector & 13.48 & 0.320 & 0.667 & 0.376 & 0.493 & 65.43 & 60.83 & 51.96 & 67.77 & 57.70 & 99.52 & 79.29 & 75.33 & 69.73 \\
    
    Encoder & 16.46 & 2.341 & 0.607 & 0.646 & 5.899 & 66.80 & 97.64 & 89.61 & 83.42 & 88.89 & 98.89 & 98.58 & 94.39 & 89.78 \\
    
    %\hline \hline
    \bottomrule
    %\vspace{-2em}
    \end{tabular}}
\end{table*}



%\subsection{Additional ID dataset}
%% TBA


%%%%%%%%%%%%%%%%%%%%%

\subsection{Fixed ETF Classifier Vs. Plastic Classifier}

\begin{table*}[t]
\centering
  \caption{\textbf{Fixed ETF Classifier Vs. Plastic Classifier.} The VGG17 models are trained on \textbf{ImageNet-100} dataset (ID) and evaluated on 8 OOD datasets. 
  The same color highlights the rows to compare. All metrics except NC are reported in percentage. For OOD transfer we report $\boldsymbol{\mathcal{E}}_{\text{GEN}}$ (\%) whereas for OOD detection we report $\boldsymbol{\mathcal{E}}_{\text{DET}}$ (\%).
  } 
  \label{tab:fixed_vs_plastic_head}
  \centering
  \resizebox{\linewidth}{!}{
     \begin{tabular}{cc|cccc|ccccccccc}
     \hline %\hline
     \multicolumn{1}{c}{\textbf{Classifier}} &
     \multicolumn{1}{c|}{$\boldsymbol{\mathcal{E}}_{\text{ID}} \downarrow$} &
     \multicolumn{4}{c|}{\textbf{Neural Collapse}} &
     \multicolumn{9}{c}{\textbf{OOD Datasets}} \\
    & IN & $\mathcal{NC}1$ & $\mathcal{NC}2$ & $\mathcal{NC}3$ & $\mathcal{NC}4$ & IN-R & CIFAR & Flowers & NINCO & CUB & Aircrafts & Pets & STL & Avg. \\
    & 100 &  &  &  &  & 200 & 100 & 102 & 64 & 200 & 100 & 37 & 10 & \\    
    \hline
    %% CE Loss
    \textbf{\textcolor{orange}{Transfer Error $\downarrow$}} \\
    %\rowcolor[gray]{0.9}
    \textcolor{blue}{\textbf{Fixed ETF}} \\
    Projector & 13.56 & 0.088 & 0.702 & 0.374 & 0.379 & 98.18 & 84.28 & 92.25 & 96.94 & 96.86 & 97.60 & 72.23 & 36.59 & 84.37 \\

    \rowcolor{yellow!50}
    Encoder & 16.40 & 3.794 & 0.773 & 0.786 & 54.24 & 82.47 & 63.19 & 55.98 & 36.31 & 81.00 & 88.36 & 31.18 & 20.88 & 57.42 \\
    \hline %\hline

    %\rowcolor[gray]{0.9}
    \textcolor{blue}{\textbf{Plastic (Ours)}} \\

    Projector & \textbf{12.62} & 0.393 & 0.490 & 0.468 & 0.316 & 91.38 & 65.72 & 64.51 & 64.97 & 82.22 & 97.42 & 43.17 & 21.51 & 66.36 \\
    
    \rowcolor{yellow!50}
    %% ENCODER
    \textbf{Encoder} & 15.52 & 2.175 & 0.603 & 0.616 & 5.364 & \textbf{71.52} & \textbf{47.24} & \textbf{25.10} & \textbf{24.32} & \textbf{63.67} & \textbf{67.81} & \textbf{21.56} & \textbf{13.55} & \textbf{41.85} \\ % error

    \hline \hline
    \textbf{\textcolor{orange}{Detection Error $\downarrow$}} \\
    %\rowcolor[gray]{0.9}
    \textcolor{blue}{\textbf{Fixed ETF}} \\
    \rowcolor{green!25}
    
    Projector & 13.56 & 0.088 & 0.702 & 0.374 & 0.379 & 73.80 & \textbf{26.45} & 73.04 & 68.20 & \textbf{55.80} & 98.98 & 96.05 & \textbf{63.56} & 69.49 \\
    
    Encoder & 16.40 & 3.794 & 0.773 & 0.786 & 54.24 & 81.03 & 98.98 & 81.57 & 87.25 & 97.29 & 99.01 & 86.48 & 93.11 & 90.59 \\

    \hline
    %\rowcolor[gray]{0.9}
    \textcolor{blue}{\textbf{Plastic (Ours)}} \\
    %% PROJECTOR
    \rowcolor{green!25}
    \textbf{Projector} & \textbf{12.62} & 0.393 & 0.490 & 0.468 & 0.316 & \textbf{60.85} & 48.23 & \textbf{42.35} & \textbf{67.69} & 56.51 & 99.04 & \textbf{76.32} & 69.84 & \textbf{65.10} \\
    %\hline
    
    %% ENCODER
    Encoder & 15.52 & 2.175 & 0.603 & 0.616 & 5.364 & 67.17 & 98.14 & 81.76 & 84.95 & 84.57 & 99.70 & 97.36 & 87.34 & 87.62 \\
    
    %\hline \hline
    \bottomrule
    %\vspace{-2em}
    \end{tabular}}
\end{table*}
%\begin{table}[t]
\centering
  \caption{\textbf{Plastic Vs. Fixed ETF Classifier.} The evaluation is based on ImageNet-100 pre-trained VGG17 network with a plastic classifier or a fixed ETF classifier. NC values correspond to projector embeddings. OOD-Error and FPR are averaged over 8 OOD datasets.}
  \label{tab:etf_clf_results}
  \centering
  \resizebox{\linewidth}{!}{
     \begin{tabular}{c|c|cccc|c|c}
     \hline %\hline
     \multicolumn{1}{c|}{\textbf{Classifier}} &
     \multicolumn{1}{c|}{\textbf{ID-Err}} &
     \multicolumn{4}{c|}{\textbf{Neural Collapse}} &
     \multicolumn{1}{c|}{\textbf{OOD-Err}} &
     \multicolumn{1}{c}{\textbf{FPR}} \\
     & $\downarrow$ & $\mathcal{NC}1$ & $\mathcal{NC}2$ & $\mathcal{NC}3$ & $\mathcal{NC}4$ & Avg. $\downarrow$ & Avg. $\downarrow$ \\
    %\hline \hline
    \toprule
    Fixed ETF & 13.56 & 0.088 & 0.702 & 0.374 & 0.379 & 57.42 & 69.49 \\
    %\hline
    
    \rowcolor{yellow!50}
    \textbf{Plastic (Ours)} & \textbf{12.62} & 0.393 & 0.490 & 0.468 & 0.316 & \textbf{41.85} & \textbf{65.10} \\
    %\hline \hline
    \bottomrule
    \end{tabular}}
\end{table}

We investigate how using a fixed ETF classifier head impacts NC and OOD detection/generalization performance.
We train two identical models consisting of our proposed mechanisms to control NC, the only thing we vary is the classifier head. One model consists of a plastic (learnable) classifier head which is our proposed model and the other consists of an ETF classifier head. The ETF classifier head is configured with Simplex ETF and frozen during training. We train VGG17 networks on ImageNet-100 (ID) and evaluate them on 8 OOD datasets.

%Table~\ref{tab:etf_clf_results} 
Table~\ref{tab:fixed_vs_plastic_head} shows results across all OOD datasets, where the plastic classifier outperforms the fixed ETF classifier by 4.39\% (absolute) in OOD detection and by 15.6\% in OOD generalization. The plastic classifier also outperforms ETF classifier in the ID task. 
%Table~\ref{tab:fixed_vs_plastic_head} shows detailed results for each OOD dataset. 
Our results suggest that imposing NC in the classifier head is sub-optimal for enhancing OOD detection and generalization.


%%%%%%%%%%%%%%%%%%%%%


%\section{Some Research Questions}
%\begin{enumerate}
%    \item Does a fixed classifier become a better OOD detector and classifier than a learned one?
%\end{enumerate}


%%%%%%%%%%%%%%%%%%%%%
%\newpage

\section{Classes of ImageNet-100 ID Dataset}
\label{sec:imagenet_100_classes}

We list the 100 classes in the ID dataset, ImageNet-100~\cite{tian2020contrastive}. 
This list can also be found at: \url{https://github.com/HobbitLong/CMC/blob/master/imagenet100.txt}


\textit{Rocking chair, pirate, computer keyboard, Rottweiler, Great Dane, tile roof, harmonica, langur, Gila monster, hognose snake, vacuum, Doberman, laptop, gasmask, mixing bowl, robin, throne, chime, bonnet, komondor, jean, moped, tub, rotisserie, African hunting dog, kuvasz, stretcher, garden spider, theater curtain, honeycomb, garter snake, wild boar, pedestal, bassinet, pickup, American lobster, sarong, mousetrap, coyote, hard disc, chocolate sauce, slide rule, wing, cauliflower, American Staffordshire terrier, meerkat, Chihuahua, lorikeet, bannister, tripod, head cabbage, stinkhorn, rock crab, papillon, park bench, reel, toy terrier, obelisk, walking stick, cocktail shaker, standard poodle, cinema, carbonara, red fox, little blue heron, gyromitra, Dutch oven, hare, dung beetle, iron, bottlecap, lampshade, mortarboard, purse, boathouse, ambulance, milk can, Mexican hairless, goose, boxer, gibbon, football helmet, car wheel, Shih-Tzu, Saluki, window screen, English foxhound, American coot, Walker hound, modem, vizsla, green mamba, pineapple, safety pin, borzoi, tabby, fiddler crab, leafhopper, Chesapeake Bay retriever, and ski mask.}


\begin{comment}

\textbf{Is there any semantic class overlap between ID and OOD datasets?}
There is no semantic class overlap between ImageNet-100 (ID dataset) and 8 other OOD datasets e.g., CIFAR-10, CIFAR-100, NINCO-64, CUB-200, Aircrafts-100, Oxford Pets-37, Flowers-102, and STL-10. 

Only ImageNet-R (consisting of 200 classes) has 19 classes that overlap with ImageNet-100. 
This is expected and we know that ImageNet-R includes classes from ImageNet-1K dataset but incorporates significant distribution shifts using artistic renditions.
The overlapping classes are:
\textit{Gasmask, American lobster, Standard poodle, Red fox, Head cabbage, Harmonica, Ambulance, Gibbon, Pineapple, Chihuahua, Tabby, Pirate, Rottweiler, Lorikeet, Boxer, Pickup, Goose, Shih-Tzu, and Meerkat.}

\end{comment}



\end{document}

%%%%%%%%%%%%%%%%%%%%%%%%%%%%%%%%%%%%%%%%%%%%%%%%%%%%%%%%%%%%%%%%%%%%%%%%

