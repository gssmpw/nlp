\section{Introduction}
\label{sec:introduction}
% problem introduction
Given graphs of multi-behavior interactions, how can we accurately rank items that a user is likely to purchase?
Multi-behavior recommendation~\cite{LoniPLH16} aims to recommend items to be purchased by a specific user, analyzing plentiful interactions across various user behaviors. 
Unlike early recommender systems~\cite{KorenBV09,RendleRFGS12,HeHDWLZW20} that rely on single-behavior interactions, the multi-behavior recommendation can more precisely capture user preferences for the target behavior (e.g., \textit{purchase}) by leveraging rich information from auxiliary behaviors (e.g., \textit{viewing} and \textit{adding to cart}). 
As a result, it has recently gained significant attention from data mining communities~\cite{JinJGHJL20,LeeLKSJ24xxaw,LiLCYLLD24,GaoGHGCFLCJ19,ChengCHLZGP23fqvn,YanYCSSP23} across various industrial domains, including streaming services, e-commerce, social media, and content aggregation.

% personalized graph ranking for recommender systems
To provide a recommendation list for a user, it is essential to calculate personalized ranking scores on items w.r.t. that user, particularly by analyzing user-item interactions, represented as a bipartite graph between users and items, based on the assumption\footnote{Collaborative filtering assumption indicates that users with similar preferences in the past will continue to have similar preferences in the future.} underlying collaborative filtering (CF)~\cite{SchaferFHS07}.
Existing methods for obtaining personalized rankings on graph data fall into two main categories: 
\textit{representation learning methods} and \textit{graph ranking methods}.
The former focuses on extracting representation vectors of users and items from the data, which are then used to predict the scores, while the latter aims to directly compute the scores by analyzing the relationships between users and items on graphs.

% previous methods for representation learnings
Many researchers have recently made tremendous efforts to develop representation learning methods within the CF framework for multi-behavior recommendation.
Their previous studies~\cite{LeeLKSJ24xxaw, ChengCHLZGP23fqvn, MengZGGLZZTLZ23, YanYCSSP23, MengMZYZL23} extend matrix factorization (MF)~\cite{KorenBV09} and graph neural networks (GNNs)~\cite{WangWHWFC19,HeHDWLZW20} to encode user and item embeddings for multi-behavior interactions, optimizing them to rank positive items higher than negative ones~\cite{RendleRFGS12}.
In particular, this approach allows models to 1) easily incorporate the inductive bias of multi-behavior (e.g., a natural sequence of user behaviors, such as viewing, adding to cart, and purchasing) in their encoding step~\cite{ChengCHLZGP23fqvn, MengMZYZL23, LiuXWY00024}, and 2) leverage these embeddings for multi-task learning to enhance generalization power~\cite{MengZGGLZZTLZ23,ZhangBCSYGWHH24,GaoGHGCFLCJ19}.
However, their recommendation quality remains limited because these methods are prone to producing over-smoothed embeddings\footnote{Over-smoothing refers to the phenomenon where node representations become too similar to each other.} under the assumption of CF~\cite{LiuGJ20,XiaHSX23,LiLCYLLD24}, particularly when utilizing GNNs, which limits their expressiveness.
Furthermore, the optimization across all users often prioritizes  users or items with a large number of multi-behavior interactions~\cite{YinCLYC12,abs-2410-04830}, limiting the ability to discover items in the long-tail distribution.

%, unlike graph ranking, which individually optimizes the scores for each user

% previous methods for graph ranking
Unlike representation learning, graph ranking methods directly generate ranking scores for items with respect to a specific user.
Traditional methods~\cite{PanPYFD04,BrinB98,DengDLK09kqpg,HeHGKW17} for graph ranking have focused on single-behavior graphs and smooth the ranking scores of neighbors\footnote{This follows the smoothness assumption~\cite{ZhouZBLWS03,AgarwalA06}, which states that the ranking of a node is influenced by the rankings of its neighbors.}, while incorporating information from a querying user (e.g., interacted items) to generate ranking scores, each with its own unique design.
In particular, the smoothness property helps these methods effectively identify similar nodes to the querying node, whose performance in collaborative filtering with implicit feedback has been empirically shown to outperform the representation learning methods in previous studies~\cite{ParkPJK17,HeHGKW17,GoriGP07,LeeSKLL11}.
However, relying on just one behavior (e.g., view) may fail to accurately capture users’ genuine interest in the target behavior (e.g., purchase), and the graph ranking approach for multi-behavior recommendation remains underexplored.

In this work, we explore the graph ranking approach for multi-behavior recommendation, and propose \method, a novel personalized graph ranking method tailored for it.
To leverage the semantics of multi-behaviors of users, we first construct a \textit{cascading behavior graph} by linking behavior bipartite graphs between users and items in the order of a natural (or cascading) sequence of behaviors (e.g., \texttt{view} $\rightarrow$ \texttt{cart} $\rightarrow$ \texttt{buy}), where later behaviors exhibit stronger user preferences for the target behavior compared to earlier ones.
%
We then design our ranking model and its iterative algorithm that produce ranking scores along the cascading behavior graph, ensuring the smoothness for CF and fitting the query information on the current behavior while incorporating scores from the previous behavior.
Through this process, our ranking scores precisely capture users’ preferences for the target behavior while leveraging the graph structure formed by multi-behavior interactions.
%
% contribution list
Our main contributions are summarized as follows:
\begin{itemize}[leftmargin=9mm,noitemsep]
    \item {
        We propose \method, a new ranking model on a cascading behavior graph for accurate multi-behavior recommendation, and develop an iterative algorithm for computing our ranking scores.
    }
    \item {
        We theoretically analyze our ranking model and algorithm in terms of cascading effect, convergence, and scalability.
    }
    \item {
        We conduct extensive experiments on three real-world datasets for multi-behavior recommendation, comparing \method with state-of-the-art ranking and representation learning methods.
    }
\end{itemize}

%\footnote{We will make the datasets used in this work, as well as the code for \method, publicly available upon acceptance.}
Our research findings in this paper reveal the following strengths of \method:
\begin{itemize}[leftmargin=9mm,noitemsep]
    \item{
        \textbf{Accurate}:
        Our method provides accurate multi-behavior recommendation, achieving higher accuracy than competitors in HR@$k$ and NDCG@$k$ across various values of $k$, with improvements of up to 9.56\% in HR@10 and 7.16\% in NDCG@10.
    }
    \item{
        \textbf{Reliable}:
        The convergence of our iterative algorithm is guaranteed, producing reliable rankings.
    }
    \item{
        \textbf{Scalable}:
        The running time of our algorithm scales linearly with the number of interactions.
    }
\end{itemize}

\def\UrlFont{\ttfamily}
For reproducibility, the code and the datasets are publicly available
at \url{https://github.com/geonwooko/CascadingRank}.
The rest of the paper is organized as follows. 
We review previous methods in Section~\ref{sec:related}.
After introducing preliminaries in Section~\ref{sec:preliminaires}, we describe our proposed \method in Section~\ref{sec:proposed}.
In Sections~\ref{sec:experiments}~and~\ref{sec:conclusion}, we present our experimental results and conclusions, respectively.

