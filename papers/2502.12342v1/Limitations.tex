
\clearpage
\section{Limitations} \label{sec:limitations}

While REAL-MM-RAG-Bench presents a comprehensive and realistic evaluation framework for multi-modal retrieval, several limitations remain:

\noindent
\emph{Query Variability:} Our queries are generated using a Vision-Language Model (VLM), which, while effective, may not fully capture the full range of plausible user queries.

\noindent
\emph{LLM and VLM Limitations:} Despite the strength of modern Large Language Models (LLMs) and VLMs, our filtering strategies and labeling process remain subject to their limitations. While our human evaluation confirms the effectiveness of our approach, errors in labeling and query selection may still occur. As LLMs and VLMs continue to improve, future benchmarks could leverage more accurate models to refine dataset construction further.

\noindent
\emph{Multi-Page Reasoning Queries:} Our benchmark is designed to best evaluate the retrieval component of Retrieval-Augmented Generation (RAG). While the dataset can be used for the generation step as well, it does not explicitly assess multi-page reasoning. Future work could explore automated query generation that combines multiple pages using LLMs and/or VLMs to construct multi-page reasoning tasks, enhancing the benchmark’s ability to evaluate complex retrieval scenarios.

\noindent
REAL-MM-RAG-Bench provides a realistic, reliable, and challenging retrieval benchmark, helping to identify critical weaknesses in current multi-modal retrieval models and paving the way for future improvements in both evaluation and model development.
