\section{Related Work}
% A utilização de Deep Learning no contexto de monitoramento em conflito humano, é relativamente novo. Pois os dados disponíveis de forma pública tem uma volumetria pequena e tem uma qualidade baixa nos frames dos vídeos. A seguir são mostrados alguns trabalhos relacionados ao contexto de monitoramento em conflito humano com técnicas de Deep Learning.

% The use of Deep Learning in the context of human conflict monitoring is relatively new, because the data available publicly has a small volume and has low quality in the video frames. Below are some works related to the context of monitoring human conflict with Deep Learning techniques.

% (Dashdamirov,2024) [0] avalia técnicas de aprendizado profundo na detecção de violência em vídeos, destacando que aumentar o conjunto de dados de 500 para 1.600 vídeos melhora a precisão média dos modelos em 6\%. Ele demonstra a importância de grandes conjuntos de dados e aprendizado por transferência para sistemas de vigilância mais eficazes. Datta et al. [1] analisaram a trajetória de movimentos e a orientação dos membros do corpo para detectar comportamentos violentos. Nguyen et al. [2] introduziram um modelo hierárquico de Markov oculto (HHMM), mostrando que ele pode ser útil para reconhecer atitudes agressivas, especialmente por meio de uma abordagem padrão de HHMM voltada à identificação de violência. Kim e Grauman [3] combinaram a Análise de Componentes Principais probabilística (PCA), usada para identificar padrões de fluxo em áreas locais, com Campos Aleatórios de Markov (MRF), que ajudam a manter a coerência global do modelo. Por outro lado, Mahadevan et al. [4] argumentaram que as representações baseadas em fluxo óptico não são adequadas para detectar alterações incomuns tanto na aparência quanto no movimento. Eles propuseram uma técnica que identifica cenas violentas avaliando elementos como presença de sangue, chamas, intensidade do movimento e volume sonoro.

The use of Deep Learning in the context of human conflict monitoring is relatively new, because the data available publicly has a small volume and has low quality in the video frames.
____ evaluates deep learning techniques in detecting violence in videos, highlighting that increasing the dataset from 500 to 1,600 videos improves the average accuracy of the models by 6\%. It demonstrates the importance of large data sets and transfer learning for more effective surveillance systems.

____ analyzed the trajectory of movements and orientation of body limbs to detect violent behavior. ____ introduced a hierarchical hidden Markov model (HHMM), showing that it can be useful for recognizing aggressive attitudes, especially through a standard HHMM approach aimed at identifying violence.

____ combined probabilistic Principal Component Analysis (PCA), used to identify flow patterns in local areas, with Markov Random Fields (MRF), which help maintain global model coherence. On the other hand, ____ argued that optical flow-based representations are not suitable for detecting unusual changes in both appearance and motion. They proposed a technique that identifies violent scenes by evaluating elements such as the presence of blood, flames, intensity of movement and sound volume.

%\label{sec:firstpage}