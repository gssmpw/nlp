\section{Related Work}
\label{rw}

\paragraph{Collaborative LLM-MA.}
LLM-MA systems enable agents to collaborate on complex tasks through dynamic role allocation, communication, and task execution____. Recent advancements include agent profiling____, hierarchical communication____, and integration of reasoning and intentions____. % Does this talk about tools?
However, challenges remain in ensuring robust communication, avoiding redundancy, and refining evaluation processes____. Standardized benchmarks and frameworks are needed to drive future progress____.

\paragraph{Communication in LLM-MA.}
Effective communication is crucial for collaborative intelligence____. While many previous works, including chain____, tree____, complete graph____, random graph____, optimizable graph____, and pruned graph____ methods have focused on communication topologies, there has been limited discussion on the optimal form of communication. Most systems rely on text-based exchanges____, which is inefficient and prone to errors as agents often lose track of subtasks or fail to recall prior outputs as tasks grow in complexity. We argue for structured communication protocols that guide subtasks with clear, context-specific instructions, ensuring coherence across interactions.
%prone to errors as tasks grow in complexity
% Agents often lose track of subtasks or fail to recall prior outputs. We argue for structured communication protocols that guide subtasks with clear, context-specific instructions, ensuring coherence and reliability across interactions.

\paragraph{Feedback-Based Refinement.}
Feedback mechanisms, such as Self-Refine____ and generator-evaluator frameworks____, improve system accuracy through iterative refinement. However, these methods face challenges in managing diverse feedback, which can lead to bias or inefficiencies if inputs are not well-organized____. Scalable, unbiased solutions are essential to enhance multi-agent evaluation processes.