%%
%% This is file `sample-sigconf-authordraft.tex',
%% generated with the docstrip utility.
%%
%% The original source files were:
%%
%% samples.dtx  (with options: `all,proceedings,bibtex,authordraft')
%% 
%% IMPORTANT NOTICE:
%% 
%% For the copyright see the source file.
%% 
%% Any modified versions of this file must be renamed
%% with new filenames distinct from sample-sigconf-authordraft.tex.
%% 
%% For distribution of the original source see the terms
%% for copying and modification in the file samples.dtx.
%% 
%% This generated file may be distributed as long as the
%% original source files, as listed above, are part of the
%% same distribution. (The sources need not necessarily be
%% in the same archive or directory.)
%%
%%
%% Commands for TeXCount
%TC:macro \cite [option:text,text]
%TC:macro \citep [option:text,text]
%TC:macro \citet [option:text,text]
%TC:envir table 0 1
%TC:envir table* 0 1
%TC:envir tabular [ignore] word
%TC:envir displaymath 0 word
%TC:envir math 0 word
%TC:envir comment 0 0
%%
%%
%% The first command in your LaTeX source must be the \documentclass
%% command.
%%
%% For submission and review of your manuscript please change the
%% command to \documentclass[manuscript, screen, review]{acmart}.
%%
%% When submitting camera ready or to TAPS, please change the command
%% to \documentclass[sigconf]{acmart} or whichever template is required
%% for your publication.
%%
%%
% \documentclass[manuscript,review,anonymous]{acmart}
\documentclass[sigconf]{acmart}

\usepackage{multirow}
\usepackage{makecell}
\usepackage{graphicx}
\usepackage{xcolor}
\usepackage{hyperref}

%%
%% \BibTeX command to typeset BibTeX logo in the docs
\AtBeginDocument{%
  \providecommand\BibTeX{{%
    Bib\TeX}}}

%% Rights management information.  This information is sent to you
%% when you complete the rights form.  These commands have SAMPLE
%% values in them; it is your responsibility as an author to replace
%% the commands and values with those provided to you when you
%% complete the rights form.
\setcopyright{acmlicensed}

\copyrightyear{2025}
\acmYear{2025}
\setcopyright{cc}
\setcctype{by}
\acmConference[CHI '25]{CHI Conference on Human Factors in Computing Systems}{April 26-May 1, 2025}{Yokohama, Japan}
\acmBooktitle{CHI Conference on Human Factors in Computing Systems (CHI '25), April 26-May 1, 2025, Yokohama, Japan}\acmDOI{10.1145/3706598.3713417}
\acmISBN{979-8-4007-1394-1/25/04}

%%
%% Submission ID.
%% Use this when submitting an article to a sponsored event. You'll
%% receive a unique submission ID from the organizers
%% of the event, and this ID should be used as the parameter to this command.
%%\acmSubmissionID{123-A56-BU3}

%%
%% For managing citations, it is recommended to use bibliography
%% files in BibTeX format.
%%
%% You can then either use BibTeX with the ACM-Reference-Format style,
%% or BibLaTeX with the acmnumeric or acmauthoryear sytles, that include
%% support for advanced citation of software artefact from the
%% biblatex-software package, also separately available on CTAN.
%%
%% Look at the sample-*-biblatex.tex files for templates showcasing
%% the biblatex styles.
%%

%%
%% The majority of ACM publications use numbered citations and
%% references.  The command \citestyle{authoryear} switches to the
%% "author year" style.
%%
%% If you are preparing content for an event
%% sponsored by ACM SIGGRAPH, you must use the "author year" style of
%% citations and references.
%% Uncommenting
%% the next command will enable that style.
%%\citestyle{acmauthoryear}

\newcommand{\amy}[1]{\textcolor{blue}{\textbf{amy:} #1}}
\newcommand{\mina}[1]{\textcolor{blue}{\textbf{mina:} #1}}
\newcommand{\mira}[1]{\textcolor{purple}{\textbf{mira:} #1}}
\newcommand{\revised}[1]{{#1}}
\newcommand{\camready}[1]{{#1}}
%%
%% end of the preamble, start of the body of the document source.
\begin{document}

%%
%% The "title" command has an optional parameter,
%% allowing the author to define a "short title" to be used in page headers.
\title{VideoDiff: Human-AI Video Co-Creation with Alternatives}

%%
%% The "author" command and its associated commands are used to define
%% the authors and their affiliations.
%% Of note is the shared affiliation of the first two authors, and the
%% "authornote" and "authornotemark" commands
%% used to denote shared contribution to the research.

\author{Mina Huh}
\authornote{This work was done during the author's Adobe research internship}
\affiliation{
  \institution{The University of Texas at Austin}
  \country{}}
\email{minahuh@cs.utexas.edu}

\author{Dingzeyu Li}
\affiliation{
  \institution{Adobe Research}
  \country{}}
\email{dinli@adobe.com}

\author{Kim Pimmel}
\affiliation{
  \institution{Adobe Research}
  \country{}}
\email{kipimmel@adobe.com}

\author{Hijung Valentina Shin}
\affiliation{
  \institution{Adobe Research}
  \country{}}
\email{vshin@adobe.com}

\author{Amy Pavel}
\affiliation{
  \institution{The University of Texas at Austin}
  \country{}}
\email{apavel@cs.utexas.edu}


\author{Mira Dontcheva}
\affiliation{
  \institution{Adobe Research}
  \country{}}
\email{mirad@adobe.com}


%%
%% By default, the full list of authors will be used in the page
%% headers. Often, this list is too long, and will overlap
%% other information printed in the page headers. This command allows
%% the author to define a more concise list
%% of authors' names for this purpose.
% \renewcommand{\shortauthors}{Trovato et al.}

%%
%% The abstract is a short summary of the work to be presented in the
%% article.
Vision-Language Models (VLMs) occasionally generate outputs that contradict input images, constraining their reliability in real-world applications. While visual prompting is reported to suppress hallucinations by augmenting prompts with relevant area inside an image, the effectiveness in terms of the area remains uncertain. This study analyzes success and failure cases of Attention-driven visual prompting in object hallucination, revealing that preserving background context is crucial for mitigating object hallucination.



%%
%% The code below is generated by the tool at http://dl.acm.org/ccs.cfm.
%% Please copy and paste the code instead of the example below.
%%
\begin{CCSXML}
<ccs2012>
   <concept>
       <concept_id>10003120.10003121.10003129</concept_id>
       <concept_desc>Human-centered computing~Interactive systems and tools</concept_desc>
       <concept_significance>500</concept_significance>
       </concept>
 </ccs2012>
\end{CCSXML}

\ccsdesc[500]{Human-centered computing~Interactive systems and tools}

%%
%% Keywords. The author(s) should pick words that accurately describe
%% the work being presented. Separate the keywords with commas.
\keywords{Video Editing, Authoring Tools, Generative AI, Human-AI Co-Creation}
%% A "teaser" image appears between the author and affiliation
%% information and the body of the document, and typically spans the
%% page.
% \begin{teaserfigure}

%   \label{fig:teaser}
% \end{teaserfigure}


\begin{teaserfigure}
  \centering
  \includegraphics[width=\textwidth]{figures/teaser_new.png}
  \caption{VideoDiff is a Human-AI co-creative system that supports video creators to explore multiple variations. 1) VideoDiff generates diverse AI recommendations for making rough cuts, inserting B-rolls, and adding text effects. 2) VideoDiff supports easy comparison by aligning videos and highlighting differences using timeline and transcript views. 3) Users can narrow down options by sorting and further customize them by refining and regenerating AI suggestions.}\label{fig:teaser}
\end{teaserfigure}

\newcommand{\ipstart}[1]{\vspace{1mm}\noindent{\textbf{\textit{#1.}}}}
\newcommand\smallverb[1]{\texttt{\small #1}}


\newcommand{\sysname}{VideoDiff}
%%
%% This command processes the author and affiliation and title
%% information and builds the first part of the formatted document.
\maketitle

Gaussian Processes (GPs) \citep{kolmogorov1940wienersche,rasmussen2006gaussian} are an important class of stochastic processes used in machine learning and statistics, with use cases including spatial data analysis \citep{liu2021missing}, time series forecasting \citep{girard2002gaussian}, bioinformatics \citep{luo2023diseasegps} and Bayesian optimization \citep{frazier2018tutorial}. GPs offer a non-parametric framework for modeling distributions over functions, enabling both flexibility and uncertainty quantification. These capabilities, combined with the ability to incorporate prior knowledge and specify relationships by choice of kernel function, make Gaussian Processes effective for both regression and classification.

However, GPs have substantial computational and memory bottlenecks. Both training and inference require computing the action of the inverse kernel Gram matrix, while training requires computing its log-determinant: both are $O(n^3)$ operations with sample size $n$. Further, storing the full Gram matrix requires $O(n^2)$ memory. These bottlenecks require scalable approximations for larger datasets.

Structured Kernel Interpolation (SKI) \cite{wilson2015kernel} helps scale Gaussian Processes (GPs) to large datasets by approximating the kernel matrix using interpolation on a set of inducing points. For stationary kernels, this requires $O(n+m \log m)$ computational complexity. The core idea is to express the original kernel as a combination of interpolation functions and a kernel matrix defined on a set of inducing points. However, despite its effectiveness, popularity (over $600$ citations, a large number for a GP paper) and high quality software availability (\cite{gardner2018gpytorch} has 3.5k stars on github), it currently lacks theoretical analysis. A key initial question is, given a fixed error bound for the SKI Gram matrix and use of cubic convolutional interpolation, how many inducing points are required to achieve that error bound? Given the required value of $m$ as a function of $n$, for what error tolerance is $O(n+m\log m)$ still linear? Following this, what do these errors imply for hyperparameter estimation and posterior inference?

\begin{table*}[h]
\centering
\begin{tabular}{|l|l|}
\hline
\textbf{Quantity} & \textbf{Bound} \\
\hline
SKI kernel error & $O(\frac{c^{2d}}{m^{3/d}})$ \\
\hline
SKI Gram matrix error & $O(\frac{nc^{2d}}{m^{3/d}})$ \\
\hline
SKI cross-kernel matrix error & $O(\frac{\max(n,T)c^{2d}}{m^{3/d}})$ \\
\hline
SKI score function error & $O(\frac{\sqrt{p}n^{2}c^{4d}}{m^{3/d}})$ \\
\hline
SKI posterior mean error & $O(c^{2d}\frac{\max(T,n)+\sqrt{Tn}n}{m^{3/d}})$ \\
\hline
SKI posterior covariance error & $O(\frac{Tn^{2}mc^{4d}+\sqrt{Tn}mc^{4d}\max(T,n)}{m^{3/d}})$ \\
\hline
\end{tabular}
\caption{Summary of Theoretical Results when using SKI with convolutional cubic interpolation. This shows the rate at which the error of using SKI (vs the exact kernel) grows as a function of important variables. Here $n$ and $T$ are the train/test sample sizes, $d$ is the dimensionality, $m$ the number of inducing points, $p$ is the number of hyperparameters and $c>0$ is a constant. Most importantly, the Gram matrix error grows linearly with the sample size, exponentially with the dimension while decaying at an $m^{3/d}$ rate in the inducing points.}
\label{table:theoretical_results}
\end{table*}

In this paper, we begin to bridge the gap between practice and a theoretical understanding of SKI. We have three primary contributions: 1) The first error analysis for the SKI kernel and relevant quantities, including the SKI gram matrix's spectral norm error. Based on this we provide \textit{a practical guide to select the number of inducing points}: they should grow as $n^{d/3}$ to control error. 2) SKI hyperparameter estimation analysis. 3) SKI inference analysis: the error of the GP posterior means and variances at test points. We find two interesting results: 1) we identify two dimensionality regimes relating SKI Gram matrix error to computational complexity. For $d\leq 3$, for \textit{any} fixed spectral norm error, we can achieve it in linear time using SKI with a sufficient sample size. For $d>3$, the error must \textit{increase} with the sample size to maintain our guarantee of linear time. 2) For a $\mu$-smooth log-likelihood, gradient ascent on the SKI log-likelihood will approach a neighborhood of a stationary point of the true log-likelihood at a $O\left(\frac{1}{K}\right)$ rate, with the neighborhood size determined by the SKI score function's error, which aside from the response variables grows \textit{linearly} with the sample size when increasing inducing points as we suggested. To obtain this, we leverage a recent result \cite{stonyakin2023stopping} from the inexact gradient descent \cite{daspremont2008smooth,devolder2014first} literature.

% By our downstream analysis, this implies that for small dimensionality and kernels exhibiting desirable regularity conditions, we can have arbitrarily small error in the SKI Gram matrix, in our estimated parameters and in posterior inference all in linear time as long as the sample size is sufficiently large.

In section \ref{sec:related} we describe related work. In section \ref{sec:ski-background} we give a brief background on SKI. In section \ref{sec:important-quantities} we bound the error of important quantities: specifically the SKI kernel, Gram matrix and cross-kernel matrix errors. In section \ref{sec:gp-applications} we use these to analyze the error of the SKI MLE and posteriors. We conclude in section \ref{sec:discussion} by summarizing our results and discussing limitations and future work.
\section{Related work}

\subsection{Human-AI video co-creation}
% \mira{I agree with Amy's comments for this subsection. The message of this section is not clear. What do you want the reader to get out of it? Right now you are listing a bunch of systems but you are not telling me why I (the reader) need to know about these systems. How are they related to VideoDiff? You don't have to tell me about the details of each one unless they are related. If your point is - AI has been part of video editing tools for a long time (cite, cite, cite, cite) but no one has looked at editing by building an experience for working with multiple variations. Or if someone has looked at it than say - X looked at the same problem, but we are different because Y.  }

% \amy{I think the thesis of this part is that there is something fundamentally different from traditional editing compared to this new "human-AI co-creation" and because of that we have to think about diffs for some reason. Ideally the high level difference between these two categories would be a more clear early on!}
Traditional video editing tools such as Premiere Pro~\cite{PremierePro} or iMovie~\cite{iMovie} require users to manually inspect the video, plan for what edits to make, and execute them by interacting with the complex UI in video editing tools. With the power of AI to understand the content and quality of videos, new forms of video editing systems have been introduced that speed up and automate the video creation process~\cite{invideo, opus, capcut, Descript}. 
Earlier systems supported users to structure narrated videos into multiple segments to help them quickly skim and select appropriate parts~\cite{truong2016quickcut, truong2019tool, wang2019write, leake2017computational, huh2023avscript}.
% \amy{<- are these the AI related systems you're talking about? might want to make that clear here and in the next sentence that there is automation}
These systems label clips with metadata such as speaker and topic to let users navigate and search~\cite{truong2019tool, huh2023avscript}, or recommend video segments that are semantically relevant to the script~\cite{truong2016quickcut, wang2019write, xia2020crosscast, leake2020generating}.
Beyond organization of clips, researchers have also proposed tools that recommend edits based on the video content and quality. B-Script~\cite{huber2019b} analyzes the video transcript and recommends possible B-roll positions and clips. 
AVscript~\cite{huh2023avscript} detects low-quality footage (\textit{e.g.,} blur, bad lighting) to cut out, and commercial tools such as Descript~\cite{Descript} and Vimeo~\cite{Vimeo} identify filler words and pauses for quick removal. 
Beyond post-editing, prior research has also proposed automatic video creation from various media including text documents~\cite{chi2022synthesis}, webpage~\cite{chi2020automatic}, markdown~\cite{chi2021automatic}, news articles~\cite{wang2024reelframer}, and music~\cite{liu2023generative}. 
% \amy{not quite clear to me how this is different than post-editing. what is happening in these works?} 

More recent tools proposed LLM-powered video editing tools that are capable of understanding users' editing intent in natural language and automatically executing them. ExpressEdit~\cite{tilekbay2024expressedit} allows users to edit videos by typing edit instructions and sketching on the video frame.
% \amy{<- might want to clarify what types of edits this does/does not enable. what's an example edit you can use?} 
LAVE~\cite{wang2024lave} provides an LLM-agent that assists users in planning and performing edit actions while also enabling users to manually refine agent actions. \
% amy{hmm it is a little hard to get a sense of what this does or does not support} 
To support the task of repurposing long videos into short and compelling videos, PodReels~\cite{wang2024podreels} and ROPE~\cite{wang2022record} explore the use of LLMs to create podcast summaries. 
% \amy{are these extractive summaries? it is useful to know the details of how they will be similar or not to tyour work}

Existing Human-AI video co-creation tools already create multiple variations for users to better explore the design space. For instance, ReelFramer~\cite{wang2024reelframer} suggests multiple narrative framing to let users explore and choose video topics, and ROPE~\cite{wang2022record} generates summaries of variable lengths. However, the new task of how users review, refine, and select among these variations of videos remains unexplored. 
% \amy{<- it is underexplored but I think it is not quite clear why it is a problem and how the prior work didn't solve it already? what is the issue that still exists here?}
In this era of AI-driven video creation, we investigate user challenges in reviewing and comparing AI-edited videos.


\subsection{Video summarization, skimming, and browsing}
% \mira{my feedback for this subsection is similar to the subsection above. We need to  say more than just "we build upon prior research. Are the specifics of our browing and skimming techniques similar? Are they novel? Do we leverage existing techniques? What is novel? Remember that the goal of this section is to make it clear to our reviewers what is novel as compared to prior work?}

Understanding and browsing video content is an important first step in video editing. Traditional video players commonly use a ~\textit{seek bar} which indicates the progress of the video and allows viewers to navigate to different parts of the video. To address the difficulty of previewing upcoming scenes, many video players on the web including YouTube provide picture-in-picture previews of thumbnails when viewers hover over the seek bar and video chapters that segment video into multiple sections and add context to each portion for quick skimming.

There is a large body of work that summarizes a video as images by selecting the important keyframes~\cite{truong2016quickcut, chang2021rubyslippers, barnes2010video, boreczky2000interactive, jin2017elasticplay}. 
Video editing tools such as Premiere Pro and iMovie utilize ~\textit{filmstrip}, a visual representation of a video's frames displayed in a linear sequence. This format allows editors to get a quick overview of the entire video at a glance and identify parts for cutting or inserting effects. To enable seamless zoom over filmstrip, Video Tapestries eliminated borders between frames for spatial continuity~\cite{barnes2010video}. 
Other works used grid-view of keyframes to summarize the videos~\cite{boreczky2000interactive, jin2017elasticplay}. Boreczkly et al. proposed a comic book presentation for videos that displayed keyframes with different sizes based on their importance~\cite{boreczky2000interactive} and ElasticPlay visualizes an interactive summary based on dynamic time budget using a grid of thumbnails~\cite{jin2017elasticplay}.

Researchers have also explored video browsing interfaces that leverage video transcripts. To support the navigation of informational videos, these systems highlight keywords in the transcript~\cite{chang2021rubyslippers}, or provide scaffolding steps~\cite{yang2022softvideo, fraser2020temporal} and concepts~\cite{liu2018conceptscape}.
Building upon prior research on video summarization and browsing techniques, our work explores visual representations to support efficient review of multiple edited videos. 

% - based on user viewing history
%     - Data-driven interaction techniques for improving navigation of educational videos
%     - catchlive


\subsection{Comparison and sensemaking}
\camready{
Comparison is crucial in creative tasks, helping users understand the design space and refine their decisions. Variation theory~\cite{marton2014necessary} suggests that learning occurs through experiencing contrasts. Thus, exposing creators to alternatives can help them understand model affordances and make more informed decisions.~\cite{gebreegziabher2024supporting}}

To visualize comparison, 3 main techniques are used: juxtaposition, superposition, and explicit encoding~\cite{gleicher2011visual}.
\textit{Juxtaposition} places distinct visual elements side-by-side to highlight similarities and differences. \textit{Superposition} overlays visuals within a common frame and  \textit{Explicit encoding} reveals the predefined relationship between visualizations. 
Prior work utilized these techniques for video comparison by building sequential and parallel video players~\cite{baker2024interaction, tharatipyakul2018towards}, and highlighting the subtle differences with color overlay~\cite{balakrishnan2015video, baker2024interaction}. 
To support the search of educational videos, VSedu supports video comparison based on topic relevance~\cite{benedetto2024visual}. Video Lens facilitated the search in a large collection of sports videos via rapid playback and event-based search~\cite{matejka2014video}. This line of work extends information visualization techniques to support the comparison of videos from multiple sources. Yet, they do not support video comparison in the editing context, where the comparison involves multiple variations of edited videos from the same source footage. 

\revised{As AI continues to speed up and automate content generation, it is becoming more common to work with multiple variations in parallel in creative tasks including writing~\cite{reza2023abscribe, suh2024luminate, buschek2021impact}, image generation~\cite{dalle3, midjourney, huh2023genassist}, video generation~\cite{sora}, and video editing~\cite{opus, capcut}. While facilitating divergent thinking and decision-making, this new workflow of working with variations demands an additional stage of comparison. This has led researchers to explore ways to compare and understand AI-generated content.} 
Tools have provided multiple levels of abstraction to support users in managing complex information generated by AI~\cite{suh2023sensecape, liu2024selenite}.
\camready{To facilitate collaboration with AI in creative tasks, researchers explored sensemaking and comparison of writing variations~\cite{reza2023abscribe, suh2024luminate, koch2014varifocalreader}, images~\cite{huh2023genassist, almeda2024prompting, koch2020imagesense, brade2023promptify}, slides~\cite{drucker2006comparing}, storyboards~\cite{benedetto2024visual}, and designs~\cite{matejka2018dream}. However, video presents unique challenges due to the temporal nature and multi-modal complexity. We contribute an interface that supports sensemaking of multiple variations of videos and provide empirical insights into how creators utilize variations when co-creating videos with AI.}

% \mira{this is ok but I am left wondering what you took away from previous work on sensemaking with AI generated content? Did you borrow any of those ideas in our design? How well did they generalize? }
% VAICo: Visual Analysis for Image Comparison

% \subsection{Working with multiple alternatives}

% \mira{I think we need a related work section on working with alternatives in other domains.}

\section{Formative study}
%In Human-AI video co-creation, we hypothesize that video creators will benefit from a system that presents multiple variations of AI-edited videos.\amy{<-- problem feels abstract, I wonder if we can concretely state what happens when you use AI and why we expect multiple versions to be helpful (e.g., multiple right answers for a high level prompt, other systems do this); I think the primary problem we want to lay out here is that comparison is non-trivial so that's why we're doing a formative study?} To understand the motivation and design considerations of such a system \amy{a bit abstract, I would make it clear that we're focusing on the challenges of comparing multiple versions early?}, 
To better understand how video creators work with alternatives and the support they need for comparing and managing them, we conducted a formative study with 8 professional video editors. The formative study consisted of semi-structured interviews and a comparison task with 3 videos.
%edited videos to explore current strategies and challenges with comparing videos.

\subsection{Method}
\ipstart{Participants}
We recruited 8 professional video creators using Upwork\footnote{\url{https://www.upwork.com/}} (P1-P8, \S{A} Table~\ref{tab:participants}). We compensated the participants with their self-set hourly rate (\$28, SD=5.3) for a 1-hour remote study conducted via Zoom. Participants had an average of 10.5 years of experience (SD=3.0) and created a wide variety of videos including commercials, interviews, documentaries, vlogs, and short-form videos.

\ipstart{Procedure}
% background interview, comparison task, exit interview
We started the study with demographic and background questions about the participants' video editing experiences and current approaches for reviewing and comparing videos.
In the comparison task, participants reviewed and compared 3 videos edited from the same source footage. We pre-uploaded videos to YouTube so that users could use a familiar interface that offered speed controls, hover thumbnails, and the video transcript. 
%, either from LLM-edited or expert-edited versions. 
To become familiar with the original version, participants first skimmed the source footage. Then, the participants reviewed the 3 edited versions for 30 minutes. 
% Participants were free to watch the videos and review the transcripts.
%we provided three edited videos and transcripts which participants reviewed and compared for 30 minutes. 
They were allowed to take notes to help them remember any details. At the end of the task, participants were asked to summarize the differences between the three videos and choose one they preferred, explaining their choice. We conducted a post-task interview to understand their strategies for comparing videos and any challenges they encountered.

\ipstart{Materials}
For the comparison task, we used 2 different source videos (F1: cooking tutorial~\cite{cooking_video}, F2: TED talk~\cite{ted_talk_video}).  We chose these videos aiming for source footage that was largely unedited and varied in duration (F1: 1hr 11min, F2: 12 min) and visual information (F1: featuring many objects, F2: primarily talking head).
We used 2 approaches to create multiple edited versions of the source video -- 1) with AI and 2) with experts. This allowed us to see how users work with variations created today with the current capabilities of AI video tools and in the future when AI capabilities improve.
For AI edits, we created 3 versions of the summary for each source video using an LLM to extract subsets of the video. 
Using the video transcript, we instructed GPT-4 to generate 3 summaries on the video topic (F1: How to make Eggs Benedict, F2: Talk highlights). 
%GPT-4 prompting (edited V1s: 152sec, SD=24; edited V2s: 283sec, SD=110) \amy{keep terminology similar: e.g., rough cut vs. edited V1}. 
%These videos only varied in the parts included.\amy{In the prior two sentences it is a bit hard to understand what is done to these videos. We might want to explicitly say that we used GPT-4 to select a subset of shots to include in the short video (i.e. extractive summarization) or something like this}
% To create variations manually, 
%To further explore the variations of end-to-end edited videos, 
We also hired 6 professional video editors to manually create 3 edited versions of F1 and F2~\footnote{See Supplementary Materials for examples.}.
%mira: removing this because it will be in table mentoned above (edited V1s: 451sec \& SD=152, edited V2s: 323sec \& SD=21).  
The video editors made a variety of editing decisions including trimming content to enhance focus, adding subtitles, B-roll footage, and motion graphics, incorporating text overlays and background music, and making audio enhancements. Each participant was given 3 edited versions from the same source video created with the same approach either with AI or manually. For F1, LLM edited videos were 152 seconds (SD=24) and expert edited videos were 451 seconds (SD=152). For F2, LLM edited videos were 283 seconds (SD=110) and expert edited videos were 323 seconds (SD=21).
%While tools like CapCut and OpusClips can generate edited videos, they only extract a single highlight moment. Thus, we sought to explore summary videos that extract and merge multiple parts to explore complex and comprehensive comparisons.


% analysis
\ipstart{Analysis}
To understand how participants explore alternatives, we analyzed the study recordings and participants' notes. We transcribed the interviews and participants' spontaneous comments during the task and grouped the transcript according to 1) current practices, 2) comparison strategies, and 3) opportunities and challenges of video editing via comparison. 

\subsection{Findings}
\ipstart{Current Practices}
Participants reported that decision-making via comparison is a common process throughout the video editing process. They mentioned having to review and select which clips to include from the footage provided by their clients (P1, P3-4, P7-P8), decide on the appropriate B-roll images (P2, P4), and experiment with different logo styles or placements (P5-P6).

Participants often had to create multiple versions of videos upon clients' requests (P1, P3-P8). 
This was because clients often found it challenging to imagine the final look of the videos from the editors' descriptions alone. P5 and P6 who created ad videos for marketing companies created two versions for companies' A/B testing to see which version is more effective. 
Clients often requested videos that have different lengths, storylines, intro versions, caption styles, or music styles. P3 and P4 mentioned that clients often liked different aspects of multiple versions so they had to combine the versions in further editing. 
While creating multiple versions helped clients narrow down and make decisions, it required a lot of time and effort from video editors. P5 said ~\textit{``I know they (clients) want to see three completely different videos, but I try not to because it takes forever. I only tweak small things that involve minor changes and can be done quickly.''}
As a workaround, participants created variations and got intermediate feedback in each stage (\textit{e.g.,} sharing multiple variations of scripts before moving on to the storyboard) so that they could minimize the amount of rework required later in the process (P1, P5, P8). Alternatively, P4 and P6 created multiple versions of the introductory part of a video instead of the full videos as introductions can capture the essence of what the rest of the videos will look like.
% + how do they review edits


\ipstart{Comparison Strategies - Watching Videos}
In the comparison task, participants watched the 3 versions of the edited videos to understand their differences. 5 participants watched each video from beginning to end before moving on to the next video (P1-P4, P6) while 3 participants switched to play other videos (P5, P7-P8). P7 first watched the introductions of all three videos because it was easier to remember and compare shorter segments and the introductions often captured the holistic style of the video. P5 and P8 switched to other versions of edited videos when they were not satisfied with the current version's edits (\textit{e.g.,} text effects) to check if other videos offered better alternatives. 
While reviewing edited videos, P5 revisited the original unedited video to check which part of the source video was missing. 4 participants re-watched the same video multiple times to identify parts they missed in the first review attempt (P3, P5-P6, P8). 

\ipstart{Comparison Strategies - Taking Notes}
7 participants took notes while reviewing edited videos to better recall the differences (P2-P8). They took notes on the content covered or omitted (\textit{e.g., yolk separation step is not included}), errors (\textit{e.g., mid-sentence jumps, flash frames}), what they liked or disliked about the edits applied (\textit{e.g., Nice intro music, Too many text effects}), how they would have edited differently (\textit{e.g., I would have emphasized the end results by showing it longer}), and how the video compared to other versions (\textit{e.g., This is better at showing the sequence of events compared to the first one}). 
Participants reviewing videos edited with LLM noted errors in the edits, such as sentence cut-offs and jump cuts.
P1 did not take any notes because pausing and taking notes would make the reviewing process much longer.

\ipstart{Challenges}
All participants mentioned the comparison process to be very time-consuming as they had to watch multiple videos to identify the differences. As a workaround, 5 participants sped up the video and skipped around the videos to find the edit points (P3-4, P6-P8).
Although the transcripts of the edited videos were provided, participants noted that simply reviewing the transcript or thumbnail previews on the timeline cannot fully replace the watching experience. P1 explained~\textit{``Simply reading the transcript isn't enough because the transcript might seem fine, but the visuals might have been messed up.''} P3 who checked for edit errors in the video noted that errors such as sentence cut-offs or jump cuts cannot be spotted using the transcript or by scrubbing through the thumbnails; she had to watch the entire video to identify them.

Comparing edited videos was also mentally demanding as participants had to compare multiple aspects of videos at the same time (\textit{e.g.,} storyline, visual effects, B-rolls, subtitles, music). 
As a result, participants often reviewed each version multiple times to focus on reviewing a single aspect at a time (P3, P5-P6, P8). P5 described ~\textit{``I cannot focus on multiple aspects at once while reviewing. When I’m checking the colors, I only look at colors.''}
When comparing the content, it was challenging to switch between multiple videos as they extracted different parts of the original video. For instance, P3 wanted to check if all videos included a specific cooking step but found it difficult as the step appeared in different timestamps across the videos. 

Participants mentioned that it requires a lot of effort to manually identify edits made throughout the video, take notes to remember details, and understand both high-level and low-level differences among videos from their notes. In the post-task interview, P5 mentioned that he wanted AI to describe the high-level theme or goal of each edited version to help him better understand what he was reviewing and what to expect. P6 wanted to have a list of all the edits made in each video so that he would not miss subtle edits (\textit{e.g.,} color correction, small text effects) and could quickly jump to those specific edit points for preview.
% Understanding high-level story differences and effectiveness will take hours (vs. identifying low-level edit differences)

% \mira{let's add something here about errors and the differences between those who saw the AI generated videos vs those who saw the manually created videos}

%\ipstart{Benefits of exploring variations}
%\mina{do we need this section?} \amy{I think it would be good to further motivate our approach if participants said anything interesting as it was sort of a novel task. I might wonder if participants saw any value in this exercise}
%removing for now since we are short on time. We could bring it back later as it makes sense

\subsection{Design Guidelines}
Video editors in our formative study currently make decisions by comparison throughout the video editing process and mentioned the benefits of having multiple variations of edited videos. We envision that in near future, Human-AI video co-creation tools will more commonly provide users with multiple variations. 
Our formative study reveals design guidelines (D1-D6) to better support video creation with alternatives. Our work aims to address D1-D4 and D6 while leaving D5 as a future work.


\begin{itemize}
    \item[\textbf{D1.}] \textbf{Minimize redundant watching by aligning variations} 
    Participants found it difficult to repeatedly watch overlapping sections of edited videos to identify edit points and differences. They wanted to compare edited variations with the original video and across alternatives but found it challenging with mismatched timelines. Aligning variations can simplify comparisons and reduce redundant viewing. 
    
    \item[\textbf{D2.}] \textbf{Support quick skimming by highlighting differences}
    Prior work explored comparison in text~\cite{reza2023abscribe, gero2024supporting} or images~\cite{huh2023genassist, almeda2024prompting}, but the temporal aspect of videos makes it challenging to skim them for comparison. This can be time-consuming and tedious as video lengths increase or as more videos are compared. A system that describes or visualizes the differences among edited videos can help users quickly make decisions.
    \item[\textbf{D3.}] \textbf{Enable independent comparison for different editing stages} Compared to traditional video editing where editors go through an interactive process of making sequential edits and reviewing, recent AI products such as CapCut or OpusClip speed up this process and generate fully edited videos. The resulting videos can be mentally demanding to compare, as users have to consider multiple aspects (\textit{e.g.,} rough cuts, subtitle styles) together. Enabling users to compare aspects independently can reduce the cognitive load.
    % Mentally demanding to do one at a time --> find related work on this
    % \item[\textbf{D4.}] \textbf{Provide structural anchors for alignment and navigation}\\
     \item[\textbf{D4.}] \textbf{Support comparison via multiple modalities} Relying on a single medium, like transcripts or thumbnails, is insufficient for comparing video variations. Edits in visuals or timing can be unnoticed when only text is reviewed. Supporting comparison through multiple modalities—such as transcripts, visual thumbnails, and audio—enables users to identify various types of differences in the most effective way and ensures a more thorough comparison of edits.
    % Multimodalitieis --> clarify what they are
    \item[\textbf{D5.}] \textbf{Support verification of edit suggestions} As AI-powered tools automate suggestions, prior work has explored verifying AI's recommendations~\cite{huh2023genassist, ferdowsi2024validating}. In video creation, automated edit suggestions can often lead to additional errors (\textit{e.g.,} trimming a clip leading to jump cut) and thus needs careful validation. Future work can explore automatically identifying and communicating such erroneous suggestions to users.
    \item[\textbf{D6.}] \textbf{Support management and customization of variations}
    Formative study participants who reviewed a small number of alternatives all mentioned that they wanted to further refine the edits as none of the versions fully aligned with their preferences. Scaling up the number of alternatives increases the chance of finding a preferred version but it also makes it more difficult to manage and select from a larger pool of variations.
    
    % \item[\textbf{D6.}] \textbf{Allow for flexible refinement and regeneration}\\

    % \mina{should we add a design goal about the organization? (our sort feature, pin, etc)}
    % agency, 
    
    % authoring perspective\\
    % solutions: further refine interactions, users can also export and make further edits\\
    % \item[\textbf{DX.}] \textbf{Support organization of multiple variations}\\
    % solutions: filter/sort, overview of design space\\
    % Downside: reduced control, distraction → Therefore, an evaluation mechanism (an ‘‘idea monitor’’) is needed to judge the appropriateness of generated responses
    % Need to check if the branches still align with our goals
    % \item[\textbf{DX.}] \textbf{Facilitate divergent and convergent thinking}\\
    % solutions: in-situ branches (alternatives), post-edit alternatives
\end{itemize}




\section{\TermName Term Detection System}
\label{sec:detection_section}


\begin{figure}[!t] 
 \centering
 \includegraphics[width=0.99\columnwidth]{imgs/plugin_design.pdf}
 \caption{\textbf{\platform Design}---
  (1) When the user activates the plugin, the current page URL is sent to the backend. (2) The terms and conditions are crawled and combined with the page information. (3) The pluggable LLM module analyzes the data to identify unfavorable financial terms. (4) Alerts are generated and displayed on the front end to warn users of potentially unfair financial terms. }
\Description[Diagram of \platform plugin workflow.]
 {This figure illustrates the design of the \platform plugin, which detects unfavorable financial terms on shopping websites. 
 (1) When a user activates the plugin, it sends the current page URL to the backend.
 (2) The backend crawls and extracts the terms and conditions associated with the page.
 (3) A pluggable LLM module processes the extracted data to detect potentially unfair financial terms.
 (4) If any unfavorable terms are identified, the plugin generates an alert and displays it on the front end to warn the user.}
\label{fig:plugin}
\end{figure}


In this section, we introduce \platform, a Chrome plugin designed to detect \termname terms on e-commerce websites. Built upon the insights gained from the \termname term template and topic modeling analysis, \platform enables efficient identification of potentially harmful financial terms, providing users with real-time protection against \termname terms.


\subsection{System Overview}



Our detection system is illustrated in \autoref{fig:plugin}. When a user activates \platform, the URL of the current page is sent to the backend. Upon receipt, the backend crawler collects the terms and conditions pages. These term pages, along with the HTML content of the current page (and a screenshot if paired with a multimodal LLM), are preprocessed and sent to the pluggable LLM module for further analysis. If the LLM module flags any terms as \termname terms, the alert generator sends the identified terms back to the frontend, where they are displayed to the user.








\myparagraph{Pluggable LLM Module} 
We parse the current page to determine if it is a payment page, improving alert accuracy by cross-checking terms with payment page details. For example, in \autoref{fig:example}, the term ``You will be charged \$6.85 for the shipping and handling of your free smartwatch'' aligns with the payment page, making it less concerning than the Immediate Automatic Subscription term, ``you will receive a subscription to the FitHabit Fitness App for only \$86,'' which is not shown on the payment page.

The Pluggable LLM Module, a key part of our system, analyzes both terms and conditions pages and the current webpage. By keeping the LLM decoupled from the backend, we allow flexibility in integrating different models. This enables multimodal models like GPT-4, GPT-4o~\citep{openai2023gpt4}, or LLaMA 3.2~\citep{llama3.2-90B-vision} to process screenshots and terms, or text-based models such as GPT-3.5~\citep{gpt35}, LLaMA~\citep{touvron2023llama}, Mistral~\citep{jiang2023mistral}, or Gemma~\citep{team2024gemma} to analyze HTML and terms.


\begin{table}[t!]
    \centering
    \footnotesize
    \caption{Statistics of annotated datasets for fine-tuning and validation for each term category.}
    \label{tab:dataset_stats}
    \begin{tabular}{p{3.5cm} p{1.5cm} p {1.5cm}}
        \toprule
        \textbf{Type} & \textbf{Fine-tuning} & \textbf{Validation} \\
        \midrule
        
        
        Post-Purchase & 51 & 48 \\
        Legal & 30 & 30 \\
        Termination and Account Recovery & 15 & 16 \\
        Purchase and Billing & 32 & 32 \\
        \midrule
        Unfavorable Terms Combined & 128 & 126 \\
        \midrule
        Benign & 116 & 119 \\
        \midrule
        Total Count & 244 & 245 \\
        \bottomrule
    \end{tabular}
\end{table}



\myparagraph{Backend Core Module}
The alert generator receives flagged \termname terms and checks if the user is on a payment page. If so, it only flags terms not displayed on that page. GPT-4o analyzes page screenshots to mimic the user's experience and guard against adversarial text-based evasion. When the page is not a payment page, all flagged financial terms are shown. Since returns and refunds are rarely disclosed on payment pages, our evaluation in~\S\ref{sec:eva} focuses on scenarios where the user is not on a payment page and seeks to assess financial risks in advance.





\section{User Evaluation}~\label{sec:user_eval}
To understand how well \sysname{} assists users in comparing multiple AI-edited videos, we conducted a within-subjects study with 12 video editors comparing \sysname{} to a baseline. 
Our study aims to investigate the following research questions:

\begin{itemize}
    \item[\textbf{RQ1.}] \textbf{How well does VideoDiff support video comparison?}
    \item[] Compared to the baseline, VideoDiff significantly decreases the time in video comparison (H1), improves comprehension and accuracy in video comparison (H2), lowers cognitive load (H3), and is more useful for video comparison (H4).
    
    \item[\textbf{RQ2.}] \textbf{How well does VideoDiff support video authoring?}
    \item[] Compared to the baseline, VideoDiff significantly increases satisfaction in the final video (H5), provides better creativity support (H6), and is more useful for video authoring (H7).
    
    \item[\textbf{RQ3.}] \textbf{How does VideoDiff impact video editing workflows?} 

\end{itemize}


\subsection{Method}
\ipstart{Participants}
We recruited 12 participants with diverse video editing experiences using mailing lists (P9-P20, Table~\ref{tab:participants}). 6 of them (P9-P14) described themselves as proficient, having 8.33 years of video editing experience (SD=2.94). The other 6 participants (P15-P20) identified as beginners with 3.83 years (SD=1.17) of experience. We compensated \$75 to professional video editors and \$30 to amateur editors for the 1.5-hour remote study conducted via Zoom.

\begin{figure}[t]
  \centering
  \includegraphics[width=\columnwidth]{figures/baseline_new.jpg}
  \caption{The baseline interface shares a similar UI design and features with existing AI video editing tools that support generating multiple videos (\textit{e.g.,} OpusClip, CapCut).}\label{fig:baseline}
\end{figure}

\ipstart{Baseline}
The baseline (Figure~\ref{fig:baseline}) shares a similar UI design and features with existing AI video editing tools that support generating multiple videos (\textit{e.g.,} OpusClip, CapCut). When the user uploads source footage, the baseline system generates 10 different videos with different rough cuts and B-rolls applied. Users can sort the clips based on the duration and the number of effects. 
% add about edit/re-generation feature



\ipstart{Materials}
We selected 3 videos (V0-V2, Table~\ref{tab:video_materials}) from YouTube that feature primarily raw footage with few edits, real-world camera footage rather than screen recordings, and have narration. V0 was used in the tutorial session for both VideoDiff and baseline conditions. 
For the main study, we selected two videos (V1-V2) from the same YouTube channel which are similar in terms of length, amount of narration, and scene changes. We asked participants to watch and familiarize with the original videos prior to the study. 


\begin{table}[h!]
\sffamily\def\arraystretch{0.9}\setlength{\tabcolsep}{0.4em}
  \centering
  \begin{tabular}{ccccc}
    \hline
%     [12:41] WEEKLY GROCERY HAUL AND MEAL PLAN | Shop with me!!
% [13:10] Huge ALDI Grocery Haul | Shop with me and haul!!
% Hiking vlogs
% [11:42] HIKING STAIRWAY TO HEAVEN
% [10:22] HIKING VLOG cave edition
    \textbf{Video ID} & \textbf{Video Type} & \textbf{Duration} & \textbf{Task} & \textbf{URL} \\ \hline
    V0 & Campus Tour & 12:28 & Tutorial & \cite{v0} \\ \hline
    V1 & Grocery Haul & 12:41 & Comparison & \cite{v1} \\ \hline
    V2 & Grocery Haul & 13:10 & Comparison & \cite{v2} \\ \hline
    % V3 & Hiking Vlog & 11:42 & Authoring & \cite{v3} \\ \hline
    % V4 & Hiking Vlog & 10:22 & Authoring & \cite{v4} \\ \hline
  \end{tabular}
  \caption{Videos used in the evaluation study.}
  \label{tab:video_materials}
  \vspace{-3pt}
\end{table}

\begin{table*}[h!]
\sffamily\def\arraystretch{0.8}\setlength{\tabcolsep}{0.4em}
  \centering
  \begin{tabular}{lll}    
    \toprule 
     Edit Stage & Category & Example Questions \\
    \midrule
    \multirow{5}{*}{Rough-cut} 
     & \multirow{3}{*}{Single-select} & (audio) Q1. \textit{Which video talks about most grocery items?} \\
     & & (audio) Q2. \textit{Which video mentions the calories of strawberry lemonade?} \\
     &  & (visual) Q3. \textit{Which video features the highest proportion of talking heads?} \\
     \cmidrule{2-3}
     & \multirow{2}{*}{Multi-select} & (audio) Q4. \textit{Which videos mention her salad recipe?} \\
     &  & (visual) Q5. \textit{Which videos show the scissors?} \\
    \midrule
    \multirow{5}{*}{\makecell{B-roll \&\\Text effects}} 
     & \multirow{3}{*}{Single-select} & (visual) Q6. \textit{Which video has the most B-rolls showing food?} \\
     &  & (visual) Q7. \textit{Which video shows the B-roll image when talking about the protein bar?} \\
     & & (both) Q8. \textit{Which video shows a B-roll image when talking about Z-bar?} \\
     \cmidrule{2-3}
     & \multirow{2}{*}{Multi-select} & (visual) Q9.\textit{Which videos show B-rolls with hamburgers?} \\
     &  &  (both) Q10.\textit{Which B-rolls are shown when talking about the restaurant?} \\
    \bottomrule
  \end{tabular}\caption{Questions used in the user study's comparison task included both single-select (choosing one answer that meets the criteria) and multi-select (selecting all answers that meet the criteria) formats. These questions required participants to check either the narration, visuals, or both, with the specific questions varying depending on the assigned video.}\label{tab:comparison_questions}
\end{table*}



 
\ipstart{Procedure}
We started the study with demographic and background questions about the participants' current video editing practices. Next, we gave a 10-minute tutorial on both \sysname{} and the baseline interface using V0. The participants then completed two tasks: the \textit{comparison task} (\textbf{RQ.1}) and the \textit{authoring task} (\textbf{RQ.2-RQ.3}). \revised{The comparison task studies how users use \sysname{} when the comparison is an end goal in itself and the authoring task explores when the comparison is the means to accomplish the higher-level goal of creating the final video.} 
In the comparison task, participants reviewed 10 different videos using \sysname{} or baseline to answer comparison questions (Table~\ref{tab:comparison_questions}). \revised{We derive these questions from the formative study to consider what aspects video editors compare to select the video they preferred: the content of the videos (\textit{e.g.,} which key visual scenes are included or which key concepts are mentioned) and the editing techniques applied (\textit{e.g.,} the style, placement, and frequency of visual effects). 
We aimed to cover diverse types of tasks, including both single-select and multi-select formats, as well as questions that required participants to assess narration, visuals, or a combination of both.
Also, our questions cover both comparison modes identified by Gleicher et al.~\cite{gleicher2011visual}: ~\textit{explicit comparison} where targets are known and available (\textit{e.g.,} Q4. \textit{Which video mention her salad recipe?}) and ~\textit{implicit comparison} where targets are hidden, requiring more exploratory search (\textit{e.g.,} Q3. \textit{Which video features the highest proportion of talking heads?}).} 
% To ensure a fair comparison with the baseline, we did not include questions where \sysname{} would immediately reveal the answers (\textit{e.g.,} finding videos that have most B-rolls or text effects).}
In each interface condition, users were assigned to one of V1-V2.
For each question type, we measured the time taken, answer accuracy, and interaction logs to understand which system components participants utilized to perform each comparison. For accuracy, we assigned a rating of 1 if the answer was completely correct, 0.5 if the participant got half or more of the answers right for multi-select questions, and 0 if less than half or none were correct.
After each interface, we conducted a post-stimulus survey that included the following ratings: mental demand, performance, effort, frustration, and usefulness of the system in understanding differences between videos. All ratings were on a 7-point Likert scale.

In the authoring task, participants reviewed the video suggestions and further edited videos using VideoDiff or baseline. In each interface condition, users were assigned to one of V1-V2.
We guided participants with an editing scenario: \textit{``You have created this initial video about grocery haul. To edit this video you can review AI's 10 different edit suggestions to select the final video or regenerate, refine, or recombine edit suggestions.''} After each interface, we conducted a post-stimulus survey to measure users' level of satisfaction with the final video, creativity support indexes (exploration, engagement, effort/reward tradeoff, expressiveness)~\cite{cherry2014quantifying}, and usefulness of the system in video authoring.
In both the comparison and authoring tasks, the order of the interfaces and videos was counterbalanced and randomly assigned to participants. 


\subsection{Comparison Task Results}~\label{sec:comparison_task_results}
Overall participants rated VideoDiff to be significantly more useful in understanding differences (Figure~\ref{fig:survey_results}, $\mu$=4.92, $\sigma$=1.16 vs. $\mu$=2.25, $\sigma$=1.06; $Z$=-2.77; $p$<0.05). In this section, we share 1) participants' strategies for reviewing and comparing videos to answer comparison task questions, 2) task results (RQ1), and 3) perceived workload (RQ1).


\ipstart{Comparison Strategies}
% overall
To narrow down the search space, 5 participants in the baseline condition (P9, P11, P14, P17, P19) and 6 participants in VideoDiff condition (P11, P14, P17-P20) first sorted variations based on the comparison question (\textit{e.g.,} sort by longest duration for Q1). With VideoDiff, 2 participants first identified which section is relevant to the question (\textit{e.g.,} ``Meal Plan'' section for Q9) then only checked within the variations that covered the section (P9, P18).
All participants using VideoDiff switched between timeline and transcript views depending on the question type. They used the edited view to find what each variation covers, and 2 participants additionally utilized the source view to align videos side by side (P10, P19). 3 participants were initially unsure which view would be more helpful for a specific comparison question and tried switching views multiple times (P12, P14-P15).


% visual comparison
To answer the questions that require visual comparison between videos, all 12 participants in the baseline condition frequently interacted with the video player of each variation directly. Participants scrubbed the video player timeline to skim through the visual content of the video or click-jumped to navigate to different parts of the video.
After finding the visual of interest in one video, 4 participants tried to navigate to the similar timestamp in other videos but realized they needed to re-search as all videos extract different parts of the video.
As searching in each variation took time, P12 took notes to remember the details in each variation. She noted ~\textit{``10 is a lot to compare at once. I need to take notes.''}
Because participants could not preview thumbnails of multiple videos in parallel, 2 participants tried playing multiple videos synchronously but found it challenging to compare them as they were not aligned (P13-P14). As it was time-consuming to review the visuals of each variation, 4 participants used a workaround by searching for relevant keywords in the transcript (P14, P17-P19). P19 searched ~\textit{``I got ''} to check what grocery items are mentioned in the narration. 
% For visual comparison of B-rolls and text effects, baseline participants first 
% system



Participants using VideoDiff mainly checked the timeline view to identify the shots included in each variation for rough cut comparison. If a timeline did not display a thumbnail of the visual they were searching for, 3 participants further ensured that the missing thumbnail was not simply due to the periodic nature of frame extraction while the visual is actually present in the video. P9 and P15 located sections that have thumbnails with similar background and shot type, navigated to the part and scrubbed the video player to skim the visuals. Similarly, P19 reviewed the ~\textit{source timeline view} to check whether the frame of interest was present outside of selected sections. 
When looking for specific B-roll images, P18 mentioned the benefit of visual keyword search using the B-roll keywords in the timeline view. To check the video context when a B-roll or text effect was applied, participants either reviewed the thumbnails below the effects or clicked on the effects to preview them.

% audio comparison 
For questions that involve checking the audio (\textit{i.e.} narration) of the video, participants in both conditions used keyword search (\smallverb{ctrl-F}) in the transcript. 
In the baseline condition, 5 participants read the transcript of each variation sequentially which was time-consuming (P9, P11, P14, P18-P19). P19, a low-vision participant noted ~\textit{``In this version [Baseline], it is mostly reading than watching so very tiring for my eyes!''}
With VideoDiff, 2 participants first searched where the relevant part appears in the transcript in one of the variations, then scrolled horizontally to quickly compare whether other variations cover the part using ~\textit{source view} (P9, P14). 



\ipstart{Time and Accuracy for Comparison Task}
We report the results of the comparison task in Figure~\ref{fig:completion_time} (task completion time) and Table~\ref{tab:comparison_accuarcy} (task accuracy). On average, participants spent 38 seconds (SD=14) for each comparison question with \sysname{}, roughly half the time spent with the baseline (74 seconds, SD=38). In 5 questions that required checking multiple parts of the video (\textit{e.g., Which video has the most B-rolls showing food?}), participants using VideoDiff were significantly faster than in Baseline. 
In the baseline, 8 participants failed to find the correct answer for one or more questions within the 3-minute limit (P9, P12-13, P15-P16, P18-P20). 5 participants provided a rough guess answer instead of carefully comparing variations after watching the first few videos (P9, P12, P16-P18). P16 notes ~\textit{``In only three minutes? I'll just have to guess as I cannot watch all these videos.''}


\begin{figure}
  \centering
  \includegraphics[width=\columnwidth]{figures/completion_time.pdf}
  \caption{Average task completion time to answer comparison questions in Table~\ref{tab:comparison_questions}.(~\textit{p} < 0.05 is marked with *)}\label{fig:completion_time}
\end{figure}




Baseline participants often provided incorrect or incomplete answers when the question involved visual comparison. 3 participants who scrubbed the timeline to search for an object missed it when it only appeared for a short amount of time (P11, P17-P18). While 2 participants searched keywords related to the visual in the transcript to speed up the search, some videos only mentioned the object and did not show it (P12, P17).
While VideoDiff's filmstrip in the timeline enabled participants to more quickly compare the visuals of multiple variations, 5 participants missed the visual search question (Q5) when the object asked only appeared briefly in the video and was not captured in the filmstrip.

\begin{table*}[h!]
\small % Adjusting font size to be smaller
\sffamily\def\arraystretch{1.2}\setlength{\tabcolsep}{0.3em}
  \centering
  \begin{tabular}{c|cccccccccc}
    \hline
    \textbf{} & \textbf{Q1} & \textbf{Q2} & \textbf{Q3} & \textbf{Q4} & \textbf{Q5} & \textbf{Q6} & \textbf{Q7} & \textbf{Q8} & \textbf{Q9} & \textbf{Q10} \\ \hline
    \textbf{Baseline} & 0.67 (0.49) & 1.00 (0.00) & 0.42 (0.51) & 0.96 (0.14) & 0.63 (0.31) & 0.50 (0.52) & 0.36 (0.50) & 0.92 (0.29) & 0.46 (0.40) & 0.67 (0.25) \\ 
    \textbf{System} & 0.92 (0.29) & 1.00 (0.00) & 1.00 (0.00) & 1.00 (0.00) & 0.75 (0.34) & 1.00 (0.00) & 0.90 (0.32) & 0.91 (0.30) & 0.96 (0.14) & 0.86 (0.23) \\ \hline
  \end{tabular}
  \caption{Task Accuracy for answering comparison questions in Table~\ref{tab:comparison_questions}. The value represents the average accuracy (SD).}
  \label{tab:comparison_accuarcy}
\end{table*}




\ipstart{Cognitive Load in Video Comparison}
Figure~\ref{fig:survey_results} shows the distribution of survey ratings. Our NASA-TLX~\cite{hart2006nasa} results indicate that 
VideoDiff required significantly less mental demand ($\mu$=4.33, $\sigma$=1.4 vs. $\mu$=2.75, $\sigma$=0.97; $Z$=-2.08; $p$<0.05), temporal demand ($\mu$=4.33, $\sigma$=1.72 vs. $\mu$=2.33, $\sigma$=1.15; $Z$=-2.36; $p$<0.01), effort ($\mu$=4.42, $\sigma$=1.08 vs. $\mu$=2.92, $\sigma$=1.08; $Z$=-2.62; $p$<0.05), and frustration ($\mu$=3.75, $\sigma$=2.01 vs. $\mu$=2.33, $\sigma$=1.37; $Z$=-1.78; $p$<0.05). We did not see any significant difference in performance ($\mu$=3.08, $\sigma$=1 vs. $\mu$=2.17, $\sigma$=0.94; $Z$=-1.66; $p$>0.05). 
Participants using baseline mentioned the challenge of finding relevant parts with a video using a video player and transcript and having to repeat the process for multiple variations. P15 noted ~\textit{``It is more tedious than demanding. All videos are not visually distinct so I have to do a lot of manual digging.''} 

After using VideoDiff, participants appreciated having separate views to compare videos across timelines and transcripts. P11 mentioned the benefit of having both in separate views ~\textit{``If I'm shown both views at once, it is too overwhelming and will not fit multiple versions in a page.''} P10, P13, and P19 highlighted that the same color coding of sections across timelines and transcripts helped them pick up information easily. P13 said \textit{``While this one [VideoDiff] has much more information with many different view options, they are not overwhelming and all very helpful and complementary.''} P10 expressed that she wants to use VideoDiff in the future to show her clients multiple versions of videos she has created. 
% - performance 
% comparable

\begin{figure*}
  \centering
  \includegraphics[width=\textwidth]{figures/stacked_bar.pdf}
  \caption{Distribution of the rating scores for the Baseline \raisebox{-1pt}{\includegraphics[scale=0.10]{figures/baseline_icon.pdf}} and VideoDiff \raisebox{-1pt}{\includegraphics[scale=0.10]{figures/system_icon.pdf}}
 (1 = negative, 7 = positive) in the two tasks. Note that a lower value indicates positive feedback and vice versa. 
  {The asterisks indicate the statistical significance as a result of Wilcoxon text} (~\textit{p} < 0.05 is marked with * and ~\textit{p} < 0.01 is marked with **). }\label{fig:survey_results}
\end{figure*}

\subsection{Authoring Task Results}
Overall participants rated VideoDiff to be significantly more useful for creating videos (Figure~\ref{fig:survey_results}, $\mu$=3.25, $\sigma$=0.87 vs. $\mu$=5.42, $\sigma$=0.51; $Z$=2.97; $p$<0.05). In the followings, we share 1) participants' strategies for reviewing and customizing variations for creating a video (RQ3) 2) how well VideoDiff supports creativity (RQ2) and 3) future improvements suggested by participants.

\ipstart{Reviewing Variations and Converging}
In the baseline condition, participants mentioned the difficulty of understanding the differences and quickly identifying which one aligns best with their preference. Thus, 3 participants quickly selected the longest video as they were likely to cover most of the original video (P12-P13, P16). P16 mentioned ~\textit{``I cannot understand what each version is about, so I'll choose the longest one and import it into my own video editor to cut down further.''} Similarly, when reviewing the baseline's B-roll recommendations, P18 chose a version with the most B-rolls as it was easier to remove than add them manually. 
4 participants who first edited videos with the baseline mentioned feeling overwhelmed by starting with 10 alternatives and wanted fewer variations (P10-P12, P14). However, when they used VideoDiff to create videos in the following sessions, P10 and P14 changed their opinion and stated that they liked having 10 variations as a start. P14 said \textit{``Because I can quickly skim the differences with visuals and zoom out, the same amount of suggestions don't feel overwhelming here. ''}

When creating videos with VideoDiff, all participants started in the timeline view and narrowed down the search. To closely compare final candidates or verify edits, 10 participants switched to the transcript view for a side-by-side comparison. P16 said~\textit{``You cannot zoom out and view all text in the transcript at once. It's possible with the timeline view.''}
Similarly, P17 compared two views: \textit{``Timeline view is helpful for creating bones of the videos, deciding what to show for how long. Transcript helps with individual details.'' }
2 participants never used the transcript view and finished editing in the timeline view (P17, P19). P19 said ~\textit{``All video editing software that I use shows timeline, so I'm not used to editing with the transcript view.''} 

In the rough cut stage, participants sorted and shortlisted variations based on the coverage of sections they were interested in. 3 professional video editors highlighted the benefit of segmenting the video into sections (P9, P11, P14). P14 mentioned~\textit{``This is exactly what I do manually every time I start video editing, using colors to differentiate my shots in Premier Pro. This system [VideoDiff] will save me so much time when I'm working on 2-3 hour-long videos.''} 
P15 who narrowed down the rough cut versions to 2 candidates first added B-rolls to both versions to inform decision-making in the rough cut stage. She explained~\textit{``If one version has a part that I can add nice B-rolls, I'll go for that [Rough cut version].''}
When reviewing B-roll and text effect recommendations, participants checked whether the effects were evenly distributed throughout the video using timeline view (P12, P17), whether the B-roll aligns with the narration in the transcript view (P9, P11), and whether the B-rolls and text effects are visually aesthetic and match the style of the video by previewing in the video player (P13). 


\ipstart{Generating and Editing Variations}
In the baseline condition, only 2 participants generated a new set of variations (P11, P13) while the other 10 participants selected the final video from the initial recommendation without further change. P19 said ~\textit{``I wouldn't generate new videos because that would require another comparison!} After generating new recommendations, P11 noted ~\textit{``It is difficult to know how these videos are different from the original variations.'' }

Participants who created videos using VideoDiff made 4.33 edits to existing variations (SD=2.42). 
2 participants combined existing versions (P13, P19) and
3 participants generated new variations (P11, P16-P17). During the rough cut stage, many participants made edits related to sections, such as adding or removing sections, adjusting the coverage of sections, or reordering them. Participants made more edits focused on visual changes in the timeline view (P20 edited \textit{``Show less talking heads, show more grocery items''}), while they made more edits related to narration in the transcript view (P17 edited \textit{``Do not mention the store name''}). While most participants iteratively made edits on the first rough cut variation they selected, P18 switched to another version after making 3 edits on a rough cut version. He described~\textit{``After making a few edits on this one, I realized that I didn't really like it. It's really useful to explore multiple paths at once, and be able to switch between them easily.''}

In the B-roll and text effects stage, participants made edits to add/remove an effect, move the effect to a different part of the video by specifying a timestamp or a transcript sentence, and change the B-roll images or text styles. Participants noted that VideoDiff’s description of specific changes makes it easy to verify new changes without having to play the videos before and after the changes.

% - one participant choose 2 videos as the last results (PX), one for YouTube and one for Shorts/Reels
% - another participant says he could use one version more on Aldi store if the client is Aldi, and one version that shows more meal plan for her youtube channel


% generation/iteration logs


\ipstart{Video Diff as a Creativity Support Tool}
We assessed VideoDiff and the baseline using creativity support index~\cite{cherry2014quantifying} (Figure~\ref{fig:survey_results}). 
% engagement $\mu$=avg, $\sigma$=stdev vs. $\mu$=avg, $\sigma$=stdev; $Z$=z-val; $p$<0.05), and effort--reward tradeoff $\mu$=avg, $\sigma$=stdev vs. $\mu$=avg, $\sigma$=stdev; $Z$=z-val; $p$<0.05). 
VideoDiff supported efficient exploration and comparison between variations ($\mu$=2.5, $\sigma$=0.9 vs. $\mu$=5.33, $\sigma$=0.78; $Z$=3.07; $p$<0.05), which enabled participants to quickly narrow down the search and spend more time customizing them. This led to higher effort--reward tradeoff ($\mu$=3.42, $\sigma$=1.24 vs. $\mu$=5.67, $\sigma$=0.89; $Z$=2.89; $p$<0.05) and higher satisfaction in the final result in VideoDiff than in the baseline ($\mu$=3.67, $\sigma$=0.98 vs. $\mu$=5.25, $\sigma$=1.06; $Z$=2.63; $p$<0.05). After creating the final video with VideoDiff, P13 said ~\textit{``I can see that this version covers all sections in a balanced way and has many aesthetic B-rolls, so I'm confident that I chose the right one.''}

Participants were also more engaged in the creative process with VideoDiff than the baseline ($\mu$=2.83, $\sigma$=0.83 vs. $\mu$=5.67, $\sigma$=0.89; $Z$=3.05; $p$<0.05). P15 compared the experiences in both conditions: \textit{``In the first session [Baseline], I had to read the redundant transcript over and over, and having to read most of the time when I'm trying to create a video is not so fun. With this one [VideoDiff], I didn't need to do repetitive work and could experiment what AI can do, so really enjoyed using it!''}
For expressiveness, we did not see any significant difference between two conditions ($\mu$=2.83, $\sigma$=1.27 vs. $\mu$=4.75, $\sigma$=1.14; $Z$=2.46; $p$>0.05). 
All participants using the baseline interface felt disengaged from the creative process due to limited control over the generation and a lack of understanding of the variations. P19 stated~\textit{``There was no creation in this process, I was just selecting like a judge.''} Similarly, P11 said~\textit{``While being able to generate 10 recommendations is impressive, I'm not providing any input so I don't feel like I'm the one creating something.''}
In the VideoDiff condition, participants showed mixed perspectives towards VideoDiff as a creativity support tool.
P11 described~\textit{``Instead of spending most of the time reviewing, it [VideoDiff] helped me to quickly move on to the creative stage and spend more time and effort there.''} P14 elaborated~\textit{``This tool [VideoDiff] helped me with tedious choices so that I can spend more time on meaningful choices.''}
On the other hand, P10 mentioned that VideoDiff does not make her feel as expressive as using her existing tools.~\textit{``While this is helpful for making things faster, it feels like I'm supervising AI instead of me being creative. I might feel less attached to what I create with this tool.''}
    


% Satisfaction
% Baseline
% - had less control, did not have enough time to explore all variation
% System
%     P16 who is an amateur in video editing mentioned that VideoDiff requires lower mental demand compared to other video editing tools with less complicated controls and AI's proactive suggestions. 


\ipstart{Future Improvements for VideoDiff}
Participants provided suggestions on how to improve \sysname{} in the future.
First, P13 and P20 suggested that VideoDiff can provide alternatives whenever participants request edits to existing variations. Current VideoDiff provides only one edited variation based on the user's edit prompt. Thus, when the new variation does not align with the user's intent, they have to refine the edit prompt multiple times. P20 described~\textit{``I cannot always be specific because I forget to specify all the details in the prompt or I often do not know what I want. When I ask it to ``shorten grocery haul'', it can generate 3-4 versions with different parts dropped.''}
% - show multiple alternatives like google image suggestions when user pins?
Second, 5 participants (P9-P11, P14, P16) wanted more direct control with VideoDiff. P11 described~\textit{``While I can change the duration or change B-rolls using prompt-based editing, some of these are easier with directly stretching the timeline or right clicking on B-rolls.''} P9 also suggested having a slider to control the number of B-rolls and text effects easily. P16, an amateur video editor mentioned~\textit{``Instead of exporting this and further refining using Premier Pro, I just want to finish everything here and directly publish without having to work twice.''}
% - want to specify what Broll images to use (general stocks doesn't preserve personal preferences and often he prepares own purposed B-rolls) \mira{this seems like a leftover bullet that never got deleted? Or maybe it's a note for something to finish later?}
Third, P14 mentioned that future VideoDiff can improve its generation pipeline to consider visuals into account when generating rough cuts and inserting B-rolls. P14 explained~\textit{``I want B-rolls to be placed where the jump cuts are or when there are no interesting visuals. Now it covers the original video's bananas with a B-roll bananas.''}
Finally, P19 a low vision participant provided insightful feedback to make VideoDiff more accessible to blind and low vision video creators. For low vision users who may find reading transcript challenging, VideoDiff can include more detailed subheadings summarizing the transcript's content, in addition to the current section titles, to reduce the amount of reading required. For screen reader users, providing more descriptive text for visuals (\textit{e.g.,} alt-text that describes differences between 2 similar B-roll images~\cite{huh2023genassist}) can support accessible visual comparison of videos.


% 4. HAI understanding each other
%     P14: users would need to understand what it can and cannot do
% then they know when to stop / when to export and edit
% discoverability (Didn't know what's possible) \& AI transparency failure
% - NL/ conversational interface -> 
    
\section{Exploratory Case Studies}
The controlled user study (Section~\ref{sec:user_eval}) showed that video creators found it easier to comprehend differences and create videos using VideoDiff compared to the baseline. To learn how creators would use VideoDiff to edit their ~\textit{own footage}, we conducted an exploratory study with 3 video creators (P9, P21-P22) who brought their footage. 


\begin{figure}[t]
  \centering
  \includegraphics[width=\columnwidth]{figures/exploratory_videos.jpg}
  \caption{In exploratory case studies, creators edited their own footage (V3-V5) using VideoDiff.}\label{fig:exploratory_videos}
\end{figure}


\subsection{Method}
We recruited 3 video editors (2 professionals, 1 amateur) using mailing lists and social media (P9, P21-P22). P9 also participated in the controlled user evaluation study. Before the study, we collected footage from each participant (Figure~\ref{fig:exploratory_videos}). P9 and P22 who wanted to use multiple clips in the study concatenated them into a single source video.
During a 1hr remote study session, we asked participants background questions, provided a tutorial of ~\sysname{}, invited participants to edit their own footage with ~\sysname{}, and asked participants semi-structured interview questions about their experience. We compensated participants \$75 via Amazon Gift Card for preparing the footage and participating in the study.

\subsection{Three Vignettes: VideoDiff in Context}
\ipstart{V3: Science Video Podcast}
P21 is a college student who creates videos for a personal YouTube channel and school projects. For his physics class, he created a video podcast that parodies golf announcers to explain Newton's law in a fun way. His goal for editing was to make an informational and entertaining video while describing all three of Newton's laws. With VideoDiff's initial rough cut suggestions, he first shortlisted versions with many sections which are likely to cover all three laws. P21 noted ~\textit{``It'd be nice if I could ask it to filter out versions that do not cover all three laws''}. While he chose a longer rough cut version to further edit, he also pinned a shorter version, which primarily consisted of jokes, to use as a teaser for his main video.
% - in transcript view, easier to skim the laws and notice sentence repetition. 
% - notice many repetition and ask it to remove repetead sentences but it left the first attempt of the sentence --> normally would keep the last as the earlier are mistakes
When reviewing VideoDiff's B-roll recommendations, he looked for a version with a more consistent style of B-rolls as he was making an educational video. He liked having abundant options for B-rolls and recombined suggestions to create a version that has many infographic illustrations. 
% some versions had the same B-rolls recommended (Because it had repeated lines)
For text effects, he wanted to insert a chapter text for each law but noticed that most versions focused on formulas or jokes. He generated a new version by specifying ~\textit{``Show a lower-third text of what each chapter is about.''}
In the future, P21 wished that VideoDiff would gradually learn from his selection of alternatives and offer more personalized alternatives. ~\textit{``Like reinforcement learning, I want it to learn my preferences over time so I can spend less time customizing.''}


% P21 highlighted the importance of having alternatives when AI cannot fully understand his intent. He said~\textit{It is a}
% Broll
% - Asking AI to find a version that has the most consistent style is tricky, because I don’t trust it 100\%, so I’d prefer to review and find on my own.

% - Keywords highlighting is really useful for skimming
% I’m not sure if what AI thinks the best will be actually best, so I need options so I can decide what is good.


\ipstart{V4: Artist Intro Video}
P22 is a professional video creator who often edits music/art videos and documentaries for clients. He is making an artist intro video with the footage he got from a client and used VideoDiff to make edits. His goal for editing was to fit the video into 2 minutes while talking about the major art pieces. 
When reviewing rough cut variations, P22 first noticed that \#7 and \#8 are both close to 2 minutes, and compare them side-by-side in the transcript to choose a version that mentions more art pieces of the artist.
While he tried to edit the video with a prompt ~\textit{``Remove shots with flashing lights.''}, he noticed that they still remained in the video.
In the B-roll stage, P22 did not apply any B-roll and explained~\textit{``I want to show the artist's own art images rather than stock images.''} He suggested that in the future, VideoDiff can recommend zoom and pan animations for B-roll, as well as movements for text effects, and display them side by side for comparison.
He wanted to continue using VideoDiff as part of his workflow as the alternatives gave him new ideas. ~\textit{``My brain starts to lock once I start manually editing. Even when I try to tweak to something different, it gets fixated easily. It will be always useful to have alternatives.''}

% ~\textit{``This will be useful especially when I work with tons of clips or longer videos as }
% Categorizing footage would be very useful, especially when I have tons of clips, longer videos (for music videos)
% % When editing documentaries – huge footage libraries

% What they liked:
% Starting with alternatives is very useful, it’s nice to be able to choose from multiple options. Brain starts to lock, … once you start editing, even when you try to change. Gets fixated at once.



\ipstart{V5: IoT Setup Tutorial}
P9 is a freelancer video creator who edits videos for YouTuber's vlogs and product reviews. One of the YouTubers asked him to edit their raw footage of a IoT setup tutorial to remove mistakes and make a polished video. 
% bigger roughcut
P9 first chose one rough cut version and tried to make a section longer by editing with a prompt~\textit{``Add more footage for hotspot installation.''}. After noticing that VideoDiff generated a too long version, he switched to the transcript view to check what specific sentences he want to add and refined the edit prompt. 
After reviewing the footage, P9 noticed a lot of repetition in the sentences as the YouTuber corrects themselves. 
He edited the video to remove the repeated sentences and also prompted VideoDiff to cut when the YouTuber is out of the frame. However, he noticed that VideoDiff hallucinated that it had cut the parts out when they were still present.
He mentioned that while VideoDiff is great at making a rough version of a video, he still needed to rely on other tools to manually remove filler words and polish the video. P9 described~\textit{``Usually making a rough cut for a video this long takes about 2 hours, but today I did it in 5 minutes. Although I still have to polish it up, this is so useful!'' }


% But first I need to familiarize with the original video, then you know which sections are important which are worth to spend long time, then I want to look at suggestions


% Broll/Text
% - B-roll style depends on the audience, Gen-Z targetted videos should have more B-rolls to make it fast-paced
% - edit to replace broll image with a specific prompt
% - when there is no good match for B-roll, want system to generate alternative concepts of B-rolls when there is no good match of what I want to show in the B-roll, or generate a new one 


% what they liked
% -  heading (premiere pro needs it, it only shows a wall of text), like that it leverages colors, visually see the distribution of sections
% If I have access to this #6 version (that he chose), this will save me 30-40% of the time


\section{Discussion}
In this section, we reflect on our findings from the design and evaluation of VideoDiff. We also discuss future opportunities to improve and extend VideoDiff. 

\ipstart{Creating Videos with Alternatives}
VideoDiff is a tool designed to support human-AI co-creation of video with alternatives. Our focus on alternatives is driven by prior work that highlights the value of divergent thinking in creative process~\cite{suh2024luminate, dow2010parallel, reza2023abscribe}. Video creators in our user study highlighted 3 benefits of having alternatives. First, it helped to consider alternative edit styles that they are not familiar with. Second, when users give the AI an ambiguous edit request, having multiple alternatives allows them to choose the one that best matches their intent, rather than relying on a single suggestion from the AI that might be incorrect. Third, with alternatives, creators can quickly prototype multiple versions of end-to-end videos adapted to different audiences, settings, and platforms. 
\revised{While editing videos with alternatives is not a one-size-fits-all solution, it represents existing practices in video editing and other generative AI tasks. In this work, we focus on minimizing the burden of working with alternatives in video creation and demonstrate how VideoDiff achieves it.
Still, some users may prefer incremental editing approaches, such as refining a single video~\cite{wang2024lave}, or workflows that prioritize error correction over creative decision-making~\cite{huh2023avscript}. In the future, we will consider diverse workflows by letting users select the number of suggestions (one vs many) and further explore how the scale of alternatives or the timing of providing alternatives can impact users' creation process.
}


\ipstart{\revised{Scope and Use Cases of VideoDiff}}
\revised{We reflect on the scope of VideoDiff based on user groups, creation stage, and control granularity based on the taxonomy of roles that creativity support tools play~\cite{chung2021intersection}. 
First, VideoDiff can support a diverse range of users. Novices or creators with accessibility needs (\textit{e.g.,} screen reader users) who find existing video editing software complicated to use can benefit from VideoDiff’s suggestions. Novices with limited experience with the authoring tool often do not know where to start~\cite{ashtari2020creating} and find it easier to start with examples rather than from scratch~\cite{kim2015motif}. 
For experienced editors, VideoDiff reduces the time spent on routine tasks such as planning and making rough cuts, allowing them to focus on fine-grained adjustments that improve the overall quality of their work.}

\revised{In terms of the creation stage, VideoDiff primarily supports idea generation and execution but not evaluation, requiring users to review and evaluate the quality. While automating execution, VideoDiff still emphasizes user control, allowing for post-editing and customization through natural-language prompts.
In terms of control, VideoDiff provides indirect control (\textit{i.e.,} text prompt) to prioritize a low learning curve over precise editing. This design choice contrasts with traditional video editing software, which offers precise but complex editing capabilities. Thus, it will be more useful in cases when users want to create videos under time constraints (\textit{e.g.,} social media content, event recaps) or budget limits (\textit{e.g.,} personal projects or prototyping for feedback before detailed editing). While users can export the EDL to further edit in other video editing tools, future VideoDiff can incorporate more direct control (\textit{e.g.,} timeline manipulation, dragging B-roll to adjust timing).}

% \revised{With this scope, VideoDiff is ideal for scenarios requiring quick drafts or iterative edits, such as creating social media content or event recaps under time constraints, producing personal or casual projects with limited budgets, or prototyping video ideas to gather feedback before detailed editing. It bridges the ease of natural-language interaction with the precision of traditional editing tools, addressing diverse use cases and user needs.}


\ipstart{Support for Flexible Comparison}
VideoDiff enables quick comparison of video alternatives by supporting users to align multiple videos and highlighting differences using timeline and transcript views. 
While creators using VideoDiff were successful with various comparison tasks, 2 participants were confused about which views to utilize for a few comparison tasks. 
To further minimize creators' mental load in search and comparison, a future iteration of VideoDiff can directly take creators' natural language comparison queries (\textit{e.g., Which video has fewer narration mistakes in the intro of the video?}). It can extract relevant parts of each video and determine the best views to communicate the differences and present them to creators. This can further reduce creators' effort to switch views and help scale up the number of variations for comparison.
Also, VideoDiff's pipeline can consider broader types of comparison beyond what is based on narration and visual content -- shot types (\textit{e.g., Which video features a higher proportion of a talking head?}), audio (\textit{e.g., Which videos use calm music?''}, errors (\textit{e.g., Which video has the least sentence cut-offs?}), or abstract comparison queries (\textit{e.g., Which video has a more hooking intro?}) 
\camready{Also, while D4 highlights comparison through multiple modalities, current VideoDiff focuses on visual and narration differences and does not consider sound or music differences. Prior work has explored audio tagging of videos~\cite{zhang2023peanut} and generating sound effects to augment video skimming~\cite{ning2024spica}. Similarly, future VideoDiff can consider audio-based comparison by surfacing differences in background noises, music inserted, or speech characteristics.}
As the pipeline evolves to identify and describe these various types of differences, finding effective ways to visualize them becomes another important challenge.


\ipstart{Implications of VideoDiff on Creativity}
% \ipstart{Implications of Quick, Parallel Video Authoring}
\revised{
While AI-powered creativity support tools risk design fixation~\cite{wadinambiarachchi2024effects} and lack of control~\cite{suh2024luminate}, we designed VideoDiff to mitigate these concerns. First, VideoDiff supports divergent thinking. Participants in our study mentioned that it is easy to get fixated on one video editing style with existing tools, whereas VideoDiff inspired them to explore alternative ideas, which they found valuable to incorporate.
Second, VideoDiff supports customization. Instead of generating a single result for users to accept as-is, it offers multiple options, enabling users to select and refine suggestions. By not delegating all edit decisions to AI, VideoDiff ensures creators remain in control of the final result. Finally, VideoDiff promotes efficiency by automating tedious edits, allowing creators to spend more time in the creative stages.}

\revised{To further support a deeper sense of agency and ownership, future iterations of VideoDiff can involve users in the planning and ideation stage~\cite{chen2019neural} and take user inputs before generating suggestions. For instance, users could specify requirements such as duration or style, provide text-based instructions, or upload reference videos to guide the editing process. They could also upload a partially edited video—for example, one with the first half completed—and request the system to finalize the remaining edits.
To support more flexible user control, future VideoDiff can allow users to directly accept or reject individual suggestions as in prior works ~\cite{laban2024beyond, han2020textlets}. To improve the transparency of AI suggestions, we can provide explanations for the generated suggestions~\cite{wan2024felt} or offer access to intermediate steps of LLM chaining used in generating them~\cite{wu2022ai}. This can enable users to better understand, utilize, and experiment with the tool~\cite{yeh2024ghostwriter}. 
Furthermore, future work can consider a personalized recommendation pipeline within VideoDiff, where the tool adapts to a user's unique style and preferences over time. This would not only speed up the editing process but also give creators a more tailored experience.} 
% not delegating everything to AI
% VideoDiff can mitigate these by supporting divergent thinking and letting users select and customize the edit suggestions flexibly. Video creators in our study highlighted that VideoDiff helped them be more creative by encouraging them to consider alternative ideas and allowing them to spend more time on the creative stage as they save time reviewing AI edits. 

As AI continues to accelerate the video creation process, there is also a risk that users may not thoroughly review or verify all the generated content. VideoDiff can help mitigate this by encouraging creators to do more sensemaking before making final selections. By offering clear visualizations of differences and allowing users to explore how each version fits their specific goals or audience, VideoDiff can assist creators in making more informed decisions.




\ipstart{\revised{Environmental and Societal Impact}}
\camready{
% To ensure that AI is developed and deployed responsibly, researchers investigated potential ethical issues including privacy~\cite{gupta2023chatgpt}, fairness~\cite{blodgett2020language}, toxicity~\cite{fortuna2018survey}, misinformation~\cite{weidinger2022taxonomy, huh2024long}, and computational cost~\cite{weidinger2022taxonomy}. In the long term, generative AI may also lead to potential challenges such as environmental impacts and job displacement.
We discuss the long-term impact of AI-powered creativity tools such as VideoDiff on the environment and society.
% We acknowledge that all tools leveraging GenAI, including ~\sysname{}, are not free from environmental and socioeconomic concerns. 
Like other AI tools, the language model that powers VideoDiff requires computational resources for training and inference which involves carbon emissions~\cite{weidinger2022taxonomy}. 
Yet, our study demonstrates that VideoDiff is highly valuable to users by streamlining complex editing tasks and enabling creative exploration. Also, VideoDiff may reduce the time users spend on traditional editing workflows, which often rely on resource-intensive video editing software and involv repeatedly rendering, storing, and transferring multiple large video files via cloud services. Instead, VideoDiff represents edit variations as structured text, eliminating the need to store multiple edited versions as full video files. We expect the financial and environmental costs of the models to change over time, and future work can explore how well smaller models can support tasks of suggesting video edits.}
% This time efficiency can result in lower overall energy consumption at the user level, offsetting some of the environmental impact associated with generative AI. 
% VideoDiff's task-specific generation and partial edits also help avoid unnecessary generations.

\revised{Generative AI tools like VideoDiff also raise concerns about job displacement in fields that rely heavily on manual video editing and production. However, VideoDiff does not automate the entire video editing process and retains key editing decisions with users who select and refine the suggestions. For novices, VideoDiff can lower the entry barrier to video editing and open doors for new roles. For experienced users, it reduces the time spent on tedious, repetitive tasks, allowing them to focus more on higher-order creative endeavors.
By fostering both accessibility and efficiency, VideoDiff can promote collaboration between humans and AI in a way that creates opportunities.}
% However, VideoDiff is designed to empower users across skill levels: it enables novices without much experience to create and edit videos effectively, lowering the barrier to entry for creative tasks. For experienced users, it reduces the time spent on tedious, repetitive tasks, allowing them to focus more on higher-order creative endeavors. By fostering both accessibility and efficiency, VideoDiff can promote collaboration between humans and AI in a way that aligns with the broader goals of responsible AI.

% \ipstart{Extending VideoDiff}
% % \ipstart{Parallel authoring}

% % \ipstart{Human-Human collaboration}
% tracking, 
% VideoDiff in Human-Human collaboration or version control setting

% % \ipstart{Beyond Video Editing}
% Video Diff in Video generation, 
% music editing --> requires novel consideration because cannot be conveyed with visual or text

% % \ipstart{Consumers curating or choosing video}
% pros: even consumers can edit the original content to their taste, 
% consumption scenarios choosing what I need from multiple videos
% (Prior work: comparing educational videos)


\ipstart{Limitations}
VideoDiff demonstrates the usefulness of video creation with alternatives in 3 representative video editing tasks: making rough cuts, inserting B-rolls, and adding text effects. 
Our study participants expressed excitement to use VideoDiff for a wider range of edits (\textit{e.g.,} motion graphics, and transitions). Future versions of VideoDiff could incorporate more diverse editing types.

While VideoDiff currently takes a single source video as input, study participants mentioned that VideoDiff will be useful for generating and comparing alternative curations of multiple clips when provided with a huge library of footage. To support this, we need to explore ways for users to efficiently skim through the coverage and sequence of different clips, while also accounting for the possibility that a single clip may appear multiple times in the final edit.

VideoDiff is powered by an LLM (GPT-4o) to generate edit recommendations and to follow users' commands to generate new variations. Our pipeline is prone to transcription errors and LLM hallucinations as few study participants have pointed out. GPT-4o also has a limit for input tokens (128k) and cannot process long videos with extensive length of transcripts.
Also, the current pipeline does not consider visual input when generating edit recommendations and thus cannot correctly follow edit requests related to the visual content. In the future, large multimodal models that take visual input (\textit{e.g.,} GPT-4V) can be integrated to improve the performance of the pipeline. 

\section{Conclusion}
We presented \sysname{}, a human-AI video co-creation tool designed to support efficient video editing with alternatives. Through a functional prototype and quantitative and qualitative user feedback, we showed that users want to work with variations when they are editing videos and would like flexible tools that offer easy ways to minimize watching and maximize creative potential. \camready{VideoDiff inspired users to explore alternative ideas while retaining key edit decisions to users who can select and refine AI suggestions.
All study participants wanted to integrate it into their workflow to create more satisfying videos.} Generative AI tools offer new promising capabilities that will only improve over time. It is up to the HCI community to create new experiences that harness that power into better and better creative tools.
\begin{acks}
We thank Columbia College for giving permission to use the video \href{https://www.youtube.com/watch?v=OkxRkNjPHIk}{\textit{Columbia University Campus Tour}}. Other videos are CC licensed (\href{https://www.youtube.com/watch?v=UApe--A1aQM}{\textit{What We Ate Today + Planting Our Garden }}). Part of this research is supported by Mina's Google Ph.D. fellowship. 
\end{acks}

\bibliographystyle{ACM-Reference-Format}
\bibliography{sample-base}
\newpage
\section{Appendix}


%\todo{to be finished!}
\subsection{Extended Background}
\label{app:deja_vu_comparison}
\paragraph{Déjà Vu Memorization in CLIP.}
The Déjà Vu memorization framework \citep{jayaraman2024} is the only existing other work that attempts to quantify memorization in vision-language models. It uses the text embedding of a training image caption to retrieve relevant images from a public dataset of images. It then measures the fraction of ground-truth objects from the original image that are present in the retrieved images. If the training pair is memorized, retrieved images have a higher overlap in ground truth objects, beyond the simple correlation.
While valuable, several aspects warrant further consideration for broader applicability of the framework. First, its focus on object-level memorization ignores non-object information like spatial relationships or visual patterns that can also influence memorization~\citep{feldman2020does,wang2024memorization}.
To perform object retrieval, the framework also relies on object detection and annotation tools, which may introduce variability based on the accuracy and robustness of these tools.
Additionally, the assumption that public datasets with similar distributions to the training data are readily available may not always hold, necessitating alternative approaches. 
Moreover, the framework does not analyze why certain images are memorized limiting detailed analysis. 
%The framework uses a guess-and-check approach to measure memorization, which may not fully leverage CLIP's multimodal nature or its contrastive learning objective. A more direct measurement could take advantage of these features to enhance evaluation.
Finally, while Déjà Vu must address the challenge of distinguishing between memorization and spurious correlations, \ours avoids this by directly assessing memorization on the output representations of the model.
One notable difference between the results of our approach and Déjà Vu's is that their findings show that their mitigation strategies can reduce memorization, but at the cost of decreased model utility. \ours, in contrast, does not observe trade-offs between memorization and performance.

    % \item \textbf{Lack of layer-wise insights.} Does not provide insights into how memorization occurs across different model layers.


% \paragraph{Shortcomings of Déjà Vu Memorization}
% \begin{enumerate}
%     \item \textbf{Limited to object-level memorization.} Focuses on object retrieval, neglecting other forms of memorization that pertain to non-object information like spatial relationships or visual patterns.
%     \item \textbf{Dependence on external models.} Relies on (inaccurate) object annotation tools, constraining the evaluation by the accuracy and robustness of these tools.
%     \item \textbf{Assumption of public dataset.} Assumes the availability of public datasets with similar distributions to the training data, which may not always be accessible. 
%     \item \textbf{No identification of memorized samples.} Does not specify which images are memorized or why.
%     \item \textbf{Lack of layer-wise insights.} Does not provide insights into how memorization occurs across different model layers.
%     \item \textbf{Indirect measurement of memorization.} Relies on a guess-and-check approach for measuring memorization that does not account for CLIP's multimodal nature or contrastive learning objective.
%     \item \textbf{Challenge with correlation vs. memorization.} Needs to distinguish between true memorization and spurious correlations, complicating the evaluation.
%     \item \textbf{Performance trade-offs.} Implements mitigation strategies that reduce memorization, but at the cost of decreased model utility.
% \end{enumerate}



\subsection{Extended Experimental Setup}
\label{app:setup}

\paragraph{General Setup.} 
All the experiments in the paper are done on a server with 4 A100 (80 GB) GPUs and a work station with one RTX 4090 GPU(24 GB).
We detail the setup for our model training, both CLIP and SSL (relying on DINO) in \Cref{tab:settings}.

\addtolength{\tabcolsep}{-2.5 pt}
\begin{table}[h]
\caption{\textbf{Experimental Setup.} We provide details on our setup for encoder training and evaluation.}        
          \label{tab:settings}
    \centering
    \scriptsize 
\begin{tabular}{cccccccc}
\toprule
                       & \multicolumn{3}{c}{Model Training}                                       &  & \multicolumn{3}{c}{Linear Probing}                      \\ \cmidrule{2-4} \cmidrule{6-8} 
                           &  CLIP  & DINO & Supervised ViT          &   & CLIP    &DINO     & Supervised ViT           \\ \midrule
Training Epoch          &  100     & 300          & 100 &   &   45    & 45          &45       \\
Warm-up Epoch          & 5  & 30        & 5& & 5          & 5          &5     \\
Batch Size             &  128  & 1024          &128  &   & 4096    & 4096 &4096      \\
Optimizer              &     Adam              & AdamW             & Adam      &  & LARS         & LARS & LARS        \\
Learning rate          &     1.2e-3               & 2e-3                 & 1e-3     &   & 1.6          & 1.6  & 1.6        \\
Learning rate Schedule &         Cos. Decay        & Cos. Decay & Cos. Decay & & Cos. Decay & Cos. Decay & Cos. Decay\\ \bottomrule 
\end{tabular}
\end{table}
\addtolength{\tabcolsep}{2.5 pt}


\paragraph{Experimental Setup for SSLMem.}
To experimentally evaluate memorization using the SSLMem framework~\citep{wang2024memorization}, the training dataset $S$ is split into four sets: \textit{shared set} ($S_S$) used for training both encoders $f$ and $g$; \textit{candidate set} ($S_C$) used only for training encoder $f$; \textit{independent set} ($S_I$) data used only for training encoder $g$; and an additional \textit{extra set} ($S_I$) from the test set not used for training either $f$ or $g$. For training encoders, encoder $f$ is trained on $S_S \cup S_C$, while encoder $g$ is trained on $S_S \cup S_I$. The alignment losses $\lalign(f, x)$ and $\lalign(g, x)$ are computed for both encoders, and the memorization score $m(x)$ for each data point is derived as the difference between these alignment losses, normalized to a range between $-1$ and $1$. A score of $0$ indicates no memorization, $+1$ indicates the strongest memorization by $f$, and $-1$ indicates the strongest memorization by $g$.

\paragraph{Normalization on \ours.}
For improved interpretability, we normalize our \ours scores to a range of $[-1,1]$. A memorization score of $0$ indicates no memorization, $+1$ indicates the strongest memorization on CLIP model f, and $-1$ indicates the strongest memorization on CLIP model g. We find the normalized CLIPMem score for a dataset using the following process:
For each image-text pair $(I,T)$, we first calculate the CLIPMem score as the difference in alignment scores between two CLIP models $f$ and $g$.
Once CLIPMem scores are computed for all data points, we normalize them by dividing each score by the range, which is the difference between the maximum and minimum scores in the dataset.
Finally, we report the normalized \ours score for a dataset as the average of these normalized values.
% Therefore, we calculate the differences in alignment loss per data sample $(I,T)$ over both CLIP encoder pairs $f$ and $g$.
% Afterwards, we normalize these differences by dividing them by the range (largest minus smallest difference),

% Finally, we report the \ours score as the average of the normalized scores over all data points.
% in $S_C$. 

\subsection{Additional Experiments}




% \subsubsection{Studying Memorization per Modality}
% \label{app:modality}
% \todo{@Adam, I find it very hard to justify the experiment below. Can you please have a look? The reviewers might just be confused.}
% We also perform an experiment where we remove memorized samples, retrain the model, and report the resulting linear probing accuracy.
% The more the linear probing accuracy is affected by the removal of a memorized sample, the more important the samples.
% Rather than taking the overall highest memorized samples and removing them, once the model is trained, we first cluster the training data in the embedding space into 50 clusters, once according to the text and once according to the image embeddings.
% Then, we remove from each cluster the highest memorized data point.
% This semantically corresponds to finding the highest memorized data points per-class in supervised learning (with the difference that we have no class information, and therefore, have to revert to a default number of clusters, in our case 50).
% Our results in \Cref{fig:modality} highlight that removing the highest memorized samples according to the text embeddings' grouping and then re-training boosts downstream performance more than removing based on image embeddings.\todo{@Adam, here, for sure something is missing, an explanation or so. But I have none. In the worst case, we need to remove this paragraph.} 
% Hence, the overall observed trend indicates that memorization in CLIP depends more on the text than on the image.
% \begin{figure}[h]
%     \centering
%     \includegraphics[width=0.55\linewidth]{image/sample_remove_text_image.pdf}
%     \caption{\textbf{Modality-based removal.} We train a CLIP models with standard image cropping on the COCO dataset. 
%     After training, we cluster training samples according to their similarity based on CLIP's image or text embeddings into 50 clusters, remove the highest memorized sample from each cluster, and retrain with the remaining points. The linear probing accuracy on ImageNet suggests that removing based on text embeddings' clustering memorization has higher influence on model performance.}
%     \label{fig:modality}
% \end{figure}

\subsubsection{Memorization vs. Generalization in CLIP}
\label{app:generalization}


\begin{figure}[t]
    \centering
    \includegraphics[width=0.55\linewidth]{image/sample_remove_random_clipmem_5_caption.pdf}
    \caption{\textbf{Removing memorized samples.} We show the effect on downstream performance in terms of ImageNet linear probing accuracy and \ours for a CLIP model trained on COCO using 5 text captions instead of 1, like done in \Cref{fig:generalization}. We observe the same trend, with the difference that the peak is at roughly 500 removed samples rather than 100. This is likely due to the increase in captions (by factor 5) that causes increase in mis-captioned samples.}
    \label{fig:image_text_sample5}
\end{figure}
\begin{figure}[t]
    \centering
    \includegraphics[width=0.55\linewidth]{image/sample_remove_poison_cifar_imagnet_sl.pdf}
    \caption{\textbf{Removing memorized samples in supervised learning.} We train a ViT-tiny on CIFAR10~\citep{krizhevsky2009learning} using supervised learning. We use our evaluation setup with $S_C$, $S_S$, $S_I$, and $S_E$ to approximate the memorization metric from~\citet{feldman2020does}. We use 5000 samples in $S_C$, but before training, we flip the labels of 200 samples. We calculate memorization over all samples in $S_C$ and test the linear probing accuracy with ImageNet resized to 32*32 on the representations output before the original classification layer.}
    \label{fig:sample_remove_poison_cifar_imagnet_slt}
\end{figure}
\textbf{Extending evaluation.}
In \Cref{fig:image_text_sample5}, we perform the same experiment as in \Cref{fig:generalization}, but on a CLIP model trained with 5 captions instead of 1.
We observe the same trend, with the difference that the peak is at roughly 500 removed samples rather than 100. This is likely due to the increase in captions (by factor 5) that causes increase in mis-captioned samples.

\textbf{Verifying the hypothesis on memorizing mis-captioned samples through supervised learning.}
We repeat the same experiment in the supervised learning setup to understand where the increase and then decrease in linear probing accuracy stems from. To test our hypothesis that it stems from "mis-captioned" samples, we "poison" our supervised model by flipping the labels of 200 data points before training. 
Then, we approximate the memorization metric from~\citet{feldman2020does} in our setup and remove highly memorized vs. random data points.
In the same vein as in \Cref{app:generalization}, we first observe an increase in linear probing accuracy when removing memorized samples (instead of random samples). The peak is at roughly 200 data points, \ie the number of deliberately mislabeled samples.
Until the cutoff point at roughly 3200 examples, linear probing accuracy is still higher when removing most memorized rather than random samples, which might suggest that there are other outliers or inherently mislabeled samples whose removal improves model performance. After the cutoff, we observe the behavior as observed in prior work~\citep{wang2024memorization, feldman2020does} that reducing memorization harms generalization more than reducing random data points from training.

%\textcolor{red}{TBD: refer to \Cref{tab:reduce_memorization}}

%\textcolor{red}{We need to describe \Cref{fig:image_text_sample5}}








\subsubsection{The Effect of Captions}
\label{app:captions}

\begin{table}[t]
    \centering
           \caption{\textbf{Using different/multiple captions during training.} 
           We evaluate \ours how memorization on different data subsets and linear probing accuracy on ImageNet differ when using 1 caption (baseline), 5 COCO captions, one chosen at random at every round (random), and 5 COCO captions, but all chosen equally often, \ie 20 out of 100 training epochs (balanced).
           We observe that increasing the number of captions reduces highest memorization. Yet, only when we balance the usage of caption, also model performance increases.}        
          \label{tab:multi_caption_training}
    \scriptsize
    \begin{tabular}{cccc}
    \toprule
&baseline & random & balanced \\
    \midrule
Avg. \ours (Top 10 samples)&0.792&0.788 &0.790 \\
Avg. \ours (Top 20\%) &0.552&0.531&0.540\\
%Avg. \ours ($S_C$) &0.438&0.409&0.423\\
Linear Probing Acc. &63.11\% $\pm$ 0.91\%&62.44\% $\pm$ 1.18\%& 64.88\% $\pm$ 0.83\%\\
      \bottomrule     
      \end{tabular}
\end{table}
\begin{table}[t]
          \caption{\textbf{The CLIPMem and linear probing accuracy of model trained with original coco captions and captions generated by GPT3.5.} For 'Single Caption', only one caption is used during training. For 'Five Caption', all five caption are used equally during training (every caption trained for 20 epoch out of 100). The linear probing accuracy is tested on ImageNet}        
          \label{tab:coco_cpt_train}
    \centering
    \scriptsize
    \begin{tabular}{cccccc}
    \toprule 
    & \multicolumn{2}{c}{COCO} && \multicolumn{2}{c}{GPT3.5}\\
    &Single Caption &Five Caption&&Single Caption &Five Caption\\
    \midrule
    CLIPMem &0.438&0.423& &0.430&0.411\\
    LP. Acc. &63.11\% $\pm$ 0.91\%&64.88\% $\pm$ 0.83\%&&63.09\% $\pm$ 1.12\%&64.47 $\pm$ 0.72\%\\
      \bottomrule     
      \end{tabular}
\end{table}
In \Cref{tab:multi_caption_training}, we show that using more captions during training reduces memorization and that by using each caption at the same frequency over the training epochs, we can additionally improve model performance.
Additionally, we show that captions generated by GPT3.5 have the same effect as the original COCO captions on memorization and linear probing accuracy in \Cref{tab:coco_cpt_train}.

\subsection{The Effect of Model Size}
\label{app:modelsize}
In \Cref{tab:model_size}, we present how the model size affects the memorization level of CLIP models. Both models are trained using the same dataset and settings. We observe that with more parameters (larger model size), encoders have higher memorization capacity. This aligns with findings from previous research~\citep{wang2024memorization, feldman2020does, meehan2023ssl}.



\begin{table}[t]
          \caption{\textbf{
           CLIPMem and linear probing accuracy of models with different sizes.} The models are trained using identical settings and the same subset of the COCO dataset. Linear probing accuracy is tested on the ImageNet dataset as the downstream task.}        
          % The CLIPMem and linear probing accuracy of model with different size} We train the models with same settings as well as same subset of COCO dataset. The linear probing accuracy is tested on ImageNet dataset as the downstream task.
          \label{tab:model_size}
    \centering
    \scriptsize
    \begin{tabular}{ccc}
    \toprule 
     Model& \ours&Lin. Prob. Acc. (ImageNet) \\
     \midrule
    ViT-base (Baseline in main paper) & 0.438 & 63.11\% $\pm$ 0.91\%\\
    ViT-large & 0.457 &  67.04\% $\pm$ 1.05\%\\
      \bottomrule     
      \end{tabular}
\end{table}

\subsection{Verification of infinite data regimes}
\label{app:onerun}
To evaluate CLIPMem over infinite data regimes (\ie using a single training run where no data point is repeated), we use a subset $D$ (containing 7050000 samples) of YFCC100M dataset~\citep{thomee2016yfcc100m} to train another pair of ViT-Base models for only 1 epoch. Following our definition of \ours, we further divide $D$ into $S_S$ with 6950000 samples, $S_C$ with 50000 samples, and $S_I$ with 50000 samples. The reason we use 7M (6950000+50000) samples to train either model $f$ or model $g$ is to make sure the newly trained model has the same number of training samples as the model trained with K-epoch runs (70000 samples/epoch * 100 epoch). The results in \Cref{tab:YFCC7M} show that the model trained with infinite data regimes has higher linear probing accuracy on ImageNet as a downstream task and lower memorization scores, as measured by \ours. This aligns with the fact that duplicated data points increase the memorization level and make the model over-fit, hence reducing the generalization~\citep{wang2024memorization, feldman2020does}. The results in \Cref{fig:most_YFCC} show that the most memorized samples according to \ours in the model trained with infinite data regimes are also samples with imprecise or incorrect captions. This aligns with our statements in \Cref{chpt:clip_ssl_sl}. 

\begin{figure}[t]
    \centering
    \includegraphics[width=0.95\linewidth]{image/most_memorized.pdf}
    \caption{\textbf{Top 10 memorized samples according to \ours in the model trained under infinite data regimes on YFCC100M.} The model is trained for one epoch, \ie seeing each training data point exactly once. Even in this setup, the most memorized samples are still the ones with imprecise or incorrect captions.}
    \label{fig:most_YFCC}
\end{figure}

\begin{table}[t]
\centering
\scriptsize
\caption{\textbf{Evaluation of CLIPMem under infinite data regimes, \ie seeing every data point only once during training vs training with 100 epochs.} We observe that both setups reach comparable downstream accuracy and memorization.}
\label{tab:YFCC7M}
\begin{tabular}{ccc}
      \toprule 
      Model & \ours & Lin. Prob. Acc. (ImageNet) \\
      \midrule
     ViT-Base (YFCC 7M, 1 epoch) &0.425&64.83\% $\pm$ 1.04\%\\
     ViT-Base (COCO 70K, 100 epochs)&0.438&63.11\% $\pm$ 0.91\%\\
      \bottomrule
\end{tabular}
\end{table}


\subsection{Evaluation on BLIP}
\label{app:BLIP}
To verify the effectiveness of \ours over other similar multi-modal models, we train a BLIP model on COCO dataset following the same settings as the baseline model in the main paper. We present the results for \ours on BLIP over all four data subsets in \Cref{fig:4_set_hist_blip}, which is in agreement with the results of the BLIP model in \Cref{fig:memorization_subsets}. We also present the results for UnitMem in \Cref{fig:unitmem}, which is also very similar to the results of CLIP models

\begin{figure}[t]
    \centering
    \includegraphics[width=0.55\linewidth]{image/4_set_hist_blip.pdf}
    \caption{\textbf{Memorization scores across data subsets on BLIP models} We train a BLIP model on COCO standard image cropping and no text augmentation. We present the results for \ours over all 4 data subsets, which is in agreement with the results of the CLIP model in \Cref{fig:memorization_subsets}}
    \label{fig:4_set_hist_blip}
\end{figure}

\subsection{Memorization distribution During training}
We present the distributions of neurons with highest UnitMem during training in \Cref{fig:clipmem_neuron_train}. These results highly consistently indicate that in the early stages of training, neuronal memory occurs mainly in the lower layer of the clip model, while in the middle and later stages of training, neuronal memory is more concentrated in the later layer of the model.
\begin{figure*}
    \centering
    
    \begin{subfigure}[b]{0.7\textwidth}  
        \centering 
        \includegraphics[width=\textwidth]{image/clipmem_neuron_train_1.pdf}
        \caption[]%
        {\textbf{Top 1\% neurons}}    
        \label{fig:clipmem_neuron_train_1}
    \end{subfigure}
    
    \begin{subfigure}[b]{0.7\textwidth}  
        \centering 
        \includegraphics[width=\textwidth]{image/clipmem_neuron_train_3.pdf}
        \caption[]%
        {\textbf{Top 3\% neurons}}    
        \label{fig:clipmem_neuron_train_3}
    \end{subfigure}
    
        \begin{subfigure}[b]{0.7\textwidth}  
        \centering 
        \includegraphics[width=\textwidth]{image/clipmem_neuron_train_5.pdf}
        \caption[]%
        {\textbf{Top 5\% neurons}}    
        \label{fig:clipmem_neuron_train_5}
        
    \end{subfigure}
    \caption{\textbf{Distribution of top 1\%, 3\%, and 5\% neurons with highest UnitMem during training.} We train a CLIP model on COCO standard image cropping and no text augmentation following the settings of baseline model in main paper. We record the neurons with top 1\%, 3\%, and 5\% of highest UnitMem during training (every epoch). } 
    \label{fig:clipmem_neuron_train}
\end{figure*}

\subsection{Human vs Machine Generated Captions}
\label{app:GPTcaptions}

For each image in the COCO dataset, we use GPT 3.5 (specifically, gpt-3.5-turbo) to generate 5 captions (from scratch). We use the following instruction in the OpenAI API: %\todo{@Wenhao, can you update it with your code of "prompt = f"Generate a detailed descri..."}
\begin{verbatim}
def generate_description_for_image(image_caption, clip_features):
    prompt = f"Here is an image with the caption: '{image_caption}'. "
    prompt += f"Based on this caption and the visual features
    represented by this embedding '{clip_features}', 
    please generate a new detailed description."
    response = openai.ChatCompletion.create(
        model="gpt-3.5-turbo",
        messages=[
            {"role": "system", "content": "You are a helpful 
            assistant that generates captions for images."},
            {"role": "user", "content": prompt}
        ]
    )
    return response['choices'][0]['message']['content']
\end{verbatim}
We present the obtained captions in \Cref{fig:image_text_sample}.
\begin{table}[t]
          \caption{\textbf{The machine generated captions provide similar performance to the original human-generated captions.} We report the \ours and linear probing accuracy of model trained with original COCO captions and captions generated by GPT 3.5. For the 'Single Caption', only a single caption is used during training. For 'Five Captions', all five captions are used equally during training (every caption trained for 20 epochs out of 100). The linear probing accuracy is tested on the ImageNet dataset as the downstream task.}        
          \label{tab:coco_gpt3_captions_train}
    \centering
    \scriptsize
    \begin{tabular}{cccccc}
    \toprule 
    & \multicolumn{2}{c}{COCO} && \multicolumn{2}{c}{GPT 3.5}\\
    &Single Caption &Five Captions&&Single Caption &Five Captions\\
    \midrule
    \ours &0.438&0.423& &0.430&0.411\\
    Linear Probing Accuracy (ImageNet) &63.11\% $\pm$ 0.91\%&64.88\% $\pm$ 0.83\%&&63.09\% $\pm$ 1.12\%&64.47 $\pm$ 0.72\%\\
      \bottomrule     
      \end{tabular}
\end{table}
\begin{figure}[t]
    \centering
\begin{subfigure}{0.45\columnwidth}
   \includegraphics[width=1.0\columnwidth]{image/coco_cosine_old}
    \caption{\textbf{COCO (Average: 0.798)}}
   \label{fig:coco_cosine}
\end{subfigure}
\begin{subfigure}{0.45\columnwidth}
     \includegraphics[width=1.0\columnwidth]{image/gpt_cosine_new.pdf}
   \caption{\textbf{GPT3.5 (Average: 0.851)}}
   \label{fig:GPT3.5_cosine}
\end{subfigure}
\caption{\textbf{Pairwise cosine similarity of 5 captions from COCO and generated by GPT3.5.}}
     \label{fig:cosine_single}
\end{figure}
In \Cref{fig:GPT3.5_cosine}, we analyze the pairwise cosine similarity in the original COCO and the GPT3.5 generated captions. We find that the GPT3.5 generated captions are slightly more uniform than the original COCO captions, reflecting in a higher pairwise cosine similarity.



\begin{table}[ht]
\scriptsize
\centering
   \begin{tabular}{ccc}
    \toprule
    Noise & \ours & Lin. Prob. Acc. (ImageNet) \\
    \midrule
      None & 0.438 & 63.11\% $\pm$ 0.91\%\\
      $\mathcal{N}(0.01)$  & 0.435 & 63.36\% $\pm$ 0.88\%\\
      $\mathcal{N}(0.05)$ & 0.428 & 64.02\% $\pm$ 1.12\%\\
      $\mathcal{N}(0.10)$ & 0.421 & 64.95\% $\pm$ 0.96\%\\
      \boldsymbol{$\mathcal{N}(0.15)$}  & \boldsymbol{$0.417$} &  \boldsymbol{$65.34\% \ \pm \  0.84\%$}\\
      $\mathcal{N}(0.20)$ & 0.422 & 64.83\% $\pm$ 0.92\%\\
      $\mathcal{N}(0.25)$ & 0.436 & 63.28\% $\pm$ 0.79\%\\
      $\mathcal{N}(0.30)$ & 0.447 & 61.50\% $\pm$ 0.86\%\\
      $\mathcal{N}(0.50)$ & 0.491 & 57.04\% $\pm$ 1.11\%\\
      $\mathcal{N}(0.75)$ & 0.501 & 52.28\% $\pm$ 0.98\%\\
      $\mathcal{N}(1.00)$ & 0.504 & 51.92\% $\pm$ 1.03\%\\
      \bottomrule
    \end{tabular}
        \caption{\textbf{Noising text embedding during training.} We present the impact of adding noise to the text embedding during training for the ViT-base trained on COCO.}
    \label{tab:noising}
\end{table}


\subsection{Examples for Memorized Samples}
\label{app:examples}
\begin{figure}[t]
    \centering
    \includegraphics[width=0.95\linewidth]{image/10_least_5_caption.pdf}
    \caption{\textbf{The 10 samples with lowest \ours in the CLIP model trained with all 5 captions.} We can see that these samples contain clear concepts and precise captions.}
    \label{fig:examples_memorized_5_caption_least}
\end{figure}

\begin{figure}[t]
    \centering
    \includegraphics[width=0.95\linewidth]{image/10_most_5_caption.pdf}
    \caption{\textbf{The 10 samples with highest \ours in the CLIP model trained with all 5 captions.} We can see that these samples contain atypical, difficult samples with imprecise or incorrect captions.}
    \label{fig:examples_memorized_5_caption_most}
\end{figure}

\begin{figure}[t]
    \centering
    \includegraphics[width=0.95\linewidth]{image/10_least_1_caption.pdf}
    \caption{\textbf{The 10 samples with lowest \ours in the CLIP model trained with 1 caption.} We can see that these samples contain clear concepts and precise captions.}
    \label{fig:examples_memorized_1_caption_least}
\end{figure}

\begin{figure}[t]
    \centering
    \includegraphics[width=0.95\linewidth]{image/10_most_1_caption.pdf}
    \caption{\textbf{The 10 samples with highest \ours in the CLIP model trained with 1 caption.} We can see that these samples contain atypical, difficult samples with imprecise or incorrect captions.}
    \label{fig:examples_memorized_1_caption_most}
\end{figure}

\begin{figure}[t]
    \centering
    \includegraphics[width=0.95\linewidth]{image/text_image_sample.pdf}
    \caption{\textbf{Samples of images generated by Stable Diffusion.} We present the generated images based on the COCO captions.}
    \label{fig:text_image_sample}
\end{figure}

\begin{figure}[t]
    \centering
    \includegraphics[width=0.95\linewidth]{image/image_text_sample.pdf}
    \caption{\textbf{Sample captions generated by GPT3.5.} We present the generated captions and the original image and captions from COCO.}
    \label{fig:image_text_sample}
\end{figure}


\begin{figure*}
    \centering
    \begin{subfigure}[b]{0.475\textwidth}
        \centering
        \includegraphics[width=\textwidth]{image/sslmem_sc_ss_old.pdf}
        \caption[]{{SSLMem Image Encoder (1 caption)}}
        \label{fig:sslmem_sc_ss}
    \end{subfigure}
    \hfill
    \begin{subfigure}[b]{0.475\textwidth}  
        \centering 
        \includegraphics[width=\textwidth]{image/sslmem_sc_ss_5.pdf}
        \caption[]%
        {{SSLMem Image Encoder (5 captions)}}    
        \label{fig:sslmem_sc_ss_5}
    \end{subfigure}
    \begin{subfigure}[b]{0.475\textwidth}  
        \centering 
        \includegraphics[width=\textwidth]{image/sslmem_sc_ss_text_1_old.pdf}
        \caption[]%
        {{SSLMem Text Encoder (1 caption)}}    
        \label{fig:sslmem_sc_ss_text_1}
    \end{subfigure}
    \hfill
    \begin{subfigure}[b]{0.475\textwidth}  
        \centering 
        \includegraphics[width=\textwidth]{image/sslmem_sc_ss_text_5.pdf}
        \caption[]%
        {{SSLMem Text Encoder (5 captions)}}    
        \label{fig:sslmem_sc_ss_text_5}
    \end{subfigure}
        \begin{subfigure}[b]{0.475\textwidth}  
        \centering 
        \includegraphics[width=\textwidth]{image/clipmem_sc_ss_old.pdf}
        \caption[]%
        {{CLIPMem}}    
        \label{fig:clipmem_sc_ss}
    \end{subfigure}
    \hfill
        \begin{subfigure}[b]{0.475\textwidth}  
        \centering 
        \includegraphics[width=\textwidth]{image/ssl_sslnaive_clip.pdf}
        \caption[]%
        {{SSLMem, Naive Sum of SSLMem and CLIPMem}}    
        \label{fig:ssl_sslnaive_clip}
        
    \end{subfigure}
    \caption{\textbf{Evaluation of SSLMem and \ours on a CLIP model trained on COCO.}
    Extended version of \Cref{fig:ssl_vs_clip} where we also include SSLMem calculated on encoders trained with 5 captions instead of 1.
    The trends in both cases are the same. SSLMem for the CLIP Models trained with the 5 captions is slightly higher since SSLMem uses the captions as augmentations for the calculation of the memorization. Overall, our \ours reports the strongest memorization signal for CLIP.
    %Results are obtained on ViT-Base, trained with COCO
    } 
    \label{fig:mean and std of nets}
\end{figure*}




% \paragraph{General Setup.} 
% All the experiments in the paper are done on a server with 4 A100 (80 GB) GPUs and a work station with one RTX 4090 GPU(24 GB).
% We detail the setup for our model training, both CLIP and SSL (relying on DINO) in \Cref{tab:settings}.

\end{document}
\endinput
%%
%% End of file `sample-sigconf-authordraft.tex'.
