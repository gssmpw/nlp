\section{Introduction}
% Background - Importance and challenge of video editing
As video has become a mainstream form of communication and storytelling, an increasing number of end users create and share videos. YouTube, which is considered the most popular video-focused social network, now has approximately 64 million creators worldwide~\cite{YTstats}.
Yet creating compelling videos is a complex and time consuming task. Creators need to find key moments~\cite{wang2024podreels} and remove irrelevant and repetitive content~\cite{fried2019text, huh2023avscript}. They also spend time making the video more engaging with visual effects, B-roll~\cite{huber2019b}, text~\cite{xie2023wakey}, and music~\cite{rubin2014generating}. 
% Background - AI can speed up video editing
Recent advances in video understanding and generative models have shown great potential for assisting video editing.  Prior research has shown that AI tools can speed up multiple stages of video authoring including script writing~\cite{mirowski2023co}, storyboarding~\cite{wang2024reelframer}, asembling clips into rough cuts~\cite{chi2013democut, truong2016quickcut}, identifying low-quality footage~\cite{huh2023avscript, fried2019text}, and adding B-roll~\cite{huber2019b}. 
Recent AI video products such as OpusClip~\cite{opus}, CapCut~\cite{capcut}, and Vizard~\cite{vizard} further streamline video editing by automatically making cuts, and adding transition effects and subtitles. 

One powerful new capability enabled by generative AI models is the quick generation of multiple variations. This allows creators to explore many alternative stories or B-roll placements simultaneously, potentially resulting in better final videos~\cite{dow2010parallel, suh2024luminate}. \revised{While most existing video editing tools are designed to handle only one video version at a time, recent AI tools such as OpusClip~\cite{opus} and Capcut~\cite{capcut} generate multiple variations of edited videos to offer diverse options to users. Despite the benefits of exploring alternatives in creative tasks, there are potential new burdens to users:} 1) comparing the variations~\cite{huh2023genassist, gero2024supporting} and 2) managing them over time~\cite{reza2023abscribe, suh2024luminate}.
While prior work explored sensemaking and comparison of multiple AI generations in text~\cite{gero2024supporting, suh2023sensecape, reza2023abscribe}, images~\cite{almeda2024prompting, huh2023genassist}, and designs~\cite{swearngin2020scout, matejka2018dream}, comparing multiple videos presents unique challenges due to the temporal nature of video. In this work, we explore this emerging approach to video editing centered on working with multiple variations. 
% Two key challenges in working with multiple variations in parallel are 1) comparing them and 2) managing the variations over time. \mira{can we cite prior work on this?}
%These tools generate multiple versions of edited videos, letting creators explore alternatives before quickly fixating on a single direction~\cite{dow2010parallel, suh2024luminate}.
% Problem Statement
%However, creators need to manually compare and verify multiple variations to find a video that best suits their needs.
% With AI's ability to automate and accelerate the creative process, AI creativity tools 
% AI's ability to automate and accelerate the creative process enables the generation of multiple results in parallel.
% For instance, Text-to-Image models such as DALL-E3~\cite{dalle3} and Midjourney~\cite{midjourney} can generate multiple images from a single prompt. 


% Sensemaking and comparison of videos is especially challenging with multi layers of edits applied and sequential characteristic of video (cannot glance)
% With AI accelerating the video creation process, we anticipate that human-AI collaborative video editing will more frequently involve reviewing and comparing multiple video results.
% AI is capable of generating multiple results (e.g., text, images) to a single prompt or input, yet it requires users to manually inspect the results to compare and verify the generated results.
% As AI generates multiple creations, the ~\textit{comparison} task among candidates is commonly practiced for authoring slides [cite], graphic design [cite], storyboards [cite], and audio stories [cite]. 

% Formative Study
To understand the opportunities and challenges of authoring videos with multiple alternatives, we conducted a formative study where 8 professional video creators were tasked with comparing multiple edited videos of the same source content. \camready{
Creators in our study mentioned that comparison is a common practice throughout their current video editing process as they consider alternative narratives, visual assets, or lengths of videos.
They also highlighted that having alternatives helps them reflect on their preferences and further plan editing directions, but it was time-consuming to manually create multiple versions. As AI speeds up the video creation process, we
envision that future video editing tools will more commonly provide users with multiple variations.}


However, reviewing alternatives can require a lot of time and effort because creators must watch all the videos to compare them. Unlike static images or text, which can be quickly scanned for differences, videos unfold over time, requiring creators to watch, pause, and replay to detect variations in stories, transitions, and visual effects. Based on these findings, we derive 6 design goals for video editing tools centered on working with alternatives. 
A tool to support creators in exploring video alternatives should minimize redundant watching (D1) and enable quick skimming of differences (D2) across different stages of the editing process (D3) in the most appropriate modality (D4). Such a tool should also help creators easily organize and customize variations to support them in finding a version with fewer errors (D5) and best suits their needs (D6). 
% \mira{We don't explicitly mention d5 in the text. we might want to add somethign about verification but I wasn't sure how to do it in a smooth way.}
% \mira{why not surface the design guidelines here in the intro? Also I think this paragraph would be stronger if we add some structure around it - for example 1) alternatives help with the editing process 2) it's time consuming to compare them 3) it's hard to tell how the video are different 4)...}
% Also, when none of the videos fully meet their expectations, it is difficult to further customize the edits. 
% \mira{why is it difficult to further customize?} 


% comparing videos required a lot of time and effort to watch all videos to understand the differences, especially reviewing multiple quality aspects at once (\textit{e.g.,} storyline, visual effects). 
% Unlike static images or text, which can be quickly scanned for differences, videos unfold over time, requiring creators to watch, pause, and replay segments to detect variations in pacing, transitions, and visual effects. This makes comparison more time-consuming and cognitively demanding, as changes are harder to perceive without reviewing entire sections of the video.


% Creators in our study mentioned the benefits of having multiple suggestions of video: 1) In comparing videos, creators can figure out their own preferences and plan further editing directions. 2) By creating variations in parallel, creators can repurpose their content tailored to different platforms. However, creators who participated in the video comparison task mentioned several challenges in comparing AI-edited videos. 


% System
% We introduce ~\textit{VideoDiff}, an AI-powered video editing tool designed to support video authoring with alternatives. 
\revised{We introduce ~\textit{VideoDiff}, an AI-powered video editing tool designed to enhance the creative potential of working with alternatives while minimizing the burden of comparison.} VideoDiff generates diverse AI recommendations for each video editing task such as making rough cuts, inserting B-rolls, and adding text effects (Figure~\ref{fig:teaser}.1). 
For each task, creators can quickly skim the differences between variations using VideoDiff which aligns the variations and highlights the differences using timelines, transcripts, and video previews (Figure~\ref{fig:teaser}.2). 
Additionally, creators can sort variations to narrow down the options and further customize them by refining and regenerating AI suggestions (Figure~\ref{fig:teaser}.3). 
%
In this work, we focus specifically on working with narrative videos such as lifestyle vlogs, podcasts, how-to videos, and travel videos. 
% These types of videos include a lot of dialogue and are typically unscripted or loosely scripted. 
Their descriptive dialogue makes them well suited to the current strengths of text-based large language models, which are good at extractive and abstractive summarization.  Additionally, the source content for these types of videos is either one long video (\textit{e.g.,} presentation or an event) or multiple sources that can be sequenced in time because they capture chronological activities, such as a how-to demonstration or travel videos. This simplifies our visualization design to show how different versions relate to one another. 
% In future work, we plan to explore the broader space of videos.

% 1
% VideoDiff reduces the complexity of comparing fully edited videos through multi-stage comparison. In each stage, users can focus on a single edit goal (\textit{e.g.,} inserting B-rolls) to generate, compare, and iterate on variations (Figure 1.A). 
% Breakdown comparison/authoring stages
% DG 1
% targeted/selective diff (specific parts)
% 2 
% 3
% With VideoDiff, users can also refine, regenerate, and recombine AI-edited videos. VideoDiff saves users' edits as a new variation and generate a corresponding glanceable diffs to support comparison with previous iterations (Figure 1.C).
% - let users to export and support easy integration to their current workflow

% 4
% Organization (sort/search/filter)
% easy attack points for reviews? (discussion + mock up)

% 5
% encourage divergent convergent thinking
% - auto-complete branches


% Evaluation
We evaluated VideoDiff in a within-subjects study with 12 video creators who compared VideoDiff with a baseline interface designed to encompass features of existing AI video editing tools (\textit{e.g.,} CapCut, OpusClip). Participants rated VideoDiff as more useful for quickly understanding the differences of multiple videos which helped them create and consider more diverse variations. Participants also expressed higher satisfaction with the final videos created with VideoDiff.
% Contribution
The contributions of our paper are as follows:
\begin{itemize}
    \item We identify the challenges of comparing multiple variations in video editing and derive design goals for systems to support video creation with alternatives.
    % Formative study revealing the opportunities and challenges of video authoring via comparison.
    \item We propose \textit{VideoDiff}, an AI-powered video editing tool that supports video creation with alternatives by enabling efficient generation, comparison, organization, and customization of variations.
    % \item User study demonstrating how video creators use VideoDiff to co-create videos with AI
    % \mina{did we learn something new?, will the findings change earlier design decisions?}
    \item Our user study findings showcase the advantages and challenges of using VideoDiff.
\end{itemize}
% \mira{This is more of a stylistic thing but I don't typically consider a validation user study a contribution. The findings from the formative study and system seem like the main contributions to me.}


% notes from 9/10
% How to motivate alternatives
% - other tools are doing
% - creative tasks are subjective 
%     - no good answer
% - divergent thinking - helps you not fixate
% - AI not perfect so need alternatives (Technical limitations)
% in the intro write more clear on what edits we're covering what diffs we're covering 