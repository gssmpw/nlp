\section{Demonstration}
Our solution integrates backend processing and a dynamic front end to create an intuitive simulation framework. It consists of two key components: (1) a Flask-based backend server for handling simulation logic and managing agent activity data, and (2) a front-end web application built with HTML, CSS, JavaScript, and the Leaflet.js library for visualization. 
The backend server provides RESTful API endpoints, such as the \texttt{/start\_day} endpoint, which processes input parameters like the number of agents and start time to generate a JSON response containing detailed agent activities and coordinates. 
Its modular architecture enables easy integration with new data sources and generative models, ensuring scalability and adaptability.
% Interactive frontend
The front-end application offers a user-friendly interface for real-time interaction. Users can adjust parameters via sliders and input fields, visualize agent movements on a map, and access live activity logs with toggle-able controls. This combination ensures usability and interactivity, making it ideal for engaging demonstrations. The front-end HTML file is available in our GitHub repository, allowing users to explore and interact with the interface independently.

To ensure accessibility and usability during demonstrations, the framework is deployed on a web server using Flask. This setup allows conference attendees to interact with the simulation directly on their devices or via a dedicated terminal provided at the venue.

For the purpose of demonstration, we leverage the public check-ins dataset of Tokyo as the underlying POIs dataset for simulation (note that the proposed simulation platform supports other cities as well). Based on this, the personas for the agents are created
using population statistics collected from local government
website\footnote{\url{https://www.stat.go.jp/english/data/index.html}}.
The demonstration begins with attendees configuring simulation parameters, such as the number of agents and start time. The interactive front end visualizes real-time agent movements on a map, allowing users to toggle agent visibility and review activity logs. As illustrated in Figure~\ref{fig:demo}, the interface provides an intuitive layout with a map view for agent trajectories, sliders and input fields for simulation parameters, and a "Live Movement Updates" panel displaying agent-specific details such as gender, occupation, activity, and location. This setup highlights the system's capabilities for scalability, modularity, and ease of interaction, making it an effective tool for showcasing dynamic simulations in academic and professional environments.

\begin{figure}[h]
    \centering
    \includegraphics[width=\linewidth]{img/frontend.pdf}
    \caption{Real-Time Simulation of Daily Activities in Tokyo}
    \Description{This is a figure showing the frontend demonstration for the simulation of one day routines of 10 agents in Tokyo. Their movement for daily activities are represented on the map, with textual details of the activities.}
    \label{fig:demo}
\end{figure}

