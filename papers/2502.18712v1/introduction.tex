\section{Introduction}
A comprehensive understanding of human mobility patterns is essential for the development of sustainable communities. It plays a pivotal role in the control of infectious diseases, the optimization of energy consumption, and the formulation of urban planning strategies. However, the direct utilization of real-world mobility data is often constrained by privacy concerns, since it encompasses sensitive information regarding individual movement patterns and personal behaviors. Furthermore, the collection of comprehensive mobility data is expensive. In this case, simulating mobility data has emerged as a promising solution. 

Previous approaches to modeling human mobility can be broadly categorized into mechanistic and deep learning models. Mechanistic models leverage stochastic processes to characterize human mobility behaviors, while these models tend to be oversimplified and focus only on certain aspects of mobility behavior, failing to generate nuanced and comprehensive mobility patterns. By contrast, deep learning models are capable of learning and generating human mobility patterns from training data, but often fail to capture the underlying mechanisms driving mobility behavior. Additionally, they suffer from low sample efficiency when learning behavior distributions and tend to generate data that lacks semantic-aware intentions, limiting their interpretability and realism.

The advances in Large Language Models (LLMs) have shown promising capabilities in understanding and generating coherent intentions through techniques like chain-of-thought prompting and role-playing abilities. It has demonstrated remarkable capabilities in understanding humans and society, with applications in political science, economics, and other social science studies. In recent years, several models have been proposed to simulate human mobility data using LLMs, such as LLMob \cite{wang2024large} and MobiGeaR \cite{shao2024beyond}. LLMob relies solely on LLMs, utilizing historical data and personal profiles, while MobiGeaR integrates LLMs with a gravity model to simulate mobility data.

Our framework, named TrajLLM, combines the ability of LLMs to generate coherent and interpretable human activities with an innovative physical model for location mapping, minimizing reliance on real-world historical check-in data by leveraging static Point-of-Interest (POI) data. Additionally, we integrate a memory module to ensure agents exhibit consistent and realistic activity patterns over time, reflecting habitual behaviors. 
In comparison to existing solutions, this novel approach not only enables the activity-driven generation of daily mobility trajectories, but also further enhances agent alignment with human behaviors through the consolidation of agents' historical visitation data. Furthermore, the integration of fundamental physical models ensures TrajLLM's simulation efficiency, providing a balance between accuracy and performance.

% The model operates in four stages: (1) persona and intention generation, where LLMs create daily activity lists tailored to individual characteristics; (2) routine activity generation, ensuring temporal consistency by integrating generated intentions with historical patterns; (3) destination selection, where spatial and frequency models collaboratively determine activity locations; and (4) memory consolidation, which updates and refines the agent's behavioral patterns based on simulated activities.
