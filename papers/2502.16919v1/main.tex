                            % This must be in the first 5 lines to tell arXiv to use pdfLaTeX, which is strongly recommended.
\pdfoutput=1
\documentclass[11pt]{article}

% Change "review" to "final" to generate the final (sometimes called camera-ready) version.
% Change to "preprint" to generate a non-anonymous version with page numbers.
\usepackage[final]{acl}

% Standard package includes
\usepackage{times}
\usepackage{latexsym}
\usepackage{multirow}
\usepackage{amsmath}
% For proper rendering and hyphenation of words containing Latin characters (including in bib files)
\usepackage[T1]{fontenc}
% For Vietnamese characters
% \usepackage[T5]{fontenc}
% See https://www.latex-project.org/help/documentation/encguide.pdf for other character sets

% This assumes your files are encoded as UTF8
\usepackage[utf8]{inputenc}

% This is not strictly necessary, and may be commented out,
% but it will improve the layout of the manuscript,
% and will typically save some space.
\usepackage{microtype}

% This is also not strictly necessary, and may be commented out.
% However, it will improve the aesthetics of text in
% the typewriter font.
\usepackage{inconsolata}

%Including images in your LaTeX document requires adding
%additional package(s)
\usepackage{graphicx}

% If the title and author information does not fit in the area allocated, uncomment the following
%
%\setlength\titlebox{<dim>}
%
% and set <dim> to something 5cm or larger.

\title{Dependency Parsing with the Structuralized Prompt Template}
% \title{Dependency Parsing with the Structuralized Prompt Template Revision\_link: \ref{sec:revision}}

% Author information can be set in various styles:
% For several authors from the same institution:
% \author{Author 1 \and ... \and Author n \\
%         Address line \\ ... \\ Address line}
% if the names do not fit well on one line use
%         Author 1 \\ {\bf Author 2} \\ ... \\ {\bf Author n} \\
% For authors from different institutions:
% \author{Author 1 \\ Address line \\  ... \\ Address line
%         \And  ... \And
%         Author n \\ Address line \\ ... \\ Address line}
% To start a separate ``row'' of authors use \AND, as in
% \author{Author 1 \\ Address line \\  ... \\ Address line
%         \AND
%         Author 2 \\ Address line \\ ... \\ Address line \And
%         Author 3 \\ Address line \\ ... \\ Address line}

\author{Keunha Kim \\
  SungKyunKwan University \\
  Republic of Korea \\
  \texttt{keunhakim98@gmail.com} \\\And
  Youngjoong Ko\thanks{Corresponding author} \\
  SungKyunKwan University \\
  Republic of Korea \\
  \texttt{yjko@skku.edu} \\}

%\author{
%  \textbf{First Author\textsuperscript{1}},
%  \textbf{Second Author\textsuperscript{1,2}},
%  \textbf{Third T. Author\textsuperscript{1}},
%  \textbf{Fourth Author\textsuperscript{1}},
%\\
%  \textbf{Fifth Author\textsuperscript{1,2}},
%  \textbf{Sixth Author\textsuperscript{1}},
%  \textbf{Seventh Author\textsuperscript{1}},
%  \textbf{Eighth Author \textsuperscript{1,2,3,4}},
%\\
%  \textbf{Ninth Author\textsuperscript{1}},
%  \textbf{Tenth Author\textsuperscript{1}},
%  \textbf{Eleventh E. Author\textsuperscript{1,2,3,4,5}},
%  \textbf{Twelfth Author\textsuperscript{1}},
%\\
%  \textbf{Thirteenth Author\textsuperscript{3}},
%  \textbf{Fourteenth F. Author\textsuperscript{2,4}},
%  \textbf{Fifteenth Author\textsuperscript{1}},
%  \textbf{Sixteenth Author\textsuperscript{1}},
%\\
%  \textbf{Seventeenth S. Author\textsuperscript{4,5}},
%  \textbf{Eighteenth Author\textsuperscript{3,4}},
%  \textbf{Nineteenth N. Author\textsuperscript{2,5}},
%  \textbf{Twentieth Author\textsuperscript{1}}
%\\
%\\
%  \textsuperscript{1}Affiliation 1,
%  \textsuperscript{2}Affiliation 2,
%  \textsuperscript{3}Affiliation 3,
%  \textsuperscript{4}Affiliation 4,
%  \textsuperscript{5}Affiliation 5
%\\
%  \small{
%    \textbf{Correspondence:} \href{mailto:email@domain}{email@domain}
%  }
%}

\begin{document}
\maketitle
\begin{abstract}
Dependency parsing is a fundamental task in natural language processing (NLP), aiming to identify syntactic dependencies and construct a syntactic tree for a given sentence. Traditional dependency parsing models typically construct embeddings and utilize additional layers for prediction. We propose a novel dependency parsing method that relies solely on an encoder model with a text-to-text training approach. To facilitate this, we introduce a structured prompt template that effectively captures the structural information of dependency trees. Our experimental results demonstrate that the proposed method achieves outstanding performance compared to traditional models, despite relying solely on a pre-trained model. Furthermore, this method is highly adaptable to various pre-trained models across different target languages and training environments, allowing easy integration of task-specific features.
\end{abstract}
\section{Introduction}

Dependency parsing is a fundamental task in natural language processing (NLP) that analyzes syntactic relationships between words in a sentence. Figure \ref{fig:traditional} illustrates the traditional dependency parsing pipeline. Traditionally, dependency parsing is performed in two steps: (1) creating word-level embeddings, and (2) identifying the head word of each word along with its dependency relation using the generated embeddings.

\begin{figure}[h]
 \includegraphics[width=\columnwidth]{image/pre_model.png}
 \caption{A figure shows that pipeline of traditional dependency parsing}
 \label{fig:traditional}
\vskip -.24in
\end{figure}


Previously, dependency parsing primarily relied on simple pre-processed contextual vectors to initialize embeddings \cite{li-etal-2018-seq2seq,strzyz-etal-2019-viable,vacareanu-etal-2020-parsing}. With the advent of powerful pre-trained language models such as BERT \cite{bert}, recent dependency parsing approaches leverage these models to initialize word embeddings, achieving superior performance compared to earlier methods \cite{amini-etal-2023-hexatagging}. In the second step of dependency parsing, previous studies have shown that graph-based methods, such as biaffine \cite{dozat-biaffine}, yield good performance in identifying relations. Consequently, this approach was extended to learn the subtree information of the dependency tree \cite{yang-tu-2022-headed}. Since the structural characteristics of dependency trees increased training complexity and difficulties, some studies use the sequence tagging method for parsing \cite{li-etal-2018-seq2seq,amini-cotterell-2022-parsing}. These approaches add simple layers after embedding construction and label the words into structural information sequentially. In particular, the hexatagging method achieved state-of-the-art performance by generating structural information with a finite set of tags through decoding \cite{amini-etal-2023-hexatagging}. 
In addition, \citet{lin2022dependency} demonstrated that encoder-decoder models effectively generate relation unit texts from input texts, highlighting that dependency parsing can be performed solely using pre-trained language models. This study is particularly significant as it explores dependency parsing by transforming modified text input into structured dependency output. However, the notable limitation of this approach lies in the increased computational time required for parsing, as the number of tokens grows due to the generation-based method.
%&& 여기 여백 줄여주

In this paper, we propose a novel method to perform dependency parsing solely on pre-trained encoder models that are constructed by prompt engineering using additional tokens as soft prompts. We hypothesize that prompt engineering can effectively convert the text-to-structure task in dependency parsing to the text-to-text task by pre-trained language models, just like the sequence labeling method. Hence, the output text sequence of the proposed method has to reflect the tree structure of dependency parsing well. To achieve this, we design several soft prompts so that our model can identify the structural information of the tree structure, and then apply the \textbf{S}tructuralized \textbf{P}rompt \textbf{T}emplate (SPT) for each processing unit of dependency parsing using the developed soft prompt. We believe that prompt learning with the structuralized prompt template enables effective and efficient dependency parsing only on the pre-trained language models. Eventually, by learning through the structuralized prompt template, the \textbf{S}tructuralized \textbf{P}rompt \textbf{T}emplate based \textbf{D}ependency \textbf{P}arsing (SPT-DP) method achieves the high performance and simplified training by reducing the gap between pre-training and fine-tuning because it is based on only the pre-trained language models for the text-to-text task. 
As a result, the performance of the proposed method surpasses most of existing methods: 96.95 (UAS) and 95.89 (LAS) on English Penn Treebank (PTB; \citet{marcus-etal-1993-building}). On the 2.2 version of Universal Dependencies (UD 2.2; \citet{UD2.2}), it obtains the state-of-art performance in 2 languages out of 12 languages when using the cross-lingual Roberta model \cite{roberta}. Since our method utilizes only a single encoder model, our approach is faster than existing methods. In particular, it achieves a 40x speed improvement compared to DPSG, which employs a similar text-to-text approach. Furthermore, our method achieves a performance comparable to that of the SOTA model with a complicated and heavy architecture in the Korean Sejong dataset.


\section{Preliminaries}

\subsection{Dependency Parsing}
Dependency parsing is the process of identifying dependency relationships among words in a sentence. Given a sentence $S = (w_1, w_2, \ldots, w_n)$, the task is to derive the dependency relations, $R_S = (r_1, r_2, \ldots, r_n)$, for each word. Each relation, $r_i = (H_i, L_{(w_i,w_{H_i})})$, consists of two components: (1) the index of the head word $H_i$ and (2) the dependency relation label $ L_{(w_i,w_{H_i})}$, between $w_i$ and $w_{H_i}$, where $ L_{(w_i,w_{H_i})} \in \mathcal{L}$ and $\mathcal{L}$ is the set of predefined dependency relation labels. This study focuses on syntactic dependency parsing that analyzes grammatical dependency relations within sentences. Depending on the definition of the syntactic dependency tree, each word has exactly one parent node, ensuring a hierarchical structure. Therefore, the ultimate goal of dependency parsing is to analyze the sentence $S$ into the form $S_{dep} = \{(w_i, r_i) \mid r_i = (H_i, L_{(w_i,w_{H_i})}), i \in \{1, 2, \ldots, n\}\}$, where each word $w_i$ is paired with its corresponding dependency relation $r_i$. \footnote{ Table \ref{tab:notation} for a list of notations along with explanations.}

\section{Example}
\label{app:example}

Table~\ref{tab:example} shows a counseling session between a client agent and CAMI, with the client's profile shown in Table~\ref{tab:example profile}. During the session, CAMI explores Health and Relationships topics in the first 6 turns.  It subsequently performs Step Into operation from Relationships to Parenting topics in Turn 7, and from Parenting to Paternal Bond topics in Turn 9, before a Switch operation to Habituation topic in Turn 11.  The full exploration path is visualized in Figure~\ref{fig:exploration path}.

\begin{table*}[ht]
\centering
\begin{tabularx}{\textwidth}{c X}
\toprule
        {\bf Turn ID} & {\bf Utterances}                     \\ \midrule
1       & Counselor: Hello. How are you?                                                                                                                                                                                                                                                                                                                     \\ \hline
2       & Client: I am good. What about you?                                                                                                                                                                                                                                                                                                                 \\ \hline
3       & Counselor: I'm doing well, thanks for asking! Can you share how you've been feeling about your health lately, especially in relation to smoking?                                                                                                                                                                                                   \\ \hline
4       & Client: I've been pretty busy with work and juggling family stuff. Sometimes I just think about how nice it is to take a break with my coworkers outside, chatting and during those moments, life feels a bit more relaxed. You know how it is—it’s nice to unwind.                                                                                \\ \hline
5       & Counselor: It sounds like smoking is a comforting routine for you amidst your busy life with work and your daughter. I wonder how it might impact the time you spend with your family.                                                                                                                                                             \\ \hline
6       & Client: I get what you mean, but I really don't see my smoking as an issue. It helps me manage stress, and I don’t think it affects my time with my daughter. We still have good moments together, and I wouldn't want to change that.                                                                                                             \\ \hline
7       & Counselor: It sounds like you find comfort in smoking to manage stress, and that feels crucial to your parenting now. I wonder, though, how you might feel if your daughter started to see smoking as a way to unwind, too. What impact would that have on your relationship with her?                                                             \\ \hline
8       & Client: That’s a tough thought. I remember starting young and how hard it was to quit later. I wouldn’t want that for her. It makes me reflect on what I’m modeling.                                                                                                   \\ \hline
9       & Counselor: It's great that you're reflecting on your modeling for your daughter. Perhaps seeing smoking as something that may weaken your bond could help reshape how you approach stress—considering healthier ways to connect can actually strengthen your relationship and set a positive example for her.                                      \\ \hline
10      & Client: I appreciate that perspective. I do want to set a strong example for her. My experience with my own family and their struggles makes me cautious. I see the patterns and don’t want her to feel that smoking is a go-to solution for stress, like I did.                                                                                   \\ \hline
11      & Counselor: It sounds like your family’s struggles have really shaped your perspective on smoking. You want to protect your daughter from those same patterns, knowing that smoking can not only impact her stress management but also her overall learning and adaptation to challenges. That’s a powerful motivation.                             \\ \hline
12      & Client: Exactly. I want her to learn healthier coping strategies instead of following in my footsteps. I realize that I need to make a change for both our sakes. It’s not just about me; it’s about creating an environment where she can thrive without those habits. It's time to find better ways to manage stress that we can share together.   
\\ \bottomrule
\end{tabularx}
\caption{Counseling Session between CAMI and a Simulated Client.}
\label{tab:example}
\end{table*}



\begin{table*}[ht]
\centering
\begin{tabularx}{\textwidth}{l X}
\toprule
{\bf Profile Item}        & {\bf Description}                     \\ \midrule
Behavioral Problem  & Smoking \\ \hline
State of Mind       & Precontemplation \\ \hline
Personas            & You smoked when you were young.", "You have a 12-year-old daughter. \newline The friends of the your daughter have tried smoking. \newline When you are at work, you all go out and smoke and chit chat and talk about things. \newline You like to have a cigarette after eating. \\ \hline
Motivation          & You are motivated because of the risk of smoking in habituation for your daughter, as you don't want your daughter to think smoking is good thing.    \\ \hline
Beliefs             & You feel smoking enjoyable and relaxing. \newline You would be left out if you would hate to not smoke. \newline You use smoking as a way to manage weight and control eating habits. \newline You have been smoking for a long time and feels it's almost impossible to quit. \\ \hline
Motivation Topics          & Interpersonal Relationships, Parenting, Habituation
\\ \bottomrule
\end{tabularx}
\caption{Client's Profile in the Example.}
\label{tab:example profile}
\end{table*}


\begin{figure*}
    \centering
    \includegraphics[width=\linewidth]{figs/example_explore_path.pdf}
    \caption{Topic Exploration Path by the Counselor in the Example.}
    \label{fig:exploration path}
\end{figure*}




\subsection{Prompt based tuning}
Initially, the concept of a prompt emerged as a method for training models, particularly to bridge the gap between the pre-training and fine-tuning stages. PET (Pattern-Exploiting Training) proposed a method to reduce the gap between pre-training and fine-tuning by using cloze-style patterns instead of the traditional [CLS] token-based classification approach in BERT \cite{schick-schutze-2021-exploiting}. This allows the pre-trained language model to naturally adapt to classification tasks. For example, in sentiment analysis, a sentence like "This movie is really [MASK]." is used, and the predicted probability of the [MASK] token is used to determine its sentiment.

Subsequently, prompts have been explored in combination with methods like PEFT (Parameter-Efficient Fine-Tuning) to improve model efficiency and task performance \cite{li-liang-2021-prefix, lester-etal-2021-power}. In generative models like GPT-3 \cite{NEURIPS2020_1457c0d6}, text-based prompts are used to guide and shape the outputs into task-specific outputs.

In this study, prompts are used to provide additional linguistic hints, such as grammatical structures or contextual information, necessary for dependency parsing. Specifically, prompts are added to the encoder model through preprocessing. This is done by concatenating prompt phrases, such as task-specific instructions, to the original input sentence. Prompts are applied to each word in the input sentence as shown in the following formulation:
\begin{equation}
P(S) = (T(w_1), T(w_2), \ldots, T(w_n))
\end{equation}

where $P(S)$ represents the transformed sentence and $T(w_i)$ is the prompt template applied to the $i$-th word $w_i$.

\subsection{Pre-trained Language Model}

Transformer-based pre-trained models are built upon the following architectures:

\begin{itemize}  
\item \textbf{Encoder-based Models}: These models encode input text to perform tasks like masked token prediction or classification by adding a linear layer. They are also widely used in \textit{Dense Passage Retrieval (DPR)} \cite{karpukhin-etal-2020-dense} to extract representative embeddings for documents. $X \rightarrow X’$
  \item \textbf{Decoder-based Models}: These models take an initial token or prompt and generate tokens sequentially through autoregressive decoding. $X \rightarrow Y$
\end{itemize}
The traditional approach to dependency parsing involves extracting token-level embeddings from encoder-based pre-trained models, integrating them at the word level, and constructing word embeddings. These embeddings are then analyzed through various methods to perform dependency parsing. In this process, the performance of dependency parsing heavily depends on how effectively the encoder provides information relevant to syntactic structures. 

\begin{figure*}[h]
 \includegraphics[width=1\linewidth]{image/main.png}
 \caption {Overview of the Structuralized Prompt Template based Dependency Parsing (SPT-DP) method.}
  \label{fig:main}
\end{figure*}

\section{Structuralized Prompt Template}
%
As aforementioned, traditional methods approach dependency parsing by utilizing predefined embeddings processed through specific modules. However, Dependency Parsing via Sequence Generation (DPSG) departed from this paradigm and introduced a text-to-text dependency parsing method \cite{lin2022dependency}. Despite its innovative nature, DPSG faced limitations due to its generation-based approach, which could produce unintended outputs. In addition, the need to generate tokens one by one resulted in slower processing speeds, especially when processing additional tokens.
To overcome these challenges, we propose a novel approach that leverages encoder models. Inspired by the ability of encoders to facilitate text-to-text learning through pre-training tasks such as Masked Language Modeling (MLM), we devised a method to represent dependency parsing information as text and apply it during training. We call this method \textbf{S}tructuralized \textbf{P}rompt \textbf{T}emplate (SPT). Unlike traditional dependency parsing, which treats text and dependency structures as separate levels, SPT integrates dependency information directly into text sequences. This approach enables dependency parsing entirely within the textual level by eliminating additional integration steps and improving overall efficiency.


\subsection{Dependency Parsing with Text Representation}

Basically, dependency parsing is the task of finding a word with a dependency relation and determining its corresponding dependency relation label for each word. To express the structured dependency relationships through the prompt template for each word, several conditions should be satisfied: 1) each template must be distinguished by its pattern through whole training, 2) it must be able to indicate its index, 3) it must be able to refer to the other word template with dependency relationship through the output, and 4) it must be able to express the dependency relation label through the output. 

In the first condition, we ensure that the language model can distinctly recognize each template by following a consistent pattern rather than a specific token through all the training process. To satisfy the second and third conditions, we add index prompts $P_{idx}$ that serve two roles: representing the template and indicating the referred template’s index. 
Because the $P_{idx}$ prompts represent the index of templates in a formatted text sequence, each template can refer to another template within a dependency relation regardless of the input sequence using the $P_{idx}$ prompts. Finally, the fourth condition is resolved by adding the dependency relation labels as prompt $P_{\mathcal{L}}$. In addition, we add $P_{POS}$ to reflect information for part-of-speech (POS) analysis. However, several challenges arise when using this prompt approach for dependency parsing:
\begin{enumerate}
\item{\textbf{Semantic Representation Issues}: If the prompts are directly integrated into the model’s existing vocabulary without introducing new tokens, it may disrupt the original vocabulary structure, making it difficult to capture meaningful semantics.}
\item{\textbf{Variability in Length}: The explicit addition of prompts, such as numerical indices and dependency relation labels, causes variations in the length of the enhanced sentence. Masking these components can lead to further inconsistencies in sequence length, which is critical for learning in encoder models.}
\end{enumerate}

To address these issues, we propose to add each prompt as a new token to the vocabulary. 
The above prompts are individually added to the vocabulary enclosed in []. This action allows them to be added independently to the existing vocabulary without any duplications. 
This approach allows the model’s training process to be examined at the text level.
To avoid confusion, we clarify that the notations $P_{idx}$ , $P_{\mathcal{L}}$ , $P_{POS}$ , and $P_{MASK}$\footnote{to be explained in Section \ref{sec:mask}.} represent string text encapsulated in square brackets [].
\begin{equation}
T(w_i) = \text{"}\overbrace{\underbrace{[\,i\,]}_{P_{abs}}\underbrace{[H_i]}_{P_{ref}}}^{P_{idx}}\overbrace{[L_{(w_i,w_{H_i})}]}^{P_{\mathcal{L}}}\overbrace{[POS_{w_i}]}^{P_{POS}}w_i\text{"}
\label{eqa:preprocess}
\end{equation}
\begin{equation}
D = (T(w_1),T(w_2),...,T(w_n))
\label{eqa:dep}
\end{equation}




The final word-level prompt template is represented by Equation \ref{eqa:preprocess}: for each word $w_i$, four prompts are added. In the first prompt $[\,i\,]$ is $P_{abs}$, which represents the absolute index token indicating the index of each word in the sentence.
The second prompt $[H_i]$ is $P_{ref}$ that specifies the index of the word referenced by the given word $w_i$. The only distinction between $P_{abs}$ and $P_{ref}$ is their positions in the template; they serve different roles but share the same tokens. That is why they are grouped under $P_{idx}$. The third is the dependency relation label $[L_{(w_i,w_{H_i})}]$, and the fourth contains $[POS_{w_i}]$ that is POS-tag of $w_i$. The proposed \textbf{S}tructuralized \textbf{P}rompt \textbf{T}emplate (SPT) can effectively incorporate the dependency parsing information into input sentence using these prompts. That is, by applying this template to every word in the sentence, we construct a modified sentence $D$ that encapsulates dependency parsing information. From a certain perspective, performing dependency parsing can be regarded as transforming the original sentence $S$ into a Prompted sentence $D$.





 

\begin{figure}[h]
 \includegraphics[width=\columnwidth]{image/template.png}
 \caption{A figure of word level SPT example}
 \label{fig:set}
 \vskip -.12in
\end{figure}
\subsection{Prediction Task using Soft Prompts $P_{MASK}$ : [HEAD] and [DEP]}

\label{sec:mask}
For the training, $P_{MASK}$, which contains the [HEAD] and [DEP] tokens, serves as a soft prompt and plays a role similar to the masked token in the Masked Language Model (MLM) task. Since performing dependency parsing involves predicting $[H_i]$ and $[L_{(w_i,w_{H_i})}]$, the corresponding parts in the equation are masked to enable the model to learn by making predictions. In this process, $[H_i]$ is masked as [HEAD] and [$L_{(w_i,w_{H_i})}]$ is masked as [DEP], respectively.
The $P_{MASK}$ are also added in the tokenizer's vocab for prompt engineering.
\begin{equation}
M(T(w_i)) = \text{"}\overbrace{[\,i\,]}^{P_{abs}}\overbrace{[HEAD][DEP]}^{P_{MASK}}[POS_{w_i}]w_i\text{"}
\label{eqa:masking}
\end{equation}
\begin{equation}
D_M = (M(T(w_1)),M(T(w_2)),...,M(T(w_n)))
\label{eqa:masked}
\end{equation}


As aforementioned, the $P_{MASK}$ are used to infer two main prediction tasks for the head word and the dependency relation label. They are arranged in the second and third positions of the structuralized prompt templates, forming a consistent pattern in the input text sequence. In our approach, the model is fine-tuned to predict head word and dependency relation labels by the [HEAD] and [DEP] prompts.

Ultimately, the goal is to learn to reconstruct the $D$ from the masked input $D_M$, which has been structured using prompts. This involves different prediction mechanisms for encoder and decoder models:

\paragraph{Encoder-only Models} The encoder model learns by reconstructing the masked input $D_{M}$ into its original form $D$. Since the output sequence retains the same structure as the input sequence, the model effectively maps $D_{M}$ back to $D$:
$Enc(D_{M}) \rightarrow D$
\paragraph{Decoder-based Models} In encoder-decoder models, the encoder processes the masked input $D_{M}$ to generate a latent representation and the decoder uses it to reconstruct $D$. On the other hand, $D_{M}$ of decoder-only models serves as a prefix and it guides the decoder to generate the unmasked $D$, autoregressively:
$Gen(D_{M}) \rightarrow D$ 

\section{Prompt-based Training}

As mentioned earlier, in the encoder model, training is conducted in an MLM manner, where the input masked with $P_{MASK}$:[HEAD] and [DEP] are used to predict reference indices $P_{ref}$:$[H_i]$ and dependency relation label $P_{\mathcal{L}}$:$[L_{(w_i,w_{H_i})}]$ prompts. Since we have added each prompt to the tokenizer, the model is trained to output the corresponding prompt IDs through the LM head. In the decoder model, the training process involves providing masked input and learning to generate the unmasked text.

The loss function for training is calculated by the following equations. In Equation \ref{eqa:one}, $X_{input}$ is the tokenized $D_M$ that is a concatenated sequence of SPTs where $[H_i]$ and $[L_{(w_i,w_{H_i})}]$ are replaced by [HEAD] and [DEP] prompts. In Equation \ref{eqa:two}, $Y_{label}$ is the tokenized $D$ that is the prompted sequence based on SPT.

Since all the prompts are added to the tokenizer's vocabulary, the lengths of $X_{input}$ and $Y_{label}$ are always the same, which is a crucial condition in encoder-only models. In summary, the models are trained with $D_M$ as input and $D$ as the label. Encoder-only models are trained with $\mathcal{L}_{enc}$, which optimizes token prediction based on $X_{input}$ following BERT’s masked language modeling (MLM) for bidirectional contextual learning. Decoder-based models use $\mathcal{L}_{dec}$, where each token $y_i$ is generated sequentially based on $X_{input}$ and previous outputs $Y_{<i}$, following an autoregressive approach similar to GPT. $\theta$ is the parameter of model.

The objective function is based on \textbf{Cross-Entropy Loss}, defined as follows:
\begin{equation}
Tokenize(D_M) = X_{input} = [x_{1},x_{2},...,x_{N}]
\label{eqa:one}
\end{equation}
\begin{equation}
Tokenize(D) = Y_{label} = [y_{1},y_{2},...,y_{N}]
\label{eqa:two}
\end{equation}
\begin{equation}
\mathcal{L}_{enc} = -\sum_{i=1}^N \log P(y_{i} | X_{input};\theta)
\label{eqa:loss}
\end{equation}
\begin{equation}
\mathcal{L}_{dec} = -\sum_{i=1}^N \log P(y_{i} | X_{input}, Y_{<i};\theta)
\label{eqa:loss2}
\end{equation}

In summary, $\mathcal{L}_{enc}$ and $\mathcal{L}_{dec}$ correspond to Cross-Entropy Loss, where the model optimizes token prediction probabilities. In encoder-only models, $\mathcal{L}_{enc}$ applies token-wise classification in an MLM setting, while in decoder-based models, $\mathcal{L}_{dec}$ follows an autoregressive generation paradigm.

\begin{table*}
 \setlength{\abovecaptionskip}{-8pt}
	\centering
	{ 
		\makebox[\linewidth]{\resizebox{\linewidth}{!}{%
    \begin{tabular}{lccccccccccccc}\hline
            & 	bg   & 	 ca   &  cs   &  de   &  en  &  es  &  fr   &  it  & nl  & no  & ro  & ru   &  Avg. \\ \hline\hline
            \citet{dozat-biaffine}$\diamondsuit$   &  90.30   &  \textbf{94.49}   &  {92.65}   &  \textbf{85.98}   &  91.13   &  \underline{93.78}   &  \textbf{91.77}   & \underline{94.72}   & 91.04   & 94.21   & 87.24   &  94.53   & 91.82 \\ 
        \citet{wang-tu-2020-second}$\diamondsuit$   &  91.30   &  93.60   &  92.09   &  82.00   &  90.75   &  92.62   &  89.32   &  93.66   &  91.21   &  91.74   &  86.40   &  92.61    &  90.61 \\ 
         \citet{yang-tu-2022-headed}$\diamondsuit$   &  91.10   &  \underline{94.46}   & 92.57   &  \underline{85.87}  &  \underline{91.32}   &  \textbf{93.84}   &  \underline{91.69}   &  \textbf{94.78}   &  {91.65}   &  \underline{94.28}   &  {87.48}   & 94.45   &  \underline{91.96} \\  
         \citet{lin2022dependency}*   &  \textbf{93.92}   &  93.75   & 92.97   &  84.84   &  \textbf{91.49}   &  {92.37}   &  90.73   &  {94.59}   &  \underline{92.03}   & \textbf{95.30}   &  88.76   & \textbf{95.25}   &  \textbf{92.17} \\  
         \citet{amini-etal-2023-hexatagging}$\diamondsuit$ &  92.87  &  93.79   &  92.82 &  85.18   &  90.85   &  93.17   &  91.50   &  \underline{94.72}   &  91.89   &  {93.95}   &  87.54   &  94.03   &  {91.86}
         \\ \hline			
         SPT-DP (multilingual BERT)   &  91.20  &  90.81   &  92.22   &  79.68   &  87.36   &  90.33   &  88.31   &  92.00   &  89.37   &  90.64   &  86.12   &  93.17   &  89.27  \\
         SPT-DP (XLNet-large)   &  -  &  -   &  -   &  -   &  90.58   &  -   &  -   &  -  &  -  &  -   &  -   &  -   &  - \\
         SPT-DP (XLM-RoBERTa-large) & 93.11  &  92.54   &  \textbf{94.14}   &  82.11   &  88.50   &  91.69   &  88.02   &  93.16   &  91.15   &  93.13   &  \textbf{88.87}   & \underline{95.12}  & 90.90   \\
         SPT-DP (global) & \underline{93.91} & 92.85 & \underline{93.27} & 82.91 & 89.27 & 92.03 & 89.96 & 93.24 & \textbf{92.28} & 92.81 & \underline{88.84} & 94.39 & 91.31  \\
         \hline \\
    \end{tabular}}}
 }
	\caption{12 languages' LAS scores on the test sets in UD 2.2. $\diamondsuit$ use multilingual BERT for embedding and * uses T5-base model for sequence generation parsing. Best and second-best scores are in \textbf{bold} and \underline{underlined}}
	\label{tab:ud} 
\end{table*}

\begin{table}
\centering
 \setlength{\abovecaptionskip}{4pt}
\resizebox{\columnwidth}{!}{
\begin{tabular}{@{}lcc@{}}\hline
  & \multicolumn{2}{c}{PTB} \\
Model  & UAS & LAS \\ \hline\hline
\citet{zhou-zhao-2019-head}*        & 97.0  & 95.43 \\
\citet{mrini-etal-2020-rethinking}*  & 97.42  & 96.26 \\ \hline
\citet{dozat-biaffine}     \;\;\;  \;\;          & 95.74  & 94.08  \\
\citet{wang-tu-2020-second}          & 96.91  & 95.34 \\
\citet{yang-tu-2022-headed}      & \underline{97.24}  & 95.73   \\
\citet{lin2022dependency}      & 96.64  & 95.82  \\
\citet{amini-etal-2023-hexatagging}     & \textbf{97.4}  & \textbf{96.4} \\ \hline
SPT-DP (XLNet-large)             & 96.95  & \underline{95.88}\\
% SPT-DP (w/o <index>)       & 94.28  & 92.63 & - & -\\ 
% SPT-DP (w/o <pos>)                   & 96.76  & 95.66 & - & -\\
 \hline
\end{tabular}}
\caption{Results on PTB dataset. * use additional constituency parsing information so they are not comparable to other methods.}
\label{tab:ptb_results}

\end{table}

\section{Experiments}
We first apply our approach to two datasets used in previous studies, including \textbf{PTB} (Penn Treebank) and \textbf{UD 2.2} \cite{UD2.2} covering 12 languages. Since these datasets primarily consist of \textbf{fusional} languages in which a single morpheme can encode multiple grammatical features, we extended our experiments to \textbf{agglutinative} languages. Unlike fusional languages, agglutinative languages express grammatical relationships through sequences of distinct morphemes, which are carrying out their individual function. For this, we select the \textbf{Sejong} corpus, a Korean language dataset, as an agglutinative language dataset.
\subsection{Datasets}


\paragraph{PTB } This English data is preprocessed by Stanford Parser v3.3.0 \cite{de2008stanford} to convert it into CoNLL format, following the approach of \citet{mrini-etal-2020-rethinking}. 
\paragraph{UD 2.2}
\vskip -.12in
This is composed of 12 languages from UD dataset v2.2 and we follow previous work \cite{amini-etal-2023-hexatagging} for data splitting and organizing. The POS tag information is not used for the experiments by omitting $P_{POS}$ in the template.\vskip -.10in
\paragraph{Sejong}
\vskip -.12in 
This is the Korean dataset and only the POS tags of the first and last morphemes are used for this experiment because Korean words consist of multiple morphemes.

\subsection{Pre-trained Language Models}  

\paragraph{Encoder-based Models}  
XLNet-large, Multilingual BERT, XLM-RoBERTa-large, and RoBERTa
\paragraph{Decoder-based Models}  
\vskip -.10in
T5-base, BART-large, LLaMA3.2-3B, and Qwen2.5-3B  

\subsection{Comparison Models}
\citet{zhou-zhao-2019-head} and \citet{mrini-etal-2020-rethinking} used additional constituency parsing information so they are not comparable to other methods directly.
\citet{dozat-biaffine} introduced a biaffine model as a graph-based dependency parsing approach. \citet{wang-tu-2020-second}  proposed a second-order graph-based method with message passing. \citet{yang-tu-2022-headed} developed a projective parsing method based on headed spans. \citet{lin2022dependency} introduced a sequence generation-based parsing method, while \citet{amini-etal-2023-hexatagging} leveraged structural tags and sequential tag decoding. \citet{park-2019} and \citet{IM-2021} constructed a dependency parser using the Korean morpheme version of BERT. 

\begin{table}[h]
\centering
\resizebox{\columnwidth}{!}{
\begin{tabular}{@{}lcc@{}}\hline
Model  & UAS & LAS \\ \hline\hline
 \citet{park-2019} \;  & 94.06  & 92.00 \\ 
 \citet{IM-2021} \; \; \; \; \; \; \;   & \textbf{94.76}  & \textbf{92.79} \\ \hline
SPT-DP  & \underline{94.52}   & \underline{92.36}\\
 \hline
\end{tabular}}
\caption{Results on Sejong dataset}
\label{tab:sejong}

\end{table}


\subsection{Experimental Results}

\begin{table}[h!]
\centering
\caption{Performance Comparison for Different Levels}
\resizebox{0.5\textwidth}{!}{
\begin{tabular}{@{}lccc@{}}
\toprule
\textbf{Category} & \textbf{Score} & \textbf{Soft Score} & \textbf{Hard Score} \\ \midrule
\multicolumn{4}{c}{\textbf{Overall Scores}} \\
Llama-3.2-3B-Instruct\textsubscript{BASE}       & 36.85          & 43.95              & 35.23            \\
Llama-3.2-3B-Instruct\textsubscript{SFT}         & 42.27          & 54.57               & 38.50              \\
Llama-3.2-3B-Instruct\textsubscript{DPO}         & 63.84          & 57.84               & 64.86             \\ \midrule
% \multicolumn{4}{c}{\textbf{Easy (47 Samples)}} \\
% Llama-3.2-3B-Instruct\textsubscript{BASE}         & 49.89          & 58.16               & 47.63              \\
% Llama-3.2-3B-Instruct\textsubscript{SFT}         & 51.65          & 66.38               & 42.82              \\
% Llama-3.2-3B-Instruct\textsubscript{DPO}         & 51.26          & 62.31               & 44.05              \\ \midrule
% \multicolumn{4}{c}{\textbf{Medium (55 Samples)}} \\
% Llama-3.2-3B-Instruct\textsubscript{BASE}         & 45.53         & 53.48               & 39.60              \\
% Llama-3.2-3B-Instruct\textsubscript{SFT}         & 43.92          & 53.53               & 32.14              \\
% Llama-3.2-3B-Instruct\textsubscript{DPO}         & 37.47          & 57.14               & 35.16              \\ \midrule
% \multicolumn{4}{c}{\textbf{Hard (98 Samples)}} \\
% Llama-3.2-3B-Instruct\textsubscript{BASE}         & 16.65          & 30.72               & 15.73              \\
% Llama-3.2-3B-Instruct\textsubscript{SFT}         & 25.30          & 42.14               & 24.40              \\
% Llama-3.2-3B-Instruct\textsubscript{DPO}         & 24.31          & 37.72               & 23.59              \\ \bottomrule
\end{tabular}
}
\label{table:global_performance_comparison}
\end{table}

Table \ref{tab:ptb_results} presents the performance comparison of different models on the PTB dataset. Our proposed method achieves competitive performance compared to state-of-the-art (SOTA) models in spite of relying \textbf{solely on a pre-trained language model} without additional complex modules. That is, despite its \textbf{lightweight and efficient design}, our method is ranked among the top-performing models.

As shown in Table \ref{tab:ud}, our method exhibits lower performance with multilingual BERT on UD 2.2. However, using a larger model, XLM-RoBERTa-large, leads to significant performance improvements across most languages and it eventually achieves SOTA results on two languages. Similarly, our approach with XLNet-large on UD 2.2 English data shows more improved performance depending on its English-only pre-training. 
We can think that the larger models with richer contextual representations we can use, the better performance our method can obtain. This is because we can effectively leverages their expressiveness to enhance parsing performance.
In addition, we construct the \textbf{global}\footnote{This approach will be discussed in detail in \ref{sec:multi}} model by training on the entire languages of UD 2.2 dataset and it leads to overall improvements in parsing accuracy across many languages. Notably, in \textbf{Dutch (nl)}, our approach achieves an additional SOTA result. On the other hand, the results on the \textbf{Sejong} dataset (Table \ref{tab:sejong}) demonstrate that our method achieves performance on par with more complex SOTA models, even though it is based on \textbf{lighter and more efficient architecture} for \textbf{agglutinative languages}.

\section{Analysis}

\subsection{Unified Cross-lingual Dependency Parsing}
\label{sec:multi}
Furthermore, our evaluation of language-specific experiments on the UD 2.2 dataset is expanded to cross-lingual experiments. Since the composition of dependency relation labels varies across languages, we integrate dependency relation labels from 12 languages into a shared vocabulary to construct a unified model for cross-lingual dependency parsing. 
Then we train a single model using a unified training dataset based on integrated dependency relation labels and we refer to it as the \textbf{global} model in Table \ref{tab:global}. This cross-lingual model exhibits robust performance and it demonstrates its ability to generalize multi-languages while maintaining competitive results. As you can see in Table 4, it outperforms the \textbf{base}\footnote{A trained model where the test language and train language are the same.} model in many languages. In addition, we evaluate a model trained by excluding the training set of the target test language, referred to as the \textbf{unseen} model, for an out-of-domain evaluation. This \textbf{unseen} model achieves meaningful performance even on languages for which it was not explicitly trained in dependency parsing. This highlights cross-lingual correlations in dependency parsing and shows that our approach provides flexibility and scalability, particularly in multilingual or resource-constrained scenarios.



\subsection{Length Robustness}
\label{sec:length}
In previous studies, dependency parsing first attempt to represent the syntactic information of words in a sentence by feeding the final hidden states from the pre-trained model into additional modules and classifies each embedding or directly compares the output embeddings of each word to find dependency relations.
In contrast, our study newly defines and utilizes index prompts, $P_{idx} $($P_{abs}$, $P_{ref}$); $P_{ref}$ in SPT of a word indirectly refer to the $P_{abs}$ in SPT of another word to represent their dependency relations. Therefore, we have to check out how the relations between $P_{idx}$ tokens are well trained by the proposed method. Table \ref{tab:length} presents the performance according to sentence length, showing a tendency for performance to decrease as sentence length increases. Furthermore, as shown in Table \ref{tab:length2}, which details performances based on the index range of templates, predictions for dependency relations in higher indices tend to have lower accuracy. 
Statistically, because longer sentences have relatively small population, prompts corresponding to higher indices may not have been sufficiently learned. In future work, we should focus on developing improved training strategies to enhance performances for high indexed words in longer sentences.


\begin{table}[!t]
% \small
    \centering
    \resizebox{\linewidth}{!}{
    \begin{tabular}{l|cc|c}
    \toprule
    \textbf{Methods} & \textbf{Length} & \textbf{ASR} & \textbf{Length-to-ASR Ratio (\%$\uparrow$)} \\
    \midrule
    \textbf{Pre-generated} & 111.2 & 0.80 & 0.72 \\
    \midrule
    \textbf{Simple Truncation-i} & 72.15 & 0.66 & 0.91 \\
    \textbf{Simple Truncation-ii} & 50.46 & 0.57 & 1.13 \\
    \textbf{Word Frequency} & 97.43 & 0.49 & 0.50 \\
    \textbf{LLM-based} & 52.12 & 0.68 & \textbf{1.30} \\
    \bottomrule
    \end{tabular}}
    \caption{Comparison of methods of length declining. \textbf{Length} refers to the token size of the prompt, \textbf{ASR} denotes the attack success rate, and \textbf{length-to-ASR ratio} indicates the ratio of length to ASR.}
    \label{tab:length}
\end{table}
\begin{table}[h]
\centering

\resizebox{\columnwidth}{!}{
\begin{tabular}{c|c|c|c}
\hline
\multicolumn{1}{l}{\;\;Index range\;\;} & \multicolumn{1}{l}{\;\;{\#} of Indices\;\;} & \multicolumn{1}{l}{\;\;UAS\;\;} & \multicolumn{1}{l}{\;\;LAS\;\;} \\ \hline\hline
1-10    & 23269     & 97.77     & 96.84     \\
11-20   & 18283    & 96.39     & 95.14     \\
21-30   & 10124     & 96.32     & 95.22     \\
31-40   & 3829     & 96.63     & 95.59     \\
41-50   & 966     & 96.07     & 95.13     \\
51-60   & 188     & 95.74      & 95.21     \\
61-70   & 25       & 92.00     & 88.00     \\ \hline
all & 56684    & 96.95     & 95.88     \\ \hline
\end{tabular}}
\caption{A Table of statistics and performance according to index range}
\label{tab:length2}

\vskip -.14in
\end{table}

\subsection{Decoder-based Models}
\label{sec:decoder}
The proposed \textbf{SPT-DP} method entirely operates at the \textbf{text level} and it enables our model to have both of the easily applicable and trainable abilities. In this section, we validate its feasibility on different language model architectures, including encoder-decoder models and decoder-only models. In these models, their decoders generate dependency parsing results of $P_{ref}$ and $P_{\mathcal{L}}$ along with other input sequence. As you can see in Table \ref{tab:ptb_decoders_results}, decoder-based models achieve lower performance than encoder-only models. One possible reason for this lower performance is that this decoding method relies on previously generated tokens and it can lead to error propagation; mistakes in earlier tokens are accumulated and it can degrade the accuracy of subsequent predictions.

\begin{table}[h]
\centering
\resizebox{\columnwidth}{!}{
\begin{tabular}{@{}lcc@{}}\hline

Decoder-based Model \; \; \; \; \; \; \; \; \; \;   & UAS & LAS  \\ \hline
\hline
XLNet-large(encoder) &96.95 &95.88\\ \hline
T5-base              & 95.30  & 93.86 \\
Bart-large              & 95.84  & 94.73 \\
Llama3.2-3B              & 94.55  & 93.27 \\
Qwen2.5-3B                & 94.97  & 93.81  \\
\hline
\end{tabular}}
\caption{Results on PTB with Decoder-based models}
\label{tab:ptb_decoders_results}
\vskip -.12in
\end{table}


\subsection{Ablation Study for Prompts}
First, we aim to examine the impact of each proposed prompt on parsing performance. We conduct additional experiments by excluding each of $P_{abs}$ and $P_{{POS}}$ to verify how important they are for dependency parsing. 
As shown in the Table \ref{tab:prompt result}, the role of POS information is not critical but $P_{abs}$ has a significant impact on performance. Intuitively, when a physically referable index prompt $P_{abs}$ exists in the input, the model can effectively refer it through transformer's attention mechanism. A detailed analysis is provided in Appendix \ref{sec:attention}. 
\documentclass{article}


\usepackage{arxiv}

\usepackage[utf8]{inputenc} % allow utf-8 input
\usepackage[T1]{fontenc}    % use 8-bit T1 fonts
\usepackage{hyperref}       % hyperlinks
\usepackage{url}            % simple URL typesetting
\usepackage{booktabs}       % professional-quality tables
\usepackage{amsfonts}       % blackboard math symbols
\usepackage{nicefrac}       % compact symbols for 1/2, etc.
\usepackage{microtype}      % microtypography
\usepackage{lipsum}
\usepackage{authblk}
\usepackage{graphicx}
\usepackage{amsmath}
\usepackage{comment}
\usepackage{xcolor}
\usepackage{adjustbox}
\usepackage{booktabs}
\usepackage{subcaption}
\usepackage{float}
\usepackage{algorithmic}
\usepackage{caption}

\title{Neural Network-based Vehicular Channel Estimation Performance: Effect of Noise in the Training Set}

\author[1,2,3]{\textbf{S.~A.~Ngorima}}
\author[1]{\textbf{A.~S.~J.~Helberg}}
\author[1,2,3]{\textbf{M.~H.~Davel}}

\affil[1]{Faculty of Engineering, North-West University, South Africa}
\affil[2]{Centre for Artificial Intelligence Research, South Africa}
\affil[3]{National Institute for Theoretical and Computational Sciences, South Africa}

\renewcommand\Authands{ and } % Replace "and" with "and" for the last author

%\date{}
%\date{\vspace{-2em}}

\begin{document}

\maketitle

\begingroup
\renewcommand{\thefootnote}{}
\footnote{This work is a preprint of a published paper by the same name\cite{10.1007/978-3-031-78255-8_12}. The authenticated version is available online at \url{https://doi.org/10.1007/978-3-031-78255-8_12}.}
\endgroup

\begin{abstract}
Vehicular communication systems face significant challenges due to high mobility and rapidly changing environments, which affect the channel over which the signals travel. To address these challenges, neural network (NN)-based channel estimation methods have been suggested. These methods are primarily trained on high signal-to-noise ratio (SNR) with the assumption that training a NN in less noisy conditions can result in good generalisation. This study examines the effectiveness of training NN-based channel estimators on mixed SNR datasets compared to training solely on high SNR datasets, as seen in several related works. Estimators evaluated in this work include an architecture that uses convolutional layers and self-attention mechanisms; a method that employs temporal convolutional networks and data pilot-aided estimation; two methods that combine classical methods with multilayer perceptrons; and the current state-of-the-art model that combines Long-Short-Term Memory networks with data pilot-aided and temporal averaging methods as post processing. %Our results indicate that a training dataset with a high SNR is not always optimal, and the training SNR levels should be considered a hyperparameter to be adjusted. 
Our results indicate that using only high SNR data for training is not always optimal, and the SNR range in the training dataset should be treated as a hyperparameter that can be adjusted for better performance. This is illustrated by the better performance of some models in low SNR conditions when trained on the mixed SNR dataset, as opposed to when trained exclusively on high SNR data.
\end{abstract}


% keywords can be removed
\keywords{Channel estimation \and deep learning \and neural networks \and CNN-Transformer \and IEEE 802.11p \and vehicular channels.}

\section{Introduction}
The integration of network infrastructure with artificial intelligence (AI) is becoming more prevalent due to the increasing demands of data-intensive applications and the need for more efficient and reliable communication systems. Incorporating AI into these systems promises to improve network management, resource allocation, and overall system performance.

This convergence is notably evident in vehicular communication, which is an important component of intelligent transportation systems. Vehicular communication systems face unique problems due to their high mobility, rapidly varying environments, and significant latency requirements. One of the challenges in this domain is obtaining accurate channel estimates, which is critical to ensuring reliable communication between vehicles and infrastructure. Channel estimation involves determining the characteristics of a communication channel and monitoring its changes to adjust to varying conditions~\cite{875230}.
%This is particularly evident in vehicular communication, where one of the most significant challenges is accurate channel estimation. 

The dynamic nature of vehicular environments, characterised by rapidly changing channel conditions, poses a challenge to traditional channel estimation methods. These methods include methods such as least squares (LS) and minimum mean square error (MMSE), which rely on mathematically estimating the channel based on the received version of known data sent as a preamble to the signal. Although effective in more stationary environments, these techniques often struggle to keep up with the rapid changes inherent in vehicular channels~\cite{gizzini2020deep}. This limitation highlights the need for more advanced adaptive approaches to channel estimation in these high-mobility scenarios. 

In recent years, researchers have increasingly focused on using data symbols directly to estimate the channel in a method referred to as data pilot-aided (DPA) channel estimation. DPA estimation addresses the challenges posed by dynamic environments, especially in vehicular communication systems, where pilot signals are evidently insufficient. Unlike traditional methods that rely heavily on a fixed number of pilot signals for channel estimation, DPA estimation uses demapped data symbols as additional pilots. This approach allows for more flexible and adaptive tracking of channel variations without the need to allocate more resources for pilot signals, thereby improving efficiency and accuracy in rapidly changing channels. However, the performance of the DPA method can be significantly affected by errors introduced during the demapping process, which may limit its overall effectiveness~\cite{pan2021channel}.
%However, the DPA method has a demapping error that significantly limits its performance~\cite{pan2021channel}. 
To improve efficiency, DPA estimation is typically used with an error compensation scheme to reduce channel distortion and prevent errors from propagating to the next symbols in a frame. Two common error compensation approaches for channel estimation are spectral temporal averaging (STA)~\cite{fernandez2011performance} which utilises the correlations of data symbols in time and frequency domains to estimate the channel and time-domain reliable test frequency domain interpolation (TRFI)~\cite{kim2014time}, as described in more detail in Section \ref{well-known}. 
However, these methods still struggle in highly dynamic scenarios~\cite{gizzini2021temporal}.

Deep learning (DL) has recently gained attention for its ability to improve the accuracy of channel estimates. A method utilising autoencoders and training on a high signal-to-noise ratio (SNR) dataset (SNR=40 dB) was proposed to correct DPA estimate errors in the frequency domain~\cite{han2019deep}. However, it did not account for temporal changes, limiting its performance. Furthermore, incorporating feedforward networks into classical methods such as STA and TRFI methods, as proposed in~\cite{gizzini2020deep}, has shown promise in improving estimation accuracy. These methods were also trained on high SNR datasets (SNR=30 dB).
 
These methods offer valuable baselines.
%Nevertheless, these methods are not effective on all signal-to-noise ratio SNRs. 
Recently, a long-short-term memory (LSTM)-based architecture has been proposed to capture the temporal and frequency characteristics of vehicular channels~\cite{gizzini2021temporal}. This method was trained on a dataset generated at the SNR level of 40 dB and demonstrated significant improvements over previous estimators. However, this method is computationally expensive. 

This study introduces the use of a mixed SNR dataset, where the training dataset includes data generated at different SNR levels. We hypothesise that training on a mixed SNR dataset helps an neural network (NN) generalise across a wider range of real-world conditions by exposing it to both noise-dominated and signal-dominated scenarios. This approach contrasts with the methodology adopted in several studies~\cite{gizzini2020deep,gizzini2020joint,gizzini2021temporal,pan2021channel}, which propose that training the NN-based estimator using a dataset generated only at a high SNR level enhances its ability to generalise in any environment: at high SNR the impact of the channel is greater than the impact of noise, therefore it is expected that the NN develops better channel knowledge. 

We evaluate the use of the mixed SNR dataset using a CNN-Transformer-based  estimator~\cite{Ngorima2024a} which is a hybrid model that combines a one-dimensional convolutional neural network (CNN) with a transformer architecture, a TCN-DPA estimator~\cite{Ngorima2024} that uses a temporal convolutional network (TCN) to correct the DPA propagation error, two estimators based on concatenating a multilayer perceptron (MLP) with a classical method, in this case STA-MLP~\cite{gizzini2020deep} and TRFI-MLP~\cite{gizzini2020joint}, and the LSTM-DPA-TA method~\cite{gizzini2021temporal} which combines Long Short-Term Memory (LSTM) networks with Data Pilot-Aided (DPA) estimation and Temporal Averaging (TA).


\section{Background}

We describe the system model, well-known channel estimation schemes in vehicular communications, and some of the NN-based channel estimation techniques currently used. 
We also provide a brief description of the IEEE 802.11p physical layer structure that is used in all simulations, along with a description of its specifications. 

\subsection{System Model}
\label{system_model}

In this paper, we adopt the IEEE 802.11p physical layer structure, which is specifically designed for vehicular communication systems. This section follows a detailed discussion of the IEEE 802.11p specifications as outlined in Ngorima et al.~\cite{Ngorima2024}. %We also provide a detailed discussion of some of the channel estimation methods later used for this work's evaluation stage.
%
The IEEE 802.11p standard uses Orthogonal Frequency Division Multiplexing (OFDM) to transfer data by dividing the available bandwidth into several subcarrier frequencies for simultaneous transmission of multiple signals. 52 of the 64 available subcarriers are used for data transmission and pilot symbols. The remaining subcarriers are for guard bands and direct current (DC) offset. To illustrate this structure, Figure \ref{fig:1DCNN} shows the IEEE 802.11p OFDM frame, indicating pilot and data subcarriers.

\begin{figure}[ht]
    \centering
    \includegraphics[width=0.5\linewidth]{images/1DCNN.jpg}
    \caption{The IEEE 802.11p frame structure, illustrating the allocation of subcarriers for pilot and data transmission.}
    \label{fig:1DCNN}
\end{figure}
In this work, we consider a frame structure that consists of two preamble symbols for signal detection and timing synchronisation, followed by a signal field that contains transmission parameters.
The received signal on subcarrier $[k]$ at time $i$ can be represented as
%
\begin{equation*} Y_{i}[k]=H_{i}[k]X_{i}[k]+N_{i}[k],\tag{1} \label{rec_signal}\end{equation*}
%
where $H_i[k]$, $X_i[k]$ and $N_{i}[k]$ denote the respective channel response, transmitted signal, and additive Gaussian noise (AWGN), respectively. Note that in vehicular channels $H_{i}$ varies in both the time and frequency domains.

\subsection{Well-known Vehicular Channel Estimation Schemes}
\label{well-known}
This section describes some of the main vehicular channel estimation schemes that have been widely used and studied in the literature. These methods were selected based on their effectiveness in dynamic environments and their foundational role in the development of more advanced NN-based estimators.
%The following are some of the vehicular channel estimation schemes. 

\subsubsection{DPA Estimation:}
The first step of the DPA process, as described by Pan et al.~\cite{pan2021channel}, is to calculate the initial estimate of the frame. This is done by applying the least squares (LS) estimation to the preamble as shown below:

% \subsubsection{DPA Estimation}

% The first step of the DPA process is to calculate the initial estimate of the frame. This is done by applying the least squares (LS) estimation to the preamble as shown below:

\begin{equation*} \hat { {h} }_{\text {0}}[k] = \frac { {y} ^{(p)}_{1}[k] + {y} ^{(p)}_{2}[k]}{2 {p} [k]}, \tag{2}\label{eqn:ls}\end{equation*}
%
where ${y} ^{(p)}_{1}[k]$ and ${y} ^{(p)}_{2}[k]$ are the known orthogonal preambles received on the $[k]$ subcarrier and $p[k]$ is the OFDM data symbol transmitted over the k-th subcarrier.
%
The next OFDM symbols in the frame are equalised using the initial estimate as the starting point as follows:
%
\begin{equation*} y_{i}^{\text{eq}}[k]= \frac{y_{i}[k]}{\hat{h}_{i-1}^{\text{DPA}}[k]},\hat{h}_{0}^{\text{DPA}}[k]=\hat{h}_{\text {0}}[k].\tag{3}\label{eqn:yeq}\end{equation*}
%
The equalised symbol is demapped and mapped to its closest constellation point as: %(see (\ref{eqn:conse})). 
%
\begin{equation*}d_{i}[k]=\mathfrak{R}\left(\frac{y_{i}[k]}{ \hat{h}_{i-1}^{\text{DPA}}[k]}\right), \hat{h}_{0}^{\text{DPA}}[k]=\hat{h}_{\text {LS}}[k],\tag{4}\label{eqn:conse}\end{equation*}
%
Finally, the DPA channel is updated: 
%
\begin{equation*}\tilde{h}_{\mathrm{D}\mathrm{P}\mathrm{A}_{i}}[k]=\frac{y_{i}[k]}{d_{i}[k]}\tag{5}\end{equation*}
%
\subsubsection{STA Estimation Scheme:}
%
STA uses both spectral (frequency) and temporal correlations within data symbols to estimate the channel~\cite{gizzini2020deep}. STA can also be viewed as a recursive filtering process that combines multiple channel estimates into a more accurate estimate.
%
The first step is to calculate a channel estimate in the frequency domain $\hat{h}{i}^{\text{FD}}[k]$ as follows: 
%
\begin{equation*} \hat{h}{i}^{\text{FD}}[k]=\sum{\lambda=-\beta}^{\beta}\omega_{\lambda}\hat{h}_{i}^{\text{DPA}}[k+\lambda],~\omega _{\lambda } = \frac {1}{2\beta +1},\tag{6}\end{equation*}
%
where $2\beta + 1$ represents the number of subcarriers that are considered.
%
Subsequently, a temporal average is computed to arrive at the final STA channel estimate, $\hat{h}{i}^{\text{STA}}[k]$:
%
\begin{equation*}\hat{h}{i}^{\text{STA}}[k]=\left(1-\frac{1}{\alpha}\right)\hat{h}{i-1}^{\text{STA}}[k]+\frac{1}{\alpha}\hat{h}_{i}^{\text{FD}}[k],\tag{7}\end{equation*}
%
where $\alpha$ is a smoothing parameter that controls the weight given to current and previous estimates.

\subsubsection{TRFI Estimation Scheme:}
%
TRFI leverages the high correlation within OFDM symbols to improve the accuracy of the channel estimation~\cite{gizzini2020joint}. The process starts by creating multiple channel estimates for each symbol using the DPA method. The received symbol is equalised using the estimates of the current and previous symbols ($\hat{h}_{i}^{\text{DPA}}[k]$) and ($\hat{h}_{i-1}^{\text{DPA}}[k]$) respectively.
%
\begin{align*}
{y}_{i-1}^{\text{eq}\prime}[k] &= \frac{y_{i-1}[k]}{\hat{h}_{i}^{\text{DPA}}[k]}, \\
{y}_{i-1}^{\text{eq}\prime\prime}[k] &= \frac{y_{i-1}[k]}{\hat{h}_{i-1}^{\text{DPA}}[k]}, \tag{8}\label{eqn:cdp_eq}
\end{align*}
%
Subsequently, the equalised symbols are remapped to their corresponding constellation points, ${d}_{i-1}^\prime[k]$ and ${d}_{i-1}^{\prime\prime}[k]$. 
%
TRFI then categorises subcarriers into reliable and unreliable sets based on a reliability test between the remapped symbols. Subcarriers with consistent remapped symbols, that is, ${d}_{i-1}^\prime[k]= {d}_{i-1}^{\prime\prime}[k]$, are considered reliable, while those with discrepancies are classified as unreliable. TRFI then interpolates channel estimates for unreliable subcarriers by using the reliable subcarriers as anchor points. This interpolation process effectively fills in the gaps in channel knowledge, resulting in a more accurate overall channel estimate. A detailed discussion on the interpolation procedure is given in~\cite{gizzini2020deep}.
%
\subsection{MLP-based Estimators} 
To enhance the performance of conventional STA and TRFI estimators, Gizzini et al.~\cite{gizzini2020deep} introduced MLP-based channel estimation methods. In these methods, MLP layers are added to the STA and TRFI processes, allowing the model to learn complex patterns in vehicular channels and thereby improve the accuracy of the estimation. The methods STA-MLP and TRFI-MLP initially conduct DPA estimation, proceed with STA or TRFI estimation and subsequently feed the results into the MLP layers. 
%
\subsection{TCN-DPA Estimator}
%
As proposed in~\cite{Ngorima2024}, the TCN-DPA estimator integrates a Temporal Convolutional Network (TCN) with DPA estimation to enhance channel estimation in dynamic environments. The TCN processes the received OFDM symbols in the frequency domain, treating subcarriers as time steps. The extracted features are then passed to the DPA process, where the previous TCN output, \(\hat{h}_{i-1}^{\text{TCN}}[k]\), is used to equalise the current received symbol. After equalisation, the symbol is demapped and remapped to the nearest constellation point to obtain the final channel estimate, \(\hat{h}_{i}^{\text{TCN}}[k]\). This estimate is updated iteratively using the DPA process. This approach leverages the ability of the TCN to model long-range dependencies across subcarriers, making it effective for complex channel conditions.

\subsection{LSTM-DPA-TA Estimator}
%
The LSTM-DPA-TA estimator~\cite{gizzini2021temporal} integrates an LSTM network with DPA and Temporal Averaging (TA) techniques to enhance channel estimation in vehicular environments. In this approach, the LSTM processes the received OFDM symbols to extract key features, which are then passed into the DPA module. The resulting channel estimates are further refined using the TA technique. In the LSTM-DPA-TA method, OFDM symbols are treated as time steps, with subcarriers considered as features within each time-step sequence. This design enables the model to effectively capture frequency dependencies and improve the accuracy of channel estimation under dynamic conditions.

\subsection{CNN-Transformer}
The CNN-Transformer is a recently proposed hybrid architecture for vehicular channel estimation in~\cite{Ngorima2024a}. This approach combines the strengths of CNNs and Transformer networks to analyse vehicular channel data comprehensively. The channel data can be represented in both the time and frequency domains. The frequency domain represents the subcarriers of the OFDM symbols, while the time domain represents the sequence of OFDM symbols transmitted over time. Each OFDM symbol is transmitted across multiple subcarriers. %, which are essentially distinct frequency bands. 
The frequency domain captures the spatial characteristics of the channel across OFDM subcarriers, while the time domain reflects the temporal dynamics as OFDM symbols are transmitted sequentially. 

In this CNN-Transformer architecture, subcarriers are treated as the `time steps'. 
The architecture leverages CNNs to extract local features from subcarriers and uses Transformer layers to capture global dependencies across the frequency domain. The CNN component applies 1D convolutions to capture local patterns, while the Transformer component uses self-attention mechanisms to analyse patterns between subcarriers globally.

\section{Methodology}
\label{proposed}

This section outlines the methodological framework used in this study.
%to assess the best training practices for different NN-based channel estimators. 
We start with a detailed description of the datasets used for training and evaluation, specifically focussing on two approaches: the mixed SNR dataset and the high SNR dataset. Following this, we discuss the data preprocessing steps applied to transform complex received symbols into a format suitable for input into NNs. 
%
%
\subsection{Data Preparation}
\label{dataprep}
We simulate vehicle-to-vehicle communication following the `Vehicle-To-Vehicle Expressway Same Direction with Wall' (VTV-SDWW) channel model~\cite{4526014}. Our simulation specifically modelled vehicles moving at 100 km/h, with a Doppler shift of 550 Hz. We used 16QAM (16-Quadrature Amplitude Modulation) modulation for data transmission in this urban environment scenario. This modulation scheme encodes the data by varying the amplitude and phase of the carrier signal.

\subsubsection{Mixed SNR Dataset}
The mixed SNR dataset represents a wide range of noise introduced on the VTV-SDWW channel. For this dataset, we simulate 18,000 time-specific frames with SNR values ranging from 0 to 40 dB in increments of 5 dB. Each SNR level includes 2,000 frames, with 50 OFDM symbols spread across 52 active subcarriers (48 data and 4 pilot subcarriers). During training, each model is exposed to the entire range of SNR levels.
%
The validation set is created by reserving 25\% of this training data to monitor the performance of the model during training. An independent test set consisting of 2,000 frames is generated separately from the training sets. This test set includes frames at SNR levels of 0, 5, 10, 15, 20, 25, 30, 35, and 40 dB, ensuring that the evaluation correctly reflects the performance across different SNR conditions. The same test set is used to test all the models regardless of how they are trained. 
%
\subsubsection{High SNR Dataset}
The high SNR dataset focusses on a fixed SNR level, representing less noisy or nearly ideal channel conditions, as opposed to the mixed SNR dataset. This dataset is used to train models with the assumption that training NNs in an environment with a clearly defined channel can lead to better generalisation even in noisy environments~\cite{gizzini2021temporal,gizzini2020deep,gizzini2020joint,pan2021channel}. 
In this work, the high SNR dataset consists of 18,000 time-specific frames, all generated at a 40 dB SNR level, with the same frame structure as the mixed SNR data. As before, 25\% of the training data is set aside for validation. 

\subsubsection{Data preprocessing}

To preprocess the data for NN models, we transform the complex 16QAM symbols received across subcarriers and time slots into a real-valued format. Each received symbol consists of both real and imaginary components. The following steps are performed to prepare the data:

\begin{enumerate}
    \item \textbf{Separate Real and Imaginary Components:} Decompose each complex 16QAM symbol into its real and imaginary parts. Handle the data as two separate real-valued matrices instead of a single complex-valued matrix. For a given OFDM symbol, where \(y[k] = r[k] + j \cdot i[k] \) represents the received complex value at subcarrier \( k \), extract \( r[k] \) as the real part and \( i[k] \) as the imaginary part.

    \item \textbf{Interleave Components:} After separation, interleave the real and imaginary components. This means arranging the real and imaginary parts in an alternating sequence across the time slots. By interleaving, a unified structure that preserves the relationship between the real and imaginary parts within the input data is created.

    \item \textbf{Form the Input Matrix:} Structure the interleaved data into a matrix with dimensions \( 52 \times 100 \), where 52 represents the number of subcarriers and 100 represents the sequence of interleaved real and imaginary components across the time slots. This transformation enables the NN to process the complex-valued input as a series of real-valued inputs.

    \item \textbf{Prepare Input Data for NN models:} The resulting matrix is now a real-valued representation of the original complex data, ready for input into NN models. %This preprocessing step ensures that the network can effectively learn from the data without being hindered by the challenges of handling complex numbers directly.
\end{enumerate}
%
The transformed input matrix described above is visually represented in Figure~\ref{fig:frame}. Each block in the figure represents the interleaved real and imaginary components of the received 16QAM symbols for a particular subcarrier across the OFDM symbols. 
The kernel size of 3 shown in the figure indicates the width of the sliding window used by the CNN during the convolution process across the subcarriers.
\begin{figure}[ht] 
    \centering
    \includegraphics[width=\linewidth]{images/Frame.jpg}
    \caption{Transformed IEEE 802.11p frame structure used as input to deep learning models, composed of 100 interleaved complex symbols and 52 subcarriers.}
    \label{fig:frame}
\end{figure}

\subsection{Hyperparameter Optimisation}
\label{sec:hyperparameters}
We reimplement most of the NN-based estimators discussed in the previous sections to ensure compatibility with our experimental setup and to be able to optimise each individually for the specific datasets used. Specifically, the CNN-Transformer, TCN-DPA, STA-MLP, and TRFI-MLP models were reimplemented from scratch. The LSTM-DPA-TA model is implemented using existing code available online. To verify the accuracy of our reimplementations, we compare the performance of our implementations against the benchmark results reported in the original studies, ensuring that they perform as expected.

The final architecture of each model is determined through a hyperparameter tuning process, to identify the optimal configurations for each model. We optimise the hyperparameters of all models in both datasets using Optuna, a Bayesian optimisation framework that efficiently searches the hyperparameter space using a tree-structured Parzen estimator (TPE)~\cite{akiba2019optuna}. This method systematically explores different hyperparameter configurations while conserving computational resources by early termination of underperforming trials. %The optimisation process covers a set of hyperparameters that address the important components of the model. %The steps in the optimisation procedure are as follows:

For each model, we conduct multiple trials with different random seeds to ensure that results are not biased by a particular initialisation. Specifically, we conduct 50 trials per model, each with 3 seeds, allowing us to assess the robustness of the selected hyperparameters. The search space for each model was designed to cover a wide range of potential configurations. The key hyperparameters include the learning rate, the number of layers, the layer size (kernel size,  number of attention heads, or layer width, depending on model), the learning rate scheduler parameters, and hyperparameters controlling regularisation through dropout. 
%The search space was determined on the basis of prior studies, domain knowledge, and preliminary experiments, ensuring that it was sufficiently broad to capture the optimal settings. 
During the optimisation process, we monitor the convergence of validation loss. Early stopping is employed to prevent overfitting, stopping training when the validation loss ceases to improve for a predefined number of epochs. 


\begin{table}[hb!]
\caption{Optimised hyperparameters for CNN-Transformer, TCN-DPA, STA-MLP, TRFI-MLP, and LSTM-DPA-TA on the mixed SNR and high SNR datasets.}
\label{tab:all_hyperparameters}
\centering
\begin{tabular}{lccc}
\hline
\textbf{Hyperparameter} & \hspace{0.15em}\textbf{Search Space} & \hspace{0.15em}\textbf{Mixed SNR}\hspace{0.15em} & \hspace{0.25em}\textbf{40 dB}\hspace{1em} \\
\hline
\multicolumn{4}{l}{\textbf{CNN-Transformer}} \\
Learning rate & [1e-5 to 1e-2] & 0.001 & 0.001 \\
Transformer layers & [1 to 4] & 2 & 4 \\
Number of attention heads & [1 to 4] & 2 & 4 \\
Hidden dimension & [64 to 128] & 128 & 128 \\
Dropout rate & [0.001 to 0.3] & 0.1 & 0.25 \\
Number of epochs & [50 to 200] & 110 & 200 \\
Number of CNN layers & [1 to 5] & 2 & 4 \\
CNN kernel size & [2 to 5] & 3 & 3 \\
\hline
\multicolumn{4}{l}{\textbf{TCN-DPA}} \\
Learning rate & [1e-5 to 1e-2] & 0.0006 & 0.003 \\
Number of Layers & [1 to 5] & 4 & 4 \\
Kernel Size & [2 to 5] & 2 & 2 \\
Dropout & [$10^{-5}$ to 0.5] & 0.17 & 0.01 \\
StepLR Step Size & [10 to 50] & 21 & 17 \\
StepLR Gamma & [0.5 to 1] & 0.9 & 0.8 \\
Epochs & [0 to 200] & 156 & 100 \\
\hline
\multicolumn{4}{l}{\textbf{STA-MLP}} \\
Learning rate & [1e-5 to 1e-2] & 0.001 & 0.001 \\
Number of layers & [1 to 5] & 2 & 3 \\
Size of hidden layer 0 & [5 to 30] & 29 & 15 \\
Size of hidden layer 1 & [5 to 30] & 27 & 15 \\
Size of hidden layer 2 & [5 to 30] & N/A & 15 \\
Total training epochs & [50 to 500] & 133 & 300 \\
\hline
\multicolumn{4}{l}{\textbf{TRFI-MLP}} \\
Learning rate & [1e-5 to 1e-2] & 0.0004 & 0.001 \\
Number of layers & [1 to 5] & 3 & 3 \\
Size of hidden layer 0 & [5 to 30] & 23 & 15 \\
Size of hidden layer 1 & [5 to 30] & 29 & 15 \\
Size of hidden layer 2 & [5 to 30] & 21 & 15 \\
Total training epochs & [50 to 500] & 130 & 160 \\
\hline
\multicolumn{4}{l}{\textbf{LSTM-DPA-TA}} \\
Learning rate & [1e-5 to 1e-1] & 0.004 & 0.01 \\
LSTM size & [64 to 128] & 128 & 128 \\
StepLR step size & [1 to 50] & 35 & 10 \\
StepLR step gamma & [0.1 to 1] & 0.7 & 0.8 \\
Training epochs & [50 to 500] & 160 & 500 \\
\hline
\end{tabular}
\end{table}

Table~\ref{tab:all_hyperparameters} provides a summary of the best hyperparameters obtained for each model across the datasets. In addition to the parameters in the table, batch normalisation is applied to enhance model generalisation. The Adam optimiser and a batch size of 128 was used for TCN-DPA, LSTM-DPA-TA, STA-MLP and TRFI-MLP. For the CNN-Transformer model, we use the AdamW optimiser and manually set the batch size to 16. 


\section{Results}
 This section evaluates the performance of channel estimators using BER as the primary metric. We compare the effectiveness of two training approaches: one using a mixed SNR dataset and the other using a high SNR dataset.  The evaluated architectures are CNN-Transformer, TCN-DPA, STA-MLP, TRFI-MLP, and LSTM-DPA-TA.

\subsection{BER}
BER is a crucial performance measure in digital communication. It compares transmitted and received bit sequences to assess system reliability. In channel estimation, a lower BER means more accurate signal reconstruction and fewer errors in decoded data. The following analysis details the BER performance of each model at different levels of SNR, providing a direct comparison between the two training approaches.
%\subsection{BER}
\begin{figure}[ht!]
    \centering
    \begin{subfigure}[b]{0.75\linewidth}
        \centering
        \includegraphics[width=\linewidth]{images/last_ber_bad.png}
        \caption{High SNR training dataset}
        \label{fig:high_snr_ber}
    \end{subfigure}
    \hfill
    \begin{subfigure}[b]{0.75\linewidth}
        \centering
        \includegraphics[width=\linewidth]{images/last_BER.png}
        \caption{Mixed SNR training dataset}
        \label{fig:mixed_ber}
    \end{subfigure}
    \caption{Comparison of BER performance for various channel estimators trained on high SNR and mixed SNR datasets. \label{fig:ber}}
\end{figure}

%Figures \ref{fig:high_snr_ber}. and \ref{fig:mixed_ber}. 
Figure~\ref{fig:ber} presents the BER performance of the channel estimators trained using a high SNR dataset (40 dB) and a mixed SNR dataset, respectively. The ideal curve, represented by the dotted black line in these figures, represents the best possible performance in channel estimation, where the received signal is assumed to be perfectly equalised without any channel estimation errors. This curve serves as a theoretical lower bound for the BER, showing the performance in a scenario where only additive white Gaussian noise (AWGN) is present and the channel effects are perfectly compensated for. The closer the BER of a model is to the ideal curve, the better its performance in accurately estimating the channel and mitigating the effects of noise. LS serves as the lower performance baseline. 

In Figure \ref{fig:high_snr_ber}, we observe several key trends across models trained on an high SNR dataset: 
In the low SNR range (0-10 dB), the STA-MLP and the LSTM-DPA-TA models perform relatively well, maintaining a lower BER compared to other models. %The LSTM-DPA-TA model also exhibits strong performance, achieving the lowest BER across this range, demonstrating its suitability for low SNR conditions when trained on high SNR data.
%
As we move into the mid SNR range (15-25 dB), the CNN-Transformer and DPA-TCN models show an improvement, particularly from 15 dB onwards, with performance becoming more pronounced from 20 dB. %The improvement becomes more exponential from 20 dB. 
This improvement at high SNR highlights that these models have effectively adapted to high SNR conditions. %arnt to less noisy conditions but not high noisy conditions. 
%TRFI-MLP also exhibits a steady increase in BER with increasing SNR, although its performance lags behind that of CNN-Transformer and DPA-TCN. This also indicates the limitation of training NNs with the high SNR dataset as the TRFI-MLP struggles to adapt to noisy conditions.
The TRFI-MLP model also improves in this range, although it lags behind the CNN-Transformer and DPA-TCN models, indicating some limitations in its ability to adapt to low SNR conditions.
% 
In the high SNR range (30-40 dB), the CNN-Transformer and DPA-TCN models continue to show further significant reductions in BER. The LSTM-DPA-TA model continues to outperform other models across all SNR levels, showing its effectiveness in less noisy conditions. In contrast, STA-MLP shows limited improvement beyond 20 dB, indicating that it struggles to capitalise on the reduced noise at higher SNR levels.

In the mixed SNR training scenario depicted in Figure \ref{fig:mixed_ber}, we observe that CNN-Transformer and DPA-TCN models exhibit excellent performance across the entire SNR range with CNN-Transformer outperforming all models.  %This suggests that these models are well-suited for a wide range of channel conditions when trained on mixed SNR datasets.
%
The LSTM-DPA-TA and STA-MLP models perform well at lower SNR levels (0 to 15 dB) but show reduced performance as the SNR increases, indicating diminished capabilities in high SNR conditions when trained on mixed SNR data. We can also observe the improved performance of the TRFI-MLP estimator across the entire SNR range tested compared to its performance when trained on the high SNR dataset. The unsatisfactory performance by LSTM-DPA-TA and STA-MLP indicates that while these models can handle varying noise conditions, they may not leverage mixed SNR training as effectively as TRFI-MLP, CNN-Transformer and DPA-TCN.

Models trained on mixed SNR data exhibit lower BER even in low-SNR scenarios, compared to those trained on high SNR data. Mixed SNR training appears to be beneficial for models such as TRFI-MLP, TCN-DPA, and CNN-Transformer, which show a significant improvement in low SNR conditions than when trained on high SNR. In contrast, the LSTM-DPA-TA and STA-MLP models demonstrate better performance when trained on a high SNR dataset, suggesting that these models make use of the better channel statistics available at high SNR. This is analysed in more detail below.

\subsection{Difference in BER Between Models Trained on Mixed SNR and High SNR Datasets}
Figure~\ref{fig:delta_ber} illustrates the difference in BER between models trained on high SNR datasets and those trained on a mixed SNR dataset. The graph clearly shows how the training dataset influences the performance of various models across different SNR ranges. In the low SNR range (0-10 dB), the CNN-Transformer and TCN-DPA models exhibit the highest positive delta, peaking around 10 dB. This indicates a substantial improvement in performance when these models are trained on a mixed SNR dataset compared to a high SNR dataset. A positive delta suggests that mixed SNR training better equips these models to generalise in low SNR conditions. TRFI-MLP also shows a noticeable positive delta, although less pronounced, indicating that mixed SNR training offers some advantages in low SNR environments. In contrast, the LSTM-DPA-TA and STA-MLP models display a slightly negative delta across this range, implying that training on high SNR datasets offers a marginal performance advantage in these specific models. 

As the SNR increases (15-25 dB), the positive delta values for the CNN-Transformer and TCN-DPA models start to decrease but remain positive, particularly around 15 dB. This indicates that while the benefits of mixed SNR training diminish slightly, these models still gain performance advantages in this SNR range. TRFI-MLP continues to maintain positive delta values, demonstrating that mixed SNR training consistently improves its performance at varying SNR levels. Meanwhile, the LSTM-DPA-TA model continues to show a negative delta, particularly at the higher end of the mid-SNR range, around 20-25 dB. Similarly, the STA-MLP model exhibits a slight negative delta in this range.

In the high SNR range (30-40 dB), the delta values for most models, including the CNN-Transformer, TCN-DPA, and TRFI-MLP, approach zero. This suggests that as the SNR increases and the impact of noise decreases, the performance difference between models trained on high SNR and mixed SNR datasets becomes negligible. %The LS, STA-MLP, and LSTM-DPA-TA models exhibit very low or slightly negative delta values, indicating that their performance is largely unaffected by the training dataset's SNR range in higher SNR environments.
\begin{figure}[ht!]
    \centering
    \includegraphics[width=0.8\linewidth]{images/delta_ber.png}
    \caption{Difference in BER between models trained on mixed SNR and high SNR datasets.}
    \label{fig:delta_ber}
\end{figure}

\section{Conclusion} 
This study investigated the effectiveness of training NN-based channel estimators using mixed SNR datasets compared to high SNR datasets. Our results demonstrate that training with mixed SNR data significantly improves the generalisation of various estimators, especially in low SNR conditions. In particular, models such as the CNN-Transformer, DPA-TCN and TRFI-MLP exhibited substantial improvements in BER across the entire SNR range when trained on mixed SNR data compared to when trained on high SNR. Among the models tested, the CNN-Transformer, when trained on mixed SNR data, outperformed other estimators, including the current state-of-the-art LSTM-DPA-TA. However, it is important to note that some models, such as LSTM-DPA-TA and STA-MLP, showed reduced performance when trained on mixed SNR data. 
Mixed SNR training improves performance and generalisation across SNR levels for some models. The exact reason why only certain models benefit remains unclear and requires further investigation. 

These results indicate the importance of considering the SNR range as an important hyperparameter during training, rather than following the current practice of using only high SNR training data. The channel model used in this study is an example where the impact of mobility on the channel is more significant compared to other channels with lower mobility. As a result, the tested models should be able to generalise well to those channels. This will be verified in future research that will investigate the impact of mixed SNR training on other vehicular channel models.

\subsubsection{Acknowledgements}
The authors are grateful to NITheCS and the Telkom CoE at NWU for their support of this research.

 \bibliographystyle{splncs04}
 \clearpage
 \bibliography{references}

\end{document}






\subsection{Inference Efficiency}
\label{sec:efficiency}
Hexatagging \cite{amini-etal-2023-hexatagging}, which is utilizing sequential labeling, achieves a processing speed 10 times faster than the biaffine model because it does not require additional modules, and similarly to our model, DPSG \cite{lin2022dependency} also adopts a text-to-text approach on generative model.

As a result, although our approach has to increase sentence length due to added prompts, it obtains faster inference speed than other conventional methods because it relies solely on a pre-trained model. 

\begin{figure}[h]
     \centering
     % \includegraphics[width=0.45\textwidth]{figs/speed.pdf} % IEEE
     \includegraphics[width=0.8\textwidth]{figs/speed_.pdf}
     \caption{The complexity of the generative models in terms of running time in minutes. All the models are tested on \num{40000} samples.}
     \label{fig:speed}
 \end{figure}



\section{Conclusions}

In this paper, we introduce \textbf{SPT-DP}, as a structuralized prompt template-based dependency parsing method. Our approach enables text-to-text dependency parsing through prompt engineering by utilizing additional tokens while relying solely on pre-trained encoder models without requiring any additional modules. Despite relying solely on a pre-trained encoder model, our proposed method achieves performance comparable to existing models. Through experiments on the UD 2.2, we integrated dependency relation labels to develop a universal model applicable across 12 languages. This model not only enables multi-language dependency parsing within a single model but also demonstrates the ability to generalize to unseen languages to some extent. Finally, we applied our method to decoder-based models, demonstrating its applicability across different model types. Therefore, our method has several strong points; it can be easily applied to various pre-trained models appropriate for the target language or training environments, 
and it achieves fast inference speeds.

\section*{Limitations}
In our method, there is a limitation with sequence length. Although sentences with too many words are occurred in rare cases, additional prompts also increase linearly with the number of words, which can make it difficult to use for encoder models with a short maximum length. In addition, additional research is needed to perform semantic dependency parsing with a dynamic number of relationships.

\section*{Ethics Statement}
We perform dependency parsing using a pre-trained model. The datasets may contain ethical issues or biased sentences, but the model does not influence them through dependency parsing.

\bibliography{custom}

\appendix
\begin{table*}[h]
\resizebox{\textwidth}{!}{
\begin{tabular}{llll}
\hline
Notations & Components & Type & Description \\
\hline
\hline
$w_i$& - & String & Text representation of i'th word in sentence \\
$S$ & $w_i$ & String& Sentence \\
$H_i$& - & Integer& Index of the head word\\
$L_{(w_i,w_{H_i})}$& - &String & Dependency relation label between $w_i$$w_{H_i}$ \\
$r_i$& $H_i,L_{(w_i,w_{H_i})}$& Set & Dependency relation of $w_i$\\
$R_S$& $r_i$ & Set & Dependency relations of sentence S \\
$\mathcal{L}$& $L_{(w_i,w_{H_i})}$ & Set & Predefined dependency relation labels\\
$S_{dep}$& $w_i,r_i$ & Set & Sentence that include dependency parsing information\\
$P_{abs}$ & "[$1$]","[$2$]",\ldots,"[$n$]" & Set of strings & absolute index prompts that located at the first position of the prompt set. \\
$P_{ref}$ & "[$H_1$]","[$H_2$]",\ldots,"[$H_n$]" & Set of strings & reference index prompts that located in the second position of the prompt set.\\
$P_{idx}$ & \{$P_{abs}$, $P_{ref}$\} & Set of strings & String form of index prompts, indies are encapsulated with "[]" \\
$P_{\mathcal{L}}$ & "[acomp]", "[advcl]", \ldots , "[xcomp]" & Set of strings & Prompt tokens of Dependency relation labels , string of labels are encapsulated with "[]"\\
$P_{POS}$ & "[NN]", "[NNP]", \ldots , "[WRB]" & Set of strings & Prompt tokens of pos-tags $\mathcal{P}$, string of pos-tags are encapsulated with "[]"\\
$P_{MASK}$ & "[HEAD]","[DEP]" & Set of strings & Masking prompts for training\\
$D$ &$P_{idx}$,$P_{\mathcal{L}}$,$P_{POS}$,$S$ & String & Structuralized Prompt Template\\ 
$D_M$ & $P_{MASK}$,$P_{POS}$,$S$ & String & Masked Structuralized Prompt Template \\
\hline
\end{tabular}}
\caption{Notations}
\label{tab:notation}
\end{table*}

\section{Notations}

\section{Implementation Details}

For experiments for PTB, xlnet-large-cased\footnote{\url{https://huggingface.co/xlnet-large-cased}} are used. For experiments for UD 2.2, bert-multilingual-cased\footnote{\url{https://huggingface.co/bert-base-multilingual-cased}}, xlm-roberta-large\footnote{\url{https://huggingface.co/FacebookAI/xlm-roberta-large}} and xlnet-large-cased are used. For decoder-based models, T5-base\footnote{\url{https://huggingface.co/google-t5/t5-base}}, Bart-large\footnote{\url{https://huggingface.co/facebook/bart-large}}, Qwen2.5-3B\footnote{\url{https://huggingface.co/Qwen/Qwen2.5-3B}}, Llama3.2-3B\footnote{\url{https://huggingface.co/meta-llama/Llama-3.2-3B}} are used. For the Korean Sejong dataset, RoBERTa-large\footnote{\url{https://huggingface.co/klue/roberta-large}}, a pre-trained model for the Korean language, is used. Experiments are conducted on an NVIDIA RTX A6000. The models are fine-tuned with a batch size of 8, a learning rate of 1e-5, and 10 training epochs. Training is performed using the linear scheduler and AdamW optimizer.

% \section{Data statistics}

% \begin{table}[h]
\resizebox{\columnwidth}{!}{
\begin{tabular}{cccc}
\hline
 &
  \begin{tabular}[c]{@{}c@{}}Data\\ Count\end{tabular} &
  \begin{tabular}[c]{@{}c@{}}Processed\\ Data Count\end{tabular} &
  \begin{tabular}[c]{@{}c@{}}Entries\\ Removed\end{tabular} \\ \hline
Dataset & \multicolumn{3}{c}{bg\_btb} \\ \hline
Train   & 39832     & 39829    & 3    \\ \hline
Dev     & 1700      & 1699     & 1    \\ \hline
Test    & 2146      & 2146     & 0    \\ \hline
\end{tabular}}

\caption{Data split for PTB dataset}
\end{table}
% \begin{table}[h]
\resizebox{\columnwidth}{!}{
\begin{tabular}{ccccccc}

\hline
 &
  \begin{tabular}[c]{@{}c@{}}Data\\ Count\end{tabular} &
  \begin{tabular}[c]{@{}c@{}}Processed\\ Data Count\end{tabular} &
  \begin{tabular}[c]{@{}c@{}}Entries\\ Removed\end{tabular} &
  \begin{tabular}[c]{@{}c@{}}Data\\ Count\end{tabular} &
  \begin{tabular}[c]{@{}c@{}}Processed\\ Data Count\end{tabular} &
  \begin{tabular}[c]{@{}c@{}}Entries\\ Removed\end{tabular} \\ \hline
Dataset & \multicolumn{3}{c}{bg\_btb}    & \multicolumn{3}{c}{ca\_ancora}    \\ \hline
Train   & 8907      & 8907      & 0      & 13123       & 13087      & 36     \\ \hline
Dev     & 1115      & 1115      & 0      & 1709        & 1702       & 7      \\ \hline
Test    & 1116      & 1116      & 0      & 1846        & 1837       & 9      \\ \hline
Dataset & \multicolumn{3}{c}{cs\_pdt}    & \multicolumn{3}{c}{de\_gsd}       \\ \hline
Train   & 68495     & 68484     & 11     & 13814       & 13810      & 4      \\ \hline
Dev     & 9270      & 9269      & 1      & 799         & 799        & 0      \\ \hline
Test    & 10148     & 10146     & 2      & 977         & 977        & 0      \\ \hline
Dataset & \multicolumn{3}{c}{en\_ewt}    & \multicolumn{3}{c}{es\_ancora}    \\ \hline
Train   & 12543     & 12538     & 5      & 14305       & 14287      & 18     \\ \hline
Dev     & 2002      & 2002      & 0      & 1654        & 1653       & 1      \\ \hline
Test    & 2077      & 2077      & 0      & 1721        & 1717       & 4      \\ \hline
Dataset & \multicolumn{3}{c}{fr\_gsd}    & \multicolumn{3}{c}{it\_isdt}      \\ \hline
Train   & 14554     & 14542     & 12     & 13121       & 13098      & 23     \\ \hline
Dev     & 1478      & 1477      & 1      & 564         & 564        & 0      \\ \hline
Test    & 416       & 416       & 0      & 482         & 481        & 1      \\ \hline
Dataset & \multicolumn{3}{c}{nl\_alpino} & \multicolumn{3}{c}{no\_bokmaal}   \\ \hline
Train   & 12269     & 12269     & 0      & 15696       & 15696      & 0      \\ \hline
Dev     & 718       & 718       & 0      & 2410        & 2410       & 0      \\ \hline
Test    & 596       & 596       & 0      & 1939        & 1939       & 0      \\ \hline
Dataset & \multicolumn{3}{c}{ro\_rrt}    & \multicolumn{3}{c}{ru\_syntagrus} \\ \hline
Train   & 8043      & 8032      & 11     & 48814       & 48803      & 11     \\ \hline
Dev     & 752       & 750       & 2      & 6584        & 6579       & 5      \\ \hline
Test    & 729       & 724       & 5      & 6491        & 6490       & 1      \\ \hline
\end{tabular}}

\caption{Data split for UD2.2 dataset}
\end{table}


\section{Efficiency test}
For the efficiency test between Hexatagging and SPT-DP, we used the PTB test dataset to evaluate the speed of dependency parsing. For the efficiency test between DPSG and SPT-DP, we used the UD 2.2-bg dev dataset to evaluate the speed of dependency parsing. We set the batch size to 1 and conducted the experiment under the same conditions using a single A6000 GPU.

\section{Decoder-based Model}
Unlike encoders, the decoder-based model required constrained generation. During the inference stage, contents other than $P_{ref}$ and $P_{\mathcal{L}}$ were forcibly inserted into the sequence at intervals, allowing the model to perform accurate dependency parsing in a restricted environment.


\section{Analysis : Ablation Study for Prompts}
\label{sec:attention}
To elaborate further, in experiments without $P_{abs}$, the order of the template implicitly replaced $P_{abs}$ and was used for prediction.
The experimental results (Table \ref{tab:prompt result}) indicate that the explicit presence of $P_{abs}$ (physically exists), allowing for direct reference, plays a crucial role in dependency parsing through attention.
In Figure \ref{fig:heatmap}, presents a heatmap representation of attention scores across layers and the cosine similarity of each hidden state. The attention scores on the left show that as the layers progress, the values converge, with the token in $P_{ref}$ exhibiting high attention scores toward $P_{abs}$ , which it is supposed to reference.

\begin{figure}[h]
 \includegraphics[width=\columnwidth]{image/heatmap.png}
 \caption{A heatmap of Attention scores and cosine similarities in hidden layer}
 \label{fig:heatmap}
\end{figure}

\section{Licenses}
The PTB dataset is licensed under LDC User Agreement. The UD 2.2 dataset is licensed under the Universal Dependencies License Agreement.


\end{document}
