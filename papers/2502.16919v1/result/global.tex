\begin{table*}[h]
 \setlength{\abovecaptionskip}{-4pt}
	\centering
	{ 
		\makebox[\linewidth]{\resizebox{\linewidth}{!}{%
 \begin{tabular}{c|cccccccccccc|c}\hline
  & 	bg& 	 ca&  cs&  de&  en  &  es  &  fr&  it  & nl  & no  & ro  & ru&  Avg. \\ \hline\hline
base & 93.11  &  92.54   &  \textbf{94.14}   &  82.11   &  88.50   &  91.69   &  88.02   &  93.16   &  91.15   &  \textbf{93.13}   &  \textbf{88.87}   & \textbf{95.12}  & 90.90   \\

\hline
global & \textbf{93.91} & \textbf{92.85} & 93.27 & \textbf{82.91} & \textbf{89.27} & \textbf{92.03} & \textbf{89.96} & \textbf{93.24} & \textbf{92.28} & 92.81& 88.84 & 94.39 & \textbf{91.31} \\
unseen & 81.12 & 88.49 & 79.83 & 75.80 &  72.59 & 87.26 & 81.30 & 79.63& 78.17 & 77.86 & 77.80 & 77.05 & 79.74\\
\hline
\end{tabular}}}
 }
 \vspace{10pt}
\caption{LAS scores for the UD2.2 dataset. The first row lists the languages in the test set. \textbf{base} represents the performance of a model trained individually for each language. \textbf{global} represents the performance of a single multilingual model trained on all languages. \textbf{unseen} represents the performance of individual models trained on all languages except for the test language, evaluating zero-shot transfer performance.}
	\label{tab:global} 
\vskip -.14in
\end{table*}