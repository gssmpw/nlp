\section{Problem Definition}
\label{sec:lin}

In this section, we formally define the semantics of the replicated data store
on top of which the MRDT implementations operate (\S \ref{subsec:os}),
the notion of RA-linearizability for MRDTs (\S \ref{subsec:lin_def}), and the process of bottom-up linearization (\S \ref{subsec:bottom-up}).

\subsection{Semantics of the Replicated Data Store}
\label{subsec:os}

\begin{figure}[ht]
	\scriptsize
	\raggedright$\textrm{\sc{[CreateBranch]}}$
	\[
	\inferrule{r\in dom (H) \\ r'\notin dom (H) \\  v\notin dom (N) \\
		N' = N [v \mapsto N(H(r))] \\ H' = H[r' \mapsto v] \\
		L' = L [v \mapsto L(H(r))] \\ G' = (dom(N) \cup \{v\}, {E} \cup \{(H(r), v)\})}
	{(N, H, L, G, vis) \xrightarrow{\F{createBranch(r',r)}} (N', H', L', G', vis)\\}
	\]
		\raggedright$\textrm{\sc{[Apply]}}$
	\[
	\inferrule{e = (t,r,o) \\  o \in O_\tau \\  \forall e' \in \bigcup range(L).\ time (e') \neq t  \\ r\in dom (H) \\ v\notin dom (N) \\
		N' = N [v \mapsto \F{do} (N(H(r)), e)] \\
		H' = H [r \mapsto v] \\ L' = L [v \mapsto L(H(r)) \cup \{e\}] \\
		G' = (dom(N'), {E} \cup \{(H(r), v)\}) \\
		vis' = vis \cup (L (H(r)) \times \{e\})}
	{(N, H, L, G, vis) \xrightarrow{\F{apply(t,r,o)}} (N', H', L', G', vis')\\}
	\]

		\raggedright$\textrm{\sc{[Merge]}}$
	\[
	\inferrule{r_{1}, r_{2} \in dom (H) \\ v\notin dom (N)  \\ v_{\top} = LCA (H(r_{1}), H(r_{2})) \\
		N' = N [v \mapsto \F{merge} (N(v_{\top}), N(H(r_{1})), N(H(r_{2}))] \\
		H' = H [r_{1} \mapsto v] \\ L' = L [v \mapsto L(H(r_{1})) \cup L(H(r_{2}))] \\
		G' = (dom(N'), {E} \cup \{(H(r_{1}), v), (H(r_{2}), v)\})}
	{(N, H, L, G, vis) \xrightarrow{\F{merge (r_1, r_2)}} (N', H', L', G', vis)\\}
	\]
	\raggedright$\textrm{\sc{[Query]}}$
	\[
	\inferrule{r\in dom (H) \\ q \in Q_\tau \\ a = \F{query}(N(H(r)),q)} 
	{(N, H, L, G, vis) \xrightarrow{\F{query(r,q,a)}} (N, H, L, G, vis)\\}
	\]
	\vspace{-2em}
	\caption{Semantics of the replicated datastore}
	\vspace{-3.25em}
	\label{fig:sem}
\end{figure}

The semantics of the replicated store defines all possible executions of an
MRDT implementation.  Formally, the semantics are parameterized by an MRDT
implementation $\M{D} = \langle \Sigma, \sigma_0, \F{do}, \F{merge}, \F{query},\\
\F{rc}\rangle$ of type $\tau = \langle O_\tau, Q_\tau, Val_\tau \rangle$ and are represented by a labeled transition system
$\M{S_\M{D}}$ = ($\Phi$, $\rightarrow$). Each configuration in $\Phi$ maintains a set of
versions, where each version is created either by applying an MRDT operation to
an existing version, or by merging two versions. Each replica is associated
with a head version, which is the most recent version seen at the replica.
Formally, each configuration $C$ in $\Phi$ is a tuple $\langle N, H, L, G, vis
\rangle$, where:
 
\begin{itemize}
	\item $N : \F{Version} \rightharpoonup \Sigma$ is a partial function
that maps versions to their states ($\F{Version}$ is the set of all possible
versions). 
	\item $H: \M{R} \rightharpoonup \F{Version}$ is also a partial function
that maps replicas to their head versions. A replica is considered active if it 
is in the domain of $H$ of the configuration.
	\item $L: \F{Version} \rightharpoonup \mathbb{P}(\M{E})$ maps a version to the set of events 
that led to this version. Each event $e \in \M{E}$ is an update
operation instance, uniquely identified by a timestamp value (we define $\M{E}
= \M{T} \times \M{R} \times O$).
	\item $G = (dom(N),E)$ is the version graph, whose
vertices represent the versions in the configuration (i.e. those in the domain of
$N$) and whose edges represent a relationship between different versions (we explain the different types of edges below).
	\item $vis \subseteq \M{E} \times \M{E}$ is a partial order over events.
\end{itemize}


Figure~\ref{fig:sem} gives a formal description of the transition rules. $\textrm{\sc{CreateBranch}}$ forks a new replica $r'$ from an existing replica
$r$, installing a new version $v$ at $r'$ with the same state as the head
version $H(r)$ of $r$, and adding an edge $(H(r),v')$ in the version graph.
$\textrm{\sc{Apply}}$ applies an update operation $o$ on some replica $r$,
generating a new event $e$ with a timestamp different than all events generated
so far. $\bigcup\F{range}(L)$ denotes the set of events witnessed across all versions.
A new version $v$ is also created whose state is obtained by applying
$o$ on the current state of the replica $r$. The version graph is updated by
adding the edge $(H(r),v)$. The $vis$ relation as well as the function
$L$, which tracks events applied at each version, are also updated. In
particular, each event $e'$ already applied at $r$, i.e. $e' \in L(H(r))$, is
made visible to $e$: $(e',e) \in vis$, while $L'(v)$ is obtained by adding $e$
to $L(H(r))$.

$\textrm{\sc{Merge}}$ takes two replicas $r_1$ and $r_2$, applies the $\F{merge}$
function on the states of their head versions to generate a new version $v$, which is
installed as the new head version at $r_1$. Edges are added in the version graph
from the previous head versions of $r_1$ and $r_2$ to $v$. $L(v)$ is obtained by
taking a union of $L(r_1)$ and $L(r_2)$, and there is no change in the visibility
relation. $\textrm{\sc{Query}}$ takes a replica $r$ and a query operation $q$ and applies $q$ to the state at the head version of 
$r$, returning an output value $a$. Note that the $\textrm{\sc{Query}}$ transition 
does not modify the configuration and the return value of the query is stored as part of the transition label. While our operational semantics is based on and inspired by previous works 
\cite{Kaki2022,Vimala}, we note that it is more general and precisely
captures the MRDT system model as opposed to previous
works. In particular, \citet{Kaki2022} place
significant restrictions on the $\textrm{\sc{Merge}}$ transition, disallowing
arbitrary replicas to be merged to ensure that there is a total order on the
merge transitions. While the semantics in the work by \citet{Vimala}
does allow arbitrary merges, it is more abstract and high-level, and does not
even keep track of versions and the version graph. 

\textbf{Notation:} We now introduce some notation that will be used throughout the paper. Given a configuration $C$, we use $X(C)$ to project the component $X$ of $C$. For a relation $R$, we use $x \xrightarrow{R}
y$ to signify that $(x,y) \in R$. We use $R_{\mid S}$ to indicate the relation
as given by $R$ but restricted to elements of the set $S$. Let $R^*$ denote the
reflexive-transitive closure of $R$, and let $R^+$ denote the
transitive closure of $R$. For an event $e$, we use the projection functions
$\F{op}, \F{time}, \F{rep}$ to obtain the update operation, timestamp and replica
resp. For a sequence of events $\pi$, $\pi_{\mid S}(\sigma)$ denotes
application of the sub-sequence of $\pi$ restricted to events in $S$. For a
configuration $C$, we use $e_1 \mid\mid_C e_2$ to denote that $e_1$ and $e_2$
are concurrent, that is $\neg (e_1 \xrightarrow{\F{vis}(C)} e_2 \vee e_2
\xrightarrow{\F{vis}(C)} e_1)$. Given a total order over a set of events $\M{E}$,
represented by a sequence $\pi$, and $\F{lo} \subseteq \M{E} \times \M{E}$, we say that
$\pi$ extends $\F{lo}$ if $\F{lo} \subseteq \pi$. The relation $\F{rc}$ orders
update operations, but for convenience we sometime use it for ordering events, with
the intention that it is actually being applied to the underlying update operations.
We use $e_1 \neq e_2$ to indicate that $ \F{time}(e_1) \neq  \F{time}(e_2)$.


We define the initial configuration of $\M{S_\M{D}}$ as $C_0 =
\langle N_0, H_0, L_0, G_0, \emptyset \rangle$, which consists of only one
replica $r_0$.  Here, $H_0 = [r_0 \mapsto v_0]$, $N_0 = [v_0 \mapsto
\sigma_0]$, where $\sigma_0$ is the initial state as given by
$\M{D_{\tau}}$, while $v_0$ denotes the initial version and $L_0 = [v_0
\mapsto \emptyset]$. The graph $G_0 = (\{v_0\}, \emptyset)$ is the initial
version graph. An execution of $\M{S_D}$ is defined to be a finite sequence of
transitions, $C_0 \xrightarrow{t_1} C_1 \xrightarrow{t_2} C_2 \ldots
\xrightarrow{t_n} C_n$. Note that the label of a transition corresponds to its
type. 
Let $\llbracket \M{S_\M{D}} \rrbracket$ denote the set of all possible
executions of $\M{S_D}$.

Finally, as mentioned earlier, $\F{merge}$ is a ternary function, taking as
input the states of two versions to be merged, and the state of the lowest
common ancestor (LCA) of the two versions.
Version $v_1 \in V$ is defined to be a causal ancestor of version $v_2 \in V$
if and only if  $(v_1, v_2) \in E^*$.


\begin{definition}[LCA]
	Given a version graph $G = (V,E)$ and versions $v_1, v_2 \in V$, $v_\top \in
	V$ is defined to be the lowest common ancestor of $v_1$ and $v_2$ (denoted by
	$LCA(v_1,v_2)$) if (i) $(v_\top,v_1) \in E^*$ and $(v_\top,v_2) \in E^*$,
	(ii) $\forall v \in V. (v,v_1) \in E^* \wedge (v,v_2) \in E^* \implies
	(v,v_\top) \in E^*$.
\end{definition}

Note that the version history graph at any point in any execution is guaranteed
to be acyclic (i.e. a DAG),  and hence the LCA (if it exists) is guaranteed to
be unique. We now present an important property linking the LCA of two versions
with events applied at each version.

\begin{lemma}\label{lem:LCA}
	Given a configuration $C = \langle N,H,L,G,vis \rangle$ reachable in some
	execution $\tau \in \llbracket \M{S_D} \rrbracket$ and two versions $v_1,v_2
	\in dom(N)$, if $v_\top$ is the LCA of $v_1$ and $v_2$ in $G$, then
	$L(v_\top) = L(v_1) \cap L(v_2)$\footnote{All proofs are in the Appendix \S \ref{sec:app}}.
\end{lemma}

\begin{wrapfigure}{r}{0.3\textwidth}
	\vspace{-1.5em}
	\begin{center}
		\includegraphics[angle=0, width=0.8\linewidth]{LCA}
	\end{center}
	\vspace{-1em}
	\caption{Version Graph with no LCA for $v_5$ and $v_6$}
	\label{fig:LCA}
	\vspace{-1em}
\end{wrapfigure}

Thus, the events of the LCA are exactly those applied at both the versions.
This intuitively corresponds to the fact that $LCA(v_1,v_2)$ is the most recent
version from which the two versions $v_1$ and $v_2$ diverged. Note that it is possible that the LCA may not exist for two versions. Fig.
\ref{fig:LCA} depicts the version graph of such an execution. Vertices with
in-degree 1 (i.e. $v_1,v_2,v_3,v_4$) have been generated by applying a new update
operation (with the orange edges labeled by the corresponding events $e_1,e_2,e_3,e_4$),
while vertices with in-degree 2 have been obtained by merging two other
versions (depicted by blue edges). The merge of $v_1$ and $v_4$ (leading to
$v_6$) has a unique LCA $v_0$, similarly, merge of $v_2$ and $v_3$ (leading to
$v_5$) also has a unique LCA $v_0$. However, if we now want to merge $v_5$ and
$v_6$, both $v_1$ and $v_2$ are ancestors, but there is no LCA. We note that
this execution will actually be prohibited by the semantics of
\citet{Kaki2022}, since the two merges leading to $v_5$ and $v_6$ are
concurrent.

Notice that $L(v_5) = \{e_1,e_2,e_3\}$, while $L(v_6) = \{e_1,e_2,e_4\}$.
Hence, by Lemma~\ref{lem:LCA}, $L(LCA(v_5,v_6)) = \{e_1,e_2\}$, but such a
version is not generated during the execution. To resolve this issue, we introduce the notion of \textit{potential} LCAs. 

\begin{definition}[Potential LCAs]
Given a version graph $G = (V,E)$ and versions $v_1, v_2 \in V$, $v_\top \in
	V$ is defined to be a potential LCA of $v_1$ and $v_2$  if 
	(i) $(v_\top,v_1) \in E^*$ and $(v_\top,v_2) \in E^*$,
	(ii) $\neg (\exists v. (v,v_1) \in E^* \wedge (v,v_2) \in E^* \wedge (v_\top,v) \in E^*)$.
\end{definition}

For merging $v_1$ and $v_2$, we first find all the potential LCAs, and recursively merge them to obtain the actual
LCA state. For the execution in Fig. \ref{fig:LCA}, the potential LCAs of $v_5$ and $v_6$ would be
$v_1$ and $v_2$ (with $L(v_1) = \{e_1\}$ and
$L(v_2) = \{e_2\}$); merging them would get us the actual LCA. 
  In \S \ref{subsec:lcaproof}, we prove that this recursive merge-based strategy is guaranteed to generate the
actual LCA.

\subsection{Replication-aware Linearizability for MRDTs}
\label{subsec:lin_def}

As mentioned in \S \ref{sec:overview}, our goal is to show that the state of every version $v$
generated during an execution is a linearization of the events in $L(v)$. We
use the notation $\F{lo}$ to indicate the linearization relation, which is a
binary relation over events. For an execution in $\M{S}_\M{D}$, we
want $\F{lo}$ between the events of the execution to satisfy certain desirable
properties: (i) $\F{lo}$ between two events should not change during an execution, (ii) $\F{lo}$ should obey the conflict resolution policy
for concurrent events and (iii) $\F{lo}$ should obey the replica-local
$\F{vis}$ ordering for non-concurrent events. This would ensure that two
versions which have observed the same set of events will have the same state (i.e. \textit{strong eventual consistency}), and this state would
also be a linearization of update operations of the data type satisfying the
conflict resolution policy.

While the $\F{lo}$ relation in classical linearizability literature is
typically a total order, in our context, we take advantage of commutativity
of update operations, and only define $\F{lo}$ over non-commutative events. As we
will see later, this flexibility allows us to have different sequences of
events which extend the same $\F{lo}$ relation between non-commutative events, and hence are guaranteed
to lead to the same state. We use the notation $e \rightleftarrows e'$ to
indicate that events $e$ and $e'$ commute with each other. Formally, this means
that $\forall \sigma.\;e(e'(\sigma)) = e'(e(\sigma))$. Two update operations
$o,o'$ commute if $\forall e,e'.\;\F{op}(e) = o \wedge \F{op}(e') = o'
\implies e \rightleftarrows e'$.  As mentioned earlier, the $\F{rc}$ relation
is also only defined between non-commutative update operations.

\begin{lemma}\label{lem:non-comm}
	Given a set of events $\M{E}$, if $\F{lo} \subseteq \M{E} \times \M{E}$ is defined over
	every pair of non-commutative events in $\M{E}$, then for any two sequences
	$\pi_1, \pi_2$ which extend $\F{lo}$, for any state $\sigma$, $\pi_1(\sigma)
	= \pi_2(\sigma)$.
\end{lemma}

Given a configuration $C = \langle N, H, L, G, vis \rangle$, let $\M{E}_C = \bigcup
\F{range}(L(C))$ denote the set of events witnessed across all versions in C.
Then, our goal is to define an appropriate linearization relation $\F{lo}_C
\subseteq \M{E}_C \times \M{E}_C$, which adheres to the $\F{rc}$ relation for concurrent
events, the $\F{vis}$ relation for non-concurrent events, and for every version $v
\in dom(N)$, $N(v)$ should be obtained by sequentializing the events in $L(v)$,
with the sequence extending $\F{lo}_C$. Note that this requires $\F{lo}^+$ to
be irreflexive\footnote{$\F{lo}$ need not be transitive, as we only want to
define $\F{lo}$ between non-commutative events, and non-commutativity is not a
transitive property}.

We now demonstrate that an $\F{lo}$ relation with all the desirable properties
may not exist for all executions. Suppose there are MRDT update operations $o,o'$ such
that $o \xrightarrow{\F{rc}} o'$. Fig. \ref{fig:conditional-commutativity}
contains a part of the version graph generated during some execution,
containing two instances of both $o$ and $o'$. We use $e_i:o_i$ to denote that
event $\F{op}(e_i) = o_i$. Notice that $e_1$ and $e_4$,
$e_2$ and $e_3$ are concurrent, while $e_1$ and $e_3$, $e_2$ and $e_4$ are
non-concurrent. Applying the $\F{rc}$ ordering on concurrent events, we would
want $e_3 \xrightarrow{\F{lo}} e_2$ and $e_4 \xrightarrow{\F{lo}} e_1$, while
applying $\F{vis}$ ordering, we would want $e_1 \xrightarrow{\F{lo}} e_3$ and
$e_2 \xrightarrow{\F{lo}} e_4$. However, this results in a $\F{lo}$-cycle, thus
making it impossible to construct a sequence of update operations for the merge of
$v_5$ and $v_6$, which adheres to the $\F{lo}$ ordering.

\begin{wrapfigure}{r}{0.3\textwidth}
	\vspace{-1em}
	\begin{center}
		\includegraphics[angle=0, width=0.8\linewidth]{conditional-commutativity}
	\end{center}
	\vspace{-1em}
	\caption{Example demonstrating cycle in $\F{lo}$}
	\vspace{-1.5em}
	\label{fig:conditional-commutativity}
\end{wrapfigure}

Notice that the above execution only requires the $\F{rc}$ relation to be
non-empty (i.e. there should exist some $(o,o') \in \F{rc}$). If the $\F{rc}$
relation is empty, then all update operations would commute with each other, and hence
the $\F{lo}$ relation would also be empty. If $\F{rc}$ is non-empty, $\F{rc}^+$
should be irreflexive to ensure irreflexivity of $\F{lo}^+$. Note that
$\F{rc}^+$ being irreflexive means that for any MRDT update operation $o$, $(o,o)
\notin \F{rc}$, and hence $o$ must commute with itself, since $\F{rc}$ relation
is defined for all pairs of non-commutative update operations. Furthermore, Fig.
\ref{fig:conditional-commutativity} shows that even if $\F{rc}^+$ is
irreflexive, it may still not be possible to construct an $\F{lo}$ relation
which can be extended to a total order and which adheres to the $\F{rc}$ relation
between all pairs of concurrent events. To ensure existence of an $\F{lo}$
relation such that $\F{lo}^+$ is irreflexive when $\F{rc}^+$ is irreflexive, we define it as follows:

\begin{definition}[Linearization relation]\label{def:lin-relation}
	Let $C$ be a configuration reachable in some execution in $\llbracket \M{S_D}
	\rrbracket$. Let $\M{E}_C$ be the set of events in $C$. Then, $\F{lo_C}$ is defined as:
	 \begin{align*}
		\forall e_1,e_2 \in \M{E}_C.\ e_1 \xrightarrow{\F{lo_C}} e_2  \Leftrightarrow &
		(e_1 \xrightarrow{\F{vis(C)}} e_2 \wedge \neg e_1 \rightleftarrows e_2) \\
		& \vee (e_1 \mid\mid_C e_2 \wedge e_1 \xrightarrow{\F{rc}} e_2 \wedge
		\neg(\exists e_3 \in \M{E}.\ e_2 \xrightarrow{\F{vis(C)}} e_3 \wedge \neg e_2
		\rightleftarrows e_3 ))
	\end{align*}
\end{definition}

$\F{lo}_C$ follows the visibility relation only between non-commutative
events. For concurrent non-commutative events $e_1$ and $e_2$ with $e_1
\xrightarrow{\F{rc}} e_2$, $\F{lo}_C$ follows the $\F{rc}$ relation only if
there is no event $e_3$ such that $e_2$ is visible to $e_3$ and $e_2$ does not commute with $e_3$. Applying this
definition to the execution in Fig. \ref{fig:conditional-commutativity}, for
the configuration obtained after merge, we would have neither $e_4
\xrightarrow{\F{lo}} e_1$, nor $e_3 \xrightarrow{\F{lo}} e_2$, thus avoiding
the cycle in $\F{lo}$.

\begin{lemma}\label{lem:irreflexive}
For an MRDT $\M{D}$ such that $\F{rc}^+$ is irreflexive, for any configuration $C$ reachable in
	$\M{S}_\M{D}$, $\F{lo}_C^+$ is irreflexive.
\end{lemma}

Going forward, we will assume that $\F{rc}^+$ is irreflexive for any MRDT $\M{D}$. 
We note that restricting $\F{lo}$ to not always obey the $\F{rc}$ relation by considering 
non-commutative update operations happening locally (and thus related by $\F{vis}$) is also
sensible from a practical perspective. For example, in the case of OR-set, even
though we have $\F{rem}_a \xrightarrow{\F{rc}} \F{add}_a$, if $\F{add}_a$ is
locally followed by another $\F{rem}_a$, it does not make sense to order a
concurrent $\F{rem}_a$ event before the $\F{add}_a$ event. More generally, if
an event $e_2$ is visible to another event $e_3$ with which it does not commute,
then $e_2$ is effectively "overwritten" by $e_3$, and hence there is no need to
linearize a concurrent event $e_1$ before $e_2$.

While $\F{lo}_C$ is now guaranteed to be irreflexive for any configuration $C$,
and hence can be extended to a sequence, it now no longer enforces an ordering
among all non-commutative pairs of events. Thus, there could exist sequences
$\pi_1,\pi_2$ extending an $\F{lo}_C$ relation which may contain a pair of
non-commutative events in different orders. For example, in Fig.
\ref{fig:conditional-commutativity}, for the configuration $C$ obtained after
the merge, $\F{lo}_C = \{(e_1,e_3), (e_2,e_4)\}$, resulting in sequences $\pi_1
= e_1 e_2 e_3 e_4$ and $\pi_2 = e_1 e_3 e_2 e_4$ which both extend $\F{lo}_C$,
but contain the non-commutative events $e_2$ and $e_3$ in different orders.
Thus, Lemma \ref{lem:non-comm} can no longer be applied, and it is not
guaranteed that $\pi_1$ and $\pi_2$ would lead to the same state. Notice that
in the sequences $\pi_1$ and $\pi_2$ above, even though $e_2$ and $e_3$ appear
in different orders, $e_4$ always appears after both. Indeed, $e_4$ must appear
after $e_2$ due to visibility relation, and since $e_3$ and $e_4$ commute with
each other (since both correspond to the same operation $o$), it is enough to
consider sequences where $e_4$ appears after $e_3$. Based on the above
observation, we now introduce a notion called conditional commutativity to
ensure that sequences such as $\pi_1,\pi_2$ would lead to the same state:

\begin{definition}[Conditional Commutativity]
	Events $e$ and $e'$ are said to conditionally commute with respect to event
	$e''$ (denoted by $e \overset{e''}{\rightleftarrows} e'$) if $\forall \sigma
	\in \Sigma.\ \forall \pi \in \M{E}^*.\ e''(\pi(e(e'(\sigma)))) =
	e''(\pi(e'(e(\sigma))))$.
\end{definition}

Update operations $o$ and $o'$ conditionally commute w.r.t. update operation $o''$ if
$\forall e,e',e''. \F{op}(e) = o \wedge \F{op}(e') = o' \wedge \F{op}(e'') =
o'' \Rightarrow e \overset{e''}{\rightleftarrows} e'$. For example, for the
OR-set MRDT of Fig. \ref{fig:orset_impl}, $\F{add}_a
\overset{\F{rem}_a}{\rightleftarrows} \F{rem}_a$. Even though \textit{add} and
\textit{remove} operations of the same element do not commute with each other,
if there is guaranteed to be a future \textit{remove} operation, then they do
commute. For the execution in Fig. \ref{fig:conditional-commutativity}, if
$e_2$ and $e_3$ conditionally commute w.r.t. $e_4$, then both the sequences
$\pi_1$ and $\pi_2$ will lead to the same state. For non-commutative update operations
that are not ordered by $\F{lo}$, we enforce their conditional commutativity
through the following property:
\begin{align*}
	\textrm{\sc{cond-comm}}(\M{D}) &  \triangleq  \forall o_1,o_2,o_3 \in O.\
	(o_1 \xrightarrow{\F{rc}} o_2 \wedge \neg o_2 \rightleftarrows o_3)
	\Rightarrow o_1 \overset{o_3}{\rightleftarrows} o_2
\end{align*}
$\textrm{\sc{cond-comm}}(\M{D})$ is a property of an MRDT $\M{D}$, enforcing
conditional commutativity of update operations $o_1$ and $o_2$ w.r.t. $o_3$ if $o_2$
does not commute with $o_3$. Connecting this with the definition of
linearization relation, if there are events $e_1,e_2,e_3$ performing operations
$o_1,o_2,o_3$ resp., and if $e_1 \xrightarrow{\F{rc}} e_2$, $e_2
\xrightarrow{\F{vis}} e_3$ and $\neg e_2 \rightleftarrows e_3$, then there
will not be a linearization relation between $e_1$ and $e_2$. However,
$\textrm{\sc{cond-comm}}(\M{D})$ would then ensure that the ordering of $e_1$
and $e_2$ will not matter, due to the presence of the event $e_3$. We also
formalize the requirement of an $\F{rc}$ relation between all pairs of
non-commutative update operations:
\begin{align*}
	\textrm{\sc{rc-non-comm}}(\M{D}) & \triangleq \forall o_1,o_2 \in O.\neg o_1
	\rightleftarrows o_2 \Leftrightarrow o_1 \xrightarrow{\F{rc}} o_2 \vee  o_2
	\xrightarrow{\F{rc}} o_1 \\
\end{align*}
\vspace{-3em}
\begin{lemma}\label{lem:convergence}
	For an MRDT $\M{D}$ which satisfies $\textrm{\sc{rc-non-comm}}(\M{D})$ and
	$\textrm{\sc{cond-comm}}(\M{D})$, for any reachable configuration $C$ in
	$\M{S}_\M{D}$, for any two sequences $\pi_1,\pi_2$ over $\M{E}_C$ which extend
	$\F{lo}_C$, for any state $\sigma$, $\pi_1(\sigma) = \pi_2(\sigma)$.
\end{lemma}


\begin{definition}[RA-linearizability of MRDT]
	\label{def:lin}
	Let $\M{D}$ be an MRDT which satisfies $\textrm{\sc{rc-non-comm}}(\M{D})$ and $\textrm{\sc{cond-comm}}(\M{D})$. Then, a configuration $C = \langle N, H, L, G, vis \rangle$ of $\M{S_D}$ is RA-linearizable if, for every active replica $r \in range(H)$, there exists a sequence $\pi$ consisting of all events in $L(H(r))$ such that  $\F{lo}(C)_{\mid L(H(r))} \subseteq \pi$ and $N(H(r)) = \pi(\sigma_0)$. 
An execution $\tau \in \llbracket \M{S_D} \rrbracket$ is RA-linearizable if all of its configurations are RA-linearizable. 
Finally, $\M{D}$ is RA-linearizable if all of its executions are RA-linearizable.
\end{definition}

For a configuration to be RA-linearizable, every active replica must have a state which can be obtained by applying a sequence of events witnessed at that replica, and that sequence must obey the linearization relation of the configuration. 
For an execution to be RA-linearizable, all of its configurations must be RA-linearizable.  Lemma \ref{lem:irreflexive} ensures the existence of a sequence extending the linearization relation, while Lemma \ref{lem:convergence} ensures that two versions which have witnessed the same set of events will have the same state (i.e. strong eventual consistency). Further, we also show that if an MRDT is RA-linearizable, then for any query operation in any execution, the query result is derived from the state obtained by applying the update events seen at the corresponding replica right before the query:

\begin{lemma}\label{lem:query}
	If MRDT $\mathcal{D}$ is RA-linearizable, then for all executions $\tau \in \llbracket \mathcal{S}_\mathcal{D} \rrbracket$, for all transitions $C \xrightarrow{query(r,q,a)} C'$ in $\tau$ where $C = \langle N, H, L, G, vis\rangle$, there exists a sequence $\pi$ consisting of all events in $L(H(r))$ such that $\F{lo}(C)_{\mid L(H(r))} \subseteq \pi$ and $a = \F{query}(\pi(\sigma_0),q)$.
\end{lemma}

Compared to the definition of RA-linearizability in the work by Wang et. al. \cite{Wang}, there is one major difference: Wang et. al. also consider a sequential specification in the form of a set of valid sequences of data-type operations, and requires the linearization sequence to belong to the specification. Our definition simply requires the state of a replica to be a linearization of the update operations applied to the replica, without appealing to a separate sequential specification.  Once this is done, we can separately show that a linearization of the MRDT operations obeys the sequential specification. For this, we can ignore the presence of the merge operation as well as the MRDT system model (which are taken care of by the RA-linearizability definition), thus boiling down to proving a specification over a sequential functional implementation, which is a well-studied problem.

\subsection{Bottom-up Linearization}
\label{subsec:bottom-up}

As demonstrated in \S \ref{sec:overview}, our approach to show RA-linearizability
of an MRDT implementation is based on using algebraic properties of merge
(specifically, commutativity of merge and update operation application) which allows
us to show that the result of a merge operation is a linearization of the
events in each of the versions being merged. We first describe a generic
template for the algebraic properties which can be used to prove
RA-linearizability:
	\[
\inferrule{\forall j.\ \pi_j \in \M{E} \cup \{\epsilon\} \quad l,a,b \in \Sigma \quad \pi \in \{\pi_0,\pi_1, \pi_2\}  \quad \forall j.\ \pi_j' = \pi_j - \pi }
{\F{merge}(\pi_0(l), \pi_1(a), \pi_2(b)) = \pi(\F{merge}(\pi_0'(l), \pi_1'(a)), \pi_2'(b)) ) )  }  \quad \quad \textrm{\sc{[BottomUpTemplate]}}
\]

The template for the algebraic property is given in the conclusion of the above
rule, while the premises describe certain conditions. Each $\pi_j$ for
$j \in \{0,1,2\}$ is a sequence of 0 or 1 event (i.e. either $\epsilon$ or a
single event $e_j$), while $l,a,b$ are arbitrary states of the MRDT. Note that applying the $\epsilon$ event on a state leaves it unchanged (i.e. $\epsilon(s) = s$). Then, we
can select one event $\pi$ which has been applied to the arguments of merge on
the LHS, and bring it outside, i.e. remove the event from each argument on
which it was applied, and instead apply the event to the result of merge. Note
that the notation $\pi_j^{'} = \pi_j - \pi$ means that if $\pi = \pi_j$, then
$\pi_j^{'} = \epsilon$, else $\pi_j^{'} = \pi_j - \pi$.

\begin{wrapfigure}{r}{0.3\textwidth}
	\vspace{-2em}
	\begin{center}
		\includegraphics[angle=0, width=0.8\linewidth]{no-rc-chain}
	\end{center}
	\vspace{-1em}
	\caption{Example demonstrating the failure of bottom-up linearization in the presence of an $\F{rc}$-chain}
	\vspace{-1em}
	\label{fig:no-rc-chain}
\end{wrapfigure}

The rule (P1$'$) given in \S \ref{subsec:lin} can be seen as an instantiation of the above
template with $\pi_0 = \epsilon, \pi_1 = e_1, \pi_2 = e_2$ and $\pi = e_2$
where $e_1 \xrightarrow{\F{rc}} e_2$. Similarly, (P1-1) is another instantiation
with $\pi_0 = \pi_2 = e_1$, $\pi_1 = e_3$ and $\pi = e_3$ where $e_3 \neq e_1$.
Assuming that the input arguments to merge are obtained through sequences of events $\tau_0,
\tau_1, \tau_2$, the template rule builds the linearization sequence
$\tau = \tau' e$ where $e$ is the last event in one of the $\tau_i$s, and $\tau'$ is
recursively generated by applying the rule on $\tau^{'} = \tau - e$.
We call this procedure as \emph{bottom-up linearization}.  
The event $e$ should be chosen in such a way that the sequence $\tau$ is an
extension of the linearization relation (Def. \ref{def:lin-relation}).

However, bottom-up linearization might fail if the last event in the merge output 
is not the last event in any of the three arguments to merge.
For example, consider the execution shown in Fig. \ref{fig:no-rc-chain},
where there exists an $\F{rc}$-chain: $o_2 \xrightarrow{\F{rc}} 
o_3 \xrightarrow{\F{rc}} o_1$, and $o_1$ and $o_2$ are non-commutative.
$e_1$ is visible to $e_2$, while event $e_3$ is
concurrent to $e_1$ and $e_2$. Now, for the version obtained after merging
$v_3$ and $v_4$, the linearization relation would be $e_1 \xrightarrow[\F{vis}]{\F{lo}}
e_2$ and $e_2 \xrightarrow[\F{rc}]{\F{lo}} e_3$. Notably, even though
$e_1$ and $e_3$ are also concurrent, and $\F{rc}$ orders $o_3$ before $o_1$,
this will not result in a linearization relation from $e_3$ to $e_1$, due to
the presence of a non-commutative update operation $e_2$ to which $e_1$ is visible. 
The bottom-up linearization for the merge of $v_3$ and $v_4$, will result in
the sequence $e_1 e_2 e_3$, which is an extension of the linearization order.

However, suppose we first merge versions $v_2$ and $v_4$, to obtain the
version $v_5$, where the linearization relation is $e_3
\xrightarrow[\F{rc}]{\F{lo}} e_1$. Merging $v_3$ and $v_5$ (with LCA $v_2$) 
would have the same linearization relation as merging $v_3$ and $v_4$. 
However, the sequences
leading to $v_3$ and $v_5$ are $e_1 e_2$ and $e_3 e_1$ respectively, while the
only sequence which extends the linearization relation for their merge is $e_1
e_2 e_3$. Bottom-up linearization will then be constrained to pick either $e_1$
or $e_2$ to appear at the end, but such a sequence will not extend the linearization relation
resulting in the failure of bottom-up linearization. 
To avoid such cases, we place an additional constraint which prohibits the
presence of an $\F{rc}$-chain:
\begin{align*}
	\textrm{\sc{no-rc-chain}}(\M{D}) & \triangleq  \neg (\exists o_1,o_2,o_3 \in O.\ o_1 \xrightarrow{\F{rc}} o_2 \xrightarrow{\F{rc}} o_3)
\end{align*}
If there is an $\F{rc}$-chain, executions such as Fig. \ref{fig:no-rc-chain}
are possible, resulting in infeasibility of bottom-up linearization. However,
we will show that if an MRDT satisfies $\textrm{\sc{no-rc-chain}}(\M{D})$, then
we can use bottom-up linearization to prove that $\M{D}$ is linearizable. We
note that \textrm{\sc{no-rc-chain}} is a pragmatic restriction and consistent
with standard conflict-resolution strategies such as add/remove-wins,
enable/disable-wins, update/delete-wins, etc. which are typically used in MRDT
implementations.