\vspace{-1em}
\section{Experimental Evaluation}
\label{sec:results}

We have implemented our verification technique in the \fstar programming
language and verified several MRDTs using it. We also extracted OCaml code from
the verified implementations and ran them as part of Irmin~\cite{Irmin}, a
Git-like distributed database which follows the MRDT system model described in
\S 3. This distinguishes our work from prior works in automated RDT
verification~\cite{Kartik} which focuses on verifying abstract models rather
than actual implementations.

Our framework in \fstar consists of an \fstar interface that
defines signatures for an MRDT implementation (Fig. \ref{fig:orset_impl}) and
the VCs described in Table \ref{tbl:vc}; these are encoded as \fstar lemmas.
This interface contains 200 lines of \fstar code.  An MRDT developer
instantiates the interface with their specific MRDT implementation and calls upon \fstar to prove
the lemmas (i.e., the VCs). Once this is done, our metatheory
(Theorem \ref{thm:2}) guarantees that the MRDT implementation is linearizable.

\begin{table}[htpb]
\centering
\footnotesize
	\vspace{-1em}
	\caption{Verified MRDTs. $^*$ denotes MRDT implementations not present in prior work.}
	\vspace{-0.5em}
	\begin{tabular}{@{}llcS[table-format=3.2]@{}}
		\toprule
		\textbf{MRDT} & \textbf{$\F{rc}$ Policy} & \textbf{\#LOC} & \textbf{Verification Time (s)} \\
		\midrule
		Increment-only counter~\cite{Kaki2019} & $\F{none}$ & 6 & 0.72 \\
		PN counter~\cite{Vimala} & $\F{none}$ & 10 & 1.64 \\
		Enable-wins flag$^*$ & $\F{disable} \xrightarrow{\F{rc}} \F{enable}$ & 30 & 29.80 \\
		Disable-wins flag$^*$ & $\F{enable} \xrightarrow{\F{rc}} \F{disable}$ & 30 & 37.91 \\
		Grows-only set~\cite{Kaki2019} & $\F{none}$ & 6 & 0.45 \\
		Grows-only map~\cite{Vimala} & $\F{none}$ & 11 & 4.65 \\
		OR-set~\cite{Vimala} & $\F{rem_a} \xrightarrow{\F{rc}} \F{add_a}$ & 20 & 4.53 \\
		OR-set (efficient)$^*$ & $\F{rem_a} \xrightarrow{\F{rc}} \F{add_a}$ & 34 & 660.00 \\
		Remove-wins set$^*$ & $\F{add_a} \xrightarrow{\F{rc}} \F{rem_a}$ & 22 & 9.60 \\
		Set-wins map$^*$ & $\F{del_k} \xrightarrow{\F{rc}} \F{set_k}$ & 20 & 5.06 \\
		Replicated Growable Array~\cite{Attiya} & $\F{none}$ & 13 & 1.51 \\
		Optional register$^*$ & $\F{unset} \xrightarrow{\F{rc}} \F{set}$ & 35 & 200.00 \\
		Multi-valued Register$^*$ & $\F{none}$ & 7 & 0.65 \\
		JSON-style MRDT$^*$ & $\F{Fig}.~\ref{fig:json_impl}$ & 26 & 148.84 \\
		\bottomrule
	\end{tabular}
	\label{tab:mrdts}
	\vspace{-1em}
\end{table}

We instantiate the interface with MRDT implementations of several datatypes
such as counter, flag, set, map, and list (Table~\ref{tab:mrdts}).  All the results were obtained on a
Intel\textregistered Xeon\textregistered Gold 5120 x86-64 machine running
Ubuntu 22.04 with 64GB of main memory.  While some of the MRDTs have been taken
from previous works \cite{Vimala,Kaki2019,Attiya} or translated from their CRDT
counterparts, we also develop some new implementations, denoted by $^*$ in
Table \ref{tab:mrdts}.  We also uncovered bugs in previous MRDT implementations
(Enable-wins flag and Efficient OR-set) from \cite{Vimala}, which we fixed (more
details in \S \ref{subsec:bug}). We note that in all our experiments, all the
VCs were automatically discharged by \fstar in a reasonable amount of time. 

While our approach ensures that the MRDT implementations are verified in the 
\fstar framework, it is important to note that the user is obligated to trust the \fstar
language implementation, the extraction mechanism, the OCaml language implementation, 
the OCaml runtime, and the hardware.

We now highlight several notable features about our verified MRDTs. We have
designed and developed the first correct implementations of both an enable-wins
and disable-wins flag MRDT. Our implementation of efficient
OR-set maintains a per-replica, per-element counter instead of adding different
versions of the same element (as done by the OR-set implementation of Fig.
\ref{fig:orset_impl}), thus matching the theoretical lower bound in terms of
space-efficiency for any OR-set CRDT implementation (as proved in
\cite{Burckhardt}). 
We have developed the first known MRDT implementation of a remove-wins set datatype.
Finally, as a demonstration of vertical compositionality, we have developed a
JSON MRDT which is composed of several component MRDTs, with its correctness
guarantee being directly derived from the correctness of the component MRDTs.

\subsection{Case study: A verified polymorphic JSON-style MRDT}
\label{subsec:json}

JSON is a notable example of a data type which is composed of several other
datatypes. JSON is widely used as a data interchange format in many databases
and web services ~\cite{Json}.  Our JSON MRDT is modeled as an unordered collection of key/value pairs, where the values
can be any primitive types, such as counter, list, etc., or they can be JSON
type themselves. We assume that keys are update-only; that is, key-value mappings
can be added and modified, but once a key is added, it cannot be deleted.
Previous works, such as Automerge ~\cite{Automerge}, have developed  similar
JSON-style CRDT models. However, these models are monomorphic, which means that
the data type of the values must be known in advance.  Our goal is to develop a
more generic JSON-style MRDT that supports polymorphic values, i.e. we leave
the value data type as an abstract type which can be instantiated with
any concrete MRDT.

\begin{figure}[ht]
	\footnotesize
	\vspace{-1.5em}
	\begin{algorithmic} [1]
		\State $\Sigma_\F{json} : (k:(\F{string }\times \Omega)) \rightarrow \Sigma_{\F{snd}(k)}$
		\State $O_\F{json} = \{\F{set}(k, o) \mid ~ o  \in O_{\F{snd}(k)} \}$
		\State $Q_\F{json} = \{\F{get}(k, q) \mid ~ q \in Q_{\F{snd}(k)} \}$
		\State $\sigma_{0_\F{json}} = \lambda (k: \F{string} \times \Omega).\ \sigma_{0_{\F{snd} (k)}}$
		\State $\F{do}(\sigma,t,r,\F{set}(k,o)) =  \sigma [k \mapsto o (\sigma(k), t, r)]$
		\State $\F{merge}_\F{json}(\sigma_{\top}, \sigma_1, \sigma_2) = \lambda (k: \F{string} \times \Omega).\ \F{merge}_{\F{snd}(k)}(\sigma_\top(k), \sigma_1(k), \sigma_2(k))$
		\State $\F{query}_\F{json} (\sigma, get(k, q)) = \F{query}_{\F{snd}(k)}(\sigma(k), q)$
		\State $\F{rc}_\F{json} = \{(\F{set}(k_1,o_1),\F{set}(k_2,o_2)) \in O_\F{json} \times O_\F{json}\ \mid\ k_1 = k_2 \wedge (o_1,o_2) \in \F{rc}_{\F{snd}(k_1)} \}$
	\end{algorithmic}
	\vspace{-1em}
	\caption{JSON-style MRDT implementation}
	\label{fig:json_impl}
	\vspace{-1.5em}
\end{figure}

Fig.~\ref{fig:json_impl} shows the implementation of the JSON MRDT. It uses a map to maintain the association between keys and values. Notice that the key is a tuple consisting of the identifier string and an MRDT type $\alpha \in \Omega$ which denotes the type of the value. The type $\alpha$ can be any arbitrary MRDT with implementation $\M{D}_\alpha = (\Sigma_\alpha, \sigma_{0_\alpha}, \F{merge}_\alpha, \F{query}_\alpha,\F{rc}_\alpha)$. Different key strings can now map to different value MRDT types. We also allow overloading: the same key string can be associated with multiple values of different types. The JSON MRDT allows update operations of the form $\F{set}(k,o)$ where $o$ is an operation of the underlying value MRDT associated with the key $k$. $\F{set}(k,o)$ simply applies the operation $o$ on the value associated with $k$, leaving the other key-value pairs unchanged. The JSON merge calls the underlying MRDT merge on the values associated with each key. The query operation of the form $\F{get}(k,q)$ retrieves the value associated with $k$ in $\sigma$ and applies the query operation $q$ of the underlying data type to it. The conflict resolution policy of JSON operations ($\F{rc}_\F{json}$) depends on the conflict resolution of the value types
when two operations update the same key (i.e. same identifier and value type). Every other pair of JSON operations commute with each other.

Notably, the proof of RA-linearizability of the JSON MRDT is directly derived from the proofs of the underlying value MRDT types. If all the MRDTs in $\Omega$ are linearizable, then the JSON MRDT is also linearizable. We have proved all the VCs for the JSON MRDT in \fstar by using the VCs of the underlying value MRDTs. We can now instantiate $\Omega$ with any set of verified MRDTs, thereby obtaining the verified JSON MRDT for free.

\vspace{-0.5em}
\subsection{Buggy MRDT Implementation in~\cite{Vimala}}
\label{subsec:bug}

We now present some details of one of the buggy MRDTs, Enable-wins flag, that we discovered using our framework in the work by \citet{Vimala}. The state of the enable-wins flag MRDT consists of a pair: a counter and a flag. The counter tracks the number of
the enable events, while the flag is set to true on an enable event. The desired specification for this flag is that it should be true when there is at least one enable event not visible to any disable event. In our framework, we can express this specification as $\F{disable} \xrightarrow{\F{rc}}\F{enable}$, linearizing the enable operation after a concurrent disable.
When we attempted to verify this implementation in our framework, we discovered that one of the
VCs, $\psi^{L_2^b}_\F{ind2-1op}$, was failing. Our investigation revealed that the implementation
violated the specification. The bug appeared in an execution with intermediate merges.

\begin{wrapfigure}{r}{0.35\textwidth}
	\vspace{-1em}
	\begin{center}
		\includegraphics[width=0.36\textwidth]{ewflag-counterexample}
	\end{center}
	\vspace{-2em}
	\caption{An enable-wins flag execution}
	\label{fig:ewflag-counter}
\end{wrapfigure}

Consider the execution depicted in Fig.~\ref{fig:ewflag-counter}.  When merging
versions $v_3$ and $v_5$ (with LCA $v_1$), since the counter value of $v_5$ is greater than $v_1$,
the flag in the merged version $v_6$ is set to true.  However, this contradicts the Enable-wins flag
specification, which states that the flag should be true only when there is an
enable event that is not visible to any disable event.  All enable
events in the execution are disabled by subsequent disable events on their individual replicas, yet the flag is true at
$v_6$. Notice that the version $v_5$ is obtained due to an intermediate merge. 
We discovered that \citet{Vimala} had an implementation bug in the framework. 
The framework expects a simulation relation from the MRDT developer, in addition to the specification and the implementation. 
This simulation relation serves as a proof artefact. \citet{Vimala} check whether the developer-provided simulation relation is valid
and the bug occurred during the validity-checking procedure. Due to this, \citet{Vimala} admitted the buggy enable-wins flag implementation\footnote{Buggy implementation can be found in \S \ref{subsec:bug_impl}}. 

We further note that this buggy implementation does not even satisfy strong eventual consistency.
In Fig.~\ref{fig:ewflag-counter}, merging $v_3$ and $v_4$ results in $v_7$, where the flag
is false. Note that both versions $v_6$ and $v_7$ have observed the same set of updates on both replicas,
yet they lead to divergent states. This violates strong eventual consistency.
We fixed this implementation by maintaining a counter-flag pair for every replica, 
i.e. changing the state to a map from replica-IDs to counter-flag pair.

\subsection{Verifying state-based CRDTs}
\label{subsec:crdts}
Although the developement in the paper so far has focused on  verifying MRDTs, we note that our framework can also directly verify state-based CRDTs. The only difference between the two is that state-based CRDTs do not maintain the LCA, and merge is a binary function. Our VCs (Table \ref{tbl:vc}) can be directly applied on state-based CRDTs, by simply ignoring the LCA argument for all merges. Note that while the merge function in state-based CRDTs does not use the LCA, our VCs still use the LCA to determine whether an event is local or common to both replicas, and appropriately linearize events taking into account both $\F{rc}$ and $\F{vis}$ relations. The entire set of VCs retrofitted for state-based CRDTs can be found in Table \ref{tab:full_crdts}. We have also successfully implemented and verified 7 state-based CRDTs in our framework: Increment-only counter, PN counter, Observed-Remove set, Two-Phase set, Grows-only set, Grows-only map and Multi-valued register.

\vspace{-0.5em}
\subsection{Limitations}

Our framework is currently unable to verify some MRDT implementations such as
Queue from previous works \cite{Vimala,Kaki2019}. The Queue MRDT follows at-least-once semantics for dequeues, which allows concurrent dequeue operations to return the same element from the queue, thereby having the effect of a single dequeue. Such an implementation is clearly not linearizable as per our definition, since we cannot omit any event while constructing the linearization. It would be possible to modify our notion of linearization to also allow events to be omitted; we leave this investigation as part of future work. Our verification technique is also not complete, but in practice we have been able to successfully verify all MRDT implementations (except Queue) from earlier works. %We have also discovered subtle bugs in MRDTs from previous works using our technique.