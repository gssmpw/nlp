\documentclass[acmsmall]{acmart}

\AtBeginDocument{%
  \providecommand\BibTeX{{%
    Bib\TeX}}}

\setcopyright{cc}
\setcctype{by}
\acmJournal{PACMPL}
\acmYear{2025} \acmVolume{9} \acmNumber{OOPSLA1} \acmArticle{111} \acmMonth{4} \acmPrice{}\acmDOI{10.1145/3720452}

\usepackage[T1]{fontenc}
\usepackage{graphicx}
\usepackage{color}
\renewcommand\UrlFont{\color{blue}\rmfamily}
\usepackage{booktabs}   %% For formal tables:
                        %% http://ctan.org/pkg/booktabs
\usepackage{subcaption} %% For complex figures with subfigures/subcaptions
\captionsetup{compatibility=false}
                        %% http://ctan.org/pkg/subcaption
\usepackage{algorithm}% http://ctan.org/pkg/algorithms
\usepackage{algpseudocode}% http://ctan.org/pkg/algorithmicx
\let\oldemptyset\emptyset
\usepackage{mathpartir}
\usepackage{listings}
\usepackage{mathpartir}
\usepackage{amsmath,amsthm}
\usepackage{stmaryrd}
\usepackage{xspace}
\usepackage{tabularx}
\usepackage{wrapfig}
\usepackage{enumitem}
\usepackage{xcolor} % For using colors
\usepackage{longtable}
\usepackage{enumitem}
\usepackage{siunitx}
\setlistdepth{9}

\newenvironment{nop}{}{}
\newenvironment{smathpar}{
\begin{nop}\small\begin{mathpar}}{
\end{mathpar}\end{nop}\ignorespacesafteren}
\definecolor{Bittersweet}{RGB}{255,102,102}
\definecolor{BrightBlue}{RGB}{0,0,255}
\definecolor{midnightblue}{rgb}{0.1, 0.1, 0.44}

\lstset{
      language=Caml,
      mathescape=true,
      breaklines=true,
      basicstyle=\ttfamily\footnotesize,
      flexiblecolumns=false,
      tabsize=2,
      escapechar=',
      commentstyle=\color{BrightBlue},
      stringstyle=\color{red},
      deletekeywords={do},
      keywordstyle=\color{Bittersweet},
      morekeywords={val, fun},
      keywords=[2]{Lemma},
      keywordstyle=[2]\color{blue},
      keywords=[3]{requires, ensures},
      keywordstyle=[3]\color{green!50!black},
      keywords=[4]{necessary},
      keywordstyle=[4]\color{red}\bfseries,
      numberstyle=\tiny\color{gray},
      numbersep=5pt
}

\lstMakeShortInline[columns=fullflexible]|
\newcommand{\ml}[1]{\lstinline[language=caml]{#1}}
\newcommand{\F}[1]{\mathsf{#1}}
\newcommand{\M}[1]{\mathcal{#1}}
\newcommand{\name}{\textsc{XYZ}\xspace}
\newcommand{\fstar}{F$^{\star}$\xspace}


\begin{document}

\title{Automatically Verifying Replication-aware Linearizability}
\author{Vimala Soundarapandian}
\affiliation{
	\institution{IIT Madras}
	\city{Chennai}
  \country{India}
}
\email{cs19d750@cse.iitm.ac.in}

\author{Kartik Nagar}
\affiliation{
	\institution{IIT Madras}
	\city{Chennai}
  \country{India}
}
\email{nagark@cse.iitm.ac.in}

\author{Aseem Rastogi}
\affiliation{
	\institution{Microsoft Research}
	\city{Bangalore}
  \country{India}
}
\email{aseemr@microsoft.com}

\author{KC Sivaramakrishnan}
\affiliation{
	\institution{IIT Madras and Tarides}
	\city{Chennai}
  \country{India}
}
\email{kcsrk@cse.iitm.ac.in}

\begin{abstract}
      Data replication is crucial for enabling fault tolerance and uniform low
      latency in modern decentralized applications. Replicated Data Types
      (RDTs) have emerged as a principled approach for developing replicated
      implementations of basic data structures such as counter, flag, set, map,
      etc. While the correctness of RDTs is generally specified using the notion of
      strong eventual consistency--which guarantees that replicas that have
      received the same set of updates would converge to the same state--a more
      expressive specification which relates the converged state to updates
      received at a replica would be more beneficial to RDT users.
      Replication-aware linearizability is one such specification, which
      requires all replicas to always be in a state which can be obtained by
      linearizing the updates received at the replica. In this work, we develop
      a novel fully automated technique for verifying replication-aware linearizability
      for Mergeable Replicated Data Types (MRDTs). We identify
      novel algebraic properties for MRDT operations and the merge function
      which are sufficient for proving an implementation to be linearizable and
      which go beyond the standard notions of commutativity, associativity, and
      idempotence. We also develop a novel inductive technique called bottom-up
      linearization to automatically verify the required algebraic properties.
      Our technique can be used to verify both MRDTs and state-based CRDTs. We
      have successfully applied our approach to a number of complex MRDT and
      CRDT implementations including a novel JSON MRDT.
\end{abstract}

\begin{CCSXML}
<ccs2012>
   <concept>
       <concept_id>10011007.10011074.10011099.10011692</concept_id>
       <concept_desc>Software and its engineering~Formal software verification</concept_desc>
       <concept_significance>500</concept_significance>
       </concept>
   <concept>
       <concept_id>10010147.10010919.10010177</concept_id>
       <concept_desc>Computing methodologies~Distributed programming languages</concept_desc>
       <concept_significance>500</concept_significance>
       </concept>
 </ccs2012>
\end{CCSXML}

\ccsdesc[500]{Software and its engineering~Formal software verification}
\ccsdesc[500]{Computing methodologies~Distributed programming languages}

\keywords{MRDTs, Eventual consistency, Automated verification, Replication-aware linearizability}

\maketitle

\section{Introduction}


\begin{figure}[t]
\centering
\includegraphics[width=0.6\columnwidth]{figures/evaluation_desiderata_V5.pdf}
\vspace{-0.5cm}
\caption{\systemName is a platform for conducting realistic evaluations of code LLMs, collecting human preferences of coding models with real users, real tasks, and in realistic environments, aimed at addressing the limitations of existing evaluations.
}
\label{fig:motivation}
\end{figure}

\begin{figure*}[t]
\centering
\includegraphics[width=\textwidth]{figures/system_design_v2.png}
\caption{We introduce \systemName, a VSCode extension to collect human preferences of code directly in a developer's IDE. \systemName enables developers to use code completions from various models. The system comprises a) the interface in the user's IDE which presents paired completions to users (left), b) a sampling strategy that picks model pairs to reduce latency (right, top), and c) a prompting scheme that allows diverse LLMs to perform code completions with high fidelity.
Users can select between the top completion (green box) using \texttt{tab} or the bottom completion (blue box) using \texttt{shift+tab}.}
\label{fig:overview}
\end{figure*}

As model capabilities improve, large language models (LLMs) are increasingly integrated into user environments and workflows.
For example, software developers code with AI in integrated developer environments (IDEs)~\citep{peng2023impact}, doctors rely on notes generated through ambient listening~\citep{oberst2024science}, and lawyers consider case evidence identified by electronic discovery systems~\citep{yang2024beyond}.
Increasing deployment of models in productivity tools demands evaluation that more closely reflects real-world circumstances~\citep{hutchinson2022evaluation, saxon2024benchmarks, kapoor2024ai}.
While newer benchmarks and live platforms incorporate human feedback to capture real-world usage, they almost exclusively focus on evaluating LLMs in chat conversations~\citep{zheng2023judging,dubois2023alpacafarm,chiang2024chatbot, kirk2024the}.
Model evaluation must move beyond chat-based interactions and into specialized user environments.



 

In this work, we focus on evaluating LLM-based coding assistants. 
Despite the popularity of these tools---millions of developers use Github Copilot~\citep{Copilot}---existing
evaluations of the coding capabilities of new models exhibit multiple limitations (Figure~\ref{fig:motivation}, bottom).
Traditional ML benchmarks evaluate LLM capabilities by measuring how well a model can complete static, interview-style coding tasks~\citep{chen2021evaluating,austin2021program,jain2024livecodebench, white2024livebench} and lack \emph{real users}. 
User studies recruit real users to evaluate the effectiveness of LLMs as coding assistants, but are often limited to simple programming tasks as opposed to \emph{real tasks}~\citep{vaithilingam2022expectation,ross2023programmer, mozannar2024realhumaneval}.
Recent efforts to collect human feedback such as Chatbot Arena~\citep{chiang2024chatbot} are still removed from a \emph{realistic environment}, resulting in users and data that deviate from typical software development processes.
We introduce \systemName to address these limitations (Figure~\ref{fig:motivation}, top), and we describe our three main contributions below.


\textbf{We deploy \systemName in-the-wild to collect human preferences on code.} 
\systemName is a Visual Studio Code extension, collecting preferences directly in a developer's IDE within their actual workflow (Figure~\ref{fig:overview}).
\systemName provides developers with code completions, akin to the type of support provided by Github Copilot~\citep{Copilot}. 
Over the past 3 months, \systemName has served over~\completions suggestions from 10 state-of-the-art LLMs, 
gathering \sampleCount~votes from \userCount~users.
To collect user preferences,
\systemName presents a novel interface that shows users paired code completions from two different LLMs, which are determined based on a sampling strategy that aims to 
mitigate latency while preserving coverage across model comparisons.
Additionally, we devise a prompting scheme that allows a diverse set of models to perform code completions with high fidelity.
See Section~\ref{sec:system} and Section~\ref{sec:deployment} for details about system design and deployment respectively.



\textbf{We construct a leaderboard of user preferences and find notable differences from existing static benchmarks and human preference leaderboards.}
In general, we observe that smaller models seem to overperform in static benchmarks compared to our leaderboard, while performance among larger models is mixed (Section~\ref{sec:leaderboard_calculation}).
We attribute these differences to the fact that \systemName is exposed to users and tasks that differ drastically from code evaluations in the past. 
Our data spans 103 programming languages and 24 natural languages as well as a variety of real-world applications and code structures, while static benchmarks tend to focus on a specific programming and natural language and task (e.g. coding competition problems).
Additionally, while all of \systemName interactions contain code contexts and the majority involve infilling tasks, a much smaller fraction of Chatbot Arena's coding tasks contain code context, with infilling tasks appearing even more rarely. 
We analyze our data in depth in Section~\ref{subsec:comparison}.



\textbf{We derive new insights into user preferences of code by analyzing \systemName's diverse and distinct data distribution.}
We compare user preferences across different stratifications of input data (e.g., common versus rare languages) and observe which affect observed preferences most (Section~\ref{sec:analysis}).
For example, while user preferences stay relatively consistent across various programming languages, they differ drastically between different task categories (e.g. frontend/backend versus algorithm design).
We also observe variations in user preference due to different features related to code structure 
(e.g., context length and completion patterns).
We open-source \systemName and release a curated subset of code contexts.
Altogether, our results highlight the necessity of model evaluation in realistic and domain-specific settings.





\section{Overview}

\revision{In this section, we first explain the foundational concept of Hausdorff distance-based penetration depth algorithms, which are essential for understanding our method (Sec.~\ref{sec:preliminary}).
We then provide a brief overview of our proposed RT-based penetration depth algorithm (Sec.~\ref{subsec:algo_overview}).}



\section{Preliminaries }
\label{sec:Preliminaries}

% Before we introduce our method, we first overview the important basics of 3D dynamic human modeling with Gaussian splatting. Then, we discuss the diffusion-based 3d generation techniques, and how they can be applied to human modeling.
% \ZY{I stopp here. TBC.}
% \subsection{Dynamic human modeling with Gaussian splatting}
\subsection{3D Gaussian Splatting}
3D Gaussian splatting~\cite{kerbl3Dgaussians} is an explicit scene representation that allows high-quality real-time rendering. The given scene is represented by a set of static 3D Gaussians, which are parameterized as follows: Gaussian center $x\in {\mathbb{R}^3}$, color $c\in {\mathbb{R}^3}$, opacity $\alpha\in {\mathbb{R}}$, spatial rotation in the form of quaternion $q\in {\mathbb{R}^4}$, and scaling factor $s\in {\mathbb{R}^3}$. Given these properties, the rendering process is represented as:
\begin{equation}
  I = Splatting(x, c, s, \alpha, q, r),
  \label{eq:splattingGA}
\end{equation}
where $I$ is the rendered image, $r$ is a set of query rays crossing the scene, and $Splatting(\cdot)$ is a differentiable rendering process. We refer readers to Kerbl et al.'s paper~\cite{kerbl3Dgaussians} for the details of Gaussian splatting. 



% \ZY{I would suggest move this part to the method part.}
% GaissianAvatar is a dynamic human generation model based on Gaussian splitting. Given a sequence of RGB images, this method utilizes fitted SMPLs and sampled points on its surface to obtain a pose-dependent feature map by a pose encoder. The pose-dependent features and a geometry feature are fed in a Gaussian decoder, which is employed to establish a functional mapping from the underlying geometry of the human form to diverse attributes of 3D Gaussians on the canonical surfaces. The parameter prediction process is articulated as follows:
% \begin{equation}
%   (\Delta x,c,s)=G_{\theta}(S+P),
%   \label{eq:gaussiandecoder}
% \end{equation}
%  where $G_{\theta}$ represents the Gaussian decoder, and $(S+P)$ is the multiplication of geometry feature S and pose feature P. Instead of optimizing all attributes of Gaussian, this decoder predicts 3D positional offset $\Delta{x} \in {\mathbb{R}^3}$, color $c\in\mathbb{R}^3$, and 3D scaling factor $ s\in\mathbb{R}^3$. To enhance geometry reconstruction accuracy, the opacity $\alpha$ and 3D rotation $q$ are set to fixed values of $1$ and $(1,0,0,0)$ respectively.
 
%  To render the canonical avatar in observation space, we seamlessly combine the Linear Blend Skinning function with the Gaussian Splatting~\cite{kerbl3Dgaussians} rendering process: 
% \begin{equation}
%   I_{\theta}=Splatting(x_o,Q,d),
%   \label{eq:splatting}
% \end{equation}
% \begin{equation}
%   x_o = T_{lbs}(x_c,p,w),
%   \label{eq:LBS}
% \end{equation}
% where $I_{\theta}$ represents the final rendered image, and the canonical Gaussian position $x_c$ is the sum of the initial position $x$ and the predicted offset $\Delta x$. The LBS function $T_{lbs}$ applies the SMPL skeleton pose $p$ and blending weights $w$ to deform $x_c$ into observation space as $x_o$. $Q$ denotes the remaining attributes of the Gaussians. With the rendering process, they can now reposition these canonical 3D Gaussians into the observation space.



\subsection{Score Distillation Sampling}
Score Distillation Sampling (SDS)~\cite{poole2022dreamfusion} builds a bridge between diffusion models and 3D representations. In SDS, the noised input is denoised in one time-step, and the difference between added noise and predicted noise is considered SDS loss, expressed as:

% \begin{equation}
%   \mathcal{L}_{SDS}(I_{\Phi}) \triangleq E_{t,\epsilon}[w(t)(\epsilon_{\phi}(z_t,y,t)-\epsilon)\frac{\partial I_{\Phi}}{\partial\Phi}],
%   \label{eq:SDSObserv}
% \end{equation}
\begin{equation}
    \mathcal{L}_{\text{SDS}}(I_{\Phi}) \triangleq \mathbb{E}_{t,\epsilon} \left[ w(t) \left( \epsilon_{\phi}(z_t, y, t) - \epsilon \right) \frac{\partial I_{\Phi}}{\partial \Phi} \right],
  \label{eq:SDSObservGA}
\end{equation}
where the input $I_{\Phi}$ represents a rendered image from a 3D representation, such as 3D Gaussians, with optimizable parameters $\Phi$. $\epsilon_{\phi}$ corresponds to the predicted noise of diffusion networks, which is produced by incorporating the noise image $z_t$ as input and conditioning it with a text or image $y$ at timestep $t$. The noise image $z_t$ is derived by introducing noise $\epsilon$ into $I_{\Phi}$ at timestep $t$. The loss is weighted by the diffusion scheduler $w(t)$. 
% \vspace{-3mm}

\subsection{Overview of the RTPD Algorithm}\label{subsec:algo_overview}
Fig.~\ref{fig:Overview} presents an overview of our RTPD algorithm.
It is grounded in the Hausdorff distance-based penetration depth calculation method (Sec.~\ref{sec:preliminary}).
%, similar to that of Tang et al.~\shortcite{SIG09HIST}.
The process consists of two primary phases: penetration surface extraction and Hausdorff distance calculation.
We leverage the RTX platform's capabilities to accelerate both of these steps.

\begin{figure*}[t]
    \centering
    \includegraphics[width=0.8\textwidth]{Image/overview.pdf}
    \caption{The overview of RT-based penetration depth calculation algorithm overview}
    \label{fig:Overview}
\end{figure*}

The penetration surface extraction phase focuses on identifying the overlapped region between two objects.
\revision{The penetration surface is defined as a set of polygons from one object, where at least one of its vertices lies within the other object. 
Note that in our work, we focus on triangles rather than general polygons, as they are processed most efficiently on the RTX platform.}
To facilitate this extraction, we introduce a ray-tracing-based \revision{Point-in-Polyhedron} test (RT-PIP), significantly accelerated through the use of RT cores (Sec.~\ref{sec:RT-PIP}).
This test capitalizes on the ray-surface intersection capabilities of the RTX platform.
%
Initially, a Geometry Acceleration Structure (GAS) is generated for each object, as required by the RTX platform.
The RT-PIP module takes the GAS of one object (e.g., $GAS_{A}$) and the point set of the other object (e.g., $P_{B}$).
It outputs a set of points (e.g., $P_{\partial B}$) representing the penetration region, indicating their location inside the opposing object.
Subsequently, a penetration surface (e.g., $\partial B$) is constructed using this point set (e.g., $P_{\partial B}$) (Sec.~\ref{subsec:surfaceGen}).
%
The generated penetration surfaces (e.g., $\partial A$ and $\partial B$) are then forwarded to the next step. 

The Hausdorff distance calculation phase utilizes the ray-surface intersection test of the RTX platform (Sec.~\ref{sec:RT-Hausdorff}) to compute the Hausdorff distance between two objects.
We introduce a novel Ray-Tracing-based Hausdorff DISTance algorithm, RT-HDIST.
It begins by generating GAS for the two penetration surfaces, $P_{\partial A}$ and $P_{\partial B}$, derived from the preceding step.
RT-HDIST processes the GAS of a penetration surface (e.g., $GAS_{\partial A}$) alongside the point set of the other penetration surface (e.g., $P_{\partial B}$) to compute the penetration depth between them.
The algorithm operates bidirectionally, considering both directions ($\partial A \to \partial B$ and $\partial B \to \partial A$).
The final penetration depth between the two objects, A and B, is determined by selecting the larger value from these two directional computations.

%In the Hausdorff distance calculation step, we compute the Hausdorff distance between given two objects using a ray-surface-intersection test. (Sec.~\ref{sec:RT-Hausdorff}) Initially, we construct the GAS for both $\partial A$ and $\partial B$ to utilize the RT-core effectively. The RT-based Hausdorff distance algorithms then determine the Hausdorff distance by processing the GAS of one object (e.g. $GAS_{\partial A}$) and set of the vertices of the other (e.g. $P_{\partial B}$). Following the Hausdorff distance definition (Eq.~\ref{equation:hausdorff_definition}), we compute the Hausdorff distance to both directions ($\partial A \to \partial B$) and ($\partial B \to \partial A$). As a result, the bigger one is the final Hausdorff distance, and also it is the penetration depth between input object $A$ and $B$.


%the proposed RT-based penetration depth calculation pipeline.
%Our proposed methods adopt Tang's Hausdorff-based penetration depth methods~\cite{SIG09HIST}. The pipeline is divided into the penetration surface extraction step and the Hausdorff distance calculation between the penetration surface steps. However, since Tang's approach is not suitable for the RT platform in detail, we modified and applied it with appropriate methods.

%The penetration surface extraction step is extracting overlapped surfaces on other objects. To utilize the RT core, we use the ray-intersection-based PIP(Point-In-Polygon) algorithms instead of collision detection between two objects which Tang et al.~\cite{SIG09HIST} used. (Sec.~\ref{sec:RT-PIP})
%RT core-based PIP test uses a ray-surface intersection test. For purpose this, we generate the GAS(Geometry Acceleration Structure) for each object. RT core-based PIP test takes the GAS of one object (e.g. $GAS_{A}$) and a set of vertex of another one (e.g. $P_{B}$). Then this computes the penetrated vertex set of another one (e.g. $P_{\partial B}$). To calculate the Hausdorff distance, these vertex sets change to objects constructed by penetrated surface (e.g. $\partial B$). Finally, the two generated overlapped surface objects $\partial A$ and $\partial B$ are used in the Hausdorff distance calculation step.
\section{Problem Definition}
\label{sec:lin}

In this section, we formally define the semantics of the replicated data store
on top of which the MRDT implementations operate (\S \ref{subsec:os}),
the notion of RA-linearizability for MRDTs (\S \ref{subsec:lin_def}), and the process of bottom-up linearization (\S \ref{subsec:bottom-up}).

\subsection{Semantics of the Replicated Data Store}
\label{subsec:os}

\begin{figure}[ht]
	\scriptsize
	\raggedright$\textrm{\sc{[CreateBranch]}}$
	\[
	\inferrule{r\in dom (H) \\ r'\notin dom (H) \\  v\notin dom (N) \\
		N' = N [v \mapsto N(H(r))] \\ H' = H[r' \mapsto v] \\
		L' = L [v \mapsto L(H(r))] \\ G' = (dom(N) \cup \{v\}, {E} \cup \{(H(r), v)\})}
	{(N, H, L, G, vis) \xrightarrow{\F{createBranch(r',r)}} (N', H', L', G', vis)\\}
	\]
		\raggedright$\textrm{\sc{[Apply]}}$
	\[
	\inferrule{e = (t,r,o) \\  o \in O_\tau \\  \forall e' \in \bigcup range(L).\ time (e') \neq t  \\ r\in dom (H) \\ v\notin dom (N) \\
		N' = N [v \mapsto \F{do} (N(H(r)), e)] \\
		H' = H [r \mapsto v] \\ L' = L [v \mapsto L(H(r)) \cup \{e\}] \\
		G' = (dom(N'), {E} \cup \{(H(r), v)\}) \\
		vis' = vis \cup (L (H(r)) \times \{e\})}
	{(N, H, L, G, vis) \xrightarrow{\F{apply(t,r,o)}} (N', H', L', G', vis')\\}
	\]

		\raggedright$\textrm{\sc{[Merge]}}$
	\[
	\inferrule{r_{1}, r_{2} \in dom (H) \\ v\notin dom (N)  \\ v_{\top} = LCA (H(r_{1}), H(r_{2})) \\
		N' = N [v \mapsto \F{merge} (N(v_{\top}), N(H(r_{1})), N(H(r_{2}))] \\
		H' = H [r_{1} \mapsto v] \\ L' = L [v \mapsto L(H(r_{1})) \cup L(H(r_{2}))] \\
		G' = (dom(N'), {E} \cup \{(H(r_{1}), v), (H(r_{2}), v)\})}
	{(N, H, L, G, vis) \xrightarrow{\F{merge (r_1, r_2)}} (N', H', L', G', vis)\\}
	\]
	\raggedright$\textrm{\sc{[Query]}}$
	\[
	\inferrule{r\in dom (H) \\ q \in Q_\tau \\ a = \F{query}(N(H(r)),q)} 
	{(N, H, L, G, vis) \xrightarrow{\F{query(r,q,a)}} (N, H, L, G, vis)\\}
	\]
	\vspace{-2em}
	\caption{Semantics of the replicated datastore}
	\vspace{-3.25em}
	\label{fig:sem}
\end{figure}

The semantics of the replicated store defines all possible executions of an
MRDT implementation.  Formally, the semantics are parameterized by an MRDT
implementation $\M{D} = \langle \Sigma, \sigma_0, \F{do}, \F{merge}, \F{query},\\
\F{rc}\rangle$ of type $\tau = \langle O_\tau, Q_\tau, Val_\tau \rangle$ and are represented by a labeled transition system
$\M{S_\M{D}}$ = ($\Phi$, $\rightarrow$). Each configuration in $\Phi$ maintains a set of
versions, where each version is created either by applying an MRDT operation to
an existing version, or by merging two versions. Each replica is associated
with a head version, which is the most recent version seen at the replica.
Formally, each configuration $C$ in $\Phi$ is a tuple $\langle N, H, L, G, vis
\rangle$, where:
 
\begin{itemize}
	\item $N : \F{Version} \rightharpoonup \Sigma$ is a partial function
that maps versions to their states ($\F{Version}$ is the set of all possible
versions). 
	\item $H: \M{R} \rightharpoonup \F{Version}$ is also a partial function
that maps replicas to their head versions. A replica is considered active if it 
is in the domain of $H$ of the configuration.
	\item $L: \F{Version} \rightharpoonup \mathbb{P}(\M{E})$ maps a version to the set of events 
that led to this version. Each event $e \in \M{E}$ is an update
operation instance, uniquely identified by a timestamp value (we define $\M{E}
= \M{T} \times \M{R} \times O$).
	\item $G = (dom(N),E)$ is the version graph, whose
vertices represent the versions in the configuration (i.e. those in the domain of
$N$) and whose edges represent a relationship between different versions (we explain the different types of edges below).
	\item $vis \subseteq \M{E} \times \M{E}$ is a partial order over events.
\end{itemize}


Figure~\ref{fig:sem} gives a formal description of the transition rules. $\textrm{\sc{CreateBranch}}$ forks a new replica $r'$ from an existing replica
$r$, installing a new version $v$ at $r'$ with the same state as the head
version $H(r)$ of $r$, and adding an edge $(H(r),v')$ in the version graph.
$\textrm{\sc{Apply}}$ applies an update operation $o$ on some replica $r$,
generating a new event $e$ with a timestamp different than all events generated
so far. $\bigcup\F{range}(L)$ denotes the set of events witnessed across all versions.
A new version $v$ is also created whose state is obtained by applying
$o$ on the current state of the replica $r$. The version graph is updated by
adding the edge $(H(r),v)$. The $vis$ relation as well as the function
$L$, which tracks events applied at each version, are also updated. In
particular, each event $e'$ already applied at $r$, i.e. $e' \in L(H(r))$, is
made visible to $e$: $(e',e) \in vis$, while $L'(v)$ is obtained by adding $e$
to $L(H(r))$.

$\textrm{\sc{Merge}}$ takes two replicas $r_1$ and $r_2$, applies the $\F{merge}$
function on the states of their head versions to generate a new version $v$, which is
installed as the new head version at $r_1$. Edges are added in the version graph
from the previous head versions of $r_1$ and $r_2$ to $v$. $L(v)$ is obtained by
taking a union of $L(r_1)$ and $L(r_2)$, and there is no change in the visibility
relation. $\textrm{\sc{Query}}$ takes a replica $r$ and a query operation $q$ and applies $q$ to the state at the head version of 
$r$, returning an output value $a$. Note that the $\textrm{\sc{Query}}$ transition 
does not modify the configuration and the return value of the query is stored as part of the transition label. While our operational semantics is based on and inspired by previous works 
\cite{Kaki2022,Vimala}, we note that it is more general and precisely
captures the MRDT system model as opposed to previous
works. In particular, \citet{Kaki2022} place
significant restrictions on the $\textrm{\sc{Merge}}$ transition, disallowing
arbitrary replicas to be merged to ensure that there is a total order on the
merge transitions. While the semantics in the work by \citet{Vimala}
does allow arbitrary merges, it is more abstract and high-level, and does not
even keep track of versions and the version graph. 

\textbf{Notation:} We now introduce some notation that will be used throughout the paper. Given a configuration $C$, we use $X(C)$ to project the component $X$ of $C$. For a relation $R$, we use $x \xrightarrow{R}
y$ to signify that $(x,y) \in R$. We use $R_{\mid S}$ to indicate the relation
as given by $R$ but restricted to elements of the set $S$. Let $R^*$ denote the
reflexive-transitive closure of $R$, and let $R^+$ denote the
transitive closure of $R$. For an event $e$, we use the projection functions
$\F{op}, \F{time}, \F{rep}$ to obtain the update operation, timestamp and replica
resp. For a sequence of events $\pi$, $\pi_{\mid S}(\sigma)$ denotes
application of the sub-sequence of $\pi$ restricted to events in $S$. For a
configuration $C$, we use $e_1 \mid\mid_C e_2$ to denote that $e_1$ and $e_2$
are concurrent, that is $\neg (e_1 \xrightarrow{\F{vis}(C)} e_2 \vee e_2
\xrightarrow{\F{vis}(C)} e_1)$. Given a total order over a set of events $\M{E}$,
represented by a sequence $\pi$, and $\F{lo} \subseteq \M{E} \times \M{E}$, we say that
$\pi$ extends $\F{lo}$ if $\F{lo} \subseteq \pi$. The relation $\F{rc}$ orders
update operations, but for convenience we sometime use it for ordering events, with
the intention that it is actually being applied to the underlying update operations.
We use $e_1 \neq e_2$ to indicate that $ \F{time}(e_1) \neq  \F{time}(e_2)$.


We define the initial configuration of $\M{S_\M{D}}$ as $C_0 =
\langle N_0, H_0, L_0, G_0, \emptyset \rangle$, which consists of only one
replica $r_0$.  Here, $H_0 = [r_0 \mapsto v_0]$, $N_0 = [v_0 \mapsto
\sigma_0]$, where $\sigma_0$ is the initial state as given by
$\M{D_{\tau}}$, while $v_0$ denotes the initial version and $L_0 = [v_0
\mapsto \emptyset]$. The graph $G_0 = (\{v_0\}, \emptyset)$ is the initial
version graph. An execution of $\M{S_D}$ is defined to be a finite sequence of
transitions, $C_0 \xrightarrow{t_1} C_1 \xrightarrow{t_2} C_2 \ldots
\xrightarrow{t_n} C_n$. Note that the label of a transition corresponds to its
type. 
Let $\llbracket \M{S_\M{D}} \rrbracket$ denote the set of all possible
executions of $\M{S_D}$.

Finally, as mentioned earlier, $\F{merge}$ is a ternary function, taking as
input the states of two versions to be merged, and the state of the lowest
common ancestor (LCA) of the two versions.
Version $v_1 \in V$ is defined to be a causal ancestor of version $v_2 \in V$
if and only if  $(v_1, v_2) \in E^*$.


\begin{definition}[LCA]
	Given a version graph $G = (V,E)$ and versions $v_1, v_2 \in V$, $v_\top \in
	V$ is defined to be the lowest common ancestor of $v_1$ and $v_2$ (denoted by
	$LCA(v_1,v_2)$) if (i) $(v_\top,v_1) \in E^*$ and $(v_\top,v_2) \in E^*$,
	(ii) $\forall v \in V. (v,v_1) \in E^* \wedge (v,v_2) \in E^* \implies
	(v,v_\top) \in E^*$.
\end{definition}

Note that the version history graph at any point in any execution is guaranteed
to be acyclic (i.e. a DAG),  and hence the LCA (if it exists) is guaranteed to
be unique. We now present an important property linking the LCA of two versions
with events applied at each version.

\begin{lemma}\label{lem:LCA}
	Given a configuration $C = \langle N,H,L,G,vis \rangle$ reachable in some
	execution $\tau \in \llbracket \M{S_D} \rrbracket$ and two versions $v_1,v_2
	\in dom(N)$, if $v_\top$ is the LCA of $v_1$ and $v_2$ in $G$, then
	$L(v_\top) = L(v_1) \cap L(v_2)$\footnote{All proofs are in the Appendix \S \ref{sec:app}}.
\end{lemma}

\begin{wrapfigure}{r}{0.3\textwidth}
	\vspace{-1.5em}
	\begin{center}
		\includegraphics[angle=0, width=0.8\linewidth]{LCA}
	\end{center}
	\vspace{-1em}
	\caption{Version Graph with no LCA for $v_5$ and $v_6$}
	\label{fig:LCA}
	\vspace{-1em}
\end{wrapfigure}

Thus, the events of the LCA are exactly those applied at both the versions.
This intuitively corresponds to the fact that $LCA(v_1,v_2)$ is the most recent
version from which the two versions $v_1$ and $v_2$ diverged. Note that it is possible that the LCA may not exist for two versions. Fig.
\ref{fig:LCA} depicts the version graph of such an execution. Vertices with
in-degree 1 (i.e. $v_1,v_2,v_3,v_4$) have been generated by applying a new update
operation (with the orange edges labeled by the corresponding events $e_1,e_2,e_3,e_4$),
while vertices with in-degree 2 have been obtained by merging two other
versions (depicted by blue edges). The merge of $v_1$ and $v_4$ (leading to
$v_6$) has a unique LCA $v_0$, similarly, merge of $v_2$ and $v_3$ (leading to
$v_5$) also has a unique LCA $v_0$. However, if we now want to merge $v_5$ and
$v_6$, both $v_1$ and $v_2$ are ancestors, but there is no LCA. We note that
this execution will actually be prohibited by the semantics of
\citet{Kaki2022}, since the two merges leading to $v_5$ and $v_6$ are
concurrent.

Notice that $L(v_5) = \{e_1,e_2,e_3\}$, while $L(v_6) = \{e_1,e_2,e_4\}$.
Hence, by Lemma~\ref{lem:LCA}, $L(LCA(v_5,v_6)) = \{e_1,e_2\}$, but such a
version is not generated during the execution. To resolve this issue, we introduce the notion of \textit{potential} LCAs. 

\begin{definition}[Potential LCAs]
Given a version graph $G = (V,E)$ and versions $v_1, v_2 \in V$, $v_\top \in
	V$ is defined to be a potential LCA of $v_1$ and $v_2$  if 
	(i) $(v_\top,v_1) \in E^*$ and $(v_\top,v_2) \in E^*$,
	(ii) $\neg (\exists v. (v,v_1) \in E^* \wedge (v,v_2) \in E^* \wedge (v_\top,v) \in E^*)$.
\end{definition}

For merging $v_1$ and $v_2$, we first find all the potential LCAs, and recursively merge them to obtain the actual
LCA state. For the execution in Fig. \ref{fig:LCA}, the potential LCAs of $v_5$ and $v_6$ would be
$v_1$ and $v_2$ (with $L(v_1) = \{e_1\}$ and
$L(v_2) = \{e_2\}$); merging them would get us the actual LCA. 
  In \S \ref{subsec:lcaproof}, we prove that this recursive merge-based strategy is guaranteed to generate the
actual LCA.

\subsection{Replication-aware Linearizability for MRDTs}
\label{subsec:lin_def}

As mentioned in \S \ref{sec:overview}, our goal is to show that the state of every version $v$
generated during an execution is a linearization of the events in $L(v)$. We
use the notation $\F{lo}$ to indicate the linearization relation, which is a
binary relation over events. For an execution in $\M{S}_\M{D}$, we
want $\F{lo}$ between the events of the execution to satisfy certain desirable
properties: (i) $\F{lo}$ between two events should not change during an execution, (ii) $\F{lo}$ should obey the conflict resolution policy
for concurrent events and (iii) $\F{lo}$ should obey the replica-local
$\F{vis}$ ordering for non-concurrent events. This would ensure that two
versions which have observed the same set of events will have the same state (i.e. \textit{strong eventual consistency}), and this state would
also be a linearization of update operations of the data type satisfying the
conflict resolution policy.

While the $\F{lo}$ relation in classical linearizability literature is
typically a total order, in our context, we take advantage of commutativity
of update operations, and only define $\F{lo}$ over non-commutative events. As we
will see later, this flexibility allows us to have different sequences of
events which extend the same $\F{lo}$ relation between non-commutative events, and hence are guaranteed
to lead to the same state. We use the notation $e \rightleftarrows e'$ to
indicate that events $e$ and $e'$ commute with each other. Formally, this means
that $\forall \sigma.\;e(e'(\sigma)) = e'(e(\sigma))$. Two update operations
$o,o'$ commute if $\forall e,e'.\;\F{op}(e) = o \wedge \F{op}(e') = o'
\implies e \rightleftarrows e'$.  As mentioned earlier, the $\F{rc}$ relation
is also only defined between non-commutative update operations.

\begin{lemma}\label{lem:non-comm}
	Given a set of events $\M{E}$, if $\F{lo} \subseteq \M{E} \times \M{E}$ is defined over
	every pair of non-commutative events in $\M{E}$, then for any two sequences
	$\pi_1, \pi_2$ which extend $\F{lo}$, for any state $\sigma$, $\pi_1(\sigma)
	= \pi_2(\sigma)$.
\end{lemma}

Given a configuration $C = \langle N, H, L, G, vis \rangle$, let $\M{E}_C = \bigcup
\F{range}(L(C))$ denote the set of events witnessed across all versions in C.
Then, our goal is to define an appropriate linearization relation $\F{lo}_C
\subseteq \M{E}_C \times \M{E}_C$, which adheres to the $\F{rc}$ relation for concurrent
events, the $\F{vis}$ relation for non-concurrent events, and for every version $v
\in dom(N)$, $N(v)$ should be obtained by sequentializing the events in $L(v)$,
with the sequence extending $\F{lo}_C$. Note that this requires $\F{lo}^+$ to
be irreflexive\footnote{$\F{lo}$ need not be transitive, as we only want to
define $\F{lo}$ between non-commutative events, and non-commutativity is not a
transitive property}.

We now demonstrate that an $\F{lo}$ relation with all the desirable properties
may not exist for all executions. Suppose there are MRDT update operations $o,o'$ such
that $o \xrightarrow{\F{rc}} o'$. Fig. \ref{fig:conditional-commutativity}
contains a part of the version graph generated during some execution,
containing two instances of both $o$ and $o'$. We use $e_i:o_i$ to denote that
event $\F{op}(e_i) = o_i$. Notice that $e_1$ and $e_4$,
$e_2$ and $e_3$ are concurrent, while $e_1$ and $e_3$, $e_2$ and $e_4$ are
non-concurrent. Applying the $\F{rc}$ ordering on concurrent events, we would
want $e_3 \xrightarrow{\F{lo}} e_2$ and $e_4 \xrightarrow{\F{lo}} e_1$, while
applying $\F{vis}$ ordering, we would want $e_1 \xrightarrow{\F{lo}} e_3$ and
$e_2 \xrightarrow{\F{lo}} e_4$. However, this results in a $\F{lo}$-cycle, thus
making it impossible to construct a sequence of update operations for the merge of
$v_5$ and $v_6$, which adheres to the $\F{lo}$ ordering.

\begin{wrapfigure}{r}{0.3\textwidth}
	\vspace{-1em}
	\begin{center}
		\includegraphics[angle=0, width=0.8\linewidth]{conditional-commutativity}
	\end{center}
	\vspace{-1em}
	\caption{Example demonstrating cycle in $\F{lo}$}
	\vspace{-1.5em}
	\label{fig:conditional-commutativity}
\end{wrapfigure}

Notice that the above execution only requires the $\F{rc}$ relation to be
non-empty (i.e. there should exist some $(o,o') \in \F{rc}$). If the $\F{rc}$
relation is empty, then all update operations would commute with each other, and hence
the $\F{lo}$ relation would also be empty. If $\F{rc}$ is non-empty, $\F{rc}^+$
should be irreflexive to ensure irreflexivity of $\F{lo}^+$. Note that
$\F{rc}^+$ being irreflexive means that for any MRDT update operation $o$, $(o,o)
\notin \F{rc}$, and hence $o$ must commute with itself, since $\F{rc}$ relation
is defined for all pairs of non-commutative update operations. Furthermore, Fig.
\ref{fig:conditional-commutativity} shows that even if $\F{rc}^+$ is
irreflexive, it may still not be possible to construct an $\F{lo}$ relation
which can be extended to a total order and which adheres to the $\F{rc}$ relation
between all pairs of concurrent events. To ensure existence of an $\F{lo}$
relation such that $\F{lo}^+$ is irreflexive when $\F{rc}^+$ is irreflexive, we define it as follows:

\begin{definition}[Linearization relation]\label{def:lin-relation}
	Let $C$ be a configuration reachable in some execution in $\llbracket \M{S_D}
	\rrbracket$. Let $\M{E}_C$ be the set of events in $C$. Then, $\F{lo_C}$ is defined as:
	 \begin{align*}
		\forall e_1,e_2 \in \M{E}_C.\ e_1 \xrightarrow{\F{lo_C}} e_2  \Leftrightarrow &
		(e_1 \xrightarrow{\F{vis(C)}} e_2 \wedge \neg e_1 \rightleftarrows e_2) \\
		& \vee (e_1 \mid\mid_C e_2 \wedge e_1 \xrightarrow{\F{rc}} e_2 \wedge
		\neg(\exists e_3 \in \M{E}.\ e_2 \xrightarrow{\F{vis(C)}} e_3 \wedge \neg e_2
		\rightleftarrows e_3 ))
	\end{align*}
\end{definition}

$\F{lo}_C$ follows the visibility relation only between non-commutative
events. For concurrent non-commutative events $e_1$ and $e_2$ with $e_1
\xrightarrow{\F{rc}} e_2$, $\F{lo}_C$ follows the $\F{rc}$ relation only if
there is no event $e_3$ such that $e_2$ is visible to $e_3$ and $e_2$ does not commute with $e_3$. Applying this
definition to the execution in Fig. \ref{fig:conditional-commutativity}, for
the configuration obtained after merge, we would have neither $e_4
\xrightarrow{\F{lo}} e_1$, nor $e_3 \xrightarrow{\F{lo}} e_2$, thus avoiding
the cycle in $\F{lo}$.

\begin{lemma}\label{lem:irreflexive}
For an MRDT $\M{D}$ such that $\F{rc}^+$ is irreflexive, for any configuration $C$ reachable in
	$\M{S}_\M{D}$, $\F{lo}_C^+$ is irreflexive.
\end{lemma}

Going forward, we will assume that $\F{rc}^+$ is irreflexive for any MRDT $\M{D}$. 
We note that restricting $\F{lo}$ to not always obey the $\F{rc}$ relation by considering 
non-commutative update operations happening locally (and thus related by $\F{vis}$) is also
sensible from a practical perspective. For example, in the case of OR-set, even
though we have $\F{rem}_a \xrightarrow{\F{rc}} \F{add}_a$, if $\F{add}_a$ is
locally followed by another $\F{rem}_a$, it does not make sense to order a
concurrent $\F{rem}_a$ event before the $\F{add}_a$ event. More generally, if
an event $e_2$ is visible to another event $e_3$ with which it does not commute,
then $e_2$ is effectively "overwritten" by $e_3$, and hence there is no need to
linearize a concurrent event $e_1$ before $e_2$.

While $\F{lo}_C$ is now guaranteed to be irreflexive for any configuration $C$,
and hence can be extended to a sequence, it now no longer enforces an ordering
among all non-commutative pairs of events. Thus, there could exist sequences
$\pi_1,\pi_2$ extending an $\F{lo}_C$ relation which may contain a pair of
non-commutative events in different orders. For example, in Fig.
\ref{fig:conditional-commutativity}, for the configuration $C$ obtained after
the merge, $\F{lo}_C = \{(e_1,e_3), (e_2,e_4)\}$, resulting in sequences $\pi_1
= e_1 e_2 e_3 e_4$ and $\pi_2 = e_1 e_3 e_2 e_4$ which both extend $\F{lo}_C$,
but contain the non-commutative events $e_2$ and $e_3$ in different orders.
Thus, Lemma \ref{lem:non-comm} can no longer be applied, and it is not
guaranteed that $\pi_1$ and $\pi_2$ would lead to the same state. Notice that
in the sequences $\pi_1$ and $\pi_2$ above, even though $e_2$ and $e_3$ appear
in different orders, $e_4$ always appears after both. Indeed, $e_4$ must appear
after $e_2$ due to visibility relation, and since $e_3$ and $e_4$ commute with
each other (since both correspond to the same operation $o$), it is enough to
consider sequences where $e_4$ appears after $e_3$. Based on the above
observation, we now introduce a notion called conditional commutativity to
ensure that sequences such as $\pi_1,\pi_2$ would lead to the same state:

\begin{definition}[Conditional Commutativity]
	Events $e$ and $e'$ are said to conditionally commute with respect to event
	$e''$ (denoted by $e \overset{e''}{\rightleftarrows} e'$) if $\forall \sigma
	\in \Sigma.\ \forall \pi \in \M{E}^*.\ e''(\pi(e(e'(\sigma)))) =
	e''(\pi(e'(e(\sigma))))$.
\end{definition}

Update operations $o$ and $o'$ conditionally commute w.r.t. update operation $o''$ if
$\forall e,e',e''. \F{op}(e) = o \wedge \F{op}(e') = o' \wedge \F{op}(e'') =
o'' \Rightarrow e \overset{e''}{\rightleftarrows} e'$. For example, for the
OR-set MRDT of Fig. \ref{fig:orset_impl}, $\F{add}_a
\overset{\F{rem}_a}{\rightleftarrows} \F{rem}_a$. Even though \textit{add} and
\textit{remove} operations of the same element do not commute with each other,
if there is guaranteed to be a future \textit{remove} operation, then they do
commute. For the execution in Fig. \ref{fig:conditional-commutativity}, if
$e_2$ and $e_3$ conditionally commute w.r.t. $e_4$, then both the sequences
$\pi_1$ and $\pi_2$ will lead to the same state. For non-commutative update operations
that are not ordered by $\F{lo}$, we enforce their conditional commutativity
through the following property:
\begin{align*}
	\textrm{\sc{cond-comm}}(\M{D}) &  \triangleq  \forall o_1,o_2,o_3 \in O.\
	(o_1 \xrightarrow{\F{rc}} o_2 \wedge \neg o_2 \rightleftarrows o_3)
	\Rightarrow o_1 \overset{o_3}{\rightleftarrows} o_2
\end{align*}
$\textrm{\sc{cond-comm}}(\M{D})$ is a property of an MRDT $\M{D}$, enforcing
conditional commutativity of update operations $o_1$ and $o_2$ w.r.t. $o_3$ if $o_2$
does not commute with $o_3$. Connecting this with the definition of
linearization relation, if there are events $e_1,e_2,e_3$ performing operations
$o_1,o_2,o_3$ resp., and if $e_1 \xrightarrow{\F{rc}} e_2$, $e_2
\xrightarrow{\F{vis}} e_3$ and $\neg e_2 \rightleftarrows e_3$, then there
will not be a linearization relation between $e_1$ and $e_2$. However,
$\textrm{\sc{cond-comm}}(\M{D})$ would then ensure that the ordering of $e_1$
and $e_2$ will not matter, due to the presence of the event $e_3$. We also
formalize the requirement of an $\F{rc}$ relation between all pairs of
non-commutative update operations:
\begin{align*}
	\textrm{\sc{rc-non-comm}}(\M{D}) & \triangleq \forall o_1,o_2 \in O.\neg o_1
	\rightleftarrows o_2 \Leftrightarrow o_1 \xrightarrow{\F{rc}} o_2 \vee  o_2
	\xrightarrow{\F{rc}} o_1 \\
\end{align*}
\vspace{-3em}
\begin{lemma}\label{lem:convergence}
	For an MRDT $\M{D}$ which satisfies $\textrm{\sc{rc-non-comm}}(\M{D})$ and
	$\textrm{\sc{cond-comm}}(\M{D})$, for any reachable configuration $C$ in
	$\M{S}_\M{D}$, for any two sequences $\pi_1,\pi_2$ over $\M{E}_C$ which extend
	$\F{lo}_C$, for any state $\sigma$, $\pi_1(\sigma) = \pi_2(\sigma)$.
\end{lemma}


\begin{definition}[RA-linearizability of MRDT]
	\label{def:lin}
	Let $\M{D}$ be an MRDT which satisfies $\textrm{\sc{rc-non-comm}}(\M{D})$ and $\textrm{\sc{cond-comm}}(\M{D})$. Then, a configuration $C = \langle N, H, L, G, vis \rangle$ of $\M{S_D}$ is RA-linearizable if, for every active replica $r \in range(H)$, there exists a sequence $\pi$ consisting of all events in $L(H(r))$ such that  $\F{lo}(C)_{\mid L(H(r))} \subseteq \pi$ and $N(H(r)) = \pi(\sigma_0)$. 
An execution $\tau \in \llbracket \M{S_D} \rrbracket$ is RA-linearizable if all of its configurations are RA-linearizable. 
Finally, $\M{D}$ is RA-linearizable if all of its executions are RA-linearizable.
\end{definition}

For a configuration to be RA-linearizable, every active replica must have a state which can be obtained by applying a sequence of events witnessed at that replica, and that sequence must obey the linearization relation of the configuration. 
For an execution to be RA-linearizable, all of its configurations must be RA-linearizable.  Lemma \ref{lem:irreflexive} ensures the existence of a sequence extending the linearization relation, while Lemma \ref{lem:convergence} ensures that two versions which have witnessed the same set of events will have the same state (i.e. strong eventual consistency). Further, we also show that if an MRDT is RA-linearizable, then for any query operation in any execution, the query result is derived from the state obtained by applying the update events seen at the corresponding replica right before the query:

\begin{lemma}\label{lem:query}
	If MRDT $\mathcal{D}$ is RA-linearizable, then for all executions $\tau \in \llbracket \mathcal{S}_\mathcal{D} \rrbracket$, for all transitions $C \xrightarrow{query(r,q,a)} C'$ in $\tau$ where $C = \langle N, H, L, G, vis\rangle$, there exists a sequence $\pi$ consisting of all events in $L(H(r))$ such that $\F{lo}(C)_{\mid L(H(r))} \subseteq \pi$ and $a = \F{query}(\pi(\sigma_0),q)$.
\end{lemma}

Compared to the definition of RA-linearizability in the work by Wang et. al. \cite{Wang}, there is one major difference: Wang et. al. also consider a sequential specification in the form of a set of valid sequences of data-type operations, and requires the linearization sequence to belong to the specification. Our definition simply requires the state of a replica to be a linearization of the update operations applied to the replica, without appealing to a separate sequential specification.  Once this is done, we can separately show that a linearization of the MRDT operations obeys the sequential specification. For this, we can ignore the presence of the merge operation as well as the MRDT system model (which are taken care of by the RA-linearizability definition), thus boiling down to proving a specification over a sequential functional implementation, which is a well-studied problem.

\subsection{Bottom-up Linearization}
\label{subsec:bottom-up}

As demonstrated in \S \ref{sec:overview}, our approach to show RA-linearizability
of an MRDT implementation is based on using algebraic properties of merge
(specifically, commutativity of merge and update operation application) which allows
us to show that the result of a merge operation is a linearization of the
events in each of the versions being merged. We first describe a generic
template for the algebraic properties which can be used to prove
RA-linearizability:
	\[
\inferrule{\forall j.\ \pi_j \in \M{E} \cup \{\epsilon\} \quad l,a,b \in \Sigma \quad \pi \in \{\pi_0,\pi_1, \pi_2\}  \quad \forall j.\ \pi_j' = \pi_j - \pi }
{\F{merge}(\pi_0(l), \pi_1(a), \pi_2(b)) = \pi(\F{merge}(\pi_0'(l), \pi_1'(a)), \pi_2'(b)) ) )  }  \quad \quad \textrm{\sc{[BottomUpTemplate]}}
\]

The template for the algebraic property is given in the conclusion of the above
rule, while the premises describe certain conditions. Each $\pi_j$ for
$j \in \{0,1,2\}$ is a sequence of 0 or 1 event (i.e. either $\epsilon$ or a
single event $e_j$), while $l,a,b$ are arbitrary states of the MRDT. Note that applying the $\epsilon$ event on a state leaves it unchanged (i.e. $\epsilon(s) = s$). Then, we
can select one event $\pi$ which has been applied to the arguments of merge on
the LHS, and bring it outside, i.e. remove the event from each argument on
which it was applied, and instead apply the event to the result of merge. Note
that the notation $\pi_j^{'} = \pi_j - \pi$ means that if $\pi = \pi_j$, then
$\pi_j^{'} = \epsilon$, else $\pi_j^{'} = \pi_j - \pi$.

\begin{wrapfigure}{r}{0.3\textwidth}
	\vspace{-2em}
	\begin{center}
		\includegraphics[angle=0, width=0.8\linewidth]{no-rc-chain}
	\end{center}
	\vspace{-1em}
	\caption{Example demonstrating the failure of bottom-up linearization in the presence of an $\F{rc}$-chain}
	\vspace{-1em}
	\label{fig:no-rc-chain}
\end{wrapfigure}

The rule (P1$'$) given in \S \ref{subsec:lin} can be seen as an instantiation of the above
template with $\pi_0 = \epsilon, \pi_1 = e_1, \pi_2 = e_2$ and $\pi = e_2$
where $e_1 \xrightarrow{\F{rc}} e_2$. Similarly, (P1-1) is another instantiation
with $\pi_0 = \pi_2 = e_1$, $\pi_1 = e_3$ and $\pi = e_3$ where $e_3 \neq e_1$.
Assuming that the input arguments to merge are obtained through sequences of events $\tau_0,
\tau_1, \tau_2$, the template rule builds the linearization sequence
$\tau = \tau' e$ where $e$ is the last event in one of the $\tau_i$s, and $\tau'$ is
recursively generated by applying the rule on $\tau^{'} = \tau - e$.
We call this procedure as \emph{bottom-up linearization}.  
The event $e$ should be chosen in such a way that the sequence $\tau$ is an
extension of the linearization relation (Def. \ref{def:lin-relation}).

However, bottom-up linearization might fail if the last event in the merge output 
is not the last event in any of the three arguments to merge.
For example, consider the execution shown in Fig. \ref{fig:no-rc-chain},
where there exists an $\F{rc}$-chain: $o_2 \xrightarrow{\F{rc}} 
o_3 \xrightarrow{\F{rc}} o_1$, and $o_1$ and $o_2$ are non-commutative.
$e_1$ is visible to $e_2$, while event $e_3$ is
concurrent to $e_1$ and $e_2$. Now, for the version obtained after merging
$v_3$ and $v_4$, the linearization relation would be $e_1 \xrightarrow[\F{vis}]{\F{lo}}
e_2$ and $e_2 \xrightarrow[\F{rc}]{\F{lo}} e_3$. Notably, even though
$e_1$ and $e_3$ are also concurrent, and $\F{rc}$ orders $o_3$ before $o_1$,
this will not result in a linearization relation from $e_3$ to $e_1$, due to
the presence of a non-commutative update operation $e_2$ to which $e_1$ is visible. 
The bottom-up linearization for the merge of $v_3$ and $v_4$, will result in
the sequence $e_1 e_2 e_3$, which is an extension of the linearization order.

However, suppose we first merge versions $v_2$ and $v_4$, to obtain the
version $v_5$, where the linearization relation is $e_3
\xrightarrow[\F{rc}]{\F{lo}} e_1$. Merging $v_3$ and $v_5$ (with LCA $v_2$) 
would have the same linearization relation as merging $v_3$ and $v_4$. 
However, the sequences
leading to $v_3$ and $v_5$ are $e_1 e_2$ and $e_3 e_1$ respectively, while the
only sequence which extends the linearization relation for their merge is $e_1
e_2 e_3$. Bottom-up linearization will then be constrained to pick either $e_1$
or $e_2$ to appear at the end, but such a sequence will not extend the linearization relation
resulting in the failure of bottom-up linearization. 
To avoid such cases, we place an additional constraint which prohibits the
presence of an $\F{rc}$-chain:
\begin{align*}
	\textrm{\sc{no-rc-chain}}(\M{D}) & \triangleq  \neg (\exists o_1,o_2,o_3 \in O.\ o_1 \xrightarrow{\F{rc}} o_2 \xrightarrow{\F{rc}} o_3)
\end{align*}
If there is an $\F{rc}$-chain, executions such as Fig. \ref{fig:no-rc-chain}
are possible, resulting in infeasibility of bottom-up linearization. However,
we will show that if an MRDT satisfies $\textrm{\sc{no-rc-chain}}(\M{D})$, then
we can use bottom-up linearization to prove that $\M{D}$ is linearizable. We
note that \textrm{\sc{no-rc-chain}} is a pragmatic restriction and consistent
with standard conflict-resolution strategies such as add/remove-wins,
enable/disable-wins, update/delete-wins, etc. which are typically used in MRDT
implementations.
\subsection{Lemmas}
\begin{lemma}\label{lemma: constraints do not apply if disjoint}
If $(s_t, a_t)$ and $(\tilde{s}, \tilde{a})$ have disjoint support, the monotonicity constraints \eqref{proofeq:monotonicity1} and \eqref{proofeq:monotonicity2} and counterfactual stability \eqref{proofeq:counterfactual stability} will be satisfied for all possible counterfactual transition probabilities for transitions from $(\tilde{s}, \tilde{a})$.
\end{lemma}

\begin{proof}
Because $(s_t, a_t)$ and $(\tilde{s}, \tilde{a})$ have disjoint support, $P(s_{t+1} \mid \tilde{s}, \tilde{a}) = 0$ and, for all $\tilde{s}'$ where $P(\tilde{s}' \mid \tilde{s}, \tilde{a}) > 0$, $P(\tilde{s}' \mid s_t, a_t) = 0$. This is because for all $\tilde{s}'$ where $P(\tilde{s}' \mid s_t, a_t) > 0$, $P(\tilde{s}' \mid \tilde{s}, \tilde{a}) = 0$, and, for all $\tilde{s}'$ where $P(\tilde{s}' \mid \tilde{s}, \tilde{a}) > 0$, $P(\tilde{s}' \mid s_t, a_t) = 0$. 
Because for all $\tilde{s}'$ where $P(\tilde{s}' \mid \tilde{s}, \tilde{a}) > 0$, $P(\tilde{s}' \mid s_t, a_t) = 0$, CS \eqref{proofeq:counterfactual stability} and Mon2 \eqref{proofeq:monotonicity2} will be vacuously true for all transitions from $(\tilde{s}, \tilde{a})$. Also, because $P(s_{t+1} \mid \tilde{s}, \tilde{a}) = 0$, Mon1 \eqref{proofeq:monotonicity1} is vacuously true.
\end{proof}

\begin{lemma}
\label{lemma:absolute max cf prob}
For any observed transition $s_t, a_t \rightarrow s_{t+1}$ and counterfactual transition $\tilde{s}, \tilde{a} \rightarrow \tilde{s}'$:
\[
    \sum_{u_t = 1}^{|U_t|} \mu_{\tilde{s}, \tilde{a}, u_t, \tilde{s}'} \cdot \mu_{s_t, a_t, u_t, s_{t+1}} \cdot \theta_{u_t} \leq \min(P(s_{t+1} \mid s_t, a_t), P(\tilde{s}' \mid \tilde{s}, \tilde{a}))
\]
\end{lemma}

\begin{proof}
From our interventional probability constraints \eqref{proofeq:interventional constraint}, we have:

\begin{equation}
    \label{eq: lemma 4.2 constraint 1}
   \sum_{u_t = 1}^{|U_t|} \mu_{s_t, a_t, u_t, s_{t+1}} \cdot \theta_{u_t} = P(s_{t+1} \mid s_t, a_t) 
\end{equation}

and

\begin{equation}
    \label{eq: lemma 4.2 constraint 2}
   \sum_{u_t = 1}^{|U_t|} \mu_{\tilde{s}, \tilde{a}, u_t, \tilde{s}'}\cdot \theta_{u_t} = P(\tilde{s}' \mid \tilde{s}, \tilde{a}) 
\end{equation}

$\mu_{\tilde{s}, \tilde{a}, u, \tilde{s}'} \in \{0, 1\}$ (from its definition). $\sum_{u_t = 1}^{|U_t|} \mu_{\tilde{s}, \tilde{a}, u_t, \tilde{s}'} \cdot \mu_{s_t, a_t, u_t, s_{t+1}} \cdot \theta_{u_t}$ is maximal when the probability assigned to all $u_t$ where $\mu_{\tilde{s}, \tilde{a}, u_t, \tilde{s}'} = 1$ and $\mu_{s_t, a_t, u_t, s_{t+1}}=1$ is maximal.

\[
\sum_{u_t = 1}^{|U_t|} \mu_{\tilde{s}, \tilde{a}, u_t, \tilde{s}'} \cdot (\mu_{s_t, a_t, u_t, s_{t+1}} \cdot \theta_{u_t}) \leq \sum_{u_t = 1}^{|U_t|} \mu_{s_t, a_t, u_t, s_{t+1}} \cdot \theta_{u_t} = P(s_{t+1} \mid s_t, a_t)
\]

\[
\sum_{u_t = 1}^{|U_t|} (\mu_{\tilde{s}, \tilde{a}, u_t, \tilde{s}'} \cdot \theta_{u_t}) \cdot \mu_{s_t, a_t, u_t, s_{t+1}} \leq \sum_{u_t = 1}^{|U_t|} \mu_{\tilde{s}, \tilde{a}, u_t, \tilde{s}'}\cdot \theta_{u_t} = P(\tilde{s}' \mid \tilde{s}, \tilde{a})
\]

 Because of Eq. \eqref{eq: lemma 4.2 constraint 1} and Eq. \eqref{eq: lemma 4.2 constraint 2}, the probability assigned to all $u_t$ where $\mu_{\tilde{s}, \tilde{a}, u_t, \tilde{s}'} = 1$ and $\mu_{s_t, a_t, u_t, s_{t+1}}=1$ cannot be greater than $P(s_{t+1} \mid s_t, a_t)$ or $P(\tilde{s}' \mid \tilde{s}, \tilde{a})$. Therefore,

\[
\sum_{u_t = 1}^{|U_t|} \mu_{\tilde{s}, \tilde{a}, u_t, \tilde{s}'} \cdot \mu_{s_t, a_t, u_t, s_{t+1}} \cdot \theta_{u_t} \leq \min(P(s_{t+1} \mid s_t, a_t), P(\tilde{s}' \mid \tilde{s}, \tilde{a}))
\]

%

This also means that the upper bound of any counterfactual transition probability \\$\tilde{P}_t^{UB}(\tilde{s}' \mid \tilde{s}, \tilde{a}) \leq \dfrac{\min(P(\tilde{s}' \mid \tilde{s}, \tilde{a}), P(s_{t+1} \mid s_t, a_t))}{P(s_{t+1} \mid s_t, a_t)}$.
\end{proof}

\begin{lemma}
    \label{lemma: CF probs of observed state-action pair}
    For any observed transition $s_t, a_t \rightarrow s_{t+1}$, $\tilde{P}_{t}(s_{t+1} \mid s_t, a_t) = 1$ and, $\forall \tilde{s}' \in \mathcal{S}$ where $\tilde{s}' \in \mathcal{S}\setminus \{s_{t+1}\}$, $\tilde{P}_{t}(\tilde{s}' \mid s_t, a_t) = 0$.
\end{lemma}

\begin{proof}
From \eqref{eq: counterfactual probability}, we have:
    \begin{align*}
      \tilde{P}_{t}(s_{t+1} \mid s_t, a_t)
      &= \frac{\sum_{u_t = 1}^{|U_t|} \mu_{s_t, a_t, u_t, s_{t+1}} \cdot \mu_{s_t, a_t, u_t, s_{t+1}} \cdot \theta_{u_t}}{P(s_{t+1} \mid s_t, a_t)} 
      \\ &= \frac{\sum_{u_t = 1}^{|U_t|} \mu_{s_t, a_t, u_t, s_{t+1}} \cdot \theta_{u_t}}{P(s_{t+1} \mid s_t, a_t)}
      \\ &= \frac{P(s_{t+1} \mid s_t, a_t)}{P(s_{t+1} \mid s_t, a_t)}
      \tag{from the interventional probability constraint \eqref{proofeq:interventional constraint}}
      \\ &= 1 \tag{since $s_t, a_t \rightarrow s_{t+1}$ was observed, $P(s_{t+1} \mid s_t, a_t) > 0$}
    \end{align*}

    \begin{align*}
      \tilde{P}_{t}(\tilde{s}' \mid s_t, a_t)
      &= \frac{\sum_{u_t = 1}^{|U_t|} \mu_{s_t, a_t, u_t, \tilde{s}'} \cdot \mu_{s_t, a_t, u_t, s_{t+1}} \cdot \theta_{u_t}}{P(s_{t+1} \mid s_t, a_t)} 
      \\ &=\frac{0}{P(s_{t+1} \mid s_t, a_t)}
      \tag{since at most one of $\mu_{s_t, a_t, u_t, \tilde{s}'}$ and $\mu_{s_t, a_t, u_t, s_{t+1}}$ can equal 1}
      \\ &= 0 \tag{since $s_t, a_t \rightarrow s_{t+1}$ was observed, $P(s_{t+1} \mid s_t, a_t) > 0$}
    \end{align*}

    These results are as expected, because, given the same exogenous noise and input, we must get the same output in the counterfactual world.
\end{proof}

\begin{lemma}
\label{lemma: existence of theta for overlapping case.}
Take an arbitrary counterfactual state-action pair $(\tilde{s}, \tilde{a})$ and observed state-action pair from an MDP, where $(\tilde{s}, \tilde{a}) \neq (s_t, a_t)$ and $(\tilde{s}, \tilde{a})$ has overlapping support with $(s_t, a_t)$ and $P(s_{t+1} \mid \tilde{s}, \tilde{a}) < P(s_{t+1} \mid s_t, a_t)$. Let $s_{t+1}$ be the observed next state. It can be shown that:

\[\max_{\theta} \left(\sum_{s' \in \mathcal{S}\setminus \{s_{t+1}\}}\sum_{\substack{u_t \in U_t\\f(s_t, a_t, u_t) = s_{t+1} \\ f(\tilde{s}, \tilde{a}, u_t) = s'}}{\theta_{u_t}} \geq P(s_{t+1} \mid s_t, a_t) - P(s_{t+1} \mid \tilde{s}, \tilde{a})\right)\]
\end{lemma}
\begin{proof}
First, let $S_{CS} \subset \mathcal{S}$ be the set of states $s'\in S_{CS}$ which must have a counterfactual transition probability $\tilde{P}_{t}(s' \mid \tilde{s}, \tilde{a}) = 0$ because of the counterfactual stability constraint (i.e., where $\forall s' \in S_{CS}, \dfrac{P(s_{t+1} \mid \tilde{s}, \tilde{a})}{P(s_{t+1} \mid s_t, a_t)}>\dfrac{P(s' \mid \tilde{s}, \tilde{a})}{P(s' \mid s_t, a_t)}$ and $P(s' \mid s_t, a_t) > 0$). Consider the total interventional probability of across all states $s'\in S_{CS}$. $\forall s' \in S_{CS}$:

\begin{align*}
    \dfrac{P(s' \mid \tilde{s}, \tilde{a})}{P(s' \mid s_t, a_t)} &< \dfrac{P(s_{t+1} \mid \tilde{s}, \tilde{a})}{P(s_{t+1} \mid s_t, a_t)} \text{  (from CS \eqref{proofeq:counterfactual stability})}\\
    P(s' \mid \tilde{s}, \tilde{a}) &< \dfrac{P(s_{t+1} \mid \tilde{s}, \tilde{a}) \cdot P(s' \mid s_t, a_t)}{P(s_{t+1} \mid s_t, a_t)}
\end{align*}

The sum of the interventional probabilities of these transitions is:

\[
    \sum_{s' \in S_{CS}}P(s' \mid \tilde{s}, \tilde{a}) \leq \sum_{s' \in S_{CS}}\dfrac{P(s'\mid s_t, a_t) \cdot P(s_{t+1} \mid \tilde{s}, \tilde{a})}{P(s_{t+1} \mid s_t, a_t)}
\]

Let $S_{other}$ be the set of all other states $s' \in \mathcal{S}\setminus{\{s_{t+1}\}\cup S_{CS}}$. We can calculate the sum of the interventional probabilities of the transitions going to states $s' \in S_{other}$:

\begin{align*}
    \sum_{s' \in S_{other}}P(s' \mid \tilde{s}, \tilde{a}) &= 1 - P(s_{t+1} \mid \tilde{s}, \tilde{a}) - \sum_{s'' \in S_{CS}}P(s'' \mid \tilde{s}, \tilde{a})\\
    &\geq 1 - P(s_{t+1} \mid \tilde{s}, \tilde{a}) - \sum_{s'' \in S_{CS}}\dfrac{P(s'' \mid s_t, a_t) \cdot P(s_{t+1} \mid \tilde{s}, \tilde{a})}{P(s_{t+1} \mid s_t, a_t)}\\
    &= 1 - P(s_{t+1} \mid \tilde{s}, \tilde{a}) - \dfrac{P(s_{t+1} \mid \tilde{s}, \tilde{a})}{P(s_{t+1} \mid s_t, a_t)}\cdot\sum_{s'' \in S_{CS}}P(s'' \mid s_t, a_t)\\
    &= 1 - P(s_{t+1} \mid \tilde{s}, \tilde{a}) \\&- \dfrac{P(s_{t+1} \mid \tilde{s}, \tilde{a})}{P(s_{t+1} \mid s_t, a_t)}\cdot(1 - P(s_{t+1} \mid s_t, a_t) - \sum_{s' \in S_{other}}P(s' \mid s_t, a_t))\\
    &= 1 - P(s_{t+1} \mid \tilde{s}, \tilde{a}) - \dfrac{P(s_{t+1} \mid \tilde{s}, \tilde{a})}{P(s_{t+1} \mid s_t, a_t)}\\ &+ P(s_{t+1} \mid \tilde{s}, \tilde{a}) + \dfrac{P(s_{t+1} \mid \tilde{s}, \tilde{a}) \cdot \sum_{s' \in S_{other}}P(s' \mid s_t, a_t)}{P(s_{t+1} \mid s_t, a_t)}\\
    &= 1 - \dfrac{P(s_{t+1} \mid \tilde{s}, \tilde{a})}{P(s_{t+1} \mid s_t, a_t)} + \dfrac{P(s_{t+1} \mid \tilde{s}, \tilde{a}) \cdot \sum_{s' \in S_{other}}P(s' \mid s_t, a_t)}{P(s_{t+1} \mid s_t, a_t)}
\end{align*}

Because of the Mon2 \eqref{proofeq:monotonicity2} and CS \eqref{proofeq:counterfactual stability} constraints, we know that $\forall s'' \in S_{CS}, \sum_{\substack{u_t \in U_t\\f(s_t, a_t, u_t) = s_{t+1} \\ f(\tilde{s}, \tilde{a}, u_t) = s''}}{\theta_{u_t}} =0$, and $s' \in S_{other}, \sum_{\substack{u_t \in U_t\\f(s_t, a_t, u_t) = s_{t+1} \\ f(\tilde{s}, \tilde{a}, u_t) = s'}}{\theta_{u_t}} \leq P(s' \mid \tilde{s}, \tilde{a}) \cdot P(s_{t+1} \mid s_t, a_t)$. So,

\[\max_{\theta}\left(\sum_{s'' \in S_{CS}}\sum_{\substack{u_t \in U_t\\f(s_t, a_t, u_t) = s_{t+1} \\ f(\tilde{s}, \tilde{a}, u_t) = s''}}{\theta_{u_t}}\right) = 0
\]
\[\max_{\theta}\left(\sum_{s' \in S_{other}}\sum_{\substack{u_t \in U_t\\f(s_t, a_t, u_t) = s_{t+1} \\ f(\tilde{s}, \tilde{a}, u_t) = s'}}{\theta_{u_t}}\right) \leq \sum_{s' \in S_{other}}P(s' \mid \tilde{s}, \tilde{a}) \cdot P(s_{t+1} \mid s_t, a_t)\]

Therefore, the maximum $\sum_{s' \in \mathcal{S}\setminus \{s_{t+1}\}}\sum_{\substack{u_t \in U_t\\f(s_t, a_t, u_t) = s_{t+1} \\ f(\tilde{s}, \tilde{a}, u_t) = s'}}{\theta_{u_t}}$ is as follows:

\begin{equation}
    \begin{aligned}
        \max_{\theta}\left(\sum_{s' \in \mathcal{S}\setminus \{s_{t+1}\}}\sum_{\substack{u_t \in U_t\\f(s_t, a_t, u_t) = s_{t+1} \\ f(\tilde{s}, \tilde{a}, u_t) = s'}}{\theta_{u_t}}\right) &= \max_{\theta}\left(\sum_{s'' \in S_{CS}}\sum_{\substack{u_t \in U_t\\f(s_t, a_t, u_t) = s_{t+1} \\ f(\tilde{s}, \tilde{a}, u_t) = s''}}{\theta_{u_t}}\right) + \max_{\theta}\left(\sum_{s' \in S_{other}}\sum_{\substack{u_t \in U_t\\f(s_t, a_t, u_t) = s_{t+1} \\ f(\tilde{s}, \tilde{a}, u_t) = s'}}{\theta_{u_t}}\right)\\
        &= \sum_{s' \in S_{other}}P(s'' \mid \tilde{s}, \tilde{a}) \cdot P(s_{t+1} \mid s_t, a_t)\\
        &\geq P(s_{t+1} \mid s_t, a_t) - P(s_{t+1} \mid \tilde{s}, \tilde{a}) + P(s_{t+1} \mid \tilde{s}, \tilde{a}) \cdot \sum_{s' \in S_{other}}P(s' \mid s_t, a_t)\\
        &\geq P(s_{t+1} \mid s_t, a_t) - P(s_{t+1} \mid \tilde{s}, \tilde{a})
    \end{aligned}
\end{equation} 
\end{proof}

\begin{table*}[t]
\centering
\fontsize{11pt}{11pt}\selectfont
\begin{tabular}{lllllllllllll}
\toprule
\multicolumn{1}{c}{\textbf{task}} & \multicolumn{2}{c}{\textbf{Mir}} & \multicolumn{2}{c}{\textbf{Lai}} & \multicolumn{2}{c}{\textbf{Ziegen.}} & \multicolumn{2}{c}{\textbf{Cao}} & \multicolumn{2}{c}{\textbf{Alva-Man.}} & \multicolumn{1}{c}{\textbf{avg.}} & \textbf{\begin{tabular}[c]{@{}l@{}}avg.\\ rank\end{tabular}} \\
\multicolumn{1}{c}{\textbf{metrics}} & \multicolumn{1}{c}{\textbf{cor.}} & \multicolumn{1}{c}{\textbf{p-v.}} & \multicolumn{1}{c}{\textbf{cor.}} & \multicolumn{1}{c}{\textbf{p-v.}} & \multicolumn{1}{c}{\textbf{cor.}} & \multicolumn{1}{c}{\textbf{p-v.}} & \multicolumn{1}{c}{\textbf{cor.}} & \multicolumn{1}{c}{\textbf{p-v.}} & \multicolumn{1}{c}{\textbf{cor.}} & \multicolumn{1}{c}{\textbf{p-v.}} &  &  \\ \midrule
\textbf{S-Bleu} & 0.50 & 0.0 & 0.47 & 0.0 & 0.59 & 0.0 & 0.58 & 0.0 & 0.68 & 0.0 & 0.57 & 5.8 \\
\textbf{R-Bleu} & -- & -- & 0.27 & 0.0 & 0.30 & 0.0 & -- & -- & -- & -- & - &  \\
\textbf{S-Meteor} & 0.49 & 0.0 & 0.48 & 0.0 & 0.61 & 0.0 & 0.57 & 0.0 & 0.64 & 0.0 & 0.56 & 6.1 \\
\textbf{R-Meteor} & -- & -- & 0.34 & 0.0 & 0.26 & 0.0 & -- & -- & -- & -- & - &  \\
\textbf{S-Bertscore} & \textbf{0.53} & 0.0 & {\ul 0.80} & 0.0 & \textbf{0.70} & 0.0 & {\ul 0.66} & 0.0 & {\ul0.78} & 0.0 & \textbf{0.69} & \textbf{1.7} \\
\textbf{R-Bertscore} & -- & -- & 0.51 & 0.0 & 0.38 & 0.0 & -- & -- & -- & -- & - &  \\
\textbf{S-Bleurt} & {\ul 0.52} & 0.0 & {\ul 0.80} & 0.0 & 0.60 & 0.0 & \textbf{0.70} & 0.0 & \textbf{0.80} & 0.0 & {\ul 0.68} & {\ul 2.3} \\
\textbf{R-Bleurt} & -- & -- & 0.59 & 0.0 & -0.05 & 0.13 & -- & -- & -- & -- & - &  \\
\textbf{S-Cosine} & 0.51 & 0.0 & 0.69 & 0.0 & {\ul 0.62} & 0.0 & 0.61 & 0.0 & 0.65 & 0.0 & 0.62 & 4.4 \\
\textbf{R-Cosine} & -- & -- & 0.40 & 0.0 & 0.29 & 0.0 & -- & -- & -- & -- & - & \\ \midrule
\textbf{QuestEval} & 0.23 & 0.0 & 0.25 & 0.0 & 0.49 & 0.0 & 0.47 & 0.0 & 0.62 & 0.0 & 0.41 & 9.0 \\
\textbf{LLaMa3} & 0.36 & 0.0 & \textbf{0.84} & 0.0 & {\ul{0.62}} & 0.0 & 0.61 & 0.0 &  0.76 & 0.0 & 0.64 & 3.6 \\
\textbf{our (3b)} & 0.49 & 0.0 & 0.73 & 0.0 & 0.54 & 0.0 & 0.53 & 0.0 & 0.7 & 0.0 & 0.60 & 5.8 \\
\textbf{our (8b)} & 0.48 & 0.0 & 0.73 & 0.0 & 0.52 & 0.0 & 0.53 & 0.0 & 0.7 & 0.0 & 0.59 & 6.3 \\  \bottomrule
\end{tabular}
\caption{Pearson correlation on human evaluation on system output. `R-': reference-based. `S-': source-based.}
\label{tab:sys}
\end{table*}



\begin{table}%[]
\centering
\fontsize{11pt}{11pt}\selectfont
\begin{tabular}{llllll}
\toprule
\multicolumn{1}{c}{\textbf{task}} & \multicolumn{1}{c}{\textbf{Lai}} & \multicolumn{1}{c}{\textbf{Zei.}} & \multicolumn{1}{c}{\textbf{Scia.}} & \textbf{} & \textbf{} \\ 
\multicolumn{1}{c}{\textbf{metrics}} & \multicolumn{1}{c}{\textbf{cor.}} & \multicolumn{1}{c}{\textbf{cor.}} & \multicolumn{1}{c}{\textbf{cor.}} & \textbf{avg.} & \textbf{\begin{tabular}[c]{@{}l@{}}avg.\\ rank\end{tabular}} \\ \midrule
\textbf{S-Bleu} & 0.40 & 0.40 & 0.19* & 0.33 & 7.67 \\
\textbf{S-Meteor} & 0.41 & 0.42 & 0.16* & 0.33 & 7.33 \\
\textbf{S-BertS.} & {\ul0.58} & 0.47 & 0.31 & 0.45 & 3.67 \\
\textbf{S-Bleurt} & 0.45 & {\ul 0.54} & {\ul 0.37} & 0.45 & {\ul 3.33} \\
\textbf{S-Cosine} & 0.56 & 0.52 & 0.3 & {\ul 0.46} & {\ul 3.33} \\ \midrule
\textbf{QuestE.} & 0.27 & 0.35 & 0.06* & 0.23 & 9.00 \\
\textbf{LlaMA3} & \textbf{0.6} & \textbf{0.67} & \textbf{0.51} & \textbf{0.59} & \textbf{1.0} \\
\textbf{Our (3b)} & 0.51 & 0.49 & 0.23* & 0.39 & 4.83 \\
\textbf{Our (8b)} & 0.52 & 0.49 & 0.22* & 0.43 & 4.83 \\ \bottomrule
\end{tabular}
\caption{Pearson correlation on human ratings on reference output. *not significant; we cannot reject the null hypothesis of zero correlation}
\label{tab:ref}
\end{table}


\begin{table*}%[]
\centering
\fontsize{11pt}{11pt}\selectfont
\begin{tabular}{lllllllll}
\toprule
\textbf{task} & \multicolumn{1}{c}{\textbf{ALL}} & \multicolumn{1}{c}{\textbf{sentiment}} & \multicolumn{1}{c}{\textbf{detoxify}} & \multicolumn{1}{c}{\textbf{catchy}} & \multicolumn{1}{c}{\textbf{polite}} & \multicolumn{1}{c}{\textbf{persuasive}} & \multicolumn{1}{c}{\textbf{formal}} & \textbf{\begin{tabular}[c]{@{}l@{}}avg. \\ rank\end{tabular}} \\
\textbf{metrics} & \multicolumn{1}{c}{\textbf{cor.}} & \multicolumn{1}{c}{\textbf{cor.}} & \multicolumn{1}{c}{\textbf{cor.}} & \multicolumn{1}{c}{\textbf{cor.}} & \multicolumn{1}{c}{\textbf{cor.}} & \multicolumn{1}{c}{\textbf{cor.}} & \multicolumn{1}{c}{\textbf{cor.}} &  \\ \midrule
\textbf{S-Bleu} & -0.17 & -0.82 & -0.45 & -0.12* & -0.1* & -0.05 & -0.21 & 8.42 \\
\textbf{R-Bleu} & - & -0.5 & -0.45 &  &  &  &  &  \\
\textbf{S-Meteor} & -0.07* & -0.55 & -0.4 & -0.01* & 0.1* & -0.16 & -0.04* & 7.67 \\
\textbf{R-Meteor} & - & -0.17* & -0.39 & - & - & - & - & - \\
\textbf{S-BertScore} & 0.11 & -0.38 & -0.07* & -0.17* & 0.28 & 0.12 & 0.25 & 6.0 \\
\textbf{R-BertScore} & - & -0.02* & -0.21* & - & - & - & - & - \\
\textbf{S-Bleurt} & 0.29 & 0.05* & 0.45 & 0.06* & 0.29 & 0.23 & 0.46 & 4.2 \\
\textbf{R-Bleurt} & - &  0.21 & 0.38 & - & - & - & - & - \\
\textbf{S-Cosine} & 0.01* & -0.5 & -0.13* & -0.19* & 0.05* & -0.05* & 0.15* & 7.42 \\
\textbf{R-Cosine} & - & -0.11* & -0.16* & - & - & - & - & - \\ \midrule
\textbf{QuestEval} & 0.21 & {\ul{0.29}} & 0.23 & 0.37 & 0.19* & 0.35 & 0.14* & 4.67 \\
\textbf{LlaMA3} & \textbf{0.82} & \textbf{0.80} & \textbf{0.72} & \textbf{0.84} & \textbf{0.84} & \textbf{0.90} & \textbf{0.88} & \textbf{1.00} \\
\textbf{Our (3b)} & 0.47 & -0.11* & 0.37 & 0.61 & 0.53 & 0.54 & 0.66 & 3.5 \\
\textbf{Our (8b)} & {\ul{0.57}} & 0.09* & {\ul 0.49} & {\ul 0.72} & {\ul 0.64} & {\ul 0.62} & {\ul 0.67} & {\ul 2.17} \\ \bottomrule
\end{tabular}
\caption{Pearson correlation on human ratings on our constructed test set. 'R-': reference-based. 'S-': source-based. *not significant; we cannot reject the null hypothesis of zero correlation}
\label{tab:con}
\end{table*}

\section{Results}
We benchmark the different metrics on the different datasets using correlation to human judgement. For content preservation, we show results split on data with system output, reference output and our constructed test set: we show that the data source for evaluation leads to different conclusions on the metrics. In addition, we examine whether the metrics can rank style transfer systems similar to humans. On style strength, we likewise show correlations between human judgment and zero-shot evaluation approaches. When applicable, we summarize results by reporting the average correlation. And the average ranking of the metric per dataset (by ranking which metric obtains the highest correlation to human judgement per dataset). 

\subsection{Content preservation}
\paragraph{How do data sources affect the conclusion on best metric?}
The conclusions about the metrics' performance change radically depending on whether we use system output data, reference output, or our constructed test set. Ideally, a good metric correlates highly with humans on any data source. Ideally, for meta-evaluation, a metric should correlate consistently across all data sources, but the following shows that the correlations indicate different things, and the conclusion on the best metric should be drawn carefully.

Looking at the metrics correlations with humans on the data source with system output (Table~\ref{tab:sys}), we see a relatively high correlation for many of the metrics on many tasks. The overall best metrics are S-BertScore and S-BLEURT (avg+avg rank). We see no notable difference in our method of using the 3B or 8B model as the backbone.

Examining the average correlations based on data with reference output (Table~\ref{tab:ref}), now the zero-shoot prompting with LlaMA3 70B is the best-performing approach ($0.59$ avg). Tied for second place are source-based cosine embedding ($0.46$ avg), BLEURT ($0.45$ avg) and BertScore ($0.45$ avg). Our method follows on a 5. place: here, the 8b version (($0.43$ avg)) shows a bit stronger results than 3b ($0.39$ avg). The fact that the conclusions change, whether looking at reference or system output, confirms the observations made by \citet{scialom-etal-2021-questeval} on simplicity transfer.   

Now consider the results on our test set (Table~\ref{tab:con}): Several metrics show low or no correlation; we even see a significantly negative correlation for some metrics on ALL (BLEU) and for specific subparts of our test set for BLEU, Meteor, BertScore, Cosine. On the other end, LlaMA3 70B is again performing best, showing strong results ($0.82$ in ALL). The runner-up is now our 8B method, with a gap to the 3B version ($0.57$ vs $0.47$ in ALL). Note our method still shows zero correlation for the sentiment task. After, ranks BLEURT ($0.29$), QuestEval ($0.21$), BertScore ($0.11$), Cosine ($0.01$).  

On our test set, we find that some metrics that correlate relatively well on the other datasets, now exhibit low correlation. Hence, with our test set, we can now support the logical reasoning with data evidence: Evaluation of content preservation for style transfer needs to take the style shift into account. This conclusion could not be drawn using the existing data sources: We hypothesise that for the data with system-based output, successful output happens to be very similar to the source sentence and vice versa, and reference-based output might not contain server mistakes as they are gold references. Thus, none of the existing data sources tests the limits of the metrics.  


\paragraph{How do reference-based metrics compare to source-based ones?} Reference-based metrics show a lower correlation than the source-based counterpart for all metrics on both datasets with ratings on references (Table~\ref{tab:sys}). As discussed previously, reference-based metrics for style transfer have the drawback that many different good solutions on a rewrite might exist and not only one similar to a reference.


\paragraph{How well can the metrics rank the performance of style transfer methods?}
We compare the metrics' ability to judge the best style transfer methods w.r.t. the human annotations: Several of the data sources contain samples from different style transfer systems. In order to use metrics to assess the quality of the style transfer system, metrics should correctly find the best-performing system. Hence, we evaluate whether the metrics for content preservation provide the same system ranking as human evaluators. We take the mean of the score for every output on each system and the mean of the human annotations; we compare the systems using the Kendall's Tau correlation. 

We find only the evaluation using the dataset Mir, Lai, and Ziegen to result in significant correlations, probably because of sparsity in a number of system tests (App.~\ref{app:dataset}). Our method (8b) is the only metric providing a perfect ranking of the style transfer system on the Lai data, and Llama3 70B the only one on the Ziegen data. Results in App.~\ref{app:results}. 


\subsection{Style strength results}
%Evaluating style strengths is a challenging task. 
Llama3 70B shows better overall results than our method. However, our method scores higher than Llama3 70B on 2 out of 6 datasets, but it also exhibits zero correlation on one task (Table~\ref{tab:styleresults}).%More work i s needed on evaluating style strengths. 
 
\begin{table}%[]
\fontsize{11pt}{11pt}\selectfont
\begin{tabular}{lccc}
\toprule
\multicolumn{1}{c}{\textbf{}} & \textbf{LlaMA3} & \textbf{Our (3b)} & \textbf{Our (8b)} \\ \midrule
\textbf{Mir} & 0.46 & 0.54 & \textbf{0.57} \\
\textbf{Lai} & \textbf{0.57} & 0.18 & 0.19 \\
\textbf{Ziegen.} & 0.25 & 0.27 & \textbf{0.32} \\
\textbf{Alva-M.} & \textbf{0.59} & 0.03* & 0.02* \\
\textbf{Scialom} & \textbf{0.62} & 0.45 & 0.44 \\
\textbf{\begin{tabular}[c]{@{}l@{}}Our Test\end{tabular}} & \textbf{0.63} & 0.46 & 0.48 \\ \bottomrule
\end{tabular}
\caption{Style strength: Pearson correlation to human ratings. *not significant; we cannot reject the null hypothesis of zero corelation}
\label{tab:styleresults}
\end{table}

\subsection{Ablation}
We conduct several runs of the methods using LLMs with variations in instructions/prompts (App.~\ref{app:method}). We observe that the lower the correlation on a task, the higher the variation between the different runs. For our method, we only observe low variance between the runs.
None of the variations leads to different conclusions of the meta-evaluation. Results in App.~\ref{app:results}.
\section{Related Work}
The landscape of large language model vulnerabilities has been extensively studied in recent literature \cite{crothers2023machinegeneratedtextcomprehensive,shayegani2023surveyvulnerabilitieslargelanguage,Yao_2024,Huang2023ASO}, that propose detailed taxonomies of threats. These works categorize LLM attacks into distinct types, such as adversarial attacks, data poisoning, and specific vulnerabilities related to prompt engineering. Among these, prompt injection attacks have emerged as a significant and distinct category, underscoring their relevance to LLM security.

The following high-level overview of the collected taxonomy of LLM vulnerabilities is defined in \cite{Yao_2024}:
\begin{itemize}
    \item Adversarial Attacks: Data Poisoning, Backdoor Attacks
    \item Inference Attacks: Attribute Inference, Membership Inferences
    \item Extraction Attacks
    \item Bias and Unfairness
Exploitation
    \item Instruction Tuning Attacks: Jailbreaking, Prompt Injection.
\end{itemize}
Prompt injection attacks are further classified in \cite{shayegani2023surveyvulnerabilitieslargelanguage} into the following: Goal hijacking and \textbf{Prompt leakage}.

The reviewed taxonomies underscore the need for comprehensive frameworks to evaluate LLM security. The agentic approach introduced in this paper builds on these insights, automating adversarial testing to address a wide range of scenarios, including those involving prompt leakage and role-specific vulnerabilities.

\subsection{Prompt Injection and Prompt Leakage}

Prompt injection attacks exploit the blending of instructional and data inputs, manipulating LLMs into deviating from their intended behavior. Prompt injection attacks encompass techniques that override initial instructions, expose private prompts, or generate malicious outputs \cite{Huang2023ASO}. A subset of these attacks, known as prompt leakage, aims specifically at extracting sensitive system prompts embedded within LLM configurations. In \cite{shayegani2023surveyvulnerabilitieslargelanguage}, authors differentiate between prompt leakage and related methods such as goal hijacking, further refining the taxonomy of LLM-specific vulnerabilities.

\subsection{Defense Mechanisms}

Various defense mechanisms have been proposed to address LLM vulnerabilities, particularly prompt injection and leakage \cite{shayegani2023surveyvulnerabilitieslargelanguage,Yao_2024}. We focused on cost-effective methods like instruction postprocessing and prompt engineering, which are viable for proprietary models that cannot be retrained. Instruction preprocessing sanitizes inputs, while postprocessing removes harmful outputs, forming a dual-layer defense. Preprocessing methods include perplexity-based filtering \cite{Jain2023BaselineDF,Xu2022ExploringTU} and token-level analysis \cite{Kumar2023CertifyingLS}. Postprocessing employs another set of techniques, such as censorship by LLMs \cite{Helbling2023LLMSD,Inan2023LlamaGL}, and use of canary tokens and pattern matching \cite{vigil-llm,rebuff}, although their fundamental limitations are noted \cite{Glukhov2023LLMCA}. Prompt engineering employs carefully designed instructions \cite{Schulhoff2024ThePR} and advanced techniques like spotlighting \cite{Hines2024DefendingAI} to mitigate vulnerabilities, though no method is foolproof \cite{schulhoff-etal-2023-ignore}. Adversarial training, by incorporating adversarial examples into the training process, strengthens models against attacks \cite{Bespalov2024TowardsBA,Shaham2015UnderstandingAT}.

\subsection{Security Testing for Prompt Injection Attacks}

Manual testing, such as red teaming \cite{ganguli2022redteaminglanguagemodels} and handcrafted "Ignore Previous Prompt" attacks \cite{Perez2022IgnorePP}, highlights vulnerabilities but is limited in scale. Automated approaches like PAIR \cite{chao2024jailbreakingblackboxlarge} and GPTFUZZER \cite{Yu2023GPTFUZZERRT} achieve higher success rates by refining prompts iteratively or via automated fuzzing. Red teaming with LLMs \cite{Perez2022RedTL} and reinforcement learning \cite{anonymous2024diverse} uncovers diverse vulnerabilities, including data leakage and offensive outputs. Indirect Prompt Injection (IPI) manipulates external data to compromise applications \cite{Greshake2023NotWY}, adapting techniques like SQL injection to LLMs \cite{Liu2023PromptIA}. Prompt secrecy remains fragile, with studies showing reliable prompt extraction \cite{Zhang2023EffectivePE}. Advanced frameworks like Token Space Projection \cite{Maus2023AdversarialPF} and Weak-to-Strong Jailbreaking Attacks \cite{zhao2024weaktostrongjailbreakinglargelanguage} exploit token-space relationships, achieving high success rates for prompt extraction and jailbreaking.

\subsection{Agentic Frameworks for Evaluating LLM Security}

The development of multi-agent systems leveraging large language models (LLMs) has shown promising results in enhancing task-solving capabilities \cite{Hong2023MetaGPTMP, Wang2023UnleashingTE, Talebirad2023MultiAgentCH, Wu2023AutoGenEN, Du2023ImprovingFA}. A key aspect across various frameworks is the specialization of roles among agents \cite{Hong2023MetaGPTMP, Wu2023AutoGenEN}, which mimics human collaboration and improves task decomposition.

Agentic frameworks and the multi-agent debate approach benefit from agent interaction, where agents engage in conversations or debates to refine outputs and correct errors \cite{Wu2023AutoGenEN}. For example, debate systems improve factual accuracy and reasoning by iteratively refining responses through collaborative reasoning \cite{Du2023ImprovingFA}, while AG2 allows agents to autonomously interact and execute tasks with minimal human input.

These frameworks highlight the viability of agentic systems, showing how specialized roles and collaborative mechanisms lead to improved performance, whether in factuality, reasoning, or task execution. By leveraging the strengths of diverse agents, these systems demonstrate a scalable approach to problem-solving.

Recent research on testing LLMs using other LLMs has shown that this approach can be highly effective \cite{chao2024jailbreakingblackboxlarge, Yu2023GPTFUZZERRT, Perez2022RedTL}. Although the papers do not explicitly employ agentic frameworks they inherently reflect a pattern similar to that of an "attacker" and a "judge". \cite{chao2024jailbreakingblackboxlarge}  This pattern became a focal point for our work, where we put the judge into a more direct dialogue, enabling it to generate attacks based on the tested agent response in an active conversation.

A particularly influential paper in shaping our approach is Jailbreaking Black Box Large Language Models in Twenty Queries \cite{chao2024jailbreakingblackboxlarge}. This paper not only introduced the attacker/judge architecture but also provided the initial system prompts used for a judge.
% TODO
% more detailed introduction of dataset creation
% the rumour label in such datasets
\section{Data} \label{sec:data}
We use three rumour datasets in this work, namely: PHEME~\citep{pheme2015,kochkina-etal-2018-one}, Twitter15, and Twitter16~\citep{ma-etal-2017-detect}:

% TJB: how can the number of threads be greater than the number of tweets? these numbers don't make sense
% RX: fixed, the numbers were incorrect
\paragraph{PHEME}~\citet{pheme2015} contains 6,425 tweet posts of rumours and non-rumours related to 9 events. To avoid using specific a priori keywords to search for tweet posts, PHEME used the Twitter (now X) steaming API to identify newsworthy events from breaking news and then selected from candidate rumours that met rumour criteria, finally they collected associated conversations and annotate them. They engaged journalists to annotate the threads. The data were collected between 2014 and 2015. The 9 events are split into two groups, the first being breaking news that contains rumours, including Ferguson unrest, Ottawa shooting, Sydney siege, Charlie Hebdo shooting, and Germanwings plane crash. The rest are specific rumours, namely Prince to play in Toronto, Gurlitt collection, Putin missing, and Michael Essien contracting Ebola.
% TJB: say something about the time period when this data was collected
% RX: added

\paragraph{Twitter 15}~\citet{twitter15} was constructed by collecting rumour and non-rumour posts from the tracking websites snopes.com and emergent.info. They then used the Twitter API to gather corresponding posts, resulting in 94 true and 446 false posts. This dataset further includes 1,490 root posts and their follow posts, comprising 1,116 rumours and 374 non-rumours.
% TJB: the "tweet" vs. "comment" terminology is potentially confusing and needs to be clarified
% RX: unified, used root and follow posts to refer to root posts and the comment posts, posts are used to describe tweets in general.

\paragraph{Twitter 16}
Similarly to Twitter 15, \citet{twitter16} collected rumours and non-rumours from snopes.com, resulting in 778 reported events, 64\% of which are rumours. For each event, keywords were extracted from the final part of the Snopes URL and refined manually---adding, deleting, or replacing words iteratively---until the composed queries yielded precise Twitter search results. The final dataset includes 1,490 root tweet posts and their follow posts, comprising 613 rumours and 205 non-rumours.

\begin{table*}[!t]
    \centering
    \small
    \begin{tabular}{p{0.05\linewidth}p{0.9\linewidth}}
    \toprule
    Task & Prompt \\
    \midrule
    V-oc & Categorize the text into an ordinal class that best characterizes the writer's mental state, considering various degrees of positive and negative sentiment intensity. 3: very positive mental state can be inferred. 2: moderately positive mental state can be inferred. 1: slightly positive mental state can be inferred. 0: neutral or mixed mental state can be inferred. -1: slightly negative mental state can be inferred. -2: moderately negative mental state can be inferred. -3: very negative mental state can be inferred.\\
    \midrule
    E-c & Categorize the text's emotional tone as either `neutral or no emotion' or identify the presence of one or more of the given emotions (anger, anticipation, disgust, fear, joy, love, optimism, pessimism, sadness, surprise, trust).\\
    \midrule
    E-i & Assign a numerical value between 0 (least E) and 1 (most E) to represent the intensity of emotion E expressed in the text.\\
    \bottomrule
    \end{tabular}
    \caption{Prompts used for EmoLLM to detect emotion information in tweets. V-oc = Valence Ordinal Classification, E-c = Emotion Classification, and E-i = Emotion Intensity Regression.}
    \label{tab:emollm_ins}
\end{table*}


  
%%% Local Variables:
%%% mode: latex
%%% TeX-master: "../main_anonymous"
%%% End:


\bibliographystyle{ACM-Reference-Format}
\bibliography{main}

\appendix
\subsection{Lloyd-Max Algorithm}
\label{subsec:Lloyd-Max}
For a given quantization bitwidth $B$ and an operand $\bm{X}$, the Lloyd-Max algorithm finds $2^B$ quantization levels $\{\hat{x}_i\}_{i=1}^{2^B}$ such that quantizing $\bm{X}$ by rounding each scalar in $\bm{X}$ to the nearest quantization level minimizes the quantization MSE. 

The algorithm starts with an initial guess of quantization levels and then iteratively computes quantization thresholds $\{\tau_i\}_{i=1}^{2^B-1}$ and updates quantization levels $\{\hat{x}_i\}_{i=1}^{2^B}$. Specifically, at iteration $n$, thresholds are set to the midpoints of the previous iteration's levels:
\begin{align*}
    \tau_i^{(n)}=\frac{\hat{x}_i^{(n-1)}+\hat{x}_{i+1}^{(n-1)}}2 \text{ for } i=1\ldots 2^B-1
\end{align*}
Subsequently, the quantization levels are re-computed as conditional means of the data regions defined by the new thresholds:
\begin{align*}
    \hat{x}_i^{(n)}=\mathbb{E}\left[ \bm{X} \big| \bm{X}\in [\tau_{i-1}^{(n)},\tau_i^{(n)}] \right] \text{ for } i=1\ldots 2^B
\end{align*}
where to satisfy boundary conditions we have $\tau_0=-\infty$ and $\tau_{2^B}=\infty$. The algorithm iterates the above steps until convergence.

Figure \ref{fig:lm_quant} compares the quantization levels of a $7$-bit floating point (E3M3) quantizer (left) to a $7$-bit Lloyd-Max quantizer (right) when quantizing a layer of weights from the GPT3-126M model at a per-tensor granularity. As shown, the Lloyd-Max quantizer achieves substantially lower quantization MSE. Further, Table \ref{tab:FP7_vs_LM7} shows the superior perplexity achieved by Lloyd-Max quantizers for bitwidths of $7$, $6$ and $5$. The difference between the quantizers is clear at 5 bits, where per-tensor FP quantization incurs a drastic and unacceptable increase in perplexity, while Lloyd-Max quantization incurs a much smaller increase. Nevertheless, we note that even the optimal Lloyd-Max quantizer incurs a notable ($\sim 1.5$) increase in perplexity due to the coarse granularity of quantization. 

\begin{figure}[h]
  \centering
  \includegraphics[width=0.7\linewidth]{sections/figures/LM7_FP7.pdf}
  \caption{\small Quantization levels and the corresponding quantization MSE of Floating Point (left) vs Lloyd-Max (right) Quantizers for a layer of weights in the GPT3-126M model.}
  \label{fig:lm_quant}
\end{figure}

\begin{table}[h]\scriptsize
\begin{center}
\caption{\label{tab:FP7_vs_LM7} \small Comparing perplexity (lower is better) achieved by floating point quantizers and Lloyd-Max quantizers on a GPT3-126M model for the Wikitext-103 dataset.}
\begin{tabular}{c|cc|c}
\hline
 \multirow{2}{*}{\textbf{Bitwidth}} & \multicolumn{2}{|c|}{\textbf{Floating-Point Quantizer}} & \textbf{Lloyd-Max Quantizer} \\
 & Best Format & Wikitext-103 Perplexity & Wikitext-103 Perplexity \\
\hline
7 & E3M3 & 18.32 & 18.27 \\
6 & E3M2 & 19.07 & 18.51 \\
5 & E4M0 & 43.89 & 19.71 \\
\hline
\end{tabular}
\end{center}
\end{table}

\subsection{Proof of Local Optimality of LO-BCQ}
\label{subsec:lobcq_opt_proof}
For a given block $\bm{b}_j$, the quantization MSE during LO-BCQ can be empirically evaluated as $\frac{1}{L_b}\lVert \bm{b}_j- \bm{\hat{b}}_j\rVert^2_2$ where $\bm{\hat{b}}_j$ is computed from equation (\ref{eq:clustered_quantization_definition}) as $C_{f(\bm{b}_j)}(\bm{b}_j)$. Further, for a given block cluster $\mathcal{B}_i$, we compute the quantization MSE as $\frac{1}{|\mathcal{B}_{i}|}\sum_{\bm{b} \in \mathcal{B}_{i}} \frac{1}{L_b}\lVert \bm{b}- C_i^{(n)}(\bm{b})\rVert^2_2$. Therefore, at the end of iteration $n$, we evaluate the overall quantization MSE $J^{(n)}$ for a given operand $\bm{X}$ composed of $N_c$ block clusters as:
\begin{align*}
    \label{eq:mse_iter_n}
    J^{(n)} = \frac{1}{N_c} \sum_{i=1}^{N_c} \frac{1}{|\mathcal{B}_{i}^{(n)}|}\sum_{\bm{v} \in \mathcal{B}_{i}^{(n)}} \frac{1}{L_b}\lVert \bm{b}- B_i^{(n)}(\bm{b})\rVert^2_2
\end{align*}

At the end of iteration $n$, the codebooks are updated from $\mathcal{C}^{(n-1)}$ to $\mathcal{C}^{(n)}$. However, the mapping of a given vector $\bm{b}_j$ to quantizers $\mathcal{C}^{(n)}$ remains as  $f^{(n)}(\bm{b}_j)$. At the next iteration, during the vector clustering step, $f^{(n+1)}(\bm{b}_j)$ finds new mapping of $\bm{b}_j$ to updated codebooks $\mathcal{C}^{(n)}$ such that the quantization MSE over the candidate codebooks is minimized. Therefore, we obtain the following result for $\bm{b}_j$:
\begin{align*}
\frac{1}{L_b}\lVert \bm{b}_j - C_{f^{(n+1)}(\bm{b}_j)}^{(n)}(\bm{b}_j)\rVert^2_2 \le \frac{1}{L_b}\lVert \bm{b}_j - C_{f^{(n)}(\bm{b}_j)}^{(n)}(\bm{b}_j)\rVert^2_2
\end{align*}

That is, quantizing $\bm{b}_j$ at the end of the block clustering step of iteration $n+1$ results in lower quantization MSE compared to quantizing at the end of iteration $n$. Since this is true for all $\bm{b} \in \bm{X}$, we assert the following:
\begin{equation}
\begin{split}
\label{eq:mse_ineq_1}
    \tilde{J}^{(n+1)} &= \frac{1}{N_c} \sum_{i=1}^{N_c} \frac{1}{|\mathcal{B}_{i}^{(n+1)}|}\sum_{\bm{b} \in \mathcal{B}_{i}^{(n+1)}} \frac{1}{L_b}\lVert \bm{b} - C_i^{(n)}(b)\rVert^2_2 \le J^{(n)}
\end{split}
\end{equation}
where $\tilde{J}^{(n+1)}$ is the the quantization MSE after the vector clustering step at iteration $n+1$.

Next, during the codebook update step (\ref{eq:quantizers_update}) at iteration $n+1$, the per-cluster codebooks $\mathcal{C}^{(n)}$ are updated to $\mathcal{C}^{(n+1)}$ by invoking the Lloyd-Max algorithm \citep{Lloyd}. We know that for any given value distribution, the Lloyd-Max algorithm minimizes the quantization MSE. Therefore, for a given vector cluster $\mathcal{B}_i$ we obtain the following result:

\begin{equation}
    \frac{1}{|\mathcal{B}_{i}^{(n+1)}|}\sum_{\bm{b} \in \mathcal{B}_{i}^{(n+1)}} \frac{1}{L_b}\lVert \bm{b}- C_i^{(n+1)}(\bm{b})\rVert^2_2 \le \frac{1}{|\mathcal{B}_{i}^{(n+1)}|}\sum_{\bm{b} \in \mathcal{B}_{i}^{(n+1)}} \frac{1}{L_b}\lVert \bm{b}- C_i^{(n)}(\bm{b})\rVert^2_2
\end{equation}

The above equation states that quantizing the given block cluster $\mathcal{B}_i$ after updating the associated codebook from $C_i^{(n)}$ to $C_i^{(n+1)}$ results in lower quantization MSE. Since this is true for all the block clusters, we derive the following result: 
\begin{equation}
\begin{split}
\label{eq:mse_ineq_2}
     J^{(n+1)} &= \frac{1}{N_c} \sum_{i=1}^{N_c} \frac{1}{|\mathcal{B}_{i}^{(n+1)}|}\sum_{\bm{b} \in \mathcal{B}_{i}^{(n+1)}} \frac{1}{L_b}\lVert \bm{b}- C_i^{(n+1)}(\bm{b})\rVert^2_2  \le \tilde{J}^{(n+1)}   
\end{split}
\end{equation}

Following (\ref{eq:mse_ineq_1}) and (\ref{eq:mse_ineq_2}), we find that the quantization MSE is non-increasing for each iteration, that is, $J^{(1)} \ge J^{(2)} \ge J^{(3)} \ge \ldots \ge J^{(M)}$ where $M$ is the maximum number of iterations. 
%Therefore, we can say that if the algorithm converges, then it must be that it has converged to a local minimum. 
\hfill $\blacksquare$


\begin{figure}
    \begin{center}
    \includegraphics[width=0.5\textwidth]{sections//figures/mse_vs_iter.pdf}
    \end{center}
    \caption{\small NMSE vs iterations during LO-BCQ compared to other block quantization proposals}
    \label{fig:nmse_vs_iter}
\end{figure}

Figure \ref{fig:nmse_vs_iter} shows the empirical convergence of LO-BCQ across several block lengths and number of codebooks. Also, the MSE achieved by LO-BCQ is compared to baselines such as MXFP and VSQ. As shown, LO-BCQ converges to a lower MSE than the baselines. Further, we achieve better convergence for larger number of codebooks ($N_c$) and for a smaller block length ($L_b$), both of which increase the bitwidth of BCQ (see Eq \ref{eq:bitwidth_bcq}).


\subsection{Additional Accuracy Results}
%Table \ref{tab:lobcq_config} lists the various LOBCQ configurations and their corresponding bitwidths.
\begin{table}
\setlength{\tabcolsep}{4.75pt}
\begin{center}
\caption{\label{tab:lobcq_config} Various LO-BCQ configurations and their bitwidths.}
\begin{tabular}{|c||c|c|c|c||c|c||c|} 
\hline
 & \multicolumn{4}{|c||}{$L_b=8$} & \multicolumn{2}{|c||}{$L_b=4$} & $L_b=2$ \\
 \hline
 \backslashbox{$L_A$\kern-1em}{\kern-1em$N_c$} & 2 & 4 & 8 & 16 & 2 & 4 & 2 \\
 \hline
 64 & 4.25 & 4.375 & 4.5 & 4.625 & 4.375 & 4.625 & 4.625\\
 \hline
 32 & 4.375 & 4.5 & 4.625& 4.75 & 4.5 & 4.75 & 4.75 \\
 \hline
 16 & 4.625 & 4.75& 4.875 & 5 & 4.75 & 5 & 5 \\
 \hline
\end{tabular}
\end{center}
\end{table}

%\subsection{Perplexity achieved by various LO-BCQ configurations on Wikitext-103 dataset}

\begin{table} \centering
\begin{tabular}{|c||c|c|c|c||c|c||c|} 
\hline
 $L_b \rightarrow$& \multicolumn{4}{c||}{8} & \multicolumn{2}{c||}{4} & 2\\
 \hline
 \backslashbox{$L_A$\kern-1em}{\kern-1em$N_c$} & 2 & 4 & 8 & 16 & 2 & 4 & 2  \\
 %$N_c \rightarrow$ & 2 & 4 & 8 & 16 & 2 & 4 & 2 \\
 \hline
 \hline
 \multicolumn{8}{c}{GPT3-1.3B (FP32 PPL = 9.98)} \\ 
 \hline
 \hline
 64 & 10.40 & 10.23 & 10.17 & 10.15 &  10.28 & 10.18 & 10.19 \\
 \hline
 32 & 10.25 & 10.20 & 10.15 & 10.12 &  10.23 & 10.17 & 10.17 \\
 \hline
 16 & 10.22 & 10.16 & 10.10 & 10.09 &  10.21 & 10.14 & 10.16 \\
 \hline
  \hline
 \multicolumn{8}{c}{GPT3-8B (FP32 PPL = 7.38)} \\ 
 \hline
 \hline
 64 & 7.61 & 7.52 & 7.48 &  7.47 &  7.55 &  7.49 & 7.50 \\
 \hline
 32 & 7.52 & 7.50 & 7.46 &  7.45 &  7.52 &  7.48 & 7.48  \\
 \hline
 16 & 7.51 & 7.48 & 7.44 &  7.44 &  7.51 &  7.49 & 7.47  \\
 \hline
\end{tabular}
\caption{\label{tab:ppl_gpt3_abalation} Wikitext-103 perplexity across GPT3-1.3B and 8B models.}
\end{table}

\begin{table} \centering
\begin{tabular}{|c||c|c|c|c||} 
\hline
 $L_b \rightarrow$& \multicolumn{4}{c||}{8}\\
 \hline
 \backslashbox{$L_A$\kern-1em}{\kern-1em$N_c$} & 2 & 4 & 8 & 16 \\
 %$N_c \rightarrow$ & 2 & 4 & 8 & 16 & 2 & 4 & 2 \\
 \hline
 \hline
 \multicolumn{5}{|c|}{Llama2-7B (FP32 PPL = 5.06)} \\ 
 \hline
 \hline
 64 & 5.31 & 5.26 & 5.19 & 5.18  \\
 \hline
 32 & 5.23 & 5.25 & 5.18 & 5.15  \\
 \hline
 16 & 5.23 & 5.19 & 5.16 & 5.14  \\
 \hline
 \multicolumn{5}{|c|}{Nemotron4-15B (FP32 PPL = 5.87)} \\ 
 \hline
 \hline
 64  & 6.3 & 6.20 & 6.13 & 6.08  \\
 \hline
 32  & 6.24 & 6.12 & 6.07 & 6.03  \\
 \hline
 16  & 6.12 & 6.14 & 6.04 & 6.02  \\
 \hline
 \multicolumn{5}{|c|}{Nemotron4-340B (FP32 PPL = 3.48)} \\ 
 \hline
 \hline
 64 & 3.67 & 3.62 & 3.60 & 3.59 \\
 \hline
 32 & 3.63 & 3.61 & 3.59 & 3.56 \\
 \hline
 16 & 3.61 & 3.58 & 3.57 & 3.55 \\
 \hline
\end{tabular}
\caption{\label{tab:ppl_llama7B_nemo15B} Wikitext-103 perplexity compared to FP32 baseline in Llama2-7B and Nemotron4-15B, 340B models}
\end{table}

%\subsection{Perplexity achieved by various LO-BCQ configurations on MMLU dataset}


\begin{table} \centering
\begin{tabular}{|c||c|c|c|c||c|c|c|c|} 
\hline
 $L_b \rightarrow$& \multicolumn{4}{c||}{8} & \multicolumn{4}{c||}{8}\\
 \hline
 \backslashbox{$L_A$\kern-1em}{\kern-1em$N_c$} & 2 & 4 & 8 & 16 & 2 & 4 & 8 & 16  \\
 %$N_c \rightarrow$ & 2 & 4 & 8 & 16 & 2 & 4 & 2 \\
 \hline
 \hline
 \multicolumn{5}{|c|}{Llama2-7B (FP32 Accuracy = 45.8\%)} & \multicolumn{4}{|c|}{Llama2-70B (FP32 Accuracy = 69.12\%)} \\ 
 \hline
 \hline
 64 & 43.9 & 43.4 & 43.9 & 44.9 & 68.07 & 68.27 & 68.17 & 68.75 \\
 \hline
 32 & 44.5 & 43.8 & 44.9 & 44.5 & 68.37 & 68.51 & 68.35 & 68.27  \\
 \hline
 16 & 43.9 & 42.7 & 44.9 & 45 & 68.12 & 68.77 & 68.31 & 68.59  \\
 \hline
 \hline
 \multicolumn{5}{|c|}{GPT3-22B (FP32 Accuracy = 38.75\%)} & \multicolumn{4}{|c|}{Nemotron4-15B (FP32 Accuracy = 64.3\%)} \\ 
 \hline
 \hline
 64 & 36.71 & 38.85 & 38.13 & 38.92 & 63.17 & 62.36 & 63.72 & 64.09 \\
 \hline
 32 & 37.95 & 38.69 & 39.45 & 38.34 & 64.05 & 62.30 & 63.8 & 64.33  \\
 \hline
 16 & 38.88 & 38.80 & 38.31 & 38.92 & 63.22 & 63.51 & 63.93 & 64.43  \\
 \hline
\end{tabular}
\caption{\label{tab:mmlu_abalation} Accuracy on MMLU dataset across GPT3-22B, Llama2-7B, 70B and Nemotron4-15B models.}
\end{table}


%\subsection{Perplexity achieved by various LO-BCQ configurations on LM evaluation harness}

\begin{table} \centering
\begin{tabular}{|c||c|c|c|c||c|c|c|c|} 
\hline
 $L_b \rightarrow$& \multicolumn{4}{c||}{8} & \multicolumn{4}{c||}{8}\\
 \hline
 \backslashbox{$L_A$\kern-1em}{\kern-1em$N_c$} & 2 & 4 & 8 & 16 & 2 & 4 & 8 & 16  \\
 %$N_c \rightarrow$ & 2 & 4 & 8 & 16 & 2 & 4 & 2 \\
 \hline
 \hline
 \multicolumn{5}{|c|}{Race (FP32 Accuracy = 37.51\%)} & \multicolumn{4}{|c|}{Boolq (FP32 Accuracy = 64.62\%)} \\ 
 \hline
 \hline
 64 & 36.94 & 37.13 & 36.27 & 37.13 & 63.73 & 62.26 & 63.49 & 63.36 \\
 \hline
 32 & 37.03 & 36.36 & 36.08 & 37.03 & 62.54 & 63.51 & 63.49 & 63.55  \\
 \hline
 16 & 37.03 & 37.03 & 36.46 & 37.03 & 61.1 & 63.79 & 63.58 & 63.33  \\
 \hline
 \hline
 \multicolumn{5}{|c|}{Winogrande (FP32 Accuracy = 58.01\%)} & \multicolumn{4}{|c|}{Piqa (FP32 Accuracy = 74.21\%)} \\ 
 \hline
 \hline
 64 & 58.17 & 57.22 & 57.85 & 58.33 & 73.01 & 73.07 & 73.07 & 72.80 \\
 \hline
 32 & 59.12 & 58.09 & 57.85 & 58.41 & 73.01 & 73.94 & 72.74 & 73.18  \\
 \hline
 16 & 57.93 & 58.88 & 57.93 & 58.56 & 73.94 & 72.80 & 73.01 & 73.94  \\
 \hline
\end{tabular}
\caption{\label{tab:mmlu_abalation} Accuracy on LM evaluation harness tasks on GPT3-1.3B model.}
\end{table}

\begin{table} \centering
\begin{tabular}{|c||c|c|c|c||c|c|c|c|} 
\hline
 $L_b \rightarrow$& \multicolumn{4}{c||}{8} & \multicolumn{4}{c||}{8}\\
 \hline
 \backslashbox{$L_A$\kern-1em}{\kern-1em$N_c$} & 2 & 4 & 8 & 16 & 2 & 4 & 8 & 16  \\
 %$N_c \rightarrow$ & 2 & 4 & 8 & 16 & 2 & 4 & 2 \\
 \hline
 \hline
 \multicolumn{5}{|c|}{Race (FP32 Accuracy = 41.34\%)} & \multicolumn{4}{|c|}{Boolq (FP32 Accuracy = 68.32\%)} \\ 
 \hline
 \hline
 64 & 40.48 & 40.10 & 39.43 & 39.90 & 69.20 & 68.41 & 69.45 & 68.56 \\
 \hline
 32 & 39.52 & 39.52 & 40.77 & 39.62 & 68.32 & 67.43 & 68.17 & 69.30  \\
 \hline
 16 & 39.81 & 39.71 & 39.90 & 40.38 & 68.10 & 66.33 & 69.51 & 69.42  \\
 \hline
 \hline
 \multicolumn{5}{|c|}{Winogrande (FP32 Accuracy = 67.88\%)} & \multicolumn{4}{|c|}{Piqa (FP32 Accuracy = 78.78\%)} \\ 
 \hline
 \hline
 64 & 66.85 & 66.61 & 67.72 & 67.88 & 77.31 & 77.42 & 77.75 & 77.64 \\
 \hline
 32 & 67.25 & 67.72 & 67.72 & 67.00 & 77.31 & 77.04 & 77.80 & 77.37  \\
 \hline
 16 & 68.11 & 68.90 & 67.88 & 67.48 & 77.37 & 78.13 & 78.13 & 77.69  \\
 \hline
\end{tabular}
\caption{\label{tab:mmlu_abalation} Accuracy on LM evaluation harness tasks on GPT3-8B model.}
\end{table}

\begin{table} \centering
\begin{tabular}{|c||c|c|c|c||c|c|c|c|} 
\hline
 $L_b \rightarrow$& \multicolumn{4}{c||}{8} & \multicolumn{4}{c||}{8}\\
 \hline
 \backslashbox{$L_A$\kern-1em}{\kern-1em$N_c$} & 2 & 4 & 8 & 16 & 2 & 4 & 8 & 16  \\
 %$N_c \rightarrow$ & 2 & 4 & 8 & 16 & 2 & 4 & 2 \\
 \hline
 \hline
 \multicolumn{5}{|c|}{Race (FP32 Accuracy = 40.67\%)} & \multicolumn{4}{|c|}{Boolq (FP32 Accuracy = 76.54\%)} \\ 
 \hline
 \hline
 64 & 40.48 & 40.10 & 39.43 & 39.90 & 75.41 & 75.11 & 77.09 & 75.66 \\
 \hline
 32 & 39.52 & 39.52 & 40.77 & 39.62 & 76.02 & 76.02 & 75.96 & 75.35  \\
 \hline
 16 & 39.81 & 39.71 & 39.90 & 40.38 & 75.05 & 73.82 & 75.72 & 76.09  \\
 \hline
 \hline
 \multicolumn{5}{|c|}{Winogrande (FP32 Accuracy = 70.64\%)} & \multicolumn{4}{|c|}{Piqa (FP32 Accuracy = 79.16\%)} \\ 
 \hline
 \hline
 64 & 69.14 & 70.17 & 70.17 & 70.56 & 78.24 & 79.00 & 78.62 & 78.73 \\
 \hline
 32 & 70.96 & 69.69 & 71.27 & 69.30 & 78.56 & 79.49 & 79.16 & 78.89  \\
 \hline
 16 & 71.03 & 69.53 & 69.69 & 70.40 & 78.13 & 79.16 & 79.00 & 79.00  \\
 \hline
\end{tabular}
\caption{\label{tab:mmlu_abalation} Accuracy on LM evaluation harness tasks on GPT3-22B model.}
\end{table}

\begin{table} \centering
\begin{tabular}{|c||c|c|c|c||c|c|c|c|} 
\hline
 $L_b \rightarrow$& \multicolumn{4}{c||}{8} & \multicolumn{4}{c||}{8}\\
 \hline
 \backslashbox{$L_A$\kern-1em}{\kern-1em$N_c$} & 2 & 4 & 8 & 16 & 2 & 4 & 8 & 16  \\
 %$N_c \rightarrow$ & 2 & 4 & 8 & 16 & 2 & 4 & 2 \\
 \hline
 \hline
 \multicolumn{5}{|c|}{Race (FP32 Accuracy = 44.4\%)} & \multicolumn{4}{|c|}{Boolq (FP32 Accuracy = 79.29\%)} \\ 
 \hline
 \hline
 64 & 42.49 & 42.51 & 42.58 & 43.45 & 77.58 & 77.37 & 77.43 & 78.1 \\
 \hline
 32 & 43.35 & 42.49 & 43.64 & 43.73 & 77.86 & 75.32 & 77.28 & 77.86  \\
 \hline
 16 & 44.21 & 44.21 & 43.64 & 42.97 & 78.65 & 77 & 76.94 & 77.98  \\
 \hline
 \hline
 \multicolumn{5}{|c|}{Winogrande (FP32 Accuracy = 69.38\%)} & \multicolumn{4}{|c|}{Piqa (FP32 Accuracy = 78.07\%)} \\ 
 \hline
 \hline
 64 & 68.9 & 68.43 & 69.77 & 68.19 & 77.09 & 76.82 & 77.09 & 77.86 \\
 \hline
 32 & 69.38 & 68.51 & 68.82 & 68.90 & 78.07 & 76.71 & 78.07 & 77.86  \\
 \hline
 16 & 69.53 & 67.09 & 69.38 & 68.90 & 77.37 & 77.8 & 77.91 & 77.69  \\
 \hline
\end{tabular}
\caption{\label{tab:mmlu_abalation} Accuracy on LM evaluation harness tasks on Llama2-7B model.}
\end{table}

\begin{table} \centering
\begin{tabular}{|c||c|c|c|c||c|c|c|c|} 
\hline
 $L_b \rightarrow$& \multicolumn{4}{c||}{8} & \multicolumn{4}{c||}{8}\\
 \hline
 \backslashbox{$L_A$\kern-1em}{\kern-1em$N_c$} & 2 & 4 & 8 & 16 & 2 & 4 & 8 & 16  \\
 %$N_c \rightarrow$ & 2 & 4 & 8 & 16 & 2 & 4 & 2 \\
 \hline
 \hline
 \multicolumn{5}{|c|}{Race (FP32 Accuracy = 48.8\%)} & \multicolumn{4}{|c|}{Boolq (FP32 Accuracy = 85.23\%)} \\ 
 \hline
 \hline
 64 & 49.00 & 49.00 & 49.28 & 48.71 & 82.82 & 84.28 & 84.03 & 84.25 \\
 \hline
 32 & 49.57 & 48.52 & 48.33 & 49.28 & 83.85 & 84.46 & 84.31 & 84.93  \\
 \hline
 16 & 49.85 & 49.09 & 49.28 & 48.99 & 85.11 & 84.46 & 84.61 & 83.94  \\
 \hline
 \hline
 \multicolumn{5}{|c|}{Winogrande (FP32 Accuracy = 79.95\%)} & \multicolumn{4}{|c|}{Piqa (FP32 Accuracy = 81.56\%)} \\ 
 \hline
 \hline
 64 & 78.77 & 78.45 & 78.37 & 79.16 & 81.45 & 80.69 & 81.45 & 81.5 \\
 \hline
 32 & 78.45 & 79.01 & 78.69 & 80.66 & 81.56 & 80.58 & 81.18 & 81.34  \\
 \hline
 16 & 79.95 & 79.56 & 79.79 & 79.72 & 81.28 & 81.66 & 81.28 & 80.96  \\
 \hline
\end{tabular}
\caption{\label{tab:mmlu_abalation} Accuracy on LM evaluation harness tasks on Llama2-70B model.}
\end{table}

%\section{MSE Studies}
%\textcolor{red}{TODO}


\subsection{Number Formats and Quantization Method}
\label{subsec:numFormats_quantMethod}
\subsubsection{Integer Format}
An $n$-bit signed integer (INT) is typically represented with a 2s-complement format \citep{yao2022zeroquant,xiao2023smoothquant,dai2021vsq}, where the most significant bit denotes the sign.

\subsubsection{Floating Point Format}
An $n$-bit signed floating point (FP) number $x$ comprises of a 1-bit sign ($x_{\mathrm{sign}}$), $B_m$-bit mantissa ($x_{\mathrm{mant}}$) and $B_e$-bit exponent ($x_{\mathrm{exp}}$) such that $B_m+B_e=n-1$. The associated constant exponent bias ($E_{\mathrm{bias}}$) is computed as $(2^{{B_e}-1}-1)$. We denote this format as $E_{B_e}M_{B_m}$.  

\subsubsection{Quantization Scheme}
\label{subsec:quant_method}
A quantization scheme dictates how a given unquantized tensor is converted to its quantized representation. We consider FP formats for the purpose of illustration. Given an unquantized tensor $\bm{X}$ and an FP format $E_{B_e}M_{B_m}$, we first, we compute the quantization scale factor $s_X$ that maps the maximum absolute value of $\bm{X}$ to the maximum quantization level of the $E_{B_e}M_{B_m}$ format as follows:
\begin{align}
\label{eq:sf}
    s_X = \frac{\mathrm{max}(|\bm{X}|)}{\mathrm{max}(E_{B_e}M_{B_m})}
\end{align}
In the above equation, $|\cdot|$ denotes the absolute value function.

Next, we scale $\bm{X}$ by $s_X$ and quantize it to $\hat{\bm{X}}$ by rounding it to the nearest quantization level of $E_{B_e}M_{B_m}$ as:

\begin{align}
\label{eq:tensor_quant}
    \hat{\bm{X}} = \text{round-to-nearest}\left(\frac{\bm{X}}{s_X}, E_{B_e}M_{B_m}\right)
\end{align}

We perform dynamic max-scaled quantization \citep{wu2020integer}, where the scale factor $s$ for activations is dynamically computed during runtime.

\subsection{Vector Scaled Quantization}
\begin{wrapfigure}{r}{0.35\linewidth}
  \centering
  \includegraphics[width=\linewidth]{sections/figures/vsquant.jpg}
  \caption{\small Vectorwise decomposition for per-vector scaled quantization (VSQ \citep{dai2021vsq}).}
  \label{fig:vsquant}
\end{wrapfigure}
During VSQ \citep{dai2021vsq}, the operand tensors are decomposed into 1D vectors in a hardware friendly manner as shown in Figure \ref{fig:vsquant}. Since the decomposed tensors are used as operands in matrix multiplications during inference, it is beneficial to perform this decomposition along the reduction dimension of the multiplication. The vectorwise quantization is performed similar to tensorwise quantization described in Equations \ref{eq:sf} and \ref{eq:tensor_quant}, where a scale factor $s_v$ is required for each vector $\bm{v}$ that maps the maximum absolute value of that vector to the maximum quantization level. While smaller vector lengths can lead to larger accuracy gains, the associated memory and computational overheads due to the per-vector scale factors increases. To alleviate these overheads, VSQ \citep{dai2021vsq} proposed a second level quantization of the per-vector scale factors to unsigned integers, while MX \citep{rouhani2023shared} quantizes them to integer powers of 2 (denoted as $2^{INT}$).

\subsubsection{MX Format}
The MX format proposed in \citep{rouhani2023microscaling} introduces the concept of sub-block shifting. For every two scalar elements of $b$-bits each, there is a shared exponent bit. The value of this exponent bit is determined through an empirical analysis that targets minimizing quantization MSE. We note that the FP format $E_{1}M_{b}$ is strictly better than MX from an accuracy perspective since it allocates a dedicated exponent bit to each scalar as opposed to sharing it across two scalars. Therefore, we conservatively bound the accuracy of a $b+2$-bit signed MX format with that of a $E_{1}M_{b}$ format in our comparisons. For instance, we use E1M2 format as a proxy for MX4.

\begin{figure}
    \centering
    \includegraphics[width=1\linewidth]{sections//figures/BlockFormats.pdf}
    \caption{\small Comparing LO-BCQ to MX format.}
    \label{fig:block_formats}
\end{figure}

Figure \ref{fig:block_formats} compares our $4$-bit LO-BCQ block format to MX \citep{rouhani2023microscaling}. As shown, both LO-BCQ and MX decompose a given operand tensor into block arrays and each block array into blocks. Similar to MX, we find that per-block quantization ($L_b < L_A$) leads to better accuracy due to increased flexibility. While MX achieves this through per-block $1$-bit micro-scales, we associate a dedicated codebook to each block through a per-block codebook selector. Further, MX quantizes the per-block array scale-factor to E8M0 format without per-tensor scaling. In contrast during LO-BCQ, we find that per-tensor scaling combined with quantization of per-block array scale-factor to E4M3 format results in superior inference accuracy across models. 


\end{document}
\typeout{get arXiv to do 4 passes: Label(s) may have changed. Rerun}
\endinput
