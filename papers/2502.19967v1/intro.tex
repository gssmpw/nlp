\section{Introduction}
\label{sec:intro}
Modern decentralized applications often employ data replication across
geographically distributed locations to enhance fault tolerance, minimize data
access latency, and improve scalability. This practice is crucial for mitigating
the impact of network failures and reducing data transmission delays to end
users.  However, these systems encounter the challenge of concurrent
conflicting data updates across different replicas.

Recently, Mergeable Replicated Data Types (MRDTs) \cite{Kaki2019, Kaki2022,
Vimala} have emerged as a systematic approach to the problem of ensuring that
replicas remain eventually consistent despite concurrent conflicting updates.
MRDTs draw inspiration from the Git version control system, where each update
creates a new version, and any two versions can be merged explicitly through a
user-defined $\F{merge}$ function.  $\F{merge}$ is a ternary function that
takes as input the two versions to be merged and their Lowest Common Ancestor
(LCA), i.e., the most recent version from which the two versions diverged. As
opposed to Conflict-Free Replicated Data Types (CRDTs)\cite{Shapiro}, which may
have to carry around causal context metadata to ensure consistency, MRDTs can
rely on the underlying system model to provide the causal context through the
LCA.  This results in implementations that are comparatively simpler and also
more efficient. For example, if we consider state-based CRDTs, which are the
closest analogue to the MRDT model, then any counter CRDT implementation would
require $O(n)$ space, where $n$ is the number of replicas (a lower bound proved
by \cite{Burckhardt}), whereas a counter MRDT implementation only requires
$O(1)$ space. The states maintained by CRDT implementations need to form a join
semi-lattice, with all CRDT operations restricted to being monotonic functions
and merge restricted to the lattice join. While these restrictions simplify the
task of reasoning about correctness \cite{Verifx,Katara,EneaArxiv}, crafting
correct and efficient CRDT implementations itself becomes much harder.

MRDTs do not require any of the above restrictions, which helps in developing
implementations with better space and time complexity. However, reasoning about
correctness now becomes harder. Indeed, the MRDT system model allows arbitrary
replicas to merge their states at arbitrary points in time, and this can result
in subtle bugs requiring a very specific orchestration of merge actions. As
part of this work, we discovered such subtle bugs in MRDT implementations
claimed to be verified by previous works \cite{Vimala} (more details can be
found in \S\ref{subsec:bug}).  The MRDT state as well as the implementation of
data type operations and the $\F{merge}$ function have to be cleverly designed
to ensure strong eventual consistency. That is, despite concurrent conflicting
updates and arbitrary ordering of merges, all replicas will eventually converge
to the same state. Further, we would also like to show that an MRDT satisfies
the functional behavior of the data type, along with the user-defined conflict
resolution policy for concurrent conflicting updates (e.g., for a set data
type, an \emph{add-wins} policy that favors the $\F{add}$ operation over a
concurrent $\F{remove}$ of the same element at different replicas). There have
been a few works \cite{Kaki2019,Kaki2022,Vimala} that have looked at the problem
of specifying and verifying MRDTs. However, they either restrict the system
model by disallowing concurrent merges \cite{Kaki2019}, focus only on
convergence as the correctness specification \cite{Kaki2022,Kaki2019}, or do
not support automated verification \cite{Vimala}.

In this work, we couch the correctness of MRDTs using the notion of
\emph{Replication-Aware Linearizability} (RA-linearizability) \cite{Wang}, which says that the state
at any replica must be obtained by linearizing (i.e., constructing a sequence
of) update operations that have been applied at the replica. As a first contribution,
we adapt RA-linearizability to the MRDT system model (\S
\ref{sec:lin}), and develop a simple specification framework for MRDTs based on
conflict resolution policy for concurrent update operations. We show that an MRDT
implementation can be linearized only under certain technical constraints on
the conflict resolution policy and if the merge operation satisfies a weaker
notion of commutativity called \emph{conditional commutativity}. By ensuring
that the linearization order obeys the conflict resolution policy for
concurrent update operations and it remains the same across all replicas, we guarantee
both strong eventual consistency and adherence to the user-provided
specification.

%We then prove that if the linearization order remains the same across
%replicas, obeys the user-specified conflict resolution policy as well as the
%replica-local ordering, then linearizability guarantees both functional
%correctness and convergence (\S3). \aseem{I tweaked the last sentence to say
%"We then prove", hope this ok. Before the sentence was just starting with "If
%the linearization ...", so it wasn't clear if we were stating a known fact or
%we prove this.} \aseem{Also, the part about guaranteeing functional
%correctness, our proof in S3 is common for all datatypes, so I am wondering
%how can we cover functional correctness guarantees there?}

Next, we propose a sound but not complete technique for proving
RA-linearizability for MRDT implementations. The main challenge
lies in showing that the $\F{merge}$ function generates a state which is a
linearization of its inputs. We develop a technique called \emph{bottom-up
linearization}, which relies on certain simple algebraic properties of the
$\F{merge}$ function to prove that it generates the correct linearization. We
then design an induction scheme to \emph{automatically} verify the required
algebraic properties of $\F{merge}$ for an arbitrary MRDT implementation. Our
main insight here is to leverage the fact that the merge inputs are themselves
linearizations, and hence, we can use induction over their operation sequences.
We extract a set of verification conditions (VCs) that are amenable to
automated reasoning, and prove that if an MRDT implementation satisfies the
VCs, it is linearizable  (\S \ref{sec:lemmas}). While our development is
focussed on MRDTs, our technique can be directly applied on state-based CRDTs.
State-based CRDTs also have a merge-based system model which is slightly
simpler than MRDTs as the merge function does not require any LCA. 

%Our VCs can be modified by essentially ignoring the LCA component to also verify state-based CRDTs.

% The technique uses induction on operation sequences and
%generates \emph{verification conditions} over the MRDT operations and the merge function.
%The main challenge in proving linearizability is for
%the $\F{merge}$ function, where we need to show that the output is a
%linearization of the operations in the inputs. Our main insight here is that
%the merge inputs are themselves linearizations, and therefore, we can use
%induction on their operation sequences to prove the desired property.  We prove
%the soundness of our algorithm, providing the guarantee that if the generated
%verification conditions hold for an MRDT implementation, then it is
%linearizable).

Finally, we develop a framework in the \fstar~\cite{fstar} programming language
that allows implementing MRDTs and automatically mechanically proving the VCs
required by our technique. The framework provides several advantages over
previous works. First, we require the programmer to specify only the MRDT
operations, the merge function, and the conflict resolution policy, in contrast
to the earlier work that also requires proof constructs such as abstract
simulation relations~\cite{Vimala}. Second, the VCs are simple enough that in
\emph{all} the case studies we have done, including data types such as counter,
set, map, boolean flag, and list, they are automatically discharged by \fstar.
Finally, we extract the verified implementations to OCaml using the \fstar
extraction pipeline and run them (\S \ref{sec:results}). We have also
implemented and verified a few state-based CRDTs using our framework. In the next
section, we present the main ideas of our work informally through a series of
examples.

% In this work, our goal is to fully automate the verification problem for MRDTs, and also provide a simpler specification framework which is based on only providing an ordering relation for conflicting operations (i.e. the conflict resolution policy), but at the same time can guarantee both convergence and functional correctness. Towards this end, we introduce the notion of linearizability for MRDTs, which ensures that the state at any replica can always be obtained by linearizing (i.e. constructing a sequence of) operations which have been applied at the replica. By ensuring that the linearization order remains the same across replicas, obeys the developer-specified conflict resolution policy as well as replica-local ordering, linearizability can guarantee both convergence and functional correctness. Further, we leverage the definition of linearizability itself to develop a fully automated verification procedure, which uses induction on operation sequences. The main challenge in proving linearizability is during the merge operation, where we need to demonstrate that the merge output is a linearization of the updates in the merge inputs. Here, we leverage the fact that the merge inputs are themselves linearizations, and use induction on their operation sequences to prove the desired property.
% \vspace{-1em}

% \subsubsection*{\textbf{Contributions}}
% The contributions of this paper are summarized below:
% \begin{itemize}
% 	\item We develop a definition of linearizability for MRDTs and use it as a
% 		correctness criterion to specify and verify MRDTs. Our definition
% 		encompasses both convergence and functional correctness, and also allows
% 		simple specifications for the conflict resolution policy. We also identify
% 		sufficient conditions that the specification and implementation should
% 		satisfy to allow linearization.
% 	\item We develop a fully automated verification procedure for proving
% 		linearizability of MRDTs, which only requires the MRDT implementation and
% 		the conflict resolution policy. We use induction on operation sequences,
% 		and leverage commutativity and a new notion called \emph{conditional
% 		commutativity} for MRDT operations.
% 	\item We have implemented our approach in the \fstar~\cite{fstar} programming language, and
% 		applied it successfully to a number of complex MRDT implementations of
% 		diverse data types such as counter, set, map, boolean flag and list.
% \end{itemize}
% \vspace{-0.5em}
% The rest of the paper is structured as follows: In \S2, we illustrate our
% technique using the example of a set MRDT. We formally define the system model
% and our notion of linearizability in \S3. \S4 describes our automated
% verification procedure. \S5 contains details of our implementation and
% experiments. We discuss related work and conclude in \S6.

%Ensuring that the MRDT implementations are correct is a crucial challenge. These
%implementations need to preserve the intended behaviour of the data type and also guarantee
%convergence. But mere presence of a merge function doesn't ensure convergence.
%Even in cases where MRDTs exhibit clear and intuitive merge semantics, certain executions
%can lead to divergence \aseem{(needs explaination/illustration)}. What we require is a formal
%methodology to verify that the MRDT
%implementations are functionally correct and convergent. There have been a few works
% that explore this issue~\cite{Vimala,Kaki2022}. ~\cite{Kaki2022} focuses primarily on
%achieving convergence but doesn't emphasise ensuring functional correctness.
%They achieve convergence by modifying the runtime environment to ensure
%well-structured executions, where LCAs for concurrent versions are unique, and merge
%operations are linearized \aseem{(what does linearized mean)}. However, this approach has a
%downside: it can lead to longer
%convergence times for replicas, which is undesirable, as it undermines the core purpose
%of state replication. It introduces significant latency, reducing the system's
%availability \aseem{(can we cite something here?)}.

%\cite{Vimala} avoids the imposition of runtime constraints that would necessitate the
%linearization of merges. Instead, it establishes sufficient conditions for proving
%both the functional correctness and convergence of MRDTs. This technique employs
%a replication-aware simulation relation to connect the specifications of MRDTs to their
%implementations. Unlike ~\cite{Katara}, where the specification is based on a sequence of
%operations and a reference implementation, the specification in this approach is more intricate.
%Users are expected to define specifications based on an abstract state, which includes
%events and their relationships in terms of visibility. This complexity can be challenging for non-experts.
%Additionally, verifying MRDTs using this approach demands complex reasoning because
%the simulation relation connects the data structure in its concrete domain to an abstract domain
%which has a different execution model. Users need to prove that the simulation relation
%holds true at every step of the program's execution. The difficulty increases as the MRDT becomes
%more complex. For example, consider the proof of Queues in ~\cite{Vimala} which involves
%more than 1000 lines of proof and takes over 1.5 hours to verify. Writing correct simulation
%relations can be quite challenging. Sometimes, a seemingly natural simulation relation may
%not fully capture the complexities in the execution. There is an example of this in Section~\ref{sec:overview}
%where an intuitive simulation relation for an MRDT couldn't disprove a divergent MRDT.
%
%Our goal is to make developers write simpler specifications in terms of ordering constraint
%that specifies how to resolve conflicts between non-commutative operations. By using these
%ordering constraints, users can describe the behaviour of distributed systems as if they were
%operating in a sequential execution model. The users are expected to prove simple conditions
%based on the conflict resolution of non-commutative operations. This makes it possible for
%non-experts to use our framework and the MRDTs are verified automatically or with just a
%few lines of code. The framework, using these simple conditions will prove that the state at
%any replica is obtained by applying a linearization of operations that led to that state,
%consistent with the ordering constraint. This means that there is a total order of updates,
%and eventually, when all operations are visible to all replicas, all the replicas will return the
%same value which implies convergence. \aseem{(We need to summarize the framework and the insights
%of our framework better. For example, is MRDT categories is the main insight? And then we work
%towards finding sufficient conditions for each of them, and proving the necessary metatheory?
%Also the distinction between functional correctness and linearizability.)}
%\aseem{Reading some more, is it the case that we observe what are the ways in which one
%can construct linearizations, and design our lemmas accordingly. If so, we should say it.}




%There are two possible ways of
%dealing with such data modifications. The first approach involves replicas coordinating
%with one another to determine when to apply the modifications. This approach
%ensures strong consistency but may adversely impact system performance,
%as it necessitates replicas to await coordination. In contrast, the second approach
%allows users to independently modify data on any replica and asynchronously propagate the
%update to all other replicas. This method can lead to conflicts when there are concurrent updates to different replicas.

%Conflict-Free Replicated Data Types (CRDTs)~\cite{Shapiro},
%provides a systematic approach to the problem by ensuring that replicas are eventually
%consistent despite the asynchronous delivery of updates. To handle conflicting
%updates, CRDTs typically carry causal context metadata~\cite{Yu20}, a process \aseem{(what
	%process are we referring to? carrying the causal context metadata?)} that can be
%complex and resource-intensive. CRDTs guarantee that regardless of the data modifications
%made on different replicas, the data can be automatically merged into a consistent state
%without requiring any special conflict resolution code \aseem{(code?)} or user intervention.

%As an alternative to CRDTs, Mergeable Replicated Data Types (MRDTs)~\cite{Kaki} have been
%proposed which draw inspiration from the Git version control system. This replication model
%revolves around the idea of versioned states and explicit merging. In MRDTs, the data
%type definition is extended with a three-way merge function that reconciles concurrent
%versions of the type taking into account the context information provided by the lowest common
%ancestor (LCA) version of the merging versions. This enables any regular data type,
%equipped with a three-way merge function, to function as a distributed data type.
%\aseem{We could give a small example at this point of an MRDT and its merge function.
	%In the next paragraph, we talk about correctness, intended behavior, and convergence
	%very abstractly, With this small example, we can also illustrate these concepts.}

