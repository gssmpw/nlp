\section{Appendix}
\label{sec:app}

\subsection{Proofs of \S \ref{sec:lin}}
\label{subsec:lcaproof}
\noindent \textbf{Lemma~\ref{lem:LCA}} Given a configuration $C = \langle N,H,L,G,vis \rangle$ reachable in some execution $\tau \in \llbracket \M{S_D} \rrbracket$ and two versions $v_1,v_2 \in dom(N)$, if $v_\top$ is the LCA of $v_1$ and $v_2$ in $G$, then $L(v_\top) = L(v_1) \cap L(v_2)$.
\begin{proof}
If $(v,v') \in E$, then $L(v) \subseteq L(v')$. This is because either $L(v') = L(v) \cup \{e\}$ for some event $e$ due to the $\F{apply}$ transition, or $L(v') = L(v) \cup L(v'')$ due to the $\F{merge}$ transition. 

Hence, if $(v,v') \in E^*$, then $L(v) \subseteq L(v')$.

Since $(v_\top,v_1) \in E^*$ and $(v_\top,v_2) \in E^*$, hence $L(v_\top) \subseteq L(v_1)$ and $L(v_\top) \subseteq L(v_2)$. Hence, $L(v_\top) \subseteq L(v_1) \cap L(v_2)$.

Consider vertices $u,w$ and event $e$ such that $(u,w) \in E$, $e \notin L(u)$, $e \in L(w)$ and in-degree of $w$ is 1. Then $w$ is called the $generator$ vertex of event $e$. Note that there will always be a unique generator vertex for each event.

\begin{proposition}\label{prop:lca}
	For all versions $v$, events $e$, if $e \in L(v)$, and $w$ is the generator version of $e$, then $(w,v) \in E^*$.
\end{proposition}
 
Consider $e \in L(v_1) \cap L(v_2)$. Then if $w$ is the generator version of $e$, by Proposition \ref{prop:lca} $(w,v_1) \in E^*$ and $(w,v_2) \in E^*$. Then, by definition of LCA, $(w,v_\top) \in E^*$. Hence, $L(w) \subseteq L(v_\top)$. This implies that $e \in L(v_\top)$. Thus, $ L(v_1) \cap L(v_2)  \subseteq L(v_\top)$.

We now prove Proposition \ref{prop:lca}. If $v$ has in-degree 2, then suppose $(w_1,v) \in E$, $(w_2,v) \in E$ and $L(v) = L(w_1) \cup L(w_2)$. Then either $e \in L(w_1)$ or $e \in L(w_2)$. WLOG, suppose $e \in L(w_1)$. We now recursively apply Proposition \ref{prop:lca} on $w_1$. Then, $(w,w_1) \in E^*$, which implies $(w,v) \in E^*$. 

If $v$ has in-degree 1, then suppose $(u,v) \in E$. If $e \in L(u)$, we recursively apply Proposition \ref{prop:lca} on $u$. If $e \notin L(u)$, then $v$ itself is the generator version of $e$, and clearly, $(v,v) \in E^*$.

Note that everytime we move backwards along an edge by recursively applying Proposition \ref{prop:lca}, we are either decreasing the number of events in the source vertex, or the number of unvisited vertices in the graph while still retaining $e$. Since the graph is acyclic and finite, and the number of events are also finite, eventually, we will hit the generator version. \qed

\end{proof}

\noindent \textbf{Recursive Merge Strategy}: For a given version graph $G = (V,E)$, for versions $v_1,v_2$, if the LCA does not exist, then our strategy is to find potential LCAs. For each potential LCA $v_p$, $(v_p,v_1) \in E^*$, $(v_p,v_2) \in E^*$ and $\nexists v.\ (v,v_1) \in E^* \wedge (v,v_2) \in E^* \wedge (v_p,v) \in E^*$. Note that since the version graph is rooted at the initial version $v_0$, a common ancestor of any two versions $v_1$ and $v_2$ always exist. Let $V_c$ be the set of all common ancestors of $v_1$ and $v_2$.
$$
V_c = \{v \in V\ \mid\ (v,v_1) \in E^* \wedge (v,v_2) \in E^*\}
$$

For two common ancestors $v,v' \in V_c$, either there is a path between them or there isn't. If there is a path, say $(v,v') \in E^*$, then $v$ can neither be a potential or actual LCA. In this way, we eliminate all common ancestors which cannot be potential or actual LCAs. Finally, we are left with the set of potential LCAs $V_p$. Hence, for any $v,v' \in V_p$, $(v,v') \notin E^*$ and $(v',v) \notin E^*$. It is then clear to see that if $V_p = \{v_\top\}$, i.e. $V_p$ is singleton, then $v_\top$ must be the actual LCA, because every other common ancestor $v$ must have been eliminated due to $(v,v_\top) \in E^*$.

Otherwise, if $V_p$ is not singleton, we pairwise invoke $\F{merge}$ on every pair of versions in $V_p$. Note that we would have to repeat the same merge strategy while merging any two versions in $V_p$. We now show that if $v_m$ is the version obtained by merging all the versions in $V_p$, then $L(v_m) = L(v_1) \cap L(v_2)$. Since every version $v \in V_p$ is a common ancestor of $v_1$ and $v_2$, $L(v) \subseteq L(v_1) \cap L(v_2)$, and hence $L(v_m) \subseteq  L(v_1) \cap L(v_2)$. Consider $e \in  L(v_1) \cap L(v_2)$. Now, consider the generator version $w$ of $e$. By Proposition \ref{prop:lca}, $w$ is a common ancestor of $v_1$ and $v_2$. Either $w \in V_p$, in which case by merging $w$ to get $v_m$, we would have $e \in L(v_m)$. Or else, $w$ would have been eliminated, in which case there will exist some version $v \in V_p$ such that $(w,v) \in E^*$. Hence, $e \in L(v)$, which implies $e \in L(v_m)$.  \\

\noindent \textbf{Lemma~\ref{lem:non-comm}}
	Given a set of events $\M{E}$, if $\F{lo} \subseteq \M{E} \times \M{E}$ is defined over
	every pair of non-commutative events in $\M{E}$, then for any two sequences
	$\pi_1, \pi_2$ which extend $\F{lo}$, for any state $\sigma$, $\pi_1(\sigma)
	= \pi_2(\sigma)$.
\begin{proof}
	If $\pi_1=\pi_2$, then the result trivially holds. Consider the first point of difference between $\pi_1$ and $\pi_2$.\\
	$\pi_1 = \tau.e_1.\tau_1, \pi_2 = \tau.e_2.\tau_2$.\\
	Then $e_1$ must appear somewhere in $\tau_2$.\\
	$\pi_2 = \tau.e_2.\tau_3.e_1.\tau_4$.\\
	We consider two cases here:\\
	\textbf{Case 1: $(\tau_3 = \phi)$}\\
	Since $e_1$ and $e_2$ are in different orders in $\pi_1$ and $\pi_2$, neither $e_1 \xrightarrow{\F{lo}} e_2$ nor
	$e_2 \xrightarrow{\F{lo}} e_1$. Since $\F{lo}$ is defined over every pair of non-commutative events, but is not defined between $e_1$ and $e_2$, they must commute, Hence, we can flip $e_2$ and $e_1$ in $\pi_2$, leading to the same state.\\
\textbf{Case 2: $(\tau_3 \neq \phi)$}\\
	$\pi_2 = \tau.e_2.\tau_5.e_3.e_1.\tau_4$. Then in $\pi_1$, $e_3$ is not present in $\tau$, hence it must be present after $e_1$.
	Now $e_1$ and $e_3$ are in different orders in $\pi_1$ and $\pi_2$, hence neither $e_1 \xrightarrow{\F{lo}} e_3$ nor
	$e_3 \xrightarrow{\F{lo}} e_1$. 
	
	By the same argument as above applied on $e_1$ and $e_2$, we can flip $e_1$ and $e_3$ in $\pi_2$.
	We keep doing this for all events in $\tau_5$ until $e_2$ is adjacent to $e_1$ after which we can flip them.
	Thus we can change $\pi_2$ such that $e_1$ will appear in the same position in $\pi_1$. We can keep 
	doing this until $\pi_1$ and $\pi_2$ are identical.
\end{proof}

\noindent \textbf{Lemma~\ref{lem:irreflexive}} For an MRDT $\mathcal{D}$ such that $\F{rc}^+$ is irreflexive, for any configuration $C$ reachable in
	$\mathcal{S}_\mathcal{D}$, $\F{lo}_C^+$ is irreflexive.\\

To prove that $\F{lo}_C^+$ is irreflexive, we need to prove that there cannot be cycles formed out of $\F{lo}_C$ edges.
\begin{proof}
A cycle cannot be formed using only $\F{vis}$ edges, as $\F{vis}^+$ is irreflexive. Similarly, a cycle cannot be formed using only $\F{rc}$ edges, as $\F{rc}^+$ is irreflexive. Therefore, any potential cycle must consist of adjacent $\xrightarrow{\F{rc}}$ and $\xrightarrow{\F{vis}}$ edges.
Consider three events $e_1, e_2, e_3$ such that $e_1 \xrightarrow[\F{rc}]{\F{lo}} e_2 \xrightarrow[\F{vis}]{\F{lo}} e_3$.
Since $e_1 \xrightarrow[\F{rc}]{\F{lo}} e_2$, this implies $e_1 \xrightarrow{\F{rc}} e_2 \wedge e_1 \mid\mid_C e_2$. Given that $e_2 \xrightarrow{\F{vis}} e_3$, the relation $e_1 \xrightarrow[\F{rc}]{\F{lo}} e_2$ is not possible. Thus, this case is also not feasible.
Hence, there cannot be cycles formed out of $\F{lo}_C$ edges. 
\end{proof}

\noindent \textbf{Lemma~\ref{lem:convergence}}  For an MRDT $\M{D}$ which satisfies $\textrm{\sc{rc-non-comm}}(\M{D})$ and $\textrm{\sc{cond-comm}}(\M{D})$, for any reachable configuration $C$ in $\M{S}_\M{D}$, for any two sequences $\pi_1,\pi_2$ over $E_C$ which extend $\F{lo}_C$, for any state $\sigma$, $\pi_1(\sigma) = \pi_2(\sigma)$.
\begin{proof}
Consider the first point of difference between $\pi_1$ and $\pi_2$.\\
$\pi_1 = \tau.e_1.\tau_1, \pi_2 = \tau.e_2.\tau_2$.\\
Then $e_1$ must appear somewhere in $\tau_2$.\\
$\pi_2 = \tau.e_2.\tau_3.e_1.\tau_4$.\\
We consider two cases here:\\
\textbf{Case 1: $(\tau_3 = \phi)$}\\
Since $e_1$ and $e_2$ are in different orders in $\pi_1$ and $\pi_2$, neither $e_1 \xrightarrow{\F{lo}} e_2$ nor
$e_2 \xrightarrow{\F{lo}} e_1$. If either $e_1 \xrightarrow{\F{vis}} e_2$  or $e_2 \xrightarrow{\F{vis}} e_1$, it 
must be the case that $e_1 \rightleftarrows e_2$. In this case, we can flip the order of $e_1$ and $e_2$ in $\pi_2$
leading to the same state. 
Suppose $e_1 \mid\mid_C e_2$, if neither $e_1 \xrightarrow{\F{rc}} e_2$ nor $e_2 \xrightarrow{\F{rc}} e_1$, $e_1 \rightleftarrows e_2$.
In this case, we can again flip them in $\pi_2$.
Suppose $e_1 \xrightarrow{\F{rc}} e_2$, since $\neg (e_1 \xrightarrow{\F{lo}} e_2)$, by definition of $\F{lo}$,
$\exists e_3. e_2 \xrightarrow{\F{lo}} e_3$. Then $\neg (e_2 \rightleftarrows e_3)$. By \textrm{\sc{cond-comm}},
it must be the case that $e_1  \overset{e_3}{\rightleftarrows}  e_2$. Since $e_2 \xrightarrow{\F{lo}} e_3$, $e_3$ must be 
present in $\tau_4$. By definition of \textrm{\sc{cond-comm}}, we can flip $e_2$ and $e_1$ in $\pi_2$, leading to the same state.
Similar argument can be applied to $e_2 \xrightarrow{\F{rc}} e_1$.\\
\textbf{Case 2: $(\tau_3 \neq \phi)$}\\
$\pi_2 = \tau.e_2.\tau_5.e_3.e_1.\tau_4$. Then in $\pi_1$, $e_3$ is not present in $\tau$, hence it must be present after $e_1$.
Now $e_1$ and $e_2$ are in different orders in $\pi_1$ and $\pi_2$, hence neither $e_1 \xrightarrow{\F{lo}} e_3$ nor
$e_3 \xrightarrow{\F{lo}} e_1$. 

By the same argument as above applied on $e_1$ and $e_2$, we can flip $e_1$ and $e_3$ in $\pi_2$.
We keep doing this for all events in $\tau_5$ until $e_2$ is adjacent to $e_1$ after which we can flip them.
Thus we can change $\pi_2$ such that $e_1$ will appear in the same position in $\pi_1$. We can keep 
doing this until $\pi_1$ and $\pi_2$ are identical.
\end{proof}


\noindent \textbf{Lemma~\ref{lem:query}} If MRDT $\mathcal{D}$ is RA-linearizable, then for all executions $\tau \in \llbracket \mathcal{S}_\mathcal{D} \rrbracket$, and for all transitions $C \xrightarrow{query(r,q,a)} C'$ in $\tau$, where $C = \langle N, H, L, G, vis\rangle$, there exists a sequence $\pi$ consisting of all events in $L(H(r))$ such that $\F{lo}(C)_{\mid L(H(r))} \subseteq \pi$ and $a = \F{query}(\pi(\sigma_0), q)$.

\begin{proof}
Consider an MRDT $\mathcal{D}$ that is RA-linearizable. Let $\tau = C_0 \xrightarrow{t_1} C_1 \xrightarrow{t_2} C_2 \ldots \xrightarrow{t_n} C$ 
be an execution of $\mathcal{S}_\mathcal{D}$, where $\{t_1, \dots, t_n\}$ are the labels of the transition system. For a transition $C \xrightarrow{query(r, q, a)} C'$ in $\tau$, where $C = \langle N, H, L, G, vis\rangle$, we know that $C$ is RA-linearizable from Def.~\ref{def:lin}. That is, for every active replica $r \in \mathrm{range}(H)$, there exists a sequence $\pi$ such that $\F{lo}(C)_{\mid L(H(r))} \subseteq \pi$ and $N(H(r)) = \pi(\sigma_0)$.
According to the semantics, we have $a = \F{query}(N(H(r)), q)$. Thus $a = \F{query}(\pi(\sigma_0), q)$.
\end{proof}


\subsection{Proofs of \S \ref{sec:lemmas}}

\noindent \textbf{Lemma \ref{lem:pi1}}
(1) For events $e \in L_1^a \cup L_2^a$, $e' \in L_1^b \cup L_2^b$, $\neg (e \xrightarrow{\F{lo_m}} e')$.
\begin{proof}
Suppose $e \xrightarrow{\F{lo_m}} e'$ is true. There are 2 possibilities:
\begin{enumerate}
	\item $e \xrightarrow[\F{vis}]{\F{lo}} e':$  
	By definition of $L_i^b$, there are 2 cases:
	\begin{enumerate}
		\item $\exists e_\top \in L_\top. e' \xrightarrow{\F{lo_m}} e_\top:$ But this would require $e$ to be in $L_1^b \cup L_2^b$.
		\item $\exists e_\top \in L_\top, e'' \in L_1' \cup L_2'. e' \xrightarrow{\F{lo_m}} e'' \xrightarrow{\F{lo_m}} e_\top:$
		\begin{enumerate}
			\item $e' \xrightarrow[\F{vis}]{\F{lo}} e'':$ Due to transitivity of $\F{vis}$, $e \xrightarrow{\F{vis}} e''$. This would require $e \in L_1^b \cup L_2^b$.
			\item $e'' \xrightarrow[\F{vis}]{\F{lo}} e_\top$ is not possible as $L_\top^a$ is causally closed.
			\item $e' \xrightarrow[\F{rc}]{\F{lo}} e'' \xrightarrow[\F{rc}]{\F{lo}} e_\top$ is not possible due to \textrm{\sc{no-rc-chain}} restriction.
		\end{enumerate}
	\end{enumerate}
	
	\item $e \xrightarrow[\F{rc}]{\F{lo}} e':$ 
	By definition of $L_i^b$, there are 2 cases:
	\begin{enumerate}
		\item $\exists e_\top \in L_\top. e' \xrightarrow{\F{lo_m}} e_\top:$ 
			\begin{enumerate}
				\item $e' \xrightarrow[\F{vis}]{\F{lo}} e_\top$ is not possible as $L_\top^a$ is causally closed. 
				\item $e' \xrightarrow[\F{rc}]{\F{lo}} e_\top$ is not possible due to \textrm{\sc{no-rc-chain}} restriction.
				Since $e \mid\mid_C e_\top$, we have $e \xrightarrow[\F{rc}]{\F{lo}} e_\top$ which requires $e \in L_1^b \cup L_2^b$.
			\end{enumerate}
		\item $\exists e_\top \in L_\top, e'' \in L_1' \cup L_2'. e' \xrightarrow{\F{lo_m}} e'' \xrightarrow{\F{lo_m}} e_\top:$
		\begin{enumerate}
			\item $e'' \xrightarrow[\F{vis}]{\F{lo}} e_\top$ is not possible as $L_\top^a$ is causally closed.
			\item $e' \xrightarrow[\F{rc}]{\F{lo}} e''$ is not possible due to \textrm{\sc{no-rc-chain}} restriction.
			\item $e' \xrightarrow[\F{vis}]{\F{lo}} e'' \xrightarrow[\F{rc}]{\F{lo}} e_\top:$ 
				$e' \xrightarrow{\F{rc}} e''$ creates RC-chain.
				Since $e' \mid\mid_C e_\top$, we have $e' \xrightarrow[\F{rc}]{\F{lo}} e_\top$
				which violates the \textrm{\sc{no-rc-chain}} restriction.
				$e'' \xrightarrow{\F{rc}} e'$ would requires $e$ and $e'$ to conditionally commute with each other.
				So $e \xrightarrow[\F{rc}]{\F{lo}} e'$ does not hold true.
		\end{enumerate}
	\end{enumerate}
\end{enumerate}
\end{proof}

(2) For events $e \in L_\top^a$, $e' \in L_\top^b$, $\neg (e \xrightarrow{\F{lo_m}} e')$.
\begin{proof}
By definition of $L_\top^a$, $\exists e'' \in L_1^b \cup L_2^b. e'' \xrightarrow{\F{lo_m}} e$.
	$e'' \xrightarrow{\F{vis}} e$ is not possible as $L_\top^a$ is causally closed.
Suppose $e \xrightarrow{\F{lo_m}} e'$ is true. There are 3 possibilities:
\begin{enumerate}
	\item $e \xrightarrow[\F{vis}]{\F{lo}} e':$  
	\begin{enumerate}
			\item $e'' \xrightarrow[\F{rc}]{\F{lo}} e:$ $e \xrightarrow{\F{rc}} e'$ causes RC-chain.
			Since $e'' \mid\mid_C e'$, we have
			 $e'' \xrightarrow[\F{rc}]{\F{lo}} e'$ which requires $e' \in L_\top^a$.
			 $e' \xrightarrow{\F{rc}} e$ cause $e''$ and $e$ to conditionally commute with each other. So this case does not hold true.
	\end{enumerate}
	
	\item $e \xrightarrow[\F{rc}]{\F{lo}} e':$  
	$e'' \xrightarrow{\F{vis}} e$ is not possible as $L_\top^a$ is causally closed.
	\begin{enumerate}
			\item $e'' \xrightarrow[\F{rc}]{\F{lo}} e:$ causes RC-chain.		 
	\end{enumerate}
\end{enumerate}
\end{proof}

\noindent \textbf{Lemma \ref{lem:pi2}}
(1) For events $e_i^{\top}, e_j^{\top} \in L_\top^a$, where $L_\top^a = \{e_1^{\top}, \ldots, e_m^{\top}\}$, $\neg (e_i^{\top} \xrightarrow{\F{lo_m}} e_j^{\top})$.
\begin{proof}
By definition of $L_\top^a$, $\exists e \in L_1^b(e_i^{\top}) \cup L_2^b(e_i^{\top}). e \xrightarrow{\F{lo_m}} e_i^{\top}$. 
$e \xrightarrow{\F{vis}} e_i^{\top}$ is not possible as $L_\top^a$ is causally closed.
Suppose $e_i^{\top} \xrightarrow{\F{lo_m}} e_j^{\top}$. There are 3 possibilities.
\begin{enumerate}
	\item $e_i^{\top} \xrightarrow[\F{vis}]{\F{lo}} e_j^{\top}:$  
	\begin{enumerate}
		\item $e \xrightarrow[\F{rc}]{\F{lo}} e_i^{\top}:$ $e_i^{\top} \xrightarrow{\F{rc}} e_j^{\top}$ causes RC-chain. Since $e \mid\mid_C e_j^{\top}$, we have
			 $e \xrightarrow[\F{rc}]{\F{lo}} e_j^{\top}$ which requires $e \in L_1^b(e_j^{\top}) \cup L_2^b(e_j^{\top})$.
			 But $e$ belongs to $L_1^b(e_i^{\top}) \cup L_2^b(e_i^{\top})$.
			 $e_j^{\top} \xrightarrow{\F{rc}} e_i^{\top}$ cause $e$ and $e_i^{\top}$ to conditionally commute with each other. So this case does not hold true.
		
	\end{enumerate}

	\item $e_i^{\top} \xrightarrow[\F{rc}]{\F{lo}} e_j^{\top}:$
	\begin{enumerate}
		\item $e \xrightarrow[\F{rc}]{\F{lo}} e_i^{\top}:$ By \textrm{\sc{no-rc-chain}} restriction, this case cannot happen.
	\end{enumerate}
\end{enumerate}
\end{proof}

(2) For events $e \in L_1^b(e_i^{\top}) \cup L_2^b(e_i^{\top})$, $e' \in L_1^b(e_j^{\top}) \cup L_2^b(e_j^{\top})$ where $j<i$, $\neg (e \xrightarrow{\F{lo_m}} e')$.
\begin{proof}
Suppose $e \xrightarrow{\F{lo_m}} e'$, $\neg (e \rightleftarrows e')$.
By definition of $L_1^b(e_i^{\top})$ and $L_2^b(e_i^{\top})$, we know that $e \xrightarrow{\F{lo}} e_i^{\top}$ and $e' \xrightarrow{\F{lo}} e_j^{\top}$.
We consider several possibilities based on this:
\begin{enumerate}
	\item Neither $e \xrightarrow[\F{vis}]{\F{lo}} e_i^{\top}$ nor $e' \xrightarrow[\F{vis}]{\F{lo}} e_j^{\top}$ is true because
		$L_\top^a$ is causally closed.
	\item $e \xrightarrow[\F{rc}]{\F{lo}} e_i^{\top} \wedge e' \xrightarrow[\F{rc}]{\F{lo}} e_j^{\top}:$
	\begin{enumerate}
		\item $e \xrightarrow{\F{rc}} e' \vee e' \xrightarrow{\F{rc}} e$ creates RC chain.
	\end{enumerate}	
\end{enumerate}
\end{proof}

\noindent \textbf{Theorem~\ref{thm:1}}
If an MRDT $\M{D}$ satisfies  $\textrm{\sc{BottomUp-2-OP}}$,  $\textrm{\sc{BottomUp-1-OP}}$,  $\textrm{\sc{BottomUp-0-OP}}$, 
\\$\textrm{\sc{MergeIdempotence}}$ and $\textrm{\sc{MergeCommutativity}}$, then $\M{D}$ is linearizable.
\begin{proof}
To prove that $\M{D}$ is linearizable, we will prove that any execution $\tau \in \llbracket \M{S_D} \rrbracket$ is linearizable, for which we will show that all of its configurations are linearizable.
Let $\tau = C_0 \xrightarrow{t_1} C_1 \xrightarrow{t_2} C_2 \ldots \xrightarrow{t_n} C$
be an execution of $\M{S_D}$, where $\{t_1,\dots, t_n\}$ are labels of the transition system. 
We prove by induction on the length of $\tau$. Base case of $C_0$ which consists of only 
one replica $r_0$ is trivially satisfied, as no operations are applied on the head version $v_0$ at $r_0$. 
Assuming the required result holds in the execution $C_0 \rightarrow^{*} C$, 
and suppose there is a new transition $C \rightarrow C^{'}$, we need to prove that $C$ is linearizable.
There are four cases corresponding to the four transition rules given in Fig. 8.

\subsubsection{Case $({\textrm{\sc{CreateBranch}})}$:}
Assume that a new replica $r^{'}$ is forked off from the origin replica $r$. Let $C = \langle N, H, L, G, vis \rangle$ 
and $C^{'} = \langle N^{'}, H^{'}, L^{'}, G^{'}, vis \rangle$ be the configurations of the replica before and after the branch creation.
According to the semantics, we have $L(H(r)) = L^{'}(H^{'}(r^{'}))$ and $N(H(r)) = N^{'}(H^{'}(r^{'}))$. We need to prove that Def.~\ref{def:lin} 
holds for $C^{'}$. This is obvious since Def.~\ref{def:lin} holds for C by the induction assumption.

\subsubsection{Case $({\textrm{\sc{Apply}}})$:}
Assume that an event $e$ is applied on a replica $r$. Let $C = \langle N, H, L, G, vis \rangle$ 
and $C^{'} = \langle N^{'}, H^{'}, L^{'}, G^{'}, vis^{'} \rangle$ be the configurations of the replica before and after the apply operation.
By semantics we have $L^{'}(H^{'}(r)) = L(H(r)) \cup \{e\}$. We need to prove that Def.~\ref{def:lin} holds for $C^{'}$. 
By induction assumption, $\exists \pi.\ \F{lo}(C)_{\mid L(H(r))} \subseteq \pi \wedge N(H(r)) = \pi(\sigma_0)$. 
Here $\F{lo}(C^{'})_{\mid L^{'}(H^{'}(r))}$ is the linearization order $\F{lo}(C)_{\mid L(H(r))}.e$ and $\pi^{'} = \pi.e$.
We need to show that $\pi^{'}$ extends  $\F{lo}(C^{'})_{\mid L^{'}(H^{'}(r))}$. 
We have $N^{'}(H^{'}(r)) = e(\pi(\sigma_0))$. Event $e$ is visible to all events in $\pi$ according to the semantics of $\F{apply}$. 
Since $\forall e^{'} \in \pi. e^{'} \xrightarrow[\F{vis}]{\F{lo}} e$,  $e \xrightarrow[\F{vis}]{\F{lo}} e^{'}$ is not possible due to
anti-symmetry of $\F{vis}$. $e \xrightarrow[\F{rc}]{\F{lo}} e^{'}$ is also
not possible as it would require $e$ and $e^{'}$ to be concurrent events. 
Hence, $\pi^{'}$ is a total order which extends $\F{lo}(C^{'})_{\mid L^{'}(H^{'}(r))}$. This proves the required result.

\subsubsection{Case $({\textrm{\sc{Merge}}})$:}
Consider there is a $\F{merge}(r_1,r_2)$ transition to $C^{'}$ where $r_2$ merges with $r_1$.
Let $C = \langle N, H, L, G, vis \rangle$, $C' = \langle N', H', L', G', vis \rangle$ , and let $H(r_1) = v_1, H(r_2) = v_2$. Let $v_\top$ be the 
LCA of $v_1$ and $v_2$ in $G$. Let $N(v_1) = a$, $N(v_2) = b$, $N(v_\top) = l$. 
The transition will install a version $v_m$ with state $m = \F{merge}(l, a, b)$ at 
the replica $r_1$, leaving the other replicas unchanged. Also, $L'(v_m) = L(v_1) \cup L(v_2)$. We need to show that 
there exists a sequence $\pi_m$ of events in $L'(v_m)$ such that $\pi_m$ extends $\F{lo}(C')_{\mid L'(v_m)}$ and $m = \pi(\sigma_0)$.
For ease of readability, we use $L_1$ for $L(v_1)$, $L_2$ for $L(v_2)$ and $L_\top$ for $L(v_\top)$, and $\F{lo_m}$ for $\F{lo}(C')_{\mid L'(v_m)}$. 

We repeat the definitions of various event sets below:
\begin{align*}
	& L_1' = L_1 \setminus L_\top \quad \quad  L_2' = L_2 \setminus L_\top \\
	& L_1^b = \{e \in L_1^{'}\ \mid\ \exists e_\top \in L_\top.\ (e \xrightarrow{\F{lo_m}} e_\top \vee  \exists e' \in L_1^{'}.\ e \xrightarrow{\F{lo_m}} e^{'} \xrightarrow{\F{lo_m}} e_\top)\}\\
	& L_2^b = \{e \in L_2^{'}\ \mid\ \exists e_\top \in L_\top.\ (e \xrightarrow{\F{lo_m}} e_\top \vee \exists e' \in L_2^{'}.\ e \xrightarrow{\F{lo_m}} e^{'} \xrightarrow{\F{lo_m}} e_\top)\}\\
	& L_\top^{a} = \{e_\top \in L_\top \mid \exists e \in L_1^{b} \cup L_2^{b}. e  \xrightarrow{\F{lo_m}} e_\top\}\\
	&  L_1^a = L_1^{'} \setminus L_1^b \quad  \quad L_2^a = L_2^{'} \setminus L_2^b \quad \quad L_\top^{b} = L_\top \setminus L_\top^{a}
\end{align*}

Let $\F{lo_i} = \F{lo}(C')_{\mid L_i}$ for $i=1,2$.

First we will prove that $\F{lo}$ between two events should remain the same in all versions.
$\forall e, e^{'} \in L_i. e \xrightarrow{\F{lo_i}} e^{'} \Leftrightarrow e \xrightarrow{\F{lo_m}} e^{'}$. Note that $\F{vis}$ and $\F{rc}$ ordering between events remains same in
$L_i$ and $L'(v_m)$. 
\begin{itemize}
	\item If $e \xrightarrow{\F{rc}} e^{'}, e \mid\mid_C e^{'}$ and $\neg(\exists e^{''} \in L(v_i). e^{'} \xrightarrow{\F{vis}} e^{''} \wedge \neg e^{'} \rightleftarrows e^{''})$, 
	then these constraints will continue to hold in $L_m$. Because it is not possible that $e^{'} \in L_1^{'} , e^{''} \in L_2^{'}$
	such that $e^{'} \xrightarrow{\F{vis}} e^{''}$. Because otherwise $e^{'} \in L_2^{'} \Rightarrow e^{'} \in L_\top$. 
	\item If $e\xrightarrow{\F{vis}} e^{'} \wedge \neg e \rightleftarrows e^{'}$ in $L_i$, then it continues to hold in $L_m$.
\end{itemize}

By induction assumption, we know that\\
$\exists \pi_a.\ \F{lo}(C)_{\mid L(v_1)} \subseteq \pi_a \wedge a = \pi_a(\sigma_0)$\\
$\exists \pi_b.\ \F{lo}(C)_{\mid L(v_2)} \subseteq \pi_b \wedge b = \pi_b(\sigma_0)$\\
$\exists \pi_\top.\ \F{lo}(C)_{\mid L(v_\top)} \subseteq \pi_\top \wedge l = \pi_\top(\sigma_0)$\\

To start off, let's consider the set $L_1^a \cup L_2^a$. These are all local events of $v_1$ and $v_2$, which are not linearized before events of the LCA. 
We consider different cases depending on the size if this set.\\

\noindent\textrm{\sc{Case 1}}: $(\mid L_1^{a} \cup L_2^{a} \mid = 0)$\\
We note that in this case, $a,b$ can be defined as follows:
$a= {\pi_a}_{\mid (L_\top^{b} \cup L_1^{b} \cup L_\top^{a})}(\sigma_0)$,
$b = {\pi_b}_{\mid (L_\top^{b} \cup L_2^{b} \cup L_\top^{a})}(\sigma_0)$.\\
We need to show that there exists a sequence $\pi_m$ that extends $\F{lo_m}$ such that $\F{merge}(l,a,b) = \pi_m(\sigma_0)$.
Here, we induct on the size of the set $L_\top^a$.\\

\noindent\textrm{\sc{Base Case 1}}: $(\mid L_\top^a \mid  = 0)$\\
Then $L_1^b \cup L_2^b = \phi$. So $l = a = b$. $\F{merge}(l, l, l) = l$ is inferred by $\textrm{\sc{MergeIdempotence}}$. We know that $l$ is correctly linearized, hence the required result follows.\\

\noindent\textrm{\sc{Inductive Case 1}}: $(\mid L_\top^a  \mid > 0)$\\
Let $L_\top^a = \{e_1^{\top}, \dots, e_{m-1}^{\top}, e_m^{\top}\}$. Let $S = \{e_1^{\top}, \dots, e_{m-1}^{\top}\}$.
By IH, for the set $S$, we have the required result. We define $l',a',b'$ based on the above set $S$: $l'=\pi_{l_{\mid L_\top^b \cup S}}(\sigma_0)$, $a' = \pi_{a_{\mid L_\top^b \cup \bigcup_{e \in S} L_1^b(e) \cup S}}(\sigma_0)$, $b' = \pi_{b_{\mid L_\top^b \cup \bigcup_{e \in S} L_2^b(e) \cup S}(\sigma_0)}$. Note that in this case, all the LCA events which are linearized after local events are already taken as part of the states $l',a',b'$. Now, suppose we add one more LCA event $e_m^{\top}$ to all states. We define $a'',b''$ such that
$a'' = {\pi_a}_{\mid L_1^b(e_m^{\top})} (a')$,
$b'' = {\pi_b}_{\mid L_2^b(e_m^{\top})} (b')$. 

Then, $l = e_m^{\top}(l'), a = e_m^{\top}(a''), b = e_m^{\top}(b'')$.  $e_m^{\top}$ is not linearized before any of the events in $L_\top^{b} \cup L_1^{b} \cup L_2^b \cup S$  based on the definition of $L_\top^a$.

Now, by $\textrm{\sc{BottomUp-0-OP}}$ rule,

\begin{equation}\label{eq:0op}
	\F{merge}(e_m^{\top}(l'), e_m^{\top}(a''), e_m^{\top}(b'')) = e_m^{\top}(\F{merge}(l',a'',b''))
\end{equation}

Now that we have linearized $e_m^{\top}$, we need to linearize the events that led to $\F{merge}(l',a'',b'')$. 
Let's denote $L_1^b(e_m^{\top})$ as $M_1^a$ and $L_2^b(e_m^{\top})$ as $M_2^a$. Now we induct on the size of the set $M_1^{a} \cup M_2^{a}$.\\

\noindent\textrm{\sc{Base Case 1.1}}:$(\mid M_1^{a} \cup M_2^{a} \mid = 0)$\\
$a'' = a', b'' = b'$. By induction assumption, $\exists \pi.\ \F{lo}(C)_{\mid (L_\top^b \cup \bigcup_{e \in S} L_1^b(e) \cup \bigcup_{e \in S} L_2^b(e) \cup S)} \subseteq \pi$\\ 
and $\F{merge}(l',a',b') = \pi(\sigma_0)$. Hence, $\pi_m = \pi. e_m^{\top}$.\\

\noindent\textrm{\sc{Inductive Case 1.1}}:$(\mid M_1^{a} \cup M_2^{a} \mid > 0)$\\
We have 2 cases here: \\(1.1.1) Either of $M_1^{a}$ or $M_2^{a}$ is $\phi$ \\(1.1.2) Both $M_1^{a}$ and $M_2^{a}$ are not $\phi$.\\

\noindent\textrm{\sc{Case 1.1.1}}:$(M_1^a \neq \phi \wedge M_2^a = \phi)$\\
Consider $e_1 \in M_1^a$ such that there does not exist $e \in M_1^a$ and $e_1 \xrightarrow{\F{lo_m}} e$, i.e. $e_1$ is the maximal event according to $\F{lo_m}$. Since $\F{lo}$ ordering between events remains the same in all versions, and since versions $v_1$ and $v_2$ (which are being merged) were already linearizable, there would exist sequences leading to the states $a$ such that $e_1$ would appear at the end. Hence, there exists $a'''$ such that $a'' = e_1(a''')$. Since $M_2^a$ is empty, all local events in $L_2$ are linearized before the rest of the  LCA events.
Suppose $L_\top^a \setminus {e_m^{\top}} \neq \phi$ or $L_\top^b \neq \phi$, the last event which leads to the state $l',b''$ must be an LCA event.  Let's consider $e_\top$ to be the maximal event in $L_\top$ according to $\F{lo_m}$.
Hence there exists states $l'', b'''$ such that $l' = e_\top(l''), b'' = e_\top(b''')$.
By $\textrm{\sc{BottomUp-1-OP}}$ rule
\begin{equation}\label{eq:1op9}
	\F{merge}(e_\top(l''), e_1(a'''), e_\top(b''')) = e_1(\F{merge}(e_\top(l''),a''',e_\top(b''')))
\end{equation}

If both  $L_\top^a \setminus {e_m^{\top}} = \phi$ and $L_\top^b = \phi$, then $l' = b'' = \sigma_0$. By $\textrm{\sc{BottomUp-1-OP}}$

\begin{equation*}
	\F{merge}(\sigma_0, e_1(a'''), \sigma_0) = e_1(\F{merge}(\sigma_0,a''', \sigma_0))
\end{equation*}

From the induction assumption, we get that $\F{merge}(e_\top(l''),a''',e_\top(b'''))$ is already obtained by 
the linearization of events applied on the initial state $\sigma_0$. That is, there exists a sequence $\pi'$ over events in $L_\top^b \cup \bigcup_{e \in S} L_1^b(e) \cup \bigcup_{e \in S} L_2^b(e) \cup S \cup M_1^a \setminus e_1$ which extends $\F{lo_m}$ relation such that $\F{merge}(e_\top(l''),a''',e_\top(b''')) = \pi'(\sigma_0)$. Now, $\pi = \pi'. e_1$ is the required linearization. 

Let $\F{lo}_1$ be the linearization relation for $\F{merge}(e_\top(l''),a''',e_\top(b'''))$ (i.e. from the RHS in Eq. \eqref{eq:1op9}, without the event $e_1$) and let $\F{lo}_2$ be the linearization relation for $\F{merge}(l',a'',b'')$ (i.e. the LHS in Eq. \eqref{eq:1op9}). Then $\pi'$ according to the IH extends $\F{lo}_1$. We will show that for any pair of events $e,e'$ in $\F{merge}(e_\top(l''),a''',e_\top(b'''))$ , $e \xrightarrow{\F{lo}_2} e' \implies e \xrightarrow{\F{lo}_1} e'$. This ensures that if $\pi$ extends $\F{lo}_2$. Now, the $\F{vis}$ and $\F{rc}$ relation between $e'$ and $e$ remains the same while determining both $\F{lo}_1$ and $\F{lo}_2$. If $e \xrightarrow[\F{rc}]{\F{lo}_2} e'$, then $e'$ cannot be visible to any non-commutative event while calculating $\F{lo}_2$, but then the same should be true for $\F{lo}_1$ as well. If $e \xrightarrow[\F{vis}]{\F{lo}_2} e'$, then clearly $e \xrightarrow[\F{vis}]{\F{lo}_1} e'$. This concludes the proof that $\pi = \pi'. e_1$ must extend $\F{lo}_2$.\\

\noindent\textrm{\sc{Case 1.1.2}}:$(M_1^a \neq \phi \wedge M_2^a \neq \phi)$\\
Consider $e_1 \in M_1^a, e_2 \in M_2^a$ such that there does not exist $e \in M_i^a$ and $e_i \xrightarrow{\F{lo_m}} e$ (for $i=1,2$), i.e. each of the $e_i$s are maximal events according to $\F{lo_m}$. Since $\F{lo}$ ordering between events remains the same in all versions, and since versions $v_1$ and $v_2$ (which are being merged) were already linearizable, there would exist sequences leading to the states $a''$ and $b''$ such that $e_1$ and $e_2$ would appear at the end resp. Hence, there exists $a'''$ and $b'''$ such that $a'' = e_1(a''')$ and $b'' = e_1(b''')$. Since $e_1 \mid\mid_C e_2$, they are related to each other by $\F{rc}$ relation or they commute with each other i.e., $e_1 \xrightarrow{\F{rc}} e_2 \vee e_2 \xrightarrow{\F{rc}} e_1 \vee e_1 \rightleftarrows e_2$.
We will consider the case when $e_2 \xrightarrow{\F{rc}} e_1 \vee e_1 \rightleftarrows e_2$. $e_1 \xrightarrow{\F{rc}} e_2$ is handled by $\textrm{\sc{MergeCommutativity}}$.
The equation becomes 
\begin{equation}\label{eq:2op9}
\F{merge}(l', e_1(a'''), e_2(b''')) = e_1(\F{merge}(l', a''', e_2(b''')))
\end{equation}
which is the $\textrm{\sc{BottomUp-2-OP}}$ rule.\\

From the induction assumption, we get that $\F{merge}(l', a''', e_2(b'''))$ is already obtained by 
the linearization of events applied on the initial state $\sigma_0$. If $\pi'$ is the linearization for this merge, then $\pi = \pi'. e_1$ is the required linearization.

For this, we prove that $e_1$ is not linearized before any of the events in $M_1^a \textbackslash \{e_1\} \cup M_2^a$.
Clearly, $e_1$ is not linearized before any event in $M_1^a \textbackslash \{e_1\}$ because it is the maximal event on that branch.
Since $e_2 \xrightarrow{\F{rc}} e_1, e_1 \xrightarrow{\F{vis}} e_2$ is not possible.
$e_1 \xrightarrow{\F{rc}} e_2$ is not possible as $\F{rc}^+$ is irreflexive.
So $e_1 \xrightarrow{\F{lo}} e_2$ is not possible.
Let's assume there is some event $e$ in $ M_2^a \textbackslash \{e_2\}$ that comes $\F{lo}$ after $e_1$. There are 2 possibilities.
\begin{itemize}
	\item $e_1 \xrightarrow{\F{rc}} e:$ Since $e_2 \xrightarrow{\F{rc}} e_1$, this case is not possible due to $\textrm{\sc{no-rc-chain}}$ restriction.
	\item $e_1 \xrightarrow{\F{vis}} e:$ This is not possible as events in $M_2^a \textbackslash \{e_2\}$ are concurrent with $e_1$. 
		This is because every version is causally closed.
\end{itemize}


\noindent\textrm{\sc{Case 2}}: $(\mid L_1^{a} \cup L_2^{a} \mid > 0)$\\
The proof here will be identical to the proof of Inductive Case 1.1, substituting $L_1^a$ and $L_2^a$ for $M_1^a$ and $M_2^a$, and using the rules $\textrm{\sc{BottomUp-1-OP}}$, $\textrm{\sc{MergeCommutativity}}$ and  $\textrm{\sc{BottomUp-2-OP}}$.

\subsubsection{Case $({\textrm{\sc{Query}})}$:}
Assume that a query operation is applied on a replica $r$. Let $C = \langle N, H, L, G, vis \rangle$ 
be the configuration of the replica before the operation. According to the semantics, the configuration of the 
replica remains same after the query operation. By the induction hypothesis, Def.~\ref{def:lin} holds for the configuration $C$.
\end{proof}


\noindent \textbf{Theorem~\ref{thm:2}}
If an MRDT $\M{D}$ satisfies  the VCs $\psi^*(\textrm{\sc{BottomUp-2-OP}})$,  $\psi^*(\textrm{\sc{BottomUp-1-OP}})$,  \\$\psi^*(\textrm{\sc{BottomUp-0-OP}})$, $\textrm{\sc{MergeIdempotence}}$ and $\textrm{\sc{MergeCommutativity}}$, then $\M{D}$ is linearizable.

\begin{proof}
To prove that $\M{D}$ is linearizable, we will prove that any execution $\tau \in \llbracket \M{S_D} \rrbracket$ is linearizable, for which we will show that all of its configurations are linearizable.
Let $\tau = C_0 \xrightarrow{t_1} C_1 \xrightarrow{t_2} C_2 \ldots \xrightarrow{t_n} C$
be an execution of $\M{S_D}$, where $\{t_1,\dots, t_n\}$ are labels of the transition system. 
We prove by induction on the length of $\tau$. Base case of $C_0$ which consists of only 
one replica $r_0$ is trivially satisfied, as no operations are applied on the head version $v_0$ at $r_0$. 
Assuming the required result holds in the execution $C_0 \rightarrow^{*} C$, 
and suppose there is a new transition $C \rightarrow C^{'}$, we need to prove that $C$ is linearizable.
There are four cases corresponding to the four transition rules given in Fig. 8.

\subsubsection{Case $({\textrm{\sc{CreateBranch}})}$:}
Assume that a new replica $r^{'}$ is forked off from the origin replica $r$. Let $C = \langle N, H, L, G, vis \rangle$ 
and $C^{'} = \langle N^{'}, H^{'}, L^{'}, G^{'}, vis \rangle$ be the configurations of the replica before and after the branch creation.
According to the semantics, we have $L(H(r)) = L^{'}(H^{'}(r^{'}))$ and $N(H(r)) = N^{'}(H^{'}(r^{'}))$. We need to prove that Def.~\ref{def:lin}
holds for $C^{'}$. This is obvious since Def.~\ref{def:lin} holds for C by the induction assumption.

\subsubsection{Case $({\textrm{\sc{Apply}}})$:}
Assume that an event $e$ is applied on a replica $r$. Let $C = \langle N, H, L, G, vis \rangle$ 
and $C^{'} = \langle N^{'}, H^{'}, L^{'}, G^{'}, vis^{'} \rangle$ be the configurations of the replica before and after the apply operation.
By semantics we have $L^{'}(H^{'}(r)) = L(H(r)) \cup \{e\}$. We need to prove that Def.~\ref{def:lin} holds for $C^{'}$. 
By induction assumption, $\exists \pi.\ \F{lo}(C)_{\mid L(H(r))} \subseteq \pi \wedge N(H(r)) = \pi(\sigma_0)$. 
Here $\F{lo}(C^{'})_{\mid L^{'}(H^{'}(r))}$ is the linearization order $\F{lo}(C)_{\mid L(H(r))}.e$ and $\pi^{'} = \pi.e$.
We need to show that $\pi^{'}$ extends  $\F{lo}(C^{'})_{\mid L^{'}(H^{'}(r))}$. 
We have $N^{'}(H^{'}(r)) = e(\pi(\sigma_0))$. Event $e$ is visible to all events in $\pi$ according to the semantics of $\F{apply}$. 
Since $\forall e^{'} \in \pi. e^{'} \xrightarrow[\F{vis}]{\F{lo}} e$,  $e \xrightarrow[\F{vis}]{\F{lo}} e^{'}$ is not possible due to
anti-symmetry of $\F{vis}$. $e \xrightarrow[\F{rc}]{\F{lo}} e^{'}$ is also
not possible as it would require $e$ and $e^{'}$ to be concurrent events. 
Hence, $\pi^{'}$ is a total order which extends $\F{lo}(C^{'})_{\mid L^{'}(H^{'}(r))}$. This proves the required result.

\subsubsection{Case $({\textrm{\sc{Merge}}})$:}
Consider there is a $\F{merge}(r_1,r_2)$ transition to $C^{'}$ where $r_2$ merges with $r_1$.
Let $C = \langle N, H, L, G, vis \rangle$, $C' = \langle N', H', L', G', vis \rangle$ , and let $H(r_1) = v_1, H(r_2) = v_2$. Let $v_\top$ be the 
LCA of $v_1$ and $v_2$ in $G$. Let $N(v_1) = a$, $N(v_2) = b$, $N(v_\top) = l$. 
The transition will install a version $v_m$ with state $m = \F{merge}(l, a, b)$ at 
the replica $r_1$, leaving the other replicas unchanged. Also, $L'(v_m) = L(v_1) \cup L(v_2)$. We need to show that 
there exists a sequence $\pi_m$ of events in $L'(v_m)$ such that $\pi_m$ extends $\F{lo}(C')_{\mid L'(v_m)}$ and $m = \pi(\sigma_0)$.
For ease of readability, we use $L_1$ for $L(v_1)$, $L_2$ for $L(v_2)$ and $L_\top$ for $L(v_\top)$, and $\F{lo_m}$ for $\F{lo}(C')_{\mid L'(v_m)}$. 

By induction assumption, we know that\\
$\exists \pi_a.\ \F{lo}(C)_{\mid L(v_1)} \subseteq \pi_a \wedge a = \pi_a(\sigma_0)$\\
$\exists \pi_b.\ \F{lo}(C)_{\mid L(v_2)} \subseteq \pi_b \wedge b = \pi_b(\sigma_0)$\\
$\exists \pi_\top.\ \F{lo}(C)_{\mid L(v_\top)} \subseteq \pi_\top \wedge l = \pi_\top(\sigma_0)$\\

To start off, let's consider the set $L_1^a \cup L_2^a$. These are all local events of $v_1$ and $v_2$, which are not linearized before events of the LCA. 
We consider different cases depending on the size of this set.\\

\noindent\textrm{\sc{Case 1}}: $(\mid L_1^{a} \cup L_2^{a} \mid = 0)$\\
We note that in this case, $a,b$ can be defined as follows:
$a= {\pi_a}_{\mid (L_\top^{b} \cup L_1^{b} \cup L_\top^{a})}(\sigma_0)$,
$b = {\pi_b}_{\mid (L_\top^{b} \cup L_2^{b} \cup L_\top^{a})}(\sigma_0)$.\\
We need to show that there exists a sequence $\pi_m$ that extends $\F{lo_m}$ such that $\F{merge}(l,a,b) = \pi_m(\sigma_0)$.
Here, we induct on the size of the set $L_\top^a$.\\

\noindent\textrm{\sc{Base Case 1}}: $(L_\top^a  = \phi)$\\
Then $L_1^b \cup L_2^b = \phi$. So $l = a = b$. $\F{merge}(l, l, l) = l$ is handled by $\textrm{\sc{MergeIdempotence}}$.
We know that $l$ is correctly linearized, hence the required result follows.\\

\noindent\textrm{\sc{Inductive Case 1}}: $(\mid L_\top^a  \mid > 0)$\\
Let $L_\top^a = \{e_1^{\top}, \dots, e_{m-1}^{\top}, e_m^{\top}\}$. Let $S = \{e_1^{\top}, \dots, e_{m-1}^{\top}\}$.
By IH, for the set $S$, we have the required result. We define $l',a',b'$ based on the above set $S$: $l'=\pi_{l_{\mid L_\top^b \cup S}}(\sigma_0)$, $a' = \pi_{a_{\mid L_\top^b \cup \bigcup_{e \in S} L_1^b(e) \cup S}}(\sigma_0)$, $b' = \pi_{b_{\mid L_\top^b \cup \bigcup_{e \in S} L_2^b(e) \cup S}(\sigma_0)}$. Note that in this case, all the LCA events which are linearized after local events are already taken as part of the states $l',a',b'$. Now, suppose we add one more LCA event $e_m^{\top}$ to all states. We define $a'',b''$ such that
$a'' = {\pi_a}_{\mid L_1^b(e_m^{\top})} (a')$,
$b'' = {\pi_b}_{\mid L_2^b(e_m^{\top})} (b')$. 

Then, $l = e_m^{\top}(l'), a = e_m^{\top}(a''), b = e_m^{\top}(b'')$.  $e_m^{\top}$ is not linearized before any of the events in $L_\top^{b} \cup L_1^{b} \cup L_2^b \cup S$  based on the definition of $L_\top^a$.

Now, by $\textrm{\sc{BottomUp-0-OP}}$ rule,

\begin{equation}\label{eq:0}
	\F{merge}(e_m^{\top}(l'), e_m^{\top}(a''), e_m^{\top}(b'')) = e_m^{\top}(\F{merge}(l',a'',b''))
\end{equation}

We will now show prove $\textrm{\sc{BottomUp-0-OP}}$ rule, i.e. Eqn. \eqref{eq:0}:

\noindent\textrm{\sc{Proof of } Eq. \eqref{eq:0}}: 

Let $l_b = \pi_{l_{\mid L_\top^b}}(\sigma_0)$.

We first induct on $\mid L_\top^b \mid$ to show that $\F{merge}(e_m^{\top}(l_b), e_m^{\top}(l_b), e_m^{\top}(l_b)) = e_m^{\top}(\F{merge}(l_b, l_b, l_b))$

For the base case, we use $\psi^{L_\top^b}_\F{base-0op}$. For the inductive case, we use $\psi^{L_\top^b}_\F{ind-0op}$, whose pre-condition will be satisfied by the IH. 

Next, we induct on $\mid L_\top^a \setminus \{e_m^{\top}\} \mid$ to show Eqn. \eqref{eq:0}.

For the base case, we have $\mid L_\top^a \setminus \{e_m^{\top}\} \mid = 0$. In this case, the set $S = \emptyset$. Also, $l' = a' = b' = l_b$. Hence, we need to show the following: 

\begin{equation}\label{eq:100}
	\F{merge}(e_m^{\top}(l_b), e_m^{\top}({\pi_a}_{\mid L_1^b(e_m^{\top})} (a')), e_m^{\top}({\pi_b}_{\mid L_2^b(e_m^{\top})} (b'))) = e_m^{\top}(\F{merge}(l',{\pi_a}_{\mid L_1^b(e_m^{\top})} (a'),{\pi_b}_{\mid L_2^b(e_m^{\top})} (b')))
\end{equation}

We will now induct on $\mid L_1^b(e_m^{\top}) \cup L_2^b(e_m^{\top}) \mid$ to show Eqn. \eqref{eq:100}.

For the base case where $\mid L_1^b(e_m^{\top}) \cup L_2^b(e_m^{\top}) \mid = 0$, it directly follows from the outcome of the induction on $\mid L_\top^b \mid$.

For the inductive case, we use one of $\psi^{L_1^b}_\F{ind1-0op}$, $\psi^{L_1^b}_\F{ind2-0op}$, $\psi^{L_2^b}_\F{ind1-0op}$ or $\psi^{L_2^b}_\F{ind2-0op}$ depending on the event $e_b$ or $e$ to be added to $L_1^b(e_m^{\top})$ or $L_2^b(e_m^{\top})$, with the pre-condition of these VCs being inferred from the IH.

This completes the proof of Eqn. \eqref{eq:100}.

Now, we consider the inductive case for $\mid L_\top^a \setminus \{e_m^{\top}\} \mid$ to show Eqn. \eqref{eq:101}.  By IH, we get the following:
\begin{equation}\label{eq:3}
	\F{merge}(e_m^{\top}(l'''), e_m^{\top}(a'''), e_m^{\top}(b''')) = e_m^{\top}(\F{merge}(l''',a''',b'''))
\end{equation}

where for the set $S' = S \setminus e_{m-1}^{\top}$, $l'''=\pi_{l_{\mid L_\top^b \cup S'}}(\sigma_0)$, $a''' = \pi_{a_{\mid L_\top^b \cup \bigcup_{e \in S'} L_1^b(e) \cup S'}}(\sigma_0)$, $b''' = \pi_{b_{\mid L_\top^b \cup \bigcup_{e \in S'} L_2^b(e) \cup S'}(\sigma_0)}$. That is, we consider the effects of all event in $S$ except $e_{m-1}^{\top}$. 

Now, we first use $\psi^{L_\top^a}_\F{ind-0op}$ to apply $e_{m-1}^{\top}$ to $l'''$, $a'''$ and $b'''$. Note that the pre-condition for  $\psi^{L_\top^a}_\F{ind-0op}$ is satisfied due to Eqn. \eqref{eq:3}. 

Next, we use induct on $\mid L_1^b(e_{m-1}^{\top}) \cup L_2^b(e_{m-1}^{\top}) \mid$ using the VCs $\psi^{L_1^b}_\F{ind1-0op}$, $\psi^{L_1^b}_\F{ind2-0op}$, $\psi^{L_2^b}_\F{ind1-0op}$ or $\psi^{L_2^b}_\F{ind2-0op}$ to add all events in these sets. Finally, we induct on $\mid L_1^b(e_{m}^{\top}) \cup L_2^b(e_{m}^{\top}) \mid$ to again add all these events, thereby proving Eqn. \eqref{eq:101}.

Now that we have linearized $e_m^{\top}$ using Eqn. \eqref{eq:101}, we need to linearize the events that led to $\F{merge}(l',a'',b'')$. 
Let's denote $L_1^b(e_m^{\top})$ as $M_1^a$ and $L_2^b(e_m^{\top})$ as $M_2^a$. Now we induct on the size of the set $M_1^{a} \cup M_2^{a}$.\\

\noindent\textrm{\sc{Base Case 1.1}}:$(\mid M_1^{a} \cup M_2^{a} \mid = 0)$\\
$a'' = a', b'' = b'$. By induction assumption, $\exists \pi.\ \F{lo}(C)_{\mid (L_\top^b \cup \bigcup_{e \in S} L_1^b(e) \cup \bigcup_{e \in S} L_2^b(e) \cup S)} \subseteq \pi$\\ 
and $\F{merge}(l',a',b') = \pi(\sigma_0)$. Hence, $\pi_m = \pi. e_m^{\top}$.\\

\noindent\textrm{\sc{Inductive Case 1.1}}:$(\mid M_1^{a} \cup M_2^{a} \mid > 0)$\\
We have 2 cases here: \\(1.1.1) Either of $M_1^{a}$ or $M_2^{a}$ is $\phi$ \\(1.1.2) Both $M_1^{a}$ and $M_2^{a}$ are not $\phi$.\\

\noindent\textrm{\sc{Case 1.1.1}}:$(M_1^a \neq \phi \wedge M_2^a = \phi)$\\
Consider $e_1 \in M_1^a$ such that there does not exist $e \in M_1^a$ and $e_1 \xrightarrow{\F{lo_m}} e$, i.e. $e_1$ is the maximal event according to $\F{lo_m}$. Since $\F{lo}$ ordering between events remains the same in all versions, and since versions $v_1$ and $v_2$ (which are being merged) were already linearizable, there would exist sequences leading to the states $a$ such that $e_1$ would appear at the end. Hence, there exists $a'''$ such that $a'' = e_1(a''')$. Since $M_2^a$ is empty, all local events in $L_2$ are linearized before the rest of the  LCA events.
Suppose $L_\top^a \setminus \{e_m^{\top}\} \neq \phi$ or $L_\top^b \neq \phi$, the last event which leads to the state $l',b''$ must be an LCA event.  Let's consider $e_\top$ to be the maximal event in $L_\top$ according to $\F{lo_m}$.
Hence there exists states $l'', b'''$ such that $l' = e_\top(l''), b'' = e_\top(b''')$.
By $\textrm{\sc{BottomUp-1-OP}}$ rule
\begin{equation}\label{eq:1op}
	\F{merge}(e_\top(l''), e_1(a'''), e_\top(b''')) = e_1(\F{merge}(e_\top(l''),a''',e_\top(b''')))
\end{equation}

Again, we prove $\textrm{\sc{BottomUp-1-OP}}$ rule using the same induction scheme that we showed for $\textrm{\sc{BottomUp-0-OP}}$. Briefly, we use $\psi^{L_\top^b}_\F{base-1op}$ and $\psi^{L_\top^b}_\F{ind-1op}$ for induction on $\mid L_\top^b \mid$. Then, we use $\psi^{L_\top^a}_\F{ind-1op}$, $\psi^{L_1^b}_\F{ind1-1op}$, $\psi^{L_1^b}_\F{ind2-1op}$, $\psi^{L_2^b}_\F{ind1-1op}$ and $\psi^{L_2^b}_\F{ind2-1op}$ to build the event sets $L_\top^a \setminus \{e_m^{\top}\}$ and $\sqcup_{e \in L_\top^a \setminus \{e_m^{\top}\}} L_1^b(e) \cup \sqcup_{e \in L_\top^a \setminus \{e_m^{\top}\}} L_2^b(e)$.


From the induction assumption, we get that $\F{merge}(e_\top(l''),a''',e_\top(b'''))$ is already obtained by 
the linearization of events applied on the initial state $\sigma_0$. That is, there exists a sequence $\pi'$ over events in $L_\top^b \cup \bigcup_{e \in S} L_1^b(e) \cup \bigcup_{e \in S} L_2^b(e) \cup S \cup M_1^a \setminus e_1$ which extends $\F{lo_m}$ relation such that $\F{merge}(e_\top(l''),a''',e_\top(b''')) = \pi'(\sigma_0)$. Now, $\pi = \pi' e_1$ is the required linearization. \\


\noindent\textrm{\sc{Case 1.1.2}}:$(M_1^a \neq \phi \wedge M_2^a \neq \phi)$\\
Consider $e_1 \in M_1^a, e_2 \in M_2^a$ such that there does not exist $e \in M_i^a$ and $e_i \xrightarrow{\F{lo_m}} e$ (for $i=1,2$), i.e. each of the $e_i$s are maximal events according to $\F{lo_m}$. Since $\F{lo}$ ordering between events remains the same in all versions, and since versions $v_1$ and $v_2$ (which are being merged) were already linearizable, there would exist sequences leading to the states $a''$ and $b''$ such that $e_1$ and $e_2$ would appear at the end resp. Hence, there exists $a'''$ and $b'''$ such that $a'' = e_1(a''')$ and $b'' = e_1(b''')$. Since $e_1 \mid\mid_C e_2$, they are related to each other by $\F{rc}$ relation or they commute with each other i.e., $e_1 \xrightarrow{\F{rc}} e_2 \vee e_2 \xrightarrow{\F{rc}} e_1 \vee e_1 \rightleftarrows e_2$.
We will consider the case when $e_2 \xrightarrow{\F{rc}} e_1 \vee e_1 \rightleftarrows e_2$. $e_1 \xrightarrow{\F{rc}} e_2$ is handled by $\textrm{\sc{MergeCommutativity}}$.
The equation becomes 
\begin{equation}\label{eq:2op10}
	\F{merge}(l', e_1(a'''), e_2(b''')) = e_1(\F{merge}(l', a''', e_2(b''')))
\end{equation}
which is the $\textrm{\sc{BottomUp-2-OP}}$ rule.\\

Again, we prove $\textrm{\sc{BottomUp-2-OP}}$ rule using the same induction scheme that we showed for $\textrm{\sc{BottomUp-1-OP}}$. Briefly, we use $\psi^{L_\top^b}_\F{base-2op}$ and $\psi^{L_\top^b}_\F{ind-2op}$ for induction on $\mid L_\top^b \mid$. Then, we use $\psi^{L_\top^a}_\F{ind-2op}$, $\psi^{L_1^b}_\F{ind1-2op}$, $\psi^{L_1^b}_\F{ind2-2op}$, $\psi^{L_2^b}_\F{ind1-2op}$ and $\psi^{L_2^b}_\F{ind2-2op}$ to build the event sets $L_\top^a \setminus \{e_m^{\top}\}$ and $\sqcup_{e \in L_\top^a \setminus \{e_m^{\top}\}} L_1^b(e) \cup \sqcup_{e \in L_\top^a \setminus \{e_m^{\top}\}} L_2^b(e)$.\\


From the induction assumption, we get that $\F{merge}(l', a''', e_2(b'''))$ is already obtained by 
the linearization of events applied on the initial state $\sigma_0$. If $\pi'$ is the linearization for this merge, then $\pi = \pi' e_1$ is the required linearization.\\

\noindent\textrm{\sc{Case 2}}: $(\mid L_1^{a} \cup L_2^{a} \mid > 0)$\\
The proof here will be identical to the proof of Inductive Case 1.1, substituting $L_1^a$ and $L_2^a$ for $M_1^a$ and $M_2^a$, and using the rules $\textrm{\sc{BottomUp-1-OP}}$, $\textrm{\sc{MergeCommutativity}}$ and  $\textrm{\sc{BottomUp-2-OP}}$.


\subsubsection{Case $({\textrm{\sc{Query}})}$:}
Assume that a query operation is applied on a replica $r$. Let $C = \langle N, H, L, G, vis \rangle$ 
be the configuration of the replica before the operation. According to the semantics, the configuration of the 
replica remains same after the query operation. By the induction hypothesis, Def.~\ref{def:lin} holds for the configuration $C$.


\begin{comment}



Let $L_\top^a = \{e_1^{\top}, \dots, e_m^{\top}\}$ be the linearization of events in $L_\top^a$. 
We will induct on the sequence of the form $L_1^b(e_i^{\top}) \cup L_2^b(e_i^{\top}). e_i^{\top}$.\\

\noindent\textrm{\sc{Base Case 1.1}}: $(\mid L_\top^a \mid = 1)$\\
Let $L_\top^a = \{e_\top\}$. We need to show that 
\begin{equation}\label{eq:0op}
	\F{merge}(e_\top(l'), e_\top(a'), e_\top(b')) = e_\top(\F{merge}(l',a',b'))
\end{equation}
where $l'=\pi_{\mid L_\top^b \cup S}$, \\
$a' = \pi_{\mid L_\top^b \cup L_1^b(e_\top)}(\sigma_0)$, \\
$b' = \pi_{\mid L_\top^b \cup L_2^b(e_\top)}(\sigma_0)$\\
$l = e_\top(l'), a = e_\top(a'), b = e_\top(b')$. 
According to the definition of $L_\top^a$, no other event in $L_\top^b, L_1^b, L_2^b$ 
can come before $e_\top$ in the linearization order.
Now, we will induct on $\mid L_1^b \cup L_2^b \mid.$\\

\noindent\textrm{\sc{Base Case 1.1.1}}:$(\mid L_1^b \cup L_2^b \mid = 1)$\\
For the base case, we consider single events in $L_1^b$ and $L_2^b$.
\begin{itemize}
	\item $(\mid L_1^b \mid = 1  ~\wedge \mid L_2^b \mid = 0):$ Let $L_1^b = \{e_b\}$. We need to prove that
	$\F{merge}(e_\top(l'), e_\top(e_b(l')), e_\top(l')) = e_\top(\F{merge}(l',e_b(l'),l'))$ which is handled by $\psi^{L_1^b}_\F{ind1-0op}$. 
	$\F{merge}(l',e_b(l'),l')$ is linearized by $\textrm{\sc{BottomUp-1-OP}}$ rule.
	\item $(\mid L_1^b \mid = 0 ~\wedge \mid L_2^b \mid = 1):$ We use \textrm{\sc{MergeCommutativity}} to prove this case.	\\
\end{itemize}

\noindent\textrm{\sc{Inductive Case 1.1.1}}:$(\mid L_1^b \cup L_2^b \mid > 1)$\\
$\psi^{L_1^b}_\F{ind2-0op}$ and $\psi^{L_2^b}_\F{ind2-0op}$ are the VCs for the inductive case on $\mid L_1^b \cup L_2^b \mid$. Focusing on $\psi^{L_2^b}_\F{ind2-0op}$, in the pre-condition it assumes the presence of an event $e_b$ which is linearized before $e_\top$ on the right branch, and in the post-condition, it adds a new event $e$ before $e_b$. As per the definition of $L_2^b$, the pre-condition also asserts that the new event $e$ should either not commute with $e_b$ (which takes care of events of the form $e \xrightarrow{\F{lo_m}} e_b \xrightarrow{\F{lo_m}} e_\top$), or it should be in $\F{rc}$ before $e_\top$. $\psi^{L_1^b}_\F{ind2-0op}$ works similarly for the left branch.\\

\noindent\textrm{\sc{Inductive Case 1.1}}: $(\mid L_\top^a \mid > 1)$\\
Assume that for set $S=\{e_1^{\top}, \ldots, e_{i-1}^{\top}\}$, we have the required result. We re-define $l', a', b'$ based on the above set $S$: $l'=\pi_{\mid L_\top^b \cup S}$, $a' = \pi_{\mid L_\top^b \cup \bigcup_{e \in S} L_1^b(e) \cup S}(\sigma_0)$, $b' = \pi_{\mid L_\top^b \cup \bigcup_{e \in S} L_2^b(e) \cup S}(\sigma_0)$. Note that in this case, all the LCA events which are linearized after local events are already taken as part of the states $l', a', b'$. Now, suppose we add one more LCA event $e_i^{\top}$. If $L_1^b(e_i^{\top}) \cup L_2^b(e_i^{\top}) = \emptyset$ (which is possible since these local events might already have been linearized before some other $e_j^{\top}$). $\psi^{L_\top^a}_\F{ind-0op}$ handles this case, by establishing in the postcondition that adding one more LCA event to all states. The pre-condition requires the existence of some other event which needs to be in $\F{rc}$-order to $e_i^{\top}$. If $L_1^b(e_i^{\top}) \cup L_2^b(e_i^{\top}) \neq \emptyset$, we can first use $\psi^{L_\top^a}_\F{ind-0op}$, and then the other $\psi_\F{0op}$ VCs to build this set.\\

\noindent\textrm{\sc{Case 2}}: $(\mid L_1^{a} \cup L_2^{a} \mid > 0)$\\
We have 2 cases here: \\(2.1) Either of $L_1^{a}$ or $L_2^{a}$ is $\phi$ \\
(2.2) Both $L_1^{a}$ and $L_2^{a}$ are not $\phi$.\\

\noindent\textrm{\sc{Case 2.1}}:$(L_1^a \neq \phi \wedge L_2^a = \phi)$\\
Consider $e_1 \in L_1^a$ such that there does not exist $e \in L_1^a$ and $e_1 \xrightarrow{\F{lo_m}} e$, i.e. $e_1$ is the maximal event according to $\F{lo_m}$. Since $\F{lo}$ ordering between events remains the same in all versions, and since versions $v_1$ and $v_2$ (which are being merged) were already linearizable, there would exist sequences leading to the states $a$ such that $e_1$ would appear at the end. Hence, there exists $a'$ such that $a = e_1(a')$. Since $L_2^a$ is empty, all local events in $L_2$ are linearized before the LCA events.
We need to show that
\begin{equation}\label{eq:1op}
	\F{merge}(l, e_1(a'), b) = e_1(\F{merge}(l,a',b))
\end{equation}

We prove that $e_1$ is not linearized before any of the events in $L_1 \textbackslash \{e_1\} \cup L_2^b$.
$e_1$ is not linearized before any event in $L_1 \textbackslash \{e_1\}$ according to the induction assumption.
Let's assume there is some event $e$ in $L_2^b \textbackslash \{e_2\}$ that comes $\F{lo}$ after $e_1$. There are 2 possibilities.
\begin{itemize}
	\item $e_1 \xrightarrow{\F{vis}} e:$ This is not possible as events in $L_2^b \{e_2\}$ are concurrent with $e_1$. 
		This is because every version is causally closed, which means that $e_1 \in L_2^b \textbackslash \{e_2\}$,
		which would make $e_1$ part of the LCA which is not true according to the definition of $L_1^a$.
	\item $e_1 \xrightarrow{\F{rc}} e:$ This event $e$ must be in $\F{lo}$ order before some
		some LCA event say $e_m^i$. There are 2 possibilities for this:
		\begin{itemize}
			\item $e \xrightarrow{\F{vis}} e_m^i:$ Not possible as every version is causally closed.
			\item $e \xrightarrow{\F{rc}} e_m^i:$ Not possible due to $\textrm{\sc{no-rc-chain}}$ restriction.
		\end{itemize}
\end{itemize}
\VS{We also show that if $\F{lo}$ order between 2 events that led to the state in LHS should remain 
the same in the linearization of events that led to this state RHS is pending.}\\
We do induction on the size of $L_1^a \textbackslash \{e_1\}$  to prove Eqn.~\ref{eq:1op}.\\

\noindent\textrm{\sc{Base Case 2.1}}:$(L_1^a  \textbackslash \{e_1\} = \phi)$\\
We now induct on the event sequence for the LCA, but we also need to consider local events which have been 
linearized before LCA events. For this purpose, we consider two cases depending on whether $L_1^b \cup L_2^b$ is empty or not.\\

\noindent\textrm{\sc{Case 2.1.1}}:$(L_1^b \cup L_2^b = \phi)$\\
Then $L_\top^a = \emptyset$, $L_\top^b = L_\top$. In this case, we need to show that 
\begin{equation*}
	\F{merge}(l, e_1(l), l) = e_1(\F{merge}(l,l,l))
\end{equation*}
For showing this, we induct on the event sequence leading to the LCA state $l$, 
and the inductive and base case for this are described by the VCs $\psi^{L_\top^b}_\F{ind-1op}$ and $\psi^{L_\top^b}_\F{base-1op}$. \\

\noindent\textrm{\sc{Case 2.1.2}}:$(L_1^b \cup L_2^b \neq \phi)$\\
We induct on the size of the set $L_\top^a$. \\

\noindent\textrm{\sc{Base Case 2.1.2}}:$(\mid L_\top^a \mid = 1)$\\
Suppose $L_\top^a = \{e_\top\}$.  Let $l = \pi_{\mid L_\top^b}(\sigma_0)$. 
Now, in this scenario, $a' = \pi_{\mid L_\top^b \cup L_1^b}(\sigma_0)$, 
$b = \pi_{\mid L_\top^b \cup L_2^b}(\sigma_0)$. 
Then, our goal is to show the following:
\begin{equation}\label{eq:1op1}
	\F{merge}(e_\top(l), e_1(e_\top(a')), e_\top(b)) = e_1(\F{merge}(e_\top(l),e_\top(a'),e_\top(b)))
\end{equation}
That is, the LCA corresponds to all events in $L_\top^b \cup \{e_\top\}$, while for either branch to be merged, 
we have all local events of each branch which are linearized before the LCA event $e_\top$. 
Here, we induct on the set $L_1^b \cup L_2^b$.\\

\noindent\textrm{\sc{Base Case 2.1.2.1}}:$(\mid L_1^b \cup L_2^b \mid = 1)$\\
For the base case, we consider single events in $L_1^b$ and $L_2^b$.
\begin{itemize}
	\item $\mid L_1^b \mid = 1$ and  $\mid L_2^b \mid = 0$ case is handed by $\psi^{L_1^b}_\F{ind1-1op}$ 
	\item $\mid L_1^b \mid = 0$ and $\mid L_2^b \mid = 1$ case is handed by $\psi^{L_2^b}_\F{ind1-1op}$ 
\end{itemize}
Focusing on the post-condition of $\psi^{L_1^b}_\F{ind1-1op}$ , it adds a single event which is supposed to be linearized before $e_\top$ on the left branch. Similarly, $\psi^{L_2^b}_\F{ind1-1op}$ , it adds a single event $e_b$ which is supposed to be linearized before $e_\top$ on the right branch.\\

\noindent\textrm{\sc{Inductive Case 2.1.2.1}}:$(\mid L_1^b \cup L_2^b \mid > 1)$\\
$\psi^{L_1^b}_\F{ind2-1op}$ and $\psi^{L_2^b}_\F{ind2-1op}$ are the VCs for the inductive case on $\mid L_1^b \cup L_2^b \mid$. Focusing on $\psi^{L_2^b}_\F{ind2-1op}$, in the pre-condition it assumes the presence of an event $e_b$ which is linearized before $e_\top$ on the right branch, and in the post-condition, it adds a new event $e$ before $e_b$. As per the definition of $L_2^b$, the pre-condition also asserts that the new event $e$ should either not commute with $e_b$ (which takes care of events of the form $e \xrightarrow{\F{lo_m}} e_b \xrightarrow{\F{lo_m}} e_\top)$, or it should be in $\F{rc}$-order before $e_\top$. $\psi^{L_1^b}_\F{ind2-1op}$ works similarly for the left branch.\\

\noindent\textrm{\sc{Indcutive Case 2.1.2}}:$(\mid L_\top^a \mid > 1)$\\
Assume that for set $S=\{e_1^{\top}, \ldots, e_{i-1}^{\top}\}$, we have the required result. We re-define $l, a', b$ based on the above set $S$: $l=\pi_{\mid L_\top^b \cup S}$, $a'= \pi_{\mid L_\top^b \cup \bigcup_{e \in S} M_1^b(e) \cup S}(\sigma_0)$, $b = \pi_{\mid L_\top^b \cup \bigcup_{e \in S} M_2^b(e) \cup S}(\sigma_0)$. By IH, we have the Eqn.~\ref{eq:1op}. Note that in this case, all the LCA events which are linearized after local events are already taken as part of the states $a', b$. Now, suppose we add one more LCA event $e_i^{\top}$. If $L_1^b(e_i^{\top}) \cup L_2^b(e_i^{\top}) = \emptyset$ (which is possible since these local events might already have been linearized before some other $e_j^{\top}$). $\psi^{L_\top^a}_\F{ind-1op}$ handles this case, by establishing in the postcondition that adding one more LCA event to all states. The pre-condition requires the existence of some other event which needs to be in $\F{rc}$-order to $e_i^{\top}$. If $M_1^b(e_i^{\top}) \cup M_2^b(e_i^{\top}) \neq \emptyset$, we can first use $\psi^{L_\top^a}_\F{ind-1op}$ and then the other $\psi_\F{1op}$ VCs to build this set.\\

\noindent\textrm{\sc{Inductive Case 2.2}}:$(L_1^a  \textbackslash \{e_1\} \neq \phi)$\\
$\psi_\F{ind-1op}^{L_1^a}$ and $\psi_\F{ind-1op}^{L_2^a}$ are the VCs used for this purpose. Focusing $\psi_\F{ind-1op}^{L_1^a}$, in the pre-condition it assumes Eqn.~\eqref{eq:1op}, and in the post-condition, it adds a new event $e_1^{'}$ before $e_1$. $\psi_\F{ind-1op}^{L_2^a}$ works similar to the left branch.\\

The other case where $L_1^a = \phi \wedge L_2^a \neq \phi$ is handled by $\textrm{\sc{MergeCommutativity}}$.\\


\noindent\textrm{\sc{Case 2.2}}:$(L_1^a \neq \phi \wedge L_2^a \neq \phi)$\\
Consider $e_1 \in L_1^a, e_2 \in L_2^a$ such that there does not exist $e \in L_i^a$ and $e_i \xrightarrow{\F{lo_m}} e$ (for $i=1,2$), i.e. each of the $e_i$s are maximal events according to $\F{lo_m}$. Since $\F{lo}$ ordering between events remains the same in all versions, and since versions $v_1$ and $v_2$ (which are being merged) were already linearizable, there would exist sequences leading to the states $a$ and $b$ such that $e_1$ and $e_2$ would appear at the end resp. Hence, there exists $a'$ and $b'$ such that $a = e_1(a')$ and $b = e_2(b')$. Since $e_1 \mid\mid_C e_2$, they are related to each other by $\F{rc}$ relation or they commute with each other i.e., $e_1 \xrightarrow{\F{rc}} e_2 \vee e_2 \xrightarrow{\F{rc}} e_1 \vee e_1 \rightleftarrows e_2$.
We will consider the case when $e_2 \xrightarrow{\F{rc}} e_1 \vee e_1 \rightleftarrows e_2$. $e_1 \xrightarrow{\F{rc}} e_2$ is handled by $\textrm{\sc{MergeCommutativity}}$.
The equation becomes 
\begin{equation}\label{eq:2op}
\F{merge}(l, e_1(a'), e_2(b')) = e_1(\F{merge}(l, a', e_2(b')))
\end{equation}

We prove that $e_1$ is not linearized before any of the events in $L_1 \textbackslash \{e_1\} \cup L_2$.
$e_1$ is not linearized before any event in $L_1 \textbackslash \{e_1\}$ according to the induction assumption.
Since $e_2 \xrightarrow{\F{rc}} e_1, e_1 \xrightarrow{\F{vis}} e_2$ is not possible.
$e_1 \xrightarrow{\F{rc}} e_2$ is not possible as $\F{rc}^+$ is irreflexive.
So $e_1 \xrightarrow{\F{lo}} e_2$ is not possible.
Let's assume there is some event $e$ in $L_2^b \cup L_2^a \textbackslash \{e_2\}$ that comes $\F{lo}$ after $e_1$. There are 2 possibilities.
\begin{itemize}
	\item $e_1 \xrightarrow{\F{rc}} e:$ Since $e_2 \xrightarrow{\F{rc}} e_1$, this case is not possible due to $\textrm{\sc{no-rc-chain}}$ restriction.
	\item $e_1 \xrightarrow{\F{vis}} e:$ This is not possible as events in $L_2^b \cup L_2^a \textbackslash \{e_2\}$ are concurrent with $e_1$. 
		This is because every version is causally closed, which means that $e_1 \in L_2^b \cup L_2^a \textbackslash \{e_2\}$,
		which would make $e_1$ part of the LCA which is not true according to the definition of $L_1^a$.
\end{itemize}
\VS{We also show that if $\F{lo}$ order between 2 events that led to the state in LHS should remain 
the same in the linearization of events that led to this state RHS is pending.}\\
We do induction on the size of $L_1^a \textbackslash \{e_1\} \cup L_2^a \textbackslash \{e_2\}$ to prove Eqn.~\ref{eq:2op}.\\

\noindent\textrm{\sc{Base Case 2.2}}:$(L_1^a  \textbackslash \{e_1\} \cup L_2^a \textbackslash \{e_2\} = \phi)$\\
We now induct on the event sequence for the LCA, but we also need to consider local events which have been 
linearized before LCA events. For this purpose, we consider two cases depending on whether $L_1^b \cup L_2^b$ is empty or not.\\

\noindent\textrm{\sc{Case 2.2.1}}:$(L_1^b \cup L_2^b = \phi)$\\
Then $L_\top^a = \emptyset$, $L_\top^b = L_\top$. In this case, we need to show that 
\begin{equation*}
	\F{merge}(l, e_1(l), e_2(l)) = e_1(\F{merge}(l, l,e_2(l))
\end{equation*}
For showing this, we induct on the event sequence leading to the LCA state $l$, 
and the inductive and base case for this are described by the VCs $\psi^{L_\top^b}_\F{ind-2op}$ and $\psi^{L_\top^b}_\F{base-2op}$. \\

\noindent\textrm{\sc{Case 2.2.2}}:$(L_1^b \cup L_2^b \neq \phi)$\\
We induct on the size of the set $L_\top^a$. \\

\noindent\textrm{\sc{Base Case 2.2.2}}:$(\mid L_\top^a \mid = 1)$\\
$L_\top^a = \{e_\top\}$.  We induct on the size of $L_1^b \cup L_2^b$.\\

\noindent\textrm{\sc{Base Case 2.2.2.1}}:$(\mid L_1^b \cup L_2^b \mid = 1)$\\
For the base case, we consider single events in $L_1^b$ and $L_2^b$.
\begin{itemize}
	\item $\mid L_1^b \mid = 1$ and  $\mid L_2^b \mid = 0$ case is handed by $\psi^{L_1^b}_\F{ind1-2op}$ 
	\item $\mid L_1^b \mid = 0$ and $\mid L_2^b \mid = 1$ case is handed by $\psi^{L_2^b}_\F{ind1-2op}$ 
\end{itemize}
Focusing on the post-condition of $\psi^{L_1^b}_\F{ind1-2op}$ , it adds a single event which is supposed to be linearized before $e_\top$ on the left branch. Similarly, $\psi^{L_2^b}_\F{ind1-2op}$ , it adds a single event $e_b$ which is supposed to be linearized before $e_\top$ on the right branch.\\

\noindent\textrm{\sc{Inductive Case 2.2.2.1}}:$(\mid L_1^b \cup L_2^b \mid > 1)$\\
$\psi^{L_1^b}_\F{ind2-2op}$ and $\psi^{L_2^b}_\F{ind2-2op}$ are the VCs for the inductive case on $\mid L_1^b \cup L_2^b \mid$. Focusing on $\psi^{L_2^b}_\F{ind2-2op}$, in the pre-condition it assumes the presence of an event $e_b$ which is linearized before $e_\top$ on the right branch, and in the post-condition, it adds a new event $e$ before $e_b$. As per the definition of $L_2^b$, the pre-condition also asserts that the new event $e$ should either not commute with $e_b$ (which takes care of events of the form $e \xrightarrow{\F{lo_m}} e_b \xrightarrow{\F{lo_m}} e_\top)$, or it should be in $\F{rc}$-order before $e_\top$. $\psi^{L_1^b}_\F{ind2-2op}$ works similarly for the left branch.\\

\noindent\textrm{\sc{Inductive Case 2.2.2}}:$(\mid L_\top^a \mid > 1)$\\
Assume that for set $S=\{e_1^{\top}, \ldots, e_{i-1}^{\top}\}$, we have the required result. We re-define $l, a', b'$ based on the above set $S$: $l =\pi_{\mid L_\top^b \cup S}$, $a' = \pi_{\mid L_\top^b \cup \bigcup_{e \in S} L_1^b(e) \cup S}(\sigma_0)$, $b' = \pi_{\mid L_\top^b \cup \bigcup_{e \in S} M_2^b(e) \cup S}(\sigma_0)$. By IH, we have the Eqn.~\ref{eq:2op}. Note that in this case, all the LCA events which are linearized after local events are already taken as part of the states $a', b'$. Now, suppose we add one more LCA event $e_i^{\top}$. If $M_1^b(e_i^{\top}) \cup M_2^b(e_i^{\top}) = \emptyset$ (which is possible since these local events might already have been linearized before some other $e_j^{\top}$). $\psi^{L_\top^a}_\F{ind-2op}$ handles this case, by establishing in the postcondition that adding one more LCA event to all states. The pre-condition requires the existence of some other event which needs to be in $\F{rc}$-order to $e_i^{\top}$. If $M_1^b(e_i^{\top}) \cup M_2^b(e_i^{\top}) \neq \emptyset$, we can first use $\psi^{L_\top^a}_\F{ind-2op}$ and then the other $\psi_\F{2op}$ VCs to build this set.\\

\noindent\textrm{\sc{Inductive Case 2.2}}:$(L_1^a  \textbackslash \{e_1\} \cup L_2^a \textbackslash \{e_2\} \neq \phi)$\\
$\psi_\F{ind-2op}^{L_1^a}$ and $\psi_\F{ind-2op}^{L_2^a}$ are the VCs used for this purpose. Focusing $\psi_\F{ind-2op}^{L_1^a}$, in the pre-condition it assumes Eqn.~\eqref{eq:2op}, and in the post-condition, it adds a new event $e_1^{'}$ before $e_1$. $\psi_\F{ind-2op}^{L_2^a}$ works similar to the left branch.\\




\end{comment}




\begin{comment}
\noindent\textrm{\sc{Case 2}}:$(\mid L_1^{a} \cup L_2^{a} \mid > 0)$\\
We have 2 cases here: \\(2.1) Either of $L_1^{a}$ or $L_2^{a}$ is $\phi$ \\(2.2) Both $L_1^{a}$ and $L_2^{a}$ are not $\phi$.\\

\noindent\textrm{\sc{Case 2.1}}:$(L_1^a \neq \phi \wedge L_2^a = \phi)$\\
Consider $e_1 \in L_1^a$ such that there does not exist $e \in L_1^a$ and $e_1 \xrightarrow{\F{lo_m}} e$, i.e. $e_1$ is the maximal event according to $\F{lo_m}$. Since $\F{lo}$ ordering between events remains the same in all versions, and since versions $v_1$ and $v_2$ (which are being merged) were already linearizable, there would exist sequences leading to the states $a$ such that $e_1$ would appear at the end. Hence, there exists $a'$ such that $a = e_1(a')$. Since $L_2^a$ is empty, all local events in $L_2$ are linearized before the LCA events. In this case, the last event which leads to the state $b$ and $l$ must be an LCA event. Let's consider $e_\top$ to be the maximal event in $L_\top^a$ according to $\F{lo_m}$. Hence there exists states $l', b'$ such that $l = e_\top(l'), b = e_\top(b')$. Note that $e_\top$ can be $\epsilon$. If that is the case, then $l' = b'$.
We need to show that
\begin{equation}\label{eq:vc100}
\F{merge}(e_\top(l'), e_1(a'), e_\top(b')) = e_1(\F{merge}(e_\top(l'), a', e_\top(b')))
\end{equation}
where $l = {\pi_m}_{\mid  (L_\top)}(\sigma_0), \\ a = {\pi_m}_{\mid (L_\top^{b} \cup L_1^{b} \cup L_\top^{a} \cup L_1^a)}(\sigma_0), \\b = {\pi_m}_{\mid (L_\top^{b} \cup L_2^{b} \cup L_\top^{a})}(\sigma_0),\\
\F{merge}(e_\top(l'), a', e_\top(b')) = {\pi_m}_{\mid (L_\top^{b} \cup L_1^{b} \cup L_2^b \cup L_1^a \textbackslash \{e_1\})}(\sigma_0)$.\\
We do induction on the size of $L_1^a \textbackslash \{e_1\}$ to prove Eqn.~\eqref{eq:vc100}.\\

\noindent\textrm{\sc{Case 2.1.1}}:$(L_1^a \textbackslash \{e_1\} = \phi)$\\
We now induct on the event sequence for the LCA, but we also need to consider local events which have been linearized before LCA events. For this purpose, we consider two cases depending on whether $L_1^b \cup L_2^b$ is empty or not.\\

\noindent\textrm{\sc{Case 2.1.1.1}}:$(L_1^b \cup L_2^b = \phi)$\\
If $L_1^b \cup L_2^b = \emptyset$ then $L_\top^a = \emptyset$ and $L_\top^b = L_\top$. In this case, we need to show that $\F{merge}(e_\top(l), e_1(l), e_\top(l)) = e_1(\F{merge}(e_\top(l), l, e_\top(l)))$. Note that since LCA and the third argument to merge is the same, $e_\top$ can be $\epsilon$. Hence it is enough if we prove, $\F{merge}(l, e_1(l), l) = e_1(\F{merge}(l, l, l))$. For showing this, we induct on the event sequence leading to the LCA state $\sigma_\top$, and the inductive and base case for this are described by the VCs $\psi^{L_\top^b}_\F{base-1op}$ and $\psi^{L_\top^b}_\F{ind-1op}$.\\

\noindent\textrm{\sc{Case 2.1.1.2}}:$(L_1^b \cup L_2^b \neq \phi)$\\
If $L_1^b \cup L_2^b \neq \emptyset$, we need to consider local events in each branch which were linearized before the LCA events. Suppose $a'$ is the state of the left branch obtained by excluding the events $e_1$. Hence, these states would contain the effect of all events in $L_1^b$. Then, our goal is to show that $\F{merge}(e_\top(l), e_1(a'), e_\top(b)) = e_1(\F{merge}(e_\top(l), a', e_\top(b)))$.
For this purpose, we now induct on the size of the set $L_\top^a$. \\

\noindent\textrm{\sc{Case 2.1.1.2.1}}:$(\mid L_\top^a \mid = 1)$\\
$L_\top^a = \{e_\top\}$. Let $l = \pi_{\mid L_\top^b}(\sigma_0)$. Now, in this scenario, $a'' = \pi_{\mid L_\top^b \cup L_1^b}(\sigma_0)$, $b = \pi_{\mid L_\top^b \cup L_2^b}(\sigma_0)$. Then, our goal is to show the following:
\begin{align}\label{eq:vc101}
	\F{merge}(e_\top(l), e_1(e_\top(a'')), e_2(e_\top(b))) = e_1(\F{merge}(e_\top(l), e_\top(a''), e_2(e_\top(b))) )
\end{align}
That is, the LCA corresponds to all events in $L_\top^b \cup \{e_\top\}$, while for either branch to be merged, we have all local events of each branch which are linearized before the LCA event $e_\top$. To prove Eqn.~\ref{eq:vc101}, we induct on the set $L_1^b \cup L_2^b$.\\

\noindent\textrm{\sc{Case 2.1.1.2.1.1}}:$(\mid L_1^b \cup L_2^b \mid = 1)$\\
For the base case, we consider single events in $L_1^b$ and $L_2^b$.
\begin{itemize}
	\item $\mid L_1^b \mid = 1$ and  $\mid L_2^b \mid = 0$ case is handed by $\psi^{L_1^b}_\F{ind1-1op}$
	\item $\mid L_1^b \mid = 0$ and $\mid L_2^b \mid = 1$ case is handed by $\psi^{L_2^b}_\F{ind1-1op}$
\end{itemize}
Focusing on the post-condition of $\psi^{L_1^b}_\F{ind1-1op}$, it adds a single event which is supposed to be linearized before $e_\top$ on the left branch. Similarly, $\psi^{L_2^b}_\F{ind1-1op}$, it adds a single event $e_b$ which is supposed to be linearized before $e_\top$ on the right branch.\\

\noindent\textrm{\sc{Case 2.1.1.2.1.2}}:$(\mid L_1^b \cup L_2^b \mid > 1)$\\
$\psi^{L_1^b}_\F{ind2-1op}$ and $\psi^{L_2^b}_\F{ind2-1op}$are the VCs for the inductive case on $\mid L_1^b \cup L_2^b \mid$. Focusing on $\psi^{L_2^b}_\F{ind2-1op}$, in the pre-condition it assumes the presence of an event $e_b$ which is linearized before $e_\top$ on the right branch, and in the post-condition, it adds a new event $e$ before $e_b$. As per the definition of $L_2^b$, the pre-condition also asserts that the new event $e$ should either not commute with $e_b$ (which takes care of events of the form $e \xrightarrow{\F{lo_m}} e_b \xrightarrow{\F{lo_m}} e_\top$), or it should be in $\F{rc}$-order before $e_\top$. $\psi^{L_1^b}_\F{ind2-1op}$ works similarly for the left branch.\\

\noindent\textrm{\sc{Case 2.1.1.2.2}}:$(\mid L_\top^a \mid > 1)$\\
Assume that for set $S=\{e_1^{\top}, \ldots, e_{i-1}^{\top}\}$, we have the required result. We re-define $l', a', b'$ based on the above set $S$: $l' =\pi_{\mid L_\top^b \cup S}$, $a' = \pi_{\mid L_\top^b \cup \bigcup_{e \in S} L_1^b(e) \cup S}(\sigma_0)$, $b' = \pi_{\mid L_\top^b \cup \bigcup_{e \in S} L_2^b(e) \cup S}(\sigma_0)$. By IH, $\F{merge}(l',a',b') = \pi_{\mid L_\top^b \cup \bigcup_{e \in S} L_1^b(e) \cup \bigcup_{e \in S} L_1^b(e) \cup S}(\sigma_0)$. Note that in this case, all the LCA events which are linearized after local events are already taken as part of the states $l',a',b'$. Now, suppose we add one more LCA event $e_i^{\top}$. If $L_1^b(e_i^{\top}) \cup L_2^b(e_i^{\top}) = \emptyset$ (which is possible since these local events might already have been linearized before some other $e_j^{\top}$). $\psi^{L_\top^a}_\F{ind-1op}$ handles this case, by establishing in the postcondition that adding one more LCA event to all states. The pre-condition requires the existence of some other event which needs to be in $\F{rc}$-order to $e_i^{\top}$. If $L_1^b(e_i^{\top}) \cup L_2^b(e_i^{\top}) \neq \emptyset$, we can first use $\psi^{L_\top^a}_\F{ind-1op}$, and then the other $\psi_\F{1op}$ VCs to build this set.\\

\noindent\textrm{\sc{Case 2.1.2}}:$(L_1^a \textbackslash \{e_1\} \neq \phi)$\\
$\psi_\F{ind-1op}^{L_1^a}$ is the used for this purpose. In the pre-condition it assumes Eqn.~\eqref{eq:vc100}, and in the post-condition, it adds a new event $e_1^{'}$ before $e_1$. The case where $(L_1^a = \phi \wedge L_2^a \neq \phi)$ is handled by $\textrm{\sc{MergeCommutativity}}$.\\

\end{comment}
\end{proof}



\begin{comment}
\noindent \textbf{Proof of Eqn~\ref{eq:vct1}}
\begin{proof}
The pre-condition is that $(L_1^a \neq \phi \wedge L_2^a = \phi)$.
We do induction on the size of $L_1^a \textbackslash \{e_1\}$. \\

\noindent\textrm{\sc{Case 1}}:$(L_1^a \textbackslash \{e_1\} = \phi)$\\
We now induct on the event sequence for the LCA, but we also need to consider local events which have been linearized before LCA events. For this purpose, we consider two cases depending on whether $L_1^b \cup L_2^b$ is empty or not.\\

\noindent\textrm{\sc{Case 1.1}}:$(L_1^b \cup L_2^b = \phi)$\\
If $L_1^b \cup L_2^b = \emptyset$, then $L_\top^a = \emptyset$ and $L_\top^b = L_\top$. In this case, we need to show that 
\begin{equation}\label{eq:vc100}
\F{merge}(l, e_1(l), l) = e_1(\F{merge}(l, l, l))
\end{equation}
For showing Eqn. \eqref{eq:vc100}, we induct on the event sequence leading to the LCA state $l$, and the inductive and base case for this are described by the VCs $\psi^{L_\top^b}_\F{ind-1op}$ and $\psi^{L_\top^b}_\F{base-1op}$. \\

\noindent\textrm{\sc{Case 1.2}}:$(L_1^b \cup L_2^b \neq \phi)$\\
If $L_1^b \cup L_2^b \neq \emptyset$, we need to consider local events in each branch which were linearized before the LCA events. Suppose $a'$ is the state of the left branch obtained by excluding the events $e_1$. Hence, this states would contain the effect of all events in $L_1^b$. Let  $e_\top$ be the maximal event in $L_\top^a{}$ according to $\F{lo_m}$. Then, our goal is to show that 
\begin{equation}\label{eq:vc101}
\F{merge}(e_\top(l), e_1(e_\top(a')), e_\top(b')) = e_1(\F{merge}(e_\top(l), e_\top(a'), e_\top(b')))
\end{equation}
For this purpose, we induct on the size of the set $L_\top^a$.\\

\noindent\textrm{\sc{Case 1.2.1}}:$(\mid L_\top^a \mid = 1)$\\
Then $L_\top^ a = \{e_\top\}$. Here, we induct on the size of the set $L_1^b \cup L_2^b$.\\ 

\noindent\textrm{\sc{Case 1.2.1.1}}:$(\mid L_1^b \cup L_2^b \mid = 1)$\\
For the base case, we consider single events in $L_1^b$ and $L_2^b$.
\begin{itemize}
	\item $\mid L_1^b \mid = 1$ and  $\mid L_2^b \mid = 0$ case is handed by $\psi^{L_1^b}_\F{ind1-1op}$
	\item $\mid L_1^b \mid = 0$ and $\mid L_2^b \mid = 1$ case is handed by $\psi^{L_2^b}_\F{ind1-1op}$
\end{itemize}
Focusing on the post-condition of $\psi^{L_1^b}_\F{ind1-1op}$, it adds a single event which is supposed to be linearized before $e_\top$ on the left branch. Similarly, $\psi^{L_2^b}_\F{ind1-1op}$, it adds a single event $e_b$ which is supposed to be linearized before $e_\top$ on the right branch.\\

\noindent\textrm{\sc{Case 1.2.1.2}}:$(\mid L_1^b \cup L_2^b \mid > 1)$\\
$\psi^{L_1^b}_\F{ind2-1op}$ and $\psi^{L_2^b}_\F{ind2-1op}$ are the VCs for the inductive case on $\mid L_1^b \cup L_2^b \mid$. Focusing on $\psi^{L_2^b}_\F{ind2-1op}$, in the pre-condition it assumes the presence of an event $e_b$ which is linearized before $e_\top$ on the right branch, and in the post-condition, it adds a new event $e$ before $e_b$. As per the definition of $L_2^b$, the pre-condition also asserts that the new event $e$ should either not commute with $e_b$ (which takes care of events of the form $e \xrightarrow{\F{lo_m}} e_b \xrightarrow{\F{lo_m}} e_\top$), or it should be in $\F{rc}$-order before $e_\top$. $\psi^{L_1^b}_\F{ind2-1op}$ works similarly for the left branch.\\

\noindent\textrm{\sc{Case 1.2.2}}:$(\mid L_\top^a \mid > 1)$\\
By IH, we have the following equation $\F{merge}(l, e_1(a), b) = e_1(\F{merge}(l,a,b))$. Note that in this case, all the LCA events which are linearized after local events are already taken as part of the states $l,a,b$. Now, we add one more LCA event $e_\top$ to all three arguments of merge from pre-condition to post-condition using the VC $\psi^{L_\top^a}_\F{ind}$. Notice that we also have another pre-condition which requires the existence of some event $e$ which should come $\F{rc}$-before $e_\top$, which is necessary for $e_\top$ to be in $L_\top^a$.\\

\noindent\textrm{\sc{Case 2}}:$(L_1^a \textbackslash \{e_1\} \neq \phi)$\\
$\psi^{L_1^a}_\F{ind-1op}$ and $\psi^{L_2^a}_\F{ind-1op}$ are the VCs used for this purpose. Focusing $\psi^{L_1^a}_\F{ind-1op}$, in the pre-condition it assumes Eqn.~\eqref{eq:vct1}, and in the post-condition, it adds a new event $e_1^{'}$ before $e_1$. $\psi^{L_2^a}_\F{ind-1op}$ works similar to the right branch.
\end{proof}


\noindent \textbf{Proof of Eqn~\ref{eq:vct2}}
\begin{proof}
The pre-condition is that $(L_1^a \neq \phi \wedge L_2^a \neq \phi)$.
We do induction on the size of $L_1^a \textbackslash \{e_1\} \cup L_2^a \textbackslash \{e_2\}$ to prove Eqn.~\eqref{eq:vct2}. \\

\noindent\textrm{\sc{Case 1}}:$(L_1^a \textbackslash \{e_1\} \cup L_2^a \textbackslash \{e_2\} = \phi)$\\
We now induct on the event sequence for the LCA, but we also need to consider local events which have been linearized before LCA events. For this purpose, we consider two cases depending on whether $L_1^b \cup L_2^b$ is empty or not.\\

\noindent\textrm{\sc{Case 1.1}}:$(L_1^b \cup L_2^b = \phi)$\\
If $L_1^b \cup L_2^b = \emptyset$, then $L_\top^a = \emptyset$ and $L_\top^b = L_\top$. In this case, we need to show that 
\begin{equation}\label{eq:vc200}
\F{merge}(l, e_1(l), e_2(l)) = e_1(\F{merge}(l, l, e_2(l)))
\end{equation}
For showing Eqn. \eqref{eq:vc200}, we induct on the event sequence leading to the LCA state $l$, and the inductive and base case for this are described by the VCs $\psi^{L_\top^b}_\F{ind-2op}$ and $\psi^{L_\top^b}_\F{base-2op}$. \\

\noindent\textrm{\sc{Case 1.2}}:$(L_1^b \cup L_2^b \neq \phi)$\\
If $L_1^b \cup L_2^b \neq \emptyset$, we need to consider local events in each branch which were linearized before the LCA events. Suppose $a'$ and $b'$ are the states of the left branch and right branch obtained by excluding the events $e_1$ and $e_2$ respectively. Hence, these states would contain the effect of all events in $L_1^b$ and $L_2^b$. Let  $e_\top$ be the maximal event in $L_\top^a{}$ according to $\F{lo_m}$. Then, our goal is to show that 
\begin{equation}\label{eq:vc201}
\F{merge}(e_\top(l), e_1(e_\top(a')), e_2(e_\top(b'))) = e_1(\F{merge}(e_\top(l), e_\top(a'), e_2(e_\top(b'))))
\end{equation}
For this purpose, we induct on the size of the set $L_\top^a$.\\

\noindent\textrm{\sc{Case 1.2.1}}:$(\mid L_\top^a \mid = 1)$\\
Then $L_\top^ a = \{e_\top\}$. Here, we induct on the set $L_1^b \cup L_2^b$.\\ 

\noindent\textrm{\sc{Case 1.2.1.1}}:$(\mid L_1^b \cup L_2^b \mid = 1)$\\
For the base case, we consider single events in $L_1^b$ and $L_2^b$.
\begin{itemize}
	\item $\mid L_1^b \mid = 1$ and  $\mid L_2^b \mid = 0$ case is handed by $\psi^{L_1^b}_\F{ind1-2op}$
	\item $\mid L_1^b \mid = 0$ and $\mid L_2^b \mid = 1$ case is handed by $\psi^{L_2^b}_\F{ind1-2op}$
\end{itemize}
Focusing on the post-condition of $\psi^{L_1^b}_\F{ind1-2op}$, it adds a single event which is supposed to be linearized before $e_\top$ on the left branch. Similarly, $\psi^{L_2^b}_\F{ind1-2op}$, it adds a single event $e_b$ which is supposed to be linearized before $e_\top$ on the right branch.\\

\noindent\textrm{\sc{Case 1.2.1.2}}:$(\mid L_1^b \cup L_2^b \mid > 1)$\\
$\psi^{L_1^b}_\F{ind2-2op}$ and $\psi^{L_2^b}_\F{ind2-2op}$ are the VCs for the inductive case on $\mid L_1^b \cup L_2^b \mid$. Focusing on $\psi^{L_2^b}_\F{ind2-2op}$, in the pre-condition it assumes the presence of an event $e_b$ which is linearized before $e_\top$ on the right branch, and in the post-condition, it adds a new event $e$ before $e_b$. As per the definition of $L_2^b$, the pre-condition also asserts that the new event $e$ should either not commute with $e_b$ (which takes care of events of the form $e \xrightarrow{\F{lo_m}} e_b \xrightarrow{\F{lo_m}} e_\top$), or it should be in $\F{rc}$-order before $e_\top$. $\psi^{L_1^b}_\F{ind2-2op}$ works similarly for the left branch.\\

\noindent\textrm{\sc{Case 1.2.2}}:$(\mid L_\top^a \mid > 1)$\\
By IH, we have the following equation $\F{merge}(l, e_1(a), e_2(b)) = e_1(\F{merge}(l,a,e_2(b)))$. Note that in this case, all the LCA events which are linearized after local events are already taken as part of the states $l,a,b$. Now, we add one more LCA event $e_\top$ to all three arguments of merge from pre-condition to post-condition using the VC $\psi^{L_\top^a}_\F{ind}$. Notice that we also have another pre-condition which requires the existence of some event $e$ which should come $\F{rc}$-before $e_\top$, which is necessary for $e_\top$ to be in $L_\top^a$.\\


\noindent\textrm{\sc{Case 2}}:$(L_1^a \textbackslash \{e_1\} \cup L_2^a \textbackslash \{e_2\} \neq \phi)$\\
$\psi^{L_1^a}_\F{ind-2op}$ and $\psi^{L_2^a}_\F{ind-2op}$ are the VCs used for this purpose. Focusing $\psi^{L_1^a}_\F{ind-2op}$, in the pre-condition it assumes Eqn.~\eqref{eq:vct2}, and in the post-condition, it adds a new event $e_1^{'}$ before $e_1$. $\psi^{L_2^a}_\F{ind-2op}$ works similar to the left branch.

\end{proof}
\end{comment}

\begin{table}[ht]
\scriptsize
\makebox[\textwidth][c]{%
\begin{tabular}{|>{\raggedright\arraybackslash}p{2.2cm}|>{\raggedright\arraybackslash}p{3cm}|>{\raggedright\arraybackslash}p{3.5cm}|>{\raggedright\arraybackslash}p{4cm}|}%{|c|c|c|c|}
		 
  		\hline
  		\textbf{VC Name} & \multicolumn{2}{c|}{\textbf{Pre-condition}} & \textbf{Post-condition} \\
		\hline

		$\textrm{\sc{MergeCommutativity}}$ & 
		$ $ & 
		$ $ &
		$\mu(l, a, b)= \mu(l, b, a)$\\
		\hline
		
		$\textrm{\sc{MergeIdempotence}}$ & 
		$ $ & 
		$ $ &
		$\mu(s, s, s)= s$\\
		\hline

		$\psi^{L_\top^b}_\F{base-2op}$ &
		$e_2 \xrightarrow{\F{rc}} e_1 \vee e_2 \rightleftarrows e_1$ & 
		$ $ &
		$\mu(\sigma_0, e_1(\sigma_0), e_2(\sigma_0)) = e_1(\mu(\sigma_0, \sigma_0, e_2(\sigma_0)))$\\
		\hline

		$\psi^{L_\top^b}_\F{ind-2op}$ &
		$e_2 \xrightarrow{\F{rc}} e_1 \vee e_2 \rightleftarrows e_1$ &
		$\mu(l, e_1(l), e_2(l)) = e_1(\mu(l, l, e_2(l)))$ & 
		$\mu(e_\top(l), e_1 \comp e_\top(l), e_2 \comp e_\top(l)) = e_1(\mu(e_\top(l), e_\top(l), e_2 \comp e_\top(l)))$\\
		\hline
		
		$\psi^{L_\top^a}_\F{ind-2op}$ &
		$(e_2 \xrightarrow{\F{rc}} e_1 \vee e_2 \rightleftarrows e_1) \wedge (\exists e. e \xrightarrow{\F{rc}} e_\top)$ &
		$\mu(l, e_1(a), e_2(b)) = e_1(\mu(l, a, e_2(b)))$ & 
		$\mu(e_\top(l), e_1 \comp e_\top(a), e_2 \comp e_\top(b)) = e_1(\mu(e_\top(l), e_\top(a), e_2 \comp e_\top(b)))$\\
		\hline

		$\psi^{L_1^b}_\F{ind1-2op}$ & 
		$(e_2 \xrightarrow{\F{rc}} e_1 \vee e_2 \rightleftarrows e_1) \wedge e_b \xrightarrow{\F{rc}} e_\top$ & 
		$\mu(e_\top(l), e_1 \comp e_\top(a), e_2 \comp e_\top(b)) = e_1(\mu(e_\top(l), e_\top(a), e_2 \comp e_\top(b)))$ & 
		$\mu(e_\top(l), e_1 \comp e_\top \comp e_b(a), e_2 \comp e_\top(b)) = e_1(\mu(e_\top(l), e_\top \comp e_b(a), e_2 \comp e_\top(b)))$\\
		\hline
		
		$\psi^{L_1^b}_\F{ind2-2op}$ & 
		$(e_2 \xrightarrow{\F{rc}} e_1 \vee e_2 \rightleftarrows e_1) \wedge e_b \xrightarrow{\F{rc}} e_\top \wedge (\neg e \rightleftarrows e_b \vee e \xrightarrow{\F{rc}} e_\top)$ &
		$\mu(e_\top(l), e_1 \comp e_\top \comp e_b(a), e_2 \comp e_\top(b)) = e_1(\mu(e_\top(l), e_\top \comp e_b(a), e_2 \comp e_\top(b)))$ & 
		$\mu(e_\top(l), e_1 \comp e_\top \comp e_b(e(a)), e_2 \comp e_\top(b)) = e_1(\mu(e_\top(l), e_\top \comp e_b \comp e(a), e_2 \comp e_\top(b)))$\\
		\hline

		$\psi^{L_2^b}_\F{ind1-2op}$ & 
		$(e_2 \xrightarrow{\F{rc}} e_1 \vee e_2 \rightleftarrows e_1) \wedge e_b \xrightarrow{\F{rc}} e_\top$ &
		$\mu(e_\top(l), e_1 \comp e_\top(a), e_2 \comp e_\top(b)) = e_1(\mu(e_\top(l), e_\top(a), e_2 \comp e_\top(b)))$ & 
		$\mu(e_\top(l), e_1 \comp e_\top(a), e_2 \comp e_\top \comp e_b(b)) = e_1(\mu(e_\top(l), e_\top(a), e_2 \comp e_\top \comp e_b(b)))$\\
		\hline

		$\psi^{L_2^b}_\F{ind2-2op}$ & 
		$(e_2 \xrightarrow{\F{rc}} e_1 \vee e_2 \rightleftarrows e_1) \wedge e_b \xrightarrow{\F{rc}} e_\top$ & 
		$\mu(e_\top(l), e_1 \comp e_\top(a), e_2 \comp e_\top(b)) = e_1(\mu(e_\top(l), e_\top(a), e_2 \comp e_\top(b)))$ & 
		$\mu(e_\top(l), e_1 \comp e_\top \comp e_b(a), e_2 \comp e_\top(b)) = e_1(\mu(e_\top(l), e_\top \comp e_b(a), e_2 \comp e_\top(b)))$\\
		\hline
				
		$\psi_\F{ind-2op}^{L_1^a}$ & 
		$e_2 \xrightarrow{\F{rc}} e_1 \vee e_2 \rightleftarrows e_1$ & 
		$\mu(l, e_1(a), e_2(b)) = e_1(\mu(l, a, e_2(b)))$ & 
		$\mu(l, e_1 \comp e_1'(a), e_2(b)) = e_1(\mu(l, e_1'(a), e_2(b)))$\\
		\hline
		
		$\psi_\F{ind-2op}^{L_2^a}$ & 
		$e_2 \xrightarrow{\F{rc}} e_1 \vee e_2 \rightleftarrows e_1$ &
		$\mu(l, e_1(a), e_2(b)) = e_1(\mu(l, a, e_2(b)))$ & 
		$\mu(l, e_1(a), e_2 \comp e_2'(b)) = e_1(\mu(l, a, e_2 \comp e_2'(b)))$\\
		\hline		
		
		
		
		$\psi^{L_\top^b}_\F{base-1op}$ &
		$ $ & 
		$ $ &
		$\mu(\sigma_0, e_1(\sigma_0), \sigma_0) = e_1(\mu(\sigma_0, \sigma_0, \sigma_0))$\\
		\hline

		$\psi^{L_\top^b}_\F{ind-1op}$ &
		$ $ &
		$\mu(l, e_1(l), l) = e_1(\mu(l, l, l))$ & 
		$\mu(e_\top(l), e_1 \comp e_\top(l), e_\top(l)) = e_1(\mu(e_\top(l), e_\top(l), e_\top(l)))$\\
		\hline

		$\psi^{L_\top^a}_\F{ind-1op}$ &
		$\exists e. e \xrightarrow{\F{rc}} e_\top$ &
		$\mu(e_\top'(l), e_1(a), e_\top'(b)) = e_1(\mu(e_\top'(l), a, e_\top'(b)))$ & 
		$\mu(e_\top \comp e_\top'(l), e_1 \comp e_\top(a), e_\top \comp e_\top'(b)) = e_1(\mu(e_\top \comp e_\top'(l), e_\top(a), e_\top \comp e_\top'(b)))$\\
		\hline
		
		$\psi^{L_1^b}_\F{ind1-1op}$ &
		$e_b \xrightarrow{\F{rc}} e_\top$ & 
		$\mu(e_\top(l), e_1 \comp e_\top(a), e_\top(b))) = e_1(\mu(e_\top(l), e_\top(a), e_\top(b)))$ & 
		$\mu(e_\top(l), e_1 \comp e_\top \comp e_b(a), e_\top(b)) = e_1(\mu(e_\top(l), e_\top \comp e_b(a), e_\top(b)))$\\
		\hline

		$\psi^{L_1^b}_\F{ind2-1op}$ &
		$e_b \xrightarrow{\F{rc}} e_\top \wedge (\neg e \rightleftarrows e_b \vee e \xrightarrow{\F{rc}} e_\top)$ &
		$\mu(e_\top(l), e_1 \comp e_\top \comp e_b(a), e_\top(b)) = e_1(\mu(e_\top(l), e_\top \comp e_b(a), e_\top(b)))$ & 
		$\mu(e_\top(l), e_1 \comp e_\top \comp e_b \comp e(a), e_\top(b)) = e_1(\mu(e_\top(l), e_\top \comp e_b \comp e(a), e_\top(b)))$\\
		\hline
		
		$\psi^{L_2^b}_\F{ind1-1op}$ & 
		$e_b \xrightarrow{\F{rc}} e_\top$ &
		$\mu(e_\top(l), e_1 \comp e_\top(a), e_\top(b)) = e_1(\mu(e_\top(l), e_\top(a), e_\top(b)))$ & 
		$\mu(e_\top(l), e_1 \comp e_\top(a), e_\top \comp e_b(b)) = e_1(\mu(e_\top(l), e_\top(a), e_\top \comp e_b(b)))$\\
		\hline

		$\psi^{L_2^b}_\F{ind2-1op}$ &
		$e_b \xrightarrow{\F{rc}} e_\top \wedge (\neg e \rightleftarrows e_b \vee e \xrightarrow{\F{rc}} e_\top)$ & 
		$\mu(e_\top(l), e_1 \comp e_\top(a), e_\top \comp e_b(b)) = e_1(\mu(e_\top(l), e_\top(a), e_\top \comp e_b(b)))$ & 
		$\mu(e_\top(l), e_1 \comp e_\top(a), e_\top \comp e_b \comp e(b)) = e_1(\mu(e_\top(l), e_\top(a), e_\top \comp e_b \comp e(b)))$\\
		\hline	
		
		$\psi_\F{ind-1op}^{L_1^a}$ & 
		$ $ &
		$\mu(e_\top(l), e_1(a), e_\top(b)) = e_1(\mu(e_\top(l), a, e_\top(b)))$ & 
		$\mu(e_\top(l), e_1 \comp e_1'(a), e_\top(b)) = e_1(\mu(e_\top(l), e_1'(a), e_\top(b)))$\\
		\hline
		
		$\psi^{L_\top^b}_\F{base-0op}$ &
		$ $ & 
		$ $ &
		$\mu(e_1(\sigma_0), e_1(\sigma_0), e_1(\sigma_0)) = e_1(\mu(\sigma_0, \sigma_0, \sigma_0))$\\
		\hline
		
		$\psi^{L_\top^b}_\F{ind-0op}$ &
		$ $ &
		$\mu(e_1(l), e_1(l), e_1(l)) = e_1(\mu(l, l, l))$ & 
		$\mu(e_1\comp e_\top(l), e_1 \comp e_\top(l), e_1\comp e_\top(l)) = e_1(\mu(e_\top(l), e_\top(l), e_\top(l)))$\\
		\hline
		
		$\psi^{L_\top^a}_\F{ind-0op}$ &
		$\exists e. e \xrightarrow{\F{rc}} e_\top$ &
		$\mu(e_1(l), e_1(a), e_1(b)) = e_1(\mu(l, a, b))$ &  
		$\mu(e_1\comp e_\top(l), e_1 \comp e_\top(a), e_1\comp e_\top(b)) = e_1(\mu(e_\top(l), e_\top(a), e_\top(b)))$\\
		\hline
		
		$\psi^{L_1^b}_\F{ind1-0op}$ &
		$ $ & 
		$\mu(e_1(l), e_1(a), e_1(b)) = e_1(\mu(l, a, b))$ &  
		$\mu(e_1(l), e_1 \comp e_b(a)), e_1(b)) = e_1(\mu(l, e_b(a), b))$\\
		\hline
		
		$\psi^{L_1^b}_\F{ind2-0op}$ &
		$\neg e \rightleftarrows e_b \vee e \xrightarrow{\F{rc}} e_1$ &
		$\mu(e_1(l), e_1 \comp e_b(a)), e_1(b)) = e_1(\mu(l, e_b(a), b))$ & 
		$\mu(e_1(l), e_1 \comp e_b \comp e(a))), e_1(b)) = e_1(\mu(l, e_b \comp e(a)), b))$\\
		\hline
		
		$\psi^{L_2^b}_\F{ind1-0op}$ & 
		$ $ &
		$\mu(e_1(l), e_1(a), e_1(b)) = e_1(\mu(l, a, b))$ &  
		$\mu(e_1(l), e_1(a), e_1 \comp e_b(b))) = e_1(\mu(l, a, e_b(b)))$\\
		\hline
		
		$\psi^{L_2^b}_\F{ind2-0op}$ &
		$\neg e \rightleftarrows e_b \vee e \xrightarrow{\F{rc}} e_1$ & 
		$\mu(e_1(l), e_1(a), e_1 \comp e_b(b))) = e_1(\mu(l, a, e_b(b)))$ & 
		$\mu(e_1(l), e_1(a), e_1 \comp e_b \comp e(b)))) = e_1(\mu(l, a, e_b \comp e(b))))$\\
		\hline	
		
\end{tabular}%
}
\caption{Complete set of Verification Conditions for MRDTs}
\label{tab:full}
\end{table}


\begin{table}[ht]
\scriptsize
\makebox[\textwidth][c]{%
\begin{tabular}{|>{\raggedright\arraybackslash}p{2.2cm}|>{\raggedright\arraybackslash}p{3cm}|>{\raggedright\arraybackslash}p{3.5cm}|>{\raggedright\arraybackslash}p{4cm}|}%{|c|c|c|c|}
		 
  		\hline
  		\textbf{VC Name} & \multicolumn{2}{c|}{\textbf{Pre-condition}} & \textbf{Post-condition} \\
		\hline

		$\textrm{\sc{MergeCommutativity}}$ & 
		$ $ & 
		$ $ &
		$\mu(a, b)= \mu(b, a)$\\
		\hline
		
		$\textrm{\sc{MergeIdempotence}}$ & 
		$ $ & 
		$ $ &
		$\mu(s, s)= s$\\
		\hline

		$\psi^{L_\top^b}_\F{base-2op}$ &
		$e_2 \xrightarrow{\F{rc}} e_1 \vee e_2 \rightleftarrows e_1$ & 
		$ $ &
		$\mu(e_1(\sigma_0), e_2(\sigma_0)) = e_1(\mu(\sigma_0, e_2(\sigma_0)))$\\
		\hline

		$\psi^{L_\top^b}_\F{ind-2op}$ &
		$e_2 \xrightarrow{\F{rc}} e_1 \vee e_2 \rightleftarrows e_1$ &
		$\mu(e_1(l), e_2(l)) = e_1(\mu(l, e_2(l)))$ & 
		$\mu(e_1 \comp e_\top(l), e_2 \comp e_\top(l)) = e_1(\mu(e_\top(l), e_2 \comp e_\top(l)))$\\
		\hline
		
		$\psi^{L_\top^a}_\F{ind-2op}$ &
		$(e_2 \xrightarrow{\F{rc}} e_1 \vee e_2 \rightleftarrows e_1) \wedge (\exists e. e \xrightarrow{\F{rc}} e_\top)$ &
		$\mu(e_1(a), e_2(b)) = e_1(\mu(a, e_2(b)))$ & 
		$\mu(e_1 \comp e_\top(a), e_2 \comp e_\top(b)) = e_1(\mu(e_\top(a), e_2 \comp e_\top(b)))$\\
		\hline

		$\psi^{L_1^b}_\F{ind1-2op}$ & 
		$(e_2 \xrightarrow{\F{rc}} e_1 \vee e_2 \rightleftarrows e_1) \wedge e_b \xrightarrow{\F{rc}} e_\top$ & 
		$\mu(e_1 \comp e_\top(a), e_2 \comp e_\top(b)) = e_1(\mu(e_\top(a), e_2 \comp e_\top(b)))$ & 
		$\mu(e_1 \comp e_\top \comp e_b(a), e_2 \comp e_\top(b)) = e_1(\mu(e_\top \comp e_b(a), e_2 \comp e_\top(b)))$\\
		\hline
		
		$\psi^{L_1^b}_\F{ind2-2op}$ & 
		$(e_2 \xrightarrow{\F{rc}} e_1 \vee e_2 \rightleftarrows e_1) \wedge e_b \xrightarrow{\F{rc}} e_\top \wedge (\neg e \rightleftarrows e_b \vee e \xrightarrow{\F{rc}} e_\top)$ &
		$\mu(e_1 \comp e_\top \comp e_b(a), e_2 \comp e_\top(b)) = e_1(\mu(e_\top \comp e_b(a), e_2 \comp e_\top(b)))$ & 
		$\mu(e_1 \comp e_\top \comp e_b(e(a)), e_2 \comp e_\top(b)) = e_1(\mu(e_\top \comp e_b \comp e(a), e_2 \comp e_\top(b)))$\\
		\hline

		$\psi^{L_2^b}_\F{ind1-2op}$ & 
		$(e_2 \xrightarrow{\F{rc}} e_1 \vee e_2 \rightleftarrows e_1) \wedge e_b \xrightarrow{\F{rc}} e_\top$ &
		$\mu(e_1 \comp e_\top(a), e_2 \comp e_\top(b)) = e_1(\mu(e_\top(a), e_2 \comp e_\top(b)))$ & 
		$\mu(e_1 \comp e_\top(a), e_2 \comp e_\top \comp e_b(b)) = e_1(\mu(e_\top(a), e_2 \comp e_\top \comp e_b(b)))$\\
		\hline

		$\psi^{L_2^b}_\F{ind2-2op}$ & 
		$(e_2 \xrightarrow{\F{rc}} e_1 \vee e_2 \rightleftarrows e_1) \wedge e_b \xrightarrow{\F{rc}} e_\top$ & 
		$\mu(e_1 \comp e_\top(a), e_2 \comp e_\top(b)) = e_1(\mu(e_\top(a), e_2 \comp e_\top(b)))$ & 
		$\mu(e_1 \comp e_\top \comp e_b(a), e_2 \comp e_\top(b)) = e_1(\mu(e_\top \comp e_b(a), e_2 \comp e_\top(b)))$\\
		\hline
				
		$\psi_\F{ind-2op}^{L_1^a}$ & 
		$e_2 \xrightarrow{\F{rc}} e_1 \vee e_2 \rightleftarrows e_1$ & 
		$\mu(e_1(a), e_2(b)) = e_1(\mu(a, e_2(b)))$ & 
		$\mu(e_1 \comp e_1'(a), e_2(b)) = e_1(\mu(e_1'(a), e_2(b)))$\\
		\hline
		
		$\psi_\F{ind-2op}^{L_2^a}$ & 
		$e_2 \xrightarrow{\F{rc}} e_1 \vee e_2 \rightleftarrows e_1$ &
		$\mu(e_1(a), e_2(b)) = e_1(\mu(a, e_2(b)))$ & 
		$\mu(e_1(a), e_2 \comp e_2'(b)) = e_1(\mu(a, e_2 \comp e_2'(b)))$\\
		\hline		
		
		
		
		$\psi^{L_\top^b}_\F{base-1op}$ &
		$ $ & 
		$ $ &
		$\mu(e_1(\sigma_0), \sigma_0) = e_1(\mu(\sigma_0, \sigma_0))$\\
		\hline

		$\psi^{L_\top^b}_\F{ind-1op}$ &
		$ $ &
		$\mu(e_1(l), l) = e_1(\mu(l, l))$ & 
		$\mu(e_1 \comp e_\top(l), e_\top(l)) = e_1(\mu(e_\top(l), e_\top(l)))$\\
		\hline

		$\psi^{L_\top^a}_\F{ind-1op}$ &
		$\exists e. e \xrightarrow{\F{rc}} e_\top$ &
		$\mu(e_1(a), e_\top'(b)) = e_1(\mu(a, e_\top'(b)))$ & 
		$\mu(e_1 \comp e_\top(a), e_\top \comp e_\top'(b)) = e_1(\mu(e_\top(a), e_\top \comp e_\top'(b)))$\\
		\hline
		
		$\psi^{L_1^b}_\F{ind1-1op}$ &
		$e_b \xrightarrow{\F{rc}} e_\top$ & 
		$\mu(e_1 \comp e_\top(a), e_\top(b))) = e_1(\mu(e_\top(a), e_\top(b)))$ & 
		$\mu(e_1 \comp e_\top \comp e_b(a), e_\top(b)) = e_1(\mu(e_\top \comp e_b(a), e_\top(b)))$\\
		\hline

		$\psi^{L_1^b}_\F{ind2-1op}$ &
		$e_b \xrightarrow{\F{rc}} e_\top \wedge (\neg e \rightleftarrows e_b \vee e \xrightarrow{\F{rc}} e_\top)$ &
		$\mu(e_1 \comp e_\top \comp e_b(a), e_\top(b)) = e_1(\mu(e_\top \comp e_b(a), e_\top(b)))$ & 
		$\mu(e_1 \comp e_\top \comp e_b \comp e(a), e_\top(b)) = e_1(\mu(e_\top \comp e_b \comp e(a), e_\top(b)))$\\
		\hline
		
		$\psi^{L_2^b}_\F{ind1-1op}$ & 
		$e_b \xrightarrow{\F{rc}} e_\top$ &
		$\mu(e_1 \comp e_\top(a), e_\top(b)) = e_1(\mu(e_\top(a), e_\top(b)))$ & 
		$\mu(e_1 \comp e_\top(a), e_\top \comp e_b(b)) = e_1(\mu(e_\top(a), e_\top \comp e_b(b)))$\\
		\hline

		$\psi^{L_2^b}_\F{ind2-1op}$ &
		$e_b \xrightarrow{\F{rc}} e_\top \wedge (\neg e \rightleftarrows e_b \vee e \xrightarrow{\F{rc}} e_\top)$ & 
		$\mu(e_1 \comp e_\top(a), e_\top \comp e_b(b)) = e_1(\mu(e_\top(a), e_\top \comp e_b(b)))$ & 
		$\mu(e_1 \comp e_\top(a), e_\top \comp e_b \comp e(b)) = e_1(\mu(e_\top(a), e_\top \comp e_b \comp e(b)))$\\
		\hline	
		
		$\psi_\F{ind-1op}^{L_1^a}$ & 
		$ $ &
		$\mu(e_1(a), e_\top(b)) = e_1(\mu(a, e_\top(b)))$ & 
		$\mu(e_1 \comp e_1'(a), e_\top(b)) = e_1(\mu(e_1'(a), e_\top(b)))$\\
		\hline
		
		$\psi^{L_\top^b}_\F{base-0op}$ &
		$ $ & 
		$ $ &
		$\mu(e_1(\sigma_0), e_1(\sigma_0)) = e_1(\mu(\sigma_0, \sigma_0))$\\
		\hline
		
		$\psi^{L_\top^b}_\F{ind-0op}$ &
		$ $ &
		$\mu(e_1(l), e_1(l)) = e_1(\mu(l, l))$ & 
		$\mu(e_1 \comp e_\top(l), e_1\comp e_\top(l)) = e_1(\mu(e_\top(l), e_\top(l)))$\\
		\hline
		
		$\psi^{L_\top^a}_\F{ind-0op}$ &
		$\exists e. e \xrightarrow{\F{rc}} e_\top$ &
		$\mu(e_1(a), e_1(b)) = e_1(\mu(a, b))$ &  
		$\mu(e_1 \comp e_\top(a), e_1\comp e_\top(b)) = e_1(\mu(e_\top(a), e_\top(b)))$\\
		\hline
		
		$\psi^{L_1^b}_\F{ind1-0op}$ &
		$ $ & 
		$\mu(e_1(a), e_1(b)) = e_1(\mu(a, b))$ &  
		$\mu(e_1 \comp e_b(a)), e_1(b)) = e_1(\mu(e_b(a), b))$\\
		\hline
		
		$\psi^{L_1^b}_\F{ind2-0op}$ &
		$\neg e \rightleftarrows e_b \vee e \xrightarrow{\F{rc}} e_1$ &
		$\mu(e_1 \comp e_b(a)), e_1(b)) = e_1(\mu(e_b(a), b))$ & 
		$\mu(e_1 \comp e_b \comp e(a))), e_1(b)) = e_1(\mu(e_b \comp e(a)), b))$\\
		\hline
		
		$\psi^{L_2^b}_\F{ind1-0op}$ & 
		$ $ &
		$\mu(e_1(a), e_1(b)) = e_1(\mu(a, b))$ &  
		$\mu(e_1(a), e_1 \comp e_b(b))) = e_1(\mu(a, e_b(b)))$\\
		\hline
		
		$\psi^{L_2^b}_\F{ind2-0op}$ &
		$\neg e \rightleftarrows e_b \vee e \xrightarrow{\F{rc}} e_1$ & 
		$\mu(e_1(a), e_1 \comp e_b(b))) = e_1(\mu(a, e_b(b)))$ & 
		$\mu(e_1(a), e_1 \comp e_b \comp e(b)))) = e_1(\mu(a, e_b \comp e(b))))$\\
		\hline	
		
\end{tabular}%
}
\caption{Complete set of Verification Conditions for CRDTs}
\label{tab:full_crdts}
\end{table}

\subsection{Buggy MRDT implementation in~\cite{Vimala}}
\label{subsec:bug_impl}
\begin{figure}[ht]
\small
\begin{algorithmic} [1]
	\State $\Sigma = (\mathbb{N} \times \F{bool})$
	\State $O = \{\F{enable}, \F{disable}\}$
	\State $Q = \{\F{rd} \}$
	\State $\sigma_0 = (0, \F{false})$
	\State ${\F{do}(\sigma,\_,\_,\F{enable}}) = (\F{fst}(\sigma) + 1, \F{true})$
	\State ${\F{do}(\sigma,\_,\_,\F{disable}}) = (\F{fst}(\sigma), \F{false})$
	\State $\F{merge\_flag}((lc,lf), (ac,af), (bc,bf)) =
   			 \begin{cases}
      				\F{true}, & \F{if}\ af = \text{true}~\&\&~bf = \F{true} \\
				\F{false}, & \F{else\ if}\ af = \text{false}~\&\&~bf = \F{false} \\
				ac > lc, & \F{else\ if}\ af = \text{true}\\
				bc > lc, & \F{otherwise}
   			 \end{cases}$

	\State $\F{merge}(\sigma_\top,\sigma_a,\sigma_b) = (\F{fst}(\sigma_a) + \F{fst}(\sigma_b) - \F{fst}(\sigma_\top), \F{merge\_flag} (\sigma_\top, \sigma_a, \sigma_b))$
	\State $\F{query}(\sigma,rd) = \F{snd}(\sigma)$
	\State $\F{rc} = \{(\F{disable}, \F{enable})\}$
\end{algorithmic}
\caption{Enable-wins flag MRDT implementation from \cite{Vimala}}
\label{fig:ewflag_impl}
\end{figure}