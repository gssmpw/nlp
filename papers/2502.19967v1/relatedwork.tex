\section{Related Work and Conclusion}
\label{sec:related}

Reconciling concurrent updates is a challenging problem in distributed systems.
CRDTs~\cite{Annette,Roh,Shapiro} (and more recently MRDTs) have emerged as a
principled approach for building correct and efficient replicated
implementations. Numerous works have focused on specifying and verifying
CRDTs~\cite{Attiya, Burckhardt, Gotsman, Gomes, Liu, Kartik, Wang, Nair,
Zeller, Katara, Verifx}. Op-based CRDTs have a considerably different system
model than MRDTs, where every operation instance at a replica is individually
sent to other replicas. Hence, verification efforts targeting them
\cite{Gomes,Liu,Kartik,Wang,Nair} are mostly orthogonal to our work.  

The system model of state-based CRDTs is similar to MRDTs, as it also requires 
a merge function to be implemented for reconciling concurrent updates. However, state-based CRDTs have 
stricter requirements for convergence and consistency: CRDT states must form a join-semilattice, 
updates must be monotonic, and the merge function must be the lattice join operation. 
The three algebraic properties of a semilattice: idempotence, commutativity, and 
associativity guarantee convergence.

Some CRDT works focus solely on ensuring convergence 
without addressing functional correctness. For instance, ~\citet{Verifx} do
not fully capture the user intent when verifying state-based CRDTs. Consider a
Counter CRDT with only an increment operation and an \emph{incorrect} merge
function that ignores its input states and always returns 0. Such an
implementation is still convergent. 
However, it clearly does not capture the developer intent, which is
that the value of the counter should be equal to the number of increment
operations. Functional correctness is as important as convergence for 
replicated data types. Our framework addresses both
by couching both in terms of RA-linearizability. We will flag the above
implementation as incorrect, since the state after merge cannot be obtained by
linearizing the operations performed on both the replicas. 

In the context of CRDTs, \citet{Wang} proposed the notion of replication-aware
linearizability, which requires all replicas to have a state which can be
obtained by linearizing the update operations visible to the replica according
to the sequential specification.  However, they do not propose any automated
verification methodology for RA-linearizability. Further, though the main paper
\citet{Wang} focuses on op-based CRDTs, the extended version \citet{EneaArxiv}
does address state-based CRDTs, but they also require a semi-lattice-based
formulation of the CRDT states for proving RA-linearizability.

Few works \cite{Vimala,Kaki2022} have explored the problem of verifying MRDT
implementations. ~\citet{Kaki2022} only focus on verifying convergence, but not
functional correctness. Moreover, they significantly restrict the underlying
system model by synchronizing all merge operations, which as mentioned in the
paper itself could lead to longer convergence times. ~\citet{Vimala} verify both
convergence and functional correctness, without requiring synchronized merges.
However, their approach is not fully automated, and
requires developers to provide a simulation relation linking concrete MRDT
states with an abstract state which is based on a event-based declarative
model. Their specification language is also based on an event-based model and
is not very intuitive or developer-friendly. A few MRDT implementations from
\cite{Vimala} were found to be buggy, and these errors were due to faulty
simulation relations.

To conclude, in this work, we present the first, fully-automated verification
methodology for MRDTs. We introduce the notion of replication-aware
linearizability for MRDTs, as well as a simple specification framework based on
ordering non-commutative update operations. We identify certain restrictions on the
specification to ensure existence of a consistent linearization. We then
leverage the definition of replication-aware linearizability to propose an automated verification
methodology based on induction on operation sequences. We have successfully
applied the technique on a number of complex MRDTs. While the foundations have
been laid in this work, we believe there is a lot of scope for enriching the
technique even further by considering more complex linearization strategies.
\vspace{-2em}