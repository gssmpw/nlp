\section{Verifying RA-linearizability of MRDTs}
\label{sec:lemmas}
In this section, we present our verification strategy for proving RA-linearizability of MRDTs using bottom-up linearization. According to Def. \ref{def:lin}, in order to prove that an MRDT $\M{D}$ is linearizable, we need to consider every configuration $C$ reachable in any execution, and show that all replicas in $C$ have states which can be obtained by linearizing the events applied to the replica, i.e. finding a sequence which obeys the linearization relation (Def. \ref{def:lin-relation}). We will assume that $\M{D}$ satisfies the three constraints ($\textrm{\sc{rc-non-comm}}$, $\textrm{\sc{cond-comm}}$ and $\textrm{\sc{no-rc-chain}}$) necessary for an MRDT to be linearizable, and for bottom-up linearization to succeed.

Our overall proof strategy is to use induction on the length of the execution and to extract generic verification conditions (VCs) which help us to discharge the inductive case. These VCs would essentially be instantiations of the $\textrm{\sc{BottomUpTemplate}}$ rule, proving that the merge operation results in a linearization of the events of the two versions being merged. Proving these VCs for arbitrary MRDTs is not straightforward (as discussed in \S \ref{subsec:induction}), and hence we propose another induction scheme over event sequences. We first discuss the instantiations of the $\textrm{\sc{BottomUpTemplate}}$ rule required for linearizing merges.

\subsection{Linearizing Merge Operations}

Consider an execution $\tau \in \llbracket \M{S_D} \rrbracket$ such that all
configurations in $\tau$ are linearizable. Suppose $\tau$ ends in the
configuration $C$. Now, we extend $\tau$ by one more transition, resulting in
the new configuration $C'$; we need to prove that $C'$ is also linearizable.
Let $C = \langle N, H, L, G, vis \rangle$, $C' = \langle N', H', L', G', vis'
\rangle$. It is easy to see if that this transition is caused due to
$\textrm{\sc{CreateBranch}}$ or $\textrm{\sc{Apply}}$ rules, then $C'$ will be
linearizable. For example, in the $\textrm{\sc{[Apply]}}$  transition, where a
new update operation $o$ is applied on a replica $r$ (generating a new event $e$),
only the state at $r$ changes, and this new state is obtained by directly
applying $e$ on the original state $\sigma$ at $r$. Since $\sigma$ was assumed
to be linearizable, there exists a sequence $\pi$ which extends
$\F{lo}(C)_{\mid L(H(r))}$, with $\sigma = \pi(\sigma_0)$ (recall that
$L(H(r))$ denotes the set of events applied at $r$). Then, the new state
$e(\sigma)$ is clearly linearizable through the sequence $\pi e$ which extends
$\F{lo}(C')_{\mid L'(H'(r))}$. 

We focus on the difficult case when there is a $\textrm{\sc{Merge}}$ transition
from $C$ to $C'$ which merges the replicas $r_1$ and $r_2$. Let $\sigma_1$ and
$\sigma_2$ be the states of the head versions $v_1$ and $v_2$ at $r_1$ and
$r_2$ respectively. Let $\sigma_\top$ be the state of the LCA version $v_\top$
of $v_1$ and $v_2$. Recall that $L(v_\top) = L(v_1) \cap L(v_2)$.  The
transition will install a new version with state $\sigma_m =
\F{merge}(\sigma_\top, \sigma_1, \sigma_2)$ at the replica $r_1$, leaving the
other replicas unchanged. Also, $L'(v_m) = L(v_1) \cup L(v_2)$. We need to show
that there exists a sequence $\pi$ of events in $L'(v_m)$ such that $\pi$
extends $\F{lo}(C')_{\mid L'(v_m)}$ and $\sigma_m = \pi(\sigma_0)$.

We first describe the structure of a sequence $\pi$ which extends
$\F{lo}(C')_{\mid L'(v_m)}$. For ease of readability, we use $L_1$ for
$L(v_1)$, $L_2$ for $L(v_2)$ and $L_\top$ for $L(v_\top)$, and $\F{lo_m}$ for
$\F{lo}(C')_{\mid L'(v_m)}$. We define the following sets of events:
\begin{align*}
	& L_1' = L_1 \setminus L_\top \quad \quad  L_2' = L_2 \setminus L_\top \\
	& L_1^b = \{e \in L_1^{'}\ \mid\ \exists e_\top \in L_\top.\ (e \xrightarrow{\F{lo_m}} e_\top \vee  \exists e' \in L_1^{'}.\ e \xrightarrow{\F{lo_m}} e^{'} \xrightarrow{\F{lo_m}} e_\top)\}\\
	& L_2^b = \{e \in L_2^{'}\ \mid\ \exists e_\top \in L_\top.\ (e \xrightarrow{\F{lo_m}} e_\top \vee \exists e' \in L_2^{'}.\ e \xrightarrow{\F{lo_m}} e^{'} \xrightarrow{\F{lo_m}} e_\top)\}\\
	& L_\top^{a} = \{e_\top \in L_\top \mid \exists e \in L_1^{b} \cup L_2^{b}. e  \xrightarrow{\F{lo_m}} e_\top\} \quad
	  L_1^a = L_1^{'} \setminus L_1^b \quad  \quad L_2^a = L_2^{'} \setminus L_2^b \quad \quad L_\top^{b} = L_\top \setminus L_\top^{a}
\end{align*}


\begin{wrapfigure}{r}{0.3\textwidth}
	\vspace{-1.5em}
	\begin{center}
		\includegraphics[angle=0, width=0.75\linewidth]{order}
	\end{center}
	\vspace{-1em}
	\caption{Structure of sequence $\pi$ extending $\F{lo}_m$}
	\vspace{-2em}
	\label{fig:order}
\end{wrapfigure}

$L_1^{'}$ and $L_2^{'}$ are the local events in each version. Note that any pair of events $e_1 \in L_1', e_2 \in L_2'$ will necessarily be concurrent. This is because, in any reachable configuration, any version $v$ is always \textbf{causally closed}, which means that if $e_1 \xrightarrow{\F{vis}} e_2$ and $e_2 \in L(v)$, then $e_1 \in L(v)$. Hence, for events $e_1 \in L_1', e_2 \in L_2'$, if $e_1 \xrightarrow{\F{vis}} e_2$ then $e_1 \in L_2'$, which would make $e_1$ a non-local event (i.e. part of the LCA). Bottom-up linearization first linearizes the local events across the two versions using the $\F{rc}$ relation for non-commutative events, and then linearizes events of the LCA. However, as demonstrated by the example in \S \ref{subsec:inter}, local events may also need to be linearized before events of the LCA (due to possible intermediate merges), and these events are collected in the sets $L_1^b$ and $L_2^b$. Specifically, $L_i^b (i = 1,2)$  contains those local events $e$ in $L_i'$ which either occur $\F{lo_m}$ before some event in the LCA, or which occur $\F{lo_m}$ before another local event $e'$ which occurs $\F{lo_m}$ before an LCA event. The events of the LCA which need to be linearized after local events are collected in $L_\top^a$. Finally, $L_1^a$ and $L_2^a$ contain local events which do not occur $\F{lo_m}$ before an LCA event.


\begin{example}\label{ex:1}
	Consider the execution in Fig. \ref{fig:inter_merge}, and the merge of versions $v_3$ and $v_4$, for which the LCA is $v_1$. For this merge, $L_1' = \{e_3\}$, $L_2' = \{e_2\}$, $L_1^b = \emptyset$, $L_2^b = \{e_2\}$, $L_\top^{a} = \{e_1\}$. For the merge of versions $v_1$ and $v_2$ (whose LCA is $v_0$), $L_1' = \{e_1\}$, $L_2' = \{e_2\}$, while $L_1^b,L_2^b,L_\top^a$ will all be empty (since no local event comes $\F{lo}$-before an LCA event).
\end{example}


We now show that there exists a sequence $\pi$ which extends $\F{lo_m}$ and which has events in $S_1 = L_\top^b$ followed by $S_2 = L_\top^a \cup L_1^b \cup L_2^b$ followed by $S_3 = L_1^a \cup L_2^a$ (later, we will discuss the ordering of events inside each set $S_i$). To prove this, we will demonstrate that there is no $\F{lo_m}$ from events in $S_i$ to events in $S_{i-1}$.  Based on the definitions of the $S_i$ sets, we can deduce some obvious facts: (i) there cannot be events $e \in S_3$, $e' \in L_\top$ such that $e \xrightarrow{\F{lo_m}} e'$, because otherwise, such an event $e$ would be in $L_1^b \cup L_2^b$ (and hence not in $S_3$), (ii) there cannot be events $e \in L_1^b \cup L_2^b$, $e' \in L_\top^b$ such that $e \xrightarrow{\F{lo_m}} e'$, because otherwise, such an event $e'$ would be in $L_\top^a$. In addition, using \textsc{no-rc-chain} and \textsc{rc-non-comm}, we also prove the following:

\begin{lemma}\label{lem:pi1}
	\begin{enumerate}
		\item For events $e \in L_1^a \cup L_2^a$, $e' \in L_1^b \cup L_2^b$, $\neg (e \xrightarrow{\F{lo_m}} e')$.
		\item For events $e \in L_\top^a$, $e' \in L_\top^b$, $\neg (e \xrightarrow{\F{lo_m}} e')$.
	\end{enumerate}
\end{lemma}

(1) from the above lemma ensures that there is no $\F{lo_m}$ relation from $S_3$ to $S_2$, while (2) ensures the same from $S_2$ to $S_1$. Hence a sequence with the structure $S_1\ S_2\ S_3$ would extend $\F{lo_m}$. Let us now consider the ordering among events in each set. First, for $S_3$, this set contains local events which are guaranteed to not come $\F{lo_m}$ before any event of the LCA. An event in $L_1^a$ will be concurrent with an event in $L_2^a$, and the linearization relation between them will depend upon the $\F{rc}$ relation between the underlying operations (if the events don't commute). We now instantiate $\textrm{\sc{BottomUpTemplate}}$ for the case where both $L_1^a$ and $L_2^a$ are non-empty in the rule $\textrm{\sc{BottomUp-2-OP}}$ in Fig. \ref{fig:bottom-up}, so that the linearization needs to consider the $\F{rc}$ relation between events in the two sets.

\begin{figure}[ht]
\vspace{-1em}
		\small
	\begin{align*}
		& \textrm{\sc{[BottomUp-2-OP]}} & &\textrm{\sc{[BottomUp-1-OP]}} & \\
		& \inferrule{e_1 \neq e_2 \quad e_1 \xrightarrow{\F{rc}} e_2 \vee e_1 \rightleftarrows e_2}{\F{merge}(l, e_1(a), e_2(b)) = e_2(\F{merge}(l, e_1(a), b))} & &\inferrule{(e_\top \neq \epsilon \wedge e_1 \neq e_\top) \vee (e_\top = \epsilon \wedge l = b) }{\F{merge}(e_\top(l), e_1(a), e_\top(b)) = e_1(\F{merge}(e_\top(l), a, e_\top(b)))}  
	\end{align*}
	\begin{align*}
		& \textrm{\sc{[BottomUp-0-OP]}} &    &\textrm{\sc{[MergeIdempotence]}} & 
		&\textrm{\sc{[MergeCommutativity]}} &\\
		& \inferrule{}{\F{merge}(e_\top(l), e_\top(a), e_\top(b)) = e_\top(\F{merge}(l, a, b))} & & \F{merge}(a,a,a) = a & & \F{merge}(l,a,b) = \F{merge}(l,b,a)
	\end{align*}
	\caption{Bottom-up Linearization}
	\vspace{-0.5em}
	\label{fig:bottom-up}
	\vspace{-1em}
\end{figure}

Note that $e_1,e_2$ and $l,a,b$ are all universally quantified. The $\textrm{\sc{BottomUp-2-OP}}$ rule is an algebraic property of $\F{merge}$ which needs to be separately shown for each MRDT implementation. For our case where we are trying to linearize $\F{merge}(\sigma_\top, \sigma_1, \sigma_2)$, we can apply $\textrm{\sc{BottomUp-2-OP}}$ with $l = \sigma_\top$, $e_1(a) = \sigma_1$ and $e_2(b) = \sigma_2$. Note that since $L_1^a$ and $L_2^a$ are both non-empty, $e_1\in L_1^a$, $e_2 \in L_2^b$ (in fact, $e_1$ and $e_2$ would be the maximal events in $L_1^a$ and $L_2^b$ according to $\F{lo_m}$). $\textrm{\sc{BottomUp-2-OP}}$ would then linearize $e_2$ at the end of the sequence. If $ e_1 \xrightarrow{\F{rc}} e_2$, then  $e_1 \xrightarrow{\F{lo_m}} e_2$, and thus linearizing $e_2$ at the end obeys the $\F{lo_m}$ ordering. Note that due to the $\textrm{\sc{no-rc-chain}}$ constraint, $e_2$ cannot come $\F{lo_m}$ before another concurrent event $e_3$. $\textrm{\sc{BottomUp-2-OP}}$  can now be recursively applied on $\F{merge}(l, e_1(a), b)$, by considering $e_1$ and the last event leading to the state $b$. By repeatedly applying $\textrm{\sc{BottomUp-2-OP}}$ all the remaining events in $L_1^a$ and $L_2^a$ can be linearized until one of the sets becomes empty.

Let us now consider the scenario where exactly one of  $L_1^a$ and $L_2^a$ is empty. WLOG, let $L_1^a$ be non-empty. We instantiate $\textrm{\sc{BottomUpTemplate}}$ for the case where $L_1^a$ is non-empty and $L_2^a$ is empty in the rule $\textrm{\sc{BottomUp-1-OP}}$ in Fig. \ref{fig:bottom-up}, so that the linearization orders all events of $L_1^a$ after events of $S_2$.

Let us consider the first clause in the premise where $e_\top \neq \epsilon$. To understand $\textrm{\sc{BottomUp-1-OP}}$, note that if $L_2^a$ is empty, then all local events in $L_2'$ are linearized before the LCA events. In this case, the last event which leads to the state $\sigma_2$ must be an LCA event. $\textrm{\sc{BottomUp-1-OP}}$ uses this observation, with $e_\top(l)=\sigma_\top$, $e_1(a)=\sigma_1$ and $e_\top(b)=\sigma_2 $. Notice that the last event in both the LCA and the second argument to merge are exactly the same. $e_\top$ will be the maximal event (according to $\F{lo_m}$ relation) in $L_\top^a$, while $e_1$ will be the maximal event in $L_1^a$. $\textrm{\sc{BottomUp-1-OP}}$ then linearizes $e_1$ at the end of the sequence, thus ensuring that all $L_1^a$ events are linearized after events in $S_1$ and $S_2$. It is possible that $L_\top^a$ is empty, in which case $L_2'$ will be empty, which is covered by the second clause where $e_\top = \epsilon$ and $l = b$ since there is no local event in the second state.

\begin{example}\label{ex:2}
Referring to Example \ref{ex:1} for the execution in Fig. \ref{fig:inter_merge}, recall that for the merge of $v_3$ and $v_4$, we have $L_1^a = \{e_3\}$, $L_2^a = \emptyset$ and $L_\top = \{e_1\}$. $\textrm{\sc{BottomUp-1-OP}}$ can be applied in this scenario to linearize $e_3$ at the end of the sequence.
\end{example}

$\textrm{\sc{BottomUp-2-OP}}$  and $\textrm{\sc{BottomUp-1-OP}}$ can thus be used to linearize all events in $S_3$. Let us now consider $S_2$, which contains both local events in $L_1^b \cup L_2^b$ and LCA events in $L_\top^a$. We first provide a more fine-grained structure of $\F{lo_m}$ among events in the set $S_2$.  Let $L_\top^a = \{e_1^{\top}, \ldots, e_m^{\top}\}$. For each $e_i^{\top}$, we collect all local events from $L_1^b$ and $L_2^b$ which need to be linearized before $e_i^{\top}$. For local events which need to be linearized before multiple $e_i^{\top}$s, we associate them with the smallest such $i$. We use $L_1^b(e_i^{\top})$ and $L_2^b(e_i^{\top})$ to denote these sets. Formally:
\begin{equation*}
	\begin{split}
		\forall e_i^{\top} \in L_\top^{a}.\ L_1^{b}(e_i^{\top}) = \{e \in L_1^{'} \mid (\forall j.\ j < i \implies e \notin L_1^{b}(e_j^{\top})) \wedge
		e \xrightarrow{\F{lo_m}} e_i^{\top} \vee \exists e^{'} \in L_1^{'}. e \xrightarrow{\F{lo_m}} e^{'} \xrightarrow{\F{lo_m}} e_i^{\top}\}
	\end{split}
\end{equation*}

$L_2^b(e_i^{\top})$ is defined in a similar manner. We now prove the following lemma using \textsc{no-rc-chain} and \textsc{rc-non-comm}:

\begin{lemma}\label{lem:pi2} \begin{enumerate}
		\item For all events $e_i^{\top}, e_j^{\top} \in L_\top^a$, where $L_\top^a = \{e_1^{\top}, \ldots, e_m^{\top}\}$, $\neg (e_i^{\top} \xrightarrow{\F{lo_m}} e_j^{\top})$
		\item 	For events $e \in L_1^b(e_i^{\top}) \cup L_2^b(e_i^{\top})$, $e' \in L_1^b(e_j^{\top}) \cup L_2^b(e_j^{\top})$ where $j<i$, $\neg (e \xrightarrow{\F{lo_m}} e')$.
	\end{enumerate}
\end{lemma}

From (1) in the above lemma, since there is no $\F{lo_m}$ relation among events in $L_\top^a$, consider the sequence $e_1^{\top} e_2^{\top} \ldots e_m^{\top}$ as a starting point for the sequence of events in $S_2$ which extends $\F{lo_m}$. We then inject $L_1^b(e_i^{\top}) \cup L_2^b(e_i^{\top})$ before each $e_i^{\top}$ in the sequence $e_1^{\top} e_2^{\top} \ldots e_m^{\top}$, as shown in Fig. \ref{fig:order}.  Note that in Fig.\ref{fig:order}, we have only presented various segments of the sequence, with the ordering within those segments determined by $\F{vis}$ and $\F{rc}$. By (2) in Lemma \ref{lem:pi2}, we can show that such a sequence will extend $\F{lo}_m$ among the events in $S_2$.

To show that $\F{merge}$ follows the sequence $\pi$ for $S_2$, we now instantiate  $\textrm{\sc{BottomUpTemplate}}$ for the case where $L_1^a$ and $L_2^a$ are empty (i.e. $S_3$ has already been linearized) in the rule $\textrm{\sc{Bottom-0-OP}}$ in Fig. \ref{fig:bottom-up}. Following the structure of $\pi$ in Fig. \ref{fig:order}, $e_\top$ would be the event $e_m^{\top} \in L_\top^a$. Note that since $e_m^{\top}$ is an LCA event, it will be present in both states being merged. $\textrm{\sc{BottomUp-0-OP}}$ then allows this event to be linearized first at the end.

\begin{example}\label{ex:3}
	Following on from Example \ref{ex:2} for the execution in Fig. \ref{fig:inter_merge} for the merge of $v_3$ and $v_4$, after $\textrm{\sc{BottomUp-1-OP}}$ is applied to linearize $e_3$, the states to be merged would be the versions $v_1$ and $v_4$ (with LCA $v_1$), both of whose last operation is $e_1$. Hence, $\textrm{\sc{BottomUp-0-OP}}$ would be applicable, which would linearize $e_1$.
\end{example}

After applying $\textrm{\sc{BottomUp-0-OP}}$ to linearize the LCA event $e_m^{\top}$, we then need to linearize events in $L_1^b(e_m^{\top}) \cup L_2^b(e_m^{\top})$. However, the event $e_m^{\top}$ has already been linearized, so none of the events in $L_1^b(e_m^{\top}) \cup L_2^b(e_m^{\top})$ appear $\F{lo_m}$ after an LCA event. This scenario can now be handled using $\textrm{\sc{BottomUp-2-OP}}$ (if both $L_1^b(e_m^{\top})$ and $L_2^b(e_m^{\top})$ are non-empty) or  $\textrm{\sc{BottomUp-1-OP}}$ (if one of 2 sets is empty). These rules will appropriately linearize the events in $L_1^b(e_m^{\top}) \cup L_2^b(e_m^{\top})$ taking into account the $\F{rc}$ relation for concurrent events and $\F{vis}$ relation for non-concurrent events. Once $L_1^b(e_m^{\top}) \cup L_2^b(e_m^{\top})$ becomes empty, we then encounter the next LCA event in $L_\top^a$, which can again be linearized using $\textrm{\sc{BottomUp-0-OP}}$.

The three instantiations of $\textrm{\sc{BottomUpTemplate}}$ can thus be repeatedly applied to linearize the rest of the events in $S_2$. Following this, all the local events would have been linearized, leaving only the LCA events in $S_1$. This would result in all three arguments to $\F{merge}$ being equal, in which case we can use the
$\textrm{\sc{MergeIdempotence}}$ rule in Fig. \ref{fig:bottom-up}. Using $\textrm{\sc{MergeIdempotence}}$, we can equate the output of $\F{merge}$ to it's argument, which has already been assumed to be appropriately linearized.

In order to avoid mirrored versions of $\textrm{\sc{BottomUp-2-OP}}$ and $\textrm{\sc{BottomUp-1-OP}}$ where the second and third arguments are swapped, we also require the $\textrm{\sc{MergeCommutativity}}$ property in Fig. \ref{fig:bottom-up}. We now state our soundness theorem linking the various properties with RA-linearizability of MRDT:

\begin{theorem}\label{thm:1}
	If an MRDT $\M{D}$ satisfies  $\textrm{\sc{BottomUp-2-OP}}$,  $\textrm{\sc{BottomUp-1-OP}}$,  $\textrm{\sc{BottomUp-0-OP}}$, $\textrm{\sc{MergeIdempotence}}$ and $\textrm{\sc{MergeCommutativity}}$, then $\M{D}$ is linearizable.
\end{theorem}

The proof closely follows the informal arguments that we have presented in this sub-section, using induction on the size of the various sets $L_1^a, L_2^a, L_1^b \cup L_2^b, L_\top^a$.

\subsection{Automated Verification}
While we have identified the sufficient conditions to show RA-linearizability of an MRDT using bottom-up linearization, proving these conditions for arbitrary MRDTs is not straightforward. Further, while the $\textrm{\sc{BottomUp-X-OP}}$ properties as shown in the previous sub-section had universal quantification over MRDT states $l,a,b$, in general, for proving RA-linearizability, we only need to show these properties for feasible states that may arise during an actual execution. 

We now leverage the fact that the feasible states would have been obtained through linearization of the visible events at the respective versions. In particular, we can characterize the states on which merge can be invoked through the various events sets $L_1^a, L_2^a, L_1^b, L_2^b, L_\top^a, L_\top^b$ that we had defined in the previous sub-section. We only need to prove the $\textrm{\sc{BottomUp-X-OP}}$ properties for states which have been obtained through linearizations of events in these event sets. For this purpose, we propose an induction scheme which establishes the required properties while traversing the event sets as depicted in Fig. \ref{fig:order} in a top-down fashion.
\newcommand{\comp}{\!\cdot\!}

\begin{table}[htpb]
	\caption{Induction scheme for  $\textrm{\sc{BottomUpTemplate}}$. For clarity, we use $\cdot$ for function composition, and $\mu$ for merge.}
	\scriptsize
	\vspace{-1em}
	\makebox[\textwidth][c]{
		\begin{tabular}{|>{\raggedright\arraybackslash}p{0.5cm}|>{\raggedright\arraybackslash}p{2.2cm}|>{\raggedright\arraybackslash}p{3.8cm}|>{\raggedright\arraybackslash}p{4.5cm}|}

			\hline
			\textbf{VC Name} & \multicolumn{2}{c|}{\textbf{Pre-condition}} & \textbf{Post-condition} \\
			\hline


			$\psi^{L_\top^b}_\F{base}$ &
			&
			$ $ &
			$\mu(\pi_0(\sigma_0), \pi_1(\sigma_0), \pi_2(\sigma_0)) = \pi \comp \mu(\pi_0'(\sigma_0),\pi_1'( \sigma_0), \pi_2'(\sigma_0))$\\
			\hline

			$\psi^{L_\top^b}_\F{ind}$&
			&
			$\mu(\pi_0(l), \pi_1(l), \pi_2(l)) = \pi \comp \mu(\pi_0'(l), \pi_1'(l), \pi_2'(l))$ &
			$\mu (\pi_0 \comp e_\top(l), \pi_1 \comp e_\top(l), \pi_2 \comp e_\top(l)) = \pi \comp \mu(\pi_0' \comp e_\top(l), \pi_1' \comp e_\top(l), \pi_2' \comp e_\top(l))$\\
			\hline

			$\psi^{L_\top^a}_\F{ind}$ &
			$\exists e.\ e \xrightarrow{\F{rc}} e_\top$ &
			$\mu(\pi_0(l), \pi_1(a), \pi_2(b)) = \pi \comp \mu(\pi_0'(l), \pi_1'(a), \pi_2'(b))$ &
			$\mu (\pi_0 \comp e_\top(l), \pi_1 \comp e_\top(a), \pi_2 \comp e_\top(b)) = \pi \comp \mu(\pi_0' \comp e_\top(l), \pi_1' \comp e_\top(a), \pi_2' \comp e_\top(b))$\\
			\hline


			$\psi^{L_1^b}_\F{ind1}$ &
			$e_b \xrightarrow{\F{rc}} e_\top$ &
			$\mu(\pi_0 \comp e_\top(l), \pi_1 \comp e_\top(a), \pi_2 \comp e_\top(b)) = \pi \comp \mu(\pi_0' \comp e_\top(l), \pi_1' \comp e_\top(a), \pi_2' \comp e_\top(b))$ &
			$\mu(\pi_0 \comp e_\top(l), \pi_1 \comp e_\top \comp e_b(a), \pi_2 \comp e_\top(b)) = \pi \comp \mu(\pi_0' \comp e_\top(l), \pi_1' \comp e_\top \comp e_b(a), \pi_2' \comp e_\top(b))$\\
			\hline

			$\psi_\F{ind2}^{L_1^b}$ &
			$e_b \xrightarrow{\F{rc}} e_\top \wedge \neg e \rightleftarrows e_b$ &
			$\mu(\pi_0 \comp e_\top(l), \pi_1 \comp e_\top \comp e_b(a), \pi_2 \comp e_\top(b)) = \pi \comp \mu(\pi_0' \comp e_\top(l), \pi_1' \comp e_\top \comp e_b(a), \pi_2' \comp e_\top(b))$ &
			$\mu(\pi_0 \comp e_\top(l), \pi_1 \comp e_\top \comp e_b \comp e(a), \pi_2 \comp e_\top(b)) = \pi \comp \mu(\pi_0' \comp e_\top(l), \pi_1' \comp e_\top \comp e_b \comp e(a), \pi_2' \comp e_\top(b))$ \\
			\hline


			$\psi^{L_2^b}_\F{ind1}$ &
			$ e_b \xrightarrow{\F{rc}} e_\top$ &
			$\mu(\pi_0 \comp e_\top(l), \pi_1 \comp e_\top(a), \pi_2 \comp e_\top(b)) = \pi \comp \mu(\pi_0' \comp e_\top(l), \pi_1' \comp e_\top(a), \pi_2' \comp e_\top(b))$ &
			$\mu(\pi_0 \comp e_\top(l), \pi_1 \comp e_\top(a), \pi_2 \comp e_\top \comp e_b(b)) = \pi \comp \mu(\pi_0' \comp e_\top(l), \pi_1' \comp e_\top(a), \pi_2' \comp e_\top \comp e_b(b))$\\
			\hline


			$\psi_\F{ind2}^{L_2^b}$ &
			$ e_b \xrightarrow{\F{rc}} e_\top \wedge \neg e \rightleftarrows e_b$ &
			$\mu(\pi_0 \comp e_\top(l), \pi_1 \comp e_\top \comp e_b(a), \pi_2 \comp e_\top(b)) = \pi \comp \mu(\pi_0' \comp e_\top(l), \pi_1' \comp e_\top \comp e_b(a), \pi_2' \comp e_\top(b))$ &
			$\mu(\pi_0 \comp e_\top(l), \pi_1 \comp e_\top \comp e_b(a), \pi_2 \comp e_\top \comp e_b \comp e(b)) = \pi \comp \mu(\pi_0' \comp e_\top(l), \pi_1' \comp e_\top \comp e_b(a), \pi_2' \comp e_\top \comp e_b \comp e(b))$ \\
			\hline


			$\psi_\F{ind}^\F{L_1^a}$ &
			&
			$\mu(\pi_0(l), \pi_1(a), \pi_2(b)) = \pi \comp \mu(\pi_0'(l), \pi_1'(a), \pi_2'(b))$ &
			$\mu(\pi_0(l), \pi_1 \comp e(a), \pi_2(b)) = \pi \comp \mu(\pi_0'(l), \pi_1' \comp e(a), \pi_2'(b))$\\
			\hline

			$\psi_\F{ind}^\F{L_2^a}$ &
			&
			$\mu(\pi_0(l), \pi_1(a), \pi_2(b)) = \pi \comp \mu(\pi_0'(l), \pi_1'(a), \pi_2'(b))$ &
			$\mu(\pi_0(l), \pi_1(a), \pi_2 \comp e(b)) = \pi \comp \mu(\pi_0'(l), \pi_1'(a), \pi_2' \comp e(b))$\\
			\hline

		\end{tabular}
	}
	\label{tbl:vc}
	\vspace{-1em}
\end{table}
Here, we present the induction scheme for the generic $\textrm{\sc{BottomUpTemplate}}$ rule. The scheme can then be instantiated for all the three $\textrm{\sc{BottomUp-X-OP}}$ rules. Table \ref{tbl:vc} contains the verification conditions corresponding to the base case and inductive case over the different event sets. Every VC has the form $(\text{pre-condition} \implies \text{post-condition})$, and all variables are universally quantified. Our goal is to show the $\textrm{\sc{BottomUpTemplate}}$ rule for all feasible MRDT states $l,a,b$, where $l$ is the state of the LCA of $a$ and $b$. Let $L_\top,L_1,L_2$ be the event sets corresponding to $l,a,b$ respectively. We define the event sets $L_1^a, L_2^a, L_1^b, L_2^b, L_\top^a, L_\top^b$ in exactly the same manner as the previous sub-section, based on the linearization relation of the configuration obtained by the $\F{merge}(l,a,b)$ transition. Note that the events in $\pi_0,\pi_1,\pi_2$ (used in the $\textrm{\sc{BottomUpTemplate}}$ rule) would also come from the above event sets, but in the following discussion, we freeze these events, i.e. all our assertions about the events sets will be modulo these events.

We start with the VC $\psi^{L_\top^b}_\F{base}$, which corresponds to the case where every event set is empty. There is no pre-condition, and the post-condition requires $\textrm{\sc{BottomUpTemplate}}$ to hold on the initial MRDT state $\sigma_0$. For example, for the $\textrm{\sc{BottomUp-2-OP}}$ rule, $\psi^{L_\top^b}_\F{base}$ VC would be $\F{merge}(\sigma_0, e_1(\sigma_0), e_2(\sigma_0)) = e_2(\F{merge}(\sigma_0, e_1(\sigma_0), \sigma_0))$, where $e_1 \xrightarrow{\F{rc}} e_2$ or $e_1$ and $e_2$ commute. Notice that $e_1$ and $e_2$ would be events in $L_1^a$ and $L_2^a$, and our assertion about all event sets being empty is modulo these events.

Next, the VC $\psi^{L_\top^b}_\F{ind}$ corresponds to the inductive case on $L_\top^b$, where we assume every event set except $L_\top^b$ to be empty. The pre-condition corresponds to the inductive hypothesis, where we assume the property to hold for some event set $L_\top^b$, and the post-condition asserts that the property holds while adding another event $e_\top$ to $L_\top^b$. Recall that $L_\top^b$ corresponds to the LCA events which come $\F{lo}$ before all local events. Since all the other event sets are empty, this translates to the same state $l$ for all the three arguments to merge in the pre-condition, and applying the LCA event $e_\top$ to all three arguments in the post-condition.

Next, we induct on the set $L_\top^a$, i.e. the set of LCA events which occur $\F{lo}$ after a local event. The base case, where $\mid L_\top^a \mid = \emptyset$  exactly corresponds to the result of the induction on $L_\top^b$. The inductive case is covered by the VC $\psi^{L_\top^a}_\F{ind}$, which adds an LCA event $e_\top$ to all three arguments of merge from pre-condition to post-condition. Notice that we also have another pre-condition which requires the existence of some event $e$ which should come $\F{rc}$-before $e_\top$, which is necessary for $e_\top$ to be in $L_\top^a$. The post-condition just adds a new LCA event $e_\top$. The events in $L_1^b(e_\top)$ and $L_2^b(e_\top)$ will be added by the next 4 VCs.

$\psi^{L_1^b}_\F{ind1}$ and $\psi^{L_1^b}_\F{ind2}$ add an event in $L_1^b$ from the pre-condition to the post-condition. $\psi^{L_1^b}_\F{ind1}$ considers an event $e_b$ which occurs $\F{rc}$-before the LCA event $e_\top$. Notice that the pre-condition of  $\psi^{L_1^b}_\F{ind1}$  is exactly the same as the post-condition of $\psi^{L_\top^a}_\F{ind}$. In the post-condition of $\psi^{L_1^b}_\F{ind1}$, the event $e_b$ is applied before $e_\top$ on the argument $a$ to merge, thus reflecting that this is an event in $L_1^b$. $\psi^{L_1^b}_\F{ind2}$ adds an event $e \in L_1^b$ which does not commute with an existing event $e_b \in L_1^b$ (see the definition of $L_1^b$).   $\psi^{L_2^b}_\F{ind1}$ and $\psi^{L_2^b}_\F{ind2}$ are analogous and do the same thing for the argument $b$ to merge.

Finally, $\psi_\F{ind}^\F{L_1^a}$ and $\psi_\F{ind}^\F{L_2^a}$ add events from $L_1^a$ and $L_2^a$. The base cases for the two sets would exactly correspond to the result of the induction carried out so far on the rest of the event sets. For the inductive case, in $\psi_\F{ind}^\F{L_1^a}$ (resp. $\psi_\F{ind}^\F{L_2^a}$), a new event $e$ is added on the second argument $a$ (resp. third argument $b$) from the pre-condition to the post-condition. This establishes the rule $\textrm{\sc{BottomUpTemplate}}$ for any feasible input arguments to merge during any execution. We denote the set of VCs in Table \ref{tbl:vc} by $\psi^*(\textrm{\sc{BottomUpTemplate}})$.

\begin{theorem}\label{thm:2}
	If an MRDT $\M{D}$ satisfies  the VCs $\psi^*(\textrm{\sc{BottomUp-2-OP}})$,  $\psi^*(\textrm{\sc{BottomUp-1-OP}})$,\\  $\psi^*(\textrm{\sc{BottomUp-0-OP}})$, $\textrm{\sc{MergeIdempotence}}$ and $\textrm{\sc{MergeCommutativity}}$, then $\M{D}$ is linearizable.
\end{theorem}