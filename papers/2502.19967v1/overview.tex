\section{Overview}
\label{sec:overview}


\subsection{System Model}
\label{model}

The MRDT system model resembles a distributed version control system, such as
Git~\cite{Git}, with replication centred around versioned states in branches
and explicit merges.  A replicated data store handles multiple objects
independently \cite{Riak,Irmin}; in our presentation, we focus on modeling a
store with a single object. The state of the object is replicated across
multiple replicas $r_{1}, r_{2},\ldots \in \mathcal{R}$ in the store. 
Clients interact with the store by performing query or update operations 
on one of the replicas, with update operations modifying its state.
These replicas operate concurrently, allowing
independent modifications without synchronization. They periodically (and
non-deterministically) exchange updates with each other through a process
called \emph{merge}. Due to concurrent operations happening at multiple
replicas, conflicts may arise, which must be resolved by the merge operation in
an appropriate and consistent manner.
An object has a type $\tau \in Type$, whose type signature  $\langle O_\tau, Q_\tau, Val_\tau \rangle$
contains the set of supported update operations $O_{\tau}$, 
query operations $Q_{\tau}$ and 
their return values $Val_{\tau}$.

\begin{definition}
An MRDT implementation for a data type $\tau$ is a tuple $\mathcal{D}_{\tau}$ =
$\langle \Sigma, \sigma_0, \F{do}, \F{merge}, \F{query}, \\
\F{rc} \rangle$, where:
\begin{itemize}
	\item $\Sigma$ is the set of states, $\sigma_0 \in \Sigma$ is the initial state.
	\item $\F{do}$ : $\Sigma \times \mathcal{T} \times \mathcal{R} \times O_{\tau} \rightarrow \Sigma$
			implements all update operations in $O_\tau$, 
			where $\mathcal{T}$ is the set of timestamps.
	\item $\F{merge}$ : $\Sigma \times \Sigma \times \Sigma \rightarrow \Sigma$
				is a three-way merge function.
	\item $\F{query}$: $\Sigma \times Q_{\tau} \rightarrow Val_{\tau}$ implements all query operations in $Q_\tau$, returning a value in $Val_\tau$.
	\item $\F{rc} \subseteq O_{\tau} \times O_{\tau}$  is the conflict resolution policy to be followed for
			concurrent update operations.
\end{itemize}
\end{definition}

An MRDT $\mathcal{D}_\tau$ provides implementations of $\F{do}, \F{merge}$ and $\F{query}$ which will be invoked by the data store appropriately.
A client request to perform an update operation
$o \in O_{\tau}$ at a replica $r$ triggers the call $\F{do}(\sigma,t,r,o)$. 
This takes as input the current state $\sigma \in \Sigma$ of $r$, a unique timestamp $t \in \mathcal{T}$
and produces an updated
state which is then installed at $r$. The data store ensures that timestamps are unique across all operations
(which can be achieved through e.g. Lamport timestamps \cite{Lamport78}).

Replicas can also receive states from other replicas, which are merged with the
receiver's state using \emph{$\F{merge}$}. The \emph{$\F{merge}$}
function is called with the current states of both the sender and receiver
replicas and their lowest common ancestor (LCA), which represents the most
recent common state from which the two replicas diverged. 
Clients can query the state of the MRDT using the $\F{query}$ method.
This takes a MRDT state $\sigma \in \Sigma$ and 
a query operation as input and produces a return value. Note that a query operation cannot change the state at a replica.

\begin{wrapfigure}{r}{0.36\textwidth}
	\footnotesize
	\begin{algorithmic} [1]
		\State $\Sigma = \mathbb{N}$
		\State $O = \{\F{inc} \}$
		\State $Q = \{\F{rd} \}$
		\State $\sigma_0 = 0$
		\State $\F{do}(\sigma, \_, \_,\F{inc}) = \sigma+1$
		\State $\F{merge}(\sigma_\top,\sigma_1,\sigma_2) = \sigma_1 + \sigma_2 - \sigma_\top$
		\State $\F{query}(\sigma,rd) = \sigma$
		\State $\F{rc} = \emptyset$
	\end{algorithmic}
	\vspace{-0.5em}
	\caption{Counter MRDT implementation}
	\vspace{-1.5em}
	\label{fig:counter_impl}
\end{wrapfigure}

While merging, it may
happen that conflicting update operations may have been performed on the two states,
in which case, the implementation also provides a conflict resolution policy
$\F{rc}$.  The merge function should make sure that this policy is followed
while computing the merged state. To illustrate, we now present a couple of
MRDT implementations: an increment-only counter and an observed-remove set.

The counter MRDT implementation is given in Fig. \ref{fig:counter_impl}. The
state space of the counter MRDT is simply the set of natural numbers, and it
allows clients to perform only one update operation ($\F{inc}$) which increments the
value of the counter. For merging two counter states $\sigma_1$ and $\sigma_2$,
whose lowest common ancestor is $\sigma_\top$, intuitively, we want to find the
total number of increment operations across $\sigma_1$ and $\sigma_2$. Since
$\sigma_\top$ already accounts for the effect of the common increments in
$\sigma_1$ and $\sigma_2$, we need to count the newer increments and then add
them to $\sigma_\top$. This is achieved by adding $\sigma_1 - \sigma_\top$ and
$\sigma_2 - \sigma_\top$ to $\sigma_\top$, which simplifies to the merge definition
in Fig. \ref{fig:counter_impl}. For example, suppose we have replicas $r_1$ and
$r_2$ whose initial state was $\sigma_\top = 2$. Now, if there are 2 $\F{inc}$
operations at $r_1$ and 3 $\F{inc}$ operation at $r_2$, their states will be
$\sigma_1 = 4$ and $\sigma_2 = 5$. On merging $r_2$ at $r_1$,
$\F{merge}(\sigma_\top,\sigma_1,\sigma_2)$ will return 7, which reflects the total
number of increments. The $\F{query}$ method simply returns the 
current state of the counter. Finally, the increment operation commutes with itself, so
there is no need to define a conflict resolution policy.


\begin{wrapfigure}{l}{0.4\textwidth}
\footnotesize 
\begin{algorithmic} [1]
	\State $\Sigma = \mathcal{P}(\mathbb{E} \times \mathcal{T})$
	\State $O = \{\F{add}_a, \F{rem}_a \mid a \in \mathbb{E}\}$
	\State $Q = \{\F{rd} \}$
	\State $\sigma_0 = \{\}$
	\State ${\F{do}(\sigma,t,\_,\F{add}_{a}}) = \sigma\cup\{(a,t)\}$
	\State ${\F{do}(\sigma,\_,\_,\F{rem}_{a}}) = \sigma\textbackslash\{(a,i) \mid (a,i)\in \sigma\}$
	\State $\F{merge}(\sigma_\top,\sigma_1,\sigma_2) =$
	\Statex $ \quad (\sigma_\top \cap \sigma_1 \cap \sigma_2) \cup (\sigma_1 \textbackslash \sigma_\top) \cup (\sigma_2 \textbackslash \sigma_\top)$
	\State $\F{query}(\sigma,rd) = \{a \mid (a,\_) \in \sigma\}$
	\State $\F{rc} = \{(\F{rem}_a, \F{add}_a)\ \mid\ a \in \mathbb{E}\}$					
\end{algorithmic}
\caption{OR-set MRDT implementation}
\vspace{-1em}
\label{fig:orset_impl}
\end{wrapfigure}

An observed-remove set (OR-set) \cite{Shapiro} is an implementation of a set
data type that employs an add-wins conflict-resolution strategy, prioritizing
addition in cases of concurrent addition and removal of the same element.
Fig.~\ref{fig:orset_impl} shows the OR-set MRDT implementation. This
implementation is quite similar to the operation-based (op-based) CRDT
implementation of OR-set \cite{shapiro2011comprehensive}.  The state of the
OR-set is a set of element-timestamp pairs, with the initial state being an
empty set.  Clients can perform two operations for every element $a \in
\mathbb{E}$: $\F{add}_a$ and $\F{rem}_a$. The $\F{add}_a$ method adds the
element $a$ along with the (unique) timestamp at which the operation was
performed.  The $\F{rem}_a$ method removes all entries in the set corresponding
to the element $a$. An element $a$ is considered to be present in the set if
there is some $(a,t)$ in the state.

The \emph{$\F{merge}$} method takes as input the LCA set $\sigma_\top$ and the two
sets $\sigma_1$ and $\sigma_2$ to be merged, retains elements of $\sigma_\top$
that were not removed in both sets, and includes the newly added elements from
both sets. Since $\sigma_\top$ is the most recent state from which the two sets
diverged, the intersection of all three sets is the set of elements that were
not removed from $\sigma_\top$ in either branch, while the difference of either
set with the $\sigma_\top$ corresponds to the newly added elements.
The $\F{query}$ operation $rd$ returns all the elements in the set. The conflict
resolution relation $\F{rc}$ orders $\F{rem}_a$ before $\F{add}_a$ of the same
element in order to achieve the add-wins semantics. Note that all other pairs of
operations ($\F{add}\_$ and $\F{add}\_$, $\F{rem}\_$ and $\F{rem}\_$, 
and $\F{add}_x$ and $\F{rem}_y$ with $x \neq y$) commute with each other, 
hence $\F{rc}$ does not specify their ordering. We now consider whether 
the merge operation adheres to the conflict resolution policy.

\subsection{RA-linearizability for MRDTs}
\label{subsec:lin}
We would like to verify that an MRDT implementation is correct, in the sense
that in every execution, (a) replicas which have observed the same set of
update operations converge to the same state, and (b) this state reflects the
semantics of the implemented data type and the conflict resolution policy. 
Note that an update operation $o$ is considered to be visible to a replica
$r$ either if $o$ is directly applied by a client at $r$, or indirectly through
merge with another replica $r'$ on which $o$ was visible. To specify MRDT
correctness, we propose to use the notion of RA-linearizability
\cite{Wang}: the state at any replica during any execution must be achievable
by applying a sequence (or linearization) of the update operations visible to the
replica.  Further, this linearization should obey the user-specified conflict
resolution policy for concurrent operations, and the local replica order for
non-concurrent operations.

Our definition of RA-linearizability allows viewing the state of an MRDT replica as a sequence of update operations applied on the initial state, thus abstracting over the merge function and how it handles concurrent operations. Consequently, any formal reasoning (e.g. assertion checking, functional correctness, equivalence checking etc.) can now essentially forget about the presence of merges, and only focus on update operations, with the additional guarantee that operations would have been correctly linearized, taking into account the conflict resolution policy and local replica ordering.

Proving RA-linearizability for MRDTs is straightforward when there is only a
single replica on which all operations are performed, since there is no
interleaving among operations on a single replica. Complexity arises when update
operations happen concurrently across replicas, which are then merged. For a
merge operation, we need to show that the output can be obtained by applying a
linearization of update operations witnessed by both replicas being merged. However,
the states being merged would have been obtained after an arbitrary number of update
operations or even other merges. Further, the MRDT framework maintains only the
states, but not the update operations leading to those states, thus requiring the
verification technique to somehow infer the update operations leading to a state, and
then show that merge constructs the correct linearization.

We break down this difficult problem gradually with a series of observations.
We will start with an intuitively correct approach, show how it could be broken
through examples, and gradually refine it to make it work. As a starting point,
we first observe that we can leverage the following algebraic properties of the
MRDT update operations and the merge function: (i) commutativity of merge and
update operations, (ii) commutativity of merge, (iii) idempotence of merge, and (iv)
commutativity of update operations. To motivate this, we first introduce some
terminology. An event $e = \langle t,r,o \rangle$ is generated for every update
operation instance, where $t$ is the event's timestamp and $r$ is the replica
on which the update operation $o$ is applied.  Applying an event $e$ on a replica with
state $\sigma$ changes the replica state to $e(\sigma) = \F{do}(\sigma,t,r,o)$ using
the implementation of the operation $o$. Given a sequence of events $\pi = e_1
e_2 \ldots e_n$, we use the notation $\pi(\sigma)$ to denote
$e_n(\ldots(e_2(e_1(\sigma))))$. Now, the properties described above can be
formally defined as follows (forall $ \sigma_\top, \sigma_1, \sigma_2, e, e'$):

{\small
	\begin{enumerate}
	\item[(P1)] $\F{merge}(\sigma_\top,e(\sigma_1), \sigma_2) = e(\F{merge}(\sigma_\top, \sigma_1, \sigma_2 ))$
	\item[(P2)]$\F{merge}(\sigma_\top,\sigma_1, \sigma_2) = \F{merge}(\sigma_\top, \sigma_2, \sigma_1)$
	\item[(P3)] $\F{merge}(\sigma_\top, \sigma_\top, \sigma_\top) = \sigma_\top$
	\item[(P4)] $e(e'(\sigma)) = e'(e(\sigma))$
\end{enumerate}}

As per our proposed definition of RA-linearizability, we need to show that there exists a linearization of events visible at the replica such that the state of the replica can be obtained by applying this linearization. As mentioned earlier, an event can become visible at a replica either by a direct client application, or
by merging with another replica. To illustrate this, consider the scenario
\begin{wrapfigure}{r}{0.3\textwidth}
	\vspace{-1em}
	\begin{center}
		\includegraphics[angle=0, width=0.8\linewidth]{Merge-linearization}
	\end{center}
	\vspace{-1em}
	\caption{Linearizing a merge operation}
	\vspace{-1em}
	\label{fig:merge-lin}
\end{wrapfigure}
shown in Fig. \ref{fig:merge-lin} where two replicas with states $\sigma_1$ and
$\sigma_2$ are being merged. These states were obtained by applying a sequence
of events $\pi_1$ and $\pi_2$ respectively on the LCA state $\sigma_\top$. We call
the events in $\pi_1$ and $\pi_2$ as local to their respective replicas. Now,
when the two states are merged to create a new state $\sigma_m$ we would need
to show that the state $\sigma_m$ ($= \F{merge}(\sigma_\top, \sigma_1, \sigma_2)$)
can be obtained by linearizing all the events in $\pi_1$ and $\pi_2$, and
applying this linearization on the state $\sigma_\top$.

To show that the merge function constructs a linearization, we can take advantage of
properties (P1)-(P4). In particular, commutativity of merge and update operation
application (P1) allows us to move an event from the second argument of merge
to outside, and we can then repeatedly apply this property to peel off all the
events in $\pi_1$. More formally, by performing induction on the sequence
$\pi_1$ and using (P1), we can show that $\F{merge}(\sigma_\top,\pi_1(\sigma_\top),
\sigma_2 ) = \pi_1(\F{merge}(\sigma_\top, \sigma_\top, \sigma_2))$.  We can then use
commutativity of merge (P2) to swap the last two arguments of merge, and then
apply (P1) again to peel off all the events in $\pi_2$, thus establishing that
$\F{merge}(\sigma_\top, \sigma_\top, \pi_2(\sigma_\top)) =
\pi_2(\F{merge}(\sigma_\top,\sigma_\top, \sigma_\top))$. Finally, using merge
idempotence (P3), and combining all the previous results, we can infer that
$\F{merge}(\sigma_\top,\sigma_1, \sigma_2) = \pi_2(\pi_1(\sigma_\top))$.
Commutativity of update operations (P4) ensures that all linearizations of events in
$\pi_1$ and $\pi_2$ lead to the same state, thus ratifying the specific
linearization order $\pi_1 \pi_2$ that we constructed using properties P1-P3.
We call this process as bottom-up linearization, since we built the sequence
from end through property (P1), linearizing one event at a time.

\begin{wrapfigure}{r}{0.35\textwidth}
	\vspace{-1.8em}
	\begin{center}
		\includegraphics[angle=0, width=1\linewidth]{orset_exec}
	\end{center}
	\vspace{-1.5em}
	\caption{OR-set execution}
	\label{fig:orset_exec}
\end{wrapfigure}

It is also easy to see that the counter MRDT implementation in Fig.
\ref{fig:counter_impl} satisfies (P1)-(P4). In particular, commutativity of
integer addition and subtraction essentially gives us (P1)-(P4) for free. While
this strategy works for the counter MRDT, commutativity of all update operations is in
general a very strong requirement, and would fail for other
datatypes. For example, the OR-set MRDT of Fig. \ref{fig:orset_impl}
does not satisfy (P4), as the $\F{add}_a$ and $\F{rem}_a$ operations do not
commute.

In the presence of non-commutative update operations, the property (P1) now needs to
be altered, as we need to consider the conflict resolution policy
to decide the replica from which an event needs to be peeled off. To
illustrate this, consider an OR-set execution depicted in
Fig.~\ref{fig:orset_exec}. We show the version graph of the execution, where
each oval represents a version. The state of the version is depicted inside the
oval. The versions $v_1$ and $v_2$ are obtained by applying $\F{rem}_a$  and
$\F{add}_a$ operations to the version $v_\top$ on two different replicas ($r_1$
and $r_2$). Each edge is labeled with the event corresponding to the
application of an operation. Let $\sigma_\top = \{\}$ denote the state of the LCA
$v_\top$. The versions $v_1$ and $v_2$ are then merged at $r_2$ which gives rise
to a new version $v_m$ with state $\F{merge}(\sigma_\top,e_1(\sigma_\top),
e_2(\sigma_\top))$. Now, since $e_1$ and $e_2$ do not commute, the conflict
resolution policy of OR-set places $e_1$ (i.e. the remove operation)
before $e_2$ (i.e. the add operation). Hence, we want the merged version to
follow the linearization order $e_2(e_1(\sigma_\top))$. This requires us to
first peel off the event $e_2$ from the third argument of $\F{merge}$. To
achieve this, we can alter the property (P1) by making it aware of the conflict
resolution policy as follows:

\begin{itemize}
	\item[(P1$'$)]
		$(e_1,e_2) \in \F{rc} \implies
			\F{merge}(\sigma_\top, e_1(\sigma_1), e_2(\sigma_2)) =
			e_2(\F{merge}(\sigma_\top, e_1(\sigma_1), \sigma_2 ))$\footnote{Note that we are abusing the $\F{rc}$ notation slightly, since $\F{rc}$ is a
		relation over operations $O$, but we are considering it over operation
		instances (i.e. events)}
\end{itemize}

Property (P1$'$) would then allow us to establish the
required linearization order. Property (P4) also needs to be altered due to the presence of non-commutative update operations. 
We modify (P4) to enforce commutativity for non-$\F{rc}$ related events, which gives us flexibility to include such events in any order while constructing the linearization sequence:

\begin{itemize}
	\item[(P4$'$)] $(e_1,e_2) \notin \F{rc} \wedge (e_2,e_1) \notin \F{rc}  \implies e_1(e_2(\sigma)) = e_2(e_1(\sigma))$
\end{itemize}

However, we now face another major challenge: proving (P1$'$) for the OR-set
MRDT. For the counter MRDT, the operations and merge function used integer
addition and subtraction, which commute with each other. But for the OR-set,
$\F{add}_a$ uses set union, while $\F{merge}$ uses set difference and
intersection, which do not commute in general.  Hence, (P1$'$) does not hold
for arbitrary $\sigma_\top,\sigma_1,\sigma_2$.

To illustrate this concretely, consider the same execution of
Fig.~\ref{fig:orset_exec}, except assume that the state $\sigma_\top$ of the LCA
$v_\top$ is $\{(a,1)\}$. Let us try to establish (P1$'$) for the merge of versions
$v_1$ and $v_2$. First, note that as per the OR-set $\F{rc}$, the antecedent of
(P1$'$) is satisfied, as $(e_1,e_2) \in \F{rc}$. Now, the RHS in the consequent
must contain the tuple $(a,1)$, since the event $e_2$ adds $(a,1)$ to the
result of the merge. Does the LHS also contain $(a,1)$? Expanding the
definition of merge in the LHS, $(a,1)$ will not be present in $(\sigma_\top
\cap e_1(\sigma_\top) \cap e_2(\sigma_\top))$ (because $(a,1) \notin e_1(\sigma_\top)$,
as $e_1$ removes $a$). Similarly, since $(a,1)$ is in $\sigma_\top$, it will not be
present in $e_2(\sigma_\top) \setminus \sigma_\top$. It will not be in $e_1(\sigma_\top)
\setminus \sigma_\top$, as $e_1$ removes $a$. To conclude, $(a,1)$ will not be
present in the LHS, thus invalidating the consequent of (P1$'$).

However, we note that this particular execution is actually spurious, because
the tuple $(a,1)$ in the LCA could only have been added by another $\F{add}_a$
operation whose timestamp is the same as $e_2$. But this is not possible as the
data store ensures that timestamps are unique across all events. In the general
case, we would not be able to show (P1$'$) for OR-set because the tuple $(a,t)$
being added by the $\F{add}_a$ operation (event $e_2$) could also be present in
the LCA state. However, this situation cannot occur.

Thus, it is possible to show (P1$'$) for all \emph{feasible} states $\sigma_\top,
\sigma_1, \sigma_2$ that may occur during an actual execution. In
the case of OR-set, there are two arguments which are required to infer this:
(i) timestamps are unique across all events and (ii) if a tuple $(a,t)$ is
present in the state $\sigma$, then there must have been an $\F{add}_a$
operation with timestamp $t$ in the history of events leading to $\sigma$.
While the first argument is a property of the data store, the second argument
is an invariant linking a state with the history of events leading to that
state. Such arguments are in general hard to infer, and would also change
across different MRDTs. We now present our second major observation which
allows us to automatically verify (P1$'$) for feasible states without requiring
invariants like argument (ii) linking MRDT states and events.

\subsection{Verification using Induction on Event Sequences}
\label{subsec:induction}
In order to show property (P1$'$) for an MRDT implementation, we need to consider the feasible states which would be given as input to the merge function during an actual execution. We observe that we can leverage the RA-linearizability of the MRDT implementation, and hence characterize these feasible states by sequences of MRDT update operations (more precisely, events corresponding to update operation instances). We can now use induction over these sequences to establish property (P1$'$). Note that the input states to merge may themselves have been obtained through prior merges, but we can inductively assume that these prior merges resulted in correct linearizations. Since merge takes as input three states ($\sigma_\top, \sigma_1, \sigma_2$), we need to consider three sequences that led to these states and induct on all the three separately.


Concretely, let $\pi_\top$ be a sequence of events which when applied on the
initial MRDT state $\sigma_0$ results in the state $\sigma_\top$. Since the LCA
state always contains events which are common to the states $\sigma_1$ and
$\sigma_2$, $\pi_\top$ will be the common prefix of the sequences leading to both
$\sigma_1$ and $\sigma_2$. We consider the
sequences $\pi_1$ and $\pi_2$ that consist of the local events which when
applied on $\sigma_\top$ led to $\sigma_1$ and $\sigma_2$ respectively. Fig.
\ref{fig:induction} depicts the situation. Notice that the last two events on
each replica before the merge are fixed to be $e_1$ and $e_2$, which would be
related by the $\F{rc}$ relation, as per the requirement of property (P1$'$).

\vspace{-2em}
\begin{align}
	& \F{merge} (\sigma_0, e_1(\sigma_0), e_2(\sigma_0)) = e_2 (\F{merge} (\sigma_0, e_1(\sigma_0), \sigma_0)) \label{eq:1}\\
	&\F{merge} (\sigma_\top, e_1(\sigma_\top), e_2(\sigma_\top)) = e_2 (\F{merge} (\sigma_\top, e_1(\sigma_\top), \sigma_\top))\nonumber \\
	&\implies \F{merge} (e(\sigma_\top), e_1(e(\sigma_\top)), e_2(e(\sigma_\top))) = e_2 (\F{merge} (e(\sigma_\top), e_1(e(\sigma_\top)), e(\sigma_\top)))  \label{eq:2}
\end{align}
\vspace{-1.5em}

\begin{wrapfigure}{r}{0.28\textwidth}
	\vspace{-1.5em}
	\begin{center}
		\includegraphics[angle=0, width=0.9\linewidth]{induction}
	\end{center}
	\vspace{-1em}
	\caption{Induction on event sequences}
	\vspace{-1em}
	\label{fig:induction}
\end{wrapfigure}

We first induct on the sequence $\pi_\top$ which leads to the state $\sigma_\top$.
For this, we assume that $\pi_1 = \pi_2 = \epsilon$, and hence $\sigma_\top =
\sigma_1 = \sigma_2 = \pi_\top(\sigma_0)$. We also assume the antecedent of
property (P1$'$), i.e. $(e_1,e_2) \in \F{rc}$, and hence our goal is to show
its consequent. For the OR-set, $e_1$ will be a $\F{rem}_a$ event, while $e_2$
will be an $\F{add}_a$ event (say with timestamp $t$).

Eqn. \eqref{eq:1} is the base-case of the induction (where $\pi_\top = \epsilon$),
and this can be now directly discharged since $\sigma_0$ is an empty set, and
hence clearly won$'$t contain $(a,t)$. Eqn. \eqref{eq:2} is the inductive case,
which assumes that (P1$'$) is true for some LCA state $\sigma_\top$, and tries to
prove the property when one more update operation (signified by the event $e$) is
applied on the LCA (and also on both $\sigma_1$ and $\sigma_2$, since LCA
operations are common to both states to be merged). This can also be
automatically discharged with the property that events $e,e_1,e_2$ have
different timestamps. Intuitively, the inductive hypothesis establishes that
$(a,t) \notin \sigma_\top$, and since the timestamp of event $e$ is different from
$e_1$ and $e_2$, it cannot add $(a,t)$ to the LCA, thus preserving the property
that $(a,t) \notin e(\sigma_\top)$, thereby implying the consequent. This
completes the proof for property (P1$'$) for any arbitrary LCA state $\sigma_\top$
that may be feasible in an actual execution. A similar inductive strategy is used for proving property (P1$'$) for feasible states $\sigma_1$ and $\sigma_2$ (more details in \S \ref{sec:lemmas}).

\vspace{-0.5em}
\subsection{Intermediate Merges}
\label{subsec:inter}
In our linearization strategy for merges (given by properties (P1$'$-P4$'$)), we first considered the local update operations of each branch,
linearized them according to the conflict-resolution policy, and then applied
this sequence on the LCA. This effectively orders the update operations that led
to the LCA before the update operations local to each branch.

\begin{wrapfigure}{r}{0.35\textwidth}
	\vspace{-3em}
	\begin{center}
		\includegraphics[angle=0, width=1\linewidth]{inter_merge}
	\end{center}
	\caption{Intermediate merge}
	\vspace{-1em}
	\label{fig:inter_merge}
\end{wrapfigure}

However, in a Git-based execution model, due to a phenomenon known as
intermediate merges, it may happen that update operations of the LCA may need to be
linearized after update operations local to a branch. To
illustrate this, consider an execution of the OR-set MRDT as shown in Fig.
~\ref{fig:inter_merge}. There are 3 operations and 2 merges being performed in
this execution, with the events $e_1,e_3$ at replica $r_1$ and event $e_2$ at
replica $r_2$.

Instead of merging with the latest version $v_3$ at replica $r_1$, replica
$r_2$ first merges with an intermediate version $v_1$ to generate the version
$v_4$. Next, this version $v_4$ is merged with the latest version $v_3$ of
replica $r_1$. However, note that for this merge, the LCA will be version
$v_1$. This is because the set of events associated with version $v_3$ is
$\{e_1,e_3\}$, while for version $v_4$, it is $\{e_1,e_2\}$. Hence, the set of
common events among both versions would be $\{e_1\}$, which corresponds to the
version $v_1$. Indeed, in the version graph, both $v_1$ and $v_0$ are ancestors
of $v_3$ and $v_4$, but $v_1$ is the lowest common ancestor\footnote{in \S \ref{sec:lin},
we will formally prove that the LCA of two versions according to the version
graph contains the intersection of events in both versions.}.

In Fig. \ref{fig:inter_merge}, we have also provided the linearization of
events associated with each version. Notice that for version $v_4$, which is
obtained through a merge of $v_1$ and $v_2$, the conflict resolution policy of
the OR-set linearizes $e_2$ before $e_1$. Now, for the merge of $v_3$ and
$v_4$, we have a situation where a local event ($e_2$ in $v_4$) needs to be
linearized before an event of the LCA ($e_1$ in $v_1$). This does not fit our
linearization strategy. Let us see why. If we were to try to apply (P1$'$), it would linearize $e_1$ after $e_3$,
since these are the last operations in the two states to be merged and the
conflict resolution policy orders $\F{add}_a (e_1)$ after $\F{rem}_a (e_3)$.
However, in the execution, $e_1$ and $e_3$ are causally related, i.e. $e_1$
occurs before $e_3$ on the same replica, and hence they should be linearized in
that order. Intuitively, property (P1$'$) does not work because it does not consider the possibility that the last event in one replica could be visible to the last event in another replica, and hence the linearization must obey the visibility relation.

In order to handle this situation, we consider another algebraic property (P1-1), which explicitly forces visibility relation among the last events by making one of them part of the LCA:

\begin{itemize}
	\item[(P1-1)] $\F{merge}(e_1(\sigma_0),e_3(\sigma_1), e_1(\sigma_2)) = e_3(\F{merge}(e_1(\sigma_0), \sigma_1, e_1(\sigma_2 )))$
\end{itemize}

Note that events in the LCA are visible to events on both replicas being merged. Hence, by having the same event $e_1$ in both the first and third argument to $\F{merge}$ in the LHS, $e_3$ would have to be linearized after $e_1$ to respect the visibility order, thus over-riding the $\F{rc}$ ordering among them. Property (P1-1) can be directly applied to the execution in Fig.
\ref{fig:inter_merge} for the merge of $v_3$ and $v_4$ (with $\sigma_0$ as the
state of $v_0$, $\sigma_1$ as the state of $v_1$ and $\sigma_2$ as the state of
$v_2$), constructing the correct linearization.

We will revisit the example in Fig. \ref{fig:inter_merge} and properties (P1$'$) and (P1-1) in a more formal setting in \S \ref{sec:lemmas}, renaming them as $\textrm{\sc{BottomUp-2-OP}}$ and $\textrm{\sc{BottomUp-1-OP}}$. We will also identify the conditions under which these properties can guarantee the existence of a correct linearization.