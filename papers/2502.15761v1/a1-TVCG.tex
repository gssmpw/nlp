%\documentclass[lettersize,journal]{IEEEtran}
\documentclass[10pt,journal, compsoc]{IEEEtran}
\usepackage{amsmath,amsfonts}
% \usepackage{algorithmic}
% \usepackage{algorithm}
% \usepackage{array}
\usepackage[caption=false,font=normalsize,labelfont=sf,textfont=sf]{subfig}
\usepackage{textcomp}
\usepackage{stfloats}
\usepackage{url}
\usepackage{verbatim}
\usepackage{graphicx}
\usepackage{cite}
\hyphenation{op-tical net-works semi-conduc-tor IEEE-Xplore}
% updated with editorial comments 8/9/2021
 \usepackage{times}                     % we use Times as the main font
\renewcommand*\ttdefault{txtt}         % a nicer typewriter font

%% Only used in the template examples. You can remove these lines.
\usepackage{tabu}                      % only used for the table example
\usepackage{booktabs}                  % only used for the table example
\usepackage{lipsum}                    % used to generate placeholder text
\usepackage{mwe}                       % used to generate placeholder figures
\usepackage{multirow}
%% We encourage the use of mathptmx for consistent usage of times font
%% throughout the proceedings. However, if you encounter conflicts
%% with other math-related packages, you may want to disable it.
\usepackage{mathptmx}                  % use matching math font
\usepackage{glossaries}
\usepackage{graphicx}
%\usepackage{graphicx}
\usepackage{xspace}                    % used to manage spacing at the end of macros (commands.tex)
\usepackage{svg}
\usepackage{verbatim}

\usepackage{placeins}
\usepackage{adjustbox}
\usepackage{mathptmx}                  % use matching math font
\usepackage{glossaries}
\usepackage{graphicx}
%\usepackage{graphicx}
\usepackage{xspace}                    % used to manage spacing at the end of macros (commands.tex)
\usepackage{svg}
\usepackage{verbatim}

\usepackage{placeins}
\usepackage{adjustbox}

  \usepackage{enumitem}
 \usepackage{rotating}
\usepackage{longtable}

\usepackage{orcidlink}


 
%%%%%%%%%%%%%%%%%%%%%%%%%%%%%%%%%%%%%%%%%%%%%%%%%%%%%%%%%%%%%%%%%%%%%%%%%%%%%%

%% Beautiful mathematics
\usepackage{amsmath, amssymb, amsfonts} 
\usepackage{nicefrac}
\usepackage{mathtools}
\usepackage{bm, bbm}
\usepackage[scr=boondoxo,scrscaled=1.05]{mathalfa}

%% References in the correct format 
%\usepackage[square,numbers]{natbib}
%\def\bibfont{\footnotesize} % fix to have the same font size as without natbib

\usepackage[sort, compress, space]{cite}            


%% Enumerate nicely 
\usepackage{enumitem}

%% Different color comments and commenting large parts of the text
\usepackage{xcolor}
\usepackage{comment}
\usepackage{soul}

%% Hyper references
\usepackage{hyperref}
\usepackage{cleveref}
%\usepackage[numbers]{natbib}

\usepackage{tikz}
%\usepackage{thm-restate}
%% Appendix package
%\usepackage{appendix}

%% Random text to test spacing 
\usepackage{blindtext}

\usepackage{afterpage}

\usepackage{algorithm, algorithmic}    



\usepackage{dsfont}

\usepackage{tikz}
\usepackage{graphicx}
\usepackage{tikzscale}
\usepackage{pgfplots}
\pgfplotsset{compat=newest}
\usepackage{xfrac}

\usepackage{thm-restate}

%\usepackage{subcaption}

\usepackage{balance}

\usepackage{cite}
\usepackage{amsmath,amssymb,amsfonts}
\usepackage{balance}
\usepackage{algorithmic}
\usepackage{graphicx}
\usepackage{textcomp}
\usepackage{xcolor}
\usepackage{amsmath}
\usepackage{amssymb}
\usepackage[mathscr]{euscript}
\usepackage{comment}
\usepackage{xcolor}
\usepackage{enumitem} 
\usepackage{amsthm}


%%% REVIEW
\newcommand{\tocite}{{\color{red}CITE} }
\newcommand{\toref}{{\color{red}REF} }

%%% LOGO
\newcommand{\usc}{\raisebox{-1pt}{\includegraphics[height=0.8em]{figures/usc_logo.png}}}
\newcommand{\vuam}{\raisebox{-1pt}{\includegraphics[height=0.8em]{figures/vu_logo.png}}}

%%% SIGNS and SYMBOLS
\newcommand{\grad}{\texttt{grad-CROP}}
\newcommand{\att}{\texttt{att-CROP}}
\newcommand{\seg}{\texttt{seg}}
\newcommand{\clip}{\texttt{clip-CROP}}
\newcommand{\sam}{\texttt{sam-CROP}}
\newcommand{\yolo}{\texttt{yolo-CROP}}
\newcommand{\hc}{\texttt{human-CROP}}
\newcommand{\zsvqa}{\texttt{ZSVQA}}
\newcommand{\vic}{\textbf{ViCrop}}
\newcommand{\xmark}{\text{\ding{55}}}
\newcommand{\cmark}{\text{\ding{51}}}
\newcommand{\success}{\texttt{\color{green} \cmark}}
\newcommand{\failure}{\texttt{\color{red} \xmark}}
\newcommand{\rel}{\texttt{rel-att}}
\newcommand{\gra}{\texttt{grad-att}}
\newcommand{\pgra}{\texttt{pure-grad}}
\newcommand{\relh}{\texttt{rel-att$^h$}}
\newcommand{\grah}{\texttt{grad-att$^h$}}
\newcommand{\pgrah}{\texttt{pure-grad$^h$}}


%%% Text Abb.
\makeatletter
\DeclareRobustCommand\onedot{\futurelet\@let@token\@onedot}
\def\@onedot{\ifx\@let@token.\else.\null\fi\xspace}

\def\aka{\emph{a.k.a}\onedot} \def\Eg{\emph{E.g}\onedot}
\def\eg{\emph{e.g}\onedot} \def\Eg{\emph{E.g}\onedot}
\def\ie{\emph{i.e}\onedot} \def\Ie{\emph{I.e}\onedot}
\def\cf{\emph{c.f}\onedot} \def\Cf{\emph{C.f}\onedot}
\def\etc{\emph{etc}\onedot} \def\vs{\emph{vs}\onedot}
\def\wrt{w.r.t\onedot} \def\dof{d.o.f\onedot}
\def\etal{\emph{et al}\onedot}
\makeatletter



\definecolor{myred}{HTML}{FF8577}
\definecolor{mygreen}{HTML}{0FA958}
\definecolor{myblue}{HTML}{1982C4}
\definecolor{codegreen}{rgb}{0,0.5,0}
\definecolor{codegray}{rgb}{0.5,0.5,0.5}
\definecolor{codepurple}{rgb}{0.07,0,0.53}
\definecolor{codered}{RGB}{189,41,0}
\definecolor{codecomment}{RGB}{153,153,153}
\definecolor{backcolour}{rgb}{0.96,0.96,0.96}
\definecolor{royalblue}{rgb}{0.0, 0.14, 0.4}
\definecolor{egyptianblue}{rgb}{0.06, 0.2, 0.65}
\definecolor{royalazure}{rgb}{0.0, 0.22, 0.66}
\definecolor{portlandorange}{rgb}{1.0, 0.35, 0.21}
\definecolor{sienna}{RGB}{183,105,68}
\definecolor{saddlebrown}{RGB}{139,69,19}
\definecolor{mediumbrown}{RGB}{83,41,11}
\definecolor{darkbrown}{RGB}{58,28,7}
\hypersetup{
    colorlinks=true,
    linkcolor=sienna,
    urlcolor=royalblue,
    citecolor=royalblue,
}  
\makeglossaries
\begin{acronym}
\acro{gan}[GANs]{Generative Adversarial Networks}
\acro{rl}[RL]{Reinforcement Learning}
\acro{pae}[PAE]{Periodic Autoencoder}
\acro{fld}[FLD]{Fourier Latent Dynamics}
\acro{ppo}[PPO]{Proximal Policy Optimization}
\acro{fft}[FFT]{Fast Fourier Transform}
\acro{pca}[PCA]{Principal Component Analysis}
\acro{dfm}[DFM]{Deep Fourier Mimic}
\acro{dof}[DoF]{Degrees of Freedom}
\acro{mlp}[MLPs]{Multi-Layer Perceptrons}
\end{acronym}


\begin{document}
\title{\dk{LoXR: Performance Evaluation of Locally Executing LLMs on XR Devices}}
%\title{LLMs on XR (LoXR): Performance Evaluation of LLMs Executed Locally on XR Devices }
  \vskip -0.365cm
\author{ Dawar Khan$\orcidlink{0000-0001-5864-1888}$, Xinyu Liu , Omar Mena\orcidlink{https://orcid.org/0009-0002-6564-0175}, Donggan Jia\orcidlink{https://orcid.org/0000-0002-1358-8718}, lexandre Kouyoumdjian\orcidlink{https://orcid.org/0009-0005-0915-369X}, Ivan Viola\orcidlink{https://orcid.org/0000-0003-4248-6574}

\IEEEcompsocitemizethanks{ 
\IEEEcompsocthanksitem Dawar~Khan, Omar Mena, Donggang Jia, Alexandre Kouyoumdjian, Ivan~Viola are with King Abdullah University of Science and Technology (KAUST), Saudi Arabia. E-mail: \{xinyu.liu, dawar.khan, omar.mena, donggang.jia, alexandre.kouyoumdjian, ivan.viola\}@kaust.edu.sa. 
\IEEEcompsocthanksitem Xinyu Liu is with King Abdullah University of Science and Technology (KAUST), Saudi Arabia, and also with University of Electronic Science and Technology of China, Chengdu, China.
\IEEEcompsocthanksitem  D. Khan and X. Liu are joint first authors with equal contributions.}
\thanks{Manuscript received MM dd, YYYY; revised MM dd, YYYY.\\ (Corresponding author: Dawar Khan.)}}

        % <-this % stops a space
% \thanks{This paper was produced by the IEEE Publication Technology Group. They are in Piscataway, NJ.}% <-this % stops a space
% \thanks{Manuscript received April 19, 2021; revised August 16, 2021.}}

% The paper headers
\markboth{Arxiv: Feb.~2025}%
{Shell \MakeLowercase{\textit{et al.}}: A Sample Article Using IEEEtran.cls for IEEE Journals}

% \IEEEpubid{0000--0000/00\$00.00~\copyright~2021 IEEE}
% Remember, if you use this you must call \IEEEpubidadjcol in the second
% column for its text to clear the IEEEpubid mark.

\IEEEtitleabstractindextext{ 
\begin{abstract} 
\begin{abstract}
Role-Playing Agent (RPA) is an increasingly popular type of LLM Agent that simulates human-like behaviors in a variety of tasks. 
However, evaluating RPAs is challenging due to diverse task requirements and agent designs.
This paper proposes an evidence-based, actionable, and generalizable evaluation design guideline for LLM-based RPA by systematically reviewing $1,676$ papers published between Jan. 2021 and Dec. 2024.
Our analysis identifies six agent attributes, seven task attributes, and seven evaluation metrics from existing literature.
Based on these findings, we present an RPA evaluation design guideline to help researchers develop more systematic and consistent evaluation methods.

\end{abstract}


% to synthesize what agent attributes and task attributes prior literature have considered influence the selection of evaluation metrics, as well as the relationships between these factors.
% For each agent attribute and task category, we summarize its distinct associations with RPLA's evaluation metrics, providing practical guidance on comprehensive based on their RPLA's design. Additionally, we explore the  between agent attributes and downstream tasks to support researchers in refining RPLA design choices.
\end{abstract}
\begin{IEEEkeywords}
Extended reality, large language models, human AI interactions,  performance evaluation, conversational user interfaces
\end{IEEEkeywords}
 } 
\maketitle

% \begin{abstract}
% This document describes the most common article elements and how to use the IEEEtran class with \LaTeX \ to produce files that are suitable for submission to the IEEE.  IEEEtran can produce conference, journal, and technical note (correspondence) papers with a suitable choice of class options. 
% \end{abstract}

% \begin{IEEEkeywords}
% Article submission, IEEE, IEEEtran, journal, \LaTeX, paper, template, typesetting.
% \end{IEEEkeywords}

%\section{Introduction}
%\IEEEPARstart{T}{his} 

 \IEEEdisplaynontitleabstractindextext 
 \IEEEpeerreviewmaketitle
%\maketitle
% % Use this for peer review papers
% \IEEEpeerreviewmaketitle
\IEEEraisesectionheading{\section{Introduction}}




%\IEEEraisesectionheading{\section{Introduction}}

The increasing reliance on LLMs for multimodal tasks across far-reaching sectors such as healthcare, finance, and manufacturing underscores the need to assess the accuracy and reliability of the information they generate. Vision-Language Models (VLM) have achieved state-of-the-art (SoTA) performance on Visual Question-Answering (VQA) benchmarks, and these models often utilize Retrieval-Augmented Generation (RAG) to maintain factual accuracy and relevance in a dynamic information environment. However, this has led to uncertainty in the information the LLM bases its answer on, as it may choose between parametric memory and retrieved sources. When models rely on memorized information instead of dynamically retrieving information, they may inadvertently propagate outdated or incorrect information, causing serious legal and ethical risks and undermining trust and reliability in AI systems \citep{huang2023survey}.
% The ability to strike a balance between generalization and specialization in AI systems is therefore crucial for ensuring the safe, reliable use of these technologies in real-world applications.

Despite these concerns, the way that Vision-Language models (VLMs) memorize and retrieve information, particularly in complex multimodal tasks, remains under-explored. Current research often focuses on either the general capabilities of large language models (LLMs) or the specialized retrieval mechanisms in retrieval augmented generation systems (RAG) \citep{incontext_rag,chen_murag_2022,liu_universal_2023}. Particularly in the context of multimodal retrieval and multihop reasoning, few studies analyze the tradeoff between finetuning for specialized tasks and zero-shot prompting for general-purpose vision-language capabilities. A lack of consensus on how to approach this tradeoff motivates the development of measures to quantify reliance on parametric memory, as well as metrics for quantifying the potential performance impact of extending LLMs with RAG systems.

To address this gap, we investigate how multimodal QA models balance accuracy with memorization on the WebQA benchmark. We compare finetuned multimodal systems against zero-shot VLMs, analyzing how retrieval performance influences QA accuracy. In particular, we focus on cases where retrieval fails, allowing us to measure reliance on parametric memory through two proposed metrics---the \ppr (\PPR) which quantifies how much model accuracy is influenced by retrieval quality, contrasting performance in best-case versus worst-case retrieval scenarios, and the \ucr (\UCR) which measures how often correct QA responses are generated when the retriever fails, providing a proxy for memorization.

To enable this analysis, we make several methodological contributions. For the finetuned QA models, we investigate Vision-Transformer (ViT) architectures, which allow for multihop reasoning over multiple sources. To investigate the impact of retrieval performance on trained LMs, we propose a variable-input Fusion-in-Decoder (FiD) model \cite{tanaka_slidevqa_2023, nlvr2}, building upon the VoLTA architecture \citep{pramanick_volta_2023}. For the zero-shot case, we build upon previous research on In-Context Retrieval \citep{incontext_rag} by demonstrating that LLMs such as GPT-4o are capable of performing the final ranking step of the retrieval process. In doing so, we find that GPT-4o, a general-purpose LLM, achieves SoTA performance on the WebQA task, outperforming existing finetuned RAG models by a significant margin (7\% higher accuracy). 

Crucially, our results reveal that while retrieval-augmented models reduce memorization, the training paradigm plays an important role. Finetuned models exhibit higher reliance on parametric memory, whereas zero-shot RAG approaches have lower memorization scores at the cost of accuracy. This suggests that while retrieval modules may mitigate the risks associated with outdated or incorrect information, SoTA performance requires that they be coupled with specialized QA models. Our memorization measures contribute to the development of transparent and reliable AI systems, particularly in applications where the sourcing of up-to-date, factual information is critical.



% We investigate the impact of question complexity on the ability of these models to integrate multiple data sources—such as images, text, and external retrievers—and produce coherent and accurate answers. We also explore whether in-context retrieval can be a viable alternative to traditional retrieval-augmented systems, offering a more streamlined approach to multimodal QA.

% To achieve this, we first compare zero-shot prompting multimodal LLMs with finetuned multimodal systems. We evaluate both types of models on the WebQA benchmark, a dataset designed for complex question answering that requires reasoning across both image and text sources. For the finetuned models, we use a Fusion-in-Decoder (FiD) architecture, which allows for multihop reasoning over multiple sources. Additionally, we introduce the concept of In-Context Retrieval Language Modeling (RLM), where the LLM itself performs retrieval tasks without the need for external retrievers. This method builds upon existing research in in-context learning  and aims to explore the viability of LLMs retrieving relevant sources and generating accurate answers directly from their context window.

% In order to investigate source utilization in finetuned multimodal models and LLMs, three lines of inquiry are established; 
% \begin{itemize}
%     \item Study 1: retrieval vs QA performance on webQA (motivating example, does QA answer correctly even with incorrect sources?)
%     \item Study 2: performance on adversarial examples where parametric knowledge would be incorrect by design
%     \item Study 3: improving performance on adversarial examples by fine-tuning (i.e model robustness)
% \end{itemize}

% Note, there is one weakness in this plan which is tying in the work we've already done. 
% If we added something from adversarial generation to the retrieval experiment (like a combination of study 1 + 3) it would be complete. So for instance we could try fine-tuning the retriever with adversarial examples (and not just the QA model)

% \begin{figure}
%     \centering
%     \includegraphics[width=0.95\linewidth]{figures/segmentation/webqa_segment_infill.png}
%     \caption{Example of the segmentation substitution pipeline from the WebQA task.}
%     % d5c76d760dba11ecb1e81171463288e9
%     \label{fig:seg_sub_pipeline}
% \end{figure}



% Retrieval augmented generation (RAG) with zero-shot prompting and fine-tuning Large Language Models (LLMs) have become the go-to methods for tasks relying on information retrieval and text generation. In many cases the LLMs parametric memory can sufficiently generalize to answer questions without being provided with retrieval mechanisms for out-of-domain knowledge. However, LLMs often hallucinate and provide wrong information in certain scenarios. This problem is amplified even further on open-domain Question Answering (QA) tasks involving multiple modalities. Grounded text generation using retrieved sources \citep{lewis2021retrievalaugmented} has been extensively studied for text-to-text QA tasks, but its application in multimodal settings has not been studied as much.


% Multimodal reasoning and question answering have gained prominence in recent research endeavors, with an increasing emphasis on handling various forms of data, particularly text and images. In this study, we address a specific gap in the existing literature by focusing on the development of a versatile multihop model capable of accommodating varying numbers of input images.

% Our motivation for this research lies in the growing complexity of answering questions using information on the web, where the challenge of navigating the open-domain setting is further complicated by the presence of multiple modalities and sometimes requires reasoning over multiple sources. WebQA is an ideal dataset on which to compare performance of finetuned RAG systems against general purpose LLMs; it is multimodal, with correct answers requiring reasoning over image and text sources. It is multihop, requiring a complex reasoning process over multiple sources. Finally, WebQA questions from different categories can be broken down into subdomains to analyze performance over domains of varying cardinality.

% Motivated by the real-world challenges of building retrieval and question answering (QA) systems, we design and finetune a closed domain, multimodal, multihop QA model, that is capable of reasoning over a varying number of sources taken as input from an external retriever module. This research contributes to the relatively underexplored domain of multihop reasoning across various input sources and modalities. Our goal is to explore the challenges posed by these scenarios and develop strategies that enable QA models to retrieve relevant information, conduct logical or numerical reasoning across diverse modalities, and generate coherent responses in natural language. To our knowledge, this is the first application of the Fusion-in-Decoder (FiD) architecture \cite{tanaka_slidevqa_2023, nlvr2} that is shown to work with a variable number of inputs, enabling multi-hop reasoning over sources.

% In-Context Learning refers to the ability of LLMs to perform any task by simply providing examples in the input prompt \citep{dong2022survey,min2022rethinking}. Inspired by this research, we propose a method to use the LLM itself as a multimodal retriever, potentially eschewing the requirement of a distinct retrieval module, thereby allowing the design of simpler retrieval-augmented QA systems. We dub this method In-Context Retrieval Language Modeling (RLM). To the best of the authors knowledge, In-Content RLM is disparate from other retrieval augmented approaches which utilize external retrieval modules \citep{incontext_rag,chen_murag_2022,liu_universal_2023}. Despite being a natural extension of In-Context learning, In-Context RLM has not yet been studied empirically.

% To expand on our contribution of In-Context Retrieval, this stems from the well-researched in-context learning of LLMs. In-context learning is the ability of a model to perform any task given a sufficient context window \citep{dong2022survey,min2022rethinking}. Such tasks could include retrieval and ranking, but typically, the go-to solution for tasks requiring retrieval has been RAG. To the best of the authors knowledge, In-Context Retrieval is distinct from In-Context Retrieval Augmented Language Modelling (RALM), and despite being a natural extension of In-Context learning, In-Context Retrieval has not yet been shown empirically.

% Finally, we explore the tradeoff between using zero-shot prompting LLMs and the fine-tuning approach. While we find that, overall, GPT-4o obtains SoTA performance on the WebQA task, outperforming the accuracy of existing finetuned RAG approaches by 7\%, finetuned approaches still perform better on more restricted subdomains\footnote{``In-Context RLM" @ \url{https://eval.ai/web/challenges/challenge-page/1255/leaderboard/3168}}. Finally, we validate that GPT-4o is relying on retrieval abilities to solve the task; we find that GPT-4o is capable of retrieving relevant sources in the presence of distractors and furthermore, when GPT-4o fails to retrieve correct sources, it answers incorrectly 75\% of the time, meaning that it is not relying on parametric memory for this task.

% \paragraph{Contributions}
% Based on our experimentation and analysis on the WebQA benchmark, we make the following contributions:
% \begin{itemize}
%     \item Propose a new architecture for multimodal multihop QA that takes variable number of input sources inspired by the Fusion-in-Decoder method.
%     \item Comparison of general purpose LLMs vs specialized models on the WebQA benchmark.
%     \item Observation of In-Context Multimodal Retrieval abilities of GPT-4o and that it does not rely on parametric memory for multimodal QA.
%     \item Analysis of relationship between retrieval and QA task performance.
%     \item Analysis of task and query complexity on the performance of retrieval and QA tasks.
% \end{itemize}
















% Throughout this paper, we will present our methodology, experiments, and findings, emphasizing our approach to multihop reasoning over varying numbers of input images. We believe that our work contributes to a deeper understanding of multimodal reasoning and has the potential to enhance the capabilities of question-answering systems in the intricate, multimodal landscape of web-based information.
\section{Related Work}
\label{sec:related-work}

We review related work with respect to two major research directions. The first research direction is exploring the use of LLMs on XR devices, and the second research direction explores how to make LLMs directly deployable on low-resource hardware.
% \h{Research on LLMs:}\\
% \cite{Fan2024ReviewLLMs}\\
% \cite{meta2023introducingLLaMA}\\
% \cite{Brown2020NEURIPSLMareFewshotLearner}\\
% \h{On-device LLMs:}\\
% mobileLLM\\
% \h{LLM-based applications on VR/XR}\\
% \h{AI and XR:}\\
% \cite{Bozkir2024Review:LLMXR}
% \cite{HirzleXRandAIReview}\\

\subsection{LLM-powered XR Applications}
Owing to the powerful semantic understanding and extensive general knowledge of LLMs, numerous studies have explored their application in assisting various tasks within XR scenarios. 

Torre \etal~\cite{de2024llmr} introduced \emph{LLMR}, a framework that leverages LLMs for the real-time creation and modification of interactive mixed reality experiences, enabling tasks such as generating new content or editing existing works on VR/AR devices. Jia \etal~\cite{Jia2024_1} developed the \emph{VOICE} framework, which employs a two-layer agent system for conversational interaction and explanation in scientific communication, with a prototype deployable on VR devices. Kurai \etal~\cite{kurai2024magicitem} proposed \emph{MagicItem}, a tool that allows users with limited programming experience to define object behaviors within the metaverse platform. Zhang \etal~\cite{zhang2024odoragent} introduced \emph{OdorAgent}, which combines an LLM with a text-image model to automate video-odor matching. Yin \etal~\cite{yin2024text2vrscene} identified potential limitations in LLM-based automated systems and proposed the systematic framework \emph{Text2VRScene} to address them. Chen \etal~\cite{chen2024supporting} leveraged the extensive capabilities of LLMs in context perception and text prediction to enhance text entry efficiency by reducing manual keystrokes in VR scenarios. Giunchi \etal~\cite{giunchi2024dreamcodevr} developed \emph{DreamCodeVR}, a tool designed to help users, regardless of coding experience, create basic object behavior in VR environments by translating spoken language into code within an active application. Wan \etal~\cite{wan2024building} presented an LLM-based AI agent for human-agent interaction in VR, involving GPT-4 to simulate realistic NPC behavior, including context-aware responses, facial expressions, and body gestures.

As they stand, these previous contributions still require the deployed devices to connect to a cloud server for LLM inference, raising concerns about user privacy, latency, costs, and internet access requirements. In our work, we evaluate the local inference performance of various LLMs and propose a prototype capable of using local LLM inference.

\subsection{On-device LLMs}
Running LLMs on edge devices, commonly called on-device LLMs, has garnered significant research interest due to their advantages in enhancing privacy, reducing latency, and operating without the need for internet connectivity. Because of the limited memory and computing capabilities, on-device LLMs usually require resource-efficient
LLM deployment~\cite{qu2024mobile}: a trade-off between performance and model size.

Cheng \etal~\cite{cheng2023optimize} introduced the \emph{SignRound} method, which leverages signed gradient descent to optimize both rounding values and weight clipping. This approach achieves outstanding performance in 2- to 4-bit quantization while maintaining low tuning costs and eliminating additional inference overhead. Ma \etal~\cite{ma2023llm} developed the \emph{LLM-Pruner} method, which uses structural pruning to selectively remove non-essential coupled structures based on gradient information, effectively preserving the core functionality of the LLM. Their results demonstrate that the compressed models continue to perform well in tasks such as zero-shot classification and generation. Gu \etal~\cite{gu2024minillm} introduced a novel knowledge distillation method that compresses LLMs into smaller models, resulting in the student model \emph{MINILLM}. \emph{MINILLM} demonstrates superior performance compared to baseline models, producing more precise responses with reduced exposure bias, improved calibration, and enhanced long-text generation capabilities. Liu \etal~\cite{liu2024kivi} developed a tuning-free 2-bit KV cache quantization algorithm, named \emph{KIVI}, which independently quantizes the key cache per channel and the value cache per token. This approach enabled up to a 4× increase in batch size, resulting in a $2.35\times$ to $3.47\times$ improvement in throughput on real LLM inference workloads. Liu \etal~\cite{liu2024mobilellm} introduced \emph{MobileLLM}, which explores the importance of model architecture for sub-billion-parameter LLMs by leveraging deep and narrow architectures, combined with embedding sharing and grouped-query attention mechanisms. 

Although these studies primarily focused on reducing model size efficiently for deployment on various devices, they did not evaluate the models' performance in real-world on-device scenarios, which is the focus of our paper.


\section{Hierarchical Scene Synthesis with LLM}

Given a text description $t_r$ and the scene size $s_r$ as conditions, our approach is composed of three stages, as illustrated in Figure~\ref{pipeline}. First, we prompt the pre-trained LLM to generate the hierarchical structure with text descriptions. Second, we train a hierarchy-aware graph neural network to infer the relative placement coordinates between objects.
% the fine-grained relative placements between objects corresponding to the LLM-generated textual spatial relations. 
Third, we design a divide-and-conquer optimization which optimizes the sub-layout for each functional area and then arranges their placements to form the entire scene.  

% \vspace{-5pt}
\subsection{Hierarchical Structure Generation with LLM} 
Given the user requirement, the pre-trained LLM takes the constructed prompt as input and outputs structured text to describe the hierarchical scene representation, including the node attributes. The key challenge is to generate reasonable and informative spatial relations to specify the scene layout.

Although existing works define dense object relations to describe layouts, the more detailed the descriptions are, the more incorrect or self-contradictory results they make, due to the lack of spatial reasoning ability of the LLM. Therefore, in our approach, we require the LLM to generate a hierarchical structure to ground the objects and only the spatial relations between objects belonging to the same area, only to roughly specify their arrangements.

We construct the input prompt with three components: 1) a description of the LLM's role and task, including a brief definition of the hierarchical structure with the meaning of the nodes and connections; 2) a description of the preferred data format and pre-defined constraints, including the types of functional areas, possible anchor objects, and spatial relations; and 3) an example of a simple scene in the preferred format and the specific user requirements. We don't require the example to be selected corresponding to the user requirement, but only to demonstrate the output format. In this stage, the LLM generates textual descriptions and size attributes of the functional areas and objects, as well as the textual descriptions of spatial relations.

% \subsection{Data-Driven Inference of Relative Placements}
% \subsection{Hierarchy-Aware Network Inference}
\subsection{Fine-Grained Relative Placement Inference}
We propose a hierarchy-aware graph neural network to infer the fine-grained relative placements between correlated objects. The relative placements within each functional area exhibit a more compact and generalizable prior, allowing us to train a network to infer the placements for various scenes.

% As illustrated in Figure~\ref{pipeline}, given the LLM-generated hierarchy, we construct the input graph $G=(V,E)$ with the nodes as objects and edges connecting all the objects belonging to the same functional area. Although the input includes all the objects in the scene, the functional areas are isolated from each other. We use Linear embeddings for the object sizes $s_o$ and the relative placement coordinates $[p_e, \theta_e, d_e]$, where $d_e$ is a binary indicator of the alignment between two objects, and the pre-trained CLIP text encoder~\cite{radford2021learningtransferablevisualmodels} for descriptions of objects and spatial relations,
As illustrated in Figure~\ref{pipeline}, given the LLM-generated hierarchy, we construct the input graph $G=(O,E)$ with the nodes as objects and edges connecting all objects belonging to the same functional area. Although the input includes all objects in the scene, the functional areas are isolated from each other. We use Linear embeddings for the object sizes $s_o$ and the ground truth relative placement coordinates $[p_e, \theta_e, d_e]$, where $d_e$ is a binary indicator of the alignment between two objects, and the pre-trained CLIP text encoder~\cite{radford2021learningtransferablevisualmodels} for descriptions of objects $t_o$ and spatial relations $t_e$. They are organized as node features $h_o$ and edge features $h_e$, i.e.
%Note that the $[p_e, \theta_e, d_e]$ is only used during the training process. We incorporate this information into each node and edge of the $G$ to conceptualize it as the contextual graph $G_c$ with the node embedding $h_o$ and edge embedding $h_e$, 

\begin{equation}
\begin{aligned}
& h_o = [\mathrm{CLIP_t}(t_o), \mathrm{LINEAR(s_o)}], \\
& h_e = [\mathrm{CLIP_t}(t_e), \mathrm{LINEAR}(p_e, \theta_e, d_e)],
\end{aligned}
\end{equation}
which forms the contextual graph for the following network processing. Note that since we only have textual spatial relations between the anchor object and the others, we use all-zero vectors as the text embeddings for the edges without corresponding textual spatial relations (dotted arrows). 

We adopt the variational graph neural network~\cite{zhai2024commonscenes} for the contextual graph with the $h_o$ and $h_e$. Both the encoder and decoder are composed of several MLPs for 5 rounds of message passing, including $g_e^{(k)}$ for updating the edge features with connected node features in the $k$th round and $g_o^{(k)}$ for updating the node features with the 1-ring neighbor nodes, i.e. 
\begin{equation}
\begin{aligned}
h_{e_{i\xrightarrow{} j}}^{(k+1)}&=g_e^{(k)}(h_{o_i}^{(k)}, h_{e_{i\xrightarrow{} j}}^{(k)}, h_{o_j}^{(k)}) \\
h_{o_i}^{(k+1)} &=h_{o_i}^{(k)} + g_o^{(k)}(\mathrm{AVG}(h_{o_j}^{(k)}|o_j \in N_ \mathcal{G}(o_i))),
\end{aligned}
\end{equation}
where $e_{i\xrightarrow{} j}$ represents an edge connecting two objects $o_i$ and $o_j$, $N_\mathcal{G}(o_i)$ represents the set of neighbor nodes connected with object $o_i$. The encoder takes the contextual graph as input and outputs the graph with updated features, where the edge features (specifically the relative placement components of edge features, as shown in Figure~\ref{pipeline}) are parameterized as a Gaussian distribution. The decoder takes the updated graph as input and randomly samples from the Gaussian distribution. Finally, we use separate MLPs to decode the relative placement $[\hat{p}_e, \hat{\theta}_e, \hat{d}_e]$.

During training, we freeze the CLIP text encoder and update all other network layers. The loss function is 
\begin{equation}
L=L_{KL} + L_{ep} + L_{e\theta} + L_{ed},
\end{equation}
where $L_{KL}$ is the Kullback-Liebler divergence between the Gaussian distribution and
posterior distribution of the edge feature components. $L_{ep}$ is L1 loss on the relative positions $p_e$. $L_{e\theta}$ and $L_{ed}$ are cross-entropy loss on the discretized relative orientation angles and the binary alignment indicator.

\subsection{Divide-and-Conquer Layout Optimization} 

Given the hierarchical scene with the relative placements between correlated objects, we develop a divide-and-conquer optimization to solve for the final layout. Our solution includes a local optimization for each functional area and then a global optimization to organize the areas into scenes. This optimization produces reasonable and physically feasible layouts more effectively than a simple global optimization or iteratively optimizing each object's placements.

\begin{figure*}
\centering
\includegraphics[width=0.95\linewidth]{CameraReady/Figures/comparison.pdf}
\caption{The scenes generated from different approaches. The end-to-end data-driven approaches often produce infeasible object placements including overlap and out-of-boundary cases (red boxes). On the other hand, HOLODECK sometimes produces unreasonable results such as two nightstands on the same side of the bed (top row) or a tv stand on the left side of the sofa (second row). By contrast, our approach is able to more effectively produce reasonable and physically feasible scene layouts.}
\label{fig:comparison_topview}
\end{figure*}

\noindent \textbf{Local optimization.} For each functional area, we use local optimization to solve the object placements w.r.t. the bounding box of the functional area. The local optimization is formulated to minimize the objects' relative placements and those inferred by the network, with constraints to avoid object overlap and out-of-boundary, i.e.
\begin{equation}
\begin{aligned}
\min_{o'_i\in O} &  \sum_{o_i\in N_\mathcal{G}(o_a)} |\mathrm{REL}(o'_i, o'_a)-[p_{e_{i\xrightarrow{} a}},\theta_{e_{i\xrightarrow{} a}}]|, \quad \\
s.t. \quad & C_{overlap}(o'_i, o'_j), \quad \forall o_i,o_j \in A \\
& C_{OOB}(o'_i, s_a), \quad \forall o_i \in A 
\end{aligned}
\end{equation}
where $o'_i$ and $o'_a$ refers to the placements (center positions and orientations) w.r.t. the functional area of an object $o_i$ and the anchor object $o_a$, respectively. $\mathrm{REL}$ computes the relative placements between two objects and $[p_{e_{i\xrightarrow{} a}},\theta_{e_{i\xrightarrow{} a}}]$ is the relative positions between object $o_i$ and $o_a$ predicted by the network. $A$ represents the set of objects within the area. $C_{overlap}$ constrains the overlap between oriented bounding boxes of any two objects as small as possible, and $C_{OOB}$ aims to avoid the object boxes lying out of the area boundary, whose size $s_a$ is generated from pre-trained LLM in stage 1.


\noindent \textbf{Global optimization.} We then organize the areas to form scenes with global optimization. Each functional area takes the orientation of its anchor object as its own orientation. Based on observations in our daily life, the optimization is formulated to place the functional areas against the walls and far from each other with orientations pointing inside the scene, while avoiding object overlap and out-of-boundary:
\begin{equation}
\begin{aligned}
\min_{a_i} & \sum_{a_i} |\mathrm{D_w}(a_i, s_r)| - \sum_{a_i, a_j} |\mathrm{D_a}(a_i, a_j)|, \\
s.t. \quad & C_{overlap}(a_i, a_j), \quad \forall a_i,a_j \in S \\
& C_{OOB}(a_i, s_r), \quad \forall a_i \in S  
\end{aligned}
\end{equation}
where $a_i$ denotes area placement (center position and orientation). $D_w$ is the distance between back side of the area and the boundary of the scene. $D_a$ is the distance between two areas' bounding boxes. $S$ is the set of areas within the scene. $C_{overlap}$ and $C_{OOB}$ are same as in local optimization.

After the optimizations, we transform the coordinate systems to obtain object positions and orientations in the frame of scenes. Finally, we retrieve 3D object models from Objaverse~\cite{objaverseXL} and 3D-Front datasets~\cite{fu20213d} based on the CLIP scores, i.e. cosine similarity between object images and text embeddings. The object models are then scaled and placed according to the scene layouts.



\section{Experimental Setup}
\label{sec:exp:design}
\dk{This section describes the experimental setup and methodology for evaluating processing speed and error rates. The results are presented in \cref{sec:results}. Before diving into the detailed setup, we first discuss the two evaluation metrics: processing speed and error count.}\\
\noindent \textbf{Processing Speed:}
\dk{In this study, we measure the processing speed for each model-device pair \((m_i, d_j)\) by recording the time \(t_{ijk}\) taken by model \(m_i\) running on device \(d_j\) to process a string of length \(L\). The instantaneous processing speed, \(S_{inst.}\), expressed in tokens per second, is calculated as follows:}

\begin{equation}
\label{eq:instSpeed}
S_{inst.} = \frac{L}{t_{ij}}.
\end{equation}

\dk{However, instead of relying on a single run, we perform 5 runs for each model-device pair to ensure accuracy and measure stability. If the coefficient of variance (CV) across the 5 runs is less than \(40\%\), we calculate the 5-Run-Mean Speed (referred to as consistent speed, \(S_c\)) as:}

\begin{equation}
\label{eq:mean_speed}
S_c = \frac{1}{5} \sum_{k=1}^{5} S_{inst.}(k),
\end{equation}
\dk{where \( S_{\text{inst.}}(k) \) represents the instantaneous speed during the \( k^{\text{th}} \) run of each experiment, calculated via \cref{eq:instSpeed}.  This approach minimizes the impact of factors such as initialization delays, hardware fluctuations, or other environmental inconsistencies, providing a more reliable measure of processing speed.}

\noindent \textbf{Error Count:}
\dk{If the CV exceeds \(33\%\), the run is considered unstable, and the corresponding data point is excluded from further analysis. The count of such unstable runs is recorded as an error during the experiment.}

\dk{We conducted the following four types of experiments.}


\subsection{Models Quality Analysis}
\label{sec:modelsQA}
Due to the storage and computational limitations of edge devices, deploying \gls{llm} locally often requires a trade-off between model size and performance. To put our performance results in perspective, we first conducted an empirical analysis of the models we have used in this study. \dk{This section offers insights into the models evaluated in this work, presenting a comparative analysis. While this section is not directly related to device performance, we believe providing detailed information about the models is essential for understanding their capabilities and limitations.}
We select six benchmarks to evaluate each model and calculate different metrics for each run:

\begin{itemize}[noitemsep,leftmargin=8pt, topsep=3pt]
    \item Hellaswag~\cite{hellaswag}: commonsense reasoning tasks, focusing on selecting the most plausible ending to a given situation from multiple options.
    \item mmlu~\cite{hendrycks2020measuring}: evaluating a model’s knowledge and reasoning across 57 different subjects, ranging from elementary-level topics to advanced professional knowledge.
    \item ARC~\cite{clark2018thinksolvedquestionanswering}: evaluating model's ability to answer complex, grade-school science questions that require reasoning and problem-solving beyond simple fact retrieval.
    \item Truthful-qa~\cite{lin2021truthfulqa}: evaluate how accurately AI models generate truthful answers to questions, especially in cases where common misconceptions or false information could lead to incorrect responses.
    \item winogrande~\cite{ai2:winogrande}: large-scale benchmark for commonsense reasoning, specifically designed to test a model's ability to resolve ambiguous pronouns in sentences.
    \item WikiText-2~\cite{merity2016pointer}: containing over 100 million tokens, sourced from verified Wikipedia articles, designed to support research in language modeling and text generation tasks.

\end{itemize}

For the first five benchmarks, which consist of multiple-choice questions, we use accuracy as the evaluation metric. However, for WikiText-2, the task is to predict the probability of each word in a given text, reflecting the model's natural language understanding. Therefore, we use perplexity as the evaluation metric, defined as:

\begin{equation} \text{Perplexity}(P) = e^{-\frac{1}{N} \sum_{i=1}^{N} \log P(w_i)} \end{equation}

where $P(w_i)$ represents the probability assigned by the model to the $i$-th word, and $N$ is the total number of words in the sequence.

\subsection{Evaluating Performance Consistency}
\label{sec:setStability}
\dk{Some devices may experience performance degradation over time. In these experiments, we evaluate how processing speed fluctuates across multiple runs. For each model \( m_i \) and device \( d_j \), we perform \( N \) runs and compute the mean $(\mu_{ij})$ of the speed $(S_c)$, standard deviation \( \sigma_{ij} \), and coefficient of variation ($CV_{ij}$). We also record the error count, as well as the maximum and minimum speeds observed during each experiment.}\\
\dk{In our experiments, we set \( N = 20 \), \ie each model-device pair is tested 20 times to evaluate performance variance. The mean speed across \( N \) runs is calculated as:}

\begin{equation}
\label{eq:time_efficiency}
\mu = \frac{\sum_{i=1}^{N} S_c}{N},
\end{equation}
\dk{where \( S_c \) represents the 5-Run-Mean speed measured during each run, calculated as in~\cref{eq:mean_speed}. In other words, each \( S_c \) itself is the mean of 5 runs, ensuring that the measured value is not too affected by errors or hidden factors.}


To let the devices cool down and conduct each run in comparable circumstances, we leave each device idle for two minutes between each run of the same model. We find this to be adequate, but just to be safer, between different models, we leave each device idle for ten minutes. We noted that the Meta Quest is not very prone to heating up. Heating was more noticeable on the Vivo X100s Pro, so it is always placed on a metal plate for heat dissipation. This is not representative of actual use, but this device is included in our evaluation as a baseline, not as an example of an actual XR headset. Magic Leap features a cooling fan, as does Vision Pro.\\
\dk{For each test, we conducted both PP and TG evaluations with the string length set to 64, \ie \( \text{PP} = 64 \) and \( \text{TG} = 64 \). For each model-device pair, we calculated the standard deviation, mean, and CV. The results are presented in~\cref{sec:resultsStability}.}


% \dk{The processing speed is not recorded instantaneously after a single run. Instead, we perform 5 runs, and if the coefficient of variance (CV) is less than $33\%$, we compute the mean speed for these runs as Instentenious Speed (S_i):}
% \begin{equation}
% \label{eq:speed}
% S_i= \frac{\sum_{k=1}^{5} t_{ijk}}{5}.
% \end{equation}
% \dk{However, if the CV is $33\% $ or higher, the run is considered erroneous and excluded from the analysis.}


% \dk{In our experiments, we set \( N = 20 \), \ie each model-device pair was tested 20 times to evaluate performance variance. The mean is calculated as: 

% \begin{equation}
% \label{eq:time_efficiency}
% S_{mean} = \frac{\sum_{N=1}^{20}{\frac{1}{5} \sum_{k=1}^{5} t_{ijk}}}{N}
% \end{equation}

% For each test, we conducted both PP and TG evaluations with the string length set to 64, \ie \( \text{PP} = 64 \) and \( \text{TG} = 64 \).} The results are presented in~\cref{sec:resultsStability}.  

\subsection{Processing Speed and String Length}
\label{sec:settimePPTG}
\dk{String length, comprising both \textit{Prompt Length} and \textit{Size of the Token Set}, significantly affects performance. We evaluate the performance of LLMs on XR devices with varying string lengths to assess their processing speed. For each experiment, the processing speed is calculated as in~\cref{eq:mean_speed}.}\\
\dk{We conducted experiments for five different string lengths (64, 128, 256, 512, and 1024) under the following PP and TG setups.}
\begin{itemize}[noitemsep,leftmargin=8pt,topsep=0pt]
    \item \textbf{\gls{pp}:} The PP values were set to 64, 128, 256, 512, and 1024, while TG was fixed at 0, with all other parameters set to their default values.
    \item \textbf{\gls{tg}:} The TG values were set to 64, 128, 256, 512, and 1024, while PP was fixed at 0, with all other parameters set to their default values.
\end{itemize}
The results are shown in~\cref{sec:resultsPPTG}.
\subsection{\dk{Parallelization with Thread Count and Batch Size}}
\label{sec:setTTBT}
\dk{The parameters \textit{Batch Size} and \textit{Thread Count}, which control parallelization, also significantly affect performance. We evaluate the performance of LLMs on XR devices with varying batch sizes and thread counts to assess their processing speed, calculated using~\cref{eq:mean_speed}. The two tests conducted are described below:}
\begin{itemize}[noitemsep,leftmargin=8pt,topsep=0pt]
    \item \textbf{\gls{bt}:} \dk{We set PP to 64, TG to 0, and varied the batch sizes between 128, 256, 512, and 1024, with all other parameters kept at their default values.}
    \item \textbf{\gls{tt}:} \dk{We set PP to 64 and TG to 0, while varying the thread counts between 1, 2, 4, 8, 16, and 32. All other parameters were kept at their default values.}
\end{itemize}
The results are shown in~\cref{sec:resultsTTBT}.
 
\subsection{Evaluation of Memory Consumption}
\label{sec:resultsMemory}
Memory consumption \( M_{ij} \) is measured in terms of the \gls{rss} which represents the actual physical memory usage of the relevant processes. 
For each model-device pair  ($m_i, d_j$), we measure memory usage over three runs, and report the average.
 \dk{The memory usage for PP and TG parameters was evaluated using the same approach as the performance consistency assessment. In each test, a fixed prompt was provided, and the model was instructed to generate outputs. The experiments were conducted with batch sizes of 128, 256, 512, and 1024, and their average memory consumption was recorded.}   The results are shown in~\cref{sec:resultsMemory}.
\subsection{Battery Consumption}
\label{sec:resultsBattery}  \dk{
This section presents battery consumption rates during extended \gls{pp} and \gls{tg} tests. From our initial investigation, we observed that only \gls{avp} exhibited a slight impact of model size on battery consumption, while the other three devices did not show significant variations with changing model sizes. 
Given this observation, instead of testing all 17 models, we conducted battery tests on selected models. Specifically, we chose $m_1$ as the sole model from the Qwen Series. For the remaining series, we selected the smallest and largest models: $m_2$ and $m_5$ from the Vikhr-Gemma Series, $m_6$ and $m_{11}$ from the Phi-3.1 Series, $m_{12}$ and $m_{13}$ from the LLaMA-2 Series, and $m_{14}$ and $m_{17}$ from the Mistral-7B Series. 
For each experiment, we recorded the battery level at the start and end of the experiment, allowing us to calculate battery consumption over a fixed duration of $600$ seconds (10 minutes). Each experiment was repeated three times, and we reported the mean values.  To ensure consistent conditions, we took a $600$-second break between experiments, allowing the devices to cool down. The results are presented in~\cref{sec:resultsBattery}.}
\subsection{Pareto Optimality}
Although we evaluate our five performance metrics across various devices and models, the metrics remain fragmented and cannot be directly compared with other metrics. To address this, we turn to \textit{Pareto efficiency} theory~\cite{chinchuluun2007survey}, which provides a measure of efficiency in multi-objective contexts and has been widely applied in various system design approaches~\cite{brisset2015approaches, santoro2018design}. A choice is considered Pareto optimal if no other choice exists that can improve any of its objective criteria without deteriorating at least one other criterion. In our case, $x_{1}$ is considered dominated by $x_{2}$ through objects $f$ if :

\begin{multline} \label{equ:pareto}
f_i(x_1) \leq f_i(x_2) \quad \forall i \in \{1, \dots, m\} \\
\text{and} \quad \exists j \in \{1, \dots, m\} \mid f_j(x_1) < f_j(x_2)
\end{multline}

where $i$, $j$ represent different objective indices.

A choice $x^*$ is considered Pareto optimal if no other feasible option dominates it. The set of all non-dominated designs forms the Pareto front, which represents the optimal trade-offs between all objectives.

To identify the optimal choices across devices and models, we define \dk{three objectives: \textit{quality}, \textit{performance}, and \textit{stability}. } To calculate the final score for each objective, we propose~\cref{equ:score}. For a given device-model pair and a specific objective, we first apply min-max normalization to each metric across all device-model pairs to eliminate the impact of different scales. Then, we compute a weighted sum of all metrics for that pair to obtain a single score:

\begin{equation} \label{equ:score}
    f_{o}(x) = \sum_{i=1}^{n} w_i \cdot \frac{m_i(x) - \min(m_i)}{\max(m_i) - \min(m_i)}
\end{equation}

where $o$ is the objective index, $n$ is the number of metrics used in the objective category, $w_i$ defines the weight for each metric, and $\min$ and $\max$ calculate the minimum and maximum values across the entire set of results for metric $m_i$.

In practice, some metrics, such as perplexity, are better when lower, which is the opposite of the direction in~\cref{equ:pareto}. Therefore, we take the reciprocal of each perplexity value. 
% \begin{figure}[!htb]
%     \centering
%     \includesvg[width=\columnwidth]{figures/Apple_Vision_Pro_average_Prompt_Processing_Speed.svg}
%     %\vskip -0.5cm
%     \includesvg[width=\columnwidth]{figures/Magic_Leap_average_Prompt_Processing_Speed.svg}
%     \includesvg[width=\columnwidth]{figures/Meta_Q3_average_Prompt_Processing_Speed.svg}
%     \includesvg[width=\columnwidth]{figures/Vivo_average_Prompt_Processing_Speed.svg}
%     \caption{Prompt processing speeds (in tokens/second) for different devices, averaged over five prompt sizes: 64, 128, 256, 512, 1024, for each model. Error bars indicate standard deviations. \todo{vision pro results have now been updated}}
    
%     \label{fig:pp_speed_three_devices}
% \end{figure}


% \begin{figure}[!htb]
%     \centering
%     \includesvg[width=\columnwidth]{figures/Prompt_Processing_Speed_slope_heatmap.svg}
%     \includesvg[width=\columnwidth]{figures/Token_Generation_Speed_slope_heatmap.svg}
%     \includesvg[width=\columnwidth]{figures/Batched_Token_Generation_Speed_slope_heatmap.svg}
%     \caption{ \todo{I think we should change graph type: Heatmaps do not show a big difference.. } Heatmaps showing the effects of prompt size, token set size and batch size (respectively) on prompt processing, token generation and token generation performance (respectively) for each device-model pair. Brighter colors indicate stronger effects, though absolute values remain small.}
%     \label{fig:pptgbs_heatmap}
% \end{figure}
\begin{figure}[!hbp]
    \centering 
    \includegraphics[width=1.0\linewidth]{images/mqa.png}  
 \caption{Model Quality Analysis: The models were evaluated based on six different benchmarks. We chose accuracy (\textbf{$\uparrow$}) for the five benchmarks on the left and perplexity (\textbf{$\downarrow$}) for the WikiText-2 benchmark.  Each group in the bar chart represents models $m_1$ to $m_{17}$ from left to right.}
    \label{fig:Modelsqty}
\end{figure}  

%===================== ======================================
\begin{figure*}[!htbp]
    \centering 
    \vskip -0.1cm   
      % \includegraphics[width=\linewidth]{images/conssistencyNewdk.png}  
      \includegraphics[width=\linewidth]{images/CPU_Consistency_Plot.pdf}  
    \vskip -0.2478cm
    \label{fig:StabDD}
\caption{\dk{Consistency results of the four devices over time: PP (top) and TG (bottom) speeds in tokens per second across 20 sorted runs (X-axis: run number, Y-axis: speed in $t/s$).}}
    \label{fig:StabDD}
\end{figure*}
% ==================================================================
\begin{figure*}[!htbp] 
 \vskip -0.145cm 
     \centering  
     \includegraphics[width=0.807\linewidth]{images/GPU_Consistency_Plot.pdf}  
\vskip -0.245cm 
    \caption{\dk{Consistency results of the Apple Vision Pro on GPU. Left: PP results. Right: TG results. Each model was tested 20 times, and the results are plotted in sorted order.}}
    \label{fig:VPGPU}
\end{figure*}
%\begin{sidewaystable}[!htbp]%
\begin{table*}%[!hbp]
%\begin{adjustbox}
\centering
\caption{\dk{Performance consistency results for the four devices: Magic Leap 2 (ML2), Meta Quest 3 (MQ3), Vivo X100 Pro (Vivo), and Apple Vision Pro (VPro for CPU and VPro* for GPU). Processing speed was calculated from 20 runs for each model-device pair. The results are reported in terms of the mean ($\mu$) speed (t/s) of 20 runs, its standard deviation ($\sigma$), coefficient of variation (CV \%), and the range of values: $[min, max]$.  }}
\label{tab:consistency}
\vskip -0.45cm 
%\begin{adjustbox}{angle=90} % Rotate table by 90 degrees
\resizebox{1.02\linewidth}{!}{
\begin{tabular}{@{\hskip -0.05cm}c|@{\hskip -0.05cm}c@{\hskip -0.05cm}|l@{\hskip0.05cm}|c@{\hskip0.05cm}c@{\hskip 0.05cm}c@{\hskip 0.05cm}c@{\hskip 0.05cm}c@{\hskip 0.05cm}c@{\hskip 0.05cm}c@{\hskip 0.05cm}c@{\hskip 0.05cm}c@{\hskip 0.05cm}c@{\hskip 0.05cm}c@{\hskip 0.05cm}c@{\hskip 0.05cm}c@{\hskip 0.05cm}c@{\hskip 0.05cm}c@{\hskip 0.05cm}c@{\hskip 0.05cm}}
%\hline
%\begin{tabular}{l|l|cccccccccccccccc}
%\toprule
\hline 
% \multicolumn{18}{c}{\textbf{PP Results}} \\  % Single-row header spanning all columns
%  \hline
 &D& Metric & m2 & m3 & m4 & m5 & m6 & m7 & m8 & m9 & m10 & m11 & m12 & m13 & m14 & m15 & m16 & m17 \\ 
        \hline
 \multirow{15}{*}{\rotatebox[origin=c]{90}{PP Results}}  % Left-most column label 
    &\multirow{3}{*}{ML2}% 2 decimal points 
    & $\mu$$\pm\sigma$ & $17.01 \pm 4.67$ & $13.99 \pm 4.57$ & $10.38 \pm 2.74$ & $20.82 \pm 0.87$ & $16.81 \pm 0.07$ & $13.16 \pm 0.47$ & $14.22 \pm 0.31$ & $12.63 \pm 0.38$ & $13.12 \pm 0.29$ & $11.45 \pm 0.18$ & $7.92 \pm 0.05$ & $7.37 \pm 0.19$ & $2.20 \pm 0.03$ & $3.14 \pm 0.08$ & $1.15 \pm 0.05$ & $1.52 \pm 0.09$ \\
    && Range & $[10.25, 22.24]$ & $[7.47, 18.75]$ & $[6.27, 13.32]$ & $[18.28, 21.36]$ & $[16.68, 16.95]$ & $[12.04, 13.51]$ & $[12.98, 14.36]$ & $[11.73, 12.87]$ & $[12.00, 13.27]$ & $[10.93, 11.62]$ & $[7.73, 7.99]$ & $[6.73, 7.54]$ & $[2.14, 2.23]$ & $[2.98, 3.27]$ & $[1.10, 1.33]$ & $[1.36, 1.72]$ \\
   & & CV (\%) & $27.46$ & $32.66$ & $26.41$ & $4.16$ & $0.43$ & $3.56$ & $2.19$ & $2.98$ & $2.22$ & $1.57$ & $0.66$ & $2.54$ & $1.35$ & $2.41$ & $4.70$ & $5.99$ \\%ml2 pp done ....  
 %\cline{2-19}
 &\multirow{3}{*}{MQ3} 
    & $\mu$$\pm\sigma$ & $21.00 \pm 2.16$ & $25.61 \pm 0.10$ & $21.33 \pm 1.47$ & $16.33 \pm 1.84$ & $10.99 \pm 1.65$ & $11.37 \pm 1.80$ & $13.74 \pm 1.72$ & $10.24 \pm 1.34$ & $11.07 \pm 0.61$ & $8.17 \pm 2.19$ & $7.12 \pm 0.04$ & $6.09 \pm 0.32$ & $2.65 \pm 0.21$ & $2.37 \pm 0.12$ & $1.57 \pm 0.06$ & $2.04 \pm 0.09$ \\
    && Range & $[17.42, 23.59]$ & $[25.03, 25.77]$ & $[14.79, 23.12]$ & $[13.66, 18.72]$ & $[7.62, 13.89]$ & $[7.91, 13.86]$ & $[10.88, 16.74]$ & $[7.79, 11.87]$ & $[7.19, 11.28]$ & $[4.31, 13.05]$ & $[5.96, 7.17]$ & $[5.64, 6.61]$ & $[2.10, 2.92]$ & $[2.12, 2.57]$ & $[1.44, 1.68]$ & $[1.85, 2.34]$ \\
   && CV (\%) & $10.27$ & $0.37$ & $6.89$ & $11.24$ & $14.99$ & $15.80$ & $12.51$ & $13.06$ & $5.55$ & $26.81$ & $0.56$ & $5.23$ & $7.84$ & $5.24$ & $3.93$ & $4.23$ \\ 
% Meta q3 PP done 

 %\cline{2-19} 
    &\multirow{3}{*}{Vivo} 
    & $\mu$$\pm\sigma$ & $15.70 \pm 2.88$ & $15.90 \pm 2.76$ & $14.98 \pm 2.98$ & $13.77 \pm 2.13$ & $8.66 \pm 0.80$ & $8.27 \pm 1.08$ & $10.35 \pm 1.10$ & $8.48 \pm 1.08$ & $8.82 \pm 1.61$ & $11.14 \pm 1.21$ & $4.47 \pm 0.19$ & $4.35 \pm 0.23$ & $2.44 \pm 0.07$ & $4.28 \pm 0.44$ & $2.61 \pm 0.12$ & $5.89 \pm 0.47$ \\
    && Range & $[11.16, 25.55]$ & $[9.39, 17.81]$ & $[10.52, 24.36]$ & $[10.65, 15.55]$ & $[7.32, 9.34]$ & $[5.88, 9.53]$ & $[8.27, 11.05]$ & $[7.19, 11.72]$ & $[5.97, 11.60]$ & $[9.13, 12.04]$ & $[3.98, 4.92]$ & $[3.80, 4.86]$ & $[2.27, 2.52]$ & $[2.61, 4.80]$ & $[2.23, 2.76]$ & $[5.17, 6.59]$ \\
   & & CV (\%) & $18.33$ & $17.38$ & $19.88$ & $15.50$ & $9.26$ & $13.07$ & $10.61$ & $12.75$ & $18.31$ & $10.84$ & $4.24$ & $5.39$ & $3.00$ & $10.30$ & $4.55$ & $7.92$ \\ 
    % Vivoo PP done 
 %\cline{2-19}
&\multirow{3}{*}{VPro} 
    & $\mu$$\pm\sigma$ & $31.03 \pm 2.20$ & $41.22 \pm 3.04$ & $33.11 \pm 1.35$ & $32.96 \pm 1.21$ & $20.62 \pm 0.56$ & $19.34 \pm 0.64$ & $23.14 \pm 1.09$ & $19.52 \pm 0.73$ & $19.56 \pm 0.60$ & $29.22 \pm 1.22$ & $9.15 \pm 0.40$ & $9.06 \pm 0.28$ & $5.52 \pm 0.21$ & $7.44 \pm 0.40$ & $5.01 \pm 0.26$ & $14.05 \pm 0.56$ \\
    && Range & $[29, 34.98]$ & $[39.44, 47.03]$ & $[32.98, 34.79]$ & $[32.15, 35.93]$ & $[20.19, 21.63]$ & $[19.09, 20.43]$ & $[22.65, 24.92]$ & $[19.04, 20.98]$ & $[19.15, 20.74]$ & $[28.58, 31.25]$ & $[8.77, 9.67]$ & $[8.88, 9.77]$ & $[5.4, 5.93]$ & $[7.21, 8.13]$ & $[4.83, 5.47]$ & $[13.63, 15.02]$ \\
   & & CV (\%) & $7.08$ & $7.37$ & $4.07$ & $3.68$ & $2.72$ & $3.31$ & $4.69$ & $3.72$ & $3.05$ & $4.17$ & $4.41$ & $3.12$ & $3.73$ & $5.38$ & $5.16$ & $3.99$ \\

 %\cline{2-19}
     & \multirow{4}{*}{VPro* } 
    & $\mu$$\pm\sigma$ & $378 \pm 3.7$ & $432 \pm 3.7$ & $382\pm 4.8$ & $374\pm 4.6$ & $241\pm 1.5$ & $230\pm 2.3$ & $237\pm 2.0$ & $218\pm 5.8$ & $229\pm 4.3$ & $264\pm 3.4$ & $136\pm 9.9$ & $128\pm 12.5$ & $135\pm 2.3$ & $134\pm 2.3$ & $133\pm 1.4$ & $130\pm 6.6$ \\
    && Range & $[369, 384]$ & $[424, 439]$ & $[367, 389]$ & $[363, 380]$ & $[237, 243]$ & $[225, 234]$ & $[232, 240]$ & $[198, 223]$ & $[221, 237]$ & $[255, 268]$ & $[109, 145]$ & $[100, 143]$ & $[130, 138]$ & $[129, 136]$ & $[129, 134]$ & $[109, 135]$ \\
   & & CV (\%) & $1.0$ & $0.9$ & $1.2$ & $1.2$ & $0.6$ & $1.0$ & $0.8$ & $2.6$ & $1.9$ & $1.3$ & $7.3$ & $9.8$ & $1.7$ & $1.7$ & $1.1$ & $5.1$ \\ 
    \hline

     
  \multirow{15}{*}{\rotatebox[origin=c]{90}{TG Results}}
    %%============
    %%
    %%--------------------------
%\multirow{12}{*}{\rotatebox[origin=c]{90}{TG Test}}  
   &\multirow{3}{*}{ML2}% 2 decimal points 
    & $\mu$$\pm\sigma$ & $9.14 \pm 1.92$ & $8.22 \pm 1.92$ & $6.53 \pm 1.77$ & $8.43 \pm 0.07$ & $11.69 \pm 0.04$ & $8.52 \pm 0.02$ & $7.88 \pm 0.01$ & $6.92 \pm 0.01$ & $6.25 \pm 0.10$ & $5.05 \pm 0.02$ & $5.90 \pm 0.03$ & $5.59 \pm 0.14$ & $2.05 \pm 0.01$ & $2.84 \pm 0.06$ & $0.24 \pm 0.01$ & $0.21 \pm 0.01$ \\
    && Range & $[6.92, 11.25]$ & $[5.69, 10.27]$ & $[4.41, 8.39]$ & $[8.33, 8.55]$ & $[11.61, 11.74]$ & $[8.48, 8.55]$ & $[7.85, 7.91]$ & $[6.89, 6.95]$ & $[5.95, 6.31]$ & $[5.01, 5.08]$ & $[5.83, 5.94]$ & $[5.13, 5.73]$ & $[2.04, 2.06]$ & $[2.77, 2.97]$ & $[0.22, 0.27]$ & $[0.19, 0.23]$ \\
   & & CV (\%) & $20.95$ & $23.36$ & $27.10$ & $0.77$ & $0.32$ & $0.23$ & $0.18$ & $0.21$ & $1.62$ & $0.39$ & $0.48$ & $2.49$ & $0.32$ & $2.24$ & $5.32$ & $5.20$ \\  
    % ml2 tg DONE 
 %\cline{2-19} 
    &\multirow{3}{*}{MQ3}
    & $\mu$$\pm\sigma$ & $12.37 \pm 1.45$ & $14.19 \pm 0.30$ & $11.58 \pm 0.98$ & $9.47 \pm 1.29$ & $9.00 \pm 1.12$ & $8.57 \pm 0.70$ & $9.30 \pm 0.51$ & $7.61 \pm 0.65$ & $8.30 \pm 0.26$ & $5.47 \pm 0.10$ & $5.62 \pm 0.23$ & $5.01 \pm 0.24$ & $2.95 \pm 0.21$ & $3.06 \pm 0.05$ & $1.67 \pm 0.15$ & $3.38 \pm 0.09$ \\
    && Range & $[9.11, 13.89]$ & $[13.52, 14.67]$ & $[7.77, 12.80]$ & $[7.64, 10.87]$ & $[7.38, 10.39]$ & $[6.85, 9.55]$ & $[7.66, 9.86]$ & $[6.13, 8.24]$ & $[7.40, 8.53]$ & $[5.19, 5.61]$ & $[5.26, 6.17]$ & $[4.61, 5.57]$ & $[2.23, 3.40]$ & $[2.96, 3.13]$ & $[1.40, 1.84]$ & $[3.11, 3.53]$ \\
   & & CV (\%) & $11.73$ & $2.11$ & $8.43$ & $13.65$ & $12.50$ & $8.17$ & $5.52$ & $8.51$ & $3.18$ & $1.87$ & $4.09$ & $4.83$ & $7.27$ & $1.68$ & $8.93$ & $2.80$ \\  
    % Meta Q3 TG done 

 %\cline{2-19}
   &\multirow{3}{*}{Vivo} 
    & $\mu$$\pm\sigma$ & $9.39 \pm 0.98$ & $9.85 \pm 0.42$ & $9.44 \pm 1.22$ & $8.66 \pm 0.30$ & $6.91 \pm 0.08$ & $6.83 \pm 0.22$ & $8.21 \pm 0.12$ & $6.84 \pm 0.61$ & $6.61 \pm 0.71$ & $7.51 \pm 0.11$ & $3.70 \pm 0.05$ & $3.59 \pm 0.11$ & $2.09 \pm 0.01$ & $3.71 \pm 0.04$ & $2.27 \pm 0.02$ & $4.72 \pm 0.17$ \\
    && Range & $[8.54, 13.35]$ & $[9.17, 10.67]$ & $[7.58, 13.86]$ & $[7.97, 9.26]$ & $[6.75, 7.00]$ & $[6.39, 7.21]$ & $[7.52, 8.35]$ & $[4.37, 7.11]$ & $[5.70, 7.68]$ & $[7.25, 7.64]$ & $[3.60, 3.77]$ & $[3.41, 3.71]$ & $[2.06, 2.11]$ & $[3.65, 3.79]$ & $[2.23, 2.31]$ & $[4.48, 5.00]$ \\
   & & CV (\%) & $10.46$ & $4.22$ & $12.91$ & $3.48$ & $1.13$ & $3.25$ & $1.48$ & $8.92$ & $10.75$ & $1.41$ & $1.27$ & $3.04$ & $0.68$ & $1.15$ & $1.06$ & $3.61$ \\  
 %\cline{2-19} 
&\multirow{3}{*}{VPro} 
& $\mu$$\pm\sigma$ & $17.90 \pm 1.00$ & $21.15 \pm 0.82$ & $18.92 \pm 0.74$ & $17.71 \pm 0.50$ & $15.72 \pm 0.37$ & $14.76 \pm 0.26$ & $16.63 \pm 0.49$ & $13.93 \pm 0.50$ & $14.07 \pm 0.19$ & $12.03 \pm 0.21$ & $7.20 \pm 0.21$ & $7.22 \pm 0.17$ & $4.53 \pm 0.17$ & $6.24 \pm 0.27$ & $4.17 \pm 0.18$ & $10.22 \pm 0.30$ \\
    && Range & $[16.59, 19.72]$ & $[19.4, 22.49]$ & $[17.31, 19.8]$ & $[16.89, 18.77]$ & $[14.88, 16.36]$ & $[14.32, 15.28]$ & $[15.77, 17.56]$ & $[12.89, 14.99]$ & $[13.71, 14.49]$ & $[11.26, 12.28]$ & $[6.74, 7.52]$ & $[6.98, 7.7]$ & $[4.11, 4.93]$ & $[5.52, 6.72]$ & $[3.86, 4.52]$ & $[9.82, 10.93]$ \\
   & & CV (\%) & $5.58$ & $3.86$ & $3.89$ & $2.84$ & $2.35$ & $1.73$ & $2.97$ & $3.61$ & $1.35$ & $1.77$ & $2.95$ & $2.41$ & $3.84$ & $4.35$ & $4.31$ & $2.91$ \\  
 %\cline{2-19}
&\multirow{3}{*}{VPro*} 
    & $\mu$$\pm\sigma$   & $23.1 \pm 0.1$ & $27.5 \pm 0.1$ & $23.5 \pm 0.1$ & $22.6 \pm 0.1$ & $23.3 \pm 0.1$ & $19.6 \pm 0.8$ & $19.9 \pm 0.1$ & $18.0 \pm 0.4$ & $17.7 \pm 0.2$ & $14.6 \pm 0.2$ & $13.7 \pm 0.8$ & $12.5 \pm 1.0$ & $14.5 \pm 0.3$ & $15.1 \pm 0.2$ & $15.2 \pm 0.2$ & $14.7 \pm 0.4$ \\
    && Range  & $[23, 23.3]$ & $[27.1, 27.7]$ & $[23.2, 23.8]$ & $[22.3, 22.7]$ & $[23.0, 23.5]$ & $[19.2, 22.8]$ & $[19.8, 20.2]$ & $[16.8, 18.5]$ & $[17.2, 17.9]$ & $[14.3, 14.9]$ & $[11.4, 14.6]$ & $[10.2, 13.9]$ & $[13.9, 14.9]$ & $[14.6, 15.4]$ & $[14.9, 15.4]$ & $[13.5, 15.1]$ \\
   & & CV(\%) & $0.4$ & $0.5$ & $0.5$ & $0.3$ & $0.6$ & $4.3$ & $0.6$ & $2.1$ & $1.0$ & $1.1$ & $5.8$ & $8.3$ & $1.9$ & $1.4$ & $1.1$ & $2.5$ \\ 
    %& Errors &  m2 & m3 & m4 & m5 & m6 & m7 & m8 & m9 & m10 & m11 & m12 & m13 & m14 & m15 & m16 & m17 \\
\hline
\end{tabular}
}
%\end{adjustbox}
%\end{sidewaystable}%
\end{table*}
\section{Performance Evaluation Results}
\label{sec:results}%----
This section presents the results of four types of experiments described in \cref{sec:exp:design}.
\subsection{Model Quality Analysis}
As shown in \cref{tab:models}, we select five different model architectures, and for each architecture, we choose various quantization settings. While lower-bit quantization reduces model size, speeds up inference, and lowers power consumption, it typically comes at the cost of reduced model performance. The results of the model quality are illustrated in \cref{fig:Modelsqty}.
\label{sec:resModelsQty}
As shown in \cref{fig:Modelsqty}, a consistent trend is observed across all benchmarks: applying lower-bit quantization settings results in reduced model performance. Models with more parameters exhibit better language understanding capabilities, leading to higher performance under the same quantization settings. Additionally, different benchmarks exhibit varying levels of sensitivity to the quantization settings. For example, Hellaswag, ARC and WikiText-2 are more sensitive compared to the other three benchmarks.

\subsection{\dk{Performance Consistency Results}}
\label{sec:resultsStability}
\dk{\Cref{fig:StabDD,fig:VPGPU} illustrate the consistency of performance across all model-device pairs, while \Cref{tab:consistency} provides the corresponding quantitative results. Across all 20 runs, the Apple Vision Pro demonstrates the most consistent results compared to other devices, with the corresponding speed also being the fastest. Notably, the GPU results for the Apple Vision Pro exhibit greater stability and smaller variance. Among the remaining devices, the Magic Leap 2 also shows a reasonable degree of stability; however, for the first three models, its variance is surprisingly high, reaching 27\%, 32\%, and 26\% for \(m_2\), \(m_3\), and \(m_3\) under the PP setup, and 20\%, 23\%, and 27\% under the TG setup, respectively. In contrast, the Meta Quest 3 and Vivo X100 Pro exhibit relatively poor performance both in terms of speed and variance. Between these two devices, the Meta Quest 3 achieves higher speeds than the Vivo X100 Pro, but there is no clear difference in variance. For some models, one device shows lower variance, while for others, the variance is higher.}\\ 
\dk{Furthermore, as shown in \Cref{tab:perfConsistencyresultsForm1} performance varies significantly for \(m_1\) over Magic Leap 2 and Meta Quest 3, but it is generally consistent across the other two devices. These findings suggest that, provided model-device pairs are carefully selected, LLM performance can be considered reliable, strengthening the case for on-device LLM use in XR environments.}
% \begin{table*}[htbp]
% \centering
% \caption{\dk{Performance consistency results for \(m_1\) across the four devices. For the results of other models and explanation of the metrics used see \Cref{tab:consistency}.}}
% \vskip -0.45cm 
% \label{tab:perfConsistencyresultsForm1}
% \resizebox{\linewidth}{!}{
% \begin{tabular}{l|ccccc|ccccc}
% \hline
% \multirow{2}{*}{Metric} & \multicolumn{5}{c|}{PP Results}     & \multicolumn{5}{c}{TG Results}      \\ \cline{2-11} 
%  & ML2     & Vivoo   & Meta Q3  & VisPro   & VisPro*  & ML2     & Vivoo   & Meta Q3  & VisPro   & VisPro*  \\\hline
% \(\mu \pm \sigma\)               & 51.68 $\pm$ 4.22 & 62.68 $\pm$ 16.64 & 39.32 $\pm$ 15.66 & 292.48 $\pm$ 12.87 & 1603.59 $\pm$ 56.12 & 20.26 $\pm$ 0.05 & 22.16 $\pm$ 0.90 & 16.76 $\pm$ 3.76 & 41.94 $\pm$ 1.77 & 42.94 $\pm$ 0.80 \\ %\hline
% Range                            & [43.62, 54.59]   & [21.63, 72.04]   & [24.25, 72.59]    & [288.58, 312.17]  & [1516.96, 1696.13] & [20.17, 20.34]   & [20.54, 23.41]   & [11.61, 22.53]    & [38.90, 46.37]   & [41.76, 44.35]   \\ %\hline
% CV (\%)                          & 8.17             & 26.55            & 39.82             & 4.40              & 3.50               & 0.23             & 4.05             & 22.44             & 4.21              & 1.87              \\   
% \hline
% \end{tabular}}
% \end{table*} 
% ==============================================================================
\begin{table}[htbp]
\centering
\caption{\dk{Performance consistency results for \(m_1\) across the four devices. For the results of other models and description of metrics used, see \Cref{tab:consistency}.}}
\vskip -0.45cm 
\label{tab:perfConsistencyresultsForm1}
\resizebox{\linewidth}{!}{
\begin{tabular}{l|l|ccccc}
\hline
Test& Metric & ML2     & Vivoo   & Meta Q3  & VisPro   & VisPro*  \\ \hline 
\multirow{3}{*}{PP} 
 & \(\mu \pm \sigma\) & 51.68 $\pm$ 4.22 & 62.68 $\pm$ 16.64 & 39.32 $\pm$ 15.66 & 292.48 $\pm$ 12.87 & 1603.59 $\pm$ 56.12 \\ 
 & Range  & [43.62, 54.59]   & [21.63, 72.04]   & [24.25, 72.59]    & [288.5, 312.1]  & [1516.9, 1696.1] \\ 
 & CV (\%)  & 8.17             & 26.55            & 39.82             & 4.40              & 3.50               \\   
\hline
\multirow{3}{*}{TG}% & Metric & ML2     & Vivoo   & Meta Q3  & VisPro   & VisPro*  \\ \cline{2-7}
 & \(\mu \pm \sigma\) & 20.26 $\pm$ 0.05 & 22.16 $\pm$ 0.90 & 16.76 $\pm$ 3.76 & 41.94 $\pm$ 1.77 & 42.94 $\pm$ 0.80 \\ 
 & Range  & [20.17, 20.34]   & [20.54, 23.41]   & [11.61, 22.53]    & [38.90, 46.37]   & [41.76, 44.35]   \\ 
 & CV (\%)  & 0.23             & 4.05             & 22.44             & 4.21              & 1.87              \\   
\hline
\end{tabular}}
\end{table} 

% ========================= ==================================
\begin{figure*}%[!htp]
     \centering   
     \includegraphics[width=1.0\linewidth]{images/PP_TG_Results_Main.pdf}  
\vskip -0.3645cm 
    \caption{\dk{Processing speed of the four devices in the PP test (top) and TG test (bottom) with varying string lengths: 64, 128, 256, 512, and 1024. The x-axis represents the model, while the y-axis represents the processing speed in t/s.}}
    \label{fig:resPPTG}
\end{figure*}

%-------------------------------------------------
\begin{figure}[!hbp]
     \centering   
     \includegraphics[width=1.0\linewidth]{images/PP_TG_Results_GPU.pdf}  
\vskip -0.45cm 
    \caption{\dk{Processing speed of Apple Vision Pro (GPU) in the PP test (left) and TG test (rigth). The x-axis represents the model, while the y-axis represents the processing speed in t/s.}}
    \label{fig:resPPTG-GPU}
\end{figure}
%--------------------------------------

%-------------------------------------------------
\begin{figure}[!htbp]
     \centering   
     \includegraphics[width=1.0\linewidth]{images/Errors_Count.pdf}  
\vskip -0.355cm 
 \caption{\dk{Error counts for Meta Quest 3 (left) and Apple Vision Pro (right) in PP (top) and TG (bottom). Magic Leap 2, with zero errors, and Vivo X100 Pro, with a total of four errors, one in $m_1$ (PP-64), two in $m_{10}$ (PP-64, PP-128), and one in $m_{11}$ (TG-1024) are excluded.}}
    \label{fig:resErrorsCount}
\end{figure}
%-------------------------------------------------

%-------------------------------------------------
\begin{figure*}%[!htbp]
    \centering 
   % % \vskip -0.396478cm 
   %   \includegraphics[width=0.73\linewidth]{images/BT_DK.pdf} %
   %   \includegraphics[width=0.26\linewidth]{images/BT_GPUAVP.pdf} %
    \includegraphics[width=\linewidth]{images/BT_DK5devices.pdf} %
    \vskip -0.34780cm
    \caption{\dk{Batch Test results: Batch size (X-axis) vs. processing speed (Y-axis).}}
    % \vskip -0.478cm
    \label{fig:resBT}
\end{figure*}
%-------------------------------------------------
%-------------------------------------------------
\begin{figure*}%[!htbp]
    \centering 
  %  \vskip -0.478cm
    %\includegraphics[width=1.0\linewidth]{images/threadsBB.png} %
    % \includegraphics[width=1.0\linewidth]{images/TT_DK.pdf} %
    \includegraphics[width=1.0\linewidth]{images/TT_DK5Devices.pdf} %
    \vskip -0.32478cm
\caption{\dk{Thread Test (TT) results: The X-axis represents the thread count, while the Y-axis shows the speed in tokens per second. Threads 4 and 8 deliver the fastest results. Note: Apple Vision Pro (CPU) fails for thread counts of 16 and 32.}}
    \label{fig:resThreadA}
\end{figure*}
%------------------------------------------------- 

\subsection{Results of Processing Speed and String Length}
\label{sec:resultsPPTG} 
 % \dk{\cref{tab:PPTGresults} and
 \dk{\cref{fig:resPPTG,fig:resPPTG-GPU} present the processing speed results of PP and TG with varying string lengths (prompt length and token sets). Both PP and TG were tested with values of 64, 128, 256, 512, and 1024. The processing speed varies significantly across string lengths, devices, and models. \\
 % %\renewcommand{\arraystretch}{0.9} 
\begin{table*}[tb]
\centering
\caption{\dk{Processing speed (tokens per second) of 17 models across four XR devices for varying prompt/string lengths (PP/TG), along with the mean ($\mu$), standard deviation ($\sigma$), and coefficient of variation (CV). Here, $*$ denotes a single error occurrence, while $*2$ and $*3$ indicate two and three repetitions due to errors, respectively. Note that the CV is computed across different parameters (PP/TG = 64, 128, 256, 512, 1024), meaning a higher CV does not indicate an error (as in CV for 5-run speed) but rather reflects the impact of parameter variation on results.}
\label{tab:PPTGresults}}
\vskip -0.245cm 
\resizebox{\linewidth}{!}{ 
%\begin{tabular}{l|p{20pt}|l|ccccccccccccccccc|l}
\begin{tabular}{l|l|l|ccccccccccccccccc|l}
 \toprule
Test & Device & $PP|TG$ & \multicolumn{17}{c}{Processing speed for each model}& \multicolumn{1}{|l}{Errors} \\ \cline{4-20}
      &        &   Metrics     & $m_1$ & $m_2$ & $m_3$ & $m_4$ & $m_5$ & $m_6$ & $m_7$ & $m_8$ & $m_9$ & $m_{10}$ & $m_{11}$ & $m_{12}$ & $m_{13}$ & $m_{14}$ & $m_{15}$ & $m_{16}$ & $m_{17}$ &Count\\  
\hline
\multirow{36}{*}{\rotatebox[origin=c]{90}{PP Test}} 
& \multirow{8}{*}{\rotatebox[origin=c]{90}{Magic Leap 2}} 
    & 64    & 68.86 & 25.23 & 20.80 & 14.28 & 24.94 & 17.46 & 14.06 & 15.03 & 13.37 & 14.17 & 12.50 & 8.10 & 7.69 & 2.24 & 3.28 & 2.16 & 6.82 &0\\ 
&   & 128   & 69.07 & 24.97 & 20.78 & 14.25 & 24.56 & 17.43 & 13.93 & 14.96 & 13.46 & 14.04 & 12.43 & 8.07 & 7.66 & 2.23 & 3.27 & 2.16 & 6.80 &0\\ 
&   & 256   & 67.55 & 24.46 & 20.35 & 14.10 & 24.26 & 17.02 & 13.65 & 14.72 & 13.13 & 13.81 & 12.13 & 7.96 & 7.55 & 2.23 & 3.26 & 2.15 & 6.75 &0\\ 
&   & 512   & 64.58 & 23.31 & 19.52 & 13.67 & 23.04 & 16.22 & 13.19 & 14.08 & 12.71 & 13.33 & 11.77 & 7.75 & 7.37 & 2.22 & 3.23 & 2.14 & 6.55&0 \\ 
&   & 1024  & 62.20 & 22.25 & 18.73 & 13.30 & 22.01 & 15.57 & 12.74 & 13.59 & 12.25 & 12.87 & 11.43 & 7.54 & 7.15 & 2.20 & 3.20 & 2.13 & 6.45 &0\\ 
&   & $\mu$  & 66.85 & 24.24 & 20.04 & 14.14 & 23.76 & 16.74 & 13.91 & 14.88 & 13.38 & 13.84 & 12.45 & 7.88 & 7.47 & 2.23 & 3.25 & 2.15 & 6.73 &-\\ 
&   & $\sigma$ & 2.53 & 1.19 & 0.80 & 0.33 & 1.18 & 0.74 & 0.65 & 0.58 & 0.42 & 0.55 & 0.40 & 0.23 & 0.16 & 0.01 & 0.03 & 0.01 & 0.14 &-\\ 
&   & CV (\%) & 3.78 & 4.91 & 4.00 & 2.34 & 4.96 & 4.42 & 4.66 & 3.91 & 3.14 & 3.95 & 3.21 & 2.92 & 2.14 & 0.45 & 1.02 & 0.54 & 2.08 &-\\ 
%&   & Errors & 0 & 1 & 1 & 1 & 1 & $1,1,1,1$ & $m_7$ & $m_8$ & $m_9$ & $m_{10}$ & $m_{11}$ & $m_{12}$ & $m_{13}$ & $m_{14}$ & $m_{15}$ & $m_{16}$ & $m_{17}$ &Count\\ 
\cline{2-21}
& \multirow{8}{*}{\rotatebox[origin=c]{90}{Meta Quest 3}}  
    & 64    & 118.63 & 26.39 & 28.63 & 23.14 & 20.53 & 14.41 & 13.77 & 16.74 & 11.95 & 11.60 & 18.78 & 6.37 & 6.27 & 3.58 & 5.62 & 2.83 & 6.83$^{*}$ &1\\ 
&   & 128   & 121.99 & 26.17 & 27.04 & 21.51 & 19.48 & 13.08 & 12.15 & 14.94 & 10.52 & 10.29 & 16.34 & 5.98 & 5.71 & 3.06$^{*}$ & 5.00 & 2.22 & 5.98 &1\\ 
&   & 256   & 113.77 & 23.54 & 24.48 & 18.99 & 17.98 & 12.03 & 10.93 & 13.44 & 9.46  & 9.57  & 11.84 & 4.78 & 4.56$^{*}$ & 2.35 & 3.74 & 2.14 & 4.95$^{*}$ &2\\ 
&   & 512   & 101.85 & 21.34 & 21.97 & 17.44 & 16.02 & 9.49  & 8.39  & 10.33$^{*}$ & 7.05  & 7.87$^{*}$  & 11.01 & 4.46 & 3.82 & 2.19 & 3.37 & 1.92 & 4.06 &2\\ 
&   & 1024  & 88.49  & 19.05 & 19.77 & 12.67 & 12.66$^{*}$ & 7.81  & 7.19  & 8.68  & 6.24  & 6.38  & 9.58  & 4.04 & 3.65 & 2.18 & 3.28 & 1.87 & 3.82 &1\\ 
&   & $\mu$  & 108.95 & 23.98 & 24.77 & 18.75 & 17.33 & 11.35 & 10.09 & 12.83 & 9.05  & 9.62  & 13.91 & 5.33 & 4.92 & 2.63 & 4.80 & 2.20 & 4.93 &-\\ 
&   & $\sigma$ & 12.77 & 3.00  & 3.65  & 3.54  & 2.92  & 2.46  & 2.37  & 3.07  & 1.84  & 2.02  & 3.55  & 0.96 & 0.92 & 0.48 & 0.86 & 0.14 & 1.07 &-\\ 
&   & CV (\%) & 11.72 & 12.51 & 14.72 & 18.90 & 16.84 & 21.63 & 23.53 & 23.94 & 20.34 & 21.01 & 25.55 & 18.06 & 18.67 & 18.25 & 17.96 & 6.28 & 21.68 &-\\ 
%&   & Errors & $m_1$ & $m_2$ & $m_3$ & $m_4$ & $m_5$ & $m_6$ & $m_7$ & $m_8$ & $m_9$ & $m_{10}$ & $m_{11}$ & $m_{12}$ & $m_{13}$ & $m_{14}$ & $m_{15}$ & $m_{16}$ & $m_{17}$ &Count\\ 

\cline{2-21}
& \multirow{8}{*}{\rotatebox[origin=c]{90}{Vivo X100s Pro }}  
    & 64    & 94.55 & 16.43 & 18.40 & 15.95 & 15.90 & 9.16 & 9.36 & 10.83 & 9.00 & 9.23 & 12.02 & 4.80 & 4.74 & 2.56 & 4.64 & 2.80 & 6.43 &0\\ 
&   & 128   & 103.07$^{*}$ & 16.27 & 18.35 & 16.05 & 15.94 & 9.17 & 9.35 & 10.78 & 9.07 & 12.47$^{*}$ & 12.05 & 4.85 & 4.75 & 2.57 & 4.63 & 2.81 & 6.46 &2\\ 
&   & 256   & 109.44 & 16.48 & 18.08 & 15.92 & 15.83 & 9.11 & 9.25 & 10.75 & 8.97 & 11.89$^{*}$ & 11.70 & 4.80 & 4.74 & 2.55  & 4.61 & 2.80 & 6.42 &1\\ 
&   & 512   & 101.79 & 15.86 & 17.57 & 15.45 & 15.37 & 8.90 & 9.09 & 10.47 & 8.50& 8.25 & 11.12 & 4.76& 4.68 & 2.53 & 4.44 & 2.78 & 6.29 &0\\ 
&   & 1024  & 96.52 & 15.25 & 16.73 & 14.85 & 14.73  & 8.66  & 8.83  & 10.15 & 8.05 & 8.10 & 10.38 & 4.68 & 4.40 & 2.50 & 4.32 & 2.75 & 5.93 &0\\ 
&   & $\mu$  & 101.07 & 16.05 & 17.83 & 15.65 & 15.56 & 9.00 & 9.18 & 10.60 & 8.72 & 10.59 & 11.45 & 4.78 & 4.67 & 2.54 & 4.53 & 2.79 & 6.31 &-\\ 
&   & $\sigma$ & 5.87 & 0.48 & 0.68 & 0.48 & 0.47 & 0.22 & 0.22 & 0.30 & 0.44 & 1.69 & 0.84 & 0.06 & 0.13 & 0.03 & 0.12 & 0.03 & 0.19 &-\\ 
&   & CV (\%) & 5.81 & 2.99 & 3.82 & 3.05 & 3.02 & 2.44 & 2.37 & 2.83 & 5.07 & 15.96 & 7.32 & 1.26 & 2.83 & 1.06 & 2.57 & 1.06 & 3.01 &-\\ 
\cline{2-21}
& \multirow{8}{*}{\rotatebox[origin=c]{90}{Apple Visions Pro }}   
    & 64    & 306.88 & 34.81 & 48.79 & 36.14 & 36.69 & 21.89 & 20.46 & 25.38 & 21.07 & 20.04 & 31.46 & 10.01 & 9.54 & 5.81 & 8.41 & 5.23 & 15.05 &0\\ 
&   & 128   & 313.30 & 34.87 & 48.13 & 35.88 & 36.64 & 21.39 & 20.09 & 25.08 & 20.74 & 20.37 & 31.58 & 9.74 & 9.32 & 5.50 & 7.66 & 4.87 & 14.18 &0\\ 
&   & 256   & 309.84 & 32.62 & 44.46 & 32.44 & 33.59 & 19.68 & 19.27 & 22.95 & 18.66 & 18.36 & 28.49 & 9.05 & 8.47 & 5.35$^*$ & 7.35 & 4.75$^*$ & 13.32 &2\\ 
&   & 512   & 287.89 & 29.74 & 40.52 & 29.57 & 30.63 & 18.39$^*$ & 17.89$^*$ & 21.16 & 17.21$^*$ & 17.27$^*$ & 25.72$^*$ & 8.70$^*$ & 8.37$^*$ & 5.24 & 7.12 & 4.66 & 12.83 &7\\ 
&   & 1024  & 254.60 & 29.28$^*$ & 38.45$^*$ & 28.39$^*$ & 30.10$^*$ & 17.55$^*$ & 17.15$^*$ & 20.29 & 16.50 & 16.33 & 24.10 & 8.29 & 8.04 & 5.04 & 6.17 & 4.57 & 12.31$^*$ &7\\ 
&   & $\mu$  & 294.70 & 32.26 & 44.07 & 32.08 & 33.74 & 19.79 & 18.97 & 23.57 & 18.44 & 18.67 & 28.67 & 9.56 & 8.75 & 5.39 & 7.54 & 4.82 & 13.54 &--\\ 
&   & $\sigma$ & 21.77 & 2.61 & 3.79 & 3.27 & 2.79 & 1.78 & 1.32 & 2.11 & 2.07 & 1.85 & 3.06 & 0.67 & 0.51 & 0.21 & 0.92 & 0.21 & 1.02 &--\\ 
&   & CV (\%) & 7.39 & 8.09 & 8.60 & 10.19 & 8.28 & 8.99 & 6.96 & 8.95 & 11.21 & 9.90 & 10.69 & 7.01 & 5.85 & 3.96 & 12.21 & 4.36 & 7.53 &--\\ 
\cline{2-21}
& \multirow{8}{*}{\rotatebox[origin=c]{90}{Vision Pro (GPU)}} 
    & 64    & 1529.67 & 360.11 & 417.73 & 368.71 & 360.02 & 237.96 & 218.35 & 225.86 & 216.24 & 230.30 & 256.45 & 143.86 & 144.12 & 128.94 & 125.73 & 129.77 & 134.74 & 0\\ 
&   & 128   & 2047.04 & 394.42 & 458.48 & 402.89 & 397.81 & 249.57 & 226.23 & 230.42 & 220.94 & 243.01 & 263.26 & 150.84 & 149.97 & 141.49 & 130.77 & 137.79 & 140.35 & 0\\ 
&   & 256   & 2304.05 & 418.75 & 474.66 & 416.41 & 421.77 & 259.63 & 233.24 & 235.00 & 227.00 & 244.50 & 270.05 & 153.13 & 151.58 & 145.29 & 135.54 & 141.05 & 141.90 & 0\\ 
&   & 512   & 2255.74 & 423.83 & 484.57 & 421.13 & 423.35 & 258.17 & 232.29 & 233.41 & 233.52 & 244.26 & 276.66 & 151.08 & 150.39 & 146.56 & 142.89 & 141.09 & 141.64 & 0\\ 
&   & 1024  & 2117.98 & 416.88 & 474.03 & 409.38 & 418.41 & 246.34 & 224.61 & 232.08 & 227.66 & 225.57 & 265.67 & 139.65 & 140.05 & 142.85 & 137.78 & 138.76 & 141.32$^{*}$ & 1\\ 
&   & $\mu$  & 2050.90 & 402.80 & 461.89 & 403.70 & 404.27 & 250.34 & 226.94 & 231.35 & 225.07 & 237.53 & 266.42 & 147.71 & 147.22 & 141.03 & 134.54 & 137.69 & 139.99 &--\\ 
&   & $\sigma$ & 309.13 & 26.40  & 26.39  & 20.75  & 26.79  & 8.91   & 6.08   & 3.50   & 6.65   & 8.93   & 7.55   & 5.71   & 4.94   & 7.04   & 6.58   & 4.65   & 2.99   & --\\ 
&   & CV (\%) & 15.07  & 6.55   & 5.71   & 5.14   & 6.63   & 3.56   & 2.68   & 1.51   & 2.95   & 3.76   & 2.83   & 3.86   & 3.36   & 4.99   & 4.89   & 3.38   & 2.14   & --\\ 

\midrule

\multirow{32}{*}{\rotatebox[origin=c]{90}{TG Test}} 
& \multirow{8}{*}{\rotatebox[origin=c]{90}{Magic Leap 2}} 
    & 64    & 20.42 & 11.37 & 10.31 & 8.45 & 8.51 & 12.38 & 8.88 & 8.26 & 7.14 & 6.47 & 5.17 & 6.13 & 5.89 & 2.07 & 3.00 & 1.99 & 4.92 &0\\ 
&   & 128   & 20.41 & 11.27 & 10.32 & 8.40 & 8.47 & 12.21 & 8.80 & 8.20 & 7.09 & 6.44 & 5.13 & 6.08 & 5.86 & 2.07 & 2.99 & 1.98 & 4.89 &0\\ 
&   & 256   & 20.30 & 11.13 & 10.21 & 8.34 & 8.39 & 11.91 & 8.67 & 8.07 & 6.99 & 6.35 & 5.08 & 5.97 & 5.77 & 2.07 & 2.99 & 1.98 & 4.87 &0\\ 
&   & 512   & 20.11 & 10.92 & 10.02 & 8.23 & 8.28 & 11.43 & 8.40 & 7.83 & 6.82 & 6.21 & 5.01 & 5.81 & 5.61 & 2.06 & 2.97 & 1.97 & 4.82 &0\\ 
&   & 1024  & 19.81 & 10.53 & 9.70  & 7.98 & 8.04 & 10.56 & 7.93 & 7.43 & 6.49 & 5.94 & 4.83 & 5.47 & 5.32 & 2.06 & 2.93 & 1.96 & 4.71 &0\\ 
&   & $\mu$  & 20.21 & 11.08 & 10.11 & 8.29 & 8.34 & 11.71 & 8.74 & 8.16 & 6.91 & 6.47 & 5.06 & 5.99 & 5.67 & 2.07 & 2.98 & 1.98 & 4.84 &-\\ 
&   & $\sigma$ & 0.22 & 0.33 & 0.24 & 0.17 & 0.17 & 0.35 & 0.34 & 0.32 & 0.23 & 0.21 & 0.13 & 0.29 & 0.18 & 0.00 & 0.02 & 0.01 & 0.07 &-\\ 
&   & CV (\%) & 1.09 & 2.96 & 2.37 & 2.05 & 2.04 & 2.97 & 3.93 & 3.90 & 3.36 & 3.27 & 2.62 & 4.79 & 3.16 & 0.23 & 0.57 & 0.34 & 1.48 &-\\ 

\cline{2-21}
& \multirow{8}{*}{\rotatebox[origin=c]{90}{Meta Quest 3}}    
    & 64    & 22.96 & 13.55 & 13.92 & 11.85 & 9.29  & 11.13 & 9.59  & 10.13 & 8.49  & 8.32  & 6.66  & 5.11  & 4.95  & 3.10  & 4.77  & 2.01  & 5.30 &0\\ 
&   & 128   & 22.79 & 12.01 & 11.83 & 10.03 & 8.51  & 9.06$^{*}$  & 7.41$^{*2}$  & 8.57  & 6.57  & 7.21  & 5.98  & 3.93$^{*2}$  & 3.67$^{*}$  & 2.27  & 3.12  & 1.61  & 3.57$^{*}$ &7\\ 
&   & 256   & 21.14 & 10.37 & 9.66$^{*}$  & 7.09$^{*2}$  & 6.40  & 6.11  & 5.14$^{*2}$  & 5.74$^{*2}$  & 4.29$^{*2}$  & 5.10$^{*2}$  & 3.43$^{*}$  & 3.07$^{*2}$  & 2.41  & 1.73  & 2.42$^{*}$  & 1.43  & 2.40 &14\\ 
&   & 512   & 19.62 & 7.78$^{*3}$  & 5.53  & 4.49  & 4.18  & 4.79  & 3.85  & 4.01  & 3.29  & 3.53  & 2.66  & 2.55  & 2.16  & 1.63  & 2.15  & 1.37  & 2.14 &3\\ 
&   & 1024  & 12.65$^{*}$ & 5.82  & 4.52  & 3.75  & 3.53  & 3.99  & 3.32  & 3.41  & 2.92  & 2.88$^{*}$  & 2.34  & 2.31  & 2.03  & 1.61  & 2.04  & 1.32  & 2.01 &2\\ 
&   & $\mu$  & 19.43 & 9.71  & 8.81  & 7.04  & 6.38  & 7.02  & 5.87  & 6.37  & 5.11  & 5.41  & 4.21  & 3.79  & 3.44  & 2.27  & 3.30  & 1.75  & 3.40 &-\\ 
&   & $\sigma$ & 4.45  & 3.47  & 3.33  & 3.08  & 2.62  & 3.13  & 2.48  & 2.82  & 2.35  & 2.11  & 1.92  & 1.29  & 1.13  & 0.72  & 1.04  & 0.35  & 1.29 &-\\ 
&   & CV (\%) & 22.91 & 35.76 & 37.76 & 43.72 & 41.12 & 44.57 & 42.25 & 44.26 & 45.99 & 38.94 & 45.66 & 34.14 & 32.95 & 31.73 & 31.52 & 19.87 & 38.06 &-\\ 
\cline{2-21} 
& \multirow{8}{*}{\rotatebox[origin=c]{90}{Vivo X100s Pro}}   
    & 64    & 23.21 & 9.42 & 10.29 & 9.25 & 9.08 & 7.31 & 7.42 & 8.62 & 7.22 & 7.56 & 7.80 & 3.84 & 3.81 & 2.05 & 3.80 & 2.34 & 4.90 &0\\ 
&   & 128   & 22.55 & 9.26 & 10.06 & 9.14 & 8.93 & 7.25 & 7.36 & 8.53 & 7.13 & 7.49 & 7.66 & 3.83 & 3.78 & 2.07 & 3.70 & 2.33 & 5.00 &0\\ 
&   & 256   & 22.71 & 9.11 & 9.98 & 9.03 & 8.88 & 7.18 & 7.18 & 8.31 & 6.67 & 6.95 & 7.21 & 3.80 & 3.66 & 2.04 & 3.55 & 2.24 & 4.84 &0\\ 
&   & 512   & 22.35 & 8.85 & 9.53 & 8.66 & 8.50 & 6.70 & 6.80 & 7.67 & 5.97 & 6.27 & 6.59 & 3.61 & 3.50 & 1.97 & 3.44 & 2.20 & 4.48 &0\\ 
&   & 1024  & 21.69 & 8.20 & 8.78 & 8.00 & 7.82 & 6.24 & 6.15 & 6.96 & 5.18 & 5.32 & 5.24$^{*}$ & 3.44 & 3.27 & 1.93 & 3.29 & 2.07 & 4.25 &1\\ 
&   & $\mu$  & 22.10 & 8.77 & 9.71 & 8.81 & 8.65 & 6.93 & 7.00 & 8.02 & 6.43 & 6.92 & 6.70 & 3.70 & 3.60 & 2.01 & 3.56 & 2.24 & 4.69 &-\\ 
&   & $\sigma$ & 0.74 & 0.46 & 0.55 & 0.47 & 0.42 & 0.38 & 0.42 & 0.63 & 0.88 & 1.04 & 0.94 & 0.16 & 0.25 & 0.05 & 0.22 & 0.10 & 0.25 &-\\ 
&   & CV (\%) & 3.35 & 5.24 & 5.67 & 5.34 & 4.85 & 5.48 & 5.97 & 7.87 & 13.71 & 15.04 & 14.03 & 4.39 & 6.92 & 2.49 & 6.18 & 4.56 & 5.33 &-\\ 
\cline{2-21}
& \multirow{8}{*}{\rotatebox[origin=c]{90}{Apple Visions Pro}}    
    & 64    & 45.14 & 18.76 & 23.17 & 17.56 & 18.97 & 15.97 & 14.84 & 17.03 & 14.87 & 14.22 & 12.29 & 7.87 & 7.65 & 4.94 & 6.82 & 6.83 & 10.72 &0\\ 
&   & 128   & 45.17 & 19.32 & 22.30 & 18.48 & 18.41 & 15.15 & 14.15 & 16.97 & 13.76 & 13.41 & 12.11 & 7.27 & 7.16 & 4.57 & 6.05 & 6.24 & 10.48$^*$ &1\\ 
&   & 256   & 47.83 & 18.15 & 20.88 & 16.85 & 17.15 & 14.04 & 12.83 & 15.55 & 12.54 & 12.12 & 12.23 & 6.74 & 6.83 & 4.46 & 6.03 & 6.11$^{*3}$ & 9.02$^{*2}$ &5\\ 
&   & 512   & 47.24 & 17.97$^{*3}$ & 19.69 & 16.94$^{*2}$ & 16.52$^{*2}$ & 13.26 & 11.63 & 14.17 & 11.61$^{*3}$  & 10.59 & 11.84$^{*2}$ & 3.03$^{*2}$  & 3.12$^{*2}$ & 4.16 & 5.52 & 2.88$^{*}$ & 4.06$^{*2}$ &19\\ 
&   & 1024  & 46.35 & 16.58$^{*3}$ & 19.33 & 15.37$^{*3}$   & 13.43$^{*2}$ & 11.69 & 10.19 & 12.54$^{*}$ & 5.68$^{*2}$  & 9.80  & 5.73$^{*}$  & 2.83$^{*}$ & 5.87$^{*2}$ & 4.38$^{*}$ & 5.31  & 2.86 & 4.57$^{*2}$ &18\\ 
&   & $\mu$  & 46.75 & 18.56 & 21.47 & 17.04& 17.90 & 14.82 & 12.73 & 15.65 & 11.69 & 12.83 & 10.84 & 5.55 & 6.13 & 4.50 & 6.35 & 4.98 & 8.57 &--\\ 
&   & $\sigma$ & 1.15  & 0.97  & 1.43  & 1.14  & 1.89  & 1.33  & 1.63  & 1.55  & 3.58  & 1.82  & 2.47  & 2.42 & 1.48 & 0.54 & 0.72 & 1.35 & 2.16 &--\\ 
&   & CV (\%) & 2.46  & 5.23  & 6.67  & 6.68  & 10.57 & 8.97  & 12.81 & 9.91  & 30.61 & 14.20 & 22.79 & 43.69 & 24.15 & 12.00 & 11.34 & 27.11 & 25.20 &--\\  
\cline{2-21}
& \multirow{8}{*}{\rotatebox[origin=c]{90}{Visions Pro (GPU)}} 
    & 64    & 43.34 & 22.71 & 28.64 & 22.58 & 22.14 & 23.41 & 19.23 & 19.47 & 17.24 & 16.78 & 14.68 & 14.48 & 13.80 & 14.74 & 15.16 & 14.89 & 15.06 & 0\\ 
    && 128   & 42.35 & 23.17 & 28.00 & 23.42 & 22.64 & 23.50 & 19.31 & 19.92 & 17.58 & 17.78 & 14.66 & 14.03 & 13.57 & 14.51 & 14.99 & 14.91 & 14.89 & 0\\ 
    && 256   & 41.85 & 23.08 & 27.01 & 22.89 & 22.62 & 23.81 & 19.31 & 19.12$^{*}$ & 17.39 & 17.67 & 14.30 & 12.24 & 12.16 & 14.34 & 15.16 & 14.62 & 14.70 & 1\\ 
    && 512   & 40.73 & 22.08$^{*}$ & 27.67$^{*}$  & 23.36 & 21.85 & 23.08$^{*}$  & 18.08 & 18.99$^{*}$  & 16.75 & 16.97 & 13.75 & 11.37 & 9.91  & 13.84 & 14.65 & 13.00 & 13.15 & 4\\ 
    && 1024  & 39.80 & 19.95$^{*}$  & 26.66$^{*}$  & 21.06$^{*}$  & 19.49 & 19.87$^{*}$  & 15.72 & 17.88 & 15.58 & 14.55$^{*}$  & 13.55 & 10.51 & 9.87$^{*}$   & 12.78 & 13.02 & 12.50 & 11.66$^{*}$  & 7\\ 
    && $\mu$  & 41.61 & 22.20 & 27.60 & 22.66 & 21.75 & 22.73 & 18.33 & 19.08 & 16.91 & 16.75 & 14.19 & 12.53 & 11.86 & 14.04 & 14.60 & 13.98 & 13.89 & --\\ 
    && $\sigma$ & 1.38 & 1.33  & 0.79  & 0.96  & 1.31  & 1.62  & 1.55  & 0.76  & 0.80  & 1.30  & 0.52  & 1.70  & 1.91  & 0.78  & 0.91  & 1.15  & 1.46  & --\\ 
    && CV (\%) & 3.32 & 5.98  & 2.85  & 4.24  & 6.00  & 7.13  & 8.45  & 3.98  & 4.75  & 7.78  & 3.66  & 13.57 & 16.07 & 5.55  & 6.20  & 8.20  & 10.52 & --\\ 

\bottomrule
\end{tabular}
}
\end{table*}

\noindent{}\textbf{String length and PP vs TG Speed}: PP speed is consistently higher than TG speed, with a speedup of approximately two to three times across the four (CPU-based) devices, while on \gls{avp} (GPU), the PP speed is significantly higher, ranging from 10 to 19 times the TG speed. Similarly, increasing the prompt size slightly reduces the prompt processing speed in PP, whereas in TG, the speed reduction is more significant with a growing token set. This suggests that TG has a higher computational cost due to its sequential execution of tokens, whereas PP is somewhat parallelized, requiring only the loading of prompts. Though there is minor variation in the 64, 128, and 256 string lengths, the last two, 512 and 1024, are consistently the slowest. When examining the variation across string lengths, \gls{mq3} exhibits the largest variance, whereas \gls{ml2} and \gls{avp} have the smallest variance.  \\
\noindent{}\textbf{Device-Based Analysis:} If we examine the devices, the overall performance trends reveal that \gls{avp} consistently shows the highest speed in both PP and TG tests. \gls{ml2} ranks second in both PP and TG performance, while Vivo X100s Pro outperforms \gls{mq3} in TG (particularly in longer strings 512, 1024) but ranks fourth in PP. This ranking highlights the varying computational capabilities of XR devices, with AVP (both CPU and GPU) standing out as the fasted one across all conditions. \gls{ml2} shows stable performance across all variations of string lengths, with lower $CV\%$ values, indicating that it does not vary significantly with changing string lengths. In contrast, \gls{mq3} shows a higher $CV\%$, indicating greater variance with increasing string length.  \\
%-------------------------------------------------
\noindent{}\textbf{Error Counts:} \Cref{fig:resErrorsCount} shows the errors counts during the PP and TG tests. Errors mostly occurred in \gls{avp} (CPU) and \gls{mq3}, particularly at longer prompt lengths of 512 and 1024 tokens. There are a few reasons for this. First, longer strings process more slowly, leading to abrupt changes and occasional failures. Second, memory constraints or inefficiencies in sustained generation contribute to these errors. With \gls{avp}, the general user experience is also suboptimal, as the device must be actively mounted on the head, which can cause errors if not handled carefully. Overall, \gls{mq3} records the highest count of errors, with frequent retries affecting its reliability, whereas \gls{ml2} remains highly stable, leading to minimal inconsistencies (zero errors). As expected, AVP (GPU) also reported zero errors. \\
\textbf{Model-Based Analysis:} In the model-wise analysis, smaller models such as Qwen2-0.5B and Vikhr-Gemma-2B achieve the highest speeds across devices, particularly in PP. In contrast, larger models like the LLaMA-2-7B series and Mistral-7B series exhibit significantly lower processing speeds, with higher variability and instability, especially in TG. Model size also impacts processing speed, as smaller models are faster. Here, $m_1$ is the fastest (though omitted from later analysis), while $m_2$, $m_3$, and $m_5$ are the fastest among the remaining models, whereas $m_{14}$ and $m_{16}$ are the slowest.}
\subsection{Results of Parallelization with BT and TT}
\label{sec:resultsTTBT} 
\dk{This section presents the results of BT and TT, both of which are used to achieve concurrency and parallelization. Regardless of the parallelization method, the three models $m_2$, $m_3$, and $m_5$ consistently rank among the fastest across all devices, while $m_{14}$ and $m_{16}$ are the slowest.\\
\noindent\textbf{Results of BT:}  
\cref{fig:resBT} presents the results of the batch test with varying batch sizes of 128, 256, 512, and 1024. Generally, increasing the batch size leads to a decrease in performance across all four devices (except for \gls{avp} (GPU)) due to increased computational overhead. \gls{avp} (GPU), with its strong computational resources, does not show a significant performance drop with varying batch sizes, although there are some fluctuations at a batch size of 256. This indicates that GPU-based processing on \gls{avp} (GPU) is highly optimized for parallel execution and can effectively manage larger batches of input without affecting processing speed. \\
More importantly, \gls{avp} (CPU) experiences a noticeable performance drop as the batch size increases. However, the remaining devices do not follow this trend. For instance, on \gls{mq3}, batch size 128 yields the best performance, whereas for the other batch sizes, there is no significant change. Similarly, for \gls{ml2} and \gls{vivo}, variations in batch size do not appear to have a substantial impact on performance.\\
\noindent\textbf{Results of TT:}  
\cref{fig:resThreadA} presents the TT results for thread counts of 1, 2, 4, 8, 16, and 32. Across all four devices, processing speed sees the most significant increase when increasing the thread count from 1 to 4. The speed gain continues at 8 threads (or shows minor degradation), but beyond 8 threads, performance begins to decline slightly, with further degradation beyond 16 threads. Apple Vision Pro (CPU) fails at thread counts of 16 and 32, indicating its limitations in achieving this level of parallelism.\\
\gls{avp} (GPU) follows a different trend, maintaining consistently high processing speed across all thread counts. Unlike CPU-based devices, its performance remains stable even at 32 threads, demonstrating its superior ability to handle concurrent tasks. These results highlight that while the four CPU-based devices benefit from moderate threading, \gls{avp} (GPU) is significantly more efficient at scaling concurrency without experiencing notable performance degradation. Since our study primarily focuses on CPU-based implementation, we conclude that using a moderate thread count of 4, 6, or 8 yields optimal results.}
\begin{figure}[!htbp]
    \centering
    \includegraphics[width=1\linewidth]{images/memoryresults.pdf}
    \vskip -0.396478cm 
    \caption{ \dk{Memory consumption for each model-device pair. The results represent the mean values across batch sizes 128, 256, 512, and 1024. }}
    \label{fig:Memory_plot}
    \vskip -0.396478cm 
\end{figure}
\subsection{Memory Consumption Results }
\label{sec:resultsMemory}
\dk{\cref{fig:Memory_plot} presents the memory consumption results for each model-device pair. The values represent the mean memory consumption from five experiments with varying batch sizes of 128, 256, 512, and 1024. As expected, memory consumption varies across models, with most showing consistent usage across different devices.    Generally, within each model series, memory consumption increases with model size. However, minor exceptions exist, such as $m_7$ on Apple Vision Pro (GPU), which consumes more memory than other models in the Phi-3.1 series $[m_6-m_{11}]$. This size-based trend does not necessarily hold across series, even for models of similar sizes. For example, \textit{m2} (1.36 GB) consumes less memory than  $m_6$ (1.32 GB) despite having a larger size, as they belong to different series. Similarly,  $m_7$ (1.94 GB) requires more memory than \textit{$m_5$} (2.0 GB) and \textit{$m_{15}$} (2.05 GB), as all three belong to different series. Another example is  $m_{12}$ (2.63 GB) consuming more memory than $m_{16}$ (2.81 GB), despite being smaller in size.} \\
\dk{From a device-specific perspective, Apple Vision Pro demonstrates significantly lower memory consumption than the other devices, averaging $0.5782$ GB on the CPU and $0.5488$ GB on the GPU across all $17$ models. Among the remaining devices, Meta Quest 3 consumes the least memory at $2.3520$ GB, followed by Vivo X100 Pro at $2.3540$ GB, while Magic Leap 2 exhibits the highest at $2.3748$ GB. These findings suggest that Apple Vision Pro (both CPU and GPU) is approximately four times more memory-efficient than Magic Leap 2, Meta Quest 3, and Vivo X100 Pro. Its superior memory management and hardware optimizations make it the most efficient device in terms of memory consumption.}
\subsection{Battery Consumption Results}
\label{sec:resultsBattery}
\Cref{fig:BatteryTEST} presents the battery consumption results over a 10-minute experiment. \dk{We observe that for the first two series, the Qwen Series and Vikhr-Gemma Series, battery consumption remains relatively low and with lower variation across all four devices. However, for larger models in the LLaMA-2 Series and Mistral-7B Series, battery consumption increases significantly. 
\gls{avp} shows a strong correlation between model size and battery consumption, both for GPU and CPU usage. In contrast, the remaining three devices do not exhibit a significant variation with respect to model size. \gls{vivo} demonstrates the best battery life, with an average loss of only $2.5\%$ over 10 minutes. \gls{ml2} follows with a $8.5\%$ average battery loss, while \gls{mq3} records a slightly higher battery loss at $9.7\%$. \gls{avp} shows the worst in this regard, with GPU usage resulting in a $10.1\%$ battery loss and with CPU usage leading to a $12.6\%$ battery loss in 10 minuets.} 

\begin{figure}[!htbp]
    \centering 
    \includegraphics[width=1\linewidth]{images/battery.pdf}  
    \vskip -0.396478cm 
    \caption{ \dk{Battery Consumption in 10-minute experiments. }  }
    \label{fig:BatteryTEST}
\end{figure}

\subsection{Pareto Front Results}

Since we use \dk{both CPU and GPU} for inference in the Vision Pro experiments, while the other three devices use CPU only for inference, we exclude \dk{Apple Vision Pro GPU} from the Pareto calculation to ensure a fair comparison. \dk{With 4 devices and 16 models (m2-m17), we have a total of 64 device-model pairs.} Using \cref{equ:pareto} and \cref{equ:score}, we calculated the \textit{quality}, \dk{\textit{stability} and \textit{performance}}  scores for each pair, then identified the Pareto fronts and visualized them in \cref{fig:pareto}. The quality of a model can be evaluated using various benchmarks, such as accuracy and perplexity, as detailed in \cref{sec:modelsQA}. \dk{ For the performance objective, we evaluate performance using \textit{PP}, \textit{TG}, \textit{memory consumption}, and \textit{battery consumption}, with respective weights of 0.35, 0.35, 0.2, and 0.1. For the stability objective, we assess it based on the coefficient of variation (\textit{CV}) and error count of \textit{PP} and \textit{TG}, assigning 0.7 to CV and 0.3 to error count.
} 
\begin{figure}[!htbp]
    \centering
    \includegraphics[width=1.0\linewidth]{images/pareto.png}
    \vskip -0.578cm 
    \caption{\dk{Pareto fronts for various device-model pairs, with different colors representing distinct device types. Performance is evaluated based on processing speed, memory consumption, and battery consumption. Stability is assessed using the \textit{CV} and \textit{error count} values of \textit{PP} and \textit{TG}. Quality is measured through evaluation results on six benchmark datasets.
    } }
    \label{fig:pareto}
\end{figure}

\dk{\Cref{fig:pareto} illustrates the Pareto front points, highlighting the optimal choices across different model sizes and devices. For example, the AVP (CPU) device achieves the highest speed among the models but exhibits relatively low stability, whereas the ML 2 device offers the greatest stability. Meanwhile, both Vivo and Meta Q3 deliver a balanced trade-off between speed and stability. At the model-pair level, $m_17$ and $m_11$ provide high quality on AVP (CPU) and Vivo, respectively. At the same time, $m_3$ is selected as a Pareto-optimal choice due to its high performance on AVP (CPU) and robust stability on Meta Q3. A similar trend is observed with $m_8$, which attains strong performance and quality on AVP (CPU) as well as commendable quality and stability on ML 2.}



 
% \input{content/05-Demo}
% \newpage 
% \newpage


\section{Discussion}
\label{sec:discussion}


\subsection{RPA: an Algorithm v.s. a System}

Unlike traditional algorithmic innovations in NLP, the design of RPAs can not only support technical innovations to improve LLMs' humanoid capabilities but also enable RPA-based simulation systems for practical benefits.
For instance, from the perspective of psychology, RPAs support the exploration of human cognitive and behavioral activities in controlled yet highly scalable experiments, even in hypothetical scenarios.
In social science, RPAs can deployed as proxies or pilot experiments to analyze and audit social systems, power dynamics, and human societal behaviors at scale.
For the machine learning community, RPAs shed light on dynamic and human-centered model evaluations that are aligned with real-world scenarios by incorporating human and societal factors into consideration.
Last but not least, HCI researchers are particularly intrigued by the implications of RPA systems that can provide personalized assistance with human-centered applications in various sectors, such as medicine, healthcare, and education. 

Nevertheless, RPAs' capability and flexibility are a double-edged sword; they not only have the potential to bring benefits to stakeholders but also expose potential risks and even harm if not responsibly designed.
To what extent do RPAs' responses align with genuine human cognitive activities, whether the cultural, linguistic, and contextual biases learned from the training data of LLMs impact predicted behaviors, and how to ensure RPAs' robustness and consistency under different scenarios, are critical but under-explored challenges for both technical developers and system designers.


As a result, the design of RPAs should incorporate system design considerations while advancing technical explorations.
For instance, RPA design should focus on target users from the very beginning of system design, emphasize the diversity of user backgrounds and perspectives, and iteratively refine the system, as suggested by \citet{gould1985designing} and \citet{shneiderman2010designing} in established design guidelines for system usability.
Nevertheless, differences in cultural norms, linguistic subtleties, and domain-specific knowledge can introduce variability in how RPAs are designed and perceived.
Designers and developers must focus on a balance between generalization and specificity to ensure RPAs are both adaptable and effective across a wide range of scenarios. 
% Establishing best practices for defining agent characteristics and integrating contextual information is a critical area for future research, as it will help standardize the development of RPAs while ensuring their relevance in diverse applications.


\subsection{The Design of RPA Persona}

One of RPAs' key strengths is their ability to adapt to diverse personas, tasks, and environments. But how can RPA personas be designed to ensure that LLMs faithfully and believably reflect the agents' cognitive behaviors within a given task?
Persona descriptions must strike a careful balance between intrinsic agent characteristics and contextual factors, ensuring thoughtful consideration of both the agents' intrinsic characteristics and the contextual information of the specific environments for which the agents are designed. 


% persona related
% \subsection{Designing RPAs: Characteristics and Contextual Descriptions}

% How to design RPAs' descriptions that can guide LLMs faithfully and believably reflect the agents' cognitive and physical behaviors? 
% The descriptions of RPAs have to be sufficiently detailed in both intrinsic and extrinsic factors but not consist of irrelevant content. 
% In particular, the design of RPAs requires careful consideration of both the agents' \textbf{intrinsic characteristics} and the \textbf{contextual information} of the specific environments where the agents are located. 


The \textit{intrinsic characteristics} of RPAs, such as their personal characteristics, education experience, domain expertise, emotional expressiveness, and decision-making processes, must be \textit{aligned with the purpose} of the applications of RPAs.
For example, an RPA designed for psychological experiments should prioritize cognitive characteristics like personality and empathy ability, whereas an RPA developed for economic simulations might emphasize negotiation tactics, competitive reasoning, and adaptability to changing conditions.



% A key strength of RPAs lies in their ability to adapt to diverse personas, tasks, and scenarios, but achieving this flexibility demands thoughtful design strategies. 
% The characteristics of RPAs, such as their personal attributes, knowledge boundaries, reasoning style, emotional expressiveness, and decision-making processes, must align with the goals of the application. 
% For example, an RPA designed for psychological experiments might prioritize empathy, consistency, and transparency, whereas an RPA developed for economic simulations might emphasize negotiation tactics, competitive reasoning, and adaptability to changing conditions.
On the other hand, \textit{contextual information}, such as task- and scenario-specific details, factors, and specifications, is equally critical in shaping the behaviors of RPAs. 
In healthcare applications, for instance, RPAs may simulate caregivers' emotional responses to patients' changing health status but still operate under clinical protocols, such as the ICU visitor rules.
The granularity and fidelity of contextual information heavily influence the believability and effectiveness of the agents' behaviors.



% The role of contextual descriptions is equally critical in shaping the behavior of RPAs. 
% These descriptions provide the scaffolding for the agent’s responses that incorporate task-specific details, user expectations, and environmental factors into their behavior. 
% In healthcare applications, for instance, RPAs may need to simulate a caregiver’s sensitivity to patient emotions while operating within the constraints of clinical protocols. 
% Conversely, in educational settings, RPAs might focus on fostering engagement and curiosity, tailoring their explanations to the learner’s background. 
% The granularity and fidelity of these contextual descriptions heavily influence the believability and effectiveness of the agent’s behavior.



% Evaluation Challenges
\subsection{The Challenges of RPA Evaluation}


% Evaluating RPAs systematically within and across tasks and user scenarios presents a complex challenge. 
The versatility of RPAs, which allows them to function in diverse roles and contexts, makes it infeasible to have a ``one-solution-fits-all'' evaluation metric for systematically evaluate RPAs both within and across tasks and user scenarios.
One major difficulty lies in designing and determining task-oriented and agent-oriented evaluation metrics. 
Despite our work recommending an RPA evaluation design guideline based on a comprehensive review of the literature, existing evaluation metrics may not be sufficient to measure the performance of RPAs for different domain-specific applications.


% \textit{Task-oriented} metrics, such as the accuracy of decision-making, capture the agent’s effectiveness in specific applications. 
% Conversely, \textit{agent-oriented} metrics, which assess qualities like consistency, robustness, and diversity, are essential to understand the role-play capabilities of the agents.

The diversity of user scenarios further exacerbates the evaluation challenge. 
Different tasks may prioritize different aspects of RPAs, making it difficult to develop a one-size-fits-all evaluation framework.
For instance, RPAs designed for psychological research focus on believable emotional responses, whereas RPAs for policymaking simulations underscore robustness to policy changes.


Moreover, cross-task evaluations pose significant challenges due to inconsistencies in how metrics are designed and applied across studies. 
The lack of standardized evaluation criteria complicates systematic benchmarking in RPA development and impedes interdisciplinary collaboration.

Addressing these challenges will require the development of systematic, multi-faceted evaluation frameworks that can accommodate the diverse applications and capabilities of RPAs while providing consistency and comparability across studies.


% \subsection{Practical Implications of RPAs in Various Research Domains}

% Role-Playing Language Agents (RPAs) represent a transformative leap in leveraging large language models to simulate human behavior and cognitive activities from both single-agent and multi-agents' perspectives, and being further extended to interaction patterns and societal dynamics.
% RPAs' ability to role-play diverse personas and perform complex, non-predefined tasks opens a broad avenue of opportunities and challenges across multiple research domains. 
% As this technical advance evolves, critical considerations must be addressed to maximize its potential while mitigating associated potential risks.

% \subsubsection{Psychological and Behavioral Research}
% From the perspective of psychology, for example, RPAs offer unprecedented opportunities to \textbf{explore human cognitive and behavioral activities in controlled yet highly scalable experiments}. 
% By simulating cognitive processes such as reasoning, decision-making, and empathy, RPAs offer researchers a novel approach to replicate and extend psychological studies without suffering from a number of critical but inherent limitations of traditional human subject studies.
% In particular, RPA simulations are ethical and risk-free because they do not rely on human participants, which is a game-changer for experiments with vulnerable populations. 
% Further, RPA simulations can easily mitigate sampling biases and recruitment difficulties by augmenting agent personas and executing the same or different settings on the same group of personas repeatedly and ethically.
% RPAs can also simulate responses in hypothetical scenarios, which was hardly possible for human subject studies and offer new avenues for topics like personality, social interactions, and mental health. 
% However, this potential also raises concerns about experiment validity and ethical boundaries. 
% While RPAs can mimic human behavior, the extent to which their responses align with genuine human cognitive activities remains underexplored. 
% Over-reliance on such tools in psychological research risks conflating simulation with reality, which could potentially lead to flawed conclusions about human behavior.

% \subsubsection{Social Science Research and Policy Analysis}
% In social science, RPAs provide a powerful means to \textbf{analyze and audit social systems, power dynamics, and human societal behaviors at scale}. 
% For instance, applications in studying opinion dynamics or testing ethical decision-making scenarios have already demonstrated RPAs' utility. 
% Additionally, RPAs can be deployed as proxies or pilot experiments to explore sensitive or ethically challenging scenarios, such as understanding the impact of bias or discrimination in institutional settings and testing different policies in virtual environments without harm to any human. 
% However, RPAs' integration into social science research is not without challenges. 
% The interpretability of agent-generated behavior and the potential biases, which are learned from the models' training data and embedded within the underlying models, could amplify existing social inequities rather than mitigate them. 
% Moreover, the deployment of RPAs in the real world must carefully consider cultural, linguistic, and contextual factors to ensure their applicability and fairness across diverse populations.

% ref bibliometrics: AI paper vs HCI paper
% \subsubsection{Machine Learning Development and Evaluation}
% For the machine learning community, RPAs serve as a testbed to advance models and their applications in real-world, human-centered scenarios. 
% Inherent divergences exist between the standardized model development process with static training and testing datasets versus the complicated real-world users' workflows with constantly changing environments.
% In particular, human and societal factors are critical to the technologies' real-world usability when models are integrated into a user-facing system, but none of these factors were considered in the earlier model development stage.
% To bridge the gaps, RPAs shed light on \textbf{dynamic and human-centered model evaluations} that are aligned with real-world user scenarios to support the development and evaluation of robust, generalizable, and adaptive technologies.
% Despite the promising landscape, few have been explored, and challenges are still multifold. 
% The inherent complexity of human cognition poses significant obstacles in creating agents that are realistic and contextually aware.
% How to responsibly develop robust, consistent, and systematic evaluation benchmarks while benefiting from the flexibility of RPAs will be challenging but of significant practical benefits to the broader research community. 

% % human over-expectation of persona

% \subsubsection{Human-Computer Interaction and Design Practices}
% Human-computer interaction (HCI) researchers are particularly intrigued by the implications of RPAs for \textbf{designing systems that seamlessly integrate into human workflows and interactions}. 
% Their ability to simulate empathy, adapt to user preferences, and engage in natural conversations offers immense potential to provide personalized assistance with human-centered applications in various sectors, such as medicine, healthcare, and education. 
% For instance, RPAs can act as virtual tutors, providing personalized feedback and support to learners, or as virtual companions for mental health support, offering empathic interactions that mimic human therapists.
% In healthcare, RPAs can help train clinicians by simulating diverse patient scenarios, testing different intervention strategies harmlessly and ethically, and helping clinicians develop diagnostic and communication skills for hard conversations, such as in palliative care. 
% Furthermore, RPAs' role in designing inclusive and accessible systems is particularly promising, as they can simulate interactions for users with disabilities or language barriers. 

% Nevertheless, RPAs' high degree of flexibility is a double-edged sword; not only has the potential to bring a lot of benefits to stakeholders' workflow but also exposes potential risks and even harm if not responsibly designed.
% Established responsible design guidelines will be indispensable in the interaction design of RPAs, such as ensuring transparency and trustworthiness and preventing misuse, biases and harm. 
% New human-AI collaboration frameworks may emerge to guide the user interface design that enables end-users to understand, control, and collaborate effectively with these novel agents. 
% In addition, the ethical implications of anthropomorphism in RPAs must be carefully examined, as overly human-like behaviors may blur the line between machine and human interaction and lead to unintended emotional or social consequences.

% disciplinary divide


% \subsection{Toward a Unified Research Agenda}


% Looking forward, the interdisciplinary nature of RPAs demands a collaborative effort among different disciplines, particularly between domain experts and technical experts, to explore the aforementioned opportunities while addressing the potential challenges. 
% First, the development of a systematic guideline for RPAs' evaluation frameworks is critical.
% The unified guideline outlined in this study for the evaluation and deployment of RPAs is essential to ensure their effectiveness, safety, and ethical alignment that accounts for task-specific performance and agent-specific attributes. 
% % Moreover, it is crucial to incorporate perspectives from diverse stakeholders, including researchers, policymakers, and end-users, to establish guidelines that are inclusive and contextually sensitive.



% Ethical and inclusive practices must also be prioritized to mitigate the biases embedded in LLMs.
% Ensuring that RPAs are designed and deployed in ways that promote fairness and inclusivity is essential, particularly as they are increasingly applied in sensitive contexts such as healthcare, education, and social research. 
% The anthropomorphic nature of RPAs introduces additional ethical complexities, as overly human-like behaviors performed by RPAs may blur the line between machine and human interaction and lead to unintended emotional or social consequences.
% Careful consideration of these aspects will help ensure that RPAs are used responsibly and ethically.

% Interdisciplinary collaboration plays a critical role in advancing the potential of RPAs. Psychology, social science, machine learning, HCI, and more disciplines bring unique perspectives and methodologies to the table. 
% By promoting conversation and collaboration between researchers from these fields, it is possible to leverage their respective strengths to address both technical and societal challenges. 
% Such collaboration would facilitate the integration of human-centered design principles with advanced computational techniques for the design of RPAs that are technically robust and aligned with user needs.

% Finally, exploring real-world applications of RPAs is an essential step in understanding their broader implications. 
% Investigating how RPAs can enhance human-AI collaboration in practical scenarios, such as education, mental health support, or policy simulation, will provide valuable insights into their usability, safety, and societal impact. 
% These efforts must be guided by an overarching commitment to ensuring that RPAs contribute meaningfully to human welfare and knowledge creation.

% % In summary, a unified research agenda for RPAs should focus on establishing systematic evaluation frameworks, promoting ethical and inclusive practices, fostering interdisciplinary collaboration, and exploring real-world applications. By addressing these priorities, the research community can unlock the transformative potential of RPAs to advance knowledge, drive innovation, and create meaningful societal impact.

% % The evaluation still needs the effort of human/experts

% % Potential risks of anthropomorphism 
% % intentionally misuse the RPA to do harmful things

% \subsection{From Human-in-the-Loop to Human-on-the-Loop and Beyond}

% The advent of RPAs marks a significant shift in the traditional paradigms of human-AI collaboration. 
% Unlike conventional AI systems, which typically operate within a pre-defined prediction space for AI models and rely on human-in-the-loop mechanisms to monitor and evaluate AI systems' performance, RPAs offer new paradigms of interaction enabled by the freedom and generative capabilities of LLMs.
% This shift introduces two emerging collaboration paradigms: \textbf{human-on-the-loop}, where humans monitor and guide AI systems executing semi-autonomous actions, and \textbf{human-out-of-the-loop}, where AI agents operate independently without real-time human intervention. 
% Both paradigms present unique opportunities and challenges that require careful exploration.

% The human-on-the-loop paradigm positions RPAs as semi-autonomous collaborators that can execute a series of tasks with minimal but critical human supervision. 
% This paradigm offers significant advantages in dynamic or high-stakes environments where continuous human monitoring of AI performance is expensive, inaccessible, or impractical. 
% For example, RPAs can autonomously simulate negotiation strategies, generate alternative solutions, or interact with multiple stakeholders in real time, which allows human supervisors to intervene only when necessary. 
% This paradigm can enhance efficiency by reducing cognitive and operational burdens on human users while still maintaining human supervision to ensure the safety and reliability of RPA-based AI systems in humans' workflow. 
% % In domains such as healthcare, policy simulation, and collaborative problem-solving, human-on-the-loop systems can streamline workflows and enable more scalable applications of AI.
% However, this paradigm introduces concerns regarding the scope and reliability of human supervision. Effective human-on-the-loop collaboration depends on the system’s ability to communicate its reasoning, limitations, and uncertainties to the human supervisor. 
% Without adequate explainability and transparency, humans may either over-rely on the AI or intervene unnecessarily, undermining the benefits of the paradigm. 

% The human-out-of-the-loop paradigm, where RPAs operate independently without real-time human intervention, represents an even more transformative opportunity. 
% In scenarios such as large-scale simulations, long-term automated interactions, or environments where immediate human involvement is infeasible, human-out-of-the-loop RPA systems can autonomously generate and evaluate solutions, collect data, and adapt their strategies based on dynamically changing contextual information. 
% This capability is particularly valuable in domains such as economic modeling, environmental simulations, and social system analysis, where the complexity and scale of the tasks make human real-time intervention impractical.
% Despite its potential, the human-out-of-the-loop paradigm also raises significant concerns. 
% The autonomy granted to RPAs increases the risk of unintended or undesirable outcomes, particularly when the agent encounters scenarios that fall outside its training data or reasoning capabilities. 
% Ensuring that RPAs adhere to ethical principles and domain-specific constraints without direct human involvement requires robust pre-deployment safeguards, including rigorous testing, built-in fail-safes, and post-hoc evaluation mechanisms. 
% Moreover, the potential for RPAs to make decisions that could have real-world consequences—such as in policy recommendations or behavioral studies—necessitates the development of regulatory frameworks to govern their deployment.

% These new paradigms also shift the traditional role of humans in AI collaborations. Instead of actively intervening in decision-making processes, humans are increasingly tasked with designing, monitoring, and evaluating the systems that perform autonomous or semi-autonomous actions. 
% This shift emphasizes the importance of interdisciplinary research, combining expertise from HCI, machine learning, and ethics to ensure that these new collaboration paradigms are effective, transparent, and aligned with human values. 
% As RPAs continue to evolve, understanding how to balance autonomy with accountability will be a critical factor in their success across various domains. 


% \section*{Acknowledgments}
% This should be a simple paragraph before the References to thank those individuals and institutions who have supported your work on this article.


%{\appendices
%\section*{Proof of the First Zonklar Equation}
%Appendix one text goes here.
% You can choose not to have a title for an appendix if you want by leaving the argument blank
%\section*{Proof of the Second Zonklar Equation}
%Appendix two text goes here.}



% \section{References Section}
% You can use a bibliography generated by BibTeX as a .bbl file.
%  BibTeX documentation can be easily obtained at:
%  http://mirror.ctan.org/biblio/bibtex/contrib/doc/
%  The IEEEtran BibTeX style support page is:
%  http://www.michaelshell.org/tex/ieeetran/bibtex/
 
 % argument is your BibTeX string definitions and bibliography database(s)
%\bibliography{IEEEabrv,../bib/paper}
%
% \section{Simple References}
% You can manually copy in the resultant .bbl file and set second argument of $\backslash${\tt{begin}} to the number of references
%  (used to reserve space for the reference number labels box).

%\begin{thebibliography}{1}
\bibliographystyle{IEEEtran}  
%\bibliographystyle{abbrv}
%\bibliographystyle{abbrv-doi}
%\bibliographystyle{abbrv-doi-narrow}
%\bibliographystyle{abbrv-doi-hyperref}
%\bibliographystyle{abbrv-doi-hyperref-narrow}

\bibliography{Llamaxr}
%\end{thebibliography}

% \newpage

% \section{Biography Section}
% If you have an EPS/PDF photo (graphicx package needed), extra braces are
%  needed around the contents of the optional argument to biography to prevent
%  the LaTeX parser from getting confused when it sees the complicated
%  $\backslash${\tt{includegraphics}} command within an optional argument. (You can create
%  your own custom macro containing the $\backslash${\tt{includegraphics}} command to make things
%  simpler here.)
 
% \vspace{11pt}

% \bf{If you include a photo:}\vspace{-33pt}
% \begin{IEEEbiography}[{\includegraphics[width=1in,height=1.25in,clip,keepaspectratio]{fig1}}]{Michael Shell}
% Use $\backslash${\tt{begin\{IEEEbiography\}}} and then for the 1st argument use $\backslash${\tt{includegraphics}} to declare and link the author photo.
% Use the author name as the 3rd argument followed by the biography text.
% \end{IEEEbiography}

% \vspace{11pt}

% \bf{If you will not include a photo:}\vspace{-33pt}
% \begin{IEEEbiographynophoto}{John Doe}
% Use $\backslash${\tt{begin\{IEEEbiographynophoto\}}} and the author name as the argument followed by the biography text.
% \end{IEEEbiographynophoto}
%\vfill
\vspace{-33pt} 
\begin{IEEEbiographynophoto}{Jane Doe}
    Biography text here without a photo.
\end{IEEEbiographynophoto}

\begin{IEEEbiography}[
    %{\includegraphics[width=1in,height=1.25in,clip,keepaspectratio]{}}
    ]{IEEE Publications Technology Team}
    In this paragraph you can place your educational, professional background and research and other interests.\end{IEEEbiography}

\vfill
\end{document}


