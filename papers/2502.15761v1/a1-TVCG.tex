%\documentclass[lettersize,journal]{IEEEtran}
\documentclass[10pt,journal, compsoc]{IEEEtran}
\usepackage{amsmath,amsfonts}
% \usepackage{algorithmic}
% \usepackage{algorithm}
% \usepackage{array}
\usepackage[caption=false,font=normalsize,labelfont=sf,textfont=sf]{subfig}
\usepackage{textcomp}
\usepackage{stfloats}
\usepackage{url}
\usepackage{verbatim}
\usepackage{graphicx}
\usepackage{cite}
\hyphenation{op-tical net-works semi-conduc-tor IEEE-Xplore}
% updated with editorial comments 8/9/2021
 \usepackage{times}                     % we use Times as the main font
\renewcommand*\ttdefault{txtt}         % a nicer typewriter font

%% Only used in the template examples. You can remove these lines.
\usepackage{tabu}                      % only used for the table example
\usepackage{booktabs}                  % only used for the table example
\usepackage{lipsum}                    % used to generate placeholder text
\usepackage{mwe}                       % used to generate placeholder figures
\usepackage{multirow}
%% We encourage the use of mathptmx for consistent usage of times font
%% throughout the proceedings. However, if you encounter conflicts
%% with other math-related packages, you may want to disable it.
\usepackage{mathptmx}                  % use matching math font
\usepackage{glossaries}
\usepackage{graphicx}
%\usepackage{graphicx}
\usepackage{xspace}                    % used to manage spacing at the end of macros (commands.tex)
\usepackage{svg}
\usepackage{verbatim}

\usepackage{placeins}
\usepackage{adjustbox}
\usepackage{mathptmx}                  % use matching math font
\usepackage{glossaries}
\usepackage{graphicx}
%\usepackage{graphicx}
\usepackage{xspace}                    % used to manage spacing at the end of macros (commands.tex)
\usepackage{svg}
\usepackage{verbatim}

\usepackage{placeins}
\usepackage{adjustbox}

  \usepackage{enumitem}
 \usepackage{rotating}
\usepackage{longtable}

\usepackage{orcidlink}


 
% basic
%\usepackage{color,xcolor}
\usepackage{color}
\usepackage{epsfig}
\usepackage{graphicx}
\usepackage{algorithm,algorithmic}
% \usepackage{algpseudocode}
%\usepackage{ulem}

% figure and table
\usepackage{adjustbox}
\usepackage{array}
\usepackage{booktabs}
\usepackage{colortbl}
\usepackage{float,wrapfig}
\usepackage{framed}
\usepackage{hhline}
\usepackage{multirow}
% \usepackage{subcaption} % issues a warning with CVPR/ICCV format
% \usepackage[font=small]{caption}
\usepackage[percent]{overpic}
%\usepackage{tikz} % conflict with ECCV format

% font and character
\usepackage{amsmath,amsfonts,amssymb}
% \let\proof\relax      % for ECCV llncs class
% \let\endproof\relax   % for ECCV llncs class
\usepackage{amsthm} 
\usepackage{bm}
\usepackage{nicefrac}
\usepackage{microtype}
\usepackage{contour}
\usepackage{courier}
%\usepackage{palatino}
%\usepackage{times}

% layout
\usepackage{changepage}
\usepackage{extramarks}
\usepackage{fancyhdr}
\usepackage{lastpage}
\usepackage{setspace}
\usepackage{soul}
\usepackage{xspace}
\usepackage{cuted}
\usepackage{fancybox}
\usepackage{afterpage}
%\usepackage{enumitem} % conflict with IEEE format
%\usepackage{titlesec} % conflict with ECCV format

% ref
% commenting these two out for this submission so it looks the same as RSS example
% \usepackage[breaklinks=true,colorlinks,backref=True]{hyperref}
% \hypersetup{colorlinks,linkcolor={black},citecolor={MSBlue},urlcolor={magenta}}
\usepackage{url}
\usepackage{quoting}
\usepackage{epigraph}

% misc
\usepackage{enumerate}
\usepackage{paralist,tabularx}
\usepackage{comment}
\usepackage{pdfpages}
% \usepackage[draft]{todonotes} % conflict with CVPR/ICCV/ECCV format



% \usepackage{todonotes}
% \usepackage{caption}
% \usepackage{subcaption}

\usepackage{pifont}% http://ctan.org/pkg/pifont

% extra symbols
\usepackage{MnSymbol}


%%%%%%%%%%%---SETME-----%%%%%%%%%%%%%
%replace @@ with the submission number submission site.
\newcommand{\thiswork}{INF$^2$\xspace}
%%%%%%%%%%%%%%%%%%%%%%%%%%%%%%%%%%%%


%\newcommand{\rev}[1]{{\color{olivegreen}#1}}
\newcommand{\rev}[1]{{#1}}


\newcommand{\JL}[1]{{\color{cyan}[\textbf{\sc JLee}: \textit{#1}]}}
\newcommand{\JW}[1]{{\color{orange}[\textbf{\sc JJung}: \textit{#1}]}}
\newcommand{\JY}[1]{{\color{blue(ncs)}[\textbf{\sc JSong}: \textit{#1}]}}
\newcommand{\HS}[1]{{\color{magenta}[\textbf{\sc HJang}: \textit{#1}]}}
\newcommand{\CS}[1]{{\color{navy}[\textbf{\sc CShin}: \textit{#1}]}}
\newcommand{\SN}[1]{{\color{olive}[\textbf{\sc SNoh}: \textit{#1}]}}

%\def\final{}   % uncomment this for the submission version
\ifdefined\final
\renewcommand{\JL}[1]{}
\renewcommand{\JW}[1]{}
\renewcommand{\JY}[1]{}
\renewcommand{\HS}[1]{}
\renewcommand{\CS}[1]{}
\renewcommand{\SN}[1]{}
\fi

%%% Notion for baseline approaches %%% 
\newcommand{\baseline}{offloading-based batched inference\xspace}
\newcommand{\Baseline}{Offloading-based batched inference\xspace}


\newcommand{\ans}{attention-near storage\xspace}
\newcommand{\Ans}{Attention-near storage\xspace}
\newcommand{\ANS}{Attention-Near Storage\xspace}

\newcommand{\wb}{delayed KV cache writeback\xspace}
\newcommand{\Wb}{Delayed KV cache writeback\xspace}
\newcommand{\WB}{Delayed KV Cache Writeback\xspace}

\newcommand{\xcache}{X-cache\xspace}
\newcommand{\XCACHE}{X-Cache\xspace}


%%% Notions for our methods %%%
\newcommand{\schemea}{\textbf{Expanding supported maximum sequence length with optimized performance}\xspace}
\newcommand{\Schemea}{\textbf{Expanding supported maximum sequence length with optimized performance}\xspace}

\newcommand{\schemeb}{\textbf{Optimizing the storage device performance}\xspace}
\newcommand{\Schemeb}{\textbf{Optimizing the storage device performance}\xspace}

\newcommand{\schemec}{\textbf{Orthogonally supporting Compression Techniques}\xspace}
\newcommand{\Schemec}{\textbf{Orthogonally supporting Compression Techniques}\xspace}



% Circular numbers
\usepackage{tikz}
\newcommand*\circled[1]{\tikz[baseline=(char.base)]{
            \node[shape=circle,draw,inner sep=0.4pt] (char) {#1};}}

\newcommand*\bcircled[1]{\tikz[baseline=(char.base)]{
            \node[shape=circle,draw,inner sep=0.4pt, fill=black, text=white] (char) {#1};}}  
\makeglossaries
\newacronym{rl}{RL}{Reinforcement Learning}
\newacronym{drl}{DRL}{Deep Reinforcement Learning}
\newacronym{mdp}{MDP}{Markov Decision Process}
\newacronym{ppo}{PPO}{Proximal Policy Optimization}
\newacronym{sac}{SAC}{Soft Actor-Critic}
\newacronym{epvf}{EPVF}{Explicit Policy-conditioned Value Function}
\newacronym{unf}{UNF}{Universal Neural Functional}
\begin{document}
\title{\dk{LoXR: Performance Evaluation of Locally Executing LLMs on XR Devices}}
%\title{LLMs on XR (LoXR): Performance Evaluation of LLMs Executed Locally on XR Devices }
  \vskip -0.365cm
\author{ Dawar Khan$\orcidlink{0000-0001-5864-1888}$, Xinyu Liu , Omar Mena\orcidlink{https://orcid.org/0009-0002-6564-0175}, Donggan Jia\orcidlink{https://orcid.org/0000-0002-1358-8718}, lexandre Kouyoumdjian\orcidlink{https://orcid.org/0009-0005-0915-369X}, Ivan Viola\orcidlink{https://orcid.org/0000-0003-4248-6574}

\IEEEcompsocitemizethanks{ 
\IEEEcompsocthanksitem Dawar~Khan, Omar Mena, Donggang Jia, Alexandre Kouyoumdjian, Ivan~Viola are with King Abdullah University of Science and Technology (KAUST), Saudi Arabia. E-mail: \{xinyu.liu, dawar.khan, omar.mena, donggang.jia, alexandre.kouyoumdjian, ivan.viola\}@kaust.edu.sa. 
\IEEEcompsocthanksitem Xinyu Liu is with King Abdullah University of Science and Technology (KAUST), Saudi Arabia, and also with University of Electronic Science and Technology of China, Chengdu, China.
\IEEEcompsocthanksitem  D. Khan and X. Liu are joint first authors with equal contributions.}
\thanks{Manuscript received MM dd, YYYY; revised MM dd, YYYY.\\ (Corresponding author: Dawar Khan.)}}

        % <-this % stops a space
% \thanks{This paper was produced by the IEEE Publication Technology Group. They are in Piscataway, NJ.}% <-this % stops a space
% \thanks{Manuscript received April 19, 2021; revised August 16, 2021.}}

% The paper headers
\markboth{Arxiv: Feb.~2025}%
{Shell \MakeLowercase{\textit{et al.}}: A Sample Article Using IEEEtran.cls for IEEE Journals}

% \IEEEpubid{0000--0000/00\$00.00~\copyright~2021 IEEE}
% Remember, if you use this you must call \IEEEpubidadjcol in the second
% column for its text to clear the IEEEpubid mark.

\IEEEtitleabstractindextext{ 
\begin{abstract} 
\begin{abstract}
Role-Playing Agent (RPA) is an increasingly popular type of LLM Agent that simulates human-like behaviors in a variety of tasks. 
However, evaluating RPAs is challenging due to diverse task requirements and agent designs.
This paper proposes an evidence-based, actionable, and generalizable evaluation design guideline for LLM-based RPA by systematically reviewing $1,676$ papers published between Jan. 2021 and Dec. 2024.
Our analysis identifies six agent attributes, seven task attributes, and seven evaluation metrics from existing literature.
Based on these findings, we present an RPA evaluation design guideline to help researchers develop more systematic and consistent evaluation methods.

\end{abstract}


% to synthesize what agent attributes and task attributes prior literature have considered influence the selection of evaluation metrics, as well as the relationships between these factors.
% For each agent attribute and task category, we summarize its distinct associations with RPLA's evaluation metrics, providing practical guidance on comprehensive based on their RPLA's design. Additionally, we explore the  between agent attributes and downstream tasks to support researchers in refining RPLA design choices.
\end{abstract}
\begin{IEEEkeywords}
Extended reality, large language models, human AI interactions,  performance evaluation, conversational user interfaces
\end{IEEEkeywords}
 } 
\maketitle

% \begin{abstract}
% This document describes the most common article elements and how to use the IEEEtran class with \LaTeX \ to produce files that are suitable for submission to the IEEE.  IEEEtran can produce conference, journal, and technical note (correspondence) papers with a suitable choice of class options. 
% \end{abstract}

% \begin{IEEEkeywords}
% Article submission, IEEE, IEEEtran, journal, \LaTeX, paper, template, typesetting.
% \end{IEEEkeywords}

%\section{Introduction}
%\IEEEPARstart{T}{his} 

 \IEEEdisplaynontitleabstractindextext 
 \IEEEpeerreviewmaketitle
%\maketitle
% % Use this for peer review papers
% \IEEEpeerreviewmaketitle
\IEEEraisesectionheading{\section{Introduction}}




%\IEEEraisesectionheading{\section{Introduction}}

% humans are sensitive to the way information is presented.

% introduce framing as the way we address framing. say something about political views and how information is represented.

% in this paper we explore if models show similar sensitivity.

% why is it important/interesting.



% thought - it would be interesting to test it on real world data, but it would be hard to test humans because they come already biased about real world stuff, so we tested artificial.


% LLMs have recently been shown to mimic cognitive biases, typically associated with human behavior~\citep{ malberg2024comprehensive, itzhak-etal-2024-instructed}. This resemblance has significant implications for how we perceive these models and what we can expect from them in real-world interactions and decisionmaking~\citep{eigner2024determinants, echterhoff-etal-2024-cognitive}.

The \textit{framing effect} is a well-known cognitive phenomenon, where different presentations of the same underlying facts affect human perception towards them~\citep{tversky1981framing}.
For example, presenting an economic policy as only creating 50,000 new jobs, versus also reporting that it would cost 2B USD, can dramatically shift public opinion~\cite{sniderman2004structure}. 
%%%%%%%% 图1:  %%%%%%%%%%%%%%%%
\begin{figure}[t]
    \centering
    \includegraphics[width=\columnwidth]{Figs/01.pdf}
    \caption{Performance comparison (Top-1 Acc (\%)) under various open-vocabulary evaluation settings where the video learners except for CLIP are tuned on Kinetics-400~\cite{k400} with frozen text encoders. The satisfying in-context generalizability on UCF101~\cite{UCF101} (a) can be severely affected by static bias when evaluating on out-of-context SCUBA-UCF101~\cite{li2023mitigating} (b) by replacing the video background with other images.}
    \label{fig:teaser}
\end{figure}


Previous research has shown that LLMs exhibit various cognitive biases, including the framing effect~\cite{lore2024strategic,shaikh2024cbeval,malberg2024comprehensive,echterhoff-etal-2024-cognitive}. However, these either rely on synthetic datasets or evaluate LLMs on different data from what humans were tested on. In addition, comparisons between models and humans typically treat human performance as a baseline rather than comparing patterns in human behavior. 
% \gabis{looks good! what do we mean by ``most studies'' or ``rarely'' can we remove those? or we want to say that we don't know of previous work doing both at the same time?}\gili{yeah the main point is that some work has done each separated, but not all of it together. how about now?}

In this work, we evaluate LLMs on real-world data. Rather than measuring model performance in terms of accuracy, we analyze how closely their responses align with human annotations. Furthermore, while previous studies have examined the effect of framing on decision making, we extend this analysis to sentiment analysis, as sentiment perception plays a key explanatory role in decision-making \cite{lerner2015emotion}. 
%Based on this, we argue that examining sentiment shifts in response to reframing can provide deeper insights into the framing effect. \gabis{I don't understand this last claim. Maybe remove and just say we extend to sentiment analysis?}

% Understanding how LLMs respond to framing is crucial, as they are increasingly integrated into real-world applications~\citep{gan2024application, hurlin2024fairness}.
% In some applications, e.g., in virtual companions, framing can be harnessed to produce human-like behavior leading to better engagement.
% In contrast, in other applications, such as financial or legal advice, mitigating the effect of framing can lead to less biased decisions.
% In both cases, a better understanding of the framing effect on LLMs can help develop strategies to mitigate its negative impacts,
% while utilizing its positive aspects. \gabis{$\leftarrow$ reading this again, maybe this isn't the right place for this paragraph. Consider putting in the conclusion? I think that after we said that people have worked on it, we don't necessarily need this here and will shorten the long intro}


% If framing can influence their outputs, this could have significant societal effects,
% from spreading biases in automated decision-making~\citep{ghasemaghaei2024understanding} to reducing public trust in AI-generated content~\citep{afroogh2024trust}. 
% However, framing is not inherently negative -- understanding how it affects LLM outputs can offer valuable insights into both human and machine cognition.
% By systematically investigating the framing effect,


%It is therefore crucial to systematically investigate the framing effect, to better understand and mitigate its impact. \gabis{This paragraph is important - I think that right now it's saying that we don't want models to be influenced by framing (since we want to mitigate its impact, right?) When we talked I think we had a more nuanced position?}




To better understand the framing effect in LLMs in comparison to human behavior,
we introduce the \name{} dataset (Section~\ref{sec:data}), comprising 1,000 statements, constructed through a three-step process, as shown in Figure~\ref{fig:fig1}.
First, we collect a set of real-world statements that express a clear negative or positive sentiment (e.g., ``I won the highest prize'').
%as exemplified in Figure~\ref{fig:fig1} -- ``I won the highest prize'' positive base statement. (2) next,
Second, we \emph{reframe} the text by adding a prefix or suffix with an opposite sentiment (e.g., ``I won the highest prize, \emph{although I lost all my friends on the way}'').
Finally, we collect human annotations by asking different participants
if they consider the reframed statement to be overall positive or negative.
% \gabist{This allows us to quantify the extent of \textit{sentiment shifts}, which is defined as labeling the sentiment aligning with the opposite framing, rather then the base sentiment -- e.g., voting ``negative'' for the statement ``I won the highest prize, although I lost all my friends on the way'', as it aligns with the opposite framing sentiment.}
We choose to annotate Amazon reviews, where sentiment is more robust, compared to e.g., the news domain which introduces confounding variables such as prior political leaning~\cite{druckman2004political}.


%While the implications of framing on sensitive and controversial topics like politics or economics are highly relevant to real-world applications, testing these subjects in a controlled setting is challenging. Such topics can introduce confounding variables, as annotators might rely on their personal beliefs or emotions rather than focusing solely on the framing, particularly when the content is emotionally charged~\cite{druckman2004political}. To balance real-world relevance with experimental reliability, we chose to focus on statements derived from Amazon reviews. These are naturally occurring, sentiment-rich texts that are less likely to trigger strong preexisting biases or emotional reactions. For instance, a review like ``The book was engaging'' can be framed negatively without invoking specific cultural or political associations. 

 In Section~\ref{sec:results}, we evaluate eight state-of-the-art LLMs
 % including \gpt{}~\cite{openai2024gpt4osystemcard}, \llama{}~\cite{dubey2024llama}, \mistral{}~\cite{jiang2023mistral}, \mixtral{}~\cite{mistral2023mixtral}, and \gemma{}~\cite{team2024gemma}, 
on the \name{} dataset and compare them against human annotations. We find  that LLMs are influenced by framing, somewhat similar to human behavior. All models show a \emph{strong} correlation ($r>0.57$) with human behavior.
%All models show a correlation with human responses of more than $0.55$ in Pearson's $r$ \gabis{@Gili check how people report this?}.
Moreover, we find that both humans and LLMs are more influenced by positive reframing rather than negative reframing. We also find that larger models tend to be more correlated with human behavior. Interestingly, \gpt{} shows the lowest correlation with human behavior. This raises questions about how architectural or training differences might influence susceptibility to framing. 
%\gabis{this last finding about \gpt{} stands in opposition to the start of the statement, right? Even though it's probably one of the largest models, it doesn't correlate with humans? If so, better to state this explicitly}

This work contributes to understanding the parallels between LLM and human cognition, offering insights into how cognitive mechanisms such as the framing effect emerge in LLMs.\footnote{\name{} data available at \url{https://huggingface.co/datasets/gililior/WildFrame}\\Code: ~\url{https://github.com/SLAB-NLP/WildFrame-Eval}}

%\gabist{It also raises fundamental philosophical and practical questions -- should LLMs aim to emulate human-like behavior, even when such behavior is susceptible to harmful cognitive biases? or should they strive to deviate from human tendencies to avoid reproducing these pitfalls?}\gabis{$\leftarrow$ also following Itay's comment, maybe this is better in the dicsussion, since we don't address these questions in the paper.} %\gabis{This last statement brings the nuance back, so I think it contradicts the previous parapgraph where we talked about ``mitigating'' the effect of framing. Also, I think it would be nice to discuss this a bit more in depth, maybe in the discussion section.}






\section{Related Work}
\label{sec:related-work}

We review related work with respect to two major research directions. The first research direction is exploring the use of LLMs on XR devices, and the second research direction explores how to make LLMs directly deployable on low-resource hardware.
% \h{Research on LLMs:}\\
% \cite{Fan2024ReviewLLMs}\\
% \cite{meta2023introducingLLaMA}\\
% \cite{Brown2020NEURIPSLMareFewshotLearner}\\
% \h{On-device LLMs:}\\
% mobileLLM\\
% \h{LLM-based applications on VR/XR}\\
% \h{AI and XR:}\\
% \cite{Bozkir2024Review:LLMXR}
% \cite{HirzleXRandAIReview}\\

\subsection{LLM-powered XR Applications}
Owing to the powerful semantic understanding and extensive general knowledge of LLMs, numerous studies have explored their application in assisting various tasks within XR scenarios. 

Torre \etal~\cite{de2024llmr} introduced \emph{LLMR}, a framework that leverages LLMs for the real-time creation and modification of interactive mixed reality experiences, enabling tasks such as generating new content or editing existing works on VR/AR devices. Jia \etal~\cite{Jia2024_1} developed the \emph{VOICE} framework, which employs a two-layer agent system for conversational interaction and explanation in scientific communication, with a prototype deployable on VR devices. Kurai \etal~\cite{kurai2024magicitem} proposed \emph{MagicItem}, a tool that allows users with limited programming experience to define object behaviors within the metaverse platform. Zhang \etal~\cite{zhang2024odoragent} introduced \emph{OdorAgent}, which combines an LLM with a text-image model to automate video-odor matching. Yin \etal~\cite{yin2024text2vrscene} identified potential limitations in LLM-based automated systems and proposed the systematic framework \emph{Text2VRScene} to address them. Chen \etal~\cite{chen2024supporting} leveraged the extensive capabilities of LLMs in context perception and text prediction to enhance text entry efficiency by reducing manual keystrokes in VR scenarios. Giunchi \etal~\cite{giunchi2024dreamcodevr} developed \emph{DreamCodeVR}, a tool designed to help users, regardless of coding experience, create basic object behavior in VR environments by translating spoken language into code within an active application. Wan \etal~\cite{wan2024building} presented an LLM-based AI agent for human-agent interaction in VR, involving GPT-4 to simulate realistic NPC behavior, including context-aware responses, facial expressions, and body gestures.

As they stand, these previous contributions still require the deployed devices to connect to a cloud server for LLM inference, raising concerns about user privacy, latency, costs, and internet access requirements. In our work, we evaluate the local inference performance of various LLMs and propose a prototype capable of using local LLM inference.

\subsection{On-device LLMs}
Running LLMs on edge devices, commonly called on-device LLMs, has garnered significant research interest due to their advantages in enhancing privacy, reducing latency, and operating without the need for internet connectivity. Because of the limited memory and computing capabilities, on-device LLMs usually require resource-efficient
LLM deployment~\cite{qu2024mobile}: a trade-off between performance and model size.

Cheng \etal~\cite{cheng2023optimize} introduced the \emph{SignRound} method, which leverages signed gradient descent to optimize both rounding values and weight clipping. This approach achieves outstanding performance in 2- to 4-bit quantization while maintaining low tuning costs and eliminating additional inference overhead. Ma \etal~\cite{ma2023llm} developed the \emph{LLM-Pruner} method, which uses structural pruning to selectively remove non-essential coupled structures based on gradient information, effectively preserving the core functionality of the LLM. Their results demonstrate that the compressed models continue to perform well in tasks such as zero-shot classification and generation. Gu \etal~\cite{gu2024minillm} introduced a novel knowledge distillation method that compresses LLMs into smaller models, resulting in the student model \emph{MINILLM}. \emph{MINILLM} demonstrates superior performance compared to baseline models, producing more precise responses with reduced exposure bias, improved calibration, and enhanced long-text generation capabilities. Liu \etal~\cite{liu2024kivi} developed a tuning-free 2-bit KV cache quantization algorithm, named \emph{KIVI}, which independently quantizes the key cache per channel and the value cache per token. This approach enabled up to a 4× increase in batch size, resulting in a $2.35\times$ to $3.47\times$ improvement in throughput on real LLM inference workloads. Liu \etal~\cite{liu2024mobilellm} introduced \emph{MobileLLM}, which explores the importance of model architecture for sub-billion-parameter LLMs by leveraging deep and narrow architectures, combined with embedding sharing and grouped-query attention mechanisms. 

Although these studies primarily focused on reducing model size efficiently for deployment on various devices, they did not evaluate the models' performance in real-world on-device scenarios, which is the focus of our paper.


\section{Method}

We conduct a systematic literature review to address our research question. Following prior method~\cite{nightingale2009guide}, we aim to identify relevant research papers on RPAs and provide a comprehensive summary of the literature. We selected four widely used academic databases: Google Scholar, ACM Digital Library, IEEE Xplore, and ACL Anthology. These databases encompass a broad spectrum of research across AI, human-computer interaction, and computational linguistics. Given the rapid advancements in LLM research, we included both peer-reviewed and preprint studies (e.g., from arXiv) to capture the latest developments. Below, we detail our paper selection and annotation process.

\begin{table*}[t]
\small

\caption{Definition and examples of six agent attributes.}
% \vspace{-0.5em}
\resizebox{\textwidth}{!}{%
\begin{tabular}{@{}p{0.21\textwidth}p{0.45\textwidth}p{0.33\textwidth}@{}}
\toprule
\textbf{Agent attributes}     & \textbf{Definition}     & \textbf{Examples} \\ 
\midrule
Activity History        & A record of past actions, behaviors, and engagements, including schedules, browsing history, and lifestyle choices. & Backstory, plot, weekly schedule, browsing history, social media posts, lifestyle       \\ 
Belief and Value        & The principles, attitudes, and ideological stances that shape an individual's perspectives and decisions.           & Stances, beliefs, attitudes, values, political leaning, religion                            \\ 
Demographic Information & Personal identifying details such as name, age, education, career, and location.                                    & Name, appearance, gender, age, date of birth, education, location, career, household income \\ 
Psychological Traits    & Characteristics related to personality, emotions, interests, and cognitive tendencies.                              & Personality, hobby and interest, emotional                                                  \\ 
Skill and Expertise     & The knowledge level, proficiency, and capability in specific domains or technologies.                             & Knowledge level, technology proficiency, skills                                            \\ 
Social Relationships & The nature and dynamics of interactions with others, including roles, connections, and communication styles.        & Parenting styles, interactions with players                                                \\ 
\bottomrule
\end{tabular}
}
\vspace{-1em}
\label{attr_def}
\end{table*}

\subsection{Literature Search and Screening Method}

\begin{figure}
    \includegraphics[width=\linewidth]{Figures/simple-PRISMA-1.png}
    % \vspace{-1em}
    \caption{Screening process of literature review. We initially retrieved $1,676$ papers published between 2021 and 2024, and narrowed down to $122$ final selections.}
    \vspace{-1em}
    \label{fig:prisma}
\end{figure}

Our literature review focuses on LLM agents that role-play human behaviors, such as decision-making, reasoning, and deliberate actions. We specifically focus on studies where LLM agents demonstrate the ability to simulate human-like cognitive processes in their objectives, methodologies, or evaluation techniques. To ensure methodological rigor, we define explicit inclusion and exclusion criteria (Tab.~\ref{tab:criteria} in Appendix~\ref{tab: inclusion and exclusion criteria}). 

The inclusion criteria require that an LLM agent in the study exhibits human-like behavior, engages in cognitive activities such as decision-making or reasoning, and operates in an open-ended task environment. We excluded studies where LLM agents primarily serve as chatbots, task-specific assistants, evaluators, or agents operating within predefined and finite action spaces. Additionally, studies focusing solely on perception-based tasks (e.g., computer vision or sensor-based autonomous driving) without cognitive simulation were also excluded.

Using this scope, we searched four databases using the query string provided in Appendix~\ref{query string}, retrieving $1,676$ papers published between January 2021 to December 2024. After removing duplicates, $1,573$ unique papers remained. Two authors independently screened the paper titles and abstracts based on the inclusion criteria. If at least one author deemed a paper relevant, it proceeded to full-text screening, where two authors reviewed the paper in detail and resolved any disagreements through discussion (Fig.~\ref{fig:prisma}). The final set of selected studies comprised $122$ publications.


\subsection{Paper Annotation Method}
Our team followed established open coding procedures \cite{brod2009qualitative} to conduct an inductive coding process to identify key themes. Three co-authors with extensive experience in LLM agents (``annotators,'' hereinafter) collaboratively annotated the papers on three dimensions: \textbf{agent attributes}, \textbf{task attributes}, and \textbf{evaluation metrics}. 

To ensure consistency, two annotators independently annotated the same 20\% of articles and then held a meeting to discuss and refine an initial set of categories for the three dimensions. After reaching a consensus, each annotator annotated half of the remaining papers and cross-validated the other half annotated by the other annotator. Once the annotations were completed, a third annotator reviewed the coded data and identified potential discrepancies. 
Any discrepancies were discussed among the annotators to ensure consistency until disagreements were resolved, ensuring reliability and validity through an iterative refinement process.
\section{Experimental Setup}
\label{sec:exp:design}
\dk{This section describes the experimental setup and methodology for evaluating processing speed and error rates. The results are presented in \cref{sec:results}. Before diving into the detailed setup, we first discuss the two evaluation metrics: processing speed and error count.}\\
\noindent \textbf{Processing Speed:}
\dk{In this study, we measure the processing speed for each model-device pair \((m_i, d_j)\) by recording the time \(t_{ijk}\) taken by model \(m_i\) running on device \(d_j\) to process a string of length \(L\). The instantaneous processing speed, \(S_{inst.}\), expressed in tokens per second, is calculated as follows:}

\begin{equation}
\label{eq:instSpeed}
S_{inst.} = \frac{L}{t_{ij}}.
\end{equation}

\dk{However, instead of relying on a single run, we perform 5 runs for each model-device pair to ensure accuracy and measure stability. If the coefficient of variance (CV) across the 5 runs is less than \(40\%\), we calculate the 5-Run-Mean Speed (referred to as consistent speed, \(S_c\)) as:}

\begin{equation}
\label{eq:mean_speed}
S_c = \frac{1}{5} \sum_{k=1}^{5} S_{inst.}(k),
\end{equation}
\dk{where \( S_{\text{inst.}}(k) \) represents the instantaneous speed during the \( k^{\text{th}} \) run of each experiment, calculated via \cref{eq:instSpeed}.  This approach minimizes the impact of factors such as initialization delays, hardware fluctuations, or other environmental inconsistencies, providing a more reliable measure of processing speed.}

\noindent \textbf{Error Count:}
\dk{If the CV exceeds \(33\%\), the run is considered unstable, and the corresponding data point is excluded from further analysis. The count of such unstable runs is recorded as an error during the experiment.}

\dk{We conducted the following four types of experiments.}


\subsection{Models Quality Analysis}
\label{sec:modelsQA}
Due to the storage and computational limitations of edge devices, deploying \gls{llm} locally often requires a trade-off between model size and performance. To put our performance results in perspective, we first conducted an empirical analysis of the models we have used in this study. \dk{This section offers insights into the models evaluated in this work, presenting a comparative analysis. While this section is not directly related to device performance, we believe providing detailed information about the models is essential for understanding their capabilities and limitations.}
We select six benchmarks to evaluate each model and calculate different metrics for each run:

\begin{itemize}[noitemsep,leftmargin=8pt, topsep=3pt]
    \item Hellaswag~\cite{hellaswag}: commonsense reasoning tasks, focusing on selecting the most plausible ending to a given situation from multiple options.
    \item mmlu~\cite{hendrycks2020measuring}: evaluating a model’s knowledge and reasoning across 57 different subjects, ranging from elementary-level topics to advanced professional knowledge.
    \item ARC~\cite{clark2018thinksolvedquestionanswering}: evaluating model's ability to answer complex, grade-school science questions that require reasoning and problem-solving beyond simple fact retrieval.
    \item Truthful-qa~\cite{lin2021truthfulqa}: evaluate how accurately AI models generate truthful answers to questions, especially in cases where common misconceptions or false information could lead to incorrect responses.
    \item winogrande~\cite{ai2:winogrande}: large-scale benchmark for commonsense reasoning, specifically designed to test a model's ability to resolve ambiguous pronouns in sentences.
    \item WikiText-2~\cite{merity2016pointer}: containing over 100 million tokens, sourced from verified Wikipedia articles, designed to support research in language modeling and text generation tasks.

\end{itemize}

For the first five benchmarks, which consist of multiple-choice questions, we use accuracy as the evaluation metric. However, for WikiText-2, the task is to predict the probability of each word in a given text, reflecting the model's natural language understanding. Therefore, we use perplexity as the evaluation metric, defined as:

\begin{equation} \text{Perplexity}(P) = e^{-\frac{1}{N} \sum_{i=1}^{N} \log P(w_i)} \end{equation}

where $P(w_i)$ represents the probability assigned by the model to the $i$-th word, and $N$ is the total number of words in the sequence.

\subsection{Evaluating Performance Consistency}
\label{sec:setStability}
\dk{Some devices may experience performance degradation over time. In these experiments, we evaluate how processing speed fluctuates across multiple runs. For each model \( m_i \) and device \( d_j \), we perform \( N \) runs and compute the mean $(\mu_{ij})$ of the speed $(S_c)$, standard deviation \( \sigma_{ij} \), and coefficient of variation ($CV_{ij}$). We also record the error count, as well as the maximum and minimum speeds observed during each experiment.}\\
\dk{In our experiments, we set \( N = 20 \), \ie each model-device pair is tested 20 times to evaluate performance variance. The mean speed across \( N \) runs is calculated as:}

\begin{equation}
\label{eq:time_efficiency}
\mu = \frac{\sum_{i=1}^{N} S_c}{N},
\end{equation}
\dk{where \( S_c \) represents the 5-Run-Mean speed measured during each run, calculated as in~\cref{eq:mean_speed}. In other words, each \( S_c \) itself is the mean of 5 runs, ensuring that the measured value is not too affected by errors or hidden factors.}


To let the devices cool down and conduct each run in comparable circumstances, we leave each device idle for two minutes between each run of the same model. We find this to be adequate, but just to be safer, between different models, we leave each device idle for ten minutes. We noted that the Meta Quest is not very prone to heating up. Heating was more noticeable on the Vivo X100s Pro, so it is always placed on a metal plate for heat dissipation. This is not representative of actual use, but this device is included in our evaluation as a baseline, not as an example of an actual XR headset. Magic Leap features a cooling fan, as does Vision Pro.\\
\dk{For each test, we conducted both PP and TG evaluations with the string length set to 64, \ie \( \text{PP} = 64 \) and \( \text{TG} = 64 \). For each model-device pair, we calculated the standard deviation, mean, and CV. The results are presented in~\cref{sec:resultsStability}.}


% \dk{The processing speed is not recorded instantaneously after a single run. Instead, we perform 5 runs, and if the coefficient of variance (CV) is less than $33\%$, we compute the mean speed for these runs as Instentenious Speed (S_i):}
% \begin{equation}
% \label{eq:speed}
% S_i= \frac{\sum_{k=1}^{5} t_{ijk}}{5}.
% \end{equation}
% \dk{However, if the CV is $33\% $ or higher, the run is considered erroneous and excluded from the analysis.}


% \dk{In our experiments, we set \( N = 20 \), \ie each model-device pair was tested 20 times to evaluate performance variance. The mean is calculated as: 

% \begin{equation}
% \label{eq:time_efficiency}
% S_{mean} = \frac{\sum_{N=1}^{20}{\frac{1}{5} \sum_{k=1}^{5} t_{ijk}}}{N}
% \end{equation}

% For each test, we conducted both PP and TG evaluations with the string length set to 64, \ie \( \text{PP} = 64 \) and \( \text{TG} = 64 \).} The results are presented in~\cref{sec:resultsStability}.  

\subsection{Processing Speed and String Length}
\label{sec:settimePPTG}
\dk{String length, comprising both \textit{Prompt Length} and \textit{Size of the Token Set}, significantly affects performance. We evaluate the performance of LLMs on XR devices with varying string lengths to assess their processing speed. For each experiment, the processing speed is calculated as in~\cref{eq:mean_speed}.}\\
\dk{We conducted experiments for five different string lengths (64, 128, 256, 512, and 1024) under the following PP and TG setups.}
\begin{itemize}[noitemsep,leftmargin=8pt,topsep=0pt]
    \item \textbf{\gls{pp}:} The PP values were set to 64, 128, 256, 512, and 1024, while TG was fixed at 0, with all other parameters set to their default values.
    \item \textbf{\gls{tg}:} The TG values were set to 64, 128, 256, 512, and 1024, while PP was fixed at 0, with all other parameters set to their default values.
\end{itemize}
The results are shown in~\cref{sec:resultsPPTG}.
\subsection{\dk{Parallelization with Thread Count and Batch Size}}
\label{sec:setTTBT}
\dk{The parameters \textit{Batch Size} and \textit{Thread Count}, which control parallelization, also significantly affect performance. We evaluate the performance of LLMs on XR devices with varying batch sizes and thread counts to assess their processing speed, calculated using~\cref{eq:mean_speed}. The two tests conducted are described below:}
\begin{itemize}[noitemsep,leftmargin=8pt,topsep=0pt]
    \item \textbf{\gls{bt}:} \dk{We set PP to 64, TG to 0, and varied the batch sizes between 128, 256, 512, and 1024, with all other parameters kept at their default values.}
    \item \textbf{\gls{tt}:} \dk{We set PP to 64 and TG to 0, while varying the thread counts between 1, 2, 4, 8, 16, and 32. All other parameters were kept at their default values.}
\end{itemize}
The results are shown in~\cref{sec:resultsTTBT}.
 
\subsection{Evaluation of Memory Consumption}
\label{sec:resultsMemory}
Memory consumption \( M_{ij} \) is measured in terms of the \gls{rss} which represents the actual physical memory usage of the relevant processes. 
For each model-device pair  ($m_i, d_j$), we measure memory usage over three runs, and report the average.
 \dk{The memory usage for PP and TG parameters was evaluated using the same approach as the performance consistency assessment. In each test, a fixed prompt was provided, and the model was instructed to generate outputs. The experiments were conducted with batch sizes of 128, 256, 512, and 1024, and their average memory consumption was recorded.}   The results are shown in~\cref{sec:resultsMemory}.
\subsection{Battery Consumption}
\label{sec:resultsBattery}  \dk{
This section presents battery consumption rates during extended \gls{pp} and \gls{tg} tests. From our initial investigation, we observed that only \gls{avp} exhibited a slight impact of model size on battery consumption, while the other three devices did not show significant variations with changing model sizes. 
Given this observation, instead of testing all 17 models, we conducted battery tests on selected models. Specifically, we chose $m_1$ as the sole model from the Qwen Series. For the remaining series, we selected the smallest and largest models: $m_2$ and $m_5$ from the Vikhr-Gemma Series, $m_6$ and $m_{11}$ from the Phi-3.1 Series, $m_{12}$ and $m_{13}$ from the LLaMA-2 Series, and $m_{14}$ and $m_{17}$ from the Mistral-7B Series. 
For each experiment, we recorded the battery level at the start and end of the experiment, allowing us to calculate battery consumption over a fixed duration of $600$ seconds (10 minutes). Each experiment was repeated three times, and we reported the mean values.  To ensure consistent conditions, we took a $600$-second break between experiments, allowing the devices to cool down. The results are presented in~\cref{sec:resultsBattery}.}
\subsection{Pareto Optimality}
Although we evaluate our five performance metrics across various devices and models, the metrics remain fragmented and cannot be directly compared with other metrics. To address this, we turn to \textit{Pareto efficiency} theory~\cite{chinchuluun2007survey}, which provides a measure of efficiency in multi-objective contexts and has been widely applied in various system design approaches~\cite{brisset2015approaches, santoro2018design}. A choice is considered Pareto optimal if no other choice exists that can improve any of its objective criteria without deteriorating at least one other criterion. In our case, $x_{1}$ is considered dominated by $x_{2}$ through objects $f$ if :

\begin{multline} \label{equ:pareto}
f_i(x_1) \leq f_i(x_2) \quad \forall i \in \{1, \dots, m\} \\
\text{and} \quad \exists j \in \{1, \dots, m\} \mid f_j(x_1) < f_j(x_2)
\end{multline}

where $i$, $j$ represent different objective indices.

A choice $x^*$ is considered Pareto optimal if no other feasible option dominates it. The set of all non-dominated designs forms the Pareto front, which represents the optimal trade-offs between all objectives.

To identify the optimal choices across devices and models, we define \dk{three objectives: \textit{quality}, \textit{performance}, and \textit{stability}. } To calculate the final score for each objective, we propose~\cref{equ:score}. For a given device-model pair and a specific objective, we first apply min-max normalization to each metric across all device-model pairs to eliminate the impact of different scales. Then, we compute a weighted sum of all metrics for that pair to obtain a single score:

\begin{equation} \label{equ:score}
    f_{o}(x) = \sum_{i=1}^{n} w_i \cdot \frac{m_i(x) - \min(m_i)}{\max(m_i) - \min(m_i)}
\end{equation}

where $o$ is the objective index, $n$ is the number of metrics used in the objective category, $w_i$ defines the weight for each metric, and $\min$ and $\max$ calculate the minimum and maximum values across the entire set of results for metric $m_i$.

In practice, some metrics, such as perplexity, are better when lower, which is the opposite of the direction in~\cref{equ:pareto}. Therefore, we take the reciprocal of each perplexity value. 
% \begin{figure}[!htb]
%     \centering
%     \includesvg[width=\columnwidth]{figures/Apple_Vision_Pro_average_Prompt_Processing_Speed.svg}
%     %\vskip -0.5cm
%     \includesvg[width=\columnwidth]{figures/Magic_Leap_average_Prompt_Processing_Speed.svg}
%     \includesvg[width=\columnwidth]{figures/Meta_Q3_average_Prompt_Processing_Speed.svg}
%     \includesvg[width=\columnwidth]{figures/Vivo_average_Prompt_Processing_Speed.svg}
%     \caption{Prompt processing speeds (in tokens/second) for different devices, averaged over five prompt sizes: 64, 128, 256, 512, 1024, for each model. Error bars indicate standard deviations. \todo{vision pro results have now been updated}}
    
%     \label{fig:pp_speed_three_devices}
% \end{figure}


% \begin{figure}[!htb]
%     \centering
%     \includesvg[width=\columnwidth]{figures/Prompt_Processing_Speed_slope_heatmap.svg}
%     \includesvg[width=\columnwidth]{figures/Token_Generation_Speed_slope_heatmap.svg}
%     \includesvg[width=\columnwidth]{figures/Batched_Token_Generation_Speed_slope_heatmap.svg}
%     \caption{ \todo{I think we should change graph type: Heatmaps do not show a big difference.. } Heatmaps showing the effects of prompt size, token set size and batch size (respectively) on prompt processing, token generation and token generation performance (respectively) for each device-model pair. Brighter colors indicate stronger effects, though absolute values remain small.}
%     \label{fig:pptgbs_heatmap}
% \end{figure}
\begin{figure}[!hbp]
    \centering 
    \includegraphics[width=1.0\linewidth]{images/mqa.png}  
 \caption{Model Quality Analysis: The models were evaluated based on six different benchmarks. We chose accuracy (\textbf{$\uparrow$}) for the five benchmarks on the left and perplexity (\textbf{$\downarrow$}) for the WikiText-2 benchmark.  Each group in the bar chart represents models $m_1$ to $m_{17}$ from left to right.}
    \label{fig:Modelsqty}
\end{figure}  

%===================== ======================================
\begin{figure*}[!htbp]
    \centering 
    \vskip -0.1cm   
      % \includegraphics[width=\linewidth]{images/conssistencyNewdk.png}  
      \includegraphics[width=\linewidth]{images/CPU_Consistency_Plot.pdf}  
    \vskip -0.2478cm
    \label{fig:StabDD}
\caption{\dk{Consistency results of the four devices over time: PP (top) and TG (bottom) speeds in tokens per second across 20 sorted runs (X-axis: run number, Y-axis: speed in $t/s$).}}
    \label{fig:StabDD}
\end{figure*}
% ==================================================================
\begin{figure*}[!htbp] 
 \vskip -0.145cm 
     \centering  
     \includegraphics[width=0.807\linewidth]{images/GPU_Consistency_Plot.pdf}  
\vskip -0.245cm 
    \caption{\dk{Consistency results of the Apple Vision Pro on GPU. Left: PP results. Right: TG results. Each model was tested 20 times, and the results are plotted in sorted order.}}
    \label{fig:VPGPU}
\end{figure*}
%\begin{sidewaystable}[!htbp]%
\begin{table*}%[!hbp]
%\begin{adjustbox}
\centering
\caption{\dk{Performance consistency results for the four devices: Magic Leap 2 (ML2), Meta Quest 3 (MQ3), Vivo X100 Pro (Vivo), and Apple Vision Pro (VPro for CPU and VPro* for GPU). Processing speed was calculated from 20 runs for each model-device pair. The results are reported in terms of the mean ($\mu$) speed (t/s) of 20 runs, its standard deviation ($\sigma$), coefficient of variation (CV \%), and the range of values: $[min, max]$.  }}
\label{tab:consistency}
\vskip -0.45cm 
%\begin{adjustbox}{angle=90} % Rotate table by 90 degrees
\resizebox{1.02\linewidth}{!}{
\begin{tabular}{@{\hskip -0.05cm}c|@{\hskip -0.05cm}c@{\hskip -0.05cm}|l@{\hskip0.05cm}|c@{\hskip0.05cm}c@{\hskip 0.05cm}c@{\hskip 0.05cm}c@{\hskip 0.05cm}c@{\hskip 0.05cm}c@{\hskip 0.05cm}c@{\hskip 0.05cm}c@{\hskip 0.05cm}c@{\hskip 0.05cm}c@{\hskip 0.05cm}c@{\hskip 0.05cm}c@{\hskip 0.05cm}c@{\hskip 0.05cm}c@{\hskip 0.05cm}c@{\hskip 0.05cm}c@{\hskip 0.05cm}}
%\hline
%\begin{tabular}{l|l|cccccccccccccccc}
%\toprule
\hline 
% \multicolumn{18}{c}{\textbf{PP Results}} \\  % Single-row header spanning all columns
%  \hline
 &D& Metric & m2 & m3 & m4 & m5 & m6 & m7 & m8 & m9 & m10 & m11 & m12 & m13 & m14 & m15 & m16 & m17 \\ 
        \hline
 \multirow{15}{*}{\rotatebox[origin=c]{90}{PP Results}}  % Left-most column label 
    &\multirow{3}{*}{ML2}% 2 decimal points 
    & $\mu$$\pm\sigma$ & $17.01 \pm 4.67$ & $13.99 \pm 4.57$ & $10.38 \pm 2.74$ & $20.82 \pm 0.87$ & $16.81 \pm 0.07$ & $13.16 \pm 0.47$ & $14.22 \pm 0.31$ & $12.63 \pm 0.38$ & $13.12 \pm 0.29$ & $11.45 \pm 0.18$ & $7.92 \pm 0.05$ & $7.37 \pm 0.19$ & $2.20 \pm 0.03$ & $3.14 \pm 0.08$ & $1.15 \pm 0.05$ & $1.52 \pm 0.09$ \\
    && Range & $[10.25, 22.24]$ & $[7.47, 18.75]$ & $[6.27, 13.32]$ & $[18.28, 21.36]$ & $[16.68, 16.95]$ & $[12.04, 13.51]$ & $[12.98, 14.36]$ & $[11.73, 12.87]$ & $[12.00, 13.27]$ & $[10.93, 11.62]$ & $[7.73, 7.99]$ & $[6.73, 7.54]$ & $[2.14, 2.23]$ & $[2.98, 3.27]$ & $[1.10, 1.33]$ & $[1.36, 1.72]$ \\
   & & CV (\%) & $27.46$ & $32.66$ & $26.41$ & $4.16$ & $0.43$ & $3.56$ & $2.19$ & $2.98$ & $2.22$ & $1.57$ & $0.66$ & $2.54$ & $1.35$ & $2.41$ & $4.70$ & $5.99$ \\%ml2 pp done ....  
 %\cline{2-19}
 &\multirow{3}{*}{MQ3} 
    & $\mu$$\pm\sigma$ & $21.00 \pm 2.16$ & $25.61 \pm 0.10$ & $21.33 \pm 1.47$ & $16.33 \pm 1.84$ & $10.99 \pm 1.65$ & $11.37 \pm 1.80$ & $13.74 \pm 1.72$ & $10.24 \pm 1.34$ & $11.07 \pm 0.61$ & $8.17 \pm 2.19$ & $7.12 \pm 0.04$ & $6.09 \pm 0.32$ & $2.65 \pm 0.21$ & $2.37 \pm 0.12$ & $1.57 \pm 0.06$ & $2.04 \pm 0.09$ \\
    && Range & $[17.42, 23.59]$ & $[25.03, 25.77]$ & $[14.79, 23.12]$ & $[13.66, 18.72]$ & $[7.62, 13.89]$ & $[7.91, 13.86]$ & $[10.88, 16.74]$ & $[7.79, 11.87]$ & $[7.19, 11.28]$ & $[4.31, 13.05]$ & $[5.96, 7.17]$ & $[5.64, 6.61]$ & $[2.10, 2.92]$ & $[2.12, 2.57]$ & $[1.44, 1.68]$ & $[1.85, 2.34]$ \\
   && CV (\%) & $10.27$ & $0.37$ & $6.89$ & $11.24$ & $14.99$ & $15.80$ & $12.51$ & $13.06$ & $5.55$ & $26.81$ & $0.56$ & $5.23$ & $7.84$ & $5.24$ & $3.93$ & $4.23$ \\ 
% Meta q3 PP done 

 %\cline{2-19} 
    &\multirow{3}{*}{Vivo} 
    & $\mu$$\pm\sigma$ & $15.70 \pm 2.88$ & $15.90 \pm 2.76$ & $14.98 \pm 2.98$ & $13.77 \pm 2.13$ & $8.66 \pm 0.80$ & $8.27 \pm 1.08$ & $10.35 \pm 1.10$ & $8.48 \pm 1.08$ & $8.82 \pm 1.61$ & $11.14 \pm 1.21$ & $4.47 \pm 0.19$ & $4.35 \pm 0.23$ & $2.44 \pm 0.07$ & $4.28 \pm 0.44$ & $2.61 \pm 0.12$ & $5.89 \pm 0.47$ \\
    && Range & $[11.16, 25.55]$ & $[9.39, 17.81]$ & $[10.52, 24.36]$ & $[10.65, 15.55]$ & $[7.32, 9.34]$ & $[5.88, 9.53]$ & $[8.27, 11.05]$ & $[7.19, 11.72]$ & $[5.97, 11.60]$ & $[9.13, 12.04]$ & $[3.98, 4.92]$ & $[3.80, 4.86]$ & $[2.27, 2.52]$ & $[2.61, 4.80]$ & $[2.23, 2.76]$ & $[5.17, 6.59]$ \\
   & & CV (\%) & $18.33$ & $17.38$ & $19.88$ & $15.50$ & $9.26$ & $13.07$ & $10.61$ & $12.75$ & $18.31$ & $10.84$ & $4.24$ & $5.39$ & $3.00$ & $10.30$ & $4.55$ & $7.92$ \\ 
    % Vivoo PP done 
 %\cline{2-19}
&\multirow{3}{*}{VPro} 
    & $\mu$$\pm\sigma$ & $31.03 \pm 2.20$ & $41.22 \pm 3.04$ & $33.11 \pm 1.35$ & $32.96 \pm 1.21$ & $20.62 \pm 0.56$ & $19.34 \pm 0.64$ & $23.14 \pm 1.09$ & $19.52 \pm 0.73$ & $19.56 \pm 0.60$ & $29.22 \pm 1.22$ & $9.15 \pm 0.40$ & $9.06 \pm 0.28$ & $5.52 \pm 0.21$ & $7.44 \pm 0.40$ & $5.01 \pm 0.26$ & $14.05 \pm 0.56$ \\
    && Range & $[29, 34.98]$ & $[39.44, 47.03]$ & $[32.98, 34.79]$ & $[32.15, 35.93]$ & $[20.19, 21.63]$ & $[19.09, 20.43]$ & $[22.65, 24.92]$ & $[19.04, 20.98]$ & $[19.15, 20.74]$ & $[28.58, 31.25]$ & $[8.77, 9.67]$ & $[8.88, 9.77]$ & $[5.4, 5.93]$ & $[7.21, 8.13]$ & $[4.83, 5.47]$ & $[13.63, 15.02]$ \\
   & & CV (\%) & $7.08$ & $7.37$ & $4.07$ & $3.68$ & $2.72$ & $3.31$ & $4.69$ & $3.72$ & $3.05$ & $4.17$ & $4.41$ & $3.12$ & $3.73$ & $5.38$ & $5.16$ & $3.99$ \\

 %\cline{2-19}
     & \multirow{4}{*}{VPro* } 
    & $\mu$$\pm\sigma$ & $378 \pm 3.7$ & $432 \pm 3.7$ & $382\pm 4.8$ & $374\pm 4.6$ & $241\pm 1.5$ & $230\pm 2.3$ & $237\pm 2.0$ & $218\pm 5.8$ & $229\pm 4.3$ & $264\pm 3.4$ & $136\pm 9.9$ & $128\pm 12.5$ & $135\pm 2.3$ & $134\pm 2.3$ & $133\pm 1.4$ & $130\pm 6.6$ \\
    && Range & $[369, 384]$ & $[424, 439]$ & $[367, 389]$ & $[363, 380]$ & $[237, 243]$ & $[225, 234]$ & $[232, 240]$ & $[198, 223]$ & $[221, 237]$ & $[255, 268]$ & $[109, 145]$ & $[100, 143]$ & $[130, 138]$ & $[129, 136]$ & $[129, 134]$ & $[109, 135]$ \\
   & & CV (\%) & $1.0$ & $0.9$ & $1.2$ & $1.2$ & $0.6$ & $1.0$ & $0.8$ & $2.6$ & $1.9$ & $1.3$ & $7.3$ & $9.8$ & $1.7$ & $1.7$ & $1.1$ & $5.1$ \\ 
    \hline

     
  \multirow{15}{*}{\rotatebox[origin=c]{90}{TG Results}}
    %%============
    %%
    %%--------------------------
%\multirow{12}{*}{\rotatebox[origin=c]{90}{TG Test}}  
   &\multirow{3}{*}{ML2}% 2 decimal points 
    & $\mu$$\pm\sigma$ & $9.14 \pm 1.92$ & $8.22 \pm 1.92$ & $6.53 \pm 1.77$ & $8.43 \pm 0.07$ & $11.69 \pm 0.04$ & $8.52 \pm 0.02$ & $7.88 \pm 0.01$ & $6.92 \pm 0.01$ & $6.25 \pm 0.10$ & $5.05 \pm 0.02$ & $5.90 \pm 0.03$ & $5.59 \pm 0.14$ & $2.05 \pm 0.01$ & $2.84 \pm 0.06$ & $0.24 \pm 0.01$ & $0.21 \pm 0.01$ \\
    && Range & $[6.92, 11.25]$ & $[5.69, 10.27]$ & $[4.41, 8.39]$ & $[8.33, 8.55]$ & $[11.61, 11.74]$ & $[8.48, 8.55]$ & $[7.85, 7.91]$ & $[6.89, 6.95]$ & $[5.95, 6.31]$ & $[5.01, 5.08]$ & $[5.83, 5.94]$ & $[5.13, 5.73]$ & $[2.04, 2.06]$ & $[2.77, 2.97]$ & $[0.22, 0.27]$ & $[0.19, 0.23]$ \\
   & & CV (\%) & $20.95$ & $23.36$ & $27.10$ & $0.77$ & $0.32$ & $0.23$ & $0.18$ & $0.21$ & $1.62$ & $0.39$ & $0.48$ & $2.49$ & $0.32$ & $2.24$ & $5.32$ & $5.20$ \\  
    % ml2 tg DONE 
 %\cline{2-19} 
    &\multirow{3}{*}{MQ3}
    & $\mu$$\pm\sigma$ & $12.37 \pm 1.45$ & $14.19 \pm 0.30$ & $11.58 \pm 0.98$ & $9.47 \pm 1.29$ & $9.00 \pm 1.12$ & $8.57 \pm 0.70$ & $9.30 \pm 0.51$ & $7.61 \pm 0.65$ & $8.30 \pm 0.26$ & $5.47 \pm 0.10$ & $5.62 \pm 0.23$ & $5.01 \pm 0.24$ & $2.95 \pm 0.21$ & $3.06 \pm 0.05$ & $1.67 \pm 0.15$ & $3.38 \pm 0.09$ \\
    && Range & $[9.11, 13.89]$ & $[13.52, 14.67]$ & $[7.77, 12.80]$ & $[7.64, 10.87]$ & $[7.38, 10.39]$ & $[6.85, 9.55]$ & $[7.66, 9.86]$ & $[6.13, 8.24]$ & $[7.40, 8.53]$ & $[5.19, 5.61]$ & $[5.26, 6.17]$ & $[4.61, 5.57]$ & $[2.23, 3.40]$ & $[2.96, 3.13]$ & $[1.40, 1.84]$ & $[3.11, 3.53]$ \\
   & & CV (\%) & $11.73$ & $2.11$ & $8.43$ & $13.65$ & $12.50$ & $8.17$ & $5.52$ & $8.51$ & $3.18$ & $1.87$ & $4.09$ & $4.83$ & $7.27$ & $1.68$ & $8.93$ & $2.80$ \\  
    % Meta Q3 TG done 

 %\cline{2-19}
   &\multirow{3}{*}{Vivo} 
    & $\mu$$\pm\sigma$ & $9.39 \pm 0.98$ & $9.85 \pm 0.42$ & $9.44 \pm 1.22$ & $8.66 \pm 0.30$ & $6.91 \pm 0.08$ & $6.83 \pm 0.22$ & $8.21 \pm 0.12$ & $6.84 \pm 0.61$ & $6.61 \pm 0.71$ & $7.51 \pm 0.11$ & $3.70 \pm 0.05$ & $3.59 \pm 0.11$ & $2.09 \pm 0.01$ & $3.71 \pm 0.04$ & $2.27 \pm 0.02$ & $4.72 \pm 0.17$ \\
    && Range & $[8.54, 13.35]$ & $[9.17, 10.67]$ & $[7.58, 13.86]$ & $[7.97, 9.26]$ & $[6.75, 7.00]$ & $[6.39, 7.21]$ & $[7.52, 8.35]$ & $[4.37, 7.11]$ & $[5.70, 7.68]$ & $[7.25, 7.64]$ & $[3.60, 3.77]$ & $[3.41, 3.71]$ & $[2.06, 2.11]$ & $[3.65, 3.79]$ & $[2.23, 2.31]$ & $[4.48, 5.00]$ \\
   & & CV (\%) & $10.46$ & $4.22$ & $12.91$ & $3.48$ & $1.13$ & $3.25$ & $1.48$ & $8.92$ & $10.75$ & $1.41$ & $1.27$ & $3.04$ & $0.68$ & $1.15$ & $1.06$ & $3.61$ \\  
 %\cline{2-19} 
&\multirow{3}{*}{VPro} 
& $\mu$$\pm\sigma$ & $17.90 \pm 1.00$ & $21.15 \pm 0.82$ & $18.92 \pm 0.74$ & $17.71 \pm 0.50$ & $15.72 \pm 0.37$ & $14.76 \pm 0.26$ & $16.63 \pm 0.49$ & $13.93 \pm 0.50$ & $14.07 \pm 0.19$ & $12.03 \pm 0.21$ & $7.20 \pm 0.21$ & $7.22 \pm 0.17$ & $4.53 \pm 0.17$ & $6.24 \pm 0.27$ & $4.17 \pm 0.18$ & $10.22 \pm 0.30$ \\
    && Range & $[16.59, 19.72]$ & $[19.4, 22.49]$ & $[17.31, 19.8]$ & $[16.89, 18.77]$ & $[14.88, 16.36]$ & $[14.32, 15.28]$ & $[15.77, 17.56]$ & $[12.89, 14.99]$ & $[13.71, 14.49]$ & $[11.26, 12.28]$ & $[6.74, 7.52]$ & $[6.98, 7.7]$ & $[4.11, 4.93]$ & $[5.52, 6.72]$ & $[3.86, 4.52]$ & $[9.82, 10.93]$ \\
   & & CV (\%) & $5.58$ & $3.86$ & $3.89$ & $2.84$ & $2.35$ & $1.73$ & $2.97$ & $3.61$ & $1.35$ & $1.77$ & $2.95$ & $2.41$ & $3.84$ & $4.35$ & $4.31$ & $2.91$ \\  
 %\cline{2-19}
&\multirow{3}{*}{VPro*} 
    & $\mu$$\pm\sigma$   & $23.1 \pm 0.1$ & $27.5 \pm 0.1$ & $23.5 \pm 0.1$ & $22.6 \pm 0.1$ & $23.3 \pm 0.1$ & $19.6 \pm 0.8$ & $19.9 \pm 0.1$ & $18.0 \pm 0.4$ & $17.7 \pm 0.2$ & $14.6 \pm 0.2$ & $13.7 \pm 0.8$ & $12.5 \pm 1.0$ & $14.5 \pm 0.3$ & $15.1 \pm 0.2$ & $15.2 \pm 0.2$ & $14.7 \pm 0.4$ \\
    && Range  & $[23, 23.3]$ & $[27.1, 27.7]$ & $[23.2, 23.8]$ & $[22.3, 22.7]$ & $[23.0, 23.5]$ & $[19.2, 22.8]$ & $[19.8, 20.2]$ & $[16.8, 18.5]$ & $[17.2, 17.9]$ & $[14.3, 14.9]$ & $[11.4, 14.6]$ & $[10.2, 13.9]$ & $[13.9, 14.9]$ & $[14.6, 15.4]$ & $[14.9, 15.4]$ & $[13.5, 15.1]$ \\
   & & CV(\%) & $0.4$ & $0.5$ & $0.5$ & $0.3$ & $0.6$ & $4.3$ & $0.6$ & $2.1$ & $1.0$ & $1.1$ & $5.8$ & $8.3$ & $1.9$ & $1.4$ & $1.1$ & $2.5$ \\ 
    %& Errors &  m2 & m3 & m4 & m5 & m6 & m7 & m8 & m9 & m10 & m11 & m12 & m13 & m14 & m15 & m16 & m17 \\
\hline
\end{tabular}
}
%\end{adjustbox}
%\end{sidewaystable}%
\end{table*}
\section{Performance Evaluation Results}
\label{sec:results}%----
This section presents the results of four types of experiments described in \cref{sec:exp:design}.
\subsection{Model Quality Analysis}
As shown in \cref{tab:models}, we select five different model architectures, and for each architecture, we choose various quantization settings. While lower-bit quantization reduces model size, speeds up inference, and lowers power consumption, it typically comes at the cost of reduced model performance. The results of the model quality are illustrated in \cref{fig:Modelsqty}.
\label{sec:resModelsQty}
As shown in \cref{fig:Modelsqty}, a consistent trend is observed across all benchmarks: applying lower-bit quantization settings results in reduced model performance. Models with more parameters exhibit better language understanding capabilities, leading to higher performance under the same quantization settings. Additionally, different benchmarks exhibit varying levels of sensitivity to the quantization settings. For example, Hellaswag, ARC and WikiText-2 are more sensitive compared to the other three benchmarks.

\subsection{\dk{Performance Consistency Results}}
\label{sec:resultsStability}
\dk{\Cref{fig:StabDD,fig:VPGPU} illustrate the consistency of performance across all model-device pairs, while \Cref{tab:consistency} provides the corresponding quantitative results. Across all 20 runs, the Apple Vision Pro demonstrates the most consistent results compared to other devices, with the corresponding speed also being the fastest. Notably, the GPU results for the Apple Vision Pro exhibit greater stability and smaller variance. Among the remaining devices, the Magic Leap 2 also shows a reasonable degree of stability; however, for the first three models, its variance is surprisingly high, reaching 27\%, 32\%, and 26\% for \(m_2\), \(m_3\), and \(m_3\) under the PP setup, and 20\%, 23\%, and 27\% under the TG setup, respectively. In contrast, the Meta Quest 3 and Vivo X100 Pro exhibit relatively poor performance both in terms of speed and variance. Between these two devices, the Meta Quest 3 achieves higher speeds than the Vivo X100 Pro, but there is no clear difference in variance. For some models, one device shows lower variance, while for others, the variance is higher.}\\ 
\dk{Furthermore, as shown in \Cref{tab:perfConsistencyresultsForm1} performance varies significantly for \(m_1\) over Magic Leap 2 and Meta Quest 3, but it is generally consistent across the other two devices. These findings suggest that, provided model-device pairs are carefully selected, LLM performance can be considered reliable, strengthening the case for on-device LLM use in XR environments.}
% \begin{table*}[htbp]
% \centering
% \caption{\dk{Performance consistency results for \(m_1\) across the four devices. For the results of other models and explanation of the metrics used see \Cref{tab:consistency}.}}
% \vskip -0.45cm 
% \label{tab:perfConsistencyresultsForm1}
% \resizebox{\linewidth}{!}{
% \begin{tabular}{l|ccccc|ccccc}
% \hline
% \multirow{2}{*}{Metric} & \multicolumn{5}{c|}{PP Results}     & \multicolumn{5}{c}{TG Results}      \\ \cline{2-11} 
%  & ML2     & Vivoo   & Meta Q3  & VisPro   & VisPro*  & ML2     & Vivoo   & Meta Q3  & VisPro   & VisPro*  \\\hline
% \(\mu \pm \sigma\)               & 51.68 $\pm$ 4.22 & 62.68 $\pm$ 16.64 & 39.32 $\pm$ 15.66 & 292.48 $\pm$ 12.87 & 1603.59 $\pm$ 56.12 & 20.26 $\pm$ 0.05 & 22.16 $\pm$ 0.90 & 16.76 $\pm$ 3.76 & 41.94 $\pm$ 1.77 & 42.94 $\pm$ 0.80 \\ %\hline
% Range                            & [43.62, 54.59]   & [21.63, 72.04]   & [24.25, 72.59]    & [288.58, 312.17]  & [1516.96, 1696.13] & [20.17, 20.34]   & [20.54, 23.41]   & [11.61, 22.53]    & [38.90, 46.37]   & [41.76, 44.35]   \\ %\hline
% CV (\%)                          & 8.17             & 26.55            & 39.82             & 4.40              & 3.50               & 0.23             & 4.05             & 22.44             & 4.21              & 1.87              \\   
% \hline
% \end{tabular}}
% \end{table*} 
% ==============================================================================
\begin{table}[htbp]
\centering
\caption{\dk{Performance consistency results for \(m_1\) across the four devices. For the results of other models and description of metrics used, see \Cref{tab:consistency}.}}
\vskip -0.45cm 
\label{tab:perfConsistencyresultsForm1}
\resizebox{\linewidth}{!}{
\begin{tabular}{l|l|ccccc}
\hline
Test& Metric & ML2     & Vivoo   & Meta Q3  & VisPro   & VisPro*  \\ \hline 
\multirow{3}{*}{PP} 
 & \(\mu \pm \sigma\) & 51.68 $\pm$ 4.22 & 62.68 $\pm$ 16.64 & 39.32 $\pm$ 15.66 & 292.48 $\pm$ 12.87 & 1603.59 $\pm$ 56.12 \\ 
 & Range  & [43.62, 54.59]   & [21.63, 72.04]   & [24.25, 72.59]    & [288.5, 312.1]  & [1516.9, 1696.1] \\ 
 & CV (\%)  & 8.17             & 26.55            & 39.82             & 4.40              & 3.50               \\   
\hline
\multirow{3}{*}{TG}% & Metric & ML2     & Vivoo   & Meta Q3  & VisPro   & VisPro*  \\ \cline{2-7}
 & \(\mu \pm \sigma\) & 20.26 $\pm$ 0.05 & 22.16 $\pm$ 0.90 & 16.76 $\pm$ 3.76 & 41.94 $\pm$ 1.77 & 42.94 $\pm$ 0.80 \\ 
 & Range  & [20.17, 20.34]   & [20.54, 23.41]   & [11.61, 22.53]    & [38.90, 46.37]   & [41.76, 44.35]   \\ 
 & CV (\%)  & 0.23             & 4.05             & 22.44             & 4.21              & 1.87              \\   
\hline
\end{tabular}}
\end{table} 

% ========================= ==================================
\begin{figure*}%[!htp]
     \centering   
     \includegraphics[width=1.0\linewidth]{images/PP_TG_Results_Main.pdf}  
\vskip -0.3645cm 
    \caption{\dk{Processing speed of the four devices in the PP test (top) and TG test (bottom) with varying string lengths: 64, 128, 256, 512, and 1024. The x-axis represents the model, while the y-axis represents the processing speed in t/s.}}
    \label{fig:resPPTG}
\end{figure*}

%-------------------------------------------------
\begin{figure}[!hbp]
     \centering   
     \includegraphics[width=1.0\linewidth]{images/PP_TG_Results_GPU.pdf}  
\vskip -0.45cm 
    \caption{\dk{Processing speed of Apple Vision Pro (GPU) in the PP test (left) and TG test (rigth). The x-axis represents the model, while the y-axis represents the processing speed in t/s.}}
    \label{fig:resPPTG-GPU}
\end{figure}
%--------------------------------------

%-------------------------------------------------
\begin{figure}[!htbp]
     \centering   
     \includegraphics[width=1.0\linewidth]{images/Errors_Count.pdf}  
\vskip -0.355cm 
 \caption{\dk{Error counts for Meta Quest 3 (left) and Apple Vision Pro (right) in PP (top) and TG (bottom). Magic Leap 2, with zero errors, and Vivo X100 Pro, with a total of four errors, one in $m_1$ (PP-64), two in $m_{10}$ (PP-64, PP-128), and one in $m_{11}$ (TG-1024) are excluded.}}
    \label{fig:resErrorsCount}
\end{figure}
%-------------------------------------------------

%-------------------------------------------------
\begin{figure*}%[!htbp]
    \centering 
   % % \vskip -0.396478cm 
   %   \includegraphics[width=0.73\linewidth]{images/BT_DK.pdf} %
   %   \includegraphics[width=0.26\linewidth]{images/BT_GPUAVP.pdf} %
    \includegraphics[width=\linewidth]{images/BT_DK5devices.pdf} %
    \vskip -0.34780cm
    \caption{\dk{Batch Test results: Batch size (X-axis) vs. processing speed (Y-axis).}}
    % \vskip -0.478cm
    \label{fig:resBT}
\end{figure*}
%-------------------------------------------------
%-------------------------------------------------
\begin{figure*}%[!htbp]
    \centering 
  %  \vskip -0.478cm
    %\includegraphics[width=1.0\linewidth]{images/threadsBB.png} %
    % \includegraphics[width=1.0\linewidth]{images/TT_DK.pdf} %
    \includegraphics[width=1.0\linewidth]{images/TT_DK5Devices.pdf} %
    \vskip -0.32478cm
\caption{\dk{Thread Test (TT) results: The X-axis represents the thread count, while the Y-axis shows the speed in tokens per second. Threads 4 and 8 deliver the fastest results. Note: Apple Vision Pro (CPU) fails for thread counts of 16 and 32.}}
    \label{fig:resThreadA}
\end{figure*}
%------------------------------------------------- 

\subsection{Results of Processing Speed and String Length}
\label{sec:resultsPPTG} 
 % \dk{\cref{tab:PPTGresults} and
 \dk{\cref{fig:resPPTG,fig:resPPTG-GPU} present the processing speed results of PP and TG with varying string lengths (prompt length and token sets). Both PP and TG were tested with values of 64, 128, 256, 512, and 1024. The processing speed varies significantly across string lengths, devices, and models. \\
 % %\renewcommand{\arraystretch}{0.9} 
\begin{table*}[tb]
\centering
\caption{\dk{Processing speed (tokens per second) of 17 models across four XR devices for varying prompt/string lengths (PP/TG), along with the mean ($\mu$), standard deviation ($\sigma$), and coefficient of variation (CV). Here, $*$ denotes a single error occurrence, while $*2$ and $*3$ indicate two and three repetitions due to errors, respectively. Note that the CV is computed across different parameters (PP/TG = 64, 128, 256, 512, 1024), meaning a higher CV does not indicate an error (as in CV for 5-run speed) but rather reflects the impact of parameter variation on results.}
\label{tab:PPTGresults}}
\vskip -0.245cm 
\resizebox{\linewidth}{!}{ 
%\begin{tabular}{l|p{20pt}|l|ccccccccccccccccc|l}
\begin{tabular}{l|l|l|ccccccccccccccccc|l}
 \toprule
Test & Device & $PP|TG$ & \multicolumn{17}{c}{Processing speed for each model}& \multicolumn{1}{|l}{Errors} \\ \cline{4-20}
      &        &   Metrics     & $m_1$ & $m_2$ & $m_3$ & $m_4$ & $m_5$ & $m_6$ & $m_7$ & $m_8$ & $m_9$ & $m_{10}$ & $m_{11}$ & $m_{12}$ & $m_{13}$ & $m_{14}$ & $m_{15}$ & $m_{16}$ & $m_{17}$ &Count\\  
\hline
\multirow{36}{*}{\rotatebox[origin=c]{90}{PP Test}} 
& \multirow{8}{*}{\rotatebox[origin=c]{90}{Magic Leap 2}} 
    & 64    & 68.86 & 25.23 & 20.80 & 14.28 & 24.94 & 17.46 & 14.06 & 15.03 & 13.37 & 14.17 & 12.50 & 8.10 & 7.69 & 2.24 & 3.28 & 2.16 & 6.82 &0\\ 
&   & 128   & 69.07 & 24.97 & 20.78 & 14.25 & 24.56 & 17.43 & 13.93 & 14.96 & 13.46 & 14.04 & 12.43 & 8.07 & 7.66 & 2.23 & 3.27 & 2.16 & 6.80 &0\\ 
&   & 256   & 67.55 & 24.46 & 20.35 & 14.10 & 24.26 & 17.02 & 13.65 & 14.72 & 13.13 & 13.81 & 12.13 & 7.96 & 7.55 & 2.23 & 3.26 & 2.15 & 6.75 &0\\ 
&   & 512   & 64.58 & 23.31 & 19.52 & 13.67 & 23.04 & 16.22 & 13.19 & 14.08 & 12.71 & 13.33 & 11.77 & 7.75 & 7.37 & 2.22 & 3.23 & 2.14 & 6.55&0 \\ 
&   & 1024  & 62.20 & 22.25 & 18.73 & 13.30 & 22.01 & 15.57 & 12.74 & 13.59 & 12.25 & 12.87 & 11.43 & 7.54 & 7.15 & 2.20 & 3.20 & 2.13 & 6.45 &0\\ 
&   & $\mu$  & 66.85 & 24.24 & 20.04 & 14.14 & 23.76 & 16.74 & 13.91 & 14.88 & 13.38 & 13.84 & 12.45 & 7.88 & 7.47 & 2.23 & 3.25 & 2.15 & 6.73 &-\\ 
&   & $\sigma$ & 2.53 & 1.19 & 0.80 & 0.33 & 1.18 & 0.74 & 0.65 & 0.58 & 0.42 & 0.55 & 0.40 & 0.23 & 0.16 & 0.01 & 0.03 & 0.01 & 0.14 &-\\ 
&   & CV (\%) & 3.78 & 4.91 & 4.00 & 2.34 & 4.96 & 4.42 & 4.66 & 3.91 & 3.14 & 3.95 & 3.21 & 2.92 & 2.14 & 0.45 & 1.02 & 0.54 & 2.08 &-\\ 
%&   & Errors & 0 & 1 & 1 & 1 & 1 & $1,1,1,1$ & $m_7$ & $m_8$ & $m_9$ & $m_{10}$ & $m_{11}$ & $m_{12}$ & $m_{13}$ & $m_{14}$ & $m_{15}$ & $m_{16}$ & $m_{17}$ &Count\\ 
\cline{2-21}
& \multirow{8}{*}{\rotatebox[origin=c]{90}{Meta Quest 3}}  
    & 64    & 118.63 & 26.39 & 28.63 & 23.14 & 20.53 & 14.41 & 13.77 & 16.74 & 11.95 & 11.60 & 18.78 & 6.37 & 6.27 & 3.58 & 5.62 & 2.83 & 6.83$^{*}$ &1\\ 
&   & 128   & 121.99 & 26.17 & 27.04 & 21.51 & 19.48 & 13.08 & 12.15 & 14.94 & 10.52 & 10.29 & 16.34 & 5.98 & 5.71 & 3.06$^{*}$ & 5.00 & 2.22 & 5.98 &1\\ 
&   & 256   & 113.77 & 23.54 & 24.48 & 18.99 & 17.98 & 12.03 & 10.93 & 13.44 & 9.46  & 9.57  & 11.84 & 4.78 & 4.56$^{*}$ & 2.35 & 3.74 & 2.14 & 4.95$^{*}$ &2\\ 
&   & 512   & 101.85 & 21.34 & 21.97 & 17.44 & 16.02 & 9.49  & 8.39  & 10.33$^{*}$ & 7.05  & 7.87$^{*}$  & 11.01 & 4.46 & 3.82 & 2.19 & 3.37 & 1.92 & 4.06 &2\\ 
&   & 1024  & 88.49  & 19.05 & 19.77 & 12.67 & 12.66$^{*}$ & 7.81  & 7.19  & 8.68  & 6.24  & 6.38  & 9.58  & 4.04 & 3.65 & 2.18 & 3.28 & 1.87 & 3.82 &1\\ 
&   & $\mu$  & 108.95 & 23.98 & 24.77 & 18.75 & 17.33 & 11.35 & 10.09 & 12.83 & 9.05  & 9.62  & 13.91 & 5.33 & 4.92 & 2.63 & 4.80 & 2.20 & 4.93 &-\\ 
&   & $\sigma$ & 12.77 & 3.00  & 3.65  & 3.54  & 2.92  & 2.46  & 2.37  & 3.07  & 1.84  & 2.02  & 3.55  & 0.96 & 0.92 & 0.48 & 0.86 & 0.14 & 1.07 &-\\ 
&   & CV (\%) & 11.72 & 12.51 & 14.72 & 18.90 & 16.84 & 21.63 & 23.53 & 23.94 & 20.34 & 21.01 & 25.55 & 18.06 & 18.67 & 18.25 & 17.96 & 6.28 & 21.68 &-\\ 
%&   & Errors & $m_1$ & $m_2$ & $m_3$ & $m_4$ & $m_5$ & $m_6$ & $m_7$ & $m_8$ & $m_9$ & $m_{10}$ & $m_{11}$ & $m_{12}$ & $m_{13}$ & $m_{14}$ & $m_{15}$ & $m_{16}$ & $m_{17}$ &Count\\ 

\cline{2-21}
& \multirow{8}{*}{\rotatebox[origin=c]{90}{Vivo X100s Pro }}  
    & 64    & 94.55 & 16.43 & 18.40 & 15.95 & 15.90 & 9.16 & 9.36 & 10.83 & 9.00 & 9.23 & 12.02 & 4.80 & 4.74 & 2.56 & 4.64 & 2.80 & 6.43 &0\\ 
&   & 128   & 103.07$^{*}$ & 16.27 & 18.35 & 16.05 & 15.94 & 9.17 & 9.35 & 10.78 & 9.07 & 12.47$^{*}$ & 12.05 & 4.85 & 4.75 & 2.57 & 4.63 & 2.81 & 6.46 &2\\ 
&   & 256   & 109.44 & 16.48 & 18.08 & 15.92 & 15.83 & 9.11 & 9.25 & 10.75 & 8.97 & 11.89$^{*}$ & 11.70 & 4.80 & 4.74 & 2.55  & 4.61 & 2.80 & 6.42 &1\\ 
&   & 512   & 101.79 & 15.86 & 17.57 & 15.45 & 15.37 & 8.90 & 9.09 & 10.47 & 8.50& 8.25 & 11.12 & 4.76& 4.68 & 2.53 & 4.44 & 2.78 & 6.29 &0\\ 
&   & 1024  & 96.52 & 15.25 & 16.73 & 14.85 & 14.73  & 8.66  & 8.83  & 10.15 & 8.05 & 8.10 & 10.38 & 4.68 & 4.40 & 2.50 & 4.32 & 2.75 & 5.93 &0\\ 
&   & $\mu$  & 101.07 & 16.05 & 17.83 & 15.65 & 15.56 & 9.00 & 9.18 & 10.60 & 8.72 & 10.59 & 11.45 & 4.78 & 4.67 & 2.54 & 4.53 & 2.79 & 6.31 &-\\ 
&   & $\sigma$ & 5.87 & 0.48 & 0.68 & 0.48 & 0.47 & 0.22 & 0.22 & 0.30 & 0.44 & 1.69 & 0.84 & 0.06 & 0.13 & 0.03 & 0.12 & 0.03 & 0.19 &-\\ 
&   & CV (\%) & 5.81 & 2.99 & 3.82 & 3.05 & 3.02 & 2.44 & 2.37 & 2.83 & 5.07 & 15.96 & 7.32 & 1.26 & 2.83 & 1.06 & 2.57 & 1.06 & 3.01 &-\\ 
\cline{2-21}
& \multirow{8}{*}{\rotatebox[origin=c]{90}{Apple Visions Pro }}   
    & 64    & 306.88 & 34.81 & 48.79 & 36.14 & 36.69 & 21.89 & 20.46 & 25.38 & 21.07 & 20.04 & 31.46 & 10.01 & 9.54 & 5.81 & 8.41 & 5.23 & 15.05 &0\\ 
&   & 128   & 313.30 & 34.87 & 48.13 & 35.88 & 36.64 & 21.39 & 20.09 & 25.08 & 20.74 & 20.37 & 31.58 & 9.74 & 9.32 & 5.50 & 7.66 & 4.87 & 14.18 &0\\ 
&   & 256   & 309.84 & 32.62 & 44.46 & 32.44 & 33.59 & 19.68 & 19.27 & 22.95 & 18.66 & 18.36 & 28.49 & 9.05 & 8.47 & 5.35$^*$ & 7.35 & 4.75$^*$ & 13.32 &2\\ 
&   & 512   & 287.89 & 29.74 & 40.52 & 29.57 & 30.63 & 18.39$^*$ & 17.89$^*$ & 21.16 & 17.21$^*$ & 17.27$^*$ & 25.72$^*$ & 8.70$^*$ & 8.37$^*$ & 5.24 & 7.12 & 4.66 & 12.83 &7\\ 
&   & 1024  & 254.60 & 29.28$^*$ & 38.45$^*$ & 28.39$^*$ & 30.10$^*$ & 17.55$^*$ & 17.15$^*$ & 20.29 & 16.50 & 16.33 & 24.10 & 8.29 & 8.04 & 5.04 & 6.17 & 4.57 & 12.31$^*$ &7\\ 
&   & $\mu$  & 294.70 & 32.26 & 44.07 & 32.08 & 33.74 & 19.79 & 18.97 & 23.57 & 18.44 & 18.67 & 28.67 & 9.56 & 8.75 & 5.39 & 7.54 & 4.82 & 13.54 &--\\ 
&   & $\sigma$ & 21.77 & 2.61 & 3.79 & 3.27 & 2.79 & 1.78 & 1.32 & 2.11 & 2.07 & 1.85 & 3.06 & 0.67 & 0.51 & 0.21 & 0.92 & 0.21 & 1.02 &--\\ 
&   & CV (\%) & 7.39 & 8.09 & 8.60 & 10.19 & 8.28 & 8.99 & 6.96 & 8.95 & 11.21 & 9.90 & 10.69 & 7.01 & 5.85 & 3.96 & 12.21 & 4.36 & 7.53 &--\\ 
\cline{2-21}
& \multirow{8}{*}{\rotatebox[origin=c]{90}{Vision Pro (GPU)}} 
    & 64    & 1529.67 & 360.11 & 417.73 & 368.71 & 360.02 & 237.96 & 218.35 & 225.86 & 216.24 & 230.30 & 256.45 & 143.86 & 144.12 & 128.94 & 125.73 & 129.77 & 134.74 & 0\\ 
&   & 128   & 2047.04 & 394.42 & 458.48 & 402.89 & 397.81 & 249.57 & 226.23 & 230.42 & 220.94 & 243.01 & 263.26 & 150.84 & 149.97 & 141.49 & 130.77 & 137.79 & 140.35 & 0\\ 
&   & 256   & 2304.05 & 418.75 & 474.66 & 416.41 & 421.77 & 259.63 & 233.24 & 235.00 & 227.00 & 244.50 & 270.05 & 153.13 & 151.58 & 145.29 & 135.54 & 141.05 & 141.90 & 0\\ 
&   & 512   & 2255.74 & 423.83 & 484.57 & 421.13 & 423.35 & 258.17 & 232.29 & 233.41 & 233.52 & 244.26 & 276.66 & 151.08 & 150.39 & 146.56 & 142.89 & 141.09 & 141.64 & 0\\ 
&   & 1024  & 2117.98 & 416.88 & 474.03 & 409.38 & 418.41 & 246.34 & 224.61 & 232.08 & 227.66 & 225.57 & 265.67 & 139.65 & 140.05 & 142.85 & 137.78 & 138.76 & 141.32$^{*}$ & 1\\ 
&   & $\mu$  & 2050.90 & 402.80 & 461.89 & 403.70 & 404.27 & 250.34 & 226.94 & 231.35 & 225.07 & 237.53 & 266.42 & 147.71 & 147.22 & 141.03 & 134.54 & 137.69 & 139.99 &--\\ 
&   & $\sigma$ & 309.13 & 26.40  & 26.39  & 20.75  & 26.79  & 8.91   & 6.08   & 3.50   & 6.65   & 8.93   & 7.55   & 5.71   & 4.94   & 7.04   & 6.58   & 4.65   & 2.99   & --\\ 
&   & CV (\%) & 15.07  & 6.55   & 5.71   & 5.14   & 6.63   & 3.56   & 2.68   & 1.51   & 2.95   & 3.76   & 2.83   & 3.86   & 3.36   & 4.99   & 4.89   & 3.38   & 2.14   & --\\ 

\midrule

\multirow{32}{*}{\rotatebox[origin=c]{90}{TG Test}} 
& \multirow{8}{*}{\rotatebox[origin=c]{90}{Magic Leap 2}} 
    & 64    & 20.42 & 11.37 & 10.31 & 8.45 & 8.51 & 12.38 & 8.88 & 8.26 & 7.14 & 6.47 & 5.17 & 6.13 & 5.89 & 2.07 & 3.00 & 1.99 & 4.92 &0\\ 
&   & 128   & 20.41 & 11.27 & 10.32 & 8.40 & 8.47 & 12.21 & 8.80 & 8.20 & 7.09 & 6.44 & 5.13 & 6.08 & 5.86 & 2.07 & 2.99 & 1.98 & 4.89 &0\\ 
&   & 256   & 20.30 & 11.13 & 10.21 & 8.34 & 8.39 & 11.91 & 8.67 & 8.07 & 6.99 & 6.35 & 5.08 & 5.97 & 5.77 & 2.07 & 2.99 & 1.98 & 4.87 &0\\ 
&   & 512   & 20.11 & 10.92 & 10.02 & 8.23 & 8.28 & 11.43 & 8.40 & 7.83 & 6.82 & 6.21 & 5.01 & 5.81 & 5.61 & 2.06 & 2.97 & 1.97 & 4.82 &0\\ 
&   & 1024  & 19.81 & 10.53 & 9.70  & 7.98 & 8.04 & 10.56 & 7.93 & 7.43 & 6.49 & 5.94 & 4.83 & 5.47 & 5.32 & 2.06 & 2.93 & 1.96 & 4.71 &0\\ 
&   & $\mu$  & 20.21 & 11.08 & 10.11 & 8.29 & 8.34 & 11.71 & 8.74 & 8.16 & 6.91 & 6.47 & 5.06 & 5.99 & 5.67 & 2.07 & 2.98 & 1.98 & 4.84 &-\\ 
&   & $\sigma$ & 0.22 & 0.33 & 0.24 & 0.17 & 0.17 & 0.35 & 0.34 & 0.32 & 0.23 & 0.21 & 0.13 & 0.29 & 0.18 & 0.00 & 0.02 & 0.01 & 0.07 &-\\ 
&   & CV (\%) & 1.09 & 2.96 & 2.37 & 2.05 & 2.04 & 2.97 & 3.93 & 3.90 & 3.36 & 3.27 & 2.62 & 4.79 & 3.16 & 0.23 & 0.57 & 0.34 & 1.48 &-\\ 

\cline{2-21}
& \multirow{8}{*}{\rotatebox[origin=c]{90}{Meta Quest 3}}    
    & 64    & 22.96 & 13.55 & 13.92 & 11.85 & 9.29  & 11.13 & 9.59  & 10.13 & 8.49  & 8.32  & 6.66  & 5.11  & 4.95  & 3.10  & 4.77  & 2.01  & 5.30 &0\\ 
&   & 128   & 22.79 & 12.01 & 11.83 & 10.03 & 8.51  & 9.06$^{*}$  & 7.41$^{*2}$  & 8.57  & 6.57  & 7.21  & 5.98  & 3.93$^{*2}$  & 3.67$^{*}$  & 2.27  & 3.12  & 1.61  & 3.57$^{*}$ &7\\ 
&   & 256   & 21.14 & 10.37 & 9.66$^{*}$  & 7.09$^{*2}$  & 6.40  & 6.11  & 5.14$^{*2}$  & 5.74$^{*2}$  & 4.29$^{*2}$  & 5.10$^{*2}$  & 3.43$^{*}$  & 3.07$^{*2}$  & 2.41  & 1.73  & 2.42$^{*}$  & 1.43  & 2.40 &14\\ 
&   & 512   & 19.62 & 7.78$^{*3}$  & 5.53  & 4.49  & 4.18  & 4.79  & 3.85  & 4.01  & 3.29  & 3.53  & 2.66  & 2.55  & 2.16  & 1.63  & 2.15  & 1.37  & 2.14 &3\\ 
&   & 1024  & 12.65$^{*}$ & 5.82  & 4.52  & 3.75  & 3.53  & 3.99  & 3.32  & 3.41  & 2.92  & 2.88$^{*}$  & 2.34  & 2.31  & 2.03  & 1.61  & 2.04  & 1.32  & 2.01 &2\\ 
&   & $\mu$  & 19.43 & 9.71  & 8.81  & 7.04  & 6.38  & 7.02  & 5.87  & 6.37  & 5.11  & 5.41  & 4.21  & 3.79  & 3.44  & 2.27  & 3.30  & 1.75  & 3.40 &-\\ 
&   & $\sigma$ & 4.45  & 3.47  & 3.33  & 3.08  & 2.62  & 3.13  & 2.48  & 2.82  & 2.35  & 2.11  & 1.92  & 1.29  & 1.13  & 0.72  & 1.04  & 0.35  & 1.29 &-\\ 
&   & CV (\%) & 22.91 & 35.76 & 37.76 & 43.72 & 41.12 & 44.57 & 42.25 & 44.26 & 45.99 & 38.94 & 45.66 & 34.14 & 32.95 & 31.73 & 31.52 & 19.87 & 38.06 &-\\ 
\cline{2-21} 
& \multirow{8}{*}{\rotatebox[origin=c]{90}{Vivo X100s Pro}}   
    & 64    & 23.21 & 9.42 & 10.29 & 9.25 & 9.08 & 7.31 & 7.42 & 8.62 & 7.22 & 7.56 & 7.80 & 3.84 & 3.81 & 2.05 & 3.80 & 2.34 & 4.90 &0\\ 
&   & 128   & 22.55 & 9.26 & 10.06 & 9.14 & 8.93 & 7.25 & 7.36 & 8.53 & 7.13 & 7.49 & 7.66 & 3.83 & 3.78 & 2.07 & 3.70 & 2.33 & 5.00 &0\\ 
&   & 256   & 22.71 & 9.11 & 9.98 & 9.03 & 8.88 & 7.18 & 7.18 & 8.31 & 6.67 & 6.95 & 7.21 & 3.80 & 3.66 & 2.04 & 3.55 & 2.24 & 4.84 &0\\ 
&   & 512   & 22.35 & 8.85 & 9.53 & 8.66 & 8.50 & 6.70 & 6.80 & 7.67 & 5.97 & 6.27 & 6.59 & 3.61 & 3.50 & 1.97 & 3.44 & 2.20 & 4.48 &0\\ 
&   & 1024  & 21.69 & 8.20 & 8.78 & 8.00 & 7.82 & 6.24 & 6.15 & 6.96 & 5.18 & 5.32 & 5.24$^{*}$ & 3.44 & 3.27 & 1.93 & 3.29 & 2.07 & 4.25 &1\\ 
&   & $\mu$  & 22.10 & 8.77 & 9.71 & 8.81 & 8.65 & 6.93 & 7.00 & 8.02 & 6.43 & 6.92 & 6.70 & 3.70 & 3.60 & 2.01 & 3.56 & 2.24 & 4.69 &-\\ 
&   & $\sigma$ & 0.74 & 0.46 & 0.55 & 0.47 & 0.42 & 0.38 & 0.42 & 0.63 & 0.88 & 1.04 & 0.94 & 0.16 & 0.25 & 0.05 & 0.22 & 0.10 & 0.25 &-\\ 
&   & CV (\%) & 3.35 & 5.24 & 5.67 & 5.34 & 4.85 & 5.48 & 5.97 & 7.87 & 13.71 & 15.04 & 14.03 & 4.39 & 6.92 & 2.49 & 6.18 & 4.56 & 5.33 &-\\ 
\cline{2-21}
& \multirow{8}{*}{\rotatebox[origin=c]{90}{Apple Visions Pro}}    
    & 64    & 45.14 & 18.76 & 23.17 & 17.56 & 18.97 & 15.97 & 14.84 & 17.03 & 14.87 & 14.22 & 12.29 & 7.87 & 7.65 & 4.94 & 6.82 & 6.83 & 10.72 &0\\ 
&   & 128   & 45.17 & 19.32 & 22.30 & 18.48 & 18.41 & 15.15 & 14.15 & 16.97 & 13.76 & 13.41 & 12.11 & 7.27 & 7.16 & 4.57 & 6.05 & 6.24 & 10.48$^*$ &1\\ 
&   & 256   & 47.83 & 18.15 & 20.88 & 16.85 & 17.15 & 14.04 & 12.83 & 15.55 & 12.54 & 12.12 & 12.23 & 6.74 & 6.83 & 4.46 & 6.03 & 6.11$^{*3}$ & 9.02$^{*2}$ &5\\ 
&   & 512   & 47.24 & 17.97$^{*3}$ & 19.69 & 16.94$^{*2}$ & 16.52$^{*2}$ & 13.26 & 11.63 & 14.17 & 11.61$^{*3}$  & 10.59 & 11.84$^{*2}$ & 3.03$^{*2}$  & 3.12$^{*2}$ & 4.16 & 5.52 & 2.88$^{*}$ & 4.06$^{*2}$ &19\\ 
&   & 1024  & 46.35 & 16.58$^{*3}$ & 19.33 & 15.37$^{*3}$   & 13.43$^{*2}$ & 11.69 & 10.19 & 12.54$^{*}$ & 5.68$^{*2}$  & 9.80  & 5.73$^{*}$  & 2.83$^{*}$ & 5.87$^{*2}$ & 4.38$^{*}$ & 5.31  & 2.86 & 4.57$^{*2}$ &18\\ 
&   & $\mu$  & 46.75 & 18.56 & 21.47 & 17.04& 17.90 & 14.82 & 12.73 & 15.65 & 11.69 & 12.83 & 10.84 & 5.55 & 6.13 & 4.50 & 6.35 & 4.98 & 8.57 &--\\ 
&   & $\sigma$ & 1.15  & 0.97  & 1.43  & 1.14  & 1.89  & 1.33  & 1.63  & 1.55  & 3.58  & 1.82  & 2.47  & 2.42 & 1.48 & 0.54 & 0.72 & 1.35 & 2.16 &--\\ 
&   & CV (\%) & 2.46  & 5.23  & 6.67  & 6.68  & 10.57 & 8.97  & 12.81 & 9.91  & 30.61 & 14.20 & 22.79 & 43.69 & 24.15 & 12.00 & 11.34 & 27.11 & 25.20 &--\\  
\cline{2-21}
& \multirow{8}{*}{\rotatebox[origin=c]{90}{Visions Pro (GPU)}} 
    & 64    & 43.34 & 22.71 & 28.64 & 22.58 & 22.14 & 23.41 & 19.23 & 19.47 & 17.24 & 16.78 & 14.68 & 14.48 & 13.80 & 14.74 & 15.16 & 14.89 & 15.06 & 0\\ 
    && 128   & 42.35 & 23.17 & 28.00 & 23.42 & 22.64 & 23.50 & 19.31 & 19.92 & 17.58 & 17.78 & 14.66 & 14.03 & 13.57 & 14.51 & 14.99 & 14.91 & 14.89 & 0\\ 
    && 256   & 41.85 & 23.08 & 27.01 & 22.89 & 22.62 & 23.81 & 19.31 & 19.12$^{*}$ & 17.39 & 17.67 & 14.30 & 12.24 & 12.16 & 14.34 & 15.16 & 14.62 & 14.70 & 1\\ 
    && 512   & 40.73 & 22.08$^{*}$ & 27.67$^{*}$  & 23.36 & 21.85 & 23.08$^{*}$  & 18.08 & 18.99$^{*}$  & 16.75 & 16.97 & 13.75 & 11.37 & 9.91  & 13.84 & 14.65 & 13.00 & 13.15 & 4\\ 
    && 1024  & 39.80 & 19.95$^{*}$  & 26.66$^{*}$  & 21.06$^{*}$  & 19.49 & 19.87$^{*}$  & 15.72 & 17.88 & 15.58 & 14.55$^{*}$  & 13.55 & 10.51 & 9.87$^{*}$   & 12.78 & 13.02 & 12.50 & 11.66$^{*}$  & 7\\ 
    && $\mu$  & 41.61 & 22.20 & 27.60 & 22.66 & 21.75 & 22.73 & 18.33 & 19.08 & 16.91 & 16.75 & 14.19 & 12.53 & 11.86 & 14.04 & 14.60 & 13.98 & 13.89 & --\\ 
    && $\sigma$ & 1.38 & 1.33  & 0.79  & 0.96  & 1.31  & 1.62  & 1.55  & 0.76  & 0.80  & 1.30  & 0.52  & 1.70  & 1.91  & 0.78  & 0.91  & 1.15  & 1.46  & --\\ 
    && CV (\%) & 3.32 & 5.98  & 2.85  & 4.24  & 6.00  & 7.13  & 8.45  & 3.98  & 4.75  & 7.78  & 3.66  & 13.57 & 16.07 & 5.55  & 6.20  & 8.20  & 10.52 & --\\ 

\bottomrule
\end{tabular}
}
\end{table*}

\noindent{}\textbf{String length and PP vs TG Speed}: PP speed is consistently higher than TG speed, with a speedup of approximately two to three times across the four (CPU-based) devices, while on \gls{avp} (GPU), the PP speed is significantly higher, ranging from 10 to 19 times the TG speed. Similarly, increasing the prompt size slightly reduces the prompt processing speed in PP, whereas in TG, the speed reduction is more significant with a growing token set. This suggests that TG has a higher computational cost due to its sequential execution of tokens, whereas PP is somewhat parallelized, requiring only the loading of prompts. Though there is minor variation in the 64, 128, and 256 string lengths, the last two, 512 and 1024, are consistently the slowest. When examining the variation across string lengths, \gls{mq3} exhibits the largest variance, whereas \gls{ml2} and \gls{avp} have the smallest variance.  \\
\noindent{}\textbf{Device-Based Analysis:} If we examine the devices, the overall performance trends reveal that \gls{avp} consistently shows the highest speed in both PP and TG tests. \gls{ml2} ranks second in both PP and TG performance, while Vivo X100s Pro outperforms \gls{mq3} in TG (particularly in longer strings 512, 1024) but ranks fourth in PP. This ranking highlights the varying computational capabilities of XR devices, with AVP (both CPU and GPU) standing out as the fasted one across all conditions. \gls{ml2} shows stable performance across all variations of string lengths, with lower $CV\%$ values, indicating that it does not vary significantly with changing string lengths. In contrast, \gls{mq3} shows a higher $CV\%$, indicating greater variance with increasing string length.  \\
%-------------------------------------------------
\noindent{}\textbf{Error Counts:} \Cref{fig:resErrorsCount} shows the errors counts during the PP and TG tests. Errors mostly occurred in \gls{avp} (CPU) and \gls{mq3}, particularly at longer prompt lengths of 512 and 1024 tokens. There are a few reasons for this. First, longer strings process more slowly, leading to abrupt changes and occasional failures. Second, memory constraints or inefficiencies in sustained generation contribute to these errors. With \gls{avp}, the general user experience is also suboptimal, as the device must be actively mounted on the head, which can cause errors if not handled carefully. Overall, \gls{mq3} records the highest count of errors, with frequent retries affecting its reliability, whereas \gls{ml2} remains highly stable, leading to minimal inconsistencies (zero errors). As expected, AVP (GPU) also reported zero errors. \\
\textbf{Model-Based Analysis:} In the model-wise analysis, smaller models such as Qwen2-0.5B and Vikhr-Gemma-2B achieve the highest speeds across devices, particularly in PP. In contrast, larger models like the LLaMA-2-7B series and Mistral-7B series exhibit significantly lower processing speeds, with higher variability and instability, especially in TG. Model size also impacts processing speed, as smaller models are faster. Here, $m_1$ is the fastest (though omitted from later analysis), while $m_2$, $m_3$, and $m_5$ are the fastest among the remaining models, whereas $m_{14}$ and $m_{16}$ are the slowest.}
\subsection{Results of Parallelization with BT and TT}
\label{sec:resultsTTBT} 
\dk{This section presents the results of BT and TT, both of which are used to achieve concurrency and parallelization. Regardless of the parallelization method, the three models $m_2$, $m_3$, and $m_5$ consistently rank among the fastest across all devices, while $m_{14}$ and $m_{16}$ are the slowest.\\
\noindent\textbf{Results of BT:}  
\cref{fig:resBT} presents the results of the batch test with varying batch sizes of 128, 256, 512, and 1024. Generally, increasing the batch size leads to a decrease in performance across all four devices (except for \gls{avp} (GPU)) due to increased computational overhead. \gls{avp} (GPU), with its strong computational resources, does not show a significant performance drop with varying batch sizes, although there are some fluctuations at a batch size of 256. This indicates that GPU-based processing on \gls{avp} (GPU) is highly optimized for parallel execution and can effectively manage larger batches of input without affecting processing speed. \\
More importantly, \gls{avp} (CPU) experiences a noticeable performance drop as the batch size increases. However, the remaining devices do not follow this trend. For instance, on \gls{mq3}, batch size 128 yields the best performance, whereas for the other batch sizes, there is no significant change. Similarly, for \gls{ml2} and \gls{vivo}, variations in batch size do not appear to have a substantial impact on performance.\\
\noindent\textbf{Results of TT:}  
\cref{fig:resThreadA} presents the TT results for thread counts of 1, 2, 4, 8, 16, and 32. Across all four devices, processing speed sees the most significant increase when increasing the thread count from 1 to 4. The speed gain continues at 8 threads (or shows minor degradation), but beyond 8 threads, performance begins to decline slightly, with further degradation beyond 16 threads. Apple Vision Pro (CPU) fails at thread counts of 16 and 32, indicating its limitations in achieving this level of parallelism.\\
\gls{avp} (GPU) follows a different trend, maintaining consistently high processing speed across all thread counts. Unlike CPU-based devices, its performance remains stable even at 32 threads, demonstrating its superior ability to handle concurrent tasks. These results highlight that while the four CPU-based devices benefit from moderate threading, \gls{avp} (GPU) is significantly more efficient at scaling concurrency without experiencing notable performance degradation. Since our study primarily focuses on CPU-based implementation, we conclude that using a moderate thread count of 4, 6, or 8 yields optimal results.}
\begin{figure}[!htbp]
    \centering
    \includegraphics[width=1\linewidth]{images/memoryresults.pdf}
    \vskip -0.396478cm 
    \caption{ \dk{Memory consumption for each model-device pair. The results represent the mean values across batch sizes 128, 256, 512, and 1024. }}
    \label{fig:Memory_plot}
    \vskip -0.396478cm 
\end{figure}
\subsection{Memory Consumption Results }
\label{sec:resultsMemory}
\dk{\cref{fig:Memory_plot} presents the memory consumption results for each model-device pair. The values represent the mean memory consumption from five experiments with varying batch sizes of 128, 256, 512, and 1024. As expected, memory consumption varies across models, with most showing consistent usage across different devices.    Generally, within each model series, memory consumption increases with model size. However, minor exceptions exist, such as $m_7$ on Apple Vision Pro (GPU), which consumes more memory than other models in the Phi-3.1 series $[m_6-m_{11}]$. This size-based trend does not necessarily hold across series, even for models of similar sizes. For example, \textit{m2} (1.36 GB) consumes less memory than  $m_6$ (1.32 GB) despite having a larger size, as they belong to different series. Similarly,  $m_7$ (1.94 GB) requires more memory than \textit{$m_5$} (2.0 GB) and \textit{$m_{15}$} (2.05 GB), as all three belong to different series. Another example is  $m_{12}$ (2.63 GB) consuming more memory than $m_{16}$ (2.81 GB), despite being smaller in size.} \\
\dk{From a device-specific perspective, Apple Vision Pro demonstrates significantly lower memory consumption than the other devices, averaging $0.5782$ GB on the CPU and $0.5488$ GB on the GPU across all $17$ models. Among the remaining devices, Meta Quest 3 consumes the least memory at $2.3520$ GB, followed by Vivo X100 Pro at $2.3540$ GB, while Magic Leap 2 exhibits the highest at $2.3748$ GB. These findings suggest that Apple Vision Pro (both CPU and GPU) is approximately four times more memory-efficient than Magic Leap 2, Meta Quest 3, and Vivo X100 Pro. Its superior memory management and hardware optimizations make it the most efficient device in terms of memory consumption.}
\subsection{Battery Consumption Results}
\label{sec:resultsBattery}
\Cref{fig:BatteryTEST} presents the battery consumption results over a 10-minute experiment. \dk{We observe that for the first two series, the Qwen Series and Vikhr-Gemma Series, battery consumption remains relatively low and with lower variation across all four devices. However, for larger models in the LLaMA-2 Series and Mistral-7B Series, battery consumption increases significantly. 
\gls{avp} shows a strong correlation between model size and battery consumption, both for GPU and CPU usage. In contrast, the remaining three devices do not exhibit a significant variation with respect to model size. \gls{vivo} demonstrates the best battery life, with an average loss of only $2.5\%$ over 10 minutes. \gls{ml2} follows with a $8.5\%$ average battery loss, while \gls{mq3} records a slightly higher battery loss at $9.7\%$. \gls{avp} shows the worst in this regard, with GPU usage resulting in a $10.1\%$ battery loss and with CPU usage leading to a $12.6\%$ battery loss in 10 minuets.} 

\begin{figure}[!htbp]
    \centering 
    \includegraphics[width=1\linewidth]{images/battery.pdf}  
    \vskip -0.396478cm 
    \caption{ \dk{Battery Consumption in 10-minute experiments. }  }
    \label{fig:BatteryTEST}
\end{figure}

\subsection{Pareto Front Results}

Since we use \dk{both CPU and GPU} for inference in the Vision Pro experiments, while the other three devices use CPU only for inference, we exclude \dk{Apple Vision Pro GPU} from the Pareto calculation to ensure a fair comparison. \dk{With 4 devices and 16 models (m2-m17), we have a total of 64 device-model pairs.} Using \cref{equ:pareto} and \cref{equ:score}, we calculated the \textit{quality}, \dk{\textit{stability} and \textit{performance}}  scores for each pair, then identified the Pareto fronts and visualized them in \cref{fig:pareto}. The quality of a model can be evaluated using various benchmarks, such as accuracy and perplexity, as detailed in \cref{sec:modelsQA}. \dk{ For the performance objective, we evaluate performance using \textit{PP}, \textit{TG}, \textit{memory consumption}, and \textit{battery consumption}, with respective weights of 0.35, 0.35, 0.2, and 0.1. For the stability objective, we assess it based on the coefficient of variation (\textit{CV}) and error count of \textit{PP} and \textit{TG}, assigning 0.7 to CV and 0.3 to error count.
} 
\begin{figure}[!htbp]
    \centering
    \includegraphics[width=1.0\linewidth]{images/pareto.png}
    \vskip -0.578cm 
    \caption{\dk{Pareto fronts for various device-model pairs, with different colors representing distinct device types. Performance is evaluated based on processing speed, memory consumption, and battery consumption. Stability is assessed using the \textit{CV} and \textit{error count} values of \textit{PP} and \textit{TG}. Quality is measured through evaluation results on six benchmark datasets.
    } }
    \label{fig:pareto}
\end{figure}

\dk{\Cref{fig:pareto} illustrates the Pareto front points, highlighting the optimal choices across different model sizes and devices. For example, the AVP (CPU) device achieves the highest speed among the models but exhibits relatively low stability, whereas the ML 2 device offers the greatest stability. Meanwhile, both Vivo and Meta Q3 deliver a balanced trade-off between speed and stability. At the model-pair level, $m_17$ and $m_11$ provide high quality on AVP (CPU) and Vivo, respectively. At the same time, $m_3$ is selected as a Pareto-optimal choice due to its high performance on AVP (CPU) and robust stability on Meta Q3. A similar trend is observed with $m_8$, which attains strong performance and quality on AVP (CPU) as well as commendable quality and stability on ML 2.}



 
% \input{content/05-Demo}
\subsection{General Discussion}
Our
study provided empirical evidence that the intervention improved some aspects of users' general media literacy, their behavioral intentions to use different strategies for checking misinformation and their performance in discriminating between fake and real news. However, it did not substantially enhance players’ prosuming skills, despite the mechanics of the game being tailored to support content generation. One possible explanation of this result would be that pre-survey data indicated that participants %already 
entered the study with high levels of confidence in their prosuming abilities. This "ceiling effect" likely constrained measurable gains, as participants had or perceived less room for improvement.
Another explanation is that high-level skills of critical prosuming require collaborative efforts and collective intelligence between participants \cite{lin2013understanding}, which was not possible to achieve in the PvP model of the game. Finally, the game only provided four rounds of content creation for each person, which could be sufficient to apply a critical perspective on the content, but not enough to
train creation effectiveness.

Another important finding is that, similarly to the previous studies of another game-based intervention \cite{leder2024feedback} we did not find significant effects from the intervention on the person's self-confidence in tackling misinformation.
Interestingly, game log data indicated that players' proficiency improved during gameplay; by the third and fourth rounds, they typically produced longer and more comprehensive messages compared to the initial rounds. Social Cognitive Theory \cite{bandura1997self}. may explain this discrepancy. According to it, enactive mastery experiences — ie., successful task completion— are the most influential sources of self-efficacy\cite{bandura1997self}. Positive experiences bolster self-efficacy, while repeated failures undermine it. In our game, success was determined by evaluations from LLM-simulated public opinion. The qualitative results revealed that the players found that no single strategy was effective for all characters, requiring frequent adjustments based on the unique characteristics of each character. This unpredictability made it difficult to achieve consistent success. While players developed greater proficiency during the game, the difficulty in achieving consistent success may have limited their perceived self-efficacy. However, this outcom can also be considered through the lens of the educational effects of the game. Previous studies about differential challenges of misinformation showed that young adults %might lack the critical thinking skills needed to assess information effectively and 
often overestimate their ability to assess information effectively\cite{papapicco2022adolescents,porat2018measuring}; in this context, the reaction of our participants, most of whom were young adults could be a positive signal that they became aware of the complexity of misinformation and the absence of one-size-fit-all solutions. 
%I deleted this explanation
%Furthermore, as character evaluations responded to the actions of both players, scores fluctuated frequently, with the gains of one player often neutralized by the moves of their opponent. This dynamic likely hindered the confidence of the players in their ability to discern misinformation successfully, even though their actual performance improved, as evidenced by significant gains in discriminative test results.Thus, while players developed greater proficiency during the game, the lack of sustained clear success may have limited their perceived self-efficacy. 

%For example, in our qualitative findings, participants reported scepticism toward information that purely relied on authoritative sources because they found their opponents used fabricated evidence from these sources to gain LLM-simulated characters’ trust and increase scores. This gameplay experience reminded them of real-life situations, where misinformation often exploits trust by citing credible authorities. As a result, players learned to examine the intent behind messages instead of automatically trusting authoritative sources.
%\textcolor{blue}{This evolution in how participants evaluated information aligns with the Elaboration Likelihood Model (ELM)\cite{petty1984source}, which describes how individuals process persuasive messages via central(critical evaluation of content) or peripheral routes(reliance on heuristic cues). In the gameplay, participants appeared to shift from peripheral processing (trusting authority as a shortcut) to central processing (evaluating the intent and content of the message) when exposed to the misuse of authority in gameplay. Notably, many participants suggested cross-checking sources as a practical solution for verifying information, demonstrating their enhanced critical thinking skills}.

Lastly, In our qualitative findings, participants reported skepticism toward information that purely relied on authoritative sources because they found their opponents used fabricated evidence from these sources to gain the trust of the LLM-simulated characters, with a view to increasing their game score. This gameplay experience reminded them of real-life situations, where misinformation often exploits trust by citing credible authorities. As a result, players learned to examine the intent behind messages instead of automatically trusting authoritative sources. This evolution in how participants evaluated information aligns with the Elaboration Likelihood Model \cite{petty1984source}, which describes how individuals process persuasive messages via central (critical evaluation of content) or peripheral routes (reliance on heuristic cues). In the gameplay, participants appeared to shift from peripheral processing (trusting authority as a shortcut) to central processing (evaluating the intent and content of the message) when exposed to the misuse of authority in gameplay.
Furthermore, our findings align with studies showing that gamified inoculation techniques for pre-bunking misinformation can trigger skepticism related to both false and real news\cite{hameleers2023intended,modirrousta2023gamified}. While this induced skepticism might seem limiting —potentially thus undermining trust in high-credibility sources— we believe it supports the goal of fostering critical media literacy. By encouraging players to evaluate the trustfulness of content, source and intent,  the game develops essential skills for navigating today's complex information landscape. Rather than promoting cynicism, this skepticism cultivates constructive inquiry that helps individuals better discern reliable information. Notably, many participants pointed to the need to cross-check sources as a practical solution for verifying information, demonstrating their enhanced critical thinking skills.  
%Additionally, recent game interventions have explored ways to improve individuals' ability to assess both real and false information. Strategies include teaching players to identify credible versus low-credibility media \cite{micallef2021fakey}, rewarding accurate judgments \cite{barzilai2023misinformation}, and incorporating feedback tests after gameplay \cite{leder2024feedback}. }

\subsection{Outcomes of the Game Mechanics}

\subsubsection{PvP model for Media-Literacy Game}
Unlike prior misinformation education game which broadly employ single-player mechanics \cite{roozenbeek2019fake,camCambridgeGame,harmonysquare,jeon2021chamberbreaker}; "Breaking the News" applies PvP mechanics. While in general, previous studies in serious games showed that PvP games are more engaging and motivating \cite{cagiltay2015effect}; at the same time in competitive environments, the motivation to “win” may overshadow educational goals.  In addition, in recent studies of another misinformation game it showed that some students may not find %gamification or
competition enjoyable or motivating\cite{axelsson2024bad}. 
%our results also indicate players learnd from the opponents. 
Our results demonstrated that, in general, our participants were highly motivated to play one against another. Moreover, they indicated that they learned from each other's strategies, and it helped them better understand the dynamic of misinformation. Therefore, we can conclude that, in our case, this approach was beneficial to fulfilling the intended purpose of the game. 
However, our study was conducted on East Asian participants, who are part of a collectivist-oriented rather than competitive culture \cite{chung1999social}. Previous studies have shown that cultural factors play an important role in the degree of competitiveness in gamified interventions \cite{oyibo2017investigation}. Therefore, it is also possible that in other cultural settings, the game's incentives can trigger more intense competition, which can negatively affect educational results. 


%These games frame the narrative as either misinformation creator or misinformation debunker and asks the player to adopt one of the roles, tasked by discerning misinformation using choice-based interaction. 


%\subsubsection{Free-form Input Generation}
%\textcolor{blue}{One notable feature of Breaking the News is its open world narrative and free-form response format, which contrasts with the linear choice-based formats commonly used in prior game intervention [REF]. 
%Previous works in the field of role-playing games showed that this approach makes game interaction more natural and increases user engagement \cite{csepregi2021effect,ashby2023personalized}.


%This format positively influences players’ learning outcomes by enhancing engagement and replayability. 

%In choice-based formats,interactions are typically brief, requiring players to passively select predetermined options. This structure can leads to "guesswork," where learners may choose correct answers without fully understanding the underlying concepts. In contrast, free-form responses force players to actively reason and articulate original ideas, fostering deeper engagement and critical thinking. 

%Additionally, by allowing players to develop their own narratives and gameplay experiences, each playthrough feels unique, encouraging them to return to the game.

%This variability not only boosts intrinsic motivation but also contributes to the game's educational value. When players repeatedly engage with core mechanics through varied narratives, they continue practicing essential skills, thereby reinforcing and enhancing their learning outcomes\cite{kucklich2004play}}.

%\textcolor{blue}{An additional strength of this format lies in the autonomy it provides. Players can craft personalized narratives that align with their individual learning styles, a factor proven to result in better learning outcomes in serious game design\cite{hwang2012development}. Furthermore, the freedom to explore and shape their own gameplay experiences not only deepens engagement but also excites players, acting as a powerful motivator for repeated play\cite{ravyse2017success}. By promoting intrinsic motivation and repeated practice, the free-form and open-world design demonstrate their potential to enrich both player experience and educational effectiveness. Given that inoculation interventions often show diminishing effectiveness over time, developing an engaging, replayable game that can consistently reinforce players' resistance to misinformation.}

%Free form rewriting: 
\subsubsection{Free-form Input Generation}
One notable feature of Breaking the News is its free-form response format, which contrasts with the linear choice-based formats commonly used in prior game intervention\cite{roozenbeek2019fake,camCambridgeGame,harmonysquare,jeon2021chamberbreaker,micallef2021fakey}. 
Previous works in the field of role-playing games showed that this approach makes game interaction more natural and increases user engagement \cite{csepregi2021effect,ashby2023personalized}.
Based on our data, we can say that this format positively influences players’ learning outcomes by enhancing engagement and replayability. 
By allowing players to develop narratives themselves, each playthrough feels unique, encouraging players to return to the game. This variability not only boosts intrinsic motivation but also contributes to the game's educational value. When players repeatedly engage with core mechanics through varied narratives, they continue practicing essential skills, thereby reinforcing and enhancing their learning outcomes\cite{kucklich2004play}. 

In an educational setting, previous works showed that in choice-based formats, interactions are typically brief, requiring students to select predetermined options. This structure could lead to "guesswork," where learners may choose correct answers without fully understanding the underlying concepts. In contrast, free-form responses force students to actively reason and articulate original ideas, fostering deeper engagement and critical thinking \cite{bryfczynski2012besocratic}. This autonomy allows players to craft responses based on their understanding of their roles, making this another motivating factor for returning to the game\cite{ravyse2017success}.
 
%\textcolor{blue}{An additional strength of this format lies in the autonomy it provides, allowing them to craft responses based on their own understanding of their roles. For example, during gameplay,  we observed that one player in the journalist-debunker role chose to mimic social media comments rather than rely on scientific strategies, crafting responses that resembled typical user comments. The freedom to explore and shape their own gameplay experiences not only increase engagement but also motivates players to return\cite{ravyse2017success}. Given the memory-strengthening effects of repeated learning, requiring people to recall what they have learned helps them relearn and reinforce these lessons\cite{nader2009single}.Therefore, developing an engaging, replayable game can consistently reinforce players' resistance to misinformation.}
%Players can craft personalized narratives that align with their individual learning styles, a factor proven to result in better learning outcomes in serious game design\cite{hwang2012development}. Furthermore, 


\subsubsection{LLM-Powered Feedback}
%\textcolor{blue}{Another innovation of Breaking the News lies in its interaction and feedback mechanisms. The game incorporates LLM-based role-play as evaluators, creating an interactive feedback for players. LLMs demonstrate the ability to simulate human behavior and reactions, consistent with findings from prior research[REF]. To the best of our knowledge, this is the first attempt to integrate LLM role play into a misinformation education game, opening new possibilities for interactive learning experiences in media literacy game interventions.} 

%\textcolor{blue}{In the game, the LLMs mimic human reactions and behaviors by leveraging demographic, personality, behavioral, and psychological features towards misinformation. This approach provides a personalized gameplay experience, fostering greater engagement and enhanced learning outcomes compared to binary feedback mechanisms (e.g., true or false responses). Players interact with LLM personas by crafting strategies based on their perceived characteristics. When players’ actions get the expected results(eg, persona's reaction towards on the side), they experience a sense of achievement. Conversely, when the outcomes deviate from expectations, players adapt by employing alternative persuasive strategies tailored to the persona. This dynamic interaction encourages active learning, problem solving, and critical thinking. Notably, players exhibit higher levels of engagement with personas they found relatable due to personal experiences, focusing more attentively on their feedback. This level of engagement is difficult to achieve with traditional binary feedback mechanisms, which provide limited insights beyond correctness.}

%\textcolor{blue}{Feedback plays an important role in shaping game-based learning outcomes(REF). In our case, users observed that personas react more obviously to emotional elements. This is because persona’s feedback begin with emotional feelings (such as panic), followed by behaviors (like sharing news with friends or taking no action). These emotional responses also reflect their trust level scores. To get better game performance, players used various tactics to trigger emotions or address these emotions in gameplay. As a results, most participants reported that they learned to detect emotional manipulation techniques. To further enhance learning outcomes, future games could adjust LLM outputs to teach diverse strategies more effectively. }
%Another innovation of Breaking the News is the interaction and feedback component. Compared to choice-based format where player present predetermined options and get binary and concise feedback(eg, true or false)Players tasks actively craft their responses and then receive evaluation by five different LLM-stimulated personas. These personas are based on their characteristics, offering unique feedback toward the player’s output. By engaging with these nuanced evaluations, players gain insights into why certain arguments might appear trustworthy or untrustworthy, encouraging them to refine their strategies dynamically. Furthermore, Since the game is last multiple rounds, players will get to know personas and adjust their misinformation narratives or debunking approaches based on the personas’ reactions. Thus, they learn and practice various tactics to identify and debunk misinformation.
%innovation in LLM as evaluator in game and other settings. 
%feedback in serious game's advantage and disadvantage

Another innovation of Breaking the News lies in incorporating LLM-based "characters" as evaluators, creating interactive feedback for players. Although there are other attempts to incorporate LLMs into helping users
learn about misinformation \cite{danry2023don, hsu2024enhancing} (including gamified attempts\cite{tang2024mystery}), in these approaches, AI was mostly the source of correcting information. While this approach has benefits, it is can also be criticised due to  the possibility of LLMs creating incorrect by plausible text \cite{kim2024can,agarwal2024faithfulness}. In this case, it is possible that the intervention will disinform people to an even greater extent. In contrast, in our study, we incorporate AI not as a source of information but as non-playable characters with their own opinions. Therefore, the educational part of the game becomes more robust to resist the negative effects of erroneous generations, as they can only affect the opinion of the "character" but not the main narrative of the game. In general, LLMs demonstrate the ability to simulate human behavior and reactions, consistent with findings from prior research\cite{1park2023generative}; we also find, that the system is capable of emulating the opinion of 5 different characters at the same time and still keeping them consistently different during the game.
 
One advantage of dynamic LLM feedback is that it achieves greater engagement compared to binary feedback (e.g., true or false). Based on our observation, players adapted by employing alternative persuasive strategies tailored to the character and concentrated on the character's feedback. Notably, players exhibit higher levels of engagement with characters they found relatable due to personal experiences, focusing more attentively on their feedback. This level of engagement is difficult to achieve with traditional binary feedback mechanisms, which provide limited insights beyond checking their correctness.%\textcolor{blue}{Feedback plays an important role in shaping game-based learning outcomes(REF). In our case, users observed that characters react more obviously to emotional elements. This is because character’s feedback begin with emotional feelings (such as panic), followed by behaviors (like sharing news with friends or taking no action). These emotional responses also reflect their trust level scores. To get better game performance, players used various tactics to trigger emotions or address these emotions in gameplay. As a results, most participants reported that they learned to detect emotional manipulation techniques.} %To further enhance learning outcomes, future games could adjust LLM outputs to teach diverse strategies more effectively.

%In the game, the LLMs mimic human reactions and behaviours by leveraging demographic, personality, behavioural, and psychological features towards misinformation. Based on our observation, players were able to catch these different clues and crafted strategies based on these characteristics; players adapted by employing alternative persuasive strategies tailored to the character and concentrated on the character's feedback. This level of engagement is difficult to achieve with traditional binary feedback mechanisms, which provide limited insights beyond correctness.

\subsection{Design Implications for the Further Development of Serious Games with LLM-components}
Based on our experience designing and testing the game, we summarize the following recommendations to practitioners working on similar projects.

\subsubsection {Balance Between Freedom and Guidance}
In our game, we sought to challenge the existing choice-based approach in misinformation education games by providing users with free-form input. We found this approach triggers reflection, which helps to build hands-on experience and make the game more enjoyable. Yet, we also found that it relies on players’ existing knowledge of misinformation. For example, players might incorporate unverified information they’ve encountered on social media into the game, which is specifically problematic for the debunker role. While we provided the players with comprehensible instructions to guide their role's actions (how to act as a debunker or an evil influencer), it would be better to incorporate more context-specific tips in each stage of the game to help users explore different ways of winning the game and deepening their learning. We suggest using the approach used in the free-input educational interventions (e.g. \cite{bryfczynski2012besocratic}) to build clear, understandable criteria for free-form answers. These will not stop creativity but help people tailor their answers to the context of the game. We also suggest adding a preliminary stage to the game in the form of a simple choice-based questionnaire, helping people understand the role and the guidelines for the role. For example, we can introduce to the individual playing the debunker role the Debunking Ethics Standards  \cite{edmo2024,afp2024,eeas2024}, and ask them a series of questions about understanding the content of the guidelines before starting the main game. In addition, we can incorporate an additional AI-based mechanic to check the quality of answers, not only in terms of its influentiality, but also of its coherence with debunking practices guidelines. Similarly, we can demonstrate a "Score of Manipulativeness" to the influencer role. This better shows us how the ideas players implement into the message can be judged, from the point of view of dealing with misinformation.
\subsubsection {Replayability and Feedback}
%LLM 
One of the critical challenges in serious games is maintaining replayability, as this is important to facilitate the learning process \cite{adetunji2024unlocking} and making interventions more sustainable \cite{silveira2016open}. Moreover, a lack of replayability in educational games can limit both educational and behavioral change \cite{epstein2021tabletop}. To address this, we incorporated elements designed to ensure the game can be played multiple times, such as offering two distinct player roles and providing a free-form input mechanism that allows participants to explore a broader decision space. However, we recommend that future interventions consider additional methods to further enhance replayability.

One such approach involves leveraging opportunities to introduce various characters to represent public opinion. In our game, we observed that players finds characters' feedback more engaging when this resonates with them personally. By introducing more characters, or by allowing players to customize characters to better reflect their own experience, the game could encourage players to return and interact with new characters. The result would be a more engaging experience. For instance, research has shown that debunking misinformation often occurs within families and can sometimes lead to conflict \cite{scott2023figured}. In such a scenario, players could customize a character based on their previous experiences with family members, thereby practicing their own debunking strategies in a risk-free environment.
%Character feedback can be designed to elicit deeper cognitive engagement and reflective thinking. For example, when a player attempts to spread rumors intended to evoke fear and anxiety, the character might respond with questions such as: “What evidence supports this claim?” “Why might it be spreading so rapidly?” or “What would happen if I believe this information?” Studies have shown that AI-framed questioning can stimulate user reasoning and enable independent information evaluation(REF).This approach can also motivate players to return to the game equipped with new strategies. Over multiple playthroughs, as players become more skilled in critical reasoning, they gain access to increasingly sophisticated dialogues and insights from the characters, further enhancing replayability.

While in-game feedback guides player behavior during play, post-game debriefing sessions help consolidate learning and improve future performance. Research has found that debriefing is a crucial opportunity for players to process and integrate their learning experiences\cite{crookall2014engaging,leder2024feedback,barzilai2024learning}. After gameplay, we suggest arranging debriefing sessions that allow players to review their strategies, assess their effectiveness, and receive constructive feedback, potentially improving learning outcomes. For instance, after a session focused on combating misinformation, a post-game review might present an ideal debunking response or a well-supported counterargument. Such structured reflection enables players to internalize lessons and increases the likelihood that they will re-enter the game with newly gained insights, thereby reinforcing both learning and replayability.
%For example, some characters aligned closely with players’ prior experiences, prompting players to focus on that particular feedback. In other cases, a character proved difficult to persuade, sparking the player’s interest in challenging that character and crafting more tailored responses, sometimes at the expense of others.

\subsection{Limitations and Future Work}
We acknowledge the following limitations. 
% I still think it's need cultural context in limiation
Our participants were mainly from an East Asian country, which can limit the generalizability of our findings. Previous studies have shown that cultural contexts influence individuals' perceptions of misinformation and interventions \cite{noman2024designing}. For example, study revealed that Mexican and Spanish users were more likely to trust -related misinformation compared to users from Ireland, the UK and the USA \cite{roozenbeek2020susceptibility}. At the same time, non-western participants demonstrated a significantly higher willingness to challenge misinformation. The effect of proposed misinformation design interventions was significantly greater than that seen in analysis of the behaviors and attitudes in the UK population\cite{noman2024designing}. 
Previous studies also suggest that certain populations may face greater challenges in being able to critically evaluate information. For example, a large-scale study observed that Asian individuals encounter more difficulties in assessing health information from social media compared to other populations \cite{chandrasekaran2024racial}. 
%This increased difficulty could be attributed to various factors, such as differing educational approaches[REF], access to resources[REF], or cultural norms surrounding information consumption. 
In addition, individuals from an Asian background are more likely to incorporate social media information into their health-related decisions, potentially increasing their susceptibility to misinformation \cite{chandrasekaran2024racial}. Therefore, future work should explore the cross-cultural applicability of our findings in different cultural contexts. However, as the cultural background of our participants can potentially make them more vulnerable to misinformation than other populations, we believe that our intervention is valuable as providing insight into the outlook, activity and the relevance of certain interventions for these populations.
%Serious games have the potential to address this challenge by geographic and cultural boundaries. By tailoring narrative elements and characters to resonate with specific cultural contexts, these games can enhance both engagement and learning outcomes. In our game, we incorporated traditional medicine into the narrative(News in the game), knowing that it is a topic that resonates strongly with East Asian participants. For example, 48.4\% of Hong Kong residents reported using traditional medicine before the COVID-19 pandemic\cite{lam2021public}. Similarly, a national survey in South Korea found a 74.8\% prevalence of traditional medicine use overall\cite{ock2009use}.In China, traditional medicine is formally integrated into the healthcare system as one of the most widely practiced modalities\cite{chung2023implementation}.This cultural relevance provided a familiar backdrop for the game's content, allowing players to engage more intuitively and craft responses based on their lived experiences. 

%Since the game's news scenarios are easily modifiable, future adaptations could incorporate culturally specific settings to enhance relatability and impact for other audiences. By customizing content to align with different cultural contexts, the game can potentially improve its effectiveness in media literacy and reducing susceptibility to misinformation across diverse populations.}

%In a large-scale study of US populations, Chandrasekaran et al. \cite{chandrasekaran2024racial} found that people of Asian origin are more likely to incorporate social media information into their health-related decisions, potentially increasing their susceptibility to misinformation\cite{chandrasekaran2024racial}. At the same time, previous work showed that non-western participants demonstrated a significantly higher willingness to challenge misinformation and that the effect of proposed misinformation design interventions was significantly greater compared to the UK population.
%Taking together, the further studies required to determine if the intervention will be equally efficient on other ethnical groups, however, the current results showed that the game can be used to help the populations potentially more vulnerable to misinformation.
 
Our sample was relatively homogeneous in age. A recent meta-analysis of articles about different intervention approaches showed that neither age nor gender significantly impacts the effectiveness of media literacy interventions \cite{lu2024can}. However, previous work has suggested that media literacy interventions designed for certain age groups (e.g., older adults and adolescents) achieved greater effects\cite{moore2022digital,hartwig2024adolescents}. Future work should determine if our approach is efficient in other age groups of users and, if necessary, tailor scenarios to suit the various needs of different age groups.

Thirdly, our study provided only a one-time intervention and observed immediate learning effects; 
%we are unsure how long these effects will persist.
Previous work showed that even a one-time interaction with an educational game can provide long-term improvement in misinformation recognition. For instance, Maertens et al. tested the game "Bad News" and found that inoculation effects lasted for at least 13 weeks. This suggests the potential for the long-term effectiveness of active inoculation interventions with regular assessment\cite{maertens2021long}. Still, future research should include multiple time points to assess the long-term effectiveness of our game intervention. There should also be comparisons between one-time and multiple play sessions, with explorations of the impacts of players assuming different roles within the games.
%Suppose we had incorporated a control group or compared different types of training. In that case, we might have identified elements that could be better addressed through lectures, potentially reducing gameplay time and preventing participant exhaustion.
%two players learning diff things: spread vs control, diff play flow so how can you assume they played the same game?
In this study, each player was limited to a single role, either a misinformation creator or a debunker.  This resulted in different learning experiences depending on their assigned role. %While participants reported learning from their opponents, the educational outcomes between roles may have varied. 
The primary reason for not having role-switching in our study was the length of the game and its cognitive demands, which we feared would lead to player exhaustion if roles were switched mid-game. In future iterations, we aim to improve the design by allowing players to save their progress and switch roles during subsequent sessions. This could offer a more immersive experience, as players would gain perspectives from both the misinformation creator and debunker roles. Additionally, we aim to introduce new modes, such as a family mode, where players interact with two LLM-simulated characters. This option would reduce cognitive load while maintaining engagement. 
%how is this representing range of possible misinformation? survey is not comprehensive, should be finding that they learned about one subtopic but not another that was not addressed.

The current game is also limited in its sole focus on a pandemic scenario. In reality, misinformation spans multiple domains, with health-related misinformation able to influence political events such as elections. Our game only addressed text-based misinformation, while visual and video-based misinformation pose even greater challenges and are harder to detect. Future work could include multimedia content, such as images and videos, to more accurately simulate the diverse forms of misinformation that exist in the real world.

The current game approach may unintentionally foster skepticism toward both true and false news, a common issue in misinformation pre-bunking interventions\cite{hameleers2023intended,modirrousta2023gamified}. %Recent efforts aim to improve individuals' ability to assess information credibility. These include teaching players to distinguish between credible and low-credibility media \cite{micallef2021fakey}, rewarding accurate judgments \cite{barzilai2023misinformation}, and using feedback tests after gameplay \cite{leder2024feedback}. 
While we believe that the benefits of promoting critical thinking towards sources are very important in prebunking interventions,
%discuss feedback mechanisms in the design implications, 
we further recommend incorporating features that clearly differentiate high- and low-credibility sources during gameplay.

While LLMs like ChatGPT-4o were used to simulate human reactions in the game, these models do not fully replicate the complexities of human behavior. Human reactions are often nuanced and influenced by multiple factors, including culture, history, and personal experience. LLM-generated characters may oversimplify human emotions and fail to grasp the full context of certain situations. Despite us using an advanced ChatGPT-4o model, it may have missed these subtleties, leading to interactions that feel artificial or incomplete. For example, in the gameplay, players employ strong emotional manipulation strategies to provoke specific responses. However, the LLM-generated characters do not react as expected, where participants reported feeling frustrated when their emotional manipulation strategies did not yield the anticipated reactions. This can limit the diversity and depth of the simulated interactions and detract from the realism and fairness of the experience for certain audiences.

LLMs are trained on large datasets that may contain intrinsic biases, which can manifest themselves in unintended ways during gameplay\cite{kasneci2023chatgpt}. Studies have shown that LLMs are prone to so-called "hallucinations" and can reflect stereotypes or skewed perspectives, which could affect how certain characters respond in the game\cite{xie2024can}. For instance, if biased training data influence a simulated character’s reaction, it may inadvertently reinforce player stereotypes about certain groups of people. 

%Finally, the game was primarily designed in English, which may have affected immersion and comprehension for participants who were not fluent in the language.

%how is this representing range of possible misinformation? survey is not comprehensive, should be finding that they learned about one subtopic but not another that was not addressed.

%LLM do not truly represent the public opinion, biased, misrepresented.

%two players learning diff things: spread vs control, diff play flow so how can you assume they played the same game?


%\subsection{Future Work}
%There are several directions for future work. One potential expansion is to incorporate more diverse topics of misinformation beyond pandemics and health, such as political misinformation or climate change, to enhance replayability. Although misinformation tactics share common features across different domains, each topic presents unique characteristics and challenges. We have already demonstrated that games are an effective medium for confronting misinformation, and the mechanics of \textit{Breaking the News} are designed to simulate the complexity of misinformation in the real world. By changing the in-game news topics, we could tailor the educational impact for different audiences relatively easily.

%Another direction is to allow players to switch roles, experiencing both the misinformation creator and debunker perspectives. This could provide a fuller, more immersive experience. To further enhance the educational value and replayability, we also plan to incorporate more tools to help players investigate evidence. For instance, as in MathE, which designed artificial search engines for its game, we aim to introduce investigative techniques like lateral reading and source evaluation to support both player roles.

%Additionally, future work could explore different formats of misinformation within the game, including multimedia content like images and videos. This would address the limitation of focusing mainly on text-based misinformation and would reflect the growing challenges posed by visual misinformation in real life. Allowing players to create or debunk multimedia misinformation would make the game more relevant to the evolving nature of misinformation. 


% \section*{Acknowledgments}
% This should be a simple paragraph before the References to thank those individuals and institutions who have supported your work on this article.


%{\appendices
%\section*{Proof of the First Zonklar Equation}
%Appendix one text goes here.
% You can choose not to have a title for an appendix if you want by leaving the argument blank
%\section*{Proof of the Second Zonklar Equation}
%Appendix two text goes here.}



% \section{References Section}
% You can use a bibliography generated by BibTeX as a .bbl file.
%  BibTeX documentation can be easily obtained at:
%  http://mirror.ctan.org/biblio/bibtex/contrib/doc/
%  The IEEEtran BibTeX style support page is:
%  http://www.michaelshell.org/tex/ieeetran/bibtex/
 
 % argument is your BibTeX string definitions and bibliography database(s)
%\bibliography{IEEEabrv,../bib/paper}
%
% \section{Simple References}
% You can manually copy in the resultant .bbl file and set second argument of $\backslash${\tt{begin}} to the number of references
%  (used to reserve space for the reference number labels box).

%\begin{thebibliography}{1}
\bibliographystyle{IEEEtran}  
%\bibliographystyle{abbrv}
%\bibliographystyle{abbrv-doi}
%\bibliographystyle{abbrv-doi-narrow}
%\bibliographystyle{abbrv-doi-hyperref}
%\bibliographystyle{abbrv-doi-hyperref-narrow}

\bibliography{Llamaxr}
%\end{thebibliography}

% \newpage

% \section{Biography Section}
% If you have an EPS/PDF photo (graphicx package needed), extra braces are
%  needed around the contents of the optional argument to biography to prevent
%  the LaTeX parser from getting confused when it sees the complicated
%  $\backslash${\tt{includegraphics}} command within an optional argument. (You can create
%  your own custom macro containing the $\backslash${\tt{includegraphics}} command to make things
%  simpler here.)
 
% \vspace{11pt}

% \bf{If you include a photo:}\vspace{-33pt}
% \begin{IEEEbiography}[{\includegraphics[width=1in,height=1.25in,clip,keepaspectratio]{fig1}}]{Michael Shell}
% Use $\backslash${\tt{begin\{IEEEbiography\}}} and then for the 1st argument use $\backslash${\tt{includegraphics}} to declare and link the author photo.
% Use the author name as the 3rd argument followed by the biography text.
% \end{IEEEbiography}

% \vspace{11pt}

% \bf{If you will not include a photo:}\vspace{-33pt}
% \begin{IEEEbiographynophoto}{John Doe}
% Use $\backslash${\tt{begin\{IEEEbiographynophoto\}}} and the author name as the argument followed by the biography text.
% \end{IEEEbiographynophoto}
%\vfill
\vspace{-33pt} 
\vskip -2.5\baselineskip plus -1fil
\begin{IEEEbiographynophoto}{Krithika Iyer} is a computing PhD candidate at the Kahlert School of Computing and 
the Scientific Computing and Imaging Institute at the University of Utah in the Image Analysis track. She received her B.E. in Electronics \& Telecommunication in 2015. Her research interests include medical image analysis, probabilistic modeling, deep learning, and statistical shape modeling.\end{IEEEbiographynophoto}
\vskip -2.5\baselineskip plus -1fil
\begin{IEEEbiographynophoto}{Mokshagna Sai Teja Karanam} is a computing PhD student at the Kahlert School of Computing and the Scientific Computing and Imaging Institute at the University of Utah in the Image Analysis track. He received his B.E. in Computer Science in 2020 and his Masters in 2024. His research interests include deep learning, computer vision and statistical shape modeling.\end{IEEEbiographynophoto}
\vskip -2.5\baselineskip plus -1fil
\begin{IEEEbiographynophoto}{Shireen Elhabian} is a faculty member at the Kahlert School of Computing and the Scientific Computing and Imaging Institute at the University of Utah. Her research focuses on enhancing diagnostic accuracy through deep learning, probabilistic modeling, and advanced computer vision algorithms. She has published over 100 peer-reviewed publications in prestigious journals and conferences, including IEEE-TMI, MedIA, ICLR, CVPR, ICCV, MICCAI, IPMI, AAAI. \end{IEEEbiographynophoto}
\vskip -3\baselineskip plus -1fil

\vfill
\end{document}


