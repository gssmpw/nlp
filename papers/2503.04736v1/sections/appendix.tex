
\section{Hierarchy of the Standard Development Process}
\label{app:hierarchy}

%%
%% FIGURE - HIERARCHY FIGURE
%% 
\begin{figure}[!htbp]
    \centering
    \includegraphics[width=0.50\linewidth]{fig/hierarchy.pdf}
    %\vspace{-0.8em}
    \caption{A supporting visualization of the general process of developing standards. Standards can be created either by government and regulatory bodies as a product of legislation or through industry associations and academic expert groups to ensure interoperability and quality of systems and processes.}
    \label{fig:standard-hierarchy}
    %\vspace{-0.60cm}
\end{figure}


\section{Technical Enhancements for Standard Alignment}
\label{app:technical}
%\section{Proposed Method}
\textbf{Problem Statement: } Let $\mathcal{X}$ and $\mathcal{Y}$ denote the input and output spaces, respectively, and $D = \{(x_i, y_i)\}_{i=1}^n$ the dataset, where $x_i \in \mathcal{X}$ and $y_i \in \mathcal{Y}$ are the $i^{th}$ question-answer pair. For each $x_i$, the goal is to generate a response $\hat{y}_i$ that maximizes the overall accuracy. The goal is to achieve this using a pool  of $N$ pretrained foundational LLMs without additional training or fine-tuning.


\begin{comment}
    

Let $\mathcal{X}$ and $\mathcal{Y}$ denote input and output spaces respectively and  $\mathcal{D}= \{(x_i,y_i)\}_{i=1}^n$ denote Dataset, where $x_i \in \mathcal{X}$ and $y_i \in \mathcal{Y}$ is the $i^{th}$ question-answer(QA) pair. For each $x_i$ we want to generate a response $\hat{y}_i$  such that overall accuracy denoted by $\frac{1}{n}\sum_{i=1}^n\mathbf{I}(y_i=\hat{y}_i)$ is maximized.  We want to maximize this using ensemble of $N$ LLMs without any additional training or fine-tuning.
\end{comment}
\renewcommand{\algorithmicrequire}{\textbf{Input:}}
\renewcommand{\algorithmicensure}{\textbf{Output:}}

\begin{algorithm}[t]
    \caption{Components of UAF - SELECTOR and FUSER}
    \label{alg:selector}
    
    \begin{algorithmic}
    \REQUIRE $D_{val}$, Pool of LLMs $\mathcal{M}= \{M^j\}_{j=1}^{N}$, Uncertainty function $U_f(.,.,.)$, Ensemble size $K$, Test data point $x_{test}$
    \ENSURE Test data response $\hat{y}_{test}$
    \newline

    \STATE \textbf{procedure }SELECTOR ($D_{val}, \mathcal{M}, U_f, K$)
    \FOR{$M^j \in \mathcal{M}$}
    %\STATE $L_j, U_j = \emptyset,\emptyset$
    \FOR{$(x_i, y_i) \in D_{val}$}
    \STATE $\hat{y}_i^j = M^j(x_i)$   \quad;\quad  $s_i^j = \mathbf{1}(\hat{y}_i^j == y_i)$
    \STATE $u_i^j = U_f(M^j, x_i, \hat{y}_i^j)$
    %\STATE $L_j \leftarrow L_j \cup IsCorrect_i^j$   \quad;\quad    $U_j \leftarrow U_j \cup u_{val_i}^j$
    \ENDFOR

    \STATE $Acc_j = \frac{1}{|D_{val}|}\sum_i s_i^j$  ;\quad $SAH_j = ROC\_AUC\_score(\{s_i^j,u_i^j\}_i)$
    \STATE $Cscore_j = Acc_j \times SAH_j$
    \ENDFOR
    \STATE \textbf{return:} $\mathcal{M}^{sel} =  TopK (\{Cscore_j\}_{j=1}^N)$,  \hfill\COMMENT{//Top $K$ LLMs}
    \STATE \quad  \quad      $Acc^{sel} = \{Acc_j| j \in \mathcal{M}^{sel}\}$ \hfill\COMMENT{//Accuracy of selected $K$ LLMs}
    %\STATE $j_1, j_2, \dots, j_K = \arg\max_{j \in \{1, 2, \dots, N\}} Cscore_j$ 
    %\ENSURE $j_1, j_2, \dots, j_K$    \hfill\COMMENT{//Indices of Top K llms}
    \newline
\STATE \textbf{procedure } FUSER ($x_{test}, \mathcal{M}^{sel}, Acc^{sel}, U_f, K$)
\FOR{$M^k \in \mathcal{M}^{sel}$}
\STATE $\hat{y}_{test}^k = M^k(x_{test})$
\STATE $u_{test}^k = U_f(M^k, x_{test}, \hat{y}_{test}^k)$
\ENDFOR
\STATE $\hat{y}_{test} = \hat{y}_{test}^{k^*} \text{, where } k^* = argmax_{k \in \{1,\dots K\}} Acc_k \times (1-u_{test}^k)$
\STATE \textbf{return:} $\hat{y}_{test}$

\end{algorithmic}
\end{algorithm}

\subsection{Uncertainty Aware Fusion (UAF)}\label{sec:uaf}
Figure \ref{fig:fuser} provides an overview of our UAF framework. At a high level, UAF consists of two modules: SELECTOR and FUSER. Given a specific task, the SELECTOR selects the top $K$ LLMs from a pool of $N$ LLMs based on performance metric. FUSER then combines the outputs of these $K$ LLMs to produce the final response.

\subsubsection{SELECTOR}
Given a pool of $N$ LLMs denoted by $\mathcal{M}$, SELECTOR selects $K$ LLMs (where $K<N$) to optimize computational efficiency and enhance overall factual accuracy by pruning underperforming LLMs. Selection is based on two criteria: (1) task-specific accuracy and (2) self-assessment of hallucinations based on an given uncertainty measure. Given a specified uncertainty measure $U_f(\cdot)$, and  a validation set $D_{val}$, we prompt each LLM with input $x_i$,   obtaining response  $\hat{y}_i^j$  and corresponding uncertainty score $u_i^j$ from the $j^{th}$ LLM $M^j$.  We compute the accuracy $Acc_j$ of $M^j$ as the fraction of correct responses. We also measure the LLMs ability for  self-assessment of hallucinations $SAH_j$ as the area under the ROC curve for the binary classification of truthful vs. hallucinatory responses using uncertainty scores. We then compute a combined score $Cscore_j = Acc_j \times SAH_j$ for each LLM. The top K models with the highest combined scores are selected greedily, where K is a   hyperparameter tuned for specific tasks.

\begin{comment}
To achieve this we sample $10\%$ of examples from $D$ and denote it as  $D_{val}$. We use the rest denoted by $D_{test}$ for evaluation. For each $(xval_i,yval_i)\in D_{val}$ we prompt each of the $N$ LLMs with $xval_i$. Let $\hat{yval}_i^j$ denote the response and $u_i^j$ its corresponding uncertainty score computed with a particular uncertainty method for $j^{th}$ LLM. Accuracy of $j^{th}$ LLM, denoted by $Acc_j$ is defined as percentage of correct responses. We also measure area under the receiver operator characteristic curve  $Unc\_auroc_j$ of $j^{th}$ LLM, by measuring the performance of classifying its own correct(truthful) from incorrect(hallucinatory) responses by varying the thresholds on the set of uncertainty scores $\{u_i^j\}_{i=1}^{nval}$. For each $j^{th}$ LLM we compute a combined score $Cscore_j = Acc_j*Unc\_auroc_j$. We use greedy method to select $K$ LLMs based on top $K$ highest $Cscore$. Here $K$ is the hyperparameter which we tune using $D_{val}$. Algorithm \ref{alg:selector} presents the pseudo-code for this method.
\end{comment}

\subsubsection{FUSER}
Given the selected ensemble of $K$ models $\mathcal{M}^{sel}$, 
%with respective accuracies $\{Acc_1, \dots, Acc_K \}$,
for each unseen example $x_{test}$, we generate outputs from the $K$ LLMs denoted by $\{\hat{y}_{test}^1,\dots,\hat{y}_{test}^K\}$ along with the  corresponding instance-specific uncertainty scores.  denoted by $\{u_{test}^1,\dots,u_{test}^K\}$. 
While there can be several fusion strategies, since we are dealing with natural language responses, the simplest one is to example-specific selection from the candidate outputs, i.e., 
\[
\hat{y}_{test} = \hat{y}_{test}^{k^*}, \quad \text{where} \quad {k^*} = \arg\max_{k \in \{1, \dots, K\}} f^k.
\]
Selection criterion $f^k$ could be based on validation set accuracy alone, inverse uncertainty or some combination of both such as $\text{Acc}_k \cdot (1 - u_{test}^k)$  or $\frac{\text{Acc}_k}{u_{test}^k}$.
The first strategy essentially reduces the ensemble to a single most accurate model, while the second one elevates the most confident one. However, both of these approaches are sub-optimal compared to combined criteria, specifically 
\[
f^k = \text{Acc}_k \cdot (1 - u_{test}^k),
\]
which yields the best performance. Experiments with  other combined selection criteria shows similar behavior to the aforementioned one and hence,  we omit the results for brevity.


%There are multiple ways to choose the final answer from these candidate responses - $\hat{y}_{test} = \hat{y}_{test}^j where j = argmax_{k \in \{1,\dots K\}}f^k$. We can define $f^k$ as $Acc_k$ , $1-u_{test}^k$ ,$Acc_k*(1-u_{test}^k)$ or $Acc_k/u_{test}^k$. First strategy essentially reduces the ensemble to a single model having highest accuracy, while second one picks the final answer as the one with the least uncertainty. Both of these are suboptimal compared to  $f_k = Acc_k*(1-u_{test}^k)$ with which we experiment here in this work. Alternative strategy $f_k = Acc_k/u_{test}^k$ shows similar behaviour as above and we omit it's analysis for brevity.

%Although there are multiple ways to aggregate these candidate responses we propose a simple aggregation technique where we choose the final answer as the one with the least uncertainty ie. $\hat{y}_{test} = \hat{y}_{test}^j where j = argmin_{k \in \{1,\dots K\}}u_{test}^k$. Algorithm \ref{alg:selector} presents the pseudo-code of UAF components.



We provide supplementary information on promising novel technical advancements that can be explored (but are not limited to) to further improve how GenAI models can align with standards and cater to their inherent complexities as discussed in Section~\ref{sec:challenges}. Some of these approaches may have been preliminarily explored by recent works for selected domains with available machine-readable standards data. Collaborations between AI and interdisciplinary areas can foster further advancements and open more novel approaches toward standard alignment.

\textbf{Constraint and Knowledge Representations}
Specifications from standards can be represented as a form of a structured set of constraints either as input or as part of an embedded training regime for aligning GenAI models. The simplest example of this is through in-context learning (ICL), where constraints are framed as prompts in an instruction-like manner with specific informative examples provided to show the target output required from the model
\cite{brown2020language,mishra-etal-2022-reframing}. This has been done on works such as the production of high-quality standard-conforming educational content with CEFR and Bloom's Taxonomy \cite{imperial-etal-2024-standardize,malik-etal-2024-tarzan,elkins2024teachers} and rewriting complex texts to conform to government-mandated plain languages guidelines \cite{da2022redactor,joseph-etal-2024-factpico}. The advantage of ICL is its simplicity, which can easily be explored by non-technical domain users with any arbitrary GenAI-based chat interface (e.g., ChatGPT) and can be attributed to how the paradigm shift started (see Section~\ref{sec:paradigm_shift}). More advanced levels of representation focus on transforming standards into knowledge graphs and ontologies. An example of this is the work by \citet{hernandez2024open}, where they transformed the text content of the EU AI Act into a high-level knowledge graph to show links between defined terms and their associated requirements from the Act's statements for compliance checking.

\textbf{Post-Training Improvements}
Post-training has become one of the major foci in ML research since the release of preference optimization techniques like RLHF \cite{ouyang2022training} and DPO \cite{rafailov2024direct} combined with supervised finetuning (SFT) to improve the response alignment of GenAI models with respect to task requirements. All Baseline models in Table~\ref{tab:model_classification} have undergone post-training through SFT. For standard alignment, post-processing can potentially be emulated by aggregating prompt and response pairs that conform to the specifications of a standard and then finetuning a pretrained model with this data. An example reference for this is OpenAI's Deliberative Alignment method \cite{guan2024deliberative} where an LLM is trained with chain-of-thought (CoT) style prompts \cite{wei2022chain} can identify whether which safety policy specification is applicable when identifying whether to respond to a prompt or not. However, for researchers who want to explore this approach, one caveat is that it will require at least 1,000 instances of very high-quality, expert-level pairwise data to achieve relatively decent performance \cite{zhou2023less}. Nonetheless, combinations of post-training techniques (e.g., SFT + DPO with standard-aligned preference pairs + CoT prompting) are viable approaches for improving a GenAI model's standard compliance capabilities.

\textbf{Synthetic Data Generation}
Synthetic data generation using modern GenAI models that have undergone processes such as larger scaling, instruction-tuning, and preference optimization often outperform other data augmentation techniques for downstream tasks \cite{ye-etal-2022-zerogen,li-etal-2023-synthetic}. In standard alignment of GenAI models, researchers can explore several options related to synthetic data generation to improve compliance capabilities. First, in parallel with post-training enhancements from above, using synthetic data in the form of standard compliant and non-compliant examples can be a practical choice to optimize a model's generation qualities. This approach has been applied by \citet{fan-etal-2024-goldcoin}, where they generated synthetic case scenarios for GDPR and HIPAA Privacy Rules to finetune smaller LLMs for compliance detection. Moreover, recent works have documented higher performance for models that have been finetuned with a combination of high-quality expert data and machine-generated data in alignment tasks which makes the process of compiling standard-aligned feedback data relatively easier \cite{miranda2024hybrid,ivison2024unpacking,lee2024rlaif}.

\textbf{Retrieval and Tool Augmentation}
Augmenting GenAI models, particularly LLMs, with external tools to enhance their problem-solving capabilities has gained increasing research attention in recent years. The use of tools such as calculators, search engines, and API function calls has been shown to improve the zero-shot performance of LLMs across question-answering downstream tasks requiring up-to-date information \cite{hao2023toolkengpt,schick2023toolformer}. In the case of standard alignment, syntactic content-based standards such as the CEFR and CCS standards (see Table~\ref{tab:standards_classification} for reference) requiring specific characteristics of texts such as sentence lengths to measure complexity can greatly benefit from a GenAI model that knows how to call a calculator tool to approximate how long sentences should be generated. Moreover, a search engine tool can also help GenAI models access updated versions of standard specifications from its original web sources as prior version checking. On the other hand, another approach that can improve GenAI models' domain knowledge is to encapsulate it in a retrieval-augmented generation (RAG) ecosystem where auxiliary retrievers that have access to external knowledge bases that can be added to as context to prompts \cite{lewis2020retrieval,ram-etal-2023-context}. 

\textbf{Reasoning Capabilities}
Whether GenAI models, such as LLMs, can reason or not is a highly debated topic in current ML research. Reasoning is an inherent ability that plays a crucial role in how humans solve problems through critical thinking \cite{huang-chang-2023-towards}. Previous research often claims that such capability can be triggered in different ways, such as providing intermediary reasoning steps to prompts \cite{wei2022chain} or using arbitrary models to select and infer reasoning steps from context information \cite{creswell2022faithful}. Assuming GenAI models can actually reason, in standard alignment, such skill may play an important role in deciding which specifications of a standard are required to be followed and which ones can be disregarded safely, given the additional context information of a task. Preliminary work in this direction includes OpenAI's Deliberative Alignment method \cite{guan2024deliberative} where an LLM is optimized to reason over which safety policy specifications should be followed and which can be ignored using their o-series models. As such, GenAI models rated Advanced and Adaptive in \textsc{C3F} should document convincing reasoning capabilities across applicable tasks.


\section{Full Assessment Results of GenAI Models and Standards with \textsc{C3F}}
\label{app:framework_tables}

As discussed in Section~\ref{sec:framework}, we provide the full list of the 15 selected foundational and domain-finetuned GenAI models assessed based on their compliance capabilities and the 34 selected standards across multiple disciplines for their criticality levels.

%%
%% TABLE - COMPLIANCE CAPABILITIES 
%% 
\onecolumn

% ACCESS
\definecolor{lightgray}{RGB}{217, 215, 213}  % Subscription
\definecolor{lightblue}{RGB}{230, 235, 245}     % Open Weight
\definecolor{lightpurple}{RGB}{240, 230, 245}   % Open Code

% Create commands for the styled text boxes
\newcommand{\subscription}{Subscription}
\newcommand{\openweight}{Open Weight}
\newcommand{\opencode}{Open Code}

\renewcommand{\arraystretch}{1.5}
\begin{table}[!t]
\small
\centering
\begin{tabular}{@{}
  >{\raggedright\arraybackslash}p{3.3cm}
  >{\raggedright\arraybackslash}p{2.5cm}
  >{\raggedright\arraybackslash}p{2.3cm}
  >{\raggedright\arraybackslash}p{1.8cm}
  >{\centering\arraybackslash}p{2cm}
  @{}}
\toprule
\textbf{\textsc{Model}} & 
\textbf{\textsc{Organization}} & 
\textbf{\textsc{Domain}} & 
\textbf{\textsc{Openness}} & 
\textbf{\textsc{Compliance}} \\ \midrule
\textsc{o-series}                & OpenAI                & General                 & \subscription              & \advanced    \\
\textsc{GPT-4}             & OpenAI                & General                 & \subscription              & \specialized    \\
\textsc{DeepSeek-R1}                & DeepSeek-AI                & General                 & \openweight              & \baseline    \\
\textsc{Claude Opus}       & Anthropic             & General                 & \subscription              & \baseline    \\
\textsc{Gemini 2.0}        & Google                & General                 & \subscription              & \baseline    \\
\textsc{Llama 3.1 405B}    & Meta                  & General                 & \openweight               & \baseline    \\
\textsc{Mistral Large}     & Mistral               & General                 & \subscription & \baseline    \\
\textsc{Command-R 105B}    & Cohere                & General                 & \openweight & \baseline    \\
\textsc{Midjourney 6.1}    & Midjourney                & General                 & \subscription & \baseline    \\
\textsc{DALL-E}    & OpenAI                & General                 & \subscription & \baseline    \\\midrule
\textsc{Meditron 70B}      & \citet{chen2023meditron}     & Healthcare              & \opencode                 & \specialized \\
\textsc{GoldCoin-Llama}    & \citet{fan-etal-2024-goldcoin}      & Healthcare, Legal       & \opencode                 & \specialized \\
\textsc{LegiLM}            & \citet{zhu2024legilm}      & Legal                   & \opencode                 & \specialized \\
\textsc{ChemCrow}          & \citet{m2024augmenting}     & Chemistry               & \opencode                 & \specialized \\
\textsc{Standardize-Llama} & \citet{imperial-etal-2024-standardize} & Education               & \opencode                 & \specialized \\ 

\bottomrule
\end{tabular}
\vspace{10pt}
\caption{We used the \textbf{compliance capabilities} component of \textsc{C3F} in Figure~\ref{fig:framework} to assess the latest versions of 15 foundational and specialized GenAI models both for text and image. We also include information on their respective domains (in the case of Specialized models) and accessibility (Open Weight, Open Code, or Subscription). The first section of the table includes industry-released GenAI models (mostly Baseline except for o-series models by OpenAI) while the second section is more focused on Specialized models commonly led by academic and research groups targeting specific domains and sectors. Models and their compliance capabilities in this table serve as a non-exhaustive example and can be extended or re-assessed as supporting literature are released.}
\label{tab:model_classification}
\end{table}



%%
%% TABLE - CRITICALITY LEVEL
%%

\small
\renewcommand{\arraystretch}{1.5}

\begin{longtable}{@{}
  >{\raggedright\arraybackslash}p{4cm}
  >{\raggedright\arraybackslash}p{2cm}
  >{\raggedright\arraybackslash}p{2.8cm}
  >{\raggedright\arraybackslash}p{1.5cm}
  >{\centering\arraybackslash}p{1.5cm}
  @{\hspace{10pt}}@{}}

\toprule
\textbf{\textsc{Standard}} &
  \textbf{\textsc{Domain}} &
  \textbf{\textsc{Organization}} &
  \textbf{\textsc{Openness}} &
  \textbf{\textsc{Criticality}} \\ \midrule
\endfirsthead

\toprule
\textbf{\textsc{Standard}} &
  \textbf{\textsc{Domain}} &
  \textbf{\textsc{Organization}} &
  \textbf{\textsc{Openness}} &
  \textbf{\textsc{Criticality}} \\ \midrule
\endhead

\bottomrule
\multicolumn{5}{r}{\textit{Continued on next page}} \\
\endfoot
\bottomrule
\caption{We used the \textbf{criticality levels} component of \textsc{C3F} in Figure~\ref{fig:framework} to classify a wide variety of standards across domains and sectors based on their sensitivity which translates into the margin of permissible errors a hypothetical GenAI model can potentially commit when assisting with standard compliance tasks. We include supporting information including the organization or initiative which led to the development of the standard as well as its accessibility (Public or Subscription). For standards classified from the domains of healthcare and engineering, we obtained direct recommended assessments from expert practitioners and professionals with respect to the criticality classification criteria in \textsc{C3F}.}\label{tab:standards_classification} 
\endlastfoot

% Table content
AILuminate Assessing Safety Standard &
  Software and Technology &
  MLCommons &
  Public &
  \minimal \\
Simplified Molecular Input Line Entry System (SMILES) &
  Chemistry &
  David Weininger &
  Public &
  \minimal \\
Associated Press Stylebook &
  Media and Communications &
  Associated Press &
  Subscription &
  \minimal \\
Plain Language Standards &
  Media and Communications &
  (region-specific) &
  Public &
  \minimal \\
Text Encoding Initiative &
  Software and Technology &
  TEI Consortium &
  Public &
  \minimal \\
ISO/IEC Systems and Software Engineering Documentation &
  Software and Technology &
  International Electrotechnical Commission &
  Subscription &
  \minimal \\
FAIR Data Principles Documentation &
  Software and Technology &
  GO FAIR Initiative &
  Public &
  \minimal \\
OpenAPI Specification (OAS) &
  Software and Technology &
  SmartBear Software &
  Public &
  \minimal \\
Section 508 Compliance Documentation &
  Software and Technology &
  US Government &
  Public &
  \minimal \\
Python Enhancement Proposals &
  Software and Technology &
  Python &
  Public &
  \minimal \\
Web Standards &
  Software and Technology &
  W3C &
  Public &
  \minimal \\
ASD-STE100 Simplified Technical English (STE) &
  Engineering &
  Aerospace, Security and Defence Industries Association of Europe &
  Public &
  \moderate \\
Common European Framework of Reference for Languages (CEFR) &
  Education &
  Council of Europe &
  Public &
  \moderate \\
Common Core Standards (CCS) &
  Education &
  National Governors Association, Council of Chief State School Officers &
  Public &
  \moderate \\
Preferred Reporting Items for Systematic Reviews and Meta-Analyses (PRISMA) &
  Healthcare &
  PRISMA Initiative &
  Public &
  \moderate \\
Standards for Quality Improvement Reporting Excellence (SQUIRE) &
  Healthcare &
  SQUIRE Initiative &
  Public &
  \moderate \\
IFRS Accounting Standards &
  Finance &
  International Financial Reports Standards &
  Public &
  \moderate \\
Systematized Nomenclature of Medicine Clinical Terms (SNOMED CT) &
  Healthcare &
  National Health Service &
  Public &
  \high \\
NHS Health Content Standards &
  Healthcare, Government &
  National Health Service &
  Public &
  \high \\
HIPAA Privacy Rule &
  Healthcare, Government &
  US Government &
  Public &
  \high \\
GDPR Documentation Standard &
  Legal, Government &
  European Union &
  Public &
  \high \\
International Classification of Diseases (ICD) Standard &
  Healthcare, Government &
  Centers for Disease Control and Prevention &
  Public &
  \high \\
Strengthening the Reporting of Observational Studies in Epidemiology (STROBE) Guidelines &
  Healthcare &
  STROBE Initiative &
  Public &
  \high \\
Case Report Guidelines (CARE) &
  Healthcare &
  CARE Initiative &
  Public &
  \high \\
Standard Protocol Items: Recommendations for Interventional Trials (SPIRIT) &
  Healthcare &
  SPIRIT Initiative &
  Public &
  \high \\
Digital Imaging and Communications in Medicine (DICOM) &
  Software and Technology, Healthcare &
  National Electrical Manufacturers Association &
  Public &
  \high \\
USDA Food Safety Documentation &
  Healthcare, Government &
  US Government &
  Public &
  \high \\
AGREE Reporting Checklist &
  Healthcare &
  International Appraisal of Guidelines, Research and Evaluation (AGREE) &
  Public &
  \high \\
NICE Process and Methods &
  Healthcare &
  National Institute for Health and Care Excellence &
  Public &
  \high \\
Operational Design Domains &
  Engineering &
  (company-specific) &
  Public &
  \high \\
BBC Content Standards &
  Media and Communications &
  Office of Communications &
  Public &
  \high \\
Consolidated Standards of Reporting Trials (CONSORT) &
  Healthcare &
  CONSORT Group &
  Public &
  \high \\
IAEA Safety Standards &
  CBRN &
  International Atomic Energy Agency &
  Public &
  \extreme \\
Responding To A CBRN Event: Joint Operating Principles for the Emergency Services &
  CBRN &
  Joint Emergency Services Interoperability Programme &
  Public &
  \extreme 
\\      
% Add more rows here as needed

\end{longtable}
\clearpage 
\twocolumn

