
%%
%% FIGURE 1 
%% 

\begin{figure}[!t]
    \centering
    \includegraphics[width=0.70\linewidth, trim=0 7 0 0, clip]{fig/fig1.pdf}
    %\vspace{-0.8em}
    \caption{We describe an emerging \textbf{paradigm shift} where domain experts from interdisciplinary areas such as education, engineering, and healthcare are using advanced generative AI models (e.g., GPT-4) to assist them with regulatory and operational compliance through standards. We pattern the temporal observation of the paradigm shift within the near-to-midterm realized capabilities of GenAI as described in \citet{pmlr-v235-eiras24b}.}
    \label{fig:paradigm-shift}
    %\vspace{-0.60cm}
\end{figure}

%% Paragraph - Broad concepts, rising trend in GenAI across domains

The industries of the modern world rely on systematic processes for the efficient production of goods and delivering services guided through \textbf{standards}. According to the International Organization for Standardization (ISO)\footnote{\url{https://www.iso.org/standards.html}}, standards refer to a general form of documented specifications, rules, and norms specialized across various domains and sectors such as healthcare, education, engineering, science, communication, security and defense. 
For example, in the aerospace engineering domain, any technical instruction manual produced must conform to recognized technical language standards such as ASD-STE100 Simplified Technical English (STE)\footnote{\url{https://www.asd-ste100.org/}} developed and maintained by the Aerospace, Security and Defence Industry Association of Europe (ASD, formerly AECMA). In the education and language proficiency assessment domain, on the other hand, teachers in charge of developing curriculum materials must follow education standard frameworks such as the Common European Framework of Reference for Languages (CEFR)\footnote{\url{https://www.coe.int/en/web/common-european-framework-reference-languages}} in Europe or the Common Core State Standards (CCS)\footnote{\url{https://www.thecorestandards.org/read-the-standards/}} in North America to produce high-quality classroom content.

Standards are typically composed of carefully defined technical \textit{specifications} for measurements, design, and performance quotas that can be used to check or validate regulatory and operational compliance while preserving interoperability, quality, and accuracy \cite{astm2025}. In recent years, there has been observable interest from users across various domains in adopting these instruction-following generative AI (GenAI) technologies, such as ChatGPT, due to their documented capabilities, including the ability to follow complex instructions and generate human-like writing. For example, a survey by the Department of Education revealed that 62\% of primary and secondary teachers in the UK have reported using GenAI tools to create new educational content and lessons for classroom use\footnote{n = 230. Data gathering was conducted around August and September 2023 with respondents from 23 educational settings.} \cite{DfE2024GenAI}. Complementary to this, recent empirical works on benchmarking GenAI models for automatic content generation using CEFR and CCS standards as references for control show that off-the-shelf commercial and open-weight LLMs such as Llama2 \cite{touvron2023llama} and GPT-4 \cite{achiam2023gpt} can be systematically steered to produce high-quality content through methods such as in-context learning (ICL) and reinforcement learning (RL) while preserving high automatic and human expert evaluations \cite{imperial-etal-2024-standardize, malik-etal-2024-tarzan}. Thus, this promising research direction of aligning GenAI models with standards underscores the need for greater attention from both the AI and interdisciplinary research communities to examine how GenAI is transforming regulatory and operational compliance across standard-driven domains.

In this paper, we analyze the changing landscape---an emerging \textit{paradigm shift}---of regulatory and operational compliance through standards across various sectors and domains. We propose a joint \textbf{\textsc{Criticality and Compliance Capabilities Framework (C3F)}} for assessing the measured capabilities of 15 recent and community-recognized foundational and specialized GenAI models for standard-compliance tasks and the criticality levels of 34 standards from various domains based on their sensitivity and potential consequences in case of non-compliance. We cover a variety of case studies supporting the paradigm shift, including healthcare, education, safety, finance, and engineering. We outline possible challenges and opportunities with learning standard compliance using GenAI models, the benefits if done successfully, and recommendations for various stakeholders involved. Finally, we take the following position that \textbf{aligning GenAI with standards through computational methods can help strengthen regulatory and operational compliance}. Thus, leading to enhanced control, oversight, and trust among these systems in real-world settings.

