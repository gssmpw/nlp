
To set the stage, we discuss preliminary information on the common definitions and processes comprising the development of standards, including common characteristics and entities involved in their conception and development.

\subsection{Defining A Standard}
Standards are established to provide specifications for measuring quality and proof of compliance with regulations. In line with this, we consider standards as \textit{expert-defined} documents since one typically needs substantial knowledge within a domain or sector to propose measurable requirements and regulatory orders for compliance from users, industries, and organizations \cite{demortain2008standardising}. The typology of standards can be generalized into either \textit{product-based} standards that specify target characteristics for physical or digital products or \textit{non-product} standards that govern and specify target efficiency, operation, and performance-based measures for processes and services \cite{nsf2018}.

To cater to the diverse overlapping notions of standards, in this paper, we do not restrict the scope of standards to those created by international and regional standard developing organizations (SDOs) such as ISO, IEEE, ETSI, CEN-CENELEC, or NIST. We also include formally documented rules coined under related terms like \textit{frameworks}, \textit{guidelines}, and \textit{checklists}, which are often used in interdisciplinary domains as forms of standards themselves since they observe objectively similar nature and usage. In summary, to unify the overlapping characteristics, a documented set of guidelines can be considered a standard if it conforms to the following below:

\textbf{Anatomy}: Standards are composed of well-defined specifications and procedures documenting measurable requirements for a given product or process.

\textbf{Purpose}: Standards serve specific purposes for various domains and sectors, such as introducing a formal language of communication, defining recognized procedures, ensuring compatibility checks, specifying performance requirements, and assuring compliance with regulations.

\textbf{Recognition}: Standards are developed and recognized by members and constituents of a private or public domain, sector, organization, or regulatory body.

% We see a parallel between process evaluations for AI and cases such as pharmaceuticals and nuclear power, where the entire design process must be overseen. In all of these industries, it is impossible to tell just from looking at the final result whether it is safe and effective (Mannheim et al - The Necessity of AI Audit Standards Boards)

\subsection{Standards as Co-Regulation and Co-Integration Tools}
A standard may be developed publicly or privately through initiatives by the government or regulatory bodies, unions, organizations, and expert groups. In legislation, a standard may be paired with specific laws as a form of a \textbf{co-regulation tool} which, if successfully fulfilled, can serve as a \textit{form of compliance with related state or nation-wide jurisdictions} \cite{pouget2024ai}. In this case, a regulatory body may appoint one or more SDOs to initiate the workflow of creating a specific standard that contains the legal requirements that must be included in line with the law. An example of these standards includes the well-known ISO/IEC 27001:2022 which defines measures for Personally Identifiable Information (PII) controllers and processors of any information security management system (ISMS) in response to the EU General Data Protection Regulation (GDPR) for data privacy and protection \cite{iso27001_2022}.
%In a more recent example, the European Commission has tasked and endorsed the European Committee for Standardization (CEN) and the European Committee for Electrotechnical Standardization (CENELEC) to develop standards in accordance with the EU AI Act, which will then serve as a \textit{presumption of conformity} for AI systems \cite{pouget2024ai,cencenelec2024}.

On the other hand, standards developed through non-legislative initiatives can serve as a \textbf{co-integration tool} which focuses on organization- or community-wide \textit{interoperability and harmonization of systems and processes}. A well-known example is the Web Standards developed by the World Wide Web Consortium (W3C)\footnote{\url{https://www.w3.org/standards/}} which maintains all technical specifications, guidelines, and protocols for web-based technologies including HTML, CSS, and XML. These standards and technologies are globally recognized and form the core building blocks of the Web or Internet. 

%In the same vein, the Guidelines for Electronic Text Encoding and Interchange are maintained by the Text Encoding Initiative (TEI) Consortium\footnote{\url{https://tei-c.org/}} which is extensively used in digital humanities, particularly in supporting infrastructures such as digital encoding for preserving machine-readable cultural heritage texts. 
%This interdisciplinary text encoding standard is widely used by a global community of people working in libraries, publication industries, museums, and academia.


