\section{Pseudo Code of \RPC Method}
\label{sec:appendix-rpc}

In this section, we provide the pseudo code of \RPC. 
The output of Algorithm~\ref{alg:rpc} is a set of reasoning paths with the highest confidence.
The extraction function $g(\cdot)$ can be used to transform the reasoning paths to answers.

\renewcommand{\algorithmicrequire}{\textbf{Input:}}
\renewcommand{\algorithmicensure}{\textbf{Output:}}
\begin{algorithm}[ht]
    \caption{Reasoning-pruning Perplexity Consistency}
    \label{alg:rpc}
    \begin{algorithmic}[1]
    \REQUIRE 
        \STATE Sampled Reasoning paths $\tilde{t}_1, \ldots, \tilde{t}_n$ 
        \STATE LLM Internal Probabilities $p_1, \ldots, p_n$
        \STATE Consisitency function $\mathbb{I}_C(\cdot, \cdot)$
    \ENSURE Most-confident reasoning paths with probabilities
    
    \STATE \COMMENT{\textbf{\underline{Phase 1}: Reasoning Pruning}}
    \STATE $(k_1, \lambda_1, k_2, \lambda_2, w_1, w_2) \gets$ Model probability distribution parameters from $p_1, \ldots, p_n$ \COMMENT{Using \autoref{eq:weibull-mix}}
    \STATE $p_{\text{mean}} \gets \frac{1}{n}\sum_{i=1}^n p_i$
    \STATE $\mathrm{I}_{\text{retain}} \gets \{i \mid P_{\text{High}}(p_i) > 0.5 \text{ or } p_i \geq p_{\text{mean}}\}$ \COMMENT{$P_{\text{High}}$ is defined in \autoref{eq:weibull-prob}}
    \STATE \COMMENT{\textbf{\underline{Phase 2}: Perplexity Consistency}}
    \STATE $\mathrm{U} \gets \text{Set}(\tilde{t}_1, \ldots, \tilde{t}_n)$ 
    \STATE $\mathrm{I}_{\text{unique}} \gets \{i \mid \tilde{t}_i \in \mathrm{U} \text{ and } i \in \mathrm{I}_{\text{retain}}\}$ 
    \FOR{each reasoning path $\tilde{t} \in \mathrm{U}$}
        \STATE $\mathrm{C}(\tilde{t}) \gets \sum_{i \in \mathcal{I}_{\text{retain}}} \mathbb{I}_C[\tilde{t}, \tilde{t}_i] p_i$
    \ENDFOR

    \STATE $\text{C}_{\text{max}} \gets \max_{\tilde{t} \in \mathrm{U}} \mathrm{C}(\tilde{t})$
    \STATE {\bfseries return} $\{(\tilde{t}, \mathrm{C}(\tilde{t})) \mid \tilde{t} \in \mathrm{U}, \text{C}(\tilde{t}) = \mathrm{C}_{\text{max}}\}$
 \end{algorithmic}
\end{algorithm}

% \section{Details of \PC and \RPC methods}

% \algrenewcommand\algorithmicrequire{\textbf{Input:}}
% \algrenewcommand\algorithmicensure{\textbf{Output:}}

% \begin{algorithm}
% \caption{The pseudo-code of \PC approach. }\label{alg:PC}
% \begin{algorithmic}
% \Require Response and corresponding probability pairs $ \mathbf{R} = \left \{ (\mathrm{resp}_i, \mathrm{prob}_i) \right \}_{i=1}^n$ of reasoning paths; \\
% ~~~~~~~~~Consisitency function $\mathbb{I}_C(\cdot, \cdot)$; Answer extraction function $G(\cdot)$.
% \Ensure Answer and corresponding confidence pairs $\{ (\mathrm{ans}_i, \mathrm{conf}_i) \}_{i=1}^m$. 
% \State $\mathbf{ans} \gets \left \{G(\mathrm{resp}_i) ~|~ (\mathrm{resp}_i, \mathrm{prob}_i) \in \mathbf{R} \right \}$
% \State $m \gets |\mathbf{ans}|$
% \State $\mathbf{conf} \gets $ A $m$-sized float array initialized by 0
% \State $\mathbf{R}^{\star} \gets \mathrm{Unique}(\mathbf{R})$ 
% \For{$i = 1 \ldots |\mathbf{R}^{\star}|$}

% \State $k \gets$ Find 
% \EndFor
% % \For{$i=1 \ldots n$}
% % \State 
% % \EndFor
% \State $y \gets 1$
% \State $X \gets x$
% \State $N \gets n$
% \While{$N \neq 0$}
% \If{$N$ is even}
%     \State $X \gets X \times X$
%     \State $N \gets \frac{N}{2}$  \Comment{This is a comment}
% \ElsIf{$N$ is odd}
%     \State $y \gets y \times X$
%     \State $N \gets N - 1$
% \EndIf
% \EndWhile
% \end{algorithmic}
% \end{algorithm}
