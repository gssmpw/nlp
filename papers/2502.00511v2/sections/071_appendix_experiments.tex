\section{Detailed Experimental Results}
\label{app:detailed-results}

\subsection{Performance on GSM8k dataset.} 
In Section~\ref{sec:problem}, we analyze the GSM8k dataset, which is relatively easy that allows accurate estimation of ground-truth probabilities with 100 samples. Due to this characteristic, we exclude the GSM8k dataset from our main experiments, as a limited number of samples is sufficient for accurate confidence evaluation.
\autoref{tab:gsm8k} shows the performance of InternLM-2-MATH-Plus 7B model on GSM8k with various sampling temperatures, where the number of samples is set to $n=6$ to maintain consistency with \autoref{fig:motivation-estimation-error}. The results show that the \RPC method outperforms comparison methods.

\begin{table}[ht]
    \centering
    \caption{Performance on GSM8k datasets using InternLM-2-MATH-Plus 7B model with diverse sampling temperatures when sampling size $n=6$. The best performance is highlighted in \textbf{bold}. The results show that the \RPC approach outperforms the \SC method. }
    \vskip 0.15in
    \begin{center}
    \begin{small}
    \begin{sc}
    \label{tab:gsm8k}
    \begin{tabular}{l|cccc}
    \bottomrule
    \toprule
     & \PP & \Verb & \SC & \RPC  \\
    \midrule
    T=1.0 & 86.97 $\pm$ 0.29 & 63.67 $\pm$ 0.98 & 89.32 $\pm$ 0.26 & \textbf{89.45 $\pm$ 0.38} \\
    T=1.1 & 86.78 $\pm$ 0.43 & 62.25 $\pm$ 1.00 & 89.44 $\pm$ 0.35 & \textbf{89.51 $\pm$ 0.36} \\
    T=1.3 & 86.65 $\pm$ 0.57 & 61.29 $\pm$ 0.84 & 88.92 $\pm$ 0.32 & \textbf{89.08 $\pm$ 0.57} \\
    \bottomrule
    \toprule
    \end{tabular}
    \end{sc}
    \end{small}
    \end{center}
    \vskip -0.1in
\end{table}

\subsection{Results with High Sampling Temperature. }

Using a high sampling temperature enables language models to produce more diverse outputs, potentially enhancing reasoning performance. 
However, it also leads to an increase in estimation error. 
To investigate the effectiveness of our approaches in addressing the estimation error issue, we conducted experiments with higher sampling temperatures (T = 1.1 and T = 1.3) using the InternLM-2-MATH-Plus 7B model. The results in \autoref{tab:temp-performance} indicate that our \RPC approach consistently surpasses baseline methods. Notably, a significant performance gap persists between \RPC and \SC, indicating that our methods effectively tackle the estimation error issue even under high-temperature sampling conditions.

\begin{table}[t]
    \centering
    \caption{Performance Comparison of different models and different parameter scales. The accuracy is reported as the mean and stdev. The best performance is highlighted in \textbf{bold}. The results show that our \RPC approach consistently outperforms existing methods.}
    \label{tab:temp-performance}
    % \vskip 0.15in
    \begin{center}
    \begin{small}
    \begin{sc}
    % \resizebox{\linewidth}{!}{
    \begin{tabular}{l|cccc}
    \bottomrule
    \toprule
    \multirow{2}{*}{Method} & \multicolumn{4}{c}{\textbf{ Temperature = 1.1}} \\
    \cmidrule(lr){2-5}      & MATH & MathOdyssey & OlympiadBench & AIME \\
    \midrule
    \PP & 47.35 $\pm$ 0.16 & 28.59 $\pm$ 1.30 & ~~7.27 $\pm$ 0.23 & 6.02 $\pm$ 0.34 \\
    \Verb & 25.51 $\pm$ 0.23 & ~~9.41 $\pm$ 0.44 & ~~3.66 $\pm$ 0.16 & 3.07 $\pm$ 0.15 \\
    \SC & 50.66 $\pm$ 0.22 & 27.89 $\pm$ 0.43 & 10.74 $\pm$ 0.15 & 8.73 $\pm$ 0.24 \\
    \hline
    \rowcolor{gray!20} \RPC & \textbf{52.58 $\pm$ 0.14} & \textbf{32.98 $\pm$ 0.69} & \textbf{11.00 $\pm$ 0.24} & \textbf{9.30 $\pm$ 0.29} \\
    \bottomrule
    \toprule
    \multirow{2}{*}{Method} & \multicolumn{4}{c}{\textbf{Temperature = 1.3}} \\
    \cmidrule(lr){2-5}      & MATH & MathOdyssey & OlympiadBench & AIME \\
    \midrule
    \PP & 47.58 $\pm$ 0.31 & 26.38 $\pm$ 1.41 & ~~7.76 $\pm$ 0.46 & 6.50 $\pm$ 0.41 \\
    \Verb & 24.62 $\pm$ 0.33 & ~~8.60 $\pm$ 0.26 & ~~3.11 $\pm$ 0.17 & 2.29 $\pm$ 0.12 \\
    \SC & 50.65 $\pm$ 0.14 & 27.61 $\pm$ 0.67 & 10.49 $\pm$ 0.18 & 8.02 $\pm$ 0.20 \\
    \hline
    \rowcolor{gray!20} \RPC & \textbf{53.12 $\pm$ 0.14} & \textbf{33.19 $\pm$ 0.56} & \textbf{10.91 $\pm$ 0.18} & \textbf{8.83 $\pm$ 0.23} \\
    \bottomrule
    \toprule
    \end{tabular}
    \end{sc}
    \end{small}
    \end{center}
    \vskip -0.2in
\end{table}

\subsection{Performance with Diverse Number of Samplings}

In the \autoref{fig:InternLM2-7B-Accuracy}, we plot the accuracy of the InternLM-2-MATH-Plus 7B model on four mathematical reasoning datasets with different sample sizes $n$. Here, we give the detailed results on different models and different sampling temperatures.

\paragraph{Different Models Scales.} 
The performance of relateively small model, InternLM-2-MATH-Plus 1.8B, is presented in \autoref{fig:InternLM2-1_8B-Accuracy}.  Similar conclusions can be drawn from these results. For the MathOdyssey dataset, the \PP method shows superior performance compared to other methods, which can be attributed to the relatively low model error of \PP on this dataset, allowing the perplexity-based approach to function effectively. Furthermore, the \RPC method consistently outperforms the \SC method, which demonstrates its ability to enhance the convergence properties of the \SC method. 

\begin{figure}[ht]
    \vskip 0.2in
    \begin{center}
        \begin{minipage}[t]{\textwidth}
            \centering
            \begin{subfigure}[t]{0.23\textwidth}
                \centering
        \includegraphics[width=\linewidth]{figures/Samplings-I18B/MATH-Accuracy.pdf}
                \vskip -0.3em
                \caption{MATH}
                \label{fig:MATH-Accuracy}
            \end{subfigure}
            \hfill
            \begin{subfigure}[t]{0.23\textwidth}
                \centering
                \includegraphics[width=\linewidth]{figures/Samplings-I18B/Odyssey-Accuracy.pdf}
                \vskip -0.3em
                \caption{MathOdyssey}
                \label{fig:MathOdyssey-Accuracy}
            \end{subfigure}
            \hfill
            \begin{subfigure}[t]{0.23\textwidth}
                \centering
                \includegraphics[width=\linewidth]{figures/Samplings-I18B/OlympiadBench-Accuracy.pdf}
                \vskip -0.3em
                \caption{OlympiadBench}
            \end{subfigure}
            \hfill
            \begin{subfigure}[t]{0.23\textwidth}
                \centering
                \includegraphics[width=\linewidth]{figures/Samplings-I18B/AIME_1983_2024-Accuracy.pdf}
                \vskip -0.3em
                \caption{AIME}
            \end{subfigure}
            \vskip -0.5em
            \caption{The accuracy of InternLM-2-MATH-Plus 1.8B model on four mathematical reasoning datasets with different sample sizes $n$.}
            \label{fig:InternLM2-1_8B-Accuracy}
        \end{minipage}
    \end{center}
    \vskip -0.2in
\end{figure}

\paragraph{Different Sampling Temperatures.} We also evaluate the InternLM-2-MATH-Plus 7B model with sampling temperatures $T=1.1$ and $T=1.3$. The results are presented in \autoref{fig:InternLM2-7B-Accuracy-T1.1} and \autoref{fig:InternLM2-7B-Accuracy-T1.3}. 
The results demonstrate that the \RPC method effectively improves the model's reasoning performance at higher temperatures, as it leverages the increased diversity in sampling to enhance self-consistency. In contrast, the \SC method's performance deteriorates due to increased estimation errors at higher temperatures.

\begin{figure}[ht]
    \vskip 0.2in
    \begin{center}
        \begin{minipage}[t]{\textwidth}
            \centering
            \begin{subfigure}[t]{0.23\textwidth}
                \centering
        \includegraphics[width=\linewidth]{figures/Samplings/MATH-Accuracy-T1.1.pdf}
                \vskip -0.3em
                \caption{MATH}
                \label{fig:MATH-Accuracy}
            \end{subfigure}
            \hfill
            \begin{subfigure}[t]{0.23\textwidth}
                \centering
                \includegraphics[width=\linewidth]{figures/Samplings/Odyssey-Accuracy-T1.1.pdf}
                \vskip -0.3em
                \caption{MathOdyssey}
                \label{fig:MathOdyssey-Accuracy}
            \end{subfigure}
            \hfill
            \begin{subfigure}[t]{0.23\textwidth}
                \centering
                \includegraphics[width=\linewidth]{figures/Samplings/OlympiadBench-Accuracy-T1.1.pdf}
                \vskip -0.3em
                \caption{OlympiadBench}
            \end{subfigure}
            \hfill
            \begin{subfigure}[t]{0.23\textwidth}
                \centering
                \includegraphics[width=\linewidth]{figures/Samplings/AIME_1983_2024-Accuracy-T1.1.pdf}
                \vskip -0.3em
                \caption{AIME}
            \end{subfigure}
            \vskip -0.5em
            \caption{The accuracy of InternLM-2-MATH-Plus 7B model on four mathematical reasoning datasets with different sample sizes $n$. The sampling temperature is set to 1.1.}
            \label{fig:InternLM2-7B-Accuracy-T1.1}
        \end{minipage}
    \end{center}
    \vskip -0.2in
\end{figure}

\begin{figure}[ht]
    \vskip 0.2in
    \begin{center}
        \begin{minipage}[t]{\textwidth}
            \centering
            \begin{subfigure}[t]{0.23\textwidth}
                \centering
        \includegraphics[width=\linewidth]{figures/Samplings/MATH-Accuracy-T1.3.pdf}
                \vskip -0.3em
                \caption{MATH}
                \label{fig:MATH-Accuracy}
            \end{subfigure}
            \hfill
            \begin{subfigure}[t]{0.23\textwidth}
                \centering
                \includegraphics[width=\linewidth]{figures/Samplings/Odyssey-Accuracy-T1.3.pdf}
                \vskip -0.3em
                \caption{MathOdyssey}
                \label{fig:MathOdyssey-Accuracy}
            \end{subfigure}
            \hfill
            \begin{subfigure}[t]{0.23\textwidth}
                \centering
                \includegraphics[width=\linewidth]{figures/Samplings/OlympiadBench-Accuracy-T1.3.pdf}
                \vskip -0.3em
                \caption{OlympiadBench}
            \end{subfigure}
            \hfill
            \begin{subfigure}[t]{0.23\textwidth}
                \centering
                \includegraphics[width=\linewidth]{figures/Samplings/AIME_1983_2024-Accuracy-T1.3.pdf}
                \vskip -0.3em
                \caption{AIME}
            \end{subfigure}
            \vskip -0.5em
            \caption{The accuracy of InternLM-2-MATH-Plus 7B model on four mathematical reasoning datasets with different sample sizes $n$. The sampling temperature is set to 1.3.}
            \label{fig:InternLM2-7B-Accuracy-T1.3}
        \end{minipage}
    \end{center}
    \vskip -0.2in
\end{figure}
