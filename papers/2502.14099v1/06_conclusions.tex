\section{Conclusions}
\label{sec:conclusions}
This paper introduces \gls{esqh}, a novel joint resolution and quality scalability scheme implemented and validated for geometry coding in the \gls{jpeg-pcc} standard. \gls{esqh} enables the encoding of point clouds into scalable bitstreams, supporting compressed representations at various compression qualities and resolutions.
The proposed method demonstrates minimal to no rate-distortion performance loss compared to non-scalable \gls{jpeg-pcc}, indicating that the added scalability functionality incurs a negligible rate cost.

A key advantage of \gls{esqh} is its implementation in the latent space, which circumvents common drawbacks associated with spatial domain scalability, such as residual sparsity and the necessity to decode point clouds at each target quality. This approach results in limited additional complexity relative to the JPEG PCC baseline. Moreover, \gls{esqh} enhances the capabilities of \gls{sqh} by incorporating resolution scalability alongside quality scalability while reducing the required network parameters. The modular nature of \gls{esqh}, implemented through the \gls{squlpe} model, allows for flexible usage of the feature. Users can still utilize \gls{jpeg-pcc} in a non-scalable manner when scalability is not required, maintaining backward compatibility and versatility.

The latent space alignment principle underlying \gls{esqh} is readily achievable through sequential training, making this approach adaptable to other learning-based codecs.

Future research directions include integrating \gls{esqh} with JPEG-AI to enable attribute domain scalability in \gls{jpeg-pcc} and exploring additional applications of latent alignment beyond scalability. Preliminary findings suggest that low-quality latents may serve as superior side information compared to hyper-latents in certain scenarios, potentially leading to improvements in current entropy models. Additionally, extending support for arbitrary block sizes across various enhancement layers would enhance the algorithm's adaptability in practical applications.
