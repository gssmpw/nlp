\documentclass{ieeeaccess}
\usepackage{cite}
\usepackage{amsmath,amssymb,amsfonts}
\usepackage{algorithmic}
\usepackage{graphicx}
\usepackage{textcomp}
\usepackage{glossaries}
\usepackage{bm}
\usepackage{enumitem}
\usepackage{subcaption}
\usepackage{multirow}
\usepackage{float}
\makeatletter
\AtBeginDocument{\DeclareMathVersion{bold}
\SetSymbolFont{operators}{bold}{T1}{times}{b}{n}
\SetSymbolFont{NewLetters}{bold}{T1}{times}{b}{it}
\SetMathAlphabet{\mathrm}{bold}{T1}{times}{b}{n}
\SetMathAlphabet{\mathit}{bold}{T1}{times}{b}{it}
\SetMathAlphabet{\mathbf}{bold}{T1}{times}{b}{n}
\SetMathAlphabet{\mathtt}{bold}{OT1}{pcr}{b}{n}
\SetSymbolFont{symbols}{bold}{OMS}{cmsy}{b}{n}
\renewcommand\boldmath{\@nomath\boldmath\mathversion{bold}}}
\makeatother

\def\BibTeX{{\rm B\kern-.05em{\sc i\kern-.025em b}\kern-.08em
    T\kern-.1667em\lower.7ex\hbox{E}\kern-.125emX}}

% \def\x{{\boldsymbol x}}
% \def\y{{\boldsymbol y}}
% \def\z{{\boldsymbol z}}
% \def\c{{\boldsymbol c}}
% \def\K{{\boldsymbol \kappa}}
\newcommand{\x}{\ifmmode\bm{x}\else\textbf{x}\fi}
\newcommand{\y}{\ifmmode\bm{y}\else\textbf{y}\fi}
\newcommand{\z}{\ifmmode\bm{z}\else\textbf{z}\fi}
\newcommand{\C}{\ifmmode\bm{c}\else\textbf{c}\fi}
\newcommand{\K}{\ifmmode\bm{K}\else\textbf{K}\fi}


%Your document starts from here ___________________________________________________
\begin{document}
\newglossaryentry{AI}{
    name=AI: Artificial Intelligence,
    description={The simulation of human intelligence}
}

\newglossaryentry{LLM}{
    name=LLM: Large Language Model,
    description={A type of artificial intelligence designed to understand and generate human-like text}
}

\newglossaryentry{API}{
    name=API: Application Programming Interface,
    description={A software interface for offering a service to other pieces of software}
}

\newglossaryentry{AST}{
    name=AST: Abstract Syntax Tree,
    description={A tree representation of the abstract syntactic structure of source code written in a programming language}
}

\newglossaryentry{ReACC}{
    name=ReACC: Retrieval-Augmented Code Completion,
    description={A framework that enhances code completion by leveraging external context from a large codebase}
}

\newglossaryentry{CMSIS}{
    name=CMSIS: Cortex Microcontroller Software Interface Standard,
    description={A hardware abstraction layer independent of vendor for the Cortex-M processor series}
}

\newglossaryentry{HAL}{
    name=HAL: Hardware Abstraction Layer,
    description={A layer of programming that allows a computer operating system to interact with a hardware device at an abstract level}
}

\newglossaryentry{RAG}{
    name=RAG: Retrieval-Augmented Generation,
    description={A process that enhances large language models by allowing them to respond to prompts using a specified set of documents}
}

\newglossaryentry{STM32F407}{
    name=STM32F407: High-\allowbreak performance Microcontroller,
    description={A microcontroller that offers the performance of the Cortex-M4 core}
}

\newglossaryentry{AURIX TC334}{
    name=AURIX TC334: 32-bit Microcontroller from Infineon,
    description={A microcontroller designed for automotive and industrial applications, featuring a 32-bit TriCore-\allowbreak architecture}
}

\newglossaryentry{LED}{
    name=LED: Light-Emitting Diode,
    description={A semiconductor light source that emits light when current flows through it}
}

\newglossaryentry{CortexM4}{
    name=Cortex-M4: 32-bit processor design from ARM,
    description={A 32-bit processor design optimized for real-time applications with low power consumption}
}

\newglossaryentry{GPIO}{
    name=GPIO: General-Purpose Input/Output Pin,
    description={A versatile pin on a microcontroller that can be configured as either an input or an output. As an \textbf{input}, it can read external signals such as button presses. As an \textbf{output}, it can control devices}
}

\newglossaryentry{Offset}{
    name=Offset: Relative distance of a specific register,
    description={The relative distance or position of a specific register or memory location within a hardware block, measured from a base address}
}

\newglossaryentry{Clock}{
    name=Clock: Synchronization signal for operations,
    description={The signal used to synchronize operations within a microcontroller or hardware system, ensuring consistent timing and execution of tasks}
}

\newglossaryentry{GPT4oMini}{
    name=GPT-4o Mini: Large Language Model from OpenAI,
    description={A compact variant of the GPT-4 language model designed for cost-efficient and versatile tasks}
}

\newglossaryentry{FAISS}{
    name=FAISS: Facebook AI Similarity Search,
    description={An open-source library for efficient similarity search and clustering of high-dimensional vectors}
}



\history{Date of publication xxxx 00, 0000, date of current version xxxx 00, 0000.}
\doi{10.1109/ACCESS.2024.0429000}

\title{Point Cloud Geometry Scalable Coding Using a Resolution and Quality-conditioned Latents Probability Estimator}
\author{\uppercase{Daniele Mari}\authorrefmark{1},
\uppercase{André F. R. Guarda}\authorrefmark{2}, \IEEEmembership{Member, IEEE}, \uppercase{Nuno M. M. Rodrigues}\authorrefmark{2}\authorrefmark{4}, \IEEEmembership{Senior Member, IEEE}, \uppercase{Simone Milani}\authorrefmark{1}, \IEEEmembership{Member, IEEE}, and \uppercase{Fernando Pereira}\authorrefmark{2}\authorrefmark{3}, \IEEEmembership{Fellow, IEEE}}

\address[1]{University of Padova, Department of Information Engineering, Padova, 35131 Italy (e-mail: name.surname@dei.unipd.it)}
\address[2]{Instituto de Telecomunicações, 1049-001 Lisbon, Portugal}
\address[3]{Instituto Superior Técnico, Universidade de Lisboa, 1049-001 Lisbon, Portugal)}
\address[4]{ESTG, Politécnico de Leiria, 2411-901 Leiria, Portugal}
\tfootnote{This work was partially supported by the European Union under the Italian National Recovery and Resilience Plan (NRRP) of NextGenerationEU, with a partnership on “Telecommunications of the Future” (PE00000001 - program “RESTART”). This study was also carried out within the Future Artificial Intelligence Research (FAIR) and received funding from the European Union Next-GenerationEU (PNRR - Piano Nazionale di Ripresa e Resilienza - Missione 4, Componente 2, Investimento 1.3 - D.D. 1555 11/10/2022, PE00000013). This work was also funded by the Fundação para a Ciência e a Tecnologia (FCT, Portugal) through the research project PTDC/EEI-COM/1125/2021, entitled “Deep Learning-based Point Cloud Representation”, and by FCT/MECI through national funds and when applicable co-funded EU funds under UID/50008: Instituto de Telecomunicações. This manuscript reflects only the authors’ views and opinions, neither the EU nor the European Commission can be considered responsible for them. Daniele Mari's activities were supported by Fondazione CaRiPaRo under the grants “Dottorati di Ricerca” 2021/2022.}

\markboth
{D. Mari \headeretal: Preparation of Papers for IEEE TRANSACTIONS and JOURNALS}
{D. Mari \headeretal: Preparation of Papers for IEEE TRANSACTIONS and JOURNALS}

\corresp{Corresponding author: Daniele Mari (e-mail: daniele.mari@dei.unipd.it).}


\begin{abstract}
In the current age, users consume multimedia content in very heterogeneous scenarios in terms of network, hardware, and display capabilities. A naive solution to this problem is to encode multiple independent streams, each covering a different possible requirement for the clients, with an obvious negative impact in both storage and computational requirements. These drawbacks can be avoided by using codecs that enable scalability, i.e., the ability to generate a progressive bitstream, containing a base layer followed by multiple enhancement layers, that allow decoding the same bitstream serving multiple reconstructions and visualization specifications.
While scalable coding is a well-known and addressed feature in conventional image and video codecs, this paper focuses on a new and very different problem, notably the development of scalable coding solutions for deep learning-based \gls{pc} coding. The peculiarities of this 3D representation make it hard to implement flexible solutions that do not compromise the other functionalities of the codec.
This paper proposes a joint quality and resolution scalability scheme, named \gls{esqh}, that, contrary to previous solutions, can model the relationship between latents obtained with models trained for different RD tradeoffs and/or at different resolutions.
Experimental results obtained by integrating \gls{esqh} in the emerging JPEG Pleno learning-based PC coding standard show that \gls{esqh}  allows decoding the PC at different qualities and resolutions with a single bitstream while incurring only in a limited RD penalty and increment in complexity w.r.t. non-scalable JPEG PCC that would require one bitstream per coding configuration.
\end{abstract}

\begin{keywords}
Point cloud geometry coding, JPEG Pleno PCC, deep learning-based codec, scalable coding.  
\end{keywords}

\titlepgskip=-21pt

\maketitle

\section{Introduction}\label{sec:Intro} 


Novel view synthesis offers a fundamental approach to visualizing complex scenes by generating new perspectives from existing imagery. 
This has many potential applications, including virtual reality, movie production and architectural visualization \cite{Tewari2022NeuRendSTAR}. 
An emerging alternative to the common RGB sensors are event cameras, which are  
 bio-inspired visual sensors recording events, i.e.~asynchronous per-pixel signals of changes in brightness or color intensity. 

Event streams have very high temporal resolution and are inherently sparse, as they only happen when changes in the scene are observed. 
Due to their working principle, event cameras bring several advantages, especially in challenging cases: they excel at handling high-speed motions 
and have a substantially higher dynamic range of the supported signal measurements than conventional RGB cameras. 
Moreover, they have lower power consumption and require varied storage volumes for captured data that are often smaller than those required for synchronous RGB cameras \cite{Millerdurai_3DV2024, Gallego2022}. 

The ability to handle high-speed motions is crucial in static scenes as well,  particularly with handheld moving cameras, as it helps avoid the common problem of motion blur. It is, therefore, not surprising that event-based novel view synthesis has gained attention, although color values are not directly observed.
Notably, because of the substantial difference between the formats, RGB- and event-based approaches require fundamentally different design choices. %

The first solutions to event-based novel view synthesis introduced in the literature demonstrate promising results \cite{eventnerf, enerf} and outperform non-event-based alternatives for novel view synthesis in many challenging scenarios. 
Among them, EventNeRF \cite{eventnerf} enables novel-view synthesis in the RGB space by assuming events associated with three color channels as inputs. 
Due to its NeRF-based architecture \cite{nerf}, it can handle single objects with complete observations from roughly equal distances to the camera. 
It furthermore has limitations in training and rendering speed: 
the MLP used to represent the scene requires long training time and can only handle very limited scene extents or otherwise rendering quality will deteriorate. 
Hence, the quality of synthesized novel views will degrade for larger scenes. %

We present Event-3DGS (E-3DGS), i.e.,~a new method for novel-view synthesis from event streams using 3D Gaussians~\cite{3dgs} 
demonstrating fast reconstruction and rendering as well as handling of unbounded scenes. 
The technical contributions of this paper are as follows: 
\begin{itemize}
\item With E-3DGS, we introduce the first approach for novel view synthesis from a color event camera that combines 3D Gaussians with event-based supervision. 
\item We present frustum-based initialization, adaptive event windows, isotropic 3D Gaussian regularization and 3D camera pose refinement, and demonstrate that high-quality results can be obtained. %

\item Finally, we introduce new synthetic and real event datasets for large scenes to the community to study novel view synthesis in this new problem setting. 
\end{itemize}
Our experiments demonstrate systematically superior results compared to EventNeRF \cite{eventnerf} and other baselines. 
The source code and dataset of E-3DGS are released\footnote{\url{https://4dqv.mpi-inf.mpg.de/E3DGS/}}. 





\section{Related Works}
\label{sec:related}
State-of-the-art PCC encompasses various approaches, ranging from traditional signal processing techniques to modern learning-based solutions. Among the conventional approaches, the most relevant are G-PCC and V-PCC \cite{graziosi2020overview}, the two MPEG standards for \gls{pcc}.
G-PCC, or Geometry-based Point Cloud Compression, leverages octree representations for efficient geometry coding, and uses predictive or hierarchical transforms for attribute coding. On the other hand, V-PCC, or Video-based Point Cloud Compression, projects 3D PC data into the 2D domain, creating images that represent the geometry and texture information, which are compressed using very efficient and established video codecs like HEVC and VVC \cite{bross2021overview}.
While G-PCC inherently supports resolution scalability for both geometry and attributes, achieving scalability in V-PCC presents challenges due to its reliance on video coding frameworks. Although MPEG has initiated investigations into various scalability techniques for V-PCC \cite{vpccscal}, these features remain to be specified in the current version of the standard.

More recently, the advent of \gls{dl} has revolutionized PC compression, yielding numerous high-performing solutions \cite{guarda2024jpeg,quach2020improved,wang2021lossy, wang2022sparse, liu2022pcgformer}. Nevertheless, among the many \gls{dl}-based PCC solutions, few approaches address any form of scalability. DL-PCSC \cite{guarda2020point} implements quality scalability by channelwise partitioning of latent representations, enabling progressive quality enhancement through incremental transmission. However, this approach faces limitations: the requirement for zero-padding untransmitted latents constrains the latent space design, and the reduced latent space dimensionality at lower rates leads to reduced modeling capabilities \cite{balle2018variational}. These constraints significantly impact the rate-distortion performance making scalability less appealing.

GRASP-Net \cite{pang2022grasp} offers an alternative approach, by implementing a \gls{dl}-based enhancement layer atop a G-PCC base layer. However, scalability is limited to this two-layer structure, and the extremely low-resolution base layer may prove impractical for real-world applications.

The work by Ulhaq et al. \cite{ulhaq2024scalable} implements scalability by adapting content for human and machine consumption. In particular, the base layer can be used to solve computer vision tasks (e.g. PC classification) while the enhancement layer allows reconstructing the PC for human visualization. This approach thus addresses scalability requirements that are different from those explored in this work.

SparsePCGC \cite{wang2022sparse} and its successor, Unicorn \cite{wang2024versatile1,wang2024versatile2}, represent significant advances in resolution scalability for PCC, due to their inherently multiscale nature. At the encoder side, Unicorn employs a hierarchical downscaling approach, encoding the information necessary for losslessly upscaling it at the decoder.
When this enhancement information is unavailable, the decoder employs a lossy thresholding strategy for upscaling.
By dividing the bitstream in different enhancement layers, required to upscale the PC, Unicorn achieves resolution scalability. Furthermore, Unicorn has a very competitive coding performance when compared with other PC codecs. The intrinsic scalability mechanism of Unicorn, which results from its architecture, contrasts with \gls{esqh}, which offers a modular, plug-and-play solution applicable to various codecs.

Additionally, it is also important to mention that recently the MPEG group issued the call for proposal for the new AI-based PCC \cite{aigccfp} which is currently under development.
\section{JPEG Pleno Point Cloud Geometry Codec and Previous Extension for Quality Scalability}
\label{sec:vm}

\begin{figure*}
    \centering
    \includegraphics[width=.9\textwidth]{figures/jpeg_pleno_high_level.pdf}
    \caption{High-level scheme of the coding procedure for PC geometry in \gls{jpeg-pcc}.}
    \label{fig:high-level-scheme}
\end{figure*}

\label{sec:jpegpleno}
The \gls{esqh} method proposed in this paper is implemented on top of the verification model software for the \gls{jpeg-pcc} standard \cite{jpeg-pleno}, which will be presented next in this section. After that, \gls{sqh} \cite{mari2024point}, a previously proposed solution for implementing quality scalability in \gls{jpeg-pcc} for geometry coding, which served as the basis for the \gls{esqh} method, will be described.

\subsection{JPEG Pleno Point Cloud Codec}

\gls{jpeg-pcc} is the JPEG standard for PC coding, which uses a learning-based approach for coding both PC geometry and color attributes \cite{guarda2023point}.
The geometry coding in \gls{jpeg-pcc} utilizes a deep learning model structured as an autoencoder, complemented with a variational autoencoder model that determines a mean and scale hyperprior that improves the performance for entropy coding the compressed domain latent representation \cite{minnen2018joint}. To enhance compression performance, particularly for sparse PCs and lower-rate coding scenarios, \gls{jpeg-pcc} incorporates two additional tools:
\begin{enumerate}
    \item A down-scaling module using a scaling factor parameter, $sf$.
    \item A deep learning-based super-resolution (SR) module to improve reconstruction quality when down-scaling is applied.
\end{enumerate}

\gls{jpeg-pcc} adopts a sparse tensor representation \cite{choy20194d} for geometry coding, offering advantages in both computational complexity and rate-distortion performance. In this representation, PCs are described as a tuple $\bm{x}=(\bm{x}_C, \bm{x}_F)$, where $\bm{x}_C$ represents the coordinates of non-empty voxels, and $\bm{x}_{F}$ denotes the corresponding features (initially set to "1" to indicate occupied voxels).

For encoding the color attributes, JPEG
PCC projects texture patches onto an image (similarly to V-PCC \cite{V-PCC}) which is then coded using the emerging JPEG AI codec \cite{jpeg-ai}.
Since the focus of this paper lies in geometry coding, the remaining of this section will focus only on this component.

A high-level description of the full geometry coding and decoding procedures is shown in Fig.~\ref{fig:high-level-scheme}. Specifically, to encode the geometry of a point cloud $\bm{P} \in \mathbb{R}^3$, the encoder performs the following operations:


\begin{figure*}
    \centering
    \includegraphics[width=.9\textwidth]{figures/jpeg-pcc.pdf}
    \caption{Model architecture of the deep learning based codec in \gls{jpeg-pcc} (\gls{dl}-based Geometry Encoder and \gls{dl}-based Geometry Decoder in Fig.~\ref{fig:high-level-scheme}).}
    \label{fig:dl-scheme}
\end{figure*}

\begin{enumerate}[label=E\arabic*.]
    \item \textit{Downscaling}: The input PC is downscaled by a factor $sf$ through the operation $\boldsymbol{P}^\prime = \lceil \boldsymbol{P}/sf \rfloor$.
    \item \textit{Block Split}: The downscaled points are divided into non-overlapping blocks $\x_{l, C} \in \mathbb{R}^3, l \in {1, \dots, N}$ of size $bs$, such that $\boldsymbol{P}^\prime = \bigcup_{l=1}^{N} \x_{l, C}$.
    \item \textit{Sparse Tensor Construction}: For each block, a sparse tensor representation $\x_l = (\x_{l, C}, \x_{l, F})$ is created, where $\x_{l, F}$ contains ones to indicate occupied voxels.
    \item \textit{DL-Based Encoding}: The blocks are processed through the deep learning-based coding procedure to generate the bitstream
    \item \textit{Distortion Optimization}: Two parameters per block, $k_l$ and $k_{SR, l}$, are computed and added to the bitstream. These parameters represent the optimal number of points to be retained in the decoded block (with and without super-resolution) to minimize a chosen distortion metric.
\end{enumerate}

At the decoder side, the PC reconstruction process consists of the following operations:
\begin{enumerate}[label=D\arabic*.]
    \item \textit{DL-Based Decoding}: The decoder reconstructs the blocks $\hat{\x}_l$ by inputting the compressed domain latent representation, extracted from the bitstream, in the DL-based decoder.
    \item \textit{Top-K Points Selection}: For each decoded block $\hat{\x}_l$, only the $k_l$ points with the highest occupancy probabilities are retained, ensuring optimal point selection.
    \item \textit{Upscaling}: The blocks are upscaled according to scaling factor $sf$ (included in the bitstream) to restore the original spatial resolution.
    \item \textit{Super-Resolution}: When super-resolution is enabled ($SR=1$), the upscaled blocks are processed through the SR network to obtain enhanced blocks $\hat{\x}_{SR, l}$.
    \item \textit{Post-SR Top-K Point Selection}: From each super-resolved block, the $k_{SR,l}$ points with the highest probability values are selected, ensuring optimal point selection.
    \item \textit{Block Merge}: Finally, all processed blocks are merged to reconstruct the complete point cloud geometry $\hat{\boldsymbol{P}}$.
\end{enumerate}

The deep learning-based encoding (and decoding) process for each block $x_l$, is illustrated in Fig.~\ref{fig:dl-scheme}. It consists of the following sequence of operations:

\begin{enumerate}[label=E\arabic*.]
    \item \textit{Latents Generation}: Generate latents $\y_l = (\y_{l, C}, \y_{l, F})$ through the analysis transform $\mathcal{G}_a$, expressed as $\y_l=\mathcal{G}_a(\x_l)$.
    \item \textit{Coordinates Encoding}: code the latent coordinates $\y_{l, C}$ using an octree encoder to generate the coordinates bitstream.
    \item \textit{Hyper-Latents Generation}: Generate hyper-latents $\z_l$ using the hyper-analysis transform $\mathcal{HG}_a$, where $\z_l=\mathcal{HG}_a(\y_l)$.
    \item \textit{Hyper-Latents Quantization}: Quantize the hyper-latent features to obtain $\hat{\z}_{l, F} = \lceil \z_{l, F} \rfloor$.
    \item \textit{Entropy Coding}: Apply rANS entropy coding to the hyper-latents according to a fully factorized prior $p(\hat{\z}_{l, F})$ to generate the hyper-latents bitstream.
    \item \textit{Sparse Tensor Construction}: Reconstruct the quantized hyper-latents' sparse representation as $\hat{\z}_l = (\z_{l, C}, \hat{\z}_{l, F})$.
    \item \textit{Latents Distribution Estimation}: Process $\hat{\z}_l$ through hyper-synthesis transforms $\mathcal{HG}_{s, \mu}$ and $\mathcal{HG}_{s, \sigma}$ to estimate Gaussian parameters $\boldsymbol{\mu}_l = \mathcal{HG}_{s, \mu}(\hat{\z}_l), \boldsymbol{\sigma }_l= \mathcal{HG}_{s, \sigma}(\hat{\z}_l)$. 
    \item \textit{Residual Encoding}: Calculate and encode quantized residuals $\boldsymbol{r}_l= \lceil \y_{l,F} - \boldsymbol{\mu}_l \rfloor$ using $\mathcal{N}(\boldsymbol{0}, \boldsymbol{\sigma}_l)$ to produce the final latents bitstream.
\end{enumerate}

Conversely, a receiver that needs to decode the blocks from the bitstream will have to:

\begin{enumerate}[label=D\arabic*.]
    \item \textit{Coordinates Decoding}: Losslessly decode $\y_{l, C}$ from the coordinates bitstream.
    \item \textit{Hyper-latents Decoding}: Entropy decode $\hat{\z}_{l, F}$ from the hyper-latents bitstream using the probability distribution $p(\hat{\z}_{l, F})$.
    \item \textit{Coordinates Down-scaling}: Down-scale $\y_{l, C}$ by a factor of 4 (as determined by the stride parameters in $\mathcal{HG}_a$'s convolutional layers) to obtain $\z_{l, C}$.
    \item \textit{Hyper-Latents Sparse Tensor Construction}:  Build the sparse representation of hyper-latents as $\hat{\z}_l = (\z_{l, C}, \hat{\z}_{l, F})$.
    \item \textit{Latents Distribution Estimation}:  Compute Gaussian parameters using hyper-synthesis transforms as $\boldsymbol{\mu}_l = \mathcal{HG}_{s, \mu}(\hat{\z}_l), \boldsymbol{\sigma }_l= \mathcal{HG}_{s, \sigma}(\hat{\z}_l)$.
    \item \textit{Residuals Decoding}: Entropy decode $\boldsymbol{r}_l$ from the latents' bitstream using $\mathcal{N}(\boldsymbol{0}, \boldsymbol{\sigma}_l)$.
    \item \textit{Latent Features Reconstruction}: Recover the latent features $\hat{\y}_{l, F} = \boldsymbol{\mu}_l + \boldsymbol{r}_l$.
    \item \textit{Latents Sparse Tensor Construction}: Reconstruct the sparse representation of latents as $\hat{\y}_l = (\y_{l, C}, \hat{\y}_{l, F})$.
    \item \textit{Block Reconstruction}: Apply the synthesis transform $\mathcal{G}_s$ to the decoded latents to determine the probability for the occupancy state of each voxel in the reconstruct the block: $\hat{\x}_l = \mathcal{G}_s(\hat{\y}_l)$. 
\end{enumerate}

The model training follows an end-to-end approach incorporating all previously described operations except for two differences: quantization is replaced by a differentiable approximation and entropy coding is removed, to ensure full model differentiability. The training utilizes a rate-distortion optimization framework defined by the loss function:
\begin{equation}
    \mathcal{L}(\bm{x}, \hat{\bm{x}}, \bm{y}, \bm{z}) = \mathcal{D}(\bm{x}, \hat{\bm{x}}) + \lambda\mathcal{H}(\bm{y}, \bm{z}),
\end{equation}
where $\mathcal{D}(\cdot, \cdot)$ is the distortion, measured as the focal loss \cite{lin2017focal}, $\mathcal{H}(\cdot, \cdot)$ denotes the entropy of the bitstream components under the probability distributions $p(\hat{\bm{z}})$ and $p(\bm{y}|\hat{\bm{z}})$, and $\lambda$ controls the rate-distortion trade-off.
Generally, one model is trained for each RD point corresponding to one value of $\lambda$. In \gls{jpeg-pcc}, five different coding models are trained to support the defined range of tradeoffs.
The training procedure is carried out by sequentially spanning the chosen values of $\lambda \in \{0.0025, 0.005, 0.01, 0.025, 0.05\}$, using the checkpoint for the previous $\lambda$ as a starting point, progressively moving from the lowest value (highest rate/quality) to the highest one (lowest rate/quality). These five models naturally define a quality parameter $qp \in \{1, \dots , 5\}$, with $qp = 1$ corresponding to $\lambda = 0.05$ (lowest rate/quality) and $qp = 5$ to $\lambda = 0.0025$ (highest rate/quality).

\subsection{Scalable Quality Hyperprior}

The \gls{esqh} method proposed in this paper follows a previous work \cite{mari2024point} that introduced a quality scalability algorithm, known as \gls{sqh}. \gls{sqh} constructs a quality scalable bitstream by leveraging information from latents $\y_i$ obtained at a lower \gls{qp} ($qp=i$) to predict probability distributions for latents $\y_j$ at a higher \gls{qp} ($qp=j$).

Starting from a low-quality base layer of latents $\y_i$, which have already been encoded, the encoder must execute the following sequence of steps to generate a new enhancement layer:

\begin{enumerate}[label=E\arabic*.]
    \item \textit{Higher Quality Latents Generation}: Generate new latents $\y_{j}$ using the \gls{jpeg-pcc} coding model with $qp = j > i$.
    \item \textit{Latents Distribution Estimation}: Predict the means and standard deviations of the
latents $\y_{j}$ based on the previous latents $\y_{i}$, using the \gls{dl}-based \gls{qulpe} model
(detailed in \cite{mari2024point}) as $\boldsymbol{\mu}_j$, $\boldsymbol{\sigma}_j = QuLPE(\hat{\y}_i, i, j)$, under the assumption of independently distributed Gaussian latents, $P(\y_j|\hat{\y}_i)$.
    \item \textit{Entropy Coding}: Generate the \gls{sqh} bitstream by entropy encoding $\y_j$ using $\boldsymbol{\mu}_j$, $\boldsymbol{\sigma}_j$.
\end{enumerate}

\gls{sqh} employs a Mean and Scale Hyperprior entropy model, analogous to \gls{jpeg-pcc}. The key distinction lies in \gls{sqh}'s utilization of previously decoded latents $\hat{\y}_i$ as side information, rather than hyper-latents $\hat{\z}_j$.


To reconstruct the higher rate/quality PC, the decoder, which can access the base layer information $\hat{\y}_i$, performs the following decoding procedure:
\begin{enumerate}[label=D\arabic*.]
    \item \textit{Latents Distribution Estimation}: Derive $\boldsymbol{\mu}_{j}$, $\boldsymbol{\sigma}_{j} = QuLPE(\hat{\y}_i, i, j)$ from the base layer information $\hat{\y}_i$ using the \gls{qulpe} model.
    \item \textit{Higher Quality Latents Decoding}: Decode the higher quality latents $\hat{\y}_{j}$ by applying a rANS decoder to the \gls{sqh} bitstream using the estimated distribution.
    \item \textit{Higher Quality PC Reconstruction}: Reconstruct the higher quality PC by processing the decoded latents through the \gls{jpeg-pcc} synthesis transform as $\hat{\x}_{j} = \mathcal{G}_{s, j}(\hat{\y}_{j})$.
    \item \textit{Super Resolution}: If specified in the coding parameters the Super Resolution model is used to enhance the decoded blocks
\end{enumerate}

While \gls{sqh} effectively handles quality scalability through latents refinement, it faces limitations when dealing with \gls{jpeg-pcc}'s downscaling strategy. The challenge arises because varying $sf$ produces latents at different resolutions, a scenario not supported by \gls{sqh}'s U-Net-based \gls{qulpe} model, which requires consistent input-output dimensions. This architectural constraint, coupled with the absence of a multi-resolution handling strategy, restricts \gls{sqh}'s practical applicability. 

The next sections introduce and evaluate \gls{esqh}, an enhanced framework that addresses these limitations by enabling joint quality and resolution scalability in the latent domain. These functional advantages are relevant in the framework of the \gls{jpeg-pcc} codec, but also for the generalization of the \gls{esqh} method for other autoencoder-based codecs.

\section{The E-3DGS Method}\label{sec:Method} 
Our aim is to learn a 3D representation of a static scene using only a color event stream, where each pixel observes changes in brightness corresponding to one of the red, green, or blue channels according to a Bayer pattern, with known camera intrinsics $K_t~\in~\mathbb{R}^{3 \times 3}$, and noisy initial poses~$P_t~\in~\mathbb{R}^{3 \times 4}$, at reasonably high-frequency time steps indexed by $t$. 
Following 3DGS~\cite{3dgs}, we represent our scene by anisotropic 3D Gaussians. Our methodology comprises a technique to initialize Gaussians in the absence of a Structure from Motion (SfM) point cloud, adaptive event frame supervision of 3DGS, and a pose refinement module. 
An overview of our method is provided in Fig.~\ref{fig:methodology}.


Our E-3DGS method is not restricted to scenes of a certain size and can handle unbounded environments. It does not rely on any assumptions regarding the background color, type of camera motion, or speed. Thus, it ensures robust performance across a wide range of scenarios. 

\subsection{Event Stream Supervision} 

There are two main categories of approaches to learning 3D scene representations from event streams. 
Some apply the loss to single events~\cite{robust_enerf} based on Eq.~\eqref{eq:egm}. Others use the sum of events~\( E_{\x}(t_1,t_2) \) from Eq.~\eqref{eq:egm_sum}. We choose the second approach, as rasterization in 3DGS is well suited to efficiently render entire images rather than individual pixels. 

To optimize our Gaussian scene representation using event data, we can make a logical equivalence between the observed event stream and the scene renderings. 
To do so, we replace the true logarithmic intensities~\( L_{\x} \) in Eq.~\eqref{eq:egm_sum} with the rendered logarithmic intensities~\(\hat{L}_{\x} \) from our scene, and the times $\tau$ with the camera poses $P_t$ that were used to render the scene at the respective time steps. 
Following the approach used in~\cite{eventnerf}, the log difference is then point-wise multiplied with a Bayer filter $F$ to obtain the respective color channel. We can finally calculate the error between the logarithmic change from our model and the actual change observed from the event stream, and define the following per-pixel loss: 
\begin{equation}
    \begin{split}
    &\mathcal{L}_{\x}\left(t_1, t_2\right) = \\
    &\left\| 
    F \odot (\hat{L}_{\x}(P_{t_2}) - \hat{L}_{\x}(P_{t_1})) 
    - F \odot E_{\x}\left(t_1, t_2\right)\right\|_1, 
    \end{split}
    \label{eq:L_recon_per_pixel} 
\end{equation}
where ``$\odot$'' denotes pixelwise multiplication. 


\subsection{Frustum-Based Initialization}
\label{sec:frustum_init}

In the original 3DGS \cite{3dgs}, the Gaussians are initialized using a point cloud obtained from applying SfM on the input images. 
The authors also experimented with initializing the Gaussians at random locations within a cube. While this worked for them with a slight performance drop, it requires an assumption about the extent of the scene. 

Applying SfM directly to event streams is more challenging than RGB inputs \cite{Kim2016} and exploring this aspect is not the primary focus of this paper. 
In the absence of an SfM point cloud, we use the randomly initialized Gaussians and extend this approach to unbounded scenes. 
To this end, we initialize a specified number of Gaussians (on the order of \qty{d4}{}) in the frustum of each camera. 
This gives two benefits: 1) All the initialized Gaussians are within the observable area, and 2) We only need one loose assumption about the scene, which is the maximum depth $z_\mathrm{far}$. 


\subsection{Adaptive Event Window}\label{subsec:adaptive_window}

Rudnev et al.~\cite{eventnerf} demonstrated in EventNeRF that using a fixed event window duration results in suboptimal reconstruction. They find that larger windows are essential for capturing low-frequency color and structure, and smaller ones are essential for optimization of finer high-frequency details. While they randomly sampled the event window duration, a drawback is that it does not consider the camera speed and event rate, thus the sampled windows may contain too many or too few events.  
As our dataset features variable camera speeds, we improve upon this by sampling the number of events rather than the window duration.  
To achieve this, for each time step we randomly sample a target number of events from within the range $[N_\mathrm{min}, N_\mathrm{max}]$. 
Given a time step~$t$, we search for a previous time step~$t_s$ such that the number of events in the event frame $E(t_s, t)$ is approximately equal to the desired number. 

When determining $N_\mathrm{max}$, we find that for values where details and low-frequency structure are optimal, 3DGS tends to get unstable and sometimes prunes away Gaussians in homogeneous areas.
While this can be mitigated by choosing a much larger $N_\mathrm{max}$, this again deteriorates the details. 
Therefore, we propose a strategy to incorporate both, small and large windows. For each $t$, we choose two earlier time steps~$t_{s_1}$ and~$t_{s_2}$. The ranges for sampling the event counts for both are empirically chosen to be $[\frac{N_\mathrm{max}}{10}, N_\mathrm{max}]$ and $[\frac{N_{max}}{300}, \frac{N_\mathrm{max}}{30}]$. We then render frames from our model at times $t$, $t_{s_1}$ and $t_{s_2}$, and use two concurrent losses for the event windows $E_{\x}\left(t_{s_1}, t\right)$ and $E_{\x}\left(t_{s_2}, t\right)$. 

\subsection{As-Isotropic-As-Possible Regularization} 
\label{ssec:IsotropicReg} 

In 3DGS, Gaussians are unconstrained in the direction perpendicular to the image plane. 
This lack of constraint can result in elongated and overfitted Gaussians. 
And while they may appear correct from the training views, they introduce significant artifacts when rendered from novel views by manifesting as floaters and distortions of object surfaces. 
We also observe that the lack of multi-view consistency and tendency to overfit destabilize the pose refinement. 

To mitigate these issues, we draw inspiration from Gaussian Splatting SLAM~\cite{3dgsslam} and SplaTAM~\cite{splatam}, and apply isotropic regularization:
\begin{equation}
    \mathcal{L}_{\text{iso}} = \frac{1}{|\mathcal{G}|} \sum_{g \in \mathcal{G}} \left\| S_g - \bar{S}_g \right\|_1
    \label{eq:L_iso} \mathrm{\,,}
\end{equation}
where~$\mathcal{G}$ is the set of Gaussians visible in the image. Eq.~\eqref{eq:L_iso} imposes a soft constraint on the Gaussians to be as isotropic as possible.
We find that it helps to improve pose refinement, minimizes floaters and enhances generalizability. 

\subsection{Pose Refinement} 
\label{sec:pose_refinement}

To obtain the most accurate results, we allow the poses to be refined during optimization
by modeling the refined pose as $P'_t = P^e_t P_t$, where  $P^e_t$ is an error correction transform. 
Instead of directly optimizing~$P^e_t$ as a~$3 \times 3$ matrix, following Hempel et al.~\cite{6d_rotation} we represent it as $[r_1\,\, r_2\,\, T]$, where $r_1$ and $r_2$ represent two rotation vectors of the rotation matrix~$R = [r_1\,\, r_2\,\, r_3]$, while $T$ is the translation.
We can then obtain the~$P^e_t$ matrix from the representation using Gram-Schmidt orthogonalization (see details in Supplement~\ref{sec:supp_pose_refinement}), hence ensuring that during optimization, our error correction transform always represents a valid transformation matrix. 
$P^e_t$ is initialized to be the identity transform. Since the loss function from Eq.~\eqref{eq:L_recon_per_pixel} depends on the camera pose as well, it allows us to use the same loss to backpropagate and obtain gradients for pose refinement. 

As our goal is to refine the estimated noisy poses rather than perform SLAM, this training signal is sufficient for our needs. Moreover, we observe that poses tend to diverge with 3DGS due to the periodic opacity reset.
To combat this, we impose a soft constraint with an additional pose regularization, that encourages the matrices~$P^e_t$ to stay close to the identity matrix $I$:
\begin{equation}
    \mathcal{L}_{\text{pose}} = \| P^e_{t_{s_1}} - I \|_2 + \| P^e_{t_{s_2}} - I \|_2 + \| P^e_{t} - I \|_2
    \label{eq:L_pose} \mathrm{\,,}
\end{equation}
with all terms weighted equally. 


\subsection{Optimization}
\label{ssec:Optimization} 

Eq.~\eqref{eq:L_recon_per_pixel} defines the reconstruction loss per pixel for a single event frame. However, naively averaging these per-pixel losses over whole images leads to problems. For small event windows, most pixels have no events, which are not very informative but will then make up the majority of the loss. 
To address this, we compute separate averages of the losses for pixels with events~$\mathcal{X}_\text{evs}$ and pixels without events~$\mathcal{X}_\text{noevs}$. 
These averages are then scaled by the hyperparameter~$\alpha=0.3$ to obtain the complete weighted reconstruction loss:
\begin{equation}
    \begin{split}
        \mathcal{L}_{\text{recon}}\left(t_s, t\right) = \,\,&
        \frac{\alpha}{|\mathcal{X}_{\text{noevs}}|} \cdot 
        \left(\sum_{\x\in \mathcal{X}_{\text{noevs}}} \mathcal{L}_{\x}\left(t_s, t\right)\right) + \\
        + \,\,& \,\, \frac{1 - \alpha}{|\mathcal{X}_{\text{evs}}|} \,\,\, \cdot 
        \left(\sum_{\x\in \mathcal{X}_{\text{evs}}} \mathcal{L}_{\x}\left(t_s, t\right)\right). 
    \end{split}
    \label{eq:L_recon}
\end{equation}
To obtain the final loss, we take a weighted sum of the reconstruction losses for the two event windows from Sec.~\ref{subsec:adaptive_window} along with the isotropic and pose regularization: 
\begin{equation}
    \begin{split}
        \mathcal{L} =\,\,\,\, & 
        \lambda_1 \mathcal{L}_{\text{recon}}\left(t_{s_1}, t\right) \,\,+  \,\,
        \lambda_2 \mathcal{L}_{\text{recon}}\left(t_{s_2}, t\right)  \\&
        +\,\, \lambda_\text{iso} \mathcal{L}_{\text{iso}} \,\,+ \,\,
        \lambda_\text{pose} \mathcal{L}_{\text{pose}}
    \end{split}
    \label{eq:loss} \mathrm{\,,}
\end{equation}
where $\lambda_1$, $\lambda_2$ and $\lambda_{\text{iso}}$ are hyper-parameters. In our experiments, we use  $\lambda_1=\lambda_2=0.65$, and $\lambda_{\text{iso}}$ is set to $10$ initially and reduced to $1$ after $\qty{d4}{}$ iterations. 






\section{Results}
We first describe communication patterns within the full chronological context of the game in \textit{League of Legends (LoL)}, separated into four sections based on changing coordination dynamics. Based on this context, we identify core factors players assess to decide when to participate in communication with other teammates. Afterward, we discuss how communication shapes player perceptions toward their teammates, showing player's wariness towards players actively engaging in communication. 

\subsection{Communication Patterns in Context}

We discuss the communication patterns among teammates within the game. We organize the data into chronological phases of the game for a structured analysis of how the context shapes communication patterns. 

\subsubsection{Pre-game stage}
Before gameplay begins, team communication opens with \textit{team drafting}, where players are assigned roles (Top, Mid, Bot, Support, or Jungle) and take turns picking or banning champions. In Solo Ranked mode, roles are pre-assigned based on player preferences selected before queueing. Once teams are set, all players enter \textit{champion select} stage, alternating champion picks and banning up to five champions per team. During this stage, communication is limited to text chat. The usernames are anonymized (i.e., replacing the name with aliases) to prevent queue dodging by checking third-party stats sites such as OP.GG\footnote{https://www.op.gg/}, leaving the chat as the only option to inform individual strengths and preferences. 

Team composition in \textit{LoL} is crucial to the strategy and outcome of the game~\cite{ong2015player}, setting the basis for future interactions. Most participants acknowledged the importance of balanced and synergistic team composition, especially as players move into higher ranks where team coordination outweighs individual excellence. Yet, we observed a distinct lack of verbal communication between the members during this period across all ranks. Participants attributed the lack of willingness to initiate a conversation on the dangers of starting the game on a bad footing. They prioritized ``not creating friction'' during this stage as negative impressions can propagate throughout the game. Some participants attempted communication to reduce such friction, such as P14, who stated,``\textit{If I had the time, I wanted to say that I will be banning [this Champion], just in case a player on my team wanted to play them.}'' However, several participants viewed any communication during the pre-game phase with wariness, as dissatisfaction or conflict at this step portended negative interactions between players in the game (P3, P9, P15). Thus, even when participants expressed doubt about other teammates' unconventional or non-meta champion picks, they refrained from entering into discourse. This contrasts with findings by Kou and Gui~\cite{kou2014}, which showed players attempt to maintain a harmonious and constructive atmosphere through greetings and introductions.

Another emergent code of the reason for not engaging in communication in the pre-game stage stems from different purposes of playing the game (P1, P5, P13, P16, P17). Despite being in ranked mode, which is more prone to increased competitiveness and effort, participants showed differing goals and levels of interest in winning the game. Several players stated that they had previously exerted great mental load in coordinating synergistic plays, but stopped as they gave less importance to winning at all costs (``\textit{I don't really play to win. I play \textit{LoL} to relieve stress, so I don't engage in chat.}'', P5). These players saw verbal communication with the goal of coordination as an unnecessary or even cumbersome component of the pre-game stage.


\subsubsection{Structured phase}
In many MOBAs, including \textit{LoL}, the early stages of the game play out in a formulaic manner: players join their lanes (Top, Mid, and Bot/Support), defeat minions to gain gold, buy items towards certain ``builds'', kill or assist in early objectives (Jungle), and battle counterparts in their respective lanes. Participants at this stage expressed that most players possessed tacit knowledge of what must be done, such as knowing when to aid their Jungle to capture a jungle monster, choosing the opportune moments to leave their lanes, or positioning wards (i.e., a deployable unit which provides a vision of the surrounding area) at the ideal placements. The participants assumed each player knew their ``role'' to fulfill, often comparing it to ``doing their share'' (P1, P3, P7, P19). In line with this belief, players rarely initiated preemptive or proactive verbal communication for strategic or social purposes at the early stage. 

Pings, on the other hand, constantly permeated the game. At this stage, players used ping to provide information relevant to others from their position, such as letting others know if an enemy went missing from their lane. As the players are largely separated and independent from one another, pings (coupled with the minimap and scoreboard) served as the primary channel for maintaining context over the game rather than as warnings or direct guidance to the players. For other non-verbal gestures, while objective votes would occasionally appear, they were rarely answered. Instead, relevant players near the objective would place pings or move toward it to help out their teammates.

Participants viewed the structured phase as a routine, but uncertain period of the game where the pendulum could swing in either team's favor. Players --- especially Jungles who roam the board looking for opportunities to ambush the enemy team in lanes (``gank'') --- sometimes felt hesitant to make calls and demands at this stage since ``\textit{[they] could make a call, but if I fail, they'll start blaming my decisions down the line.}'' (P7) But at this stage, participants believed that they held personal agency over the final game outcome. P1 and P6 stated that they entered the game with the mindset that only they had to succeed regardless of the performance of their teammates. This belief was reflected in their chatting behavior, where players prioritized focusing on their circumstances over the team's (``\textit{I mute the chat so that I don't get swayed by the team, as I can win the game if I do well.}'', P9).


\subsubsection{Group engagement phase}
As the game enters its middle phase, it provides opportunities for more diverse decision-making. Players may swap lanes, seize or trade crucial objectives, and fight in large battles involving multiple champions. At this point, teams typically have a clear outlook on which players and team have the advantage, requiring more team-driven decisions to maintain or overcome their current standing. Thus, players used verbal communication to discuss more complicated tactics that could not be effectively conveyed through pings.

But more often than not, chat messages became judgment-based. As enemy engagement with larger groups occurred more frequently, the availability for chatting would come after death, which led to comments on past actions rather than future choices. Additionally, the respawn timer for deaths becomes longer as the game progresses, providing more time to observe other players than in earlier phases. This gave players more opportunities to express dissatisfaction specifically towards certain plays, such as placing Enemy Missing pings on the map where other teammates are located to bring attention to their questionable play.

This stage also gave much more exposure of each other to the allies as the team would gather at a single point, giving way to greater scrutiny by their teammates. Repeated or critical mistakes put participants on edge, as they braced for criticism from their teammates. They expressed relief or surprise when the chat remained silent or civil, with P8 stating ``\textit{I messed up there. No one is saying anything, thankfully.}''


\subsubsection{Point of no return}
Meanwhile, verbal communication flowed out when the game had a clear trajectory to the end. Previous research has shown that both toxic and non-toxic communication skyrockets near the end of the game~\cite{kwak2015linguistic} when the players have determined the game outcome with certainty. We saw that this phase opened up both positive and negative sides of communication for guaranteed win and loss, respectively. The winning team would compliment and cheer each other through chat messages and emotes, while the losing side devolved into arguments and calling out. The communication at this stage was driven by emotion, showing excitement or venting frustration.


\subsection{Communication Assessment Process}

We describe the factors that users mainly focus on to assess when or when not to involve themselves in communication with their teammates. 

\subsubsection{Calculating communication cost}
One of the most proximate factors behind when communication is performed is the limited action economy of the game. In \textit{LoL} and other MOBAs, players can rarely afford time to type out messages due to the fast-paced nature of the game. In time-sensitive scenarios, the time pressure makes communication particularly costly. It is therefore unsurprising that much of the communication occurs after major events (e.g., battles and objective hunting), as players are given more downtime while waiting for teammates or enemies to respawn or regroup.

For periods where players were still actively involved in gameplay, the players made conscious decisions on choosing which communication media to use based on the perceived action availability and the importance of communicating the message. Players relied on pings for non-critical indications, believing that the mutual understanding of the game would get the message across. However, many players recognized that pings were prone to be missed, ignored, or misinterpreted by their allies (P2, P9, P16, P17, P20). Subsequently, participants typed out information considered to be too important to the situation to be misunderstood or missed by other players even if it caused delays in their gameplay (P10, P11, P14). Simultaneously, the priority of importance constantly shifted --- we observed multiple times participants start to type, but stop to react to an ongoing play, only to never send out their message.

\subsubsection{Evaluating relevance and responsiveness}
When the brief window of communication opportunity is missed, players are unlikely to ever send out that information. In \textit{LoL}, situations can change within seconds and certain communication media cannot keep up with the changing state of the game. For example, almost all study participants did not participate in votes for objectives. Among the tens of objective votes initiated among all the games in this study, no objective vote saw more than three votes, frequently being left with no vote beyond the player who initiated the vote. Some players, when asked why they did not participate, stated that the votes they made often became irrelevant as the game state had changed during the time it took to vote (P2, P11). Other players also discussed how information conveyed through communication can get outdated fast (P1, P8, P9). 

\begin{quote}
I can't always follow through with what I say [in the chat] since the game is really dynamic. My teammates don't understand such situations, so I tend to not chat proactively. - P9
\end{quote}

Thus, some players instead preferred to react through direct action (P8, P10, P11, P16, P20). P10 stated, ``\textit{I think it's enough to show through action rather than [using objective voting]. I can look out for how the player reacts when I request something from them.}''

On the other hand, such action-based responses left the player to assess whether and how the communication was received. P10 stated that they tried to predict whether a player understood their ping direction by how they moved, but it was hard to interpret their intent: ``\textit{members sometimes seem to move towards me but then turn around, and sometimes they even ping back but don't come.}''. P16 discussed how they weren't sure whether the ping was received, but performed it anyway since it felt helpful.

Similarly, participation in surrender votes (or lack thereof) carried different intent by the player. During most of the games that ended in a loss, one or more surrender votes were called by the participant's team. However, only two surrender votes achieved four or more players' participation. However, the reasons why a player chose to not participate varied. Some had decided to wait and see how other teammates voted, which may have paradoxically led many members to not participate in the vote (P4, P9). Meanwhile, others didn't reply as they didn't think the vote was actually calling for a response: P13 stated, ``\textit{I didn't vote because they were just showing their anger. It's just a member venting through a surrender vote that they're not doing well.}''

\subsubsection{Balancing information access and psychological safety}
While recognizing that communication would be useful or even necessary in certain situations, participants also put their psychological safety first over information access. Some players, worn down by the normalization of toxic communication such as flaming, muted the chat (P1, P9).

Many participants expressed the sentiment of ``protecting [their] mentality'', describing how certain communication harmed their psychological well-being. This communication did not always refer to negative communication; P9 often muted players who gave commands as they did not want to be ``swept up'' by others' play-related judgments. This separation even extended to other more widely considered essential communication forms, such as pings. Even after acknowledging that pings were vital and useful to the game, P9 went as far as muting the ping of the support player in the same lane after they sent a barrage of Enemy Missing pings that signified aggression and criticism. 

Additionally, the abundance and high frequency of communication also strained the limited mental capacity of the players. Many players, when asked why they had not replied to an objective vote or other chat messages, stated that they simply did not notice them among other events happening (P1, P2, P3, P9, P12, P15, P18, P20, P21). The information overload caused stress and became distracting to players.

\subsubsection{Reducing potential friction}
As demonstrated in the pre-game stage of the game, players sometimes used communication to minimize friction between their teammates. Some participants sacrificed time to apologize to other players when they believed themselves to be at fault. When asked why, P12 replied, ``\textit{There are too many people who don't come to help gank if I don't apologize.}''. Similarly, P5 sacrificed time typing in an apology after a teammate had died despite still being in the middle of a fight as they didn't wish to give the other player a reason to start an attack.

However, some noted that silence is sometimes the best answer to a negative situation. P4, after dying to the enemy, put into chat ``Fighting!'' (roughly meaning, ``We can do it!''). They stated ``\textit{I don't know why I do it... it probably angers [my teammates] more.
}'' They also stated that ``\textit{for certain people, talking in the chat only spurs them more. You just have to let them be.}'' Other players shared similar sentiments that being quiet and dedicating focus to the game was a better choice (P1, P11, P14).

For female players, the fear of gender-based harassment shaped their communication patterns. While \textit{LoL} does not provide any demographic information of a player to other players, almost all female participants noted experiences of receiving derogatory remarks or doubts about their abilities based on other players' assumptions of their gender, a trend frequently seen in male-dominated online gaming cultures~\cite{fox2016women, norris2004gender, mclean2019female}. They noted that the players were able to correctly guess their gender when the participant's role and champion fit into the preconceived notions of what women ``tended to play'' (i.e., female-identifying support champions, such as Lulu) or their username ``seemed feminine'' (P18, P19, P20, P21, P22). This led to certain players adopting tactics that signaled male-like behavior, such as changing their speech style to be more gender-neutral or male-like (P19, P21) and changing their username to sound more gender-neutral. Cote describes similar instances of ``camouflaging gender'' as one of five main strategies for women coping with harassment~\cite{cote2017coping}. However, some players opted to keep playing their preferred character or maintaining their username even if it signaled their gender, such as P21 who expressed, ``\textit{I cherish and feel attached to my username, so I don’t want to change it just because of [harassment and inappropriate comments].}'' These players valued self-expression and identity even at the risk of increased risk to unpleasant communication experiences.


\subsubsection{Forming performance-based hierachy}
Naturally formed leadership has often been observed in other works on \textit{LoL} teams~\cite{kou2014}. Kim et al. showed that more hierarchy in in-game decision-making led to higher collective intelligence~\cite{kim2017}. While they used ``hierarchy'' to mean varying amounts of communication throughout the game, we observed that the hierarchy extends further to performance-based hierarchy, where teammates in more advantageous positions are given greater weight when communicating with other players. Players actively chose to refrain from suggesting strategic plans when they were ``holding down the team'', recognizing that they held less power and trust among the team members (P8, P10, P12, P14, P22). The player who was losing against the enemy team was viewed as having no ``right'' to lead the team, which was reserved for well-performing players.


\subsubsection{Enforcing norms and habits}
One of the most common answers to why players performed certain communication actions, especially non-verbal actions such as pings and emotes, was ``a force of habit'' (P6, P7, P8, P9, P10, P12, P17). Players formed learned practices of using communication channels at certain points by observing other players exhibit the same behaviors. This promoted, for example, replying to an emote sent by the teammate with their own or pinging readied skills and items to emphasize relevant information for other players throughout the game. 

On the other hand, this meant that players were averse to communication patterns outside of the norm --- participants stated that they had a hard time adapting to new forms of communication, seeing no immediate benefit or impact from using them (P1, P8, P14, P13, P15, P17). Most egregiously, the recently introduced objective pings were largely viewed to be awkward to use and unnecessary (P1, P4, P8, P12).


\subsection{Impact of Communication Assessment}
We describe how the communication patterns and assessment of the players impact the individual players' perspectives on team dynamics.

\subsubsection{Relationship between trust and communication frequency}
Most participants saw value in constant and well-informed communication but with an important distinction: verbal communication with strangers rarely ended well. Players largely recognized frequent verbal communication to burgeon conflict, regardless of the message within. Even when players understood the helpful intent behind positive messages from the players, they compared actively talking players to be possible bad actors who were likely to exhibit toxic behaviors when the game turned against them. (P1, P4, P8, P12, P14)

\begin{quote}
I need to make sure to not disturb Twisted Fate. I saw him start to flame. It's not because I don't want to hear more criticism. I know these types. The more I react and chat with them, the more deviant they will become. - P4  
\end{quote}

Similarly, P19 lamented that players used to socialize more in the chat during the pre-game phase to build a fun and prosocial environment, noting a memorable example of encouraging each other to do well on their academic exams, but noted that such prosocial behavior has become much rarer during the recent seasons. They noted that there are inevitably players ``who take it negatively'' and thus stopped proactively typing non-game related messages in the chat.

Ultimately, players desired assurance and trust of player commitment. The participants trusted actions more than words to prove that the player remained dedicated to the game. Both P10 and P17 pointed out that it was easy to tell who was still ``in the game'' and motivated to try their best and that ``staying on the keyboard'' likely meant that they weren't invested or focused on the game. Players viewed such commitment to be the most important aspect of a ``good'' teammate, sometimes even more than their skill or performance (P9, P14). It is interesting to note that unlike what previous literature may suggest~\cite{marlow2018}, players' averseness to talkative teammates had less to do with the cognitive overload or distraction caused by the frequent communication, but rather due to the threats of future team breakdown. This view in turn also affected how players decided to communicate or not, as they believed that players would not take their suggestions or comments in a positive light. 


\subsubsection{Perception of player commitment and fortitude}

Communication also acted as a mirror of their teammates' mental fortitude. A number of players mentioned how they valued a resilient mindset in their teammates playing the game, referring to players who remained committed to the game until the very end. They saw players who provoked or complained to teammates as ``having a weak mentality'' who had been altered by the bad outcomes of the game to act in an unhelpful manner towards the team through their communication. The communication actions of the teammate informed the participants of how steadfast their teammate remained in disadvantageous situations.  

\begin{quote}
It's not like I constantly reply in the chat or anything, but I pay attention [to the chat] to grasp the overall atmosphere of the team. If the team doesn't collaborate well then we lose, so I try to have a rough understanding of the mentality of the other players. - P13
\end{quote}

There were also instances of communication that helped players maintain a positive view of their teammates. For example, P11 mentioned near the beginning of the game, ``\textit{Looking at the chat, Varus player has strong mentality [for being so positive]. There were lots of points [in his support's] plays that he could have criticized.}'' Unfortunately, this view quickly soured when the Varus player devolved into criticism later in the late game phase where the Varus player started criticizing the support and other players. P11 then noted that the Varus player seemed to merely be ``bearing through the game''.
\section{Conclusions}
\label{sec:conclusions}

In this work, we proposed \plangen{}, an easily scalable multi-agent approach incorporating three key components: constraint, verification, and selection agents. We leveraged these agents to improve the verification process of existing inference algorithms and proposed three frameworks: Multi-Agent Best of $\mathcal{N}$, ToT, and REBASE. Further, we introduced a Mixture of Algorithms, an iterative framework that integrates the selection agent (Figure \ref{fig:teaser}) to dynamically choose the best algorithm. We evaluated our frameworks on NATURAL PLAN, OlympiadBench, GPQA, and DocFinQA. Experimental results demonstrate that \plangen{} outperforms strong baselines, achieving SOTA results across datasets. Furthermore, our findings suggest that the proposed frameworks are scalable and generalizable to different LLMs, improving their natural language planning ability.


\section*{Limitations}

Despite the strong performance of our frameworks, an area of improvement is the reliance on predefined heuristics for selecting inference-time algorithms, which may not always generalize optimally across all tasks and domains. Additionally, while our frameworks demonstrate strong performance, their computational overhead could be further optimized for efficiency in real-world applications. We believe that our frameworks can be useful in further boosting the planning and reasoning capabilities of existing models such as o1 and Gemini-thinking. In addition, the use of reinforcement learning or meta-learning techniques to dynamically adapt agent strategies based on task complexity could be an interesting area to explore. Moreover, broadening the scope to multi-modal and multi-lingual reasoning would significantly expand the applicability of our approach, and exploring the use of generated planning trajectories for model training offers valuable direction.

\section*{Ethics Statement}

The use of proprietary LLMs such as GPT-4, Gemini, and Claude-3 in this study adheres to their policies of usage. We have used AI assistants (Grammarly and Gemini) to address the grammatical errors and rephrase the sentences.

% \section*{Acknowledgments}
\section*{Acknowledgment}
After the first draft, the text in the various sections of the manuscript (except the Abstract) was improved using  Claude 3.5 Sonnet through the Perplexity AI web app. After this process, the paper underwent multiple sets of reviews from the authors and was thus further improved and modified.
\bibliographystyle{IEEEtran}
\bibliography{biblio}

\begin{IEEEbiography}[{\includegraphics[width=1in,height=1.25in,clip,keepaspectratio]{figures/Daniele.jpg}}]{Daniele Mari} received the B.S. degree in information engineering and M.S. degree in ICT for Internet and Multimedia from the University of Padova, Italy, in 2019 and 2021 respectively. He is currently pursuing his Ph.D. in information engineering  and is currently a Ph.D. student in Padova since 2021. Additionally, he has spent 6 months as a visiting Ph.D. student in Instituto Superior Técnico, Universidade de Lisboa, Portugal, in 2023. His main research interests are learned point cloud and image coding. He has authored several publications in top conferences and journals in this field.
\end{IEEEbiography}


\begin{IEEEbiography}[{\includegraphics[width=1in,height=1.25in,clip,keepaspectratio]{figures/Andre.png}}]{ANDRÉ F. R. GUARDA} (Member, IEEE) received his B.Sc. and M.Sc. degrees in electrotechnical engineering from Instituto Politécnico de Leiria, Portugal, in 2013 and 2016, respectively, and the Ph.D. degree in electrical and computer engineering from Instituto Superior Técnico, Universidade de Lisboa, Portugal, in 2021. He has been a researcher at Instituto de Telecomunicações since 2011, where he currently holds a Post-Doctoral position. His main research interests include multimedia signal processing and coding, with particular focus on point cloud coding with deep learning. He has authored several publications in top conferences and journals in this field and is actively contributing to the standardization efforts of JPEG and MPEG on learning-based point cloud coding.
\end{IEEEbiography}

\newpage

%If you do not have or do not want to include a photo, you can use IEEEbiographynophoto as shown below:

\begin{IEEEbiography}[{\includegraphics[width=1in,height=1.25in,clip,keepaspectratio]{figures/Nuno_Rodrigues_4_by_3.jpg}}]{NUNO M. M. RODRIGUES} 
 (Senior Member,
IEEE) graduated in electrical engineering in 1997, received the M.Sc. degree from the Universidade de Coimbra, Portugal, in 2000, and the Ph.D. degree from the Universidade de Coimbra, Portugal, in 2009, in collaboration with the Universidade Federal do Rio de Janeiro, Brazil. He is a Professor in the Department of Electrical Engineering, in the School of Technology and Management of the Polytechnic University of Leiria, Portugal and a Senior Researcher in Instituto de Telecomunicações, Portugal. He has coordinated and participated as a researcher in various national and international funded projects. He has supervised three concluded PhD theses and several MSc theses. He is co-author of a book and more than 100 publications, including book chapters and papers in national and international journals and conferences. His research interests include several topics related with image and video coding and processing, for different signal modalities and applications. His current research is focused on deep learning-based techniques for point cloud coding and processing.
\end{IEEEbiography}

\begin{IEEEbiography}[{\includegraphics[width=1in,height=1.25in,clip,keepaspectratio]{figures/simone-milani-bg.png}}]{Simone Milani} 
(Member, IEEE) received the
Laurea degree in telecommunication engineering
and the Ph.D. degree in electronics and telecommunication engineering from the University of
Padova, Padova, Italy, in 2002 and 2007, respectively. He was a Visiting Ph.D. Student at the University of California at Berkeley, Berkeley, CA,
USA, in 2006. He was a Consultant at STMicroelectronics, Agrate, Italy. He was a Postdoctoral Researcher at the University of Udine, Udine,
Italy, the University of Padova, and the Politecnico di Milano, Milan, Italy,
from 2007 to 2013. From 2013 to 2020, he was an Assistant Professor with
the Department of Information Engineering, University of Padova, where
he is an Associate Professor. His research interests include digital signal
processing, image and video coding, 3-D video processing and compression,
joint source-channel coding, robust video transmission, distributed source
coding, multiple description coding, and multimedia forensics.
\end{IEEEbiography}

\begin{IEEEbiography}[{\includegraphics[width=1in,height=1.25in,clip,keepaspectratio]{figures/Fernando.png}}]{Fernando Pereira} (Fellow, IEEE) graduated in electrical and computer engineering in 1985 and received the M.Sc. and Ph.D. degrees in 1988 and 1991, respectively, from Instituto Superior Técnico, Technical University of Lisbon. He is with the Department of Electrical and Computers Engineering of Instituto Superior Técnico, University of Lisbon, and Instituto de Telecomunicações, Lisbon, Portugal. He is or has been Associate Editor of IEEE Transactions of Circuits and Systems for Video Technology, IEEE Transactions on Image Processing, IEEE Transactions on Multimedia, IEEE Signal Processing Magazine and EURASIP Journal on Image and Video Processing, and Area Editor of the Signal Processing: Image Communication Journal. In 2013-2015, he was the Editor-in-Chief of the IEEE Journal of Selected Topics in Signal Processing. He was an IEEE Distinguished Lecturer in 2005 and elected as an IEEE Fellow in 2008 for “contributions to object-based digital video representation technologies and standards”. He has been elected to serve on the IEEE Signal Processing Society Board of Governors in the capacity of Member-at-Large for 2012 and 2014-2016 terms. He has been IEEE Signal Processing Society Vice-President for Conferences in 2018-2020 and IEEE Signal Processing Society Awards Board Member in 2017. He was the recipient of the 2023 Leo L. Beranek Meritorious Service Award. Since 2013, he is also a EURASIP Fellow for “contributions to digital video representation technologies and standards”. He has been elected to serve on the European Signal Processing Society Board of Directors for a 2015-2018 term. He was the recipient of the 2023 EURASIP Meritorious Service Award. Since 2015, he is also an IET Fellow. He has also held key leadership roles in numerous IEEE Signal Processing Society conferences and workshops, mostly notably serving twice as ICIP Technical Chair in two continents, Hong Kong (2010) and Phoenix (2016). He has been MPEG Requirements Subgroup Chair and is currently JPEG Requirements Subgroup Chair. Recently, he has been one of the key designers of the JPEG Pleno and JPEG AI standardization projects. He has contributed more than 300 papers in international journals, conferences and workshops, and made several tens of invited talks and tutorials at conferences and workshops. His areas of interest are video analysis, representation, coding, description and adaptation, and advanced multimedia services.
\end{IEEEbiography}

\EOD

\end{document}
