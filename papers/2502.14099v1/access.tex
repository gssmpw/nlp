\documentclass{ieeeaccess}
\usepackage{cite}
\usepackage{amsmath,amssymb,amsfonts}
\usepackage{algorithmic}
\usepackage{graphicx}
\usepackage{textcomp}
\usepackage{glossaries}
\usepackage{bm}
\usepackage{enumitem}
\usepackage{subcaption}
\usepackage{multirow}
\usepackage{float}
\makeatletter
\AtBeginDocument{\DeclareMathVersion{bold}
\SetSymbolFont{operators}{bold}{T1}{times}{b}{n}
\SetSymbolFont{NewLetters}{bold}{T1}{times}{b}{it}
\SetMathAlphabet{\mathrm}{bold}{T1}{times}{b}{n}
\SetMathAlphabet{\mathit}{bold}{T1}{times}{b}{it}
\SetMathAlphabet{\mathbf}{bold}{T1}{times}{b}{n}
\SetMathAlphabet{\mathtt}{bold}{OT1}{pcr}{b}{n}
\SetSymbolFont{symbols}{bold}{OMS}{cmsy}{b}{n}
\renewcommand\boldmath{\@nomath\boldmath\mathversion{bold}}}
\makeatother

\def\BibTeX{{\rm B\kern-.05em{\sc i\kern-.025em b}\kern-.08em
    T\kern-.1667em\lower.7ex\hbox{E}\kern-.125emX}}

% \def\x{{\boldsymbol x}}
% \def\y{{\boldsymbol y}}
% \def\z{{\boldsymbol z}}
% \def\c{{\boldsymbol c}}
% \def\K{{\boldsymbol \kappa}}
\newcommand{\x}{\ifmmode\bm{x}\else\textbf{x}\fi}
\newcommand{\y}{\ifmmode\bm{y}\else\textbf{y}\fi}
\newcommand{\z}{\ifmmode\bm{z}\else\textbf{z}\fi}
\newcommand{\C}{\ifmmode\bm{c}\else\textbf{c}\fi}
\newcommand{\K}{\ifmmode\bm{K}\else\textbf{K}\fi}


%Your document starts from here ___________________________________________________
\begin{document}
% acronyms go here

\newacronym{av}{AV}{autonomous vehicle}
\newacronym{nn}{NN}{neural network}
\newacronym{tsp}{TSP}{Travelling Salesman problem}
\newacronym{vrp}{VRP}{Vehicle Routing Problem}
\newacronym{cvrp}{CVRP}{Capacitated Vehicle Routing Problem}
\newacronym{rl}{RL}{reinforcement learning}
\newacronym{gnn}{GNN}{Graph Neural Net}
\newacronym{gat}{GAT}{Graph Attention Net}
\newacronym{rnn}{RNN}{Recurrent Neural Net}
\newacronym{drl}{DRL}{Deep Reinforcement Learning}
\newacronym{mlp}{MLP}{Multi-Layer Perceptron}

\newacronym{co}{CO}{Combinatorial Optimization}
\newacronym{ndp}{TNDP}{Transit Network Design Problem}
\newacronym{fsp}{FSP}{Frequency-Setting Problem}
\newacronym{dfsp}{DFSP}{Design and Frequency-Setting Problem}
\newacronym{sp}{SP}{Scheduling Problem}
\newacronym{tp}{TP}{Timetabling Problem}
\newacronym{ndsp}{NDSP}{Network Design and Scheduling Problem}

\newacronym{mod}{MoD}{Mobility on Demand}
\newacronym{amod}{AMoD}{Autonomous Mobility on Demand}
\newacronym{imodp}{IMoDP}{Intermodal Mobility-on-Demand Problem}
\newacronym{matsim}{MATSim}{Multi-Agent Transport Simulation}
\newacronym{od}{OD}{Origin-Destination}
\newacronym{csa}{CSA}{Connection Scan Algorithm}
\newacronym{mdp}{MDP}{Markov Decision Process}
\newacronym{dqn}{DQN}{Deep Q-Networks}
\newacronym{acer}{ACER}{Actor-Critic with Experience Replay}
\newacronym{ppo}{PPO}{Proximal Policy Optimization}
\newacronym{artm}{ARTM}{Metropolitan Regional Transportation Authority}
\newacronym{stl}{STL}{Soci\'et\'e de Transport de Laval}
\newacronym{cda}{CDA}{Census Dissemination Area}



\history{Date of publication xxxx 00, 0000, date of current version xxxx 00, 0000.}
\doi{10.1109/ACCESS.2024.0429000}

\title{Point Cloud Geometry Scalable Coding Using a Resolution and Quality-conditioned Latents Probability Estimator}
\author{\uppercase{Daniele Mari}\authorrefmark{1},
\uppercase{André F. R. Guarda}\authorrefmark{2}, \IEEEmembership{Member, IEEE}, \uppercase{Nuno M. M. Rodrigues}\authorrefmark{2}\authorrefmark{4}, \IEEEmembership{Senior Member, IEEE}, \uppercase{Simone Milani}\authorrefmark{1}, \IEEEmembership{Member, IEEE}, and \uppercase{Fernando Pereira}\authorrefmark{2}\authorrefmark{3}, \IEEEmembership{Fellow, IEEE}}

\address[1]{University of Padova, Department of Information Engineering, Padova, 35131 Italy (e-mail: name.surname@dei.unipd.it)}
\address[2]{Instituto de Telecomunicações, 1049-001 Lisbon, Portugal}
\address[3]{Instituto Superior Técnico, Universidade de Lisboa, 1049-001 Lisbon, Portugal)}
\address[4]{ESTG, Politécnico de Leiria, 2411-901 Leiria, Portugal}
\tfootnote{This work was partially supported by the European Union under the Italian National Recovery and Resilience Plan (NRRP) of NextGenerationEU, with a partnership on “Telecommunications of the Future” (PE00000001 - program “RESTART”). This study was also carried out within the Future Artificial Intelligence Research (FAIR) and received funding from the European Union Next-GenerationEU (PNRR - Piano Nazionale di Ripresa e Resilienza - Missione 4, Componente 2, Investimento 1.3 - D.D. 1555 11/10/2022, PE00000013). This work was also funded by the Fundação para a Ciência e a Tecnologia (FCT, Portugal) through the research project PTDC/EEI-COM/1125/2021, entitled “Deep Learning-based Point Cloud Representation”, and by FCT/MECI through national funds and when applicable co-funded EU funds under UID/50008: Instituto de Telecomunicações. This manuscript reflects only the authors’ views and opinions, neither the EU nor the European Commission can be considered responsible for them. Daniele Mari's activities were supported by Fondazione CaRiPaRo under the grants “Dottorati di Ricerca” 2021/2022.}

\markboth
{D. Mari \headeretal: Preparation of Papers for IEEE TRANSACTIONS and JOURNALS}
{D. Mari \headeretal: Preparation of Papers for IEEE TRANSACTIONS and JOURNALS}

\corresp{Corresponding author: Daniele Mari (e-mail: daniele.mari@dei.unipd.it).}


\begin{abstract}
In the current age, users consume multimedia content in very heterogeneous scenarios in terms of network, hardware, and display capabilities. A naive solution to this problem is to encode multiple independent streams, each covering a different possible requirement for the clients, with an obvious negative impact in both storage and computational requirements. These drawbacks can be avoided by using codecs that enable scalability, i.e., the ability to generate a progressive bitstream, containing a base layer followed by multiple enhancement layers, that allow decoding the same bitstream serving multiple reconstructions and visualization specifications.
While scalable coding is a well-known and addressed feature in conventional image and video codecs, this paper focuses on a new and very different problem, notably the development of scalable coding solutions for deep learning-based \gls{pc} coding. The peculiarities of this 3D representation make it hard to implement flexible solutions that do not compromise the other functionalities of the codec.
This paper proposes a joint quality and resolution scalability scheme, named \gls{esqh}, that, contrary to previous solutions, can model the relationship between latents obtained with models trained for different RD tradeoffs and/or at different resolutions.
Experimental results obtained by integrating \gls{esqh} in the emerging JPEG Pleno learning-based PC coding standard show that \gls{esqh}  allows decoding the PC at different qualities and resolutions with a single bitstream while incurring only in a limited RD penalty and increment in complexity w.r.t. non-scalable JPEG PCC that would require one bitstream per coding configuration.
\end{abstract}

\begin{keywords}
Point cloud geometry coding, JPEG Pleno PCC, deep learning-based codec, scalable coding.  
\end{keywords}

\titlepgskip=-21pt

\maketitle

\section{Introduction}
\label{sec:intro}


\ps{Challenges of technology scaling}

The growing demand for computing performance has always been met by increasing the number of transistors per chip, which is only possible due to CMOS technology scaling.
However, as we keep pushing the boundaries of technology scaling, we encounter multiple challenges.
Firstly, whenever we transition to a more advanced technology node, the non-recurring cost due to physical design, verification, software, mask sets, and prototyping almost doubles \cite{cost-tech-node}.
As a result, designing a chip in an advanced technology node is only economically viable if the chip is manufactured in vast quantities.
Secondly, many chip components such as I/O drivers, analog circuits, or \gls{srams} have reached their scaling limit.
This means that we cannot shrink these components further, even if we use a more advanced technology with a smaller feature size.
Thirdly, advanced technology nodes suffer from high defect rates, diminishing the yield and inflating the recurring cost.
To tackle these challenges, new chip-design paradigms have been developed.

\ps{Why 2.5D integration?}

One of these new paradigms is 2.5D integration, where multiple silicon dies called chiplets are integrated into the same package.
Once designed, a single chiplet can be reused in multiple 2.5D stacked chips, which increases the ratio of production volume to non-recurring cost.
Another advantage is that multiple chiplets - fabricated in different technologies - can be integrated into the same package.
This means that only components that can take full advantage of technology scaling are built in bleeding-edge technologies.
Components that have reached their scaling limit are fabricated in more mature and hence less costly technology nodes.
Furthermore, chiplets are smaller than monolithic chips.
Therefore, manufacturing chiplets results in less silicon area loss due to fabrication defects and hence a higher yield.
Due to these economic advantages, chip vendors such as AMD \cite{amd-chiplet} and NVIDIA \cite{chiplet-book} have adopted the 2.5D integration paradigm.  

\ps{Challenges of 2.5D integration}

An important challenge when designing 2.5D stacked chips is the construction of a low-latency and high-throughput \gls{ici}. 
To build an \gls{ici}, we connect different chiplets using \gls{d2d} links.
These links are fabricated in an organic package substrate, silicon bridge, or silicon interposer, and they are connected to the chiplets using \gls{c4} bumps or microbumps.
The number of bumps per chiplet is limited, and so is the bandwidth of \gls{d2d} links.
In addition to having lower bandwidth than links in monolithic chips, \gls{d2d} links also have higher latency.
This latency is caused by wire delay and by \gls{phys} that are necessary in both the sending and the receiving chiplet.
\gls{phys} are needed to convert between protocols, voltage levels, and frequencies, which are usually different between on-chiplet links and \gls{d2d} links.
Due to these limitations, the \gls{ici} can quickly become a bottleneck.

\ps{How we solve these challenges differently than the related work does.}

Existing approaches to maximize the performance of the \gls{ici} either optimize the placement of chiplets (with potentially heterogeneous shapes) for a predetermined \gls{ici} topology 
\cite{ho,liu,seemuth,eris,osmolovskyi,tap25d,chiou}, select one topology out of a set of candidates \cite{coskun-1, coskun-2}, or they optimize the \gls{ici} topology for a 2D grid of homogeneously shaped chiplets on an active interposer \cite{butterdonut, cluscross, kite}.
To the best of our knowledge, there is no prior work on \gls{ici} topologies for chips with heterogeneously shaped chiplets or with passive silicon interposers or silicon bridges.
To fill this gap, we propose \name, a novel optimization methodology to jointly optimize the chiplet placement and \gls{ici} topology of such architectures.
\ifnb
\else
\newpage
\fi

\ps{Details on \name~and the key idea}

The key idea is as follows: 
We optimize the chiplet placement without a predetermined topology.
For each placement generated by an optimization algorithm, we infer a placement-based \gls{ici} topology by connecting chiplets that are in close proximity in that specific placement.
We then compute the latency and throughput of this combination of placement and topology for different traffic types.
These latencies and throughputs together with the total chip area are used to compute a user-defined quality-score of the placement, which is returned to the optimization algorithm.
Based on this quality score, the algorithm can further optimize the placement.
By following this iterative process, we jointly optimize the chiplet placement and the \gls{ici} topology.

\ps{Short evaluation-summary}

We provide our open-source framework implementing the proposed placement and topology co-optimization methodology, which we evaluate using both synthetic traffic and traffic traces.
A 2D grid of chiplets with a mesh topology is used as a baseline since many proposals for 2.5D stacked chips \cite{dataflow_accel_dnn, cifher, simba, hecaton, dojo} use such an architecture.
We reduce the latency of synthetic L1-to-L2 and L2-to-memory traffic, the two most important traffic types for cache coherency traffic, by up to 28\% and 62\% respectively.
For real traffic traces, we reduce the average packet latency for almost all traces and architectures considered (reduced by an 8\% or 18\% on average depending on the configuration of \gls{phys} within a chiplet).

\section{Related Works}
\label{sec:related}
State-of-the-art PCC encompasses various approaches, ranging from traditional signal processing techniques to modern learning-based solutions. Among the conventional approaches, the most relevant are G-PCC and V-PCC \cite{graziosi2020overview}, the two MPEG standards for \gls{pcc}.
G-PCC, or Geometry-based Point Cloud Compression, leverages octree representations for efficient geometry coding, and uses predictive or hierarchical transforms for attribute coding. On the other hand, V-PCC, or Video-based Point Cloud Compression, projects 3D PC data into the 2D domain, creating images that represent the geometry and texture information, which are compressed using very efficient and established video codecs like HEVC and VVC \cite{bross2021overview}.
While G-PCC inherently supports resolution scalability for both geometry and attributes, achieving scalability in V-PCC presents challenges due to its reliance on video coding frameworks. Although MPEG has initiated investigations into various scalability techniques for V-PCC \cite{vpccscal}, these features remain to be specified in the current version of the standard.

More recently, the advent of \gls{dl} has revolutionized PC compression, yielding numerous high-performing solutions \cite{guarda2024jpeg,quach2020improved,wang2021lossy, wang2022sparse, liu2022pcgformer}. Nevertheless, among the many \gls{dl}-based PCC solutions, few approaches address any form of scalability. DL-PCSC \cite{guarda2020point} implements quality scalability by channelwise partitioning of latent representations, enabling progressive quality enhancement through incremental transmission. However, this approach faces limitations: the requirement for zero-padding untransmitted latents constrains the latent space design, and the reduced latent space dimensionality at lower rates leads to reduced modeling capabilities \cite{balle2018variational}. These constraints significantly impact the rate-distortion performance making scalability less appealing.

GRASP-Net \cite{pang2022grasp} offers an alternative approach, by implementing a \gls{dl}-based enhancement layer atop a G-PCC base layer. However, scalability is limited to this two-layer structure, and the extremely low-resolution base layer may prove impractical for real-world applications.

The work by Ulhaq et al. \cite{ulhaq2024scalable} implements scalability by adapting content for human and machine consumption. In particular, the base layer can be used to solve computer vision tasks (e.g. PC classification) while the enhancement layer allows reconstructing the PC for human visualization. This approach thus addresses scalability requirements that are different from those explored in this work.

SparsePCGC \cite{wang2022sparse} and its successor, Unicorn \cite{wang2024versatile1,wang2024versatile2}, represent significant advances in resolution scalability for PCC, due to their inherently multiscale nature. At the encoder side, Unicorn employs a hierarchical downscaling approach, encoding the information necessary for losslessly upscaling it at the decoder.
When this enhancement information is unavailable, the decoder employs a lossy thresholding strategy for upscaling.
By dividing the bitstream in different enhancement layers, required to upscale the PC, Unicorn achieves resolution scalability. Furthermore, Unicorn has a very competitive coding performance when compared with other PC codecs. The intrinsic scalability mechanism of Unicorn, which results from its architecture, contrasts with \gls{esqh}, which offers a modular, plug-and-play solution applicable to various codecs.

Additionally, it is also important to mention that recently the MPEG group issued the call for proposal for the new AI-based PCC \cite{aigccfp} which is currently under development.
\section{JPEG Pleno Point Cloud Geometry Codec and Previous Extension for Quality Scalability}
\label{sec:vm}

\begin{figure*}
    \centering
    \includegraphics[width=.9\textwidth]{figures/jpeg_pleno_high_level.pdf}
    \caption{High-level scheme of the coding procedure for PC geometry in \gls{jpeg-pcc}.}
    \label{fig:high-level-scheme}
\end{figure*}

\label{sec:jpegpleno}
The \gls{esqh} method proposed in this paper is implemented on top of the verification model software for the \gls{jpeg-pcc} standard \cite{jpeg-pleno}, which will be presented next in this section. After that, \gls{sqh} \cite{mari2024point}, a previously proposed solution for implementing quality scalability in \gls{jpeg-pcc} for geometry coding, which served as the basis for the \gls{esqh} method, will be described.

\subsection{JPEG Pleno Point Cloud Codec}

\gls{jpeg-pcc} is the JPEG standard for PC coding, which uses a learning-based approach for coding both PC geometry and color attributes \cite{guarda2023point}.
The geometry coding in \gls{jpeg-pcc} utilizes a deep learning model structured as an autoencoder, complemented with a variational autoencoder model that determines a mean and scale hyperprior that improves the performance for entropy coding the compressed domain latent representation \cite{minnen2018joint}. To enhance compression performance, particularly for sparse PCs and lower-rate coding scenarios, \gls{jpeg-pcc} incorporates two additional tools:
\begin{enumerate}
    \item A down-scaling module using a scaling factor parameter, $sf$.
    \item A deep learning-based super-resolution (SR) module to improve reconstruction quality when down-scaling is applied.
\end{enumerate}

\gls{jpeg-pcc} adopts a sparse tensor representation \cite{choy20194d} for geometry coding, offering advantages in both computational complexity and rate-distortion performance. In this representation, PCs are described as a tuple $\bm{x}=(\bm{x}_C, \bm{x}_F)$, where $\bm{x}_C$ represents the coordinates of non-empty voxels, and $\bm{x}_{F}$ denotes the corresponding features (initially set to "1" to indicate occupied voxels).

For encoding the color attributes, JPEG
PCC projects texture patches onto an image (similarly to V-PCC \cite{V-PCC}) which is then coded using the emerging JPEG AI codec \cite{jpeg-ai}.
Since the focus of this paper lies in geometry coding, the remaining of this section will focus only on this component.

A high-level description of the full geometry coding and decoding procedures is shown in Fig.~\ref{fig:high-level-scheme}. Specifically, to encode the geometry of a point cloud $\bm{P} \in \mathbb{R}^3$, the encoder performs the following operations:


\begin{figure*}
    \centering
    \includegraphics[width=.9\textwidth]{figures/jpeg-pcc.pdf}
    \caption{Model architecture of the deep learning based codec in \gls{jpeg-pcc} (\gls{dl}-based Geometry Encoder and \gls{dl}-based Geometry Decoder in Fig.~\ref{fig:high-level-scheme}).}
    \label{fig:dl-scheme}
\end{figure*}

\begin{enumerate}[label=E\arabic*.]
    \item \textit{Downscaling}: The input PC is downscaled by a factor $sf$ through the operation $\boldsymbol{P}^\prime = \lceil \boldsymbol{P}/sf \rfloor$.
    \item \textit{Block Split}: The downscaled points are divided into non-overlapping blocks $\x_{l, C} \in \mathbb{R}^3, l \in {1, \dots, N}$ of size $bs$, such that $\boldsymbol{P}^\prime = \bigcup_{l=1}^{N} \x_{l, C}$.
    \item \textit{Sparse Tensor Construction}: For each block, a sparse tensor representation $\x_l = (\x_{l, C}, \x_{l, F})$ is created, where $\x_{l, F}$ contains ones to indicate occupied voxels.
    \item \textit{DL-Based Encoding}: The blocks are processed through the deep learning-based coding procedure to generate the bitstream
    \item \textit{Distortion Optimization}: Two parameters per block, $k_l$ and $k_{SR, l}$, are computed and added to the bitstream. These parameters represent the optimal number of points to be retained in the decoded block (with and without super-resolution) to minimize a chosen distortion metric.
\end{enumerate}

At the decoder side, the PC reconstruction process consists of the following operations:
\begin{enumerate}[label=D\arabic*.]
    \item \textit{DL-Based Decoding}: The decoder reconstructs the blocks $\hat{\x}_l$ by inputting the compressed domain latent representation, extracted from the bitstream, in the DL-based decoder.
    \item \textit{Top-K Points Selection}: For each decoded block $\hat{\x}_l$, only the $k_l$ points with the highest occupancy probabilities are retained, ensuring optimal point selection.
    \item \textit{Upscaling}: The blocks are upscaled according to scaling factor $sf$ (included in the bitstream) to restore the original spatial resolution.
    \item \textit{Super-Resolution}: When super-resolution is enabled ($SR=1$), the upscaled blocks are processed through the SR network to obtain enhanced blocks $\hat{\x}_{SR, l}$.
    \item \textit{Post-SR Top-K Point Selection}: From each super-resolved block, the $k_{SR,l}$ points with the highest probability values are selected, ensuring optimal point selection.
    \item \textit{Block Merge}: Finally, all processed blocks are merged to reconstruct the complete point cloud geometry $\hat{\boldsymbol{P}}$.
\end{enumerate}

The deep learning-based encoding (and decoding) process for each block $x_l$, is illustrated in Fig.~\ref{fig:dl-scheme}. It consists of the following sequence of operations:

\begin{enumerate}[label=E\arabic*.]
    \item \textit{Latents Generation}: Generate latents $\y_l = (\y_{l, C}, \y_{l, F})$ through the analysis transform $\mathcal{G}_a$, expressed as $\y_l=\mathcal{G}_a(\x_l)$.
    \item \textit{Coordinates Encoding}: code the latent coordinates $\y_{l, C}$ using an octree encoder to generate the coordinates bitstream.
    \item \textit{Hyper-Latents Generation}: Generate hyper-latents $\z_l$ using the hyper-analysis transform $\mathcal{HG}_a$, where $\z_l=\mathcal{HG}_a(\y_l)$.
    \item \textit{Hyper-Latents Quantization}: Quantize the hyper-latent features to obtain $\hat{\z}_{l, F} = \lceil \z_{l, F} \rfloor$.
    \item \textit{Entropy Coding}: Apply rANS entropy coding to the hyper-latents according to a fully factorized prior $p(\hat{\z}_{l, F})$ to generate the hyper-latents bitstream.
    \item \textit{Sparse Tensor Construction}: Reconstruct the quantized hyper-latents' sparse representation as $\hat{\z}_l = (\z_{l, C}, \hat{\z}_{l, F})$.
    \item \textit{Latents Distribution Estimation}: Process $\hat{\z}_l$ through hyper-synthesis transforms $\mathcal{HG}_{s, \mu}$ and $\mathcal{HG}_{s, \sigma}$ to estimate Gaussian parameters $\boldsymbol{\mu}_l = \mathcal{HG}_{s, \mu}(\hat{\z}_l), \boldsymbol{\sigma }_l= \mathcal{HG}_{s, \sigma}(\hat{\z}_l)$. 
    \item \textit{Residual Encoding}: Calculate and encode quantized residuals $\boldsymbol{r}_l= \lceil \y_{l,F} - \boldsymbol{\mu}_l \rfloor$ using $\mathcal{N}(\boldsymbol{0}, \boldsymbol{\sigma}_l)$ to produce the final latents bitstream.
\end{enumerate}

Conversely, a receiver that needs to decode the blocks from the bitstream will have to:

\begin{enumerate}[label=D\arabic*.]
    \item \textit{Coordinates Decoding}: Losslessly decode $\y_{l, C}$ from the coordinates bitstream.
    \item \textit{Hyper-latents Decoding}: Entropy decode $\hat{\z}_{l, F}$ from the hyper-latents bitstream using the probability distribution $p(\hat{\z}_{l, F})$.
    \item \textit{Coordinates Down-scaling}: Down-scale $\y_{l, C}$ by a factor of 4 (as determined by the stride parameters in $\mathcal{HG}_a$'s convolutional layers) to obtain $\z_{l, C}$.
    \item \textit{Hyper-Latents Sparse Tensor Construction}:  Build the sparse representation of hyper-latents as $\hat{\z}_l = (\z_{l, C}, \hat{\z}_{l, F})$.
    \item \textit{Latents Distribution Estimation}:  Compute Gaussian parameters using hyper-synthesis transforms as $\boldsymbol{\mu}_l = \mathcal{HG}_{s, \mu}(\hat{\z}_l), \boldsymbol{\sigma }_l= \mathcal{HG}_{s, \sigma}(\hat{\z}_l)$.
    \item \textit{Residuals Decoding}: Entropy decode $\boldsymbol{r}_l$ from the latents' bitstream using $\mathcal{N}(\boldsymbol{0}, \boldsymbol{\sigma}_l)$.
    \item \textit{Latent Features Reconstruction}: Recover the latent features $\hat{\y}_{l, F} = \boldsymbol{\mu}_l + \boldsymbol{r}_l$.
    \item \textit{Latents Sparse Tensor Construction}: Reconstruct the sparse representation of latents as $\hat{\y}_l = (\y_{l, C}, \hat{\y}_{l, F})$.
    \item \textit{Block Reconstruction}: Apply the synthesis transform $\mathcal{G}_s$ to the decoded latents to determine the probability for the occupancy state of each voxel in the reconstruct the block: $\hat{\x}_l = \mathcal{G}_s(\hat{\y}_l)$. 
\end{enumerate}

The model training follows an end-to-end approach incorporating all previously described operations except for two differences: quantization is replaced by a differentiable approximation and entropy coding is removed, to ensure full model differentiability. The training utilizes a rate-distortion optimization framework defined by the loss function:
\begin{equation}
    \mathcal{L}(\bm{x}, \hat{\bm{x}}, \bm{y}, \bm{z}) = \mathcal{D}(\bm{x}, \hat{\bm{x}}) + \lambda\mathcal{H}(\bm{y}, \bm{z}),
\end{equation}
where $\mathcal{D}(\cdot, \cdot)$ is the distortion, measured as the focal loss \cite{lin2017focal}, $\mathcal{H}(\cdot, \cdot)$ denotes the entropy of the bitstream components under the probability distributions $p(\hat{\bm{z}})$ and $p(\bm{y}|\hat{\bm{z}})$, and $\lambda$ controls the rate-distortion trade-off.
Generally, one model is trained for each RD point corresponding to one value of $\lambda$. In \gls{jpeg-pcc}, five different coding models are trained to support the defined range of tradeoffs.
The training procedure is carried out by sequentially spanning the chosen values of $\lambda \in \{0.0025, 0.005, 0.01, 0.025, 0.05\}$, using the checkpoint for the previous $\lambda$ as a starting point, progressively moving from the lowest value (highest rate/quality) to the highest one (lowest rate/quality). These five models naturally define a quality parameter $qp \in \{1, \dots , 5\}$, with $qp = 1$ corresponding to $\lambda = 0.05$ (lowest rate/quality) and $qp = 5$ to $\lambda = 0.0025$ (highest rate/quality).

\subsection{Scalable Quality Hyperprior}

The \gls{esqh} method proposed in this paper follows a previous work \cite{mari2024point} that introduced a quality scalability algorithm, known as \gls{sqh}. \gls{sqh} constructs a quality scalable bitstream by leveraging information from latents $\y_i$ obtained at a lower \gls{qp} ($qp=i$) to predict probability distributions for latents $\y_j$ at a higher \gls{qp} ($qp=j$).

Starting from a low-quality base layer of latents $\y_i$, which have already been encoded, the encoder must execute the following sequence of steps to generate a new enhancement layer:

\begin{enumerate}[label=E\arabic*.]
    \item \textit{Higher Quality Latents Generation}: Generate new latents $\y_{j}$ using the \gls{jpeg-pcc} coding model with $qp = j > i$.
    \item \textit{Latents Distribution Estimation}: Predict the means and standard deviations of the
latents $\y_{j}$ based on the previous latents $\y_{i}$, using the \gls{dl}-based \gls{qulpe} model
(detailed in \cite{mari2024point}) as $\boldsymbol{\mu}_j$, $\boldsymbol{\sigma}_j = QuLPE(\hat{\y}_i, i, j)$, under the assumption of independently distributed Gaussian latents, $P(\y_j|\hat{\y}_i)$.
    \item \textit{Entropy Coding}: Generate the \gls{sqh} bitstream by entropy encoding $\y_j$ using $\boldsymbol{\mu}_j$, $\boldsymbol{\sigma}_j$.
\end{enumerate}

\gls{sqh} employs a Mean and Scale Hyperprior entropy model, analogous to \gls{jpeg-pcc}. The key distinction lies in \gls{sqh}'s utilization of previously decoded latents $\hat{\y}_i$ as side information, rather than hyper-latents $\hat{\z}_j$.


To reconstruct the higher rate/quality PC, the decoder, which can access the base layer information $\hat{\y}_i$, performs the following decoding procedure:
\begin{enumerate}[label=D\arabic*.]
    \item \textit{Latents Distribution Estimation}: Derive $\boldsymbol{\mu}_{j}$, $\boldsymbol{\sigma}_{j} = QuLPE(\hat{\y}_i, i, j)$ from the base layer information $\hat{\y}_i$ using the \gls{qulpe} model.
    \item \textit{Higher Quality Latents Decoding}: Decode the higher quality latents $\hat{\y}_{j}$ by applying a rANS decoder to the \gls{sqh} bitstream using the estimated distribution.
    \item \textit{Higher Quality PC Reconstruction}: Reconstruct the higher quality PC by processing the decoded latents through the \gls{jpeg-pcc} synthesis transform as $\hat{\x}_{j} = \mathcal{G}_{s, j}(\hat{\y}_{j})$.
    \item \textit{Super Resolution}: If specified in the coding parameters the Super Resolution model is used to enhance the decoded blocks
\end{enumerate}

While \gls{sqh} effectively handles quality scalability through latents refinement, it faces limitations when dealing with \gls{jpeg-pcc}'s downscaling strategy. The challenge arises because varying $sf$ produces latents at different resolutions, a scenario not supported by \gls{sqh}'s U-Net-based \gls{qulpe} model, which requires consistent input-output dimensions. This architectural constraint, coupled with the absence of a multi-resolution handling strategy, restricts \gls{sqh}'s practical applicability. 

The next sections introduce and evaluate \gls{esqh}, an enhanced framework that addresses these limitations by enabling joint quality and resolution scalability in the latent domain. These functional advantages are relevant in the framework of the \gls{jpeg-pcc} codec, but also for the generalization of the \gls{esqh} method for other autoencoder-based codecs.

 \section{Method}
\label{sec:method}











Given a set $\{x_{1_i},c_i\}_{i=1}^m$ of input samples and their corresponding conditioning states, our goal is to construct a flow-matching model that samples from $q(x_1|c)$ that start from our conditional prior distribution (CPD). 

\subsection{Flow Matching from Conditional Prior Distribution}
\label{sec:conditional_fm_joint}

We generalize the framework of  Sec.~\ref{sec:flow_matching} to a construction that uses an arbitrary conditional joint distribution of $\rho(x_0, x_1, c)$ which satisfy the marginal constraints:
\begin{equation*}
\label{eq:conditional_marginal}
    \int \rho(x_0, x_1, c)dx_0 = q(x_1, c),  \int \rho(x_0, x_1, c)dx_1dc = p(x_0)
\end{equation*}
Then, building on flow matching, we propose to modify the conditional probability path so that at $t=0$, we define:
\begin{equation}
    \rho_0(x_0|x_1, c) = p(x_0|x_1, c) 
\end{equation}
where $p(x_0|x_1, c)$ is the conditional distribution $\frac{\rho(x_0, x_1, c)}{q(x_1, c)}$. 
Using this construction, we satisfy the boundary condition of Eq.~\ref{eq:boundary_conditions}: 
\begin{align}
    \rho_0(x_0) &= \int\rho_0(x_0|x_1, c)q(x_1, c)dx_1dc  \\
                &=  \int p(x_0|x_1, c)dx_1dc = p(x_0)
\end{align}




The conditional probability path $\rho_t(x|x_1, c)$ does not need to be explicitly formulated. Instead, only its corresponding conditional vector field $u_t(x|x_1, c)$ needs to be defined such that points $x_0$ drawn from the conditional prior distribution $\rho_0(x_0|x_1, c) $, reach $x_1$ at $t=1$, i.e., reach distribution $\rho_1(x|x_1, c) = \delta(x - x_1)$.  We thus purpose the \emph{Conditional Generation Joint FM} $\gL_{\rm cgjfm}(\theta)$ objective:
\begin{equation}\label{eq:conditionl_joint_cfm}
    \mathbb{E}_{t\sim \mathcal{U}(0,1), q(x_0,x_1,c)} \|v_\theta(t, x, c) - u_t(x | x_1, c)\|^2
\end{equation}
where $x = \psi_t(x_0|x_1,c)$.
Training only involves sampling from $q(x_0,x_1,c)$ and does not require explicitly defining the densities $q(x_0,x_1,c)$ and $\rho_t(x|x_1,c)$.
We note that this objective is reduced to the CGFM objective Eq.~\ref{eq_conditional_generative_fm_objective} when $q(x_0,x_1,c) = q(x_1, c)p(x_0)$.

\subsection{Conditional Prior Distribution}
\label{sec:prior_distribution}

We now describe our choice of a condition-specific prior distribution. 
When choosing a conditional prior distribution we want to adhere to the following design principles:
(i) \emph{Easy to sample}: can be efficiently sampled from.
(ii) Well represents the target conditional modes. 
We design a condition-specific prior distribution based on a parametric \emph{Mixture Model} where each mode of the mixture is correlated to a specific conditional distribution $p(x_1|c)$. 
Specifically, we choose the prior distribution to be the following, \emph{easy to sample}, \emph{Gaussian Mixture Model} (GMM):
\begin{equation}\label{eq:gmm_formula}
    p_0 = \mathrm{GMM}(\gN(\mu_i, \Sigma_i)_{i=1}^n, \pi)
\end{equation}

where $\pi\in\R^n$ is a probability vector associated with the number of conditions $n$ (could be $\infty$) and $\mu_i, \Sigma_i$ are parameters determined by the conditional distribution $q(x_1|c_i)$ statistics, \emph{i.e.} 
 \begin{equation}\label{eq:gmm_parameters}
     \mu_i = \E[x_1|c_i], \quad \Sigma_i = \mathrm{cov}[x_1|c_i]
 \end{equation}
To sample from the marginal distribution $p(x_0|x_1, c_i)$, we sample from the cluster $\gN(\mu_i, \Sigma_i)$ associated with the condition $c_i$.

\noindent \textbf{Obtaining a Lower Global Truncation Error.} \quad 
Our CPD fits a GMM to the data distribution in a favorable setting, where the association between samples and clusters is given. 
\begin{equation}\label{eq:wasserstein_definition}
    d_1 \left(X, Y \right) \coloneqq \sup_{h \in \mathrm{Lip_1}} \mathbb{E}[h(X) - h(Y)] .
\end{equation}

In this process, we fit a dedicated Gaussian distribution to data points with the same condition. If the latter are close to being unimodal, this approximation is expected to be tight, in terms of the average distances between samples from the condition data mode and the fitted Gaussian. 
Tab.~\ref{tab:wasserstein_table} provides the average distances between pairs of samples from the prior and data distributions (i.e. the \emph{transport cost}) of CondOT~\cite{lipman2022flow}, BatchOT~\cite{pooladian2023multisample} and our CPD over the ImageNet-64~\cite{deng2009imagenet} and MS-COCO~\cite{lin2014microsoft} datasets. 
As expected, BatchOT which minimizes this exact measure within mini-batches, obtains better scores than the naïve pairing used in CondOT, while our CPD, which approximates the data using a GMM exploits the conditioning available in these datasets, and obtains considerably lower average distances.

As noted in \cite{pooladian2023multisample}, lower transport cost is generally associated with straighter flow trajectories, more efficient sampling and lower training time. We want to substantiate this claim from the viewpoint of cumulative errors in numerical integration.
Sampling from flow-based models consists of solving a time-dependent ODE of the form $\dot{x}_t =u_t(x_t)$, where $u_t$ is the velocity field. This equation is solved by the following integral $x_t = \int_{0}^t u_s(x_s)ds$, where the initial condition $x_0 $ is sampled from the prior distribution. Numerical integration over discrete time steps accumulate an error at each step $n$ which is known as the \emph{local truncation error $\tau_n$}, which accumulates into what is known as the \emph{global truncation error $e_n$}.  This error is bounded by ~\cite{suli2003introduction}
\begin{equation}
    |e_n| \leq \frac{max_j\tau_j}{hL}\big(e^{L(t_n-t_0)} - 1\big)
\end{equation}\label{eq:truncation_error_bound} 
where $h$ is the step size and $L$ is the Lipschitz constant of the velocity $u_t$. 
Accordingly, the distance between the endpoints of a path $\Delta = |x_1  - x_0|$  is given by $|\int_0^1 u_s(x_s)ds|$ which can be interpreted as the magnitude of the average velocity along the path $x_t$. Hence, the longer the path $\Delta$ is, the larger the integrated flow vector field $u_t$ is.
For example, if we scale a path uniformly by a factor $C>1$, i.e., $x_t \rightarrow C(x_t)$, we get,  $\frac{d}{dt}C(x_t) = C(u_t)$ in which case the Lipschitz constant $L$ is also multiplied by $C$.

By shortening the distance between the prior and and data distribution, as our CPD does, we lower the integration errors which permits the use of coarser integration steps, which in turn yield smaller global errors. Thus, our construction allows for fewer integration steps during sampling.

\subsubsection{Construction}


Next, we explain how we construct $p_0$ (Eq.~\ref{eq:gmm_formula}) for both the discrete case (e.g., class conditional generation) and continuous case (e.g., text conditional generation). 

\noindent \textbf{Discrete Condition.} \quad
In the setup of discrete conditional generation, we are given data $\{x_{1_i}, c_i\}_{i=1}^m$ where there are a finite set of conditions $c_i$.
We approximate the statistics of Eq.~\ref{eq:gmm_parameters} using the training data statistics. That is, we compute the mean and covariance matrix of each class (potentially in some latent represntation of a pretrained auto-encoder).  Since the classes at inference time are the same as in training, we use the same statistics at inference. 

\noindent \textbf{Continuous Condition.} \quad
While in the discrete case we can directly approximate the statistics in Eq.~\ref{eq:gmm_parameters} from the training data, in the continuous case (\emph{e.g.} text-conditional) we need to find those statistics also for conditions that were not seen during training. To this end, we first consider a joint representation space for training samples $\{x_{1_i}, c_i\}_{i=1}^m$, which represents the semantic distances between the conditions $c_i$ and the samples $x_{1_i}$. In the setting where $c_i$ is text, we choose a pretrained CLIP embedding. 
$c_i$ is then mapped to this representation space, and then mapped to the 
data space (which could be a latent representation of an auto-encoder), using a learned mapper $\gP_\theta$. 
Specifically, $\gP_\theta$ is trained to minimize the objective:
\begin{equation}
    \gL_{\rm prior}(\theta) = \mathbb{E}_{q(x_1,c)} \|\gP_\theta(E(c)) - x_1\|^2_2.
\end{equation}
where $E$ is the pre-trained mapping to the joint condition-sample space (e.g. CLIP). $\gP_\theta$ can be seen as approximating $\E[x_1|c]$, which is used as the mean for the condition specific Gaussian.  
At inference, where new conditions (e.g., texts) may appear, we first encode the condition $c_i$ to the joint representation space (e.g., CLIP) followed by $\gP_\theta$. This mapping provides us with the center $\mu_i$ of each Gaussian. %
We also define $\Sigma_i = \sigma_i^2\mathrm{I}$ where $\sigma_i$ is a hyper-parameter, ablated in Sec.~\ref{sec:results_quantitative} 

\subsection{Training and Inference}

Given the prior $p_0$ (either using the data statistics or by training $\gP_\theta$), for each condition $c$, we have its associated Gaussian parameters $\mu_c$ and $\Sigma_c$. The map $\psi_t(x|x_1,c)$ must be defined in order to minimize Eq.~\ref{eq:conditionl_joint_cfm} above. This corresponds to the interpolating maps between this Gaussian at $t=0$ and a small Gaussian around $x_1$ at $t=1$, defined by:
\begin{align}
    \psi_{t}(x|x_1,c) &= \sigma_t(x_1,c)x + \mu_t(x_1,c), \\ 
    \sigma_t(x_1,c) &= t (\sigma_{\min} \mathrm{I}) + (1-t)\Sigma_{c}^{1/2}, \quad \text{and} \\
    \mu_t(x_1,c) &= t x_1 + (1-t) \mu_c.
\end{align}
This results in the following target flow vector field 
\begin{equation*}
    u_t(\psi_{t}(x|x_1,c)) = \frac{d}{dt}\psi_t (x|x_1,c)  =   \big(\sigma_{\min}  \mathrm{I} - \Sigma_c^{1/2}\big)x +  x_1 - \mu_c.
\end{equation*}

During inference we are given a condition $c$ and want to sample from $q(x_1|c)$. Similarly to the training, we sample $x_0\sim p(x_0|c)$ and solve the ODE 
\begin{equation}
    \frac{d}{dt} \psi_t(x) = v_\theta \left(t, \psi_t(x), c \right), \quad \psi_0(x) = x_0
\end{equation}
Training and implementation details are in the appendix.








\begin{table*}[t]
    \centering
    \resizebox{\textwidth}{!}{
\begin{tabular}{l|rrllrrll}
\toprule
\textbf{Dataset} & \multicolumn{4}{c}{\textbf{GSM8K}} & \multicolumn{4}{c}{\textbf{MATH}} \\
\cmidrule(lr){1-1} \cmidrule(lr){2-5} \cmidrule(lr){6-9}
\textbf{Method} & Acc & Len & Rel. Acc & Rel. Len & Acc & Len & Rel. Acc & Rel. Len \\
\midrule
\multicolumn{9}{l}{\textit{Zero-Shot Prompting}} \\
\midrule
\hspace{12pt}Baseline & 78.06 & 241.87 & 100.00 \small{(0.00)} & 100.00 \small{(0.00)} & 46.40 & 480.37 & 100.00 \small{(0.00)} & 100.00 \small{(0.00)} \\
\hspace{12pt}Be Concise & 77.98 & 214.87 & 99.85 \small{(1.18)} & 88.46 \small{(10.37)} & 47.76 & 446.09 & 102.71 \small{(7.59)} & 92.66 \small{(7.46)} \\
\hspace{12pt}Hand Crafted 2 (ours) & 76.72 & 184.13 & 98.27 \small{(3.67)} & 77.10 \small{(22.27)} & 46.84 & 404.85 & 101.62 \small{(4.79)} & 85.26 \small{(15.97)} \\
\midrule
\multicolumn{9}{l}{\textit{FT - External Data}} \\
\midrule
\hspace{12pt}Direct Answer & 19.70 & 3.17 & 24.88 \small{(5.03)} & 1.36 \small{(0.40)} & 15.08 & 6.98 & 35.16 \small{(10.34)} & 1.44 \small{(0.73)} \\
\hspace{12pt}Human CoT & 65.73 & 127.85 & 83.82 \small{(7.28)} & 54.95 \small{(13.17)} & 33.88 & 243.54 & 75.61 \small{(13.56)} & 53.14 \small{(13.87)} \\
\hspace{12pt}GPT4o CoT & 76.36 & 156.24 & 97.65 \small{(3.63)} & 67.60 \small{(16.70)} & 40.44 & 399.80 & 90.52 \small{(15.07)} & 87.21 \small{(22.22)} \\
\midrule
\multicolumn{9}{l}{\textit{FT - Best-of-N Self-Generation}} \\
\midrule
\hspace{12pt}Naive BoN & 77.12 & 214.22 & 98.79 \small{(1.64)} & 87.17 \small{(8.79)} & 47.64 & 433.26 & 101.74 \small{(7.04)} & 89.89 \small{(3.99)} \\
\hspace{12pt}Rational Metareasoning & 76.15 & 207.49 & 97.21 \small{(5.74)} & 84.93 \small{(5.09)} & 47.56 & 432.56 & 103.02 \small{(6.56)} & 90.56 \small{(5.25)} \\
\midrule
\multicolumn{9}{l}{\textit{FT - Few-Shot Conditioned Self-Generation (ours)}} \\
\midrule
\hspace{12pt}FS-Human & 76.66 & 161.72 & 98.06 \small{(3.28)} & 67.96 \small{(16.62)} & 46.44 & 421.54 & 99.69 \small{(6.97)} & 87.78 \small{(5.98)} \\
\hspace{12pt}FS-GPT4o & 78.07 & 175.54 & 99.94 \small{(1.69)} & 73.15 \small{(13.49)} & 47.36 & 421.21 & 101.87 \small{(5.33)} & 87.58 \small{(6.60)} \\
\hspace{12pt}FS-Self & 77.27 & 190.03 & 98.86 \small{(2.51)} & 77.51 \small{(9.18)} & 48.00 & 426.67 & 102.67 \small{(5.24)} & 88.50 \small{(4.49)} \\
\midrule
\multicolumn{9}{l}{\textit{FT - Few-Shot Conditioned Best-of-N Self-Generation (ours)}} \\
\midrule
% GPT4o Best-of-16 (Naive) & 75.48 & 153.51 & 96.56 \small{(3.79)} & 64.12 \small{(16.35)} & 47.28 & 367.49 & 101.50 \small{(9.81)} & 76.96 \small{(11.42)} \\
\hspace{12pt}FS-GPT4o-BoN & 75.88 & 153.38 & 97.00 \small{(4.11)} & 64.25 \small{(16.66)} & 47.36 & 364.33 & 102.56 \small{(6.24)} & 76.30 \small{(10.56)} \\
\hspace{24pt}\raisebox{0.5ex}{$\llcorner$}\hspace{4pt}\textit{Budget-Matched} & 76.24 & 160.59 & 97.44 \small{(3.67)} & 67.15 \small{(16.41)} & 47.52 & 384.43 & 101.58 \small{(9.53)} & 80.43 \small{(9.04)} \\

\bottomrule
\end{tabular}
    }
    \caption{
        Evaluation of zero-shot prompting, fine-tuning (FT), and our proposed methods on GSM8K and MATH datasets, averaged across five main models.
        We report absolute accuracy (\%) and length (tokens), along with relative accuracy (\%) and length (\%) compared to baseline prompting. Values in parentheses show standard deviations.
    }
    \label{tab:main_results}
\end{table*}

\section{Results}
\label{results}

% \begin{table*}[t]
% \resizebox{\textwidth}{!}{
% \centering
% \small
% \begin{tabular}{m{2.5cm}|m{15cm}}
% \toprule
% \textbf{Method} & \textbf{Content} \\ %& \# \textbf{Tokens}\\
% \midrule
% \midrule[0.05em]
% \multicolumn{2}{c}{Question} \\ 
% \midrule
% - & A robe takes 2 bolts of blue fiber and half that much white fiber.  How many bolts in total does it take?  \\
% \midrule
% \midrule[0.05em]
% \multicolumn{2}{c}{Reasoning Paths} \\
% \midrule
% Zero-Shot & To find the total number of bolts needed, we need to calculate the amount of white fiber first, since it's half the amount of blue fiber. Step 1: Determine the amount of blue fiber needed. The robe takes 2 bolts of blue fiber. Step 2: Calculate the amount of white fiber needed. The robe takes half the amount of white fiber as blue fiber. Half of 2 bolts is $2/2$ = 1 bolt. Step 3: Add the amount of blue fiber and white fiber to find the total. Total bolts = blue fiber + white fiber. Total bolts = 2 bolts + 1 bolt. Total bolts = 3 bolts. The answer is \textbf{\textit{3}}. \\ %& 213 \\
% \midrule
% FS-GPT4o-BoN & The robe takes 2 bolts of blue fiber. It takes half that much white fiber, which is $2 / 2$ = 1 bolt. Add the blue and white fiber together: 2 + 1 = 3 bolts. The answer is \textbf{\textit{3}}. \\ %57 \\
% \bottomrule
% \end{tabular}
% }
% \caption{Example rationales generated from the original Llama-3.1-8B model and the model fine-tuned using our method. The question is randomly sampled from GSM8K. For the same question, one can observe that our rationale is much more concise, removing redundant parts that explain the question.}
% \label{tab:samples}
% \vspace{-0.1in}
% \end{table*}


\begin{table*}[t]
\resizebox{\textwidth}{!}{
\centering
\normalsize
\begin{tabular}{m{17.5cm}}
\toprule
\multicolumn{1}{c}{\textbf{Question}} \\ 
\midrule
A robe takes 2 bolts of blue fiber and half that much white fiber.  How many bolts in total does it take? \\
\midrule
\multicolumn{1}{c}{\textbf{Reasoning Paths}} \\
\midrule
\textbf{[Zero-Shot]} 
To find the total number of bolts needed, we need to calculate the amount of white fiber first, since it's half the amount of blue fiber. Step 1: Determine the amount of blue fiber needed. The robe takes 2 bolts of blue fiber. Step 2: Calculate the amount of white fiber needed. The robe takes half the amount of white fiber as blue fiber. Half of 2 bolts is $2/2$ = 1 bolt. Step 3: Add the amount of blue fiber and white fiber to find the total. Total bolts = blue fiber + white fiber. Total bolts = 2 bolts + 1 bolt. Total bolts = 3 bolts. The answer is 3. \\
\midrule
\textbf{[FS-GPT4o-BoN]} 
The robe takes 2 bolts of blue fiber. It takes half that much white fiber, which is $2 / 2$ = 1 bolt. Add the blue and white fiber together: 2 + 1 = 3 bolts. The answer is 3. \\
\bottomrule
\end{tabular}
}
\caption{Example rationales generated from the original Llama-3.1-8B model (\textbf{Zero-Shot}) and the model fine-tuned using our method (\textbf{FS-GPT4o-BoN}). The question is randomly sampled from GSM8K. For the same question, one can observe that our rationale is much more concise, removing redundant parts that explain the question.}
\label{tab:samples}
\vspace{-0.1in}
\end{table*}


\subsection{Main results}

Our main results, presented in \autoref{tab:main_results} and \autoref{fig:main_methods_comparison}, demonstrate the performance of our self-training methods against baseline approaches.
% We highlight key observations from these results below.

\paragraph{Naive BoN fine-tuning is effective but sample inefficient.}
Naive BoN fine-tuning effectively reduces output length without significantly degrading model performance. 
This also holds true for Qwen2.5-Math-1.5B and DeepSeekMath-7B (\autoref{tab:main_results_full_gsm8k} and \autoref{tab:main_results_full_math}), which failed to achieve length reduction through zero-shot prompting.
% However, while naive BoN does reduce output length, the reduction is limited to 12\%.
However, the length reduction from naive BoN with $N=16$ is limited to 12\% on average.
Furthermore, as illustrated in Figure~\ref{fig:bon_sample_efficiency}, achieving more compression with BoN becomes progressively less efficient.

\paragraph{Iterative baseline yields similar results as naive BoN fine-tuning.}
% We compare our single-step naive BoN approach with Rational Metareasoning \cite{de2024rational}, an iterative approach using expert iteration \cite{zelikman2022star}  which incorporates an additional \textit{utility reward} to balance efficiency and accuracy in BoN sampling.
Rational Metareasoning, an iterative baseline, yields similar relative length reduction and relative accuracy to BoN fine-tuning. 
This suggests that the utility reward proposed by \citet{de2024rational} may not effectively achieve both accuracy gains and token length reduction.

\begin{figure}[t] % "h" places the figure roughly here
    \centering
    \includegraphics[width=\columnwidth]{figures/main_methods_comparison.pdf} % Adjust width as needed
    \caption{Tradeoff between relative accuracy and length reduction for main methods. Results are averaged over GSM8K and MATH across five main models. Matching colors and shapes indicate the same FS prompt. FS conditioning without augmentation (†) are marked with lighter colors. 
    Relative length reduction refers to 100 - relative length (\%).}
    \label{fig:main_methods_comparison} % Label for referencing in text
\end{figure}
% \red{TODO - shorten this}

\paragraph{Few-shot conditioning outperforms BoN in length reduction.}
The results demonstrate that few-shot conditioning achieves a greater relative length reduction compared to naive BoN, including math-specialized models (\autoref{tab:main_results_full_gsm8k} and \autoref{tab:main_results_full_math}).
% This reduction is attributed to the fact that the fine-tuning datasets generated through few-shot conditioning contain shorter reasoning paths compared to those generated by naive BoN, as illustrated in \autoref{fig:bon_sample_efficiency}.  % too long
This is in line with the superior length reduction of few-shot conditioning, compared to naive BoN as shown in \autoref{fig:bon_sample_efficiency}.
Notably, self-training on generations conditioned on human-annotated examples (FS-Human) achieves an average relative length of 67.96\% on GSM8K, compared to 87.17\% with naive BoN.  % good to have some specific numbers in the text
% We further analyze the effect of fine-tuning on length reduction in \autoref{analysis}.



\paragraph{Self-training better preserves accuracy than training with external data.} 
\autoref{tab:main_results} shows fine-tuning with external data (\textit{FT-External Data}) leads to a significant reduction in relative length but causes a severe drop in relative accuracy. 
% \autoref{fig:main_methods_comparison} further highlights that while fine-tuning with GPT-4o CoT (FT-GPT4o) achieves slightly greater reduction in relative length than fine-tuning with self-generated data using few-shots from GPT-4o (FS-GPT4o), it results in substantially lower relative accuracy.  % a bit complicated / not concrete (conrete evidence = one where we beat external FT in both accuracy and reduction)
\autoref{fig:main_methods_comparison} further highlights the accuracy preservation of self-training: fine-tuning with external concise reasoning supervision from GPT-4o (FT-GPT4o) lies below the Pareto-curve of relative accuracy and relative length reduction, established by our self-training methods.
% NAMGYU - TODO add some commentary

\paragraph{Few-shot conditioned BoN achieves best length reduction while maintaining accuracy.}
% Few-shot conditioned BoN enables substantial length reduction compared to all other BoN and few-shot methods while maintaining relative accuracy.
FS-BoN elicits the largest length reduction among our self-training methods, while maintaining relative accuracy, on average.
Notably, for math-specialized models, FS-GPT4o-BoN achieves the greatest reduction among all methods, except those fine-tuned on external data which greatly sacrifice the accuracy (\autoref{tab:main_results_full_gsm8k} and \autoref{tab:main_results_full_math}). 
% This result reflects how applying BoN to few-shot conditioning further reduces the relative length of the training data while also increasing the proportion of correct samples.  % unnecessary

\paragraph{Augmentation boosts accuracy for few-shot conditioning.}
\autoref{fig:main_methods_comparison} compares few-shot conditioning, i.e., FS and FS-BoN, with and without augmentation (†). 
Augmentation improves accuracy by providing solutions for previously unsolvable hard questions as discussed in \autoref{sample_augmentation}. 
While augmentation may slightly affect reduction rates, they remain superior to naive BoN and RM.
% Similar effect is observed for augmentation in FS-BoN.
% Even when matching the budget (\textit{Budget-Matched}) with other fine-tuning methods using self-generated data in \autoref{tab:main_results}, it achieves the greatest length reduction among them with minimal accuracy degradation.
Even when matching the budget (\textit{Budget-Matched}) with other self-training methods in \autoref{tab:main_results}, it achieves the greatest length reduction among them with minimal accuracy degradation.
The effect of augmentation on training data length is analyzed in \autoref{appx_augmentation_length}.
% Furthermore, as shown in Figure \ref{fig:main_methods_comparison}, augmentation on few-shot conditioned BoN enhances accuracy similar to naive BoN and Meta-Reasoning while achieving greater length reduction.

\begin{figure}[t]
    \centering
    \includegraphics[width=\columnwidth]{figures/length_by_difficulty.pdf} % Adjust width as needed
    \caption{\textbf{Tokens are reduced adaptively according to question difficulty.} 
    Token reduction rate for each difficulty level on MATH, for 4 models individually and averaged.
    % Higher difficulty levels show lower reduction rates.
    Relative length reduction refers to 100 - relative length (\%).
    }
    \label{fig:length_difficulty} % Label for referencing in text
\end{figure}

\subsection{Analysis}
\label{analysis}
% This section analyzes length reduction: transfer from generation to fine-tuning, reduction by question difficulty, qualitative analysis, and consistency across model sizes. DeepSeekMath-7B is excluded from quantitative analysis due to cost.
% let's keep this short
In this section, we analyze the length reduction effects in depth.
We exclude DeepSeekMath-7B from quantiative analysis due to cost.


% \paragraph{Analysis on sample efficiency}
% As shown in \autoref{fig:bon_sample_efficiency}, best-of-n (BoN) sampling requires a substantial number of samples to be generated to achieve a level of reasoning length reduction comparable to that achievable through few-shot conditioning.
% In other words, it is infeasible to reach the reasoning length reduction performance of few-shot conditioning using BoN alone, without generating a prohibitively large number of samples.
% However, our experiments consistently demonstrate that combining few-shot conditioning with BoN sampling is more effective in reducing reasoning length than using either technique in isolation.
% Specifically, few-shot conditioning helps to guide the model towards generating more concise reasoning paths, while BoN sampling allows us to select the shortest and most accurate path from a diverse set of candidates.
% This synergistic effect results in a more efficient and effective approach to concise reasoning.


% \paragraph{FT can reduce generation length effectively.}
% As shown in \autoref{fig:ft_length_scatter}, after fine-tuning, the models tend to follow the length of the training data, suggesting that reasoning length reduction can be achieved through simple supervised fine-tuning on short reasoning samples.
% Note that test generation length is relatively longer than the training data length, as the models can generate lengthy incorrect answers, while the training data consists of correct answers.
% Correctly generated answers align more closely with training data length as shown in (Appendix~\ref{appx_length_scatter_correct}).

% \paragraph{Length reduction through generation and fine-tuning}
% Our method reduces reasoning length in two stages: generation and fine-tuning.
% First, as shown in \autoref{fig:ft_length_scatter}, 
% % generation length for training data varies depending on the method. 
% few-shot conditioning methods produce shorter outputs than naive BoN, with few-shot conditioned BoN achieving the shortest. 
% Second, fine-tuning with shorter rationales results in shorter model outputs, showing a strong correlation between test and training lengths\footnote{Test generation lengths are generally longer than training data lengths due to the possibility of lengthy incorrect answers during testing. Test outputs that are correct align more closely with training data lengths, as shown in Appendix~\ref{appx_length_scatter_correct}.}.
% Overall, FS-GPT4o-BoN effectively generates and trains for shorter reasoning paths.
% Additionally, unlike zero-shot methods, our approach significantly reduces token length in math-tuned models like Qwen2.5-Math-1.5B with FS-GPT4o-BoN, achieving 54.7\% relative length after fine-tuning. (See \autoref{tab:main_results_full_gsm8k} and \autoref{tab:main_results_full_math}).

\paragraph{Tokens are reduced adaptively according to question complexity.} 
The MATH dataset's difficulty levels range from 1 (basic algebra) to 5 (advanced calculus and complex mathematical reasoning).
As shown in \autoref{fig:length_difficulty}, our method adaptively reduces tokens based on question difficulty, with higher difficulty leading to less reduction.
% Most models achieve their peak reduction (around 20\%--40\%) at difficulty levels 1-2, where simple concepts allow for more concise explanations.
% The reduction rate gradually declines at levels 3-5, indicating our method's ability to preserve necessary details for complex problems automatically.
%  -> not precise. simple concepts allow for more concise explanations *in absolute terms*, but this does not necessarily mean that length reduction *relative to the default* should be high. E.g., if the model already uses very few tokens for easy questions, then relative reduction would be low.
The higher reduction (20\%--40\%) at easier difficulty levels (1--2) suggests that the original model outputs for these easier questions contained unnecessary tokens.
This reveals a gap in current models' ability to tailor their inference budget to problem complexity.
Our method effectively closes this gap by reducing redundancy, allowing for more precise token allocation based on question difficulty.

\begin{figure}[t] % "h" places the figure roughly here
    \centering
    \includegraphics[width=\columnwidth]{figures/scaling_methods_comparison.pdf} % Adjust width as needed
    \caption{Scaling study on baseline and few-shot conditioned self-training methods. Results are averaged over GSM8K and MATH for Llama 1B, 3B, and 8B.
    % Accuracy tends to be maintained, with greater length reduction using our FS-GPT4o(-BoN) method.
    Relative length reduction refers to 100 - relative length (\%).
    }
    \label{fig:scaling_methods_comparison} % Label for referencing in text
\end{figure}

\paragraph{Self-training maintains consistency across model scales.}
We conduct a scaling study on Llama-3.2-1B, 3B, and Llama-3.1-8B to examine consistency across different model sizes (\autoref{fig:scaling_methods_comparison}). 
Overall, token reduction increases as the model size increases, while the maintenance of accuracy does not show a strong correlation with model size. 
RM exhibits lower reduction rates compared to our few-shot conditioned self-training methods across all models and shows a decrease in accuracy with increasing model size. 
% The few-shot method also shows a similar trend in length reduction, but it achieves the best relative accuracy in the 3B model.
Our standalone few-shot conditioning method (FS-GPT4o) also shows a similar trend in length reduction, but better preserves accuracy.
Our joint FS-GPT4o-BoN method achieves the greatest reduction across all models, maintaining relative accuracy across different model sizes, especially in the largest 8B model.



\paragraph{Sample study}
\autoref{tab:samples} presents qualitative examples of reasoning paths generated by the model before and after fine-tuning with our method. 
The original reasoning exhibits verbosity, containing redundant processes such as question confirmation and repeated instructions. 
In contrast, the reasoning generated by our method includes only the necessary steps, significantly reducing the number of tokens while still arriving at the correct answer. 
% These examples demonstrate the effectiveness of our method in reducing token count. 
More examples are provided in the \autoref{appx_sample_studies}.

\begin{figure}[t]
    \centering
    \includegraphics[width=\columnwidth]{figures/both_length_scatter.pdf} % Adjust width as needed
    \caption{\textbf{Fine-tuning effectively transfers the length reduction to the model.} Correlation between the relative length of train data and test output averaged over GSM8K and MATH across 4 models. Training length includes only correct solutions. Solid points represent test lengths including all generated outputs, while lighter points indicate test lengths of correct solutions only.}
    \label{fig:ft_length_scatter} % Label for referencing in text
\end{figure}

\paragraph{Length reduction is transferred through fine-tuning.}
As shown in \autoref{fig:ft_length_scatter}, fine-tuning with shorter rationales results in shorter model outputs, showing a strong correlation between test and training lengths.
% Test generation lengths (solid datapoints) are generally longer than training data lengths due to the possibility of lengthy incorrect answers during testing.
% However, when comparing with test generation lengths that are correct (lighter datapoints), they align more closely with training data lengths.
We note that the length of test outputs (incorrect and correct) are longer than the length of training samples (only correct) on average.
This is mainly because incorrect paths are generally longer than correct ones.
We find a closer correspondence between train length and test length of correct samples only, indicated by the lighter datapoints.
This discrepancy suggests the need to terminate incorrect paths early to minimize redundant inference overhead.
We consider this for future work.

% !TEX root = template.tex

\section{Conclusion}
\label{sec:conclusion}
This work focuses on MAS coordination and synchronization under recurring LTL. We extended the bottom-up scheme for distributed motion and task coordination of MAS in \cite{meng_paper}, reducing computational complexity to enhance scalability and enable deployment on robotic hardware. The package was developed in ROS2, with a synchronization mechanism to handle action delays in experiments. Future work will focus on developing additional actions and incorporating human-in-the-loop scenarios.

\section*{Acknowledgment}
After the first draft, the text in the various sections of the manuscript (except the Abstract) was improved using  Claude 3.5 Sonnet through the Perplexity AI web app. After this process, the paper underwent multiple sets of reviews from the authors and was thus further improved and modified.
\bibliographystyle{IEEEtran}
\bibliography{biblio}

\begin{IEEEbiography}[{\includegraphics[width=1in,height=1.25in,clip,keepaspectratio]{figures/Daniele.jpg}}]{Daniele Mari} received the B.S. degree in information engineering and M.S. degree in ICT for Internet and Multimedia from the University of Padova, Italy, in 2019 and 2021 respectively. He is currently pursuing his Ph.D. in information engineering  and is currently a Ph.D. student in Padova since 2021. Additionally, he has spent 6 months as a visiting Ph.D. student in Instituto Superior Técnico, Universidade de Lisboa, Portugal, in 2023. His main research interests are learned point cloud and image coding. He has authored several publications in top conferences and journals in this field.
\end{IEEEbiography}


\begin{IEEEbiography}[{\includegraphics[width=1in,height=1.25in,clip,keepaspectratio]{figures/Andre.png}}]{ANDRÉ F. R. GUARDA} (Member, IEEE) received his B.Sc. and M.Sc. degrees in electrotechnical engineering from Instituto Politécnico de Leiria, Portugal, in 2013 and 2016, respectively, and the Ph.D. degree in electrical and computer engineering from Instituto Superior Técnico, Universidade de Lisboa, Portugal, in 2021. He has been a researcher at Instituto de Telecomunicações since 2011, where he currently holds a Post-Doctoral position. His main research interests include multimedia signal processing and coding, with particular focus on point cloud coding with deep learning. He has authored several publications in top conferences and journals in this field and is actively contributing to the standardization efforts of JPEG and MPEG on learning-based point cloud coding.
\end{IEEEbiography}

\newpage

%If you do not have or do not want to include a photo, you can use IEEEbiographynophoto as shown below:

\begin{IEEEbiography}[{\includegraphics[width=1in,height=1.25in,clip,keepaspectratio]{figures/Nuno_Rodrigues_4_by_3.jpg}}]{NUNO M. M. RODRIGUES} 
 (Senior Member,
IEEE) graduated in electrical engineering in 1997, received the M.Sc. degree from the Universidade de Coimbra, Portugal, in 2000, and the Ph.D. degree from the Universidade de Coimbra, Portugal, in 2009, in collaboration with the Universidade Federal do Rio de Janeiro, Brazil. He is a Professor in the Department of Electrical Engineering, in the School of Technology and Management of the Polytechnic University of Leiria, Portugal and a Senior Researcher in Instituto de Telecomunicações, Portugal. He has coordinated and participated as a researcher in various national and international funded projects. He has supervised three concluded PhD theses and several MSc theses. He is co-author of a book and more than 100 publications, including book chapters and papers in national and international journals and conferences. His research interests include several topics related with image and video coding and processing, for different signal modalities and applications. His current research is focused on deep learning-based techniques for point cloud coding and processing.
\end{IEEEbiography}

\begin{IEEEbiography}[{\includegraphics[width=1in,height=1.25in,clip,keepaspectratio]{figures/simone-milani-bg.png}}]{Simone Milani} 
(Member, IEEE) received the
Laurea degree in telecommunication engineering
and the Ph.D. degree in electronics and telecommunication engineering from the University of
Padova, Padova, Italy, in 2002 and 2007, respectively. He was a Visiting Ph.D. Student at the University of California at Berkeley, Berkeley, CA,
USA, in 2006. He was a Consultant at STMicroelectronics, Agrate, Italy. He was a Postdoctoral Researcher at the University of Udine, Udine,
Italy, the University of Padova, and the Politecnico di Milano, Milan, Italy,
from 2007 to 2013. From 2013 to 2020, he was an Assistant Professor with
the Department of Information Engineering, University of Padova, where
he is an Associate Professor. His research interests include digital signal
processing, image and video coding, 3-D video processing and compression,
joint source-channel coding, robust video transmission, distributed source
coding, multiple description coding, and multimedia forensics.
\end{IEEEbiography}

\begin{IEEEbiography}[{\includegraphics[width=1in,height=1.25in,clip,keepaspectratio]{figures/Fernando.png}}]{Fernando Pereira} (Fellow, IEEE) graduated in electrical and computer engineering in 1985 and received the M.Sc. and Ph.D. degrees in 1988 and 1991, respectively, from Instituto Superior Técnico, Technical University of Lisbon. He is with the Department of Electrical and Computers Engineering of Instituto Superior Técnico, University of Lisbon, and Instituto de Telecomunicações, Lisbon, Portugal. He is or has been Associate Editor of IEEE Transactions of Circuits and Systems for Video Technology, IEEE Transactions on Image Processing, IEEE Transactions on Multimedia, IEEE Signal Processing Magazine and EURASIP Journal on Image and Video Processing, and Area Editor of the Signal Processing: Image Communication Journal. In 2013-2015, he was the Editor-in-Chief of the IEEE Journal of Selected Topics in Signal Processing. He was an IEEE Distinguished Lecturer in 2005 and elected as an IEEE Fellow in 2008 for “contributions to object-based digital video representation technologies and standards”. He has been elected to serve on the IEEE Signal Processing Society Board of Governors in the capacity of Member-at-Large for 2012 and 2014-2016 terms. He has been IEEE Signal Processing Society Vice-President for Conferences in 2018-2020 and IEEE Signal Processing Society Awards Board Member in 2017. He was the recipient of the 2023 Leo L. Beranek Meritorious Service Award. Since 2013, he is also a EURASIP Fellow for “contributions to digital video representation technologies and standards”. He has been elected to serve on the European Signal Processing Society Board of Directors for a 2015-2018 term. He was the recipient of the 2023 EURASIP Meritorious Service Award. Since 2015, he is also an IET Fellow. He has also held key leadership roles in numerous IEEE Signal Processing Society conferences and workshops, mostly notably serving twice as ICIP Technical Chair in two continents, Hong Kong (2010) and Phoenix (2016). He has been MPEG Requirements Subgroup Chair and is currently JPEG Requirements Subgroup Chair. Recently, he has been one of the key designers of the JPEG Pleno and JPEG AI standardization projects. He has contributed more than 300 papers in international journals, conferences and workshops, and made several tens of invited talks and tutorials at conferences and workshops. His areas of interest are video analysis, representation, coding, description and adaptation, and advanced multimedia services.
\end{IEEEbiography}

\EOD

\end{document}
