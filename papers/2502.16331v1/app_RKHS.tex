\section{Properties of Gaussian RKHS on $\mathbb{R}^d$}\label{app: properties}
\rahul{don't need this appendix at all.}
In this Appendix, we will show the necessary properties of the RKHS $\cH$ defined \secref{sec: setup}. To simplify the notation, we show the properties for the case when the underlying matrix $\textbf{M}$ is identity, and thus we use the notion $k$ for the kernel $k_{\textbf{M}}$. We state all the necessary properties and their proofs in the following subsections: %\akash{some of the properties hold similar as before so remove those discussions}


\subsection{Point Evaluation Functional}

\begin{lemma}
For any $x \in \mathbb{R}^d$, the point evaluation functional $\boldsymbol{\delta}_x: \mathcal{H} \to \mathbb{R}$ defined by $\boldsymbol{\delta}_x(f) = f(x)$ is continuous.
\end{lemma}

\begin{proof}
Let $f \in \mathcal{H}$. By the reproducing property:
$$|\boldsymbol{\delta}_x(f)| = |f(x)| = |\langle f, k(\cdot,x)\rangle_\mathcal{H}| \leq \|f\|_\mathcal{H} \|k(\cdot,x)\|_\mathcal{H}$$

Now, 
$$\|k(\cdot,x)\|^2_\mathcal{H} = \langle k(\cdot,x), k(\cdot,x)\rangle_\mathcal{H} = k(x,x) = 1$$

Therefore, $|f(x)| \leq \|f\|_\mathcal{H}$, showing that $\boldsymbol{\delta}_x$ is bounded which completes the proof of the lemma. 
\end{proof}

\subsection{Inner Product Structure}
\begin{lemma}
The inner product $\langle \cdot, \cdot \rangle_\mathcal{H}$ satisfies all required properties and generates a valid norm.    
\end{lemma}

\begin{proof}

\textbf{Symmetry}: For any $x,y \in \mathbb{R}^d$:
   $$\langle k(\cdot,x), k(\cdot,y)\rangle_\mathcal{H} = k(x,y) = k(y,x) = \langle k(\cdot,y), k(\cdot,x)\rangle_\mathcal{H}$$

\textbf{Linearity}: For $f = \sum_{i=1}^\infty \alpha_i k(\cdot,x_i)$ and $g = \sum_{j=1}^\infty \beta_j k(\cdot,y_j)$ such that $\norm{f}_{\cH}$ and $\norm{g}_{\cH}$:
   $$\langle f, g\rangle_\mathcal{H} = \sum_{i=1}^\infty \sum_{j=1}^\infty \alpha_i \beta_j k(x_i,y_j)$$
   Linearity follows from the properties of summation.

\textbf{Positive Definiteness}: For $f = \sum_{i=1}^\infty \alpha_i k(\cdot,x_i)$:
   $$\langle f,f\rangle_\mathcal{H} = \sum_{i=1}^\infty \sum_{j=1}^\infty \alpha_i \alpha_j k(x_i,x_j) > 0$$
   for $f \neq 0$, due to the strict positive definiteness of the Gaussian kernel.

Similarly, it is straight-forward to show that the norm $\|f\|_\mathcal{H} = \sqrt{\langle f,f\rangle_\mathcal{H}}$ satisfies the following:
\begin{itemize}
    \item Positivity: $\|f\|_\mathcal{H} \geq 0$ with equality iff $f = 0$
    \item Homogeneity: $\|cf\|_\mathcal{H} = |c| \|f\|_\mathcal{H}$
    \item Triangle Inequality: it is straightforward to show that the triangle inequality holds for two functions $f,g$ with finite representations, which then can be used to show in the limit that even functions with infinite representations satisfy the property.
    %follows from Cauchy-Schwarz 
\end{itemize}
\end{proof}
% \subsection*{Theorem 3 (Separability)}
% The Gaussian RKHS $\mathcal{H}$ is separable.

% \paragraph{Proof:}
% Let $\mathbb{Q}^d$ be the set of points in $\mathbb{R}^d$ with rational coordinates. Consider the set:
% $$S = \{\sum_{i=1}^n \alpha_i k(\cdot,q_i) : n \in \mathbb{N}, \alpha_i \in \mathbb{Q}, q_i \in \mathbb{Q}^d\}$$

% 1. $S$ is countable as it's a countable union of finite products of countable sets.

% 2. We show $S$ is dense in $\mathcal{H}$. Let $f \in \mathcal{H}$ and $\varepsilon > 0$.
%    - Express $f = \sum_{i=1}^n \alpha_i k(\cdot,x_i)$ (finite sum exists by definition of $\mathcal{H}$)
%    - By continuity of $k$, for each $x_i$ we can find $q_i \in \mathbb{Q}^d$ such that:
%      $$\|k(\cdot,x_i) - k(\cdot,q_i)\|_\mathcal{H} < \frac{\varepsilon}{n|\alpha_i|}$$
%    - Then $\|f - \sum_{i=1}^n \alpha_i k(\cdot,q_i)\|_\mathcal{H} < \varepsilon$

% Therefore, $\mathcal{H}$ is separable. $\square$

\subsection{Completeness}
\begin{lemma}
    The Gaussian RKHS $\mathcal{H}$ is complete.
\end{lemma}

% \paragraph{Proof:}
% Assume that $(f_n)$ be a Cauchy sequence in $\mathcal{H}$. For any $x \in \mathbb{R}^d$, using the reproducing property and Cauchy-Schwarz:
%    $$|f_n(x) - f_m(x)| = |\langle f_n - f_m, k(\cdot,x)\rangle_\mathcal{H}| \leq \|f_n - f_m\|_\mathcal{H}$$

% Therefore, $(f_n(x))$ is Cauchy in $\mathbb{R}$ for each $x$. Define:
%    $$f(x) = \lim_{n \to \infty} f_n(x)$$

% For any $N \in \mathbb{N}$, we have 
%    $$\|f_N\|_\mathcal{H}^2 = \langle f_N, f_N\rangle_\mathcal{H} \leq M^2$$
%    for some $M > 0$ (as Cauchy sequences are bounded)

% 4. For any finite set $\{x_1,\ldots,x_k\} \subset \mathbb{R}^d$ and $\{\alpha_1,\ldots,\alpha_k\} \subset \mathbb{R}$:
%    $$\left|\sum_{i=1}^k \alpha_i f(x_i)\right| = \lim_{n \to \infty} \left|\sum_{i=1}^k \alpha_i f_n(x_i)\right| \leq M\left\|\sum_{i=1}^k \alpha_i k(\cdot,x_i)\right\|_\mathcal{H}$$

% 5. By the Riesz representation theorem, there exists $g \in \mathcal{H}$ such that:
%    $$f(x) = \langle g, k(\cdot,x)\rangle_\mathcal{H}$$

% Therefore, $f \in \mathcal{H}$ and $\|f_n - f\|_\mathcal{H} \to 0$. $\square$

\begin{proof}    

Let $(f_n)$ be a Cauchy sequence in $\mathcal{H}$. Each $f_n$ has a possibly infinite representation:
$$f_n = \lim_{m \to \infty} \sum_{i=1}^m \alpha_i^{(n,m)} k(\cdot,x_i^{(n,m)})$$

For any $x \in \mathbb{R}^d$, using the reproducing property and Cauchy-Schwarz inequality:
$$|f_n(x) - f_m(x)| = |\langle f_n - f_m, k(\cdot,x)\rangle_\mathcal{H}| \leq \|f_n - f_m\|_\mathcal{H}$$
Therefore, $(f_n(x))$ is Cauchy in $\mathbb{R}$ for each $x$. Define:
$$f(x) = \lim_{n \to \infty} f_n(x)$$
For any $N \in \mathbb{N}$:
$$\|f_N\|_\mathcal{H}^2 = \lim_{m \to \infty} \sum_{i,j=1}^m \alpha_i^{(N,m)} \alpha_j^{(N,m)} k(x_i^{(N,m)},x_j^{(N,m)}) \leq M^2$$
for some $M > 0$ as Cauchy sequences are bounded.
For any finite set ${y_1,\ldots,y_k} \subset \mathbb{R}^d$ and ${\beta_1,\ldots,\beta_k} \subset \mathbb{R}$:
\begin{align*}
\left|\sum_{i=1}^k \beta_i f(y_i)\right| = \lim_{n \to \infty} \left|\sum_{i=1}^k \beta_i f_n(y_i)\right| \
&= \lim_{n \to \infty} \left|\left\langle f_n, \sum_{i=1}^k \beta_i k(\cdot,y_i)\right\rangle_{\mathcal{H}}\right| \\
&\leq \limsup_{n \to \infty} \|f_n\|_{\mathcal{H}}\cdot \|\sum_{i=1}^k \beta_i k(\cdot,y_i)\|{\mathcal{H}} \\
&\leq M\|\sum_{i=1}^k \beta_i k(\cdot,y_i)\|_{\mathcal{H}}
\end{align*}
By the Riesz representation theorem, there exists $g \in \mathcal{H}$ such that:
$$f(x) = \langle g, k(\cdot,x)\rangle_\mathcal{H}$$
For any $\epsilon > 0$, choose $N$ such that $\|f_n - f_m\|_\mathcal{H} < \epsilon$ for all $n,m \geq N$. Then for any $x \in \mathbb{R}^d$:
$$|\langle f_n - f, k(\cdot,x)\rangle_\mathcal{H}| = \lim_{m \to \infty} |\langle f_n - f_m, k(\cdot,x)\rangle_\mathcal{H}| \leq \epsilon \|k(\cdot,x)\|_\mathcal{H}$$
Therefore, $f \in \mathcal{H}$ and $\|f_n - f\|_\mathcal{H} \to 0$. 

\end{proof}
% \subsection*{Theorem 5 (Additional Properties)}
% For the Gaussian RKHS $\mathcal{H}$:

% 1. Every $f \in \mathcal{H}$ is in $C^\infty(\mathbb{R}^d)$.

% \paragraph{Proof:}
% For any multi-index $\alpha$:
% $$\left|\frac{\partial^\alpha f}{\partial x^\alpha}(x)\right| = \left|\left\langle f, \frac{\partial^\alpha k(\cdot,x)}{\partial x^\alpha}\right\rangle_\mathcal{H}\right| \leq \|f\|_\mathcal{H} \left\|\frac{\partial^\alpha k(\cdot,x)}{\partial x^\alpha}\right\|_\mathcal{H}$$

% The right-hand side is finite because the Gaussian kernel is infinitely differentiable. $\square$

% 2. For any $f \in \mathcal{H}$, $f(x) \to 0$ as $\|x\| \to \infty$.

% \paragraph{Proof:}
% For $f = \sum_{i=1}^n \alpha_i k(\cdot,x_i)$:
% $$|f(x)| \leq \sum_{i=1}^n |\alpha_i| \exp\left(-\frac{\|x-x_i\|^2}{2\sigma^2}\right) \to 0$$
% as $\|x\| \to \infty$. This extends to all of $\mathcal{H}$ by density. $\square$

% 3. $\mathcal{H}$ is dense in $L^2(\mathbb{R}^d)$.

% \paragraph{Proof:}
% This follows from the spectral properties of the Gaussian kernel and Mercer's theorem. The eigenfunctions form a complete orthonormal system in $L^2(\mathbb{R}^d)$. $\square$

% \section*{Properties of Gaussian RKHS on $\mathbb{R}^d$}

% \subsection*{Definition and Setup}

% Let $k: \mathbb{R}^d \times \mathbb{R}^d \to \mathbb{R}$ be the Gaussian kernel defined by:

% $$k(x,y) = \exp\left(-\frac{\|x-y\|^2}{2\sigma^2}\right)$$

% where $\sigma > 0$ is fixed. Let $\mathcal{H}$ be the RKHS associated with this kernel.

% \subsection*{Theorem 1 (Point Evaluation Functional)}
% For any $x \in \mathbb{R}^d$, the point evaluation functional $L_x: \mathcal{H} \to \mathbb{R}$ defined by $L_x(f) = f(x)$ is continuous.

% \paragraph{Proof:}
% Let $f \in \mathcal{H}$. By the reproducing property:
% $$|f(x)| = |\langle f, k(\cdot,x)\rangle_\mathcal{H}| \leq \|f\|_\mathcal{H} \|k(\cdot,x)\|_\mathcal{H}$$

% Now, 
% $$\|k(\cdot,x)\|^2_\mathcal{H} = \langle k(\cdot,x), k(\cdot,x)\rangle_\mathcal{H} = k(x,x) = 1$$

% Therefore, $|f(x)| \leq \|f\|_\mathcal{H}$, showing that $L_x$ is bounded with $\|L_x\| \leq 1$. $\square$

% \subsection*{Theorem 2 (Inner Product Structure)}
% The inner product $\langle \cdot, \cdot \rangle_\mathcal{H}$ satisfies all required properties and generates a valid norm.

% \paragraph{Proof:}
% 1. \textbf{Symmetry}: For any $x,y \in \mathbb{R}^d$:
%    $$\langle k(\cdot,x), k(\cdot,y)\rangle_\mathcal{H} = k(x,y) = k(y,x) = \langle k(\cdot,y), k(\cdot,x)\rangle_\mathcal{H}$$

% 2. \textbf{Linearity}: For $f = \sum_{i=1}^n \alpha_i k(\cdot,x_i)$ and $g = \sum_{j=1}^m \beta_j k(\cdot,y_j)$:
%    $$\langle f, g\rangle_\mathcal{H} = \sum_{i=1}^n \sum_{j=1}^m \alpha_i \beta_j k(x_i,y_j)$$
%    Linearity follows from the properties of summation.

% 3. \textbf{Positive Definiteness}: For $f = \sum_{i=1}^n \alpha_i k(\cdot,x_i)$:
%    $$\langle f,f\rangle_\mathcal{H} = \sum_{i=1}^n \sum_{j=1}^n \alpha_i \alpha_j k(x_i,x_j) > 0$$
%    for $f \neq 0$, due to the strict positive definiteness of the Gaussian kernel.

% The norm $\|f\|_\mathcal{H} = \sqrt{\langle f,f\rangle_\mathcal{H}}$ satisfies:
% - Positivity: $\|f\|_\mathcal{H} \geq 0$ with equality iff $f = 0$
% - Homogeneity: $\|cf\|_\mathcal{H} = |c| \|f\|_\mathcal{H}$
% - Triangle Inequality: follows from Cauchy-Schwarz $\square$


\subsection{Separability of $\cH$}
\begin{lemma}
The Gaussian RKHS $\mathcal{H}$ is separable.    
\end{lemma}

\begin{proof}
Let $\mathbb{Q}^d$ denote the set of points in $\mathbb{R}^d$ with rational coordinates, and define:
$$S := \curlybracket{\sum_{i=1}^n \alpha_i k(\cdot,q_i) : n \in \mathbb{N}, \alpha_i \in \mathbb{Q}, q_i \in \mathbb{Q}^d}$$
We first show that $S$ is countable. For each $n \in \mathbb{N}$, the set of functions with $n$ terms is a finite product of countable sets:
$$S_n := \curlybracket{\sum_{i=1}^n \alpha_i k(\cdot,q_i) : \alpha_i \in \mathbb{Q}, q_i \in \mathbb{Q}^d} \cong \mathbb{Q}^n \times (\mathbb{Q}^d)^n$$
Therefore, $S = \bigcup_{n=1}^\infty S_n$ is countable.

To prove density of $S$ in $\mathcal{H}$, let $f \in \mathcal{H}$ and $\epsilon > 0$ be given. Since finite linear combinations of kernel functions are dense in $\mathcal{H}$, there exists
$$g := \sum_{i=1}^n \alpha_i k(\cdot,x_i)$$
such that
$$\|f - g\|_\mathcal{H} < \frac{\epsilon}{3}$$

Let $M := \max_{1 \leq i \leq n} \|k(\cdot,x_i)\|_\mathcal{H}$. By density of $\mathbb{Q}$ in $\mathbb{R}$, for each $i \in \{1,\ldots,n\}$, there exist $\beta_i \in \mathbb{Q}$ such that
$$|\alpha_i - \beta_i| < \frac{\epsilon}{3nM}$$

By continuity of the Gaussian kernel and density of $\mathbb{Q}^d$ in $\mathbb{R}^d$, for each $i \in \{1,\ldots,n\}$, there exist $q_i \in \mathbb{Q}^d$ such that:
$$\|k(\cdot,x_i) - k(\cdot,q_i)\|_\mathcal{H} < \frac{\epsilon}{3n\max\{1,\max_{j}|\beta_j|\}}$$

Define $h := \sum_{i=1}^n \beta_i k(\cdot,q_i) \in S$. Then:
\begin{align*}
\|f - h\|_\mathcal{H} \leq \|f - g\|_\mathcal{H} + \|g - h\|_\mathcal{H} 
&\leq \frac{\epsilon}{3} + \left\|\sum_{i=1}^n (\alpha_i k(\cdot,x_i) - \beta_i k(\cdot,q_i))\right\|_\mathcal{H} \\
&\leq \frac{\epsilon}{3} + \sum_{i=1}^n \|(\alpha_i - \beta_i)k(\cdot,x_i) + \beta_i(k(\cdot,x_i) - k(\cdot,q_i))\|_\mathcal{H} \\
&\leq \frac{\epsilon}{3} + \sum_{i=1}^n \left(|\alpha_i - \beta_i|\|k(\cdot,x_i)\|_\mathcal{H} + |\beta_i|\|k(\cdot,x_i) - k(\cdot,q_i)\|_\mathcal{H}\right) \\
&< \frac{\epsilon}{3} + n\left(\frac{\epsilon}{3nM}M + \max_{j}|\beta_j|\frac{\epsilon}{3n\max\{1,\max_{j}|\beta_j|\}}\right) \\
&\leq \epsilon
\end{align*}
Therefore, $S$ is dense in $\mathcal{H}$, proving that $\mathcal{H}$ is separable.
\end{proof}

% \subsection*{Theorem 4 (Completeness)}
% The Gaussian RKHS $\mathcal{H}$ is complete.

% \paragraph{Proof:}
% Let $(f_n)$ be a Cauchy sequence in $\mathcal{H}$.

% 1. For any $x \in \mathbb{R}^d$, using the reproducing property and Cauchy-Schwarz:
%    $$|f_n(x) - f_m(x)| = |\langle f_n - f_m, k(\cdot,x)\rangle_\mathcal{H}| \leq \|f_n - f_m\|_\mathcal{H}$$

% 2. Therefore, $(f_n(x))$ is Cauchy in $\mathbb{R}$ for each $x$. Define:
%    $$f(x) = \lim_{n \to \infty} f_n(x)$$

% 3. For any $N \in \mathbb{N}$:
%    $$\|f_N\|_\mathcal{H}^2 = \langle f_N, f_N\rangle_\mathcal{H} \leq M^2$$
%    for some $M > 0$ (as Cauchy sequences are bounded)

% 4. For any finite set $\{x_1,\ldots,x_k\} \subset \mathbb{R}^d$ and $\{\alpha_1,\ldots,\alpha_k\} \subset \mathbb{R}$:
%    $$\left|\sum_{i=1}^k \alpha_i f(x_i)\right| = \lim_{n \to \infty} \left|\sum_{i=1}^k \alpha_i f_n(x_i)\right| \leq M\left\|\sum_{i=1}^k \alpha_i k(\cdot,x_i)\right\|_\mathcal{H}$$

% 5. By the Riesz representation theorem, there exists $g \in \mathcal{H}$ such that:
%    $$f(x) = \langle g, k(\cdot,x)\rangle_\mathcal{H}$$

% Therefore, $f \in \mathcal{H}$ and $\|f_n - f\|_\mathcal{H} \to 0$. $\square$

% \subsection*{Theorem 5 (Additional Properties)}
% For the Gaussian RKHS $\mathcal{H}$:

% 1. Every $f \in \mathcal{H}$ is in $C^\infty(\mathbb{R}^d)$.

% \paragraph{Proof:}
% For any multi-index $\alpha$:
% $$\left|\frac{\partial^\alpha f}{\partial x^\alpha}(x)\right| = \left|\left\langle f, \frac{\partial^\alpha k(\cdot,x)}{\partial x^\alpha}\right\rangle_\mathcal{H}\right| \leq \|f\|_\mathcal{H} \left\|\frac{\partial^\alpha k(\cdot,x)}{\partial x^\alpha}\right\|_\mathcal{H}$$

% The right-hand side is finite because the Gaussian kernel is infinitely differentiable. $\square$

% 2. For any $f \in \mathcal{H}$, $f(x) \to 0$ as $\|x\| \to \infty$.

% \paragraph{Proof:}
% For $f = \sum_{i=1}^n \alpha_i k(\cdot,x_i)$:
% $$|f(x)| \leq \sum_{i=1}^n |\alpha_i| \exp\left(-\frac{\|x-x_i\|^2}{2\sigma^2}\right) \to 0$$
% as $\|x\| \to \infty$. This extends to all of $\mathcal{H}$ by density. $\square$

% 3. $\mathcal{H}$ is dense in $L^2(\mathbb{R}^d)$.

% \paragraph{Proof:}
% This follows from the spectral properties of the Gaussian kernel and Mercer's theorem. The eigenfunctions form a complete orthonormal system in $L^2(\mathbb{R}^d)$. $\square$

\iffalse

\section*{Embedding of the Gaussian RKHS into Sobolev Spaces}

In this note, we show that the reproducing kernel Hilbert space (RKHS) corresponding to the Gaussian kernel
\[
k(x,y)=\exp\left(-\frac{\|x-y\|^2}{2\sigma^2}\right),\quad x,y\in\mathbb{R}^d,
\]
with \(\sigma>0\) is continuously embedded in the Sobolev space \(H^s(\mathbb{R}^d)\) for every \(s>0\).

\begin{theorem}
Let \(\mathcal{H}\) denote the RKHS associated with the Gaussian kernel \(k\). Then for every \(s>0\), there exists a constant \(C=C(s,\sigma)>0\) such that
\[
\|f\|_{H^s} \le C\,\|f\|_{\mathcal{H}}, \quad \forall f\in \mathcal{H}.
\]
In particular, \(\mathcal{H}\) is continuously embedded in \(H^s(\mathbb{R}^d)\).
\end{theorem}

\begin{proof}
A standard way to characterize the norm in the RKHS of a translation-invariant kernel is via the Fourier transform. For the Gaussian kernel, one may show (using Bochner's theorem) that there exists a constant \(A>0\) (depending on \(\sigma\) and the dimension \(d\)) such that for all \(f\in \mathcal{H}\) with Fourier transform \(\hat{f}\) (normalized appropriately) the RKHS norm is given by
\begin{equation} \label{eq:RKHS-norm}
\|f\|_{\mathcal{H}}^2 = \int_{\mathbb{R}^d} |\hat{f}(\xi)|^2\,\omega(\xi)\,d\xi, \quad \text{with} \quad \omega(\xi) = e^{\frac{\sigma^2 \|\xi\|^2}{2}}.
\end{equation}

On the other hand, the Sobolev norm of order \(s>0\) is defined by
\[
\|f\|_{H^s}^2 = \int_{\mathbb{R}^d} (1+\|\xi\|^2)^s\, |\hat{f}(\xi)|^2\,d\xi.
\]

Notice that for any fixed \(s>0\) and for all \(\xi\in\mathbb{R}^d\), there exists a constant \(C(s,\sigma)>0\) such that
\begin{equation} \label{eq:poly-vs-exp}
(1+\|\xi\|^2)^s \le C(s,\sigma)\, e^{\frac{\sigma^2 \|\xi\|^2}{2}},
\end{equation}
because the exponential function grows faster than any fixed polynomial. (A formal way to see this is by noting that for any \(a>0\) and \(s>0\) there exists a constant \(C=C(a,s)>0\) such that \((1+\|\xi\|^2)^s \le C\, e^{a\|\xi\|^2}\) for all \(\xi\in\mathbb{R}^d\); here, take \(a=\sigma^2/2\).)

Combining \eqref{eq:RKHS-norm} and \eqref{eq:poly-vs-exp}, we have
\[
\|f\|_{H^s}^2 = \int_{\mathbb{R}^d} (1+\|\xi\|^2)^s\, |\hat{f}(\xi)|^2\,d\xi 
\le C(s,\sigma) \int_{\mathbb{R}^d} e^{\frac{\sigma^2 \|\xi\|^2}{2}}\, |\hat{f}(\xi)|^2\,d\xi 
= C(s,\sigma) \, \|f\|_{\mathcal{H}}^2.
\]
Taking square roots yields
\[
\|f\|_{H^s} \le \sqrt{C(s,\sigma)}\,\|f\|_{\mathcal{H}}, \quad \forall f\in \mathcal{H}.
\]
This shows that the embedding \(\mathcal{H} \hookrightarrow H^s(\mathbb{R}^d)\) is continuous.

\end{proof}

\fi

