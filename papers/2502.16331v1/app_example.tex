\section{A sequence with diverging $\ell_1$ norm and converging RKHS norm}\label{app: diverging}
In this Appendix, we provide the proof of \exmref{exam: divergence}. We state the example as a lemma and prove it.

\begin{lemma}
Let $\alpha_n = \frac{1}{n}$ and suppose that the points $x_n \in \mathbb{R}^d$ satisfy 
\[
\|x_i-x_j\| \ge |i-j|\delta \quad \text{for some } \delta>0 \text{ and for all } i,j \in \mathbb{N}.
\]
Then the function
\[
f(\bm{x})=\sum_{n=1}^\infty \frac{1}{n}\, k(\bm{x},\bm{x}_n),
\]
with the Gaussian kernel 
\[
k(\bm{x},\bm{y})=\exp\Bigl(-\frac{\|\bm{x}-\bm{y}\|^2}{2\sigma^2}\Bigr),
\]
has finite RKHS norm
\[
\|f\|_{\mathcal{H}}^2 = \sum_{i,j=1}^\infty \frac{1}{ij}\, k(\bm{x}_i,\bm{x}_j) < \infty,
\]
even though 
\[
\|a\|_1 = \sum_{n=1}^\infty \frac{1}{n} = \infty.
\]
\end{lemma}

\begin{proof}
We begin by splitting the double series defining the RKHS norm into diagonal and off--diagonal parts:
\[
\|f\|_{\mathcal{H}}^2 = \sum_{i=1}^\infty \frac{1}{i^2}\, k(\bm{x}_i,\bm{x}_i)
+\sum_{i\neq j} \frac{1}{ij}\, k(\bm{x}_i,\bm{x}_j).
\]
Since $k(\bm{x},\bm{x})=1$ for all $\bm{x}\in\mathbb{R}^d$, the diagonal contribution is
\[
S_{\text{diag}} = \sum_{i=1}^\infty \frac{1}{i^2},
\]
which converges (indeed, $\sum_{i=1}^\infty \frac{1}{i^2}=\pi^2/6$).

For the off--diagonal part, define
\[
S_{\text{off}} = \sum_{i\neq j} \frac{1}{ij}\, k(\bm{x}_i,\bm{x}_j).
\]
By symmetry and non-negativity of $k(\bm{x}_i,\bm{x}_j)$, we can write
\[
S_{\text{off}} = 2\sum_{i>j} \frac{1}{ij}\, k(\bm{x}_i,\bm{x}_j).
\]
For $i>j$, the separation condition implies
\[
\|\bm{x}_i-\bm{x}_j\| \ge (i-j)\delta,
\]
so that
\[
k(\bm{x}_i,\bm{x}_j) = \exp\Bigl(-\frac{\|\bm{x}_i-\bm{x}_j\|^2}{2\sigma^2}\Bigr)
\le \exp\Bigl(-\frac{((i-j)\delta)^2}{2\sigma^2}\Bigr).
\]
Setting 
\[
k = i - j \quad (k\ge 1)
\]
and writing $i=j+k$, we obtain
\[
S_{\text{off}} \le 2\sum_{k=1}^\infty \exp\Bigl(-\frac{(k\delta)^2}{2\sigma^2}\Bigr)
\sum_{j=1}^\infty \frac{1}{j(j+k)}.
\]

We now analyze the inner sum. Using partial fractions,
\[
\frac{1}{j(j+k)} = \frac{1}{k}\Bigl(\frac{1}{j} - \frac{1}{j+k}\Bigr).
\]
Thus,
\[
\sum_{j=1}^\infty \frac{1}{j(j+k)} = \frac{1}{k} \sum_{j=1}^\infty \left(\frac{1}{j} - \frac{1}{j+k}\right).
\]
The telescoping sum yields
\[
\sum_{j=1}^\infty \left(\frac{1}{j} - \frac{1}{j+k}\right)
=\sum_{j=1}^{k} \frac{1}{j} = H_k,
\]
where $H_k$ denotes the $k$th harmonic number. Hence,
\[
\sum_{j=1}^\infty \frac{1}{j(j+k)} = \frac{H_k}{k}.
\]
It follows that
\[
S_{\text{off}} \le 2\sum_{k=1}^\infty \exp\Bigl(-\frac{(k\delta)^2}{2\sigma^2}\Bigr) \frac{H_k}{k}.
\]

For large $k$, we have the asymptotic
\[
H_k = \ln k + \gamma + o(1),
\]
with $\gamma$ the Euler--Mascheroni constant. Moreover, the Gaussian factor 
\[
\exp\Bigl(-\frac{(k\delta)^2}{2\sigma^2}\Bigr)
\]
decays super--exponentially in $k$. Therefore, the series
\[
\sum_{k=1}^\infty \exp\Bigl(-\frac{(k\delta)^2}{2\sigma^2}\Bigr) \frac{H_k}{k}
\]
converges absolutely.

Combining the diagonal and off--diagonal parts, we deduce that
\[
\|f\|_{\mathcal{H}}^2 \le \sum_{i=1}^\infty \frac{1}{i^2} + 2\sum_{k=1}^\infty \exp\Bigl(-\frac{(k\delta)^2}{2\sigma^2}\Bigr) \frac{H_k}{k} < \infty.
\]
Thus, the RKHS norm of $f$ is finite, even though
\[
\sum_{n=1}^\infty \frac{1}{n} = \infty.
\]
\end{proof}