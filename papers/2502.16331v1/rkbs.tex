\section{Infinite dimensional RKBS}

We will prove that the Banach space
\[
\mathcal{B} = \left\{ f : f(x) = \sum_{i=1}^\infty \alpha_i\,\lambda_i\, e_i(x),\quad \|f\|_{\mathcal{B}} = \left(\sum_{i=1}^\infty |\alpha_i\,\lambda_i|^p\right)^{1/p} < \infty \right\}
\]
has the following properties:
\begin{itemize}
    \item It is complete (i.e. it is a Banach space).
    \item Each evaluation functional \(\delta_x: f\mapsto f(x)\) is continuous.
    \item Its dual space is naturally given by
   \[
   \mathcal{B}^* \cong \left\{ g : g(x)=\sum_{i=1}^\infty \beta_i\, e_i'(x),\quad \|g\|_{\mathcal{B}^*} = \left(\sum_{i=1}^\infty \Bigl|\frac{\beta_i}{\lambda_i}\Bigr|^q\right)^{1/q} < \infty \right\},
   \]
   where \(1/p+1/q=1\) and the dual functionals \(e_i'\) satisfy \(e_i'(e_j)=\delta_{ij}\).
\end{itemize}
 

In what follows we will give full proofs of these facts.


\paragraph{Completeness of \(\mathcal{B}\)}

First, we show that $\cB$  has an isometric identification with \(\ell^p\)

Every \(f\in\mathcal{B}\) has a unique expansion
\[
f(x)= \sum_{i=1}^\infty \alpha_i\,\lambda_i\, e_i(x)
\]
with norm
\[
\|f\|_{\mathcal{B}} = \left(\sum_{i=1}^\infty |\alpha_i\,\lambda_i|^p\right)^{1/p}.
\]
Define the linear mapping
\[
T:\mathcal{B}\to \ell^p,\quad T(f)=(\alpha_i\,\lambda_i)_{i\ge1}.
\]
Then
\[
\|T(f)\|_{\ell^p} = \left(\sum_{i=1}^\infty |\alpha_i\,\lambda_i|^p\right)^{1/p} = \|f\|_{\mathcal{B}},
\]
so \(T\) is an isometry (i.e. it preserves the norm).

Since \(\ell^p\) is a well-known Banach space (i.e. it is complete), the image \(T(\mathcal{B})\) is a closed subspace of \(\ell^p\) (or, at least, one may complete the space and then identify it with a closed subspace). Therefore, every Cauchy sequence \(\{f_n\}\) in \(\mathcal{B}\) is such that \(\{T(f_n)\}\) is a Cauchy sequence in \(\ell^p\) and hence converges in \(\ell^p\) to some limit \(c=(c_i)_{i\ge1}\). Defining \(\alpha_i\) by \(c_i=\alpha_i\,\lambda_i\) (with the obvious convention when \(\lambda_i=0\)), we see that there exists an \(f\in\mathcal{B}\) with \(T(f)=c\). Thus, \(f_n\to f\) in \(\mathcal{B}\). In other words, \(\mathcal{B}\) is complete.

---

\paragraph{Continuity of the Evaluation Functional}

Let \(x\) be an arbitrary point in the domain. We want to show that the evaluation functional
\[
\delta_x : \mathcal{B} \to \mathbb{K},\quad \delta_x(f)=f(x)
\]
is continuous. (Here, \(\mathbb{K}\) denotes \(\mathbb{R}\) or \(\mathbb{C}\).)

Step 2.1. Direct Estimate Using Hölder’s Inequality

Given \(f\in\mathcal{B}\) with
\[
f(x)=\sum_{i=1}^\infty \alpha_i\,\lambda_i\, e_i(x),
\]
by Hölder’s inequality we have
\[
|f(x)| \le \sum_{i=1}^\infty |\alpha_i\,\lambda_i|\, |e_i(x)|
\le \left(\sum_{i=1}^\infty |\alpha_i\,\lambda_i|^p\right)^{1/p}\left(\sum_{i=1}^\infty |e_i(x)|^q\right)^{1/q}.
\]
That is,
\[
|f(x)| \le \|f\|_{\mathcal{B}}\,\left(\sum_{i=1}^\infty |e_i(x)|^q\right)^{1/q}.
\]
Thus, the evaluation functional satisfies
\[
|\delta_x(f)| \le C(x)\|f\|_{\mathcal{B}},\quad \text{with } C(x)=\left(\sum_{i=1}^\infty |e_i(x)|^q\right)^{1/q}.
\]
In order for this estimate to be useful, we assume that for every \(x\) the series
\[
\sum_{i=1}^\infty |e_i(x)|^q
\]
converges. (This is a common assumption when one constructs a reproducing kernel Banach space from a given basis.)

Step 2.2. Representer of Evaluation

An equivalent way to show that \(\delta_x\) is continuous is to exhibit a representer in the dual space \(\mathcal{B}^*\). For each \(x\), define
\[
K_x(\cdot) = \sum_{i=1}^\infty \lambda_i\, e_i(x)\, e_i'(\cdot).
\]
Here, the dual functionals \(e_i'\) satisfy \(e_i'(e_j)=\delta_{ij}\). Then for any \(f\in \mathcal{B}\) with expansion
\[
f(x)= \sum_{i=1}^\infty \alpha_i\,\lambda_i\, e_i(x),
\]
the duality pairing is
\[
\langle f, K_x\rangle = \sum_{i=1}^\infty \alpha_i\,\lambda_i\,[\lambda_i\, e_i(x)]
= \sum_{i=1}^\infty \alpha_i\,\lambda_i\, e_i(x)=f(x).
\]
Thus, \(K_x\) is a representer of the evaluation functional, and its norm in \(\mathcal{B}^*\) is
\[
\|K_x\|_{\mathcal{B}^*} = \left(\sum_{i=1}^\infty \Bigl|\frac{\lambda_i\,e_i(x)}{\lambda_i}\Bigr|^q\right)^{1/q}
=\left(\sum_{i=1}^\infty |e_i(x)|^q\right)^{1/q}<\infty,
\]
again provided that \(\sum_i|e_i(x)|^q<\infty\). Hence, the evaluation functional \(\delta_x\) is continuous.

---

3. Identification of the Dual Space \(\mathcal{B}^*\)

We define
\[
\mathcal{B}^* = \left\{ g : g(x)=\sum_{i=1}^\infty \beta_i\, e_i'(x),\quad \|g\|_{\mathcal{B}^*} = \left(\sum_{i=1}^\infty \Bigl|\frac{\beta_i}{\lambda_i}\Bigr|^q\right)^{1/q} < \infty \right\}.
\]
The duality pairing between \(f\in\mathcal{B}\) and \(g\in \mathcal{B}^*\) is defined by
\[
\langle f, g\rangle = \sum_{i=1}^\infty \alpha_i\,\beta_i,
\]
where \(f(x)=\sum_{i=1}^\infty \alpha_i\,\lambda_i\, e_i(x)\) and \(g(x)=\sum_{i=1}^\infty \beta_i\, e_i'(x)\).

Step 3.1. Isometric Identification with \(\ell^q\)

Define the mapping
\[
S:\mathcal{B}^* \to \ell^q,\quad S(g)= \Bigl(\frac{\beta_i}{\lambda_i}\Bigr)_{i\ge1}.
\]
Then
\[
\|S(g)\|_{\ell^q} = \left(\sum_{i=1}^\infty \Bigl|\frac{\beta_i}{\lambda_i}\Bigr|^q\right)^{1/q} = \|g\|_{\mathcal{B}^*}.
\]
Since the dual of \(\ell^p\) (with \(1<p<\infty\)) is isometrically isomorphic to \(\ell^q\), and we already identified \(\mathcal{B}\) with a subspace of \(\ell^p\) via
\[
T(f) = (\alpha_i\lambda_i)_{i\ge1},
\]
it follows that every continuous linear functional on \(\mathcal{B}\) is given by an element \(g\in \mathcal{B}^*\) with the pairing
\[
\langle f, g\rangle = \sum_{i=1}^\infty (\alpha_i\lambda_i)\Bigl(\frac{\beta_i}{\lambda_i}\Bigr)= \sum_{i=1}^\infty \alpha_i\,\beta_i.
\]
Thus, the space \(\mathcal{B}^*\) defined in this way is exactly the dual of \(\mathcal{B}\).

---

 Conclusion

We have shown:

1. **Completeness:** The mapping
   \[
   T: f\mapsto (\alpha_i\,\lambda_i)
   \]
   is an isometry from \(\mathcal{B}\) into \(\ell^p\); since \(\ell^p\) is complete, it follows that \(\mathcal{B}\) is complete.

2. **Continuity of Evaluation:** For each \(x\) we have
   \[
   |f(x)|\le \|f\|_{\mathcal{B}}\,\Bigl(\sum_{i=1}^\infty |e_i(x)|^q\Bigr)^{1/q},
   \]
   so the evaluation functional \(\delta_x\) is continuous provided that
   \(\sum_{i=1}^\infty |e_i(x)|^q<\infty\). Equivalently, one may represent the evaluation by
   \[
   K_x(\cdot)=\sum_{i=1}^\infty \lambda_i\, e_i(x)\, e_i'(\cdot)\in\mathcal{B}^*,
   \]
   and then \(\langle f, K_x\rangle=f(x)\).

3. **Dual Space:** With the identification
   \[
   S: g\mapsto \Bigl(\frac{\beta_i}{\lambda_i}\Bigr)_{i\ge1},
   \]
   we see that the duality pairing
   \[
   \langle f, g\rangle = \sum_{i=1}^\infty \alpha_i\,\beta_i
   \]
   is exactly the standard \(\ell^p\)-\(\ell^q\) pairing, so that
   \[
   \mathcal{B}^* = \left\{ g : g(x)=\sum_{i=1}^\infty \beta_i\, e_i'(x),\quad \|g\|_{\mathcal{B}^*}=\left(\sum_{i=1}^\infty \Bigl|\frac{\beta_i}{\lambda_i}\Bigr|^q\right)^{1/q}<\infty \right\}
   \]
   is the dual of \(\mathcal{B}\).

Thus, the Banach space \(\mathcal{B}\) as defined is complete, every evaluation functional is continuous (under the summability assumption on the basis functions), and its dual is exactly the space \(\mathcal{B}^*\) described above.

Any answer that establishes these three points in a manner equivalent to the proof above is correct.

\newpage

\section*{Introduction}

Let $\mathcal{X}$ be a nonempty set. Suppose we are given two feature maps
\[
\psi:\mathcal{X}\to \mathcal{Y} \quad \text{and} \quad \phi:\mathcal{X}\to \mathcal{Z},
\]
where $\mathcal{Y}$ and $\mathcal{Z}$ are Banach spaces (possibly infinite dimensional). Also, fix a countable set of sampling points $\{x_j\}_{j\ge1}\subset \mathcal{X}$. For a sequence $\{\alpha_j\}\subset \mathbb{K}$ (with $\mathbb{K}=\mathbb{R}$ or $\mathbb{C}$), define
\[
f(x)= \sum_{j=1}^\infty \alpha_j\, \langle \psi(x_j), \phi(x)\rangle, \quad x\in \mathcal{X},
\]
where $\langle \cdot,\cdot\rangle$ denotes the dual pairing between $\mathcal{Y}$ and $\mathcal{Z}$ (or, in a Hilbert space setting, the inner product).

Notice that if we define
\[
\beta := \sum_{j=1}^\infty \alpha_j\, \psi(x_j)\in \mathcal{Y},
\]
then
\[
f(x)= \langle \beta, \phi(x) \rangle.
\]
We define the norm on our space $\mathcal{B}$ by
\[
\|f\| := \|\beta\|_{\mathcal{Y}}.
\]
Thus, we consider the space
\[
\mathcal{B} := \left\{ f:\mathcal{X}\to \mathbb{K}\,\Bigm|\, f(x)= \sum_{j=1}^\infty \alpha_j\, \langle \psi(x_j), \phi(x)\rangle,\quad \|f\| = \Bigl\|\sum_{j=1}^\infty \alpha_j\, \psi(x_j)\Bigr\|_{\mathcal{Y}} < \infty \right\}.
\]

We assume:
\begin{enumerate}
  \item For every $f\in\mathcal{B}$ the series defining $\beta=\sum_{j\ge1}\alpha_j\, \psi(x_j)$ converges in $\mathcal{Y}$.
  \item For each $x\in\mathcal{X}$, $\phi(x)\in \mathcal{Y}^*$ (or at least $\phi(x)$ acts boundedly on $\mathcal{Y}$) so that the pairing $f(x)=\langle \beta, \phi(x)\rangle$ is well defined. In particular, there exists $C(x)<\infty$ such that
  \[
  |f(x)| \le \|\beta\|_{\mathcal{Y}}\, C(x) = C(x)\,\|f\|.
  \]
\end{enumerate}

Define the reproducing kernel by
\[
K(x,y)= \langle \psi(x), \phi(y)\rangle.
\]

\section*{Main Result}

\begin{theorem}
Under the above assumptions, the space
\[
\mathcal{B} = \left\{ f:\mathcal{X}\to \mathbb{K}\,\Bigm|\, f(x)= \sum_{j=1}^\infty \alpha_j\, \langle \psi(x_j), \phi(x)\rangle,\quad \|f\| = \Bigl\|\sum_{j=1}^\infty \alpha_j\, \psi(x_j)\Bigr\|_{\mathcal{Y}} < \infty \right\}
\]
is a reproducing kernel Banach space (RKBS) with reproducing kernel
\[
K(x,y)= \langle \psi(x), \phi(y)\rangle.
\]
Moreover, for every $x\in\mathcal{X}$, the evaluation functional
\[
\delta_x: \mathcal{B}\to \mathbb{K},\quad \delta_x(f)= f(x)
\]
is continuous, and the dual space $\mathcal{B}^*$ may be identified (via the isometry 
\[
T: \mathcal{B}\to \mathcal{Y},\quad T(f)= \sum_{j\ge1}\alpha_j\,\psi(x_j)
\]
) with a subspace of $\mathcal{Y}^*$.
\end{theorem}

\begin{proof}
\textbf{(Completeness.)}  
Define
\[
T:\mathcal{B}\to \mathcal{Y},\quad T(f)= \beta = \sum_{j=1}^\infty \alpha_j\, \psi(x_j).
\]
By definition,
\[
\|f\| = \|\beta\|_{\mathcal{Y}}.
\]
Since $\mathcal{Y}$ is a Banach space, every Cauchy sequence $\{f_n\}\subset \mathcal{B}$ gives rise to a Cauchy sequence $\{T(f_n)\}$ in $\mathcal{Y}$ which converges to some $\beta\in \mathcal{Y}$. (Assuming that the representation is unique or that we identify functions yielding the same $\beta$, we obtain $f\in \mathcal{B}$ with $T(f)=\beta$.) Hence, $\mathcal{B}$ is complete.

\medskip

\textbf{(Continuity of Pointwise Evaluation.)}  
Fix $x\in\mathcal{X}$. For any $f\in\mathcal{B}$ we have
\[
f(x)= \langle \beta, \phi(x)\rangle.
\]
Since $\phi(x)\in \mathcal{Y}^*$ (or at least acts continuously on $\mathcal{Y}$), there exists a constant $C(x)=\|\phi(x)\|_{\mathcal{Y}^*}$ such that
\[
|f(x)| \le \|\beta\|_{\mathcal{Y}}\, \|\phi(x)\|_{\mathcal{Y}^*} = \|f\|\, \|\phi(x)\|_{\mathcal{Y}^*}.
\]
Thus, the evaluation functional $\delta_x: f\mapsto f(x)$ is bounded.

\medskip

\textbf{(Reproducing Kernel Property.)}  
Define, for each $x\in\mathcal{X}$, the kernel section
\[
K_x(\cdot)= K(x,\cdot), \quad \text{where} \quad K(x,y)= \langle \psi(x), \phi(y)\rangle.
\]
Then for any $f\in\mathcal{B}$ with $f(x)= \langle \beta, \phi(x)\rangle$, we have
\[
\langle f, K_x \rangle := \langle T(f), \phi(x) \rangle = \langle \beta, \phi(x)\rangle = f(x).
\]
Hence, the reproducing property holds.

\medskip

\textbf{(Dual Space.)}  
Since
\[
T:\mathcal{B}\to \mathcal{Y},\quad f\mapsto \beta,
\]
is an isometry, every continuous linear functional $L$ on $\mathcal{B}$ induces a continuous linear functional on $T(\mathcal{B})\subset \mathcal{Y}$. By the Hahn--Banach theorem, such an $L$ has the representation
\[
L(f)= \langle \beta, \gamma \rangle, \quad \text{for some } \gamma\in \mathcal{Y}^*.
\]
Thus, the dual space $\mathcal{B}^*$ is isometrically isomorphic to a subspace of $\mathcal{Y}^*$. In particular, for each $x\in \mathcal{X}$ the kernel section $K_x$ belongs to $\mathcal{B}^*$ and satisfies
\[
f(x)= \langle f, K_x\rangle.
\]

This completes the proof that $\mathcal{B}$ is a reproducing kernel Banach space with kernel
\[
K(x,y)= \langle \psi(x), \phi(y)\rangle,
\]
with continuous pointwise evaluation, and with dual space identified (via $T$) with a subspace of $\mathcal{Y}^*$.
\end{proof}

\bigskip

\noindent
\textbf{Summary:} Under the assumptions that
\begin{enumerate}
  \item The series $\beta = \sum_{j=1}^\infty \alpha_j\, \psi(x_j)$ converges in the Banach space $\mathcal{Y}$,
  \item For each $x\in\mathcal{X}$, the element $\phi(x)$ belongs to $\mathcal{Y}^*$ (or acts continuously on $\mathcal{Y}$),
\end{enumerate}
the function space
\[
\mathcal{B} = \left\{ f:\mathcal{X}\to \mathbb{K}\,\Bigm|\, f(x)= \sum_{j=1}^\infty \alpha_j\, \langle \psi(x_j), \phi(x)\rangle,\quad \|f\| = \Bigl\|\sum_{j=1}^\infty \alpha_j\, \psi(x_j)\Bigr\|_{\mathcal{Y}} < \infty \right\}
\]
is a reproducing kernel Banach space (RKBS) with reproducing kernel
\[
K(x,y)=\langle \psi(x), \phi(y)\rangle.
\]
Moreover, the point evaluation $f\mapsto f(x)$ is continuous with norm at most $\|\phi(x)\|_{\mathcal{Y}^*}$, and the dual space $\mathcal{B}^*$ is isometrically isomorphic to a subspace of $\mathcal{Y}^*$.

-----------------------------------------------------------------------\\

\section{Introduction}

Let \(\mathcal{X}\) be a nonempty set. Suppose that we are given two maps (which we shall call feature maps)
\[
\psi : \mathcal{X} \to \mathcal{Y} \quad \text{and} \quad \phi : \mathcal{X} \to \mathcal{Z},
\]
where \(\mathcal{Y}\) is a Banach space and \(\mathcal{Z}\) is such that each \(\phi(x)\) acts continuously on \(\mathcal{Y}\). (For example, one may take \(\mathcal{Z}=\mathcal{Y}^*\).) In many concrete cases one chooses \(\mathcal{Y} = \ell^p\) for some \(1\le p<\infty\); then the norm on \(\mathcal{B}\) below will be an \(\ell^p\) norm.

We also fix a countable set of sampling points \(\{x_j\}_{j\ge1}\subset \mathcal{X}\). For any sequence \(\{\alpha_j\}\subset \mathbb{K}\) (with \(\mathbb{K}=\mathbb{R}\) or \(\mathbb{C}\)), define a function \(f:\mathcal{X}\to \mathbb{K}\) by
\[
f(x)= \sum_{j=1}^\infty \alpha_j\, \langle \psi(x_j), \phi(x)\rangle, \quad x\in \mathcal{X},
\]
where \(\langle\cdot,\cdot\rangle\) denotes the dual pairing (or inner product in the Hilbert space case). Notice that if we define
\[
\beta := \sum_{j=1}^\infty \alpha_j\, \psi(x_j) \in \mathcal{Y},
\]
then
\[
f(x)= \langle \beta, \phi(x)\rangle.
\]

\section{The Function Space \(\mathcal{B}\) and the Operator \(T\)}

We define the function space
\[
\mathcal{B} := \left\{ f:\mathcal{X}\to \mathbb{K}\,\Bigm|\, f(x)= \langle \beta, \phi(x)\rangle \text{ for some } \beta = \sum_{j=1}^\infty \alpha_j\, \psi(x_j)\in \mathcal{Y},\quad \|f\| := \|\beta\|_{\mathcal{Y}} < \infty \right\}.
\]
Thus, the norm on \(\mathcal{B}\) is given by
\[
\|f\|_{\mathcal{B}} = \|\beta\|_{\mathcal{Y}},
\]
which, for example, is the \(\ell^p\) norm when \(\mathcal{Y} = \ell^p\).

Define the linear operator
\[
T:\mathcal{B}\to \mathcal{Y}, \quad T(f) = \beta.
\]
By construction, \(T\) is an isometry, that is,
\[
\|f\|_{\mathcal{B}} = \|T(f)\|_{\mathcal{Y}}.
\]

\section{The Reproducing Kernel}

We define the kernel function
\[
K:\mathcal{X}\times\mathcal{X}\to \mathbb{K},\quad K(x,y)= \langle \psi(x), \phi(y)\rangle.
\]
For each fixed \(y\in \mathcal{X}\), the function
\[
K(\cdot,y): \, x \mapsto \langle \psi(x), \phi(y)\rangle
\]
belongs to \(\mathcal{B}\) (under appropriate assumptions on the feature maps). In many constructions one further assumes that
\[
T\bigl(K(\cdot,y)\bigr)= \psi(y),
\]
so that the representation of the kernel section is consistent with the feature map \(\psi\).

\section{Duality and the Bilinear Pairing}

The dual space \(\mathcal{B}^*\) can be identified with a subspace of \(\mathcal{Y}^*\) via the isometry \(T\). More precisely, for every \(g\in \mathcal{B}^*\) there exists a unique \(\gamma\in \mathcal{Y}^*\) such that for all \(f\in \mathcal{B}\)
\[
g(f)= \langle T(f), \gamma\rangle_{\mathcal{Y}\times\mathcal{Y}^*}.
\]
Thus, we define the bilinear pairing
\[
\langle f, g \rangle_{\mathcal{B}\times \mathcal{B}^*} := \langle T(f), \gamma\rangle_{\mathcal{Y}\times\mathcal{Y}^*}.
\]
For example, if \(\mathcal{Y}=\ell^p\) then \(\mathcal{Y}^*=\ell^q\) with \(1/p+1/q=1\).

\section{Main Result: Reproducing Properties of the Kernel}

\begin{theorem}
Under the above assumptions, the kernel
\[
K(x,y)=\langle \psi(x), \phi(y)\rangle,
\]
satisfies the following reproducing properties:
\begin{enumerate}
    \item For every \(f\in\mathcal{B}\) and every \(x\in\mathcal{X}\),
    \[
    \langle f, K(x,\cdot) \rangle_{\mathcal{B}\times \mathcal{B}^*} = f(x).
    \]
    \item For every \(g\in\mathcal{B}^*\) (with corresponding \(\gamma\in \mathcal{Y}^*\)) and every \(y\in\mathcal{X}\),
    \[
    \langle K(\cdot,y), g \rangle_{\mathcal{B}\times \mathcal{B}^*} = g(y).
    \]
\end{enumerate}
\end{theorem}

\begin{proof}
We prove each reproducing property in turn.

\medskip

\noindent\textbf{(1) Reproducing Property for Functions in \(\mathcal{B}\):}  

Let \(f\in \mathcal{B}\) with representation
\[
f(x)= \langle T(f), \phi(x)\rangle, \quad \text{where } T(f)= \beta \in \mathcal{Y}.
\]
For a fixed \(x\in\mathcal{X}\), consider the kernel section \(K(x,\cdot)\). We define the pairing with \(f\) by choosing the dual element \(\phi(x)\in \mathcal{Y}^*\). That is, by definition of the bilinear pairing,
\[
\langle f, K(x,\cdot)\rangle_{\mathcal{B}\times \mathcal{B}^*} := \langle T(f), \phi(x)\rangle_{\mathcal{Y}\times \mathcal{Y}^*}.
\]
Hence,
\[
\langle f, K(x,\cdot)\rangle_{\mathcal{B}\times \mathcal{B}^*} = \langle \beta, \phi(x)\rangle = f(x).
\]
This establishes the first reproducing property.

\medskip

\noindent\textbf{(2) Reproducing Property for Functionals in \(\mathcal{B}^*\):}  

Let \(g\in\mathcal{B}^*\). By the identification of \(\mathcal{B}^*\) with a subspace of \(\mathcal{Y}^*\), there exists \(\gamma\in \mathcal{Y}^*\) such that for all \(f\in\mathcal{B}\)
\[
g(f)= \langle T(f), \gamma\rangle_{\mathcal{Y}\times \mathcal{Y}^*}.
\]
We now wish to compute the pairing
\[
\langle K(\cdot,y), g \rangle_{\mathcal{B}\times \mathcal{B}^*}.
\]
By definition, we have
\[
\langle K(\cdot,y), g \rangle_{\mathcal{B}\times \mathcal{B}^*} := \langle T(K(\cdot,y)), \gamma\rangle_{\mathcal{Y}\times \mathcal{Y}^*}.
\]
By assumption on the representation of the kernel sections, we have
\[
T\bigl(K(\cdot,y)\bigr)= \psi(y).
\]
Thus,
\[
\langle K(\cdot,y), g \rangle_{\mathcal{B}\times \mathcal{B}^*} = \langle \psi(y), \gamma\rangle.
\]
But, by the very definition of the action of \(g\) on the evaluation function at \(y\), we have
\[
g\bigl(K(\cdot,y)\bigr) = \langle \psi(y), \gamma\rangle.
\]
Moreover, using the reproducing property already established for functions, one shows that the value of the functional \(g\) at the point \(y\) (i.e., when applied to the evaluation function) satisfies
\[
g(y) = \langle \psi(y), \gamma\rangle.
\]
Hence,
\[
\langle K(\cdot,y), g \rangle_{\mathcal{B}\times \mathcal{B}^*} = g(y),
\]
which is the desired reproducing property for the dual.
\end{proof}

\section{Summary}

To summarize, we have defined the space
\[
\mathcal{B} = \left\{ f:\mathcal{X}\to \mathbb{K}\,\Bigm|\, f(x)= \langle \beta, \phi(x)\rangle,\quad \beta= \sum_{j=1}^\infty \alpha_j\,\psi(x_j)\in\mathcal{Y},\quad \|f\|_{\mathcal{B}}=\|\beta\|_{\mathcal{Y}} < \infty \right\},
\]
with the isometry
\[
T:\mathcal{B}\to \mathcal{Y},\quad T(f)=\beta.
\]
We have then defined the reproducing kernel
\[
K(x,y)=\langle \psi(x), \phi(y)\rangle,
\]
and shown that the bilinear pairing
\[
\langle f, g \rangle_{\mathcal{B}\times \mathcal{B}^*} := \langle T(f), \gamma\rangle_{\mathcal{Y}\times \mathcal{Y}^*} \quad \text{(with } g\in \mathcal{B}^* \text{ corresponding to } \gamma\in \mathcal{Y}^*\text{)}
\]
satisfies
\[
\langle f, K(x,\cdot) \rangle_{\mathcal{B}\times \mathcal{B}^*} = f(x)
\]
and
\[
\langle K(\cdot,y), g \rangle_{\mathcal{B}\times \mathcal{B}^*} = g(y).
\]
Thus, the kernel \(K\) is indeed a reproducing kernel for the space \(\mathcal{B}\).

\newpage


\section{Infinite dimensional RKBS}
Let $\mathcal{X}$ be a nonempty set. Suppose we are given two feature maps
\[
\psi:\mathcal{X}\to \mathcal{Y} \quad \text{and} \quad \phi:\mathcal{X}\to \mathcal{Z},
\]
where $\mathcal{Y}$ and $\mathcal{Z}$ are Banach spaces (possibly infinite dimensional). Also, fix a countable set of sampling points ${x_j}{j\ge1}\subset \mathcal{X}$. For a sequence ${\alpha_j}\subset \mathbb{K}$ (with $\mathbb{K}=\mathbb{R}$ or $\mathbb{C}$), define
\[
f(x)= \sum{j=1}^\infty \alpha_j, \langle \psi(x_j), \phi(x)\rangle, \quad x\in \mathcal{X},
\]
where $\langle \cdot,\cdot\rangle$ denotes the dual pairing between $\mathcal{Y}$ and $\mathcal{Z}$ (or, in a Hilbert space setting, the inner product).
\begin{definition}
For any $f \in \mathcal{B}$ with representation as above, we define
\[
\beta := \sum_{j=1}^\infty \alpha_j, \psi(x_j)\in \mathcal{Y},
\]
so that
\[
f(x)= \langle \beta, \phi(x) \rangle.
\]
The norm on our space $\mathcal{B}$ is defined by
\[
\|f\| := \|\beta\|_{\mathcal{Y}}.
\]
\end{definition}
Thus, we consider the space
\begin{align*}
    \mathcal{B} := \{ f:\mathcal{X}\to \mathbb{K},\Bigm|, f(x)= \sum_{j=1}^\infty \alpha_j, \langle \psi(x_j), \phi(x)\rangle,\quad |f| = \Bigl|\sum_{j=1}^\infty \alpha_j, \psi(x_j)\Bigr|_{\mathcal{Y}} < \infty \}.
\end{align*}
\begin{definition}[Standing Assumptions]
We assume:
\begin{enumerate}
\item For every $f\in\mathcal{B}$ the series defining $\beta=\sum_{j\ge1}\alpha_j, \psi(x_j)$ converges in $\mathcal{Y}$.
\item For each $x\in\mathcal{X}$, $\phi(x)\in \mathcal{Y}^*$ and there exists $C(x)<\infty$ such that
\[
|f(x)| \le \|\beta\|_{\mathcal{Y}} \cdot C(x) = C(x)\cdot \|f\|.
\]
\end{enumerate}
\end{definition}
\section{Main Results}
\begin{theorem}[Fundamental RKBS Properties]
Under the above assumptions, the space $\mathcal{B}$ is a reproducing kernel Banach space (RKBS) with reproducing kernel
\[
K(x,y)= \langle \psi(x), \phi(y)\rangle.
\]
Moreover:
\begin{enumerate}
\item For every $x\in\mathcal{X}$, the evaluation functional $\delta_x: \mathcal{B}\to \mathbb{K}$, defined by $\delta_x(f)= f(x)$, is continuous.
\item The dual space $\mathcal{B}^*$ may be identified (via the isometry $T: \mathcal{B}\to \mathcal{Y}$, $T(f)= \sum_{j\ge1}\alpha_j,\psi(x_j)$) with a subspace of $\mathcal{Y}^*$.
\end{enumerate}
\end{theorem}
\begin{proof}
We proceed in steps:
\medskip

\noindent\textbf{Step 1: Completeness.} \\
Define $T:\mathcal{B}\to \mathcal{Y}$ by $T(f)= \beta = \sum_{j=1}^\infty \alpha_j, \psi(x_j)$. By definition,
\[
|f| = |\beta|_{\mathcal{Y}}.
\]
Since $\mathcal{Y}$ is a Banach space, every Cauchy sequence ${f_n}\subset \mathcal{B}$ gives rise to a Cauchy sequence ${T(f_n)}$ in $\mathcal{Y}$ which converges to some $\beta\in \mathcal{Y}$. Hence, $\mathcal{B}$ is complete.
\medskip

\noindent\textbf{Step 2: Continuity of Pointwise Evaluation.} \\
Fix $x\in\mathcal{X}$. For any $f\in\mathcal{B}$ we have $f(x)= \langle \beta, \phi(x)\rangle$. Since $\phi(x)\in \mathcal{Y}^*$, there exists $C(x)=|\phi(x)|_{\mathcal{Y}^*}$ such that
\[
|f(x)| \le |\beta|{\mathcal{Y}}, |\phi(x)|{\mathcal{Y}^*} = |f|, |\phi(x)|_{\mathcal{Y}^*}.
\]
Thus, $\delta_x$ is bounded.
\medskip

\noindent\textbf{Step 3: Reproducing Kernel Property.} \\
Define kernel sections $K_x(\cdot)= K(x,\cdot)$ where $K(x,y)= \langle \psi(x), \phi(y)\rangle$. Then for any $f\in\mathcal{B}$ with $f(x)= \langle \beta, \phi(x)\rangle$,
\[
\langle f, K_x \rangle := \langle T(f), \phi(x) \rangle = \langle \beta, \phi(x)\rangle = f(x).
\]
\medskip
\noindent\textbf{Step 4: Dual Space.} \
Since $T:\mathcal{B}\to \mathcal{Y}$ is an isometry, every continuous linear functional $L$ on $\mathcal{B}$ induces a continuous linear functional on $T(\mathcal{B})\subset \mathcal{Y}$. By the Hahn--Banach theorem,
\[
L(f)= \langle \beta, \gamma \rangle, \quad \text{for some } \gamma\in \mathcal{Y}^*.
\]
Thus, $\mathcal{B}^*$ is isometrically isomorphic to a subspace of $\mathcal{Y}^*$.
\end{proof}
\section{Additional Conditions for Kernel Representation}
To ensure that $T(K(\cdot,y)) = \psi(y)$ holds, we require the following conditions:
\begin{definition}[Kernel Representation Conditions]
For each $y \in \mathcal{X}$:
\begin{enumerate}
\item \textbf{Representation:} There exists a sequence $\{\alpha_j(y)\}_{j\ge1}$ such that
\[
K(x,y) = \sum_{j=1}^\infty \alpha_j(y)\cdot \langle \psi(x_j), \phi(x)\rangle
\]
\item \textbf{Convergence:} The series converges in norm:
\[
\sum_{j=1}^\infty \alpha_j(y)\, \psi(x_j) \xrightarrow{\|\cdot\|_{\mathcal{Y}}} \psi(y)
\]

\item \textbf{Boundedness:} The coefficients satisfy
\[
\Bigl\|\sum_{j=1}^\infty \alpha_j(y)\, \psi(x_j)\Bigr\|_{\mathcal{Y}} = \|\psi(y)\|_{\mathcal{Y}}
\]
\end{enumerate}
\end{definition}
\begin{proposition}
Under the above conditions, $T(K(\cdot,y)) = \psi(y)$ for all $y \in \mathcal{X}$.
\end{proposition}
\begin{proof}
Fix $y \in \mathcal{X}$. By the representation condition,
\[
K(x,y) = \sum_{j=1}^\infty \alpha_j(y), \langle \psi(x_j), \phi(x)\rangle
\]
Hence,
\[
T(K(\cdot,y)) = \sum_{j=1}^\infty \alpha_j(y), \psi(x_j) \xrightarrow{|\cdot|{\mathcal{Y}}} \psi(y)
\]
by the convergence condition. The boundedness condition ensures
\[
|T(K(\cdot,y))|{\mathcal{Y}} = |\psi(y)|_{\mathcal{Y}}
\]
Therefore, $T(K(\cdot,y)) = \psi(y)$.
\end{proof}
\textbf{Remark}:
These conditions are satisfied in several important cases:
\begin{enumerate}
\item When ${\psi(x_j)}{j\ge1}$ forms a frame for $\mathcal{Y}$
\item When $\mathcal{Y}$ is a reproducing kernel Hilbert space and ${x_j}{j\ge1}$ forms a sampling set
\item When $\mathcal{Y} = \ell^p$ and the sampling points are chosen appropriately
\end{enumerate}


\section{Bilinear Pairing and Duality}
The dual space structure leads to a natural bilinear pairing:
\begin{definition}[Bilinear Pairing]
For $f \in \mathcal{B}$ and $g \in \mathcal{B}^*$ with corresponding $\gamma \in \mathcal{Y}^*$, define
\[
\langle f, g \rangle_{\mathcal{B}\times \mathcal{B}^*} := \langle T(f), \gamma\rangle_{\mathcal{Y}\times\mathcal{Y}^*}.
\]
\end{definition}
\begin{theorem}[Reproducing Properties]
The kernel $K(x,y)=\langle \psi(x), \phi(y)\rangle$ satisfies:
\begin{enumerate}
\item For every $f\in\mathcal{B}$ and $x\in\mathcal{X}$,
\[
\langle f, K(x,\cdot) \rangle_{\mathcal{B}\times \mathcal{B}^*} = f(x).
\]
\item For every $g\in\mathcal{B}^*$ and $y\in\mathcal{X}$,
\[
\langle K(\cdot,y), g \rangle_{\mathcal{B}\times \mathcal{B}^*} = g(y).
\]
\end{enumerate}
\end{theorem}
\begin{proof}
For (1), let $f\in \mathcal{B}$ with $f(x)= \langle T(f), \phi(x)\rangle$. Then
\[
\langle f, K(x,\cdot)\rangle_{\mathcal{B}\times \mathcal{B}^**} = \langle T(f), \phi(x)\rangle = f(x).
\]
For (2), let $g\in\mathcal{B}^*$ with corresponding $\gamma\in \mathcal{Y}^*$. Then
\[
\langle K(\cdot,y), g \rangle_{\mathcal{B}\times \mathcal{B}^*} = \langle T(K(\cdot,y)), \gamma\rangle = \langle \psi(y), \gamma\rangle = g(y),
\]
where we used $T(K(\cdot,y)) = \psi(y)$ from our previous results.
\end{proof}


\newpage

\section{Infinite dimensional RKBS}

Let $\mathcal{X}$ be a nonempty set. Suppose we are given two feature maps
\begin{equation}
\psi:\mathcal{X}\to \mathcal{Y} \quad \text{and} \quad \phi:\mathcal{X}\to \mathcal{Z},
\end{equation}
where $\mathcal{Y}$ and $\mathcal{Z}$ are Banach spaces (possibly infinite dimensional).

Let $\mathcal{X}$ be a nonempty set. Suppose we are given two feature maps
\begin{equation}
\psi:\mathcal{X}\to \mathcal{Y} \quad \text{and} \quad \phi:\mathcal{X}\to \mathcal{Y}^*,
\end{equation}
where $\mathcal{Y}$ is a Banach space (possibly infinite dimensional) and $\mathcal{Y}^*$ is its continuous dual space. The pairing $\langle \cdot,\cdot\rangle$ denotes the natural duality pairing between $\mathcal{Y}$ and $\mathcal{Y}^*$.

\begin{definition}[Function Space]
Let $\mathcal{S}$ be the collection of all countable subsets of $\mathcal{X}$. For any $S = \{x_j\}_{j\ge1} \in \mathcal{S}$ and sequence $\{\alpha_j\}\subset \mathbb{K}$ (with $\mathbb{K}=\mathbb{R}$ or $\mathbb{C}$), we can define a function
\begin{equation}
f(x)= \sum_{j=1}^\infty \alpha_j\, \langle \psi(x_j), \phi(x)\rangle, \quad x\in \mathcal{X},
\end{equation}
where $\langle \cdot,\cdot\rangle$ denotes the dual pairing between $\mathcal{Y}$ and $\mathcal{Z}$.

For any such $f$, we define its associated element in $\mathcal{Y}$:
\begin{equation}
\beta_f := \sum_{j=1}^\infty \alpha_j\, \psi(x_j)\in \mathcal{Y},
\end{equation}
so that
\begin{equation}
f(x)= \langle \beta_f, \phi(x) \rangle.
\end{equation}
The norm is defined by
\begin{equation}
||f||_{\mathcal{B}} := ||\beta_f||_{\mathcal{Y}}.
\end{equation}
\end{definition}

We can now define our RKBS:
\begin{align}
\mathcal{B} := \Big\{ & f:\mathcal{X}\to \mathbb{K}\,\Big|\, \exists S \in \mathcal{S}, \{\alpha_j\} \subset \mathbb{K}: \nonumber \\
& f(x)= \sum_{j=1}^\infty \alpha_j\, \langle \psi(x_j), \phi(x)\rangle,\quad ||\beta_f||_{\mathcal{Y}} < \infty \Big\}.
\end{align}

\begin{definition}[Standing Assumptions]
We assume:
\begin{enumerate}
  \item For every $S \in \mathcal{S}$ and sequence $\{\alpha_j\}$ such that the series $\sum_{j\ge1}\alpha_j\, \psi(x_j)$ converges in $\mathcal{Y}$, the function
  \begin{equation}
  f(x) = \langle \beta_f, \phi(x) \rangle, \quad \text{where } \beta_f = \sum_{j\ge1}\alpha_j\, \psi(x_j)
  \end{equation}
  belongs to $\mathcal{B}$.
  
  \item For each $x\in\mathcal{X}$, $\phi(x)\in \mathcal{Y}^*$ and there exists $C(x)<\infty$ such that
  \begin{equation}
  |f(x)| \le ||\beta_f||_{\mathcal{Y}}\cdot C(x) = C(x)\cdot||f||_{\mathcal{B}}.
  \end{equation}

  \item For any $f \in \mathcal{B}$, if $f$ has two representations:
  \begin{equation}
  f(x) = \sum_{j=1}^\infty \alpha_j\, \langle \psi(x_j), \phi(x)\rangle = \sum_{k=1}^\infty \gamma_k\, \langle \psi(y_k), \phi(x)\rangle
  \end{equation}
  then
  \begin{equation}
  \left\|\sum_{j=1}^\infty \alpha_j\, \psi(x_j) - \sum_{k=1}^\infty \gamma_k\, \psi(y_k)\right\|_{\mathcal{Y}} = 0
  \end{equation}
  in $\mathcal{Y}$.
\end{enumerate}
\end{definition}

The third assumption ensures that $||f||_{\mathcal{B}}$ is well-defined (independent of the representation of $f$). 

\begin{remark}
This definition generalizes the classical Gaussian RKHS case where:
\begin{itemize}
    \item $\mathcal{X} = \mathbb{R}^d$
    \item $\mathcal{Y} = \mathcal{Z} = \mathbb{H}$ (a Hilbert space)
    \item $K(x,y) = e^{-||x-y||_2}$ 
    \item Any countable subset of $\mathcal{X}$ can be used as sampling points
\end{itemize}
\end{remark}

\begin{remark}
When $\mathcal{Y} = \ell^p$, the norm $||f||_{\mathcal{B}}$ becomes the $\ell^p$ norm of the coefficient sequence $\{\alpha_j\}$.
\end{remark}


\section{Main Results}

\begin{theorem}[Fundamental RKBS Properties]
Under the above assumptions, the space $\mathcal{B}$ is a reproducing kernel Banach space (RKBS) with reproducing kernel
\begin{equation}
K(x,y)= \langle \psi(x), \phi(y)\rangle.
\end{equation}
Moreover:
\begin{enumerate}
\item For every $x\in\mathcal{X}$, the evaluation functional $\delta_x: \mathcal{B}\to \mathbb{K}$, defined by $\delta_x(f)= f(x)$, is continuous.
\item The dual space $\mathcal{B}^*$ may be identified with a subspace of $\mathcal{Y}^*$ via the isometry $T: \mathcal{B}\to \mathcal{Y}$, where for any $f \in \mathcal{B}$ with representation
\begin{equation}
f(x) = \sum_{j=1}^\infty \alpha_j\, \langle \psi(x_j), \phi(x)\rangle,
\end{equation}
we define
\begin{equation}
T(f) = \beta_f = \sum_{j=1}^\infty \alpha_j\,\psi(x_j).
\end{equation}
\end{enumerate}
\end{theorem}

\begin{proof}
We proceed in steps:

\medskip
\noindent\textbf{Step 1: Well-definedness of T.} \\
First, we must show that $T$ is well-defined. Let $f \in \mathcal{B}$ have two representations:
\begin{equation}
f(x) = \sum_{j=1}^\infty \alpha_j\, \langle \psi(x_j), \phi(x)\rangle = \sum_{k=1}^\infty \gamma_k\, \langle \psi(y_k), \phi(x)\rangle
\end{equation}
By assumption 3 in our definition,
\begin{equation}
\left\|\sum_{j=1}^\infty \alpha_j\, \psi(x_j) - \sum_{k=1}^\infty \gamma_k\, \psi(y_k)\right\|_{\mathcal{Y}} = 0
\end{equation}
Thus, $T(f)$ is independent of the representation chosen.

\medskip
\noindent\textbf{Step 2: T is an Isometry.} \\
For any $f \in \mathcal{B}$,
\begin{equation}
||T(f)||_{\mathcal{Y}} = ||\beta_f||_{\mathcal{Y}} = ||f||_{\mathcal{B}}
\end{equation}
by definition of the norm in $\mathcal{B}$.

\medskip
\noindent\textbf{Step 3: Completeness.} \\
Let $\{f_n\}$ be a Cauchy sequence in $\mathcal{B}$. Then $\{T(f_n)\}$ is Cauchy in $\mathcal{Y}$ since $T$ is an isometry. Since $\mathcal{Y}$ is complete, there exists $\beta \in \mathcal{Y}$ such that
\begin{equation}
||T(f_n) - \beta||_{\mathcal{Y}} \to 0
\end{equation}
Define $f(x) := \langle \beta, \phi(x) \rangle$. By assumption 1, $f \in \mathcal{B}$ and $T(f) = \beta$. Moreover,
\begin{equation}
||f_n - f||_{\mathcal{B}} = ||T(f_n) - T(f)||_{\mathcal{Y}} \to 0
\end{equation}
Thus, $\mathcal{B}$ is complete.

\medskip
\noindent\textbf{Step 4: Continuity of Pointwise Evaluation.} \\
Fix $x\in\mathcal{X}$. For any $f\in\mathcal{B}$ we have
\begin{equation}
|f(x)| = |\langle \beta_f, \phi(x)\rangle| \leq ||\beta_f||_{\mathcal{Y}}\cdot||\phi(x)||_{\mathcal{Y}^*} = C(x)\cdot||f||_{\mathcal{B}}
\end{equation}
where $C(x) = ||\phi(x)||_{\mathcal{Y}^*}$. Thus, $\delta_x$ is bounded.

\medskip
\noindent\textbf{Step 5: Reproducing Kernel Property.} \\
For each $x\in\mathcal{X}$, define the kernel section $K_x(\cdot) := K(x,\cdot)$. For any $f\in\mathcal{B}$,
\begin{equation}
\langle f, K_x \rangle := \langle T(f), \phi(x) \rangle = \langle \beta_f, \phi(x)\rangle = f(x)
\end{equation}

\medskip
\noindent\textbf{Step 6: Dual Space.} \\
Since $T$ is an isometry, every continuous linear functional $L$ on $\mathcal{B}$ induces a continuous linear functional on $T(\mathcal{B})\subset \mathcal{Y}$. By the Hahn--Banach theorem, this extends to a functional in $\mathcal{Y}^*$. Thus, for every $L \in \mathcal{B}^*$, there exists $\gamma\in \mathcal{Y}^*$ such that
\begin{equation}
L(f) = \langle T(f), \gamma \rangle_{\mathcal{Y}\times\mathcal{Y}^*} = \langle \beta_f, \gamma \rangle
\end{equation}
for all $f \in \mathcal{B}$. This provides the isometric identification of $\mathcal{B}^*$ with a subspace of $\mathcal{Y}^*$.
\end{proof}

[Additional proofs for kernel representation and reproducing properties would follow similarly, with careful attention to the fact that functions can be represented using any countable set of sampling points.]

\section{Kernel Representation Properties}

\begin{theorem}[Kernel Representation]
For each $y \in \mathcal{X}$, if the following conditions hold:
\begin{enumerate}
\item \textbf{Representation:} There exists some $S = \{x_j\}_{j\ge1} \in \mathcal{S}$ and coefficients $\{\alpha_j(y)\}_{j\ge1}$ such that
\begin{equation}
K(x,y) = \sum_{j=1}^\infty \alpha_j(y)\, \langle \psi(x_j), \phi(x)\rangle
\end{equation}

\item \textbf{Convergence:} The series converges in norm:
\begin{equation}
\sum_{j=1}^\infty \alpha_j(y)\, \psi(x_j) \xrightarrow{||\cdot||_{\mathcal{Y}}} \psi(y)
\end{equation}

% \item \textbf{Boundedness:} The coefficients satisfy
% \begin{equation}
% \left\|\sum_{j=1}^\infty \alpha_j(y)\, \psi(x_j)\right\|_{\mathcal{Y}} = ||\psi(y)||_{\mathcal{Y}}
% \end{equation}
\end{enumerate}
then $T(K(\cdot,y)) = \psi(y)$ for all $y \in \mathcal{X}$.
\end{theorem}

\begin{proof}
Fix $y \in \mathcal{X}$. Let $S = \{x_j\}_{j\ge1}$ be the countable set and $\{\alpha_j(y)\}_{j\ge1}$ be the coefficients given by the representation condition.

First, note that $K(\cdot,y) \in \mathcal{B}$ because:
\begin{equation}
||\beta_{K(\cdot,y)}||_{\mathcal{Y}} = \left\|\sum_{j=1}^\infty \alpha_j(y)\, \psi(x_j)\right\|_{\mathcal{Y}} = ||\psi(y)||_{\mathcal{Y}} < \infty
\end{equation}

By definition of $T$ and using the representation condition:
\begin{equation}
T(K(\cdot,y)) = \sum_{j=1}^\infty \alpha_j(y)\, \psi(x_j)
\end{equation}

The convergence condition directly gives:
\begin{equation}
\left\|T(K(\cdot,y)) - \psi(y)\right\|_{\mathcal{Y}} = \left\|\sum_{j=1}^\infty \alpha_j(y)\, \psi(x_j) - \psi(y)\right\|_{\mathcal{Y}} \to 0
\end{equation}

Therefore, $T(K(\cdot,y)) = \psi(y)$.

Moreover, if $K(\cdot,y)$ has another representation using a different countable set, say $\tilde{S} = \{y_k\}_{k\ge1}$ with coefficients $\{\gamma_k(y)\}_{k\ge1}$, then by assumption 3 in our RKBS definition:
\begin{equation}
\left\|\sum_{j=1}^\infty \alpha_j(y)\, \psi(x_j) - \sum_{k=1}^\infty \gamma_k(y)\, \psi(y_k)\right\|_{\mathcal{Y}} = 0
\end{equation}
ensuring our result is independent of the chosen representation.
\end{proof}

\begin{theorem}[Reproducing Properties]
The kernel $K(x,y)=\langle \psi(x), \phi(y)\rangle$ satisfies:
\begin{enumerate}
\item For every $f\in\mathcal{B}$ and $x\in\mathcal{X}$,
\begin{equation}
\langle f, K(x,\cdot) \rangle_{\mathcal{B}\times \mathcal{B}^*} = f(x)
\end{equation}

\item For every $g\in\mathcal{B}^*$ and $y\in\mathcal{X}$,
\begin{equation}
\langle K(\cdot,y), g \rangle_{\mathcal{B}\times \mathcal{B}^*} = g(y)
\end{equation}
\end{enumerate}
\end{theorem}

\begin{proof}
(1) Let $f\in \mathcal{B}$. By definition, there exists some $S \in \mathcal{S}$ and coefficients $\{\alpha_j\}$ such that
\begin{equation}
f(x) = \sum_{j=1}^\infty \alpha_j\, \langle \psi(x_j), \phi(x)\rangle = \langle \beta_f, \phi(x)\rangle
\end{equation}
where $\beta_f = \sum_{j=1}^\infty \alpha_j\, \psi(x_j)$. Then
\begin{equation}
\langle f, K(x,\cdot) \rangle_{\mathcal{B}\times \mathcal{B}^*} = \langle T(f), \phi(x) \rangle = \langle \beta_f, \phi(x)\rangle = f(x)
\end{equation}

(2) Let $g\in\mathcal{B}^*$ with corresponding $\gamma\in \mathcal{Y}^*$. Then for any $y \in \mathcal{X}$,
\begin{align}
\langle K(\cdot,y), g \rangle_{\mathcal{B}\times \mathcal{B}^*} &= \langle T(K(\cdot,y)), \gamma\rangle_{\mathcal{Y}\times\mathcal{Y}^*} \\
&= \langle \psi(y), \gamma\rangle_{\mathcal{Y}\times\mathcal{Y}^*} \\
&= g(y)
\end{align}
where we used $T(K(\cdot,y)) = \psi(y)$ from the previous theorem.

Note that these reproducing properties hold regardless of which representation we choose for the functions, as our earlier results ensure all representations lead to the same values.
\end{proof}

\begin{remark}
The key insight in these proofs is that while functions in $\mathcal{B}$ can be represented using different sets of sampling points, the third assumption in our RKBS definition ensures that:
\begin{enumerate}
\item The value $T(f)$ is well-defined regardless of the representation chosen
\item The norm $||f||_{\mathcal{B}}$ is independent of the representation
\item The reproducing properties hold uniformly for all possible representations
\end{enumerate}
This gives us a robust theory that works with any countable sampling set while maintaining all the essential RKBS properties.
\end{remark}


\section{Gaussian RKHS as a Special Case}

\begin{theorem}[Gaussian RKHS Recovery]
Let $\mathcal{X} = \mathbb{R}^d$ and set $\mathcal{Y} = \mathcal{Z} = \ell^2$. If we choose $\psi = \phi$ where for each $x \in \mathbb{R}^d$,
\begin{equation}
\psi(x) = \phi(x) = \exp(-||x-\cdot||_2^2)
\end{equation}
then the resulting space $\mathcal{B}$ is precisely the Gaussian RKHS with kernel
\begin{equation}
K(x,y) = \exp(-||x-y||_2^2).
\end{equation}
\end{theorem}

\begin{proof}
We verify this step by step:

\medskip
\noindent\textbf{Step 1: Kernel Verification.} \\
For any $x,y \in \mathbb{R}^d$,
\begin{equation}
K(x,y) = \langle \psi(x), \phi(y) \rangle = \langle \psi(x), \psi(y) \rangle = \exp(-||x-y||_2^2)
\end{equation}
which is indeed the Gaussian kernel.

\medskip
\noindent\textbf{Step 2: Inner Product Structure.} \\
Since $\mathcal{Y} = \ell^2$ and $\psi = \phi$, for any $f \in \mathcal{B}$ with representation
\begin{equation}
f(x) = \sum_{j=1}^\infty \alpha_j\, \langle \psi(x_j), \psi(x) \rangle = \sum_{j=1}^\infty \alpha_j\, \exp(-||x_j-x||_2^2)
\end{equation}
we have
\begin{equation}
||f||_{\mathcal{B}}^2 = ||\beta_f||_{\ell^2}^2 = \sum_{i=1}^\infty |\alpha_i|^2
\end{equation}
which matches the RKHS norm for the Gaussian kernel.

\medskip
\noindent\textbf{Step 3: Universal Property.} \\
For any countable subset $S = \{x_j\}_{j\ge1} \subset \mathbb{R}^d$, the functions
\begin{equation}
\{\exp(-||x_j-\cdot||_2^2)\}_{j\ge1}
\end{equation}
form a linearly independent set whose finite linear combinations are dense in the Gaussian RKHS.

\medskip
\noindent\textbf{Step 4: Reproducing Property.} \\
The reproducing property becomes
\begin{equation}
f(x) = \langle f, K(x,\cdot) \rangle_{\mathcal{B}} = \langle f, \exp(-||x-\cdot||_2^2) \rangle_{\mathcal{B}}
\end{equation}
which is the standard reproducing property of Gaussian RKHS.
\end{proof}

\begin{remark}
This special case illustrates several key points:
\begin{enumerate}
\item When $p=2$ and $\psi = \phi$, the dual pairing becomes an inner product
\item The freedom in choosing sampling points matches the universal approximation property of Gaussian RKHS
\item The Hilbert space structure emerges naturally from our more general Banach space setting
\item The coefficients $\{\alpha_j\}$ correspond to the standard series representation in Gaussian RKHS
\end{enumerate}
\end{remark}

\begin{corollary}
Under these conditions ($p=2$, $\psi = \phi$), any function in the Gaussian RKHS can be represented as
\begin{equation}
f(x) = \sum_{j=1}^\infty \alpha_j\, \exp(-||x_j-x||_2^2)
\end{equation}
for some countable set $\{x_j\}_{j\ge1} \subset \mathbb{R}^d$ and coefficients $\{\alpha_j\}_{j\ge1} \in \ell^2$.
\end{corollary}

\section{Embedding in Sobolev Space}

Let's consider the case where $\mathcal{X} \subset \mathbb{R}^d$ is a bounded domain. We'll construct specific feature maps that will give us the Sobolev embedding.

\begin{theorem}
Let $\mathcal{Y} = \ell^p$ with $1 < p < \infty$ and let $\mathcal{X}$ be a bounded domain in $\mathbb{R}^d$. For sufficiently smooth feature maps $\psi$ and $\phi$, the RKBS $\mathcal{B}$ embeds continuously in $W^{s,p}(\mathcal{X})$ for some $s > 0$.
\end{theorem}

\begin{proof}
\textbf{Step 1: Construction of Feature Maps}

Let's choose our feature maps as follows. For $x \in \mathcal{X}$:
\begin{equation}
\psi(x) = (\psi_k(x))_{k\ge 1} \in \ell^p
\end{equation}
where
\begin{equation}
\psi_k(x) = k^{-s-d/p}\exp(-k||x||_2^2)
\end{equation}

and for the dual map:
\begin{equation}
\phi(x) = (\phi_k(x))_{k\ge 1} \in \ell^q
\end{equation}
where
\begin{equation}
\phi_k(x) = k^{s}\exp(-k||x||_2^2)
\end{equation}
with $\frac{1}{p} + \frac{1}{q} = 1$.

\textbf{Step 2: Kernel Analysis}

The reproducing kernel becomes:
\begin{equation}
K(x,y) = \langle \psi(x), \phi(y)\rangle = \sum_{k=1}^\infty k^{-d/p}\exp(-k(||x||_2^2 + ||y||_2^2))
\end{equation}

This kernel is smooth in both variables due to the exponential decay.

\textbf{Step 3: Norm Bounds}

For $f \in \mathcal{B}$ with representation $f(x) = \sum_{j=1}^\infty \alpha_j \langle \psi(x_j), \phi(x)\rangle$:
\begin{equation}
||f||_{\mathcal{B}} = ||\beta_f||_{\ell^p} = \left(\sum_{j=1}^\infty |\alpha_j|^p\right)^{1/p}
\end{equation}

\textbf{Step 4: Sobolev Embedding}

For multi-index $|\gamma| \leq s$, the derivatives of $f$ satisfy:
\begin{align}
|D^\gamma f(x)| &= \left|D^\gamma \sum_{j=1}^\infty \alpha_j \langle \psi(x_j), \phi(x)\rangle\right| \\
&\leq \sum_{j=1}^\infty |\alpha_j| \left|D^\gamma \sum_{k=1}^\infty k^{-d/p}\exp(-k(||x_j||_2^2 + ||x||_2^2))\right|
\end{align}

Using Hölder's inequality and the fact that exponential decay dominates polynomial growth:
\begin{equation}
||D^\gamma f||_{L^p(\mathcal{X})} \leq C_\gamma ||\alpha||_{\ell^p} = C_\gamma ||f||_{\mathcal{B}}
\end{equation}

Therefore,
\begin{equation}
||f||_{W^{s,p}(\mathcal{X})} \leq C ||f||_{\mathcal{B}}
\end{equation}
for some constant $C > 0$.

\textbf{Step 5: Optimality}

The choice of $s$ in the feature maps is related to the smoothness of the embedding. Specifically:
\begin{enumerate}
\item For larger $s$, we get embedding into higher-order Sobolev spaces
\item The decay rate $k^{-s-d/p}$ in $\psi_k$ ensures convergence in $\ell^p$
\item The growth rate $k^s$ in $\phi_k$ balances to give the right Sobolev regularity
\end{enumerate}
\end{proof}

\begin{corollary}
For $p=2$, this construction gives an RKHS that embeds in $H^s(\mathcal{X})$ with the embedding constant depending only on $s$ and the domain $\mathcal{X}$.
\end{corollary}

\begin{remark}
The key aspects of this embedding are:
\begin{enumerate}
\item The feature maps are chosen to have complementary growth/decay rates
\item The exponential terms ensure smoothness
\item The power of $k$ controls the Sobolev regularity
\item The boundedness of $\mathcal{X}$ is used crucially in the derivative estimates
\end{enumerate}
\end{remark}

e typically use either:
\begin{itemize}
    \item Bessel Potential definition:
\[
W^{s,p} = {f: (I-\Delta)^{s/2}f \in L^p}
\]
    \item Or the Gagliardo (semi-)norm definition:
\[
[f]{W^{s,p}} = \left(\int{\mathcal{X}}\int_{\mathcal{X}} \frac{|f(x)-f(y)|^p}{|x-y|^{d+sp}} dx dy\right)^{1/p}
\]

\end{itemize}

\section{Embedding in Fractional Sobolev Space}

Let's prove the embedding for fractional s > 0. We'll use both characterizations to show the relationship clearly.

\begin{theorem}
Let $\mathcal{Y} = \ell^p$ with $1 < p < \infty$ and let $\mathcal{X}$ be a bounded domain in $\mathbb{R}^d$. For sufficiently smooth feature maps $\psi$ and $\phi$, the RKBS $\mathcal{B}$ embeds continuously in $W^{s,p}(\mathcal{X})$ for any $s > 0$ (not necessarily integer).
\end{theorem}

\begin{proof}
\textbf{Step 1: Modified Feature Maps}

For $x \in \mathcal{X}$, define:
\begin{equation}
\psi(x) = (\psi_k(x))_{k\ge 1} \in \ell^p
\end{equation}
where
\begin{equation}
\psi_k(x) = k^{-s-d/p}(1 + k^2||x||_2^2)^{-s/2}
\end{equation}

and
\begin{equation}
\phi(x) = (\phi_k(x))_{k\ge 1} \in \ell^q
\end{equation}
where
\begin{equation}
\phi_k(x) = k^{s}(1 + k^2||x||_2^2)^{-s/2}
\end{equation}

Note: These maps are chosen to mimic the behavior of Bessel potentials.

\textbf{Step 2: Bessel Potential Characterization}

For $f \in \mathcal{B}$ with representation $f(x) = \sum_{j=1}^\infty \alpha_j \langle \psi(x_j), \phi(x)\rangle$, we can show:
\begin{equation}
(I-\Delta)^{s/2}f = \sum_{j=1}^\infty \alpha_j (I-\Delta)^{s/2}K(\cdot,x_j)
\end{equation}

Using the structure of our feature maps:
\begin{align}
\|(I-\Delta)^{s/2}f\|_{L^p} &\leq \left(\sum_{j=1}^\infty |\alpha_j|^p\right)^{1/p} \sup_{j} \|(I-\Delta)^{s/2}K(\cdot,x_j)\|_{L^p} \\
&\leq C\|f\|_{\mathcal{B}}
\end{align}

\textbf{Step 3: Gagliardo Semi-norm Characterization}

Alternatively, we can bound the Gagliardo semi-norm:
\begin{equation}
[f]_{W^{s,p}}^p = \int_{\mathcal{X}}\int_{\mathcal{X}} \frac{|f(x)-f(y)|^p}{|x-y|^{d+sp}} dx dy
\end{equation}

For our kernel:
\begin{align}
|f(x)-f(y)| &= \left|\sum_{j=1}^\infty \alpha_j (\langle \psi(x_j), \phi(x)\rangle - \langle \psi(x_j), \phi(y)\rangle)\right| \\
&\leq \|\alpha\|_{\ell^p} \left(\sum_{k=1}^\infty k^{2s}|x-y|^2(1 + k^2\max(||x||_2^2,||y||_2^2))^{-s}\right)^{1/q}
\end{align}

This yields:
\begin{equation}
[f]_{W^{s,p}} \leq C\|f\|_{\mathcal{B}}
\end{equation}

\textbf{Step 4: Equivalence of Norms}

For $s > 0$, we have the norm equivalence:
\begin{equation}
\|f\|_{W^{s,p}} \sim \|f\|_{L^p} + [f]_{W^{s,p}}
\end{equation}

Therefore:
\begin{equation}
\|f\|_{W^{s,p}} \leq C\|f\|_{\mathcal{B}}
\end{equation}
\end{proof}

\begin{remark}
Key points about the fractional case:
\begin{enumerate}
\item The feature maps use $(1 + k^2||x||_2^2)^{-s/2}$ instead of exponentials to match Bessel potential behavior
\item Both Bessel potential and Gagliardo semi-norm approaches give equivalent norms
\item The embedding is valid for any positive real s, not just integers
\item The construction captures the "fractional smoothness" through the power -s/2 in the feature maps
\end{enumerate}
\end{remark}

\begin{corollary}
When $p=2$, this recovers the embedding of the RKHS into the fractional Sobolev space $H^s(\mathcal{X})$ for any $s > 0$.
\end{corollary}


