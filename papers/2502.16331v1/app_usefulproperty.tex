\section{A useful property of Hermite polynomials}\label{app: useful}
In this Appendix, we provide the proof of \lemref{lem: inter}.

\begin{proof}
Since \( H_{d+1}(y) \) is a polynomial of degree \( d+1 \), there exists a constant \( C > 0 \) (depending only on \( d \)) such that
\[
\Bigl| H_{d+1}(y) \Bigr| \le C \,(1+|y|)^{d+1} \quad \text{for all } y \in \mathbb{R}.
\]
Hence, for any \(\delta>0\) and any integer \( j\ge 2 \) we have
\[
\Bigl| H_{d+1}\bigl(j\delta\bigr) \Bigr|\, e^{-\frac{(j\delta)^2}{2}}
\le C \,(1+j\delta)^{d+1}\, e^{-\frac{(j\delta)^2}{2}}.
\]
Now, we define
\[
S(\delta) := \sum_{j=2}^\infty (1+j\delta)^{d+1}\, e^{-\frac{(j\delta)^2}{2}}.
\]
Note that we we can upper bound as follows
\[
\sum_{j=2}^\infty \Bigl| H_{d+1}\bigl(j\delta\bigr) \Bigr|\, e^{-\frac{(j\delta)^2}{2}}
\le C\, S(\delta).
\]
For each fixed \( j \ge 2 \), notice that $(1+j\delta)^{d+1}\, e^{-\frac{(j\delta)^2}{2}}$
decays super--exponentially in \( j \) (since the exponential term \( e^{-\frac{(j\delta)^2}{2}} \) dominates the polynomial growth of \((1+j\delta)^{d+1}\)). Moreover, for fixed \( j\ge2 \) we have
\[
\lim_{\delta\to\infty} (1+j\delta)^{d+1}\, e^{-\frac{(j\delta)^2}{2}} = 0.
\]
Thus, the series \( S(\delta) \) converges for every fixed \(\delta > 0\) and
\[
\lim_{\delta\to\infty} S(\delta) = 0.
\]
Hence, by the definition, there exists some \(\delta_0 > 0\) such that for all \(\delta \ge \delta_0\) we have
\[
S(\delta) < \frac{\rho}{4C}.
\]
It follows that for every \(\delta \ge \delta_0\),
\[
\sum_{j=2}^\infty \Bigl| H_{d+1}\bigl(j\delta\bigr) \Bigr|\, e^{-\frac{(j\delta)^2}{2}}
\le C\cdot S(\delta) < C\cdot \frac{\rho}{4C} = \frac{\rho}{4}.
\]

Thus for this choice of $\delta_0$ we achieve the statement of the lemma.
\end{proof}