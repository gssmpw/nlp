\section{Separability of the Gaussian RKHS}

\begin{theorem}
Let $\mathcal{H}$ be the reproducing kernel Hilbert space (RKHS) associated with the Gaussian kernel
\[
k(x,y)=\exp\Bigl(-\frac{\|x-y\|^2}{2\sigma^2}\Bigr), \quad x,y\in\mathbb{R}^d,
\]
with $\sigma>0$. Then $\mathcal{H}$ is separable.
\end{theorem}

\begin{proof}


\textbf{Step 1. Dense subspace given by kernel sections.}

By the Moore--Aronszajn theorem, the set
\[
\mathcal{F} := \operatorname{span}\{ k(x,\cdot) : x \in \mathbb{R}^d \}
\]
is dense in $\mathcal{H}$. That is, for every $f\in\mathcal{H}$ and every $\epsilon>0$, there exists a finite linear combination
\[
g = \sum_{i=1}^n a_i\, k(x_i,\cdot), \quad a_i\in\mathbb{R},\; x_i\in\mathbb{R}^d,
\]
such that
\[
\|f - g\|_{\mathcal{H}} < \frac{\epsilon}{2}.
\]

\textbf{Step 2. Construction of a countable dense subset.}

Let
\[
D = \mathbb{Q}^d \quad \text{and} \quad Q = \mathbb{Q},
\]
which are both countable. Define
\[
\mathcal{F}_0 = \left\{ \sum_{i=1}^n a_i\, k(q_i,\cdot) : n\in\mathbb{N},\; a_i\in Q,\; q_i\in D \right\}.
\]
Since finite sums of elements with coefficients from the countable sets $Q$ and centers from the countable set $D$ form a countable set, $\mathcal{F}_0$ is countable.

\textbf{Step 3. Density of $\mathcal{F}_0$ in $\mathcal{H}$.}

Let $f\in\mathcal{H}$ and $\epsilon>0$. As in Step 1, choose
\[
g(x)=\sum_{i=1}^n a_i\, k(x_i,x)
\]
with $a_i\in\mathbb{R}$, $x_i\in\mathbb{R}^d$, such that
\[
\|f-g\|_{\mathcal{H}} < \frac{\epsilon}{2}.
\]
Since $D=\mathbb{Q}^d$ is dense in $\mathbb{R}^d$, for each $i=1,\dots,n$, there exists $q_i\in D$ such that
\[
\|x_i-q_i\| < \delta,
\]
where $\delta>0$ will be chosen later. Similarly, because $Q$ is dense in $\mathbb{R}$, for each $i$ there exists $b_i\in Q$ such that
\[
|a_i - b_i| < \delta.
\]
Define
\[
g_0(x)=\sum_{i=1}^n b_i\, k(q_i,x).
\]
We now show that $g_0\in\mathcal{F}_0$ approximates $g$ well in the $\mathcal{H}$-norm.

Using the triangle inequality in $\mathcal{H}$, we have
\[
\|g - g_0\|_{\mathcal{H}} \le \sum_{i=1}^n \|a_i\, k(x_i,\cdot) - b_i\, k(q_i,\cdot)\|_{\mathcal{H}}.
\]
For each term, applying the triangle inequality again yields
\[
\|a_i\, k(x_i,\cdot) - b_i\, k(q_i,\cdot)\|_{\mathcal{H}} \le |a_i - b_i|\, \|k(x_i,\cdot)\|_{\mathcal{H}} + |b_i|\, \|k(x_i,\cdot)- k(q_i,\cdot)\|_{\mathcal{H}}.
\]
Recall that by the reproducing property, 
\[
\|k(x,\cdot)\|_{\mathcal{H}} = \sqrt{k(x,x)} = 1 \quad \text{for all } x\in\mathbb{R}^d.
\]
Thus,
\[
\|a_i\, k(x_i,\cdot) - b_i\, k(q_i,\cdot)\|_{\mathcal{H}} \le |a_i - b_i| + |b_i|\, \|k(x_i,\cdot)- k(q_i,\cdot)\|_{\mathcal{H}}.
\]

Since the Gaussian kernel is smooth, the mapping
\[
x \mapsto k(x,\cdot)
\]
is continuous in the $\mathcal{H}$-norm. In fact, one may compute
\[
\|k(x,\cdot) - k(y,\cdot)\|_{\mathcal{H}}^2 = k(x,x) - 2k(x,y) + k(y,y) = 2\Bigl(1 - k(x,y)\Bigr),
\]
and since $k(x,y)$ is continuous in both arguments, for each $\eta>0$ there exists a $\delta>0$ such that
\[
\|k(x,\cdot)- k(y,\cdot)\|_{\mathcal{H}} < \eta \quad \text{whenever } \|x-y\|<\delta.
\]

Now, choose $\delta>0$ small enough so that for each $i=1,\dots,n$ we have
\[
|a_i - b_i| < \frac{\epsilon}{2n} \quad \text{and} \quad |b_i|\, \|k(x_i,\cdot)- k(q_i,\cdot)\|_{\mathcal{H}} < \frac{\epsilon}{2n}.
\]
Then for each $i$,
\[
\|a_i\, k(x_i,\cdot) - b_i\, k(q_i,\cdot)\|_{\mathcal{H}} < \frac{\epsilon}{n},
\]
so that
\[
\|g - g_0\|_{\mathcal{H}} < \sum_{i=1}^n \frac{\epsilon}{n} = \epsilon.
\]
Finally, applying the triangle inequality again,
\[
\|f-g_0\|_{\mathcal{H}} \le \|f-g\|_{\mathcal{H}} + \|g-g_0\|_{\mathcal{H}} < \frac{\epsilon}{2} + \epsilon = \frac{3\epsilon}{2}.
\]
(If necessary, one may adjust the choices to achieve $\|f-g_0\|_{\mathcal{H}} < \epsilon$; the important point is that for any $\epsilon>0$, there exists $g_0\in \mathcal{F}_0$ with $\|f-g_0\|_{\mathcal{H}} < \epsilon$.)

Thus, $\mathcal{F}_0$ is dense in $\mathcal{H}$.

Since $\mathcal{F}_0$ is countable and dense in $\mathcal{H}$, the space $\mathcal{H}$ is separable.
\end{proof}

