\section{Example Projects}
To further illustrate the practical applications of Thoughtful AI, we present two projects that embody its core principles. The first, Inner Thoughts, explores how AI can proactively engage in conversations by developing and evaluating its internal thoughts before participating. The second, ThinkaloudLM, investigates AI-generated thoughts as an interface component that allows users to observe and interact with in real-time.

\subsection{Inner Thoughts}
\begin{figure}
    \centering
  \includegraphics[width=0.65\linewidth]{figures/03_inner_thoughts.pdf}
    \caption{Conversational Agents with Inner Thoughts: AI generates a train of thoughts and evaluates them based on their intrinsic motivation to participate.}
    \label{fig:inner_thoughts}
\end{figure}

Conversational AI often rely on turn-taking prediction techniques, where an algorithm determines the most likely next speaker and generates a response accordingly. However, these approaches struggle in multi-party conversations where turn allocation is often ambiguous. Furthermore, existing conversational agents tend to fall into two extremes: either they remain passive, requiring explicit user input to respond, or they overcompensate, generating frequent and often unnecessary interruptions.

Instead of predicting conversational turns, \textit{Inner Thoughts}~\cite{liu2025inner} introduces a new method in which AI autonomously generates a continuous stream of covert (internal) thoughts, similar to how humans process conversations. These thoughts remain internal until AI evaluates whether it has sufficient \textit{intrinsic motivation} to contribute---defined by heuristics derived from a user study on human conversational behavior. Once the AI determines that a thought is relevant and meaningful, it then strategically selects an appropriate moment to engage (\autoref{fig:inner_thoughts}).

The Inner Thoughts framework embodies multiple traits of Thoughtful AI:
\textit{Intermediate Medium}: AI does not simply generate final responses but produces an evolving series of intermediate thoughts; \textit{Full-Duplex Process}: Instead of waiting for user prompts, the AI continuously listens and generates internal responses in parallel with the ongoing conversation; \textit{Intrinsic Driver}: The AI does not just react but actively determines when to contribute based on its intrinsic motivation.

The framework was implemented in two systems, a multi-agent conversational simulation and a chatbot. Through a technical evaluation, conversational agents using Inner Thoughts significantly outperformed a traditional next-speaker prediction baseline across multiple criteria, including turn appropriateness, coherence, engagement, and adaptability.
Participants overwhelmingly preferred AI interactions using Inner Thoughts (in 82\% of the conversations).



\subsection{ThinkaloudLM}
\begin{figure}
    \centering
  \includegraphics[width=\linewidth]{figures/03_thinkaloudlm.png}
    \caption{ThinkaloudLM: AI generates intermediate, fragmented thoughts in parellel to user input.}
    \label{fig:thinkaloudlm}
\end{figure}

While Inner Thoughts explores AI's ability to think internally, ThinkaloudLM investigates a complementary question: what if users could actively observe and engage with AI's thought process?

Most AI systems present only finalized outputs, hiding the intermediate thoughts behind their decisions. This lack of transparency can lead to user distrust, inefficiencies in communication, and missed opportunities for collaboration. For example, if an AI-powered writing assistant provides a response that feels misaligned with the user’s intent, the user must iteratively refine their query, leading to unnecessary back-and-forth exchanges.

By contrast, ThinkaloudLM envisions a system where users can engage with AI-generated ideas before they fully materialize into responses. Rather than receiving a single, static reply, users might see a live preview of AI’s thinking, including key points, possible directions, or alternative solutions. They can then refine, select, or discard thoughts before finalizing the AI’s output.

This approach leverages multiple traits of Thoughtful AI:
\textit{Intermediate Medium}: The AI’s partially formed ideas are themselves the primary interface, giving users a real-time window into the AI’s thought process; \textit{Full-duplex Process}: By continuously updating these intermediate thoughts, the AI can incorporate user feedback on-the-fly while still in the ``draft'' stage; \textit{Shared cognitive space}: Users and AI co-construct the final answer, using the AI's visible thoughts as a collaborative canvas for discussion, refinement, and mutual alignment.
To explore this concept, a participatory design study is currently underway with qualitative interviews and observational studies. 