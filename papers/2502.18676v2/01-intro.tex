\section{Introduction}


\begin{figure}
    \centering
  \includegraphics[width=\linewidth]{figures/00_teaser.pdf}
    \caption{Rethinking human-AI interaction paradigms:
(A) Traditional AI is reactive, responding only when prompted.
(B) Thoughtful AI thinks continuously, proactively generating, iterating, and allowing users to interact with its thoughts.
}
    \label{fig:teaser}
\end{figure}


In 1950, Alan Turing famously asked, \textit{“Can machines think?”}~\cite{Turing1950ComputingMA} This question has inspired decades of research in artificial intelligence (AI), with efforts ranging from symbolic reasoning and expert systems to today’s large language models (LLMs).
Early philosophical and computational models of cognition viewed thought as a defining characteristic of intelligence—whether in humans or machines~\cite{descartes1996discourse, newell1972human, miller2001integrative, raichle2001default}. 
Recent LLMs have reignited discussions about what it means for an AI to ``think''.
Techniques such as Chain-of-Thought prompting~\cite{wei2022chain} demonstrate that generating intermediate reasoning steps can significantly enhance model performance on complex tasks. Building on this foundation, models like OpenAI o1~\cite{jaech2024openai}, Gemini Flash Thinking~\cite{deepmind2025geminiflashthinking}, and DeepSeek R1~\cite{guo2025deepseek} integrate reasoning into their training processes. These developments mark substantial progress in AI’s ability to problem-solve.


However, despite these advancements, current human-AI interaction paradigms remain fundamentally constrained. Existing systems still operate within a turn-based, input-output framework: Users issue a prompt, the AI generates a single response, and the cycle repeats. 
This limitation becomes evident in everyday AI interactions. Imagine a user asking ChatGPT: ``Help me plan a surprise birthday party for my friend.'' 
ChatGPT would generate a static response---perhaps listing general steps like choosing a venue, sending invitations, and selecting a menu. 
At this point, the AI has already stopped processing and just passively waits for the user’s next input. 
If the user later realizes they need a budget-friendly vegan restaurant for six people in New York, they must start a new turn, manually refining their query.

In contrast, consider instead how humans naturally collaborate during a brainstorming session. Each participant holds an internal train of thought that continuously updates in the background, even while someone else is speaking. In parallel, they may externalize immature, intermediate thoughts by sketching or annotating on a whiteboard, inviting immediate refinements and comments from others. This thinking process does not pause in discrete cycles: conversation flows as people update, erase, and reorganize ideas in real time.


In this position paper, we envision the concept of \textit{Thoughtful AI}, a new interaction paradigm in which AI functions as a continuously thinking entity. 
Unlike conventional AI that passively waits for user prompts and responds in discrete turns, Thoughtful AI autonomously generates, iterates, and selectively communicates its evolving thoughts throughout an interaction. 


At the core of Thoughtful AI is to view AI's \textit{thought} as a first-class citizen in human-AI interaction: it is not merely hidden backend computation steps but a fundamental modality that contribute to AI's capabilities to interact with humans. 
We identify four key traits that define this paradigm:
\begin{enumerate} 
    \item It provides an \textit{intermediate medium}, enabling users to observe and interact with AI’s intermediate thoughts rather than just its final outputs. 
    \item It enables a \textit{full-duplex process}, where AI and users exchange thoughts fluidly rather than in rigid, turn-based exchanges. 
    \item It serves as a \textit{intrinsic driver}, allowing AI to initiate interactions rather than merely responding to queries. 
    \item It establishes a \textit{shared cognitive space}, where AI and users build upon each other’s thoughts in a collaborative, dynamic process.
\end{enumerate}


In the sections that follow, we first outline the conceptual foundations of Thoughtful AI.
We then provide concrete examples that illustrate how a continuously thinking AI can bring new possibilities to human-AI interaction, offering benefits like enabling proactive AI behavior, facilitating continuous cognitive alignment with users, and fostering more dynamic interaction experience.
We discuss the broader implications of this paradigm shift, and explore how ``thinking’’---long considered a uniquely human trait---can transform our ways to interact with machines in the future.







% % Background: distinct perspectives on what is "thought"
% The study of thought has long been a central topic in fields like cognitive science, neuroscience and philosophy, each offering distinct perspectives on its nature. 
% Early philosophers, such as Descartes, emphasized thought as the essence of consciousness (\textit{``Cogito, ergo sum''})~\cite{descartes1996discourse}. Information-processing model~\cite{newell1972human} developed by Newell and Simon conceptualizes thought as a computational process that involves encoding, storing, and retrieving mental representations. 
% Meanwhile, neuroscientists tie these processes to brain activity, linking specific neural mechanisms to cognitive functions such as problem-solving, decision-making, and self-reflection~\cite{miller2001integrative, raichle2001default}. 
% Although the exact nature and mechanisms of thought continue to be debated, it is widely acknowledged as a fundamental factor in shaping human cognition an behavior.

% % Background: AIs are using thoughts
% More recently, as artificial intelligence (AI) systems, particularly those based on Large Language Models (LLMs), become increasingly capable, there is growing interest in replicating not only human conversational abilities but also the underlying thought processes that guide them. Techniques such as Chain-of-Thought (CoT) prompting~\cite{wei2022chain} have demonstrated that intermediate reasoning steps generated by LLMs can significantly improve the performance of the model, especially in reasoning tasks. Models like OpenAI’s o1~\cite{jaech2024openai}, Gemini’s Flash Thinking~\cite{deepmind2025geminiflashthinking}, and DeepSeek R1~\cite{guo2025deepseek} have started to integrate these reasoning mechanisms into model architectures and training procedures. 
% Many existing LLM-based chatbots also display their ``thinking process'' to users before the final response.

% % Linking back to HCI and motivate our vision
% Although thought has long been recognized as a fundamental aspect of human and is increasingly becoming essential to the design and operation of AI, its role in human-computer interaction (HCI) remains largely unexplored.
% Traditional HCI has largely centered on explicit ``input-output'' paradigms, where users issue commands and receive responses, with interaction modalities like text, speech, and gesture designed to capture observable, externalized behaviors~\cite{dourish1999embodied, ishii1997tangible}.
% We believe that this input-output paradigm imposes an inherent limitation on human-AI interaction:
% unlike humans, whose internal thought processes continuously generate, refine, and transform ideas throughout the communication, current AI systems are confined to \textit{reactive, turn-based} exchanges.
% Consider a trip planning scenario: while brainstorming potential destinations, a human mind continuously generates abstract ideas, iteratively refines them, and eventually crystallizes a plan that is then articulated.
% In contrast, AI remains largely mechanical and passive, lacking the dynamic, iterative and proactive qualities of human thought. 


% In this position paper, we propose a paradigm shift in human-AI interaction that moves beyond discrete, turn-based input-output exchanges toward engaging with AI as a continuously ``thinking'' system. 
% We introduce the concept of \textbf{thoughtful AI},
% where AI's internal cognitive processes are no longer concealed as backend computations but are integrated into the interaction itself.
% Rather than waiting for user input, thoughtful AI proactively and continuously generates, iterates, and optionally externalizes its intermediate \textit{thought} in real time, throughout the interaction process.

% To formalize this concept, we first define AI’s thought as the\textit{ continuous stream of intermediate processes and responses} that unfold throughout an interaction---encompassing everything from interpreting inputs and retrieving knowledge to generating hypotheses and refining answers.
% We further introduce four key traits of thought that distinguish thoughtful AI:
% (1) its function as an \textit{intermediate medium}, 
% (2) its capacity for \textit{full-duplex interaction}, 
% (3) its role as a \textit{source of proactivity}, 
% and (4) its potential to create a \textit{shared cognitive space}. 

% Building on these traits, we explore how thoughtful AI can transform human-AI interaction, discussing how it can shift AI from passive respondents to proactive participants, help bridge the gulf of execution and evaluation, and extend interactions beyond chat-based, turn-taking paradigms.
% Finally, we illustrate this vision with two example projects demonstrating how incorporating AI’s thought can benefit human-AI interaction in practice.

