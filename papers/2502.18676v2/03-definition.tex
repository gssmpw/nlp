\section{What makes AI ``Thoughtful''?}


Discussions around ``thought'' in LLM research often focus on a chain-of-thought approach, where a model generates intermediate reasoning steps to improve performance on reasoning tasks. From an HCI perspective, however, \textit{thought} can be understood more broadly. We define \textit{Thoughtfulness} as:

\newtheorem{thoughtfuldefinition}{Definition}
\begin{definition}
\textbf{Thoughtfulness} refers to a system's ability to continuously generate, develop, and selectively communicate its \textit{intermediate processes and responses} over the course of an interaction. 
\end{definition}


In contrast to LLM-based definitions, our conception of Thoughtful AI emphasizes how these ongoing processes shape the system’s behavior and its capacity to interact with human. Thoughts can emerge at any point in an interaction, triggered by external stimuli or intrinsic reflection, may be expressed or remain internal, and they can take various forms, from abstract keywords to visual or auditory representations.


To better understand how Thoughtful AI differs from conventional systems, we examine four key traits of this interaction paradigm (\autoref{tab:02-traits}), comparing them with the turn-based model to highlight the key distinctions.


\subsection{Traits of Thoughtful AI}

% Please add the following required packages to your document preamble:
% \usepackage{booktabs}
% \usepackage{graphicx}
% \usepackage[normalem]{ulem}
% \useunder{\uline}{\ul}{}

\begin{table}
  \resizebox{\columnwidth}{!}{%
    \begin{tabular}{@{}lll@{}}
      \toprule
      \textbf{Trait} & \textbf{Current AI} & \textbf{Thoughtful AI} \\ 
      \midrule
      \textit{Intermediate Medium} & 
        \begin{tabular}[c]{@{}l@{}}
          Reveals only\\
          final outputs
        \end{tabular} & 
        \begin{tabular}[c]{@{}l@{}}
          Surfaces intermediate\\
          thoughts in real time
        \end{tabular} \\
      \addlinespace
      \textit{Full-Duplex Process} & 
        \begin{tabular}[c]{@{}l@{}}
          Waits for user\\
          prompts
        \end{tabular} & 
        \begin{tabular}[c]{@{}l@{}}
          Continuously thinks in\\
          parallel with user activity
        \end{tabular} \\
      \addlinespace
      \textit{Intrinsic Driver} & 
        \begin{tabular}[c]{@{}l@{}}
          Responds only\\
          when asked
        \end{tabular} & 
        \begin{tabular}[c]{@{}l@{}}
          Self-initiates actions\\
          based on thoughts
        \end{tabular} \\
      \addlinespace
      \textit{Shared Cognitive Space} & 
        \begin{tabular}[c]{@{}l@{}}
          Turn-based\\
          exchanges
        \end{tabular} & 
        \begin{tabular}[c]{@{}l@{}}
          Builds on collaborative\\
          fragmented ideas
        \end{tabular} \\
      \bottomrule
    \end{tabular}%
  }
  \caption{Comparison of Thoughtful AI and Current AI across key traits.}
  \label{tab:02-traits}
\end{table}

\subsubsection{Intermediate Medium}
A fundamental distinction between a \textit{thought} and a \textit{response} is that a thought is \textit{in-progress}, potentially fragmented, and not necessarily intended for final output. In Thoughtful AI, the system’s processing, whether it involves brainstorming ideas, weighing pros and cons, or exploring parallel strategies, can be treated as an \textit{intermediate medium} that the AI may optionally share.

In most existing AI systems, users see only a final answer. This one-shot approach is limiting: once the user receives an answer, any clarifications or follow-up requests often require multiple back-and-forth queries. If the AI’s thinking was flawed or incomplete, the user has no window into how or why the system arrived at its conclusion.


\subsubsection{Full-duplex Process}
Human conversation does not pause for one party to ``finish thinking'', nor do humans stop thinking when others start to speak. Similarly, Thoughtful AI enables a \textit{full-duplex process} where thinking is continuous and can occur any point in the interaction, rather than locked into turn-based cycles. AI thoughts may be triggered by user inputs or by the system’s own internal reflections; they can also evolve without any external stimuli, for example, during periods of user silence.

Conventional AI interactions are typically half-duplex. After producing an answer, the AI stops listening or processing until another user query arrives. It remains idle, unaware of changing contexts and does not develop anything internally during that downtime.


\subsubsection{Intrinsic Driver}
One of the key aspects of human intelligence is that thought is not merely reactive: it also serves as an \textit{intrinsic driver} of action. Similarly, in Thoughtful AI, thinking can serve as a mechanism that enables the system to self-initiate its interactions and be \textit{proactive}.
Rather than being solely triggered by user input, Thoughtful AI can independently generate ideas, monitor evolving contexts, and identify opportunities to intervene or contribute.

Traditional turn-based systems are \emph{reactive}: they respond only once the user issues a query. Even so-called ``proactive'' assistants typically rely on predefined triggers or simple rule-based heuristics. In contrast, a Thoughtful AI continuously \emph{thinks} in the background, allowing it to model its intrinsic motivation to take actions based on the intermediate thoughts. Its continuous thought process is the primary engine for proactive behavior.

\subsubsection{Shared Cognitive Space}

Finally, a \textit{shared cognitive space} could emerge when AI and user co-exist in an ongoing thought process. Rather than the AI presenting a single answer and the user responding with a single set of followups, both parties iteratively build upon each other’s partial ideas. We envision that the AI’s intermediate thoughts, user feedback, clarifying questions, and real-time refinements etc form a collaborative ``thinking canvas.''

Traditional interactions are linear and unidirectional: the AI provides an answer, the user reacts, and so on. Even if the user can add further prompts, there is no persistent collaborative workspace where partial ideas accumulate and evolve. 


\subsection{Implications for HCI}
The four traits of Thoughtful AI---an \textit{intermediate medium}, \textit{full-duplex process}, \textit{intrinsic driver}, and a \textit{shared cognitive space}---open up new possibilities for how humans interact with AI. Below, we discuss four implications for Human-AI Interaction.

\subsubsection{From Passive Respondents to Proactive Participants}
A first implication of Thoughtful AI is the transition from AI as a passive respondent to an active participant in interactions. Conventional AI systems operate in a reactive manner, awaiting user input before generating a response. In contrast, Thoughtful AI continuously generates thoughts, enabling it to self-initiate actions and engage more dynamically with users.

This proactivity manifests in two ways. First, the \textit{full-duplex process} (Trait 2) allows the AI to generate thoughts in parallel with user input, meaning it does not need to remain idle between interactions. It can detect emerging needs, anticipate user queries, and even interrupt when necessary. Second, thoughts equip AI with an \textit{intrinsic drive} (Trait 3) to contribute based on its internal cognitive processes. This ability to self-motivate and intervene resembles human-like initiative, distinguishing it from traditional systems that rely on heuristics or predefined triggers~\cite{horvitz1999principles}.


\subsubsection{Continuous Cognitive Alignment}

A second major implication for HCI arises from Thoughtful AI's ability to reveal \emph{in-progress} ideas (Trait 1: \textit{Intermediate Medium}) continuously (Trait 2: \textit{Full-duplex Process}). Rather than presenting a single, static final output, the system selectively shares partial thoughts. Users can then not only observe these thoughts for better interpretability, but also guide the AI's thinking early in the process, leading to more dynamic collaboration.

By showing its intermediate reasoning, Thoughtful AI helps bridge the ``Gulf of Evaluation,'' where users struggle to understand how or why a system arrived at a particular result~\cite{norman2013design}. Here, the AI’s partial thoughts provide visibility into its evolving path, allowing users to catch misconceptions or supply additional context before errors compound. Similarly, users can steer the AI's next steps more effectively, shrinking the ``Gulf of Execution,'' since they can articulate what they want \emph{as} they see the AI’s tentative directions.

This also helps build common ground~\cite{clark1991grounding} between human and AI, aligning with established theories of communication that emphasize the importance of shared context for effective collaboration. By understanding the AI's intermediate thought, users gain a clearer understanding of the system's current assumptions and partial conclusions. This shared cognitive workspace drives continuous alignment between the user's thinking and AI's thinking process.


\subsubsection{Beyond Turn-based Interaction}
Thoughtful AI prompts us to reconsider the fundamental structure of human–AI interaction. Traditional chatbot interfaces operate in discrete, back-and-forth messages, with the AI effectively going idle after sending each response. By contrast, a \emph{full-duplex process} (Trait 2) enables the AI to continuously think and listen, even when the user is not actively providing input.
This opens the door to interaction models that transcend simple chat boxes. For instance, an AI planning assistant could silently update its suggestions as it overhears new constraints in a virtual meeting, intervening only when necessary (Traits 2 and 3). 
More interestingly, AI and users can build a \emph{shared cognitive space} (Trait 4), adding thoughts, annotations, and partial ideas to a collective interface. Rather than a linear log of exchanges, this collaborative canvas allows ideas to branch, merge, and evolve continuously---resembling a dynamic mind-map or sketchnote more than a turn-based chat.


\subsubsection{Messy, Fragmented and Informal Interaction}
Traditional AI systems demand users structure their inputs as complete, well-formed queries or commands. This formality creates friction: human thinking is inherently non-linear, often involving half-formed ideas and abrupt shifts in focus. 
Thoughtful AI embraces this messiness by using intermediate thoughts as the first-class citizen of interaction. 
This shift towards a more natural, fragmented, and informal interaction style: AI thinking can be messy, incomplete, and in-progress. 
In addition, instead of requiring users to provide polished, fully-formed queries or responses, users can also input partial ideas, rough thoughts, or even keywords, and the system can process and respond to these in a similarly fragmented manner. 
This approach mirrors how humans often think and communicate: by iterating on ideas, refining them over time, and bouncing off incomplete or tentative thoughts. 