\section{Conclusion}
In this position paper, we proposed the concept of Thoughtful AI, a new paradigm in which AI continuously generates, develops, and selectively communicates its evolving thoughts throughout an interaction. By moving beyond the conventional turn-based, input-output framework, Thoughtful AI enables more proactive, adaptive, and collaborative engagement between humans and AI. We identified four key traits, \textit{intermediate medium, full-duplex process, intrinsic driver}, and \textit{shared cognitive space}, and demonstrated their potential through two example projects.

However, Thoughtful AI is not merely about enhancing existing AI assistants; it represents a broader shift toward more \textit{fluid, intuitive, and adaptive} paradigms of human–AI interaction. At its core, this approach challenges the decades-old model of AI as a passive respondent confined to chat interfaces---a structure that has persisted since the era of ELIZA~\cite{Weizenbaum1966ELIZAaCP}. 
We advocate for a fundamental reimagining of AI systems and human-AI interaction, ones that enables richer collaboration, fosters deeper trust, and ultimately creates a more natural way for humans and machines to think together.