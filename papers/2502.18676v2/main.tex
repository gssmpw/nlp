%%
%% This is file `sample-sigconf.tex',
%% generated with the docstrip utility.
%%
%% The original source files were:
%%
%% samples.dtx  (with options: `all,proceedings,bibtex,sigconf')
%% 
%% IMPORTANT NOTICE:
%% 
%% For the copyright see the source file.
%% 
%% Any modified versions of this file must be renamed
%% with new filenames distinct from sample-sigconf.tex.
%% 
%% For distribution of the original source see the terms
%% for copying and modification in the file samples.dtx.
%% 
%% This generated file may be distributed as long as the
%% original source files, as listed above, are part of the
%% same distribution. (The sources need not necessarily be
%% in the same archive or directory.)
%%
%%
%% Commands for TeXCount
%TC:macro \cite [option:text,text]
%TC:macro \citep [option:text,text]
%TC:macro \citet [option:text,text]
%TC:envir table 0 1
%TC:envir table* 0 1
%TC:envir tabular [ignore] word
%TC:envir displaymath 0 word
%TC:envir math 0 word
%TC:envir comment 0 0
%%
%% The first command in your LaTeX source must be the \documentclass
%% command.
%%
%% For submission and review of your manuscript please change the
%% command to \documentclass[manuscript, screen, review]{acmart}.
%%
%% When submitting camera ready or to TAPS, please change the command
%% to \documentclass[sigconf]{acmart} or whichever template is required
%% for your publication.
%%
%%
\documentclass[sigconf, screen]{acmart}
%%
%% \BibTeX command to typeset BibTeX logo in the docs
\AtBeginDocument{%
  \providecommand\BibTeX{{%
    Bib\TeX}}}

%% Rights management information.  This information is sent to you
%% when you complete the rights form.  These commands have SAMPLE
%% values in them; it is your responsibility as an author to replace
%% the commands and values with those provided to you when you
%% complete the rights form.
\setcopyright{acmlicensed}
\copyrightyear{2025}
\acmYear{2025}
\acmDOI{XXXXXXX.XXXXXXX}
%% These commands are for a PROCEEDINGS abstract or paper.
\acmConference[Conference acronym 'XX]{Make sure to enter the correct
  conference title from your rights confirmation email}{June 03--05,
  2025}{Woodstock, NY}
%%
%%  Uncomment \acmBooktitle if the title of the proceedings is different
%%  from ``Proceedings of ...''!
%%
%%\acmBooktitle{Woodstock '18: ACM Symposium on Neural Gaze Detection,
%%  June 03--05, 2025, Woodstock, NY}
\acmISBN{978-1-4503-XXXX-X/2025/02}


%%
%% Submission ID.
%% Use this when submitting an article to a sponsored event. You'll
%% receive a unique submission ID from the organizers
%% of the event, and this ID should be used as the parameter to this command.
%%\acmSubmissionID{123-A56-BU3}

%%
%% For managing citations, it is recommended to use bibliography
%% files in BibTeX format.
%%
%% You can then either use BibTeX with the ACM-Reference-Format style,
%% or BibLaTeX with the acmnumeric or acmauthoryear sytles, that include
%% support for advanced citation of software artefact from the
%% biblatex-software package, also separately available on CTAN.
%%
%% Look at the sample-*-biblatex.tex files for templates showcasing
%% the biblatex styles.
%%

%%
%% The majority of ACM publications use numbered citations and
%% references.  The command \citestyle{authoryear} switches to the
%% "author year" style.
%%
%% If you are preparing content for an event
%% sponsored by ACM SIGGRAPH, you must use the "author year" style of
%% citations and references.
%% Uncommenting
%% the next command will enable that style.
%%\citestyle{acmauthoryear}


%%
%% end of the preamble, start of the body of the document source.
\begin{document}

%%
%% The "title" command has an optional parameter,
%% allowing the author to define a "short title" to be used in page headers.
\title{Interacting with \textit{Thoughtful} AI}

%%
%% The "author" command and its associated commands are used to define
%% the authors and their affiliations.
%% Of note is the shared affiliation of the first two authors, and the
%% "authornote" and "authornotemark" commands
%% used to denote shared contribution to the research.

\author{Xingyu Bruce Liu}
\affiliation{%
  \institution{University of California, Los Angeles}
  \city{Los Angeles}
  \state{CA}
  \country{USA}}
 \email{xingyuliu@ucla.edu}

 \author{Haijun Xia}
\affiliation{%
  \institution{University of California, San Diego}
  \city{La Jolla}
  \state{CA}
  \country{USA}}
 \email{haijunxia@ucsd.edu}

\author{Xiang `Anthony' Chen}
\affiliation{%
  \institution{University of California, Los Angeles}
  \city{Los Angeles}
  \state{California}
  \country{USA}}
\email{xac@ucla.edu}

%%
%% By default, the full list of authors will be used in the page
%% headers. Often, this list is too long, and will overlap
%% other information printed in the page headers. This command allows
%% the author to define a more concise list
%% of authors' names for this purpose.
\renewcommand{\shortauthors}{Liu et al.}

%%
%% The abstract is a short summary of the work to be presented in the
%% article.
\begin{abstract}

During the early stages of interface design, designers need to produce multiple sketches to explore a design space.  Design tools often fail to support this critical stage, because they insist on specifying more details than necessary. Although recent advances in generative AI have raised hopes of solving this issue, in practice they fail because expressing loose ideas in a prompt is impractical. In this paper, we propose a diffusion-based approach to the low-effort generation of interface sketches. It breaks new ground by allowing flexible control of the generation process via three types of inputs: A) prompts, B) wireframes, and C) visual flows. The designer can provide any combination of these as input at any level of detail, and will get a diverse gallery of low-fidelity solutions in response. The unique benefit is that large design spaces can be explored rapidly with very little effort in input-specification. We present qualitative results for various combinations of input specifications. Additionally, we demonstrate that our model aligns more accurately with these specifications than other models. 

% OLD ABSTRACT
%When sketching Graphical User Interfaces (GUIs), designers need to explore several aspects of visual design simultaneously, such as how to guide the user’s attention to the right aspects of the design while making the intended functionality visible. Although current Large Language Models (LLMs) can generate GUIs, they do not offer the finer level of control necessary for this kind of exploration. To address this, we propose a diffusion-based model with multi-modal conditional generation. In practice, our model optionally takes semantic segmentation, prompt guidance, and flow direction to generate multiple GUIs that are aligned with the input design specifications. It produces multiple examples. We demonstrate that our approach outperforms baseline methods in producing desirable GUIs and meets the desired visual flow.

% Designing visually engaging Graphical User Interfaces (GUIs) is a challenge in HCI research. Effective GUI design must balance visual properties, like color and positioning, with user behaviors to ensure GUIs easy to comprehend and guide attention to critical elements. Modern GUIs, with their complex combinations of text, images, and interactive components, make it difficult to maintain a coherent visual flow during design.
% Although current Large Language Models (LLMs) can generate GUIs, they often lack the fine control necessary for ensuring a coherent visual flow. To address this, we propose a diffusion-based model that effectively handles multi-modal conditional generation. Our model takes semantic segmentation, optional prompt guidance, and ordered viewing elements to generate high-fidelity GUIs that are aligned with the input design specifications.
% We demonstrate that our approach outperforms baseline methods in producing desirable GUIs and meets the desired visual flow. Moreover, a user study involving XX designers indicates that our model enhances the efficiency of the GUI design ideation process and provides designers with greater control compared to existing methods.    



% %%%%%%%%%%%%%%%%%%%%%%%%%%%%%%%%%%%%%%%%%%%%%%%%%%%%%%
% % Writing Clinic Comments:
% %%%%%%%%%%%%%%%%%%%%%%%%%%%%%%%%%%%%%%%%%%%%%%%%%%%%%%
% % Define: Effective UI design
% % Motivate GANs and write in full form.
% % LLMs vs ControlNet vs GANs
% % Say something about the Figma plugin?
% % Write the work is novel or what has been done before
% % What is desirable UI and how to evalutate that?
% % Visual Flow - main theme (center around it)
% % Re-Title: use word Flow!
% % Use ControlNet++ & SPADE for abstract.
% % Write about input/output. 
% % Why better than previous work?
% %%%%%%%%%%%%%%%%%%%%%%%%%%%%%%%%%%%%%%%%%%%%%%%%%%%%%

% % v2:
% % \noindent \textcolor{red}{\textbf{NEW Abstract!} (Post Writing Clinic 1 - 25-Jun)}

% % \noindent \textcolor{red}{----------------------------------------------------------------------}

% % \noindent Designing user interfaces (UIs) is a time-consuming process, particularly for novice designers. 
% % Creating UI designs that are effective in market funneling or any other designer defined goal requires a good understanding of the visual flow to guide users' attention to UI elements in the desired order. 
% % While current Large Language Models (LLMs) can generate UIs from just prompts, they often lack finer pixel-precise control and fail to consider visual flow. 
% % In this work, we present a UI synthesis method that incorporates visual flow alongside prompts and semantic layouts. 
% % Our efficient approach uses a carefully designed Generative Adversarial Network (GAN) optimized for scenarios with limited data, making it more suitable than diffusion-based and large vision-language models.
% % We demonstrate that our method produces more "desirable" UIs according to the well-known contrast, repetition, alignment, and proximity principles of design. 
% % We further validate our method through comprehensive automatic non-reference, human-preference aligned network scoring and subjective human evaluations.
% % Finally, an evaluation with xx non-expert designers using our contributed Figma plugin shows that <method-name> improves the time-efficiency as well as the overall quality of the UI design development cycle.

% % \noindent \textcolor{red}{----------------------------------------------------------------------}


% \noindent \textcolor{blue}{\textbf{NEW Abstract!} (Pre Writing Clinic 9-July)}

% \noindent \textcolor{blue}{----------------------------------------------------------------------}

% \noindent Exploring different graphical user interface (GUI) design ideas is time-consuming, particularly for novice designers. 
% Given the segmentation masks, design requirement as prompt, and/or preferred visual flow, we aim to facilitate creative exploration for GUI design and generate different UI designs for inspiration.
% While current Vision Language Models (VLMs) can generate GUIs from just prompts, they often lack control over visual concepts and flow that are difficult to convey through language during the generation process. 
% In this work, we present FlowGenUI, a semantic map-guided GUI synthesis method that optionally incorporates visual flow information based on the user's choice alongside language prompts. 
% We demonstrate that our model not only creates more realistic GUIs but also creates "predictable" (how users pay attention to and order of looking at GUI elements) GUIs.
% Our approach uses Stable Diffusion (SD), a large paired image-text pretrained diffusion model with a rich latent space that we steer toward realistic GUIs using a trainable copy of SD's encoder for every condition (segmentation masks, prompts, and visual flow). 
% We further provide a semantic typography feature to create custom text-fonts and styles while also alleviating SD's inherent limitations in drawing coherent, meaningful and correct aspect-ratio text. 
% Finally, a subjective evaluation study of XX non-expert and expert designers demonstrates the efficiency and fidelity of our method.


% This process encourages creativity and prevents designers from falling into habitual patterns.


% ------------------------------------------------------------------
% Joongi Why is it important to create realistic GUI?
% I do not see how the Visual Flow given on the left hand side is reflected in the results on the right hand side. 
% I’d avoid making unsubstantiated claims about designers (falling into habitual patterns).
% The UIs you generate do not “align with users’ attention patterns” but rather try to control it (that’s what visual flow means)
% ------------------------------------------------------------------
% Comments - Writing Clinic - 9th July:
% Improve title. More names: FlowGen
% Figure 1: Use an inference time hand-drawn mask
% Figure 1: Show both workflows. Add a designer --> Input.
% Figure 1: Make them more diverse
% ------------------------------------------------------------------
% Designing graphical user interfaces (GUIs) requires human creativity and time. Designers often fall into habitual patterns, which can limit the exploration of new ideas. 
% To address this, we introduce FlowGenUI, a method that facilitates creative exploration and generates diverse GUI designs for inspiration. By using segmentation masks, design requirements as prompts, and/or selected visual flows, our approach enhances control over the visual concepts and flows during the generation process, which current Vision Language Models (VLMs) often lack.
% FlowGenUI uses Stable Diffusion (SD), a largely pretrained text-to-image diffusion model, and guides it to create realistic GUIs. 
% We achieve this by using a trainable copy of SD's encoder for each condition (segmentation masks, prompts, and visual flow). 
% This method enables the creation of more realistic and predictable GUIs that align with users' attention patterns and their preferred order of viewing elements.
% We also offer a semantic typography feature that creates custom text fonts and styles while addressing SD's limitations in generating coherent, meaningful, and correctly aspect-ratio text.
% Our approach's efficiency and fidelity are evaluated through a subjective user study involving XX designers. 
% The results demonstrate the effectiveness of FlowGenUI in generating high-quality GUI designs that meet user requirements and visual expectations.

% ---------------------------------------


%A critical and general issue remains while using such deep generative priors: creating coherent, meaningful and correct aspect-ratio text. 
%We tackle this issue within our framework and additionally provide a semantic typography feature to create custom text-fonts and styles. 


% %Creating UI designs that are effective in market funneling or any other designer-defined goal requires a good understanding of the visual flow to guide users' attention to UI elements in the desired order. 
% %While current largely pre-trained Vision Language Models (VLMs) can generate GUIs from just prompts, they often lack finer or pixel-precise control which can be crucial for many easy-to-understand visual concepts but difficult to convey through language. 
% % However, obtaining such pixe-level labels is an extremely expensive so we
% % For example - overlaying text on images with certain aspect ratios and two equally separated buttons 
% Additionally, all prior GUI generation work fails to consider visual flow information during the generation process. 
% We demonstrate that visual flow-informed generation not only creates more realistic and human-friendly GUIs but also creates "predictable" (how users pay attention to and order of looking at GUI elements) UIs that could be beneficial for designers for tasks like creating effective market funnels.
% In this work, we present a semantic map-guided GUI synthesis method that optionally incorporates visual flow information based on the user's choice alongside language prompts. 
% Our approach uses Stable Diffusion, a large (billions) paired image-text pretrained diffusion model with a rich latent space that we steer toward realistic GUIs using an ensemble of ControlNets. 
% % TODO: Mention it in 1 sentence:
% A critical and general issue remains while using such deep generative priors: creating coherent, meaningful and correct aspect-ratio text. 
% We tackle this issue within our framework and additionally provide a semantic typography feature to create custom text-fonts and styles. 
% To evaluate our method, we demonstrate that our method produces more "desirable" UIs according to the well-known contrast, repetition, alignment, and proximity principles of design. 
% % We further validate our method through comprehensive automatic non-reference and human-preference aligned scores. (TODO: Maybe Unskip if we get UIClip from Jason!)
% % TODO: Re-word this and only keep ideation cycles and time-efficiency.
% Finally, a subjective evaluation study of XX non-expert and expert designers demonstrates the efficiency and fidelity of our method.
% % improves the time-efficiency by quick iterations of the UI design ideation process.
% %Finally, an evaluation with xx non-expert designers using our contributed <method-name> improves the time-efficiency by quick iterations of the UI design ideation cycle.

%\noindent \textcolor{blue}{----------------------------------------------------------------------}


%In an evaluation with xx designers, we found that GenerativeLayout: 1) enhances designers' exploration by expanding the coverage of the design space, 2) reduces the time required for exploration, and 3) maintains a perceived level of control similar to that of manual exploration.



% Present-day graphical user interfaces (GUIs) exhibit diverse arrangements of text, graphics, and interactive elements such as buttons and menus, but representations of GUIs have not kept up. They do not encapsulate both semantic and visuo-spatial relationships among elements. %\color{red} 
% To seize machine learning's potential for GUIs more efficiently, \papername~ exploits graph neural networks to capture individual elements' properties and their semantic—visuo-spatial constraints in a layout. The learned representation demonstrated its effectiveness in multiple tasks, especially generating designs in a challenging GUI autocompletion task, which involved predicting the positions of remaining unplaced elements in a partially completed GUI. The new model's suggestions showed alignment and visual appeal superior to the baseline method and received higher subjective ratings for preference. 
% Furthermore, we demonstrate the practical benefits and efficiency advantages designers perceive when utilizing our model as an autocompletion plug-in.


% Overall pipeline: Maybe drop semantic typography / visual flow?
\end{abstract}

%%
%% The code below is generated by the tool at http://dl.acm.org/ccs.cfm.
%% Please copy and paste the code instead of the example below.
%%
\begin{CCSXML}
<ccs2012>
   <concept>
       <concept_id>10003120.10003121.10003126</concept_id>
       <concept_desc>Human-centered computing~HCI theory, concepts and models</concept_desc>
       <concept_significance>500</concept_significance>
       </concept>
   <concept>
       <concept_id>10003120.10003121.10003129</concept_id>
       <concept_desc>Human-centered computing~Interactive systems and tools</concept_desc>
       <concept_significance>500</concept_significance>
       </concept>
   <concept>
       <concept_id>10010147.10010178.10010216</concept_id>
       <concept_desc>Computing methodologies~Philosophical/theoretical foundations of artificial intelligence</concept_desc>
       <concept_significance>500</concept_significance>
       </concept>
 </ccs2012>
\end{CCSXML}

\ccsdesc[500]{Human-centered computing~HCI theory, concepts and models}
\ccsdesc[500]{Human-centered computing~Interactive systems and tools}


%%
%% Keywords. The author(s) should pick words that accurately describe
%% the work being presented. Separate the keywords with commas.
\keywords{Thoughtful AI, Large Language Model, Agent, Human-AI Interaction, Thought, Theory, Interaction Paradigm}
%% A "teaser" image appears between the author and affiliation
%% information and the body of the document, and typically spans the
%% page.
\begin{teaserfigure}
\end{teaserfigure}

% \received{20 February 2007}
% \received[revised]{12 March 2009}
% \received[accepted]{5 June 2009}

%%
%% This command processes the author and affiliation and title
%% information and builds the first part of the formatted document.
\maketitle

\section{Introduction}

Tutoring has long been recognized as one of the most effective methods for enhancing human learning outcomes and addressing educational disparities~\citep{hill2005effects}. 
By providing personalized guidance to students, intelligent tutoring systems (ITS) have proven to be nearly as effective as human tutors in fostering deep understanding and skill acquisition, with research showing comparable learning gains~\citep{vanlehn2011relative,rus2013recent}.
More recently, the advancement of large language models (LLMs) has offered unprecedented opportunities to replicate these benefits in tutoring agents~\citep{dan2023educhat,jin2024teach,chen2024empowering}, unlocking the enormous potential to solve knowledge-intensive tasks such as answering complex questions or clarifying concepts.


\begin{figure}[t!]
\centering
\includegraphics[width=1.0\linewidth]{Figs/Fig.intro.pdf}
\caption{An illustration of coding tutoring, where a tutor aims to proactively guide students toward completing a target coding task while adapting to students' varying levels of background knowledge. \vspace{-5pt}}
\label{fig:example}
\end{figure}

\begin{figure}[t!]
\centering
\includegraphics[width=1.0\linewidth]{Figs/Fig.scaling.pdf}
\caption{\textsc{Traver} with the trained verifier shows inference-time scaling for coding tutoring (detailed in \S\ref{sec:scaling_analysis}). \textbf{Left}: Performance vs. sampled candidate utterances per turn. \textbf{Right}: Performance vs. total tokens consumed per tutoring session. \vspace{-15pt}}
\label{fig:scale}
\end{figure}


Previous research has extensively explored tutoring in educational fields, including language learning~\cite{swartz2012intelligent,stasaski-etal-2020-cima}, math reasoning~\cite{demszky-hill-2023-ncte,macina-etal-2023-mathdial}, and scientific concept education~\cite{yuan-etal-2024-boosting,yang2024leveraging}. 
Most aim to enhance students' understanding of target knowledge by employing pedagogical strategies such as recommending exercises~\cite{deng2023towards} or selecting teaching examples~\cite{ross-andreas-2024-toward}. 
However, these approaches fall short in broader situations requiring both understanding and practical application of specific pieces of knowledge to solve real-world, goal-driven problems. 
Such scenarios demand tutors to proactively guide people toward completing targeted tasks (e.g., coding).
Furthermore, the tutoring outcomes are challenging to assess since targeted tasks can often be completed by open-ended solutions.



To bridge this gap, we introduce \textbf{coding tutoring}, a promising yet underexplored task for LLM agents.
As illustrated in Figure~\ref{fig:example}, the tutor is provided with a target coding task and task-specific knowledge (e.g., cross-file dependencies and reference solutions), while the student is given only the coding task. The tutor does not know the student's prior knowledge about the task.
Coding tutoring requires the tutor to proactively guide the student toward completing the target task through dialogue.
This is inherently a goal-oriented process where tutors guide students using task-specific knowledge to achieve predefined objectives. 
Effective tutoring requires personalization, as tutors must adapt their guidance and communication style to students with varying levels of prior knowledge. 


Developing effective tutoring agents is challenging because off-the-shelf LLMs lack grounding to task-specific knowledge and interaction context.
Specifically, tutoring requires \textit{epistemic grounding}~\citep{tsai2016concept}, where domain expertise and assessment can vary significantly, and \textit{communicative grounding}~\citep{chai2018language}, necessary for proactively adapting communications to students' current knowledge.
To address these challenges, we propose the \textbf{Tra}ce-and-\textbf{Ver}ify (\textbf{\model}) agent workflow for building effective LLM-powered coding tutors. 
Leveraging knowledge tracing (KT)~\citep{corbett1994knowledge,scarlatos2024exploring}, \model explicitly estimates a student's knowledge state at each turn, which drives the tutor agents to adapt their language to fill the gaps in task-specific knowledge during utterance generation. 
Drawing inspiration from value-guided search mechanisms~\citep{lightman2023let,wang2024math,zhang2024rest}, \model incorporates a turn-by-turn reward model as a verifier to rank candidate utterances. 
By sampling more candidate tutor utterances during inference (see Figure~\ref{fig:scale}), \model ensures the selection of optimal utterances that prioritize goal-driven guidance and advance the tutoring progression effectively. 
Furthermore, we present \textbf{Di}alogue for \textbf{C}oding \textbf{T}utoring (\textbf{\eval}), an automatic protocol designed to assess the performance of tutoring agents. 
\eval employs code generation tests and simulated students with varying levels of programming expertise for evaluation. While human evaluation remains the gold standard for assessing tutoring agents, its reliance on time-intensive and costly processes often hinders rapid iteration during development. 
By leveraging simulated students, \eval serves as an efficient and scalable proxy, enabling reproducible assessments and accelerated agent improvement prior to final human validation. 



Through extensive experiments, we show that agents developed by \model consistently demonstrate higher success rates in guiding students to complete target coding tasks compared to baseline methods. We present detailed ablation studies, human evaluations, and an inference time scaling analysis, highlighting the transferability and scalability of our tutoring agent workflow.

% \section{Related Work}

% Our vision relates to piror research in Explainable AI (XAI), Mixed-Initiative Interaction, and Feed-Forward Interaction, but introduces a fundamentally new paradigm of a continuously thinking AI in human-AI interaction.

% Explainable AI (XAI) has focused on making AI’s decision-making processes transparent to users, often through post-hoc explanations such as model interpretability techniques \cite{lipton2016mythos, doshi2017towards}. However, these explanations are typically static and retrospective, providing insight only after a decision is made. In contrast, we propose Thoughtful AI, where AI’s internal thought processes are continuously evolving and interactively shared with users in real-time.

% Our work is also related to Mixed-Initiative Interaction, where both human and AI agents can take turns leading an interaction based on situational needs \cite{horvitz1999principles, amershi2014power}. While mixed-initiative AI systems, such as intelligent assistants, enable more fluid, adaptive collaborations, they still operate in turn-based modes, intervening only at opportune moments. In contrast, Thoughtful AI introduces a full-duplex model of interaction, where AI and humans think in parallel, continuously iterating on ideas rather than exchanging discrete turns.

% Additionally, Feed-Forward Interaction enables users to see AI-generated previews of possible outcomes before committing to an action \cite{djajadiningrat2002feedforward, verduyn2020feedforward}. While this approach enhances predictability, it does not allow AI to actively refine or evolve its thoughts dynamically over time. Thoughtful AI moves beyond passive previewing by establishing a shared cognitive space, where both AI and human thought processes interact, shape, and refine each other throughout the interaction.

% By integrating continuous AI thought, our work introduces a new paradigm that reimagines AI not as a reactive tool but as an active thought partner.
\section{What makes AI ``Thoughtful''?}


Discussions around ``thought'' in LLM research often focus on a chain-of-thought approach, where a model generates intermediate reasoning steps to improve performance on reasoning tasks. From an HCI perspective, however, \textit{thought} can be understood more broadly. We define \textit{Thoughtfulness} as:

\newtheorem{thoughtfuldefinition}{Definition}
\begin{definition}
\textbf{Thoughtfulness} refers to a system's ability to continuously generate, develop, and selectively communicate its \textit{intermediate processes and responses} over the course of an interaction. 
\end{definition}


In contrast to LLM-based definitions, our conception of Thoughtful AI emphasizes how these ongoing processes shape the system’s behavior and its capacity to interact with human. Thoughts can emerge at any point in an interaction, triggered by external stimuli or intrinsic reflection, may be expressed or remain internal, and they can take various forms, from abstract keywords to visual or auditory representations.


To better understand how Thoughtful AI differs from conventional systems, we examine four key traits of this interaction paradigm (\autoref{tab:02-traits}), comparing them with the turn-based model to highlight the key distinctions.


\subsection{Traits of Thoughtful AI}

% Please add the following required packages to your document preamble:
% \usepackage{booktabs}
% \usepackage{graphicx}
% \usepackage[normalem]{ulem}
% \useunder{\uline}{\ul}{}

\begin{table}
  \resizebox{\columnwidth}{!}{%
    \begin{tabular}{@{}lll@{}}
      \toprule
      \textbf{Trait} & \textbf{Current AI} & \textbf{Thoughtful AI} \\ 
      \midrule
      \textit{Intermediate Medium} & 
        \begin{tabular}[c]{@{}l@{}}
          Reveals only\\
          final outputs
        \end{tabular} & 
        \begin{tabular}[c]{@{}l@{}}
          Surfaces intermediate\\
          thoughts in real time
        \end{tabular} \\
      \addlinespace
      \textit{Full-Duplex Process} & 
        \begin{tabular}[c]{@{}l@{}}
          Waits for user\\
          prompts
        \end{tabular} & 
        \begin{tabular}[c]{@{}l@{}}
          Continuously thinks in\\
          parallel with user activity
        \end{tabular} \\
      \addlinespace
      \textit{Intrinsic Driver} & 
        \begin{tabular}[c]{@{}l@{}}
          Responds only\\
          when asked
        \end{tabular} & 
        \begin{tabular}[c]{@{}l@{}}
          Self-initiates actions\\
          based on thoughts
        \end{tabular} \\
      \addlinespace
      \textit{Shared Cognitive Space} & 
        \begin{tabular}[c]{@{}l@{}}
          Turn-based\\
          exchanges
        \end{tabular} & 
        \begin{tabular}[c]{@{}l@{}}
          Builds on collaborative\\
          fragmented ideas
        \end{tabular} \\
      \bottomrule
    \end{tabular}%
  }
  \caption{Comparison of Thoughtful AI and Current AI across key traits.}
  \label{tab:02-traits}
\end{table}

\subsubsection{Intermediate Medium}
A fundamental distinction between a \textit{thought} and a \textit{response} is that a thought is \textit{in-progress}, potentially fragmented, and not necessarily intended for final output. In Thoughtful AI, the system’s processing, whether it involves brainstorming ideas, weighing pros and cons, or exploring parallel strategies, can be treated as an \textit{intermediate medium} that the AI may optionally share.

In most existing AI systems, users see only a final answer. This one-shot approach is limiting: once the user receives an answer, any clarifications or follow-up requests often require multiple back-and-forth queries. If the AI’s thinking was flawed or incomplete, the user has no window into how or why the system arrived at its conclusion.


\subsubsection{Full-duplex Process}
Human conversation does not pause for one party to ``finish thinking'', nor do humans stop thinking when others start to speak. Similarly, Thoughtful AI enables a \textit{full-duplex process} where thinking is continuous and can occur any point in the interaction, rather than locked into turn-based cycles. AI thoughts may be triggered by user inputs or by the system’s own internal reflections; they can also evolve without any external stimuli, for example, during periods of user silence.

Conventional AI interactions are typically half-duplex. After producing an answer, the AI stops listening or processing until another user query arrives. It remains idle, unaware of changing contexts and does not develop anything internally during that downtime.


\subsubsection{Intrinsic Driver}
One of the key aspects of human intelligence is that thought is not merely reactive: it also serves as an \textit{intrinsic driver} of action. Similarly, in Thoughtful AI, thinking can serve as a mechanism that enables the system to self-initiate its interactions and be \textit{proactive}.
Rather than being solely triggered by user input, Thoughtful AI can independently generate ideas, monitor evolving contexts, and identify opportunities to intervene or contribute.

Traditional turn-based systems are \emph{reactive}: they respond only once the user issues a query. Even so-called ``proactive'' assistants typically rely on predefined triggers or simple rule-based heuristics. In contrast, a Thoughtful AI continuously \emph{thinks} in the background, allowing it to model its intrinsic motivation to take actions based on the intermediate thoughts. Its continuous thought process is the primary engine for proactive behavior.

\subsubsection{Shared Cognitive Space}

Finally, a \textit{shared cognitive space} could emerge when AI and user co-exist in an ongoing thought process. Rather than the AI presenting a single answer and the user responding with a single set of followups, both parties iteratively build upon each other’s partial ideas. We envision that the AI’s intermediate thoughts, user feedback, clarifying questions, and real-time refinements etc form a collaborative ``thinking canvas.''

Traditional interactions are linear and unidirectional: the AI provides an answer, the user reacts, and so on. Even if the user can add further prompts, there is no persistent collaborative workspace where partial ideas accumulate and evolve. 


\subsection{Implications for HCI}
The four traits of Thoughtful AI---an \textit{intermediate medium}, \textit{full-duplex process}, \textit{intrinsic driver}, and a \textit{shared cognitive space}---open up new possibilities for how humans interact with AI. Below, we discuss four implications for Human-AI Interaction.

\subsubsection{From Passive Respondents to Proactive Participants}
A first implication of Thoughtful AI is the transition from AI as a passive respondent to an active participant in interactions. Conventional AI systems operate in a reactive manner, awaiting user input before generating a response. In contrast, Thoughtful AI continuously generates thoughts, enabling it to self-initiate actions and engage more dynamically with users.

This proactivity manifests in two ways. First, the \textit{full-duplex process} (Trait 2) allows the AI to generate thoughts in parallel with user input, meaning it does not need to remain idle between interactions. It can detect emerging needs, anticipate user queries, and even interrupt when necessary. Second, thoughts equip AI with an \textit{intrinsic drive} (Trait 3) to contribute based on its internal cognitive processes. This ability to self-motivate and intervene resembles human-like initiative, distinguishing it from traditional systems that rely on heuristics or predefined triggers~\cite{horvitz1999principles}.


\subsubsection{Continuous Cognitive Alignment}

A second major implication for HCI arises from Thoughtful AI's ability to reveal \emph{in-progress} ideas (Trait 1: \textit{Intermediate Medium}) continuously (Trait 2: \textit{Full-duplex Process}). Rather than presenting a single, static final output, the system selectively shares partial thoughts. Users can then not only observe these thoughts for better interpretability, but also guide the AI's thinking early in the process, leading to more dynamic collaboration.

By showing its intermediate reasoning, Thoughtful AI helps bridge the ``Gulf of Evaluation,'' where users struggle to understand how or why a system arrived at a particular result~\cite{norman2013design}. Here, the AI’s partial thoughts provide visibility into its evolving path, allowing users to catch misconceptions or supply additional context before errors compound. Similarly, users can steer the AI's next steps more effectively, shrinking the ``Gulf of Execution,'' since they can articulate what they want \emph{as} they see the AI’s tentative directions.

This also helps build common ground~\cite{clark1991grounding} between human and AI, aligning with established theories of communication that emphasize the importance of shared context for effective collaboration. By understanding the AI's intermediate thought, users gain a clearer understanding of the system's current assumptions and partial conclusions. This shared cognitive workspace drives continuous alignment between the user's thinking and AI's thinking process.


\subsubsection{Beyond Turn-based Interaction}
Thoughtful AI prompts us to reconsider the fundamental structure of human–AI interaction. Traditional chatbot interfaces operate in discrete, back-and-forth messages, with the AI effectively going idle after sending each response. By contrast, a \emph{full-duplex process} (Trait 2) enables the AI to continuously think and listen, even when the user is not actively providing input.
This opens the door to interaction models that transcend simple chat boxes. For instance, an AI planning assistant could silently update its suggestions as it overhears new constraints in a virtual meeting, intervening only when necessary (Traits 2 and 3). 
More interestingly, AI and users can build a \emph{shared cognitive space} (Trait 4), adding thoughts, annotations, and partial ideas to a collective interface. Rather than a linear log of exchanges, this collaborative canvas allows ideas to branch, merge, and evolve continuously---resembling a dynamic mind-map or sketchnote more than a turn-based chat.


\subsubsection{Messy, Fragmented and Informal Interaction}
Traditional AI systems demand users structure their inputs as complete, well-formed queries or commands. This formality creates friction: human thinking is inherently non-linear, often involving half-formed ideas and abrupt shifts in focus. 
Thoughtful AI embraces this messiness by using intermediate thoughts as the first-class citizen of interaction. 
This shift towards a more natural, fragmented, and informal interaction style: AI thinking can be messy, incomplete, and in-progress. 
In addition, instead of requiring users to provide polished, fully-formed queries or responses, users can also input partial ideas, rough thoughts, or even keywords, and the system can process and respond to these in a similarly fragmented manner. 
This approach mirrors how humans often think and communicate: by iterating on ideas, refining them over time, and bouncing off incomplete or tentative thoughts. 
\section{Example Projects}
To further illustrate the practical applications of Thoughtful AI, we present two projects that embody its core principles. The first, Inner Thoughts, explores how AI can proactively engage in conversations by developing and evaluating its internal thoughts before participating. The second, ThinkaloudLM, investigates AI-generated thoughts as an interface component that allows users to observe and interact with in real-time.

\subsection{Inner Thoughts}
\begin{figure}
    \centering
  \includegraphics[width=0.65\linewidth]{figures/03_inner_thoughts.pdf}
    \caption{Conversational Agents with Inner Thoughts: AI generates a train of thoughts and evaluates them based on their intrinsic motivation to participate.}
    \label{fig:inner_thoughts}
\end{figure}

Conversational AI often rely on turn-taking prediction techniques, where an algorithm determines the most likely next speaker and generates a response accordingly. However, these approaches struggle in multi-party conversations where turn allocation is often ambiguous. Furthermore, existing conversational agents tend to fall into two extremes: either they remain passive, requiring explicit user input to respond, or they overcompensate, generating frequent and often unnecessary interruptions.

Instead of predicting conversational turns, \textit{Inner Thoughts}~\cite{liu2025inner} introduces a new method in which AI autonomously generates a continuous stream of covert (internal) thoughts, similar to how humans process conversations. These thoughts remain internal until AI evaluates whether it has sufficient \textit{intrinsic motivation} to contribute---defined by heuristics derived from a user study on human conversational behavior. Once the AI determines that a thought is relevant and meaningful, it then strategically selects an appropriate moment to engage (\autoref{fig:inner_thoughts}).

The Inner Thoughts framework embodies multiple traits of Thoughtful AI:
\textit{Intermediate Medium}: AI does not simply generate final responses but produces an evolving series of intermediate thoughts; \textit{Full-Duplex Process}: Instead of waiting for user prompts, the AI continuously listens and generates internal responses in parallel with the ongoing conversation; \textit{Intrinsic Driver}: The AI does not just react but actively determines when to contribute based on its intrinsic motivation.

The framework was implemented in two systems, a multi-agent conversational simulation and a chatbot. Through a technical evaluation, conversational agents using Inner Thoughts significantly outperformed a traditional next-speaker prediction baseline across multiple criteria, including turn appropriateness, coherence, engagement, and adaptability.
Participants overwhelmingly preferred AI interactions using Inner Thoughts (in 82\% of the conversations).



\subsection{ThinkaloudLM}
\begin{figure}
    \centering
  \includegraphics[width=\linewidth]{figures/03_thinkaloudlm.png}
    \caption{ThinkaloudLM: AI generates intermediate, fragmented thoughts in parellel to user input.}
    \label{fig:thinkaloudlm}
\end{figure}

While Inner Thoughts explores AI's ability to think internally, ThinkaloudLM investigates a complementary question: what if users could actively observe and engage with AI's thought process?

Most AI systems present only finalized outputs, hiding the intermediate thoughts behind their decisions. This lack of transparency can lead to user distrust, inefficiencies in communication, and missed opportunities for collaboration. For example, if an AI-powered writing assistant provides a response that feels misaligned with the user’s intent, the user must iteratively refine their query, leading to unnecessary back-and-forth exchanges.

By contrast, ThinkaloudLM envisions a system where users can engage with AI-generated ideas before they fully materialize into responses. Rather than receiving a single, static reply, users might see a live preview of AI’s thinking, including key points, possible directions, or alternative solutions. They can then refine, select, or discard thoughts before finalizing the AI’s output.

This approach leverages multiple traits of Thoughtful AI:
\textit{Intermediate Medium}: The AI’s partially formed ideas are themselves the primary interface, giving users a real-time window into the AI’s thought process; \textit{Full-duplex Process}: By continuously updating these intermediate thoughts, the AI can incorporate user feedback on-the-fly while still in the ``draft'' stage; \textit{Shared cognitive space}: Users and AI co-construct the final answer, using the AI's visible thoughts as a collaborative canvas for discussion, refinement, and mutual alignment.
To explore this concept, a participatory design study is currently underway with qualitative interviews and observational studies. 
\section{Discussion}\label{sec:disc}

While diffusion models can generate highly realistic images, most of the images they produce still contain visible artifacts. In particular, we find that only 17\% of diffusion model-generated images are misclassified as real at rates consistent with random guessing. Notably, this misclassification rate increases to 43\% when the viewing duration is restricted to 1 second. By curating a dataset of 599 images and conducting a large scale digital experiment, we can begin to answer fundamental questions about what drives the appearance of photorealism in diffusion model-generated images. 

First, we find that images with greater scene complexity tend to introduce more opportunities for artifacts to appear, making it easier for participants to detect AI-generated images. Our results reveal that participants were less accurate at identifying AI-generated portraits compared to more complex scenes, such as those involving multiple people in candid settings.  Based on qualitative analysis of the images, we identify three main reasons for this difference. First, portraits often feature a single person against a blurred background, which can obscure details and provide fewer cues compared to full-body or group images. Second, portraits typically involve fewer and simpler poses, focusing only on the face and torso, leaving fewer opportunities for errors or inconsistencies to be apparent. Third, the prevalence of edited and retouched portraits in real-world photography complicates the distinction between real and AI-generated portraits, addressing the question of how subject type and context (e.g., unknown people vs. public figures) influence the perceived authenticity of an image. In contrast, more complex images, like full-body or group shots, involve a greater number of elements, increasing the likelihood of noticeable errors or inconsistencies. Similar to our results on AI-generated images, we find that real images with lower scene complexity are also harder to identify as real. 

Second, we identify five high-level categories of artifacts and implausibilities and find that the easiest images to identify as diffusion model generated are the ones with anatomical implausibilities, such as unrealistic body proportions and stylistic artifacts like overly glossy or waxy features.

Third, by randomizing display time, we identify the relationship between how long an individual looks at an image and their accuracy at distinguishing between real and AI-generated images. Specifically, we find that participants' accuracy at identifying an AI-generated image upon a quick glance of 1 second is 72\% and increases by 5 percentage points with just an additional 4 seconds of viewing time and 10 percentage points when unconstrained by time. Given the nature of rapid scrolling on social media and how much time people have to see advertisements as they pass by billboards on a highway, these results reveal the importance of attentive viewing of images before making judgments about an image's veracity. 


Fourth, we find that human curation had a notable negative impact on participants' accuracy compared to uncurated images generated by the same prompts as the human-curated AI-generated images. In particular, the images curated by our research team were harder to identify as AI-generated than 84\% of the uncurated images generated using the same prompts as the curated images. This finding reveals the limitation of state-of-the-art diffusion models in producing images of consistent quality. It also suggests that human curation is a bottleneck to generating fake images at scale. The process of generating high-quality AI images is inherently iterative---users refine prompts and select outputs until they achieve their desired result. This fundamental aspect of AI image generation is evident across all applications, from advertising and marketing to education and beyond. While concerns exist about fake images being used to mislead or impersonate, many use cases exist for business and educational applications~\cite{vartiainen2023using, hartmann2023power, gvirtz2023text}. The critical role of human curation in this iterative process further emphasizes how the photorealism of images produced by diffusion models depends not only on the capabilities of the diffusion model but also on the quality of human curation, choice of prompts, and context of the scene. Given the importance of these factors beyond the generative AI model, these results reveal the importance of considering these factors in research examining human perception of AI-generated images. Without considering these elements, it is possible to produce biased findings showing AI-generated images are more or less realistic than they really appear in real-world settings. 

The taxonomy offers a practical framework on which to build AI literacy tools for the general public. We synthesized information from diverse sources such as social media posts, scientific literature, and our online behavioral study with 50,444 participants to systematically categorize artifacts in AI-generated images. Through this process, we identify five key categories: anatomical implausibilities, which involve unlikely artifacts in individual body parts or inconsistent proportions, particularly in images with multiple people;  stylistic artifacts, referring to overly glossy, waxy, or picturesque qualities of specific elements of an image; functional implausibilities, arising from a lack of understanding of real-world mechanics and leading to objects or details that appear impossible or nonsensical; violations of physics, which include inconsistencies in shadows, reflections, and perspective that defy physical logic; and sociocultural implausibilities, focusing on scenarios that violate social norms, cultural context, or historical accuracy. Our taxonomy builds upon the Borji 2023 taxonomy \cite{borji2023qualitative} and focuses on images that appear more realistic at first glance, which is useful for comparing and contrasting real photographs with diffusion model generated images for revealing the nuances of the artifacts and implausibilities~\cite{kamali2024distinguish}. Moreover, this taxonomy offers a shared language by which practitioners and researchers can communicate about artifacts commonly seen in AI-generated images and exposes the persistent challenges that can help people identify AI-generated images. 

\subsection{Future Work and Limitations}

In addition to aiding in identifying AI-generated content, the taxonomy offers insights into the open problems for producing realistic AI-generated images. Future work may explore integrating such taxonomies into model evaluation frameworks to provide iterative feedback during the development of generative models. As models advance to address the weaknesses presented in this taxonomy, new and more subtle artifacts may emerge, requiring future updates to this taxonomy. This dynamic interplay between detection and generation capabilities demonstrates why we need to maintain robust human detection abilities even as models evolve. We acknowledge the potential dual use of these insights to create more deceptive synthetic media, and we believe that transparent documentation of artifacts does more good than harm by offering detection strategies and an opportunity to develop general awareness in the public.

Large-scale digital experiments with participants who participate based on their own interests come with certain limitations. First, we did not collect demographic data from participants. Participants were not recruited for this experiment; instead, participants found the experiment organically and participated. Given the organic nature of the participation, we prioritized maximizing engagement, which involves questions unrelated to distinguishing AI-generated and real images like demographic questions. While this approach enabled substantial data collection, it limits analysis by excluding factors like age, gender, and cultural background that may influence detection. 

Second, we provided feedback on the correct answer after each participant made an observation, which has the potential to introduce learning effects. Future research could address these open questions by collecting demographic data to design more inclusive AI literacy tools and evaluating how performance changes with and without feedback. 

This research focused on images generated by state-of-the-art generative models available in 2024, and the findings are inherently tied to the state of diffusion models and generative AI technologies as of 2024. In the future, models are likely to change, and the somewhat visible errors that emerge will also likely change. Past state-of-the-art GAN models such as StyleGAN2~\cite{karras2020analyzingimprovingimagequality} and BigGAN~\cite{brock2018biggan}, often produced more noticeable artifacts in facial features, color balance, and overall photorealism, making their outputs more easily distinguishable. Nonetheless, the current taxonomy on diffusion models points out elements like anatomical implausibilities and stylistic artifacts that can be mapped to the facial feature and color balance cues. These recurring issues offer evidence of the taxonomy’s robustness to differences across model generations, but future studies should explore how the taxonomy may need to adapt to these changes, which may involve adding or removing categories or may involve further identifying nuances within these categories. As an example of how this taxonomy may be applied to AI-generated video, Figure~\ref{fig:sora} presents an example of an anatomical implausibility that we never saw in diffusion model-generated images because it involves a temporal inconsistency. Future research on the realism of AI-generated audio and video may also consider following the three-step process involved in building this taxonomy for images generated by diffusion models. Based on first surveying AI literacy resources, academic literature, and social media, second generating media with state-of-the-art models, and third collecting empirical data on the human ability to distinguish AI-generated media from authentically recorded media, researchers can build empirical insights towards characterizing realism and categorizing the artifacts in AI-generated media.  
 
The empirical insights on the photorealism of AI-generated images and the resulting taxonomy designed to help people better navigate real and synthetic images online lead to a practical research question: How can AI literacy interventions improve people's ability to distinguish real photographs and AI-generated images? Future research may address this question via randomized experiments comparing a control group with no intervention to a treatment group that receives training based on the taxonomy presented in this paper. Likewise, future research may explore this with just-in-time interventions to direct people's attention to the cues identified in the taxonomy.
\section{Conclusion and Suggestions}

Our work, including the creation of \texttt{ScholarLens} and the proposal of \texttt{LLMetrica}, provides methods for assessing LLM penetration in scholarly writing and peer review. By incorporating diverse data types and a range of evaluation techniques, we consistently observe the growing influence of LLMs across various scholarly processes, raising concerns about the credibility of academic research. As LLMs become more integrated into scholarly workflows, it is crucial to establish strategies that ensure their responsible and ethical use, addressing both content creation and the peer review process. 

Despite existing guidelines restricting LLM-generated content in scholarly writing and peer review,\footnote{\href{https://aclrollingreview.org/acguidelines\#-task-3-checking-review-quality-and-chasing-missing-reviewers}{Area Chair} \&  \href{https://aclrollingreview.org/reviewerguidelines\#q-can-i-use-generative-ai}{Reviewer} \& \href{https://www.aclweb.org/adminwiki/index.php/ACL_Policy_on_Publication_Ethics\#Guidelines_for_Generative_Assistance_in_Authorship}{Author} guidelines.} challenges still remain. 
To address these, we propose the following based on our work and findings: 
(i) \textbf{Increase transparency in LLM usage within scholarly processes} by incorporating LLM assistance into review checklists, encouraging explicit acknowledgment of LLM support in paper acknowledgments, and 
reporting LLM usage patterns across diverse demographic groups;
% reporting LLM penetration based on social demographic features;
(ii) \textbf{Adopt policies to prevent irresponsible LLM reviewers} by establishing feedback channels for authors on LLM-generated reviews and developing fine-grained LLM detection models~\cite{abassy-etal-2024-llm, cheng2024beyond, artemova2025beemobenchmarkexperteditedmachinegenerated} to distinguish acceptable LLM roles (e.g., language improvement vs. content creation);
(iii) \textbf{Promote data-driven research in scholarly processes} by supporting the collection of review data for further robust analysis~\cite{dycke-etal-2022-yes}.\footnote{\url{https://arr-data.aclweb.org/}}

% make LLM usage transparent in scholarly processes: such as incorporating LLM usage into review checklists, encouraging explicit acknowledgment of LLM assistance in paper acknowledgments, and reporting LLM penetration based on social demographic features; (ii) Adopt policies to prevent irresponsible LLM reviewers: such as providing authors feedback on LLM-assisted reviews, and developing fine-grained LLM detection models~\cite{cheng2024beyond} to distinguish acceptable LLM roles (e.g., language improvement vs. content creation); (iii) Encourage data-driven research in scholarly processes: such as supporting review data collection for further research.

 



\bibliographystyle{ACM-Reference-Format}
\bibliography{refs}


\end{document}
\endinput
%%
%% End of file `sample-sigconf.tex'.
