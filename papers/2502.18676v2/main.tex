%%
%% This is file `sample-sigconf.tex',
%% generated with the docstrip utility.
%%
%% The original source files were:
%%
%% samples.dtx  (with options: `all,proceedings,bibtex,sigconf')
%% 
%% IMPORTANT NOTICE:
%% 
%% For the copyright see the source file.
%% 
%% Any modified versions of this file must be renamed
%% with new filenames distinct from sample-sigconf.tex.
%% 
%% For distribution of the original source see the terms
%% for copying and modification in the file samples.dtx.
%% 
%% This generated file may be distributed as long as the
%% original source files, as listed above, are part of the
%% same distribution. (The sources need not necessarily be
%% in the same archive or directory.)
%%
%%
%% Commands for TeXCount
%TC:macro \cite [option:text,text]
%TC:macro \citep [option:text,text]
%TC:macro \citet [option:text,text]
%TC:envir table 0 1
%TC:envir table* 0 1
%TC:envir tabular [ignore] word
%TC:envir displaymath 0 word
%TC:envir math 0 word
%TC:envir comment 0 0
%%
%% The first command in your LaTeX source must be the \documentclass
%% command.
%%
%% For submission and review of your manuscript please change the
%% command to \documentclass[manuscript, screen, review]{acmart}.
%%
%% When submitting camera ready or to TAPS, please change the command
%% to \documentclass[sigconf]{acmart} or whichever template is required
%% for your publication.
%%
%%
\documentclass[sigconf, screen]{acmart}
%%
%% \BibTeX command to typeset BibTeX logo in the docs
\AtBeginDocument{%
  \providecommand\BibTeX{{%
    Bib\TeX}}}

%% Rights management information.  This information is sent to you
%% when you complete the rights form.  These commands have SAMPLE
%% values in them; it is your responsibility as an author to replace
%% the commands and values with those provided to you when you
%% complete the rights form.
\setcopyright{acmlicensed}
\copyrightyear{2025}
\acmYear{2025}
\acmDOI{XXXXXXX.XXXXXXX}
%% These commands are for a PROCEEDINGS abstract or paper.
\acmConference[Conference acronym 'XX]{Make sure to enter the correct
  conference title from your rights confirmation email}{June 03--05,
  2025}{Woodstock, NY}
%%
%%  Uncomment \acmBooktitle if the title of the proceedings is different
%%  from ``Proceedings of ...''!
%%
%%\acmBooktitle{Woodstock '18: ACM Symposium on Neural Gaze Detection,
%%  June 03--05, 2025, Woodstock, NY}
\acmISBN{978-1-4503-XXXX-X/2025/02}


%%
%% Submission ID.
%% Use this when submitting an article to a sponsored event. You'll
%% receive a unique submission ID from the organizers
%% of the event, and this ID should be used as the parameter to this command.
%%\acmSubmissionID{123-A56-BU3}

%%
%% For managing citations, it is recommended to use bibliography
%% files in BibTeX format.
%%
%% You can then either use BibTeX with the ACM-Reference-Format style,
%% or BibLaTeX with the acmnumeric or acmauthoryear sytles, that include
%% support for advanced citation of software artefact from the
%% biblatex-software package, also separately available on CTAN.
%%
%% Look at the sample-*-biblatex.tex files for templates showcasing
%% the biblatex styles.
%%

%%
%% The majority of ACM publications use numbered citations and
%% references.  The command \citestyle{authoryear} switches to the
%% "author year" style.
%%
%% If you are preparing content for an event
%% sponsored by ACM SIGGRAPH, you must use the "author year" style of
%% citations and references.
%% Uncommenting
%% the next command will enable that style.
%%\citestyle{acmauthoryear}


%%
%% end of the preamble, start of the body of the document source.
\begin{document}

%%
%% The "title" command has an optional parameter,
%% allowing the author to define a "short title" to be used in page headers.
\title{Interacting with \textit{Thoughtful} AI}

%%
%% The "author" command and its associated commands are used to define
%% the authors and their affiliations.
%% Of note is the shared affiliation of the first two authors, and the
%% "authornote" and "authornotemark" commands
%% used to denote shared contribution to the research.

\author{Xingyu Bruce Liu}
\affiliation{%
  \institution{University of California, Los Angeles}
  \city{Los Angeles}
  \state{CA}
  \country{USA}}
 \email{xingyuliu@ucla.edu}

 \author{Haijun Xia}
\affiliation{%
  \institution{University of California, San Diego}
  \city{La Jolla}
  \state{CA}
  \country{USA}}
 \email{haijunxia@ucsd.edu}

\author{Xiang `Anthony' Chen}
\affiliation{%
  \institution{University of California, Los Angeles}
  \city{Los Angeles}
  \state{California}
  \country{USA}}
\email{xac@ucla.edu}

%%
%% By default, the full list of authors will be used in the page
%% headers. Often, this list is too long, and will overlap
%% other information printed in the page headers. This command allows
%% the author to define a more concise list
%% of authors' names for this purpose.
\renewcommand{\shortauthors}{Liu et al.}

%%
%% The abstract is a short summary of the work to be presented in the
%% article.
\begin{abstract}
Humor is a social binding agent. It is an act of creativity that can provoke emotional reactions on a broad range of topics. Humor has long been thought to be “too human” for AI to generate. However, humans are complex, and humor requires our complex set of skills: cognitive reasoning, social understanding, a broad base of knowledge, creative thinking, and audience understanding. We explore whether giving AI such skills enables it to write humor. We target one audience: Gen Z humor fans. We ask people to rate meme caption humor from three sources: highly upvoted human captions, 2) basic LLMs, and 3) LLMs captions with humor skills. We find that users like LLMs captions with humor skills more than basic LLMs and almost on par with top-rated humor written by people. We discuss how giving AI human-like skills can help it generate communication that resonates with people. 

\end{abstract}

%%
%% The code below is generated by the tool at http://dl.acm.org/ccs.cfm.
%% Please copy and paste the code instead of the example below.
%%
\begin{CCSXML}
<ccs2012>
   <concept>
       <concept_id>10003120.10003121.10003126</concept_id>
       <concept_desc>Human-centered computing~HCI theory, concepts and models</concept_desc>
       <concept_significance>500</concept_significance>
       </concept>
   <concept>
       <concept_id>10003120.10003121.10003129</concept_id>
       <concept_desc>Human-centered computing~Interactive systems and tools</concept_desc>
       <concept_significance>500</concept_significance>
       </concept>
   <concept>
       <concept_id>10010147.10010178.10010216</concept_id>
       <concept_desc>Computing methodologies~Philosophical/theoretical foundations of artificial intelligence</concept_desc>
       <concept_significance>500</concept_significance>
       </concept>
 </ccs2012>
\end{CCSXML}

\ccsdesc[500]{Human-centered computing~HCI theory, concepts and models}
\ccsdesc[500]{Human-centered computing~Interactive systems and tools}


%%
%% Keywords. The author(s) should pick words that accurately describe
%% the work being presented. Separate the keywords with commas.
\keywords{Thoughtful AI, Large Language Model, Agent, Human-AI Interaction, Thought, Theory, Interaction Paradigm}
%% A "teaser" image appears between the author and affiliation
%% information and the body of the document, and typically spans the
%% page.
\begin{teaserfigure}
\end{teaserfigure}

% \received{20 February 2007}
% \received[revised]{12 March 2009}
% \received[accepted]{5 June 2009}

%%
%% This command processes the author and affiliation and title
%% information and builds the first part of the formatted document.
\maketitle

%!TEX root=main.tex

\section{Introduction}
% Decision-makers, analysts, data scientists, and policymakers frequently rely on data to draw conclusions and extract insights. This data-driven approach helps them identify actionable recommendations aimed at influencing an outcome of interest, such as increasing product satisfaction or income levels or decreasing the likelihood of experiencing serious health conditions \cite{galhotra2022hyper,lakkaraju2016interpretable,agrawal1994fast}. 
\revc{Prescriptions, or actionable recommendations, are commonly generated across various fields to influence key outcomes such as improving product satisfaction, enhancing economic policies, or increasing business efficiency. 
%Decision- or policy-makers, analysts, data scientists, and 
Policymakers in government, decision-makers in businesses, and data scientists in various fields, often rely on data-driven approaches to identify 
%actionable recommendations 
potential actions to influence an outcome of interest, such as increasing income levels or loan approval rates}.
% , or decreasing the likelihood of experiencing serious health conditions. 
%
While association or prediction-based methods are extensively used in practice to draw useful insights from data, they typically identify correlations among variables and may fail to reveal the underlying causal factors, i.e., which actions may result in an improved outcome, needed for informed decision-making. 
%For recommendations to be truly impactful, there must be a clear  explanation that justifies why a particular decision is appropriate for a specific subpopulation~\cite{sun2021treatment,plecko2022causal}. 

\emph{Causal analysis} or {\em causal inference}, therefore, is considered one of the most important requirements to generate prescriptions that are {\em actionable} and aligned with human reasoning~\cite{imbens2024causal}. Causal inference, and in particular {\em observational studies} for causal inference on collected data (when controlled trials are impossible due to cost or ethical reasons), have been extensively studied in the statistics and artificial intelligence (AI) literature for several decades \cite{rubin2005causal, pearl2009causal}. Motivated by this foundational work on causal inference, the notion of causality has also influenced the field of database research. The causal models from AI have been extended to relational databases \cite{salimi2020causal},  and causality has been incorporated into various data management tasks such as finding responsibilities of inputs toward query answers ~\cite{meliou2010causality, meliou2009so, meliou2014causality}, explanations for query answers \cite{roy2014formal, DBLP:journals/pacmmod/YoungmannCGR24}, data discovery~\cite{galhotra2023metam,youngmann2023causal}, data cleaning~\cite{pirhadi2024otclean,salimi2019interventional}, hypothetical reasoning \cite{galhotra2022causal}, and large system diagnostics~\cite{markakis2024sawmill,causalsim,sage, gudmundsdottir2017demonstration}. 


\revc{If-then rules are generally considered interpretable by humans~\cite{lakkaraju2016interpretable,guidotti2018local,van2021evaluating,pradhan2022interpretable,chen2018optimization}.
We give a concrete example of the difference between association and causation in generating prescriptions or recommended actions in the form of if-then rules below}:
\begin{example}	%
\label{example:ex1} {\bf Importance of causal prescriptions:}
Consider the Stack Overflow (SO) annual developer survey
\cite{stackoverflowreport}, where respondents from around the world answer
questions about their jobs and demographics. A sample of the dataset \reva{with a subset of the
attributes (there are 20 attributes)} is presented in \cref{tab:data}.
%
Alice, a researcher in the United Nations (UN) finance department, is interested in discovering ways to increase the salaries of high-tech employees worldwide. She is looking for a set of actionable recommendations 
%(that we call a prescription rules) 
to raise the overall average salary.
%
Using association-based approaches~\cite{chen2018optimization,lakkaraju2016interpretable}, she may discover that individuals residing in the US who identify as straight or heterosexual tend to earn higher salaries (see \cref{exp:quality} for full details). However, this observation merely indicates a correlation: people living in the US, for example, generally earn more than those outside the country. Their comparatively higher salaries are primarily attributable to the country's economy and are unrelated to their sexual orientation. Thus, this observation cannot be used as a prescription rule to increase salary. 
Our causal analysis, on the other hand, reveals that individuals aged 25-34 with dependents would benefit from working as front-end developers.
This results in a \$44,009 annual salary increase on average. \reva{Another observation is that students should pursue an
undergraduate major in CS. %Computer Science (CS). 
This can boost their salary by \$22,174 per year} (see details in \cref{sec:casestudy}).
\end{example}

%It has been incorporated into various tasks including . 
%Causal interventions are often more relatable and easier to understand, as they offer insight into the underlying reasons behind the recommendations and allow unraveling complex cause-effect relationships that govern our world~\cite{pearl2009causality}. Furthermore, causal interventions often have long-lasting effects~\cite{imbens2024causal}.

%, making it essential that the prescribed actions are not only actionable but also 

%causally consistent. 

%Decision makings, in particular, high-stak

\cut{
In this work, {we study the problem of generating causal insights (referred to as \emph{prescription rules}), which serve as actionable recommendations} to improve an outcome of interest.
Recent works have introduced causality to the field of database research~\cite{meliou2010causality,  meliou2014causality,salimi2020causal,10.14778/3554821.3554902}. It has been incorporated into various tasks including data discovery~\cite{galhotra2023metam,youngmann2023causal}, data cleaning~\cite{pirhadi2024otclean,salimi2019interventional}, and large system diagnostics~\cite{markakis2024sawmill,causalsim,sage, gudmundsdottir2017demonstration}. 
We propose using causal inference to generate prescription rules that are both actionable and justifiable.
}

While generating prescriptions based on causal inference may help in robust decision-making, causal prescriptions that solely consider the betterment of an outcome (like salary) are not enough in practice. 
It is well-known that decision-making in many high-stake applications (like hiring policy, or policy for approving loans by banks) may lead to disparate societal or economic impact on different sub-populations. 
As a shocking example from a recent work called 
%For example, 
CauSumX~\cite{DBLP:journals/pacmmod/YoungmannCGR24} that generates a set of causal explanations for an aggregated view, the explanations generated %by CauSumX %recommendations which 
suggest that male individuals do a Bachelor's degree to increase their salary while %suggesting that 
being an unmarried woman 
%the recommendation for women includes getting married 
has the most adverse effect on salary
(borrowed directly 
from Fig.~19 in~\cite{youngmann2024summarizedcausalexplanationsaggregate}). 
%We demonstrate the advantage of using causal reasoning to generate actionable recommendations and the limitations of not considering fairness requirements in the following example. 
We explored this further in the context of generating prescriptions and observed that prescriptions that are not fairness-aware can generate unfair outcomes to some subpopulations which we refer to as the {\em protected group}. Examples include women, Black, Latino, or Native Americans, individuals with a disability, countries with a weaker economy, or other protected groups specific to an application. %Here is a concrete example:


% Understanding the causal factors behind these recommendations is crucial to ensuring that decisions lead to fair and equitable outcomes, particularly in sensitive applications where biased decisions can perpetuate or even exacerbate societal inequalities.
% While prior work has extensively explored techniques for association rule mining~\cite{kumbhare2014overview}, recent efforts have focused on deriving causal explanations for individual data points or entire datasets~\cite{salimi2018bias,youngmann2022explaining,ma2023xinsight}. Although some of these methods produce causally consistent insights, the absence of fairness considerations in the process can lead to unfair outcomes, further reinforcing existing biases. For example, CauSumX~\cite{DBLP:journals/pacmmod/YoungmannCGR24} generates causal recommendation which suggest male individuals to do a Bachelor's degree to increase salary while the recommendation for women include getting married (borrowed directly from Figure~19 in the paper~\cite{youngmann2024summarizedcausalexplanationsaggregate}). 





%\emph{Causal inference} has been thoroughly studied in AI and Statistics~\cite{pearl2009causal,rubin2005causal}. Causal analysis is a vital tool in determining the effect of a \emph{treatment} on an \emph{outcome}, and has been used in decision-making in medicine \cite{robins2000marginal}, economics \cite{banerjee2011poor}, biology \cite{shipley2016cause}, and in high-stakes areas such as identifying the root causes of failures in critical infrastructure systems to prevent catastrophic outcomes. Recent works have introduced causality to the field of database research~\cite{meliou2010causality,  meliou2014causality,salimi2020causal,10.14778/3554821.3554902}. It has been incorporated into various tasks including data discovery~\cite{galhotra2023metam,youngmann2023causal}, query result explanation~\cite{salimi2018bias,youngmann2022explaining,DBLP:journals/pacmmod/YoungmannCGR24}, and large system diagnostics~\cite{markakis2024sawmill,causalsim,sage, gudmundsdottir2017demonstration}. We propose leveraging causal inference to generate interpretable and justifiable insights (referred to as \emph{prescription rules}), which serve as actionable recommendations to improve an outcome of interest. Causal reasoning is considered one of the most important requirements,  to generate insights that are actionable and aligned with human reasoning.




\begin{table*}[]
\footnotesize
    \centering
    	\caption{\textnormal{A subset of the Stack Overflow dataset.}}
         \label{tab:data}
    	% \vspace{-4mm}
  			\begin{tabular}[b]{|l|l|l|c|l|l|c|l|c|}
  			
				%\multicolumn{9}{c}{\textbf{Users}}\\ 
				\hline

				\textbf{ID}
    
    % \textbf{Country}& \textbf{Continent} 
    
    &\textbf{Gender} &\textbf{Ethnicity}&
				\textbf{Age} &\textbf{Role} &
				\textbf{Education} &\textbf{Country}&\textbf{Undergrad Major}&\textbf{Salary}
				\\ \hline

				1 &Male&White&26&Data Scientist & PhD& US&Computer Science&180k\\
    		2 &Non-binary&White&32&QA developer & Bachelor's degree& US&Mechanical Eng.&83k\\

 3 &Male&South Asian&29&C-suite executive  & Bachelor's degree & India&Computer Science&24k\\

  % 4 &Female&South Asian&25&Back-end developer  & Master's degree & India&Mathematics&7.5k\\

  4 &Female&East Asian&21&Back-end developer & Bachelor's degree & China&Computer Science&19k\\
  

        % $\ldots$ &  $\ldots$&  $\ldots$&  $\ldots$&  $\ldots$&  $\ldots$&  $\ldots$&  $\ldots$&  $\ldots$&  $\ldots$&  $\ldots$\\
    \hline
			\end{tabular}
            \vspace{-5mm}
\end{table*}




\begin{example}	%
\label{example:ex2}
{\bf Importance of fair prescriptions:}
Continuing Example~\ref{example:ex1}, while those causal prescription rules are highly beneficial for the overall population, they are considerably less effective for individuals residing in countries with a low GDP (indicating a weaker economy). For this group, the average expected increase in salary is only approximately \$13,000 per year (in contrast to \$44,009 for the entire group). % \sr{add which rule 44k or 25k} 
Consequently, implementing these rules would exacerbate the disparity between those living in countries with strong economies and those in countries with weaker economies.
\end{example}




% Our objective is to generate a small set of prescription rules aimed at increasing (or decreasing) an outcome of interest. This is framed as an optimization problem where the goal is to select the fewest prescription rules that maximize utility (i.e., the expected increase or decrease in the outcome). However, 

The example above shows that focusing solely on maximizing utility (\revc{i.e., increasing income}) can result in a scenario where only some of the population receive significant improvement, while others experience no benefit (\revc{only a small benefit for individuals from countries with weaker economies in our example}). Additionally, even if a large portion of the population receives recommendations, a protected subpopulation might not share the benefits and, worse, their situation could deteriorate, exacerbating inequalities.

Examples~\ref{example:ex1} and \ref{example:ex2} show that it is crucial to provide recommendations that are (1) {\em causal} for the outcome (beyond associations),  and (2) also {\em fair or equitable} in terms of the outcome for both the protected and non-protected groups. While recent work in database research
has focused on deriving {\em causal explanations} for individual data points, aggregated view, or entire datasets~\cite{salimi2018bias,youngmann2022explaining,ma2023xinsight, DBLP:journals/pacmmod/YoungmannCGR24}, and in particular \cite{DBLP:journals/pacmmod/YoungmannCGR24} has considered generating a set of causal explanations for an aggregated view that resemble a ruleset, 
%Although some of these methods produce causally consistent insights, 
the absence of fairness considerations in generating these causal explanations can lead to unfair outcomes for the protected group.
%further reinforcing existing biases.


%\red{We, therefore, enable users to incorporate various \emph{coverage and fairness constraints} along with the overall objective of improving an outcome of interest. }

\medskip
\noindent
\textbf{Our contributions.~} 
Motivated by the dual goals of generating causal and fair prescriptions for the betterment of an outcome, we introduce a {\em fairness-aware framework leveraging causal reasoning for generating a set of actionable prescription rules (ruleset)} called \sysName\ (\underline{Fair} \underline{CA}usal \underline{P}rescription).
%
Following research on fairness in data management~\cite{stoyanovich2020responsible,galhotra2022causal}, we assume the existence of a \emph{protected subpopulation}, defined by an attribute such as gender or race for people, or GDP of a country. Motivated by the causal explanation rules for an aggregated view \cite{DBLP:journals/pacmmod/YoungmannCGR24}, each prescription rule in our ruleset applies to a sub-population defined by a {\em grouping attribute}, and prescribes a {\em treatment or intervention} to improve the {\em outcome} for this sub-population. Fairness constraints ensure that the expected utility of the protected population is {\em comparable} to the utility of the unprotected individuals. We borrow the notions of \emph{group and individual fairness} from the fairness literature but tailor them for prescription rules. In addition to the fairness constraints, our coverage constraints ensure that a substantial fraction of the population and protected subpopulation receives at least one recommendation. 
%We demonstrate how such constraints ensure that the generated rules apply to a large portion of the population and ensure fairness through the following example.

\begin{example}
\label{ex:intro_example_3}
Continuing Examples~\ref{example:ex1} and \ref{example:ex2}, Alice uses our proposed system, called \sysName, to impose fairness and coverage constraints to discover useful and equitable recommendations for increasing salaries worldwide. In particular,
Alice chooses to implement a coverage constraint to ensure that the selected rules apply to a significant portion of people worldwide, including a sufficiently large number of individuals from countries with low GDP (the protected group). She also imposes a fairness constraint to ensure that the expected gains for both protected and non-protected groups are comparable.
\reva{She discovers, for example, that for individuals with 6-8 years of coding experience (a subpopulation comprising 21\% of the entire dataset and 25\% of the protected group), pursuing a bachelor’s degree in computer science will increase the expected salary by $\$14.9k$ for protected and by $\$17.8k$ for non-protected}. (See \cref{sec:casestudy} for more details.) This prescription rule applies to a large portion of the population and ensures fairness by providing a similar expected gain for both protected and non-protected groups, and the allowed difference of outcomes between these two populations may be adjusted by choosing appropriate thresholds in the fairness definitions. 
\end{example}


\noindent
Our main contributions are as follows. \\
%\begin{itemize}[leftmargin=*,topsep=0pt]
{\bf (1)} We {\bf develop a framework that generates a set of prescription rules to enhance an outcome of interest (Section~\ref{sec:problem})}. A prescription rule consists of a \emph{grouping pattern} and an \emph{intervention pattern}, representing the target subpopulation and the actionable recommendation for that group, respectively. The strength of the {\em conditional causal effect} (Section~\ref{sec:background-causal}) of this intervention on the subgroup is used to measure the expected utility of a rule. Our objective is to identify the smallest set of rules that maximizes overall expected utility. We refer to this problem as the {\em \probName} problem.
We adopt several notions of fairness (individual vs. group, statistical parity vs. bounded group loss) from the literature to define the {\bf fairness constraints} for our problem. In addition, {\bf coverage constraints} (for individual rules or for a group) ensure that the solution for the \probName\ problem is applied to a sufficient number of individuals and to minimize inequalities. We show NP-hardness for different variants of the problems and properties (matroid) useful in our algorithms. 
%We establish several definitions for group and individual fairness constraints tailored for prescription rules.
\smallskip
    \par
    \noindent
{\bf (2)} We {\bf develop a general three-step algorithm named \sysName to solve the optimization problem of selecting a fair prescription ruleset (Section~\ref{sec:algo})}. The first step involves mining frequent grouping patterns using the Apriori algorithm~\cite{agrawal1994fast}. In the second step, we employ a lattice-based algorithm to find high utility and fair intervention patterns for grouping patterns identified in the previous step. Finally, the third step applies a greedy approach to determine a solution. \sysName\ can be easily adapted to accommodate all variants of the \probName\ problem.

\smallskip
\par
\noindent
{\bf (3) We provide a detailed  case study  (Section~\ref{sec:casestudy}) and experimental analysis (Section~\ref{sec:experiments}) to evaluate our framework and algorithms.}
The case study shows the qualitative difference of different variants of our problem for different choices of the fairness and coverage constraints. The experiments include two datasets, three baselines, and 18 variations of our problem with different constraints. Our evaluations suggest that fairness may come at the cost of expected
utility for everyone. However, without fairness constraints, we often observe a significant disparity between the protected and non-protected. We also observe that
achieving individual fairness is harder than group fairness,
as most high-utility or high-coverage rules are unfair. Lastly, we show that \sysName\ can generate  prescription rules over large datasets in a reasonable time. 

%\end{itemize}


%\paragraph*{Paper outline} 
We discuss related work in \cref{sec:related}, review background on causal inference in \Cref{sec:background-causal}, %and our problem formulation can be found in \cref{sec:problem}. Our algorithmic framework is presented in \cref{sec:algo}. A case study demonstrating the impact of different constraint configurations on the solution is given in \cref{exp:problem_variants}, and our experimental evaluation is detailed in \cref{sec:experiments}. Finally, we 
and discuss the limitations of our framework and future work in \cref{sec:conc}.

% \noindent
% \boxed{\parbox{\columnwidth}{$\bullet$ 
% For people with a professional degree, move to the United Kingdom
%  (coverage = 435 (20), coverage-protected = 20 (13), utility = 186855, utility-protected = 0.)\\
% $\bullet$ For graphic developers, move to the	United States
%  (coverage = 116 (29), coverage-protected = 8 (2), utility = 169431, utility-protected = 0).\\
% $\bullet$ For people who have no formal education, move to the United States
%  (coverage = 123 (34), coverage-protected = 7 (2), utility = 206742, utility-protected = 0).\\
% % \textcolor{red}{size = 38, length = 76, overlap = 64029181, utility = 1659307}\\
% \textcolor{blue}{overall coverage =674, expected utility = 187485
% coverage-protected = 35, expected utility-protected = 0}
% \sr{should mention protected group, and possibly not mention coverage in the intro or just intuitively like high coverage}
% }}


% Alice notes that although these rules result in a \$187,485 increase in the overall salary for those to whom they apply, they only affect a small fraction of the population, specifically 674 individuals. Additionally, although the expected salary increase is substantial, there is no expected increase in salary for non-males, a subpopulation of particular interest to Alice. In other words, applying these rules would result in no gain for non-males.
% \end{example}

% \begin{example}[Episode 2 - coverage and fairness constraints]
% Alice introduces coverage and fairness constraints to ensure that enough people will benefit from the rules and that they will be \emph{fair} with respect to non-males. Specifically, she demands that the benefit for a randomly chosen individual to whom one of the rules applies is nearly the same as the benefit for a randomly chosen individual who identifies as non-male and to whom one of the rules applies.

% After adding these constraints, \sysName\ recommends the following set of prescription rules:



% \noindent
% \boxed{\parbox{\columnwidth}{$\bullet$ 
% For people who have no formal education, move to the United States
%  (coverage = 123 (34), coverage-protected = 7 (2), utility = 206742, utility-protected = 0)\\
% $\bullet$ 
% For females, change role to	DevOps specialist (coverage = 2256 (47), coverage-protected = 2256 (47), utility = 90023, utility-protected = 90023).\\
% $\bullet$ For people with a Master's degree, move to the	United States
%  (coverage = 9097 (2222), coverage-protected = 642 (236), utility = 85390, utility-protected = 84201).\\
% % \textcolor{red}{size = 38, length = 76, overlap = 64029181, utility = 1659307}\\
% \textcolor{blue}{overall coverage =11476	
% , expected utility = 87601,
% coverage-protected = 2905, expected utility-protected = 88519}
% }} 







% \begin{figure}[t]
%         \centering
%         \begin{minipage}[b]{1.0\linewidth}
%             \small
%             \begin{tcolorbox}[colback=white]
%             \vspace{-2mm}
% $\bullet$ For backend developers, the treatment with the highest effect on salary is “Country = US” effect size = 78646
% \begin{itemize}
%     \item For non-male the effect is only: 59429
%     \item For male the effect is 80454
% \end{itemize}

% $\bullet$ For frontend developers, the treatment with the highest effect is :Formal Education = Bachelor's degree” effect size: 17340
% \begin{itemize}
%     \item For white the effect is 33464
%     \item For non-white the effect is 15320
% \end{itemize}


% $\bullet$ For people in Europe, the treatment with the highest effect on salary is “DevType = C-suite executive” effect size = 53254
% \begin{itemize}
%     \item For white the effect is 55112
%     \item For non-white 35249
% \end{itemize}



%             \vspace{-2mm}
%             \end{tcolorbox}
%         \end{minipage}%%
%          % \vspace{-4mm}
%         \caption{Set of prescription rules.}
%         \label{fig:so-explanation}
%     \end{figure}

\input{02-literature}
\section{What makes AI ``Thoughtful''?}


Discussions around ``thought'' in LLM research often focus on a chain-of-thought approach, where a model generates intermediate reasoning steps to improve performance on reasoning tasks. From an HCI perspective, however, \textit{thought} can be understood more broadly. We define \textit{Thoughtfulness} as:

\newtheorem{thoughtfuldefinition}{Definition}
\begin{definition}
\textbf{Thoughtfulness} refers to a system's ability to continuously generate, develop, and selectively communicate its \textit{intermediate processes and responses} over the course of an interaction. 
\end{definition}


In contrast to LLM-based definitions, our conception of Thoughtful AI emphasizes how these ongoing processes shape the system’s behavior and its capacity to interact with human. Thoughts can emerge at any point in an interaction, triggered by external stimuli or intrinsic reflection, may be expressed or remain internal, and they can take various forms, from abstract keywords to visual or auditory representations.


To better understand how Thoughtful AI differs from conventional systems, we examine four key traits of this interaction paradigm (\autoref{tab:02-traits}), comparing them with the turn-based model to highlight the key distinctions.


\subsection{Traits of Thoughtful AI}

% Please add the following required packages to your document preamble:
% \usepackage{booktabs}
% \usepackage{graphicx}
% \usepackage[normalem]{ulem}
% \useunder{\uline}{\ul}{}

\begin{table}
  \resizebox{\columnwidth}{!}{%
    \begin{tabular}{@{}lll@{}}
      \toprule
      \textbf{Trait} & \textbf{Current AI} & \textbf{Thoughtful AI} \\ 
      \midrule
      \textit{Intermediate Medium} & 
        \begin{tabular}[c]{@{}l@{}}
          Reveals only\\
          final outputs
        \end{tabular} & 
        \begin{tabular}[c]{@{}l@{}}
          Surfaces intermediate\\
          thoughts in real time
        \end{tabular} \\
      \addlinespace
      \textit{Full-Duplex Process} & 
        \begin{tabular}[c]{@{}l@{}}
          Waits for user\\
          prompts
        \end{tabular} & 
        \begin{tabular}[c]{@{}l@{}}
          Continuously thinks in\\
          parallel with user activity
        \end{tabular} \\
      \addlinespace
      \textit{Intrinsic Driver} & 
        \begin{tabular}[c]{@{}l@{}}
          Responds only\\
          when asked
        \end{tabular} & 
        \begin{tabular}[c]{@{}l@{}}
          Self-initiates actions\\
          based on thoughts
        \end{tabular} \\
      \addlinespace
      \textit{Shared Cognitive Space} & 
        \begin{tabular}[c]{@{}l@{}}
          Turn-based\\
          exchanges
        \end{tabular} & 
        \begin{tabular}[c]{@{}l@{}}
          Builds on collaborative\\
          fragmented ideas
        \end{tabular} \\
      \bottomrule
    \end{tabular}%
  }
  \caption{Comparison of Thoughtful AI and Current AI across key traits.}
  \label{tab:02-traits}
\end{table}

\subsubsection{Intermediate Medium}
A fundamental distinction between a \textit{thought} and a \textit{response} is that a thought is \textit{in-progress}, potentially fragmented, and not necessarily intended for final output. In Thoughtful AI, the system’s processing, whether it involves brainstorming ideas, weighing pros and cons, or exploring parallel strategies, can be treated as an \textit{intermediate medium} that the AI may optionally share.

In most existing AI systems, users see only a final answer. This one-shot approach is limiting: once the user receives an answer, any clarifications or follow-up requests often require multiple back-and-forth queries. If the AI’s thinking was flawed or incomplete, the user has no window into how or why the system arrived at its conclusion.


\subsubsection{Full-duplex Process}
Human conversation does not pause for one party to ``finish thinking'', nor do humans stop thinking when others start to speak. Similarly, Thoughtful AI enables a \textit{full-duplex process} where thinking is continuous and can occur any point in the interaction, rather than locked into turn-based cycles. AI thoughts may be triggered by user inputs or by the system’s own internal reflections; they can also evolve without any external stimuli, for example, during periods of user silence.

Conventional AI interactions are typically half-duplex. After producing an answer, the AI stops listening or processing until another user query arrives. It remains idle, unaware of changing contexts and does not develop anything internally during that downtime.


\subsubsection{Intrinsic Driver}
One of the key aspects of human intelligence is that thought is not merely reactive: it also serves as an \textit{intrinsic driver} of action. Similarly, in Thoughtful AI, thinking can serve as a mechanism that enables the system to self-initiate its interactions and be \textit{proactive}.
Rather than being solely triggered by user input, Thoughtful AI can independently generate ideas, monitor evolving contexts, and identify opportunities to intervene or contribute.

Traditional turn-based systems are \emph{reactive}: they respond only once the user issues a query. Even so-called ``proactive'' assistants typically rely on predefined triggers or simple rule-based heuristics. In contrast, a Thoughtful AI continuously \emph{thinks} in the background, allowing it to model its intrinsic motivation to take actions based on the intermediate thoughts. Its continuous thought process is the primary engine for proactive behavior.

\subsubsection{Shared Cognitive Space}

Finally, a \textit{shared cognitive space} could emerge when AI and user co-exist in an ongoing thought process. Rather than the AI presenting a single answer and the user responding with a single set of followups, both parties iteratively build upon each other’s partial ideas. We envision that the AI’s intermediate thoughts, user feedback, clarifying questions, and real-time refinements etc form a collaborative ``thinking canvas.''

Traditional interactions are linear and unidirectional: the AI provides an answer, the user reacts, and so on. Even if the user can add further prompts, there is no persistent collaborative workspace where partial ideas accumulate and evolve. 


\subsection{Implications for HCI}
The four traits of Thoughtful AI---an \textit{intermediate medium}, \textit{full-duplex process}, \textit{intrinsic driver}, and a \textit{shared cognitive space}---open up new possibilities for how humans interact with AI. Below, we discuss four implications for Human-AI Interaction.

\subsubsection{From Passive Respondents to Proactive Participants}
A first implication of Thoughtful AI is the transition from AI as a passive respondent to an active participant in interactions. Conventional AI systems operate in a reactive manner, awaiting user input before generating a response. In contrast, Thoughtful AI continuously generates thoughts, enabling it to self-initiate actions and engage more dynamically with users.

This proactivity manifests in two ways. First, the \textit{full-duplex process} (Trait 2) allows the AI to generate thoughts in parallel with user input, meaning it does not need to remain idle between interactions. It can detect emerging needs, anticipate user queries, and even interrupt when necessary. Second, thoughts equip AI with an \textit{intrinsic drive} (Trait 3) to contribute based on its internal cognitive processes. This ability to self-motivate and intervene resembles human-like initiative, distinguishing it from traditional systems that rely on heuristics or predefined triggers~\cite{horvitz1999principles}.


\subsubsection{Continuous Cognitive Alignment}

A second major implication for HCI arises from Thoughtful AI's ability to reveal \emph{in-progress} ideas (Trait 1: \textit{Intermediate Medium}) continuously (Trait 2: \textit{Full-duplex Process}). Rather than presenting a single, static final output, the system selectively shares partial thoughts. Users can then not only observe these thoughts for better interpretability, but also guide the AI's thinking early in the process, leading to more dynamic collaboration.

By showing its intermediate reasoning, Thoughtful AI helps bridge the ``Gulf of Evaluation,'' where users struggle to understand how or why a system arrived at a particular result~\cite{norman2013design}. Here, the AI’s partial thoughts provide visibility into its evolving path, allowing users to catch misconceptions or supply additional context before errors compound. Similarly, users can steer the AI's next steps more effectively, shrinking the ``Gulf of Execution,'' since they can articulate what they want \emph{as} they see the AI’s tentative directions.

This also helps build common ground~\cite{clark1991grounding} between human and AI, aligning with established theories of communication that emphasize the importance of shared context for effective collaboration. By understanding the AI's intermediate thought, users gain a clearer understanding of the system's current assumptions and partial conclusions. This shared cognitive workspace drives continuous alignment between the user's thinking and AI's thinking process.


\subsubsection{Beyond Turn-based Interaction}
Thoughtful AI prompts us to reconsider the fundamental structure of human–AI interaction. Traditional chatbot interfaces operate in discrete, back-and-forth messages, with the AI effectively going idle after sending each response. By contrast, a \emph{full-duplex process} (Trait 2) enables the AI to continuously think and listen, even when the user is not actively providing input.
This opens the door to interaction models that transcend simple chat boxes. For instance, an AI planning assistant could silently update its suggestions as it overhears new constraints in a virtual meeting, intervening only when necessary (Traits 2 and 3). 
More interestingly, AI and users can build a \emph{shared cognitive space} (Trait 4), adding thoughts, annotations, and partial ideas to a collective interface. Rather than a linear log of exchanges, this collaborative canvas allows ideas to branch, merge, and evolve continuously---resembling a dynamic mind-map or sketchnote more than a turn-based chat.


\subsubsection{Messy, Fragmented and Informal Interaction}
Traditional AI systems demand users structure their inputs as complete, well-formed queries or commands. This formality creates friction: human thinking is inherently non-linear, often involving half-formed ideas and abrupt shifts in focus. 
Thoughtful AI embraces this messiness by using intermediate thoughts as the first-class citizen of interaction. 
This shift towards a more natural, fragmented, and informal interaction style: AI thinking can be messy, incomplete, and in-progress. 
In addition, instead of requiring users to provide polished, fully-formed queries or responses, users can also input partial ideas, rough thoughts, or even keywords, and the system can process and respond to these in a similarly fragmented manner. 
This approach mirrors how humans often think and communicate: by iterating on ideas, refining them over time, and bouncing off incomplete or tentative thoughts. 
\section{Example Projects}
To further illustrate the practical applications of Thoughtful AI, we present two projects that embody its core principles. The first, Inner Thoughts, explores how AI can proactively engage in conversations by developing and evaluating its internal thoughts before participating. The second, ThinkaloudLM, investigates AI-generated thoughts as an interface component that allows users to observe and interact with in real-time.

\subsection{Inner Thoughts}
\begin{figure}
    \centering
  \includegraphics[width=0.65\linewidth]{figures/03_inner_thoughts.pdf}
    \caption{Conversational Agents with Inner Thoughts: AI generates a train of thoughts and evaluates them based on their intrinsic motivation to participate.}
    \label{fig:inner_thoughts}
\end{figure}

Conversational AI often rely on turn-taking prediction techniques, where an algorithm determines the most likely next speaker and generates a response accordingly. However, these approaches struggle in multi-party conversations where turn allocation is often ambiguous. Furthermore, existing conversational agents tend to fall into two extremes: either they remain passive, requiring explicit user input to respond, or they overcompensate, generating frequent and often unnecessary interruptions.

Instead of predicting conversational turns, \textit{Inner Thoughts}~\cite{liu2025inner} introduces a new method in which AI autonomously generates a continuous stream of covert (internal) thoughts, similar to how humans process conversations. These thoughts remain internal until AI evaluates whether it has sufficient \textit{intrinsic motivation} to contribute---defined by heuristics derived from a user study on human conversational behavior. Once the AI determines that a thought is relevant and meaningful, it then strategically selects an appropriate moment to engage (\autoref{fig:inner_thoughts}).

The Inner Thoughts framework embodies multiple traits of Thoughtful AI:
\textit{Intermediate Medium}: AI does not simply generate final responses but produces an evolving series of intermediate thoughts; \textit{Full-Duplex Process}: Instead of waiting for user prompts, the AI continuously listens and generates internal responses in parallel with the ongoing conversation; \textit{Intrinsic Driver}: The AI does not just react but actively determines when to contribute based on its intrinsic motivation.

The framework was implemented in two systems, a multi-agent conversational simulation and a chatbot. Through a technical evaluation, conversational agents using Inner Thoughts significantly outperformed a traditional next-speaker prediction baseline across multiple criteria, including turn appropriateness, coherence, engagement, and adaptability.
Participants overwhelmingly preferred AI interactions using Inner Thoughts (in 82\% of the conversations).



\subsection{ThinkaloudLM}
\begin{figure}
    \centering
  \includegraphics[width=\linewidth]{figures/03_thinkaloudlm.png}
    \caption{ThinkaloudLM: AI generates intermediate, fragmented thoughts in parellel to user input.}
    \label{fig:thinkaloudlm}
\end{figure}

While Inner Thoughts explores AI's ability to think internally, ThinkaloudLM investigates a complementary question: what if users could actively observe and engage with AI's thought process?

Most AI systems present only finalized outputs, hiding the intermediate thoughts behind their decisions. This lack of transparency can lead to user distrust, inefficiencies in communication, and missed opportunities for collaboration. For example, if an AI-powered writing assistant provides a response that feels misaligned with the user’s intent, the user must iteratively refine their query, leading to unnecessary back-and-forth exchanges.

By contrast, ThinkaloudLM envisions a system where users can engage with AI-generated ideas before they fully materialize into responses. Rather than receiving a single, static reply, users might see a live preview of AI’s thinking, including key points, possible directions, or alternative solutions. They can then refine, select, or discard thoughts before finalizing the AI’s output.

This approach leverages multiple traits of Thoughtful AI:
\textit{Intermediate Medium}: The AI’s partially formed ideas are themselves the primary interface, giving users a real-time window into the AI’s thought process; \textit{Full-duplex Process}: By continuously updating these intermediate thoughts, the AI can incorporate user feedback on-the-fly while still in the ``draft'' stage; \textit{Shared cognitive space}: Users and AI co-construct the final answer, using the AI's visible thoughts as a collaborative canvas for discussion, refinement, and mutual alignment.
To explore this concept, a participatory design study is currently underway with qualitative interviews and observational studies. 
We analyze the experimental results in this section. We highlight two interesting findings: Test-time scaling in Multi-Agent system (Section \ref{sec:result_test_time_scaling}), and modality-tailored critiques enhance the self-correction ability (Section \ref{sec:result_multimodal_self_critique}). Additionally, we discuss the advantage of \model{} (Section \ref{sec:why_metal}), and the benefit of agentic design (Section \ref{sec:result_agentic_vs_modular})


\subsection{Test-Time Scaling} 
\label{sec:result_test_time_scaling}

We investigate the relationship between the test-time computational budget and model performance. As illustrated in Figure~\ref{fig:acc_by_iter}, our analysis reveals an interesting trend:  increasing the logarithm of the computational budget leads to continuous performance improvements. This near-linear relationship indicates the test-time scaling phenomenon, demonstrating that allowing more iterations during inference could potentially enhance performance.

One potential reason for this phenomenon is the strong self-improvement capability of \model{}. Our framework is designed so that specialized agents iteratively collaborate, allowing each agent to refine its output based on feedback from others. With each iteration, errors are corrected and insights from different modalities are integrated, leading to incremental performance gains. This continual refinement process leverages the strengths of individual agents, resulting in the self-improvement capability that drives the observed performance enhancements as computational resources increase. 

Due to limited resources, we have not extended the experiment range further. However, the observed scaling implies that the framework can benefit from more iterations of collaborative self-improvement. We leave a more comprehensive exploration of this potential to the future work.

% Figure~\ref{fig:acc_by_iter} reveals a near-linear relationship between performance and the logarithm of the computational budget. \kw{one question might be why don't we further increase the budget and stop at 4096 tokens.} As more computations are performed, performance improves continuously, suggesting that prolonged multi-agent collaboration could eventually achieve near-perfect results. This finding confirms that the test-time scaling law applies to the chart generation task with the multi-agent framework. \kw{I would not use "confirm" as it might be too strong. We can say we observe this trend in our experiments. }


\subsection{Modality-Tailored Critiques} \label{sec:result_multimodal_self_critique}

\begin{figure} [tbp]
    \centering
    \includegraphics[width=1\linewidth]{figs/average_improvement_by_difficulty.pdf}
    \caption{Performance gain after 5 compute recurrences of \model{} over different  difficulty.}
    \vspace{-0.2in}
    \label{fig:improv_by_diff}
\end{figure}

From the ablation study result shown in Table \ref{tbl:ablation_study}, we observed that separating visual and code critiques enhances the model's self-correction capabilities. In contrast, \model{}$_S$ struggles to effectively self-improve in the chart-to-code generation task.

We identify two potential reasons for this observation. First, combining both visual and code inputs results in an extended context that can overwhelm the model, leading to information loss. This dilution makes it difficult to capture key details from each modality, resulting in less accurate critiques and a reduction in overall self-correction effectiveness. Second, the self-critique process for chart generation involves distinct requirements: visual data demands spatial understanding, color analysis, and fine detail recognition, while code data requires strict adherence to syntax and logical consistency. A unified critique approach is ill-suited to address these differing needs. Without modality-specific feedback, the model struggles to detect and correct errors unique to each data type.

These findings suggest that self-correction in the multimodal context can be enhanced by leveraging tailored critique strategies for each modality.

\subsection{Why \model{}} 
\label{sec:why_metal}

We believe \model{} provides three advantages. 
First, by assigning specialized tasks to individual agents, the system effectively reduces error propagation. During inference, each agent evaluates whether to take action based on the available information and insights from other agents. This process enables each agent to serve as a safeguard, detecting and correcting mistakes before they escalate. 

Second, the modular design of \model{} enables easy modification and adaptation. For instance, one can integrate different base models tailored for specific tasks—such as employing a critique-trained model for critique agents and a generation-trained model for generation agents—to maximize overall performance. 

Third, \model{} is robust with the strong base model. Figure~\ref{fig:improv_by_diff} compares the performance of \model{} to that of Direct Prompting over five iterations across varying chart difficulty levels. \model{} with the \gpt base model achieved consistent improvements regardless of difficulty. When using \llama as the base model, the performance gains tend to diminish with increasing reference chart complexity, but the improvements remain substantial. This drop might be due to the limited critique capabilities of the \llama base model. Nonetheless, the flexibility of \model{} to replace the base model for different agents allows us to tailor the system optimally—using, for example, a critique-optimized model for critique agents and a generation-focused model for generation agents—to maximize overall performance.

\begin{figure*}[ht]
    \centering
    \includegraphics[width=1\linewidth]{figs/case_study.pdf}
    \caption{Case study of \model{}'s progressive refinement from initial generation to perfect. Starting from Round 0's initial generation (60\% color score , 84\% text score), the system iteratively improves the output. In Round 1, the system identifies and corrects Y-axis scale issues and missing annotations, achieving 100\% text score. Round 2 refines the color representations of distributions, achieving perfect F1 score across all metrics.}
    \label{fig:case_study}
    \vspace{-0.08in}
\end{figure*}

\subsection{Multi-Agent System vs. Modular System}
\label{sec:result_agentic_vs_modular}

We further investigate the impact of agentic behavior of \model{} on final performance. We think self-decision-making and code execution abilities are key features that distinguish the multi-agent system from a modular system. We implement a self-revision modular system without these two key abilities,  and conduct an additional ablation study on a subset of 50 data points to examine the impact of these agentic behaviors on final performance.

The results show that, compared to \model{}, there is a 4.51\% reduction in average performance gain over direct prompting. The absence of decision-making and code execution abilities in the modular system hinders its capacity to refine generated charts effectively. Specifically, the inability to execute code for chart rendering significantly diminishes the quality of the critique, and the absence of self-decision-making ability potentially leads to error propagation that further negatively impacts the self-correction process.

This comparison underscores the critical role of the agentic approach.


\section{Conclusion}\label{sec:con}

Our work contributes empirical insights on the photorealism of AI-generated images and a taxonomy of artifacts commonly found in AI-generated images, organized into five categories: anatomical implausibilities, stylistic artifacts, functional implausibilities, violations of physics, and sociocultural implausibilities. We find that the photorealism of AI-generated images depends on the scene complexity of the image, the kind of artifacts and implausibilities, if any, detectable in an image, the duration of visual attention to an image, and the quality of human effort to select appropriate prompts and curate images. A question such as ``How photorealistic are state-of-the-art diffusion models'' may sound simple, but the answer is more complex and depends on many details, including what images are generated and selected, how photorealism is measured, what real images are included in the experiment, and how much time, skill, and effort a human participant has and willing to offer. This paper offers an initial exploration into how we can address this question and develops a practical taxonomy that offers scaffolding for building AI--literacy interventions to help people navigate the capabilities and limitations of diffusion models and whether an image is AI-generated or not. 

\begin{acks}

This material is based upon work supported by Robert Pozen, and in part with funding from the Department of Defense (DoD). Any opinions, findings, conclusions, or recommendations expressed in this material are those of the authors and do not necessarily reflect the views of the DoD or any agency or entity of the United States Government. We thank Will Thompson from Kellogg Research Support for performing a replication check.
\end{acks}

\bibliographystyle{ACM-Reference-Format}
\bibliography{refs}


\end{document}
\endinput
%%
%% End of file `sample-sigconf.tex'.
