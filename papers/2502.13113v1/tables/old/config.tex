\begin{scriptsize}
    \begin{table}[h!]
\begin{scriptsize}
    
  \begin{center}
  \caption{Configurations for workloads and architecture} 
  %\Rav{There is space for remarks too if needed}}
  %\TK{@Raveesh - by x do you mean you can do either s or t for that dimension? So does that mean each row is actually multiple datapoints? Thats confusing.}\Rav{It means that the datapoint that we choose for evaluation of that dataflow can have varaible tile sizes for dimension marked by X. S and T on the other hand mean that its NECESSARILY spatial or temporal}}
  \label{table:config}
  \begin{centering}
  \begin{tabular}{|l|l|}
    \hline
    \textbf{Parameter} & \textbf{Value}  \\
    \hline
    %Seq & $V\times F$ & $t_{AGG}+t_{CMB}$\\
    %\hline
    Bytes per word/element & 1B (8 bits)\\\hline
    PE array size & 32$\times$32\\ 
    \hline
    PE dot product size\footnote{No reduction for grouped convolutions} & 8\\\hline
  %  PE array size at scale (\autoref{fig:scale}) & 256$\times$256\\ \hline
  % PE dot product size at scale (\autoref{fig:scale}) & 32\\\hline
    SRAM capacity & 1MB \\\hline
 %  SRAM capacity at scale (\autoref{fig:scale}) & 8MB\\ \hline
%    Off-chip bandwidth & \\\hline
 %   On-chip bandwidth & \\\hline
  %  Inter-cluster bandwidth & \\\hline
   Memory bandwidth & 256 GB/s\\\hline
 %   Memory bandwidth at scale (\autoref{fig:scale}) & 1TB/s\\\hline
    %Total L2 capacity & 32MB\\ \hline
    %Total L3 capacity & 256MB\\ \hline
  \end{tabular}
\vspace{-3mm}
\end{centering}
\end{center}
\end{scriptsize}

\end{table}
\end{scriptsize}