
\section{Experimental Methodology}

\label{sec:expt}

\subsection{Evaluation Framework: Timeloop for \HHPName}

\autoref{fig:framework} shows the evaluation framework built on Timeloop~\cite{timeloop}. The inputs to the framework include configuration based on the proposed \HHPName taxonomy, cascades of operations, and hardware parameters. Operations are allocated based on reuse to sub-accelerators, and based on the taxonomy notation and resource partitioning, sub-accelerator architecture files are generated. These sub-accelerator files are based on Timeloop v0.4 and include a detailed hardware model with the ability to specify mapping constraints and resources available to each sub-accelerator. Timeloop mapper is called for each operation with the architecture files corresponding sub-accelerators executing that operation. We use a wrapper to compute the statistics of the HHP configuration from statistics of operations executed on individual sub-accelerators to return the final results corresponding to the cascade.

\subsection{Workloads}

We evaluate HHP's on mixed-intensity cascades of operations as shown in~\autoref{tables:workloads}. We focus on transformers, where we consider an attention layer from encoder-only transformers as intra-cascade workload partition, and prefill and decode stages from decoder-only transformer as inter-cascade workload partition. 

\begin{tabularx}{0.98\textwidth}{@{}XllllS[table-format=2.6]S[table-format=6.0]@{}}
	\toprule
                              &              &              &     &      & {Validation} & {\Runtime} \\ 
\textbf{Task} & \textbf{Dataset} & \textbf{Model} & \textbf{Loss} & \textbf{Metric} & \textbf{Target} & \textbf{Budget} \\ \midrule
Clickthrough rate \newline prediction  & \criteo      & \dlrmsmall   & CE  & CE   & 0.123735   & 7703    \\ \addlinespace
MRI reconstruction            & \fastmri     & \unet        & L1  & SSIM & 0.7344     & 8859    \\ \addlinespace
Image                         & \imagenet    & \resnetfifty & CE  & ER   & 0.22569    & 63008   \\
classification                &              & \vit         & CE  & ER   & 0.22691    & 77520   \\ \addlinespace
Speech                        & \librispeech & \conformer   & CTC & WER  & 0.085884   & 61068  \\
recognition                   &              & \deepspeech  & CTC & WER  & 0.119936   & 55506   \\ \addlinespace
Molecular property \newline prediction & \ogbg        & \gnn         & CE  & mAP  & 0.28098    & 18477   \\ \addlinespace
Translation                   & \wmt         & \transformer & CE  & BLEU & 30.8491    & 48151   \\ \bottomrule
\end{tabularx}


\subsection{Evaluation Configurations}

We evaluate the common configurations configurations based on the taxonomy shown in~\autoref{fig:taxonomy} (a-d) to study leaf vs hierarchical and homogeneous vs cross-node heterogeneous vs intra-node heterogeneous vs cross-depth heterogeneous. The hardware parameters are described in~\autoref{tables:config}. Other parameter changes for specific sensitivity studies are mentioned along with the studies.

\usepackage[utf8]{inputenc} % allow utf-8 input
\usepackage[T1]{fontenc}    % use 8-bit T1 fonts
\usepackage{microtype,inconsolata}
\usepackage{times,latexsym}
\usepackage{graphicx} \graphicspath{{figures/}}
\usepackage{amsmath,amssymb,mathabx,mathtools,amsthm,nicefrac}
\usepackage[linesnumbered,ruled,vlined]{algorithm2e}
\usepackage{acronym}
\usepackage{enumitem}
\usepackage[pagebackref,breaklinks,colorlinks]{hyperref}
\usepackage{balance}
\usepackage{xspace}
\usepackage{setspace}
\usepackage[skip=3pt,font=small]{subcaption}
\usepackage[skip=3pt,font=small]{caption}
\usepackage[capitalise,noabbrev,nameinlink]{cleveref}
\usepackage{booktabs,tabularx,colortbl,multirow,multicol,array,makecell,tabularray}
\usepackage{overpic,wrapfig}
\usepackage{dblfloatfix}
\usepackage[misc]{ifsym}
\usepackage{pifont}
\usepackage{fancyvrb}

% Add a period to the end of an abbreviation unless there's one
% already, then \xspace.
\makeatletter
\DeclareRobustCommand\onedot{\futurelet\@let@token\@onedot}
\def\@onedot{\ifx\@let@token.\else.\null\fi\xspace}

\def\eg{\emph{e.g}\onedot} \def\Eg{\emph{E.g}\onedot}
\def\ie{\emph{i.e}\onedot} \def\Ie{\emph{I.e}\onedot}
\def\cf{\emph{c.f}\onedot} \def\Cf{\emph{C.f}\onedot}
\def\etc{\emph{etc}\onedot} \def\vs{\emph{vs}\onedot}
\def\wrt{w.r.t\onedot} \def\dof{d.o.f\onedot}
\def\etal{\emph{et al}\onedot}

\makeatother

\acrodef{sota}[SOTA]{State-of-the-Art}
\acrodef{method}[\textsc{PRA}]{Preference-based Robot Assistant}
\acrodef{pbp}[\textsc{PbP}]{Preference-based Planning}
\acrodef{vln}[VLN]{Vision-and-Language Navigation}
\acrodef{llm}[LLM]{Large Language Model}
\acrodef{EILEV}[EILEV]{Efficient In-context Learning on Egocentric Videos}
\acrodef{vlm}[VLM]{Vision-Language Model}
\acrodef{vivit}[ViViT]{Video Vision Transformer}
\acrodef{llava}[LLaVA]{Large Language and Vision Assistant}
\acrodef{ai}[AI]{Artificial Intelligence}
\acrodef{ik}[IK]{Inverse Kinematics}
\acrodef{ompl}[OMPL]{Open Motion Planning Library}
\acrodef{sem}[SEM]{Structural Equation Model}

% Spacing
% \medmuskip=2mu   % reduce spacing around binary operators
% \thickmuskip=3mu % reduce spacing around relational operators
\setlength{\abovedisplayskip}{3pt}
\setlength{\belowdisplayskip}{3pt}
\setlength{\abovecaptionskip}{3pt}
\setlength{\belowcaptionskip}{3pt}
% \setlength\floatsep{1\baselineskip plus 3pt minus 2pt}
% \setlength\textfloatsep{1\baselineskip plus 3pt minus 2pt}
% \setlength\dbltextfloatsep{1\baselineskip plus 3pt minus 2pt}
% \setlength\intextsep{1\baselineskip plus 3pt minus 2pt}

\newcolumntype{x}{>{\columncolor{LightCyan1}}c}
\newcolumntype{y}{>{\columncolor{MistyRose}}c}


