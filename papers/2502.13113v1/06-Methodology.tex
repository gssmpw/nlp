
\section{Experimental Methodology}

\label{sec:expt}

\subsection{Evaluation Framework: Timeloop for \HHPName}

\autoref{fig:framework} shows the evaluation framework built on Timeloop~\cite{timeloop}. The inputs to the framework include configuration based on the proposed \HHPName taxonomy, cascades of operations, and hardware parameters. Operations are allocated based on reuse to sub-accelerators, and based on the taxonomy notation and resource partitioning, sub-accelerator architecture files are generated. These sub-accelerator files are based on Timeloop v0.4 and include a detailed hardware model with the ability to specify mapping constraints and resources available to each sub-accelerator. Timeloop mapper is called for each operation with the architecture files corresponding sub-accelerators executing that operation. We use a wrapper to compute the statistics of the HHP configuration from statistics of operations executed on individual sub-accelerators to return the final results corresponding to the cascade.

\subsection{Workloads}

We evaluate HHP's on mixed-intensity cascades of operations as shown in~\autoref{tables:workloads}. We focus on transformers, where we consider an attention layer from encoder-only transformers as intra-cascade workload partition, and prefill and decode stages from decoder-only transformer as inter-cascade workload partition. 

\begin{table}[h]
\begin{scriptsize}
    \centering
    \caption{Transformer workload configuration. Sequence lengths for decoder-only transformers are listed as prefill/decode}
    \begin{tabular}{|c|c|c|c|c|}
    \hline
         \textbf{Workload} & \textbf{Models} & \textbf{Partitioning} & {$\mathbf{d_{model}}$} &
         \textbf{Seq length} \\\hline
         Encoder (translation) & BERT-large & Intra-cascade & 1024 & 256\\\hline
         Decoder (chatbot~\cite{genz}) & Llama-2 & Inter-cascade & 4096 & 3000/1000\\\hline
         Decoder (chatbot~\cite{genz}) & GPT3  & Inter-cascade & 12288 & 3000/1000\\\hline
          
    \end{tabular}
    \label{tables:workloads}
\end{scriptsize}
\end{table}

\subsection{Evaluation Configurations}

We evaluate the common configurations configurations based on the taxonomy shown in~\autoref{fig:taxonomy} (a-d) to study leaf vs hierarchical and homogeneous vs cross-node heterogeneous vs intra-node heterogeneous vs cross-depth heterogeneous. The hardware parameters are described in~\autoref{tables:config}. Other parameter changes for specific sensitivity studies are mentioned along with the studies.

% Common packages
\usepackage[utf8]{inputenc} % allow utf-8 input
\usepackage[T1]{fontenc}    % use 8-bit T1 fonts
\usepackage{microtype,inconsolata}
\usepackage{times,latexsym}
\usepackage{graphicx} \graphicspath{{figures/}}
\usepackage{amsmath,amssymb,mathabx,mathtools,amsthm,nicefrac}
% \usepackage{algorithmic}
\usepackage[linesnumbered,ruled,vlined]{algorithm2e}
\usepackage{acronym}
\usepackage{enumitem}
%\usepackage[pagebackref,breaklinks,colorlinks]{hyperref}
\usepackage{balance}
\usepackage{xspace}
\usepackage{setspace}
\usepackage[skip=3pt,font=small]{subcaption}
\usepackage[skip=3pt,font=small]{caption}
%\usepackage[dvipsnames,svgnames,x11names,table]{xcolor}
\usepackage[capitalise,noabbrev,nameinlink]{cleveref}
\usepackage{booktabs,tabularx,colortbl,multirow,multicol,array,makecell,tabularray}
\usepackage{overpic,wrapfig}
\usepackage[misc]{ifsym}
\usepackage{pifont}
\usepackage{diagbox}


% Handy shorthand
\makeatletter
\DeclareRobustCommand\onedot{\futurelet\@let@token\@onedot}
\def\@onedot{\ifx\@let@token.\else.\null\fi\xspace}
\def\eg{\emph{e.g}\onedot} 
\def\Eg{\emph{E.g}\onedot}
\def\ie{\emph{i.e}\onedot} 
\def\Ie{\emph{I.e}\onedot}
\def\cf{\emph{c.f}\onedot} 
\def\Cf{\emph{C.f}\onedot}
\def\etc{\emph{etc}\onedot} 
\def\vs{\emph{vs}\onedot}
\def\aka{a.k.a\onedot}
\def\wrt{w.r.t\onedot} 
\def\dof{d.o.f\onedot}
\def\etal{\emph{et al}\onedot}
\makeatother

% Handy math ops
%\DeclareMathOperator*{\argmax}{arg\,max}
%\DeclareMathOperator*{\argmin}{arg\,min}
\DeclareMathOperator*{\kl}{KL}
\newcommand\energy{\mathcal{E}}
\newcommand{\norm}[1]{\left\Vert #1 \right\Vert}

% Handy table symbols
\newcommand{\cmark}{\ding{51}}%
\newcommand{\xmark}{\ding{55}}%

% Spacing
\frenchspacing
\makeatletter
\renewcommand{\paragraph}{%
  \@startsection{paragraph}{4}%
  {\z@}{0ex \@plus 0ex \@minus 0ex}{-1em}%
  {\hskip0em\normalfont\normalsize\bfseries}%
}
\makeatother

% Clever references
\crefname{algorithm}{Alg.}{Algs.}
\Crefname{algocf}{Algorithm}{Algorithms}
\crefname{section}{Sec.}{Secs.}
\Crefname{section}{Section}{Sections}
\crefname{table}{Tab.}{Tabs.}
\Crefname{table}{Table}{Tables}
\crefname{figure}{Fig.}{Figs.}
\Crefname{figure}{Figure}{Figures}
\crefname{equation}{Eq.}{Eqs.}
\Crefname{equation}{Equation}{Equations}
\crefname{appendix}{Appx.}{Appxs.}
\Crefname{appendix}{Appendix}{Appendices}

% Handy Colors
\definecolor{gblue}{HTML}{4285F4}
\definecolor{gred}{HTML}{DB4437}
\definecolor{ggreen}{HTML}{0F9D58}

\hypersetup{
  %citecolor=gray %Colour of citations
}

% Spacing
% \frenchspacing
% \medmuskip=2mu   % reduce spacing around binary operators
% \thickmuskip=3mu % reduce spacing around relational operators
% \setlength{\abovedisplayskip}{3pt}
% \setlength{\belowdisplayskip}{3pt}
% \setlength{\abovecaptionskip}{3pt}
% \setlength{\belowcaptionskip}{3pt}
% \setlength\floatsep{0.5\baselineskip plus 3pt minus 2pt}
% \setlength\textfloatsep{0.5\baselineskip plus 3pt minus 2pt}
% \setlength\dbltextfloatsep{0.5\baselineskip plus 3pt minus 2pt}
% \setlength\intextsep{0.5\baselineskip plus 3pt minus 2pt}

\newcolumntype{P}[1]{>{\centering\arraybackslash}p{#1}}
\newcolumntype{M}[1]{>{\centering\arraybackslash}m{#1}}

\acrodef{metaicl-w}[Minnow]{Meta-training for IN-context learNing Of Words}
\acrodef{metaicl}[MetaICL]{Meta-training for In-Context Learning}
\acrodef{icl}[ICL]{in-context learning}
\author{%
  Wentao Wang$^1$ \hspace{1em} Guangyuan Jiang$^2$ \hspace{1em} Tal Linzen$^1$ \hspace{1em} Brenden M.\ Lake$^1$ \\
  $^1$New York University \hspace{1em} $^2$Peking University \\
  \texttt{\{ww2135, linzen, brenden\}@nyu.edu} \hspace{1em} \texttt{jgy@stu.pku.edu.cn}
}


