%%%%%%%%%%%%%%%%%%%%%%%%%%%%%%%%%%%%%%%%%%%%%%%%%%%%%%%%%%%%%%%%%%%%%%%%%%%%%%%
%%%%%%%%%%%%%%%%%%%%%%%%%%%%%%%%%%%%%%%%%%%%%%%%%%%%%%%%%%%%%%%%%%%%%%%%%%%%%%%
% APPENDIX
%%%%%%%%%%%%%%%%%%%%%%%%%%%%%%%%%%%%%%%%%%%%%%%%%%%%%%%%%%%%%%%%%%%%%%%%%%%%%%%
%%%%%%%%%%%%%%%%%%%%%%%%%%%%%%%%%%%%%%%%%%%%%%%%%%%%%%%%%%%%%%%%%%%%%%%%%%%%%%%
\newpage
\appendix

%Code listing style named "mystyle"
\lstdefinestyle{mystyle}{
  % backgroundcolor=\color{backcolour},   commentstyle=\color{codegreen},
  % keywordstyle=\color{magenta},
  % numberstyle=\tiny\color{codegray},
  stringstyle=\color{codepurple},
  basicstyle=\ttfamily\footnotesize,
  % moredelim=[is][\color{red}]{@}{@}
  escapeinside={@}{@},
  breakatwhitespace=false,         
  breaklines=true,                 
  captionpos=b,                    
  keepspaces=true,                 
  % numbers=left,                    
  % numbersep=5pt,                  
  showspaces=false,                
  showstringspaces=false,
  showtabs=false,                  
  tabsize=2
}

%"mystyle" code listing set
\lstset{style=mystyle}

\onecolumn
This appendix presents supplementary materials. \Cref{sec:pal} provides a comparison between PEA and PAL. \Cref{sec:eval} contains additional empirical results omitted from the main text. \Cref{sec:prompts} details the prompts utilized for PEA in our evaluation.

\section{PEA vs PAL}\label{sec:pal}
PAL~\citep{pal} and our PEA framework framework both utilize code generation to enhance LLM performance on reasoning tasks. From a first-order logic perspective, PAL addresses less complex tasks involving direct concrete predicate evaluation without quantifiers, while PEA is designed to handle more complex logical structures. To illustrate, consider the following example of a PAL task:
\begin{tcolorbox}[title=PAL Example]
Q: Olivia has $23$. She bought five bagels for $3$ each. How much money does she have left?

\# solution in Python:
\begin{lstlisting}[language=Python]

def solution():
    """Olivia has $23. She bought five bagels for $3 each. 
    How much money does she have left?"""
    money_initial = 23
    bagels = 5
    bagel_cost = 3
    money_spent = bagels * bagel_cost
    money_left = money_initial - money_spent
    result = money_left
    return result
\end{lstlisting}

\end{tcolorbox}
In such problems, the solution merely requires extracting predicates from the natural language description and evaluating them directly. This contrasts with the quantified predicate problems primarily addressed by PEA, which necessitate more sophisticated enumeration strategies.
%%%%%%%%%%%%%%%%%%%%%%%%%%%%%%%%%%%%%%%%%%%%%%%%%%%%%%%%%%%%%%%%%%%%%%%%%%%%%%%
%%%%%%%%%%%%%%%%%%%%%%%%%%%%%%%%%%%%%%%%%%%%%%%%%%%%%%%%%%%%%%%%%%%%%%%%%%%%%%%

\section{Additional empirical results}\label{sec:eval}
\subsection{Cartesian product combination}
The results of the Cartesian product-combination evaluation are presented in~\cref{tab:prod}. These findings demonstrate a similar pattern to those observed in the permutation evaluation.


\begin{table}[!ht]
\begin{center}
\caption{Coverage of Cartesian product combinations: Percentage of total product combinations returned in LLM responses. Notation $(m, n)$ represents $n$ variables, each selecting from a pool of $m$ random string values, yielding $m^n$ product combinations. Values reported are means from three independent trials. For coverage below $10\%$, LLMs provide Cartesian product generation instructions rather than explicit results.
%\twr{I don't understand the phrase ``combination-generation instruction''.}
}
\begin{tabular}{ ccccc } 
\toprule
\bf Model & \bf (4,4) & \bf (5,4) & \bf (4,5) & \bf (5,5) \\  
\hline
GPT-4o & 100\%  & 100\% & 57.8\% & 26.8\% \\  
\hline
o1-mini &  100\% &  32.5\% &  1.3\% &  0.4\% \\  
\hline
o1 &  100\% &  67.7\% &  6.1\% &  0.9\% \\ 
  \bottomrule
\end{tabular}
\label{tab:prod}
\end{center}
\vspace{-0.3cm}
\end{table}

\subsection{2-SAT}
The results for the 2-SAT problem are presented in~\cref{tab:2-cnf}. Given that 2-SAT is a $\p$-class problem with efficient algorithmic solutions, it is noteworthy that all models demonstrated commendable performance in this evaluation.

\begin{table}[h!]
\begin{center}
\caption{Performance comparison of methods on 2-SAT problems. Results show the number of correctly solved formulas out of \textbf{100} satisfiable 2-CNF instances.
}
\begin{tabular}{ ccccc } 
\toprule
\bf Model & \bf PEA & \bf CoT & \bf DQ  \\  
\hline
GPT-4o & 100  & 88 & 77  \\  
\hline
o1-mini & 100 &  100  & 100  \\  
\hline
o1 &  100 &  100 &  100  \\ 
  \bottomrule
\end{tabular}
\label{tab:2-cnf}
\end{center}
\vspace{-0.3cm}
\end{table}



\subsection{Deepseek preliminary results}

Preliminary small-scale experiments were conducted utilizing DeepSeek models, specifically DeepSeek-V3 (\textit{deepseek-chat}) and DeepSeek-R1 (\textit{deepseek-reasoner}), for Blocksworlds and Logistics planning tasks via the DeepSeek API~\citep{deepseekai2024deepseekv3technicalreport,deepseekai2025deepseekr1incentivizingreasoningcapability}. PEA when applied to these models, necessitated multiple iterations to synthesize programs meeting sanity check criteria. However, the generated programs failed to achieve sufficient accuracy. While the web interface demonstrated rapid response times for direct queries and CoT approaches, API requests exhibited slow performance and poor connection stability. The obtained results did not match the quality of those from OpenAI o1. Consequently, despite intentions to conduct a comprehensive evaluation, the decision was made to defer full-scale testing pending improved service reliability.

\newcommand{\hlc}[2][yellow]{{%
    \colorlet{foo}{#1}%
    \sethlcolor{foo}\hl{#2}}%
}
\section{PEA Prompts}\label{sec:prompts}
This section presents the core PEA prompts utilized in our evaluation, \hlc[pink]{highlighting in pink} the PEA components. For prompts incorporating enumeration optimization strategies, the strategies are \hlc[yellow]{highlighted in yellow} and noted to be LLM-generated rather than user-crafted. The majority of the prompt content for planning problems is directly from the problem descriptions provided in the dataset.
\begin{tcolorbox}[title=SAT Prompt]
\begin{lstlisting}
You are given a boolean formula with 10 boolean variables and conjunctive normal form. Find a satisfiable assignment for this formula and return the assignment to me as a list of 10 True/False values in a bracket [], for example, [True, False, True, True, False, False, False, True, True, False]; If this formula is not satisfiable, then return me UNSAT.

Now the tasks are to generate three Python functions to answer this question. We will use a Predicate-Enumeration-Aggregation template to solve this problem.

@\hlc[pink]{For the predicate, implement a function "evaluate\_formula" to with inputs: a python function "formula" and a bool variable list "vals". Evaluate "formula" with "vals" and return True or False.}@

@\hlc[pink]{For the enumeration, implement a function "enumerate\_boolean" that generates all possible enumerations of 10 boolean variables.}@

@\hlc[pink]{For the aggregation, implement a function "can\_evaluate" to find either valid truth assignment or UNSAT. "can\_evaluate" takes a python function "formula" as input and use the previously generated functions by checking whether any of the boolean assignment from "enumerate\_boolean" can satisfy formula by passing the assignment and "formula" to "evaluate\_formula". If such an assignment exists, print out the assignment and return True; else, return False.}@

You may add comments to the generated code to show how the implementation reflects the instruction.
\end{lstlisting}
\end{tcolorbox}

\begin{tcolorbox}[title=G24 Prompt]
\begin{lstlisting}
Given any four positive numbers which may have repetitions; Using just the +, -, *, and / operators; and the possible use of parenthesis, (), show how to make an answer of 24. You will use each number once and only once. Return me the arithmetic expression in a bracket, for example,[6+2*(4+5)]; if it cannot make an answer 24, then tell me it cannot.

Now the tasks are to generate Python functions to answer this question. We will use a Predicate-Enumeration-Aggregation template to solve this problem.

@\hlc[pink]{For the predicate, implement a function "evaluate\_to\_24" to with input: an expression "expression" and check whether the expression can be evaluated to 24. Notice arithemtical errors like divide-by-zero and evaluation precision. Return True if it is evaluated to 24; or False otherwise.}@

@\hlc[pink]{For the enumeration, implement a function "generate\_expressions" that with input: four numbers "n1, n2, n3, n4" and generates all possible arithmetical expressions of the four input numbers, that uses +, -, *, and /; and all possible parenthesis patterns.}@

@\hlc[pink]{For the aggregation, implement a function "can\_evaluate" to find either a valid 24 expression or impossible. "can\_evaluate" takes four numbers as input and uses the previously generated functions by checking whether any of the expressions from "generate\_expressions" can be evaluated to 24 by calling "evaluate\_to\_24".
If yes, print the expression and return True;
If no expression evaluates to 24, then return False.}@

You may add comments to the generated code to show how the implementation reflects the instruction.
\end{lstlisting}

\end{tcolorbox}

\begin{tcolorbox}[breakable, title=Blocksworld Prompt]
\begin{lstlisting}
I am playing with a set of blocks where I need to arrange the blocks into stacks. Here are the actions I can do:

Pick up a block. It takes 1 minute to pick up a block.
Unstack a block from on top of another block. It takes 1 minute to unstack a block from on top of another block.
Put down a block. It takes 1 minute to put down a block.
Stack a block on top of another block. It takes 1 minute to stack a block on top of another block.

I have the following restrictions on my actions:
I can only pick up or unstack one block at a time.
I can only pick up or unstack a block if my hand is empty.
I can only pick up a block if the block is on the table and the block is clear. A block is clear if the block has no other blocks on top of it and if the block is not picked up.
I can only unstack a block from on top of another block if the block I am unstacking was really on top of the other block.
I can only unstack a block from on top of another block if the block I am unstacking is clear.
Once I pick up or unstack a block, I am holding the block.
I can only put down a block that I am holding.
I can only stack a block on top of another block if I am holding the block being stacked.
I can only stack a block on top of another block if the block onto which I am stacking the block is clear.
Once I put down or stack a block, my hand becomes empty.
Once you stack a block on top of a second block, the second block is no longer clear.

[STATEMENT]
As initial conditions I have that, the red block is clear, the blue block is clear, the yellow block is clear, the hand is empty, the blue block is on top of the orange block, the red block is on the table, the orange block is on the table and the yellow block is on the table.
My goal is to have that the orange block is on top of the blue block. I want to minimize the time taken to achieve my goal.

My plan is as follows:

[PLAN]
unstack the blue block from on top of the orange block
put down the blue block
pick up the orange block
stack the orange block on top of the blue block
[PLAN END]
The total time to execute the plan is 4 minutes.

[STATEMENT]
As initial conditions I have that, the red block is clear, the yellow block is clear, the hand is empty, the red block is on top of the blue block, the yellow block is on top of the orange block, the blue block is on the table and the orange block is on the table.
My goal is to have that the orange block is on top of the red block. I want to minimize the time taken to achieve my goal.
My plan is as follows:

[PLAN]

@\hlc[pink]{Generate a class `BlockWorldState` to represent the state of the blocks world. The state includes the positions of the blocks, the clear state of the blocks, and the state of the hand.}@

@\hlc[pink]{For the predicate, check function `check\_goal` to determine whether a state satified the goal conditions.}@

@\hlc[pink]{For the enumeration, implement function `generate\_actions` to generate valid actions at the current state. You can almost have four possible actions but should filter out with given restrictions. Also implement function `apply\_action` to apply an action to the current state. The function should return a new state after applying the action.}@

@\hlc[pink]{For the aggregation, implement a function `find\_plan` to find a plan to reach the target goal. The function should take the initial state and the goal  conditions as input and return a list of actions. For the implementation, the function should use BFS. The function should use functions generated above. Also implement a function `verify\_plan` to verify a given plan. The function should take the initial state, the goal conditions, and the plan as input and return a boolean value.
}@

\end{lstlisting}
\end{tcolorbox}

\begin{tcolorbox}[breakable, title=Logistics Prompt]
\begin{lstlisting}
I have to plan logistics to transport packages within cities via trucks and between cities via airplanes. Locations within a city are directly connected (trucks can move between any two such locations), and so are the cities. In each city there is exactly one truck and each city has one location that serves as an airport.
Here are the actions that can be performed:

Load a package into a truck. For example, load package_1 into truck_1 at location_1_1.
Load a package into an airplane. For example, load package_1 into airplane_1 at location_1_1.
Unload a package from a truck. For example, unload package_1 from truck_1 at location_1_1.
Unload a package from an airplane. For example, unload package_1 from airplane_1 at location_1_1.
Drive a truck from one location to another location. For example, drive truck_1 from location_1_1 to location_1_2 in city_1.
Fly an airplane from one city to another city. For example, fly airplane_1 from location_1_1 to location_2_1. Here location_1_1 is the airport in city_1 and location_2_1 is the airport in city_2.

The following are the restrictions on the actions:
A package can be loaded into a truck only if the package and the truck are in the same location.
Once a package is loaded into a truck, the package is not at the location and is in the truck.   
A package can be loaded into an airplane only if the package and the airplane are in the same location.
Once a package is loaded into an airplane, the package is not at the location and is in the airplane.
A package can be unloaded from a truck only if the package is in the truck.
Once a package is unloaded from a truck, the package is not in the truck and is at the location of the truck.
A package can be unloaded from an airplane only if the package in the airplane.
Once a package is unloaded from an airplane, the package is not in the airplane and is at the location of the airplane.   
A truck can be driven from one location to another if the truck is at the from-location and both from-location and to-location are locations in the same city.
Once a truck is driven from one location to another, it is not at the from-location and is at the to-location.
An airplane can be flown from one city to another if the from-location and the to-location are airports and the airplane is at the from-location.
Once an airplane is flown from one city to another the airplane is not at the from-location and is at the to-location.

[STATEMENT]
As initial conditions I have that, location_0_0 is an airport, location_1_0 is an airport, airplane_0 is at location_1_0, package_0 is at location_1_0, truck_0 is at location_0_0, truck_1 is at location_1_0, location_0_0 is in the city city_0, location_0_1 is in the city city_0, location_1_0 is in the city city_1 and location_1_1 is in the city city_1.
My goal is to have that package_0 is at location_1_1.

My plan is as follows:

[PLAN]
load package_0 into truck_1 at location_1_0
drive truck_1 from location_1_0 to location_1_1 in city_1
unload package_0 from truck_1 at location_1_1
[PLAN END]

[STATEMENT]
As initial conditions I have that, location_0_0 is an airport, location_1_0 is an airport, airplane_0 is at location_0_0, package_0 is at location_0_1, truck_0 is at location_0_0, truck_1 is at location_1_1, location_0_0 is in the city city_0, location_0_1 is in the city city_0, location_1_0 is in the city city_1 and location_1_1 is in the city city_1.
My goal is to have that package_0 is at location_1_0.

My plan is as follows:

[PLAN]

@\hlc[pink]{Generate a class `LogisticsState` to represent the state of the world. Remember to import the deepcopy if you use it.}@

@\hlc[pink]{For the predicate, check function `check\_goal` to determine whether a state satified the goal conditions.}@

@\hlc[pink]{For the enumeration, implement the function `generate\_actions` to generate valid actions at the current state.}@

@\hlc[pink]{For the aggregation, implement a function `find\_plan` to find a plan to reach the target goal. The function should take the initial state and the goal conditions as input and return a list of actions. For the implementation, the function should use BFS. The function should use functions generated above. Also implement a function `verify\_plan` to verify a given plan. The function should take the initial state, the goal conditions, and the plan condition as input and return a boolean value.}@
\end{lstlisting}
\end{tcolorbox}

\begin{tcolorbox}[breakable, title=Logistics Prompt for Optimization Strategies]
\begin{lstlisting}
I have to plan logistics to transport packages within cities via trucks and between cities via airplanes. Locations within a city are directly connected (trucks can move between any two such locations), and so are the cities. In each city there is exactly one truck and each city has one location that serves as an airport.
Here are the actions that can be performed:

Load a package into a truck. For example, load package_1 into truck_1 at location_1_1.
Load a package into an airplane. For example, load package_1 into airplane_1 at location_1_1.
Unload a package from a truck. For example, unload package_1 from truck_1 at location_1_1.
Unload a package from an airplane. For example, unload package_1 from airplane_1 at location_1_1.
Drive a truck from one location to another location. For example, drive truck_1 from location_1_1 to location_1_2 in city_1.
Fly an airplane from one city to another city. For example, fly airplane_1 from location_1_1 to location_2_1. Here location_1_1 is the airport in city_1 and location_2_1 is the airport in city_2.

The following are the restrictions on the actions:
A package can be loaded into a truck only if the package and the truck are in the same location.
Once a package is loaded into a truck, the package is not at the location and is in the truck.   
A package can be loaded into an airplane only if the package and the airplane are in the same location.
Once a package is loaded into an airplane, the package is not at the location and is in the airplane.
A package can be unloaded from a truck only if the package is in the truck.
Once a package is unloaded from a truck, the package is not in the truck and is at the location of the truck.
A package can be unloaded from an airplane only if the package in the airplane.
Once a package is unloaded from an airplane, the package is not in the airplane and is at the location of the airplane.   
A truck can be driven from one location to another if the truck is at the from-location and both from-location and to-location are locations in the same city.
Once a truck is driven from one location to another, it is not at the from-location and is at the to-location.
An airplane can be flown from one city to another if the from-location and the to-location are airports and the airplane is at the from-location.
Once an airplane is flown from one city to another the airplane is not at the from-location and is at the to-location.

[STATEMENT]
As initial conditions I have that, location_0_0 is an airport, location_1_0 is an airport, airplane_0 is at location_1_0, package_0 is at location_1_0, truck_0 is at location_0_0, truck_1 is at location_1_0, location_0_0 is in the city city_0, location_0_1 is in the city city_0, location_1_0 is in the city city_1 and location_1_1 is in the city city_1.
My goal is to have that package_0 is at location_1_1.

My plan is as follows:

[PLAN]
load package_0 into truck_1 at location_1_0
drive truck_1 from location_1_0 to location_1_1 in city_1
unload package_0 from truck_1 at location_1_1
[PLAN END]

[STATEMENT]
As initial conditions I have that, location_0_0 is an airport, location_1_0 is an airport, airplane_0 is at location_0_0, package_0 is at location_0_1, truck_0 is at location_0_0, truck_1 is at location_1_1, location_0_0 is in the city city_0, location_0_1 is in the city city_0, location_1_0 is in the city city_1 and location_1_1 is in the city city_1.
My goal is to have that package_0 is at location_1_0.

My plan is as follows:

[PLAN]

@\hlc[pink]{Generate a class `LogisticsState` to represent the state of the world. Remember to import the deepcopy if you use it.}@

@\hlc[pink]{For the predicate, check function `check\_goal` to determine whether a state satified the goal conditions.}@

@\hlc[pink]{For the enumeration, implement the function `generate\_actions` to generate valid actions at the current state.}@

@\hlc[pink]{For the aggregation, implement a function `find\_plan` to find a plan to reach the target goal. The function should take the initial state and the goal conditions as input and return a list of actions. For the implementation, the function should use BFS. The function should use functions generated above. Also implement a function `verify\_plan` to verify a given plan. The function should take the initial state, the goal conditions, and the plan condition as input and return a boolean value.}@

@\hlc[pink]{Instead of generating the above code, for function generate\_actions,  generate a structured and succinct instruction description for generating one deterministic action for reaching packages' goals, trying to think about all possible actions according to the package, truck and airplane's location and remove unhelpful actions.  I will insert it in the later prompt for code generation so you don't need to have a summary.}@

\end{lstlisting}
\end{tcolorbox}

\begin{tcolorbox}[breakable, title=Logistics Prompt with Enumeration Optimization Strategies]
\begin{lstlisting}
I have to plan logistics to transport packages within cities via trucks and between cities via airplanes. Locations within a city are directly connected (trucks can move between any two such locations), and so are the cities. In each city there is exactly one truck and each city has one location that serves as an airport.
Here are the actions that can be performed:

Load a package into a truck. For example, load package_1 into truck_1 at location_1_1.
Load a package into an airplane. For example, load package_1 into airplane_1 at location_1_1.
Unload a package from a truck. For example, unload package_1 from truck_1 at location_1_1.
Unload a package from an airplane. For example, unload package_1 from airplane_1 at location_1_1.
Drive a truck from one location to another location. For example, drive truck_1 from location_1_1 to location_1_2 in city_1.
Fly an airplane from one city to another city. For example, fly airplane_1 from location_1_1 to location_2_1. Here location_1_1 is the airport in city_1 and location_2_1 is the airport in city_2.

The following are the restrictions on the actions:
A package can be loaded into a truck only if the package and the truck are in the same location.
Once a package is loaded into a truck, the package is not at the location and is in the truck.   
A package can be loaded into an airplane only if the package and the airplane are in the same location.
Once a package is loaded into an airplane, the package is not at the location and is in the airplane.
A package can be unloaded from a truck only if the package is in the truck.
Once a package is unloaded from a truck, the package is not in the truck and is at the location of the truck.
A package can be unloaded from an airplane only if the package in the airplane.
Once a package is unloaded from an airplane, the package is not in the airplane and is at the location of the airplane.   
A truck can be driven from one location to another if the truck is at the from-location and both from-location and to-location are locations in the same city.
Once a truck is driven from one location to another, it is not at the from-location and is at the to-location.
An airplane can be flown from one city to another if the from-location and the to-location are airports and the airplane is at the from-location.
Once an airplane is flown from one city to another the airplane is not at the from-location and is at the to-location.

[STATEMENT]
As initial conditions I have that, location_0_0 is an airport, location_1_0 is an airport, airplane_0 is at location_1_0, package_0 is at location_1_0, truck_0 is at location_0_0, truck_1 is at location_1_0, location_0_0 is in the city city_0, location_0_1 is in the city city_0, location_1_0 is in the city city_1 and location_1_1 is in the city city_1.
My goal is to have that package_0 is at location_1_1.

My plan is as follows:

[PLAN]
load package_0 into truck_1 at location_1_0
drive truck_1 from location_1_0 to location_1_1 in city_1
unload package_0 from truck_1 at location_1_1
[PLAN END]

[STATEMENT]
As initial conditions I have that, location_0_0 is an airport, location_1_0 is an airport, airplane_0 is at location_0_0, package_0 is at location_0_1, truck_0 is at location_0_0, truck_1 is at location_1_1, location_0_0 is in the city city_0, location_0_1 is in the city city_0, location_1_0 is in the city city_1 and location_1_1 is in the city city_1.
My goal is to have that package_0 is at location_1_0.

My plan is as follows:

[PLAN]

@\hlc[pink]{Generate a class `LogisticsState` to represent the state of the world. Remember to import the deepcopy if you use it.}@

@\hlc[pink]{For the predicate, check function `check\_goal` to determine whether a state satified the goal conditions.}@

@\hlc[yellow]{For the enumeration, implement the function `generate\_actions` to generate useful and valid actions at the current state. The function should follow these rules:}@

@\hlc[yellow]{
Below is a concise step‐by‐step procedure for selecting **one** “helpful” action that moves at least one package closer to its goal location (no unnecessary detours). You can integrate this logic into `generate\_actions` to produce exactly one deterministic next action (rather than enumerating everything) that progresses a single package:}@

@\hlc[yellow]{
\#\#\# 1. Identify a Package to Move
}@
@\hlc[yellow]{1. Gather all packages **not** at their goal locations.}@
@\hlc[yellow]{2. Pick **one** such package (e.g., the first in a list).}@

@\hlc[yellow]{
\#\#\# 2. Check If Package Is in the Correct City}@

@\hlc[yellow]{1. **If the package is not in its goal city**:}@

    @\hlc[yellow]{1. **Not Loaded,** at some location (possibly an airport):}@
     @\hlc[yellow]{- If a **truck** is co‐located and this location is **not** an airport, use \`{}['load-truck', package, truck, location]\`{}.}@

     @\hlc[yellow]{- Else if an **airplane** is co‐located and this location **is** an airport, use \`{}['load-airplane', package, airplane, airport\_location]\`{}.}@

     @\hlc[yellow]{- If neither vehicle is co‐located, pick a valid action to bring the vehicle to the package (e.g., drive the city’s truck over, or fly the airplane in if it is an airport) — returning the **first** such valid movement action.}@

   @\hlc[yellow]{2. **Loaded in a truck**, but not at the airport:}@
     @\hlc[yellow]{- Drive that truck toward its city’s airport to prepare for intercity shipment:  }@
        @\hlc[yellow]{\`{}['drive-truck', truck, current\_loc, airport\_loc, city]\`{}.}@
        
   @\hlc[yellow]{3. **Loaded in an airplane**:}@
     @\hlc[yellow]{- Fly to the destination city’s airport:  }@
        @\hlc[yellow]{\`{}['fly-airplane', airplane, current\_airport, destination\_airport]\`{}.}@

@\hlc[yellow]{2. **If the package is in its correct city, but not at its goal location**:}@
   @\hlc[yellow]{1. **Not Loaded** and co‐located with the city’s truck:}@
     @\hlc[yellow]{- Load it:  
        \`{}['load-truck', package, truck, location]\`{}.}@
     @\hlc[yellow]{- If the truck is **not** at the same location yet, drive the truck there first.}@

   @\hlc[yellow]{2. **Loaded in the city’s truck**:}@
     @\hlc[yellow]{- Drive the truck to the package’s goal location: \`{}['drive-truck', truck, current\_loc, goal\_loc, city]\`{}.}@

   @\hlc[yellow]{3. **In the truck at goal location**:}@
     @\hlc[yellow]{ Unload it:  
        \`{}['unload-truck', package, truck, goal\_loc]\`{}.}@

@\hlc[yellow]{
\#\#\# 3. Return the Single Action
- As soon as you identify a valid step (load, drive, fly, or unload) that progresses your chosen package **toward** its goal, **return** that action immediately (rather than continuing to look for others). This ensures only the “helpful” action is generated.}@

   @\hlc[yellow]{- As soon as you identify a valid step (load, drive, fly, or unload) that progresses your chosen package **toward** its goal, **return** that action immediately (rather than continuing to look for others). This ensures only the “helpful” action is generated.}@

@\hlc[yellow]{This procedure avoids superfluous moves: it always moves a relevant vehicle/piece of cargo step by step closer to the correct final location, ignoring any action that does not help a package reach its goal.}@

@\hlc[pink]{For the aggregation, implement a function `find\_plan` to find a plan to reach the target goal. The function should take the initial state and the goal conditions as input and return a list of actions. For the implementation, the function should use BFS. The function should use functions generated above. Also implement a function `verify\_plan` to verify a given plan. The function should take the initial state, the goal conditions, and the plan condition as input and return a boolean value. }@

Let's think step by step.
\end{lstlisting}
\end{tcolorbox}

% @\hlc[yellow]{
% 1. **Focus on Packages Not at Goal Locations**:
%    We should only consider the actions for the just one package that are not at the goal locations.}@

% @\hlc[yellow]{
% 2. **Choose Between Truck and Airplane Actions**: }@

%    @\hlc[yellow]{- Consider a **truck action** if any of the following holds:}@
%      @\hlc[yellow]{- The package is already loaded in a truck. }@
%      @\hlc[yellow]{- The package is not loaded, and its current location and goal location are in the same city. }@
%      @\hlc[yellow]{- The package is not loaded, and its current location is not at an airport. }@
     
%    @\hlc[yellow]{- Otherwise, consider an **airplane action**. }@

% @\hlc[yellow]{
% 3. **Truck Action Rules**:}@
%    @\hlc[yellow]{- Determine the target location for a truck:}@
%      @\hlc[yellow]{- If the package's goal is in the same city, the target is the goal location.}@
%      @\hlc[yellow]{- Otherwise, the target is the airport in the current city.}@
%    @\hlc[yellow]{- Generate the following actions based on the state:}@
%      @\hlc[yellow]{- If the package is loaded in a truck and the truck is not at the target, move the truck to the target.}@
%      @\hlc[yellow]{- If the package is loaded in a truck and the truck is at the target, unload the package from the truck.}@
%      @\hlc[yellow]{- If the package is not loaded and there is no truck is at the package's current location, move one truck in package;s current city to this location.}@
%      @\hlc[yellow]{- If the package is not loaded and a truck is at the package;s current location, load the package into the truck.}@

% @\hlc[yellow]{
% 4. **Airplane Action Rules**:}@
%    @\hlc[yellow]{- Determine the target city for a airplane:}@
%      @\hlc[yellow]{- The target city is the city of the package's goal location.}@
%    @\hlc[yellow]{- Generate the following actions based on the state:}@
%      @\hlc[yellow]{- If the package is loaded in the airplane and the airplane is not at the target city's airport, move the airplane to the target city's airport.}@
%      @\hlc[yellow]{- If the package is loaded in the airplane and the airplane is at the target city's airport, unload the package from the airplane.}@
%      @\hlc[yellow]{- If the package is not loaded and there is no airplane at the current airport, move one airplane to this airport.}@
%      @\hlc[yellow]{- If the package is not loaded and a airplane is at the current airport, load the package into the airplane.}@

% @\hlc[yellow]{5. **Deterministic Choice**:}@
%    @\hlc[yellow]{- Generate **one deterministic action** at a time, following the above rules.}@
