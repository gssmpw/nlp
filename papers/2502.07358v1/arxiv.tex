%\documentclass[acmtog]{formats/acmart} 
\documentclass[acmtog]{acmart}
\makeatletter
\let\@authorsaddresses\@empty
\makeatother

%\documentclass[10pt, format=manuscript,nonacm]{acmart}

%\documentclass[acmtog,anonymous,review]{acmart}


\usepackage{booktabs} % For formal tables
\citestyle{acmauthoryear}
\setcitestyle{square}
\usepackage{ifthen}
\usepackage{enumitem}
\usepackage{algorithm}
%\usepackage{algorithmicx}
%\usepackage{algcompatible}
\usepackage[noend]{algpseudocode}
%\usepackage{algorithm2e}
\usepackage{float}
\usepackage{syntax}
\usepackage{amsfonts}
\let\Bbbk\relax
\usepackage{listings}
\usepackage{fancyvrb}
\usepackage{wrapfig}
\usepackage{graphicx}
\usepackage{subfigure}
\usepackage{xspace}
\usepackage{colortbl}
\usepackage{gensymb}
\usepackage{verbatim}
\usepackage{colortbl}
\usepackage{booktabs}
\usepackage{multirow}
\usepackage{cleveref}
\usepackage{pifont}
\usepackage{circledsteps}



%\lstset{style=mystyle}
\newcommand{\etal}{et al.}
\newcommand\TODO[1]{{\color{red} A: #1}}
\newcommand{\yuexin}[1]{{\color{blue} YX: #1}}
\newcommand{\ZY}[1]{{\color{cyan} ZY: #1}}
\newcommand{\manyi}[1]{{\color{brown} MY: #1}}
\newcommand{\haoran}[1]{{\color{olive} HR: #1}}
\newcommand{\yiteng}[1]{{\color{red} YT: #1}}
\newcommand{\yiming}[1]{{\color{orange} YM: #1}}
\newcommand{\changhe}[1]{{\color{orange} CH: #1}}


\usepackage{amsthm,amsmath,amssymb}
\usepackage{mathrsfs}
\usepackage[normalem]{ulem}
\begin{document}
\settopmatter{printacmref=false}
\setcopyright{none}
\renewcommand\footnotetextcopyrightpermission[1]{}
\pagestyle{empty} 

%\setcopyright{acmlicensed}
%\acmJournal{TOG}
%\acmYear{XXX} \acmVolume{XX} \acmNumber{XX} \acmArticle{XX} \acmMonth{XX} %\acmPrice{15.00}\acmDOI{10.1145/3618324}
\title{SymbioSim: Human-in-the-loop Simulation Platform for Bidirectional Continuing Learning in Human-Robot Interaction}


\author{Haoran Chen}
\authornote{Joint first authors.}
\affiliation{%
  \institution{Shandong University}
  \country{China}
}
\author{Yiteng Xu}
\authornotemark[1]
\affiliation{%
  \institution{ShanghaiTech University}
  \country{China}
}
\author{Yiming Ren, Yaoqin Ye}
\affiliation{%
  \institution{ShanghaiTech University}
  \country{China}
}
%\author{Yaoqin Ye}
%\affiliation{%
%  \institution{ShanghaiTech University}
%}
\author{Xinran Li, Ning Ding}
\affiliation{%
  \institution{Shandong University}
  \country{China}
}
%\author{Ning Ding}
%\affiliation{%
%  \institution{Shandong University}
%}
\author{Peishan Cong, Ziyi Wang, Bushi Liu, Yuhan Chen}
\affiliation{%
  \institution{ShanghaiTech University}
  \country{China}
}
%\author{Ziyi Wang}
%\affiliation{%
%  \institution{ShanghaiTech University}
%}
%\author{Bushi Liu}
%\affiliation{%
%  \institution{ShanghaiTech University}
%}
%\author{Yuhan Chen}
%\affiliation{%
%  \institution{ShanghaiTech University}
%}
\author{Zhiyang Dou}
\affiliation{%
  \institution{The University of Hong Kong}
  \country{China}
}
\author{Xiaokun Leng}
\affiliation{%
  \institution{LEJU(Shenzhen) Robotics Co., Ltd}
  \country{China}
}
\author{Manyi Li}
\authornote{Corresponding authors.}
\affiliation{%
  \institution{Shandong University}
  \country{China}
}
\author{Yuexin Ma}
\authornotemark[2]
\affiliation{%
  \institution{ShanghaiTech University}
  \country{China}
}
\author{Changhe Tu}
\affiliation{%
  \institution{Shandong University}
  \country{China}
}

\begin{abstract}
The development of intelligent robots seeks to seamlessly integrate them into the human world, providing assistance and companionship in daily life and work, with the ultimate goal of achieving human-robot symbiosis. To realize this vision, robots must continuously learn and evolve through consistent interaction and collaboration with humans, while humans need to gradually develop an understanding of and trust in robots through shared experiences. However, training and testing algorithms directly on physical robots involve substantial costs and safety risks. Moreover, current robotic simulators fail to support real human participation, limiting their ability to provide authentic interaction experiences and gather valuable human feedback.
In this paper, we introduce \textit{SymbioSim}, a novel human-in-the-loop robotic simulation platform designed to enable the safe and efficient development, evaluation, and optimization of human-robot interactions. By leveraging a carefully designed system architecture and modules, \textit{SymbioSim} delivers a natural and realistic interaction experience, facilitating bidirectional continuous learning and adaptation for both humans and robots. Extensive experiments and user studies demonstrate the platform's promising performance and highlight its potential to significantly advance research on human-robot symbiosis.
\end{abstract}


%\ccsdesc[500]{Computing methodologies~Shape modeling}
%\ccsdesc[300]{Computing methodologies~Graphics systems and interfaces}

%\acmJournal{TOG}

\keywords{Robotic Simulation, Augmented Reality, Human-robot Interaction, Real-time Motion Generation}

\begin{teaserfigure}
%\vspace{-5pt}
  \includegraphics[width=1.0\linewidth]{images/teaser.jpg}
	\vspace{-15pt}
\caption {Demonstration of the human-robot interaction process in SymbioSim. In this system, human interacts with a virtual robot in 3D space through AR. The robot perceives human actions and generates real-time responses. Feedback from the human experience is continuously collected to optimize the robot’s model, improving its performance. Concurrently, humans gradually develop a deeper understanding and trust in the robot. Ultimately, SymbioSim fosters bidirectional learning and adaptation, with the potential to promote human-robot symbiosis.}
  \label{fig:teaser}    
\end{teaserfigure}

%
%
\setlength\unitlength{1mm}
\newcommand{\twodots}{\mathinner {\ldotp \ldotp}}
% bb font symbols
\newcommand{\Rho}{\mathrm{P}}
\newcommand{\Tau}{\mathrm{T}}

\newfont{\bbb}{msbm10 scaled 700}
\newcommand{\CCC}{\mbox{\bbb C}}

\newfont{\bb}{msbm10 scaled 1100}
\newcommand{\CC}{\mbox{\bb C}}
\newcommand{\PP}{\mbox{\bb P}}
\newcommand{\RR}{\mbox{\bb R}}
\newcommand{\QQ}{\mbox{\bb Q}}
\newcommand{\ZZ}{\mbox{\bb Z}}
\newcommand{\FF}{\mbox{\bb F}}
\newcommand{\GG}{\mbox{\bb G}}
\newcommand{\EE}{\mbox{\bb E}}
\newcommand{\NN}{\mbox{\bb N}}
\newcommand{\KK}{\mbox{\bb K}}
\newcommand{\HH}{\mbox{\bb H}}
\newcommand{\SSS}{\mbox{\bb S}}
\newcommand{\UU}{\mbox{\bb U}}
\newcommand{\VV}{\mbox{\bb V}}


\newcommand{\yy}{\mathbbm{y}}
\newcommand{\xx}{\mathbbm{x}}
\newcommand{\zz}{\mathbbm{z}}
\newcommand{\sss}{\mathbbm{s}}
\newcommand{\rr}{\mathbbm{r}}
\newcommand{\pp}{\mathbbm{p}}
\newcommand{\qq}{\mathbbm{q}}
\newcommand{\ww}{\mathbbm{w}}
\newcommand{\hh}{\mathbbm{h}}
\newcommand{\vvv}{\mathbbm{v}}

% Vectors

\newcommand{\av}{{\bf a}}
\newcommand{\bv}{{\bf b}}
\newcommand{\cv}{{\bf c}}
\newcommand{\dv}{{\bf d}}
\newcommand{\ev}{{\bf e}}
\newcommand{\fv}{{\bf f}}
\newcommand{\gv}{{\bf g}}
\newcommand{\hv}{{\bf h}}
\newcommand{\iv}{{\bf i}}
\newcommand{\jv}{{\bf j}}
\newcommand{\kv}{{\bf k}}
\newcommand{\lv}{{\bf l}}
\newcommand{\mv}{{\bf m}}
\newcommand{\nv}{{\bf n}}
\newcommand{\ov}{{\bf o}}
\newcommand{\pv}{{\bf p}}
\newcommand{\qv}{{\bf q}}
\newcommand{\rv}{{\bf r}}
\newcommand{\sv}{{\bf s}}
\newcommand{\tv}{{\bf t}}
\newcommand{\uv}{{\bf u}}
\newcommand{\wv}{{\bf w}}
\newcommand{\vv}{{\bf v}}
\newcommand{\xv}{{\bf x}}
\newcommand{\yv}{{\bf y}}
\newcommand{\zv}{{\bf z}}
\newcommand{\zerov}{{\bf 0}}
\newcommand{\onev}{{\bf 1}}

% Matrices

\newcommand{\Am}{{\bf A}}
\newcommand{\Bm}{{\bf B}}
\newcommand{\Cm}{{\bf C}}
\newcommand{\Dm}{{\bf D}}
\newcommand{\Em}{{\bf E}}
\newcommand{\Fm}{{\bf F}}
\newcommand{\Gm}{{\bf G}}
\newcommand{\Hm}{{\bf H}}
\newcommand{\Id}{{\bf I}}
\newcommand{\Jm}{{\bf J}}
\newcommand{\Km}{{\bf K}}
\newcommand{\Lm}{{\bf L}}
\newcommand{\Mm}{{\bf M}}
\newcommand{\Nm}{{\bf N}}
\newcommand{\Om}{{\bf O}}
\newcommand{\Pm}{{\bf P}}
\newcommand{\Qm}{{\bf Q}}
\newcommand{\Rm}{{\bf R}}
\newcommand{\Sm}{{\bf S}}
\newcommand{\Tm}{{\bf T}}
\newcommand{\Um}{{\bf U}}
\newcommand{\Wm}{{\bf W}}
\newcommand{\Vm}{{\bf V}}
\newcommand{\Xm}{{\bf X}}
\newcommand{\Ym}{{\bf Y}}
\newcommand{\Zm}{{\bf Z}}

% Calligraphic

\newcommand{\Ac}{{\cal A}}
\newcommand{\Bc}{{\cal B}}
\newcommand{\Cc}{{\cal C}}
\newcommand{\Dc}{{\cal D}}
\newcommand{\Ec}{{\cal E}}
\newcommand{\Fc}{{\cal F}}
\newcommand{\Gc}{{\cal G}}
\newcommand{\Hc}{{\cal H}}
\newcommand{\Ic}{{\cal I}}
\newcommand{\Jc}{{\cal J}}
\newcommand{\Kc}{{\cal K}}
\newcommand{\Lc}{{\cal L}}
\newcommand{\Mc}{{\cal M}}
\newcommand{\Nc}{{\cal N}}
\newcommand{\nc}{{\cal n}}
\newcommand{\Oc}{{\cal O}}
\newcommand{\Pc}{{\cal P}}
\newcommand{\Qc}{{\cal Q}}
\newcommand{\Rc}{{\cal R}}
\newcommand{\Sc}{{\cal S}}
\newcommand{\Tc}{{\cal T}}
\newcommand{\Uc}{{\cal U}}
\newcommand{\Wc}{{\cal W}}
\newcommand{\Vc}{{\cal V}}
\newcommand{\Xc}{{\cal X}}
\newcommand{\Yc}{{\cal Y}}
\newcommand{\Zc}{{\cal Z}}

% Bold greek letters

\newcommand{\alphav}{\hbox{\boldmath$\alpha$}}
\newcommand{\betav}{\hbox{\boldmath$\beta$}}
\newcommand{\gammav}{\hbox{\boldmath$\gamma$}}
\newcommand{\deltav}{\hbox{\boldmath$\delta$}}
\newcommand{\etav}{\hbox{\boldmath$\eta$}}
\newcommand{\lambdav}{\hbox{\boldmath$\lambda$}}
\newcommand{\epsilonv}{\hbox{\boldmath$\epsilon$}}
\newcommand{\nuv}{\hbox{\boldmath$\nu$}}
\newcommand{\muv}{\hbox{\boldmath$\mu$}}
\newcommand{\zetav}{\hbox{\boldmath$\zeta$}}
\newcommand{\phiv}{\hbox{\boldmath$\phi$}}
\newcommand{\psiv}{\hbox{\boldmath$\psi$}}
\newcommand{\thetav}{\hbox{\boldmath$\theta$}}
\newcommand{\tauv}{\hbox{\boldmath$\tau$}}
\newcommand{\omegav}{\hbox{\boldmath$\omega$}}
\newcommand{\xiv}{\hbox{\boldmath$\xi$}}
\newcommand{\sigmav}{\hbox{\boldmath$\sigma$}}
\newcommand{\piv}{\hbox{\boldmath$\pi$}}
\newcommand{\rhov}{\hbox{\boldmath$\rho$}}
\newcommand{\upsilonv}{\hbox{\boldmath$\upsilon$}}

\newcommand{\Gammam}{\hbox{\boldmath$\Gamma$}}
\newcommand{\Lambdam}{\hbox{\boldmath$\Lambda$}}
\newcommand{\Deltam}{\hbox{\boldmath$\Delta$}}
\newcommand{\Sigmam}{\hbox{\boldmath$\Sigma$}}
\newcommand{\Phim}{\hbox{\boldmath$\Phi$}}
\newcommand{\Pim}{\hbox{\boldmath$\Pi$}}
\newcommand{\Psim}{\hbox{\boldmath$\Psi$}}
\newcommand{\Thetam}{\hbox{\boldmath$\Theta$}}
\newcommand{\Omegam}{\hbox{\boldmath$\Omega$}}
\newcommand{\Xim}{\hbox{\boldmath$\Xi$}}


% Sans Serif small case

\newcommand{\Gsf}{{\sf G}}

\newcommand{\asf}{{\sf a}}
\newcommand{\bsf}{{\sf b}}
\newcommand{\csf}{{\sf c}}
\newcommand{\dsf}{{\sf d}}
\newcommand{\esf}{{\sf e}}
\newcommand{\fsf}{{\sf f}}
\newcommand{\gsf}{{\sf g}}
\newcommand{\hsf}{{\sf h}}
\newcommand{\isf}{{\sf i}}
\newcommand{\jsf}{{\sf j}}
\newcommand{\ksf}{{\sf k}}
\newcommand{\lsf}{{\sf l}}
\newcommand{\msf}{{\sf m}}
\newcommand{\nsf}{{\sf n}}
\newcommand{\osf}{{\sf o}}
\newcommand{\psf}{{\sf p}}
\newcommand{\qsf}{{\sf q}}
\newcommand{\rsf}{{\sf r}}
\newcommand{\ssf}{{\sf s}}
\newcommand{\tsf}{{\sf t}}
\newcommand{\usf}{{\sf u}}
\newcommand{\wsf}{{\sf w}}
\newcommand{\vsf}{{\sf v}}
\newcommand{\xsf}{{\sf x}}
\newcommand{\ysf}{{\sf y}}
\newcommand{\zsf}{{\sf z}}


% mixed symbols

\newcommand{\sinc}{{\hbox{sinc}}}
\newcommand{\diag}{{\hbox{diag}}}
\renewcommand{\det}{{\hbox{det}}}
\newcommand{\trace}{{\hbox{tr}}}
\newcommand{\sign}{{\hbox{sign}}}
\renewcommand{\arg}{{\hbox{arg}}}
\newcommand{\var}{{\hbox{var}}}
\newcommand{\cov}{{\hbox{cov}}}
\newcommand{\Ei}{{\rm E}_{\rm i}}
\renewcommand{\Re}{{\rm Re}}
\renewcommand{\Im}{{\rm Im}}
\newcommand{\eqdef}{\stackrel{\Delta}{=}}
\newcommand{\defines}{{\,\,\stackrel{\scriptscriptstyle \bigtriangleup}{=}\,\,}}
\newcommand{\<}{\left\langle}
\renewcommand{\>}{\right\rangle}
\newcommand{\herm}{{\sf H}}
\newcommand{\trasp}{{\sf T}}
\newcommand{\transp}{{\sf T}}
\renewcommand{\vec}{{\rm vec}}
\newcommand{\Psf}{{\sf P}}
\newcommand{\SINR}{{\sf SINR}}
\newcommand{\SNR}{{\sf SNR}}
\newcommand{\MMSE}{{\sf MMSE}}
\newcommand{\REF}{{\RED [REF]}}

% Markov chain
\usepackage{stmaryrd} % for \mkv 
\newcommand{\mkv}{-\!\!\!\!\minuso\!\!\!\!-}

% Colors

\newcommand{\RED}{\color[rgb]{1.00,0.10,0.10}}
\newcommand{\BLUE}{\color[rgb]{0,0,0.90}}
\newcommand{\GREEN}{\color[rgb]{0,0.80,0.20}}

%%%%%%%%%%%%%%%%%%%%%%%%%%%%%%%%%%%%%%%%%%
\usepackage{hyperref}
\hypersetup{
    bookmarks=true,         % show bookmarks bar?
    unicode=false,          % non-Latin characters in AcrobatÕs bookmarks
    pdftoolbar=true,        % show AcrobatÕs toolbar?
    pdfmenubar=true,        % show AcrobatÕs menu?
    pdffitwindow=false,     % window fit to page when opened
    pdfstartview={FitH},    % fits the width of the page to the window
%    pdftitle={My title},    % title
%    pdfauthor={Author},     % author
%    pdfsubject={Subject},   % subject of the document
%    pdfcreator={Creator},   % creator of the document
%    pdfproducer={Producer}, % producer of the document
%    pdfkeywords={keyword1} {key2} {key3}, % list of keywords
    pdfnewwindow=true,      % links in new window
    colorlinks=true,       % false: boxed links; true: colored links
    linkcolor=red,          % color of internal links (change box color with linkbordercolor)
    citecolor=green,        % color of links to bibliography
    filecolor=blue,      % color of file links
    urlcolor=blue           % color of external links
}
%%%%%%%%%%%%%%%%%%%%%%%%%%%%%%%%%%%%%%%%%%%


\maketitle


%!TEX root = gcn.tex
\section{Introduction}
Graphs, representing structural data and topology, are widely used across various domains, such as social networks and merchandising transactions.
Graph convolutional networks (GCN)~\cite{iclr/KipfW17} have significantly enhanced model training on these interconnected nodes.
However, these graphs often contain sensitive information that should not be leaked to untrusted parties.
For example, companies may analyze sensitive demographic and behavioral data about users for applications ranging from targeted advertising to personalized medicine.
Given the data-centric nature and analytical power of GCN training, addressing these privacy concerns is imperative.

Secure multi-party computation (MPC)~\cite{crypto/ChaumDG87,crypto/ChenC06,eurocrypt/CiampiRSW22} is a critical tool for privacy-preserving machine learning, enabling mutually distrustful parties to collaboratively train models with privacy protection over inputs and (intermediate) computations.
While research advances (\eg,~\cite{ccs/RatheeRKCGRS20,uss/NgC21,sp21/TanKTW,uss/WatsonWP22,icml/Keller022,ccs/ABY318,folkerts2023redsec}) support secure training on convolutional neural networks (CNNs) efficiently, private GCN training with MPC over graphs remains challenging.

Graph convolutional layers in GCNs involve multiplications with a (normalized) adjacency matrix containing $\numedge$ non-zero values in a $\numnode \times \numnode$ matrix for a graph with $\numnode$ nodes and $\numedge$ edges.
The graphs are typically sparse but large.
One could use the standard Beaver-triple-based protocol to securely perform these sparse matrix multiplications by treating graph convolution as ordinary dense matrix multiplication.
However, this approach incurs $O(\numnode^2)$ communication and memory costs due to computations on irrelevant nodes.
%
Integrating existing cryptographic advances, the initial effort of SecGNN~\cite{tsc/WangZJ23,nips/RanXLWQW23} requires heavy communication or computational overhead.
Recently, CoGNN~\cite{ccs/ZouLSLXX24} optimizes the overhead in terms of  horizontal data partitioning, proposing a semi-honest secure framework.
Research for secure GCN over vertical data  remains nascent.

Current MPC studies, for GCN or not, have primarily targeted settings where participants own different data samples, \ie, horizontally partitioned data~\cite{ccs/ZouLSLXX24}.
MPC specialized for scenarios where parties hold different types of features~\cite{tkde/LiuKZPHYOZY24,icml/CastigliaZ0KBP23,nips/Wang0ZLWL23} is rare.
This paper studies $2$-party secure GCN training for these vertical partition cases, where one party holds private graph topology (\eg, edges) while the other owns private node features.
For instance, LinkedIn holds private social relationships between users, while banks own users' private bank statements.
Such real-world graph structures underpin the relevance of our focus.
To our knowledge, no prior work tackles secure GCN training in this context, which is crucial for cross-silo collaboration.


To realize secure GCN over vertically split data, we tailor MPC protocols for sparse graph convolution, which fundamentally involves sparse (adjacency) matrix multiplication.
Recent studies have begun exploring MPC protocols for sparse matrix multiplication (SMM).
ROOM~\cite{ccs/SchoppmannG0P19}, a seminal work on SMM, requires foreknowledge of sparsity types: whether the input matrices are row-sparse or column-sparse.
Unfortunately, GCN typically trains on graphs with arbitrary sparsity, where nodes have varying degrees and no specific sparsity constraints.
Moreover, the adjacency matrix in GCN often contains a self-loop operation represented by adding the identity matrix, which is neither row- nor column-sparse.
Araki~\etal~\cite{ccs/Araki0OPRT21} avoid this limitation in their scalable, secure graph analysis work, yet it does not cover vertical partition.

% and related primitives
To bridge this gap, we propose a secure sparse matrix multiplication protocol, \osmm, achieving \emph{accurate, efficient, and secure GCN training over vertical data} for the first time.

\subsection{New Techniques for Sparse Matrices}
The cost of evaluating a GCN layer is dominated by SMM in the form of $\adjmat\feamat$, where $\adjmat$ is a sparse adjacency matrix of a (directed) graph $\graph$ and $\feamat$ is a dense matrix of node features.
For unrelated nodes, which often constitute a substantial portion, the element-wise products $0\cdot x$ are always zero.
Our efficient MPC design 
avoids unnecessary secure computation over unrelated nodes by focusing on computing non-zero results while concealing the sparse topology.
We achieve this~by:
1) decomposing the sparse matrix $\adjmat$ into a product of matrices (\S\ref{sec::sgc}), including permutation and binary diagonal matrices, that can \emph{faithfully} represent the original graph topology;
2) devising specialized protocols (\S\ref{sec::smm_protocol}) for efficiently multiplying the structured matrices while hiding sparsity topology.


 
\subsubsection{Sparse Matrix Decomposition}
We decompose adjacency matrix $\adjmat$ of $\graph$ into two bipartite graphs: one represented by sparse matrix $\adjout$, linking the out-degree nodes to edges, the other 
by sparse matrix $\adjin$,
linking edges to in-degree nodes.

%\ie, we decompose $\adjmat$ into $\adjout \adjin$, where $\adjout$ and $\adjin$ are sparse matrices representing these connections.
%linking out-degree nodes to edges and edges to in-degree nodes of $\graph$, respectively.

We then permute the columns of $\adjout$ and the rows of $\adjin$ so that the permuted matrices $\adjout'$ and $\adjin'$ have non-zero positions with \emph{monotonically non-decreasing} row and column indices.
A permutation $\sigma$ is used to preserve the edge topology, leading to an initial decomposition of $\adjmat = \adjout'\sigma \adjin'$.
This is further refined into a sequence of \emph{linear transformations}, 
which can be efficiently computed by our MPC protocols for 
\emph{oblivious permutation}
%($\Pi_{\ssp}$) 
and \emph{oblivious selection-multiplication}.
% ($\Pi_\SM$)
\iffalse
Our approach leverages bipartite graph representation and the monotonicity of non-zero positions to decompose a general sparse matrix into linear transformations, enhancing the efficiency of our MPC protocols.
\fi
Our decomposition approach is not limited to GCNs but also general~SMM 
by 
%simply 
treating them 
as adjacency matrices.
%of a graph.
%Since any sparse matrix can be viewed 

%allowing the same technique to be applied.

 
\subsubsection{New Protocols for Linear Transformations}
\emph{Oblivious permutation} (OP) is a two-party protocol taking a private permutation $\sigma$ and a private vector $\xvec$ from the two parties, respectively, and generating a secret share $\l\sigma \xvec\r$ between them.
Our OP protocol employs correlated randomnesses generated in an input-independent offline phase to mask $\sigma$ and $\xvec$ for secure computations on intermediate results, requiring only $1$ round in the online phase (\cf, $\ge 2$ in previous works~\cite{ccs/AsharovHIKNPTT22, ccs/Araki0OPRT21}).

Another crucial two-party protocol in our work is \emph{oblivious selection-multiplication} (OSM).
It takes a private bit~$s$ from a party and secret share $\l x\r$ of an arithmetic number~$x$ owned by the two parties as input and generates secret share $\l sx\r$.
%between them.
%Like our OP protocol, o
Our $1$-round OSM protocol also uses pre-computed randomnesses to mask $s$ and $x$.
%for secure computations.
Compared to the Beaver-triple-based~\cite{crypto/Beaver91a} and oblivious-transfer (OT)-based approaches~\cite{pkc/Tzeng02}, our protocol saves ${\sim}50\%$ of online communication while having the same offline communication and round complexities.

By decomposing the sparse matrix into linear transformations and applying our specialized protocols, our \osmm protocol
%($\prosmm$) 
reduces the complexity of evaluating $\numnode \times \numnode$ sparse matrices with $\numedge$ non-zero values from $O(\numnode^2)$ to $O(\numedge)$.

%(\S\ref{sec::secgcn})
\subsection{\cgnn: Secure GCN made Efficient}
Supported by our new sparsity techniques, we build \cgnn, 
a two-party computation (2PC) framework for GCN inference and training over vertical
%ly split
data.
Our contributions include:

1) We are the first to explore sparsity over vertically split, secret-shared data in MPC, enabling decompositions of sparse matrices with arbitrary sparsity and isolating computations that can be performed in plaintext without sacrificing privacy.

2) We propose two efficient $2$PC primitives for OP and OSM, both optimally single-round.
Combined with our sparse matrix decomposition approach, our \osmm protocol ($\prosmm$) achieves constant-round communication costs of $O(\numedge)$, reducing memory requirements and avoiding out-of-memory errors for large matrices.
In practice, it saves $99\%+$ communication
%(Table~\ref{table:comm_smm}) 
and reduces ${\sim}72\%$ memory usage over large $(5000\times5000)$ matrices compared with using Beaver triples.
%(Table~\ref{table:mem_smm_sparse}) ${\sim}16\%$-

3) We build an end-to-end secure GCN framework for inference and training over vertically split data, maintaining accuracy on par with plaintext computations.
We will open-source our evaluation code for research and deployment.

To evaluate the performance of $\cgnn$, we conducted extensive experiments over three standard graph datasets (Cora~\cite{aim/SenNBGGE08}, Citeseer~\cite{dl/GilesBL98}, and Pubmed~\cite{ijcnlp/DernoncourtL17}),
reporting communication, memory usage, accuracy, and running time under varying network conditions, along with an ablation study with or without \osmm.
Below, we highlight our key achievements.

\textit{Communication (\S\ref{sec::comm_compare_gcn}).}
$\cgnn$ saves communication by $50$-$80\%$.
(\cf,~CoGNN~\cite{ccs/KotiKPG24}, OblivGNN~\cite{uss/XuL0AYY24}).

\textit{Memory usage (\S\ref{sec::smmmemory}).}
\cgnn alleviates out-of-memory problems of using %the standard 
Beaver-triples~\cite{crypto/Beaver91a} for large datasets.

\textit{Accuracy (\S\ref{sec::acc_compare_gcn}).}
$\cgnn$ achieves inference and training accuracy comparable to plaintext counterparts.
%training accuracy $\{76\%$, $65.1\%$, $75.2\%\}$ comparable to $\{75.7\%$, $65.4\%$, $74.5\%\}$ in plaintext.

{\textit{Computational efficiency (\S\ref{sec::time_net}).}} 
%If the network is worse in bandwidth and better in latency, $\cgnn$ shows more benefits.
$\cgnn$ is faster by $6$-$45\%$ in inference and $28$-$95\%$ in training across various networks and excels in narrow-bandwidth and low-latency~ones.

{\textit{Impact of \osmm (\S\ref{sec:ablation}).}}
Our \osmm protocol shows a $10$-$42\times$ speed-up for $5000\times 5000$ matrices and saves $10$-2$1\%$ memory for ``small'' datasets and up to $90\%$+ for larger ones.

\section{Related Work}

\subsection{Advancements in AI and Agentic Workflows for Code Generation}

Since the training process of GPT-3.5 incorporated a substantial amount of code data to enhance the logical reasoning capabilities of language models \cite{chenEvaluatingLargeLanguage2021}, code generation has become closely intertwined with language modeling. With the emergence of models that place a stronger emphasis on reasoning, these capabilities continue to evolve. According to the SWE-bench benchmark, which simulates human programmers' problem-solving workflows, AI programming performance increased from below 2\% in December 2023 \cite{jimenez2024swebench} to over 60\% by February 2025\footnote{https://www.swebench.com/}.

However, simply reinforcing the reasoning ability of language models primarily advances lower-level software development tasks such as auto-completion and refactoring. To enhance automation in real-world software and system development, researchers have introduced various agentic workflows, including OpenHands \cite{openhands}, an open-source coding agent designed for end-to-end development, and Agent Company, which simulates the operation of a software company \cite{xu2024theagentcompany}. Nonetheless, as of February 2025, even the most sophisticated agentic workflows remain unable to fully realize end-to-end programming\footnote{https://www.swebench.com/}, let alone organization-level agency\footnote{https://the-agent-company.com/}. 

Within code generation and system development, front-end code generation—such as website development—often demonstrates stronger performance than back-end development. Research in this domain has examined reconstructing HTML/CSS structures from UI screenshots using computer vision techniques \cite{soseliaLearningUItoCodeReverse2023}, implementing hierarchical decomposition strategies for interface elements to optimize UI code generation \cite{wanAutomaticallyGeneratingUI2024}, and improving model specialization through domain-specific fine-tuning for UI generation \cite{wuUICoderFinetuningLarge2024}. To systematically evaluate front-end code generation, specialized benchmarks have been developed to assess the quality of HTML, CSS, and JavaScript implementations \cite{siDesign2CodeHowFar2024}. To investigate the societal impact of this notable improvement in AI programming capabilities, we focus on the task of website generation, where current AI systems are relatively close to achieving near end-to-end automation.

\subsection{Beyond Templates: AI-Powered, User-Centric UI}

With the continuing development of AI-driven user interface (UI) generation, users increasingly seek more personalized and diverse expressions rather than relying solely on conventional template reuse. Recent advances have led to adaptive UI generation systems like FrameKit, which allows end users to manually design keyframes and generate multiple interface variants \cite{wu_framekit_2024}. PromptInfuser goes a step further by enabling runtime dynamic input and generation of UI content \cite{petridisPromptInfuserHowTightly2024}. In this context, AI tools have been shown to offer inspiration for professional designers \cite{luBridgingGapUX2022}. For instance, DesignAID \cite{cai_designaid_2023} demonstrates that generative AI can provide conceptual directions and stimulate creativity at early design stages. Misty supports remixing concepts by allowing users to blend example images with the current UI, thereby enabling flexible conceptual exploration \cite{luMistyUIPrototyping2024}.

Beyond offering inspiration, AI can also provide real-time design feedback to guide iterative refinement and error correction \cite{duan_towards_2023}, such as handling CSS styling in simple websites and optimizing specific UI components \cite{liUsingLLMsCustomize2023}. It is capable of evaluating UI quality and relevance, offering suggestions at various design stages \cite{wuUIClipDatadrivenModel2024}, and even detecting potential development or UI issues in advance \cite{petridisPromptInfuserHowTightly2024}. Automated heuristic evaluations generated by AI can provide more precise assessments and recommendations, thereby streamlining the iterative process \cite{duanGeneratingAutomaticFeedback2024}. When combined with traditional heuristic rules, AI has been shown to increase the effectiveness of UI error detection and correction \cite{lu_ai_2024}. Integrating prototype-checking techniques into the UI generation workflow can further enhance automatic repair capabilities \cite{xiaoPrototype2CodeEndtoendFrontend2024}.

\subsection{Improving the Creative Workflow with AI}

In many creativity workflows, a prolonged progression from ideation, prototyping, and development to iteration is required \cite{palaniEvolvingRolesWorkflows2024}. Those creative processes are frequently constrained by multiple intricate steps that limit users' expressive capabilities. For example, the complexity and associated costs of developing a personal website often deter individuals from undertaking this process, prompting many to resort to standardized website templates for personal websites. However, GenAI can assist with the creativity workflow from various angles \cite{wanItFeltHaving2024,palaniEvolvingRolesWorkflows2024,longNotJustNovelty2024}. First, GenAI such as text-to-image generation can reduce the time needed to produce high-fidelity outcomes. This enables creators to focus on refining the gap between the high-fidelity results and their envisioned expectations, rather than expending effort on how to achieve high fidelity in the first place \cite{edwardsSketch2PrototypeRapidConceptual2024}. Besides, AI lowers the cost of experimenting with new ideas, thereby minimizing the psychological barriers to conducting trial and error with unconventional concepts \cite{palaniEvolvingRolesWorkflows2024}. When users are uncertain about what they want or have only a broad concept lacking specific details, AI can offer inspiration \cite{rickSupermindIdeatorExploring2023}. Moreover, AI can facilitate parallel prototyping by presenting multiple design directions simultaneously, allowing creators to compare and refine a range of diverse design solutions \cite{dowParallelPrototypingLeads2010}.
\section{System Framework}


%Our system's objective is to develop a human-robot interaction simulator for the study of embodied intelligence. It is characterized by its ability to provide real-time and authentic feedback during the simulation process. For human-robot interaction, this feedback enables bidirectional continuing learning. On one hand, it allows humans to adapt to interacting with robots and improves their ability to engage with them. On the other hand, it continuously enhances the intelligent interaction capabilities of robots, making their performance more natural and realistic.

%To facilitate real-time and authentic feedback for human-robot interactions, the key to our system lies in the \emph{human-in-the-loop simulation platform}, which is manifested in three aspects:

%Firstly, our system simulates human-robot interaction. As different robots have diverse physical structures and degrees of freedom, human actions can be retargeted to robots in the simulation environment, teaching robots to interact with humans.

%Secondly, our system enables real-time and realistic human-robot interaction. Specifically, human motions are captured in real time and fed into a foundational interactive model to generate corresponding robot actions, which are adjusted in a physical simulation environment. Augmented reality technology integrates the dynamic virtual robot into the real environment, to provide natural and realistic human-robot interaction experiences.

%Thirdly, our system collects instant human feedback during interactions to further enhance robot interaction capabilities. When interacting with virtual robots, human ratings, speeches, and facial expressions are recorded. This feedback is used to organize interaction data and fine-tune the foundational interactive model.

To facilitate human-in-the-loop robotic simulation for bidirectional continuous learning in human-robot interaction, SymbioSim consists of six key modules, as illustrated in Fig.~\ref{fig:system}. In Section~\ref{sec:components}, we first provide a detailed implementation of each component, followed by a discussion in Section~\ref{sec:capabilities} on how these components ensure the system's capabilities in terms of realism, real-time performance, and adaptability.

\subsection{System Components}
\label{sec:components}

\textbf{Motion Capture} For pose-level human-robot interaction in 3D space, accurate and efficient human motion capture is necessary. Because LiDAR can provide precise depth information and is not affected by light conditions. We employ a LiDAR device~\cite{OusterLidar2022} to scan 3D point clouds and estimate human motions in the scene. The human motion is represented as the pose parameters of the SMPL model~\cite{SMPL}, which describes the global transformation and joint angles of the human body. To achieve this, we utilize LiveHPS++~\cite{ren2025livehps++}, a portable, highly flexible, and easy-to-configure system designed for capturing human motion across a wide range of activities. It is capable of processing the scanned point clouds in real-time, providing high-quality motion data for further analysis or integration. 
The motion capture system captures a frame of human motion every 0.1 second.

\textbf{Interactive Model} The interactive model is responsible for providing realistic robot actions based on the captured human motion. The human motion is first converted into joint positions of the human body skeleton with a forward kinematic algorithm. Then, for every frame, the interactive model takes historical and current human poses and robot poses, as well as the user's action command such as "handshake" as input, and outputs a sequence of the predicted robot joint skeletons, which are then converted to the robot pose representation via an inverse kinematics solver. The technical details of the interactive model are described in Section~\ref{sec:interactivemodel}. 
It is worth noting that we provide this foundational model to evaluate the effectiveness of SymbioSim, and it can be easily replaced with any other human-robot interaction models that users expect to test. In this paper, we also explore an alternative approach by replacing the interactive model with a retargeting model, which maps captured human motion to corresponding robot poses, allowing the virtual robot to mimic human movements. 
This module operates at a stable frequency, synchronized with the motion capture system.

\textbf{Physical Simulation} 
% We employ the physics-based constrained optimization approach presented in~\cite{luo2023perpetual} to compute the robot motion sequences within a physical simulation environment.
We employ the deep reinforcement learning technique presented in~\cite{luo2023perpetual} to control simulated robots in a physical simulation environment, i.e., IssacGym~\cite{makoviychuk2021isaac}. The objective is to achieve maximum alignment between predicted robot poses with the real robot poses (also as known as motion tracking for physical agents) while maintaining physical plausibility. It is able to produce real-time and high-fidelity robot actions in the physical simulation environment, enhancing the authentic interaction experience and reducing the sim-real gaps. Specifically, the physics-based humanoid controller samples the predicted robot poses and outputs the simulated action at a frequency of 50FPS to ensure the fluency and continuity of robot movements for a natural interaction experience.

\textbf{AR System} The AR system aims to provide a realistic visualization of the interaction between the real human and virtual robot. To facilitate a high-fidelity interaction experience for the human user, we employ an AR device, i.e. PICO4 Ultra~\cite{PicoInteractive2022}, to visualize virtual robots from the first-person perspective. It communicates with the physical simulation module through TCP protocol to update robot poses, which are seamlessly integrated into the real environment background. In addition, we also provide the visualization of the third-person perspective to show the panoramic view of the interaction between the real human and the virtual robot. To achieve this, we set up an RGBD camera and calibrated its intrinsic and extrinsic parameters. After that, we render the virtual robot from the calibrated viewpoint and fuse it with the image captured by the camera based on their depth maps.
 
\textbf{Human Feedback} 
We collect user feedback throughout their interactions with the virtual robot to improve the system's adaptability. Users can provide feedback in various forms, such as direct ratings or detailed language descriptions of their interaction experience. Additionally, implicit feedback can be captured through users' actions, which may reflect their comfort level, engagement, or preferences during the interaction. The collected feedback, combined with the recorded human-robot interaction data, forms comprehensive human-in-the-loop realistic data for fine-tuning the foundational interactive model.

\textbf{Model Fine-tuning}
Based on user feedback, we classify the interaction data into positive and negative samples. Positive samples represent behaviors or interaction methods that are well-received by users, while negative samples correspond to interactions that users find unsatisfactory or unnatural. Using these samples, the system leverages a supervised learning paradigm to continuously refine and adjust the interaction model by fine-tuning the model on positive samples. This allows the robot to better align with user needs and preferences. Through this closed-loop feedback mechanism, the robot progressively improves its interaction quality, enabling more natural and efficient collaboration with humans.



\subsection{System Capabilities}
\label{sec:capabilities}
%\manyi{Design Principle as subtitle?}
Aiming to achieve human-robot symbiosis, SymbioSim is the first human-robot interaction closed-loop simulation platform, seamlessly integrating both virtual and real environments. It provides a robust foundation for the rapid and safe development, validation, and optimization of human-robot interaction algorithms. SymbioSim benefits the learning and improvement for both humans and robots. On one hand, through authentic interactive scenarios, humans can gradually understand the robot’s intentions, develop effective collaboration and getting along strategies, and ultimately integrate the robot more naturally into daily life. On the other hand, SymbioSim facilitates expansion to diverse testing scenarios while continuously collecting human feedback, enabling real-time adjustments and evolution of the robot's models. This allows for swift iterations, aligning robot's behavior with human preferences and habits. In the end, SymbioSim supports mutual adaptation and continuous learning between humans and robots, achieving the goal of human-robot symbiosis in a safe, efficient, and cost-effective manner. To enhance user experience and reduce the gap between simulation and reality, the advantages of SymbioSim system are reflected in three aspects: realism, real-time capability, and adaptability.


\textbf{Realism}
By leveraging AR technology, we seamlessly integrate virtual robots into real-world environments, providing users with an authentic interaction experience. Furthermore, we build upon human-human interaction data to transfer human behaviors and habits onto humanoid robots, enabling their actions to transcend the traditional boundaries between humans and machines. This fosters more natural and intuitive human-robot interactions, enhancing the sense of realism. Additionally, we apply physics-based constrained optimization to ensure that the robot’s behaviors remain consistent with its inherent physical constraints. This approach effectively bridges the gap between simulation and reality, enabling the effective deployment of simulation results onto real-world robots.

\textbf{Real-time Capability}
Real-time performance is essential for establishing a closed-loop feedback system between humans and robots. To achieve this, we utilize cutting-edge LiDAR-based 3D motion capture technology, enabling real-time perception for the robot. Moreover, we have developed an innovative interactive motion generation model that facilitates real-time regulation and control of the robot’s actions. The real-time execution of core algorithms, coupled with efficient inter-module data transmission, ensures the overall system’s responsiveness and reliability in real-time operation.


\textbf{Adaptability} SymbioSim not only facilitates the testing of existing models but also integrates user evaluations and feedback. These insights are automatically collected to fine-tune and optimize the model, allowing the robot to become progressively more adaptive and efficient. Simultaneously, users can make adaptive adjustments throughout their interaction, resulting in an increasingly refined experience.

\section{Foundational Interactive Model}
\label{sec:interactivemodel}
% \begin{figure*}[]
%   \includegraphics[width=0.8\linewidth]{images/ActModel.drawio.pdf}
%   \caption{Foundational Interactive Model.}
% \end{figure*}
\begin{figure}
\centering
\includegraphics[width=0.95\linewidth]{images/InteractionModel.pdf}
\vspace{-5pt}
\caption{The network architecture of our human-robot interactive model.}
\label{fig:interactiveModel}
\vspace{-10pt}
\end{figure}


% \iffalse
% \begin{table*}[]
% \centering
% \caption{Ablation studies on Foundational Interactive Model.}
% \resizebox{\linewidth}{!}{ 
% \begin{tabular}{ccc|cccccc|cccccc}
% \hline
% \multicolumn{3}{c|}{Constraint}                                & \multicolumn{6}{c|}{Unitree H1 robot}                                                                                                                                                                           & \multicolumn{6}{c}{LEJU Kuavo robot}                                                                                                                                                                      \\ \hline
% \multicolumn{1}{c|}{Pos} & \multicolumn{1}{c|}{Vel} & Interact & \multicolumn{1}{c|}{Contact}        & \multicolumn{1}{c|}{FID}             & \multicolumn{1}{c|}{Traj}           & \multicolumn{1}{c|}{Orientation}    & \multicolumn{1}{c|}{Relation}         & MPJPE          & \multicolumn{1}{c|}{Contact}        & \multicolumn{1}{c|}{FID}             & \multicolumn{1}{c|}{Traj}           & \multicolumn{1}{c|}{Orientation}    & \multicolumn{1}{c|}{Relation}         & MPJPE          \\ \hline
% \multicolumn{1}{c|}{$\checkmark$}   & \multicolumn{1}{c|}{$\times$}   & $\times$        & \multicolumn{1}{c|}{0.17}           & \multicolumn{1}{c|}{374.13}          & \multicolumn{1}{c|}{0.302}          & \multicolumn{1}{c|}{40.88}          & \multicolumn{1}{c|}{0.00051}          & 0.308          & \multicolumn{1}{c|}{0.02}           & \multicolumn{1}{c|}{432.25}          & \multicolumn{1}{c|}{0.433}          & \multicolumn{1}{c|}{58.8}           & \multicolumn{1}{c|}{0.00034}          & 0.444          \\ \hline
% \multicolumn{1}{c|}{$\checkmark$}   & \multicolumn{1}{c|}{$\checkmark$}   & $\times$        & \multicolumn{1}{c|}{0.034}          & \multicolumn{1}{c|}{\textbf{229.67}} & \multicolumn{1}{c|}{\textbf{0.176}} & \multicolumn{1}{c|}{\textbf{29.13}} & \multicolumn{1}{c|}{0.00051}          & \textbf{0.198} & \multicolumn{1}{c|}{\textbf{0.025}} & \multicolumn{1}{c|}{\textbf{228.08}} & \multicolumn{1}{c|}{\textbf{0.179}} & \multicolumn{1}{c|}{\textbf{29.17}} & \multicolumn{1}{c|}{0.00034}          & \textbf{0.193} \\ \hline
% \multicolumn{1}{c|}{$\checkmark$}   & \multicolumn{1}{c|}{$\checkmark$}   & $\checkmark$        & \multicolumn{1}{c|}{\textbf{0.022}} & \multicolumn{1}{c|}{251.61}          & \multicolumn{1}{c|}{0.194}          & \multicolumn{1}{c|}{31.94}          & \multicolumn{1}{c|}{\textbf{0.00051}} & 0.213          & \multicolumn{1}{c|}{0.026}          & \multicolumn{1}{c|}{240.06}          & \multicolumn{1}{c|}{0.193}          & \multicolumn{1}{c|}{31.75}          & \multicolumn{1}{c|}{\textbf{0.00034}} & 0.204          \\ \hline
% \end{tabular}
% }
% \end{table*}
% \fi

%\begin{table*}[]
%\centering
%\caption{Ablation Studies on Interx-HRI.}
%\resizebox{\linewidth}{!}{ 
%\begin{tabular}{c|cl|cccccccccc}
%\hline
%$L_{\text{pos}} $          & \multicolumn{2}{c|}{$L_{\text{vel}}$}          & PA-MPJPE↓      & MPJPE↓         & C\_prec↑       & C\_rec↑        & Acc↑           & F1↑            & Traj↓          & Orie↓          & FID↓          & R-score↑      \\ \hline 
%$\checkmark$ & \multicolumn{2}{c|}{$\times$}     & 3.958          & 14.32          & 0.243          & 0.326          & 0.763          & 0.278          & 0.302          & 40.79          & 0.60          & 0.40          \\ \hline
%$\times$     & \multicolumn{2}{c|}{$\checkmark$} & 3.194          & 12.04          & 0.360          & 0.357          & 0.821          & 0.359          & 0.206          & 32.04          & 0.34          & 0.46          \\ \hline
%$\checkmark$ & \multicolumn{2}{c|}{$\checkmark$} & \textbf{2.874} & \textbf{10.51} & \textbf{0.375} & \textbf{0.380} & \textbf{0.825} & \textbf{0.377} & \textbf{0.176} & \textbf{29.13} & \textbf{0.29} & \textbf{0.48} \\ \hline
%\end{tabular}
%}
%\label{tab:ab-hhi}
%\end{table*}


We propose a foundational human-robot interactive model, which autoregressively generates humanoid robot motion, ensuring reasonable and consistent real-time interaction with humans. Fig.~\ref{fig:interactiveModel} illustrates the pipeline of our model.

%by autoregressively utilizing past and present interaction data through maintaining a rollout queue and applying a series of optimization designs.


\subsection{Model Input and Output}
The interactive model receives three inputs at current time $t$: historical robot motion skeletons $J^R_{t-n \sim t}$, observed human motion skeletons $J^H_{t-n \sim t}$, and user's action command $T^A$.
Specifically, $J^R_{t-n \sim t}$ comes from the model output stored in the robot's rollout queue at time \( t-1 \), $J^H_{t-n \sim t}$ is derived from human motion capture data $P^H_{t-n \sim t}$ represented by SMPL~\cite{SMPL}, which is then processed through a forward kinematic module, and \( T^A \) represents a user-specified phrase, such as ``High-five'' or ``Handshake''.
The output is the predicted robot motion sequences $J^R_{t+1 \sim t+k}$ in future $k$ frames, because sequence supervision can benefit the consistency of predicted robot motion. However, to ensure timely response, we only feed $J^R_{t+1}$ to the following physical simulation module and store it in the robot's rollout queue. In our experiment, we set $n=10$ and $k=10$.
% \begin{enumerate}
%     \item Robot joint skeleton $J^R_t$;
%     \item Human joint skeleton $J^H_t$;
%     \item User's action command $T^A$,
% \end{enumerate}

\subsection{Real-time Interactive Feature Learning}

%\paragraph{\textbf{History Information Supplementary}}
%First, $J^R_t$ and $J^H_t$ are concatenated with recent historical data from the rollout queues to form continuous motion sequences, $J^R_{t-n \sim t}$ and $J^H_{t-n \sim t}$, respectively. 
%The role of rollout queues is to generate temporally consistent future motions by maintaining the continuous sequences that includes recent historical information.



\noindent{\textbf{Feature Encoder}}$\ $
% $F^R_{t-n \sim t}$ and $F^H_{t-n \sim t}$
Both human and robot motion sequences are encoded by convolution and attention-based hybrid network blocks for feature extraction. Meanwhile, $T^A$ is processed using a pre-trained CLIP model~\cite{radford2021learning} to obtain text embeddings $F^A$, which are promptly integrated into $F^R_{t-n \sim t}$ and $F^H_{t-n \sim t}$ by cross-attention blocks to guide the robot performing user-specified interactive action type. The Robot Encoder and Human Encoder are repeatedly stacked for $L$ times, and we set $L=3$:
\begin{equation}
F^A = CLIP(T^A),
\end{equation}
\begin{equation}
F^R_{t-n \sim t} = \{ \textbf{Robot Encoder}(J^R_{t-n \sim t}, F^A)\}_{\times L},
\end{equation}
\begin{equation}
F^H_{t-n \sim t} = \{ \textbf{Human Encoder}(J^H_{t-n \sim t}, F^A)\}_{\times L}.
\end{equation}
\noindent{\textbf{Human-robot Feature Interaction}}$\ $
To enable the robot to adapt to changes of human position and posture for effective human-robot interaction, $F^H_{t-n \sim t}$ is integrated with $F^R_{t-n \sim t}$ through Human-Robot Feature Interaction Module (HR Interaction), which contains
$M$ stacked self-attention and cross-attention blocks:
\begin{equation}
F'^R_{t+1 \sim t+k} = \{ \textbf{HR Interaction}(F^R_{t-n \sim t}, F^H_{t-n \sim t})\}_{\times M},
\end{equation}
where $M=2$, $F^{'R}_{t+1 \sim t+k}$ is the generated future robot motion features containing interactive representations.


\noindent{\textbf{Robot Motion Decoder}}$\ $
After the integration of human-robot interactive information, $F^{'R}_{t+1 \sim t+k}$ is then decoded into the future robot joint skeleton $J^R_{t+1 \sim t+k}$ by the Robot Decoder, which is similar to Robot Encoder:
\begin{equation}
J^R_{t+1 \sim t+k} = \{ \textbf{Robot Decoder}(F'^R_{t+1 \sim t+k}, F^A)\}_{\times L},
\end{equation}

% \subsection{Real-time Output}
\noindent{\textbf{Rollout Strategy}} To ensure real-time responsiveness, we maintain only $J^R_{t+1}$ from the set $J^R_{t+1 \sim t+k}$ as the output of the interactive model. This output is then pushed to the robot's rollout queue as the input for the subsequent $t+1$ frame. Different humanoid robots have their own unique joint representation $J^R_{t+1}$. It is then transformed into the specific robot's pose, $P^R_{t+1}$, via a dedicated Inverse Kinematic Solver (IK-Solver) from LiveHPS++~\cite{ren2025livehps++}, and subsequently used as the input for the following robot simulation.


%\begin{equation}
%P^R_{t+1} = \textbf{IK-Solver}(J^R_{t+1}).
%\end{equation}

\subsection{Training Strategy}
% \paragraph{\textbf{Training Strategy}}$\ $
In order to generate consistent future robot motion without jitteriness or sliding, our interactive model predicts the displacement of each skeleton joint from the previous frame to the current frame (i.e. Joint Velocity, $Vel^R$), rather than directly predicting the joint position of the current frame (Joint Position, $J^R$). 
We apply $L_{vel}$, which is the $l_2$ loss to constraint Joint Velocity between the predicted result and ground truth:
\begin{equation}
    L_{\text{vel}}  = \left\| Vel_{t+1 \sim t+k}^{\text{pred}}- Vel_{t+1 \sim t+k}^{\text{gt}} \right\|_2.
\end{equation}
Then, we sequentially accumulate $Vel_{t+1 \sim t+k}^{\text{pred}}$ from the initial frame's position to obtain the predicted Joint Position $P_{t+1 \sim t+k}^{\text{pred}}$ of each frame.
Besides, to prevent the accumulation of errors caused by the sequential summation, we also apply $L_{pos}$ to the Joint Position:
\begin{equation}
    L_{\text{pos}}  = \left\| P_{t+1 \sim t+k}^{\text{pred}} - P_{t+1 \sim t+k}^{\text{gt}} \right\|_2,
\end{equation}
Finally, the total training loss is:
\begin{equation}
    L = L_{\text{vel}} + L_{\text{pos}}.
\end{equation}









% \subsection{Initialization}
% The Foundational Interactive Model initializes the position and orientation of the robot based on those of the person in the first frame, ensuring that the human and robot face each other.
% Then the initial state of both person and robot will be replicated $n-1$ times to fill the rollout queue, which aims to generate temporally consistent future motions by maintaining a continuous sequence that includes recent historical information.
\subsection{Robot-object Interaction}
In addition to human-robot interaction, robot-object interaction is also a crucial skill for robots to live and work alongside humans, such as collaborating with a person to carry a box together. Therefore, building on the human-robot interaction model, we have also extended it to incorporate a robot-object interaction model. For Robot-Object Interaction task, the input consists of the observed object point cloud at the current time step \( t \), represented as \( \mathbf{P}_t \in \mathbb{R}^{N \times 3} \), where \( N \) is the number of points, and the output is responsive robot motion. To encode the object geometry, we adopt the Basis Point Set (BPS) representation~\cite{prokudin2019efficient}, following the approach in~\cite{li2023object}. A three-layer MLP is then used to encode the point cloud features.
For the remaining feature interaction blocks and robot control decoding, the model shares the same framework with the human-robot interactive model. 

Specifically, in addition to predicting the future robot's joint skeleton position \( J^R \) and velocity \( Vel^R \), the robot needs to approach the object and ensure that its hands make close contact with it. To achieve this, we further introduce an interactive decoder branch that predicts the future position of the object. We apply $L_{obj}$, which is the $l_2$ loss to minimize the discrepancy between the predicted object future position and the ground truth future position. Additionally, we compute the $L_{interactive}$  which measures the distance between the human hand's position $\mathbf{h}_t^{\text{pred}}$ and the closest point on the object surface $\mathbf{P}_t^{\text{surface}}$. This encourages the human hand to remain in close proximity to the object during interactions. The final loss function is a combination of these terms:

$$
L_{\text{obj}} = \left\| \mathbf{Obj}_{t+1 \sim t+k}^{\text{pred}} - \mathbf{Obj}_{t+1 \sim t+k}^{\text{gt}} \right\|_2,
$$
$$
L_{\text{interactive}} = \left\| \mathbf{h}_{t+1 \sim t+k}^{\text{pred}} - \mathbf{P}_{t+1 \sim t+k}^{\text{surface}} \right\|_2,
$$
$$
L = L_{\text{pos}} + L_{\text{vel}} + L_{\text{interactive}} + L_{\text{obj}}.
$$




\begin{table*}[]
\centering
\caption{Comparison results of human-robot interaction on Inter-HRI.}
\resizebox{\linewidth}{!}{ 
\begin{tabular}{c|c|ccccccccccc}
\hline
Robot Type                      & Methods                                                & PA-MPJPE↓     & MPJPE↓         & C\_prec↑       & C\_rec↑        & Acc↑           & F1↑            & Traj↓          & Orie↓          & FID↓          & R-score↑      & AITS↓          \\ \hline
\multirow{5}{*}{Unitree H1} & JRT~\cite{xu2023joint}          & 3.47          & 12.66          & 0.291          & 0.341          & 0.791          & 0.314          & 0.473          & 55.14          & 0.76          & 0.38          & 0.080          \\ \cline{2-13} 
                            & MRT~\cite{wang2021multi}         & 3.16          & 11.69          & 0.358          & 0.368          & 0.819          & 0.363          & 0.224          & 34.96          & 0.37          & 0.44          & 0.035          \\ \cline{2-13} 
                            & ReGenNet~\cite{xu2024regennet}   & 3.91          & 14.82          & 0.251          & 0.269          & 0.785          & 0.260          & 0.292          & 48.49          & 0.55          & 0.40          & 0.934          \\ \cline{2-13} 
                            & SAST~\cite{mueller2024massively} & 5.21          & 19.82          & 0.239          & 0.316          & 0.764          & 0.272          & 0.380          & 42.90          & 0.74          & 0.40          & 14.852         \\ \cline{2-13} 
                            & Ours                                                   & \textbf{2.87} & \textbf{10.51} & \textbf{0.375} & \textbf{0.380} & \textbf{0.825} & \textbf{0.377} & \textbf{0.176} & \textbf{29.13} & \textbf{0.29} & \textbf{0.48} & \textbf{0.023} \\ \hline
\multirow{5}{*}{LEJU Kuavo} & JRT~\cite{xu2023joint}           & 2.66          & 10.51          & 0.231          & 0.339          & 0.843          & 0.275          & 0.284          & 46.28          & 0.40          & 0.41          & 0.083          \\ \cline{2-13} 
                            & MRT~\cite{wang2021multi}         & 2.61          & 10.26          & 0.261          & 0.362          & 0.853          & 0.304          & 0.227          & 33.76          & 0.40          & 0.44          & 0.035          \\ \cline{2-13} 
                            & ReGenNet~\cite{xu2024regennet}   & 3.95          & 14.00          & 0.158          & 0.255          & 0.814          & 0.195          & 0.285          & 51.53          & 0.65          & 0.40          & 0.934          \\ \cline{2-13} 
                            & SAST~\cite{mueller2024massively} & 4.81          & 20.22          & 0.153          & 0.346          & 0.774          & 0.213          & 0.374          & 43.52          & 0.84          & 0.40          & 14.852         \\ \cline{2-13} 
                            & Ours                                                   & \textbf{2.36} & \textbf{9.34}  & \textbf{0.287} & \textbf{0.389} & \textbf{0.861} & \textbf{0.330} & \textbf{0.179} & \textbf{29.14} & \textbf{0.31} & \textbf{0.47} & \textbf{0.023} \\ \hline
\end{tabular}
}
\label{tab:hhi}
\end{table*}



\begin{table*}[]
\centering
\caption{Comparison results of robot-object interaction on Inter-ROI.}
\resizebox{\linewidth}{!}{ 
\begin{tabular}{c|c|ccccccccccc}
\hline
Robot Type                      & Methods                                                & PA-MPJPE↓     & MPJPE↓         & C\_prec↑       & C\_rec↑        & Acc↑           & F1↑            & Traj↓          & Orie↓          & FID↓          & R-score↑      & AITS↓             \\ \hline
\multirow{4}{*}{Unitree H1} & InterDiff~\cite{xu2023interdiff}                                              & 6.27           & 23.36          & 0.463          & 0.214          & 0.435          & 0.280          & 0.627          & 60.27         & 3.99          & 0.28           & 19.000             \\ \cline{2-13} 
& MRT~\cite{wang2021multi}         & 5.99           & 21.94          & 0.559          & \textbf{0.332}          & 0.540           & \textbf{0.361} & \textbf{0.315}  & \textbf{29.62} & 3.08          & 0.49           & \textbf{0.033} \\ \cline{2-13} 
& SAST~\cite{mueller2024massively} &  7.44              &        29.69        &    0.335            &  0.129              &       0.489         &    0.162           & 0.631               &     45.48          & 3.98          & \textbf{0.56}  & 14.896         \\ \cline{2-13} 
& Ours                                                   & \textbf{4.43} & \textbf{20.05} & \textbf{0.592} & 0.253 & \textbf{0.540}  & 0.320         & 0.486          & 53.79         & \textbf{2.02} & 0.36           & 0.035          \\ \hline
\multirow{4}{*}{LEJU Kuavo} & InterDiff~\cite{xu2023interdiff}                                              & 6.34           & 23.14          & 0.184          & 0.109          & 0.710           & 0.112         & 0.684          & 64.71          & 3.95          & 0.25          & 19.000             \\ \cline{2-13} 
& MRT ~\cite{wang2021multi}         & 5.80            & 22.10           & 0.161          & 0.100          & 0.733          & \textbf{0.170} & \textbf{0.330} & \textbf{33.61} & 2.49          & 0.39          & \textbf{0.033} \\ \cline{2-13} 
& SAST~\cite{mueller2024massively}  &    7.06            &     27.82           &    0.123            &    0.061           &   0.720             &     0.062          &  0.642              &    50.90           & 4.91          & \textbf{0.45} & 14.890         \\ \cline{2-13} 
& Ours                                                   & \textbf{4.04}  & \textbf{17.72} & \textbf{0.225} & \textbf{0.126} & \textbf{0.734} & 0.136         & 0.427          & 50.03         & \textbf{1.17} & 0.34          & 0.035          \\ \hline
\end{tabular}}
\label{tab:hoi}
\end{table*}


\begin{table*}[]
\centering
\caption{Ablation studies on Inter-HRI and Inter-ROI.}
\resizebox{\linewidth}{!}{ 
\begin{tabular}{c|c|c|c|c|cccccccccc}
\hline
Dataset  & 
$L_{\text{pos}} $          & $L_{\text{vel}}$          & $L_{\text{interactive}}$     & $L_{\text{obj}}$   &  PA-MPJPE↓     & MPJPE↓         & C\_prec↑       & C\_rec↑        & Acc↑           & F1↑            & Traj↓          & Orie↓          & FID↓          & R-score↑         \\ \hline \multirow{2}{*}{Inter-HRI} & 
$\checkmark$ &  & &   & 3.96          & 14.32          & 0.243          & 0.326          & 0.763          & 0.278          & 0.302          & 40.79          & 0.60          & 0.40           
\\ \cline{2-15}  & 
$\checkmark$ & $\checkmark$ & &&\textbf{2.87} & \textbf{10.51} & \textbf{0.375} & \textbf{0.380} & \textbf{0.825} & \textbf{0.377} & \textbf{0.176} & \textbf{29.13} & \textbf{0.29} & \textbf{0.48} 
\\ \hline \hline \multirow{4}{*}{Inter-ROI} & 
$\checkmark$ &              &              &              & 4.63           & 20.62          & 0.572           & 0.242           & 0.531         & 0.300           & \textbf{0.461} & 53.51          & 2.69          & 0.35          \\ \cline{2-15} & 
$\checkmark$ & $\checkmark$ &              &              & 4.62           & 20.92          & 0.561          & 0.251          & 0.531         & 0.308         & 0.476          & 53.68         & \textbf{1.92} & 0.37          \\ \cline{2-15} & 
$\checkmark$ & $\checkmark$ & $\checkmark$ &              & 4.45          & 20.72          & 0.579          & 0.242          & 0.533         & 0.308         & 0.495          & \textbf{53.30} & 2.14          & \textbf{0.38} \\ \cline{2-15}  & 
$\checkmark$ & $\checkmark$ & $\checkmark$ & $\checkmark$ & \textbf{4.43} & \textbf{20.05} & \textbf{0.592} & \textbf{0.253} & \textbf{0.54} & \textbf{0.320} & 0.486          & 53.79         & 2.02          & 0.36          \\ \hline
\end{tabular}
}
\label{tab:ab-hoi}
\end{table*}


% \subsection{Fine-tuning \color{olive}{(Yufei)}\color{black}}


% \noindent \TODO{Fune-tuning strategy (offline and online?)}

\section{Experiments of Interactive Model}

%\subsection{Foundational Interactive model}
% \paragraph{\textbf{Evaluation Metrics}}$\ $
\subsection{Datasets} 
%To the best of our knowledge, there is no available dataset specifically focused on human-robot interaction (HRI). We transfer human-human interaction data to human-robot interaction data, enabling the robot to learn realistic and natural interactive behaviors with humans. Specifically, we introduce the Interx-HRI dataset, which is built on the open human-human interaction dataset Inter-X~\cite{xu2024inter} by retargeting one human pose to robot joint and pose representations. Interx-HRI covers a wide range of interaction types and category annotations. It provides a robust foundation for developing and evaluating human-robot interaction models. Additionally, we introduce Object-ROI dataset based on FullBodyManipulation~\cite{li2023object}, which focuses on robot-object interaction (ROI), further expanding the scope of interactive tasks that can be studied in the context of human-robot interactions.
\begin{figure*}[t]
\centering
\includegraphics[width=1.0\linewidth]{images/vis-interactive-model.pdf}
\caption{Visualization results of our foundational interactive model. We show five actions (High-five, Pat on back, Handshake, Block, and Push) of human-robot interaction and two actions (Kick box and Rotate box) of robot-object interaction. By observing dynamic real human motions (orange), our model can generate reasonable corresponding actions of the robot (white) in real time.}
\label{fig:visintermodel}
\end{figure*}
To the best of our knowledge, no existing dataset specifically focuses on human-robot interaction (HRI). To address this gap, we adapt human-human interaction data for human-robot interaction, enabling the robot to learn realistic and natural behaviors in interaction with humans. Specifically, we introduce the \textit{Inter-HRI} dataset, which extends the open human-human interaction dataset Inter-X~\cite{xu2024inter} by retargeting one of two interactive human poses to robot joint and pose representations. \textit{Inter-HRI} encompasses a wide variety of interaction types and category annotations, providing a solid foundation for the development and evaluation of human-robot interaction models. Additionally, we introduce the \textit{Inter-ROI} dataset, derived from FullBodyManipulation~\cite{li2023object}, which focuses on robot-object interaction (ROI), further broadening the range of interactive tasks that can be studied in the context of human-robot collaboration.


\noindent \textbf{Inter-X~\cite{xu2024inter}} is a large-scale human-human interaction dataset with over 8.1 million frames, 11,000 motion sequences, and 40 interaction categories, captured at 120 fps. It provides detailed annotations, including hand gestures, interaction order, and social relationships, for both perception and generation tasks.

\noindent \textbf{FullBodyManipulation~\cite{li2023object}}
provides paired object and human motion with a total duration of approximately
10 hours, captured at 120 fps. It provides the interactions between 17 subjects and 15 different objects with text descriptions.

\noindent\textbf{Inter-HRI \& Inter-ROI} 
We build Inter-HRI and Inter-ROI by re-targeting the angle-axis of one person in original datasets to that of specific robot types, such as Unitree H1~\cite{unitreeh1robot} and LEJU Kuavo~\cite{lejukuavorobot}, following PHC~\cite{luo2023perpetual}. It is important to note that Inter-HRI and Inter-ROI datasets include all the data from the original datasets while preserving the division between the training and testing sets.
Additionally, we downsample the processed data to a frequency of 10 frames per second to better adapt to the setting of our motion capture system.


\subsection{Evaluation Metrics}

\textbf{MPJPE} represents the mean per joint position error, computed using the average Euclidean distance between predicted and ground truth joint positions. \textbf{PA-MPJPE} represents the Procrustes-Aligned MPJPE,  differs in that it first aligns for rotation, translation, and scale, following the ~\cite{deitke2020robothor}. For measuring the contact quality between interactive targets, we employ contact metrics including \textbf{precision($C\_prec$), recall($C\_rec$), accuracy($C\_acc$)} and \textbf{F1 score($C\_F1$)} following the ~\cite{li2023object}.
\textbf{Traj} represents the trajectory error, computed using the average Euclidean distance between root joints of the predictions and ground truths, 
\textbf{Orie} represents the orientation error, computed using the angular difference of the speed of root joints between the predictions and ground truths. The two error metrics are following the ~\cite{dai2024motionlcmrealtimecontrollablemotion}.
\textbf{FID } calculates the Fréchet distance between the features extracted by the inception network. \textbf{R-score} measures the text and motion matching accuracy, computed by calculating the Euclidean distances between text embeddings and predicted motion embeddings and counting Top-3 accuracy of motion-to-text retrieval. \textbf{AITS} represents the average inference time to evaluate the real-time efficiency of models.

% \paragraph{\textbf{Comparative Experiment}}$\ $


\subsection{Baselines}
% We chose prediction-based and generation-based methods to validate the advantages of our Foundational Interactive model in terms of performance and real-time efficiency. 
To validate the performance and efficiency of our interactive model, we compare it against current state-of-the-art human-related interaction methods using both the Inter-HRI and Inter-ROI datasets. For a fair comparison, we adapt these methods to our specific setting and incorporate action-guided interaction by providing them with action text input through a CLIP-based text encoder. Additionally, we enable the prediction of the robot's future motion based on real-time human motion input, processed frame by frame.
JRT~\cite{xu2023joint} is a multi-person motion prediction method, which explicitly models the relations between joints and predicts for multiple humans simultaneously.
MRT~\cite{wang2021multi} processes individual motion and social interactions simultaneously by introducing a local-range encoder and a global-range encoder, and then aggregates the multi-range representations to the primary person for future motion prediction.
SAST~\cite{mueller2024massively} is a diffusion-based method, which models interactions among various targets such as multiple people, objects in a scene by a denoising diffusion model.
% However, during inference, it requires many denoising iterations, which results in poor real-time efficiency.
ReGenNet~\cite{xu2024regennet} proposes a reaction model that generates one person's motion based on the action and motion of another person, without using historical information like prediction-based methods.
% It analyzes the dynamic and detailed nature of human-human interactions. However, it does not incorporate a design that ensures the generated robot motion remains consistent with the previous robot motion.
InterDiff~\cite{xu2023inter} is a diffusion-based generative model that predicts the next step in human-object interaction tasks based on past motion. 
% Since it is not specifically designed for text-guided scenarios, we adapt its framework by incorporating a CLIP-based text encoder to extract textual features, enabling text-guided interaction prediction.

% cps original
% InterDiff~\cite{xu2023inter} is a diffusion-based generative model that predicts the next step in human-object interaction tasks based on past motion. Since it is not specifically designed for text-guided scenarios, we adapt its framework by incorporating a CLIP-based text encoder to extract textual features, enabling text-guided interaction prediction.

\subsection{Result Analysis}

\noindent\textbf{Results on the Human-robot Interaction}
As shown in Tab.~\ref{tab:hhi}, our foundational human-robot interactive model has achieved state-of-the-art performance across all evaluation metrics on two types of robots. 
Our model achieves the best performance in MPJPE and PA-MPJPE, demonstrating its effectiveness in local pose estimation.
It also outperforms others in trajectory (traj) and orientation (orie), validating its superior performance in global trajectory prediction.
Additionally, our model outperforms others in contact metrics, demonstrating its capacity to facilitate more precise interactions between the generated robot and the human.
Additionally, our model achieves the best FID and R-score, indicating that the distribution of predicted motion is close to that of the ground truth.
Finally, our model leads in AITS, indicating its high efficiency and application in real-time scenarios.
% Compared to JRT~\cite{xu2023joint} which explicitly models the relationship of joints between multiple person, our model demonstrates superior accuracy in joint position prediction, as evidenced by MPJPE and PA-MPJPE. 
% Furthermore, the contact metrics ($C\_prec$,$C\_rec$,
% $C\_acc$,$C\_F1$) show that our model generates more precise interactions between the robot and human.
% Compared to MRT~\cite{wang2021multi}, which uses transformers with different ranges to model local pose and global trajectory separately, our model outperforms in both pose and trajectory aspects as shown in MPJPE, PA-MPJPE, Trajectory Error, and Orientation Error.



\noindent\textbf{Results on the Robot-object Interaction}
To test our model's ability to predict the results of the robot-object interaction task, we further conduct the experiment on Inter-ROI. The result is illustrated in Tab.~\ref{tab:hoi}, our proposed method outperforms other methods on most metrics. Notably, while MRT achieves better performance on trajectory (traj) and orientation (orie) metrics, this is primarily because MRT tends to produce more translational results, which in turn leads to higher FID scores.  We also provide our visualization results are illustrated in Fig.~\ref{fig:visintermodel}.



\subsection{Ablation Study}
To further verify the effectiveness of our model, we conduct ablation studies to assess the impact of each loss design in Tab.~\ref{tab:ab-hoi} on HRI and ROI. Adding the $L_{\text{vel}}$ loss improves the trajectory alignment by encouraging smoother and more accurate robot motion results. For robot-object interaction, we have two specific losses. Incorporating the $L_{\text{interactive}}$ loss enhances the quality of interactions by ensuring close proximity between the robot's hands and the object. Introducing the object branch and $L_{\text{obj}}$ boosts the model's capability to predict object behavior, resulting in more realistic dynamic robot-object interactions.


% \paragraph{\textbf{Ablation Study}}$\ $
% Goal: the proposed method outperforms the SOTA human-human (robot) interactive model in terms of performance and inference speed.

% \noindent \TODO{SOTA methods???}

% \noindent \TODO{Metrics???}


% \subsection{Human-Object Interactive model}


% \begin{table*}[]
% \centering
% \caption{Ablation studies on FullBodyManipulation.}
% \resizebox{\linewidth}{!}{ 
% \begin{tabular}{c|ccccccccccc}
% \hline
% & PA-MPJPE & MPJPE & C\_prec & Contact recall & Acc   & F1     & percent & Traj & Orientation & FID & R-score \\ \hline
% basline          &          &       &                   &                &       &        &         &            &             &     &         \\ \hline
% w/o BPS          & 7.73     & 20.03 & 0.61              & 0.28           & 0.49  & 0.35   & 0.27    & 0.36      & 48.98       &     &         \\ \hline
% w/o Joint Loss       &          &       &                   &                &       &        &         &            &             &     &         \\ \hline
% w/o Interactive Loss &          &       &                   &                &       &        &         &            &             &     &         \\ \hline
% w/o Inter-Branch     & 4.64     & 20.4  & 0.67              & 0.29           & 0.51  & 0.37   & 0.28    & 0.34       & 46.7        &     &         \\ \hline
% ours             & 4.32     & 19.83 & 0.70               & 0.31          & 0.52 & 0.38 & 0.27   & 0.38       & 47.2        &     &         \\ \hline
% \end{tabular}
% }
% \end{table*}
% 

%\subsection{Motion Re-targeting}
%Goal: the simulated robot action performs better when equipped with PHC.

%\noindent \TODO{Metrics}

%\subsection{Fine-tuning}
%Goal: the fine-tuned model performs better than the primary version.

%\noindent \TODO{Metrics}

%\noindent \TODO{Table: online fine-tuning, offline fine-tuning}

\section{System Evaluation}
\label{sec:eval}

\begin{figure}
\centering
\includegraphics[width=1\linewidth]{images/userstudy1.jpg}
\caption{User study results of group A before model fine-tuning.}
\label{fig:userstudy1}
\end{figure}
\begin{figure}
\centering
\includegraphics[width=\linewidth]{images/userstudy2.png}
\caption{User study results of group B after model fine-tuning.}
\label{fig:userstudy2}
\end{figure}

\subsection{Case Study}
In this section, we present examples to demonstrate the functionality of our system, i.e. human motion imitation, human-robot interaction, and robot-object interaction. For each functionality, we present the visualization from both the first-person perspective and the third-person perspective. The corresponding videos
are provided in the supplementary material.

\textbf{Human Motion Imitation}
\begin{figure*}
\centering
\includegraphics[width=1\linewidth]{images/taiji.png}
\caption{Example of robot imitating human motion.}
\label{fig:Human Motion Imitation}
\end{figure*}
Fig.~\ref{fig:Human Motion Imitation} presents the key frames of a LEJU Kuavo robot imitating real human motion. In this process, the participant is required to perform Tai-Chi actions, which consist of intricate and harmonious movements. To realize this imitation, our system utilizes motion capture to acquire human motion and invokes the inverse kinematic solver to map it onto the robot. Subsequently, the robot's action is optimized in the physical simulation module to ensure stable and reliable movements. The imitation of the particularly selected Tai-Chi actions effectively verifies the capacity of our system to stably and reliably transfer human behaviors onto humanoid robots. 

\textbf{Human-robot Interaction}
\begin{figure*}
\centering
\includegraphics[width=1\linewidth]{images/block.png}
\caption{Example of robot blocking human.}
\label{fig:Human-Robot-Interaction}
\end{figure*}
%Figure 5 demonstrates a complete interaction process in which a robot attempts to block a user's path.
%At the start of the interaction, the user approaches the robot, moving closer. In response, the robot raises and extends its arms to block the user’s path. In the fifth set of images, the robot lowers its arms and returns to its initial position as the user steps backward. This indicates that the interactive model has stopped blocking the user. However, the robot repeats the blocking action when the user approaches again, illustrating the adaptive behavior of the interaction model.
Fig.~\ref{fig:Human-Robot-Interaction} showcases the complete process of a robot attempting to block a user's path during an interaction.
At the start of the interaction, the user walks toward the robot. As the user gets closer, the robot progressively raises and extends its arms to block the way. This forces the user to step back. Immediately after, the robot lowers its arms and returns to its initial position. When the user approaches again, the robot raises one of its arms once more to block the user's path, clearly demonstrating its attention to blocking the repeated approach. Driven by the interactive model and physical simulation module, the virtual robot automatically performs actions in response to human movement. The participant's reaction, i.e. stepping back once being blocked, strongly verifies our system's realism and real-time capability, since in a real-world scenario, a person would react similarly when their path is suddenly obstructed. This not only validates the accuracy of the motion capture but also highlights the effectiveness of the interactive model and physical simulation module. The module ensures that the robot's movements are not only functional in blocking the user but also natural-looking, mimicking the physical constraints and capabilities of a real-world robot agent.


%In the fifth set of images, as the user steps backward, the robot lowers its arms and goes back to its original position. This indicates that the interactive model has ceased blocking the user. Nevertheless, when the user approaches again, the robot repeats the blocking action, highlighting the adaptive behavior of the interaction model.


\textbf{Robot-object Interaction} 
\begin{figure*}
\centering
\includegraphics[width=1\linewidth]{images/Human-object.png}
\caption{Example of robot kicking box.}
\label{fig:Human-object Interaction}
\end{figure*} 
Fig.~\ref{fig:Human-object Interaction} presents an example of the interaction between virtual robots and objects. In contrast to traditional training frameworks where humans are limited to observing data via monitors with fixed viewpoints, our system enables human users, equipped with AR glasses, to observe the entire process of virtual robots interacting with objects from any perspective within the real-world environment, as depicted in Fig.~\ref{fig:Human-object Interaction}.
Simultaneously, users can offer evaluative feedback on this interaction process. Our system is capable of leveraging the collected feedback information to further optimize the interactive model for the interaction between virtual robots and objects. Owing to the high-fidelity visualization provided by the AR module, our system empowers users to provide more reliable and accurate evaluations. This, in turn, furnishes a human-involved platform for the training and evaluation of the human-robot and robot-object interaction model.





\subsection{User Study}
%The objective of this user study is to validate the bidirectional learning capacity of our system. On one hand, we examine whether the participants can become increasingly familiar with the interaction with the robot during the continuous human-robot interaction process, thereby achieving more efficient skill improvement. On the other hand, we check whether the virtual robot's interaction capacity is enhanced with the recorded interaction data and collected user feedbacks.

The user study aims to validate the bidirectional learning capacity of SymbioSim for human-robot interactions. On the human side, we check whether participants can get more experience interacting with a virtual robot via the AR glasses and be more skilled in later interactions. On the robot side, we see if the ability of the interactive model can be enhanced by collected data and feedback from users. As these two aspects are interlinked, we meticulously design and conduct a series of user studies to comprehensively validate the effectiveness of our system.


In this human-robot interaction user study, we recruited two groups of participants, i.e. group A and group B, each with 10 participants. We selected five types of actions from daily human-human interactions to conduct the experiments. These five actions are handshake, high-five, pat on the back, block, and push.
Each participant is required to repeatedly conduct three rounds of interactions with the virtual robot for each action type. First, the participants in group A interact with the virtual robot driven by the directly trained foundational interactive model and the physics-based controller in the simulation environment. The interaction data and user ratings of these participants are collected during their interaction process for fine-tuning the interactive model. Then, the participants in group B follow the same instructions as group A, but employ the fine-tuned interactive model to drive the robot for interaction. The user feedback during the interaction process of group B is also collected.

After each interaction, participants are required to rate the interaction on a scale of 1-5 in three aspects, i.e. Q1: legibility of robot's action; Q2: satisfaction with the interaction; Q3: ease of the interaction. %Q1: whether the robot's actions match the corresponding action types; Q2: the satisfaction of the interaction experience; Q3: the naturalness and relaxation of the interaction process. 
We show the average user rating of Group A and Group B for each round of interactions in Fig.~\ref{fig:userstudy1} and Fig.~\ref{fig:userstudy2} respectively. %The highest ratings among the three rounds corresponding to the same action type are marked with bold font.


\textbf{SymbioSim Improves Human Performance} Fig.~\ref{fig:userstudy1} reports the user ratings w.r.t. the three aspects during the human-robot interactions of participants from group A, which employs the interactive model trained on our retargeted Inter-HRI dataset. %The "handshake" and "pat on back" actions exhibit a monotonous increase of the average rating on the three aspects. Similarly, the "high-five" and "block" actions reach the highest or nearly the highest ratings in the final round. Comparatively, since the "push" action received relatively high ratings from the very beginning, it didn't obtain the highest score in the third round. Nevertheless, after three rounds of interactions, the participants were still able to steadily and proficiently complete all kinds of interactions. 
In Fig.~\ref{fig:userstudy1}, the positions of curves exhibit a nearly monotonous increase in the order of blue, orange, and green, indicating an overall upward trend over the three rounds of interactions. It confirms that during the continuous interaction process, users become increasingly adapted to the robot's behaviors. As a result, they can more easily identify the action characteristics of the robot, efficiently and successfully complete the interaction with the robot, and maintain a relaxed interaction state throughout the process.

%We present the quantitative result of three aspects in five actions in Table~\ref{Table:UserStudy1}. 
%Overall, the ratings show a positive trend across iterations for most actions. 

%For example, the "Handshake" action started with an average rating of 2.9, 3.0 and 3.1 for the three aspects in round 1 and gradually increased to 4.4, 3.7 and 3.6 in the third iteration. Similar improvements were observed in the other actions, with participants consistently rating their experience higher after each iteration. The highest ratings were generally seen in the "Walk and block" action, with an increase from 4.0 to 4.2 in Iteration 3. Conversely, "High-five" had the lowest ratings in the first round, but also showed improvements over time. This suggests that participants became more comfortable and satisfied with their interactions with the virtual robot as the study progressed. Although, during three iterations, we use same interactive model for robot motion generation, the first question also shows an upward trend. With users having clearer understanding of robot interactive habit, user also giving higher quality motion as feedback through motion capture module, producing a higher quality robot motion sequence.



\textbf{SymbioSim Improves Robot Performance}
Similarly, Fig.~\ref{fig:userstudy2} presents the average user ratings of participants from group B. This group utilizes the interactive model that has been fine-tuned based on the collected user feedback from group A. We emphasize the statistics from two aspects.

Firstly, when comparing the ratings of round 1 for group A (in Fig.~\ref{fig:userstudy1}) and group B (in Fig.~\ref{fig:userstudy2}), it can be observed that the participants in group B generally commence with a higher rating. This verifies that the fine-tuned interactive model is able to predict more reasonable and reliable robot actions, thereby better adapting to human users.
Secondly, despite having a high initial rating, the ratings of group B still show an upward trend as the number of interaction rounds increases. This demonstrates the bidirectional continuous learning ability of our system, meaning that both the human and the robot's capabilities are continuously enhanced throughout the series of human-robot interactions.

%Table~\ref{Table:UserStudy2} records the average score from users interact with LEJU Kuavo robot driven by the finetuned interactive model. Similar to the result of the first user study, users tend to give higher rating in later iterations. Focusing on the result of first iteration, in the second user study, within 5 actions, "high-five", "push", and "pat on back" receive better feedback in all three aspect comparing with result from first iteration in the first user study. "Handshake" action has notable higher result in first and third aspect, despite of the slight decrease on the second aspect.

%In another words, both user studies prove the human side learning capacity of our system. The comparison between first iteration of two user studies, illustrates the capability of allowing robot or the interactive model to improve. The interactive model in the second user study is able to avoid the motion that cause lower grade since it learns from the feedback provided by previous users.

%\begin{figure*}[]
  %\includegraphics[width=0.8\linewidth]{images/example-comparison.png}
%  \caption{Example Comparison.}
%\end{figure*}



%\begin{figure*}[]
  %\includegraphics[width=0.8\li%newidth]{images/smaeModelexample.png}
%  \caption{Same model user study}
%\end{figure*}


%\begin{figure*}[]
  %\includegraphics[width=0.8\linewidth]{images/Fine-tune example.png}
 % \caption{User Study with fine-tuning}
%\end{figure*}




\section{Conclusion}

In this paper, we introduce STeCa, a novel agent learning framework designed to enhance the performance of LLM agents in long-horizon tasks. 
STeCa identifies deviated actions through step-level reward comparisons and constructs calibration trajectories via reflection. 
These trajectories serve as critical data for reinforced training. Extensive experiments demonstrate that STeCa significantly outperforms baseline methods, with additional analyses underscoring its robust calibration capabilities.

\section{Acknowledgement}
This work is supported by the National Natural Science Foundation of China (U23A20312, No.62302269, No.62206173), Shandong Provincial Natural Science Foundation (No.ZR2023QF077), Shandong Province Excellent Young Scientists Fund Program (Overseas) (No.2023HWYQ-034), Shanghai Frontiers Science Center of Human-centered Artificial Intelligence (ShangHAI), MoE Key Laboratory of Intelligent Perception and Human-Machine Collaboration (KLIP-HuMaCo).


\bibliographystyle{formats/ACM-Reference-Format}
\bibliography{Reference} 
% \appendix
%\subsection{Lloyd-Max Algorithm}
\label{subsec:Lloyd-Max}
For a given quantization bitwidth $B$ and an operand $\bm{X}$, the Lloyd-Max algorithm finds $2^B$ quantization levels $\{\hat{x}_i\}_{i=1}^{2^B}$ such that quantizing $\bm{X}$ by rounding each scalar in $\bm{X}$ to the nearest quantization level minimizes the quantization MSE. 

The algorithm starts with an initial guess of quantization levels and then iteratively computes quantization thresholds $\{\tau_i\}_{i=1}^{2^B-1}$ and updates quantization levels $\{\hat{x}_i\}_{i=1}^{2^B}$. Specifically, at iteration $n$, thresholds are set to the midpoints of the previous iteration's levels:
\begin{align*}
    \tau_i^{(n)}=\frac{\hat{x}_i^{(n-1)}+\hat{x}_{i+1}^{(n-1)}}2 \text{ for } i=1\ldots 2^B-1
\end{align*}
Subsequently, the quantization levels are re-computed as conditional means of the data regions defined by the new thresholds:
\begin{align*}
    \hat{x}_i^{(n)}=\mathbb{E}\left[ \bm{X} \big| \bm{X}\in [\tau_{i-1}^{(n)},\tau_i^{(n)}] \right] \text{ for } i=1\ldots 2^B
\end{align*}
where to satisfy boundary conditions we have $\tau_0=-\infty$ and $\tau_{2^B}=\infty$. The algorithm iterates the above steps until convergence.

Figure \ref{fig:lm_quant} compares the quantization levels of a $7$-bit floating point (E3M3) quantizer (left) to a $7$-bit Lloyd-Max quantizer (right) when quantizing a layer of weights from the GPT3-126M model at a per-tensor granularity. As shown, the Lloyd-Max quantizer achieves substantially lower quantization MSE. Further, Table \ref{tab:FP7_vs_LM7} shows the superior perplexity achieved by Lloyd-Max quantizers for bitwidths of $7$, $6$ and $5$. The difference between the quantizers is clear at 5 bits, where per-tensor FP quantization incurs a drastic and unacceptable increase in perplexity, while Lloyd-Max quantization incurs a much smaller increase. Nevertheless, we note that even the optimal Lloyd-Max quantizer incurs a notable ($\sim 1.5$) increase in perplexity due to the coarse granularity of quantization. 

\begin{figure}[h]
  \centering
  \includegraphics[width=0.7\linewidth]{sections/figures/LM7_FP7.pdf}
  \caption{\small Quantization levels and the corresponding quantization MSE of Floating Point (left) vs Lloyd-Max (right) Quantizers for a layer of weights in the GPT3-126M model.}
  \label{fig:lm_quant}
\end{figure}

\begin{table}[h]\scriptsize
\begin{center}
\caption{\label{tab:FP7_vs_LM7} \small Comparing perplexity (lower is better) achieved by floating point quantizers and Lloyd-Max quantizers on a GPT3-126M model for the Wikitext-103 dataset.}
\begin{tabular}{c|cc|c}
\hline
 \multirow{2}{*}{\textbf{Bitwidth}} & \multicolumn{2}{|c|}{\textbf{Floating-Point Quantizer}} & \textbf{Lloyd-Max Quantizer} \\
 & Best Format & Wikitext-103 Perplexity & Wikitext-103 Perplexity \\
\hline
7 & E3M3 & 18.32 & 18.27 \\
6 & E3M2 & 19.07 & 18.51 \\
5 & E4M0 & 43.89 & 19.71 \\
\hline
\end{tabular}
\end{center}
\end{table}

\subsection{Proof of Local Optimality of LO-BCQ}
\label{subsec:lobcq_opt_proof}
For a given block $\bm{b}_j$, the quantization MSE during LO-BCQ can be empirically evaluated as $\frac{1}{L_b}\lVert \bm{b}_j- \bm{\hat{b}}_j\rVert^2_2$ where $\bm{\hat{b}}_j$ is computed from equation (\ref{eq:clustered_quantization_definition}) as $C_{f(\bm{b}_j)}(\bm{b}_j)$. Further, for a given block cluster $\mathcal{B}_i$, we compute the quantization MSE as $\frac{1}{|\mathcal{B}_{i}|}\sum_{\bm{b} \in \mathcal{B}_{i}} \frac{1}{L_b}\lVert \bm{b}- C_i^{(n)}(\bm{b})\rVert^2_2$. Therefore, at the end of iteration $n$, we evaluate the overall quantization MSE $J^{(n)}$ for a given operand $\bm{X}$ composed of $N_c$ block clusters as:
\begin{align*}
    \label{eq:mse_iter_n}
    J^{(n)} = \frac{1}{N_c} \sum_{i=1}^{N_c} \frac{1}{|\mathcal{B}_{i}^{(n)}|}\sum_{\bm{v} \in \mathcal{B}_{i}^{(n)}} \frac{1}{L_b}\lVert \bm{b}- B_i^{(n)}(\bm{b})\rVert^2_2
\end{align*}

At the end of iteration $n$, the codebooks are updated from $\mathcal{C}^{(n-1)}$ to $\mathcal{C}^{(n)}$. However, the mapping of a given vector $\bm{b}_j$ to quantizers $\mathcal{C}^{(n)}$ remains as  $f^{(n)}(\bm{b}_j)$. At the next iteration, during the vector clustering step, $f^{(n+1)}(\bm{b}_j)$ finds new mapping of $\bm{b}_j$ to updated codebooks $\mathcal{C}^{(n)}$ such that the quantization MSE over the candidate codebooks is minimized. Therefore, we obtain the following result for $\bm{b}_j$:
\begin{align*}
\frac{1}{L_b}\lVert \bm{b}_j - C_{f^{(n+1)}(\bm{b}_j)}^{(n)}(\bm{b}_j)\rVert^2_2 \le \frac{1}{L_b}\lVert \bm{b}_j - C_{f^{(n)}(\bm{b}_j)}^{(n)}(\bm{b}_j)\rVert^2_2
\end{align*}

That is, quantizing $\bm{b}_j$ at the end of the block clustering step of iteration $n+1$ results in lower quantization MSE compared to quantizing at the end of iteration $n$. Since this is true for all $\bm{b} \in \bm{X}$, we assert the following:
\begin{equation}
\begin{split}
\label{eq:mse_ineq_1}
    \tilde{J}^{(n+1)} &= \frac{1}{N_c} \sum_{i=1}^{N_c} \frac{1}{|\mathcal{B}_{i}^{(n+1)}|}\sum_{\bm{b} \in \mathcal{B}_{i}^{(n+1)}} \frac{1}{L_b}\lVert \bm{b} - C_i^{(n)}(b)\rVert^2_2 \le J^{(n)}
\end{split}
\end{equation}
where $\tilde{J}^{(n+1)}$ is the the quantization MSE after the vector clustering step at iteration $n+1$.

Next, during the codebook update step (\ref{eq:quantizers_update}) at iteration $n+1$, the per-cluster codebooks $\mathcal{C}^{(n)}$ are updated to $\mathcal{C}^{(n+1)}$ by invoking the Lloyd-Max algorithm \citep{Lloyd}. We know that for any given value distribution, the Lloyd-Max algorithm minimizes the quantization MSE. Therefore, for a given vector cluster $\mathcal{B}_i$ we obtain the following result:

\begin{equation}
    \frac{1}{|\mathcal{B}_{i}^{(n+1)}|}\sum_{\bm{b} \in \mathcal{B}_{i}^{(n+1)}} \frac{1}{L_b}\lVert \bm{b}- C_i^{(n+1)}(\bm{b})\rVert^2_2 \le \frac{1}{|\mathcal{B}_{i}^{(n+1)}|}\sum_{\bm{b} \in \mathcal{B}_{i}^{(n+1)}} \frac{1}{L_b}\lVert \bm{b}- C_i^{(n)}(\bm{b})\rVert^2_2
\end{equation}

The above equation states that quantizing the given block cluster $\mathcal{B}_i$ after updating the associated codebook from $C_i^{(n)}$ to $C_i^{(n+1)}$ results in lower quantization MSE. Since this is true for all the block clusters, we derive the following result: 
\begin{equation}
\begin{split}
\label{eq:mse_ineq_2}
     J^{(n+1)} &= \frac{1}{N_c} \sum_{i=1}^{N_c} \frac{1}{|\mathcal{B}_{i}^{(n+1)}|}\sum_{\bm{b} \in \mathcal{B}_{i}^{(n+1)}} \frac{1}{L_b}\lVert \bm{b}- C_i^{(n+1)}(\bm{b})\rVert^2_2  \le \tilde{J}^{(n+1)}   
\end{split}
\end{equation}

Following (\ref{eq:mse_ineq_1}) and (\ref{eq:mse_ineq_2}), we find that the quantization MSE is non-increasing for each iteration, that is, $J^{(1)} \ge J^{(2)} \ge J^{(3)} \ge \ldots \ge J^{(M)}$ where $M$ is the maximum number of iterations. 
%Therefore, we can say that if the algorithm converges, then it must be that it has converged to a local minimum. 
\hfill $\blacksquare$


\begin{figure}
    \begin{center}
    \includegraphics[width=0.5\textwidth]{sections//figures/mse_vs_iter.pdf}
    \end{center}
    \caption{\small NMSE vs iterations during LO-BCQ compared to other block quantization proposals}
    \label{fig:nmse_vs_iter}
\end{figure}

Figure \ref{fig:nmse_vs_iter} shows the empirical convergence of LO-BCQ across several block lengths and number of codebooks. Also, the MSE achieved by LO-BCQ is compared to baselines such as MXFP and VSQ. As shown, LO-BCQ converges to a lower MSE than the baselines. Further, we achieve better convergence for larger number of codebooks ($N_c$) and for a smaller block length ($L_b$), both of which increase the bitwidth of BCQ (see Eq \ref{eq:bitwidth_bcq}).


\subsection{Additional Accuracy Results}
%Table \ref{tab:lobcq_config} lists the various LOBCQ configurations and their corresponding bitwidths.
\begin{table}
\setlength{\tabcolsep}{4.75pt}
\begin{center}
\caption{\label{tab:lobcq_config} Various LO-BCQ configurations and their bitwidths.}
\begin{tabular}{|c||c|c|c|c||c|c||c|} 
\hline
 & \multicolumn{4}{|c||}{$L_b=8$} & \multicolumn{2}{|c||}{$L_b=4$} & $L_b=2$ \\
 \hline
 \backslashbox{$L_A$\kern-1em}{\kern-1em$N_c$} & 2 & 4 & 8 & 16 & 2 & 4 & 2 \\
 \hline
 64 & 4.25 & 4.375 & 4.5 & 4.625 & 4.375 & 4.625 & 4.625\\
 \hline
 32 & 4.375 & 4.5 & 4.625& 4.75 & 4.5 & 4.75 & 4.75 \\
 \hline
 16 & 4.625 & 4.75& 4.875 & 5 & 4.75 & 5 & 5 \\
 \hline
\end{tabular}
\end{center}
\end{table}

%\subsection{Perplexity achieved by various LO-BCQ configurations on Wikitext-103 dataset}

\begin{table} \centering
\begin{tabular}{|c||c|c|c|c||c|c||c|} 
\hline
 $L_b \rightarrow$& \multicolumn{4}{c||}{8} & \multicolumn{2}{c||}{4} & 2\\
 \hline
 \backslashbox{$L_A$\kern-1em}{\kern-1em$N_c$} & 2 & 4 & 8 & 16 & 2 & 4 & 2  \\
 %$N_c \rightarrow$ & 2 & 4 & 8 & 16 & 2 & 4 & 2 \\
 \hline
 \hline
 \multicolumn{8}{c}{GPT3-1.3B (FP32 PPL = 9.98)} \\ 
 \hline
 \hline
 64 & 10.40 & 10.23 & 10.17 & 10.15 &  10.28 & 10.18 & 10.19 \\
 \hline
 32 & 10.25 & 10.20 & 10.15 & 10.12 &  10.23 & 10.17 & 10.17 \\
 \hline
 16 & 10.22 & 10.16 & 10.10 & 10.09 &  10.21 & 10.14 & 10.16 \\
 \hline
  \hline
 \multicolumn{8}{c}{GPT3-8B (FP32 PPL = 7.38)} \\ 
 \hline
 \hline
 64 & 7.61 & 7.52 & 7.48 &  7.47 &  7.55 &  7.49 & 7.50 \\
 \hline
 32 & 7.52 & 7.50 & 7.46 &  7.45 &  7.52 &  7.48 & 7.48  \\
 \hline
 16 & 7.51 & 7.48 & 7.44 &  7.44 &  7.51 &  7.49 & 7.47  \\
 \hline
\end{tabular}
\caption{\label{tab:ppl_gpt3_abalation} Wikitext-103 perplexity across GPT3-1.3B and 8B models.}
\end{table}

\begin{table} \centering
\begin{tabular}{|c||c|c|c|c||} 
\hline
 $L_b \rightarrow$& \multicolumn{4}{c||}{8}\\
 \hline
 \backslashbox{$L_A$\kern-1em}{\kern-1em$N_c$} & 2 & 4 & 8 & 16 \\
 %$N_c \rightarrow$ & 2 & 4 & 8 & 16 & 2 & 4 & 2 \\
 \hline
 \hline
 \multicolumn{5}{|c|}{Llama2-7B (FP32 PPL = 5.06)} \\ 
 \hline
 \hline
 64 & 5.31 & 5.26 & 5.19 & 5.18  \\
 \hline
 32 & 5.23 & 5.25 & 5.18 & 5.15  \\
 \hline
 16 & 5.23 & 5.19 & 5.16 & 5.14  \\
 \hline
 \multicolumn{5}{|c|}{Nemotron4-15B (FP32 PPL = 5.87)} \\ 
 \hline
 \hline
 64  & 6.3 & 6.20 & 6.13 & 6.08  \\
 \hline
 32  & 6.24 & 6.12 & 6.07 & 6.03  \\
 \hline
 16  & 6.12 & 6.14 & 6.04 & 6.02  \\
 \hline
 \multicolumn{5}{|c|}{Nemotron4-340B (FP32 PPL = 3.48)} \\ 
 \hline
 \hline
 64 & 3.67 & 3.62 & 3.60 & 3.59 \\
 \hline
 32 & 3.63 & 3.61 & 3.59 & 3.56 \\
 \hline
 16 & 3.61 & 3.58 & 3.57 & 3.55 \\
 \hline
\end{tabular}
\caption{\label{tab:ppl_llama7B_nemo15B} Wikitext-103 perplexity compared to FP32 baseline in Llama2-7B and Nemotron4-15B, 340B models}
\end{table}

%\subsection{Perplexity achieved by various LO-BCQ configurations on MMLU dataset}


\begin{table} \centering
\begin{tabular}{|c||c|c|c|c||c|c|c|c|} 
\hline
 $L_b \rightarrow$& \multicolumn{4}{c||}{8} & \multicolumn{4}{c||}{8}\\
 \hline
 \backslashbox{$L_A$\kern-1em}{\kern-1em$N_c$} & 2 & 4 & 8 & 16 & 2 & 4 & 8 & 16  \\
 %$N_c \rightarrow$ & 2 & 4 & 8 & 16 & 2 & 4 & 2 \\
 \hline
 \hline
 \multicolumn{5}{|c|}{Llama2-7B (FP32 Accuracy = 45.8\%)} & \multicolumn{4}{|c|}{Llama2-70B (FP32 Accuracy = 69.12\%)} \\ 
 \hline
 \hline
 64 & 43.9 & 43.4 & 43.9 & 44.9 & 68.07 & 68.27 & 68.17 & 68.75 \\
 \hline
 32 & 44.5 & 43.8 & 44.9 & 44.5 & 68.37 & 68.51 & 68.35 & 68.27  \\
 \hline
 16 & 43.9 & 42.7 & 44.9 & 45 & 68.12 & 68.77 & 68.31 & 68.59  \\
 \hline
 \hline
 \multicolumn{5}{|c|}{GPT3-22B (FP32 Accuracy = 38.75\%)} & \multicolumn{4}{|c|}{Nemotron4-15B (FP32 Accuracy = 64.3\%)} \\ 
 \hline
 \hline
 64 & 36.71 & 38.85 & 38.13 & 38.92 & 63.17 & 62.36 & 63.72 & 64.09 \\
 \hline
 32 & 37.95 & 38.69 & 39.45 & 38.34 & 64.05 & 62.30 & 63.8 & 64.33  \\
 \hline
 16 & 38.88 & 38.80 & 38.31 & 38.92 & 63.22 & 63.51 & 63.93 & 64.43  \\
 \hline
\end{tabular}
\caption{\label{tab:mmlu_abalation} Accuracy on MMLU dataset across GPT3-22B, Llama2-7B, 70B and Nemotron4-15B models.}
\end{table}


%\subsection{Perplexity achieved by various LO-BCQ configurations on LM evaluation harness}

\begin{table} \centering
\begin{tabular}{|c||c|c|c|c||c|c|c|c|} 
\hline
 $L_b \rightarrow$& \multicolumn{4}{c||}{8} & \multicolumn{4}{c||}{8}\\
 \hline
 \backslashbox{$L_A$\kern-1em}{\kern-1em$N_c$} & 2 & 4 & 8 & 16 & 2 & 4 & 8 & 16  \\
 %$N_c \rightarrow$ & 2 & 4 & 8 & 16 & 2 & 4 & 2 \\
 \hline
 \hline
 \multicolumn{5}{|c|}{Race (FP32 Accuracy = 37.51\%)} & \multicolumn{4}{|c|}{Boolq (FP32 Accuracy = 64.62\%)} \\ 
 \hline
 \hline
 64 & 36.94 & 37.13 & 36.27 & 37.13 & 63.73 & 62.26 & 63.49 & 63.36 \\
 \hline
 32 & 37.03 & 36.36 & 36.08 & 37.03 & 62.54 & 63.51 & 63.49 & 63.55  \\
 \hline
 16 & 37.03 & 37.03 & 36.46 & 37.03 & 61.1 & 63.79 & 63.58 & 63.33  \\
 \hline
 \hline
 \multicolumn{5}{|c|}{Winogrande (FP32 Accuracy = 58.01\%)} & \multicolumn{4}{|c|}{Piqa (FP32 Accuracy = 74.21\%)} \\ 
 \hline
 \hline
 64 & 58.17 & 57.22 & 57.85 & 58.33 & 73.01 & 73.07 & 73.07 & 72.80 \\
 \hline
 32 & 59.12 & 58.09 & 57.85 & 58.41 & 73.01 & 73.94 & 72.74 & 73.18  \\
 \hline
 16 & 57.93 & 58.88 & 57.93 & 58.56 & 73.94 & 72.80 & 73.01 & 73.94  \\
 \hline
\end{tabular}
\caption{\label{tab:mmlu_abalation} Accuracy on LM evaluation harness tasks on GPT3-1.3B model.}
\end{table}

\begin{table} \centering
\begin{tabular}{|c||c|c|c|c||c|c|c|c|} 
\hline
 $L_b \rightarrow$& \multicolumn{4}{c||}{8} & \multicolumn{4}{c||}{8}\\
 \hline
 \backslashbox{$L_A$\kern-1em}{\kern-1em$N_c$} & 2 & 4 & 8 & 16 & 2 & 4 & 8 & 16  \\
 %$N_c \rightarrow$ & 2 & 4 & 8 & 16 & 2 & 4 & 2 \\
 \hline
 \hline
 \multicolumn{5}{|c|}{Race (FP32 Accuracy = 41.34\%)} & \multicolumn{4}{|c|}{Boolq (FP32 Accuracy = 68.32\%)} \\ 
 \hline
 \hline
 64 & 40.48 & 40.10 & 39.43 & 39.90 & 69.20 & 68.41 & 69.45 & 68.56 \\
 \hline
 32 & 39.52 & 39.52 & 40.77 & 39.62 & 68.32 & 67.43 & 68.17 & 69.30  \\
 \hline
 16 & 39.81 & 39.71 & 39.90 & 40.38 & 68.10 & 66.33 & 69.51 & 69.42  \\
 \hline
 \hline
 \multicolumn{5}{|c|}{Winogrande (FP32 Accuracy = 67.88\%)} & \multicolumn{4}{|c|}{Piqa (FP32 Accuracy = 78.78\%)} \\ 
 \hline
 \hline
 64 & 66.85 & 66.61 & 67.72 & 67.88 & 77.31 & 77.42 & 77.75 & 77.64 \\
 \hline
 32 & 67.25 & 67.72 & 67.72 & 67.00 & 77.31 & 77.04 & 77.80 & 77.37  \\
 \hline
 16 & 68.11 & 68.90 & 67.88 & 67.48 & 77.37 & 78.13 & 78.13 & 77.69  \\
 \hline
\end{tabular}
\caption{\label{tab:mmlu_abalation} Accuracy on LM evaluation harness tasks on GPT3-8B model.}
\end{table}

\begin{table} \centering
\begin{tabular}{|c||c|c|c|c||c|c|c|c|} 
\hline
 $L_b \rightarrow$& \multicolumn{4}{c||}{8} & \multicolumn{4}{c||}{8}\\
 \hline
 \backslashbox{$L_A$\kern-1em}{\kern-1em$N_c$} & 2 & 4 & 8 & 16 & 2 & 4 & 8 & 16  \\
 %$N_c \rightarrow$ & 2 & 4 & 8 & 16 & 2 & 4 & 2 \\
 \hline
 \hline
 \multicolumn{5}{|c|}{Race (FP32 Accuracy = 40.67\%)} & \multicolumn{4}{|c|}{Boolq (FP32 Accuracy = 76.54\%)} \\ 
 \hline
 \hline
 64 & 40.48 & 40.10 & 39.43 & 39.90 & 75.41 & 75.11 & 77.09 & 75.66 \\
 \hline
 32 & 39.52 & 39.52 & 40.77 & 39.62 & 76.02 & 76.02 & 75.96 & 75.35  \\
 \hline
 16 & 39.81 & 39.71 & 39.90 & 40.38 & 75.05 & 73.82 & 75.72 & 76.09  \\
 \hline
 \hline
 \multicolumn{5}{|c|}{Winogrande (FP32 Accuracy = 70.64\%)} & \multicolumn{4}{|c|}{Piqa (FP32 Accuracy = 79.16\%)} \\ 
 \hline
 \hline
 64 & 69.14 & 70.17 & 70.17 & 70.56 & 78.24 & 79.00 & 78.62 & 78.73 \\
 \hline
 32 & 70.96 & 69.69 & 71.27 & 69.30 & 78.56 & 79.49 & 79.16 & 78.89  \\
 \hline
 16 & 71.03 & 69.53 & 69.69 & 70.40 & 78.13 & 79.16 & 79.00 & 79.00  \\
 \hline
\end{tabular}
\caption{\label{tab:mmlu_abalation} Accuracy on LM evaluation harness tasks on GPT3-22B model.}
\end{table}

\begin{table} \centering
\begin{tabular}{|c||c|c|c|c||c|c|c|c|} 
\hline
 $L_b \rightarrow$& \multicolumn{4}{c||}{8} & \multicolumn{4}{c||}{8}\\
 \hline
 \backslashbox{$L_A$\kern-1em}{\kern-1em$N_c$} & 2 & 4 & 8 & 16 & 2 & 4 & 8 & 16  \\
 %$N_c \rightarrow$ & 2 & 4 & 8 & 16 & 2 & 4 & 2 \\
 \hline
 \hline
 \multicolumn{5}{|c|}{Race (FP32 Accuracy = 44.4\%)} & \multicolumn{4}{|c|}{Boolq (FP32 Accuracy = 79.29\%)} \\ 
 \hline
 \hline
 64 & 42.49 & 42.51 & 42.58 & 43.45 & 77.58 & 77.37 & 77.43 & 78.1 \\
 \hline
 32 & 43.35 & 42.49 & 43.64 & 43.73 & 77.86 & 75.32 & 77.28 & 77.86  \\
 \hline
 16 & 44.21 & 44.21 & 43.64 & 42.97 & 78.65 & 77 & 76.94 & 77.98  \\
 \hline
 \hline
 \multicolumn{5}{|c|}{Winogrande (FP32 Accuracy = 69.38\%)} & \multicolumn{4}{|c|}{Piqa (FP32 Accuracy = 78.07\%)} \\ 
 \hline
 \hline
 64 & 68.9 & 68.43 & 69.77 & 68.19 & 77.09 & 76.82 & 77.09 & 77.86 \\
 \hline
 32 & 69.38 & 68.51 & 68.82 & 68.90 & 78.07 & 76.71 & 78.07 & 77.86  \\
 \hline
 16 & 69.53 & 67.09 & 69.38 & 68.90 & 77.37 & 77.8 & 77.91 & 77.69  \\
 \hline
\end{tabular}
\caption{\label{tab:mmlu_abalation} Accuracy on LM evaluation harness tasks on Llama2-7B model.}
\end{table}

\begin{table} \centering
\begin{tabular}{|c||c|c|c|c||c|c|c|c|} 
\hline
 $L_b \rightarrow$& \multicolumn{4}{c||}{8} & \multicolumn{4}{c||}{8}\\
 \hline
 \backslashbox{$L_A$\kern-1em}{\kern-1em$N_c$} & 2 & 4 & 8 & 16 & 2 & 4 & 8 & 16  \\
 %$N_c \rightarrow$ & 2 & 4 & 8 & 16 & 2 & 4 & 2 \\
 \hline
 \hline
 \multicolumn{5}{|c|}{Race (FP32 Accuracy = 48.8\%)} & \multicolumn{4}{|c|}{Boolq (FP32 Accuracy = 85.23\%)} \\ 
 \hline
 \hline
 64 & 49.00 & 49.00 & 49.28 & 48.71 & 82.82 & 84.28 & 84.03 & 84.25 \\
 \hline
 32 & 49.57 & 48.52 & 48.33 & 49.28 & 83.85 & 84.46 & 84.31 & 84.93  \\
 \hline
 16 & 49.85 & 49.09 & 49.28 & 48.99 & 85.11 & 84.46 & 84.61 & 83.94  \\
 \hline
 \hline
 \multicolumn{5}{|c|}{Winogrande (FP32 Accuracy = 79.95\%)} & \multicolumn{4}{|c|}{Piqa (FP32 Accuracy = 81.56\%)} \\ 
 \hline
 \hline
 64 & 78.77 & 78.45 & 78.37 & 79.16 & 81.45 & 80.69 & 81.45 & 81.5 \\
 \hline
 32 & 78.45 & 79.01 & 78.69 & 80.66 & 81.56 & 80.58 & 81.18 & 81.34  \\
 \hline
 16 & 79.95 & 79.56 & 79.79 & 79.72 & 81.28 & 81.66 & 81.28 & 80.96  \\
 \hline
\end{tabular}
\caption{\label{tab:mmlu_abalation} Accuracy on LM evaluation harness tasks on Llama2-70B model.}
\end{table}

%\section{MSE Studies}
%\textcolor{red}{TODO}


\subsection{Number Formats and Quantization Method}
\label{subsec:numFormats_quantMethod}
\subsubsection{Integer Format}
An $n$-bit signed integer (INT) is typically represented with a 2s-complement format \citep{yao2022zeroquant,xiao2023smoothquant,dai2021vsq}, where the most significant bit denotes the sign.

\subsubsection{Floating Point Format}
An $n$-bit signed floating point (FP) number $x$ comprises of a 1-bit sign ($x_{\mathrm{sign}}$), $B_m$-bit mantissa ($x_{\mathrm{mant}}$) and $B_e$-bit exponent ($x_{\mathrm{exp}}$) such that $B_m+B_e=n-1$. The associated constant exponent bias ($E_{\mathrm{bias}}$) is computed as $(2^{{B_e}-1}-1)$. We denote this format as $E_{B_e}M_{B_m}$.  

\subsubsection{Quantization Scheme}
\label{subsec:quant_method}
A quantization scheme dictates how a given unquantized tensor is converted to its quantized representation. We consider FP formats for the purpose of illustration. Given an unquantized tensor $\bm{X}$ and an FP format $E_{B_e}M_{B_m}$, we first, we compute the quantization scale factor $s_X$ that maps the maximum absolute value of $\bm{X}$ to the maximum quantization level of the $E_{B_e}M_{B_m}$ format as follows:
\begin{align}
\label{eq:sf}
    s_X = \frac{\mathrm{max}(|\bm{X}|)}{\mathrm{max}(E_{B_e}M_{B_m})}
\end{align}
In the above equation, $|\cdot|$ denotes the absolute value function.

Next, we scale $\bm{X}$ by $s_X$ and quantize it to $\hat{\bm{X}}$ by rounding it to the nearest quantization level of $E_{B_e}M_{B_m}$ as:

\begin{align}
\label{eq:tensor_quant}
    \hat{\bm{X}} = \text{round-to-nearest}\left(\frac{\bm{X}}{s_X}, E_{B_e}M_{B_m}\right)
\end{align}

We perform dynamic max-scaled quantization \citep{wu2020integer}, where the scale factor $s$ for activations is dynamically computed during runtime.

\subsection{Vector Scaled Quantization}
\begin{wrapfigure}{r}{0.35\linewidth}
  \centering
  \includegraphics[width=\linewidth]{sections/figures/vsquant.jpg}
  \caption{\small Vectorwise decomposition for per-vector scaled quantization (VSQ \citep{dai2021vsq}).}
  \label{fig:vsquant}
\end{wrapfigure}
During VSQ \citep{dai2021vsq}, the operand tensors are decomposed into 1D vectors in a hardware friendly manner as shown in Figure \ref{fig:vsquant}. Since the decomposed tensors are used as operands in matrix multiplications during inference, it is beneficial to perform this decomposition along the reduction dimension of the multiplication. The vectorwise quantization is performed similar to tensorwise quantization described in Equations \ref{eq:sf} and \ref{eq:tensor_quant}, where a scale factor $s_v$ is required for each vector $\bm{v}$ that maps the maximum absolute value of that vector to the maximum quantization level. While smaller vector lengths can lead to larger accuracy gains, the associated memory and computational overheads due to the per-vector scale factors increases. To alleviate these overheads, VSQ \citep{dai2021vsq} proposed a second level quantization of the per-vector scale factors to unsigned integers, while MX \citep{rouhani2023shared} quantizes them to integer powers of 2 (denoted as $2^{INT}$).

\subsubsection{MX Format}
The MX format proposed in \citep{rouhani2023microscaling} introduces the concept of sub-block shifting. For every two scalar elements of $b$-bits each, there is a shared exponent bit. The value of this exponent bit is determined through an empirical analysis that targets minimizing quantization MSE. We note that the FP format $E_{1}M_{b}$ is strictly better than MX from an accuracy perspective since it allocates a dedicated exponent bit to each scalar as opposed to sharing it across two scalars. Therefore, we conservatively bound the accuracy of a $b+2$-bit signed MX format with that of a $E_{1}M_{b}$ format in our comparisons. For instance, we use E1M2 format as a proxy for MX4.

\begin{figure}
    \centering
    \includegraphics[width=1\linewidth]{sections//figures/BlockFormats.pdf}
    \caption{\small Comparing LO-BCQ to MX format.}
    \label{fig:block_formats}
\end{figure}

Figure \ref{fig:block_formats} compares our $4$-bit LO-BCQ block format to MX \citep{rouhani2023microscaling}. As shown, both LO-BCQ and MX decompose a given operand tensor into block arrays and each block array into blocks. Similar to MX, we find that per-block quantization ($L_b < L_A$) leads to better accuracy due to increased flexibility. While MX achieves this through per-block $1$-bit micro-scales, we associate a dedicated codebook to each block through a per-block codebook selector. Further, MX quantizes the per-block array scale-factor to E8M0 format without per-tensor scaling. In contrast during LO-BCQ, we find that per-tensor scaling combined with quantization of per-block array scale-factor to E4M3 format results in superior inference accuracy across models. 

%\subsection{Lloyd-Max Algorithm}
\label{subsec:Lloyd-Max}
For a given quantization bitwidth $B$ and an operand $\bm{X}$, the Lloyd-Max algorithm finds $2^B$ quantization levels $\{\hat{x}_i\}_{i=1}^{2^B}$ such that quantizing $\bm{X}$ by rounding each scalar in $\bm{X}$ to the nearest quantization level minimizes the quantization MSE. 

The algorithm starts with an initial guess of quantization levels and then iteratively computes quantization thresholds $\{\tau_i\}_{i=1}^{2^B-1}$ and updates quantization levels $\{\hat{x}_i\}_{i=1}^{2^B}$. Specifically, at iteration $n$, thresholds are set to the midpoints of the previous iteration's levels:
\begin{align*}
    \tau_i^{(n)}=\frac{\hat{x}_i^{(n-1)}+\hat{x}_{i+1}^{(n-1)}}2 \text{ for } i=1\ldots 2^B-1
\end{align*}
Subsequently, the quantization levels are re-computed as conditional means of the data regions defined by the new thresholds:
\begin{align*}
    \hat{x}_i^{(n)}=\mathbb{E}\left[ \bm{X} \big| \bm{X}\in [\tau_{i-1}^{(n)},\tau_i^{(n)}] \right] \text{ for } i=1\ldots 2^B
\end{align*}
where to satisfy boundary conditions we have $\tau_0=-\infty$ and $\tau_{2^B}=\infty$. The algorithm iterates the above steps until convergence.

Figure \ref{fig:lm_quant} compares the quantization levels of a $7$-bit floating point (E3M3) quantizer (left) to a $7$-bit Lloyd-Max quantizer (right) when quantizing a layer of weights from the GPT3-126M model at a per-tensor granularity. As shown, the Lloyd-Max quantizer achieves substantially lower quantization MSE. Further, Table \ref{tab:FP7_vs_LM7} shows the superior perplexity achieved by Lloyd-Max quantizers for bitwidths of $7$, $6$ and $5$. The difference between the quantizers is clear at 5 bits, where per-tensor FP quantization incurs a drastic and unacceptable increase in perplexity, while Lloyd-Max quantization incurs a much smaller increase. Nevertheless, we note that even the optimal Lloyd-Max quantizer incurs a notable ($\sim 1.5$) increase in perplexity due to the coarse granularity of quantization. 

\begin{figure}[h]
  \centering
  \includegraphics[width=0.7\linewidth]{sections/figures/LM7_FP7.pdf}
  \caption{\small Quantization levels and the corresponding quantization MSE of Floating Point (left) vs Lloyd-Max (right) Quantizers for a layer of weights in the GPT3-126M model.}
  \label{fig:lm_quant}
\end{figure}

\begin{table}[h]\scriptsize
\begin{center}
\caption{\label{tab:FP7_vs_LM7} \small Comparing perplexity (lower is better) achieved by floating point quantizers and Lloyd-Max quantizers on a GPT3-126M model for the Wikitext-103 dataset.}
\begin{tabular}{c|cc|c}
\hline
 \multirow{2}{*}{\textbf{Bitwidth}} & \multicolumn{2}{|c|}{\textbf{Floating-Point Quantizer}} & \textbf{Lloyd-Max Quantizer} \\
 & Best Format & Wikitext-103 Perplexity & Wikitext-103 Perplexity \\
\hline
7 & E3M3 & 18.32 & 18.27 \\
6 & E3M2 & 19.07 & 18.51 \\
5 & E4M0 & 43.89 & 19.71 \\
\hline
\end{tabular}
\end{center}
\end{table}

\subsection{Proof of Local Optimality of LO-BCQ}
\label{subsec:lobcq_opt_proof}
For a given block $\bm{b}_j$, the quantization MSE during LO-BCQ can be empirically evaluated as $\frac{1}{L_b}\lVert \bm{b}_j- \bm{\hat{b}}_j\rVert^2_2$ where $\bm{\hat{b}}_j$ is computed from equation (\ref{eq:clustered_quantization_definition}) as $C_{f(\bm{b}_j)}(\bm{b}_j)$. Further, for a given block cluster $\mathcal{B}_i$, we compute the quantization MSE as $\frac{1}{|\mathcal{B}_{i}|}\sum_{\bm{b} \in \mathcal{B}_{i}} \frac{1}{L_b}\lVert \bm{b}- C_i^{(n)}(\bm{b})\rVert^2_2$. Therefore, at the end of iteration $n$, we evaluate the overall quantization MSE $J^{(n)}$ for a given operand $\bm{X}$ composed of $N_c$ block clusters as:
\begin{align*}
    \label{eq:mse_iter_n}
    J^{(n)} = \frac{1}{N_c} \sum_{i=1}^{N_c} \frac{1}{|\mathcal{B}_{i}^{(n)}|}\sum_{\bm{v} \in \mathcal{B}_{i}^{(n)}} \frac{1}{L_b}\lVert \bm{b}- B_i^{(n)}(\bm{b})\rVert^2_2
\end{align*}

At the end of iteration $n$, the codebooks are updated from $\mathcal{C}^{(n-1)}$ to $\mathcal{C}^{(n)}$. However, the mapping of a given vector $\bm{b}_j$ to quantizers $\mathcal{C}^{(n)}$ remains as  $f^{(n)}(\bm{b}_j)$. At the next iteration, during the vector clustering step, $f^{(n+1)}(\bm{b}_j)$ finds new mapping of $\bm{b}_j$ to updated codebooks $\mathcal{C}^{(n)}$ such that the quantization MSE over the candidate codebooks is minimized. Therefore, we obtain the following result for $\bm{b}_j$:
\begin{align*}
\frac{1}{L_b}\lVert \bm{b}_j - C_{f^{(n+1)}(\bm{b}_j)}^{(n)}(\bm{b}_j)\rVert^2_2 \le \frac{1}{L_b}\lVert \bm{b}_j - C_{f^{(n)}(\bm{b}_j)}^{(n)}(\bm{b}_j)\rVert^2_2
\end{align*}

That is, quantizing $\bm{b}_j$ at the end of the block clustering step of iteration $n+1$ results in lower quantization MSE compared to quantizing at the end of iteration $n$. Since this is true for all $\bm{b} \in \bm{X}$, we assert the following:
\begin{equation}
\begin{split}
\label{eq:mse_ineq_1}
    \tilde{J}^{(n+1)} &= \frac{1}{N_c} \sum_{i=1}^{N_c} \frac{1}{|\mathcal{B}_{i}^{(n+1)}|}\sum_{\bm{b} \in \mathcal{B}_{i}^{(n+1)}} \frac{1}{L_b}\lVert \bm{b} - C_i^{(n)}(b)\rVert^2_2 \le J^{(n)}
\end{split}
\end{equation}
where $\tilde{J}^{(n+1)}$ is the the quantization MSE after the vector clustering step at iteration $n+1$.

Next, during the codebook update step (\ref{eq:quantizers_update}) at iteration $n+1$, the per-cluster codebooks $\mathcal{C}^{(n)}$ are updated to $\mathcal{C}^{(n+1)}$ by invoking the Lloyd-Max algorithm \citep{Lloyd}. We know that for any given value distribution, the Lloyd-Max algorithm minimizes the quantization MSE. Therefore, for a given vector cluster $\mathcal{B}_i$ we obtain the following result:

\begin{equation}
    \frac{1}{|\mathcal{B}_{i}^{(n+1)}|}\sum_{\bm{b} \in \mathcal{B}_{i}^{(n+1)}} \frac{1}{L_b}\lVert \bm{b}- C_i^{(n+1)}(\bm{b})\rVert^2_2 \le \frac{1}{|\mathcal{B}_{i}^{(n+1)}|}\sum_{\bm{b} \in \mathcal{B}_{i}^{(n+1)}} \frac{1}{L_b}\lVert \bm{b}- C_i^{(n)}(\bm{b})\rVert^2_2
\end{equation}

The above equation states that quantizing the given block cluster $\mathcal{B}_i$ after updating the associated codebook from $C_i^{(n)}$ to $C_i^{(n+1)}$ results in lower quantization MSE. Since this is true for all the block clusters, we derive the following result: 
\begin{equation}
\begin{split}
\label{eq:mse_ineq_2}
     J^{(n+1)} &= \frac{1}{N_c} \sum_{i=1}^{N_c} \frac{1}{|\mathcal{B}_{i}^{(n+1)}|}\sum_{\bm{b} \in \mathcal{B}_{i}^{(n+1)}} \frac{1}{L_b}\lVert \bm{b}- C_i^{(n+1)}(\bm{b})\rVert^2_2  \le \tilde{J}^{(n+1)}   
\end{split}
\end{equation}

Following (\ref{eq:mse_ineq_1}) and (\ref{eq:mse_ineq_2}), we find that the quantization MSE is non-increasing for each iteration, that is, $J^{(1)} \ge J^{(2)} \ge J^{(3)} \ge \ldots \ge J^{(M)}$ where $M$ is the maximum number of iterations. 
%Therefore, we can say that if the algorithm converges, then it must be that it has converged to a local minimum. 
\hfill $\blacksquare$


\begin{figure}
    \begin{center}
    \includegraphics[width=0.5\textwidth]{sections//figures/mse_vs_iter.pdf}
    \end{center}
    \caption{\small NMSE vs iterations during LO-BCQ compared to other block quantization proposals}
    \label{fig:nmse_vs_iter}
\end{figure}

Figure \ref{fig:nmse_vs_iter} shows the empirical convergence of LO-BCQ across several block lengths and number of codebooks. Also, the MSE achieved by LO-BCQ is compared to baselines such as MXFP and VSQ. As shown, LO-BCQ converges to a lower MSE than the baselines. Further, we achieve better convergence for larger number of codebooks ($N_c$) and for a smaller block length ($L_b$), both of which increase the bitwidth of BCQ (see Eq \ref{eq:bitwidth_bcq}).


\subsection{Additional Accuracy Results}
%Table \ref{tab:lobcq_config} lists the various LOBCQ configurations and their corresponding bitwidths.
\begin{table}
\setlength{\tabcolsep}{4.75pt}
\begin{center}
\caption{\label{tab:lobcq_config} Various LO-BCQ configurations and their bitwidths.}
\begin{tabular}{|c||c|c|c|c||c|c||c|} 
\hline
 & \multicolumn{4}{|c||}{$L_b=8$} & \multicolumn{2}{|c||}{$L_b=4$} & $L_b=2$ \\
 \hline
 \backslashbox{$L_A$\kern-1em}{\kern-1em$N_c$} & 2 & 4 & 8 & 16 & 2 & 4 & 2 \\
 \hline
 64 & 4.25 & 4.375 & 4.5 & 4.625 & 4.375 & 4.625 & 4.625\\
 \hline
 32 & 4.375 & 4.5 & 4.625& 4.75 & 4.5 & 4.75 & 4.75 \\
 \hline
 16 & 4.625 & 4.75& 4.875 & 5 & 4.75 & 5 & 5 \\
 \hline
\end{tabular}
\end{center}
\end{table}

%\subsection{Perplexity achieved by various LO-BCQ configurations on Wikitext-103 dataset}

\begin{table} \centering
\begin{tabular}{|c||c|c|c|c||c|c||c|} 
\hline
 $L_b \rightarrow$& \multicolumn{4}{c||}{8} & \multicolumn{2}{c||}{4} & 2\\
 \hline
 \backslashbox{$L_A$\kern-1em}{\kern-1em$N_c$} & 2 & 4 & 8 & 16 & 2 & 4 & 2  \\
 %$N_c \rightarrow$ & 2 & 4 & 8 & 16 & 2 & 4 & 2 \\
 \hline
 \hline
 \multicolumn{8}{c}{GPT3-1.3B (FP32 PPL = 9.98)} \\ 
 \hline
 \hline
 64 & 10.40 & 10.23 & 10.17 & 10.15 &  10.28 & 10.18 & 10.19 \\
 \hline
 32 & 10.25 & 10.20 & 10.15 & 10.12 &  10.23 & 10.17 & 10.17 \\
 \hline
 16 & 10.22 & 10.16 & 10.10 & 10.09 &  10.21 & 10.14 & 10.16 \\
 \hline
  \hline
 \multicolumn{8}{c}{GPT3-8B (FP32 PPL = 7.38)} \\ 
 \hline
 \hline
 64 & 7.61 & 7.52 & 7.48 &  7.47 &  7.55 &  7.49 & 7.50 \\
 \hline
 32 & 7.52 & 7.50 & 7.46 &  7.45 &  7.52 &  7.48 & 7.48  \\
 \hline
 16 & 7.51 & 7.48 & 7.44 &  7.44 &  7.51 &  7.49 & 7.47  \\
 \hline
\end{tabular}
\caption{\label{tab:ppl_gpt3_abalation} Wikitext-103 perplexity across GPT3-1.3B and 8B models.}
\end{table}

\begin{table} \centering
\begin{tabular}{|c||c|c|c|c||} 
\hline
 $L_b \rightarrow$& \multicolumn{4}{c||}{8}\\
 \hline
 \backslashbox{$L_A$\kern-1em}{\kern-1em$N_c$} & 2 & 4 & 8 & 16 \\
 %$N_c \rightarrow$ & 2 & 4 & 8 & 16 & 2 & 4 & 2 \\
 \hline
 \hline
 \multicolumn{5}{|c|}{Llama2-7B (FP32 PPL = 5.06)} \\ 
 \hline
 \hline
 64 & 5.31 & 5.26 & 5.19 & 5.18  \\
 \hline
 32 & 5.23 & 5.25 & 5.18 & 5.15  \\
 \hline
 16 & 5.23 & 5.19 & 5.16 & 5.14  \\
 \hline
 \multicolumn{5}{|c|}{Nemotron4-15B (FP32 PPL = 5.87)} \\ 
 \hline
 \hline
 64  & 6.3 & 6.20 & 6.13 & 6.08  \\
 \hline
 32  & 6.24 & 6.12 & 6.07 & 6.03  \\
 \hline
 16  & 6.12 & 6.14 & 6.04 & 6.02  \\
 \hline
 \multicolumn{5}{|c|}{Nemotron4-340B (FP32 PPL = 3.48)} \\ 
 \hline
 \hline
 64 & 3.67 & 3.62 & 3.60 & 3.59 \\
 \hline
 32 & 3.63 & 3.61 & 3.59 & 3.56 \\
 \hline
 16 & 3.61 & 3.58 & 3.57 & 3.55 \\
 \hline
\end{tabular}
\caption{\label{tab:ppl_llama7B_nemo15B} Wikitext-103 perplexity compared to FP32 baseline in Llama2-7B and Nemotron4-15B, 340B models}
\end{table}

%\subsection{Perplexity achieved by various LO-BCQ configurations on MMLU dataset}


\begin{table} \centering
\begin{tabular}{|c||c|c|c|c||c|c|c|c|} 
\hline
 $L_b \rightarrow$& \multicolumn{4}{c||}{8} & \multicolumn{4}{c||}{8}\\
 \hline
 \backslashbox{$L_A$\kern-1em}{\kern-1em$N_c$} & 2 & 4 & 8 & 16 & 2 & 4 & 8 & 16  \\
 %$N_c \rightarrow$ & 2 & 4 & 8 & 16 & 2 & 4 & 2 \\
 \hline
 \hline
 \multicolumn{5}{|c|}{Llama2-7B (FP32 Accuracy = 45.8\%)} & \multicolumn{4}{|c|}{Llama2-70B (FP32 Accuracy = 69.12\%)} \\ 
 \hline
 \hline
 64 & 43.9 & 43.4 & 43.9 & 44.9 & 68.07 & 68.27 & 68.17 & 68.75 \\
 \hline
 32 & 44.5 & 43.8 & 44.9 & 44.5 & 68.37 & 68.51 & 68.35 & 68.27  \\
 \hline
 16 & 43.9 & 42.7 & 44.9 & 45 & 68.12 & 68.77 & 68.31 & 68.59  \\
 \hline
 \hline
 \multicolumn{5}{|c|}{GPT3-22B (FP32 Accuracy = 38.75\%)} & \multicolumn{4}{|c|}{Nemotron4-15B (FP32 Accuracy = 64.3\%)} \\ 
 \hline
 \hline
 64 & 36.71 & 38.85 & 38.13 & 38.92 & 63.17 & 62.36 & 63.72 & 64.09 \\
 \hline
 32 & 37.95 & 38.69 & 39.45 & 38.34 & 64.05 & 62.30 & 63.8 & 64.33  \\
 \hline
 16 & 38.88 & 38.80 & 38.31 & 38.92 & 63.22 & 63.51 & 63.93 & 64.43  \\
 \hline
\end{tabular}
\caption{\label{tab:mmlu_abalation} Accuracy on MMLU dataset across GPT3-22B, Llama2-7B, 70B and Nemotron4-15B models.}
\end{table}


%\subsection{Perplexity achieved by various LO-BCQ configurations on LM evaluation harness}

\begin{table} \centering
\begin{tabular}{|c||c|c|c|c||c|c|c|c|} 
\hline
 $L_b \rightarrow$& \multicolumn{4}{c||}{8} & \multicolumn{4}{c||}{8}\\
 \hline
 \backslashbox{$L_A$\kern-1em}{\kern-1em$N_c$} & 2 & 4 & 8 & 16 & 2 & 4 & 8 & 16  \\
 %$N_c \rightarrow$ & 2 & 4 & 8 & 16 & 2 & 4 & 2 \\
 \hline
 \hline
 \multicolumn{5}{|c|}{Race (FP32 Accuracy = 37.51\%)} & \multicolumn{4}{|c|}{Boolq (FP32 Accuracy = 64.62\%)} \\ 
 \hline
 \hline
 64 & 36.94 & 37.13 & 36.27 & 37.13 & 63.73 & 62.26 & 63.49 & 63.36 \\
 \hline
 32 & 37.03 & 36.36 & 36.08 & 37.03 & 62.54 & 63.51 & 63.49 & 63.55  \\
 \hline
 16 & 37.03 & 37.03 & 36.46 & 37.03 & 61.1 & 63.79 & 63.58 & 63.33  \\
 \hline
 \hline
 \multicolumn{5}{|c|}{Winogrande (FP32 Accuracy = 58.01\%)} & \multicolumn{4}{|c|}{Piqa (FP32 Accuracy = 74.21\%)} \\ 
 \hline
 \hline
 64 & 58.17 & 57.22 & 57.85 & 58.33 & 73.01 & 73.07 & 73.07 & 72.80 \\
 \hline
 32 & 59.12 & 58.09 & 57.85 & 58.41 & 73.01 & 73.94 & 72.74 & 73.18  \\
 \hline
 16 & 57.93 & 58.88 & 57.93 & 58.56 & 73.94 & 72.80 & 73.01 & 73.94  \\
 \hline
\end{tabular}
\caption{\label{tab:mmlu_abalation} Accuracy on LM evaluation harness tasks on GPT3-1.3B model.}
\end{table}

\begin{table} \centering
\begin{tabular}{|c||c|c|c|c||c|c|c|c|} 
\hline
 $L_b \rightarrow$& \multicolumn{4}{c||}{8} & \multicolumn{4}{c||}{8}\\
 \hline
 \backslashbox{$L_A$\kern-1em}{\kern-1em$N_c$} & 2 & 4 & 8 & 16 & 2 & 4 & 8 & 16  \\
 %$N_c \rightarrow$ & 2 & 4 & 8 & 16 & 2 & 4 & 2 \\
 \hline
 \hline
 \multicolumn{5}{|c|}{Race (FP32 Accuracy = 41.34\%)} & \multicolumn{4}{|c|}{Boolq (FP32 Accuracy = 68.32\%)} \\ 
 \hline
 \hline
 64 & 40.48 & 40.10 & 39.43 & 39.90 & 69.20 & 68.41 & 69.45 & 68.56 \\
 \hline
 32 & 39.52 & 39.52 & 40.77 & 39.62 & 68.32 & 67.43 & 68.17 & 69.30  \\
 \hline
 16 & 39.81 & 39.71 & 39.90 & 40.38 & 68.10 & 66.33 & 69.51 & 69.42  \\
 \hline
 \hline
 \multicolumn{5}{|c|}{Winogrande (FP32 Accuracy = 67.88\%)} & \multicolumn{4}{|c|}{Piqa (FP32 Accuracy = 78.78\%)} \\ 
 \hline
 \hline
 64 & 66.85 & 66.61 & 67.72 & 67.88 & 77.31 & 77.42 & 77.75 & 77.64 \\
 \hline
 32 & 67.25 & 67.72 & 67.72 & 67.00 & 77.31 & 77.04 & 77.80 & 77.37  \\
 \hline
 16 & 68.11 & 68.90 & 67.88 & 67.48 & 77.37 & 78.13 & 78.13 & 77.69  \\
 \hline
\end{tabular}
\caption{\label{tab:mmlu_abalation} Accuracy on LM evaluation harness tasks on GPT3-8B model.}
\end{table}

\begin{table} \centering
\begin{tabular}{|c||c|c|c|c||c|c|c|c|} 
\hline
 $L_b \rightarrow$& \multicolumn{4}{c||}{8} & \multicolumn{4}{c||}{8}\\
 \hline
 \backslashbox{$L_A$\kern-1em}{\kern-1em$N_c$} & 2 & 4 & 8 & 16 & 2 & 4 & 8 & 16  \\
 %$N_c \rightarrow$ & 2 & 4 & 8 & 16 & 2 & 4 & 2 \\
 \hline
 \hline
 \multicolumn{5}{|c|}{Race (FP32 Accuracy = 40.67\%)} & \multicolumn{4}{|c|}{Boolq (FP32 Accuracy = 76.54\%)} \\ 
 \hline
 \hline
 64 & 40.48 & 40.10 & 39.43 & 39.90 & 75.41 & 75.11 & 77.09 & 75.66 \\
 \hline
 32 & 39.52 & 39.52 & 40.77 & 39.62 & 76.02 & 76.02 & 75.96 & 75.35  \\
 \hline
 16 & 39.81 & 39.71 & 39.90 & 40.38 & 75.05 & 73.82 & 75.72 & 76.09  \\
 \hline
 \hline
 \multicolumn{5}{|c|}{Winogrande (FP32 Accuracy = 70.64\%)} & \multicolumn{4}{|c|}{Piqa (FP32 Accuracy = 79.16\%)} \\ 
 \hline
 \hline
 64 & 69.14 & 70.17 & 70.17 & 70.56 & 78.24 & 79.00 & 78.62 & 78.73 \\
 \hline
 32 & 70.96 & 69.69 & 71.27 & 69.30 & 78.56 & 79.49 & 79.16 & 78.89  \\
 \hline
 16 & 71.03 & 69.53 & 69.69 & 70.40 & 78.13 & 79.16 & 79.00 & 79.00  \\
 \hline
\end{tabular}
\caption{\label{tab:mmlu_abalation} Accuracy on LM evaluation harness tasks on GPT3-22B model.}
\end{table}

\begin{table} \centering
\begin{tabular}{|c||c|c|c|c||c|c|c|c|} 
\hline
 $L_b \rightarrow$& \multicolumn{4}{c||}{8} & \multicolumn{4}{c||}{8}\\
 \hline
 \backslashbox{$L_A$\kern-1em}{\kern-1em$N_c$} & 2 & 4 & 8 & 16 & 2 & 4 & 8 & 16  \\
 %$N_c \rightarrow$ & 2 & 4 & 8 & 16 & 2 & 4 & 2 \\
 \hline
 \hline
 \multicolumn{5}{|c|}{Race (FP32 Accuracy = 44.4\%)} & \multicolumn{4}{|c|}{Boolq (FP32 Accuracy = 79.29\%)} \\ 
 \hline
 \hline
 64 & 42.49 & 42.51 & 42.58 & 43.45 & 77.58 & 77.37 & 77.43 & 78.1 \\
 \hline
 32 & 43.35 & 42.49 & 43.64 & 43.73 & 77.86 & 75.32 & 77.28 & 77.86  \\
 \hline
 16 & 44.21 & 44.21 & 43.64 & 42.97 & 78.65 & 77 & 76.94 & 77.98  \\
 \hline
 \hline
 \multicolumn{5}{|c|}{Winogrande (FP32 Accuracy = 69.38\%)} & \multicolumn{4}{|c|}{Piqa (FP32 Accuracy = 78.07\%)} \\ 
 \hline
 \hline
 64 & 68.9 & 68.43 & 69.77 & 68.19 & 77.09 & 76.82 & 77.09 & 77.86 \\
 \hline
 32 & 69.38 & 68.51 & 68.82 & 68.90 & 78.07 & 76.71 & 78.07 & 77.86  \\
 \hline
 16 & 69.53 & 67.09 & 69.38 & 68.90 & 77.37 & 77.8 & 77.91 & 77.69  \\
 \hline
\end{tabular}
\caption{\label{tab:mmlu_abalation} Accuracy on LM evaluation harness tasks on Llama2-7B model.}
\end{table}

\begin{table} \centering
\begin{tabular}{|c||c|c|c|c||c|c|c|c|} 
\hline
 $L_b \rightarrow$& \multicolumn{4}{c||}{8} & \multicolumn{4}{c||}{8}\\
 \hline
 \backslashbox{$L_A$\kern-1em}{\kern-1em$N_c$} & 2 & 4 & 8 & 16 & 2 & 4 & 8 & 16  \\
 %$N_c \rightarrow$ & 2 & 4 & 8 & 16 & 2 & 4 & 2 \\
 \hline
 \hline
 \multicolumn{5}{|c|}{Race (FP32 Accuracy = 48.8\%)} & \multicolumn{4}{|c|}{Boolq (FP32 Accuracy = 85.23\%)} \\ 
 \hline
 \hline
 64 & 49.00 & 49.00 & 49.28 & 48.71 & 82.82 & 84.28 & 84.03 & 84.25 \\
 \hline
 32 & 49.57 & 48.52 & 48.33 & 49.28 & 83.85 & 84.46 & 84.31 & 84.93  \\
 \hline
 16 & 49.85 & 49.09 & 49.28 & 48.99 & 85.11 & 84.46 & 84.61 & 83.94  \\
 \hline
 \hline
 \multicolumn{5}{|c|}{Winogrande (FP32 Accuracy = 79.95\%)} & \multicolumn{4}{|c|}{Piqa (FP32 Accuracy = 81.56\%)} \\ 
 \hline
 \hline
 64 & 78.77 & 78.45 & 78.37 & 79.16 & 81.45 & 80.69 & 81.45 & 81.5 \\
 \hline
 32 & 78.45 & 79.01 & 78.69 & 80.66 & 81.56 & 80.58 & 81.18 & 81.34  \\
 \hline
 16 & 79.95 & 79.56 & 79.79 & 79.72 & 81.28 & 81.66 & 81.28 & 80.96  \\
 \hline
\end{tabular}
\caption{\label{tab:mmlu_abalation} Accuracy on LM evaluation harness tasks on Llama2-70B model.}
\end{table}

%\section{MSE Studies}
%\textcolor{red}{TODO}


\subsection{Number Formats and Quantization Method}
\label{subsec:numFormats_quantMethod}
\subsubsection{Integer Format}
An $n$-bit signed integer (INT) is typically represented with a 2s-complement format \citep{yao2022zeroquant,xiao2023smoothquant,dai2021vsq}, where the most significant bit denotes the sign.

\subsubsection{Floating Point Format}
An $n$-bit signed floating point (FP) number $x$ comprises of a 1-bit sign ($x_{\mathrm{sign}}$), $B_m$-bit mantissa ($x_{\mathrm{mant}}$) and $B_e$-bit exponent ($x_{\mathrm{exp}}$) such that $B_m+B_e=n-1$. The associated constant exponent bias ($E_{\mathrm{bias}}$) is computed as $(2^{{B_e}-1}-1)$. We denote this format as $E_{B_e}M_{B_m}$.  

\subsubsection{Quantization Scheme}
\label{subsec:quant_method}
A quantization scheme dictates how a given unquantized tensor is converted to its quantized representation. We consider FP formats for the purpose of illustration. Given an unquantized tensor $\bm{X}$ and an FP format $E_{B_e}M_{B_m}$, we first, we compute the quantization scale factor $s_X$ that maps the maximum absolute value of $\bm{X}$ to the maximum quantization level of the $E_{B_e}M_{B_m}$ format as follows:
\begin{align}
\label{eq:sf}
    s_X = \frac{\mathrm{max}(|\bm{X}|)}{\mathrm{max}(E_{B_e}M_{B_m})}
\end{align}
In the above equation, $|\cdot|$ denotes the absolute value function.

Next, we scale $\bm{X}$ by $s_X$ and quantize it to $\hat{\bm{X}}$ by rounding it to the nearest quantization level of $E_{B_e}M_{B_m}$ as:

\begin{align}
\label{eq:tensor_quant}
    \hat{\bm{X}} = \text{round-to-nearest}\left(\frac{\bm{X}}{s_X}, E_{B_e}M_{B_m}\right)
\end{align}

We perform dynamic max-scaled quantization \citep{wu2020integer}, where the scale factor $s$ for activations is dynamically computed during runtime.

\subsection{Vector Scaled Quantization}
\begin{wrapfigure}{r}{0.35\linewidth}
  \centering
  \includegraphics[width=\linewidth]{sections/figures/vsquant.jpg}
  \caption{\small Vectorwise decomposition for per-vector scaled quantization (VSQ \citep{dai2021vsq}).}
  \label{fig:vsquant}
\end{wrapfigure}
During VSQ \citep{dai2021vsq}, the operand tensors are decomposed into 1D vectors in a hardware friendly manner as shown in Figure \ref{fig:vsquant}. Since the decomposed tensors are used as operands in matrix multiplications during inference, it is beneficial to perform this decomposition along the reduction dimension of the multiplication. The vectorwise quantization is performed similar to tensorwise quantization described in Equations \ref{eq:sf} and \ref{eq:tensor_quant}, where a scale factor $s_v$ is required for each vector $\bm{v}$ that maps the maximum absolute value of that vector to the maximum quantization level. While smaller vector lengths can lead to larger accuracy gains, the associated memory and computational overheads due to the per-vector scale factors increases. To alleviate these overheads, VSQ \citep{dai2021vsq} proposed a second level quantization of the per-vector scale factors to unsigned integers, while MX \citep{rouhani2023shared} quantizes them to integer powers of 2 (denoted as $2^{INT}$).

\subsubsection{MX Format}
The MX format proposed in \citep{rouhani2023microscaling} introduces the concept of sub-block shifting. For every two scalar elements of $b$-bits each, there is a shared exponent bit. The value of this exponent bit is determined through an empirical analysis that targets minimizing quantization MSE. We note that the FP format $E_{1}M_{b}$ is strictly better than MX from an accuracy perspective since it allocates a dedicated exponent bit to each scalar as opposed to sharing it across two scalars. Therefore, we conservatively bound the accuracy of a $b+2$-bit signed MX format with that of a $E_{1}M_{b}$ format in our comparisons. For instance, we use E1M2 format as a proxy for MX4.

\begin{figure}
    \centering
    \includegraphics[width=1\linewidth]{sections//figures/BlockFormats.pdf}
    \caption{\small Comparing LO-BCQ to MX format.}
    \label{fig:block_formats}
\end{figure}

Figure \ref{fig:block_formats} compares our $4$-bit LO-BCQ block format to MX \citep{rouhani2023microscaling}. As shown, both LO-BCQ and MX decompose a given operand tensor into block arrays and each block array into blocks. Similar to MX, we find that per-block quantization ($L_b < L_A$) leads to better accuracy due to increased flexibility. While MX achieves this through per-block $1$-bit micro-scales, we associate a dedicated codebook to each block through a per-block codebook selector. Further, MX quantizes the per-block array scale-factor to E8M0 format without per-tensor scaling. In contrast during LO-BCQ, we find that per-tensor scaling combined with quantization of per-block array scale-factor to E4M3 format results in superior inference accuracy across models. 


\end{document}
