\section{Conclusion and Future work}
\label{sec:conclusion}

In this paper, we propose a novel human-in-the-loop robotic simulation platform, SymbioSim, by utilizing AR technology to deliver a highly realistic human-robot interaction experience. This system enables humans to better understand the robot’s intentions and develop effective interactive strategies through authentic interactions. Simultaneously, it allows for diverse testing scenarios and real-time collection of human feedback, enabling model adjustments and robot evolution. This facilitates continuous learning and mutual adaptation between humans and robots, driving human-robot symbiosis in a safe, efficient, and cost-effective way. Extensive experiments and user studies have demonstrated the effectiveness and practicality of the SymbioSim system and its modules. 

In the future, we plan to enhance our simulation platform in four aspects. First, we aim to improve the realism of the interaction process by integrating multi-modal sensors. For instance, portable haptic sensors will be used to allow users to feel pressure and temperature changes when interacting closely with the robot. Second, we will enhance the platform's capability to support a variety of feedback mechanisms, including automatic recognition of emotional and satisfaction cues from eye movements, facial expressions, and more detailed linguistic descriptions. Third, we intend to develop reinforcement learning-based online model fine-tuning methods, enabling the robot to rapidly adjust its behavior and strategies based on real-time human feedback. Finally, we will create more robust and generalized human-robot interaction and collaboration base models, further extending the platform's application scenarios and accelerating research progress in related fields. We welcome researchers to integrate their intelligent models into our platform for testing and optimization, which is one of key objectives underlying the development of this platform.



