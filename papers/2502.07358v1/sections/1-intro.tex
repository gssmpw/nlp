\section{Introduction}
\label{sec:intro}

With rapid advancements in robotics and artificial intelligence, robots are moving from industrial settings into everyday human environments~\cite{lecun2022path,gonzalez2021service}, becoming key assistants in services, home care, and healthcare. The ultimate goal for robots is to not only handle repetitive and dangerous tasks but also collaborate with humans to enhance efficiency and quality of life, achieving human-robot symbiosis. However, given the fundamental differences between humans and robots, achieving harmonious coexistence requires effective communication and interaction, encompassing both verbal and behavioral exchanges. While the success of large language models~\cite{roumeliotis2023chatgpt,touvron2023llama} has largely eliminated barriers to language-based communication between humans and machines, relying solely on language is insufficient for real-time, precise control of robot behavior in complex interaction scenarios. As a result, human-robot behavioral interaction has emerged as a critical area of research~\cite{safavi2024emerging}, aiming to enable robots to understand human intentions through the perception and reasoning for human actions, and to adjust their own behaviors accordingly, facilitating seamless and natural collaboration in shared physical spaces.


Just as humans usually learn and grow through practice, robots are required to gain knowledge and skills through continuous real-world interactions, progressively enhancing their intelligence and capabilities. Nevertheless, due to considerations such as high efficiency, low cost, and safety, much of the current research on embodied intelligence forgo direct training and testing physical robots. Instead, they rely on physical simulators~\cite{makoviychuk2021isaac, todorov2012mujoco} or robotic simulators~\cite{igibson, nasiriany2024robocasa}. These simulators form a solid foundation for critical tasks, such as navigation, manipulation, and control, providing significant advantages in the development of robotic systems. However, these simulators do not support human-robot interaction (HRI) simulations. Although recent works~\cite{HumanTHOR, liu2024collabsphere} have begun to acknowledge the importance of integrating humans into robotic simulation platforms, these approaches rely on virtual reality (VR) to drive virtual avatars for task-level collaboration, which limits human participants to engage in authentic and fine-grained pose-level interaction with robots.
To achieve seamless human-robot coexistence through natural and reliable interaction and collaboration, it is essential to develop a human-in-the-loop robotic simulation platform, which allows real humans to provide authentic feedback, foster deeper understanding and trust in robots, and ultimately drive the refinement and mutual growth of both humans and robots.

In this paper, we present SymbioSim, a pioneering human-in-the-loop simulation platform designed for achieving human-robot symbiosis, integrating both virtual and real environments. It could facilitate continuous bidirectional learning of both humans and robots through immersive interactive experiences. Fig.~\ref{fig:teaser} demonstrates the usage process of SymbioSim. In SymbioSim, users interact with virtual robots through portable augmented reality (AR) glasses, providing feedback via verbal communication or direct ratings. The system automatically collects this feedback and uses it to fine-tune the human-robot interaction model, enabling the robot's behaviors to progressively adapt to user preferences, creating a "better with use" robotic system. Simultaneously, users develop a deeper understanding of the robot's intentions and refine their interaction strategies, fostering greater trust and enabling the robot’s seamless integration into daily life. To deliver an authentic experience and bridge the gap between simulation and reality, SymbioSim is meticulously designed with a focus on realism, real-time capability, and adaptability. It provides a robust foundation for the development, validation, and optimization of human-in-the-loop embodied intelligence algorithms in a safe, efficient, and cost-effective manner.
In particular, to facilitate the validation of SymbioSim platform, we provide foundation models for humanoid robot motion generation, enabling the robot to respond promptly to human actions and successfully complete interaction or collaboration tasks. Unlike other offline motion generation methods~\cite{xu2023interdiff,xu2024regennet}, our human-robot interaction model is the first to support real-time, online robot action generation. This capability enables the robot to dynamically adapt to rapid changes in human behavior, significantly enhancing the fluidity and naturalness of the user experience. 



To assess the effectiveness and practicality of the SymbioSim platform, we conduct comprehensive user studies for human-robot interactions, involving participants of diverse genders and body types. Through user feedback ratings, we observed an upward trend in scores over multiple interactions, even when the interaction model remained unchanged. This indicates that users were gradually adapting to the robot’s interaction style and developing a better understanding of its intentions. After fine-tuning the interactive model based on users' feedback, the updated version garnered more favorable responses of users, showing that the closed-loop feedback mechanism allows the robot to better align with human interaction habits and preferences. These experiments highlight that SymbioSim effectively promotes bidirectional learning and adaptation between humans and robots. Furthermore, through comprehensive comparisons and ablation studies, we demonstrate the superiority of our interactive model, showcasing its state-of-the-art performance in both human-robot and robot-object interactions.

