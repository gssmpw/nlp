\section{Related Works}
\label{sec:related works}

\par In this section, we review research on UAV-assisted hybrid MEC-DC architecture, joint optimization for MEC and DC, and optimization approaches. Moreover, Table~\ref{tab:works comparison} illustrates the differences between the state-of-the-art works and this work.

\subsection{UAV-Assisted Hybrid MEC-DC Architecture}

\par UAVs have been widely applied to assist MEC services in some scenarios. For example, Yu \textit{et al.} ____ studied the optimization problem of collaborative services on UAV and edge clouds, and they proposed a system to control a UAV to provide MEC service in areas where existing edge clouds are inaccessible to IoT devices. Moreover, Zhan \textit{et al.} ____ developed a framework for a multi-UAV-assisted MEC system, where multiple UAVs with edge servers offer flexible computing support to IoT devices with time-sensitive requirements. Due to their flexibility and mobility, UAVs can directly collect data from GUs by flying close to them and have drawn significant attention from researchers. For instance, Dandapat \textit{et al.} ____ studied a multi-UAV-assisted DC network, optimizing the three-dimensional (3D) trajectory of the UAVs as well as resource allocation for DC from mobile nodes. Moreover, Liu \textit{et al.} ____ investigated the trajectory optimization problem of a UAV performing DC tasks in an area containing multiple monitoring regions and multiple base stations in a UAV-assisted DC system. 

\par Previous works on UAV-assisted MEC or UAV-assisted DC primarily focused on separate studies while ignoring the requirements for the scenarios containing both MEC and DC users. Specifically, in real-world scenarios, there may not only compute-intensive tasks to perform, but also large amounts of stored data that require UAV to perform additional data collection tasks. For example, in a smart city environment, UAVs may be required to assist with real-time video analysis for traffic management while simultaneously gathering sensor data from distributed IoT devices, such as air quality sensors, temperature monitors, or noise detectors. Therefore, the UAV assisted hybrid MEC-DC scenarios need further exploration.

\par There have been some studies involving both MEC and DC. For example, Zeng \textit{et al.} ____ investigated the UAV-assisted DC and MEC scenario, constructed a new theoretical model for DC rate, and defined the quality of requirement (QoR) for real-time processing. By optimizing the UAV trajectory, resource allocation and task duration, while meeting quality of service and UAV mobility constraints, they were able to reduce the energy consumption of the UAV and task completion time. However, in this work, the UAV only supports data acquisition and relies on nearby MEC servers to fulfill computational requirements. Liu \textit{et al.}____ investigated a space-air-ground power IoT system and proposed a UAV-enabled wireless power transfer (WPT) framework, where UAVs transfer energy to devices for DC via WPT, utilize MEC for data processing, and eventually forward the data to low earth orbit satellites. Subsequently, they aimed to minimize the average age of information (AoI) of devices by optimizing the number of UAV hovering positions, hovering locations, UAV-device connections, energy transmission, DC time, UAV computational resources, flight speed, and trajectory. However, in____, both DC and MEC are completed on the same UAV, and the collected data is actually the data needed for task offloading, which did not consider the conflict between separate MEC and DC systems. Moreover, neither ____ nor ____ considered the co-channel interference among UAVs.

\par In summary, existing works mainly focused on separate MEC or DC, or collecting data from the same device for MEC, few studies investigated the case where different GUs need to perform MEC and DC separately in the same scenario, which can coordinate the resource allocation of different vendors and achieve privacy protection of data. In this case, the main challenges are the interaction of the MEC and DC subsystems in hybrid scenarios and the UAV trajectory control in the presence of interference among multiple subsystems. This motivates us to investigate these effects and propose an effective approach.


\subsection{Joint Optimization for MEC and DC}

\par Due to the limited computing and storage resources carried by UAVs, resource allocation has been extensively studied to improve the system performance of wireless networks. For example, Wang \textit{et al.}____ investigated the most efficient placement of UAV, resource allocation, and computation offloading to minimize the total delay. Du \textit{et al.} ____ studied a UAV-assisted WPT and DC network and optimized the trajectories of two UAVs, the flight speed, the safe distance of the UAVs, and the energy constraints of each IoT device to increase the minimum DC throughput of the IoT devices. Moreover, Liu \textit{et al.}____ investigated a two-layer UAV-assisted maritime communication network. They proposed a DRL-based approach to reduce the communication and computation latency. However, these studies only considered separate optimization objectives such as MEC latency or DC throughput. 

\par There are other works that studied joint optimization of multiple objectives. For instance, Yu \textit{et al.} ____ investigated the potential of UAV-assisted wireless-powered IoT network and proposed an extended deep deterministic policy gradient (DDPG) algorithm. Their aim was to achieve a joint optimization objective that maximizes the total energy and data transmission rate while reducing the energy consumption of the UAV. Chen \textit{et al.}____ studied a multi-UAV-aided MEC network and optimized the UAV movement, user association, and user transmit power to jointly minimize the energy consumption and system latency.

\par However, the optimization goals in these works are not suitable for the considered scenario since they did not study the joint optimization objectives of MEC and DC. Thus, they are not capable of solving the challenge of the trade-off between MEC and DC caused by the interference. This motivates us to jointly minimize the MEC latency and maximize the data volume of DC.


\subsection{Optimization Approaches}

\par To solve complex optimization problems, researchers have been dedicated to the design of effective algorithms. For instance, Pervez \textit{et al.}____ investigated the joint optimization of energy and delay in a UAV-assisted MEC network and proposed a novel three-tier segment-by-segment optimization scheme based on block descent method, simplistic geometric waterfilling, and gradient descent method to solve the problem. However, traditional optimization and heuristic approaches usually require high computational complexity and are difficult to adapt to large-scale and real-time application scenarios, especially when the communication environment changes significantly with the environmental dynamics.

\par To mitigate this issue, DRL-based methods have been extensively studied as an alternative. For instance, Lee \textit{et al.}____ considered a multi-UAV-MEC network and proposed an independent proximal policy optimization (IPPO) model for learning task offloading and trajectory control of UAVs. In____, the authors proposed a SAC-based algorithm that maximizes the computation amount and fairness of terminals in a UAV-assisted WPT and MEC system. Li \textit{et al.}____ investigated a three-tier multi-UAV-assisted MEC system with random task arrivals, and they proposed a new heterogeneous federated multi-agent reinforcement learning framework, which jointly optimizes task offloading, UAV trajectories, and resource allocation to minimize the AoI. However, the existing DRL-based approaches above did not consider interference among UAVs, which can lead to significant differences in the channel environment. To investigate this effect, Seid \textit{et al.}____ considered the inter-cell interference among UAVs and proposed an approach based on multi-agent deep reinforcement learning (MADRL) to ensure the QoS requirements of IoT devices or users while reducing the total computing cost of their considered network. However, this work assumes that UAV clusters provide MEC services to users at fixed positions and does not optimize UAV trajectories. 

\par To sum up, while these works can deal with resource allocation and UAV trajectory optimization in MEC or DC systems, they are not suitable for the considered scenario. The main challenges is the hybrid solution space introduced by the discrete multi-subsystem user association variables and continuous variables and the coupling of decision variables caused by co-channel interference, making it difficult to jointly optimize these variables, especially when optimizing the UAV trajectory. Therefore, this prompts us to propose an effective online optimization approach with low computational complexity to address the considered joint optimization problem.

\begin{table*}[]
\caption{List of main notations}
\label{tab:notation}
\begin{tabularx}{\textwidth}{ll}
\toprule
\textbf{Notation}                      & \textbf{Description}                                                                                                                                               \\ \hline
\multicolumn{2}{c}{System model}                                                                                                                                                                          \\ \hline
$g, N_g, \mathcal{G}$                      & The index, number, and set of GUs                                                                                                                                  \\
$m, M, \mathcal{G}^{MEC}$                        & The index, number, and set of MEC-GUs                                                                                                                              \\
$n, N, \mathcal{G}^{DC}$                       & The index, number, and set of DC-GUs                                                                                                                              \\
$f_m(t), f_g(t), F, \mathcal{F}$                        & The task generated by MEC-GU $m$, GU $g$, the number, and set of tasks                                                                                                                                \\
$u, N_U+1, \mathcal{U}$                      & The index, number, and set of UAVs                                                                                                                                 \\
$i, N_U, \mathcal{U}^{MEC}$                      & The index, number, and set of MEC-UAVs                                                                                                                             \\
$u_{dc}$                                        & The index of the DC-UAV                                                                                                                                                         \\
$b_{m,f}, l_{m,f}, t_{m,f}^{max}$                 & The task completion status, number of data bits, and maximum tolerance time limit of the MEC task $f_m(t)$ \\
$D_{m,f}$                                  & The length of deadline for task $f_m(t)$  \\
$\tau, t, T, \mathcal{T}$                  & The length, index, number, and set of time step                                                                                                                     \\
$m(t), \alpha(t)$                          & The distance of movement and the angle of deviation                                                                                                                \\
$V_u(t), V_g(t)$                           & The coordinates of the UAV $u$ and the GU $g$                                                                                                                          \\
$P_{u,g}^{LoS}(t), P_{u,g}^{NLoS}(t)$ & The connection probability of LoS and NLoS  \\
$h_{u,g}(t)$                               & The channel gain between the UAV $u$ and the GU $g$ at time step $t$                                                                                                     \\
$p_g(t), p_u^c, p_g^{max}$                 & The transmit power of the GU $g$ at time step $t$, the computation power of UAV $u$, the maximum power of GU $g$                                                           \\
$W$                                        & Total bandwidth of each UAV                                                                                                                                        \\
$C_i, \omega_i, \kappa_i$                            & The computation intensity, the CPU operating frequency, 
               and the effective switching capacitance on MEC-UAV $i$           \\
$\delta_g$                                 & Task density coefficient       \\
$X_{u,g}(t)$                      & The association indicator variable to represent whether GU $g$ is associated with UAV $u$ at time step $t$                                                        \\
$R_{u,g}(t) $                              & The data transmission rate of GU $g$ associated with UAV $u$                                                                                                           \\
$T_{i,m}^f(t), T_{i,m,f}^c(t), T_i(t)$     & The transmission and computation delay of the task $f_m(t)$ of MEC-user $m$ with MEC-UAV $i$, the total delay of\\
& MEC-UAV $i$ at time step $t$                                                   \\ 
$\mathcal{C}^{MEC}, \mathcal{C}^{DC}$      & The MEC task completion rate and the DC rate\\\hline
\multicolumn{2}{c}{Problem formulation, algorithm, and simulation}                                                                                                                                         \\  \hline
$N_u^{max}$                                & The maximum number of GUs that UAV $u$ can serve                                                                                                                     \\
$D(t), D_{min}, D_{m,f}(t), L_n^{max}$                 & The amount of collected data at time step $t$, minimum amount of data to start collecting, remaining processing time of\\
& the earliest unfinished task $f_m(t)$ of MEC-GU $m$ at time step $t$, data storage limit of the GU $n$                                            \\
$R_{M_{th}}, R_{D_{th}}$                   & Threshold rates for MEC and DC                                                                                                                                     \\
$\sigma, \rho, \delta_p$  & The coefficient in the DC reward function, and the penalty reward for UAVs out of bounds and exceeding \\
                                           & power consumption limits \\
$\varrho, \varsigma$               & The penalty reward variable of collision and the soft update constant                                               \\
$r_l(t), r_d, r_p$                         & The latency reward, DC reward, and penalty reward                                                                                                                  \\
$N_i^f, N_m^f$                             & The number of total tasks completed by the MEC-UAV $i$, the number of total tasks generated by the MEC-user $m$                                                            \\
$L_n(t), l_n(t)$                           & The amount of data storage for the DC-user $n$, the data volume generated by the DC-user $n$ at time step $t$                                                                               \\
\bottomrule
\end{tabularx}%
\end{table*}

%%%%%%%%%%%%%%%%%%%%%% System Model %%%%%%%%%%%%%%%%%%%%%%%%%%%