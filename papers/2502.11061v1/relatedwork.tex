\section{Related Work}
\label{sec:related-work}

\begin{figure}
    \centering
    \includegraphics[width=0.85\linewidth]{figures/trial_examples.pdf}
    \caption{Examples of eye movements over a single passage; top: first reading, bottom: repeated reading. Circles represent fixations, and lines represent saccades.}
    \label{fig:trial-example}
\end{figure}

When we read, the eye movement trajectory, or \emph{scanpath} over the text is divided into \emph{fixations}, prolonged periods of time during which the gaze location is relatively fixed, and \emph{saccades}, fast transitions between fixations \cite{rayner1998eye,schotter2025beginner}. Prior work in psycholinguistics has consistently demonstrated that this trajectory differs in repeated reading compared to first reading, with large \emph{facilitation} effects marked by shorter text reading times, fewer fixations, shorter fixation durations, longer saccades and fewer regressions (backward saccades) \citep{hyona1990repeated,raney1995freq,schnitzer2006,meiri2024deja}. Several studies have also examined the interaction between repeated reading and the effect of linguistic word properties such as word length, frequency and surprisal on reading times, mostly finding less sensitivity to word properties in repeated reading \citep{raney1995freq,foster2013repeated,zawoyski2015repeated,meiri2024deja}. In line with these studies, \citet{hyona1990repeated} further demonstrated a reduction in the sensitivity of eye movements to the introduction of new topics in rereading. All the above studies examined individual features aggregated across participants and texts, and it is currently unknown whether first and repeated reading can be effectively distinguished using predictive modeling at the level of an individual participant and text.

%Moreover, \citet{meiri2024deja}  found that the presence of intervening material between readings can have a subtle modulating effect on facilitation as observed in differences in global measures such as reading times and skip rate. Facilitation slightly attenuates in ordinary reading for comprehension when intervening material is present, but remains robust in information-seeking reading. The former case hints at the possibility of better memory retention in consecutive compared to non-consecutive reading. Notably, they found that the amount of intervening material, measured by the number of articles between readings, has no effect on these global eye movement measures.
%Building on these findings, our work defines a set of decoding tasks and designs features that capture both standard eye movement metrics and the sensitivity of reading times to linguistic properties of the text.

In machine learning and NLP, a nascent line of work focuses on decoding the properties of the reader and their interaction with the text, from eye movements in reading. These include, among others, decoding of linguistic knowledge \citep{berzak_predicting_2017,berzak_assessing_2018,skerath2023native}, reading comprehension \cite{ahn_towards_2020,reich_inferring_2022,meziere2023using,Shubi2024Finegrained}, subjective text difficulty \cite{reich_inferring_2022} and the reader's goals \citep{hollenstein2023zuco,shubi2024decoding}. The current study falls broadly within this area, but introduces and addresses a new task of decoding repeated reading. %We build on feature sets and models introduced in prior studies on other decoding tasks and on the psycholinguistic literature, and further introduce new features and models for the current task. 
Following \citet{sood2020improving}, our work leverages the output of \mbox{E-Z} Reader, a computational cognitive model for automatic generation of reading scanpath trajectories \cite{reichle1998toward,reichle2003ez,reichle2009using,veldre2023understanding}.