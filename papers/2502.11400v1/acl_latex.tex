% This must be in the first 5 lines to tell arXiv to use pdfLaTeX, which is strongly recommended.
\pdfoutput=1
% In particular, the hyperref package requires pdfLaTeX in order to break URLs across lines.

\documentclass[11pt]{article}

% Change "review" to "final" to generate the final (sometimes called camera-ready) version.
% Change to "preprint" to generate a non-anonymous version with page numbers.
\usepackage[preprint]{acl}


% Standard package includes
\usepackage{times}
\usepackage{latexsym}

% For proper rendering and hyphenation of words containing Latin characters (including in bib files)
\usepackage[T1]{fontenc}
% For Vietnamese characters
% \usepackage[T5]{fontenc}
% See https://www.latex-project.org/help/documentation/encguide.pdf for other character sets

% This assumes your files are encoded as UTF8
\usepackage[utf8]{inputenc}

% This is not strictly necessary, and may be commented out,
% but it will improve the layout of the manuscript,
% and will typically save some space.
\usepackage{microtype}

% This is also not strictly necessary, and may be commented out.
% However, it will improve the aesthetics of text in
% the typewriter font.
\usepackage{inconsolata}

\usepackage{enumitem}
\usepackage{amsmath, amsthm, amssymb}
\usepackage{amsfonts}
\usepackage{graphicx}
\usepackage{multirow}
\usepackage{multicol}
\usepackage{booktabs}
\usepackage{arydshln}
\usepackage{color}
\usepackage{makecell}
\usepackage{subcaption}
\usepackage{xcolor}[dvipsnames]
\usepackage{epigraph}

\usepackage{colortbl}
\definecolor{mygray}{gray}{.9}
\definecolor{mygreen}{rgb}{0,0.7,0}
\definecolor{myorange}{RGB}{255, 218, 185}
\definecolor{mycolor}{rgb}{0.8157, 0.251, 0.2196}

% If the title and author information does not fit in the area allocated, uncomment the following
%
%\setlength\titlebox{<dim>}
%
% and set <dim> to something 5cm or larger.

\title{Revisiting Robust RAG: Do We Still Need Complex Robust Training \\ in the Era of Powerful LLMs?}

\author{Hanxing Ding$^{1,2}$\thanks{\ \ Equal contributions.}\quad
Shuchang Tao\footnotemark[1] \quad
Liang Pang$^{1}$\thanks{\ \ Corresponding author}\quad
Zihao Wei$^{1,2}$\quad\\
\textbf{Liwei Chen}$^{3}$\quad
\textbf{Kun Xu}\quad
\textbf{Huawei Shen}$^{1,2}$
\textbf{Xueqi Cheng}$^{1,2}$\\
 $^{1}$Key Laboratory of AI Safety, Institute of Computing Technology, Chinese Academy of Sciences \\
 $^{2}$ University of Chinese Academy of Sciences \quad $^{3}$ Kuaishou Technology \\
 \texttt{\{dinghanxing18s, pangliang, weizihao22z, shenhuawei, cxq\}@ict.ac.cn} \\
 % \texttt{\{taoshuchang.tsc, jinyang.gjy, bolin.ding\}@alibaba-inc.com}
}

\begin{document}
\maketitle
\begin{abstract}
Retrieval-augmented generation (RAG) systems often suffer from performance degradation when encountering noisy or irrelevant documents, driving researchers to develop sophisticated training strategies to enhance their robustness against such retrieval noise. However, as large language models (LLMs) continue to advance, the necessity of these complex training methods is increasingly questioned. In this paper, we systematically investigate whether complex robust training strategies remain necessary as model capacity grows. Through comprehensive experiments spanning multiple model architectures and parameter scales, we evaluate various document selection methods and adversarial training techniques across diverse datasets. Our extensive experiments consistently demonstrate that as models become more powerful, the performance gains brought by complex robust training methods drop off dramatically.  We delve into the rationale and find that more powerful models inherently exhibit superior confidence calibration, better generalization across datasets (even when trained with randomly selected documents), and optimal attention mechanisms learned with simpler strategies. Our findings suggest that RAG systems can benefit from simpler architectures and training strategies as models become more powerful, enabling more scalable applications with minimal complexity.
\end{abstract}

\setlength{\epigraphrule}{0pt}
\epigraph{\emph{"Entities should not be multiplied unnecessarily."}}{--- Occam's razor}

\section{Introduction}

Video generation has garnered significant attention owing to its transformative potential across a wide range of applications, such media content creation~\citep{polyak2024movie}, advertising~\citep{zhang2024virbo,bacher2021advert}, video games~\citep{yang2024playable,valevski2024diffusion, oasis2024}, and world model simulators~\citep{ha2018world, videoworldsimulators2024, agarwal2025cosmos}. Benefiting from advanced generative algorithms~\citep{goodfellow2014generative, ho2020denoising, liu2023flow, lipman2023flow}, scalable model architectures~\citep{vaswani2017attention, peebles2023scalable}, vast amounts of internet-sourced data~\citep{chen2024panda, nan2024openvid, ju2024miradata}, and ongoing expansion of computing capabilities~\citep{nvidia2022h100, nvidia2023dgxgh200, nvidia2024h200nvl}, remarkable advancements have been achieved in the field of video generation~\citep{ho2022video, ho2022imagen, singer2023makeavideo, blattmann2023align, videoworldsimulators2024, kuaishou2024klingai, yang2024cogvideox, jin2024pyramidal, polyak2024movie, kong2024hunyuanvideo, ji2024prompt}.


In this work, we present \textbf{\ours}, a family of rectified flow~\citep{lipman2023flow, liu2023flow} transformer models designed for joint image and video generation, establishing a pathway toward industry-grade performance. This report centers on four key components: data curation, model architecture design, flow formulation, and training infrastructure optimization—each rigorously refined to meet the demands of high-quality, large-scale video generation.


\begin{figure}[ht]
    \centering
    \begin{subfigure}[b]{0.82\linewidth}
        \centering
        \includegraphics[width=\linewidth]{figures/t2i_1024.pdf}
        \caption{Text-to-Image Samples}\label{fig:main-demo-t2i}
    \end{subfigure}
    \vfill
    \begin{subfigure}[b]{0.82\linewidth}
        \centering
        \includegraphics[width=\linewidth]{figures/t2v_samples.pdf}
        \caption{Text-to-Video Samples}\label{fig:main-demo-t2v}
    \end{subfigure}
\caption{\textbf{Generated samples from \ours.} Key components are highlighted in \textcolor{red}{\textbf{RED}}.}\label{fig:main-demo}
\end{figure}


First, we present a comprehensive data processing pipeline designed to construct large-scale, high-quality image and video-text datasets. The pipeline integrates multiple advanced techniques, including video and image filtering based on aesthetic scores, OCR-driven content analysis, and subjective evaluations, to ensure exceptional visual and contextual quality. Furthermore, we employ multimodal large language models~(MLLMs)~\citep{yuan2025tarsier2} to generate dense and contextually aligned captions, which are subsequently refined using an additional large language model~(LLM)~\citep{yang2024qwen2} to enhance their accuracy, fluency, and descriptive richness. As a result, we have curated a robust training dataset comprising approximately 36M video-text pairs and 160M image-text pairs, which are proven sufficient for training industry-level generative models.

Secondly, we take a pioneering step by applying rectified flow formulation~\citep{lipman2023flow} for joint image and video generation, implemented through the \ours model family, which comprises Transformer architectures with 2B and 8B parameters. At its core, the \ours framework employs a 3D joint image-video variational autoencoder (VAE) to compress image and video inputs into a shared latent space, facilitating unified representation. This shared latent space is coupled with a full-attention~\citep{vaswani2017attention} mechanism, enabling seamless joint training of image and video. This architecture delivers high-quality, coherent outputs across both images and videos, establishing a unified framework for visual generation tasks.


Furthermore, to support the training of \ours at scale, we have developed a robust infrastructure tailored for large-scale model training. Our approach incorporates advanced parallelism strategies~\citep{jacobs2023deepspeed, pytorch_fsdp} to manage memory efficiently during long-context training. Additionally, we employ ByteCheckpoint~\citep{wan2024bytecheckpoint} for high-performance checkpointing and integrate fault-tolerant mechanisms from MegaScale~\citep{jiang2024megascale} to ensure stability and scalability across large GPU clusters. These optimizations enable \ours to handle the computational and data challenges of generative modeling with exceptional efficiency and reliability.


We evaluate \ours on both text-to-image and text-to-video benchmarks to highlight its competitive advantages. For text-to-image generation, \ours-T2I demonstrates strong performance across multiple benchmarks, including T2I-CompBench~\citep{huang2023t2i-compbench}, GenEval~\citep{ghosh2024geneval}, and DPG-Bench~\citep{hu2024ella_dbgbench}, excelling in both visual quality and text-image alignment. In text-to-video benchmarks, \ours-T2V achieves state-of-the-art performance on the UCF-101~\citep{ucf101} zero-shot generation task. Additionally, \ours-T2V attains an impressive score of \textbf{84.85} on VBench~\citep{huang2024vbench}, securing the top position on the leaderboard (as of 2025-01-25) and surpassing several leading commercial text-to-video models. Qualitative results, illustrated in \Cref{fig:main-demo}, further demonstrate the superior quality of the generated media samples. These findings underscore \ours's effectiveness in multi-modal generation and its potential as a high-performing solution for both research and commercial applications.
% \section{Related Work}

\subsection{Large 3D Reconstruction Models}
Recently, generalized feed-forward models for 3D reconstruction from sparse input views have garnered considerable attention due to their applicability in heavily under-constrained scenarios. The Large Reconstruction Model (LRM)~\cite{hong2023lrm} uses a transformer-based encoder-decoder pipeline to infer a NeRF reconstruction from just a single image. Newer iterations have shifted the focus towards generating 3D Gaussian representations from four input images~\cite{tang2025lgm, xu2024grm, zhang2025gslrm, charatan2024pixelsplat, chen2025mvsplat, liu2025mvsgaussian}, showing remarkable novel view synthesis results. The paradigm of transformer-based sparse 3D reconstruction has also successfully been applied to lifting monocular videos to 4D~\cite{ren2024l4gm}. \\
Yet, none of the existing works in the domain have studied the use-case of inferring \textit{animatable} 3D representations from sparse input images, which is the focus of our work. To this end, we build on top of the Large Gaussian Reconstruction Model (GRM)~\cite{xu2024grm}.

\subsection{3D-aware Portrait Animation}
A different line of work focuses on animating portraits in a 3D-aware manner.
MegaPortraits~\cite{drobyshev2022megaportraits} builds a 3D Volume given a source and driving image, and renders the animated source actor via orthographic projection with subsequent 2D neural rendering.
3D morphable models (3DMMs)~\cite{blanz19993dmm} are extensively used to obtain more interpretable control over the portrait animation. For example, StyleRig~\cite{tewari2020stylerig} demonstrates how a 3DMM can be used to control the data generated from a pre-trained StyleGAN~\cite{karras2019stylegan} network. ROME~\cite{khakhulin2022rome} predicts vertex offsets and texture of a FLAME~\cite{li2017flame} mesh from the input image.
A TriPlane representation is inferred and animated via FLAME~\cite{li2017flame} in multiple methods like Portrait4D~\cite{deng2024portrait4d}, Portrait4D-v2~\cite{deng2024portrait4dv2}, and GPAvatar~\cite{chu2024gpavatar}.
Others, such as VOODOO 3D~\cite{tran2024voodoo3d} and VOODOO XP~\cite{tran2024voodooxp}, learn their own expression encoder to drive the source person in a more detailed manner. \\
All of the aforementioned methods require nothing more than a single image of a person to animate it. This allows them to train on large monocular video datasets to infer a very generic motion prior that even translates to paintings or cartoon characters. However, due to their task formulation, these methods mostly focus on image synthesis from a frontal camera, often trading 3D consistency for better image quality by using 2D screen-space neural renderers. In contrast, our work aims to produce a truthful and complete 3D avatar representation from the input images that can be viewed from any angle.  

\subsection{Photo-realistic 3D Face Models}
The increasing availability of large-scale multi-view face datasets~\cite{kirschstein2023nersemble, ava256, pan2024renderme360, yang2020facescape} has enabled building photo-realistic 3D face models that learn a detailed prior over both geometry and appearance of human faces. HeadNeRF~\cite{hong2022headnerf} conditions a Neural Radiance Field (NeRF)~\cite{mildenhall2021nerf} on identity, expression, albedo, and illumination codes. VRMM~\cite{yang2024vrmm} builds a high-quality and relightable 3D face model using volumetric primitives~\cite{lombardi2021mvp}. One2Avatar~\cite{yu2024one2avatar} extends a 3DMM by anchoring a radiance field to its surface. More recently, GPHM~\cite{xu2025gphm} and HeadGAP~\cite{zheng2024headgap} have adopted 3D Gaussians to build a photo-realistic 3D face model. \\
Photo-realistic 3D face models learn a powerful prior over human facial appearance and geometry, which can be fitted to a single or multiple images of a person, effectively inferring a 3D head avatar. However, the fitting procedure itself is non-trivial and often requires expensive test-time optimization, impeding casual use-cases on consumer-grade devices. While this limitation may be circumvented by learning a generalized encoder that maps images into the 3D face model's latent space, another fundamental limitation remains. Even with more multi-view face datasets being published, the number of available training subjects rarely exceeds the thousands, making it hard to truly learn the full distibution of human facial appearance. Instead, our approach avoids generalizing over the identity axis by conditioning on some images of a person, and only generalizes over the expression axis for which plenty of data is available. 

A similar motivation has inspired recent work on codec avatars where a generalized network infers an animatable 3D representation given a registered mesh of a person~\cite{cao2022authentic, li2024uravatar}.
The resulting avatars exhibit excellent quality at the cost of several minutes of video capture per subject and expensive test-time optimization.
For example, URAvatar~\cite{li2024uravatar} finetunes their network on the given video recording for 3 hours on 8 A100 GPUs, making inference on consumer-grade devices impossible. In contrast, our approach directly regresses the final 3D head avatar from just four input images without the need for expensive test-time fine-tuning.


\section{Background} \label{section:LLM}

% \subsection{Large Language Model (LLM)}   

Figure~\ref{fig:LLaMA_model}(a) shows that a decoder-only LLM initially processes a user prompt in the “prefill” stage and subsequently generates tokens sequentially during the “decoding” stage.
Both stages contain an input embedding layer, multiple decoder transformer blocks, an output embedding layer, and a sampling layer.
Figure~\ref{fig:LLaMA_model}(b) demonstrates that the decoder transformer blocks consist of a self attention and a feed-forward network (FFN) layer, each paired with residual connection and normalization layers. 

% Differentiate between encoder/decoder, explain why operation intensity is low, explain the different parts of a transformer block. Discuss Table II here. 

% Explain the architecture with Llama2-70B.

% \begin{table}[thb]
% \renewcommand\arraystretch{1.05}
% \centering
% % \vspace{-5mm}
%     \caption{ML Model Parameter Size and Operational Intensity}
%     \vspace{-2mm}
%     \small
%     \label{tab:ML Model Parameter Size and Operational Intensity}    
%     \scalebox{0.95}{
%         \begin{tabular}{|c|c|c|c|c|}
%             \hline
%             & Llama2 & BLOOM & BERT & ResNet \\
%             Model & (70B) & (176B) & & 152 \\
%             \hline
%             Parameter Size (GB) & 140 & 352 & 0.17 & 0.16 \\
%             \hline
%             Op Intensity (Ops/Byte) & 1 & 1 & 282 & 346 \\
%             \hline
%           \end{tabular}
%     }
% \vspace{-3mm}
% \end{table}

% {\fontsize{8pt}{11pt}\selectfont 8pt font size test Memory Requirement}

\begin{figure}[t]
    \centering
    \includegraphics[width=8cm]{Figure/LLaMA_model_new_new.pdf}
    \caption{(a) Prefill stage encodes prompt tokens in parallel. Decoding stage generates output tokens sequentially.
    (b) LLM contains N$\times$ decoder transformer blocks. 
    (c) Llama2 model architecture.}
    \label{fig:LLaMA_model}
\end{figure}

Figure~\ref{fig:LLaMA_model}(c) demonstrates the Llama2~\cite{touvron2023llama} model architecture as a representative LLM.
% The self attention layer requires three GEMVs\footnote{GEMVs in multi-head attention~\cite{attention}, narrow GEMMs in grouped-query attention~\cite{gqa}.} to generate query, key and value vectors.
In the self-attention layer, query, key and value vectors are generated by multiplying input vector to corresponding weight matrices.
These matrices are segmented into multiple heads, representing different semantic dimensions.
The query and key vectors go though Rotary Positional Embedding (RoPE) to encode the relative positional information~\cite{rope-paper}.
Within each head, the generated key and value vectors are appended to their caches.
The query vector is multiplied by the key cache to produce a score vector.
After the Softmax operation, the score vector is multiplied by the value cache to yield the output vector.
The output vectors from all heads are concatenated and multiplied by output weight matrix, resulting in a vector that undergoes residual connection and Root Mean Square layer Normalization (RMSNorm)~\cite{rmsnorm-paper}.
The residual connection adds up the input and output vectors of a layer to avoid vanishing gradient~\cite{he2016deep}.
The FFN layer begins with two parallel fully connections, followed by a Sigmoid Linear Unit (SiLU), and ends with another fully connection.
\textbf{Datasets.} 
We consider two datasets that have been widely used to study text classification: 
%
\begin{inlineenum}
\item the Stanford Sentiment Treebank (\textbf{SST-2})~\citep{socher-etal-2013-recursive}, which consists of excerpts from written movie reviews with a binary sentiment label
\item the \textbf{AG News} dataset~\citep{Zhang2015CharacterlevelCN}, which consists of news articles labelled by category (World, Sport, Business, Sci/Tech).
\end{inlineenum}
%
In all experiments, we remove examples with less than \num{5} words, bringing the total number of examples to \num{43296} for SST-2 and \num{120000} for AG News.

\textbf{Synthetic data generation.} 
We fine-tune the pre-trained Mistral-7B model~\citep{jiang2023mistral} using low-rank adaptation (LoRa)~\citep{hulora}. We use a custom prompt template $\prompt{\cdot}$ for each dataset (see Appendix~\ref{app:prompts}). More details on the implementation and parameters are provided in Appendix~\ref{app:implementation_details}. We sample synthetic data from the fine-tuned model $\theta$ conditioned on prompts $\prompt{\synthetic{\ell}_i}$, following the same distribution of labels in the synthetic dataset $\synthetic{D}$ as in the original dataset $D$, \ie $\ell_i = \synthetic{\ell}_i$ for $i=1,...,\synthetic{N}$. To generate synthetic sequences, we sequentially sample completions using a softmax temperature of \num{1.0} and top-$p$ (aka nucleus) sampling with $p = 0.95$, \ie we sample from a vocabulary restricted to the smallest possible set of tokens whose total probability exceeds \num{0.95}. We further ensure that the synthetic data bears high utility, and is thus realistic. For this, we consider the downstream classification tasks for which the original datasets have been designed. We fine-tune RoBERTa-base~\citep{DBLP:journals/corr/abs-1907-11692} on  $D$ and $\synthetic{D}$ and compare the performance of the resulting classifiers on held-out evaluation datasets. Details are provided in Appendix~\ref{app:utility}, for synthetic data generated with and without canaries.

\textbf{Canary injection.} 
We generate canaries $\canary{x} = (\canary{s}, \canary{\ell})$ as described in Sec.~\ref{sec:method_canaries}. Unless stated otherwise, we consider $50$-word canaries. Synthetic canaries are generated using Mistral-7B~\citep{jiang2023mistral} as $\theta_0$. We consider two ways of constructing a canary label:
%
\begin{inlineenum}
\item randomly sampling label $\canary{\ell}$ from the distribution of labels in $D$, ensuring that the class distribution among canaries matches that of $D$ (\emph{Natural}) 
\item extending the set of labels with a new artificial label ($\canary{\ell}=$"canary") only used for canaries (\emph{Artificial}). 
\end{inlineenum}

\textbf{Membership inference.} 
We compute the membership scores $\beta_{\theta}(\canary{x})$ as described in Sec.~\ref{sec:membership_method}. 
For one target model $\theta$, we consider \num{1000} canaries \canary{x}, of which on average half are included in the training dataset $n_\textrm{rep}$ times (members), while the remaining half are excluded (non-members).
%
We then use the computed RMIA scores and the ground truth for membership to construct ROC curves, from which we compute AUC and true positive rate (TPR) at low false positive rate (FPR) as measures of MIA performance.
%
Across experiments, we use $M = 4$ reference models $\theta'$, each trained on a dataset $D_{\theta'}$ consisting of the dataset $D$ used to train the target model $\theta$ with canaries inserted. Note that although practical attacks rarely have this amount of information, this is allowed by the threat model of Algorithm~\ref{alg:mia} and valid as a worst-case auditing methodology. 
%
We ensure that each canary is a member in half (\ie 2) of the reference models and a non-member in the other half. For the attacks based on synthetic data, we use $n=2$ for computing scores using an $n$-gram model and $k=25$ for computing scores based on similarity. We use Sentence-BERT~\citep{reimers-2019-sentence-bert} (\texttt{paraphrase-MiniLM-L6-v2} from \texttt{sentence-transformers}) as the embedding model. 

\section{Does Sophisticated Robust Training Still Matter in Powerful Models?}
To investigate whether sophisticated document selection strategies and adversarial loss designs are still essential for robust RAG performance as LLMs continue to evolve, we conduct comprehensive experiments across multiple LMs and datasets.

% 需要解释为何golden的表现不好,为何top-1这么好

\subsection{Do Sophisticated Document Selection Strategies Matter?}
% We first conduct extensive experiments to examine whether complex document selection strategies contribute to the robustness of LLMs.
We conduct experiments to analyze the effectiveness of complex document selection strategies under Llama model families (\texttt{Llama-2-7b-chat-hf} and \texttt{Llama-3-8B-Instruct}) in Table~\ref{tab:main_result_llama}, and Qwen model families in Appendix Table~\ref{tab:main_result_qwen}.
\paragraph{Training with sophisticated documents enhances LM robustness for weak models}
The experimental results presented in Tables \ref{tab:main_result_llama} and \ref{tab:main_result_qwen} provide compelling evidence that robust training significantly improves model resilience when processing noisy documents. While base models exhibit substantial performance degradation when encountering noisy documents during inference, models that undergo robust training maintain consistent and superior QA performance across various document selection strategies. 
A notable example is the \texttt{Llama-2-7b-chat-hf} model's performance (Table~\ref{tab:main_result_llama}) on the HotpotQA dataset, 
% where training with randomly selected documents yields an improved EM score of 30.67, demonstrating enhanced resilience to document noise.
where training with golden documents improves the EM score from 3.3 (Base Model) to 30.67 (Golden Doc), indicating increased resilience to document noise.
This pattern of improvement is consistently observed across both Llama (Table~\ref{tab:main_result_llama}) and Qwen (Table~\ref{tab:main_result_qwen}) model families, strongly indicating that robust training effectively mitigates the base models' inherent vulnerability to noisy documents.

\paragraph{Training with random documents shows surprising effectiveness}
We also notice that training with randomly selected documents exhibits remarkable effectiveness across all experimental configurations. Quantitative analysis shows that with \texttt{Llama-2-7b-chat-hf}, this approach achieves superior performance on WebQuestions compared to more sophisticated strategies. Similar observations emerge from experiments with \texttt{Qwen1.5-7B-Chat}, where random document selection achieves 46.04 EM on WebQuestions, approaching the optimal performance of 47.12 EM achieved by RetRobust. The consistency of these results across distinct model architectures suggests that the efficacy of random document selection represents an inherent characteristic of contemporary RAG systems.

% \begin{figure*}[ht]
%     \centering
%     \begin{subfigure}[b]{0.48\textwidth}
%         \centering
%         \includegraphics[width=\textwidth]{fig/f1_hotpot.pdf}
%     \end{subfigure}
%     \hfill
%     \begin{subfigure}[b]{0.48\textwidth}
%         \centering
%         \includegraphics[width=\textwidth]{fig/f1_nq.pdf}
%     \end{subfigure}
%     \caption{Comparison of F1 Scores for training with random doc and golden doc across models with varying parameter sizes from 0.5B to 70B. The left figure shows results on the HotpotQA dataset, while the right figure presents results on the NQ dataset. We provide detailed results in Appendix Table~\ref{tab:vary_parameter_analysis}.}
%     \label{fig:parameter_scale}
%     \vspace{-0.3cm}
% \end{figure*}

\paragraph{Diminishing returns of sophisticated document selections as models evolve}  
Experimental results indicate that the performance gains from sophisticated document selection strategies diminish as models evolve. For instance, comparing \texttt{Llama-2-7b-chat-hf} with \texttt{Llama-3-8B-Instruct}, the improvement in performance due to advanced document selection strategies decreases significantly, with the $\Delta$ (Worst $\rightarrow$ Best) metric for NQ dropping from 21.87\% to 7.90\% EM. A similar trend is observed when comparing \texttt{Qwen1.5-7B-Chat} to \texttt{Qwen2.5-7B-Instruct}, where the performance improvement from sophisticated document selection also shows a noticeable reduction. These results suggest that as models become more advanced, their ability to process and utilize information improves independently of complex document selection strategies, leading to diminished returns from such methods.

% \paragraph{Diminishing Returns of Sophisticated Document Selection Strategies}
% Quantitative analysis demonstrates minimal performance improvements from sophisticated document selection strategies across all experimental conditions. This observation is substantiated by the $\Delta$ (Random $\rightarrow$ Best) metric, which exhibits consistently low values across datasets and model variants. Specifically, \texttt{Llama-2-7b-chat-hf} shows a maximum improvement of only 9.28\% EM on TriviaQA, while \texttt{Qwen1.5-7B-Chat} demonstrates an even smaller margin of 2.95\% EM on HotpotQA. These limited improvements raise fundamental questions about the utility of complex document selection mechanisms.

% These experimental results reveal a particularly significant trend in the diminishing impact of sophisticated selection strategies as models evolve. Comparative analysis between \texttt{Llama-2-7b-chat-hf} and \texttt{Llama-3-8B-Instruct} demonstrates a systematic decrease in $\Delta$ values across all datasets, with TriviaQA's performance differential reducing substantially from 9.28\% to 1.49\% EM. Similar observations in the progression from \texttt{Qwen1.5-7B-Chat} to \texttt{Qwen2.5-7B-Instruct} indicate that advanced language models develop enhanced capabilities for information processing, independent of document selection methodology.

% While retrieval-based selection and RetRobust occasionally achieve optimal performance metrics, their marginal improvements over random selection do not justify the additional computational overhead. This observation is further validated by the system's robust performance even under intentionally irrelevant document selection, suggesting that sophisticated document selection strategies may not be as critical as previously assumed for RAG system performance.

% 分析使用更好的loss设计是否会带来提升
\subsection{Do Adversarial Loss Functions Matter?}
% \begin{table*}[t]
\small
\centering
\resizebox{\linewidth}{!}{
\begin{tabular}{llcccccccccc}
\toprule
\multirow{2}{*}{\textbf{Model}} & \multirow{2}{*}{\textbf{RAG Scenario}} & \multicolumn{2}{c}{\textbf{HotpotQA}} & \multicolumn{2}{c}{\textbf{NQ}} & \multicolumn{2}{c}{\textbf{WebQuestion}} & \multicolumn{2}{c}{\textbf{TriviaQA}} & \multicolumn{2}{c}{\textbf{AVERAGE}}\\
\cline{3-4}\cline{5-6}\cline{7-8}\cline{9-10}\cline{11-12}
 & &  \textbf{EM} & \textbf{F1} & \textbf{EM} & \textbf{F1} & \textbf{EM} & \textbf{F1} & \textbf{EM} & \textbf{F1} & \textbf{EM} & \textbf{F1} \\
\midrule
\multirow{6}{*}{\textit{Llama-2-7b-chat-hf}} & Base Model &  3.30  &  12.34 &  1.21 &  10.61  &  0.00  &  13.08  &  4.32  &  20.27 & 2.21 & 14.08  \\
 & Top-1 Doc & 31.76 & 43.95 & 40.20 & 51.89  &  41.73  &  52.76  &  52.93  & 65.41 & 41.66 & 53.50  \\
 & Golden Doc & 30.67 & 42.78 & 36.50 & 47.77  &  39.93  &  52.11  &  50.25  & 63.28 & 39.34 & 51.49  \\
 & Random Doc & 30.94 & 43.11 & 38.16 & 49.78  &  42.45  &  53.97  &  52.72 &  65.52 & 41.07 & 53.10  \\
 \cdashline{2-12}
 & RAAT & 31.32 & 43.24 & 42.91 & 53.19  &  36.69  &  48.82 &  51.65 &  58.71 & 40.64 & 50.99  \\
 & IRM & 34.38 & 47.11 & 40.96 & 53.07  &  53.96  &  61.62 &  57.58 &  69.08 & 46.72 & 57.72  \\
\midrule
\multirow{6}{*}{\textit{Llama-3-8B-Instruct}} & Base Model & 23.31 & 32.60 & 30.04 & 41.59 & 26.98 & 43.25 & 58.80 & 66.45 & 34.78 & 45.97 \\
 & Top-1 Doc & 36.72 & 49.30 & 44.38 & 56.20  &  54.68  &  62.26 &  60.80  & 68.31 & 49.15 & 59.02  \\
 & Golden Doc & 35.52 & 48.31 & 41.35 & 53.13  &  48.92  &  58.41  &  58.26 & 66.99 &  46.01 & 56.71 \\
 & Random Doc & 35.98 & 49.05 & 43.37 & 55.43  &  53.24  &   62.55 &  60.62  & 68.64 & 48.30 & 58.92  \\
 \cdashline{2-12}
 & RAAT & 32.20 & 43.81 & 42.34 & 53.31  &  48.28  &  58.17 &  54.41 &  62.45 & 44.31 & 54.44  \\
 & IRM & 35.19 & 48.08 & 41.13 & 53.14  &  53.96  &  61.64  &  57.15 &  69.13 & 46.86 & 58.00  \\
\bottomrule
\end{tabular}}
\caption{Performance comparison of different document selection and loss design strategies (RAAT~\cite{fang-etal-2024-enhancing} and IRM~\cite{DBLP:journals/tmlr/YoshidaN24}) for various models (\texttt{Llama-2-7b-chat-hf} and \texttt{Llama-3-8B-Instruct}) across four datasets.}
\label{tab:appendix_loss_analysis}
\end{table*}

To investigate whether the design of complex adversarial loss functions contributes to model performance, in Table~\ref{tab:main_result_llama} and~\ref{tab:main_result_qwen}, we also analyze the robustness of various adversarial loss designs.

\paragraph{Adversarial loss significantly enhances performance for weaker models}  
For the weaker model (\texttt{Llama-2-7b-chat-hf}), incorporating adversarial loss functions such as RAAT and IRM leads to a substantial improvement in performance compared to the base model or alternative document selection strategies. Specifically, while the base model achieves an average EM / F1 of only 2.21 / 14.08, applying adversarial loss functions boosts the performance to 40.64 / 50.99 for RAAT and 46.72 / 57.72 for IRM. This highlights the effectiveness of adversarial loss in improving model robustness to noisy documents, significantly enhancing both robustness and downstream inference performance. Notably, in some cases, RAAT and IRM outperform traditional document selection strategies (e.g., top-1 Doc and golden doc), demonstrating their value in scenarios where the model needs stronger guidance to handle noisy retrievals.

\begin{figure}[t]
    \centering
    \includegraphics[width=\columnwidth]{fig/f1_hotpot.pdf}
    \caption{Comparison of F1 Scores on the HotpotQA dataset for training with Random Doc and Golden Doc across models with varying parameter sizes from 0.5B to 70B. We provide detailed results in Appendix Table~\ref{tab:vary_parameter_analysis}.}
    \label{fig:parameter_scale}
    \vspace{-0.5cm}
\end{figure}

\paragraph{Adversarial loss exhibits diminishing returns for stronger models}  
For the stronger models, such as \texttt{Llama-3-8B-Instruct} and \texttt{Qwen2.5-7B-Instruct}, the benefits of adversarial loss functions are less pronounced. The basic \texttt{Llama-3-8B-Instruct} already achieves an average EM / F1 of 34.78 / 45.97, and the introduction of adversarial losses (RAAT and IRM) results in only modest improvements, with average EM / F1 scores of 44.31 / 54.44 and 46.86 / 58.00, respectively. Similarly, \texttt{Qwen2.5-7B-Instruct} shows similar trends, with marginal gains from adversarial loss functions. These improvements are comparable to or even slightly worse than the performance achieved by random document selection (48.30 / 58.92) or top-1 document strategies (49.15 / 59.02). This indicates that the models’ inherent robustness reduces the impact of adversarial losses, and in some cases, may even hinder performance. We hypothesize that when the model’s internal robustness is already well-developed, additional constraints from adversarial losses may interfere with its ability to optimize on clean and relevant inputs.

% \paragraph{Conclusion: the necessity of adversarial loss.}
Based on these findings, we conclude that both adversarial loss functions and document selection strategies are more beneficial for weaker models. For smaller models, these techniques significantly improve robustness by mitigating the impact of noisy documents. For stronger models with inherently robust performance, the advantages of complex loss designs and sophisticated document selection diminish, suggesting that simpler strategies like random document selection may be sufficient. This underscores the importance of tailoring training strategies to the model’s inherent capabilities.


% 尝试更多种的噪音文档,例如反事实的,bad的,还有ISN,且可以和loss结合起来

% 加入4.3的讨论,表明random和golden的关系的讨论,证明为什么

% gold doc(按理说是sft中最好的选择,在rag的设置下不如random,所以我们在第五章接着去分析原因)
% 4.3讨论扩大model scale,绘制散点图,引入第五章分析random>golden的异常的现象

\subsection{Do Training Strategies Matter Across Model Scales?}
% Through analyzing Tables~\ref{tab:main_result_llama} and~\ref{tab:main_result_qwen}, we observed a counter-intuitive phenomenon: models trained with random documents outperformed those trained with golden documents. While golden documents containing ground truth answers typically yield optimal results in standard supervised fine-tuning (SFT), this conventional wisdom was challenged in the robust RAG task, particularly with the 7B and 8B models. To investigate whether this phenomenon is specific to models in the 7B-8B parameter range or persists across different scales, we conduct comprehensiv|e experiments using models ranging from 0.5B to 70B parameters.
Through analyzing Tables~\ref{tab:main_result_llama} and~\ref{tab:main_result_qwen}, we observe a counter-intuitive phenomenon: models trained with random documents outperformed those with golden documents, despite the latter containing ground truth answers typically yielding optimal results in standard SFT. To investigate whether this phenomenon extends beyond 7B-8B models, we conduct experiments across model scales from 0.5B to 70B parameters. 

The results (detailed in Appendix Table~\ref{tab:vary_parameter_analysis}) in Figure~\ref{fig:parameter_scale} demonstrate that for smaller models ($\le$3B parameters), training with golden documents leads to superior performance. This suggests that smaller models, limited by their inherent capabilities, benefit more from high-quality golden documents containing direct answers. However, as model size increases, we observe that training with random documents becomes more effective. This shift can be attributed to larger models' enhanced question-answering abilities and improved robustness. These models can better generalize to downstream tasks even when trained on random documents, which may contain noisier or less structured information. This finding indicates that sophisticated document selection strategies become less crucial as model size increases, revealing an important scaling property in model training.


% This unexpected observation raises a critical question: Is this phenomenon specific to models in the 7B-8B parameter range, or does it persist across different model scales? To investigate this, we need to examine whether similar patterns emerge in both smaller models (0.5B or 1.5B) and larger models (32B-70B).

% In this section, we conduct comprehensive experiments across a wide spectrum of model sizes, ranging from 0.5B to 70B parameters, to systematically investigate this counter-intuitive phenomenon.

% 同一个模型的golden和random用同一个颜色
% In order to investigate whether training with random documents can achieve better performance compared to golden documents on larger models, we conduct experiments using models with parameter sizes ranging from 0.5B to 70B.  

% The results depicted in Figure~\ref{fig:parameter_scale} reveal that when the model size is relatively small (e.g., less than or equal to 3B), using golden documents leads to superior performance. This suggests that smaller models possess limited inherent capabilities and thus benefit significantly from high-quality golden documents that contain answers, which aid the model in accurately identifying the correct responses. Conversely, as the model size increases, we observe a shift in performance dynamics where training with random documents begins to yield better results. This improvement can be attributed to the substantial enhancement in the model's intrinsic question-answering abilities, coupled with increased robustness and generalization capacity. Larger models are capable of leveraging their advanced capabilities to generalize effectively to downstream tasks, even when trained on random documents, which may include noisy or less structured data. This finding underscores the diminishing importance of complex document selection strategies as model size grows, highlighting the evolving nature of model training requirements with scale.







\section{Why Sophisticated Training No Longer Matters in Powerful Models?}
In this section, we conduct comprehensive experiments to delve into the reasons why sophisticated robust training strategies may no longer be crucial in powerful models.


\begin{figure}[t]
    \centering
    \includegraphics[width=\columnwidth]{fig/confidence.pdf}
    \caption{Confidence scores for correct and wrong answers on HotpotQA dataset, comparing Llama2 and Llama3 models across various robust training methods.}
    \label{fig:confidence}
    \vspace{-0.5cm}
\end{figure}


\subsection{Powerful Models Enable Natural Calibration}
% To investigate confidence calibration of models with different capacities, we take HotpotQA dataset as an example to compare the confidence between Llama2 (\texttt{Llama-2-7b-chat-hf}) and Llama3 (\texttt{Llama-3-8B-Instruct}) models. 
% To investigate confidence calibration of models with different capacities, 
To understand whether powerful models inherently possess the ability to distinguish reliable from unreliable answers, we take HotpotQA dataset as an example to examine the confidence calibration capabilities for Llama2 (\texttt{Llama-2-7b-chat-hf}) and Llama3 (\texttt{Llama-3-8B-Instruct}) models. 
Here, confidence is the mean of token-wise probabilities in the model's generated answer, providing a measure of the model's certainty in its predictions~\cite{DBLP:journals/corr/abs-2307-03987, DBLP:conf/iclr/XiongHLLFHH24}.
Figure~\ref{fig:confidence} reveals striking differences in their calibration patterns. In the base model, Llama2 shows poor natural calibration levels, where confidence scores for incorrect answers (95.8) abnormally exceed those for correct ones (93.8). In contrast, Llama3 demonstrates inherently better calibration, maintaining higher confidence for correct answers (97.5) than incorrect ones (91.3) without any specialized complex training.

While robust training methods (Golden Doc, Random Doc, and IRM) can effectively calibrate confidence scores and improve the gap between correct and incorrect answers for Llama2 from -2 to 12 with IRM, the marginal benefits of these complex training strategies diminish as model architectures advance.
For Llama3, which already achieves a 6.2 confidence gap naturally, the improvements from these training methods become less significant. This finding strongly suggests that advances in model architecture can effectively eliminate the need for complex robustness training procedures, as newer models come with better built-in calibration capabilities.

\begin{figure}[t]
    \centering
    \begin{subfigure}[b]{0.45\textwidth}
        \centering
        \includegraphics[width=\textwidth]{fig/llama2_cross_dataset.pdf}
    \end{subfigure}
    \begin{subfigure}[b]{0.45\textwidth}
        \centering
        \includegraphics[width=\textwidth]{fig/llama3_cross_dataset.pdf}
    \end{subfigure}
    \caption{Generalization performance comparison across different strategies trained on HotpotQA (diagonal hatches bars) and evaluated on NQ, WebQuestions, TriviaQA datasets (plain bars). More results available in Appendix Table~\ref{tab:cross_dataset_eval}.}
    \label{fig:cross_dataset_eval_part}
    \vspace{-0.6cm}
\end{figure}

\subsection{Simple Training Strategies Generalize Well in Powerful Models}
We further investigate whether powerful models can maintain robust generalization across different datasets with simple training strategies.
% To investigate why sophisticated training becomes unnecessary in modern models, we examine the generalization capabilities of different training strategies. 
We fine-tune models on HotpotQA using four document selection approaches and evaluate their transfer performance on NQ, WebQuestions, and TriviaQA.

As shown in Figure~\ref{fig:cross_dataset_eval_part}, simple strategies demonstrate surprisingly strong generalization ability. Random document selection matches or even outperforms sophisticated IRM across all evaluation datasets, with performance gaps of less than 1\%. For instance, in TriviaQA, random selection (69.5 F1) slightly surpasses both golden (68.2 F1) and IRM (68.7 F1) approaches.
This trend becomes more pronounced in the powerful \texttt{Llama-3-8B-Instruct}, where the performance gap between simple and sophisticated strategies further narrows. The consistent cross-dataset performance, regardless of training strategy, indicates that model capacity, rather than training sophistication, is the key driver of generalization ability. These findings provide strong evidence that as models become more powerful, sophisticated training strategies become increasingly unnecessary.

\begin{figure}[ht]
    \centering
    \includegraphics[width=\columnwidth]{fig/case1.pdf}
    % \caption{Attention distribution heatmaps for models. Each cell ($i$, $j$) represents the average attention assigned to tokens in document $i$ by the $j$-th attention layer when generating answers. \textcolor{mygreen}{\textbf{Doc1}} (highlighted in green) contains the correct answer.}
    \caption{Attention visualization for a QA case. Each subplot shows attention distribution heatmaps across different models, where cell ($i$, $j$) represents the average attention weight from the $j$-th attention layer to document $i$. Text highlighted in \textcolor{mygreen}{green} indicates the correct answer and corresponding \textcolor{mygreen}{\textbf{Doc1}}, {blue} indicates key terms from the query, and {red} indicates incorrect model predictions. The color intensity in the heatmaps indicates attention strength.}
    \label{fig:attention}
    \vspace{-0.3cm}
\end{figure}

% \subsection{Attention Distribution Analysis}
\subsection{Powerful Models Learn Effective Attention Patterns with Simple Training}
To provide a direct understanding of why simple training can achieve good performance, we visualize attention distributions across different training strategies. Figure~\ref{fig:attention} reveals that both sophisticated robust training methods (IRM) and simple approaches (random doc, top-1) achieve similar attention patterns, with clear focus on Doc1 (containing the correct answer) in middle layers (9-16). In contrast, the base model fails to attend to the correct document, generating a wrong answer. This finding provides direct evidence that powerful models can learn optimal attention mechanisms even with simple training strategies, making sophisticated training methods unnecessary.

\begin{figure}[t]
    \centering
    \includegraphics[width=\columnwidth]{fig/mixed_doc_hotpot.pdf}
    \caption{Performance comparison with different numbers of random documents during training. Increasing the number of random documents consistently improves model performance.}
    % \caption{Impact of random document proportion on F1 Scores. The figure shows the F1 scores for HotpotQA and NQ datasets when varying the number of random documents (0 to 3) in a total of three training documents. Increasing the proportion of random documents consistently improves model performance.}
    \label{fig:mixed_doc}
    \vspace{-0.3cm}
\end{figure}


\begin{figure}[ht]
    \centering
    \includegraphics[width=\columnwidth]{fig/training_step_f1.pdf}
    \caption{Training curves comparison between random and golden document strategies using \texttt{Llama-2-7b-chat-hf} and \texttt{LLama-3-8B-Instruct}.}
    % \caption{The relationship between training steps and F1 score during the fine-tuning process with random documents and golden documents using \texttt{Llama-2-7b-chat-hf} and \texttt{LLama-3-8B-Instruct} on the NQ dataset.}
    \label{fig:training_step_f1}
    \vspace{-0.5cm}
\end{figure}

\subsection{Training with Random Docs: Better Performance and Faster Convergence}
We investigate why random training proves good performance from two aspects.
\paragraph{More random docs lead to better performance}
 % First, we vary the proportion of random documents (0 to 3) in training instances. As shown in Figure~\ref{fig:mixed_doc}, increasing random documents consistently improves F1 scores across both HotpotQA and NQ datasets. For Llama-2-7b, F1 scores increase from 44.0 to 47.0 on HotpotQA and 48 to 52 on NQ, with Llama-3-8B showing similar improvements.
We first vary the numbers of random documents (0 to 3) in training instances to examine how increasing random documents affects model performance. As shown in Figure~\ref{fig:mixed_doc}, increasing random documents consistently improves F1 scores across both datasets. For \texttt{Llama-2-7b-chat-hf}, using 3 random documents (versus zero) improves F1 scores by 3 points on HotpotQA and 4 points on NQ. \texttt{Llama-3-8B-Instruct} shows similar gains, suggesting that powerful models can effectively learn from random documents, making sophisticated document selection relatively unnecessary.

% To investigate the effect of varying the proportion of random documents on the robustness performance of models across different datasets, we examine the results of experiments conducted with three documents per training instance, where the number of random documents ranged from 0 to 3, with the remainder being golden documents.

% The empirical results in Figure~\ref{fig:mixed_doc} demonstrate a consistent enhancement in F1 scores as the proportion of random documents increases. Specifically, for both the HotpotQA and NQ datasets, models trained with a greater number of random documents exhibit superior performance. For example, the \texttt{Llama-2-7b-chat-hf} model shows a progressive increase in F1 score from 44.0 to 47.0 on the HotpotQA dataset and from 48 to 52 on the NQ dataset as the number of random documents increases from 0 to 3. Similarly, the \texttt{Llama-3-8B-Instruct} model demonstrates an improvement from 49 to 50 on the HotpotQA dataset and from 52 to 54 on the NQ dataset. More  experimental results on the TriviaQA and WebQuestions datasets are shown in Appendix Table~\ref{tab:appendix_mixed_doc}. These findings suggest that incorporating a higher proportion of random documents during training significantly enhances model robustness and generalization capabilities. The inclusion of random documents likely introduces greater variability and noise, which may enable the model to better generalize to diverse and previously unseen data during evaluation.



\paragraph{Faster convergence with random training}
The training dynamics in Figure~\ref{fig:training_step_f1} provide another evidence for why sophisticated document selection becomes unnecessary. Random document training not only achieves higher F1 scores (2-3 points improvement) but also reaches peak performance in fewer steps compared to golden document training. This faster convergence with better performance holds true for both model scales, indicating that simpler random training actually enables more efficient learning in powerful language models.
% In Figure~\ref{fig:training_step_f1}, we plot the training curve of the relationship between training steps and F1 score during the fine-tuning process with random documents and golden documents using \texttt{Llama-2-7b-chat-hf} and \texttt{LLama-3-8B-Instruct} on the NQ dataset. Our findings reveal that training with random documents leads to superior performance, as evidenced by achieving a higher optimal F1 score. Additionally, we observe that the model reaches a high F1 score more rapidly when trained with random documents. This suggests that the random document strategy more effectively harnesses the model's inherent robustness and generalization capabilities, resulting in enhanced performance compared to the golden document approach.

% \begin{figure}[t]
%     \centering
%     \begin{subfigure}[b]{0.23\textwidth}
%         \centering
%         \includegraphics[width=\textwidth]{fig/hotpot_randcur_llama3.pdf}
%     \end{subfigure}
%     \hfill
%     \begin{subfigure}[b]{0.23\textwidth}
%         \centering
%         \includegraphics[width=\textwidth]{fig/hotpot_goodtop1_llama3.pdf}
%     \end{subfigure}
%     \centering
%     \begin{subfigure}[b]{0.23\textwidth}
%         \centering
%         \includegraphics[width=\textwidth]{fig/hotpot_top1_llama3.pdf}
%     \end{subfigure}
%     \hfill
%     \begin{subfigure}[b]{0.23\textwidth}
%         \centering
%         \includegraphics[width=\textwidth]{fig/hotpot_IRM_llama3.pdf
%     \end{subfigure}
%     \caption{Attention distribution heatmaps for models trained with (a) Random and (b) Golden documents. Each cell ($i$, $j$) represents the average attention assigned to tokens in document $i$ by the $j$-th attention layer when generating answers. \textcolor{mygreen}{\textbf{Doc1}} (highlighted in green) contains the correct answer.}
%     \label{fig:attention}
%     \vspace{-0.5cm}
% \end{figure}
\section{Insights and Future Directions}

Our findings reveal a fundamental shift in RAG system: as models become more powerful, the marginal benefits of sophisticated training strategies diminish significantly. This observation has several important insights:

\paragraph{Simplified RAG architecture design}

For powerful models, simple retrieval strategies (even random selection) can achieve comparable performance to sophisticated approaches. This enables streamlined RAG architectures by replacing elaborate document filtering mechanisms with simpler retrieval methods, substantially reducing system complexity without sacrificing performance.


\paragraph{Scalable RAG for open-domain tasks.}  
Larger models demonstrate robustness against noisy retrieval, suggesting that open-domain RAG systems can function effectively with minimal retrieval supervision. Instead of enforcing strict filtering of retrieved documents, future large-scale RAG systems can leverage weakly supervised learning, incorporating large-scale web data and noisy retrieval results to improve generalization.

\paragraph{Theoretical implications for model scaling.}

This research reveals a previously unexplored aspect of scaling laws: the diminishing returns of complex training strategies as models grow larger. This challenges current theoretical frameworks and calls for new ones that better explain how training requirements evolve with model scale.

\paragraph{Broader impact on  machine learning.}

Our findings suggest that as models become more powerful, practitioners should prioritize architectural improvements and data quality over complex training strategies. This insight could lead to more efficient resource allocation in model development across various applications, from computer vision to natural language processing.

% Our work challenges conventional wisdom about the necessity of complex training strategies and suggests a more nuanced approach to model development that could significantly impact both theoretical understanding and practical applications in machine learning.
\section{Discussion}\label{sec:discussion}



\subsection{From Interactive Prompting to Interactive Multi-modal Prompting}
The rapid advancements of large pre-trained generative models including large language models and text-to-image generation models, have inspired many HCI researchers to develop interactive tools to support users in crafting appropriate prompts.
% Studies on this topic in last two years' HCI conferences are predominantly focused on helping users refine single-modality textual prompts.
Many previous studies are focused on helping users refine single-modality textual prompts.
However, for many real-world applications concerning data beyond text modality, such as multi-modal AI and embodied intelligence, information from other modalities is essential in constructing sophisticated multi-modal prompts that fully convey users' instruction.
This demand inspires some researchers to develop multimodal prompting interactions to facilitate generation tasks ranging from visual modality image generation~\cite{wang2024promptcharm, promptpaint} to textual modality story generation~\cite{chung2022tale}.
% Some previous studies contributed relevant findings on this topic. 
Specifically, for the image generation task, recent studies have contributed some relevant findings on multi-modal prompting.
For example, PromptCharm~\cite{wang2024promptcharm} discovers the importance of multimodal feedback in refining initial text-based prompting in diffusion models.
However, the multi-modal interactions in PromptCharm are mainly focused on the feedback empowered the inpainting function, instead of supporting initial multimodal sketch-prompt control. 

\begin{figure*}[t]
    \centering
    \includegraphics[width=0.9\textwidth]{src/img/novice_expert.pdf}
    \vspace{-2mm}
    \caption{The comparison between novice and expert participants in painting reveals that experts produce more accurate and fine-grained sketches, resulting in closer alignment with reference images in close-ended tasks. Conversely, in open-ended tasks, expert fine-grained strokes fail to generate precise results due to \tool's lack of control at the thin stroke level.}
    \Description{The comparison between novice and expert participants in painting reveals that experts produce more accurate and fine-grained sketches, resulting in closer alignment with reference images in close-ended tasks. Novice users create rougher sketches with less accuracy in shape. Conversely, in open-ended tasks, expert fine-grained strokes fail to generate precise results due to \tool's lack of control at the thin stroke level, while novice users' broader strokes yield results more aligned with their sketches.}
    \label{fig:novice_expert}
    % \vspace{-3mm}
\end{figure*}


% In particular, in the initial control input, users are unable to explicitly specify multi-modal generation intents.
In another example, PromptPaint~\cite{promptpaint} stresses the importance of paint-medium-like interactions and introduces Prompt stencil functions that allow users to perform fine-grained controls with localized image generation. 
However, insufficient spatial control (\eg, PromptPaint only allows for single-object prompt stencil at a time) and unstable models can still leave some users feeling the uncertainty of AI and a varying degree of ownership of the generated artwork~\cite{promptpaint}.
% As a result, the gap between intuitive multi-modal or paint-medium-like control and the current prompting interface still exists, which requires further research on multi-modal prompting interactions.
From this perspective, our work seeks to further enhance multi-object spatial-semantic prompting control by users' natural sketching.
However, there are still some challenges to be resolved, such as consistent multi-object generation in multiple rounds to increase stability and improved understanding of user sketches.   


% \new{
% From this perspective, our work is a step forward in this direction by allowing multi-object spatial-semantic prompting control by users' natural sketching, which considers the interplay between multiple sketch regions.
% % To further advance the multi-modal prompting experience, there are some aspects we identify to be important.
% % One of the important aspects is enhancing the consistency and stability of multiple rounds of generation to reduce the uncertainty and loss of control on users' part.
% % For this purpose, we need to develop techniques to incorporate consistent generation~\cite{tewel2024training} into multi-modal prompting framework.}
% % Another important aspect is improving generative models' understanding of the implicit user intents \new{implied by the paint-medium-like or sketch-based input (\eg, sketch of two people with their hands slightly overlapping indicates holding hand without needing explicit prompt).
% % This can facilitate more natural control and alleviate users' effort in tuning the textual prompt.
% % In addition, it can increase users' sense of ownership as the generated results can be more aligned with their sketching intents.
% }
% For example, when users draw sketches of two people with their hands slightly overlapping, current region-based models cannot automatically infer users' implicit intention that the two people are holding hands.
% Instead, they still require users to explicitly specify in the prompt such relationship.
% \tool addresses this through sketch-aware prompt recommendation to fill in the necessary semantic information, alleviating users' workload.
% However, some users want the generative AI in the future to be able to directly infer this natural implicit intentions from the sketches without additional prompting since prompt recommendation can still be unstable sometimes.


% \new{
% Besides visual generation, 
% }
% For example, one of the important aspect is referring~\cite{he2024multi}, linking specific text semantics with specific spatial object, which is partly what we do in our sketch-aware prompt recommendation.
% Analogously, in natural communication between humans, text or audio alone often cannot suffice in expressing the speakers' intentions, and speakers often need to refer to an existing spatial object or draw out an illustration of her ideas for better explanation.
% Philosophically, we HCI researchers are mostly concerned about the human-end experience in human-AI communications.
% However, studies on prompting is unique in that we should not just care about the human-end interaction, but also make sure that AI can really get what the human means and produce intention-aligned output.
% Such consideration can drastically impact the design of prompting interactions in human-AI collaboration applications.
% On this note, although studies on multi-modal interactions is a well-established topic in HCI community, it remains a challenging problem what kind of multi-modal information is really effective in helping humans convey their ideas to current and next generation large AI models.




\subsection{Novice Performance vs. Expert Performance}\label{sec:nVe}
In this section we discuss the performance difference between novice and expert regarding experience in painting and prompting.
First, regarding painting skills, some participants with experience (4/12) preferred to draw accurate and fine-grained shapes at the beginning. 
All novice users (5/12) draw rough and less accurate shapes, while some participants with basic painting skills (3/12) also favored sketching rough areas of objects, as exemplified in Figure~\ref{fig:novice_expert}.
The experienced participants using fine-grained strokes (4/12, none of whom were experienced in prompting) achieved higher IoU scores (0.557) in the close-ended task (0.535) when using \tool. 
This is because their sketches were closer in shape and location to the reference, making the single object decomposition result more accurate.
Also, experienced participants are better at arranging spatial location and size of objects than novice participants.
However, some experienced participants (3/12) have mentioned that the fine-grained stroke sometimes makes them frustrated.
As P1's comment for his result in open-ended task: "\emph{It seems it cannot understand thin strokes; even if the shape is accurate, it can only generate content roughly around the area, especially when there is overlapping.}" 
This suggests that while \tool\ provides rough control to produce reasonably fine results from less accurate sketches for novice users, it may disappoint experienced users seeking more precise control through finer strokes. 
As shown in the last column in Figure~\ref{fig:novice_expert}, the dragon hovering in the sky was wrongly turned into a standing large dragon by \tool.

Second, regarding prompting skills, 3 out of 12 participants had one or more years of experience in T2I prompting. These participants used more modifiers than others during both T2I and R2I tasks.
Their performance in the T2I (0.335) and R2I (0.469) tasks showed higher scores than the average T2I (0.314) and R2I (0.418), but there was no performance improvement with \tool\ between their results (0.508) and the overall average score (0.528). 
This indicates that \tool\ can assist novice users in prompting, enabling them to produce satisfactory images similar to those created by users with prompting expertise.



\subsection{Applicability of \tool}
The feedback from user study highlighted several potential applications for our system. 
Three participants (P2, P6, P8) mentioned its possible use in commercial advertising design, emphasizing the importance of controllability for such work. 
They noted that the system's flexibility allows designers to quickly experiment with different settings.
Some participants (N = 3) also mentioned its potential for digital asset creation, particularly for game asset design. 
P7, a game mod developer, found the system highly useful for mod development. 
He explained: "\emph{Mods often require a series of images with a consistent theme and specific spatial requirements. 
For example, in a sacrifice scene, how the objects are arranged is closely tied to the mod's background. It would be difficult for a developer without professional skills, but with this system, it is possible to quickly construct such images}."
A few participants expressed similar thoughts regarding its use in scene construction, such as in film production. 
An interesting suggestion came from participant P4, who proposed its application in crime scene description. 
She pointed out that witnesses are often not skilled artists, and typically describe crime scenes verbally while someone else illustrates their account. 
With this system, witnesses could more easily express what they saw themselves, potentially producing depictions closer to the real events. "\emph{Details like object locations and distances from buildings can be easily conveyed using the system}," she added.

% \subsection{Model Understanding of Users' Implicit Intents}
% In region-sketch-based control of generative models, a significant gap between interaction design and actual implementation is the model's failure in understanding users' naturally expressed intentions.
% For example, when users draw sketches of two people with their hands slightly overlapping, current region-based models cannot automatically infer users' implicit intention that the two people are holding hands.
% Instead, they still require users to explicitly specify in the prompt such relationship.
% \tool addresses this through sketch-aware prompt recommendation to fill in the necessary semantic information, alleviating users' workload.
% However, some users want the generative AI in the future to be able to directly infer this natural implicit intentions from the sketches without additional prompting since prompt recommendation can still be unstable sometimes.
% This problem reflects a more general dilemma, which ubiquitously exists in all forms of conditioned control for generative models such as canny or scribble control.
% This is because all the control models are trained on pairs of explicit control signal and target image, which is lacking further interpretation or customization of the user intentions behind the seemingly straightforward input.
% For another example, the generative models cannot understand what abstraction level the user has in mind for her personal scribbles.
% Such problems leave more challenges to be addressed by future human-AI co-creation research.
% One possible direction is fine-tuning the conditioned models on individual user's conditioned control data to provide more customized interpretation. 

% \subsection{Balance between recommendation and autonomy}
% AIGC tools are a typical example of 
\subsection{Progressive Sketching}
Currently \tool is mainly aimed at novice users who are only capable of creating very rough sketches by themselves.
However, more accomplished painters or even professional artists typically have a coarse-to-fine creative process. 
Such a process is most evident in painting styles like traditional oil painting or digital impasto painting, where artists first quickly lay down large color patches to outline the most primitive proportion and structure of visual elements.
After that, the artists will progressively add layers of finer color strokes to the canvas to gradually refine the painting to an exquisite piece of artwork.
One participant in our user study (P1) , as a professional painter, has mentioned a similar point "\emph{
I think it is useful for laying out the big picture, give some inspirations for the initial drawing stage}."
Therefore, rough sketch also plays a part in the professional artists' creation process, yet it is more challenging to integrate AI into this more complex coarse-to-fine procedure.
Particularly, artists would like to preserve some of their finer strokes in later progression, not just the shape of the initial sketch.
In addition, instead of requiring the tool to generate a finished piece of artwork, some artists may prefer a model that can generate another more accurate sketch based on the initial one, and leave the final coloring and refining to the artists themselves.
To accommodate these diverse progressive sketching requirements, a more advanced sketch-based AI-assisted creation tool should be developed that can seamlessly enable artist intervention at any stage of the sketch and maximally preserve their creative intents to the finest level. 

\subsection{Ethical Issues}
Intellectual property and unethical misuse are two potential ethical concerns of AI-assisted creative tools, particularly those targeting novice users.
In terms of intellectual property, \tool hands over to novice users more control, giving them a higher sense of ownership of the creation.
However, the question still remains: how much contribution from the user's part constitutes full authorship of the artwork?
As \tool still relies on backbone generative models which may be trained on uncopyrighted data largely responsible for turning the sketch into finished artwork, we should design some mechanisms to circumvent this risk.
For example, we can allow artists to upload backbone models trained on their own artworks to integrate with our sketch control.
Regarding unethical misuse, \tool makes fine-grained spatial control more accessible to novice users, who may maliciously generate inappropriate content such as more realistic deepfake with specific postures they want or other explicit content.
To address this issue, we plan to incorporate a more sophisticated filtering mechanism that can detect and screen unethical content with more complex spatial-semantic conditions. 
% In the future, we plan to enable artists to upload their own style model

% \subsection{From interactive prompting to interactive spatial prompting}


\subsection{Limitations and Future work}

    \textbf{User Study Design}. Our open-ended task assesses the usability of \tool's system features in general use cases. To further examine aspects such as creativity and controllability across different methods, the open-ended task could be improved by incorporating baselines to provide more insightful comparative analysis. 
    Besides, in close-ended tasks, while the fixing order of tool usage prevents prior knowledge leakage, it might introduce learning effects. In our study, we include practice sessions for the three systems before the formal task to mitigate these effects. In the future, utilizing parallel tests (\textit{e.g.} different content with the same difficulty) or adding a control group could further reduce the learning effects.

    \textbf{Failure Cases}. There are certain failure cases with \tool that can limit its usability. 
    Firstly, when there are three or more objects with similar semantics, objects may still be missing despite prompt recommendations. 
    Secondly, if an object's stroke is thin, \tool may incorrectly interpret it as a full area, as demonstrated in the expert results of the open-ended task in Figure~\ref{fig:novice_expert}. 
    Finally, sometimes inclusion relationships (\textit{e.g.} inside) between objects cannot be generated correctly, partially due to biases in the base model that lack training samples with such relationship. 

    \textbf{More support for single object adjustment}.
    Participants (N=4) suggested that additional control features should be introduced, beyond just adjusting size and location. They noted that when objects overlap, they cannot freely control which object appears on top or which should be covered, and overlapping areas are currently not allowed.
    They proposed adding features such as layer control and depth control within the single-object mask manipulation. Currently, the system assigns layers based on color order, but future versions should allow users to adjust the layer of each object freely, while considering weighted prompts for overlapping areas.

    \textbf{More customized generation ability}.
    Our current system is built around a single model $ColorfulXL-Lightning$, which limits its ability to fully support the diverse creative needs of users. Feedback from participants has indicated a strong desire for more flexibility in style and personalization, such as integrating fine-tuned models that cater to specific artistic styles or individual preferences. 
    This limitation restricts the ability to adapt to varied creative intents across different users and contexts.
    In future iterations, we plan to address this by embedding a model selection feature, allowing users to choose from a variety of pre-trained or custom fine-tuned models that better align with their stylistic preferences. 
    
    \textbf{Integrate other model functions}.
    Our current system is compatible with many existing tools, such as Promptist~\cite{hao2024optimizing} and Magic Prompt, allowing users to iteratively generate prompts for single objects. However, the integration of these functions is somewhat limited in scope, and users may benefit from a broader range of interactive options, especially for more complex generation tasks. Additionally, for multimodal large models, users can currently explore using affordable or open-source models like Qwen2-VL~\cite{qwen} and InternVL2-Llama3~\cite{llama}, which have demonstrated solid inference performance in our tests. While GPT-4o remains a leading choice, alternative models also offer competitive results.
    Moving forward, we aim to integrate more multimodal large models into the system, giving users the flexibility to choose the models that best fit their needs. 
    


\section{Conclusion}\label{sec:conclusion}
In this paper, we present \tool, an interactive system designed to help novice users create high-quality, fine-grained images that align with their intentions based on rough sketches. 
The system first refines the user's initial prompt into a complete and coherent one that matches the rough sketch, ensuring the generated results are both stable, coherent and high quality.
To further support users in achieving fine-grained alignment between the generated image and their creative intent without requiring professional skills, we introduce a decompose-and-recompose strategy. 
This allows users to select desired, refined object shapes for individual decomposed objects and then recombine them, providing flexible mask manipulation for precise spatial control.
The framework operates through a coarse-to-fine process, enabling iterative and fine-grained control that is not possible with traditional end-to-end generation methods. 
Our user study demonstrates that \tool offers novice users enhanced flexibility in control and fine-grained alignment between their intentions and the generated images.


\section*{Limitations}  
While our study provides valuable insights into the impact of adversarial loss functions and document selection strategies on RAG model robustness, several limitations remain. First, our analysis is restricted to dense transformer-based models, leaving the effectiveness of these techniques on sparse models, such as mixture-of-experts (MoE) architectures, unexplored. Future work could investigate whether similar trends hold for sparsely activated models with dynamic routing mechanisms. Second, although we analyze the effectiveness of adversarial training, we do not explicitly examine its long-term stability or convergence properties, which may vary depending on hyperparameter choices and optimization dynamics. Additionally, while we demonstrate that stronger models exhibit diminishing returns from adversarial losses and document selection strategies, the precise mechanisms behind this phenomenon remain unclear. Further research is needed to understand how model capacity interacts with retrieval robustness.

\section*{Ethics Statements}  
Our study focuses on improving the robustness of RAG models, but several ethical considerations must be acknowledged. First, while adversarial training enhances model reliability, it does not eliminate the risk of biased or misleading outputs, particularly when retrieval sources contain inherent biases or misinformation. Future work should explore fairness-aware adversarial training to mitigate potential harms. Second, our findings suggest that stronger models require less intervention in document selection and loss design, which may influence resource allocation in real-world applications. Researchers and practitioners should ensure that model improvements do not disproportionately benefit well-resourced institutions while leaving smaller models less robust. Lastly, our experiments are conducted on widely used benchmark datasets, which may not fully reflect the diversity of real-world information needs. We encourage further research on robustness evaluation across varied domains, including low-resource languages and specialized knowledge fields, to ensure equitable advancements in RAG technology.


\bibliography{anthology,custom}

% \clearpage
\appendix
% E5数据集介绍,数据集处理过程
% 基线模型介绍

\definecolor{titlecolor}{rgb}{0.9, 0.5, 0.1}
\definecolor{anscolor}{rgb}{0.2, 0.5, 0.8}
\definecolor{labelcolor}{HTML}{48a07e}
\begin{table*}[h]
	\centering
	
 % \vspace{-0.2cm}
	
	\begin{center}
		\begin{tikzpicture}[
				chatbox_inner/.style={rectangle, rounded corners, opacity=0, text opacity=1, font=\sffamily\scriptsize, text width=5in, text height=9pt, inner xsep=6pt, inner ysep=6pt},
				chatbox_prompt_inner/.style={chatbox_inner, align=flush left, xshift=0pt, text height=11pt},
				chatbox_user_inner/.style={chatbox_inner, align=flush left, xshift=0pt},
				chatbox_gpt_inner/.style={chatbox_inner, align=flush left, xshift=0pt},
				chatbox/.style={chatbox_inner, draw=black!25, fill=gray!7, opacity=1, text opacity=0},
				chatbox_prompt/.style={chatbox, align=flush left, fill=gray!1.5, draw=black!30, text height=10pt},
				chatbox_user/.style={chatbox, align=flush left},
				chatbox_gpt/.style={chatbox, align=flush left},
				chatbox2/.style={chatbox_gpt, fill=green!25},
				chatbox3/.style={chatbox_gpt, fill=red!20, draw=black!20},
				chatbox4/.style={chatbox_gpt, fill=yellow!30},
				labelbox/.style={rectangle, rounded corners, draw=black!50, font=\sffamily\scriptsize\bfseries, fill=gray!5, inner sep=3pt},
			]
											
			\node[chatbox_user] (q1) {
				\textbf{System prompt}
				\newline
				\newline
				You are a helpful and precise assistant for segmenting and labeling sentences. We would like to request your help on curating a dataset for entity-level hallucination detection.
				\newline \newline
                We will give you a machine generated biography and a list of checked facts about the biography. Each fact consists of a sentence and a label (True/False). Please do the following process. First, breaking down the biography into words. Second, by referring to the provided list of facts, merging some broken down words in the previous step to form meaningful entities. For example, ``strategic thinking'' should be one entity instead of two. Third, according to the labels in the list of facts, labeling each entity as True or False. Specifically, for facts that share a similar sentence structure (\eg, \textit{``He was born on Mach 9, 1941.''} (\texttt{True}) and \textit{``He was born in Ramos Mejia.''} (\texttt{False})), please first assign labels to entities that differ across atomic facts. For example, first labeling ``Mach 9, 1941'' (\texttt{True}) and ``Ramos Mejia'' (\texttt{False}) in the above case. For those entities that are the same across atomic facts (\eg, ``was born'') or are neutral (\eg, ``he,'' ``in,'' and ``on''), please label them as \texttt{True}. For the cases that there is no atomic fact that shares the same sentence structure, please identify the most informative entities in the sentence and label them with the same label as the atomic fact while treating the rest of the entities as \texttt{True}. In the end, output the entities and labels in the following format:
                \begin{itemize}[nosep]
                    \item Entity 1 (Label 1)
                    \item Entity 2 (Label 2)
                    \item ...
                    \item Entity N (Label N)
                \end{itemize}
                % \newline \newline
                Here are two examples:
                \newline\newline
                \textbf{[Example 1]}
                \newline
                [The start of the biography]
                \newline
                \textcolor{titlecolor}{Marianne McAndrew is an American actress and singer, born on November 21, 1942, in Cleveland, Ohio. She began her acting career in the late 1960s, appearing in various television shows and films.}
                \newline
                [The end of the biography]
                \newline \newline
                [The start of the list of checked facts]
                \newline
                \textcolor{anscolor}{[Marianne McAndrew is an American. (False); Marianne McAndrew is an actress. (True); Marianne McAndrew is a singer. (False); Marianne McAndrew was born on November 21, 1942. (False); Marianne McAndrew was born in Cleveland, Ohio. (False); She began her acting career in the late 1960s. (True); She has appeared in various television shows. (True); She has appeared in various films. (True)]}
                \newline
                [The end of the list of checked facts]
                \newline \newline
                [The start of the ideal output]
                \newline
                \textcolor{labelcolor}{[Marianne McAndrew (True); is (True); an (True); American (False); actress (True); and (True); singer (False); , (True); born (True); on (True); November 21, 1942 (False); , (True); in (True); Cleveland, Ohio (False); . (True); She (True); began (True); her (True); acting career (True); in (True); the late 1960s (True); , (True); appearing (True); in (True); various (True); television shows (True); and (True); films (True); . (True)]}
                \newline
                [The end of the ideal output]
				\newline \newline
                \textbf{[Example 2]}
                \newline
                [The start of the biography]
                \newline
                \textcolor{titlecolor}{Doug Sheehan is an American actor who was born on April 27, 1949, in Santa Monica, California. He is best known for his roles in soap operas, including his portrayal of Joe Kelly on ``General Hospital'' and Ben Gibson on ``Knots Landing.''}
                \newline
                [The end of the biography]
                \newline \newline
                [The start of the list of checked facts]
                \newline
                \textcolor{anscolor}{[Doug Sheehan is an American. (True); Doug Sheehan is an actor. (True); Doug Sheehan was born on April 27, 1949. (True); Doug Sheehan was born in Santa Monica, California. (False); He is best known for his roles in soap operas. (True); He portrayed Joe Kelly. (True); Joe Kelly was in General Hospital. (True); General Hospital is a soap opera. (True); He portrayed Ben Gibson. (True); Ben Gibson was in Knots Landing. (True); Knots Landing is a soap opera. (True)]}
                \newline
                [The end of the list of checked facts]
                \newline \newline
                [The start of the ideal output]
                \newline
                \textcolor{labelcolor}{[Doug Sheehan (True); is (True); an (True); American (True); actor (True); who (True); was born (True); on (True); April 27, 1949 (True); in (True); Santa Monica, California (False); . (True); He (True); is (True); best known (True); for (True); his roles in soap operas (True); , (True); including (True); in (True); his portrayal (True); of (True); Joe Kelly (True); on (True); ``General Hospital'' (True); and (True); Ben Gibson (True); on (True); ``Knots Landing.'' (True)]}
                \newline
                [The end of the ideal output]
				\newline \newline
				\textbf{User prompt}
				\newline
				\newline
				[The start of the biography]
				\newline
				\textcolor{magenta}{\texttt{\{BIOGRAPHY\}}}
				\newline
				[The ebd of the biography]
				\newline \newline
				[The start of the list of checked facts]
				\newline
				\textcolor{magenta}{\texttt{\{LIST OF CHECKED FACTS\}}}
				\newline
				[The end of the list of checked facts]
			};
			\node[chatbox_user_inner] (q1_text) at (q1) {
				\textbf{System prompt}
				\newline
				\newline
				You are a helpful and precise assistant for segmenting and labeling sentences. We would like to request your help on curating a dataset for entity-level hallucination detection.
				\newline \newline
                We will give you a machine generated biography and a list of checked facts about the biography. Each fact consists of a sentence and a label (True/False). Please do the following process. First, breaking down the biography into words. Second, by referring to the provided list of facts, merging some broken down words in the previous step to form meaningful entities. For example, ``strategic thinking'' should be one entity instead of two. Third, according to the labels in the list of facts, labeling each entity as True or False. Specifically, for facts that share a similar sentence structure (\eg, \textit{``He was born on Mach 9, 1941.''} (\texttt{True}) and \textit{``He was born in Ramos Mejia.''} (\texttt{False})), please first assign labels to entities that differ across atomic facts. For example, first labeling ``Mach 9, 1941'' (\texttt{True}) and ``Ramos Mejia'' (\texttt{False}) in the above case. For those entities that are the same across atomic facts (\eg, ``was born'') or are neutral (\eg, ``he,'' ``in,'' and ``on''), please label them as \texttt{True}. For the cases that there is no atomic fact that shares the same sentence structure, please identify the most informative entities in the sentence and label them with the same label as the atomic fact while treating the rest of the entities as \texttt{True}. In the end, output the entities and labels in the following format:
                \begin{itemize}[nosep]
                    \item Entity 1 (Label 1)
                    \item Entity 2 (Label 2)
                    \item ...
                    \item Entity N (Label N)
                \end{itemize}
                % \newline \newline
                Here are two examples:
                \newline\newline
                \textbf{[Example 1]}
                \newline
                [The start of the biography]
                \newline
                \textcolor{titlecolor}{Marianne McAndrew is an American actress and singer, born on November 21, 1942, in Cleveland, Ohio. She began her acting career in the late 1960s, appearing in various television shows and films.}
                \newline
                [The end of the biography]
                \newline \newline
                [The start of the list of checked facts]
                \newline
                \textcolor{anscolor}{[Marianne McAndrew is an American. (False); Marianne McAndrew is an actress. (True); Marianne McAndrew is a singer. (False); Marianne McAndrew was born on November 21, 1942. (False); Marianne McAndrew was born in Cleveland, Ohio. (False); She began her acting career in the late 1960s. (True); She has appeared in various television shows. (True); She has appeared in various films. (True)]}
                \newline
                [The end of the list of checked facts]
                \newline \newline
                [The start of the ideal output]
                \newline
                \textcolor{labelcolor}{[Marianne McAndrew (True); is (True); an (True); American (False); actress (True); and (True); singer (False); , (True); born (True); on (True); November 21, 1942 (False); , (True); in (True); Cleveland, Ohio (False); . (True); She (True); began (True); her (True); acting career (True); in (True); the late 1960s (True); , (True); appearing (True); in (True); various (True); television shows (True); and (True); films (True); . (True)]}
                \newline
                [The end of the ideal output]
				\newline \newline
                \textbf{[Example 2]}
                \newline
                [The start of the biography]
                \newline
                \textcolor{titlecolor}{Doug Sheehan is an American actor who was born on April 27, 1949, in Santa Monica, California. He is best known for his roles in soap operas, including his portrayal of Joe Kelly on ``General Hospital'' and Ben Gibson on ``Knots Landing.''}
                \newline
                [The end of the biography]
                \newline \newline
                [The start of the list of checked facts]
                \newline
                \textcolor{anscolor}{[Doug Sheehan is an American. (True); Doug Sheehan is an actor. (True); Doug Sheehan was born on April 27, 1949. (True); Doug Sheehan was born in Santa Monica, California. (False); He is best known for his roles in soap operas. (True); He portrayed Joe Kelly. (True); Joe Kelly was in General Hospital. (True); General Hospital is a soap opera. (True); He portrayed Ben Gibson. (True); Ben Gibson was in Knots Landing. (True); Knots Landing is a soap opera. (True)]}
                \newline
                [The end of the list of checked facts]
                \newline \newline
                [The start of the ideal output]
                \newline
                \textcolor{labelcolor}{[Doug Sheehan (True); is (True); an (True); American (True); actor (True); who (True); was born (True); on (True); April 27, 1949 (True); in (True); Santa Monica, California (False); . (True); He (True); is (True); best known (True); for (True); his roles in soap operas (True); , (True); including (True); in (True); his portrayal (True); of (True); Joe Kelly (True); on (True); ``General Hospital'' (True); and (True); Ben Gibson (True); on (True); ``Knots Landing.'' (True)]}
                \newline
                [The end of the ideal output]
				\newline \newline
				\textbf{User prompt}
				\newline
				\newline
				[The start of the biography]
				\newline
				\textcolor{magenta}{\texttt{\{BIOGRAPHY\}}}
				\newline
				[The ebd of the biography]
				\newline \newline
				[The start of the list of checked facts]
				\newline
				\textcolor{magenta}{\texttt{\{LIST OF CHECKED FACTS\}}}
				\newline
				[The end of the list of checked facts]
			};
		\end{tikzpicture}
        \caption{GPT-4o prompt for labeling hallucinated entities.}\label{tb:gpt-4-prompt}
	\end{center}
\vspace{-0cm}
\end{table*}

% \begin{figure}[t]
%     \centering
%     \includegraphics[width=0.9\linewidth]{Image/abla2/doc7.png}
%     \caption{Improvement of generated documents over direct retrieval on different models.}
%     \label{fig:comparison}
% \end{figure}

\begin{figure}[t]
    \centering
    \subfigure[Unsupervised Dense Retriever.]{
        \label{fig:imp:unsupervised}
        \includegraphics[width=0.8\linewidth]{Image/A.3_fig/improvement_unsupervised.pdf}
    }
    \subfigure[Supervised Dense Retriever.]{
        \label{fig:imp:supervised}
        \includegraphics[width=0.8\linewidth]{Image/A.3_fig/improvement_supervised.pdf}
    }
    
    % \\
    % \subfigure[Comparison of Reasoning Quality With Different Method.]{
    %     \label{fig:reasoning} 
    %     \includegraphics[width=0.98\linewidth]{images/reasoning1.pdf}
    % }
    \caption{Improvements of LLM-QE in Both Unsupervised and Supervised Dense Retrievers. We plot the change of nDCG@10 scores before and after the query expansion using our LLM-QE model.}
    \label{fig:imp}
\end{figure}
\section{Appendix}
\subsection{License}
The authors of 4 out of the 15 datasets in the BEIR benchmark (NFCorpus, FiQA-2018, Quora, Climate-Fever) and the authors of ELI5 in the E5 dataset do not report the dataset license in the paper or a repository. We summarize the licenses of the remaining datasets as follows.

MS MARCO (MIT License); FEVER, NQ, and DBPedia (CC BY-SA 3.0 license); ArguAna and Touché-2020 (CC BY 4.0 license); CQADupStack and TriviaQA (Apache License 2.0); SciFact (CC BY-NC 2.0 license); SCIDOCS (GNU General Public License v3.0); HotpotQA and SQuAD (CC BY-SA 4.0 license); TREC-COVID (Dataset License Agreement).

All these licenses and agreements permit the use of their data for academic purposes.

\subsection{Additional Experimental Details}\label{app:experiment_detail}
This subsection outlines the components of the training data and presents the prompt templates used in the experiments.


\textbf{Training Datasets.} Following the setup of \citet{wang2024improving}, we use the following datasets: ELI5 (sample ratio 0.1)~\cite{fan2019eli5}, HotpotQA~\cite{yang2018hotpotqa}, FEVER~\cite{thorne2018fever}, MS MARCO passage ranking (sample ratio 0.5) and document ranking (sample ratio 0.2)~\cite{bajaj2016ms}, NQ~\cite{karpukhin2020dense}, SQuAD~\cite{karpukhin2020dense}, and TriviaQA~\cite{karpukhin2020dense}. In total, we use 808,740 training examples.

\textbf{Prompt Templates.} Table~\ref{tab:prompt_template} lists all the prompts used in this paper. In each prompt, ``query'' refers to the input query for which query expansions are generated, while ``Related Document'' denotes the ground truth document relevant to the original query. We observe that, in general, the model tends to generate introductory phrases such as ``Here is a passage to answer the question:'' or ``Here is a list of keywords related to the query:''. Before using the model outputs as query expansions, we first filter out these introductory phrases to ensure cleaner and more precise expansion results.



\subsection{Query Expansion Quality of LLM-QE}\label{app:analysis}
This section evaluates the quality of query expansion of LLM-QE. As shown in Figure~\ref{fig:imp}, we randomly select 100 samples from each dataset to assess the improvement in retrieval performance before and after applying LLM-QE.

Overall, the evaluation results demonstrate that LLM-QE consistently improves retrieval performance in both unsupervised (Figure~\ref{fig:imp:unsupervised}) and supervised (Figure~\ref{fig:imp:supervised}) settings. However, for the MS MARCO dataset, LLM-QE demonstrates limited effectiveness in the supervised setting. This can be attributed to the fact that MS MARCO provides higher-quality training signals, allowing the dense retriever to learn sufficient matching signals from relevance labels. In contrast, LLM-QE leads to more substantial performance improvements on the NQ and HotpotQA datasets. This indicates that LLM-QE provides essential matching signals for dense retrievers, particularly in retrieval scenarios where high-quality training signals are scarce.


\subsection{Case Study}\label{app:case_study}
\begin{figure}[htb]
\small
\begin{tcolorbox}[left=3pt,right=3pt,top=3pt,bottom=3pt,title=\textbf{Conversation History:}]
[human]: Craft an intriguing opening paragraph for a fictional short story. The story should involve a character who wakes up one morning to find that they can time travel.

...(Human-Bot Dialogue Turns)... \textcolor{blue}{(Topic: Time-Travel Fiction)}

[human]: Please describe the concept of machine learning. Could you elaborate on the differences between supervised, unsupervised, and reinforcement learning? Provide real-world examples of each.

...(Human-Bot Dialogue Turns)... \textcolor{blue}{(Topic: Machine learning Concepts and Types)}


[human]: Discuss antitrust laws and their impact on market competition. Compare the antitrust laws in US and China along with some case studies

...(Human-Bot Dialogue Turns)... \textcolor{blue}{(Topic: Antitrust Laws and Market Competition)}

[human]: The vertices of a triangle are at points (0, 0), (-1, 1), and (3, 3). What is the area of the triangle?

...(Human-Bot Dialogue Turns)... \textcolor{blue}{(Topic: Area and Circumcircle of Triangle)}

...(Human-Bot Dialogue Interaction on Other Topics)... 

\end{tcolorbox}
\begin{tcolorbox}[left=3pt,right=3pt,top=3pt,bottom=3pt,title=\textbf{Question:}]
What title reflects the thematic essence of the Doctor's experience with time travel as described in the story?
\end{tcolorbox}
\begin{tcolorbox}[left=3pt,right=3pt,top=3pt,bottom=3pt,title=\textbf{Ground-Truth Answer:}]
Awakening to Eternity: The Doctor's New Dawn
\end{tcolorbox}
\begin{tcolorbox}[left=3pt,right=3pt,top=3pt,bottom=3pt,title=\textbf{Retrieved Memory (Turn-Level):}]
[human]: Describe the process and write out the balanced chemical equation for the reaction that occurs when solid calcium carbonate reacts with hydrochloric acid to form aqueous calcium chloride, carbon dioxide, and water. What type of reaction is this, and what observations might indicate that the reaction is taking place?

[bot]: When solid calcium carbonate ($\text{CaCO}_{3}$) reacts with hydrochloric acid (HCl), the products formed are aqueous calcium chloride ($\text{CaCl}_{2}$), carbon dioxide ($\text{CO}_{2}$), and water ($\text{H}_{2}\text{O}$). The balanced chemical equation for this reaction is: ... \textcolor{blue}{(Topic: Calcium Carbonate and Hydrochloric Acid Reaction)}

\textcolor{brown}{
[human]: Write a title for your story.}

\textcolor{brown}{
[bot]: "Awakening to Eternity: The Doctor's New Dawn".
} \textcolor{blue}{(Topic: Time-Travel Fiction)}

[human]: Analyze the following customer reviews from different sources for three different smartphones - the latest iPhone, Samsung Galaxy, and Google Pixel - and provide an overall rating for each phone on a scale of 1 to 10. Consider the following complex and contradictory reviews: - TechRadar's review of the latest iPhone: The new iPhone is a stunning triumph of engineering that sets a new bar for smartphone performance and camera quality. However, the incremental design and high price mean it lacks the 'wow' factor of previous iPhones. Still, its power and intelligence are unrivaled. - CNET's review of the latest Samsung Galaxy: ... Return the answer as a JSON object with the overall ratings for each phone out of 10, to one decimal place.

[bot]: ```json { "iPhone": 8.5, "Samsung Galaxy": 8.0, "Google Pixel": 7.5 } ```

\textcolor{blue}{(Topic: Smartphone Review Ratings Analysis)}
\end{tcolorbox}
\begin{tcolorbox}[left=3pt,right=3pt,top=3pt,bottom=3pt,title=\textbf{Answer (Turn-Level Retrieval):}]
As an AI language model, I do not have access to the story you mentioned earlier. Please provide me with the story or the necessary details so that I can assist you better.

\textbf{\textcolor{red}{(The next part of this figure~\ref{fig: case_study_segment_vs_turn} will be continued on the next page.)}}
\end{tcolorbox}
\end{figure}

\begin{figure}[htb]
% \ContinuedFloat
\small
\begin{tcolorbox}[left=3pt,right=3pt,top=3pt,bottom=3pt,title=\textbf{Retrieved Memory (Ours):}]
[human]: Craft an intriguing opening paragraph for a fictional short story. The story should involve a character who wakes up one morning to find that they can time travel.

[bot]: In the hushed pre-dawn hours, when the world still slumbers under the cloak of night, Alex found themselves caught in the fragile boundary between sleep and wakefulness. ...

[human]: Summarize the story with three bullet points using only nouns and adjectives, without verbs.

[bot]: - Alex, pre-dawn awakening, time-travel discovery - Disorientation, vertigo, reality's malleability - Sunlight, new day, astonishing truth

[human]: Is there any famous fictional character that can time travel? If so, can you rewrite your story with him or her?

[bot]: - \textcolor{brown}{Doctor Who, a renowned time-traveling character from British television, is a perfect fit for this scenario.} In the silent embrace of early morning, where shadows merge with the fading night, the Doctor found themselves in the thin veil separating dreams from consciousness. It was here, amidst the soft whispers of time, that an unusual sensation enveloped them—a sense of time's fabric slipping through their fingers like grains of sand. As dawn's light broke through the darkness, painting the room in hues of gold and amber, the Doctor's eyes opened to what appeared to be an ordinary day. \textcolor{brown}{However, upon standing, a wave of dizziness overwhelmed them, the surroundings blurred, and in that moment of confusion, the Doctor realized a profound truth: they had not only awakened to a new dawn but had also stumbled upon the incredible ability to navigate the currents of time at will.}

\textcolor{brown}{[human]: Write a title for your story.}

\textcolor{brown}{[bot]: "Awakening to Eternity: The Doctor's New Dawn"} \textcolor{blue}{(Topic: Time-Travel Fiction)}
\end{tcolorbox}
\begin{tcolorbox}[left=3pt,right=3pt,top=3pt,bottom=3pt,title=\textbf{Answer (Ours):}]
The title "Awakening to Eternity: The Doctor's New Dawn" reflects the thematic essence of the Doctor's experience with time travel as described in the story. It captures the idea of a new beginning, a sense of eternity, and the awakening of a new power that will change the course of the Doctor's life forever.
\end{tcolorbox}
\caption{Case study comparing our \sysname\ with the \textit{turn-level} memory baseline. When constructing the memory bank at the \textit{turn-level}, the agent fails to provide a satisfactory response due to fragmentation: the retrieved dialogue turns can not capture a complete human-agent interaction related to the query, missing critical information highlighted in \textcolor{brown}{brown}. The dialogue topic is emphasized in \textcolor{blue}{blue}. Some context is omitted (replaced by ...) due to length constraints.}
\label{fig: case_study_segment_vs_turn}
\end{figure}

\newpage

\begin{figure}[htbp]
\small
\begin{tcolorbox}[left=3pt,right=3pt,top=3pt,bottom=3pt,title=\textbf{Conversation History:}]
[human]: Explain what's base rate fallacy and list five specific examples of ...

... (Human-Bot Dialogue Turns) ... \textcolor{blue}{(Topic: Base Rate Fallacy In Politics)}


[human]: How many integers are in the solution of the inequality $|x + 5| < 10$?

... (Human-Bot Dialogue Turns) ... \textcolor{blue}{(Topic: Number of Integers in Inequalities)}

[human]: How have the Alps and Rhine River influenced settlement and agriculture in Western Europe?

... (Human-Bot Dialogue Turns) ... \textcolor{blue}{(Topic: Alps and Rhine River Impact)}

[human]: In a survey conducted at a local high school, preferences for a new school color were measured: 58\% of students liked the color blue, 45\% preferred green, and 22\% liked both colors. If we randomly pick a student from the school, what's the probability that they would like neither blue nor green?

... (Human-Bot Dialogue Turns) ... \textcolor{blue}{(Topic: Probability of Liking Colors in Survey)}

[human]: Imagine you are participating in a race with a group of people. If you have just overtaken the second person, what's your current position? Where is the person you just overtook?

... (Human-Bot Dialogue Turns) ... \textcolor{blue}{(Topic: Overtaking Positions in Race)}
\end{tcolorbox}
\begin{tcolorbox}[left=3pt,right=3pt,top=3pt,bottom=3pt,title=\textbf{Question:}]
[human]: In the inequality question, how many positive integers are in the solution of the second inequality?
\end{tcolorbox}
\begin{tcolorbox}[left=3pt,right=3pt,top=3pt,bottom=3pt,title=\textbf{Ground-Truth Answer:}]
In the second inequality, $(|x + 10| < 5)$, the solution range is $(-15 < x < -5)$. Within this range, there are no positive integers, as all values of $(x)$ that satisfy the inequality are negative.
\end{tcolorbox}
\begin{tcolorbox}[left=3pt,right=3pt,top=3pt,bottom=3pt,title=\textbf{Retrieved Memory (Session-Level):}]
[human]: Explain what's base rate fallacy and list five specific examples of ...

... (Human-Bot Dialogue Turns) ... \textcolor{blue}{(Topic: Base Rate Fallacy In Politics)}

[human]: How many integers are in the solution of the inequality $|x + 5| < 10$?

[bot]: To solve the inequality $|x + 5| < 10$, ... 

\textcolor{brown}{
[human]: What about $|x + 10| < 5$?
}

\textcolor{brown}{
[bot]: To solve the inequality $|x + 10| < 5$, we ...} \textcolor{gray}{(``overwhelmed'' by other irrelevant, redundant content)}

... (Human-Bot Dialogue Turns) ...  \textcolor{blue}{(Topic: Number of Integers in Inequalities)}

% \textcolor{brown}{
% [human]: What about $|x + 10| < 5$
% }

% \textcolor{brown}{
% [bot]: To solve the inequality $|x + 10| < 5$, ...
% } \textcolor{blue}{(Topic: Number of Integers in Inequalities)}

[human]: How have the Alps and Rhine River influenced settlement and agriculture in Western Europe?

... (Human-Bot Dialogue Turns) ... \textcolor{blue}{(Topic: Alps and Rhine River Impact)}

[human]: In a survey conducted at a local high school, ...

... (Human-Bot Dialogue Turns) ... \textcolor{blue}{(Topic: Probability of Liking Colors in Survey)}
\end{tcolorbox}
\begin{tcolorbox}[left=3pt,right=3pt,top=3pt,bottom=3pt,title=\textbf{Answer (Session-Level Retrieval):}]
I apologize for the confusion earlier. To answer your question, the second inequality was not provided in our previous conversation. If you could provide the inequality, I would be happy to help you find the solution and the number of positive integers in the solution set.

\textbf{\textcolor{red}{(The next part of this figure~\ref{fig: case_study_segment_vs_session} will be continued on the next page.)}}
\end{tcolorbox}
\end{figure}
\begin{figure}[htbp]
% \ContinuedFloat
\small
\begin{tcolorbox}[left=3pt,right=3pt,top=3pt,bottom=3pt,title=\textbf{Retrieved Memory (Ours):}]
[human]: How many integers are in the solution of the inequality $|x + 5| < 10$?

[bot]: To solve the inequality $|x + 5| < 10$, we break it into two cases based on the definition of absolute value: 
1. $x + 5 < 10$ 
2. $-(x + 5) < 10$, which simplifies to $x + 5 > -10$ 
Solving each case: 
1. $x < 5$ 
2. $x > -15$ 
Combining these, we get $-15 < x < 5$. This inequality means that $x$ can be any number greater than $-15$ and less than $5$. To find how many integers satisfy this condition, we list them: $-14, -13, -12, -11, -10, -9, -8, -7, -6, -5, -4, -3, -2, -1, 0, 1, 2, 3, 4$. There are 19 integers that satisfy the inequality $|x + 5| < 10$.

\textcolor{brown}{
[human]: What about $|x + 10| < 5$?
}

\textcolor{brown}{
[bot]: To solve the inequality $|x + 10| < 5$, we again consider two cases: 
1. $x + 10 < 5$ 
2. $-(x + 10) < 5$, which simplifies to $x + 10 > -5$ 
Solving each case: 
1. $x < -5$ 
2. $x > -15$ 
Combining these, we get $-15 < x < -5$. This inequality means that $x$ can be any number greater than $-15$ and less than $-5$. To find how many integers satisfy this condition, we list them: $-14, -13, -12, -11, -10, -9, -8, -7, -6$. There are 9 integers that satisfy the inequality $|x + 10| < 5$.
}

[human]: What about $|x + 10| < 10$?

[bot]: To solve the inequality $|x + 10| < 10$, we consider two cases: 
1. $x + 10 < 10$ 
2. $-(x + 10) < 10$, which simplifies to $x + 10 > -10$ 
Solving each case: 
1. $x < 0$ 
2. $x > -20$ 
Combining these, we get $-20 < x < 0$. This inequality means that $x$ can be any number greater than $-20$ and less than $0$. To find how many integers satisfy this condition, we list them: $-19, -18, -17, -16, -15, -14, -13, -12, -11, -10, -9, -8, -7, -6, -5, -4, -3, -2, -1$. There are 19 integers that satisfy the inequality $|x + 10| < 10$. \textcolor{blue}{(Topic: Number of Integers in Inequalities)}
\end{tcolorbox}
\begin{tcolorbox}[left=3pt,right=3pt,top=3pt,bottom=3pt,title=\textbf{Answer (Ours):}]
To find the number of positive integers that satisfy the inequality $(|x + 10| < 5)$, we need to look for positive integers in the solution set $(-15 < x < -5)$. Since there are no positive integers in this interval, the answer is 0.

\end{tcolorbox}
\caption{Case study comparing our \sysname\ with the \textit{session-level} memory baseline. When constructing the memory bank at the \textit{session-level}, the agent fails to provide a satisfactory response due to the inclusion of excessive irrelevant information within a session. This distracts the generator from focusing on the truly useful information highlighted in \textcolor{brown}{brown}. The dialogue topic is emphasized in \textcolor{blue}{blue}. Some context is omitted (replaced by ...) due to length constraints.}
\label{fig: case_study_segment_vs_session}
\end{figure}



\begin{figure}[htb]
\small
\begin{tcolorbox}[left=3pt,right=3pt,top=3pt,bottom=3pt,title=\textbf{Conversation History:}]
[human]: Photosynthesis is a vital process for life on Earth. Could you outline the two main stages of photosynthesis, including where they take place within the chloroplast, and the primary inputs and outputs for each stage? ... (Human-Bot Dialogue Turns)... \textcolor{blue}{(Topic: Photosynthetic Energy Production)}

[human]: Please assume the role of an English translator, tasked with correcting and enhancing spelling and language. Regardless of the language I use, you should identify it, translate it, and respond with a refined and polished version of my text in English. 

... (Human-Bot Dialogue Turns)...  \textcolor{blue}{(Topic: Language Translation and Enhancement)}

[human]: Suggest five award-winning documentary films with brief background descriptions for aspiring filmmakers to study.

\textcolor{brown}{[bot]: ...
5. \"An Inconvenient Truth\" (2006) - Directed by Davis Guggenheim and featuring former United States Vice President Al Gore, this documentary aims to educate the public about global warming. It won two Academy Awards, including Best Documentary Feature. The film is notable for its straightforward yet impactful presentation of scientific data, making complex information accessible and engaging, a valuable lesson for filmmakers looking to tackle environmental or scientific subjects.}

... (Human-Bot Dialogue Turns)... 
\textcolor{blue}{(Topic: Documentary Films Recommendation)}

[human]: Given the following records of stock prices, extract the highest and lowest closing prices for each month in the year 2022. Return the results as a CSV string, with one line allocated for each month. Date,Open,High,Low,Close,Volume ... ... (Human-Bot Dialogue Turns)...  \textcolor{blue}{(Topic: Stock Prices Analysis)}

[human]: The city of Vega intends to build a bridge that will span the Vegona River, covering a distance of 1.8 kilometers. The proposed location falls within a seismically active area that has experienced several high-magnitude earthquakes. Given these circumstances, what would be the best approach to constructing the bridge? ... (Human-Bot Dialogue Turns)...  \textcolor{blue}{(Topic: Seismic Bridge Construction)}
\end{tcolorbox}
\begin{tcolorbox}[left=3pt,right=3pt,top=3pt,bottom=3pt,title=\textbf{Question:}]
Which documentary won two Academy Awards and what lesson does it offer to filmmakers?
\end{tcolorbox}
\begin{tcolorbox}[left=3pt,right=3pt,top=3pt,bottom=3pt,title=\textbf{Ground-Truth Answer:}]
``An Inconvenient Truth" (2006) won two Academy Awards. It offers the lesson of making complex information accessible and engaging, a valuable lesson for filmmakers looking to tackle environmental or scientific subjects.
\end{tcolorbox}
\begin{tcolorbox}[left=3pt,right=3pt,top=3pt,bottom=3pt,title=\textbf{Memory generated by RecurSum~\citep{wang2023recursively}:}]
The dialogue provided detailed information on various topics, including the stages of photosynthesis, energy production estimates through photosynthesis for different organisms, translations and enhancements of phrases in different languages, recommendations for documentary films, a pitch for a film about overcoming adversity, analysis of stock prices, and considerations for constructing a bridge in a seismically active area. Here's a summary of the key points: ...

\textbf{Documentary Film Recommendations}: Suggested five award-winning documentary films for aspiring filmmakers to study, with brief background descriptions for each. The response included:

-  "March of the Penguins" (2005) - Emperor penguins' journey in Antarctica; noted for its cinematography and portrayal of the penguins' life cycle.

-  "Citizenfour" (2014) - Edward Snowden and NSA spying scandal; known for its real-time exploration of Snowden's revelations.

-  "20 Feet from Stardom" (2013) - Stories of backup singers behind musical legends; celebrates unsung heroes in the music industry.

-  "Man on Wire" (2008) - Philippe Petit's 1974 high-wire walk between the Twin Towers; combines personal footage, interviews, and re-enactments.

-  \textcolor{brown}{"An Inconvenient Truth" (2006) - Al Gore's documentary on global warming; known for its presentation of scientific data.}
\end{tcolorbox}
\begin{tcolorbox}[left=3pt,right=3pt,top=3pt,bottom=3pt,title=\textbf{Answer (RSum):}]
I'm sorry, but based on the updated context you provided, I don't see any previous discussion related to the question you asked. However, to answer your question, the documentary that won two Academy Awards is "Free Solo," which won for Best Documentary Feature and Best Achievement in Film Editing in 2019. 
% The lesson it offers to filmmakers is the importance of taking risks and pushing boundaries in order to create something truly unique and impactful.
....
The film follows climber Alex Honnold as he attempts to free solo climb El Capitan in Yosemite National Park, and the filmmakers had to take risks themselves in order to capture the footage and tell the story.

\textbf{\textcolor{red}{(The next part of this figure~\ref{fig: case_study_segment_vs_rsum} will be continued on the next page.)}}
\end{tcolorbox}
\end{figure}
\begin{figure}[htbp]
% \ContinuedFloat
\small
\begin{tcolorbox}[left=3pt,right=3pt,top=3pt,bottom=3pt,title=\textbf{Retrieved Memory (Ours):}]
[human]: Suggest five award-winning documentary films with brief background descriptions for aspiring filmmakers to study.

[bot]: 1. **"March of the Penguins" (2005)** - Directed by Luc Jacquet, this French feature-length nature documentary delves into the annual journey of Emperor penguins across the Antarctic. It won the Academy Award for Best Documentary Feature. The film is celebrated for its stunning cinematography and its detailed, heartfelt portrayal of the penguins' life cycle, making it a must-study for filmmakers interested in wildlife documentaries. 2. **"Citizenfour" (2014)** ... 5. \textcolor{brown}{**"An Inconvenient Truth" (2006)** - Directed by Davis Guggenheim and featuring former United States Vice President Al Gore, this documentary aims to educate the public about global warming. It won two Academy Awards, including Best Documentary Feature.} The film is notable for its straightforward yet impactful presentation of scientific data, making complex information accessible and engaging, \textcolor{brown}{a valuable lesson for filmmakers looking to tackle environmental or scientific subjects.}

[human]: With the spirit in the first film, craft a succinct and persuasive pitch for a film about overcoming adversity.

[bot]: Drawing inspiration from the resilience and majestic journey of Emperor penguins in "March of the Penguins," our documentary, "Unyielding Spirits: The Human Odyssey," will explore the profound stories of individuals around the globe who have faced and overcome staggering adversities. Through breathtaking cinematography and intimate storytelling, we ...
\textcolor{blue}{(Topic: Documentary Films Recommendation)}
\end{tcolorbox}
\begin{tcolorbox}[left=3pt,right=3pt,top=3pt,bottom=3pt,title=\textbf{Answer (Ours):}]
``An Inconvenient Truth" won two Academy Awards, including Best Documentary Feature. The film offers a valuable lesson to filmmakers looking to tackle environmental or scientific subjects, as it presents complex information in a straightforward yet impactful way, making it accessible and engaging to a wider audience.


\end{tcolorbox}
\caption{Case study comparing our \sysname\ with RecurSum~\citep{wang2023recursively}. When using RecurSum, the agent fails to provide a satisfactory answer due to information loss during the summarization process. Summarization omits essential details needed to answer the user's new request, as highlighted in \textcolor{brown}{brown}. The dialogue topic is emphasized in \textcolor{blue}{blue}. Some context is omitted (replaced by ...) due to length constraints.}
\label{fig: case_study_segment_vs_rsum}
\end{figure}




\begin{figure}[htb]
\small
\begin{tcolorbox}[left=3pt,right=3pt,top=3pt,bottom=3pt,title=\textbf{Conversation History:}]
[human]: Photosynthesis is a vital process for life on Earth. Could you outline the two main stages of photosynthesis, including where they take place within the chloroplast, and the primary inputs and outputs for each stage? ... (Human-Bot Dialogue Turns)... \textcolor{blue}{(Topic: Photosynthetic Energy Production)}

[human]: Please assume the role of an English translator, tasked with correcting and enhancing spelling and language. Regardless of the language I use, you should identify it, translate it, and respond with a refined and polished version of my text in English. 

... (Human-Bot Dialogue Turns)...  \textcolor{blue}{(Topic: Language Translation and Enhancement)}

[human]: Suggest five award-winning documentary films with brief background descriptions for aspiring filmmakers to study.

\textcolor{brown}{[bot]: ...
5. \"An Inconvenient Truth\" (2006) - Directed by Davis Guggenheim and featuring former United States Vice President Al Gore, this documentary aims to educate the public about global warming. It won two Academy Awards, including Best Documentary Feature. The film is notable for its straightforward yet impactful presentation of scientific data, making complex information accessible and engaging, a valuable lesson for filmmakers looking to tackle environmental or scientific subjects.}

... (Human-Bot Dialogue Turns)... 
\textcolor{blue}{(Topic: Documentary Films Recommendation)}

[human]: Given the following records of stock prices, extract the highest and lowest closing prices for each month in the year 2022. Return the results as a CSV string, with one line allocated for each month. Date,Open,High,Low,Close,Volume ... ... (Human-Bot Dialogue Turns)...  \textcolor{blue}{(Topic: Stock Prices Analysis)}

[human]: The city of Vega intends to build a bridge that will span the Vegona River, covering a distance of 1.8 kilometers. The proposed location falls within a seismically active area that has experienced several high-magnitude earthquakes. Given these circumstances, what would be the best approach to constructing the bridge? ... (Human-Bot Dialogue Turns)...  \textcolor{blue}{(Topic: Seismic Bridge Construction)}
\end{tcolorbox}
\begin{tcolorbox}[left=3pt,right=3pt,top=3pt,bottom=3pt,title=\textbf{Question:}]
Which documentary won two Academy Awards and what lesson does it offer to filmmakers?
\end{tcolorbox}
\begin{tcolorbox}[left=3pt,right=3pt,top=3pt,bottom=3pt,title=\textbf{Ground-Truth Answer:}]
"An Inconvenient Truth" (2006) won two Academy Awards. It offers the lesson of making complex information accessible and engaging, a valuable lesson for filmmakers looking to tackle environmental or scientific subjects.
\end{tcolorbox}
\begin{tcolorbox}[left=3pt,right=3pt,top=3pt,bottom=3pt,title=\textbf{Memory generated by ConditionMem~\citep{yuan2023evolving}:}]
- \textcolor{brown}{The user is seeking recommendations for award-winning documentary films, along with brief background descriptions, for educational purposes.} This request likely stems from a desire to learn from successful examples in the documentary filmmaking field, possibly to improve their own filmmaking skills or to gain inspiration for their projects. 

- \textcolor{brown}{User is interested in award-winning documentary films for educational purposes, aiming to study their backgrounds and filmmaking techniques.}

- The user's request for crafting a succinct and persuasive pitch for a film about overcoming adversity is inspired by the spirit of "March of the Penguins," a documentary that beautifully captures the Emperor penguins' annual journey across the Antarctic, highlighting themes of endurance and survival against the odds. The user seeks to emulate the emotional depth and cinematic quality that made "March of the Penguins" compelling, aiming to create a pitch that resonates with similar themes of resilience and triumph over challenges. 

- User is interested in creating content that focuses on themes of overcoming adversity, inspired by the emotional depth and cinematic excellence of documentaries like "March of the Penguins." They value storytelling that captures the essence of endurance and survival against the odds.
% : The user's request to write a budget plan for the video production directly follows the task of creating a video script aimed at training new employees of a car wash business in Japan, focusing on business etiquette norms.
% The user is likely seeking to understand the financial requirements to bring the previously crafted script to life, indicating a move from the planning and content development phase to the practical aspects of production. 

... (Other Irrelevant Memory Records) ...
\end{tcolorbox}
\begin{tcolorbox}[left=3pt,right=3pt,top=3pt,bottom=3pt,title=\textbf{Answer (Conditional Memory):}]
The documentary that won two Academy Awards is "Free Solo," which follows climber Alex Honnold as he attempts to free solo climb El Capitan in Yosemite National Park. The film offers a lesson to filmmakers on the importance of capturing authentic and compelling human stories, as well as the technical challenges of filming in extreme environments.

\textbf{\textcolor{red}{(The next part of this figure~\ref{fig: case_study_segment_vs_condmem} will be continued on the next page.)}}
\end{tcolorbox}
\end{figure}
\begin{figure}[htbp]
% \ContinuedFloat
\small
\begin{tcolorbox}[left=3pt,right=3pt,top=3pt,bottom=3pt,title=\textbf{Retrieved Memory (Ours):}]
[human]: Suggest five award-winning documentary films with brief background descriptions for aspiring filmmakers to study.

[bot]: 1. **"March of the Penguins" (2005)** - Directed by Luc Jacquet, this French feature-length nature documentary delves into the annual journey of Emperor penguins across the Antarctic. It won the Academy Award for Best Documentary Feature. The film is celebrated for its stunning cinematography and its detailed, heartfelt portrayal of the penguins' life cycle, making it a must-study for filmmakers interested in wildlife documentaries. 2. **"Citizenfour" (2014)** ... 5. \textcolor{brown}{**"An Inconvenient Truth" (2006)** - Directed by Davis Guggenheim and featuring former United States Vice President Al Gore, this documentary aims to educate the public about global warming. It won two Academy Awards, including Best Documentary Feature.} The film is notable for its straightforward yet impactful presentation of scientific data, making complex information accessible and engaging, a valuable lesson for filmmakers looking to tackle environmental or scientific subjects.

[human]: With the spirit in the first film, craft a succinct and persuasive pitch for a film about overcoming adversity.

[bot]: Drawing inspiration from the resilience and majestic journey of Emperor penguins in "March of the Penguins," our documentary, "Unyielding Spirits: The Human Odyssey," will explore the profound stories of individuals around the globe who have faced and overcome staggering adversities. Through breathtaking cinematography and intimate storytelling, we ...
\textcolor{blue}{(Topic: Documentary Films Recommendation)}
\end{tcolorbox}
\begin{tcolorbox}[left=3pt,right=3pt,top=3pt,bottom=3pt,title=\textbf{Answer (Ours):}]
``An Inconvenient Truth" won two Academy Awards, including Best Documentary Feature. The film offers a valuable lesson to filmmakers looking to tackle environmental or scientific subjects, as it presents complex information in a straightforward yet impactful way, making it accessible and engaging to a wider audience.
\end{tcolorbox}
\caption{Case study comparing our \sysname\ with ConditionMem~\citep{yuan2023evolving}. When using ConditionMem, the agent fails to provide a satisfactory answer due to (1) information loss during the summarization process and (2) the incorrect discarding of turns that are actually useful, as highlighted in \textcolor{brown}{brown}. The dialogue topic is emphasized in \textcolor{blue}{blue}. Some context is omitted (replaced by ...) due to length constraints.}
\label{fig: case_study_segment_vs_condmem}
\end{figure}


To further demonstrate the effectiveness of LLM-QE, we conduct a case study by randomly sampling a query from the evaluation dataset. We then compare retrieval performance using the raw queries, expanded queries by vanilla LLM, and expanded queries by LLM-QE.

As shown in Table~\ref{tab:case_study}, query expansion significantly improves retrieval performance compared to using the raw query. Both vanilla LLM and LLM-QE generate expansions that include key phrases, such as ``temperature'', ``humidity'', and ``coronavirus'', which provide crucial signals for document matching. However, vanilla LLM produces inconsistent results, including conflicting claims about temperature ranges and virus survival conditions. In contrast, LLM-QE generates expansions that are more semantically aligned with the golden passage, such as ``the virus may thrive in cooler and more humid environments, which can facilitate its transmission''. This further demonstrates the effectiveness of LLM-QE in improving query expansion by aligning with the ranking preferences of both LLMs and retrievers.



\end{document}
