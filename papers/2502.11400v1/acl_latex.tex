% This must be in the first 5 lines to tell arXiv to use pdfLaTeX, which is strongly recommended.
\pdfoutput=1
% In particular, the hyperref package requires pdfLaTeX in order to break URLs across lines.

\documentclass[11pt]{article}

% Change "review" to "final" to generate the final (sometimes called camera-ready) version.
% Change to "preprint" to generate a non-anonymous version with page numbers.
\usepackage[preprint]{acl}


% Standard package includes
\usepackage{times}
\usepackage{latexsym}

% For proper rendering and hyphenation of words containing Latin characters (including in bib files)
\usepackage[T1]{fontenc}
% For Vietnamese characters
% \usepackage[T5]{fontenc}
% See https://www.latex-project.org/help/documentation/encguide.pdf for other character sets

% This assumes your files are encoded as UTF8
\usepackage[utf8]{inputenc}

% This is not strictly necessary, and may be commented out,
% but it will improve the layout of the manuscript,
% and will typically save some space.
\usepackage{microtype}

% This is also not strictly necessary, and may be commented out.
% However, it will improve the aesthetics of text in
% the typewriter font.
\usepackage{inconsolata}

\usepackage{enumitem}
\usepackage{amsmath, amsthm, amssymb}
\usepackage{amsfonts}
\usepackage{graphicx}
\usepackage{multirow}
\usepackage{multicol}
\usepackage{booktabs}
\usepackage{arydshln}
\usepackage{color}
\usepackage{makecell}
\usepackage{subcaption}
\usepackage{xcolor}[dvipsnames]
\usepackage{epigraph}

\usepackage{colortbl}
\definecolor{mygray}{gray}{.9}
\definecolor{mygreen}{rgb}{0,0.7,0}
\definecolor{myorange}{RGB}{255, 218, 185}
\definecolor{mycolor}{rgb}{0.8157, 0.251, 0.2196}

% If the title and author information does not fit in the area allocated, uncomment the following
%
%\setlength\titlebox{<dim>}
%
% and set <dim> to something 5cm or larger.

\title{Revisiting Robust RAG: Do We Still Need Complex Robust Training \\ in the Era of Powerful LLMs?}

\author{Hanxing Ding$^{1,2}$\thanks{\ \ Equal contributions.}\quad
Shuchang Tao\footnotemark[1] \quad
Liang Pang$^{1}$\thanks{\ \ Corresponding author}\quad
Zihao Wei$^{1,2}$\quad\\
\textbf{Liwei Chen}$^{3}$\quad
\textbf{Kun Xu}\quad
\textbf{Huawei Shen}$^{1,2}$
\textbf{Xueqi Cheng}$^{1,2}$\\
 $^{1}$Key Laboratory of AI Safety, Institute of Computing Technology, Chinese Academy of Sciences \\
 $^{2}$ University of Chinese Academy of Sciences \quad $^{3}$ Kuaishou Technology \\
 \texttt{\{dinghanxing18s, pangliang, weizihao22z, shenhuawei, cxq\}@ict.ac.cn} \\
 % \texttt{\{taoshuchang.tsc, jinyang.gjy, bolin.ding\}@alibaba-inc.com}
}

\begin{document}
\maketitle
\begin{abstract}
Retrieval-augmented generation (RAG) systems often suffer from performance degradation when encountering noisy or irrelevant documents, driving researchers to develop sophisticated training strategies to enhance their robustness against such retrieval noise. However, as large language models (LLMs) continue to advance, the necessity of these complex training methods is increasingly questioned. In this paper, we systematically investigate whether complex robust training strategies remain necessary as model capacity grows. Through comprehensive experiments spanning multiple model architectures and parameter scales, we evaluate various document selection methods and adversarial training techniques across diverse datasets. Our extensive experiments consistently demonstrate that as models become more powerful, the performance gains brought by complex robust training methods drop off dramatically.  We delve into the rationale and find that more powerful models inherently exhibit superior confidence calibration, better generalization across datasets (even when trained with randomly selected documents), and optimal attention mechanisms learned with simpler strategies. Our findings suggest that RAG systems can benefit from simpler architectures and training strategies as models become more powerful, enabling more scalable applications with minimal complexity.
\end{abstract}

\setlength{\epigraphrule}{0pt}
\epigraph{\emph{"Entities should not be multiplied unnecessarily."}}{--- Occam's razor}

\section{Introduction}\label{sec:intro}

In computational finance, Monte Carlo simulations are used extensively to estimate the expected value of financial payoffs based on the solution of stochastic differential equations (SDEs) which model the evolution of stock prices, interest rates, exchange rates and other quantities \cite{glasserman04}.  Monte Carlo methods are very general and flexible, but for high accuracy it requires generating a large number of costly SDE path approximations, which has motivated research into a number of variance reduction or, equivalently, cost reduction techniques. One such method is
Multilevel Monte Carlo (MLMC), which was proposed in \cite{GILES2008} and was adapted for various applications that are summarised in \cite{Giles_overview17} and successfully combined with other methods such as quasi-Monte Carlo methods. The main idea of MLMC is to approximate the payoff using different time stepping resolutions when numerically solving the underlying SDE and to generate an optimal number of samples on each level, such that the overall computational cost is minimised subject to the desired bound on the variance. %, such that the total computational cost is minimised. 
The computational savings come from the fact that most samples are computed on the coarser levels and hence are less expensive while only a few samples from the finest levels are required \cite{GILES2008}.


Among the directions in which the computational cost 
of MLMC methods could further be reduced, an important avenue is the use of lower precision calculations, especially for the first Monte Carlo levels where the targeted accuracy is relatively low. 
 An overview of the research on mixed precision for the standard Monte Carlo (MC) framework is provided in \cite{ChowMixedPrecisionStandardMC} but only a few references study the potential of low precision computation in the MLMC framework \cite{Rounding_error_oliver}. To the best of our knowledge, the only MLMC framework with customised precision in the literature is \cite{brugger2014mixed}, but they use a uniform precision for all operations on each Monte Carlo level instead of optimising 
 the precision of each intermediary variable to reduce as much as possible the cost of path generation.
 
An important motivation for an MLMC framework with variable precision would be performing the low precision computations on reconfigurable hardware devices such as Field Programmable Gate Arrays (FPGAs). FPGAs contain customizable logic blocks and connectors that make it easy to adapt the digital circuit architecture for a specific application, leading to a highly parallel and optimised implementation. Therefore they are successfully exploited in applications that require high speed and have high computational workload, such as signal processing \cite{woods2008fpga}, and real time applications like high frequency trading \cite{HFT1,HFT2}. That is why a number of previous works in hardware architecture design implemented the MLMC algorithm to price financial options using FPGAs as accelerators, which resulted in improved speed and power efficiency compared to full CPU architectures \cite{Schryver2013AMM}. The paper \cite{lindsey2016domain} also proposed 
a Domain Specific Language to automate the configuration of FPGAs for this specific application. However, only \cite{brugger2014mixed} proposed a heuristic to reduce the precision in calculations.

In addition, all aforementioned works considered that the random number generation (RNG) is performed in single or double precision. Yet in most cases an important portion of the workload in the overall MLMC simulation comes from the RNG and in \cite{brugger2014mixed} this limited the total computational savings.
To reduce the cost of MLMC simulations in particular those based on the Geometric Brownian Motion (GBM), \cite{approximateICDF_Oliver, NestedOliver} have proposed to use approximate random numbers that are generated by applying an approximation of the inverse CDF to uniform random numbers. In \cite{NestedOliver}, the authors proposed a way to integrate these lower precision random variables into a \textit{nested} MLMC framework and completed a numerical analysis to bound the resulting error at each MC level by a product of the time step and the error in the random number approximation. The same authors show in \cite{approximateICDF_Oliver} that using approximate random variables reduces the cost of path generation by a factor 7.


In this paper we propose a nested MLMC framework that combines the use of approximate random normal variables and lower precision calculations to reduce the computational cost of MLMC even further than \cite{brugger2014mixed,NestedOliver}. We illustrate the efficiency of our framework in Matlab, after making several assumptions on the cost of operations and size of the errors that we carefully justify. We focus on the case of GBM and use the approximate RNG methods presented in \cite{approximateICDF_Oliver} as well as a new slightly modified method that combines CDF inversion and the central limit theorem. To choose the precision of the variables in the low precision path generation, we introduce a novel method to optimise the bit-widths. This optimisation is performed before the main path generation loop is executed and is based on a linear model of the payoff error  
due to rounding when computing in low precision. The error model relies on algorithmic differentiation in a similar manner to \cite{unifying-bwoptim,bitwidth-AD,ADAPT}. The bit-width optimisation procedure can be performed off-line, so this stage can be excluded from the on-line time complexity of our framework. The user specified desired accuracy is then enforced by calculating on-line the number of samples that need to be generated.

In terms of hardware design, we suggest implementing the low precision path generation on FPGAs and the full-precision ones on a CPU or GPU. 
The FPGA offers enough flexibility to define a separate bit-width for every variable in the low precision path generation, and can be reconfigured periodically to update the bit-widths when the market parameters have changed considerably. 


The paper is organized as follows : \Cref{sec:MLMC} introduces MLMC and nested MLMC to make clear the estimator that is implemented in our framework. Then in \Cref{sec:RNG} we detail the methods that could be used to obtain approximate random normally distributed numbers very cheaply for the low precision path generation. In \Cref{sec:error_model} and \Cref{sec:costModel} we propose an error model and a cost model (resp.) that we then use to formulate the optimisation problem that is solved to obtain the optimal bit-widths of fixed point variables in \Cref{sec:optimisation}. Finally we summarise our results and future directions in \Cref{sec:conclusion}.



% \section{Related Work}
\label{sec:related_work}

The original investigation \cite{gibson1979ecological} on the relationship between visual perception and human action defines \emph{affordance} as the opportunities for interaction with the surrounding environment. Behavioral studies on regular and cognitively impaired persons have shown evidence that perception results in both visual and motor signals in the human brain. An extended study \cite{anderson2002attentional} shows that visual attention to the spatial characteristics of the perceived objects initiates automatic motor signals for different actions. In computer vision, human affordance learning involves novel pose prediction such that the estimated pose represents a valid human action within the scene context. The task is fundamental to many problems requiring robust semantic reasoning about the environment, such as human motion synthesis \cite{wang2021scene} and scene-aware human pose generation \cite{wang2017binge, roy2016multi, zhang2022inpaint, yao2023scene}.

Earlier methods of affordance learning have explored knowledge mining \cite{zhu2014reasoning} and multimodal feature cues \cite{roy2016multi} to address the problem. In \cite{zhu2014reasoning}, the authors use a Markov Logic Network for constructing a knowledge base by extracting several object attributes from different image and metadata sources, which can perform various downstream visual inference tasks without any additional classifier, including zero-shot affordance prediction. In \cite{roy2016multi}, the authors use depth map, surface normals, and segmentation map as multimodal cues to train a multi-scale convolutional neural network (CNN) for scene-level semantic label assignment associated with specific human actions. In \cite{do2018affordancenet}, the authors design a multi-branch end-to-end CNN with two separate pathways for object detection and affordance label assignment to achieve high real-time inference throughput. Researchers \cite{chuang2018learning} have also explored socially imposed constraints for affordance learning. In \cite{chuang2018learning}, the authors propose a graph neural network (GNN) to propagate contextual scene information from egocentric views for action-object affordance reasoning.

Probabilistic modeling of scene-aware human motion generation also involves semantic reasoning of human interaction with the environment. Initial works on human motion synthesis have taken different architectural approaches, such as sequence-to-sequence models \cite{barsoum2018hp}, generative adversarial networks (GAN) \cite{barsoum2018hp, cai2018deep, yang2018pose}, graph convolutional networks (GCN) \cite{yan2019convolutional}, and variational autoencoders (VAE) \cite{guo2020action2motion}. However, these methods have mostly ignored the role of environmental semantics. Due to potential uncertainty in human motion, in a recent approach \cite{wang2021scene}, the authors address such motion synthesis with a GAN conditioned on scene attributes and motion trajectory to predict probable body pose dynamics.

One key challenge of human affordance generation in 2D scenes is the lack of large-scale datasets with rich pose annotations. In \cite{wang2017binge}, the authors compile the only public dataset of annotated human body poses in complex 2D indoor scenes by extracting frames from sitcom videos. Aiming to generate a contextually valid human affordance at a user-defined location, the authors propose sampling the scale and deformation parameters for an existing human pose template using a VAE conditioned on the localized image patches as scene context. In \cite{zhang2022inpaint}, the authors introduce a two-stage GAN architecture for achieving a similar goal by estimating the affine bounding box parameters to localize a probable human in the scene and then generating a potential body pose at that location. The method uses the input scene, corresponding depth, and segmentation maps as semantic guidance. In \cite{yao2023scene}, the authors propose a transformer-based approach with knowledge distillation for generating human affordances in 2D indoor scenes.


\section{Background}
\label{sec:background}


\subsection{Preliminaries}

{\color{red}[TODO: LLMs? in-context learning?]}

\subsection{Problem Definition}

{\color{red}[TODO: define the problem of citation intent]}

\section{Experimental Setups}
In this section, we introduce the dataset and evaluation metrics we used to assess the robustness performance of various robust training strategies.
\subsection{Datasets and Evaluation Metrics}
For our experiment, we evaluate on four widely-used question answering datasets: (1) single-hop QA, including NaturalQuestions (NQ)~\cite{kwiatkowski-etal-2019-natural} and WebQuestions~\cite{berant-etal-2013-semantic}; and (2) multi-hop QA, including TriviaQA~\cite{joshi-etal-2017-triviaqa} and HotpotQA~\cite{yang-etal-2018-hotpotqa}. All experimental results are evaluated on their development splits using the Exact Match (EM) and F1 metrics. Detailed statistics of these datasets are listed in Appendix~\ref{sec:dataset_stats} Table~\ref{tab:dataset_stats}.

% \subsection{RAG Pipeline}
% Our experiment implements a standard two-stage RAG framework. The pipeline consists of a retrieval phase and a generation phase. For the retrieval component, we leverage Contriever~\cite{DBLP:journals/tmlr/IzacardCHRBJG22}, an advanced BERT-based dense retriever that employs unsupervised contrastive learning for document representation.

% The retrieval process operates on a Wikipedia-based knowledge corpus, specifically utilizing the Wikipedia-2018 dataset. We preprocess this corpus by segmenting Wikipedia articles into non-overlapping passages, each containing 100 words, while preserving the integrity of potentially misleading passages. To enable efficient retrieval, we generate document embeddings using Contriever and index them using FAISS\footnote{\url{https://github.com/facebookresearch/faiss}}, a powerful similarity search library. During both training and development phases, Contriever retrieves the top-20 most relevant documents for each query from our indexed Wikipedia corpus.

% For the downstream question-answering tasks, we select the top-5 retrieved documents and concatenate them with the test query to form the input context. The model then processes this enriched context to generate appropriate answers. The generator prompt we used is shown in Appendix~\ref{sec:prompt} Table~\ref{tab:generator_prompt}.

\subsection{RAG Pipeline}  
We implement a standard two-stage RAG framework with retrieval and generation phases. For retrieval, we use Contriever~\cite{DBLP:journals/tmlr/IzacardCHRBJG22}, a BERT-based dense retriever trained with unsupervised contrastive learning. Our knowledge corpus is Wikipedia-2018, preprocessed into 100-word non-overlapping passages. We encode these passages using Contriever and index them with FAISS\footnote{\url{https://github.com/facebookresearch/faiss}} for efficient retrieval. During training and development, the top-20 relevant documents are retrieved for each query. For question-answering, the top-5 retrieved documents are concatenated with the query as input for answer generation. The generator prompt is detailed in Appendix~\ref{sec:prompt} Table~\ref{tab:generator_prompt}.

% \subsection{Foundation Models}
% In our experiments, we employ four state-of-the-art foundation models: \texttt{Llama-2-7b-chat-hf}, \texttt{Llama-3-8B-Instruct}, \texttt{Qwen1.5-7B-Chat}, and \texttt{Qwen2.5-7B-Instruct}. These models represent different generations of LLMs from two prominent model families - Llama and Qwen, each with comparable model sizes around 7-8 billion parameters.

% \subsection{Robust RAG Training Scenario}

% To comprehensively evaluate the robustness of RAG systems under different document selection strategies, we design seven training scenarios across three categories.

% For basic QA capability evaluation, we directly use the \textbf{Base Model} in a zero-shot setting, where the foundation model performs inference without any RAG-specific training.

% For scenarios with relevant documents, we explore four different strategies:
% \begin{itemize}[leftmargin=0.5cm, itemindent=0cm]
%     \item \textbf{RALM}: This approach incorporates instruction tuning by prepending golden retrieval text to the context during model fine-tuning. It is worth noting that queries without golden documents in their top-20 retrieved documents are excluded from training, resulting in a smaller training set compared to other methods.
%     \item \textbf{RetRobust}: Based on \newcite{DBLP:conf/iclr/YoranWRB24}, this method enhances model robustness by exposing it to diverse retrieval qualities during training. For each query, it randomly selects between top-ranked, low-ranked, or random passages with equal probability.
%     \item \textbf{Top-1 Document}: This scenario utilizes the document with the highest similarity score from the retrieved document pool. Note that this document may not necessarily contain the correct answer, reflecting real-world retrieval scenarios where the most similar document might not always be the most helpful one.
%     \item \textbf{Golden Document}: This strategy primarily selects the document that both contains the correct answer and has the highest relevance score among the top-20 retrieved documents. In cases where no document containing the correct answer exists in the top-20 retrieved documents, we fall back to using the top-1 document to maintain consistent training sample sizes across different methods.
% \end{itemize}

% \begin{table*}[t]
\centering
\resizebox{\linewidth}{!}{
\begin{tabular}{clcccccccccc}
\toprule
\multirow{2}{*}{\textbf{Model}} & \multirow{2}{*}{\textbf{RAG Scenario}} & \multicolumn{2}{c}{\textbf{HotpotQA}} & \multicolumn{2}{c}{\textbf{NQ}} & \multicolumn{2}{c}{\textbf{WebQuestions}} & \multicolumn{2}{c}{\textbf{TriviaQA}} & \multicolumn{2}{c}{\textbf{AVERAGE}}\\
\cline{3-4}\cline{5-6}\cline{7-8}\cline{9-10}\cline{11-12}
 & &  \textbf{EM} & \textbf{F1} & \textbf{EM} & \textbf{F1} & \textbf{EM} & \textbf{F1} & \textbf{EM} & \textbf{F1} & \textbf{EM} & \textbf{F1} \\
\midrule
\multirow{10}{*}{\rotatebox{90}{\textit{Llama-2-7b-chat-hf}}} & Base Model &  3.30  &  12.34 &  1.21 &  10.61  &  0.00  &  13.08  &  4.32  &  20.27 & 2.21 & 14.08  \\
 & RALM &  26.21   & 36.42  &  32.17 &  42.68 &  33.81  &  45.85  &  50.28  &  60.17 & 35.62 & 46.28  \\
 & RetRobust &  31.29  &  43.65  &  37.71  & 49.49  &  36.33  &  47.98  &  57.61  & 67.52 & 40.74 & 52.16  \\
 & Top-1 Doc & 31.76 & 43.95 & 40.20 & 51.89  &  41.73  &  52.76  &  52.93  & 65.41 & 41.66 & 53.50  \\
 & Golden Doc & 30.67 & 42.78 & 36.50 & 47.77  &  39.93  &  52.11  &  50.25  & 63.28 & 39.34 & 51.49  \\
 \cdashline{2-12}
 & Random Doc & 30.94 & 43.11 & 38.16 & 49.78  &  42.45  &  53.97  &  52.72 &  65.52 & 41.07 & 53.10  \\
 & Irrelevant Doc & 31.01 & 42.98 & 37.08 & 48.93  &  39.21   &  50.79  &  51.97   &  64.70 & 39.82 & 51.85 \\
 \cdashline{2-12}
 & RAAT & 31.32 & 43.24 & 42.91 & 53.19  &  36.69  &  48.82 &  51.65 &  58.71 & 40.64 & 50.99  \\
 & IRM & 34.38 & 47.11 & 40.96 & 53.07  &  53.96  &  61.62 &  57.58 &  69.08 & 46.72 & 57.72  \\
 \cmidrule{2-12}
% & \textcolor{mycolor}{$\Delta$ (Random $\rightarrow$ Best)}  &  \textcolor{mycolor}{2.65\%}  & \textcolor{mycolor}{1.95\%}  &  \textcolor{mycolor}{5.35\%}  &  \textcolor{mycolor}{4.24\%}  &  \textcolor{mycolor}{0.85\%}   &  \textcolor{mycolor}{0.30\%}  &  \textcolor{mycolor}{9.28\%}  &  \textcolor{mycolor}{3.05\%} & \textcolor{mycolor}{4.53\%} & \textcolor{mycolor}{2.38\%}  \\
& \textcolor{mycolor}{$\Delta$ (Worst $\rightarrow$ Best)}  &  \textcolor{mycolor}{16.19\%}  & \textcolor{mycolor}{12.89\%}  &  \textcolor{mycolor}{21.87\%}  &  \textcolor{mycolor}{13.19\%}  &  \textcolor{mycolor}{48.53\%}   &  \textcolor{mycolor}{28.43\%}  &  \textcolor{mycolor}{13.55\%}  &  \textcolor{mycolor}{17.66\%} & \textcolor{mycolor}{15.95\%} & \textcolor{mycolor}{13.20\%}  \\
\midrule
\multirow{10}{*}{\rotatebox{90}{\textit{Llama-3-8B-Instruct}}} & Base Model & 23.31 & 32.60 & 30.04 & 41.59 & 26.98 & 43.25 & 58.80 & 66.45 & 34.78 & 45.97 \\
 & RALM &  27.64 &  38.10  &  35.19  &  46.10  &   47.84  &  56.98   &  54.75 &  63.27  & 41.36 & 51.11 \\
 & RetRobust &  36.06 &  48.99  &  43.28  &  55.04  &  52.88  &  62.10  &  59.06  & 67.77 & 47.82 & 58.48  \\
 & Top-1 Doc & 36.72 & 49.30 & 44.38 & 56.20  &  54.68  &  62.26 &  60.80  & 68.31 & 49.15 & 59.02  \\
 & Golden Doc & 35.52 & 48.31 & 41.35 & 53.13  &  48.92  &  58.41  &  58.26 & 66.99 &  46.01 & 56.71 \\
 \cdashline{2-12}
 & Random Doc & 35.98 & 49.05 & 43.37 & 55.43  &  53.24  &   62.55 &  60.62  & 68.64 & 48.30 & 58.92  \\
 & Irrelevant Doc & 35.31 & 47.92 & 42.45 & 54.41  &  46.76  &  57.67  &  58.97  & 66.57 & 45.87 & 56.64 \\
 \cdashline{2-12}
 & RAAT & 32.20 & 43.81 & 42.34 & 53.31  &  48.28  &  58.17 &  54.41 &  62.45 & 44.31 & 54.44  \\
 & IRM & 35.19 & 48.08 & 41.13 & 53.14  &  53.96  &  61.64  &  57.15 &  69.13 & 46.86 & 58.00  \\
\cmidrule{2-12}
% & \textcolor{mycolor}{$\Delta$ (Random $\rightarrow$ Best)}  &  \textcolor{mycolor}{2.06\%}  & \textcolor{mycolor}{0.51\%}  &  \textcolor{mycolor}{2.33\%}  &  \textcolor{mycolor}{1.39\%}  &  \textcolor{mycolor}{2.70\%}   &  \textcolor{mycolor}{0.00\%}  &  \textcolor{mycolor}{0.30\%}  &  \textcolor{mycolor}{0.00\%}  & \textcolor{mycolor}{1.85\%} & \textcolor{mycolor}{0.47\%} \\
& \textcolor{mycolor}{$\Delta$ (Worst $\rightarrow$ Best)}  &  \textcolor{mycolor}{14.04\%}  & \textcolor{mycolor}{12.53\%}  &  \textcolor{mycolor}{7.90\%}  &  \textcolor{mycolor}{5.76\%}  &  \textcolor{mycolor}{16.94\%}   &  \textcolor{mycolor}{8.46\%}  &  \textcolor{mycolor}{11.74\%}  &  \textcolor{mycolor}{10.70\%}  & \textcolor{mycolor}{10.92\%} & \textcolor{mycolor}{8.42\%} \\
\bottomrule
\end{tabular}}
\caption{Performance comparison of different LLMs (\texttt{Llama-2-7b-chat-hf} and \texttt{Llama-3-8B-Instruct}) across different robust RAG scenarios on four datasets (HotpotQA, NQ, WebQuestions, and TriviaQA). The row \textcolor{mycolor}{$\Delta$ (Worst $\rightarrow$ Best)} indicates the performance gain achieved by the best method compared to the worst strategy among these training strategies, representing the benefit of sophisticated robust training methods.}
\label{tab:main_result_llama}
\vspace{-0.4cm}
\end{table*}


% To evaluate the model's robustness against potentially harmful or irrelevant information, we design two adversarial scenarios:
% \begin{itemize}[leftmargin=0.5cm, itemindent=0cm]
%     \item \textbf{Random Document}: This approach randomly selects one document from the retrieved documents, simulating unpredictable retrieval quality in real-world scenarios.
%     \item \textbf{Irrelevant Document}: This scenario selects an irrelevant passage by randomly choosing from retrieval contents of other queries, ensuring no relevance to the current query.
% \end{itemize}

% These scenarios are designed to evaluate model performance across three distinct aspects: (1) the basic QA capability without retrieval augmentation, (2) the model's ability to leverage relevant documents with varying degrees of utility, and (3) the model's robustness against potentially harmful or irrelevant information. This comprehensive evaluation framework allows us to assess both the effectiveness and robustness of RAG systems under different operational conditions.

\subsection{Robust RAG Training Setups}
% We evaluate two popular robust training settings:
For our experiments, we evaluate two popular robust training settings:
\subsubsection{Document Selection Strategies}
% To assess RAG robustness under various document selection strategies, we design seven training scenarios across three categories.
We explore several document selection training strategies across three categories.

For basic QA capability evaluation, we use the \textbf{Base Model} in a few-shot setting, where the model performs inference without RAG-specific training.

For scenarios with relevant documents, we explore four strategies:
\begin{itemize}[leftmargin=0.5cm, itemindent=0cm, itemsep=0pt]
    \item \textbf{RALM}: Fine-tunes the model by prepending golden retrieval text. Queries without golden documents in the top-20 are excluded, leading to reduced training set size.
    \item \textbf{RetRobust}: Following \newcite{DBLP:conf/iclr/YoranWRB24}, this method enhances robustness by randomly selecting top-ranked, low-ranked, or random passages during training.
    \item \textbf{Top-1 Document}: Uses the highest-scoring retrieved document, which may not contain the correct answer, reflecting real-world retrieval challenges.
    \item \textbf{Golden Document}: Selects the most relevant document containing the correct answer. If none exist in the top-20, it defaults to the top-1 document for consistency.
\end{itemize}

\begin{table*}[t]
\centering
\resizebox{\linewidth}{!}{
\begin{tabular}{clcccccccccc}
\toprule
\multirow{2}{*}{\textbf{Model}} & \multirow{2}{*}{\textbf{RAG Scenario}} & \multicolumn{2}{c}{\textbf{HotpotQA}} & \multicolumn{2}{c}{\textbf{NQ}} & \multicolumn{2}{c}{\textbf{WebQuestions}} & \multicolumn{2}{c}{\textbf{TriviaQA}} & \multicolumn{2}{c}{\textbf{AVERAGE}}\\
\cline{3-4}\cline{5-6}\cline{7-8}\cline{9-10}\cline{11-12}
 & &  \textbf{EM} & \textbf{F1} & \textbf{EM} & \textbf{F1} & \textbf{EM} & \textbf{F1} & \textbf{EM} & \textbf{F1} & \textbf{EM} & \textbf{F1} \\
\midrule
\multirow{10}{*}{\rotatebox{90}{\textit{Llama-2-7b-chat-hf}}} & Base Model &  3.30  &  12.34 &  1.21 &  10.61  &  0.00  &  13.08  &  4.32  &  20.27 & 2.21 & 14.08  \\
 & RALM &  26.21   & 36.42  &  32.17 &  42.68 &  33.81  &  45.85  &  50.28  &  60.17 & 35.62 & 46.28  \\
 & RetRobust &  31.29  &  43.65  &  37.71  & 49.49  &  36.33  &  47.98  &  57.61  & 67.52 & 40.74 & 52.16  \\
 & Top-1 Doc & 31.76 & 43.95 & 40.20 & 51.89  &  41.73  &  52.76  &  52.93  & 65.41 & 41.66 & 53.50  \\
 & Golden Doc & 30.67 & 42.78 & 36.50 & 47.77  &  39.93  &  52.11  &  50.25  & 63.28 & 39.34 & 51.49  \\
 \cdashline{2-12}
 & Random Doc & 30.94 & 43.11 & 38.16 & 49.78  &  42.45  &  53.97  &  52.72 &  65.52 & 41.07 & 53.10  \\
 & Irrelevant Doc & 31.01 & 42.98 & 37.08 & 48.93  &  39.21   &  50.79  &  51.97   &  64.70 & 39.82 & 51.85 \\
 \cdashline{2-12}
 & RAAT & 31.32 & 43.24 & 42.91 & 53.19  &  36.69  &  48.82 &  51.65 &  58.71 & 40.64 & 50.99  \\
 & IRM & 34.38 & 47.11 & 40.96 & 53.07  &  53.96  &  61.62 &  57.58 &  69.08 & 46.72 & 57.72  \\
 \cmidrule{2-12}
% & \textcolor{mycolor}{$\Delta$ (Random $\rightarrow$ Best)}  &  \textcolor{mycolor}{2.65\%}  & \textcolor{mycolor}{1.95\%}  &  \textcolor{mycolor}{5.35\%}  &  \textcolor{mycolor}{4.24\%}  &  \textcolor{mycolor}{0.85\%}   &  \textcolor{mycolor}{0.30\%}  &  \textcolor{mycolor}{9.28\%}  &  \textcolor{mycolor}{3.05\%} & \textcolor{mycolor}{4.53\%} & \textcolor{mycolor}{2.38\%}  \\
& \textcolor{mycolor}{$\Delta$ (Worst $\rightarrow$ Best)}  &  \textcolor{mycolor}{16.19\%}  & \textcolor{mycolor}{12.89\%}  &  \textcolor{mycolor}{21.87\%}  &  \textcolor{mycolor}{13.19\%}  &  \textcolor{mycolor}{48.53\%}   &  \textcolor{mycolor}{28.43\%}  &  \textcolor{mycolor}{13.55\%}  &  \textcolor{mycolor}{17.66\%} & \textcolor{mycolor}{15.95\%} & \textcolor{mycolor}{13.20\%}  \\
\midrule
\multirow{10}{*}{\rotatebox{90}{\textit{Llama-3-8B-Instruct}}} & Base Model & 23.31 & 32.60 & 30.04 & 41.59 & 26.98 & 43.25 & 58.80 & 66.45 & 34.78 & 45.97 \\
 & RALM &  27.64 &  38.10  &  35.19  &  46.10  &   47.84  &  56.98   &  54.75 &  63.27  & 41.36 & 51.11 \\
 & RetRobust &  36.06 &  48.99  &  43.28  &  55.04  &  52.88  &  62.10  &  59.06  & 67.77 & 47.82 & 58.48  \\
 & Top-1 Doc & 36.72 & 49.30 & 44.38 & 56.20  &  54.68  &  62.26 &  60.80  & 68.31 & 49.15 & 59.02  \\
 & Golden Doc & 35.52 & 48.31 & 41.35 & 53.13  &  48.92  &  58.41  &  58.26 & 66.99 &  46.01 & 56.71 \\
 \cdashline{2-12}
 & Random Doc & 35.98 & 49.05 & 43.37 & 55.43  &  53.24  &   62.55 &  60.62  & 68.64 & 48.30 & 58.92  \\
 & Irrelevant Doc & 35.31 & 47.92 & 42.45 & 54.41  &  46.76  &  57.67  &  58.97  & 66.57 & 45.87 & 56.64 \\
 \cdashline{2-12}
 & RAAT & 32.20 & 43.81 & 42.34 & 53.31  &  48.28  &  58.17 &  54.41 &  62.45 & 44.31 & 54.44  \\
 & IRM & 35.19 & 48.08 & 41.13 & 53.14  &  53.96  &  61.64  &  57.15 &  69.13 & 46.86 & 58.00  \\
\cmidrule{2-12}
% & \textcolor{mycolor}{$\Delta$ (Random $\rightarrow$ Best)}  &  \textcolor{mycolor}{2.06\%}  & \textcolor{mycolor}{0.51\%}  &  \textcolor{mycolor}{2.33\%}  &  \textcolor{mycolor}{1.39\%}  &  \textcolor{mycolor}{2.70\%}   &  \textcolor{mycolor}{0.00\%}  &  \textcolor{mycolor}{0.30\%}  &  \textcolor{mycolor}{0.00\%}  & \textcolor{mycolor}{1.85\%} & \textcolor{mycolor}{0.47\%} \\
& \textcolor{mycolor}{$\Delta$ (Worst $\rightarrow$ Best)}  &  \textcolor{mycolor}{14.04\%}  & \textcolor{mycolor}{12.53\%}  &  \textcolor{mycolor}{7.90\%}  &  \textcolor{mycolor}{5.76\%}  &  \textcolor{mycolor}{16.94\%}   &  \textcolor{mycolor}{8.46\%}  &  \textcolor{mycolor}{11.74\%}  &  \textcolor{mycolor}{10.70\%}  & \textcolor{mycolor}{10.92\%} & \textcolor{mycolor}{8.42\%} \\
\bottomrule
\end{tabular}}
\caption{Performance comparison of different LLMs (\texttt{Llama-2-7b-chat-hf} and \texttt{Llama-3-8B-Instruct}) across different robust RAG scenarios on four datasets (HotpotQA, NQ, WebQuestions, and TriviaQA). The row \textcolor{mycolor}{$\Delta$ (Worst $\rightarrow$ Best)} indicates the performance gain achieved by the best method compared to the worst strategy among these training strategies, representing the benefit of sophisticated robust training methods.}
\label{tab:main_result_llama}
\vspace{-0.4cm}
\end{table*}

To assess robustness against irrelevant information, we introduce two adversarial scenarios:
\begin{itemize}[leftmargin=0.5cm, itemindent=0cm, itemsep=0pt]
    \item \textbf{Random Document}: Randomly selects a document from retrieved results, simulating unpredictable retrieval quality.
    \item \textbf{Irrelevant Document}: Chooses a passage from another query’s retrieval results, ensuring no relevance to the current query.
\end{itemize}

% These scenarios evaluate: (1) basic QA capability without retrieval, (2) the model’s ability to leverage relevant documents, and (3) robustness against irrelevant or misleading information. This framework provides a comprehensive assessment of RAG performance under diverse conditions.

\subsubsection{Adversarial Loss Design}  
We assess two popular adversarial loss strategies:
\begin{itemize}[leftmargin=0.5cm, itemindent=0cm, itemsep=0pt]
\item \textbf{RAAT}. The regularization term in RAAT reduces the performance gap between the best and worst retrieval cases. By penalizing excessive performance disparity, RAAT ensures that the model remains stable even under challenging retrieval conditions, leading to improved robustness and generalization.
\item \textbf{IRM}. The regularization in IRM minimizes the variance in performance across different retrieval environments. By enforcing consistency, IRM mitigates sensitivity to distribution shifts, ensuring that the model performs reliably across diverse retrieval scenarios.
\end{itemize}


\section{Does Sophisticated Robust Training Still Matter in Powerful Models?}
To investigate whether sophisticated document selection strategies and adversarial loss designs are still essential for robust RAG performance as LLMs continue to evolve, we conduct comprehensive experiments across multiple LMs and datasets.

% 需要解释为何golden的表现不好,为何top-1这么好

\subsection{Do Sophisticated Document Selection Strategies Matter?}
% We first conduct extensive experiments to examine whether complex document selection strategies contribute to the robustness of LLMs.
We conduct experiments to analyze the effectiveness of complex document selection strategies under Llama model families (\texttt{Llama-2-7b-chat-hf} and \texttt{Llama-3-8B-Instruct}) in Table~\ref{tab:main_result_llama}, and Qwen model families in Appendix Table~\ref{tab:main_result_qwen}.
\paragraph{Training with sophisticated documents enhances LM robustness for weak models}
The experimental results presented in Tables \ref{tab:main_result_llama} and \ref{tab:main_result_qwen} provide compelling evidence that robust training significantly improves model resilience when processing noisy documents. While base models exhibit substantial performance degradation when encountering noisy documents during inference, models that undergo robust training maintain consistent and superior QA performance across various document selection strategies. 
A notable example is the \texttt{Llama-2-7b-chat-hf} model's performance (Table~\ref{tab:main_result_llama}) on the HotpotQA dataset, 
% where training with randomly selected documents yields an improved EM score of 30.67, demonstrating enhanced resilience to document noise.
where training with golden documents improves the EM score from 3.3 (Base Model) to 30.67 (Golden Doc), indicating increased resilience to document noise.
This pattern of improvement is consistently observed across both Llama (Table~\ref{tab:main_result_llama}) and Qwen (Table~\ref{tab:main_result_qwen}) model families, strongly indicating that robust training effectively mitigates the base models' inherent vulnerability to noisy documents.

\paragraph{Training with random documents shows surprising effectiveness}
We also notice that training with randomly selected documents exhibits remarkable effectiveness across all experimental configurations. Quantitative analysis shows that with \texttt{Llama-2-7b-chat-hf}, this approach achieves superior performance on WebQuestions compared to more sophisticated strategies. Similar observations emerge from experiments with \texttt{Qwen1.5-7B-Chat}, where random document selection achieves 46.04 EM on WebQuestions, approaching the optimal performance of 47.12 EM achieved by RetRobust. The consistency of these results across distinct model architectures suggests that the efficacy of random document selection represents an inherent characteristic of contemporary RAG systems.

% \begin{figure*}[ht]
%     \centering
%     \begin{subfigure}[b]{0.48\textwidth}
%         \centering
%         \includegraphics[width=\textwidth]{fig/f1_hotpot.pdf}
%     \end{subfigure}
%     \hfill
%     \begin{subfigure}[b]{0.48\textwidth}
%         \centering
%         \includegraphics[width=\textwidth]{fig/f1_nq.pdf}
%     \end{subfigure}
%     \caption{Comparison of F1 Scores for training with random doc and golden doc across models with varying parameter sizes from 0.5B to 70B. The left figure shows results on the HotpotQA dataset, while the right figure presents results on the NQ dataset. We provide detailed results in Appendix Table~\ref{tab:vary_parameter_analysis}.}
%     \label{fig:parameter_scale}
%     \vspace{-0.3cm}
% \end{figure*}

\paragraph{Diminishing returns of sophisticated document selections as models evolve}  
Experimental results indicate that the performance gains from sophisticated document selection strategies diminish as models evolve. For instance, comparing \texttt{Llama-2-7b-chat-hf} with \texttt{Llama-3-8B-Instruct}, the improvement in performance due to advanced document selection strategies decreases significantly, with the $\Delta$ (Worst $\rightarrow$ Best) metric for NQ dropping from 21.87\% to 7.90\% EM. A similar trend is observed when comparing \texttt{Qwen1.5-7B-Chat} to \texttt{Qwen2.5-7B-Instruct}, where the performance improvement from sophisticated document selection also shows a noticeable reduction. These results suggest that as models become more advanced, their ability to process and utilize information improves independently of complex document selection strategies, leading to diminished returns from such methods.

% \paragraph{Diminishing Returns of Sophisticated Document Selection Strategies}
% Quantitative analysis demonstrates minimal performance improvements from sophisticated document selection strategies across all experimental conditions. This observation is substantiated by the $\Delta$ (Random $\rightarrow$ Best) metric, which exhibits consistently low values across datasets and model variants. Specifically, \texttt{Llama-2-7b-chat-hf} shows a maximum improvement of only 9.28\% EM on TriviaQA, while \texttt{Qwen1.5-7B-Chat} demonstrates an even smaller margin of 2.95\% EM on HotpotQA. These limited improvements raise fundamental questions about the utility of complex document selection mechanisms.

% These experimental results reveal a particularly significant trend in the diminishing impact of sophisticated selection strategies as models evolve. Comparative analysis between \texttt{Llama-2-7b-chat-hf} and \texttt{Llama-3-8B-Instruct} demonstrates a systematic decrease in $\Delta$ values across all datasets, with TriviaQA's performance differential reducing substantially from 9.28\% to 1.49\% EM. Similar observations in the progression from \texttt{Qwen1.5-7B-Chat} to \texttt{Qwen2.5-7B-Instruct} indicate that advanced language models develop enhanced capabilities for information processing, independent of document selection methodology.

% While retrieval-based selection and RetRobust occasionally achieve optimal performance metrics, their marginal improvements over random selection do not justify the additional computational overhead. This observation is further validated by the system's robust performance even under intentionally irrelevant document selection, suggesting that sophisticated document selection strategies may not be as critical as previously assumed for RAG system performance.

% 分析使用更好的loss设计是否会带来提升
\subsection{Do Adversarial Loss Functions Matter?}
% \begin{table*}[t]
\small
\centering
\resizebox{\linewidth}{!}{
\begin{tabular}{llcccccccccc}
\toprule
\multirow{2}{*}{\textbf{Model}} & \multirow{2}{*}{\textbf{RAG Scenario}} & \multicolumn{2}{c}{\textbf{HotpotQA}} & \multicolumn{2}{c}{\textbf{NQ}} & \multicolumn{2}{c}{\textbf{WebQuestion}} & \multicolumn{2}{c}{\textbf{TriviaQA}} & \multicolumn{2}{c}{\textbf{AVERAGE}}\\
\cline{3-4}\cline{5-6}\cline{7-8}\cline{9-10}\cline{11-12}
 & &  \textbf{EM} & \textbf{F1} & \textbf{EM} & \textbf{F1} & \textbf{EM} & \textbf{F1} & \textbf{EM} & \textbf{F1} & \textbf{EM} & \textbf{F1} \\
\midrule
\multirow{6}{*}{\textit{Llama-2-7b-chat-hf}} & Base Model &  3.30  &  12.34 &  1.21 &  10.61  &  0.00  &  13.08  &  4.32  &  20.27 & 2.21 & 14.08  \\
 & Top-1 Doc & 31.76 & 43.95 & 40.20 & 51.89  &  41.73  &  52.76  &  52.93  & 65.41 & 41.66 & 53.50  \\
 & Golden Doc & 30.67 & 42.78 & 36.50 & 47.77  &  39.93  &  52.11  &  50.25  & 63.28 & 39.34 & 51.49  \\
 & Random Doc & 30.94 & 43.11 & 38.16 & 49.78  &  42.45  &  53.97  &  52.72 &  65.52 & 41.07 & 53.10  \\
 \cdashline{2-12}
 & RAAT & 31.32 & 43.24 & 42.91 & 53.19  &  36.69  &  48.82 &  51.65 &  58.71 & 40.64 & 50.99  \\
 & IRM & 34.38 & 47.11 & 40.96 & 53.07  &  53.96  &  61.62 &  57.58 &  69.08 & 46.72 & 57.72  \\
\midrule
\multirow{6}{*}{\textit{Llama-3-8B-Instruct}} & Base Model & 23.31 & 32.60 & 30.04 & 41.59 & 26.98 & 43.25 & 58.80 & 66.45 & 34.78 & 45.97 \\
 & Top-1 Doc & 36.72 & 49.30 & 44.38 & 56.20  &  54.68  &  62.26 &  60.80  & 68.31 & 49.15 & 59.02  \\
 & Golden Doc & 35.52 & 48.31 & 41.35 & 53.13  &  48.92  &  58.41  &  58.26 & 66.99 &  46.01 & 56.71 \\
 & Random Doc & 35.98 & 49.05 & 43.37 & 55.43  &  53.24  &   62.55 &  60.62  & 68.64 & 48.30 & 58.92  \\
 \cdashline{2-12}
 & RAAT & 32.20 & 43.81 & 42.34 & 53.31  &  48.28  &  58.17 &  54.41 &  62.45 & 44.31 & 54.44  \\
 & IRM & 35.19 & 48.08 & 41.13 & 53.14  &  53.96  &  61.64  &  57.15 &  69.13 & 46.86 & 58.00  \\
\bottomrule
\end{tabular}}
\caption{Performance comparison of different document selection and loss design strategies (RAAT~\cite{fang-etal-2024-enhancing} and IRM~\cite{DBLP:journals/tmlr/YoshidaN24}) for various models (\texttt{Llama-2-7b-chat-hf} and \texttt{Llama-3-8B-Instruct}) across four datasets.}
\label{tab:appendix_loss_analysis}
\end{table*}

To investigate whether the design of complex adversarial loss functions contributes to model performance, in Table~\ref{tab:main_result_llama} and~\ref{tab:main_result_qwen}, we also analyze the robustness of various adversarial loss designs.

\paragraph{Adversarial loss significantly enhances performance for weaker models}  
For the weaker model (\texttt{Llama-2-7b-chat-hf}), incorporating adversarial loss functions such as RAAT and IRM leads to a substantial improvement in performance compared to the base model or alternative document selection strategies. Specifically, while the base model achieves an average EM / F1 of only 2.21 / 14.08, applying adversarial loss functions boosts the performance to 40.64 / 50.99 for RAAT and 46.72 / 57.72 for IRM. This highlights the effectiveness of adversarial loss in improving model robustness to noisy documents, significantly enhancing both robustness and downstream inference performance. Notably, in some cases, RAAT and IRM outperform traditional document selection strategies (e.g., top-1 Doc and golden doc), demonstrating their value in scenarios where the model needs stronger guidance to handle noisy retrievals.

\begin{figure}[t]
    \centering
    \includegraphics[width=\columnwidth]{fig/f1_hotpot.pdf}
    \caption{Comparison of F1 Scores on the HotpotQA dataset for training with Random Doc and Golden Doc across models with varying parameter sizes from 0.5B to 70B. We provide detailed results in Appendix Table~\ref{tab:vary_parameter_analysis}.}
    \label{fig:parameter_scale}
    \vspace{-0.5cm}
\end{figure}

\paragraph{Adversarial loss exhibits diminishing returns for stronger models}  
For the stronger models, such as \texttt{Llama-3-8B-Instruct} and \texttt{Qwen2.5-7B-Instruct}, the benefits of adversarial loss functions are less pronounced. The basic \texttt{Llama-3-8B-Instruct} already achieves an average EM / F1 of 34.78 / 45.97, and the introduction of adversarial losses (RAAT and IRM) results in only modest improvements, with average EM / F1 scores of 44.31 / 54.44 and 46.86 / 58.00, respectively. Similarly, \texttt{Qwen2.5-7B-Instruct} shows similar trends, with marginal gains from adversarial loss functions. These improvements are comparable to or even slightly worse than the performance achieved by random document selection (48.30 / 58.92) or top-1 document strategies (49.15 / 59.02). This indicates that the models’ inherent robustness reduces the impact of adversarial losses, and in some cases, may even hinder performance. We hypothesize that when the model’s internal robustness is already well-developed, additional constraints from adversarial losses may interfere with its ability to optimize on clean and relevant inputs.

% \paragraph{Conclusion: the necessity of adversarial loss.}
Based on these findings, we conclude that both adversarial loss functions and document selection strategies are more beneficial for weaker models. For smaller models, these techniques significantly improve robustness by mitigating the impact of noisy documents. For stronger models with inherently robust performance, the advantages of complex loss designs and sophisticated document selection diminish, suggesting that simpler strategies like random document selection may be sufficient. This underscores the importance of tailoring training strategies to the model’s inherent capabilities.


% 尝试更多种的噪音文档,例如反事实的,bad的,还有ISN,且可以和loss结合起来

% 加入4.3的讨论,表明random和golden的关系的讨论,证明为什么

% gold doc(按理说是sft中最好的选择,在rag的设置下不如random,所以我们在第五章接着去分析原因)
% 4.3讨论扩大model scale,绘制散点图,引入第五章分析random>golden的异常的现象

\subsection{Do Training Strategies Matter Across Model Scales?}
% Through analyzing Tables~\ref{tab:main_result_llama} and~\ref{tab:main_result_qwen}, we observed a counter-intuitive phenomenon: models trained with random documents outperformed those trained with golden documents. While golden documents containing ground truth answers typically yield optimal results in standard supervised fine-tuning (SFT), this conventional wisdom was challenged in the robust RAG task, particularly with the 7B and 8B models. To investigate whether this phenomenon is specific to models in the 7B-8B parameter range or persists across different scales, we conduct comprehensiv|e experiments using models ranging from 0.5B to 70B parameters.
Through analyzing Tables~\ref{tab:main_result_llama} and~\ref{tab:main_result_qwen}, we observe a counter-intuitive phenomenon: models trained with random documents outperformed those with golden documents, despite the latter containing ground truth answers typically yielding optimal results in standard SFT. To investigate whether this phenomenon extends beyond 7B-8B models, we conduct experiments across model scales from 0.5B to 70B parameters. 

The results (detailed in Appendix Table~\ref{tab:vary_parameter_analysis}) in Figure~\ref{fig:parameter_scale} demonstrate that for smaller models ($\le$3B parameters), training with golden documents leads to superior performance. This suggests that smaller models, limited by their inherent capabilities, benefit more from high-quality golden documents containing direct answers. However, as model size increases, we observe that training with random documents becomes more effective. This shift can be attributed to larger models' enhanced question-answering abilities and improved robustness. These models can better generalize to downstream tasks even when trained on random documents, which may contain noisier or less structured information. This finding indicates that sophisticated document selection strategies become less crucial as model size increases, revealing an important scaling property in model training.


% This unexpected observation raises a critical question: Is this phenomenon specific to models in the 7B-8B parameter range, or does it persist across different model scales? To investigate this, we need to examine whether similar patterns emerge in both smaller models (0.5B or 1.5B) and larger models (32B-70B).

% In this section, we conduct comprehensive experiments across a wide spectrum of model sizes, ranging from 0.5B to 70B parameters, to systematically investigate this counter-intuitive phenomenon.

% 同一个模型的golden和random用同一个颜色
% In order to investigate whether training with random documents can achieve better performance compared to golden documents on larger models, we conduct experiments using models with parameter sizes ranging from 0.5B to 70B.  

% The results depicted in Figure~\ref{fig:parameter_scale} reveal that when the model size is relatively small (e.g., less than or equal to 3B), using golden documents leads to superior performance. This suggests that smaller models possess limited inherent capabilities and thus benefit significantly from high-quality golden documents that contain answers, which aid the model in accurately identifying the correct responses. Conversely, as the model size increases, we observe a shift in performance dynamics where training with random documents begins to yield better results. This improvement can be attributed to the substantial enhancement in the model's intrinsic question-answering abilities, coupled with increased robustness and generalization capacity. Larger models are capable of leveraging their advanced capabilities to generalize effectively to downstream tasks, even when trained on random documents, which may include noisy or less structured data. This finding underscores the diminishing importance of complex document selection strategies as model size grows, highlighting the evolving nature of model training requirements with scale.







\section{Why Sophisticated Training No Longer Matters in Powerful Models?}
In this section, we conduct comprehensive experiments to delve into the reasons why sophisticated robust training strategies may no longer be crucial in powerful models.


\begin{figure}[t]
    \centering
    \includegraphics[width=\columnwidth]{fig/confidence.pdf}
    \caption{Confidence scores for correct and wrong answers on HotpotQA dataset, comparing Llama2 and Llama3 models across various robust training methods.}
    \label{fig:confidence}
    \vspace{-0.5cm}
\end{figure}


\subsection{Powerful Models Enable Natural Calibration}
% To investigate confidence calibration of models with different capacities, we take HotpotQA dataset as an example to compare the confidence between Llama2 (\texttt{Llama-2-7b-chat-hf}) and Llama3 (\texttt{Llama-3-8B-Instruct}) models. 
% To investigate confidence calibration of models with different capacities, 
To understand whether powerful models inherently possess the ability to distinguish reliable from unreliable answers, we take HotpotQA dataset as an example to examine the confidence calibration capabilities for Llama2 (\texttt{Llama-2-7b-chat-hf}) and Llama3 (\texttt{Llama-3-8B-Instruct}) models. 
Here, confidence is the mean of token-wise probabilities in the model's generated answer, providing a measure of the model's certainty in its predictions~\cite{DBLP:journals/corr/abs-2307-03987, DBLP:conf/iclr/XiongHLLFHH24}.
Figure~\ref{fig:confidence} reveals striking differences in their calibration patterns. In the base model, Llama2 shows poor natural calibration levels, where confidence scores for incorrect answers (95.8) abnormally exceed those for correct ones (93.8). In contrast, Llama3 demonstrates inherently better calibration, maintaining higher confidence for correct answers (97.5) than incorrect ones (91.3) without any specialized complex training.

While robust training methods (Golden Doc, Random Doc, and IRM) can effectively calibrate confidence scores and improve the gap between correct and incorrect answers for Llama2 from -2 to 12 with IRM, the marginal benefits of these complex training strategies diminish as model architectures advance.
For Llama3, which already achieves a 6.2 confidence gap naturally, the improvements from these training methods become less significant. This finding strongly suggests that advances in model architecture can effectively eliminate the need for complex robustness training procedures, as newer models come with better built-in calibration capabilities.

\begin{figure}[t]
    \centering
    \begin{subfigure}[b]{0.45\textwidth}
        \centering
        \includegraphics[width=\textwidth]{fig/llama2_cross_dataset.pdf}
    \end{subfigure}
    \begin{subfigure}[b]{0.45\textwidth}
        \centering
        \includegraphics[width=\textwidth]{fig/llama3_cross_dataset.pdf}
    \end{subfigure}
    \caption{Generalization performance comparison across different strategies trained on HotpotQA (diagonal hatches bars) and evaluated on NQ, WebQuestions, TriviaQA datasets (plain bars). More results available in Appendix Table~\ref{tab:cross_dataset_eval}.}
    \label{fig:cross_dataset_eval_part}
    \vspace{-0.6cm}
\end{figure}

\subsection{Simple Training Strategies Generalize Well in Powerful Models}
We further investigate whether powerful models can maintain robust generalization across different datasets with simple training strategies.
% To investigate why sophisticated training becomes unnecessary in modern models, we examine the generalization capabilities of different training strategies. 
We fine-tune models on HotpotQA using four document selection approaches and evaluate their transfer performance on NQ, WebQuestions, and TriviaQA.

As shown in Figure~\ref{fig:cross_dataset_eval_part}, simple strategies demonstrate surprisingly strong generalization ability. Random document selection matches or even outperforms sophisticated IRM across all evaluation datasets, with performance gaps of less than 1\%. For instance, in TriviaQA, random selection (69.5 F1) slightly surpasses both golden (68.2 F1) and IRM (68.7 F1) approaches.
This trend becomes more pronounced in the powerful \texttt{Llama-3-8B-Instruct}, where the performance gap between simple and sophisticated strategies further narrows. The consistent cross-dataset performance, regardless of training strategy, indicates that model capacity, rather than training sophistication, is the key driver of generalization ability. These findings provide strong evidence that as models become more powerful, sophisticated training strategies become increasingly unnecessary.

\begin{figure}[ht]
    \centering
    \includegraphics[width=\columnwidth]{fig/case1.pdf}
    % \caption{Attention distribution heatmaps for models. Each cell ($i$, $j$) represents the average attention assigned to tokens in document $i$ by the $j$-th attention layer when generating answers. \textcolor{mygreen}{\textbf{Doc1}} (highlighted in green) contains the correct answer.}
    \caption{Attention visualization for a QA case. Each subplot shows attention distribution heatmaps across different models, where cell ($i$, $j$) represents the average attention weight from the $j$-th attention layer to document $i$. Text highlighted in \textcolor{mygreen}{green} indicates the correct answer and corresponding \textcolor{mygreen}{\textbf{Doc1}}, {blue} indicates key terms from the query, and {red} indicates incorrect model predictions. The color intensity in the heatmaps indicates attention strength.}
    \label{fig:attention}
    \vspace{-0.3cm}
\end{figure}

% \subsection{Attention Distribution Analysis}
\subsection{Powerful Models Learn Effective Attention Patterns with Simple Training}
To provide a direct understanding of why simple training can achieve good performance, we visualize attention distributions across different training strategies. Figure~\ref{fig:attention} reveals that both sophisticated robust training methods (IRM) and simple approaches (random doc, top-1) achieve similar attention patterns, with clear focus on Doc1 (containing the correct answer) in middle layers (9-16). In contrast, the base model fails to attend to the correct document, generating a wrong answer. This finding provides direct evidence that powerful models can learn optimal attention mechanisms even with simple training strategies, making sophisticated training methods unnecessary.

\begin{figure}[t]
    \centering
    \includegraphics[width=\columnwidth]{fig/mixed_doc_hotpot.pdf}
    \caption{Performance comparison with different numbers of random documents during training. Increasing the number of random documents consistently improves model performance.}
    % \caption{Impact of random document proportion on F1 Scores. The figure shows the F1 scores for HotpotQA and NQ datasets when varying the number of random documents (0 to 3) in a total of three training documents. Increasing the proportion of random documents consistently improves model performance.}
    \label{fig:mixed_doc}
    \vspace{-0.3cm}
\end{figure}


\begin{figure}[ht]
    \centering
    \includegraphics[width=\columnwidth]{fig/training_step_f1.pdf}
    \caption{Training curves comparison between random and golden document strategies using \texttt{Llama-2-7b-chat-hf} and \texttt{LLama-3-8B-Instruct}.}
    % \caption{The relationship between training steps and F1 score during the fine-tuning process with random documents and golden documents using \texttt{Llama-2-7b-chat-hf} and \texttt{LLama-3-8B-Instruct} on the NQ dataset.}
    \label{fig:training_step_f1}
    \vspace{-0.5cm}
\end{figure}

\subsection{Training with Random Docs: Better Performance and Faster Convergence}
We investigate why random training proves good performance from two aspects.
\paragraph{More random docs lead to better performance}
 % First, we vary the proportion of random documents (0 to 3) in training instances. As shown in Figure~\ref{fig:mixed_doc}, increasing random documents consistently improves F1 scores across both HotpotQA and NQ datasets. For Llama-2-7b, F1 scores increase from 44.0 to 47.0 on HotpotQA and 48 to 52 on NQ, with Llama-3-8B showing similar improvements.
We first vary the numbers of random documents (0 to 3) in training instances to examine how increasing random documents affects model performance. As shown in Figure~\ref{fig:mixed_doc}, increasing random documents consistently improves F1 scores across both datasets. For \texttt{Llama-2-7b-chat-hf}, using 3 random documents (versus zero) improves F1 scores by 3 points on HotpotQA and 4 points on NQ. \texttt{Llama-3-8B-Instruct} shows similar gains, suggesting that powerful models can effectively learn from random documents, making sophisticated document selection relatively unnecessary.

% To investigate the effect of varying the proportion of random documents on the robustness performance of models across different datasets, we examine the results of experiments conducted with three documents per training instance, where the number of random documents ranged from 0 to 3, with the remainder being golden documents.

% The empirical results in Figure~\ref{fig:mixed_doc} demonstrate a consistent enhancement in F1 scores as the proportion of random documents increases. Specifically, for both the HotpotQA and NQ datasets, models trained with a greater number of random documents exhibit superior performance. For example, the \texttt{Llama-2-7b-chat-hf} model shows a progressive increase in F1 score from 44.0 to 47.0 on the HotpotQA dataset and from 48 to 52 on the NQ dataset as the number of random documents increases from 0 to 3. Similarly, the \texttt{Llama-3-8B-Instruct} model demonstrates an improvement from 49 to 50 on the HotpotQA dataset and from 52 to 54 on the NQ dataset. More  experimental results on the TriviaQA and WebQuestions datasets are shown in Appendix Table~\ref{tab:appendix_mixed_doc}. These findings suggest that incorporating a higher proportion of random documents during training significantly enhances model robustness and generalization capabilities. The inclusion of random documents likely introduces greater variability and noise, which may enable the model to better generalize to diverse and previously unseen data during evaluation.



\paragraph{Faster convergence with random training}
The training dynamics in Figure~\ref{fig:training_step_f1} provide another evidence for why sophisticated document selection becomes unnecessary. Random document training not only achieves higher F1 scores (2-3 points improvement) but also reaches peak performance in fewer steps compared to golden document training. This faster convergence with better performance holds true for both model scales, indicating that simpler random training actually enables more efficient learning in powerful language models.
% In Figure~\ref{fig:training_step_f1}, we plot the training curve of the relationship between training steps and F1 score during the fine-tuning process with random documents and golden documents using \texttt{Llama-2-7b-chat-hf} and \texttt{LLama-3-8B-Instruct} on the NQ dataset. Our findings reveal that training with random documents leads to superior performance, as evidenced by achieving a higher optimal F1 score. Additionally, we observe that the model reaches a high F1 score more rapidly when trained with random documents. This suggests that the random document strategy more effectively harnesses the model's inherent robustness and generalization capabilities, resulting in enhanced performance compared to the golden document approach.

% \begin{figure}[t]
%     \centering
%     \begin{subfigure}[b]{0.23\textwidth}
%         \centering
%         \includegraphics[width=\textwidth]{fig/hotpot_randcur_llama3.pdf}
%     \end{subfigure}
%     \hfill
%     \begin{subfigure}[b]{0.23\textwidth}
%         \centering
%         \includegraphics[width=\textwidth]{fig/hotpot_goodtop1_llama3.pdf}
%     \end{subfigure}
%     \centering
%     \begin{subfigure}[b]{0.23\textwidth}
%         \centering
%         \includegraphics[width=\textwidth]{fig/hotpot_top1_llama3.pdf}
%     \end{subfigure}
%     \hfill
%     \begin{subfigure}[b]{0.23\textwidth}
%         \centering
%         \includegraphics[width=\textwidth]{fig/hotpot_IRM_llama3.pdf
%     \end{subfigure}
%     \caption{Attention distribution heatmaps for models trained with (a) Random and (b) Golden documents. Each cell ($i$, $j$) represents the average attention assigned to tokens in document $i$ by the $j$-th attention layer when generating answers. \textcolor{mygreen}{\textbf{Doc1}} (highlighted in green) contains the correct answer.}
%     \label{fig:attention}
%     \vspace{-0.5cm}
% \end{figure}
\section{Insights and Future Directions}

Our findings reveal a fundamental shift in RAG system: as models become more powerful, the marginal benefits of sophisticated training strategies diminish significantly. This observation has several important insights:

\paragraph{Simplified RAG architecture design}

For powerful models, simple retrieval strategies (even random selection) can achieve comparable performance to sophisticated approaches. This enables streamlined RAG architectures by replacing elaborate document filtering mechanisms with simpler retrieval methods, substantially reducing system complexity without sacrificing performance.


\paragraph{Scalable RAG for open-domain tasks.}  
Larger models demonstrate robustness against noisy retrieval, suggesting that open-domain RAG systems can function effectively with minimal retrieval supervision. Instead of enforcing strict filtering of retrieved documents, future large-scale RAG systems can leverage weakly supervised learning, incorporating large-scale web data and noisy retrieval results to improve generalization.

\paragraph{Theoretical implications for model scaling.}

This research reveals a previously unexplored aspect of scaling laws: the diminishing returns of complex training strategies as models grow larger. This challenges current theoretical frameworks and calls for new ones that better explain how training requirements evolve with model scale.

\paragraph{Broader impact on  machine learning.}

Our findings suggest that as models become more powerful, practitioners should prioritize architectural improvements and data quality over complex training strategies. This insight could lead to more efficient resource allocation in model development across various applications, from computer vision to natural language processing.

% Our work challenges conventional wisdom about the necessity of complex training strategies and suggests a more nuanced approach to model development that could significantly impact both theoretical understanding and practical applications in machine learning.
\section{Conclusion and future directions} \label{sec:conclusion}

In this paper we proposed a nested MLMC framework that offers important computational savings by performing most calculations in low precision and exploiting approximate random normal variables for the low precision path calculations. The low precision calculations could be performed in fixed precision on an FPGA for greater efficiency, and we suggested a procedure to optimise the bit-widths of every variable at each Monte Carlo level. This is an important improvement over previous mixed precision MLMC frameworks which held the lower precision fixed \cite{Rounding_error_oliver} or defined uniform bit-width at every level heuristically \cite{brugger2014mixed}. Our numerical results suggest that for the first levels our procedure reduces the cost at these levels by a factor 5 or 7. Hence the overall savings are significant since most paths are calculated on the first levels. Our approach would be even more efficient for the Milstein scheme because its higher order strong convergence leads to a greater proportion of the computational costs being on the coarsest levels.

The next stage of the research project will be to implement the RNG methods and the nested framework on FPGAs to determine the hardware requirements and confirm the extent of the computational savings. It would also be good to compare the performance benefits to using half-precision floating point arithmetic on GPUs or CPUs for the low-accuracy computations.




\section*{Limitations}  
While our study provides valuable insights into the impact of adversarial loss functions and document selection strategies on RAG model robustness, several limitations remain. First, our analysis is restricted to dense transformer-based models, leaving the effectiveness of these techniques on sparse models, such as mixture-of-experts (MoE) architectures, unexplored. Future work could investigate whether similar trends hold for sparsely activated models with dynamic routing mechanisms. Second, although we analyze the effectiveness of adversarial training, we do not explicitly examine its long-term stability or convergence properties, which may vary depending on hyperparameter choices and optimization dynamics. Additionally, while we demonstrate that stronger models exhibit diminishing returns from adversarial losses and document selection strategies, the precise mechanisms behind this phenomenon remain unclear. Further research is needed to understand how model capacity interacts with retrieval robustness.

\section*{Ethics Statements}  
Our study focuses on improving the robustness of RAG models, but several ethical considerations must be acknowledged. First, while adversarial training enhances model reliability, it does not eliminate the risk of biased or misleading outputs, particularly when retrieval sources contain inherent biases or misinformation. Future work should explore fairness-aware adversarial training to mitigate potential harms. Second, our findings suggest that stronger models require less intervention in document selection and loss design, which may influence resource allocation in real-world applications. Researchers and practitioners should ensure that model improvements do not disproportionately benefit well-resourced institutions while leaving smaller models less robust. Lastly, our experiments are conducted on widely used benchmark datasets, which may not fully reflect the diversity of real-world information needs. We encourage further research on robustness evaluation across varied domains, including low-resource languages and specialized knowledge fields, to ensure equitable advancements in RAG technology.


\bibliography{anthology,custom}

% \clearpage
\appendix
\appendix

\section{Appendix: Prompt}
\label{sec:appendix}
``Here is a sketch of an image. 
$\{input\_color\_mask\}$, while the rest of the white space is the background. 
I need you to infer details of the image based on the given sketch.
The details should include the possible background likely to be present with the $\{input\_color\_mask\}$, the attribute of each object (like wearing, texture, color etc.), the state (including action, posture, etc.) of each object, the direction of each object and the relationships between objects.

You should first analyze the mask carefully, considering the size, location, and relative position of each object mask. Ensure that specific actions are analyzed based on the mask, and infer each aspect with a reasoning process before providing the final output.
The final output format should be: $\{format\_example\}$, and you should refer to the example: $\{few\_shot\}$. You are going to complete the "" in each item, you need to complete them in multiple short phrases based on your above reasoning.

The state and relationship should be as detailed as possible while ensuring they align with the mask, formatted as: objectA action/spatial relation objectB, with both objectA and objectB included.
You should properly refer to some examples of attributes of object $\{attributes\}$ and relationships $\{relationships\}$.
Do not include words like `or', `possibly' in your final output, there should no ambiguity in your output.
Make sure all aspects of given mask is filled.''

\end{document}
