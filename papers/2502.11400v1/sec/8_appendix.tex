\section{Related Work}

\subsection{Large 3D Reconstruction Models}
Recently, generalized feed-forward models for 3D reconstruction from sparse input views have garnered considerable attention due to their applicability in heavily under-constrained scenarios. The Large Reconstruction Model (LRM)~\cite{hong2023lrm} uses a transformer-based encoder-decoder pipeline to infer a NeRF reconstruction from just a single image. Newer iterations have shifted the focus towards generating 3D Gaussian representations from four input images~\cite{tang2025lgm, xu2024grm, zhang2025gslrm, charatan2024pixelsplat, chen2025mvsplat, liu2025mvsgaussian}, showing remarkable novel view synthesis results. The paradigm of transformer-based sparse 3D reconstruction has also successfully been applied to lifting monocular videos to 4D~\cite{ren2024l4gm}. \\
Yet, none of the existing works in the domain have studied the use-case of inferring \textit{animatable} 3D representations from sparse input images, which is the focus of our work. To this end, we build on top of the Large Gaussian Reconstruction Model (GRM)~\cite{xu2024grm}.

\subsection{3D-aware Portrait Animation}
A different line of work focuses on animating portraits in a 3D-aware manner.
MegaPortraits~\cite{drobyshev2022megaportraits} builds a 3D Volume given a source and driving image, and renders the animated source actor via orthographic projection with subsequent 2D neural rendering.
3D morphable models (3DMMs)~\cite{blanz19993dmm} are extensively used to obtain more interpretable control over the portrait animation. For example, StyleRig~\cite{tewari2020stylerig} demonstrates how a 3DMM can be used to control the data generated from a pre-trained StyleGAN~\cite{karras2019stylegan} network. ROME~\cite{khakhulin2022rome} predicts vertex offsets and texture of a FLAME~\cite{li2017flame} mesh from the input image.
A TriPlane representation is inferred and animated via FLAME~\cite{li2017flame} in multiple methods like Portrait4D~\cite{deng2024portrait4d}, Portrait4D-v2~\cite{deng2024portrait4dv2}, and GPAvatar~\cite{chu2024gpavatar}.
Others, such as VOODOO 3D~\cite{tran2024voodoo3d} and VOODOO XP~\cite{tran2024voodooxp}, learn their own expression encoder to drive the source person in a more detailed manner. \\
All of the aforementioned methods require nothing more than a single image of a person to animate it. This allows them to train on large monocular video datasets to infer a very generic motion prior that even translates to paintings or cartoon characters. However, due to their task formulation, these methods mostly focus on image synthesis from a frontal camera, often trading 3D consistency for better image quality by using 2D screen-space neural renderers. In contrast, our work aims to produce a truthful and complete 3D avatar representation from the input images that can be viewed from any angle.  

\subsection{Photo-realistic 3D Face Models}
The increasing availability of large-scale multi-view face datasets~\cite{kirschstein2023nersemble, ava256, pan2024renderme360, yang2020facescape} has enabled building photo-realistic 3D face models that learn a detailed prior over both geometry and appearance of human faces. HeadNeRF~\cite{hong2022headnerf} conditions a Neural Radiance Field (NeRF)~\cite{mildenhall2021nerf} on identity, expression, albedo, and illumination codes. VRMM~\cite{yang2024vrmm} builds a high-quality and relightable 3D face model using volumetric primitives~\cite{lombardi2021mvp}. One2Avatar~\cite{yu2024one2avatar} extends a 3DMM by anchoring a radiance field to its surface. More recently, GPHM~\cite{xu2025gphm} and HeadGAP~\cite{zheng2024headgap} have adopted 3D Gaussians to build a photo-realistic 3D face model. \\
Photo-realistic 3D face models learn a powerful prior over human facial appearance and geometry, which can be fitted to a single or multiple images of a person, effectively inferring a 3D head avatar. However, the fitting procedure itself is non-trivial and often requires expensive test-time optimization, impeding casual use-cases on consumer-grade devices. While this limitation may be circumvented by learning a generalized encoder that maps images into the 3D face model's latent space, another fundamental limitation remains. Even with more multi-view face datasets being published, the number of available training subjects rarely exceeds the thousands, making it hard to truly learn the full distibution of human facial appearance. Instead, our approach avoids generalizing over the identity axis by conditioning on some images of a person, and only generalizes over the expression axis for which plenty of data is available. 

A similar motivation has inspired recent work on codec avatars where a generalized network infers an animatable 3D representation given a registered mesh of a person~\cite{cao2022authentic, li2024uravatar}.
The resulting avatars exhibit excellent quality at the cost of several minutes of video capture per subject and expensive test-time optimization.
For example, URAvatar~\cite{li2024uravatar} finetunes their network on the given video recording for 3 hours on 8 A100 GPUs, making inference on consumer-grade devices impossible. In contrast, our approach directly regresses the final 3D head avatar from just four input images without the need for expensive test-time fine-tuning.



% \section{Background}
% In this section, we introduce the framework of robust rag.

% % 基础模型的能力进化

% \subsection{Retrieval-Augmented Generation}  
% Retrieval-Augmented Generation (RAG) combines retrieval and generation to enhance the performance of language models by incorporating external knowledge. Specifically, for a given question $q$, a retriever model $\mathcal{R}$ is employed to select the top-$k$ relevant documents $\mathcal{D} = \{d_1, d_2, \dots, d_k\}$ from a large corpus. These documents are intended to provide additional context or factual grounding to support the model's reasoning process. Once retrieved, the LM uses this augmented input—typically comprising both $q$ and $\mathcal{D}$—to generate an answer $a$. 

% While this framework has shown substantial improvements in tasks requiring external knowledge, its effectiveness heavily relies on the quality, relevance, and accuracy of the retrieved documents, which introduces challenges in noisy or imperfect retrieval scenarios.

\section{Robustness Training for RAG}
\label{sec:robust_training}
\subsection{Retrieval-augmented Adaptive Adversarial Training}
To improve the robustness of retrieval-augmented language models (RALMs) against retrieval noise, RAAT~\cite{fang-etal-2024-enhancing} incorporates an adversarial loss into the standard supervised fine-tuning (SFT). The model processes one golden context and three adversarial samples at each iteration, optimizing for the most challenging perturbation. The adversarial objective follows a min-max strategy:
\begin{equation}
\textstyle
\setlength\abovedisplayskip{0.2cm}
\setlength\belowdisplayskip{0.2cm}
\begin{aligned}
\mathcal{L}_{\text{max}} = \max_{da \in DA} \mathcal{L}(\theta, da(q), a),
\end{aligned}
\end{equation}
where $\mathcal{L}$ represents the generation loss function to noisy retrievals, and $q^{\prime}=da(q)$ represents the augmented noise context of $q$. A regularization term controls excessive sensitivity to noise:
\begin{equation}
\textstyle
\setlength\abovedisplayskip{0.2cm}
\setlength\belowdisplayskip{0.2cm}
\begin{aligned}
\mathcal{L}_{\text{ada}} = \mathcal{L}_{\text{max}} + w_{\text{reg}} \cdot |\mathcal{L}_{\text{max}} - \mathcal{L}_{\text{min}}|_2^2,
\end{aligned}
\end{equation}

The adversarial loss $\mathcal{L}_{\text{ada}}$ combines both terms, where the regularization term helps stabilize training by mitigating excessive sensitivity to retrieval noise. The regularization term, calculated as the square of the difference between $\mathcal{L}_{\text{max}}$ and $\mathcal{L}_{\text{min}}$, encourages a more balanced optimization, preventing the model from overreacting to the most challenging perturbations.

\subsection{Multiagent Iterative Tuning Optimization}
ATM~\cite{zhu-etal-2024-atm} steers the generator to have a robust perspective of useful documents for question answering with the help of an auxiliary attacker agent. The generator is trained to maximize answer correctness while minimizing its sensitivity to adversarial perturbations. It receives the user query along with a document list, which may contain both relevant and fabricated information introduced by the attacker. Its objective is to generate accurate responses as long as sufficient truthful information is present while reducing the impact of misleading or fabricated content. To achieve this, the generator learns to identify and leverage relevant documents while ignoring noisy ones, regardless of whether they originate from the original retrieval $\mathcal{D}$ or adversarial perturbations $\mathcal{D}^{\prime}$. This can be formalized as maximizing the objective:
\begin{equation}
\small
\setlength\abovedisplayskip{0.2cm}
\setlength\belowdisplayskip{0.2cm}
\begin{aligned}
G(a\mid q,\mathcal{D}^{\prime}) - \mathrm{dist}\left[G(a\mid q, \mathcal{D}), G(a\mid q,\mathcal{D}^{\prime})\right],
\end{aligned}
\end{equation}
where ${G}(\cdot)$ represents the language model probability of generating an answer, and $\mathrm{dist}\left[\cdot\right]$ measures the divergence between outputs under different document conditions. The generator and the attacker are tuned adversarially for several iterations. After rounds of multi-agent iterative tuning, the generator can eventually better discriminate useful documents amongst fabrications.

\subsection{Invariant Risk Minimization}
Invariant Risk Minimization (IRM) aims to learn representations that remain stable across different environments, improving generalization under distribution shifts. To enhance the robustness of retrieval-augmented generation (RAG) across varying retrieval conditions, the V-REx objective~\cite{DBLP:conf/icml/KruegerCJ0BZPC21} can also be adapted to enforce risk invariance across different retrieval environments. Given retrieval environments $\mathcal{E} = \{1, \dots, m\}$, where each $e \in \mathcal{E}$ corresponds to a specific retrieval scenario (e.g., golden documents, top-$k$ retrieved, noisy retrieval), we define the empirical risk $\mathcal{R}_e(\theta)$ of the model parameterized by $\theta$. The training objective is formulated as:
\begin{equation}
\small
\setlength\abovedisplayskip{0.2cm}
\setlength\belowdisplayskip{0.2cm}
\begin{aligned}
\mathcal{R}_\textrm{V-REx-RAG}(\theta) = &\beta \; \mathrm{Var}(\{\mathcal{R}_1(\theta), ..., \mathcal{R}_m(\theta)\})\\ & + \sum^m_{e=1} \mathcal{R}_e(\theta),
\end{aligned}
\end{equation}
where $\beta \geq 0$ controls the trade-off between minimizing average risk and enforcing risk invariance across retrieval environments. A higher $\beta$ reduces performance discrepancies caused by retrieval variations, improving generalization under both high-quality and noisy retrieval conditions.

\section{Appendix}
\subsection{Dataset Statistics}
\label{sec:dataset_stats}
Detailed statistics of these datasets we used are listed in Table~\ref{tab:dataset_stats}.
\begin{table}[ht]
\centering
\resizebox{0.95\linewidth}{!}{
\begin{tabular}{lccc}
\toprule
\textbf{Dataset} & \textbf{Type} & \textbf{\# Train} & \textbf{\# Dev} \\
\midrule
NQ & single-hop & 79,168 & 8,757 \\
WebQuestions & single-hop & 2,474 & 278 \\
TriviaQA & multi-hop & 78,785 & 8,837 \\
HotpotQA & multi-hop & 90,447 & 7,405 \\
\bottomrule
\end{tabular}}
\caption{Statistics of different datasets}
\label{tab:dataset_stats}
\vspace{-0.5cm}
\end{table}
\subsection{Implementation Details}
For model training, we utilize the LLaMA-Factory library to facilitate efficient LLM finetuning. We train our models with a learning rate of 1e-6 and set the maximum number of epochs to 3. To optimize the training process, we employ bfloat16 precision and DeepSpeed ZeRO-3 for distributed training. The gradient accumulation steps are set to 2 with a batch size of 8. For inference, we leverage vLLM to ensure efficient model serving. We maintain the default decoding parameters for each model during inference, including the default temperature and top-p sampling probabilities, to ensure fair comparison across different model variants. All training and inference procedures are conducted on 8x NVIDIA A100 80GB GPUs.

\subsection{Prompt Instruction}
\label{sec:prompt}
We present the prompt used to guide LM inference on noisy contexts in Table~\ref{tab:generator_prompt}. Specifically, we concatenate the top-5 retrieved documents into a single passage, append the input question afterward, and prompt the LM to generate a concise answer. 
\begin{table*}[ht]
\small
\centering
\begin{tabular}{|p{14.5cm}|}
\hline
\texttt{You need to answer my question and complete the question-and-answer pair following the format provided in the example. The answers should be short phrases or entities, not full sentences. Here are some examples to guide you.} \\
\\
\texttt{Example 1:}\\
\texttt{Question: What is the capital of France?}\\
\texttt{Answer: Paris}\\
\texttt{Example 2:}\\
\texttt{Question: Who invented the telephone?}\\
\texttt{Answer: Alexander Graham Bell}\\
\texttt{Example 3:}\\
\texttt{Question: Which element has the atomic number 1?}\\
\texttt{Answer: Hydrogen}\\
\\

\texttt{\#\#\# Retrieved Documents: \{\}} \\
\texttt{\#\#\# Question: \{\}} \\
\texttt{\#\#\# Answer:} \\
\hline
\end{tabular}
\caption{The generator prompt we used in our experiments.}
\label{tab:generator_prompt}
\end{table*}

\begin{table*}[t]
\centering
\resizebox{0.8\linewidth}{!}{
\begin{tabular}{llcccccccccc}
\toprule
\multirow{2}{*}{\textbf{Model}} & \multirow{2}{*}{\textbf{RAG Scenario}} & \multicolumn{2}{c}{\textbf{HotpotQA}} & \multicolumn{2}{c}{\textbf{NQ}} & \multicolumn{2}{c}{\textbf{WebQuestions}} & \multicolumn{2}{c}{\textbf{TriviaQA}}  & \multicolumn{2}{c}{\textbf{AVERAGE}} \\
\cline{3-4}\cline{5-6}\cline{7-8}\cline{9-10}\cline{11-12}
&  &  \textbf{EM} & \textbf{F1} & \textbf{EM} & \textbf{F1} & \textbf{EM} & \textbf{F1} & \textbf{EM} & \textbf{F1} & \textbf{EM} & \textbf{F1}  \\
\midrule
\multirow{9}{*}{\rotatebox{90}{\textit{Qwen1.5-7B-Chat}}} & Base Model & 24.15 & 34.06 & 24.31 & 35.27 & 23.02 & 38.65 & 48.43 & 57.65 & 29.98 & 41.41 \\
& RALM & 24.44 & 34.23 & 29.83 & 40.22 & 44.24 & 53.46 & 46.40 & 55.96 & 36.23 & 45.97 \\
& RetRobust & 29.05 & 40.99 & 34.82 & 45.92 & 47.12 & 55.39 & 49.16 & 59.01 & 40.04 & 50.33 \\
& Top-1 Doc & 29.68 & 41.57 & 34.16 & 45.44 & 43.88 & 52.71 & 49.61 & 59.03 & 39.33 & 49.69 \\
& Golden Doc & 28.66 & 40.55 & 32.93 & 43.99 & 43.88 & 52.74 & 49.25 & 58.87 & 38.68 & 49.04 \\
\cdashline{2-12}
& Random Doc & 28.83 & 40.66 & 33.96 & 45.09 & 46.04 & 53.21 & 48.75 & 58.54 & 39.40 & 49.38 \\
& Irrelevant Doc & 28.12 & 39.87 & 32.74 & 43.94 & 43.88 & 52.67 & 48.33 & 58.18 & 38.27 & 48.67 \\
\cdashline{2-12}
& IRM & 26.95 & 38.67 & 30.59 & 41.83 & 47.84 & 55.68 & 45.52 & 56.24 & 37.73 & 48.11 \\ 
\cmidrule{2-12}
% & \textcolor{mycolor}{$\Delta$ (Random $\rightarrow$ Best)} &  \textcolor{mycolor}{2.95\%}  & \textcolor{mycolor}{2.24\%}  &  \textcolor{mycolor}{2.53\%}  &  \textcolor{mycolor}{1.84\%}  &  \textcolor{mycolor}{2.35\%}   &  \textcolor{mycolor}{4.10\%}  &  \textcolor{mycolor}{1.76\%}  &  \textcolor{mycolor}{0.84\%} & \textcolor{mycolor}{2.40\%} & \textcolor{mycolor}{2.25\%} \\
& \textcolor{mycolor}{$\Delta$ (Worst $\rightarrow$ Best)} &  \textcolor{mycolor}{21.44\%}  & \textcolor{mycolor}{21.44\%}  &  \textcolor{mycolor}{16.73\%}  &  \textcolor{mycolor}{14.17\%}  &  \textcolor{mycolor}{9.02\%}   &  \textcolor{mycolor}{5.71\%}  &  \textcolor{mycolor}{8.99\%}  &  \textcolor{mycolor}{5.49\%} & \textcolor{mycolor}{10.52\%} & \textcolor{mycolor}{9.48\%} \\
\midrule
\multirow{9}{*}{\rotatebox{90}{\textit{Qwen2.5-7B-Instruct}}} & Base Model & 25.10 & 35.02 & 26.97 & 38.15 & 25.90 & 42.46 & 53.86 & 62.73 & 32.96 & 44.59 \\
& RALM & 26.37 & 36.30 & 31.43 & 42.00 & 41.73 & 51.98 & 53.92 & 62.74 & 38.36 & 48.26 \\
& RetRobust & 30.47 & 42.26 & 34.60 & 46.00 & 45.68 & 53.80 & 54.43 & 63.63 & 41.30 & 51.42 \\
& Top-1 Doc & 30.24 & 42.61 & 34.72 & 46.20 & 46.04 & 54.13 & 53.65 & 63.01 & 41.16 & 51.49 \\
& Golden Doc & 30.28 & 41.84 & 33.77 & 45.11 & 44.60 & 53.18 & 54.35 & 63.55 & 40.75 & 50.92 \\
\cdashline{2-12}
& Random Doc & 30.25 & 42.22 & 34.08 & 45.47 & 44.24 & 53.30 & 54.29 & 63.54 & 40.72 & 51.13 \\
& Irrelevant Doc & 29.76 & 41.67 & 33.64 & 44.87 & 42.09 & 52.72 & 53.38 & 63.01 & 39.72 & 50.57 \\
\cdashline{2-12}
& IRM & 29.58 & 41.73 & 32.79 & 44.17 & 46.40 & 54.80 & 53.45 & 64.26 & 40.56 & 51.24 \\
\cmidrule{2-12}
% & \textcolor{mycolor}{$\Delta$ (Random $\rightarrow$ Best)} &  \textcolor{mycolor}{0.73\%}  & \textcolor{mycolor}{0.92\%}  &  \textcolor{mycolor}{1.88\%}  &  \textcolor{mycolor}{1.61\%}  &  \textcolor{mycolor}{7.32\%}   &  \textcolor{mycolor}{4.93\%}  &  \textcolor{mycolor}{0.26\%}  &  \textcolor{mycolor}{0.14\%} & \textcolor{mycolor}{2.55\%}  &  \textcolor{mycolor}{1.90\%} \\
& \textcolor{mycolor}{$\Delta$ (Worst $\rightarrow$ Best)} &  \textcolor{mycolor}{15.55\%}  & \textcolor{mycolor}{17.38\%}  &  \textcolor{mycolor}{10.47\%}  &  \textcolor{mycolor}{10.00\%}  &  \textcolor{mycolor}{11.19\%}   &  \textcolor{mycolor}{5.43\%}  &  \textcolor{mycolor}{1.97\%}  &  \textcolor{mycolor}{2.42\%} & \textcolor{mycolor}{7.64\%}  &  \textcolor{mycolor}{6.70\%} \\
\bottomrule
\end{tabular}}
\caption{Performance comparison of different LLMs (\texttt{Qwen1.5-7B-Chat} and \texttt{Qwen2.5-7B-Instruct}) across different robust RAG scenarios on four datasets (HotpotQA, NQ, WebQuestions, and TriviaQA). The row \textcolor{mycolor}{$\Delta$ (Worst $\rightarrow$ Best)} represents the performance gain achieved by the best method compared to the worst strategy among these training strategies, highlighting the benefit of sophisticated robust training methods.}
\label{tab:main_result_qwen}
\vspace{-0.3cm}
\end{table*}

\begin{table*}[t]
\small
\centering
\resizebox{0.9\linewidth}{!}{
\begin{tabular}{llcccccccccc}
\toprule
\multirow{2}{*}{\textbf{Model}} & \multirow{2}{*}{\textbf{Document Distribution}} & \multicolumn{2}{c}{\textbf{HotpotQA}} & \multicolumn{2}{c}{\textbf{NQ}} & \multicolumn{2}{c}{\textbf{WebQuestion}} & \multicolumn{2}{c}{\textbf{TriviaQA}} & \multicolumn{2}{c}{\textbf{AVERAGE}}\\
\cline{3-4}\cline{5-6}\cline{7-8}\cline{9-10}\cline{11-12}
 & &  \textbf{EM} & \textbf{F1} & \textbf{EM} & \textbf{F1} & \textbf{EM} & \textbf{F1} & \textbf{EM} & \textbf{F1} & \textbf{EM} & \textbf{F1} \\
\midrule
\multirow{7}{*}{\textit{Llama-2-7b-chat-hf}} & Base Model &  3.30  &  12.34 &  1.21 &  10.61  &  0.00  &  13.08  &  4.32  &  20.27 & 2.21 & 14.08  \\
& 1 Golden Doc & 30.67 & 42.78 & 36.50 & 47.77  &  39.93  &  52.11  &  50.25  & 63.28 & 39.34 & 51.49  \\
& 1 Random Doc & 30.94 & 43.11 & 38.16 & 49.78  &  42.45  &  53.97  &  52.72 &  65.52 & 41.07 & 53.10  \\
& 0 Random + 3 Golden & 31.79 & 44.03 & 36.68 & 47.90  & 43.02  & 53.43 & 56.55 & 64.97 & 42.01 & 52.58 \\
& 1 Random + 2 Golden & 32.82 & 44.77 & 36.70 & 48.21  & 43.38  & 54.34 & 56.93 & 65.67 & 42.46 & 53.25 \\
& 2 Random + 1 Golden & 33.32 & 45.45 & 37.07 & 48.67  & 44.10  & 54.82 & 57.56 & 65.90 & 43.01 & 53.71 \\
& 3 Random + 0 Golden & 34.74 & 46.79 & 37.35 & 49.34  & 45.18  & 55.70 & 57.77 & 66.40 & 43.76 & 54.56  \\ 
\midrule
\multirow{7}{*}{\textit{Llama-3-8B-Instruct}} & Base Model & 23.31 & 32.60 & 30.04 & 41.59 & 26.98 & 43.25 & 58.80 & 66.45 & 34.78 & 45.97 \\
 & 1 Golden Doc & 35.52 & 48.31 & 41.35 & 53.13  &  48.92  &  58.41  &  58.26 & 66.99 &  46.01 & 56.71 \\
 & 1 Random Doc & 35.98 & 49.05 & 43.37 & 55.43  &  53.24  &   62.55 &  60.62  & 68.64 & 48.30 & 58.92  \\
 & 0 Random + 3 Golden & 35.96 & 49.01 & 40.88 & 52.91  &  50.25  & 59.13 &  54.95 &  66.48 & 45.51 & 56.88  \\
 & 1 Random + 2 Golden & 36.22 & 49.07 & 41.22 & 53.13 &  50.41  & 59.97 &  55.27 &  66.77 & 45.78 & 57.24  \\
 & 2 Random + 1 Golden & 36.77 & 49.59 & 41.75 & 53.48  &  51.49  & 61.02 &  55.39 &  66.74 & 46.35 & 57.71  \\
 & 3 Random + 0 Golden & 37.33 & 50.30 & 42.45 & 54.32  &  52.05  & 61.16 &  56.20 &  67.83 & 47.01 & 58.40  \\
\bottomrule
\end{tabular}}
\caption{Experimental comparison of RAG robust training using different numbers of random documents and golden documents mixed together, with \texttt{Llama-2-7b-chat-hf} and \texttt{Llama-3-8B-Instruct} models across four datasets.} 
\label{tab:appendix_mixed_doc}
\end{table*}
\begin{table*}[t]
\small
\centering
\resizebox{0.6\linewidth}{!}{
\begin{tabular}{llcccc}
\toprule
\multirow{2}{*}{\textbf{Model}} & \multirow{2}{*}{\textbf{RAG Scenario}} & \multicolumn{2}{c}{\textbf{HotpotQA}} & \multicolumn{2}{c}{\textbf{NQ}} \\
\cline{3-4}\cline{5-6}
 & &  \textbf{EM} & \textbf{F1} & \textbf{EM} & \textbf{F1}  \\
\midrule
\multirow{4}{*}{\textit{Qwen2.5-0.5B-Instruct}} & RetRobust & 17.69 & 26.63 & 17.16 & 25.44   \\
 & Top-1 Doc & 18.28 & 27.38 & 18.00 & 26.52 \\
 & Golden Doc & 19.11 & 28.04 & 19.32 & 27.49 \\
 & Random Doc & 18.10 & 27.23 & 16.27 & 24.62 \\
\midrule
\multirow{4}{*}{\textit{Qwen2.5-1.5B-Instruct}} & RetRobust &  23.35 & 34.13 & 25.01 & 35.35  \\
 & Top-1 Doc & 24.08 & 34.31 & 25.53 & 35.75   \\
 & Golden Doc & 23.70 & 34.03 & 25.61 & 35.52  \\
 & Random Doc & 23.50 & 33.68 & 24.63 & 34.89  \\
\midrule
\multirow{4}{*}{\textit{Qwen2.5-3B-Instruct}} & RetRobust &  26.29 & 37.34 & 29.46 & 40.38   \\
 & Top-1 Doc & 26.17 & 37.33 & 29.86 & 40.67  \\
 & Golden Doc & 26.17 & 37.16 & 30.08 & 40.41  \\
 & Random Doc & 26.28 & 37.26 & 29.34 & 40.09   \\
\midrule
\multirow{4}{*}{\textit{Mistral-7B-Instruct-v0.2}} & RetRobust &  35.92 & 49.03 & 43.19 & 55.51   \\
 & Top-1 Doc & 35.80 & 49.08 & 44.35 & 56.29   \\
 & Golden Doc & 34.67 & 47.46 & 41.52 & 53.41   \\
 & Random Doc & 35.52 & 48.71 & 43.87 & 55.84   \\
% \midrule
% \multirow{4}{*}{\textit{gemma-2-9b-it}} & RetRobust &  37.49 &  50.77  &  -  & - \\
%  & Top-1 Doc & 37.30 & 50.85  & - & - \\
%  & Golden Doc & 33.37 & 50.56 & 43.34 & 55.16  \\
%  & Random Doc & 37.00 & 50.39  & - & - \\
\midrule
\multirow{4}{*}{\textit{Qwen2.5-32B-Instruct}} & RetRobust &  33.96 & 46.28 & 35.67 & 48.77   \\
 & Top-1 Doc & 34.47 & 46.88 & 35.90 & 48.80   \\
 & Golden Doc & 33.68 & 45.62 & 35.29 & 47.69   \\
 & Random Doc & 33.81 & 46.31 & 36.09 & 49.03   \\
\midrule
\multirow{4}{*}{\textit{Llama-2-70b-chat-hf}} & RetRobust &  36.22 & 49.58 & 41.49 & 54.73 \\
 & Top-1 Doc & 35.45 & 48.75 & 40.39 & 53.92  \\
 & Golden Doc & 34.27 & 47.72 & 39.23 & 51.69  \\
 & Random Doc & 34.90 & 48.31 & 40.39 & 53.92  \\
\midrule
\multirow{4}{*}{\textit{Llama-3-70B-Instruct}} & RetRobust &  40.78 & 54.67 & 45.84 & 59.25  \\
 & Top-1 Doc & 40.61 & 55.37 & 46.05 & 59.61  \\
 & Golden Doc & 40.65 & 54.15 & 44.80 & 57.85  \\
 & Random Doc & 39.91 & 54.71  & 43.71 & 57.86 \\
\midrule
\multirow{4}{*}{\textit{Qwen2.5-72B-Instruct}} & RetRobust &  37.06 &  49.96  & 39.99 & 52.70 \\
 & Top-1 Doc & 37.52 & 50.30 & 39.87 & 52.81  \\
 & Golden Doc & 36.73 & 49.66 & 38.86 & 51.73  \\
 & Random Doc & 36.75 & 50.05 & 39.95 & 52.86  \\
\bottomrule
\end{tabular}}
\caption{Performance comparison of LLM with different parameter sizes (from 0.5B to 72B) in different robust RAG scenarios on HotpotQA and NQ datasets.}
\label{tab:vary_parameter_analysis}
\end{table*}
\begin{table*}[ht]
\centering
\resizebox{0.8\linewidth}{!}{
\begin{tabular}{llcccccccc}
\toprule
\multirow{2}{*}{\textbf{Models}}  & \multirow{2}{*}{\textbf{RAG Scenario}} & \multicolumn{2}{c}{\textbf{HotpotQA}} & \multicolumn{2}{c}{\textbf{NQ}} & \multicolumn{2}{c}{\textbf{WebQ}} & \multicolumn{2}{c}{\textbf{TriviaQA}} \\
\cline{3-10}
&  & \textbf{Correct} & \textbf{Wrong} & \textbf{Correct} & \textbf{Wrong} & \textbf{Correct} & \textbf{Wrong} & \textbf{Correct} & \textbf{Wrong} \\
\midrule
\multirow{4}{*}{\textit{Llama2-7b-chat-hf}} & Base Model & 93.80 & 95.82 & 95.25 & 95.87 & 0.00 & 96.21 & 94.56 & 96.04 \\
& Golden Doc & 96.33 & 86.80 & 96.28 & 87.78 & 97.11 & 89.42 & 89.69 & 79.29 \\
& Random Doc & 96.45 & 84.54 & 96.06 & 85.58 & 96.13 & 89.00 & 89.22 & 78.60 \\
& IRM  & 95.78 & 83.33 & 95.77 & 85.09 & 97.06 & 91.67 & 92.96 & 80.47 \\
\midrule
\multirow{4}{*}{\textit{Llama-3-8B-Instruct}} & Base Model & 97.53 & 91.30 & 97.59 & 91.30 & 97.90 & 89.22 & 98.15 & 90.08\\
& Golden Doc & 96.78 & 85.82 & 96.36 & 86.48 & 94.86 & 85.89 & 93.03 & 76.31 \\
& Random Doc  & 96.46 & 83.63 & 96.15 & 84.50 & 94.71 & 84.83 & 96.08 & 81.97 \\
& IRM  & 95.84 & 80.68 & 95.55 & 82.33 & 98.63 & 93.16 & 89.31 & 76.55 \\
\midrule
\multirow{4}{*}{\textit{Qwen1.5-7B-Chat}} & Base Model & 98.00 & 93.66 & 98.50 & 94.03 & 97.64 & 92.34 & 98.00 & 92.80 \\
& Golden Doc & 96.33 & 83.87 & 95.94 & 84.39 & 94.20 & 84.82 & 92.90 & 76.17 \\
& Random Doc  & 95.55 & 80.39 & 95.13 & 80.39 & 93.68 & 82.59 & 95.15 & 83.18 \\
& IRM  & 95.07 & 78.05 & 93.99 & 77.91 & 98.41 & 91.10 & 89.87 & 72.35 \\
\midrule
\multirow{4}{*}{\textit{Qwen2.5-7B-Instruct}} & Base Model & 98.52 & 94.54 & 98.98 & 95.42 & 97.15 & 95.54 & 98.62 & 93.80  \\
& Golden Doc & 96.25 & 83.91 & 96.00 & 84.83 & 95.33 & 87.23 & 96.24 & 82.33 \\
& Random Doc  & 95.92 & 80.28 & 95.57 & 80.46 & 95.28 & 84.24 & 96.51 & 82.87\\
& IRM  & 96.04 & 80.20 & 95.31 & 80.68 & 97.75 & 91.73 & 93.78 & 78.70 \\
\bottomrule
\end{tabular}}
\caption{Comparison of answer confidence scores for different models after robust training strategies on HotpotQA, NQ, WebQ and TriviaQA datasets.}
\label{tab:confidence_comparison}
\end{table*}
\begin{table*}[t]
\centering
\resizebox{0.75\linewidth}{!}{
\begin{tabular}{clcccccccc}
\toprule
\multirow{2}{*}{\textbf{Model}} & \multirow{2}{*}{\textbf{RAG Scenario}} & \multicolumn{2}{c}{\textbf{HotpotQA}} & \multicolumn{2}{c}{\textbf{NQ}} &  \multicolumn{2}{c}{\textbf{WebQuestions}} & \multicolumn{2}{c}{\textbf{TriviaQA}} \\
\cline{3-4}\cline{5-6}\cline{7-8}\cline{9-10}
 & &  \textbf{EM} & \textbf{F1} & \textbf{EM} & \textbf{F1} & \textbf{EM} & \textbf{F1} & \textbf{EM} & \textbf{F1} \\
\midrule
\multirow{8}{*}{\rotatebox{90}{\textit{Llama-2-7b-chat-hf}}} & Base Model & \cellcolor{mygray}3.30 & \cellcolor{mygray}12.34 & \cellcolor{myorange}1.21 & \cellcolor{myorange}10.61 & \cellcolor{myorange}0.00 & \cellcolor{myorange}13.08 & \cellcolor{myorange}4.32 & \cellcolor{myorange}20.27 \\
& RALM & \cellcolor{mygray}26.21 & \cellcolor{mygray}36.42 & \cellcolor{myorange}27.90 & \cellcolor{myorange}38.41 & \cellcolor{myorange}30.22 & \cellcolor{myorange}46.22 & \cellcolor{myorange}55.36 & \cellcolor{myorange}63.44 \\
& RetRobust & \cellcolor{mygray}31.29 & \cellcolor{mygray}43.65 & \cellcolor{myorange}30.72 & \cellcolor{myorange}43.69 & \cellcolor{myorange}30.94 & \cellcolor{myorange}48.52 & \cellcolor{myorange}59.56 & \cellcolor{myorange}68.58 \\
& Top-1 Doc & \cellcolor{mygray}{31.76} & \cellcolor{mygray}{43.95} & \cellcolor{myorange}{31.31} & \cellcolor{myorange}{44.02} & \cellcolor{myorange}{30.94} & \cellcolor{myorange}{49.73} & \cellcolor{myorange}59.99 & \cellcolor{myorange}68.67 \\
& Golden Doc & \cellcolor{mygray}30.67 & \cellcolor{mygray}42.78 & \cellcolor{myorange}30.59 & \cellcolor{myorange}42.52 & \cellcolor{myorange}30.22 & \cellcolor{myorange}46.58 & \cellcolor{myorange}58.59 & \cellcolor{myorange}67.25 \\
& Random Doc & \cellcolor{mygray}30.94 & \cellcolor{mygray}43.11 & \cellcolor{myorange}31.08 & \cellcolor{myorange}43.92 & \cellcolor{myorange}30.58 & \cellcolor{myorange}48.58 & \cellcolor{myorange}{60.42} & \cellcolor{myorange}{69.16} \\
& Irrelevant Doc & \cellcolor{mygray}31.01 & \cellcolor{mygray}42.98 & \cellcolor{myorange}31.10 & \cellcolor{myorange}43.41 & \cellcolor{myorange}29.14 & \cellcolor{myorange}46.70 & \cellcolor{myorange}59.25 & \cellcolor{myorange}67.92 \\
& IRM & \cellcolor{mygray}34.38 & \cellcolor{mygray}47.11 & \cellcolor{myorange}30.79 & \cellcolor{myorange}43.06 & \cellcolor{myorange}31.29 & \cellcolor{myorange}49.51 & \cellcolor{myorange}59.57 & \cellcolor{myorange}68.26 \\
\midrule
\multirow{8}{*}{\rotatebox{90}{\textit{Llama-3-8B-Instruct}}} & Base Model & \cellcolor{mygray}23.31 & \cellcolor{mygray}32.60 & \cellcolor{myorange}30.04 & \cellcolor{myorange}41.59 & \cellcolor{myorange}26.98 & \cellcolor{myorange}43.25 & \cellcolor{myorange}58.80 & \cellcolor{myorange}66.45 \\
& RALM & \cellcolor{mygray}27.64 & \cellcolor{mygray}38.10 & \cellcolor{myorange}30.34 & \cellcolor{myorange}41.25 & \cellcolor{myorange}30.22 & \cellcolor{myorange}44.94 & \cellcolor{myorange}58.72 & \cellcolor{myorange}66.90 \\
& RetRobust & \cellcolor{mygray}36.06 & \cellcolor{mygray}48.99 & \cellcolor{myorange}34.05 & \cellcolor{myorange}47.10 & \cellcolor{myorange}31.65 & \cellcolor{myorange}49.53 & \cellcolor{myorange}61.62 & \cellcolor{myorange}70.99 \\
& Top-1 Doc & \cellcolor{mygray}{36.72} & \cellcolor{mygray}{49.30} & \cellcolor{myorange}{34.32} & \cellcolor{myorange}47.12 & \cellcolor{myorange}30.22 & \cellcolor{myorange}49.96 & \cellcolor{myorange}{62.16} & \cellcolor{myorange}{71.28} \\
& Golden Doc & \cellcolor{mygray}35.52 & \cellcolor{mygray}48.31 & \cellcolor{myorange}33.78 & \cellcolor{myorange}46.45 & \cellcolor{myorange}29.86 & \cellcolor{myorange}47.26 & \cellcolor{myorange}60.72 & \cellcolor{myorange}69.89 \\
& Random Doc & \cellcolor{mygray}35.98 & \cellcolor{mygray}49.05 & \cellcolor{myorange}34.13 & \cellcolor{myorange}{47.30} & \cellcolor{myorange}28.70 & \cellcolor{myorange}47.94 & \cellcolor{myorange}61.94 & \cellcolor{myorange}71.16 \\
& Irrelevant Doc & \cellcolor{mygray}35.31 & \cellcolor{mygray}47.92 & \cellcolor{myorange}33.80 & \cellcolor{myorange}47.03 & \cellcolor{myorange}{31.65} & \cellcolor{myorange}{50.63} & \cellcolor{myorange}61.37 & \cellcolor{myorange}70.77 \\
& IRM & \cellcolor{mygray}35.19 & \cellcolor{mygray}48.08 & \cellcolor{myorange}34.29 & \cellcolor{myorange}47.15 & \cellcolor{myorange}30.94 & \cellcolor{myorange}49.09 & \cellcolor{myorange}61.68 & \cellcolor{myorange}70.80 \\
\midrule
\multirow{8}{*}{\rotatebox{90}{\textit{Qwen1.5-7B-Chat}}} & Base Model & \cellcolor{mygray}24.15 & \cellcolor{mygray}34.06 & \cellcolor{myorange}24.31 & \cellcolor{myorange}35.27 & \cellcolor{myorange}23.02 & \cellcolor{myorange}38.65 & \cellcolor{myorange}48.43 & \cellcolor{myorange}57.65  \\
& RALM & \cellcolor{mygray}24.44 & \cellcolor{mygray}34.23 & \cellcolor{myorange}25.00 & \cellcolor{myorange}35.02 & \cellcolor{myorange}23.74 & \cellcolor{myorange}37.56 & \cellcolor{myorange}49.97 & \cellcolor{myorange}58.42 \\
 & RetRobust & \cellcolor{mygray}29.05 & \cellcolor{mygray}40.99 & \cellcolor{myorange}25.84 & \cellcolor{myorange}37.67 & \cellcolor{myorange}27.34 & \cellcolor{myorange}43.27 & \cellcolor{myorange}52.97 & \cellcolor{myorange}61.93 \\
& Top-1 Doc & \cellcolor{mygray}29.68 & \cellcolor{mygray}41.57 & \cellcolor{myorange}25.21 & \cellcolor{myorange}37.12 & \cellcolor{myorange}25.18 & \cellcolor{myorange}42.39 & \cellcolor{myorange}52.27 & \cellcolor{myorange}61.90 \\
& Golden Doc & \cellcolor{mygray}28.66 & \cellcolor{mygray}40.55 & \cellcolor{myorange}25.43 & \cellcolor{myorange}36.84 & \cellcolor{myorange}24.46 & \cellcolor{myorange}41.78 & \cellcolor{myorange}51.26 & \cellcolor{myorange}60.34 \\
& Random Doc & \cellcolor{mygray}28.83 & \cellcolor{mygray}40.66 & \cellcolor{myorange}25.88 & \cellcolor{myorange}38.04 & \cellcolor{myorange}26.62 & \cellcolor{myorange}43.43 & \cellcolor{myorange}52.47 & \cellcolor{myorange}61.78 \\
& Irrelevant Doc & \cellcolor{mygray}28.12 & \cellcolor{mygray}39.87 & \cellcolor{myorange}24.67 & \cellcolor{myorange}36.46 & \cellcolor{myorange}25.54 & \cellcolor{myorange}42.16 & \cellcolor{myorange}51.45 & \cellcolor{myorange}60.55 \\
& IRM & \cellcolor{mygray}26.95 & \cellcolor{mygray}38.67 & \cellcolor{myorange}25.98 & \cellcolor{myorange}37.63 & \cellcolor{myorange}25.54 & \cellcolor{myorange}41.77 & \cellcolor{myorange}52.02 & \cellcolor{myorange}61.37 \\
\midrule
\multirow{8}{*}{\rotatebox{90}{\textit{Qwen2.5-7B-Instruct}}} & Base Model & \cellcolor{mygray}25.10 & \cellcolor{mygray}35.02 & \cellcolor{myorange}26.97 & \cellcolor{myorange}38.15 & \cellcolor{myorange}25.90 & \cellcolor{myorange}42.46 & \cellcolor{myorange}53.86 & \cellcolor{myorange}62.73 \\
& RALM & \cellcolor{mygray}26.37 & \cellcolor{mygray}36.30 & \cellcolor{myorange}26.98 & \cellcolor{myorange}37.68 & \cellcolor{myorange}26.98 & \cellcolor{myorange}42.99 & \cellcolor{myorange}54.43 & \cellcolor{myorange}62.86   \\
& RetRobust & \cellcolor{mygray}30.47 & \cellcolor{mygray}42.26 & \cellcolor{myorange}28.58 & \cellcolor{myorange}41.25 & \cellcolor{myorange}28.78 & \cellcolor{myorange}45.61 & \cellcolor{myorange}58.52 & \cellcolor{myorange}67.21  \\
& Top-1 Doc & \cellcolor{mygray}30.24 & \cellcolor{mygray}42.61 & \cellcolor{myorange}28.85 & \cellcolor{myorange}41.41 & \cellcolor{myorange}28.78 & \cellcolor{myorange}46.31 & \cellcolor{myorange}58.74 & \cellcolor{myorange}67.34  \\
& Golden Doc & \cellcolor{mygray}30.28 & \cellcolor{mygray}41.84 & \cellcolor{myorange}28.93 & \cellcolor{myorange}41.10 & \cellcolor{myorange}28.78 & \cellcolor{myorange}46.68 & \cellcolor{myorange}57.64 & \cellcolor{myorange}66.39  \\
& Random Doc & \cellcolor{mygray}30.25 & \cellcolor{mygray}42.22 & \cellcolor{myorange}28.73 & \cellcolor{myorange}41.19 & \cellcolor{myorange}28.42 & \cellcolor{myorange}44.72 & \cellcolor{myorange}58.40 & \cellcolor{myorange}67.38  \\
& Irrelevant Doc & \cellcolor{mygray}29.76 & \cellcolor{mygray}41.67 & \cellcolor{myorange}28.40 & \cellcolor{myorange}41.00 & \cellcolor{myorange}27.70 & \cellcolor{myorange}43.90 & \cellcolor{myorange}58.40 & \cellcolor{myorange}67.38  \\
& IRM & \cellcolor{mygray}29.58 & \cellcolor{mygray}41.73 & \cellcolor{myorange}28.77 & \cellcolor{myorange}41.47 & \cellcolor{myorange}25.54 & \cellcolor{myorange}42.84 & \cellcolor{myorange}58.28 & \cellcolor{myorange}67.17 \\
\bottomrule
\end{tabular}}
\caption{Comparison of different document selection strategies on three datasets (NQ, WebQuestions, TriviaQA) after \textbf{training on HotpotQA}. The \colorbox{mygray}{gray cells} indicate results evaluated on HotpotQA, while the \colorbox{myorange}{orange cells} indicate results evaluated on other datasets to assess generalization ability.}
\label{tab:cross_dataset_eval}
\end{table*}


% \begin{figure}[t]
%     \centering
%     \includegraphics[width=\columnwidth]{fig/f1_nq.pdf}
%     \caption{Comparison of F1 Scores for training with random doc and golden doc across models with varying parameter sizes from 0.5B to 70B. The left figure shows results on the HotpotQA dataset, while the right figure presents results on the NQ dataset. We provide detailed results in Appendix Table~\ref{tab:vary_parameter_analysis}.}
%     % \label{fig:parameter_scale}
%     \vspace{-0.5cm}
% \end{figure}


% \begin{figure}[ht]
%     \centering
%     \includegraphics[width=\columnwidth]{fig/training_step_f1.pdf}
%     \caption{The relationship between training steps and F1 score during the fine-tuning process with random documents and golden documents using \texttt{Llama-2-7b-chat-hf} and \texttt{LLama-3-8B-Instruct} on the NQ dataset.}
%     \label{fig:training_step_f1}
%     \vspace{-0.5cm}
% \end{figure}

% \begin{figure}[t]
%     \centering
%     \includegraphics[width=\columnwidth]{fig/llama3_cross_dataset.pdf}
%     \caption{Comparison of different document selection strategies on three datasets (NQ, WebQuestions, TriviaQA) after \textbf{training on HotpotQA}. The bars with \textit{diagonal hatches} indicate results evaluated on HotpotQA, while the plain bars indicate results evaluated on other datasets to assess generalization ability. We provide more results on \texttt{Qwen1.5-7B-Chat} and \texttt{Qwen2.5-7B-Instruct} in Appendix Table~\ref{tab:cross_dataset_eval}.}
%     % \label{fig:cross_dataset_eval_part}
%     \vspace{-0.5cm}
% \end{figure}

% \subsection{Training Curve Analysis}
% In Figure~\ref{fig:training_step_f1}, we plot the training curve of the relationship between training steps and F1 score during the fine-tuning process with random documents and golden documents using \texttt{Llama-2-7b-chat-hf} and \texttt{LLama-3-8B-Instruct} on the NQ dataset. Our findings reveal that training with random documents leads to superior performance, as evidenced by achieving a higher optimal F1 score. Additionally, we observe that the model reaches a high F1 score more rapidly when trained with random documents. This suggests that the random document strategy more effectively harnesses the model's inherent robustness and generalization capabilities, resulting in enhanced performance compared to the golden document approach.