%%
%% This is file `sample-sigconf-authordraft.tex',
%% generated with the docstrip utility.
%%
%% The original source files were:
%%
%% samples.dtx  (with options: `all,proceedings,bibtex,authordraft')
%% 
%% IMPORTANT NOTICE:
%% 
%% For the copyright see the source file.
%% 
%% Any modified versions of this file must be renamed
%% with new filenames distinct from sample-sigconf-authordraft.tex.
%% 
%% For distribution of the original source see the terms
%% for copying and modification in the file samples.dtx.
%% 
%% This generated file may be distributed as long as the
%% original source files, as listed above, are part of the
%% same distribution. (The sources need not necessarily be
%% in the same archive or directory.)
%%
%%
%% Commands for TeXCount
%TC:macro \cite [option:text,text]
%TC:macro \citep [option:text,text]
%TC:macro \citet [option:text,text]
%TC:envir table 0 1
%TC:envir table* 0 1
%TC:envir tabular [ignore] word
%TC:envir displaymath 0 word
%TC:envir math 0 word
%TC:envir comment 0 0
%%
%%
%% The first command in your LaTeX source must be the \documentclass
%% command.
%%
%% For submission and review of your manuscript please change the
%% command to \documentclass[manuscript, screen, review]{acmart}.
%%
%% When submitting camera ready or to TAPS, please change the command
%% to \documentclass[sigconf]{acmart} or whichever template is required
%% for your publication.
%%
%%
% \documentclass[sigconf,authordraft]{acmart}
\documentclass[10pt, format=manuscript,nonacm]{acmart}

\usepackage{booktabs}
\usepackage{tabularx}
\usepackage{xcolor,colortbl}
\usepackage{tabularx}
\definecolor{lightgray}{gray}{0.93}
\definecolor{slightgray}{gray}{0.98}
\definecolor{darkgray}{gray}{0.77}
\usepackage{makecell}
\usepackage{multirow}
\usepackage{pgf} 
\usepackage{threeparttable}

\usepackage{xparse}   % Advanced command definitions

% Defining the calculation command
\NewDocumentCommand{\calc}{mm}{
  \pgfmathparse{#1/#2*100}\pgfmathprintnumber[fixed,precision=2]{\pgfmathresult}
}

%%
%% \BibTeX command to typeset BibTeX logo in the docs
\AtBeginDocument{%
  \providecommand\BibTeX{{%
    Bib\TeX}}}

%% Rights management information.  This information is sent to you
%% when you complete the rights form.  These commands have SAMPLE
%% values in them; it is your responsibility as an author to replace
%% the commands and values with those provided to you when you
%% complete the rights form.
\setcopyright{acmlicensed}
\copyrightyear{2018}
\acmYear{2018}
\acmDOI{XXXXXXX.XXXXXXX}

%% These commands are for a PROCEEDINGS abstract or paper.
\acmConference[Conference acronym 'XX]{Make sure to enter the correct
  conference title from your rights confirmation emai}{June 03--05,
  2018}{Woodstock, NY}
%%
%%  Uncomment \acmBooktitle if the title of the proceedings is different
%%  from ``Proceedings of ...''!
%%
%%\acmBooktitle{Woodstock '18: ACM Symposium on Neural Gaze Detection,
%%  June 03--05, 2018, Woodstock, NY}
\acmISBN{978-1-4503-XXXX-X/18/06}


%%
%% Submission ID.
%% Use this when submitting an article to a sponsored event. You'll
%% receive a unique submission ID from the organizers
%% of the event, and this ID should be used as the parameter to this command.
%%\acmSubmissionID{123-A56-BU3}

%%
%% For managing citations, it is recommended to use bibliography
%% files in BibTeX format.
%%
%% You can then either use BibTeX with the ACM-Reference-Format style,
%% or BibLaTeX with the acmnumeric or acmauthoryear sytles, that include
%% support for advanced citation of software artefact from the
%% biblatex-software package, also separately available on CTAN.
%%
%% Look at the sample-*-biblatex.tex files for templates showcasing
%% the biblatex styles.
%%

%%
%% The majority of ACM publications use numbered citations and
%% references.  The command \citestyle{authoryear} switches to the
%% "author year" style.
%%
%% If you are preparing content for an event
%% sponsored by ACM SIGGRAPH, you must use the "author year" style of
%% citations and references.
%% Uncommenting
%% the next command will enable that style.
%%\citestyle{acmauthoryear}


%%
%% end of the preamble, start of the body of the document source.
\begin{document}
\title{Understanding Design Fixation in Generative AI}
\renewcommand{\shorttitle}{Design Fixation in GenAI}
%%
%% The "title" command has an optional parameter,
%% allowing the author to define a "short title" to be used in page headers.

%%
%% The "author" command and its associated commands are used to define
%% the authors and their affiliations.
%% Of note is the shared affiliation of the first two authors, and the
%% "authornote" and "authornotemark" commands
%% used to denote shared contribution to the research.
\author{Liuqing Chen}
\email{chenlq@zju.edu.cn}
\affiliation{%
  \institution{College of Computer Science and Technology, Zhejiang University}
  \city{Hangzhou}
  \country{China}
}
\author{Yaxuan Song}
\email{songyx23@zju.edu.cn}
\affiliation{%
  \institution{College of Computer Science and Technology, Zhejiang University}
  \city{Hangzhou}
  \country{China}
}

\author{Chunyuan Zheng}
\email{3200103594@zju.edu.cn}
\affiliation{%
  \institution{College of Computer Science and Technology, Zhejiang University}
  \city{Hangzhou}
  \country{China}
}

\author{Qianzhi Jing}
% \email{jingqz@zju.edu.cn}
\affiliation{%
  \institution{College of Computer Science and Technology, Zhejiang University}
  \city{Hangzhou}
  \country{China}
}

\author{Preben Hansen}
% \email{preben@dsv.su.se}
\affiliation{%
  \institution{Stockholm University}
  \city{Stockholm}
  \country{Sweden}
}

\author{Lingyun Sun}
% \email{sunly@zju.edu.cn}
\affiliation{%
  \institution{International Design Institute, Zhejiang University}
  \city{Hangzhou}
  \country{China}
}



%%
%% By default, the full list of authors will be used in the page
%% headers. Often, this list is too long, and will overlap
%% other information printed in the page headers. This command allows
%% the author to define a more concise list
%% of authors' names for this purpose.
\renewcommand{\shortauthors}{Trovato et al.}

%%
%% The abstract is a short summary of the work to be presented in the
%% article.
\begin{abstract}
Generative AI (GenAI) provides new opportunities for creativity support, but the phenomenon of GenAI design fixation remains underexplored. While human design fixation typically constrains ideas to familiar or existing solutions, our findings reveal that GenAI similarly experience design fixation, limiting its ability to generate novel and diverse design outcomes. To advance understanding of GenAI design fixation, we propose a theoretical framework includes the definition, causes, manifestations, and impacts of GenAI design fixation for creative design. We also conducted an experimental study to investigate the characteristics of GenAI design fixation in practice. We summarize how GenAI design fixation manifests in text generation model and image generation model respectively. Furthermore, we propose methods for mitigating GenAI design fixation for creativity support tool design. We recommend adopting the lens of GenAI design fixation for creativity-oriented HCI research, as the unique perspectives and insights it provides.

% To date, we know little about....
\end{abstract}

%%
%% The code below is generated by the tool at http://dl.acm.org/ccs.cfm.
%% Please copy and paste the code instead of the example below.
%%
\begin{CCSXML}
<ccs2012>
   <concept>
       <concept_id>10003120.10003121.10003126</concept_id>
       <concept_desc>Human-centered computing~HCI theory, concepts and models</concept_desc>
       <concept_significance>500</concept_significance>
       </concept>
   <concept>
       <concept_id>10003120.10003121.10011748</concept_id>
       <concept_desc>Human-centered computing~Empirical studies in HCI</concept_desc>
       <concept_significance>500</concept_significance>
       </concept>
   <concept>
       <concept_id>10010147.10010178</concept_id>
       <concept_desc>Computing methodologies~Artificial intelligence</concept_desc>
       <concept_significance>500</concept_significance>
       </concept>
 </ccs2012>
\end{CCSXML}

\ccsdesc[500]{Human-centered computing~HCI theory, concepts and models}
\ccsdesc[500]{Human-centered computing~Empirical studies in HCI}
\ccsdesc[500]{Computing methodologies~Artificial intelligence}

%%
%% Keywords. The author(s) should pick words that accurately describe
%% the work being presented. Separate the keywords with commas.
\keywords{Generative AI, design fixation, creativity, Human-AI interaction}
%% A "teaser" image appears between the author and affiliation
%% information and the body of the document, and typically spans the
%% page.


% \received{20 February 2007}
% \received[revised]{12 March 2009}
% \received[accepted]{5 June 2009}

%%
%% This command processes the author and affiliation and title
%% information and builds the first part of the formatted document.
\maketitle
\section{Introduction}

Video generation has garnered significant attention owing to its transformative potential across a wide range of applications, such media content creation~\citep{polyak2024movie}, advertising~\citep{zhang2024virbo,bacher2021advert}, video games~\citep{yang2024playable,valevski2024diffusion, oasis2024}, and world model simulators~\citep{ha2018world, videoworldsimulators2024, agarwal2025cosmos}. Benefiting from advanced generative algorithms~\citep{goodfellow2014generative, ho2020denoising, liu2023flow, lipman2023flow}, scalable model architectures~\citep{vaswani2017attention, peebles2023scalable}, vast amounts of internet-sourced data~\citep{chen2024panda, nan2024openvid, ju2024miradata}, and ongoing expansion of computing capabilities~\citep{nvidia2022h100, nvidia2023dgxgh200, nvidia2024h200nvl}, remarkable advancements have been achieved in the field of video generation~\citep{ho2022video, ho2022imagen, singer2023makeavideo, blattmann2023align, videoworldsimulators2024, kuaishou2024klingai, yang2024cogvideox, jin2024pyramidal, polyak2024movie, kong2024hunyuanvideo, ji2024prompt}.


In this work, we present \textbf{\ours}, a family of rectified flow~\citep{lipman2023flow, liu2023flow} transformer models designed for joint image and video generation, establishing a pathway toward industry-grade performance. This report centers on four key components: data curation, model architecture design, flow formulation, and training infrastructure optimization—each rigorously refined to meet the demands of high-quality, large-scale video generation.


\begin{figure}[ht]
    \centering
    \begin{subfigure}[b]{0.82\linewidth}
        \centering
        \includegraphics[width=\linewidth]{figures/t2i_1024.pdf}
        \caption{Text-to-Image Samples}\label{fig:main-demo-t2i}
    \end{subfigure}
    \vfill
    \begin{subfigure}[b]{0.82\linewidth}
        \centering
        \includegraphics[width=\linewidth]{figures/t2v_samples.pdf}
        \caption{Text-to-Video Samples}\label{fig:main-demo-t2v}
    \end{subfigure}
\caption{\textbf{Generated samples from \ours.} Key components are highlighted in \textcolor{red}{\textbf{RED}}.}\label{fig:main-demo}
\end{figure}


First, we present a comprehensive data processing pipeline designed to construct large-scale, high-quality image and video-text datasets. The pipeline integrates multiple advanced techniques, including video and image filtering based on aesthetic scores, OCR-driven content analysis, and subjective evaluations, to ensure exceptional visual and contextual quality. Furthermore, we employ multimodal large language models~(MLLMs)~\citep{yuan2025tarsier2} to generate dense and contextually aligned captions, which are subsequently refined using an additional large language model~(LLM)~\citep{yang2024qwen2} to enhance their accuracy, fluency, and descriptive richness. As a result, we have curated a robust training dataset comprising approximately 36M video-text pairs and 160M image-text pairs, which are proven sufficient for training industry-level generative models.

Secondly, we take a pioneering step by applying rectified flow formulation~\citep{lipman2023flow} for joint image and video generation, implemented through the \ours model family, which comprises Transformer architectures with 2B and 8B parameters. At its core, the \ours framework employs a 3D joint image-video variational autoencoder (VAE) to compress image and video inputs into a shared latent space, facilitating unified representation. This shared latent space is coupled with a full-attention~\citep{vaswani2017attention} mechanism, enabling seamless joint training of image and video. This architecture delivers high-quality, coherent outputs across both images and videos, establishing a unified framework for visual generation tasks.


Furthermore, to support the training of \ours at scale, we have developed a robust infrastructure tailored for large-scale model training. Our approach incorporates advanced parallelism strategies~\citep{jacobs2023deepspeed, pytorch_fsdp} to manage memory efficiently during long-context training. Additionally, we employ ByteCheckpoint~\citep{wan2024bytecheckpoint} for high-performance checkpointing and integrate fault-tolerant mechanisms from MegaScale~\citep{jiang2024megascale} to ensure stability and scalability across large GPU clusters. These optimizations enable \ours to handle the computational and data challenges of generative modeling with exceptional efficiency and reliability.


We evaluate \ours on both text-to-image and text-to-video benchmarks to highlight its competitive advantages. For text-to-image generation, \ours-T2I demonstrates strong performance across multiple benchmarks, including T2I-CompBench~\citep{huang2023t2i-compbench}, GenEval~\citep{ghosh2024geneval}, and DPG-Bench~\citep{hu2024ella_dbgbench}, excelling in both visual quality and text-image alignment. In text-to-video benchmarks, \ours-T2V achieves state-of-the-art performance on the UCF-101~\citep{ucf101} zero-shot generation task. Additionally, \ours-T2V attains an impressive score of \textbf{84.85} on VBench~\citep{huang2024vbench}, securing the top position on the leaderboard (as of 2025-01-25) and surpassing several leading commercial text-to-video models. Qualitative results, illustrated in \Cref{fig:main-demo}, further demonstrate the superior quality of the generated media samples. These findings underscore \ours's effectiveness in multi-modal generation and its potential as a high-performing solution for both research and commercial applications.
\section{Definition of the general method MeLoCoToN}
The core of this work is that every combinatorial problem has an explicit equation that returns its exact solution. In this section we will demonstrate how such an equation is obtained based on the use of tensor networks. This method, which we call \textit{Modulated Logical Combinatorial Tensor Networks} (MLCTN or MeLoCoToN), will consist of four steps:
\begin{enumerate}
    \item Definition of the problem variables and rewriting of the functions.
    \item Creation of the associated classical logical circuit.
    \item Creation of the associated logical tensor network.
    \item Iteration on the tensor network and contraction.
\end{enumerate}

The ideas presented can be sintetized in Fig.~\ref{fig: General Scheme}.

\begin{figure}[h]
    \centering
    \includegraphics[width=0.7\linewidth]{Images/General_Scheme.pdf}
    \caption{General scheme of ideas presented.}
    \label{fig: General Scheme}
\end{figure}

There are three general types of combinatorial problems that we can define. The first are \textit{inversion problems}, which consist in having a function $\gamma$ which associates one output combination to each input combination, and given a known output $\vec{Y}$, we search for the input $\vec{X}$ that generates it, $\vec{Y}=\gamma(\vec{X})$. An example would be the factorization of prime numbers. The second are the \textit{constraint satisfaction problems}, which consist in obtaining a solution that satisfies a set of constraints. An example is the N queens. The third are the \textit{optimization problems}, which consist in having a function which associates a cost to each input, these are not transformed, and obtaining the input with lower cost. An example is the knapsack problem. Both types can be considered combinatorial problems for the purposes of what we will discuss later.

\subsection{Quantum computing explanation of the method}
Before explaining the details of the tensor method, let us explain the quantum motivation for it, for those more familiar with quantum computing. Our method will consist in the simultaneous evaluation of all possible solutions, so that, when measuring the system, it will return with maximum probability the basis state that encodes the optimal solution. In the cases of inversion and constraint satisfaction problems, when measuring the system, only the correct solution can be obtained, while in the cases of optimization problems it can be obtained with higher probability.

For inversion problems, we have two registers. The measurement register, with the qudits we will measure, and the post-selection register, with the qudits we will operate and post-select. 
The system starts with uniform superposition on the qudits from both registers, making a Bell state between each qudit in each register.

\begin{equation}
    \ket{\psi_0} = \bigotimes_{k=0}^{N-1} \left(\sum_{x_k=0}^{d_k-1} \ket{x_k}_{m,k}\ket{x_k}_{p,k}\right)=\sum_{\vec{x}}\left(\ket{\vec{x}}_{m}\ket{\vec{x}}_{p}\right),
\end{equation}
where $\ket{x_k}_{m,k}$ is the state of the $k$-th measurement qudit and $\ket{x_k}_{p,k}$ is the state of the $k$-th post-selection qudit.

After that, we apply an oracle $\mathcal{T}$ on the post-selection register, so that for each base state its state becomes the output associated to the input of that state through the function $\gamma$ to be inverted.
\begin{equation}
    \ket{\psi_1} = \sum_{\vec{x}}\left( \ket{\vec{x}}_{m}\mathcal{T}\ket{\vec{x}}_{p}\right)=\sum_{\vec{x}} \left( \ket{\vec{x}}_{m}\ket{\gamma(\vec{x})}_{p}\right).
\end{equation}
Now, we post-select the state of the qudits in the second register, so that they are only in the state of the known output $\vec{Y}$ of the function we want to invert. This is a non-unitary operation, so we cannot perform it in a quantum system. In this way, in the measurement register, the only state $\ket{\vec{X}}$ that will remain is the one generated by the output that we have post-selected.
\begin{equation}
    \ket{\psi_2} = \left(\mathbb{I}\otimes \ket{\vec{Y}}\bra{\vec{Y}}\right)  \sum_{\vec{x}} \left( \ket{\vec{x}}_{m}\ket{\gamma(\vec{x})}_{p}\right) = \sum_{\vec{x}} \left( \ket{\vec{x}}_{m}\ket{\vec{Y}}\braket{\vec{Y}|\gamma(\vec{x})}_{p}\right) = \ket{\vec{X}}_{m}\ket{\vec{Y}}_p.
\end{equation}
Now, if we measure the first register, we can only get the correct solution.

For optimization and constraint satisfaction problems we only use one register, which starts in uniform superposition
\begin{equation}
    \ket{\psi_0} = \bigotimes_{k=0}^{N-1} \left(\sum_{x_k=0}^{d_k-1} \ket{x_k}\right)=\sum_{\vec{x}}\ket{\vec{x}}.
\end{equation}
After that, we apply an operator that performs an imaginary time evolution~\cite{ITE}, having as Hamiltonian the cost function for each combination
\begin{equation}
    \ket{\psi_1} = \sum_{\vec{x}}e^{-\tau C(\vec{x})}\ket{\vec{x}}.
\end{equation}
Finally, we apply an $\mathcal{R}$ operator that applies the constraints of the problem, cancelling the amplitude of the states that do not satisfy them
\begin{equation}
    \ket{\psi_2} = \sum_{\vec{x}}\mathcal{R}e^{-\tau C(\vec{x})}\ket{\vec{x}} = \sum_{\vec{x}\in R}e^{-\tau C(\vec{x})}\ket{\vec{x}},
\end{equation}
being $R$ the subspace of combinations satisfying the constraints. Again, these are non-unitary operations not directly implementable in a quantum system. After this, the basis state with less associated cost is the most probable state to measure.


\subsection{Definition of the problem variables and rewriting of the functions}

The first step is choosing which variables are going to be optimized to solve the problem, and rewrite it according to these variables. Many problems may have different sets of variables such that solving the problem in one set returns the same solution as solving it in another ones. For example, in the traveling salesman problem we can choose as variables to optimize the vector $\vec{y}$ whose component $y_k$ indicates the time step in which we are at node $k$, or the vector $\vec{x}$ whose component $x_t$ indicates in which node we are at time step $t$. Both formulations are equivalent, but the former allows us to express the cost function, variable transformations and dependencies in the problem in a much simpler way.

Once the variables $\vec{x}$ to optimize have been chosen, we must rewrite the problem in function to these variables. In case of inversion problems, it will be necessary to determine which operations are performed on the input variables to obtain the output variables values. For example, if the problem is to determine two numbers of a set that added result in a certain value, it is possible to take as variables the bits of each number and do the binary addition process until obtaining the output number, also in binary variables. Something similar happens in constraint satisfaction problems, but this time we need to determine when a combination is unfeasible.

In the optimization problems the cost function must be written as a function that receives the values $\vec{x}$ and returns a number. This can be expressed simply as
\begin{equation}
    C(\vec{x}) = C_{x_0,x_1,\dots,x_{N-1}},
\end{equation}
being $C(\cdot)$ the cost function and $C$ its associated cost tensor.

However, to simplify the posterior implementation of the tensor network, the cost function should be expressed as the least-number of variables dependent cost operation. An example is the cost given as a QUBO
\begin{equation}\label{eq: QUBO}
    C(\vec{x}) = \sum_{i,j}Q_{i,j}x_i x_j.
\end{equation}

Another example is the Tensorial Quadratic Unconstrained D-ary Optimization (T-QUDO) formulation, which consists in expressing the cost as a sum of tensor elements of two indexes, these being the values of the variables,
\begin{equation}\label{eq: T QUDO cost}
    C(\vec{x}) = \sum_{k} C_{k,x_{a_k},x_{b_k}},
\end{equation}
where $a_k$ and $b_k$ are the identifiers of the first and second variable involved in the $k$-th term and $\vec{x}$ is a vector of natural values.

In the case of the traveling salesman problem~\cite{TSP_General}, if the variables $x_t$ are defined as the node where we are at time $t$, it can be expressed as
\begin{equation}
    C(\vec{x}) = \sum_{t} C_{x_t,x_{t+1}}.
\end{equation}

Defined the cost function, it is necessary to define the constraints. There are many ways to define the constraints, but a convenient one is to use auxiliary variables that indicate the activation or not of a certain condition. For example, in the traveling salesman problem the constraints are to end at the same node where we start and not to repeat any node. This is
\begin{equation}
\begin{gathered}
    x_0=x_{N},\\
    x_t \neq x_{t'}\ \forall t\neq t'.
\end{gathered}
\end{equation}
Another way of expressing it with auxiliary variables is as follows
\begin{itemize}
    \item $y_r = x_0 \Rightarrow x_N=y_r$ for ending at the start.
    \item $\forall i,t$ for each node $i$ and time step $t$.
    \begin{itemize}
        \item if $i\in \{x_0, x_1, \dots, x_t\} \Rightarrow y_{i,t} = 1$
        \item else $y_{i,t} = 0$
        \item if $y_{i,t} =1 \Rightarrow x_{t'} \neq i\ \forall t'>t$
        \item else $\exists t'>t\ |\ x_{t'}=i$
    \end{itemize}
\end{itemize}
where $y_{i,t}$ takes into account if the node $i$ has being visited in some step up to the step $t$.




\subsection{Creation of the associated classical logical circuit}
Once we have determined the variables that we will use to solve the problem, we have to build a classical 
logical circuit of the problem. The type of circuit to build depends on the type of problem to solve. These circuits make use of what we call \textit{internal signals}, internal information of the circuit, which is not part of the output, comes from some operators and conditions the action of others who receive it. The internal signal is the problem relevant information that is sent between operators. It is the only information they need to perform their operations correctly, and depends directly or indirectly on the problem input. In addition, we can interpret the inputs and outputs that connect to the outside as \textit{external signals}. For this reason, the construction method is called the \textit{Signals Method}. To understand how it works, we will start with the construction for inversion problems, then the CSP and finally the optimization problems.

\subsubsection{Inversion Problem}
For an inversion problem, we have to make a circuit that implements the function to invert, receiving the inputs $\vec{x}$ and returning the corresponding output $\vec{y} = \gamma(\vec{x})$. This can always be done, as it is a known function, making use of a classical logic circuit that transforms the information it receives. This can be implemented by means of fundamental logical operators or by means of more complex ones. We call this circuits \textit{Logical Signal Transformation Circuits} (LSTC), since each operator transforms its input signals into output signals using logical rules. These circuits also serve to solve the problem of calculating $\gamma(\vec{x})$, which we call \textit{forward problems}. Let us give a few examples to make this class of circuits easier to understand.

\paragraph{Sum of two numbers in binary}
$ $

The problem is, given a number $c$, to determine two numbers $a$ and $b$ such that $c=a+b$. That is, we want to invert the addition function. To do this, we use as variables the bits of the numbers. In this way, we will have to build the LSTC that performs the binary addition function.
\begin{figure}[h]
    \centering
    \includegraphics[width=0.7\linewidth]{Images/ADD.pdf}
    \caption{LSTC to add two numbers $a$ and $b$ to obtain a number $c$.}
    \label{fig: ADD circuit}
\end{figure}

The way to build this circuit is shown in Fig.~\ref{fig: ADD circuit}, where each pair of bits, one of $a$ and the other of $b$, enters in each logical operator $ADDb$, which are in charge of doing the part of the global sum corresponding to those bits. If each binary sum is performed in an $ADDb$ operator, we can make them return both the modular sum of the bits and the carry for the sum of the following bits. This carry information is sent in what we call the \textit{internal signal}.

The $ADDb$ operator has three inputs, which are the three bits to be added, and two outputs. The first output is the function $f(x,y,z) =x\oplus y \oplus z$, the modular sum of the 3 bits, marked in orange. The second output is the function $g(x,y,z)=\left\lfloor\frac{x+y+z}{2}\right\rfloor$, which outputs the carry, marked in green. The first output is part of the circuit output, but the second is internal information that is part of the input of the next $ADDb$ operator.
This circuit uses minimal amounts of information to communicate between its parts to obtain the final output, in addition to being small in size. These are two properties necessary for the resulting tensor network to be computable.


\paragraph{Multiply two numbers in binary}
$ $

In this problem we have a number $c$ and we want to obtain two numbers $a$ and $b$ such that $c=a\times b$. To do this, we have to make the LSTC that generates the multiplication of two numbers. We generate it based on conditional binary sums. For this, we have two different internal signals. The first one is the main signal, which keeps track of how much we have added up to a certain point, and that will be the circuit output at the end, and the second signal in each sum keeps track of the carry and the condition. That is, the main signal starts with the value $0$, and the circuit adds to it $b$ if $a_0$ is equal to $1$. That is, it adds $a_0b$.
\begin{figure}
    \centering
    \includegraphics[width=0.7\linewidth]{Images/cADD.pdf}
    \caption{LSTC to add two numbers $a$ and $b$ to obtain a number $c$ if $w=1$.}
    \label{fig: cADD circuit}
\end{figure}

The $ADD$ circuit is similar as before, but now each operator has an extra input, which indicates if the addition is done or not. The $cADD$ circuit is in Fig.~\ref{fig: cADD circuit}. Thus, after doing the first conditional addition, the result is the input of another $cADD$ operator, which has to add the value $b$ multiplied by $2$, but this time if $a_1=1$. With this, the main signal is
\begin{equation}
    r_1 = a_0b + 2a_1b.
\end{equation}
Repeating this step $N$ times gives the signal
\begin{equation}
    r_{N-1} = \sum_{n=0}^{N-1} 2^n a_n b = a\times b=c.
\end{equation}

\begin{figure}
    \centering
    \includegraphics[width=0.7\linewidth]{Images/DOT.pdf}
    \caption{LSTC that performs the multiplication of two numbers $a$ and $b$.}
    \label{fig: DOT circuit}
\end{figure}
As we can see in the circuit in Fig.~\ref{fig: DOT circuit}, there is an \textit{intermediate state} that transforms until finally resulting in the final output. The intermediate state is a class of signal which is internal through the computation, and becomes external at the end of it.

\subsubsection{Constraint Satisfaction Problem}
In these problems the objective is that the circuit receives an input and returns the same value as output only if the combination satisfies the constraints. That is, if the input does not meet the constraints, we will not return any output. To do this, in addition to a set of internal and external signals, we have a value associated with the combination. This number is the \textit{amplitude} of the combination, which we will understand better in the section of optimization problems. Up to this point, it is only an internal number starting at $1$, and in case at some point it is detected that the input violates the constraints, an operator will change it to $0$.

Our circuit is composed of a set of operators that send a set of internal signals to each other, each one being in charge of analyzing a specific part of the input and detecting if any constraint is violated. We call this circuit \textit{Logical Signal Verification Circuits} (LSVC).

\paragraph{Single One Input}
$ $

This problem consists in finding a string of binary numbers such that only one of them is equal to $1$, and the rest are $0$.
\begin{figure}[h]
    \centering
    \includegraphics[width=0.7\linewidth]{Images/One_bit_circuit.pdf}
    \caption{LSVC to determine if a number has only one bit on $1$.}
    \label{fig: One bit}
\end{figure}

To do this, the input to the circuit is the value of each bit, and each one enters in an $ONE$ operator, which determines whether its value is $1$ and how many $1$ have appeared up to that point. They have two inputs and two outputs. The first input is the value of its bit, and the second is the signal indicating how many $1$ have appeared in the combination so far. The first output is the value of its bit, and the second is the signal that tells how many $1$ have appeared in the combination so far, including the current bit. If the operator receives that $1$ has already appeared in the combination, then it will make the amplitude $0$ if the input signal of its bit is also equal to $1$, since there cannot be two $1$ in the combination. The last operator forces that, if it receives that no $1$ has appeared, the amplitude of the combination is $0$ if it does not receive a $1$ on its bit. The circuit is shown in Fig.~\ref{fig: One bit}.



\subsubsection{Optimization Problem}
For optimization problems we have to change the approach slightly. In these problems we are not looking for an output nor an input which only satisfies constraints, but rather each input has an associated cost that we can calculate, but generally do not need to know. We want the state with the lowest associated cost, which satisfies the restrictions. Therefore, we will create a logical circuit whose input and output are the same, but which has a number associated with its state. We can visualize it as an optical circuit that can receive waves at discrete frequencies with a certain amplitude. Then, the circuit, depending on the frequency of that wave, changes its amplitude. Thus, the output of the circuit is a wave of the same frequency, but with a different amplitude. For example, if the circuit receives the value $x$, the output will also be $x$, but the internal value of the state will be $f(x)$. We call this internal value \textit{amplitude}, in analogy to quantum computing. It is important to note that this circuit is NOT a quantum circuit, nor does it work on superposition. It is a classical circuit that, depending on what it receives, amplifies or reduces the amplitude of the internal state. We call these circuits \textit{Logical Signal Modulation Circuits} (LSMC), since each operator transforms the internal signals it receives to modulate the amplitudes of the inputs according to logical rules. 

Due to the properties of the tensors that we will explain later, the changes in amplitude can only be multiplicative. That is, each operator can only multiply the amplitude by a number. This may seem restrictive, but it is sufficient to tackle any problem. Due to the types of existing problems and this restriction, we choose that given an input $\vec{x}$, which is a solution combination, the circuit multiplies its amplitude (initially $1$) by $e^{-\tau C(\vec{x})}$, being $\tau$ a constant. In this way, a combination with higher cost has an associated amplitude exponentially smaller than one with lower cost. This process is called \textit{imaginary time evolution}. This also allows taking advantage of the exponential property
\begin{equation}
    \prod_{i} e^{a_i}= e^{\sum_i a_i}.
\end{equation}
In case of constrained optimization problems, the circuit will also implement the LSVC logics of the constraint satisfaction problems. We will understand it better with three examples.


\paragraph{Linear function}
$ $

This combinatorial optimization problem has a cost function
\begin{equation}
    C(\vec{x})=\sum_{i=0}^{N-1} a_i x_i
\end{equation}
for a set of $a_i\in \mathbb{R}$ values, where $\vec{x}$ is a vector of binary values. The exponential of the cost function can be expressed as
\begin{equation}
    e^{-\tau C(\vec{x})} = \prod_{i=0}^{N-1} e^{-\tau a_i x_i}.
\end{equation}

\begin{figure}
    \centering
    \includegraphics[width=0.7\linewidth]{Images/Linear.pdf}
    \caption{LSMC that multiplies the amplitude of an input $\vec{x}$ by $e^{-\tau \sum_{i=0}^{N-1} a_i x_i}$.}
    \label{fig: Linear circuit}
\end{figure}

As each product only depends on one variable, its LSMC can be expressed as in Fig.~\ref{fig: Linear circuit}. If we multiply the amplitude of each part of the input by a value, the amplitude of the global input is multiplied by the product of all these values. This property allows the circuit to make the amplitude contain information from all the input without the need to transmit all the information at the same time. In this case we have not needed signals between operators as in the case of the addition, but we are going to see a more complicated case.


\paragraph{Quadratic function with a single neighbor in a linear chain}
$ $

This combinatorial optimization problem has a cost function
\begin{equation}
    C(\vec{x})=\sum_{i=0}^{N-1} (Q_{i,i} x_i^2 + Q_{i,i+1} x_ix_{i+1}).
\end{equation}
As before, the exponential can be expressed as products of exponentials. In this case, each operator needs, in addition to the information of the variable that corresponds to it, the value of the previous variable in the chain. The signal that each operator gives to the next one is its variable state.
\begin{figure}
    \centering
    \includegraphics[width=0.7\linewidth]{Images/Quadratic_linear_one_neighbor.pdf}
    \caption{LSMC that multiplies the amplitude of an input $\vec{x}$ by $e^{-\tau \sum_{i=0}^{N-1} (Q_{i,i} x_i^2 + Q_{i,i+1}x_ix_{i+1})}$.}
    \label{fig: Quadratic one neighbor circuit}
\end{figure}

The LSMC is given by Fig.~\ref{fig: Quadratic one neighbor circuit}. As noted above, this circuit is simple, where each functional part depends on few elements, and has a small size.


\paragraph{Natural sum total function problem}
$ $

The combinatorial optimization problem has a cost function
\begin{equation}
    C(\vec{x})=f\left(\sum_{i=0}^{N-1} a_i x_i\right),
\end{equation}
where $a_i\in \mathbb{N}$ and $f(\cdot)$ is some known function. In this case, we can use as a signal the sum $\sum_{i=0}^{m-1} a_i x_i$ up to the $m$-th variable, so that the last operator only does the evolution on the application of $f(\cdot)$ on the signal. This makes the last one operator the only one that multiplies the amplitude. Each previous operator only adds to the signal a value $a_i$ multiplied by its input. In this way, the LSMC is given by Fig.~\ref{fig: Natural sum circuit}.

\begin{figure}
    \centering
    \includegraphics[width=0.7\linewidth]{Images/Natural_sum.pdf}
    \caption{LSMC that multiplies the amplitude of an input $\vec{x}$ by $e^{-\tau f\left(\sum_{i=0}^{N-1} a_i x_i\right)}$.}
    \label{fig: Natural sum circuit}
\end{figure}


\subsection{Creation of the associated logical tensor network}
After understanding the three types of circuits, we have to build the tensor network associated with these circuits. The key of this tensor network is that, given its structure, it is possible to take all possible inputs at once, and return all possible corresponding outputs, with their associated amplitudes. This is because, when tensorizing, the inputs and outputs of the operators become the basis states of tensors. This allows to apply a superposition of all possible inputs, generating their corresponding outputs, and thus propagating the signals through the circuit by means of their entanglement.

Since we have all the possible outputs, in the inversion problems we only have to force the output to be the one we know. That is, having the circuit, we  put as `inverse input' the desired output, making the input of the circuit only the one which generates that output. This will result in a tensor in which the only non-zero elements are in the input values we are looking for. In cases of constraint satisfaction, we have a tensor network that represents a diagonal tensor. The only non-zero elements of this tensor will be those that satisfy the imposed constraints. In the case of optimization problems we have a similar phenomenon. As we have all the possible inputs with their amplitudes associated to their costs, we have a diagonal tensor where each element has the amplitude of that combination, so we only have to look for the one with the highest amplitude.

In both cases, the only thing we have to do is to change each operator of the circuit by a tensor with as many indexes as inputs and outputs the operator has. The values of the output indexes for the non-zero elements depend on the values of the input indexes, following the equations of the outputs of the associated operator. The values of the non-zero elements are the amplitudes of the corresponding operators, which in the case of inversion and contraint satisfaction are always 1. The inputs to the circuit are converted for each of the variables into a vector of ones, which will express the uniform superposition of all possible values of that variable. All tensors are connected to each other in exactly the same way as in the associated circuit, by the same indexes in the same way. We call this process \textit{Circuit Tensorization} (CT), and the resulting tensor network is the \textit{Tensor Logical Circuit} (TLC).

Translated into equations, it means that if we have an operator $U$ with 3 inputs $x,y,z$ and 2 outputs $\mu, \nu$, calculated as $\mu=f(x,y,z),\ \nu=g(x,y,z)$, which multiplies the amplitude of the state by $h(x,y,z)$, then its associated tensor $U$ has as non-zero elements those that satisfy
\begin{equation}
    \begin{gathered}
        \mu=f(x,y,z),\ \nu=g(x,y,z),\\
        U_{x,y,z,\mu,\nu}= h(x,y,z).
    \end{gathered}
\end{equation}
We call this process \textit{Input-Output Indexing} (IOI). As we can see, the relation between the inputs and outputs of each operator implies a constraint on the tensor representing it.

It is important to note that at this point, after creating the TLC of the problem, we are not yet going to contract it. This is because if we were to contract it at this point, we would have a tensor that collects the amplitude for all possible combinations, which is not our objective. The tensor network to be contracted is the one mentioned in the next subsection. We now present a few examples of CT.


\subsubsection{Inversion Problems}
In this case, each tensor transforms the position of the elements of the input $\vec{x}$ to the positions of the elements of the output $\vec{y}$, but not its amplitude. That is, we apply a function on the indexes of the tensor, and not on the values of the elements themselves. There is a function $\gamma$, which we do not need to know explicitly, represented through the LSTC, which receives the input and returns the output. That is, $\vec{y}=\gamma(\vec{x})$. Thus, the tensor network that replaces the circuit is such that, when contracted, it results in a $T$ tensor of elements $T_{x_0,x_1,x_2,\dots, y_0, y_1, y_2, \dots}=1$ only if $\vec{y}=\gamma(\vec{x})$, for all possible values of $\vec{x}$, and otherwise equals $0$. 

In order to force the output to be the correct one $\vec{Y}$ to get the correct input $\vec{X}$, we have to project the tensor on the subspace on that the last indexes of it are $\vec{y}=\vec{Y}$. This is equivalent to performing the contraction operation
\begin{equation}
    T'_{x_0,x_1,x_2,\dots} = \sum_{\vec{y}} T_{x_0,x_1,x_2,\dots, y_0, y_1, y_2, \dots}\delta^{Y_0}_{y_0}\delta^{Y_1}_{y_1}\delta^{Y_2}_{y_2}\dots
\end{equation}
being $\delta^{b}$ a vector of all zero elements except one equals to $1$ at position $b$. To do this, we put in each $k$-th output line a $\delta^{Y_k}$ vector. This causes the only non-zero element of the $T'$ tensor to be $T'_{X_0,X_1,X_2,\dots}=1$, for all the inputs $\vec{X}$ that generate the output $\vec{Y}$.



\paragraph{Sum of two numbers in binary}
$ $

In this case, each operator $ADDb$ must be replaced by its corresponding tensor. The TLC, and the names of the indexes are shown in Fig.~\ref{fig: TN ADD}. Note that the first tensor is different from the others, since it has no index to tell the previous carried, since it is always 0. In this case, the initial $ADDb0$ tensor is a 4-index tensor of $2\times 2\times 2\times 2$ dimensions with non-zero elements $ADDb0_{ij\mu\nu}=1$ when the following is true
\begin{equation}
\begin{gathered}
    \mu = f(i,j,0) =i\oplus j,\\
    \nu = g(i,j,0) = \left\lfloor\frac{i+j}{2}\right\rfloor.
\end{gathered}
\end{equation}

For the rest of the tensors $ADDb$, these have 5 indexes of $2\times 2\times 2\times 2 \times 2$ dimensions, whose non-zero elements $ADDb_{ijk\mu\nu}=1$ are those that satisfy
\begin{equation}
\begin{gathered}
    \mu = f(i,j,k) =i\oplus j \oplus k,\\
    \nu = g(i,j,k) = \left\lfloor\frac{i+j+k}{2}\right\rfloor.
\end{gathered}
\end{equation}

Finally, the vectors $c_b$ are those with their non-zero element at position $c_b$. 
\begin{figure}
    \centering
    \includegraphics[width=0.7\linewidth]{Images/ADD_TN.pdf}
    \caption{TLC of Fig.~\ref{fig: ADD circuit} and its index correspondence for the tensors.}
    \label{fig: TN ADD}
\end{figure}

If we contract this tensor network, the resulting tensor will have its nonzero elements in the positions in which its indexes are those corresponding to the bits of the $a$ and $b$ that generate as a result the $c$ that we have imposed.

\subsubsection{Constraint Satisfaction Problem}
In this case, the tensor network represents a diagonal tensor, in which all non-zero elements are equal to 1 when its indexes indicate a solution compatible with the constraints. Since the tensor is diagonal, we can eliminate half of the indexes, which are going to be a repetition of the other half, so we add a set of ones vectors in each output. This tensors are called the \textit{Plus Vectors} or `+' tensors. Thus, the tensor represented by the tensor network is
\begin{equation}
    T_{x_0,x_1,x_2,\dots} = 1\ \forall \vec{x} \in R.
\end{equation}

\paragraph{Single One Input}
$ $

In this case, each $ONE$ operator is replaced by the corresponding $ONE$ tensor to obtain the tensor network in Fig.~\ref{fig: One bit TN}. The non-zero elements of these tensors are
\begin{equation}
\begin{gathered}
    \mu = \nu =i,\\
    ONE(0)_{i\mu\nu}=1,
\end{gathered}
\end{equation}
\begin{equation}
\begin{gathered}
    \text{if } j=0\Rightarrow \nu = i,\quad\text{ else } \nu=j,\ i=0,\\
    \mu =i,\\
    ONE(k)_{ij\mu\nu}=1,
\end{gathered}
\end{equation}
\begin{equation}
\begin{gathered}
    i = j\oplus 1,\quad\mu =i,\\
    ONE(N-1)_{ij\mu}=1.
\end{gathered}
\end{equation}


\begin{figure}
    \centering
    \includegraphics[width=0.7\linewidth]{Images/One_bit_TN.pdf}
    \caption{TLC of Fig.~\ref{fig: One bit} for the Single One Input problem.}
    \label{fig: One bit TN}
\end{figure}




\subsubsection{Optimization problem}
As before, in optimization problems we do not want to impose a specific output. Our first objective is to create a tensor whose non-zero elements are those in which the input equals the output and whose values are the exponentials of their associated costs. That is, to perform an imaginary time evolution for a diagonal operator. After that, as the input and output information is the same, resulting redundant, we add a set of Plus Vectors in each output. This is equivalent to allowing any output, so nothing change in the problem. Each tensor that replaces an operator has to perform the multiplication of the amplitude that the operator performed. To do this, the element of the tensor associated to the corresponding input and output indexes has to have the value by which the operator multiplied the amplitude. As the contractions multiply the values of the tensor elements, this do the process we imposed on the circuit.

By doing this, the tensor $T$ represented by this tensor network has elements
\begin{equation}
    T_{x_0,x_1,x_2,\dots} = e^{-\tau C(\vec{x})}.
\end{equation}
In case of having constraints in the problem, we only have to add after the optimization circuit, a constraint circuit, so that the tensor represented will be
\begin{equation}
    T_{x_0,x_1,x_2,\dots} = e^{-\tau C(\vec{x})} \ \forall \vec{x} \in R.
\end{equation}

\paragraph{Linear problem}
$ $

In this case, we only have matrices of dimension $2\times 2$, represented in Fig.~\ref{fig: Linear TN}, whose non-zero elements are
\begin{equation}
\begin{gathered}
    \mu=i,\\
    EXPi(a,n)_{i,\mu} = e^{-\tau a_n i}.
\end{gathered}
\end{equation}

\begin{figure}
    \centering
    \includegraphics[width=0.7\linewidth]{Images/Linear_TN.pdf}
    \caption{TLC of Fig.~\ref{fig: Linear circuit} and its index correspondence for the tensors.}
    \label{fig: Linear TN}
\end{figure}


\paragraph{Quadratic function with a single neighbor in a linear chain}
$ $

Unlike the previous case, now we have tensors of 3 and 4 indexes, divided into 3 groups: the initial tensor, the intermediate tensor and the final tensor. The tensor network is represented in Fig.~\ref{fig: Quadratic TN}. All have dimension $2$ in their indexes, with non-zero elements
\begin{equation}
    \begin{gathered}
        \mu=\nu=i,\\
        EXPi(Q,0)_{i\mu\nu}=e^{-\tau Q_{00}i},\\
        EXPi(Q,n)_{ij\mu\nu}=e^{-\tau (Q_{n,n}i^2+Q_{n-1,n}ij)},
    \end{gathered}
\end{equation}
\begin{equation}
    \begin{gathered}
        \mu=i,\\
        EXPi(Q,N-1)_{ij\mu}=e^{-\tau (Q_{N-1,N-1}i^2+Q_{N-2,N-1}ij)}.
    \end{gathered}
\end{equation}
\begin{figure}
    \centering
    \includegraphics[width=0.7\linewidth]{Images/Quadratic_linear_one_neighbor_TN.pdf}
    \caption{TLC of Fig.~\ref{fig: Quadratic one neighbor circuit} and its index correspondence for the tensors.}
    \label{fig: Quadratic TN}
\end{figure}


\paragraph{Natural sum total function problem}
$ $

This case is similar to the previous one, but now the tensors have for their upper and lower indexes a dimension that allows sending any of the possible partial sums. Therefore, the $n$-th tensor has for its upper index a dimension of $\sum_{i=0}^{n-1}a_i$, while for the lower one a dimension of $\sum_{i=0}^{n}a_i$, and for the side ones a dimension of $2$. In the tensor network expressed in Fig.~\ref{fig: Natural sum TN} we have 3 types of tensors, whose non-zero elements are
\begin{equation}
    \begin{gathered}
        \mu=i,\quad \nu=a_0 i,\\
        EXPi(Q,0)_{i\mu\nu}=1,
    \end{gathered}
\end{equation}
\begin{equation}
    \begin{gathered}
        \mu=i,\quad \nu=j+a_n i,\\
        EXPi(Q,n)_{ij\mu\nu}=1,
    \end{gathered}
\end{equation}
\begin{equation}
    \begin{gathered}
        \mu=i,\\
        EXPi(Q,N-1)_{ij\mu}=e^{-\tau f(j+a_{N-1}i)}.
    \end{gathered}
\end{equation}

\begin{figure}
    \centering
    \includegraphics[width=0.7\linewidth]{Images/Natural_sum_TN.pdf}
    \caption{TLC of Fig.~\ref{fig: Natural sum circuit} and its index correspondence for the tensors.}
    \label{fig: Natural sum TN}
\end{figure}

\subsection{Iteration and contraction of the tensor network}
Once we have the tensor network that gives us the result tensor we search, we need to extract the relevant information from it without having to store it in memory. That is, we want to somehow be able to look at a reduced version of it that gives us the information we really need to get the solution. To do this, we are going to determine the correct value of each variable iteratively. Therefore, in each iteration we only want to know what is the value of the $n$-th variable in the optimal combination. To do this we  perform an integral over all the other variables, so that we only have to find the maximum in a vector of exponentially smaller dimension, and that information is included in the next iteration. We call this process \textit{Half Partial Trace}.

This can be visualized as a process similar to that performed when measuring a quantum state. We measure the qubits in order, so that the state of all the qubits that remain to be measured conditions the probability of obtaining a result on the one we are measuring, and the result of the already measured fix and alter the following probabilities in a fixed way. In our case, we do not use the amplitudes in the same way as in quantum mechanics, but the amplitudes are our `probabilities'. Instead of using the density matrix $\rho_{AB}$ of the $|\psi_{AB}\rangle$ state and performing an operation $P_{A,i} = Tr_{B}(\rho_{AB} M_i)$, we apply directly $P_{A,i} = \langle i,+^{\otimes N-1}|\psi_{AB}\rangle$. In~\cite{Escanez_Notation} simplified notation, $P_{A,i} = {}^{0}_{i}\langle +|\psi_{AB}\rangle$.

With this in mind, we will see it more clearly first in the inversion case and then in the optimization case.

\subsubsection{Inversion problem}
For simplicity, we begin by addressing the case in which there is only one solution to the problem. In this case, there is only one input $\vec{X}$ that results in that output $\vec{Y}$. This implies that there is only one non-zero element in the tensor represented by the TLC we have constructed, whose indexes give us the solution. Therefore, we can create a vector of dimension equal to that of the first index of the tensor, that is, of the number of possible values of the first variable. In the $k$-th component of this vector we store the sum of all the elements of the tensor whose first index has the value $k$. Although this operation seems computationally very expensive, it can be performed by putting a Plus Vector in all the indexes except the first one. Since there is only one correct solution, there is only one non-zero element in the tensor, so only one of these sums has a non-zero summand. This makes the position of the non-zero element of the vector match the value of the first index for the non-zero element of the global tensor.

Therefore, to determine the first variable we create the TLC and we connect Plus Vectors in all the indexes except the first one, and the value of the correct variable is the position with a 1 in the vector. To determine the second variable we can do exactly the same, but this time putting Plus Vectors in all the indexes except the second one. We can repeat this process for each variable and we will obtain the solution to the problem.

To generalize to problems in which there are several solutions, we have to take into account that the tensor may have more than one non-zero value. However, our reasoning is the same. In case we want one of the solutions, we only have to take in each iteration as the correct value of the variable the position of any of the non-zero components of the vector, since it indicates that there is a solution that has that value for that variable. One possibility is to choose the position with the largest element. However, since there are several solutions, performing the process exactly the same can lead to mixing several degenerated solutions, resulting in an incorrect result. To avoid this problem, we only need to introduce in each iteration the information of all previous results. That is, in iteration $n$ we have already determined the previous $n-1$ variables, so we can project the tensor to the solutions with the first $n-1$ indexes equal to the values already determined. We can do this just as in the projection onto the output, with vectors having only one non-zero element at the position corresponding to the value we want to impose. We call this \textit{Projection Vectors}. In this way we do not mix solutions. An example of the operation of this method is shown in Fig.~\ref{fig: Inversion iteration}.

\begin{figure}
    \centering
    \includegraphics[width=\linewidth]{Images/Inversion_TN_Iteration.pdf}
    \caption{Iterative method for the determination of the solution variables in an inversion problem, for a chain-type tensor network.}
    \label{fig: Inversion iteration}
\end{figure}

In case the variables are binary, instead of evaluating a vector we can evaluate a scalar. If we have only one solution, there are only two possible vectors for a variable: $(1,0)$ if the value has to be $0$ or $(0,1)$ if the value has to be $1$. Therefore, we can connect to the variable we are going to measure a vector $(-1,1)$, which will subtract to the amplitude associated to the value $1$ the amplitude associated to the value $0$. We call this vector \textit{Minus Vector}. If the correct value is $1$, the resulting scalar will be $1$, while if it is $0$, the scalar will be $-1$. In case of having several solutions, we choose $1$ if the scalar is positive or $0$ and $0$ if it is negative.

In the case of not being binary variables, we can perform the same process if we binarize the variable to be determined at the beginning of the circuit. That is, if we do a splitting of the free index so that we have $\log_2(d)$ binary indexes, we can determine the correct value of each of them as if they were different variables. We can also do the equivalent by placing a vector that performs the corresponding additions and subtractions between the groups.
Therefore, the resolution can always be expressed as an equation of the type
\begin{equation}
    x_i = H(\Omega_i),
\end{equation}
being $\Omega_i$ the scalar resulting from contracting the tensor network of the $i$-th variable, and $H$ the Heaviside step function. This way, if $\Omega_i<0$, $x_i=0$, else $x_i=1$. As the value to be introduced in the next iteration of the tensor network will be dependent on the previous one, through the projection tensor to the obtained result, the equation that determines the solution of the combinatorial problem (both the inversion problem and any other) is given by a nesting of Heaviside step functions within tensor networks. The constraint satisfaction problems follow the same mechanism.


\subsubsection{Optimization problems}
For the optimization cases the process is the same as described above, but with a completely different motivation. To begin with, we have to visualize the amplitude map for all possible combinations. Since we have chosen as the amplitude for each combination the negative exponential of its cost multiplied by a constant, the combination with the lowest cost will have the largest amplitude. If we increase the value of $\tau$, the amplitudes of the suboptimal combinations will decrease exponentially faster than the amplitude of the optimal combination. Thus, in the limit $\tau\rightarrow\infty$, if we renormalize the tensor by dividing each element by the sum of all the elements of the tensor, only the amplitude of the optimal combination $\vec{X}$ will remain, exactly as in the inversion case. The limit is
\begin{equation}
    \lim_{\tau\rightarrow\infty} \sum_{\vec{x}\in R}\frac{e^{-\tau C(\vec{x})}}{\sum_{\vec{x}\in R}e^{-\tau C(\vec{x})}}\ket{\vec{x}}=
    \lim_{\tau\rightarrow\infty} \sum_{\vec{x}\in R}\frac{e^{-\tau C(\vec{x})}}{e^{-\tau C(\vec{X})}}\ket{\vec{x}}=
    \lim_{\tau\rightarrow\infty}\sum_{\vec{x}\in R}e^{-\tau (C(\vec{x})-C(\vec{X}))}\ket{\vec{x}} = \ket{\vec{X}}.
\end{equation}
In case of having degeneracy, we will have the case of several solutions and we solve it as presented for inversion problems. The method is presented in Fig.~\ref{fig: Diagonal iteration}.
\begin{figure}
    \centering
    \includegraphics[width=\linewidth]{Images/Diagonal_TN_Iteration.pdf}
    \caption{Iterative method for the determination of the solution variables in an constraint satisfaction and optimization problem, for a chain-type tensor network.}
    \label{fig: Diagonal iteration}
\end{figure}

However, it is not necessary to go to the infinite limit to extract information. For a sufficiently large finite value of $\tau$, the peak amplitude in the optimal combination will be large enough so that, when summing over the other variables values to obtain the vector of amplitudes of the variable we want to determine, this amplitude will be greater than the sum of all the suboptimals. This can be seen analogously to the case of measuring a quantum system in which there is a basis state with a probability much higher than the other basis states. Since the probability of measuring that state is higher, when measuring the first qubit of the system it will be more likely to measure the first bit of the state of maximum probability, and so on with all the qubits. Therefore, with a sufficiently large value of $\tau$, this same procedure is valid if we choose in each iteration as correct the position with the largest element. However, in case of not taking a sufficiently large $\tau$ value, the suboptimal states for an incorrect value of the variable to be determined may have a sum of amplitudes greater than that of the correct value.

To dampen this error, we can replace the sum by a complex sum, changing the Plus Vectors to vectors of complex numbers of unit modulus. We call them the \textit{Phase Vectors}. In this way, the `noise' generated by the amplitudes of the suboptimal ones will not sum in the same direction. Since the amplitude of the optimal combination is always the highest, it is the dominant value in their sum. If we distribute the phases of the numbers evenly, we decrease the probability that all the suboptimal ones are in the opposite phase to the optimum, so they will cancel each other. We call this method \textit{Humbucker}, and it was presented for the first time in~\cite{QUBO_Tridiagonal}.

\subsection{General equation}
In the previous subsections we have presented the method and demonstration of how to create the tensor network that solves the problem, and extracting the solution from it. Now, we can formulate the exact equation that solves any combinatorial problem.

\begin{theorem}
    Given a combinatorial problem, be it an inversion problem, a constraint satisfaction problem or an optimization problem, there is an exact explicit equation for its solution (or solutions).
\end{theorem}

\begin{theorem}
    Given a combinatorial problem, be it an inversion problem, a constraint satisfaction problem or an optimization problem, the exact explicit equation that solves it can be obtained in a polynomial time with respect to the time needed to formulate it.
\end{theorem}

\begin{theorem}
    Due to the symmetries of tensor networks, there are infinitely many equations that solve a combinatorial problem, all equivalent to each other.
\end{theorem}

This equation is obtained by following the steps listed in this section. The steps are
\begin{enumerate}
    \item Choose a set of variables $\vec{x}$ for the problem, which will encode the solution, input and/or output. For simplicity, and without loss of generality, the variables are considered to be binary, although it is general for natural variables.

    \item Build the logical circuit corresponding to the problem. If it is an inversion problem, the LSTC, if it is a constraint satisfaction problem, the LSVC, and if it is an optimization problem, the LSMC. 

    \item Tensorize the logical circuit to obtain the TLC of the problem.

    \item Perform the half partial trace of the TLC, adding the corresponding Minus Vector for the first variable, which we will call $x_0$. This tensor network is equal to the scalar value ${\color{blue} \Omega_0}$. Thus, the correct value of the first variable is
    \begin{equation}\label{eq: first variable}
        x_0={\color{blue} H(\Omega_0)}.
    \end{equation}

    \item To determine the next variable, the half partial trace of the TLC is made by adding now the corresponding Minus Vector for the second variable. This tensor network is a function of the value already determined for $x_0$, since the input index for that variable now has a projection vector at that value. Therefore, the tensor network has a $\delta^{x_0}$ tensor of $\delta^{x_0}_{i}$ components, so all components are zero except for the $x_0$-th component. If we substitute the expression~\ref{eq: first variable}, the tensor is $\delta^{{\color{blue} H(\Omega_0)}}$. The traced TLC in this case is equal to the value ${\color{teal}\Omega_1}$, which, depending on the value of $x_0$, can be expressed as ${\color{teal}\Omega_1(}x_0{\color{teal} )}$, which by substituting the expression~\ref{eq: first variable} becomes ${\color{teal}\Omega_1(}{\color{blue} H(\Omega_0)}{\color{teal} )}$. Thus, the value of the second variable is
    \begin{equation}
        x_1 = {\color{teal} H(\Omega_1(}{\color{blue} H(\Omega_0)}{\color{teal} ))}.
    \end{equation}
    
    \item The third TLC depends on the value of the two previous ones for the same reason as in the previous step, so it is a function
    \begin{equation}
    \Omega_2(x_0,x_1)=\Omega_2({\color{blue} H(\Omega_0)},{\color{teal} H(\Omega_1(}{\color{blue} H(\Omega_0)}{\color{teal} ))}).
    \end{equation}
    This way,
    \begin{equation}
        x_2 = H(\Omega_2({\color{blue} H(\Omega_0)},{\color{teal} H(\Omega_1(}{\color{blue} H(\Omega_0)}{\color{teal} ))})).
    \end{equation}

    \item The correct value of the $n$-th variable is
    \begin{equation}
        x_n = {\color{red}H( \Omega_n(}{\color{blue} H(\Omega_0)},{\color{teal} H(\Omega_1(}{\color{blue} H(\Omega_0)}{\color{teal} ))}{\color{red},\dots,}{\color{violet} H(\Omega_{n-1}}({\color{blue} H(\Omega_0)},{\color{teal} H(\Omega_1(}{\color{blue} H(\Omega_0)}{\color{teal} ))}{\color{violet},\dots ))}{\color{red} ))}.
    \end{equation}
    Since each variable ultimately depends on the initial tensor network, we can say that its value is actually given by a function $\Xi_n$, so that $x_n=\Xi_n({\color{blue}\Omega_0})$.
    \item The solution of the problem always can be expressed as
    \begin{equation}
        \vec{x}=(\Xi_0({\color{blue}\Omega_0}), \Xi_1({\color{blue}\Omega_0}), \Xi_2({\color{blue}\Omega_0}),\dots, \Xi_{N-1}({\color{blue}\Omega_0})).
    \end{equation}
\end{enumerate}

Since the construction of the tensor network is as fast as the construction of the logic circuit, and this can be done in a polynomial time with respect to the formulation of the problem, the equation can be obtained in a polynomial time with respect to the formulation.

Moreover, since between two tensors one can always place an $A$ matrix and its inverse and have the same tensor represented, and the number of possible invertible $A$ matrices is infinite, then there are infinitely many possible TLC, and therefore infinitely many equations for each problem.


Since we already understand how the general method works to obtain the tensor network, and therefore, the equation that solves any combinatorial problem, either inversion, constraint satisfaction or optimization one, we will present a wide range of examples of the application of this method to problems. Due to the large number of examples, many of them with shared concepts, we will explain their logics in a more or less superficial way, providing the form of the tensors and tensor networks to be constructed with examples. In this way, we can be sure to explain the cases and their generalizations without overextending ourselves. In case it generates a lot of interest, concrete cases can be dealt with in greater depth in future versions. It is important to note that most of these methods have not been thoroughly investigated in an attempt to obtain either the simplest formulation or the most efficiently computable tensor network.
\section{Understanding GenAI design fixation in practice}
\label{section_experiment}
\subsection{The aim of this study}
As we have proposed the definition of GenAI design fixation in Section~\ref{section_definition}, the aim of this experimental study is to gain a practical understanding of the performances of GenAI design fixation within the human-GenAI co-ideation process. Given that this is the first instance where the lens of GenAI design fixation is applied to examine the human-GenAI co-creation process, and considering the distinct research questions concerning fixation and creativity, we conduct an experimental study to gather empirical evidence on GenAI design fixation and users' perspectives on this phenomenon. Our expected results are as follows:
\begin{itemize}
    \item \textbf{RQ1: What are the manifestations of GenAI design fixation in text generation and image generation models?}
    \item \textbf{RQ2: How do participants identify and describe design fixation in Generative AI?}
    %找个文献支持这个观点
    \item \textbf{RQ3: What impacts does GenAI design fixation have on the ideation process during human-GenAI collaboration?}
\end{itemize}

%we seek to understand how designers perceive AI’s flawed output and how AI “error” affects designer-AI co-creation

% \subsection{Preconsiderations of the experiment}
% As concluded by \cite{crilly2017next}, most of the empirical studies have focused on industrial or mechanical design, we assigned an unbounded design task to participants, which is a common tradition in design fixation research \cite{crilly2017next}. 

\subsection{Participants}

Participants were recruited for this study through social media and word-of-mouth, targeting novice designers with experience in product design and an interest in the application of Generative AI in design fields. We focused on novice designers because they represent a demographic that is still forming their design habits and are likely more open to integrating new technologies such as Generative AI into their workflow. This recruitment strategy ensured that all participants have not only a basic understanding of the methods interacting with GenAI, but also a keen interest in sharing their experience and perspectives on ideation processes facilitated by GenAI. Ultimately, ten participants were successfully recruited, consisting of four males and six females aged between 20 and 26 years (Mean = 22.7, SD = 3.01). They come from diverse academic backgrounds, including undergraduate, master’s, and doctoral students. Additionally, one participant works in a GenAI practice organization. Participants received a commemorative gift as a token of appreciation for completing the study tasks.

\subsection{Procedure}
\label{experiment_procedure}
In this study, two researchers conducted one-on-one sessions with participants to observe each individual’s interaction with Generative AI, focusing on their ability to recognize the presence of GenAI design fixation. The scheduling of these experimental sessions was based on the availability of the participants. Participants were compensated for their time. The experimental process that each participant underwent is detailed in Figure \ref{fig:experiment}.

At the start of the study, researchers explained the experimental procedure, gathered informed consent, and collected demographic information from the participants (as shown in Table~\ref{tab:participants}). Then, participants were briefed on the design task which involved creating as innovative a design as possible under the assistance of two different Generative AI tools. Additionally, we introduced CombinatorX, a representative GenAI-assisted design method \cite{chen2024foundation}, which utilizes combinational creativity—merging two concepts into a novel idea \cite{boden2004creative}. The CombinatorX process involves identifying additive embodiments and forming textual combinational ideas, followed by using text-to-image technology to visualize these ideas. It is necessary to clarify that while the CombinatorX method was provided as a reference, its use was not mandatory but offered as a scaffold for participants lacking inspiration.

In our study, designers were instructed to produce as many design proposals as possible within the allotted time, ultimately submitting all their satisfactory designs. Each submission consisted of design descriptions generated by ChatGPT (GPT-4o), images produced by Midjourney, and the accompanying prompt descriptions. Pre-experiment phase allowed participants to familiarize themselves with the use of ChatGPT, Midjourney, and our provided ideation reference method, CombinatorX. 

The formal experiment involved a specific design task requiring participants to create innovative chair designs for office settings. As chair design involves both functional and aesthetic elements that can greatly benefit from the innovative possibilities offered by AI, the experiment focused on office scenarios to refine the creative direction and functional requirements of the designs. By focusing the chair design task on office scenarios, the experiment was structured to refine the creative direction and functional requirements of the designs, making our subsequent interviews and analysis more focused. The experimental process, including the procedure details and time limits, was established based on our pilot study. 

After the experiment concluded, we conducted semi-structured interviews, each lasting approximately 20 minutes. These interviews were informed by our observations of the participants’ design processes and the outputs generated by GenAI. We structured our interview around three topics to understand participants' submitted solutions under the help of GenAI (question 1-2), designers' creation processes (question 3-4), and designers' perspective on the output (question 5).

\begin{enumerate}
    \item ``Have you noticed any repetition or similarities in the ideas or designs generated by the tools?'' (\textbf{Corresponding to RQ2})
    \item ``Do you think there are stylistic or paradigmatic limitations in the ideas generated by these tools compared to those generated through traditional design methods?'' (\textbf{Corresponding to RQ3})
    \item ``If the design process required further development, how do you think the performance of these tools would affect the originality or diversity of your final creative outputs?'' (\textbf{Corresponding to RQ3})
    \item ``What do you think are the reasons for any limitations you perceived in the model outputs? Are these limitations due to the model itself or the interaction process?'' (\textbf{Corresponding to RQ2})
    \item ``Could you share your views on the design approaches demonstrated during the experiment, including both pros and cons?'' (\textbf{Corresponding to RQ3})
\end{enumerate}

These questions were designed to explore various aspects of GenAI design fixation. While it was found in participants' responses that several participants were already familiar with the concept of design fixation, we deliberately avoided using terms like “GenAI design fixation” or “design fixation” directly in our interviews. Instead, we used alternative descriptions  (i.e. interview question 1 above) to probe for any sensations of fixation that may have occurred during their interactions with generative AI tools. At the conclusion of the experiment, we explained the study’s aims and the terminology used to all participants.


\begin{figure*}[htp]
    \centering
    \includegraphics[width=0.95\textwidth]{figures/fig_process.png}
    \caption{The process of participants engaging in our experiment.}
    \vspace{-0.05in}
    \label{fig:experiment}
    \Description{}
\end{figure*}


\begin{table}[ht]
\centering
\caption{Participant demographics in our experiment.}
\label{tab:participants}
\begin{tabular}{@{}cccccc@{}}
\toprule
\textbf{ID} & \textbf{Age} & \textbf{Gender} & \textbf{Design experience} & \textbf{Use of text generation models} & \textbf{Use of image generation models} \\ \midrule
P1 & 22 & Male & less than one year & Used before & Used before\\
P2 & 22 & Female & three years & Proficient & Used before \\
P3 & 23 & Female & One year & Proficient & Used before \\
P4 & 23 & Female & Five years & Proficient & Used before \\
P5 & 24 & Female & Five years & Proficient & Proficient \\
P6 & 20 & Male & Two years & Proficient & Used before \\
P7 & 20 & Male & One year & Proficient & Used before \\
P8 & 24 & Male & Six years & Proficient & Used before \\
P9 & 23 & Female & Six years & Used before & Used before \\
P10 & 26 & Female & Five years & Proficient & Used before \\ \bottomrule
\end{tabular}
\end{table}

\subsection{Data collection and analysis}
\subsubsection{Data collection}
We collected two sets of data. The first set of data consists of the GenAI design solutions collected in Section~\ref{experiment_procedure} (totaling 96). The second set of data comes from the chair design examples from the Red Dot design contest\footnote{\url{https://www.red-dot.org/}}. The reason why this dataset were collected is for the assessment of the originality and creativity of GenAI-generated chair designs in an office setting. We choose a design contest due to its diversity and the high quality of design innovation it represents. Commercial dataset such as IKEA were avoided due to potential stylistic uniformity linked to specific brands, which could skew the diversity of the dataset. Specifically, we crawled and filtered 105 chair design entries with "office chair" containing in their design description. For these selected cases, both the images and textual descriptions of the designs were collected. This dataset provides a comprehensive foundation for comparing traditional and AI-generated designs, enhancing the analysis of GenAI's impact on design creativity and fixation in real-world scenarios.

%解释为什么用红点奖而不是组间对照
The reason why we did not conduct between-subject analysis is that it is difficult for human designers to produce photo-like chair design schema in a limited time in a laboratory experiment.  Additionally, our research aim is to assess the novelty and diversity of GenAI-generated design outcomes, making the use of recognized design award entries as a baseline an acceptable approach for comparison.

\subsubsection{Data analysis}
\paragraph{\textbf{Text generation analysis}}
In the data preprocessing phase, to enhance the relevance analysis between descriptive content and chair design, we initially reviewed all the solutions and annotated terms specifically related to chair design such as 'seat', 'leg', 'backrest', etc. (The detailed list is provided in Appendix B in our supplementary materials). We then filtered out words representing structural design. These terms serve as indicators of the degree of alignment between the design descriptions and practical chair design aspects. The results and comparison of this part of the data will be detailed in Section~\ref{sec: result}. Our keyword extraction criteria centered around functional descriptions and aesthetic elements, following methodologies akin to those used in GenAI-assisted design process analysis, as discussed in \cite{chen2024designfusion}. To maintain fairness and consistency between two sets of data, the keyword extraction process was carried out independently by two authors. The generated results from different groups were randomly shuffled before extraction to eliminate bias. After individual completion, any disputed portions were collectively discussed, and the results were determined. After keyword extraction, we consolidated homonyms to make the following analysis more scientific. An example of our keyword extraction and homonym consolidation process is detailed in Appendix C in our supplementary materials. Following the consolidation, we quantified the frequency of terms used in the ChatGPT solutions and those in the Red Dot solutions. We then identified and calculated the unique words and shared words between the two datasets, tallying both the number of entries and their cumulative frequencies.

To evaluate the novelty in the design solutions contributed by generative AI compared to human-generated solutions, we define the proportion of novelty ($P_{novelty}$) as the ratio of unique words to the total number of word entries (unique and shared). This ratio is given by the formula:
\begin{equation}
P_{novelty} = \frac{U}{U + S}
\label{eq:novelty_ratio}
\end{equation}
where $U$ denotes the number of unique word items and $S$ denotes the number of shared word items.

\paragraph{\textbf{Image generation analysis}}
In our study, we leveraged the CLIP model for an in-depth analysis of chair images generated by Midjourney and compared these with human-designed chair data. Based on insights from \cite{gandelsman2023interpreting} regarding the CLIP-ViT image encoder, we focused on specific attention heads—Layer 22 Head 1, Layer 22 Head 11, and Layer 23 Head 12—which are linked to semantic roles such as “shape,” “color,” and “texture.” We processed chair images \(I\) through the encoder, extracting embeddings for these attributes from the class token \(z_0\) outputs at the identified heads \(z_0^{l,h}\). Additionally, we captured the general embedding from the last layer output of the class token. For visualization, we used t-SNE \cite{van2008visualizing} to reduce these high \(d\)-dimensional embeddings of all images to a 2D space, which determines the coordinates for visualizing the image distribution.

Specifically, the background of both GenAI images and human images were adjusted to white using object recognition technology to minimize distractions. The visualization results are shown in Figure~\ref{fig_figure_TSNE}.

Additionally, we calculated the pairwise distance between the two datasets. To assess the statistical significance of the differences in diversity between images generated by Midjourney and those recognized in design contests, we employed the Mann-Whitney U test for pairwise distance comparisons. The mean distance is computed using the following formula:

\begin{equation}
\overline{D} = \frac{1}{n(n-1)} \sum_{i=1}^{n} \sum_{j=i+1}^{n} d(x_i, x_j)
\end{equation}

where \(d(x_i, x_j)\) represents the distance between two elements \(x_i\) and \(x_j\) in the dataset, and \(n\) is the number of elements in the dataset.



% \subsection{Results}
% \label{subsection_quantitative_result}
% \subsubsection{Text generation models analysis results}
% \begin{table}[ht]
% \centering
% \caption{????}
% \label{???}
% \begin{tabular}{lrrrr}
% \toprule
% Group &  Unique Words &  Total Words &  Diversity (TTR) &  Novelty \\
% \midrule
% Human &          1166 &     2522.000 &            0.461 &    0.713 \\
% AI &           885 &     2379.000 &            0.371 &    0.621 \\
% \bottomrule
%  \vspace{0.05in}
% \end{tabular}
% \end{table}
% \subsection{Findings}



% Based on participants’ feedback, text generation resembles an unrestricted human brainstorming session, potentially yielding ideas that might otherwise be filtered out during an individual brainstorming process.


%\subsection{The role of expertise in design fixation}
% Many studies have selected designers with different experiences for performances comparison, offering a foundation for investigating their respective responses to GenAI design fixation. Research indicates that experts, with their depth of experience, tend to be more focused and efficient in their design processes. They often gravitate towards fewer, more targeted solutions, leveraging their extensive knowledge and past successes to streamline their creative processes \cite{kavakli2003strategic}. This efficiency, however, might also lead to a certain rigidity, as experts can become prone to overlooking novel alternatives due to confirmation bias \cite{kim2014design}. Results from the study by Jannson and Smith \cite{jansson1991design} are in agreement with this. They explain that, years of educational and professional experience may contribute to fixation.

% On the other hand, novice designers are often less constrained by past successes and are more open to experimentation, but this can also mean they are more susceptible to design fixation, as they might not recognize flawed concepts as readily as professionals \cite{ball1994cognitive, ullman1988model}. Furthermore, the tendency of novices to become fixated on specific design solutions could stem from their limited exposure to a wide range of problem-solving scenarios, making them more likely to stick to familiar ideas or directly replicate existing solutions \cite{niku2009creative}.

% The research conducted by Viswanathan explores how expertise influences the handling of open-ended design problems \cite{viswanathan2012study}. It found that both novice and expert designers are fixated on the example features, but the expert designers generated more non-redundant ideas.


%每条fixation pattern对于creativity ideation的影响
%Generates only concrete images, while ideation requires creative stimulation for both form and function


% \begin{figure*}[htp]
%     \centering
%     \includegraphics[width=0.95\textwidth]{figures/fig_figure_similarity_TNSE.png}
%     \caption{\textbf{The process and dynamics of fixation within generative AI (GenAI) systems and how it correlates with human fixation in design processes.}}
%     % \vspace{-0.25in}
%     \label{fig_figure_similarity}
%     \Description{}
% \end{figure*}





\subsection{Experimental Setup}

\begin{figure*}[ht]
\centering
\begin{minipage}{.49\textwidth}
\begin{tikzpicture}
\begin{axis}[
  width=\textwidth,
  height=0.8\textwidth,
  %xmode=log,
  grid=both,
  minor grid style={dashed},
tick label style={/pgf/number format/fixed},
  major grid style={dashed},
  xlabel={Effective MACs ($\times 10^5$)},
  ylabel={Test SI-SNR (\unit{\decibel})},
  legend style={
    at={(1.0,0.0)},   % x,y coordinates relative to the axis
    anchor=south east,
    legend columns=1
  },
  ymax=15.75,
  ymin=12.75,
  tick label style={font=\normalsize},
  label style={font=\normalsize},
  every axis legend/.append style={font=\small},
  xmin=-200,
xtick={0,100000,200000,300000,400000,500000,600000},
xticklabels={{$0$},{$1$},{$2$},{$3$},{$4$},{$5$},{$6$}},
  scaled x ticks=false, 
  xmax=600000
]

    % 2) ReLU, Sparse (blue, dashed, round marker)
    \addplot+[
      forget plot,
      color=mint,
      solid,
      thick,
      mark=*,
      mark options={fill=mint, draw=mint},
      nodes near coords,
      point meta=explicit symbolic,
      every node near coord/.append style={anchor=south, font=\small}
    ] table [meta=label, col sep=space] {
      x      y      label
 15385.396493  13.147347          2
 28589.361895  13.907744          3
 45140.258309 14.220416          4
 65621.998638  14.567575          5
 90050.190476  14.738816          6
118193.413502  14.910501          7
147635.113026  15.040557          8
183166.524831 15.155087          9
221680.248892 15.187635         10
263277.181804  15.251074         11
304559.862373 15.268791         12
    };


    \addplot [
      black,
      dashed,
      thick,
      forget plot
    ] coordinates {(8000,15.2) (9000000,15.2)} 
      node [pos=1, anchor=north east, font=\small] {Previous SotA};


    % 3) GeLU, Dense (red, solid, round marker)
    \addplot+[
      forget plot,
      color=orange,
      solid,
      thick,
      mark=*,
      mark options={fill=orange, draw=orange},
      nodes near coords,
      point meta=explicit symbolic,
      every node near coord/.append style={anchor=north, font=\small}
    ] table [meta=label, col sep=space] {
      x      y      label
  50688.0  13.333696          1
 152064.0  14.277531          2
 304128.0 14.997617          3
 506880.0 15.309213          4
    };

    
    % Weights: Dense (solid black line)
    \addlegendimage{orange, mark=*, solid, thick}
    \addlegendentry{Dense w/ GELU}

    % Weights: Sparse (dashed black line)
    \addlegendimage{mint, mark=*, solid, thick}
    \addlegendentry{Sparse w/ ReLU}

  % Now draw the big red arrows (example):
  %  -- Arrow from orange #1 (x=50688, y=13.3337) to orange #3 (x=304128, y=14.9976)
  \draw[->, thick, black!50]
    (axis cs:294128, 15.0)
    -- (axis cs:157635.113026,  15.0) ;

  %  -- Arrow from orange #3 (x=304128, y=14.9976) to green #9 (x=183166.5248, y=15.1551)
  %     (This arrow angles 'back' to the left, as in your figure.)
  \draw[->,thick, black!50]
    (axis cs:294128, 15.0)
    -- (axis cs:38589.361895,  13.907744);
  \end{axis}
\end{tikzpicture}
\end{minipage}
\begin{minipage}{.49\textwidth}
\begin{tikzpicture}
\begin{axis}[
  width=\textwidth,
  height=0.8\textwidth,
  %xmode=log,
  grid=both,
  minor grid style={dashed},
  major grid style={dashed},
  xlabel={Memory Footprint (\unit{\mega\byte})},
  ylabel={Test SI-SNR (\unit{\decibel})},
  legend style={
    at={(1,0)},   % x,y coordinates relative to the axis
    anchor=south east,
    legend columns=1
  },
  ymax=15.75,
  ymin=12.75,
  xmin=0,xmax=2,
  tick label style={font=\normalsize},
  label style={font=\normalsize},
  every axis legend/.append style={font=\small},
]

    \addplot [
      black,
      dashed,
      thick,
      forget plot
    ] coordinates {(0,15.2) (2,15.2)} 
      node [pos=1, anchor=north east, font=\small] {};

    % 4) GeLU, Sparse (red, dashed, round marker)
    \addplot+[
      forget plot,
      color=orange,
      solid,
      thick,
      mark=*,
      mark options={fill=orange, draw=orange},
      nodes near coords,
      point meta=explicit symbolic,
      every node near coord/.append style={anchor=north, font=\small}
    ] table [meta=label, col sep=space] {
      x      y      label
0.193909 13.333696          1
0.581177 14.277531          2
1.161804 14.997617          3
1.935791 15.309213          4
    };

    % 4) GeLU, Sparse (red, dashed, round marker)
    \addplot+[
      forget plot,
      color=mint,
      solid,
      thick,
      mark=*,
      mark options={fill=mint, draw=mint},
      nodes near coords,
      point meta=explicit symbolic,
      every node near coord/.append style={anchor=south, font=\small}
    ] table [meta=label, col sep=space] {
      x      y      label
0.070136 13.147347 2
0.133800 13.907744 3
0.216820 14.220416 4
0.319128 14.567575 5
0.440823 14.738816 6
0.581836 14.910501 7
0.742147 15.040557 8
0.921845 15.155087 9
1.120834 15.187635 10
1.339239 15.251074 11
1.576909 15.268791 12
    };

    % -----------------------------------------------------------
    % Manual legend entries

    % Weights: Dense (solid black line)
    \addlegendimage{orange, mark=*, solid, thick}
    \addlegendentry{Dense}

    % Weights: Sparse (dashed black line)
    \addlegendimage{mint, mark=*, solid, thick}
    \addlegendentry{Sparse}

  \draw[->, thick, black!50]
    (axis cs:1.121804, 15.0)
    -- (axis cs:0.782147,  15.0) ;

  %  -- Arrow from orange #3 (x=304128, y=14.9976) to green #9 (x=183166.5248, y=15.1551)
  %     (This arrow angles 'back' to the left, as in your figure.)
  \draw[->,thick, black!50]
    (axis cs:1.121804, 15.0)
    -- (axis cs:0.173800,  13.907744);
  \end{axis}
\end{tikzpicture}
\end{minipage}
\caption{Pareto fronts for S5 network audio denoising quality (SI-SNR) as a function of effective compute (left) and memory footprint (right) on the Intel N-DNS test set. S5 networks with  weight and activation sparsity (green) exhibit a large domain of Pareto optimality versus dense S5 networks (orange). Number annotations enumerate increasing S5 dimensionality configurations, from \qty{500}{k} to \qty{4}{\million} parameters. Dashed horizontal like marks SI-SNR of Spiking-FullSubNet XL, the previous state-of-the-art model. The horizontal arrows highlight models used for hardware deployment, the diagonal arrows highlight models of the same width. See text for details.}
\label{fig:ndns_performance_efficiency}
\end{figure*}

\paragraph{Software}
We implemented our methodology in JAX 0.4.30, building on top of the original S5 codebase \cite{DBLP:conf/iclr/SmithWL23}, with JaxPruner \cite{DBLP:journals/corr/abs-2304-14082} for the pruning algorithms and the AQT library \cite{aqt} for quantization-aware training. We implemented static quantization and a fixed-point model ourselves using only JAX.
% The implementation on the Intel Loihi 2 is based on NxKernel 0.2.0 and all characterization results are produced on a single-chip Oheo Gulch N3C1 board (accessible only to Intel Neuromorphic Research Community members).
% The implementation on the NVIDIA Jetson Orin Nano 8GB is running Jetpack 6.2, CUDA 12.4, JAX 0.4.32 and using the MAXN SUPER power mode. Power on the Jetson is reported as only CPU\_GPU\_CV through jtop 4.3.0.
% Performance results are based on testing as of Jan 2025 and may not reflect all publicly available security updates. Results may vary.

\paragraph{Audio denoising task}

We evaluated our approach on the Intel Neuromorphic Deep Noise Suppression Challenge \cite{Timcheck_2023}.
%
% AP: We should give a general understanding of the task, the pre-/post-processing steps, the acceptable latency, and the SI-SNR metric.
The objective of the Intel N-DNS Challenge is to enhance the clarity of human speech recorded on a single microphone in a noisy environment.
%
The Intel N-DNS Challenge utilizes data from the Microsoft DNS Challenge,  encompassing clean human speech audio samples and noise source samples.  \cite{reddy2020interspeech, reddy2021icassp, reddy2021interspeech, dubey2024icassp}.
Clean human speech and noise samples are mixed to produce noisy human speech with a ground truth clean human speech goal.

To train our models, we used the default Intel N-DNS Challenge training and validation sets, each consisting of \qty{60000}{} noisy audio samples of \qty{30}{\s} each, and a test set with \qty{12000}{} samples. 
%
We encoded and decoded each audio sample using the Short-Time Fourier Transform (STFT) and Inverse Short-Time Fourier Transformer (iSTFT) \cite{grochenig2013foundations}. 
%
Following the N-DNS baseline solution, NsSDNet \cite{shrestha2024efficient}, we adopted a \qty{32}{\milli\s} window length and a \qty{8}{\milli\s} hop length for the STFT/ISTFT.
%
This resulted in a nominal real-time audio processing latency of \qty{32}{\milli\s}, which allows ample time (\qty{8}{\milli\s}) for denosing network inference, as \qty{40}{\milli\s} is the standard for an acceptable latency as recognized in the Microsoft N-DNS Challenge. 

We evaluated the denoising quality of our model using the scale-invariant signal-to-noise ratio (SI-SNR)
\begin{equation}
    \text{SI-SNR} = 10\log_{10}\frac{\norm{s_\text{target}}^2}{\norm{e_\text{noise}}^2}.
\end{equation}
Importantly, SI-SNR provides a volume-agnostic measure of audio cleanliness relative to the ground truth signal. 



\subsection{Pareto Front of Performance and Efficiency}
\label{ss:pareto-front}

We studied the performance-efficiency Pareto front of dense and sparse models across inference compute budgets.
Starting from the S5 architecture \cite{DBLP:conf/iclr/SmithWL23}, we trained a family of dense models of increasing size by linearly scaling the model dimensions (i.e.\ model width and size of the SSM hidden state), while keeping the depth fixed to three S5 layers.
Similarly, we trained a family of sparse models, i.e., pruned and ReLU-fied, according to our methodology discussed above, with $90\%$ of weights pruned by the end of training (further details on the model dimensions are provided in \Cref{app:model-params}).
The results, reported in \autoref{fig:ndns_performance_efficiency}, compare de-noising performance (SI-SNR) and computational efficiency as measured by effective MACs and memory footprint (see \Cref{supp:macs}).
%
%Furthermore, we applied a hyperparameter search {\color{red} TODO: details of the hyperparameter search methodology?} to ensure a fair representation of the best performance at each network size and in either sparse or dense configuration.
%
%We computed a proxy measure of efficiency for each model by calculating the effective Multiply-And-Accumulate operations (MACs) per time step and the memory footprint (model size).

The results show that sparsification significantly degrades performance when applied to under-parametrized dense models (e.g., sparsifying dense-\qty{3}{} reduces SI-SNR by $7.3\%$).
However, task performance is recovered with increased model dimensions and the accuracy of dense models is matched by larger sparse ones, with fewer MACs and lower memory requirements.
This gives empirical support to theoretical work on the capacity of sparse-and-wide neural networks \cite{golubeva_are_2020}.
For example, sparse-\qty{8}{} model requires \textbf{$\mathbf{2}\boldsymbol{\times}$ lower compute} and \textbf{$\mathbf{36}\boldsymbol{\%}$ lower memory} than the dense-\qty{3}{} model, \textbf{while achieving the same level of accuracy}.
Overall, sparse models constitute the Pareto front of task performance and computational efficiency across compute budgets.

In terms of absolute task performance, we find that the S5 architecture provides state-of-the-art results on audio denoising out of the box.
When compared to Spiking-FullSubNet-XL \cite{10605482}, the Track 1 winner of the Intel N-DNS Challenge with \qty{15.2}{\dB} SI-SNR, our sparse-\qty{11}{} S5 model requires \textbf{$\mathbf{3.2}\boldsymbol{\times}$ lower compute} and \textbf{$\mathbf{5.37}\boldsymbol{\times}$ lower memory} \textbf{iso-accuracy}.
This finding is in line with previous research on audio modeling with state space models \cite{DBLP:conf/icml/GoelGDR22}, and provides additional evidence on the suitability of these architectures for signal processing.
%
%The XL version of the Spiking-FullSubNet network achieves \qty{15.2}{\dB} SI-SNR on the Intel N-DNS Challenge test set, as noted by the horizontal dashed line in \autoref{fig:ndns_performance_efficiency}. 
%
%Our S5 models can achieve \qty{15.2}{\dB} SI-SNR with modest computational cost and memory footprints.
%In comparison, Spiking-FullSubNet XL uses $8.4 \times 10^5$ effective MACs per \qty{8}{\milli\s} timestep and has a memory footprint of \qty{7.02}{\mega\byte}; these computational cost and memory points are much larger than those of our S5 networks shown in \autoref{fig:ndns_performance_efficiency}---beyond the domain displayed in our plots---suggesting strong competitiveness from our S5 networks, especially under consideration of resource constraints.
% 
%We note that the Spiking-FullSubNet network was trained using a loss function that includes other terms in addition to SI-SNR, catering to other audio quality metrics.
%Therefore, Spiking-FullSubNet's results in the Intel N-DNS Challenge does not represent the maximum achievable SI-SNR for the Spiking-FullSubNet architecture.
%
%Nevertheless, Spiking-FullSubNet's results provide an excellent point of comparison, as SI-SNR was one of the main metrics for which Spiking-FullSubNet was optimized.



\paragraph{Interaction of weight and activation sparsity}

An interesting question is what is the interaction between the two types of sparsity, in weights and activations.
\autoref{fig:activation_sparsity} reports the pre-activation sparsity for different layers across the model depth for two ReLU-fied models of the same size (model variant \qty{6}{}), with and without synaptic sparsity.
%
We observe that the synaptic-sparse model exhibits lower activation sparsity across the board, a finding that is consistent across model sizes.
%
In addition, activation sparsity significantly decreases with model depth, both for dense and sparse models.
These phenomena, previously observed in other models \cite{mukherji2024weight}, suggest that, during training, the model compensates the reduced information flow caused by pruning with increased levels of activation.
Additional research on more advanced activation functions would allow for the optimal allocation of MACs, especially those that provide explicit control over sparsity without cross-channel synchronization (e.g.,\ approximate top-k \cite{DBLP:journals/corr/abs-2412-04358}).
% Nonetheless, weight and activation sparsity combine constructively to result in overall effective MAC reductions greater than that of activation sparsity or weight sparsity alone.

%{\color{red} TODO: would need an ablation study or additional Pareto curves to support this statement. Not necessarily necessary to include the curves in the paper, but that we know it is true would be helpful and could be written in in some way.}

%{\color{red} TODO: may wish to move interpretation to discussion} 

\begin{figure}[t]
\centering
    \pgfplotstableread[row sep=\\,col sep=&]{
        idx     & S5Hid & S5Out & GLU   \\
        1       & 80.754554271698  & 49.60111975669861  & 71.89104557037354   \\
        2       & 58.92143249511719  & 29.815730452537537  & 82.44596123695374  \\
        3       & 51.13644599914551  & 34.60754454135895  & 62.65120506286621  \\
    }\mydata

    \pgfplotstableread[row sep=\\,col sep=&]{
        idx     & S5Hid & S5Out & GLU   \\
        1       & 81.3236653804779  & 39.38424289226532   & 60.10604500770569   \\
        2       & 54.60154414176941  & 19.661784172058105  & 70.28757929801941  \\
        3       & 45.402026176452637  & 22.966817021369934  & 47.028326988220215  \\
    }\mydatasparse

    \begin{tikzpicture}
        \begin{axis}[
                ybar,
                bar width=.28cm,
                width=\linewidth,
                height=0.8\linewidth,
                legend style={
                    at={(0,1)},
                    anchor=north west,
                    legend columns=3,
                },
                xtick=data,
                xticklabels={Layer 1, Layer 2, Layer 3},
                xmin=0.5, xmax=3.5,
                ymax=100,
                ylabel={Pre-activation Sparsity (\%)},
                xlabel={Model Depth},
                ytick pos=left,
                grid=both,
                xmajorgrids=false,
                minor grid style={dashed},
                major grid style={dashed},
              tick label style={font=\normalsize},
              label style={font=\normalsize},
              every axis legend/.append style={font=\small},
            ]
            % Define colors for consistency between data and sparse data.
            % First three plots (non-sparse data) without patterns

            \addlegendimage{area legend, fill=mint}
            \addlegendentry{Norm}
            \addlegendimage{area legend, fill=pear}
            \addlegendentry{S5 Out}
            \addlegendimage{area legend, fill=orange}
            \addlegendentry{GLU}
            \addlegendimage{area legend, fill=black}
            \addlegendentry{Dense}
            \addlegendimage{area legend, fill=black, postaction={pattern=crosshatch, pattern color=white}}
            \addlegendentry{Sparse}
            
            \addplot+[fill=mint, draw=mint] table[x=idx,y=S5Hid] {\mydata};
            \addplot+[fill=pear, draw=pear] table[x=idx,y=S5Out] {\mydata};
            \addplot+[fill=orange, draw=orange] table[x=idx,y=GLU] {\mydata};
            
            % Now add the sparse data with the same colors, but with patterns.
            \addplot+[fill=mint, draw=mint, postaction={pattern=crosshatch, pattern color=white}] table[x=idx,y=S5Hid] {\mydatasparse};
            \addplot+[fill=pear, draw=pear, postaction={pattern=crosshatch, pattern color=white}] table[x=idx,y=S5Out] {\mydatasparse};
            \addplot+[fill=orange, draw=orange, postaction={pattern=crosshatch, pattern color=white}] table[x=idx,y=GLU] {\mydatasparse};

        \end{axis}
    \end{tikzpicture}
    \caption{Activation sparsity of ReLU blocks across model depth for a dense-weight model and a sparse-weight model. The sparse-weight model exhibits significantly lower activation sparsity across layers.}
    \label{fig:activation_sparsity}
\end{figure}


\subsection{Hardware Implementation} 
\label{ss:hardware-implementation}

%In order to validate the sparsity gains on real-time inference performance, we implemented our S5 variants on the Intel Loihi 2 neuromorphic chip.


% bar on x-axis shows accuracy
% stars on x-axis (top) is memory footprint (instead of # params)
% 1) baseline - sparse & relu  ---- with or without QAT @ W8A16
% 3) static quantization conversion
% 4) fxp model in jax
% 5) nxkernel on loihi

% two bars of different shape, one with QAT, the other without
% THIS IS FOR ONE MODEL SIZE ONLY!!

\begin{figure}
    \centering
        
    \pgfplotstableread[row sep=\\,col sep=&]{
        idx     & Baseline & WithQAT  \\
        1       & 10.486  &  12.99034   \\
        2       &  11.84137 & 14.201379776000977    \\
        3       & 9.4657  & 14.5627       \\
        4       & 14.84  & 14.70684814453125   \\
    }\mydata

    \begin{tikzpicture}
        \begin{axis}[
                xbar,
                bar width=.45cm,
                  width=0.925\linewidth,
                  height=0.9\linewidth,
                legend style={at={(0,1)},
                    anchor=north west,legend columns=1,},
                ytick=data,
                yticklabels={FPX (Loihi), FXP (Sim), Static Quant, FP32},
                %xmin=0.5, xmax=3.5,
                ymax=5.3,
                ymin=0.45,
                xmax=20,
                xlabel={Test SI-SNR (\unit{\decibel})},
                xtick={10, 12, 14},
                xtick pos=bottom, % Ensure x-ticks appear only at the bottom
                ytick pos=left, % Ensure x-ticks appear only at the bottom
                grid=both,
                ymajorgrids=false,
                minor grid style={dashed},
                major grid style={dashed},
              tick label style={font=\normalsize},
              label style={font=\normalsize},
              every axis legend/.append style={font=\small},
            ]
                \addplot+[
                    sharp plot,
                    stack plots=false,
                    forget plot,
                    color = black,
                    mark = *,
                    thick,
                    fill = none,
                    draw=black,
                  nodes near coords,
                  point meta=explicit symbolic,
                  every node near coord/.append style={anchor=north west, font=\small}
                ]   table [meta=footprint, col sep=space] {
          x      y      footprint
          15.684306   1     \qty{116.7}{}
          15.684306   2     \qty{116.7}{}
          17.644938   3   \qty{451.4}{}
          17.644938   4   \qty{451.4}{}
        };
            \addlegendimage{area legend, fill=mint}
            \addlegendentry{Base}
            \addlegendimage{area legend, fill=mint, postaction={pattern=crosshatch, pattern color=white}}
            \addlegendentry{QAT}
            \addplot+[fill=mint, draw=mint, postaction={pattern=crosshatch, pattern color=white}] table[x=WithQAT,y=idx] {\mydata};
            \addplot+[fill=mint, draw=mint] table[x=Baseline,y=idx] {\mydata};

    \node[fill=white, inner sep=2pt, draw=none] at (17.45,4.75) {Memory (\unit{\kilo\byte})};
        \end{axis}
    \end{tikzpicture}
    \caption{Impact of quantization interventions on Test SI-SNR and memory footprint, with and without quantization-aware training, for model variant sparse-\qty{6}{}.}
    \label{fig:quantization_interventions}
\end{figure}

\paragraph{Impact of fixed-point conversion}

Since Loihi 2 only supports fixed-point (FXP) arithmetic, as presented in \Cref{sec:methodology}, we quantized the weights and activations of our model and implemented the network dynamics in FXP arithmetic. The effect of our quantization methodology is presented in \autoref{fig:quantization_interventions}.
%
Starting from a 32-bit floating-point (FP32) model, we apply static quantization, which rounds weights and activations using fixed scales, but still performs the actual computation in FP32. Notably, Quantization-Aware Training (QAT) is very effective in maintaining test performance (SI-SNR) from FP32 to static quantization, compared to Post-Training Quantization (PTQ).
%
The frozen scales from static quantization are imported into our FXP model implemented in JAX, which uses only int32 types and fixed-point arithmetic to compute the forward pass of the model.
We observe further performance degradation in the FXP simulation, which we analyze in more detail in \Cref{appendix:fxp-sim-mismatch}. 
%
We finally map the FXP model to Loihi 2 and perform inference on the chip, again finding a degradation in SI-SNR, which is likely due to subtle differences in the integer arithmetic performed by the FXP simulation and Loihi 2 implementation with fused layers. Another source of mismatch is that the FXP model in simulation handles overflows by clipping to the maximum value, whereas Loihi 2 ``wraps around'' the value, resulting in a sign inversion.
%
The size of the model decreases by about a factor of 4 when transitioning from FP32 weights to INT8 weights, as shown on the right side of \autoref{fig:quantization_interventions}.



\paragraph{Power and Performance}

\begin{table*}
    \centering
    \caption{Power and performance results$^*$. The Loihi 2 is running a sparse and quantized S5 model, while the Jetson Orin Nano is running a smaller dense S5 model that reaches similar test performance. All measurements are averaged over \qty{8}{} random samples from the test set, each containing \qty{3750}{} time steps. \textcolor{gray}{Gray highlights} denote violation of real-time constraints for the audio denoising task. Best real-time results are \underline{underlined}.}
    \begin{tabular}{l c r r r}
        \toprule
        & \textbf{Mode}  
        & \multicolumn{1}{c}{\textbf{Latency} ($\downarrow$)} 
        & \multicolumn{1}{c}{\textbf{Energy} ($\downarrow$)}
        & \multicolumn{1}{c}{\textbf{Throughput} ($\uparrow$)} \\
        \midrule
        \textbf{Token-by-token}  \\
        \quad Intel Loihi 2$^\dagger$ & Fall-Through              &       \underline{\qty{76}{\micro\second}} &    \underline{\qty{13}{\micro\joule/\token}} &    \underline{\qty{13178}{\token/\second}} \\
        \quad Jetson Orin Nano$^\ddagger$ & Recurrent 1-step $(b=1)$ &     \qty{2688}{\micro\second} &  \qty{15724}{\micro\joule/\token} &  \qty{372}{\token/\second} \\
        \quad Jetson Orin Nano$^\ddagger$ & Recurrent 10-step $(b=1)$ &    \qty{3224}{\micro\second} &  \qty{1936}{\micro\joule/\token} &   \qty{3103}{\token/\second} \\
        \quad Jetson Orin Nano$^\ddagger$ & Recurrent 100-step $(b=1)$ &   \textcolor{gray}{\qty{10653}{\micro\second}} & \qty{626}{\micro\joule/\token} &   \qty{9516}{\token/\second} \\
        \quad Jetson Orin Nano$^\ddagger$ & Recurrent scan $(b=1)$ &       \textcolor{gray}{\qty{236717}{\micro\second}}& \qty{404}{\micro\joule/\token} &   \qty{15845}{\token/\second} \\
        \midrule
        \textbf{Sample-by-sample} \\
        \quad Intel Loihi 2$^\dagger$ & Pipeline &                        \underline{\qty{60.58}{\milli\second}} &   \underline{\qty{185.80}{\milli\joule/\sample}} &   \underline{\qty{16.58}{\sample/\second}} \\
        \quad Jetson Orin Nano$^\ddagger$ & Scan $(b=1)$ &                                \qty{233.48}{\milli\second} &           \qty{1512.60}{\milli\joule/\sample}& \qty{4.28}{\sample/\second} \\
        \quad Jetson Orin Nano$^\ddagger$ & Scan \textcolor{gray}{$(b=b_{\text{max}})$} & \textit{\qty{226.53}{\milli\second}} &  \textit{\qty{5.89}{\milli\joule/\sample}} &  \textit{\qty{1130.09}{\sample/\second}} \\
        \bottomrule
    \end{tabular}
\centering
% \vskip 0.01em 
\begin{minipage}{.9\textwidth}{\tiny \baselineskip=8pt \setstretch{0.6}
%
$^\dagger$ Loihi 2 workloads were characterized on an Oheo Gulch system with N3C1-revision Loihi 2 chips running NxCore 2.5.8 and NxKernel 0.2.0 with on-chip IO unthrottled sequencing of inputs. Researchers interested to run S5 on Loihi 2 can gain access to the software and systems by joining \textit{Intel's Neuromorphic Research Community}.
%
$^\ddagger$ Jetson workloads were characterized on an NVIDIA Jetson Orin Nano 8GB running Jetpack 6.2, CUDA 12.4, JAX 0.4.32, using the MAXN SUPER power mode; energy values are computed based on the TOT power as reported by jtop 4.3.0. The batch size $b_{\text{max}}=256$ was chosen to be the largest that fits into memory.
%
$^*$Performance results are based on testing as of January 2025 and may not reflect all publicly available security updates; results may vary.
}
\end{minipage}
    \label{tab:pnp}
\end{table*}


To measure the empirical efficiency benefits afforded by the sparse S5 model on neuromorphic hardware, we profile inference on Loihi 2 using the fixed-point S5 model, in particular, configuration sparse-\qty{8}{} from \autoref{fig:ndns_performance_efficiency}.
%
To compare to conventional hardware, we profile the smallest dense model that achieves equivalent performance on Jetson Orin Nano\footnote{Our W8A16 fixed-point model in JAX does not provide a speedup over the FP32 model on the Jetson Orin Nano, therefore we profile the FP32 model.}, which is configuration dense-\qty{3}{} from \autoref{fig:ndns_performance_efficiency}.
%
There exist a variety of modes in which to execute a model on Loihi and Jetson, each exhibiting different tradeoffs in terms of latency, throughput, and energy.
Therefore, we present different modes for a comprehensive characterization and comparison.
We summarize our profiling results in \autoref{tab:pnp}. More details on the different execution modes on Loihi 2 are presented in \Cref{app:exmode}.

In real-time, token-by-token processing on a single input sequence, Loihi 2 processes a single STFT frame $\mathbf{35\times}$ \textbf{faster} and with $\mathbf{1200\times}$ \textbf{less energy} than the Jetson Orin Nano. % (Token-by-token; Loihi 2 Fall-Through and Jetson Orin Nano Recurrent 1-step (b=1) in \autoref{tab:pnp}). 
When the Jetson Orin Nano processes ``chunks'' of multiple time steps, its utilization increases, and energy per token improves. With the largest chunks that fit the real-time requirement of latency $\leq$\qty{8}{\milli\sec}, Loihi 2 is \textbf{$\mathbf{42}\times$ faster} and uses \textbf{$\mathbf{149} \times$ less energy} per token.

In offline processing, when many STFT frames are buffered to process in succession (or in parallel), the energy efficiency and throughput of the Jetson Orin Nano improves. Loihi 2 performs offline processing with pipelining (see \Cref{app:exmode} for further explanation). When processing single sequences, \textit{i.e.} batch size $b=1$, Loihi 2 has \textbf{$\mathbf{3.7} \times$ higher throughput} with \textbf{$\mathbf{8}\times$ less energy} per sample. 

It is important to note that the Jetson Orin Nano is only fully utilized when processing \qty{256}{} sequences in parallel, and at this level, it shows significantly higher throughput while consuming less energy per sample, compared to Loihi 2. We include these results in the last row of \autoref{tab:pnp}.

\paragraph{Energy at real-time inference rate}
The latency budget for the neural network component of the audio denoising pipeline, running either on Loihi 2 or on the Jetson, is \qty{8}{\milli\s}.
Our Loihi 2 and Jetson implementations are well below 8ms for online inference.
%
Thus, to estimate the energy consumption in real-time settings, where subsequent tokens are actually \qty{8}{\milli\s} apart, we rescale the power as:
\begin{equation*}
    P_\text{total}^\text{real-time} =  P_\text{static} + \frac{t_\text{compute}}{\qty{8}{\milli\s}} P_\text{dynamic},
\end{equation*}
based on the power measurements in token-by-token processing.
In this setting, Loihi 2 achieves \qty{1128}{\micro\joule/\token} while the Jetson achieves \qty{36528}{\micro\joule/\token} for token-by-token processing and \qty{3720}{\micro\joule/\token} when processing chunks of 10 time steps at once. Loihi 2 remains at least $3 \times$ more energy efficient than the Jetson Orin Nano.

\paragraph{Limitations}

Our Jetson Orin Nano implementation is in FP32, while our Loihi 2 implementation is in W8A16. Our fixed-point model in JAX provides no improvements in runtime or energy. More competitive Jetson energy, latency, and throughput could potentially be obtained by developing a more optimized quantized implementation. 

% \paragraph{Energy and throughput for offline processing}

% {\color{red} TODO: Distill this discussion in the PnP comparison and remove this paragraph.

% If we move from online processing to offline processing, i.e., buffer several STFT frames to rapidly process in succession, Jetson energy efficiency and throughput improves (Jetson Orin Nano, Recurrent $n$-step or Recurrent scan).
% %
% However, this comes at the cost of buffering and additional processing latency. 
% %
% Jetson Orin Nano can also perform offline S5 utilizing a parallel scan routine (Sample-by-sample, Jetson Orin Nano, Scan); here, batch processing improves Jetson throughput and energy as well (b = max). 
% %
% Nonetheless, Loihi 2 performs offline processing in pipelined mode for batch size 1 with approximately $3.8\times$ increase in throughput and approximately $24\%$ increased energy cost compared to the most efficient batch size 1 Jetson implementation, which is the recurrent scan.
% %
% When using the maximal batch size than can fit on the Jetson's memory, however, the Jetson parallel scan implementation achieves the highest throughput of approximately 4.2M tokens per second, \textcolor{red}{yet at a substantial energy cost per token}.
% }
\section{Addressing GenAI design fixation in inspiration of computational creativity}
\label{section_solution}


% Firstly, strategies include using near sources as discussed by Chan et al. \cite{chan2018best}, engaging with far sources highlighted by Chan et al. \cite{chan2011benefits}, and utilizing intermediate sources such as those mentioned by Gonçalves et al. (2013) \cite{gonccalves2013inspiration}. Additionally, less common sources and methods, like written representations and partial photographs, provide alternative perspectives and are discussed in studies like Cardoso and Badke-Schaub \citep{cardoso2009give, cardoso2009idea} and Cheng et al. \cite{cheng2014new}. Biological examples and line drawings are another approach to trigger creative thinking, as explored by Wilson et al. \cite{wilson2010effects} and Cardoso and Badke-Schaub \cite{cardoso2011influence}.

% Regarding instruction and methods, building prototypes as outlined by Kershaw et al. \cite{kershaw2011effect} and Viswanathan et al. \cite{viswanathan2014study} are practical hands-on approaches. Other techniques involve abstraction and group working, noted by Cheong and Shu \cite{cheong2013reducing} and Youmans \cite{youmans2011design}. Moreover, encouraging an incubation break as seen in the works of Cardoso and Badke-Schaub \cite{cardoso2009give} and Ttsenn et al. \cite{tsenn2014effects} allows for mental refreshment that can aid in overcoming fixation.

% Some researchers focus on the instructional aspect, like understanding design fixation itself, with Howard \cite{howard2013overcoming} providing insights on mapping rules that help navigate fixation. What's more, design theories and methods have also been used to overcome design fixation. \citep{le2011interplay} studied the relationship between creativity issues and design methods and posited that design theories and methods invent models of thought to overcome fixation. They also discussed teaching based on C-K theory which is conducive to overcome design fixation. \citep{hatchuel2011teaching} also discussed teaching concept-knowledge theory can help overcome fixation effects.

% These strategies collectively offer a comprehensive toolkit for designers and innovators seeking to navigate and overcome design fixation, promoting a more expansive and effective creative process.
%我们的研究工作表明了GenAI表现出的design fixation现象可能来源于GenAI本身的训练数据集、模型架构与目标函数以及用户输入的prompt。这样的现象可能致使生成式人工智能的创意受到限制,进而让人类设计师受到误导甚至在一定程度上减小思考更高质量的设计解决方案的可能性。已有的研究表明,人类设计师的思维容易固着在生成式人工智能所产生的结果中,尤其是新手设计师,因此减轻GenAI design fixation能够在一定程度上减轻设计师产出的同质性,提高设计方案的多元性和创造性。

According to the definition of GenAI design fixation established in this study and substantiated through our experimental investigation, this phenomenon may originate from data bias, the architectural constraints of algorithms, and the prompts provided by human users. Such occurrences have the potential to limit the creative capacities of GenAI systems, thereby potentially misleading human designers and, to some extent, obstructing access to higher-quality solutions. Recent research has demonstrated that human designers might be more likely to often exhibit fixation on the outputs generated by GenAI \cite{wadinambiarachchi2024effects}. Consequently, alleviating GenAI design fixation would be instrumental in reducing the homogeneity of GenAI design outputs, which serve as stimuli for human designers, thereby enhancing the diversity and creativity of human-AI co-design solutions.

Despite the technical recommendations provided, we propose adopting methods to alleviate GenAI design fixation. Furthermore, we suggest the HCI community to consider the phenomenon of GenAI design fixation when designing and evaluating creativity support tools based on Generative AI.

\begin{enumerate}
    \item \textbf{Reduce bias in design data:} When exploring the training and application of GenAI models, data bias is a critical issue that can't be ignored. Bias in design data can affect the fairness and accuracy of the model, potentially misleading the design process of designers \cite{zhou2024bias, ferrara2023should}. These emphasize the importance of training artificial intelligence models with diverse and balanced datasets, to ensure that the models can produce more equitable, diverse, and innovative design solutions.
    %This contrast not only underscores the need for improved training approaches that can better capture the breadth of human creativity but also highlights the potential for integrating more diverse data sources to enhance the creative capabilities of AI systems.
    \item \textbf{Optimize model architecture and objective functions:} To enhance the model's innovative capacity and adaptability while avoiding over-optimization and fixation, it is necessary to optimize the model architecture and objective functions. For text and image generation models, this means exploring new architectures capable of understanding and generating complex relationships and adopting strategies to increase the model's sensitivity to low-frequency data.
\end{enumerate}


%计算创造力
% \subsection{Computational Creativity}
% The motivation for employing computational creativity \cite{mamykina2002collaborative} stems from the purpose to leverage machine capabilities to understand and emulate forms of human creativity, thereby addressing design fixation issues inherent in GenAI. 

% \begin{itemize}
%     \item \textbf{Combinational Creativity: Implications for Training Data} \\
%     To promote combinational creativity, GenAI systems can be trained on diverse datasets that blend elements from multiple domains or styles. For example, an AI trained on both architectural designs and natural forms could generate innovative structures inspired by biological features. This approach helps break away from traditional design constraints by introducing cross-domain creativity.

%     \item \textbf{Exploratory Creativity: Implications for Interaction Methods} \\
%     For exploratory creativity, designing GenAI systems that actively explore untapped areas of the creative space can lead to novel ideas. This can be achieved by implementing algorithms that perform random or directed exploration within broader, less-defined datasets. An example is an AI that learns from an eclectic mix of unlabelled artistic styles, pushing the boundaries to develop new art forms or design concepts.

%     \item \textbf{Transformational Creativity: Implications for Workflows} \\
%     Transformational creativity can be fostered by adapting solutions from one context to another through transfer learning. This involves training a model in one domain and repurposing it for another. For instance, a language model trained to generate literary works could be adapted for music composition, creating pieces that narrate a story through their melodies.
% \end{itemize}


% In summary, utilizing computational creativity within GenAI represents a promising strategy to overcome the challenges of design fixation. By harnessing diverse data, interactive techniques, and integrated workflows, we can empower GenAI systems to produce more original, varied, and contextually relevant outputs.


%工具设计层面
\subsection{Creativity support tool design}
%Besides the technical troubleshooting methods,
In this section, we give some reference strategies regarding methods to mitigate human design fixation, researchers have actively explored various strategies to mitigate design fixation \cite{linsey2010study}. According to the solution categorization proposed by \cite{alipour2018review}, the strategies can be divided into sourece, methods and instructions. In this section, we discuss mitigation methods in inspiration of the two directions in the context of Generative AI design fixation, as well as proposing strategies for interaction design. 

\subsubsection{Choosing the right kind of source}
In human design fixation study, the term ‘source’ refers to the use of previous examples and other resources as references for the solution to the current problem \cite{cai2010extended}. Studies have recommended certain strategies for reducing the effect of fixation in design. Their results have demonstrated that some sources leave enough room for exploration in design \cite{cheng2014new} and have a positive effect on the design outcome \cite{goldschmidt2011avoiding}. 
%These sources include far sources \cite{chan2011benefits}, less-common sources \cite{perttula2007idea}, biological examples \cite{wilson2010effects}, opposite terms \cite{chiu2008use}, and effective or expansive sources \cite{agogue2014impact} etc., which  can help designers to break away from established thinking patterns and explore a broader range of possibilities.

Analogously, in GenAI design fixation, studies has shown that providing additional, non-routine knowledge bases for GenAI-based CSTs can enhance the creativity and diversity of GenAI outputs. For instance, integrating specialized biological knowledge databases into GenAI systems \cite{zhu2023biologically} has been shown to expand the scope of design possibilities, encouraging the exploration of innovative solutions that might not emerge from conventional datasets \cite{kang2024biospark}. Similarly, supplying databases from specific fields of knowledge, such as materials science, cultural studies, or even niche areas of art and literature \cite{wang2024promptcharm}, can guide GenAI to venture into new, uncharted territories of design. These enriched knowledge bases serve as catalysts, prompting GenAI to generate more varied and inventive outputs.

\subsubsection{Instructions and methods}
In addition to choosing the right kind of source, studies on human design fixation has also highlighted the effectiveness of specific design instructions or systematic methods in order to help designers overcome design fixation. Several notable strategies include group working \cite{youmans2011effects}, employing alternative representations of the problem \cite{linsey2010study}, developing instructional mapping rules \cite{cheong2013using}, and utilizing the design-by-analogy method \cite{linsey2012design}. These approaches have been shown to diversify thinking and expand the range of solutions considered during the design process.

These methods can also be transferred to the field of GenAI to alleviate its design fixation. As previous empirical studies pointed out GenAI's tendency to produce surface-level or basic information, particularly in the context requiring in-depth research or exploration \cite{kobiella2024if}. For example, by employing the method of Multi-Agent Collaboration, we can endow GenAI with critical thinking and iterative capabilities. Adopting a Human-AI Collaboration system \cite{lee2024conversational} allows us to combine human creativity with AI capabilities. For instance, use GenAI to generate initial ideas, but have humans refine, combine, or expand upon these ideas. This hybrid workflow can reduce the risk of becoming fixated on AI-generated outputs. It can also guide GenAI to make broader associations, including proposing alternative representations of the problem and using analogies.

%需要补参考文献
\subsubsection{Interaction design}
In addressing the issue of design fixation in GenAI, interactive measures and the presentation of outputs play a crucial role. Firstly, by providing inspiring example solutions, users can be guided to think critically rather than being given direct answers. This can be achieved through user interfaces or workflows that stimulate creative thinking. %提供有启发的示例解决方案,批判性思考而不是直接接受。通过UI设计和工作流设计,促进创造性思考【有无例子】

Additionally, explicitly pointing out flaws in examples and providing instructions to avoid problematic elements can help users identify and circumvent potential design pitfalls. %指出示例中的不足之处,提供如何避免有问题的元素【有无这样的交互设计】

Secondly, tool customization allows users to adjust the randomness of AI-generated outputs, explore different styles, and set specific goals, thereby preventing repetitive and predictable results. \cite{liu2022design} %能够提供不同风格的选项,调整误差

% Suggesting alternative strategies: Guides proposed different ways to approach design tasks. 【例如可以使用进一步的引导】This would help designers to better comminicate design goals to the GenAI systems and develop an intuition for harnessing the AI's capabilities. \cite{gmeiner2023exploring}

\subsubsection{Improve prompt engineering}
% 新手设计师和专业设计师都会遇到prompt撰写的问题。prompt过于简略和冗余对于GenAI生成创意都是不利的。通过GenAI design fixation的透镜,prompt engineering的方向是刺激GenAI生成更加多样和新颖的方案。
As a consensus in HCI community, GenAI are sensitive to input prompts \cite{wu2022ai}, which is also one of the causes of GenAI design fixation in our study. As observed in our experiment and in line with other HCI study conclusion , novice designers often face challenges in crafting effective prompts for GenAI \cite{zamfirescu2023johnny}, with issues arising from prompts that are either too complex or too simple. Common issues include prompts that are overly complex or overly simplistic. This challenge has also been documented in previous studies focusing on creativity support for designers \cite{chen2024designfusion, liu2022design}. From the perspective of GenAI design fixation, these insights underscore the critical role of prompt engineering in fostering more diverse and innovative design outputs.

% 之前有研究帮助生成更多样的方案吗?
Effective prompt engineering requires a balance between clarity and flexibility, enabling GenAI to explore a broad range of creative possibilities. Prompts should guide the GenAI with enough specificity to maintain relevance while allowing room for creative interpretation. For example, specifying a theme or mood can direct the AI while still permitting innovative variations.
though the GenAI's stochastic nature to explore possibilities could yield creative ideas sometimes, the trial-and-error process is time-consuming.

% 如何迭代prompt,又有什么成功经验呢。prompt attributes是怎么划分和分析的
Iterative refinement is also crucial \cite{mahdavi2024ai}. By evaluating the outputs from initial prompts, designers can adjust subsequent prompts to better harness GenAI’s creative potential. This process involves a systematic approach where experimenting with various prompts can help understand how different prompt attributes affect the creativity of the outputs. Ultimately, the goal is to develop a toolkit for designers that supports the consistent elicitation of high-quality creative outputs from GenAI systems, enhancing the technology’s role in the creative process.

% 回顾过去HCI领域关于prompt engineering的研究,研究目标大多是align GenAI models and users' intents, or improve the quality of the output \cite{mahdavi2024ai}. 对于text generation model, \cite{wu2022ai} 针对transparency, controllability, and sense of collaboration问题,提出了LLM chains的交互方式,without the need to retrain the model.对于image generation model, \cite{liu2022design} found that it focuses on subject and style keywords rather than function words. \cite{mahdavi2024ai} identify prompt structures and how users evaluating AI-generated images.


%教育层面
\subsection{Education}
\subsubsection{Understanding the phenomenon of GenAI design fixation}
Howard et al.'s study \cite{howard2013overcoming} reflects that educating students in the phenomenon and effects of fixation enables them to effectively devise their own strategies to avoid or overcome fixation. So the first step in addressing design fixation within GenAI involves ensuring that designers comprehend the nature and implications of this phenomenon. Our work in align with \cite{anderson2024homogenization}'s findings that users given a sense of what the GenAI tend to suggest in similar contexts could help mitigate homogenization effects. In the context of GenAI, design fixation specifically pertains to the GenAI's tendency to generate outputs that adhere too closely to learned patterns and examples, thereby stifling novel and diverse designs.

\subsubsection{Users need to critically evaluate AI-generated content}
To mitigate the effects of GenAI-induced design fixation, it is crucial to train users to critically evaluate AI-generated content rather than accepting it passively. The importance of guiding designers reflection on generated designs is also proposed in other HCI research, such as \cite{gmeiner2023exploring}. This training should encompass an understanding of the underlying algorithms to some extent, enabling users to recognize the inherent limitations and bias of the GenAI. By fostering a critical mindset, asking questions like \textit{“Is there a better design approach?”} and \textit{“What limitations might the GenAI-generated results have?”}, designers can more effectively assess the suitability and originality of AI outputs. This approach ensures that these tools serve as a starting point for further creative development rather than as definitive solutions.


% \subsubsection{Utilizing AI outputs as inspiration material}
% Another effective strategy is to encourage designers to use AI-generated content as inspiration material rather than adopting it as direct solutions. By treating AI outputs as a source of ideas and stimuli, designers can explore a broader range of possibilities and avoid the trap of design fixation. This approach promotes a more dynamic and iterative design process, where AI serves to enhance human creativity rather than constrain it.

%GenAI design fixation resembles a creativity support dilemma: GenAI could offer novel ideas to users, but the actual solution exploration and transformational creativity is depends on designers' creativity level

%GenAI CST工具研究的评估标准
%可能需要回顾一下过去CST评估标准
\subsection{Evaluation metrics for GenAI-based CSTs research}

To effectively evaluate GenAI-based creativity support tools (CSTs), it is essential to incorporate design fixation as a critical standard. Design fixation, the tendency to become overly influenced by existing examples or solutions, can significantly hinder creativity and innovation. Therefore, any assessment framework for GenAI-based CSTs must rigorously examine the extent to which these tools either mitigate or exacerbate design fixation.


%\cite{calderwood2020novelists}提出diversity作为评估LLM的标准之一
\subsubsection{Positioning of GenAI CST Tools}

GenAI CST tools should be positioned not merely as instruments for increasing the efficiency and quantity of design outputs but as catalysts for enhancing designers' creative and innovative capacities. The primary goal of these tools should be to stimulate designers' thinking, helping them generate more creative, innovative, and groundbreaking ideas and inspirations, which contrasts with a narrow focus on efficiency and productivity.

A key consideration is to avoid overly programmatic workflows that may strengthen design fixation. While structured processes can streamline tasks, they might also limit the creative potential of designers by promoting adherence to predefined patterns and solutions. The development of GenAI-based CST tools should not solely focus on maximizing the AI's capabilities. Given the inherent risk of fixation within GenAI, it is crucial to design these tools in a way that incorporates significant designer participation. Emphasis should be placed on how these tools can engage the designer's subjective agency and imagination, thus fostering a more collaborative and dynamic creative process.

\subsubsection{Comprehensive evaluation of GenAI CSTs}

The evaluation of GenAI CST tools should extend beyond traditional usability tests and the resolution of issues identified in formative studies or user studies. While usability and problem-solving are important metrics, they do not fully capture the tools' impact on designers' cognitive and creative processes. Research outcomes should include rigorous assessments of how well these tools stimulate designers' cognitive engagement and creative thinking. This involves evaluating whether the CST tools genuinely enhance designers' ideation processes and their ability to conceptualize innovative solutions. Evaluations should also measure the impact of CST tools on the overall vibrancy and originality of designers' thought processes. This can include metrics such as the diversity and novelty of ideas generated, the ability to break away from conventional patterns, and the overall enhancement of creative problem-solving skills.

By integrating these comprehensive evaluation criteria, research on GenAI-based CSTs can ensure that these technologies not only address practical usability concerns but also significantly contribute to the advancement of creative design practices. This holistic approach will help in developing tools that truly empower designers, fostering an environment where human creativity and AI capabilities synergistically drive innovation while effectively mitigating the risk of design fixation.
\section{Discussion}

This study presents PathFinder, a multi-modal, multi-agent AI framework designed to emulate the multi-scale, iterative diagnostic approach of expert pathologists for histopathology whole slide images (WSIs). By integrating Triage, Navigation, Description, and Diagnosis Agents, PathFinder collaboratively gathers evidence to deliver accurate, interpretable diagnoses with natural language explanations. Notably, it surpasses state-of-the-art methods and the average performance of human experts in melanoma diagnosis, setting a new benchmark in AI-driven pathology.

PathFinder has the potential to accelerate diagnostic workflows, reducing the reliance on manual examination and enabling timely patient care in clinical settings. Its natural language descriptions provide interpretability, facilitating the validation of AI-generated diagnoses by pathologists. Moreover, its integration of vision-language models (VLMs) and large language models (LLMs) highlights the promise of multi-modal AI in delivering scalable, specialized diagnostic tools that could improve access to pathology expertise.

\noindent\textbf{Limitations.} Despite its strengths, PathFinder has limitations. The framework relies on pre-existing datasets and significant computational resources, posing challenges in resource-constrained environments. Additionally, the complexity of the Navigation Agent’s decision-making process and occasional hallucinations by the Description Agent could affect transparency and accuracy of the decision-making process. Future work should address these issues by enhancing dataset diversity, computational efficiency, and patch selection strategies, further advancing PathFinder's potential as a transformative tool in AI-assisted pathology.

We present RiskHarvester, a risk-based tool to compute a security risk score based on the value of the asset and ease of attack on a database. We calculated the value of asset by identifying the sensitive data categories present in a database from the database keywords. We utilized data flow analysis, SQL, and Object Relational Mapper (ORM) parsing to identify the database keywords. To calculate the ease of attack, we utilized passive network analysis to retrieve the database host information. To evaluate RiskHarvester, we curated RiskBench, a benchmark of 1,791 database secret-asset pairs with sensitive data categories and host information manually retrieved from 188 GitHub repositories. RiskHarvester demonstrates precision of (95\%) and recall (90\%) in detecting database keywords for the value of asset and precision of (96\%) and recall (94\%) in detecting valid hosts for ease of attack. Finally, we conducted an online survey to understand whether developers prioritize secret removal based on security risk score. We found that 86\% of the developers prioritized the secrets for removal with descending security risk scores.

\bibliographystyle{ACM-Reference-Format}
\bibliography{sections/Reference}




\end{document}
\endinput
%%
%% End of file `sample-sigconf-authordraft.tex'.
