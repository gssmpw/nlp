\section{Conceptualizing GenAI design fixation}
\label{section_definition}
%We suggest that a core dimension to interpret these limitations is GenAI design fixation that 
Our preliminary definition of GenAI design fixation builds on \cite{crilly2017next}'s description of design fixation "Design fixation is a state in which someone engaged in a design task undertakes a restricted exploration of the design space due to an unconscious bias resulting from prior experiences, knowledge, or assumptions”. 

%This this definition casts the process, causes and impact of design fixation

In this study, we define and explore the phenomenon of Generative AI design fixation by drawing parallels to established
concepts of design fixation as: \textit{GenAI design fixation is the state in which a Generative AI model restricts its design exploration of the generative space due to unconscious bias stemming from technical aspects and human factors, which limits the diversity and originality of the model's design output, leading to repetitive or constrained results}.

\subsection{Preliminary causes analysis}

Our investigation into the causes of GenAI design fixation began with a comprehensive review of recent literature within the HCI and AI communities. We conducted a thorough search of key research studies, creativity support tools, and technical papers pertinent to Generative AI, utilizing resources such as the ACM Digital Library and Google Scholar. Search terms included "Generative AI", "design support", and "creativity support". To enhance our understanding of the cognitive differences between GenAI and human designers, we also explored literature on design science and design cognition. This review helped identify a corpus of relevant publications, including some recent studies that discuss the limitations of GenAI in creativity support \cite{doshi2024generative, kobiella2024if, anderson2024evaluating}. The detailed review process and results are presented in Appendix A in our supplementary materials.

Our analysis suggests that the causes of GenAI design fixation extend beyond the inherent limitations of the models; human factors also play a crucial role. Thus, we initially categorize these causes into technical aspects and human factors. In the following list, the technical aspects are specifically tailored to address the context of GenAI design fixation, while the human factors illuminate the drawbacks summarized from previous literature during the human-GenAI interaction process. A more detailed analysis of these causes will be presented in Section~\ref{section_experiment}.

\begin{enumerate}
    \item \textbf{Technical aspects}
    \begin{enumerate}
        \item \textbf{Data limitation:} Data bias arise from the dataset being unbalanced or imbued with prejudiced information \cite{manduchi2024challenges}, leading to a tendency in models to learn and perpetuate these bias. Such bias can cause the model’s outputs to lack diversity in style, content, or concept, thus influencing the range of generated solutions \cite{tan2020improving, jahanian2019steerability}.
        \item \textbf{Architecture limitation:} The architecture of a model, such as the Transformer model used in text generation \cite{vaswani2017attention} or the diffusion model employed in image generation \cite{ho2020denoising}, fundamentally shapes how inputs are processed and outputs are generated. Each architecture comes with inherent constraints that potentially predispose the models to certain types of outputs, thereby affecting the generative diversity. 
    \end{enumerate}
    \item \textbf{Human factors}
    \begin{enumerate}
        \item \textbf{Misalignment in design problem definition:} During the design ideation stage, there exists a fundamental misalignment between the requirements and design goals as understood by human designers compared to GenAI systems. This discrepancy can lead to designs that do not fully meet the intended needs or expectations.
        \item \textbf{Challenges in adhering to designerly thinking:} GenAI often struggles to produce outcomes that align with the designerly way of thinking \cite{chen2024designfusion}, which emphasizes creativity, user-centered solutions, and iterative refinement \cite{zhou2024understanding}. The AI's output may not effectively reflect these nuanced aspects of design thinking, limiting the innovation potential.
        \item \textbf{Complexities of prompt writing:} Writing effective prompts for GenAI requires a deep understanding of both the design problem and the AI's capabilities. The complexity of crafting such prompts can be a significant barrier, as poorly formulated prompts may lead to irrelevant or suboptimal design outputs \cite{brown2020language, wu2022ai}.
    \end{enumerate}
\end{enumerate}

%这一段写一写design相较于其他领域的差异
% Our analysis that stood out to us in our review was the difference in GenAI design fixation compared to other helped us identify several characteristics unique to GenAI design fixation that stood out from other creativity support domain.


\subsection{Distinctions and correlations between GenAI design fixation and human design fixation}
In discussing the distinctions and correlations between GenAI design fixation and human design fixation, we follow the framework in a design fixation review research \cite{alipour2018review}, which propose fundamental factors regarding design fixation research including source, design process and design outcomes. 

%The divergences primarily originate from foundational databases and the methodology of information processing, reflecting the difference between human and machine cognition.

Unlike humans, who are susceptible to pre-conceived ideas and concepts, GenAI design fixation predominantly stems from imbalances within the dataset. During the information processing phase, humans are influenced by three cognitive phenomena \cite{jansson1991design}: "functional fixedness," which refers to the challenge in perceiving objects beyond their conventional uses \cite{duncker1945problem}; "mental set," denoting a reluctance to deviate from known strategies \cite{luchins1959rigidity}; and "the path of least resistance," indicative of the tendency to minimize effort in creative tasks \cite{ward1994structured}. In contrast, the algorithmic architecture of GenAI dictates input processing and harbors intrinsic constraints that similarly restrict creative potential, leading to less innovative solutions.

In the manifestation of design fixation, human designers tend to adhere to pre-conceived ideas and concepts, which limits the exploration of the design space during ideation. Previous findings show that this is reflected distinctly in the similar characteristics between the example solutions for design tasks and the final design outcomes. Similarly, GenAI design fixation, while capable of generating targeted solutions, remains confined within the bounds set by its training data, which prevents the models from maximizing creative potential and leads to unoriginal design solutions.

Figure~\ref{fig:process} outlines the process and dynamics of fixation within GenAI systems and how it correlates with human fixation in design processes. The flowchart details how both types of fixation move from the initial design problem through various influencing factors like input prompts and training data (for GenAI) or inspiration sources (for humans) to the generated solution or final design outcome.


\begin{figure*}[htp]
    \centering
    \includegraphics[width=0.95\textwidth]{figures/fig_process_2.png}
    \caption{The process and dynamics of fixation within GenAI systems and how it correlates with human fixation in design processes.}
    % \vspace{-0.05in}
    \label{fig:process}
    \Description{}
\end{figure*}



%To date, the majority of fixation studies focus on initial design process, which is described as "ideation fixation".


%\subsection{What affect fixation intensity}

%It has been demonstrated that the modality, degree of abstraction of the inspiration (i.e. the fidelity), and the designer's level of expertise

