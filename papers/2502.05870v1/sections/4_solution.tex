\section{Addressing GenAI design fixation in inspiration of computational creativity}
\label{section_solution}


% Firstly, strategies include using near sources as discussed by Chan et al. \cite{chan2018best}, engaging with far sources highlighted by Chan et al. \cite{chan2011benefits}, and utilizing intermediate sources such as those mentioned by Gonçalves et al. (2013) \cite{gonccalves2013inspiration}. Additionally, less common sources and methods, like written representations and partial photographs, provide alternative perspectives and are discussed in studies like Cardoso and Badke-Schaub \citep{cardoso2009give, cardoso2009idea} and Cheng et al. \cite{cheng2014new}. Biological examples and line drawings are another approach to trigger creative thinking, as explored by Wilson et al. \cite{wilson2010effects} and Cardoso and Badke-Schaub \cite{cardoso2011influence}.

% Regarding instruction and methods, building prototypes as outlined by Kershaw et al. \cite{kershaw2011effect} and Viswanathan et al. \cite{viswanathan2014study} are practical hands-on approaches. Other techniques involve abstraction and group working, noted by Cheong and Shu \cite{cheong2013reducing} and Youmans \cite{youmans2011design}. Moreover, encouraging an incubation break as seen in the works of Cardoso and Badke-Schaub \cite{cardoso2009give} and Ttsenn et al. \cite{tsenn2014effects} allows for mental refreshment that can aid in overcoming fixation.

% Some researchers focus on the instructional aspect, like understanding design fixation itself, with Howard \cite{howard2013overcoming} providing insights on mapping rules that help navigate fixation. What's more, design theories and methods have also been used to overcome design fixation. \citep{le2011interplay} studied the relationship between creativity issues and design methods and posited that design theories and methods invent models of thought to overcome fixation. They also discussed teaching based on C-K theory which is conducive to overcome design fixation. \citep{hatchuel2011teaching} also discussed teaching concept-knowledge theory can help overcome fixation effects.

% These strategies collectively offer a comprehensive toolkit for designers and innovators seeking to navigate and overcome design fixation, promoting a more expansive and effective creative process.
%我们的研究工作表明了GenAI表现出的design fixation现象可能来源于GenAI本身的训练数据集、模型架构与目标函数以及用户输入的prompt。这样的现象可能致使生成式人工智能的创意受到限制,进而让人类设计师受到误导甚至在一定程度上减小思考更高质量的设计解决方案的可能性。已有的研究表明,人类设计师的思维容易固着在生成式人工智能所产生的结果中,尤其是新手设计师,因此减轻GenAI design fixation能够在一定程度上减轻设计师产出的同质性,提高设计方案的多元性和创造性。

According to the definition of GenAI design fixation established in this study and substantiated through our experimental investigation, this phenomenon may originate from data bias, the architectural constraints of algorithms, and the prompts provided by human users. Such occurrences have the potential to limit the creative capacities of GenAI systems, thereby potentially misleading human designers and, to some extent, obstructing access to higher-quality solutions. Recent research has demonstrated that human designers might be more likely to often exhibit fixation on the outputs generated by GenAI \cite{wadinambiarachchi2024effects}. Consequently, alleviating GenAI design fixation would be instrumental in reducing the homogeneity of GenAI design outputs, which serve as stimuli for human designers, thereby enhancing the diversity and creativity of human-AI co-design solutions.

Despite the technical recommendations provided, we propose adopting methods to alleviate GenAI design fixation. Furthermore, we suggest the HCI community to consider the phenomenon of GenAI design fixation when designing and evaluating creativity support tools based on Generative AI.

\begin{enumerate}
    \item \textbf{Reduce bias in design data:} When exploring the training and application of GenAI models, data bias is a critical issue that can't be ignored. Bias in design data can affect the fairness and accuracy of the model, potentially misleading the design process of designers \cite{zhou2024bias, ferrara2023should}. These emphasize the importance of training artificial intelligence models with diverse and balanced datasets, to ensure that the models can produce more equitable, diverse, and innovative design solutions.
    %This contrast not only underscores the need for improved training approaches that can better capture the breadth of human creativity but also highlights the potential for integrating more diverse data sources to enhance the creative capabilities of AI systems.
    \item \textbf{Optimize model architecture and objective functions:} To enhance the model's innovative capacity and adaptability while avoiding over-optimization and fixation, it is necessary to optimize the model architecture and objective functions. For text and image generation models, this means exploring new architectures capable of understanding and generating complex relationships and adopting strategies to increase the model's sensitivity to low-frequency data.
\end{enumerate}


%计算创造力
% \subsection{Computational Creativity}
% The motivation for employing computational creativity \cite{mamykina2002collaborative} stems from the purpose to leverage machine capabilities to understand and emulate forms of human creativity, thereby addressing design fixation issues inherent in GenAI. 

% \begin{itemize}
%     \item \textbf{Combinational Creativity: Implications for Training Data} \\
%     To promote combinational creativity, GenAI systems can be trained on diverse datasets that blend elements from multiple domains or styles. For example, an AI trained on both architectural designs and natural forms could generate innovative structures inspired by biological features. This approach helps break away from traditional design constraints by introducing cross-domain creativity.

%     \item \textbf{Exploratory Creativity: Implications for Interaction Methods} \\
%     For exploratory creativity, designing GenAI systems that actively explore untapped areas of the creative space can lead to novel ideas. This can be achieved by implementing algorithms that perform random or directed exploration within broader, less-defined datasets. An example is an AI that learns from an eclectic mix of unlabelled artistic styles, pushing the boundaries to develop new art forms or design concepts.

%     \item \textbf{Transformational Creativity: Implications for Workflows} \\
%     Transformational creativity can be fostered by adapting solutions from one context to another through transfer learning. This involves training a model in one domain and repurposing it for another. For instance, a language model trained to generate literary works could be adapted for music composition, creating pieces that narrate a story through their melodies.
% \end{itemize}


% In summary, utilizing computational creativity within GenAI represents a promising strategy to overcome the challenges of design fixation. By harnessing diverse data, interactive techniques, and integrated workflows, we can empower GenAI systems to produce more original, varied, and contextually relevant outputs.


%工具设计层面
\subsection{Creativity support tool design}
%Besides the technical troubleshooting methods,
In this section, we give some reference strategies regarding methods to mitigate human design fixation, researchers have actively explored various strategies to mitigate design fixation \cite{linsey2010study}. According to the solution categorization proposed by \cite{alipour2018review}, the strategies can be divided into sourece, methods and instructions. In this section, we discuss mitigation methods in inspiration of the two directions in the context of Generative AI design fixation, as well as proposing strategies for interaction design. 

\subsubsection{Choosing the right kind of source}
In human design fixation study, the term ‘source’ refers to the use of previous examples and other resources as references for the solution to the current problem \cite{cai2010extended}. Studies have recommended certain strategies for reducing the effect of fixation in design. Their results have demonstrated that some sources leave enough room for exploration in design \cite{cheng2014new} and have a positive effect on the design outcome \cite{goldschmidt2011avoiding}. 
%These sources include far sources \cite{chan2011benefits}, less-common sources \cite{perttula2007idea}, biological examples \cite{wilson2010effects}, opposite terms \cite{chiu2008use}, and effective or expansive sources \cite{agogue2014impact} etc., which  can help designers to break away from established thinking patterns and explore a broader range of possibilities.

Analogously, in GenAI design fixation, studies has shown that providing additional, non-routine knowledge bases for GenAI-based CSTs can enhance the creativity and diversity of GenAI outputs. For instance, integrating specialized biological knowledge databases into GenAI systems \cite{zhu2023biologically} has been shown to expand the scope of design possibilities, encouraging the exploration of innovative solutions that might not emerge from conventional datasets \cite{kang2024biospark}. Similarly, supplying databases from specific fields of knowledge, such as materials science, cultural studies, or even niche areas of art and literature \cite{wang2024promptcharm}, can guide GenAI to venture into new, uncharted territories of design. These enriched knowledge bases serve as catalysts, prompting GenAI to generate more varied and inventive outputs.

\subsubsection{Instructions and methods}
In addition to choosing the right kind of source, studies on human design fixation has also highlighted the effectiveness of specific design instructions or systematic methods in order to help designers overcome design fixation. Several notable strategies include group working \cite{youmans2011effects}, employing alternative representations of the problem \cite{linsey2010study}, developing instructional mapping rules \cite{cheong2013using}, and utilizing the design-by-analogy method \cite{linsey2012design}. These approaches have been shown to diversify thinking and expand the range of solutions considered during the design process.

These methods can also be transferred to the field of GenAI to alleviate its design fixation. As previous empirical studies pointed out GenAI's tendency to produce surface-level or basic information, particularly in the context requiring in-depth research or exploration \cite{kobiella2024if}. For example, by employing the method of Multi-Agent Collaboration, we can endow GenAI with critical thinking and iterative capabilities. Adopting a Human-AI Collaboration system \cite{lee2024conversational} allows us to combine human creativity with AI capabilities. For instance, use GenAI to generate initial ideas, but have humans refine, combine, or expand upon these ideas. This hybrid workflow can reduce the risk of becoming fixated on AI-generated outputs. It can also guide GenAI to make broader associations, including proposing alternative representations of the problem and using analogies.

%需要补参考文献
\subsubsection{Interaction design}
In addressing the issue of design fixation in GenAI, interactive measures and the presentation of outputs play a crucial role. Firstly, by providing inspiring example solutions, users can be guided to think critically rather than being given direct answers. This can be achieved through user interfaces or workflows that stimulate creative thinking. %提供有启发的示例解决方案,批判性思考而不是直接接受。通过UI设计和工作流设计,促进创造性思考【有无例子】

Additionally, explicitly pointing out flaws in examples and providing instructions to avoid problematic elements can help users identify and circumvent potential design pitfalls. %指出示例中的不足之处,提供如何避免有问题的元素【有无这样的交互设计】

Secondly, tool customization allows users to adjust the randomness of AI-generated outputs, explore different styles, and set specific goals, thereby preventing repetitive and predictable results. \cite{liu2022design} %能够提供不同风格的选项,调整误差

% Suggesting alternative strategies: Guides proposed different ways to approach design tasks. 【例如可以使用进一步的引导】This would help designers to better comminicate design goals to the GenAI systems and develop an intuition for harnessing the AI's capabilities. \cite{gmeiner2023exploring}

\subsubsection{Improve prompt engineering}
% 新手设计师和专业设计师都会遇到prompt撰写的问题。prompt过于简略和冗余对于GenAI生成创意都是不利的。通过GenAI design fixation的透镜,prompt engineering的方向是刺激GenAI生成更加多样和新颖的方案。
As a consensus in HCI community, GenAI are sensitive to input prompts \cite{wu2022ai}, which is also one of the causes of GenAI design fixation in our study. As observed in our experiment and in line with other HCI study conclusion , novice designers often face challenges in crafting effective prompts for GenAI \cite{zamfirescu2023johnny}, with issues arising from prompts that are either too complex or too simple. Common issues include prompts that are overly complex or overly simplistic. This challenge has also been documented in previous studies focusing on creativity support for designers \cite{chen2024designfusion, liu2022design}. From the perspective of GenAI design fixation, these insights underscore the critical role of prompt engineering in fostering more diverse and innovative design outputs.

% 之前有研究帮助生成更多样的方案吗?
Effective prompt engineering requires a balance between clarity and flexibility, enabling GenAI to explore a broad range of creative possibilities. Prompts should guide the GenAI with enough specificity to maintain relevance while allowing room for creative interpretation. For example, specifying a theme or mood can direct the AI while still permitting innovative variations.
though the GenAI's stochastic nature to explore possibilities could yield creative ideas sometimes, the trial-and-error process is time-consuming.

% 如何迭代prompt,又有什么成功经验呢。prompt attributes是怎么划分和分析的
Iterative refinement is also crucial \cite{mahdavi2024ai}. By evaluating the outputs from initial prompts, designers can adjust subsequent prompts to better harness GenAI’s creative potential. This process involves a systematic approach where experimenting with various prompts can help understand how different prompt attributes affect the creativity of the outputs. Ultimately, the goal is to develop a toolkit for designers that supports the consistent elicitation of high-quality creative outputs from GenAI systems, enhancing the technology’s role in the creative process.

% 回顾过去HCI领域关于prompt engineering的研究,研究目标大多是align GenAI models and users' intents, or improve the quality of the output \cite{mahdavi2024ai}. 对于text generation model, \cite{wu2022ai} 针对transparency, controllability, and sense of collaboration问题,提出了LLM chains的交互方式,without the need to retrain the model.对于image generation model, \cite{liu2022design} found that it focuses on subject and style keywords rather than function words. \cite{mahdavi2024ai} identify prompt structures and how users evaluating AI-generated images.


%教育层面
\subsection{Education}
\subsubsection{Understanding the phenomenon of GenAI design fixation}
Howard et al.'s study \cite{howard2013overcoming} reflects that educating students in the phenomenon and effects of fixation enables them to effectively devise their own strategies to avoid or overcome fixation. So the first step in addressing design fixation within GenAI involves ensuring that designers comprehend the nature and implications of this phenomenon. Our work in align with \cite{anderson2024homogenization}'s findings that users given a sense of what the GenAI tend to suggest in similar contexts could help mitigate homogenization effects. In the context of GenAI, design fixation specifically pertains to the GenAI's tendency to generate outputs that adhere too closely to learned patterns and examples, thereby stifling novel and diverse designs.

\subsubsection{Users need to critically evaluate AI-generated content}
To mitigate the effects of GenAI-induced design fixation, it is crucial to train users to critically evaluate AI-generated content rather than accepting it passively. The importance of guiding designers reflection on generated designs is also proposed in other HCI research, such as \cite{gmeiner2023exploring}. This training should encompass an understanding of the underlying algorithms to some extent, enabling users to recognize the inherent limitations and bias of the GenAI. By fostering a critical mindset, asking questions like \textit{“Is there a better design approach?”} and \textit{“What limitations might the GenAI-generated results have?”}, designers can more effectively assess the suitability and originality of AI outputs. This approach ensures that these tools serve as a starting point for further creative development rather than as definitive solutions.


% \subsubsection{Utilizing AI outputs as inspiration material}
% Another effective strategy is to encourage designers to use AI-generated content as inspiration material rather than adopting it as direct solutions. By treating AI outputs as a source of ideas and stimuli, designers can explore a broader range of possibilities and avoid the trap of design fixation. This approach promotes a more dynamic and iterative design process, where AI serves to enhance human creativity rather than constrain it.

%GenAI design fixation resembles a creativity support dilemma: GenAI could offer novel ideas to users, but the actual solution exploration and transformational creativity is depends on designers' creativity level

%GenAI CST工具研究的评估标准
%可能需要回顾一下过去CST评估标准
\subsection{Evaluation metrics for GenAI-based CSTs research}

To effectively evaluate GenAI-based creativity support tools (CSTs), it is essential to incorporate design fixation as a critical standard. Design fixation, the tendency to become overly influenced by existing examples or solutions, can significantly hinder creativity and innovation. Therefore, any assessment framework for GenAI-based CSTs must rigorously examine the extent to which these tools either mitigate or exacerbate design fixation.


%\cite{calderwood2020novelists}提出diversity作为评估LLM的标准之一
\subsubsection{Positioning of GenAI CST Tools}

GenAI CST tools should be positioned not merely as instruments for increasing the efficiency and quantity of design outputs but as catalysts for enhancing designers' creative and innovative capacities. The primary goal of these tools should be to stimulate designers' thinking, helping them generate more creative, innovative, and groundbreaking ideas and inspirations, which contrasts with a narrow focus on efficiency and productivity.

A key consideration is to avoid overly programmatic workflows that may strengthen design fixation. While structured processes can streamline tasks, they might also limit the creative potential of designers by promoting adherence to predefined patterns and solutions. The development of GenAI-based CST tools should not solely focus on maximizing the AI's capabilities. Given the inherent risk of fixation within GenAI, it is crucial to design these tools in a way that incorporates significant designer participation. Emphasis should be placed on how these tools can engage the designer's subjective agency and imagination, thus fostering a more collaborative and dynamic creative process.

\subsubsection{Comprehensive evaluation of GenAI CSTs}

The evaluation of GenAI CST tools should extend beyond traditional usability tests and the resolution of issues identified in formative studies or user studies. While usability and problem-solving are important metrics, they do not fully capture the tools' impact on designers' cognitive and creative processes. Research outcomes should include rigorous assessments of how well these tools stimulate designers' cognitive engagement and creative thinking. This involves evaluating whether the CST tools genuinely enhance designers' ideation processes and their ability to conceptualize innovative solutions. Evaluations should also measure the impact of CST tools on the overall vibrancy and originality of designers' thought processes. This can include metrics such as the diversity and novelty of ideas generated, the ability to break away from conventional patterns, and the overall enhancement of creative problem-solving skills.

By integrating these comprehensive evaluation criteria, research on GenAI-based CSTs can ensure that these technologies not only address practical usability concerns but also significantly contribute to the advancement of creative design practices. This holistic approach will help in developing tools that truly empower designers, fostering an environment where human creativity and AI capabilities synergistically drive innovation while effectively mitigating the risk of design fixation.