\section{Discussions}

\subsection{Distinctions about GenAI design fixation and other terms related to GenAI limitations}
In this section, we clarify the distinctions between GenAI design fixation and other terms often discussed in HCI associated with limitations within the GenAI research domain. We believe that understanding these differences would be beneficial for effectively addressing the specific challenges that arise in the development and application of Generative AI technologies

% \subsubsection{GenAI design fixation} 
% GenAI design fixation refers to the state in which a GenAI model restricts its exploration of the generative space due to unconscious bias stemming from data bias, algorithmic architecture, or input prompts, which limits the diversity and originality of the model's design output, leading to repetitive or constrained results. This phenomenon is particularly relevant in creativity tasks, where the generation of novel and diverse ideas is essential, and significantly impacts the novelty and diversity of the generated creative content, as the model is less likely to explore unconventional or innovative design spaces. Consequently, this limits the potential for groundbreaking ideas, reducing the overall quality and variety of design outputs. For design practice, this means that designers, especially those heavily reliant on GenAI tools, may find themselves constrained by the model's limited creative exploration, resulting in a homogenization of design solutions.

\subsubsection{GenAI error}
GenAI error refers to the phenomenon where GenAI produces content that contradicts factual information \cite{liu2024smart}. This issue is particularly pertinent in contexts such as code generation \cite{ebert2023generative}, where bugs or unwanted code fragments may be introduced, or in text generation, where factual inaccuracies can occur. In the domain of AI-generated text, this can manifest in various applications including reasoning, translation, summarization and paraphrasing, and content evaluation, as well as in the creation of creative content. The presence of GenAI errors in these contexts can lead to misleading outcomes, which not only diminish the reliability and accuracy of the generated content but also potentially misguide users. However, AI “error” also plays a positive role in the early ideation stage, supporting designers to find association through four forms of creativity named CETR \cite{liu2024smart}.


\subsubsection{GenAI hallucination} 
GenAI hallucination refers to the phenomenon where Generative AI produces content that is nonsensical or unfaithful to the provided source material \cite{ji2023survey}. This issue is particularly salient in several key applications, including abstractive summarization \cite{pagnoni2021understanding}, data-to-text generation \cite{wiseman2017challenges}, dialogue systems \cite{li2020slot, santhanam2021rome}, generative question answering (GQA) \cite{nguyen2016ms}, translation tasks \cite{zhou2020detecting}, and computer vision \cite{zhou2023analyzing}. In these contexts, hallucinated content can lead to misinterpretations and inaccuracies, undermining the reliability and trustworthiness of the AI-generated output. However, in the realm of creative content generation, while hallucinations can introduce misleading elements, they can also paradoxically stimulate designers' creativity by providing unexpected and novel ideas. This dual impact underscores the complexity of addressing GenAI hallucinations: while it is crucial to minimize their occurrence to ensure the fidelity and accuracy of AI-generated content, it is also important to recognize their potential to inspire innovative thinking and creative exploration.


\subsubsection{GenAI bias}
GenAI hallucination refers to the presence of systematic misrepresentations, attribution errors, or factual distortions that result in favoring certain groups or ideas, perpetuating stereotypes, or making incorrect assumptions based on learned patterns. This phenomenon is particularly relevant in the contexts of image generation \cite{zhou2024bias} and text generation \cite{ferrara2023should}. In these domains, hallucinations can manifest as biased or inaccurate representations that reinforce existing societal bias and stereotypes. By highlighting and confronting these bias, designers may be inspired to develop more inclusive and equitable creative solutions.


\subsubsection{GenAI creativity}
GenAI creativity refers to the ability of GenAI systems to produce novel and innovative outputs in creative domains such as art, design, music, and writing. These systems are increasingly used to generate unique artworks, pushing the boundaries of traditional processes by introducing fresh perspectives and solutions. However, despite these advancements, the outputs of GenAI, particularly in writing, may not always match the depth and expertise of human specialists. While large language models can produce coherent text, they often lack the nuanced quality that expert human writers possess \cite{chakrabarty2024art}. The impact of GenAI creativity is dual: it democratizes access to creative tools, fostering a more diverse creative ecosystem, but also highlights the irreplaceable value of human expertise. Thus, GenAI should be seen as a complement to human creativity, enhancing rather than replacing it.

%Li jie 那一篇有谈到

\subsection{Pros and cons of GenAI design fixation for human creativity}

With the prevailing trend in applying GenAI into huamn creativity support, and given that recent studies indicate designers are more prone to fixating on GenAI outputs \cite{wadinambiarachchi2024effects}, it is crucial to discuss the impact of GenAI design fixation on GenAI outputs, the ideation process of designers, and the final design outcomes. Similar to the perspectives on design fixation within the design community \cite{smith1993constraining, bilalic2008good}, we consider the effects of GenAI design fixation to be a double-edged sword.
%能不能同时讨论对于GenAI产出,设计师创意过程,以及最终设计成果的影响

For the advantages of GenAI design fixation:
\begin{enumerate}
    \item \textbf{Efficiency and Speed: }Similar to human designers who rely on familiar patterns or established solutions, GenAI can use design fixation to quickly generate outputs based on existing data and learned patterns. This can significantly reduce the time required to produce design variations, especially in contexts where rapid prototyping or iteration is necessary.
    \item \textbf{Consistency and Reliability: }GenAI design fixation can help maintain consistency across a series of designs by adhering to specific, proven design elements. This is particularly useful in fields where uniformity is critical, such as branding, where a cohesive visual identity must be maintained.
    \item \textbf{Leveraging Established Best Practices: }By fixating on successful designs or widely accepted principles, GenAI can ensure that its outputs are grounded in established best practices. Besides, studies have shown that access to GenAI ideas may increase the creativity and quality of the outputs,especially among less creative or novice users \cite{doshi2024generative}.
    \item \textbf{Cost-Effectiveness: }By reusing and adapting existing design elements, GenAI can reduce the resources needed for innovation, thereby saving costs related to experimentation, testing, and validation. This is similar to how repeating past solutions in human design can save time, money, and effort.
\end{enumerate}

For the disadvantage of GenAI design fixation:
%long-term影响还是比较明显的
\begin{enumerate}
    \item \textbf{Limited creativity and innovation: }One of the major drawbacks of design fixation, whether human or AI-driven, is the potential stifling of creativity. When GenAI overly relies on existing patterns or past data, it may struggle to generate truly novel or innovative designs, limiting the scope of creative exploration.
    \item \textbf{Reinforcement of bias: }If the data GenAI is trained on reflects certain bias or limitations, fixation on these patterns can perpetuate and even amplify these bias. This can lead to a lack of diversity in design outputs and may inadvertently reinforce stereotypes or outdated concepts. 
    \item \textbf{Adaptability challenges: }In rapidly changing fields, adherence to past designs may result in outputs that are not well-adapted to new trends or emerging user needs. GenAI design fixation might make it difficult for AI to pivot and adapt to new paradigms, leading to designs that feel outdated or out of touch.
    \item \textbf{Box human creativity in the long run: }As a recent empirical study find that the use of ChatGPT in creative tasks resulted in increasing homogenized contents, and this homogenization effect persisted even when ChatGPT was absence, which highlights the challenge of boxing human creative capability in the long run \cite{liu2024chatgpt}. Constantly relying on familiar solutions can lead to design outputs that are safe but unremarkable. This “safe” approach might prevent GenAI from pushing the boundaries and exploring more groundbreaking or avant-garde design possibilities, which could limit its usefulness in highly creative or forward-thinking industries.
\end{enumerate}

In summary, GenAI design fixation is not always counterproductive; rather, it can promote a more focused and thorough exploration of viable solutions, which indicate that the impact of GenAI design fixation can vary greatly depending on the context, the specific demands of the project, and the creative framework employed by the designers. The perspective of GenAI design fixation may bring new insights to balance design efficiency and design innovation in GenAI-based co-creation process.

%design fixation的理论支撑
%Just the same as the dual perspective of design fixation, (e.g. As Smith, Ward, and Schumacher \cite{smith1993constraining} suggest, relying on established solutions does not always lead to suboptimal results; its impact can depend significantly on the specific context and application, repeating past solutions can provide economic advantages by saving money, time, and effort \cite{viswanathan2014study, viswanathan2010physical} )where the GenAI design fixation 
% We begin by reviewing the literature’s evaluation of design fixation within the design field—a topic often viewed from a dual perspective. 

% Moreover, some studies highlight the positive and beneficial aspects of design fixation. For example, . Additionally, contrasting viewpoints in the literature suggest that fixation may have functional, if not entirely positive, implications for design. Bilalic, McLeod, and Gobet \cite{bilalic2008good} argue that focusing intensively on a narrow set of solutions could streamline the design process, thereby reducing the time and costs associated with a project.




%扩展研究范围:
%Without loss of generality, this work applies to the exploiting “error” in other AI technologies for inspiration, and aims to encourages further endeavors to turn AI “error” into treasure.

\subsection{Limitations and future directions}
While not exhaustive, this study marks the commencement of proposing a theoretical framework for GenAI design fixation and explores its manifestations through empirical investigation. Our research employs two representative generative models (GPT-4o and Midjourney); however, in practice, different GenAI models may produce varying results in response to the same prompts. This potential variability was not explored in our study, representing a limitation of our current methodology. Moreover, by specifically targeting novice designers as participants, this study inherently narrows the scope of feedback regarding the impact of GenAI design fixation on the ideation process. 

%Another limitation of our research is the scope of We aim to inspire future research in various domains, such as creative writing, music composition, drawing, coding.

In this study, we conducted an experimental investigation and documented the manifestations of design fixation within the context of product design, aligning with established precedents in human design fixation research \cite{crilly2017next}. Future studies should consider broadening the research scope to other areas of creativity support. We particularly recommend further investigations in the fields of creative industries, product design, and conversational agents. The manifestations of GenAI design fixation might vary across different task domains, yet they should be compatible within our proposed framework for GenAI design fixation. Moreover, future work could involve conducting literature reviews through the lens of GenAI design fixation, exemplified by \cite{chakrabarty2024art}, which demonstrates the utility of leveraging comprehensive surveys of Creative Support Tools (CSTs).

%部分领域可能对于fixation拥有更高的tolerance
