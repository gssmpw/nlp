\section{Findings}
\label{sec: result}
In this section, we analyze the manifestations of GenAI design fixation by dividing our analysis into text generation models and image generation models (RQ1). We summarize these fixation manifestations based on the results of our quantitative analysis of text and image data, as well as the design solutions submitted by the participants. Subsequently, we delve into the feedback provided by participants regarding their recognition of GenAI design fixation (RQ2) and its impact on their ideation processes (RQ3).

\subsection{Manifestations in text generation design fixation}
During the data pre-processing stage of our quantitative analysis (see detailed results in Table~\ref{tab:chair_parts_comparison}), we found that the Red Dot data contained more chair design-structure-related keywords than the ChatGPT solutions (208.57 and 142.71 respectively, normalized per 100 cases). After excluding these chair-structure-related words, we focused on analyzing function- and aesthetic-related keywords. For the Red Dot data, we extracted a total of 398 keyword items with a cumulative frequency of 1,104. In the ChatGPT data, 266 keyword items were identified with a cumulative frequency of 712. Our subsequent analysis concentrated on the frequency of these two sets of keywords.

%椅子设计相关words的词频分析
\begin{table}[ht]
\centering
\caption{Comparison of frequencies of keywords related to chair structural elements in design solutions in Red Dot and ChatGPT, normalized per 100 cases.}
\label{tab:chair_parts_comparison}
\begin{tabularx}{0.5\textwidth}{Xrr}
\toprule
\textbf{Part} & \textbf{Red Dot (per 100)} & \textbf{ChatGPT (per 100)} \\
\midrule
Seat          & \calc{70}{105}             & \calc{44}{96}                   \\
Leg           & \calc{49}{105}             & \calc{10}{96}                   \\
Backrest      & \calc{29}{105}             & \calc{33}{96}                   \\
Frame         & \calc{23}{105}             & \calc{9}{96}                    \\
Base          & \calc{22}{105}             & \calc{11}{96}                   \\
Armrest       & \calc{15}{105}             & \calc{14}{96}                   \\
Cushion       & \calc{7}{105}              & \calc{9}{96}                    \\
Footrest      & \calc{2}{105}              & \calc{5}{96}                    \\
Headrest      & \calc{2}{105}              & \calc{2}{96}                    \\
\midrule
\textbf{Total} & \textbf{\calc{219}{105}}  & \textbf{\calc{137}{96}}         \\
\bottomrule
\end{tabularx}
\end{table}

Figure~\ref{fig_word_frequency} displays the top 10 most frequent word derived from both the ChatGPT-generated and Red Dot design solutions. Notably, both datasets share five common words among their top 10 frequencies: “comfort”, “light”, “ergonomic”, “innovative”, and “aesthetic”. This overlap highlights recurrent keywords in both human and GenAI-generated design solutions.

\begin{figure*}[htp]
    \centering
    \includegraphics[width=0.95\textwidth]{figures/fig_frequency.png}
    \caption{Comparison of the top 10 most frequent word stems in design solutions generated by ChatGPT and those recognized in Red Dot award-winning designs. Common terms across both datasets are highlighted and include "comfort", "light", "ergonomic", "innovative", and "aesthetic".}
    % \vspace{-0.25in}
    \label{fig_word_frequency}
    \Description{}
\end{figure*}

We also calculated the proportion of unique items in the two set of data. The item counts and frequencies of unique and shared words for Red Dot and ChatGPT solutions are listed in Table~\ref{tab:word_frequencies}. During assessment, we focus on the counts of distinct and shared word entries rather than their total frequencies. This approach highlights the diversity of terms used rather than the volume of usage. Applying formula \ref{eq:novelty_ratio}, the proportions of novelty for both the GenAI-generated data (ChatGPT) and human-generated data (Red Dot) are calculated ($P_{novelty}^{ChatGPT}$ = 67.4\%, $P_{novelty}^{Red Dot}$ = 78.2\%). This demonstrates that the human-generated data exhibits a higher proportion of unique terms, implying a greater diversity in creative outputs compared to the ChatGPT-generated data in our experiment. 

\begin{table}[ht]
\centering
\caption{Item counts and frequencies of unique and shared words for Red Dot and ChatGPT solutions.}
\label{tab:word_frequencies}
\begin{tabular}{lrrrr}
\hline
& \multicolumn{2}{c}{Red Dot} & \multicolumn{2}{c}{ChatGPT} \\
\cline{2-5}
& Item Counts & Frequency & Item Counts & Frequency \\
\hline
Unique Words & 312 & 557 & 180 & 297 \\
Shared Words & 86 & 547 & 86 & 361 \\
\hline
\end{tabular}
\end{table}

Based on the quantitative analysis above and the manual analysis by two of the researchers, we categorized the design fixation in text generation models into four dimensions: \textit{Descriptive statements}, \textit{Repetitive theme}, \textit{Limited contextual variation}, \textit{Susceptibility to high-frequency words}. A summary of these dimensions, along with detailed explanations and examples, is provided in Table~\ref{tab:manifestations_text}.

\textbf{Descriptive Statements.} This refers to the phenomenon where the generated content often appears creative at first glance but lacks depth or clear operational mechanism, primarily focusing on wordplay rather than functional innovation. For instance, when P2 prompted ChatGPT to generate chair designs based on a puzzle structure, the solutions ChatGPT produced (e.g., \textit{An office chair designed with interlocking pieces that can be assembled or disassembled, reflecting the puzzle's structure.}) were mainly superficial descriptions that did not effectively incorporate the design characteristics or structural elements essential to chair design. \textit{P2: "It would be prudent to explore the potential of utilizing assembled or disassembled features, but how remains a question."} This manifestation aligns with previously discussed challenges concerning the limited reasoning capabilities of LLMs \cite{wu2022ai}. These challenges highlight that LLMs often only grasp the form of language without truly understanding the underlying meaning \cite{bender2020climbing}.

\textbf{Repetitive Theme.} This dimension refers to the recurrence of similar concepts or thematic elements across different outputs, where the model consistently generates alike ideas regardless of the specific input details. For instance, when P4 requested a design solution themed around "cloud," aiming for a cozy and soft atmosphere, ChatGPT continued to produce descriptions typical of modern tech offices, characterized by sleek and minimalist designs. This indicates that the model fixated on the common association of "cloud" with technology and cloud computing, neglecting the participant's intended theme of comfort and softness. Such fixation limits the diversity of ideas and can hinder the creative exploration of alternative interpretations of a theme.

\textbf{Limited Contextual Variation.} This dimension highlights the model's difficulty in adapting its responses to different contextual cues, leading to similar outputs even when the context changes significantly. For example, when P3 requested design solutions for a pet area suitable for working from home, ChatGPT continued to provide suggestions appropriate for office chair design, fixating on language and concepts typically used in that context rather than adapting to the new scenario. This suggests that the model is not effectively incorporating new contextual information into its responses, resulting in a narrow range of solutions that may not align with the user's specific needs or the unique aspects of the design challenge.

\textbf{Susceptibility to High-Frequency Words.} This reflects how GenAI often disproportionately favors words or phrases that frequently appear in its training data. This bias may lead to creative solutions that feel incongruent or limited in their inventiveness. For instance, when P6 instructed ChatGPT to generate a creative design based on bamboo, the output was unexpectedly anchored to high-frequency terms "ergonomics": "A bamboo chair designed with an ergonomic structure to enhance comfort and posture support". This response highlights the model's tendency to revert to common concepts like ergonomics, regardless of the specific creative context provided. The comparison of word frequencies between ChatGPT and Red Dot projects, as illustrated in Figure~\ref{fig_word_frequency}, further reveals this disparity. The top words in ChatGPT outputs, such as "ergonomic", "sustainable", and "modular", contrast sharply with more diverse terms found in human-generated, award-winning design descriptions. 

\begin{table}[ht]
\centering
\caption{Manifestations of design fixation in text generation by GenAI.}
\label{tab:manifestations_text}
\begin{tabular}
{p{0.2\textwidth}p{0.75\textwidth}} 
\toprule
\textbf{Dimensions} & \textbf{Explanations}\\ 
\midrule
\midrule
Descriptive statements & The generated text appears creative at first glance but lacks depth or practical application, focusing more on word play rather than functional innovation.\\
\midrule
Repetitive theme & The recurrence of concepts or thematic elements across different outputs, repeatedly churning out similar ideas or concepts regardless of the input specifics.\\
\midrule
Limited contextual variation & The model might struggle to adapt its responses to different contextual cues, leading to similar outputs across varying contextual requirement.\\
\midrule
Dependency on high-frequency words & GenAI disproportionately favors words or phrases that appear frequently in its training data, potentially overshadowing more relevant but less frequent terms.\\
\bottomrule
\end{tabular}
\end{table}

% \begin{figure*}[htp]
%     \centering
%     \includegraphics[width=0.95\textwidth]{figures/fig_text_similarity.png}
%     \caption{\textbf{Manifestations of design fixation on text generation models from our experiment displayed on the left, with participants' feedback provided on the right.}}
%     \vspace{-0.05in}
%     \label{fig:manifestation_text}
%     \Description{}
% \end{figure*}

\subsection{Manifestations in image generation model design fixation.}
Table~\ref{tab:pairwise_distance} presents the pairwise distance analysis results between Red Dot and Midjourney image datasets, which reveals significant differences across four key design attributes: global features, shape, color, and texture. Notably, the global feature distance is higher for Red Dot solutions (\(M = 16.2509, SD = 3.2638\)) compared to Midjourney-generated solutions (\(M = 16.0503, SD = 4.3814\), \(p = 0.0238^*\)). Additionally, for all other categories—shape, color, and texture—the pairwise distances are consistently larger in the Red Dot solutions compared to Midjourney, with significant \(p\)-values indicating these differences (\(p < 0.01\)). Besides, the result of exploratory analysis of the image attributes is shown in Figure~\ref{fig_figure_TSNE}, using the visualization of t-SNE dimensionality reduction applied to the embeddings from Midjourney-generated chair images. We marked several representative clustering to facilitate further analysis of the GenAI design fixation it represents.

\begin{table*}[ht]
    \centering
    \caption{Pairwise distance analysis between Red Dot and Midjourney image datasets.}
    \label{tab:pairwise_distance}
    \renewcommand{\arraystretch}{0.8}  % Reduces row spacing
    \begin{threeparttable}
        \begin{tabularx}{0.7\textwidth}{l *{2}{>{\centering\arraybackslash}X} *{2}{>{\centering\arraybackslash}X} c}
            \toprule
            Category & \multicolumn{2}{c}{Red Dot} & \multicolumn{2}{c}{Midjourney} & p-value \\
            \cmidrule(lr){2-3} \cmidrule(lr){4-5}
            & Mean & SD & Mean & SD \\
            \midrule
            Global   & \textbf{16.2509} & 3.2638 & 16.0503 & 4.3814 & 0.0238\tnote{*} \\
            Shape    & \textbf{0.0700} & 0.0003 & 0.0592 & 0.0003 & 0.0000\tnote{**} \\
            Color    & \textbf{0.0821} & 0.0003 & 0.0685 & 0.0002 & 0.0000\tnote{**} \\
            Texture  & \textbf{0.0882} & 0.0004 & 0.0862 & 0.0006 & 0.0000\tnote{**} \\
            \bottomrule
        \end{tabularx}
        \begin{tablenotes}
    \item[] \textbf{Notes:}
      \item 1. Bold font represents bigger pairwise distance.    
    \item 2. ** denotes $p < 0.01$ and * denotes $p < 0.05$.
    \end{tablenotes}
    \end{threeparttable}
\end{table*}

\begin{figure*}[htp]
    \centering
    \includegraphics[width=0.95\textwidth]{figures/fig_TSNE.png}
    \caption{Visualization of t-SNE dimensionality reduction applied to the embeddings from Midjourney-generated chair images.}
    % \vspace{-0.25in}
    \label{fig_figure_TSNE}
    \Description{}
\end{figure*}

Based on the analysis results above and the manual analysis for users' submitted solutions, we categorized the design fixation in image generation models into seven dimensions: \textit{Shooting angle}, \textit{Image generation patterns}, \textit{Image attributes}, \textit{Restrained on prompt}, \textit{Limited ability to visualize novel solution descriptions}, and \textit{Dependency on high-frequency visual motifs}. A summary of these dimensions, along with detailed explanations and examples, is provided in Table~\ref{tab:manifestations_image} with corresponding examples shown in Figure~\ref{fig_image_manifestion}.

\textbf{Shooting angle.} This indicates that generated images typically showcase a limited range of angles or viewpoints, predominantly those most commonly represented in the training data. Specifically, in our study, out of 96 user-submitted chair design solutions, only 2 depicted front views; the remainder predominantly featured 45-degree angles.

\textbf{Image generation patterns.} Our analysis revealed two distinct image generation patterns in the Midjourney outputs: surface mapping replacement of common chairs and the recombination of textures and shapes. As shown in Figure~\ref{fig_image_manifestion}-b2, these patterns represent GenAI’s tendency to focus on aesthetic alterations rather than overall structure consideration. For instance, P1 and P2’s designs replace the conventional ergonomic chair surfaces with Mondrian color blocks and puzzle-themed patterns, adding a superficial layer to the standard chair form. On the other hand, P8’s design exemplifies texture recombination by integrating spacecraft materials into the chair’s surface, which visually disrupts its traditional appearance without modifying its functional structure. In another instance, the expanded form of a folding fan was used to replace the traditional chair back, serving as a purely visual recombination that does not consider the design’s practicality or structural feasibility.

\textbf{Image attributes.} This observation comes from the clustering phenomenon observed in the t-SNE reduction of color, shape, and texture attributes. As shown in Figure~\ref{fig_image_manifestion}-b3 (consistent with the marked areas in Figure~\ref{fig_figure_TSNE}), despite different prompts like P6-LED light, P9-feather, and P10-spider net, which have varying additives and requirements, the output images exhibit similar color patterns. This phenomenon is also evident in the dimensionality reduction results for shape and texture.

\textbf{Restrained on prompt.} As exemplified in Figure~\ref{fig_image_manifestion}-b4, whether the prompts are complex or overly simplistic, the output frequently converges on a conventional office chair design. This indicates a limitation in the model’s response range to diverse prompt complexities, often defaulting to standard, familiar outputs regardless of the creative potential of the input.

%这一点对于未来CST发展还是挺有启发意义的。不能指望GPT做一切
\textbf{Limited ability to visualize novel solution descriptions.} As demonstrated in Figure~\ref{fig_image_manifestion}-b5, when P7 describes a chair featuring a neck brace—a component possibly absent from the training data—the model fails to accurately visualize this feature in the generated image. This highlights the model’s struggle to produce images that depict novel solutions or concepts not represented within its training dataset.

\textbf{Dependency on high-frequency visual motifs.} Image generation models often over-rely on visual motifs that appear frequently within their training datasets. For instance, in the generation of chair images, TNSE analysis of global features in Figure~\ref{fig_image_manifestion}-b6 reveals that the model consistently produces similar designs for armrests and bases. This pattern reflects the prevalent styles learned from the training data and indicates a limited exploration beyond these familiar designs, despite the potential for a broader array of creative interpretations.
%我们统一不考虑背景的影响“A model trained primarily on indoor scenes might include background elements like windows or plants even when generating concepts for outdoor furniture."


\begin{table}[ht]
%\scriptsize
\centering
\caption{Manifestations of design fixation in image generation by GenAI.}
\label{tab:manifestations_image}
\begin{tabular}
{p{0.2\textwidth}p{0.75\textwidth}}
\toprule
\textbf{Dimensions} & \textbf{Explanations}\\ 
\midrule
\midrule
Shooting angle & Generated images primarily feature a limited set of angles or viewpoint, often those most commonly seen in the training data. \\
\midrule
Image generation patterns & Images are mostly produced through replacing the mapping of the outer surface of the common chair, or recombination of textures and shapes, lacking more creative embodiment methods. \\
\midrule
Image attributes & Marked homogeneity in attributes such as color, shape, and texture. The model tends to replicate these familiar attributes regardless of the input prompt's requirements. \\
% \midrule
% Inability to Generate Abstract Concepts & Generates only concrete images. & xxx \\
\midrule
Restrained on prompt & Complex or overly simple prompts generally result in obtaining a conventional office chair. 
\newline
Creative combinations often require users to provide hints in the prompt. \\
\midrule
Limited ability to visualize novel solution descriptions & The model struggles to generate images that depict solutions or concepts not present within its training data. \\
\midrule
Dependency on high-frequency visual motifs &  Image generation models tend to overuse visual motifs that are frequently encountered within their training datasets. \\
% \midrule
% Limited Capability in Expressing Novel Descriptions & When generating less common design options, the output tends to use familiar ergonomic chair designs as a base & "When prompted with unique design descriptions, the AI defaulted to modifying standard ergonomic chair templates." \\
\bottomrule
\end{tabular}
\end{table}

\begin{figure*}[htp]
    \centering
    \includegraphics[width=0.95\textwidth]{figures/fig_figure_example.png}
    \caption{Manifestations of design fixation on image generation models from our experiment displayed on the left.}
    \vspace{-0.05in}
    \label{fig_image_manifestion}
    \Description{}
\end{figure*}
%\subsubsection{Predictability in GenAI design fixation}

%通过总结text generation和image generation中design fixation的范式,可以发现GenAI生成的结果具有一定的可预测性。AI与人类在创意生成过程中的不同思维模式。
\subsection{Participants' feedback}
\label{subsection_Participants'_feedback}

In this section, we discuss participants' feedback on their experience interacting with GenAI during the office chair design ideation stage. The feedback is drawn from the interview data as well as the observations made by researchers during the experiment. 

\textbf{Initial ideation.}
Approximately one-third (30\%, n=3) of the participants directly asked ChatGPT for creative ideas related to designing chairs for office settings or requested innovative "additives" during the experiment. Other participants chose to provide their own ideas and concepts to ChatGPT, asking it or GenAI-based Creativity Support Tools (CSTs) to further explore and expand upon these initial inputs. This suggests that the influence of GenAI outputs on human designers' ideas may begin to take effect early in the design process. For more experienced designers, critical thinking is applied to evaluate ChatGPT's suggestions. For instance, P8 remarked during the experiment that many of ChatGPT's creative elements, whether visual or functional, were overly focused on intelligentized or futuristic design features (e.g., adding LED lights). As a result, P8 rejected all of ChatGPT’s proposals and developed his own design ideas instead. In contrast, novice designers may be more inclined to adopt ChatGPT's ideas directly. For example, P6 requested innovative solutions from ChatGPT regarding form and color and subsequently fed these unaltered ideas into Midjourney for further development.\\

\textbf{Tool usage habits and prompt iteration.}
During the co-creation process using ChatGPT and Midjourney, nearly all participants tended to directly copy the product descriptions or Midjourney prompts generated by ChatGPT and input them into Midjourney without modification, and only two participants independently modified the Midjourney prompt provided by ChatGPT during the experiment. Some participants, such as P3, indicated that they paid little attention to the specific content of ChatGPT's product descriptions or Midjourney prompts. Instead, they focused on evaluating whether the final outcomes produced by Midjourney were sufficiently innovative. As a result, these participants often overlooked potential issues in ChatGPT's suggestions, as well as the risk of the solutions lacking creativity and diversity.\\

\textbf{Recognition of GenAI design fixation.}
Overall, when engaging with ChatGPT and Midjourney, seven and eight participants, respectively, responded affirmatively to the interview question (1): "Have you noticed any repetition or similarities in the ideas or designs generated by the tools?" This indicates a recognition of some level of GenAI design fixation during the 30-minute formal experiment involving interactions with GenAI.

Regarding the text generation model, their insights primarily focused on ‘fixation on descriptive statements’ (P2) and ‘susceptibility to high-frequency words’ (P4, P9) as cataloged in Table~\ref{tab:manifestations_text}. For instance, when P2 requested a creative chair design inspired by a fountain from ChatGPT, the response was, \textit{“An office chair designed with layered tiers for enhanced ergonomic support, inspired by the structure of a fountain”.} P2 commented, \textit{“The response from ChatGPT did not integrate well with the chair’s structure; it required further prompting to combine the tiered structure of the fountain with the cushion structure to generate a more specific answer.”} Besides, P9 noted, \textit{“The responses from ChatGPT often contain descriptions like ‘modular,’ ‘adaptive,’ ‘automatically adjustable,’ and ‘sensors,’ which I encountered even when designing chairs based on yoga balls and mechanical structures.”} P4 identified ‘ergonomics’ as a frequently mentioned term.

A greater number of participants noticed fixation phenomena in the image generation model, primarily encompassing aspects listed in Table~\ref{tab:manifestations_image}. These include b2-fixation on image generation patterns, such as replacing the mapping (P1) and recombining shapes regardless of ergonomics (P8); b5-limited ability to visualize novel solutions, exemplified by the inability to integrate a neck brace into the design (P7); and b6-dependency on high-frequency visual motifs, noted by P2 in terms of similar and commonly seen base and armrest structures. The respective cases from participants are displayed in Figure~\ref{fig_image_manifestion}, illustrating these specific manifestations of design fixation within the GenAI output.\\

%此外,对于GenAI design fixation察觉较少的,是设计经验较少以及对于生成式AI了解较少的P6,对于interview question

\textbf{The impact of GenAI design fixation on design ideation.}
Based on insights from the design process and interview data with participants, it is evident that the impact of GenAI design fixation on design ideation \textbf{exhibits individual differences.} More experienced designers not only have a broader and richer understanding of design case libraries but also possess relatively mature design thinking and established processes. These attributes aid them in developing a habit of evaluating GenAI outputs critically. For instance, P5, a GenAI enthusiast and designer with six years of experience, was initially interested in a chair design for a pet-friendly office chair that included “a cushioned area beneath or beside the seat where a small to medium-sized pet can relax.” This proposal initially inspired P5, who had been contemplating designs that accommodate pets beside the chair. However, she continued to ponder the practicality of this design and whether there could be a better solution. Consequently, she input the prompt \textit{“with smooth curves and a built-in scratching post along one side”}, which led to a more optimized solution for human-pet interaction (for a detailed example, see Appendix D in our supplementary materials).

During her response to interview question 3, P5 remarked: \textit{“GenAI can inspire creativity, but the design must serve a purpose and be practical. I assess and further iterate on the designs from a practicality standpoint.”} She also noted, “Novice designers using GenAI might forget the purpose of the design and end up selecting GenAI-generated outcomes, especially for their aesthetic effects. However, many elements in these designs have flaws,” highlighting a gap that less experienced participants often encounter.
