% 1. This is a problem that matters in real life
%  why does it matter to HCI
% 2. We believe that we find something that improves how we think about and act on that problem
% 3.  improves how we should think/act differently as a result
% 4. If you contribute a theory: New concept for understanding, studying, designing something that generates novel inferences compared to existing theory [to change how we think or act]
% remain honest about limitations and scope. precise and humble claims

\section{Introduction}

In the evolving landscape of human-computer interaction (HCI), Generative AI (GenAI)-enhanced tools have emerged as pivotal in advancing design creativity support research \cite{oh2024lumimood, choi2024creativeconnect, lin2024jigsaw}. Central to the appeal of GenAI in design is its adeptness at synthesizing contextual solutions. This strength stems from the integration of large datasets, algorithmic architectures, and users' prompt \cite{schellaert2023your}. However, these same properties also impose constraints on guiding GenAI systems to be creative, with documented limitations in generating truly novel and diverse design outcomes, particularly due to a tendency towards homogenized design perspectives \cite{doshi2024generative, kobiella2024if}.

Recent studies have begun to raise the question about the extent to which these models can produce genuinely creative outputs, since the data used to train models are highly-centralized, data-driven. Some recent user studies of Large Language Models (LLMs) provide evidence of homogenization effect in LLM-supported writing \cite{padmakumar2023does, doshi2023generative}. One such study found that the application of GenAI reduced the collective diversity of novel content \cite{doshi2024generative}. However, the authors of these articles did not deeply analyze the potential causes of this phenomenon, instead, they mainly talked about the homogenization effects on human creative ideation \cite{anderson2024homogenization, doshi2024generative}. Indeed, recent work has called for foundational research to understand what constrains GenAI’s output, and conform their existence outside of writing tasks specifically. Here, we argue that, similar to design fixation as an inherent cognitive pattern in the human design process, Generative AI also exhibits design fixation. Firstly, we suggest that current GenAI systems exhibit design fixation; understanding this phenomenon can help interpret some previous research and reveal challenges that limit the novelty and diversity of GenAI outputs. 

While the concept of fixation is well-recognized in the HCI community, research has primarily focused on developing tools to mitigate human design fixation, often discussing design goals or benefits of such systems \cite{yoo2024bi, lee2024conversational, chung2022artistic, lamiroy2022lamuse}. Another few research towards the examination of design fixation focusing on human side \cite{wadinambiarachchi2024effects}. However, there is a lack of systematic and rigorous investigation of GenAI design fixation. This oversight of the constraint of GenAI capabilities may obscure certain issues in creative support that are otherwise overlooked by other lens for design creativity support. \cite{li2023beyond, frich2019mapping, frich2018twenty}.

%Therefore, we think there is a necessity to offer a more rigorous and systematic definition of "GenAI design fixation"-the state in which a Generative AI model restricts its exploration of the generative space due to unconscious bias stemming from data bias, algorithmic architecture, or input prompt, which limits the diversity and originality of the model's design output, leading to repetitive or constrained results". 


%随机鹦鹉,没有创意GenAI system became end-to-end systems which could only automate tedious work and prevent breakdowns.

While highlighting the advancements, discussions about flaws or imperfections within the outputs in creativity support are also becoming necessary to enhance scientific balance \cite{weisz2024design}. Reflections and examinations about GenAI creativity have arisen, which mainly focus on creative writing area. For example, \cite{chakrabarty2024art} evaluating the creativity of LLM-generated stories compared to professional authors. \cite{doshi2024generative} examines the relationship between GenAI and human creativity, which demonstrates the narrower scope of novel content based on GenAI production. Meanwhile, GenAI still remains elusive. It derives opaque processing models from data patterns, and its outputs are inherently uncertain due to generative variability \cite{weisz2023toward, weisz2024design}. Generative AI's imperfection also draws the attention of HCI community, such as hallucination, AI error and bias. \cite{benjamin2021machine} proposed that AI "error" could serve as an unexpected material to help designers explore new spaces for design intervention. \cite{van2022ceci} found that AI "error" could bring design reflection through surprise. \cite{liu2024smart} summarize the design dimensions and implementation methods of AI error for creative design.

% However, concerning the close relation of creativity and fixation in many domains, we noted the dearth of and the need for a lens of fixation to complement a more comprehensive evaluation of GenAI in creativity support.

With the growing advancement and application of GenAI, the call for a fixation lens in HCI has become urgent and important. Firstly, the generative variability of GenAI brings the risk of unconscious bias. As previous creativity support methods are based on specific vocabulary-based command or data-driven methods such as semantic networks \cite{luo2019computer, shi2017data} or knowledge graphs are generally based on explicit construction algorithms, resulting in controllable and explainable recommendations. In contrast, generative models trained on large amount of datasets may incorporate unconscious bias present in the training data. Secondly, in terms of the human-computer collaboration process, traditional creative support tools primarily serve an auxiliary role, relying on humans to drive the solutions tailor to the current problem. However, GenAI could generate "situation-based" stimuli allows for the generation of complete creative outcomes with minimal or no direct human intervention \cite{zhu2023generative}. Recent study have found that this led to an over-reliance by human users on the results generated by AI, resulting in fixation on solutions produced by GenAI \cite{wadinambiarachchi2024effects}.
%在GenAI之前,少有谈论创意支持工具的original问题,尽管相关研究中已有对GenAI工具originality的质疑 (e.g. \cite{li2024user}中采访UX design experience

Towards this aim, this study proposes a previously overlooked phenomenon—GenAI design fixation. Firstly, we define the definition and analyze the causes of GenAI design fixation. Through experiments, we investigate whether novice designer can perceive the existence of GenAI design fixation, and we summarize the impact of GenAI design fixation on the human-GenAI co-ideation process, collecting empirical evidence of GenAI design fixation in the human-GenAI collaboration process. Subsequently, we propose measures to mitigate GenAI design fixation, providing references for future Generative AI-based creative support tools. Finally, we summarize the distinctions between GenAI design fixation and other GenAI flaws and imperfections, especially in the realm of creativity, and propose possible future directions for GenAI design fixation.

In summary, we have made the following contributions to the HCI community:
\begin{enumerate}
    \item \textbf{Theoretical framework of GenAI design fixation: }We have developed a preliminary theoretical framework for GenAI design fixation by combining theoretical and empirical approaches. This framework includes the definition of GenAI design fixation, its causes, manifestations, impact on creative design, and measures to mitigate it.
    \item \textbf{Design strategies for GenAI-based creative support tools: }We use the GenAI design fixation lens to identify the needs and provide concrete directions for further research into overcoming the design fixation challenges of GenAI.
\end{enumerate}


In the next sections, we define GenAI design fixation, comparing the differences and correlations between GenAI design fixation and human design fixation in Section~\ref{section_definition}; summarize how design fixation manifest in text generation and image generation models respectively through an empirical experiment, analyzing the impact of GenAI design fixation on the creative process in Section~\ref{section_experiment}, and propose ways to mitigate GenAI design fixation in Section~\ref{section_solution}.
	
% human fixation可能造成的风险:During the human-GenAI co-creation process, it has recieved attention for designers' fixation effect, 但是缺少针对前序过程GenAI design fixation的系统分析。for directly relevant or familiar features may limit designers' creativity, which traps them in information cocoons evaluating and selecting GenAI's output. As a result, this process reinforced existing ideas without guiding designers towards unexplored and new ideas during the ideation process \cite{chung2021intersection}.
