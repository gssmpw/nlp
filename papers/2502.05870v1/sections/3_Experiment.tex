\section{Understanding GenAI design fixation in practice}
\label{section_experiment}
\subsection{The aim of this study}
As we have proposed the definition of GenAI design fixation in Section~\ref{section_definition}, the aim of this experimental study is to gain a practical understanding of the performances of GenAI design fixation within the human-GenAI co-ideation process. Given that this is the first instance where the lens of GenAI design fixation is applied to examine the human-GenAI co-creation process, and considering the distinct research questions concerning fixation and creativity, we conduct an experimental study to gather empirical evidence on GenAI design fixation and users' perspectives on this phenomenon. Our expected results are as follows:
\begin{itemize}
    \item \textbf{RQ1: What are the manifestations of GenAI design fixation in text generation and image generation models?}
    \item \textbf{RQ2: How do participants identify and describe design fixation in Generative AI?}
    %找个文献支持这个观点
    \item \textbf{RQ3: What impacts does GenAI design fixation have on the ideation process during human-GenAI collaboration?}
\end{itemize}

%we seek to understand how designers perceive AI’s flawed output and how AI “error” affects designer-AI co-creation

% \subsection{Preconsiderations of the experiment}
% As concluded by \cite{crilly2017next}, most of the empirical studies have focused on industrial or mechanical design, we assigned an unbounded design task to participants, which is a common tradition in design fixation research \cite{crilly2017next}. 

\subsection{Participants}

Participants were recruited for this study through social media and word-of-mouth, targeting novice designers with experience in product design and an interest in the application of Generative AI in design fields. We focused on novice designers because they represent a demographic that is still forming their design habits and are likely more open to integrating new technologies such as Generative AI into their workflow. This recruitment strategy ensured that all participants have not only a basic understanding of the methods interacting with GenAI, but also a keen interest in sharing their experience and perspectives on ideation processes facilitated by GenAI. Ultimately, ten participants were successfully recruited, consisting of four males and six females aged between 20 and 26 years (Mean = 22.7, SD = 3.01). They come from diverse academic backgrounds, including undergraduate, master’s, and doctoral students. Additionally, one participant works in a GenAI practice organization. Participants received a commemorative gift as a token of appreciation for completing the study tasks.

\subsection{Procedure}
\label{experiment_procedure}
In this study, two researchers conducted one-on-one sessions with participants to observe each individual’s interaction with Generative AI, focusing on their ability to recognize the presence of GenAI design fixation. The scheduling of these experimental sessions was based on the availability of the participants. Participants were compensated for their time. The experimental process that each participant underwent is detailed in Figure \ref{fig:experiment}.

At the start of the study, researchers explained the experimental procedure, gathered informed consent, and collected demographic information from the participants (as shown in Table~\ref{tab:participants}). Then, participants were briefed on the design task which involved creating as innovative a design as possible under the assistance of two different Generative AI tools. Additionally, we introduced CombinatorX, a representative GenAI-assisted design method \cite{chen2024foundation}, which utilizes combinational creativity—merging two concepts into a novel idea \cite{boden2004creative}. The CombinatorX process involves identifying additive embodiments and forming textual combinational ideas, followed by using text-to-image technology to visualize these ideas. It is necessary to clarify that while the CombinatorX method was provided as a reference, its use was not mandatory but offered as a scaffold for participants lacking inspiration.

In our study, designers were instructed to produce as many design proposals as possible within the allotted time, ultimately submitting all their satisfactory designs. Each submission consisted of design descriptions generated by ChatGPT (GPT-4o), images produced by Midjourney, and the accompanying prompt descriptions. Pre-experiment phase allowed participants to familiarize themselves with the use of ChatGPT, Midjourney, and our provided ideation reference method, CombinatorX. 

The formal experiment involved a specific design task requiring participants to create innovative chair designs for office settings. As chair design involves both functional and aesthetic elements that can greatly benefit from the innovative possibilities offered by AI, the experiment focused on office scenarios to refine the creative direction and functional requirements of the designs. By focusing the chair design task on office scenarios, the experiment was structured to refine the creative direction and functional requirements of the designs, making our subsequent interviews and analysis more focused. The experimental process, including the procedure details and time limits, was established based on our pilot study. 

After the experiment concluded, we conducted semi-structured interviews, each lasting approximately 20 minutes. These interviews were informed by our observations of the participants’ design processes and the outputs generated by GenAI. We structured our interview around three topics to understand participants' submitted solutions under the help of GenAI (question 1-2), designers' creation processes (question 3-4), and designers' perspective on the output (question 5).

\begin{enumerate}
    \item ``Have you noticed any repetition or similarities in the ideas or designs generated by the tools?'' (\textbf{Corresponding to RQ2})
    \item ``Do you think there are stylistic or paradigmatic limitations in the ideas generated by these tools compared to those generated through traditional design methods?'' (\textbf{Corresponding to RQ3})
    \item ``If the design process required further development, how do you think the performance of these tools would affect the originality or diversity of your final creative outputs?'' (\textbf{Corresponding to RQ3})
    \item ``What do you think are the reasons for any limitations you perceived in the model outputs? Are these limitations due to the model itself or the interaction process?'' (\textbf{Corresponding to RQ2})
    \item ``Could you share your views on the design approaches demonstrated during the experiment, including both pros and cons?'' (\textbf{Corresponding to RQ3})
\end{enumerate}

These questions were designed to explore various aspects of GenAI design fixation. While it was found in participants' responses that several participants were already familiar with the concept of design fixation, we deliberately avoided using terms like “GenAI design fixation” or “design fixation” directly in our interviews. Instead, we used alternative descriptions  (i.e. interview question 1 above) to probe for any sensations of fixation that may have occurred during their interactions with generative AI tools. At the conclusion of the experiment, we explained the study’s aims and the terminology used to all participants.


\begin{figure*}[htp]
    \centering
    \includegraphics[width=0.95\textwidth]{figures/fig_process.png}
    \caption{The process of participants engaging in our experiment.}
    \vspace{-0.05in}
    \label{fig:experiment}
    \Description{}
\end{figure*}


\begin{table}[ht]
\centering
\caption{Participant demographics in our experiment.}
\label{tab:participants}
\begin{tabular}{@{}cccccc@{}}
\toprule
\textbf{ID} & \textbf{Age} & \textbf{Gender} & \textbf{Design experience} & \textbf{Use of text generation models} & \textbf{Use of image generation models} \\ \midrule
P1 & 22 & Male & less than one year & Used before & Used before\\
P2 & 22 & Female & three years & Proficient & Used before \\
P3 & 23 & Female & One year & Proficient & Used before \\
P4 & 23 & Female & Five years & Proficient & Used before \\
P5 & 24 & Female & Five years & Proficient & Proficient \\
P6 & 20 & Male & Two years & Proficient & Used before \\
P7 & 20 & Male & One year & Proficient & Used before \\
P8 & 24 & Male & Six years & Proficient & Used before \\
P9 & 23 & Female & Six years & Used before & Used before \\
P10 & 26 & Female & Five years & Proficient & Used before \\ \bottomrule
\end{tabular}
\end{table}

\subsection{Data collection and analysis}
\subsubsection{Data collection}
We collected two sets of data. The first set of data consists of the GenAI design solutions collected in Section~\ref{experiment_procedure} (totaling 96). The second set of data comes from the chair design examples from the Red Dot design contest\footnote{\url{https://www.red-dot.org/}}. The reason why this dataset were collected is for the assessment of the originality and creativity of GenAI-generated chair designs in an office setting. We choose a design contest due to its diversity and the high quality of design innovation it represents. Commercial dataset such as IKEA were avoided due to potential stylistic uniformity linked to specific brands, which could skew the diversity of the dataset. Specifically, we crawled and filtered 105 chair design entries with "office chair" containing in their design description. For these selected cases, both the images and textual descriptions of the designs were collected. This dataset provides a comprehensive foundation for comparing traditional and AI-generated designs, enhancing the analysis of GenAI's impact on design creativity and fixation in real-world scenarios.

%解释为什么用红点奖而不是组间对照
The reason why we did not conduct between-subject analysis is that it is difficult for human designers to produce photo-like chair design schema in a limited time in a laboratory experiment.  Additionally, our research aim is to assess the novelty and diversity of GenAI-generated design outcomes, making the use of recognized design award entries as a baseline an acceptable approach for comparison.

\subsubsection{Data analysis}
\paragraph{\textbf{Text generation analysis}}
In the data preprocessing phase, to enhance the relevance analysis between descriptive content and chair design, we initially reviewed all the solutions and annotated terms specifically related to chair design such as 'seat', 'leg', 'backrest', etc. (The detailed list is provided in Appendix B in our supplementary materials). We then filtered out words representing structural design. These terms serve as indicators of the degree of alignment between the design descriptions and practical chair design aspects. The results and comparison of this part of the data will be detailed in Section~\ref{sec: result}. Our keyword extraction criteria centered around functional descriptions and aesthetic elements, following methodologies akin to those used in GenAI-assisted design process analysis, as discussed in \cite{chen2024designfusion}. To maintain fairness and consistency between two sets of data, the keyword extraction process was carried out independently by two authors. The generated results from different groups were randomly shuffled before extraction to eliminate bias. After individual completion, any disputed portions were collectively discussed, and the results were determined. After keyword extraction, we consolidated homonyms to make the following analysis more scientific. An example of our keyword extraction and homonym consolidation process is detailed in Appendix C in our supplementary materials. Following the consolidation, we quantified the frequency of terms used in the ChatGPT solutions and those in the Red Dot solutions. We then identified and calculated the unique words and shared words between the two datasets, tallying both the number of entries and their cumulative frequencies.

To evaluate the novelty in the design solutions contributed by generative AI compared to human-generated solutions, we define the proportion of novelty ($P_{novelty}$) as the ratio of unique words to the total number of word entries (unique and shared). This ratio is given by the formula:
\begin{equation}
P_{novelty} = \frac{U}{U + S}
\label{eq:novelty_ratio}
\end{equation}
where $U$ denotes the number of unique word items and $S$ denotes the number of shared word items.

\paragraph{\textbf{Image generation analysis}}
In our study, we leveraged the CLIP model for an in-depth analysis of chair images generated by Midjourney and compared these with human-designed chair data. Based on insights from \cite{gandelsman2023interpreting} regarding the CLIP-ViT image encoder, we focused on specific attention heads—Layer 22 Head 1, Layer 22 Head 11, and Layer 23 Head 12—which are linked to semantic roles such as “shape,” “color,” and “texture.” We processed chair images \(I\) through the encoder, extracting embeddings for these attributes from the class token \(z_0\) outputs at the identified heads \(z_0^{l,h}\). Additionally, we captured the general embedding from the last layer output of the class token. For visualization, we used t-SNE \cite{van2008visualizing} to reduce these high \(d\)-dimensional embeddings of all images to a 2D space, which determines the coordinates for visualizing the image distribution.

Specifically, the background of both GenAI images and human images were adjusted to white using object recognition technology to minimize distractions. The visualization results are shown in Figure~\ref{fig_figure_TSNE}.

Additionally, we calculated the pairwise distance between the two datasets. To assess the statistical significance of the differences in diversity between images generated by Midjourney and those recognized in design contests, we employed the Mann-Whitney U test for pairwise distance comparisons. The mean distance is computed using the following formula:

\begin{equation}
\overline{D} = \frac{1}{n(n-1)} \sum_{i=1}^{n} \sum_{j=i+1}^{n} d(x_i, x_j)
\end{equation}

where \(d(x_i, x_j)\) represents the distance between two elements \(x_i\) and \(x_j\) in the dataset, and \(n\) is the number of elements in the dataset.



% \subsection{Results}
% \label{subsection_quantitative_result}
% \subsubsection{Text generation models analysis results}
% \begin{table}[ht]
% \centering
% \caption{????}
% \label{???}
% \begin{tabular}{lrrrr}
% \toprule
% Group &  Unique Words &  Total Words &  Diversity (TTR) &  Novelty \\
% \midrule
% Human &          1166 &     2522.000 &            0.461 &    0.713 \\
% AI &           885 &     2379.000 &            0.371 &    0.621 \\
% \bottomrule
%  \vspace{0.05in}
% \end{tabular}
% \end{table}
% \subsection{Findings}



% Based on participants’ feedback, text generation resembles an unrestricted human brainstorming session, potentially yielding ideas that might otherwise be filtered out during an individual brainstorming process.


%\subsection{The role of expertise in design fixation}
% Many studies have selected designers with different experiences for performances comparison, offering a foundation for investigating their respective responses to GenAI design fixation. Research indicates that experts, with their depth of experience, tend to be more focused and efficient in their design processes. They often gravitate towards fewer, more targeted solutions, leveraging their extensive knowledge and past successes to streamline their creative processes \cite{kavakli2003strategic}. This efficiency, however, might also lead to a certain rigidity, as experts can become prone to overlooking novel alternatives due to confirmation bias \cite{kim2014design}. Results from the study by Jannson and Smith \cite{jansson1991design} are in agreement with this. They explain that, years of educational and professional experience may contribute to fixation.

% On the other hand, novice designers are often less constrained by past successes and are more open to experimentation, but this can also mean they are more susceptible to design fixation, as they might not recognize flawed concepts as readily as professionals \cite{ball1994cognitive, ullman1988model}. Furthermore, the tendency of novices to become fixated on specific design solutions could stem from their limited exposure to a wide range of problem-solving scenarios, making them more likely to stick to familiar ideas or directly replicate existing solutions \cite{niku2009creative}.

% The research conducted by Viswanathan explores how expertise influences the handling of open-ended design problems \cite{viswanathan2012study}. It found that both novice and expert designers are fixated on the example features, but the expert designers generated more non-redundant ideas.


%每条fixation pattern对于creativity ideation的影响
%Generates only concrete images, while ideation requires creative stimulation for both form and function


% \begin{figure*}[htp]
%     \centering
%     \includegraphics[width=0.95\textwidth]{figures/fig_figure_similarity_TNSE.png}
%     \caption{\textbf{The process and dynamics of fixation within generative AI (GenAI) systems and how it correlates with human fixation in design processes.}}
%     % \vspace{-0.25in}
%     \label{fig_figure_similarity}
%     \Description{}
% \end{figure*}




