\newpage
\section*{Appendix}
\appendix
\label{sec:appendix}

\section{Prompts}

\subsection{Prompts for Dataset Refactoring}

\lstset{
    numbers=none,
    keywordstyle= \color{ blue!70},
    commentstyle= \color{red!50!green!50!blue!50},
    frame=none,
    rulesepcolor= \color{ red!20!green!20!blue!20} ,
    framexleftmargin=2em,
    columns=fullflexible,
    breaklines=true,
    breakindent=0pt,
    basicstyle=\ttfamily
}

\paragraph{ML:}

\begin{tcolorbox}[left=0mm,right=0mm,top=0mm,bottom=0mm,boxsep=1mm,arc=0mm,boxrule=0pt, frame empty, breakable]
    \small
    \begin{lstlisting}
I will provide you with a code problem with a solution. You need to generate a complete, executable code based on the raw json data, including all necessary package imports, the original code, the test cases, and the main function. 
You need to generate the executable code and expected result.
Please choose a test case according to the 'test' field from raw json data, and the code should print the answer of the test case.
The output should be json format, with code and expected_result fields.
Please only generate the number or string answer in 'expected_result' field without any extra description.
\end{lstlisting}
\end{tcolorbox}

\paragraph{CL:}

\begin{tcolorbox}[left=0mm,right=0mm,top=0mm,bottom=0mm,boxsep=1mm,arc=0mm,boxrule=0pt, frame empty, breakable]
    \small
    \begin{lstlisting}
I will provide you with the solution to a code problem in cpp, python, and javascript. You need to score according to the difficulty of the problem from 1 to 5, while 5 means the hardest. And generate topic keywords for the problem.
The output should only be json format, with difficulty and keywords fields.
difficulty: 1-5, integer
keywords: two or three words to best describe the problem, string list
\end{lstlisting}
\end{tcolorbox}

\paragraph{BG:}

\begin{tcolorbox}[left=0mm,right=0mm,top=0mm,bottom=0mm,boxsep=1mm,arc=0mm,boxrule=0pt, frame empty, breakable]
    \small
    \begin{lstlisting}
I will provide you with a piece of code and some test cases. You need to generate a complete, executable code based on these, including all necessary package imports, the original code, the test cases, and the main function. You should wrap the original code with ORIGINAL_CODE_START and ORIGINAL_CODE_END comments. Additionally, the program should output the results of the test cases. Do not include expected output in your answer.
\end{lstlisting}
\end{tcolorbox}

\section{Details of \bench}

\subsection{ML}


\begin{table}[!ht]
  \centering
  \caption{Language usage count across different categories in the ML subset.}
  \label{tab:ml_language_usage}
  \begin{tabular}{lccccccc}
      \toprule
      Java & C\# & Rust & Julia & Python & C++ & C \\
      \midrule
      25   & 20  & 20   & 26    & 18     & 21  & 20 \\
      \bottomrule
  \end{tabular}
\end{table}

In ML, the usage distribution of various programming languages is shown in Table \ref{tab:ml_language_usage}. We selected a variety of languages, including Java, C\#, Rust, Julia, Python, C++, and C, to evaluate the model's ability to handle multilingual code. This diverse selection helps to comprehensively assess the model's performance across different languages.

\subsubsection{System Prompts}


\paragraph{Zero-shot Chain-of-Thought:}

\begin{tcolorbox}[left=0mm,right=0mm,top=0mm,bottom=0mm,boxsep=1mm,arc=0mm,boxrule=0pt, frame empty, breakable]
    \small
    \begin{lstlisting}
Given the following code, what is the execution result?
You should think step by step.  Your answer should be in the following format:
Thought: <your thought>
Output:
<execution result>
\end{lstlisting}
\end{tcolorbox}




\paragraph{Zero-shot:}

\begin{tcolorbox}[left=0mm,right=0mm,top=0mm,bottom=0mm,boxsep=1mm,arc=0mm,boxrule=0pt, frame empty, breakable]
    \small
    \begin{lstlisting}
Given the following code, what is the execution result?
Your answer should be in the following format:
Output:
<execution result>
\end{lstlisting}
\end{tcolorbox}




\paragraph{Few-shot Chain-of-Thought:}

\begin{tcolorbox}[left=0mm,right=0mm,top=0mm,bottom=0mm,boxsep=1mm,arc=0mm,boxrule=0pt, frame empty, breakable]
    \small
    \begin{lstlisting}
Given the following code, what is the execution result?
You should think step by step.  Your answer should be in the following format:
Thought: <your thought>
Output:
<execution result>
Following are 3 examples: 
{{examples here}}

\end{lstlisting}
\end{tcolorbox}





\subsubsection{Demo Questions}

\begin{tcolorbox}[left=0mm,right=0mm,top=0mm,bottom=0mm,boxsep=1mm,arc=0mm,boxrule=0pt, frame empty, breakable]
    \small
    \begin{lstlisting}
def catalan_number(n: int) -> int:
    # Initialize an array to store the intermediate catalan numbers
    catalan = [0] * (n + 1)
    catalan[0] = 1  # Base case

    # Calculate catalan numbers using the recursive formula
    for i in range(1, n + 1):
        for j in range(i):
            catalan[i] += catalan[j] * catalan[i - j - 1]

    return catalan[n]

if __name__ == "__main__":
    # Run the test function and print the result of a specific test case
    print(catalan_number(3))
\end{lstlisting}
\end{tcolorbox}



\begin{tcolorbox}[left=0mm,right=0mm,top=0mm,bottom=0mm,boxsep=1mm,arc=0mm,boxrule=0pt, frame empty, breakable]
    \small
    \begin{lstlisting}
import java.util.*;

class Solution {
    public static int countPrefixWords(List<String> wordList, String prefix) {

        int count = 0;
        for (String word : wordList) {
            if (word.startsWith(prefix)) {
                count++;
            }
        }
        return count;
    }

    public static void main(String[] args) {
        System.out.println(countPrefixWords(Arrays.asList("dog", "dodge", "dot", "dough"), "do"));
    }
}
\end{lstlisting}
\end{tcolorbox}

\begin{tcolorbox}[left=0mm,right=0mm,top=0mm,bottom=0mm,boxsep=1mm,arc=0mm,boxrule=0pt, frame empty, breakable]
    \small
    \begin{lstlisting}
#include <assert.h>
#include <stdio.h>

long long minTotalCost(int n, int *C)
{
   return (long long)(C[n-2]) * (n - 1) + C[n-1];
}

int main() {
    int costs3[] = {5, 4, 3, 2};
    printf("%lld\n", minTotalCost(4, costs3));
    return 0;
}
\end{lstlisting}
\end{tcolorbox}


\begin{tcolorbox}[left=0mm,right=0mm,top=0mm,bottom=0mm,boxsep=1mm,arc=0mm,boxrule=0pt, frame empty, breakable]
    \small
    \begin{lstlisting}
function merge_sorted_arrays(nums1::Vector{Int}, m::Int, nums2::Vector{Int}, n::Int) :: Vector{Int}
    i = m 
    j = n 
    k = m + n
    
    while j > 0
        if i > 0 && nums1[i] > nums2[j]
            nums1[k] = nums1[i]
            i -= 1
        else
            nums1[k] = nums2[j]
            j -= 1
        end
        k -= 1
    end
    
    nums1
end

# Test case
result = merge_sorted_arrays([1, 3, 5, 0, 0, 0], 3, [2, 4, 6], 3)
println(result)
\end{lstlisting}
\end{tcolorbox}



\begin{tcolorbox}[left=0mm,right=0mm,top=0mm,bottom=0mm,boxsep=1mm,arc=0mm,boxrule=0pt, frame empty, breakable]
    \small
    \begin{lstlisting}
public class Solution {

  public static int findSmallestInteger(int n) {
    char[] characters = Integer.toString(n).toCharArray();
    int i = characters.length - 2;

    // Find the first digit that is smaller than the digit next to it.
    while (i >= 0 && characters[i] >= characters[i + 1]) {
      i--;
    }

    if (i == -1) {
      return -1; // Digits are in descending order, no greater number possible.
    }

    // Find the smallest digit on right side of (i) which is greater than characters[i]
    int j = characters.length - 1;
    while (characters[j] <= characters[i]) {
      j--;
    }

    // Swap the digits at indices i and j
    swap(characters, i, j);

    // Reverse the digits from index i+1 to the end of the array
    reverse(characters, i + 1);

    try {
      return Integer.parseInt(new String(characters));
    } catch (NumberFormatException e) {
      return -1; // The number formed is beyond the range of int.
    }
  }

  private static void swap(char[] arr, int i, int j) {
    char temp = arr[i];
    arr[i] = arr[j];
    arr[j] = temp;
  }

  private static void reverse(char[] arr, int start) {
    int end = arr.length - 1;
    while (start < end) {
      swap(arr, start, end);
      start++;
      end--;
    }
  }

  public static void main(String[] args) {
    System.out.println(findSmallestInteger(123));
  }
}
\end{lstlisting}
\end{tcolorbox}
\subsection{CL}

\begin{table}[!ht]
  \centering
  \caption{Language usage count across different categories in the CL subset.}
  \label{tab:cl_language_usage}
  \begin{tabular}{lccc}
      \toprule
      Python & C++ & JavaScript \\
      \midrule
       50     & 51  & 49         \\
      \bottomrule
  \end{tabular}
\end{table}

\begin{table}[!ht]
  \centering
  \caption{Details of problems in different languages and different difficulty levels.}
  \label{tab:stat:2}
      \begin{tabular}{cccc}
\toprule
\textbf{Difficulty} & JavaScript & CPP & Python \\
\midrule
1 & 10 & 11 & 11 \\
2 & 6 & 4 & 6 \\
3 & 12 & 14 & 12 \\
4 & 8 & 8 & 9 \\
5 & 13 & 14 & 12 \\
\bottomrule
\end{tabular}
\end{table}

In CL, we selected competition problems of varying difficulty, each with solutions in Python, C++, and JavaScript. You can see the distribution of language in Table \ref{tab:cl_language_usage}, and the distribution of problem difficulty in Table \ref{tab:stat:2}. This selection allows us to test the model's cross-language capabilities and its ability to handle problems of different difficulty levels.

% \input{sections/subrepository/subrepo-tables/table_cl}

\subsubsection{System Prompts}


\paragraph{Zero-shot Chain-of-Thought:}

\begin{tcolorbox}[left=0mm,right=0mm,top=0mm,bottom=0mm,boxsep=1mm,arc=0mm,boxrule=0pt, frame empty, breakable]
    \small
    \begin{lstlisting}
Given the following code, what is the execution result?
You should think step by step.  Your answer should be in the following format:
Thought: <your thought>
Output:
<execution result>
\end{lstlisting}
\end{tcolorbox}




\paragraph{Zero-shot:}

\begin{tcolorbox}[left=0mm,right=0mm,top=0mm,bottom=0mm,boxsep=1mm,arc=0mm,boxrule=0pt, frame empty, breakable]
    \small
    \begin{lstlisting}
Given the following code, what is the execution result?
Your answer should be in the following format:
Output:
<execution result>
\end{lstlisting}
\end{tcolorbox}




\paragraph{Few-shot Chain-of-Thought:}

\begin{tcolorbox}[left=0mm,right=0mm,top=0mm,bottom=0mm,boxsep=1mm,arc=0mm,boxrule=0pt, frame empty, breakable]
    \small
    \begin{lstlisting}
Given the following code, what is the execution result?
You should think step by step.  Your answer should be in the following format:
Thought: <your thought>
Output:
<execution result>
Following are 3 examples: 
{{examples here}}

\end{lstlisting}
\end{tcolorbox}





\subsubsection{Demo Questions}

\begin{tcolorbox}[left=0mm,right=0mm,top=0mm,bottom=0mm,boxsep=1mm,arc=0mm,boxrule=0pt, frame empty, breakable]
    \small
    \begin{lstlisting}
class TreeNode {
  constructor(val) {
    this.val = val;
    this.left = this.right = null;
  }
}

function maxDepth(root) {
  if (!root) return 0;
  const queue = [root, null];
  let depth = 1;

  while (queue.length > 0) {
    const node = queue.shift();
    if (node === null) {
      if (queue.length === 0) return depth;
      depth++;
      queue.push(null);
      continue;
    }
    if (node.left) queue.push(node.left);
    if (node.right) queue.push(node.right);
  }

  return depth;
}

// Test case
const root = new TreeNode(3);
root.left = new TreeNode(9);
root.right = new TreeNode(20);
root.right.left = new TreeNode(15);
root.right.right = new TreeNode(7);
console.log(maxDepth(root));
\end{lstlisting}
\end{tcolorbox}



\begin{tcolorbox}[left=0mm,right=0mm,top=0mm,bottom=0mm,boxsep=1mm,arc=0mm,boxrule=0pt, frame empty, breakable]
    \small
    \begin{lstlisting}
from collections import Counter
class Solution:
    def maxScoreWords(self, words, letters, score):
        self.ans = 0
        words_score = [sum(score[ord(c)-ord('a')] for c in word) for word in words]
        words_counter = [Counter(word) for word in words]

        def backtrack(start, cur, counter):
            if start > len(words):
                return
            self.ans = max(self.ans, cur)
            for j, w_counter in enumerate(words_counter[start:], start):
                if all(n <= counter.get(c,0) for c,n in w_counter.items()):
                    backtrack(j+1, cur+words_score[j], counter-w_counter)

        backtrack(0, 0, Counter(letters))
        return self.ans

solution = Solution()
print(solution.maxScoreWords(["dog","cat","dad","good"], ["a","a","c","d","d","d","g","o","o"], [1,0,9,5,0,0,3,0,0,0,0,0,0,0,2,0,0,0,0,0,0,0,0,0,0,0]))
\end{lstlisting}
\end{tcolorbox}

\begin{tcolorbox}[left=0mm,right=0mm,top=0mm,bottom=0mm,boxsep=1mm,arc=0mm,boxrule=0pt, frame empty, breakable]
    \small
    \begin{lstlisting}
#include <iostream>
#include <unordered_map>
#include <string>
using namespace std;

int findTheLongestSubstring(string s) {
    unordered_map<char, int> mapper = {{'a', 1}, {'e', 2}, {'i', 4}, {'o', 8}, {'u', 16}};
    unordered_map<int, int> seen;
    seen[0] = -1;
    int max_len = 0, cur = 0;

    for(int i = 0; i < s.size(); ++i){
        if(mapper.find(s[i]) != mapper.end()){
            cur ^= mapper[s[i]];
        }
        if(seen.find(cur) != seen.end()){
            max_len = max(max_len, i - seen[cur]);
        } else {
            seen[cur] = i;
        }
    }

    return max_len;
}

// Test case
class Solution {
public:
    void solve() {
        string input = "eleetminicoworoep";
        cout << findTheLongestSubstring(input) << endl; // Expected output: 13
    }
};

int main(){
    Solution sol;
    sol.solve();
    return 0;
}
\end{lstlisting}
\end{tcolorbox}


\begin{tcolorbox}[left=0mm,right=0mm,top=0mm,bottom=0mm,boxsep=1mm,arc=0mm,boxrule=0pt, frame empty, breakable]
    \small
    \begin{lstlisting}
class TreeNode {
    constructor(val) {
        this.val = val;
        this.left = this.right = null;
    }
}

function backtrack(root, sum, res, tempList) {
    if (root === null) return;
    if (root.left === null && root.right === null && sum === root.val)
        return res.push([...tempList, root.val]);

    tempList.push(root.val);
    backtrack(root.left, sum - root.val, res, tempList);
    backtrack(root.right, sum - root.val, res, tempList);
    tempList.pop();
}

function pathSum(root, sum) {
    if (root === null) return [];
    const res = [];
    backtrack(root, sum, res, []);
    return res;
}

// Test case setup
const root = new TreeNode(5);
root.left = new TreeNode(4);
root.right = new TreeNode(8);
root.left.left = new TreeNode(11);
root.right.left = new TreeNode(13);
root.right.right = new TreeNode(4);
root.left.left.left = new TreeNode(7);
root.left.left.right = new TreeNode(2);
root.right.right.left = new TreeNode(5);
root.right.right.right = new TreeNode(1);

console.log(pathSum(root, 22));
\end{lstlisting}
\end{tcolorbox}



\begin{tcolorbox}[left=0mm,right=0mm,top=0mm,bottom=0mm,boxsep=1mm,arc=0mm,boxrule=0pt, frame empty, breakable]
    \small
    \begin{lstlisting}
class TrieNode {
    constructor() {
        this.children = {};
        this.isEndOfWord = false;
    }
}

class Trie {
    constructor() {
        this.root = new TrieNode();
    }

    insert(word) {
        let node = this.root;
        for (let char of word) {
            if (!node.children[char]) {
                node.children[char] = new TrieNode();
            }
            node = node.children[char];
        }
        node.isEndOfWord = true;
    }

    search(stream) {
        let node = this.root;
        for (let char of stream) {
            if (!node.children[char]) {
                return false;
            }
            node = node.children[char];
            if (node.isEndOfWord) {
                return true;
            }
        }
        return false;
    }
}

class StreamChecker {
    constructor(words) {
        this.trie = new Trie();
        this.stream = [];

        for (let word of [...new Set(words)]) {
            this.trie.insert(word.split('').reverse().join(''));
        }
    }

    query(letter) {
        this.stream.unshift(letter);
        return this.trie.search(this.stream);
    }
}

// Test case
const streamChecker = new StreamChecker(["cd", "f", "kl"]);
console.log(streamChecker.query('a')); // false
console.log(streamChecker.query('b')); // false
console.log(streamChecker.query('c')); // false
console.log(streamChecker.query('d')); // true
console.log(streamChecker.query('e')); // false
console.log(streamChecker.query('f')); // true
console.log(streamChecker.query('g')); // false
console.log(streamChecker.query('h')); // false
console.log(streamChecker.query('i')); // false
console.log(streamChecker.query('j')); // false
console.log(streamChecker.query('k')); // false
console.log(streamChecker.query('l')); // true
\end{lstlisting}
\end{tcolorbox}
\chapter{Multi-Agent Reinforcement Learning for Goal-Agnostic Adaptive User
Interfaces}
\chaptermark{Multi-Agent Reinforcement Learning for AUIs}
\label{ch:control:multi}


\contribution{
End-to-end imitation learning offers a promising approach for training robot policies. However, generalizing to new settings—such as unseen scenes, tasks, and object instances—remains a significant challenge. Although large-scale robot demonstration datasets have shown potential for inducing generalization, they are resource-intensive to scale. In contrast, human video data is abundant and diverse, presenting an attractive alternative. Yet, these human-video datasets lack action labels, complicating their use in imitation learning. Existing methods attempt to extract grounded action representations (e.g., hand poses), but resulting policies struggle to bridge the embodiment gap between human and robot actions.
% our approach
We propose an alternative approach: leveraging language-based reasoning from human videos - essential for guiding robot actions - to train generalizable robot policies. Building on recent advances in reasoning-based policy architectures, we introduce Reasoning through Action-free Data (RAD). RAD learns from both robot demonstration data (with reasoning and action labels) and action-free human video data (with only reasoning labels). The robot data teaches the model to map reasoning to low-level actions, while the action-free data enhances reasoning capabilities. Additionally, we will release a new dataset of 3,377 human-hand demonstrations compatible with the Bridge V2 benchmark. This dataset includes chain-of-thought reasoning annotations and hand-tracking data to help facilitate future work on reasoning-driven robot learning.
% experiments
Our experiments demonstrate that RAD enables effective transfer across the embodiment gap, allowing robots to perform tasks seen only in action-free data. Furthermore, scaling up action-free reasoning data significantly improves policy performance and generalization to novel tasks. These results highlight the promise of reasoning-driven learning from action-free datasets for advancing generalizable robot control. 
% releasing dataset
Website: \href{https://rad-generalization.github.io}{here}.

}

\begin{figure}
    \centering
    \includegraphics[width=\textwidth]{\dir/figures/teaser.pdf}
    \caption{ 
     We formulate online user interface adaptation as a multi-agent reinforcement learning problem. Our approach comprises a user- and interface agent. The \useragent interacts with an application in order to reach a goal and the \interfaceagent learns to assist it. In the depicted example the \useragent interacts with a Virtual Reality toolbar, while the \interfaceagent assigns relevant items for the \useragent. The \interfaceagent does not know the goal of the \useragent. Crucially, our approach does not rely on labeled offline data or application-specific handcrafted heuristics.}
    \label{fig:teaser_rl}
\end{figure}

\begin{figure}[ht]
    \centering
    \includegraphics[width=0.8\linewidth]{graphs/greater_than_naive.pdf}
    \vspace{0.5cm}
    \includegraphics[width=0.8\linewidth]{graphs/p1_bottom.png}
    \vspace{-5pt}
    \caption{\textcolor{positional}{Positional} vs.\ \textcolor{nonpositional}{non-positional} circuits. In a \textcolor{nonpositional}{non-positional} circuit, the same edges must be included at all positions. A \textcolor{positional}{positional} circuit can distinguish between the same edge at different positions. This specificity yields better trade-offs between circuit size and faithfulness. It can also increase both precision and recall.}
    \label{fig:p1}
    \vspace{-5pt}
\end{figure}

\section{Introduction}

\looseness=-1
A primary goal of interpretability research is to characterize the internal mechanisms in language models (LMs) and other NLP models. 
A core approach in this area is \textbf{circuit discovery}---identifying the minimal subgraph within the model's computation graph that performs a specific task \citep{olah2021framework,olah-mech}.
Typically, the nodes of a circuit represent model components (e.g., attention heads, neurons, or layers).
While manual circuit discovery methods can yield position-specific insights \citep{wanginterpretability,goldowskydill2023localizingmodelbehaviorpath}, \emph{automatic methods often overlook positional information}, treating components as uniformly relevant across all input token positions \citep{conmytowards,syed2023attribution}. 
For instance, if an attention head is included in a circuit, it is assumed to contribute equally to the computation for every position in the input sequence.
The assumption that circuits are position-invariant ignores the fact that different positions often require distinct computations.
By ignoring positions, current methods limit their ability to capture mechanisms that operate across positions, such as interactions between attention heads across positions.

In this study, we start by demonstrating that positional agnosticism is a significant limitation (\S\ref{sec:motivating}). Then, to address these limitations, we introduce a new approach: position-aware edge attribution patching (PEAP; \S\ref{sec:full_circ_discovery}; Figure~\ref{fig:p1}). Current approaches  assume that if an edge is in a circuit, then the same edge will be in the circuit at all positions, thus leading to low precision. It is also assumed that an edge's importance should be aggregated across positions before deciding whether it should be included in the circuit; this can lead to cancellation effects, and thus low recall. PEAP instead allows us to compute the importance of cross-positional edges, and separately evaluates edge importance at each position. We show that this leads to smaller and more accurate circuits; see Figure~\ref{fig:p1}.

Incorporating positional information into circuit discovery is straightforward when inputs have the same length and structure across examples.

However, realistic datasets are not nearly this templatic.
How, then, can we incorporate positional information into automatic circuit discovery?
To address this challenge, we propose \textbf{schemas} (\S\ref{sec:schema}). 
Schemas assign semantic labels to spans of tokens, enabling information aggregation across examples even when the spans differ in length.

For example, in the input ``The \textcolor{positional}{war} lasted from 1453 to 14\underline{\hspace{1em}},'' the span ``\textcolor{positional}{war}'' could be labeled as ``\emph{Subject}''.
This enables handling spans with varying lengths: the phrase ``\textcolor{positional}{Black Plague}'' in another example can be treated as a single positional span with the same role as ``\textcolor{positional}{war}''.
In experiments with two LMs and three tasks, we find that circuits discovered using schemas achieve a better trade-off between circuit size and faithfulness to the model's behavior than position-agnostic circuits.
Importantly, position-aware circuits offer a more precise representation of the underlying mechanisms, providing a more concise foundation for mechanistic explanations.

We also present a fully automated pipeline for schema generation and application (\S\ref{sec:schema-generation}) using large language models (LLMs). 
We evaluate the quality of the generated schemas and their utility in discovering position-aware circuits (\S\ref{sec:schema-eval}).
Notably, circuits derived using automatically generated and applied schemas achieve comparable faithfulness scores to circuits discovered with human-designed and manually applied schemas.

We summarize our contributions as follows:
\begin{itemize}[noitemsep,leftmargin=*,topsep=1pt,parsep=1pt]
    \item Introduce a position-aware circuit discovery method, which obtains better faithfulness than position-agnostic discovery.  
    \item Introduce dataset schemas,  facilitating positional circuit discovery in more naturalistic settings. 
    \item Develop an automated schema generation and application pipeline with LLMs, yielding schemas that are comparable to manually-annotated ones.
\end{itemize}

\section{Framework Overview}
% Point-and-click AUIs dynamically assign relevant interactive elements to an interface, thereby decreasing number of decisions and actions users need to take. This, in turn, increases the overall usability \cite{gebhardt2019learning, lindlbauer2019context}. 

\add{We define ''adaptation'' as the online alteration of an interface given current and past user input to support users in completing their task more efficiently. As such AUIs dynamically assign relevant interactive elements to an interface, thereby decreasing number of decisions and actions users need to take.}
MARLUI is a framework to create AUIs \del{that dynamically assign the most relevant interactive elements to a point-and-click interface.
MARLUI} and can learn adaptive policies, \add{that is neural networks that given an input predict suitable adaptations}. \add{MARLUI learns} without needing data and only necessitates minimal adjustments to learn these policies for different AUIs. \add{We specifically, assume the users to be Computational Rational \cite{oulasvirta2022computational} to make the problem tractable.}
In this section, we provide a high-level overview of the workings of our proposed framework.
As shown in \Fig{overview}, MARLUI consists of three main components: a user interface as a shared environment \add{(which gets adapted)},  a \useragent \add{(which learns to interact with the AUI)}, and an \interfaceagent that learns an adaptation policy. 
Referencing the game character customization example illustrated in \Fig{teaser_rl}, we first describe the function of each component individually, followed by an explanation of how MARLUI can support real users.


\paragraph*{User Interface / Shared Environment}
Our approach models UI adaptation as a MARL problem, where an \interfaceagent learns from a simulated \useragent's interactions with a user interface, i.e., the shared environment. In this shared environment, user- and \interfaceagent take actions in a turn-based manner. The \useragent performs a point-and-click maneuver, which changes the state of the UI, e.g., altering the clothing of the game character. The \interfaceagent observes the action and the change to the interface and based on that selects a new subset of items to display, e.g., adjusting the clothing items displayed in the toolbar.
In turn, the \useragent observes this adaptation and takes a new action. This cycle continues till the task is complete, e.g., the desired clothing configuration is reached. 

\paragraph*{User Agent}
The \useragent aims to achieve its goal as fast as possible. In the context of game character creation, goals might involve selecting specific attributes for a character, such as the color of a shirt or backpack. The \useragent can observe the visible parts of the interface, and knows its internal state. Based on these information, the \useragent acts on the interface with point-and-click maneuvers, aiming to align the current state of the UI with its goal, e.g., aligning the current- with the desired clothing configuration. 
By constraining its motor behavior in a physiologically plausible manner, the \useragent is bound to exhibit human-like behavior for pointing and selecting items.
Through trial and error, the \useragent will eventually learn a policy that allows it to realize this alignment.
% The interactions continue until the task is completed, e.g., the desired clothing configuration is reached.

\paragraph*{Interface Agent}
The \interfaceagent adapts the UI in a turn-based manner to minimize the number of actions the \useragent must perform to complete a task. Despite not knowing the \useragent's specific goal, it learns the task structure \add{(i.e., sequence of states)} through observing the \useragent's interactions with the UI. 
It can then learn to select the subset of items that are most relevant to the \useragent at its current state, e.g., dynamically populating the toolbar with the cloting items that are most likely to be picked.
Through trial and error, it can learn UI adaptation policies without relying on pre-collected user data or predefined heuristics.

% \paragraph*{Joint Training}
% We train the \useragent and \interfaceagent jointly in the shared environment. This process allows the \interfaceagent to learn the underlying task structure and make relevant adjustments to the interface. Through observation and trial and error, the \interfaceagent learns to support the \useragent's efforts to achieve their goals more efficiently.

% \paragraph*{Real World Application}
% By simulating a realistic \useragent, the \interfaceagent can provide meaningful adaptations in real-world tasks. This is demonstrated through the MARLUI learning procedure, where the \useragent's goal is defined but unknown to the \interfaceagent. For instance, in game character customization, the \useragent selects attributes like backpacks, glasses, and garments using a toolbar dynamically populated by the \interfaceagent. The goal state is randomly determined, and the \useragent learns to achieve this goal through interaction by simulating human-like behavior. By assigning items to menu slots and observing the \useragent's performance, measured by the time taken for the task to be completed, the \interfaceagent learns the task structure from observing the \useragent's interactions. This collaborative exploration between the \useragent and \interfaceagent allows the latter to guide the \useragent's actions, paving the way for applications with real users.

\paragraph*{Interaction with Real Users}
By mimicking human-like point-and-click behavior through the \useragent, the \interfaceagent can learn to adapt UIs such that it also assist real users in accomplishing the same task.
To apply the learned adaptive interface to real users, the setting is changed such that \interfaceagent interacts with the actual users instead of the \useragent. 
In a turn-based fashion, it selects the most relevant next items after each click or selection of the actual user according to what it has learned through interactions with the \useragent.

% \begin{delfigure}[t]
%     \centering
%     \includegraphics[width=0.9\textwidth]{uist_2023/figures/method.pdf}
%     \caption{\textcolor{red}{\sout{Our \interfaceagent and \useragent act in the same environment. The \useragent is modeled as a two-level hierarchy with a high-level decision-making policy $\policy_d$ and a low-level motor control policy $\policy_m$. The \useragent interacts with the UI. The high-level policy of the \useragent observes that state of the environment (\Eq{sd}) and chooses a specific menu slot as target accordingly (\Eq{ad}). The lower level receives this action and computes a movement (\Sec{ll}). The \interfaceagent policy $\policy_I$ adapts the interface to assist the \useragent in achieving its task more efficiently. It observes user actions in the UI (\Eq{si}) and decides on adaptations. Note that the \interfaceagent cannot access the goal, making the problem partially observable.}}}
    
%     \Description{
%         Flow Chart with three columns labeled Hierarchical User Agent, shared Environment, and \interfaceagent.
%         The shared environment contains a GUI with different selection possibilities for different clothing. The hierarchical user agent is subdivided into 1) decision-making policy. This observes the GUI and outputs an action. 2) A motor control policy, this takes as input the action and interacts subsequently with the GUI. The \interfaceagent has a single component. This component observes the user interaction and adapts the GUI.  
%     }
%     \label{fig:overview}
% \end{delfigure}

\begin{figure*}[!t]
    \centering
    \includegraphics[width=0.9\textwidth]{chapters/05_shared_control/rl/figures/method_cr.pdf}
    \caption{Our \interfaceagent and \useragent act in the same environment. The \useragent is modeled as a two-level hierarchy with a high-level decision-making policy $\policy_d$ and a low-level motor control policy $\policy_m$. The \useragent interacts with the UI. The high-level policy of the \useragent observes that state of the environment (\Eq{sd}) and chooses a specific menu slot as target accordingly (\Eq{ad}). The lower level receives this action and computes a movement (\Sec{ll}). The \interfaceagent policy $\policy_I$ adapts the interface to assist the \useragent in achieving its task more efficiently. It observes user actions in the UI (\Eq{si}) and decides on adaptations. Note that the \interfaceagent cannot access the goal of the user, making the problem partially observable.}
    \label{fig:overview}
\end{figure*}

%!TEX root = ../main.tex

\section{Method}
We first present an outline the model of our \useragent, consisting of a high- and low-level policy. Then we present the \interfaceagent (\Fig{overview}).

\subsection{General Task Description}
We model tasks to be completed if the \useragent achieves their desired goal. For game character creation, a goal can be the desired configuration of a character with a certain shirt (red, green, blue), and backpack (pink, red, blue). We represent the goal as a one-hot vector encoding $\gattr$ of these attributes. A one-hot vector can be denoted as  $\mathbb{Z}_{2}^{j}$, where $j$ is the number of items. For the previous example, $\gattr$ would be in $\mathbb{Z}_2^6$ as it possesses six distinct items. 

Furthermore, the \useragent can access an input observation denoted by $\tools$, e.g., this can correspond to the current character configuration. The current input observation, $\tools$, and the goal state $\gattr$ are identical in dimension and type. 

The \useragent interacts with the interface and attempts to match the input observation and goal state as fast as possible, such that $\tools = \gattr$. Each interaction updates $\tools$ accordingly, and a trial terminates once they are identical. In the character creation example, this would be the case if the shirt and backpack of the edited character are the same as the desired configuration. The \interfaceagent makes online adaptations to the interface. It does \emph{not} know the specific goal of a user. Instead, it needs to observe user interactions with the interface to learn the underlying task structure that will yield the optimal adaptations, e.g., the user likely wants to configure the shoes after configuring the backpack.

\add{The \interfaceagent learns to adapt the UI to the \useragent by maximizing the same expected discounted reward. Specifically, the \interfaceagent learns to infer optimal next adaptations over an infinite horizon by updating implicit probabilities of likely next actions of the \useragent, given its current sequence of past actions. As such it does not learn an explicit or implicit goal probability distribution, but a distribution of the most likely next actions of the \useragent. An example is to suggest a white and blue shoe to the \useragent in the tool, as the \useragent has not interacted with the category of shoes thus far (\Fig{fig:teaser_rl}). }

\subsection{User Agent}
\label{sec:user_agent}
% The \useragent interacts with an environment to achieve a certain goal (e.g., select the intended attributes of a character). The agent tries to accomplish this as fast and accurately as possible. Thus, the \useragent first has to compare the goal state and input observation and then plan movements to reach the target. We model the user as a hierarchical agent with separate policies. 


The \useragent interacts with an environment to achieve a certain goal (e.g., select the intended attributes of a character). The agent tries to accomplish this as fast and accurately as possible, hence it minimizes task completion time. Thus, the \useragent first has to compare the goal state and input observation and then plan movements to reach the target. We model the \useragent as a hierarchical agent with separate policies for a two-level hierarchy \cite{Langerak:2021:Generalizing}. First, we introduce a high-level decision-making policy $\policy_d$ that computes a target for the agent (high-level decision-making), we approximate visual cost with the help of existing literature \cite{10.1145/1240624.1240723}. Second, a WHo Model Fitts'-Law-based low-level motor policy $\policy_m$ that decides on a strategy to reach this target. We now explain both policies in more detail.

\subsubsection{High-level Decision-Making Policy}
The high-level decision-making policy of the hierarchy is responsible to select the next target item in the interface. The overall goal of the policy is to complete a given task while being as fast as possible. Its actions are based on the current observation of the interface, the goal state, and the agent's current state. More specifically, the high-level state space $\StatePerPolicy_d$ is defined as:

\begin{equation}
    \StatePerPolicy_d = \left (\pos, \menu, \tools, \gattr \right ),
    \label{eq:sd}
\end{equation}
which comprises: i) the current position of the \useragent's end-effector normalized by the size of the UI, $\pos \in I^n$ (where $n$ denotes the dimensions, e.g., 2D vs 3D), ii) an encoding of the assignment of each item $\menu \in \mathbb{Z}_2^{\nitems \times \nslots}$, with $\nitems$ and $\nslots$ being the number of menu items and environment locations, respectively, iii) the current input state $\tools \in \mathbb{Z}_2^{\nitems}$, and iv) the goal state $\gattr \in \mathbb{Z}_2^{\nitems}$. Here, $I$ denotes the unit interval $[0,1]$, and 
$\mathbb{Z}_{2}^{n}$ is the previously described set of integers.
The item-location encoding $m$ represents the current state of a UI. It can be used, for instance, to model item-to-slot assignments. The action space $\ActionPerPolicy_D$ is defined as:
\begin{equation}
    \ActionPerPolicy_d = \target,
    \label{eq:ad}
\end{equation}
which indicates the next target slot $\target \in \mathbb{N}_{\nslots}$. The reward for the high-level decision-making policy consists of two weighted terms to trade-off between task completion accuracy and task completion time: i) how different the current input observation $\tools$ is from the goal state $\gattr$, and ii) the time it takes to execute an action. Therefore, the high-level policy needs to learn how items correlate with the task goal as well as how to interact with any given interface. With this, we define the reward as follows: 

\begin{equation}
    \RewardPerPolicy_d =  \alpha \underbrace{\error_{gd}}_{i)} - (1-\alpha)\underbrace{\left(\dect + \mt\right)}_{ii)} + \mathbbm{1}_{\text{success}},
    \label{eq:rd}
\end{equation}
where $\error_{gd}$ is the difference between the input observation and the goal state, $\alpha$ a weight term, $\mt$  the movement time as an output of the low-level policy, $\dect$ the decision time, and $\mathbbm{1}_{\text{success}}$ an indicator function that is 1 if the task has been successfully completed and 0 otherwise. 

In addition to movement time, we also need to determine the decision time $\dect$. To this end, we are \addiui{inspired by} the SDP model \cite{10.1145/1240624.1240723}. This model interpolates between an \addiui{approximated} linear visual search-time component \del{($T_s$)} and the Hick-Hyman decision time \cite{hick1952rate} \del{($T_{hh}$)}, both functions take into account the number of menu items and user parameters. 

We define the difference $\error_{gd}$ between the input observation $\tools$ and the goal state $\gattr$ as the number of mismatched attributes:
\begin{equation}
    \error_{gd} = - \sum_{x \in \gattr, y \in \tools }\frac{\mathbbm{1}_{x \neq y}}{n_{attr}},
\end{equation}
where $\mathbbm{1}$ is an indicator function that is $1$ if $x\neq y$ and else $0$,  $x$ and $y$ are individual entries in the vectors $\gattr$ and $\tools$ respectively, and $n_{attr}$ is the number of attributes (e.g., shirt, backpack, and glasses).  


\subsubsection{Low-Level Motor Control Policy}
\label{sec:ll}
The low-level motor control policy is a non-learned controller for the end-effector movement. In particular, given a target, it selects the parameters of an endpoint distribution (mean $\mu_\pos$ and standard deviation $\sigma_{\pos}$) . We set $\mu_\pos$ to the center of the target. The target $\target$ is the action of the higher-level decision-making policy ($\ActionPerPolicy_D$). 
Following empirical results \cite{fitts1954information}, we set $\sigma_{\pos}$ to 1/6th of a menu slot width to reach a hitrate of 96\%.

Given the current position and the endpoint parameters (mean and standard deviation), we compute the predicted movement time using the Fitts' Law derived WHo Model \cite{guiard2015mathematical}.
\begin{equation}
        \mt = \left ( \frac{k}{(\sigma_{\pos}/d_{\pos}-y_0)^{1-\beta}} \right ) ^{1/\beta} + \mt^{(0)},
\end{equation}
where $k$ and $\beta$ are parameters that describe a group of users, $\mt^{(0)}$ is the minimal movement time, and $y_0$ is equal to the minimum standard deviation. The term $d_{\pos}$ indicates the traveled distance from the current position to the new target position $\mu_\pos$. We follow literature for the values of other parameters \cite{guiard2015mathematical, jokinen2021touchscreen}. We sample a new position from a normal distribution: $\pos \sim \mathcal{N}\left(\mu_{\pos}, \sigma_{\pos}\right)$.


\subsection{Interface Agent}
The \interfaceagent makes discrete changes to the UI to maximize the performance of the \useragent. 
% For instance, it assigns items to a toolbar to simplify their selection for the \useragent.
In our running example of character customization, it assigns items to a toolbar to simplify their selection for the \useragent.
Unlike the \useragent, we model the \interfaceagent as a flat RL policy. The state space $\StatePerPolicy_I$ of the interface agent is defined as:

\begin{equation}
    \StatePerPolicy_I = \left (\pos, \tools, \menu, \stack \right ),
    \label{eq:si}
\end{equation}

which includes: i) the position of the user $\pos \in I^2$, ii) the input observation $\tools \in \mathbb{Z}_2^{\nitems}$, iii) the current state of the UI $\menu \in \mathbb{Z}_2^{\nitems \times \nslots}$, and iv) a vector including the history of interface elements the \useragent interacted with (commonly referred to as stacking). The action space $\ActionPerPolicy_I \in \mathbb{Z}$ and its dimensionality is application-specific. 
The goal of the \interfaceagent is to support the \useragent. Therefore, the reward of the \interfaceagent is directly coupled to the performance of the \useragent. We define the reward of the \interfaceagent to be the reward of the \useragent's high-level policy:

\begin{equation}
    \RewardPerPolicy_{I} = \RewardPerPolicy_{D}.
    \label{eqn:ri}
\end{equation}

Note that the \interfaceagent does \emph{not} have access to the \useragent's goal $\gattr$ or target $\target$. To accomplish its task, the \interfaceagent needs to learn to help the \useragent based on an implicit understanding of i) the objective of the \useragent, and ii) the underlying task structure. Our setting allows the \interfaceagent to gain this understanding solely by observing the changes in the interface as the result of the \useragent's actions. This makes the problem more challenging but also more realistic. 

The \interfaceagent learns an implicit distribution of possible goals, and by observing the \useragent it narrows down the distribution over goals. At every time step the \interfaceagent suggests the items that are most likely needed, given the goal distribution. 
\section{Implementation}
We train the user and \interfaceagent's policies simultaneously in a shared environment (the AUI). All policies receive an independent reward, and the actions of the policies influence a shared environment. We execute actions in the following order: (1) the \interfaceagent's action, (2) the \useragent's high-level action, followed by (3) the \useragent's low-level motor action. The reward for the two learned policies is computed after the low-level motor action has been executed. The episode is terminated when the \useragent has either completed the task or exceeded a time limit.

We implement our framework in Python 3.8 using RLLIB \cite{liang2018rllib} and Gym \cite{brockman2016openai}. We use PPO \add{\cite{schulman2017proximal, yu2022surprising}} to train our policies. We use 3 cores on an Intel(R) Xeon(R) CPU @ 2.60GHz during training \add{and an NVIDIA TITAN Xp GPU}. Training takes $\sim$36 hours. \del{We utilize an NVIDIA TITAN Xp GPU for training.} The \useragent's high-level decision-making policy $\policy_d$ is a 3-layer MLP with 512 neurons per layer and ReLU activation functions. The  \interfaceagent's policy  $\policy_I$ is a two-layer network with 256 neurons per layer and ReLU activation functions. \addiui{We sample the full state initialization (including goal) from a uniform distribution. We use stochastic sampling for our exploration-exploitation trade-off.} 

We use curriculum learning to increase the task difficulty and improve learnability. Specifically, we adjust the difficulty level every time a criteria has been met by increasing the mean number of initial attribute differences. More initial attribute differences result in longer action sequences and are therefore more complex to learn. We increase the mean by 0.01 every time the successful completion rate is above 90\% and the last level up was at least 10 epochs away.

We randomly sample the number of attribute differences from a normal distribution with standard deviation $1$, normalize the sampled number into the range $[1, n_a]$ and round it to the nearest integer, where $n_a$ is the number of attributes of a setting (in the case of game character $n_a=5$).

\del{The difference between agents of different applications is their respective state- and action spaces.}
\section{Hybrid approach and Simulation Results}
The objective of this study is to simultaneously relocate a set of TPODS modules from their current positions on a tumbling body to positions more conducive to the detumbling operation. Since the position of the RSO is dynamic, each TPODS has to accurately predict the future position of the RSO and plan safe trajectories to avoid the RSO as well as other TPODS modules. The effectiveness of the proposed approach will be a key enabler for such highly dynamic and complex autonomous operations. 

\subsection{TPODS-RSO Collision avoidance with DCOL}
The TPODS is commanded to follow a respective reference trajectory, generated using analysis presented in Figure~\ref{fig:ref_traj_a}. The differential collision detection and avoidance routine for convex polytopes summarized in Figure~\ref{fig:MPC_flow} is implemented for the ellipsoidal body and results are presented in Figure~\ref{fig:DCOL_CA} for an example reference trajectory. The TPODS module is commanded to maintain an inflation factor of $1.10$ through the motion. From Figure~\ref{fig:inflation_all}, we observe a few instances of the inflation factor being slightly lower than the target (red dotted line), as the limit is a soft target. However, the inflation factor still stays well above the actual collision event, identified as an inflation factor of $1$. The deviation of the actual trajectory from the reference ensures that the TPODS maintains a safe separation from the RSO. The collision avoidance can be switched off at the final stage of the motion to allow for relocation on the RSO. 

\begin{figure}[!t]
\centerline{\includegraphics[width=1\textwidth]{Figures/DCOL_CA.eps}}
\caption{TPODS-RSO Collision avoidance with DCOL}
\label{fig:DCOL_CA}
\end{figure}

\begin{figure}[b!]
    \centerline{\includegraphics[width=1\textwidth]{Figures/inflation_factor5.png}}
     \centering
    \caption{Inflation factor for collision avoidance using DCOL and CBF}
    \label{fig:inflation_all}
\end{figure}

\subsection{TPODS-RSO Collision avoidance with CBF}
While the CBF approach was shown to be effective in preventing head-on TPODS-TPODS collisions, the CBF approach was found to be suboptimal for TPODS-RSO collision avoidance. For a large set of the initial conditions, the myopic nature of CBFs caused it to generate control signals which were over-reactive, resulting in larger deviations from the reference trajectory when compared with the DCOL approach. \Cref{fig:inflation_all} plots the inflation factor using the CBF to assure safety. In the case examined, the CBF approach sees a much larger inflation factor than DCOL, corresponding to that overreaction.

\subsection{Hybrid Approach for TPODS-TPODS and TPODS-RSO Collision Avoidance}
As discussed in previous sections, the collision avoidance approach based on 
differential polytopes performs well while avoiding stationary obstacles but fails to avoid head-on collisions. In contrast, the CBF-based collision avoidance approach successfully navigates around head-on collisions but results in drastic corrections when approaching a stationary target. Hence, none of the collision avoidance approaches are sufficient to enable safe relocation of TPODS when applied in isolation. Consequently a hybrid approach, shown in Figure~\ref{fig:arch_flow}, that switches between CBF-based collision avoidance and DCOL is proposed and validated in this paper. First, the \eqref{eq:hocbf-qp} solves the optimization problem for a safe control signal which satisfies the HOCBFs derived from \eqref{eq:koz} and \eqref{eq:h_tpods_tpods} for each agent. If the inter-TPOD collision avoidance constraint $h_{\rm ca}$ is active (meaning a TPODS-TPODS collision is imminent), then the resulting safe control is used for each TPODS agent. Otherwise, the produced control signal is discarded, and the DCOL framework is used to prevent any TPODS-RSO collisions.

\subsection{Accounting for Uncertainty}

Standard CBF and HOCBF approaches assume perfect state information is available at all times -- an assumption that cannot be made for most real world problems. For the optimal relocation application, the autonomous TPODS agents can only access a best estimate of the true states via the MEKF. As such, measures need to be taken to robustify the safety conditions against uncertainty in state information. For this application, the position uncertainties are of upmost importance, as the state constraints are defined only in terms of these variables. Therefore, we modify the constraints in \eqref{eq:koz} and \eqref{eq:h_tpods_tpods} to be adaptive based on the uncertainty information given by the MEKF's covariance matrix, similar to \cite{vanWijk_FTRTA}. Denoting the posterior covariance for agent $k$ at any instant in time with $\boldsymbol{P}^{+}_{xx,k}$, consider a position uncertainty buffer, $\eta_k$, defined by
\begin{align*}
    \eta_k \triangleq \xi \norm{\sqrt{\texttt{diag}\{ \boldsymbol{P}^{+}_{xx,k} \}}(1:3)}
\end{align*}
where $\xi \in \mathbb{R}_{>0}$ is a constant, tunable term and the $\texttt{diag}\{ \cdot \}$ operator returns a vector containing the diagonal terms of an inputted square matrix. The buffer is a scalar term which captures an uncertainty radius around the best estimate of the state. Therefore, we modified constraints by inflating the effective radius of the TPOD geometry by this additional $\eta_k$ distance. Using $\hat{\boldsymbol{x}}_k^{\mathcal{B}}$ to denote the estimated state of agent $k$, the modified keep-out-zone constraint is written as  
\begin{align} \label{eq:koz_adaptive}
    \hat{h}_{\rm koz}(\hat{\boldsymbol{x}}_k^{\mathcal{B}}) \triangleq \frac{\hat{x}^2}{(a+r_{\rm s}+\eta_k)^2} + \frac{\hat{y}^2}{(b+r_{\rm s}+\eta_k)^2} + \frac{\hat{z}^2}{(c+r_{\rm s}+\eta_k)^2} - 1 \geq 0
\end{align}
Similarly, the inter-agent collision avoidance constraint uses the inflated effective radius for each agent and thus the TPODS-TPODS constraint can be written as
\begin{align} \label{eq:h_tpods_tpods_a}
    \hat{h}_{{\rm ca},ij}(\hat{\boldsymbol{x}}_{ij}^{\mathcal{B}}) \triangleq (\hat{x}_i - \hat{x}_j)^2 + (\hat{y}_i - \hat{y}_j)^2 + (\hat{z}_i - \hat{z}_j)^2 - (2r_{\rm s} + \eta_i + \eta_j)^2 \geq 0
\end{align}
It should be noted that enforcing these constraints using estimated states rather than true states no longer retains the safety guarantees offered by CBFs. Instead, we can only claim that using the uncertainty-based approach will result in fewer collisions than if the original constraints \eqref{eq:koz} and \eqref{eq:h_tpods_tpods} were used with estimated states. A detailed animation of the proposed approach in safe relocation of two TPODS in the vicinity of a tumbling RSO can be found here : \url{https://youtu.be/DSrAHj5wXGg}.

\subsection{Monte Carlo Simulations}

To verify the effectiveness of the proposed solution in solving the relocation task safely, a Monte Carlo simulation was performed with $500$ different sets of initial conditions. \Cref{fig:MC_hvals} plots the constraint values, \eqref{eq:koz} and \eqref{eq:h_tpods_tpods}, for each agent using the true state information and the estimated states. From a visual inspection, it is clear that there are very few cases where the value of $h_{{\rm koz},1}$, $h_{{\rm koz},2}$, or $h_{{\rm ca},12}$ decreased below $0$. Indeed, in \Cref{tab:MC_collisions} we can see that there were at most $5$ violations for any particular safety constraint, and that $97.6\%$ of the trials had no safety violations. Additionally, because the $h$ functions overapproximate the TPODS geometry, minor violations (i.e., small negative values) may not indicate that a true collision has occurred. \Cref{fig:MC_traj.eps} plots all $500$ runs in the RSO position space, showing the general trend of the agents altering their trajectories to avoid collisions.

\begin{table}
    \centering
    \begin{tabular}{|c|c|c|c|c|c|c|c|}
    \hline
        Type & Value & Type & Value & Type & Value & Type & Value \\
        \hline
        $h_{{\rm ca},12}$ & -0.0173 & $h_{{\rm koz},1}$ & -0.0021 & $h_{{\rm koz},2}$ & -0.0040 & $h_{{\rm koz},1}$ & -0.0031\\
        $h_{{\rm koz},1}$ & -0.0079 & $h_{{\rm ca},12}$ & -0.0062 & $h_{{\rm koz},2}$ & -0.0087 & $h_{{\rm koz},2}$ & -0.0054\\
        $h_{{\rm koz},1}$ & -0.0031 & $h_{{\rm ca},12}$ & -0.0063 & $h_{{\rm koz},1}$ & -0.0195 & $h_{{\rm ca},12}$ & -0.0032\\
        \hline
    \end{tabular}
    % \begin{tabular}{|c|c|}
    % \hline
    % \textbf{Constraint} & \textbf{Violations} \\ \hline
    % $h_{{\rm koz},1}$   &   5                            \\ \hline
    % $h_{{\rm koz},2}$   &   3                            \\ \hline
    % $h_{{\rm ca},12}$        &   4                            \\ \hline
    % \end{tabular}
    \caption{Constraint violations for $500$ run Monte Carlo simulation}
    \label{tab:MC_collisions}
\end{table}

\begin{figure}[h!]
    \centering
     \begin{subfigure}[b]{0.325\textwidth}
        \centering
         \includegraphics[width=\textwidth]{Figures/MC_hkoz1.eps}
         \caption{}\label{fig:MC_hkoz1}
     \end{subfigure}   
     \begin{subfigure}[b]{0.325\textwidth}
        \centering
         \includegraphics[width=\textwidth]{Figures/MC_hkoz2.eps}
         \caption{}\label{fig:MC_hkoz2.eps}
     \end{subfigure}
     \begin{subfigure}[b]{0.325\textwidth}
        \centering
         \includegraphics[width=\textwidth]{Figures/MC_hinter.eps}
         \caption{}\label{fig:MC_hinter.eps}
     \end{subfigure}
    \caption{Safety criterion for $500$ run Monte Carlo simulation}
    \label{fig:MC_hvals}
\end{figure}

\begin{figure}[!t]
\centerline{\includegraphics[width=1\textwidth]{Figures/MC_traj.eps}}
\caption{Trajectories for $500$ run Monte Carlo simulation}
\label{fig:MC_traj.eps}
\end{figure}
% \begin{figure}[ht!]
%      \begin{subfigure}[b]{0.32\textwidth}
%          \includegraphics[width=\textwidth]{Figures/scf_1.png}
%          \caption{Initial Position}\label{fig:1a}
%      \end{subfigure}   
%      \hfill
%      \begin{subfigure}[b]{0.2\textwidth}
%          \vfill   
%          \includegraphics[width=\textwidth]{Figures/scf_2.png}
%          \caption{Inflated Position}\label{fig:1b}
%      \end{subfigure}
%      \hfill
%      \begin{subfigure}[b]{0.18\textwidth}
%         \vfill
%          \includegraphics[width=\textwidth]{Figures/scf_3.png}
%          \caption{Desired Structure}\label{fig:1c}
%     \end{subfigure}
%     \caption{Scaffolding generation with control barrier functions}
%     \label{fig:scaf_sBC}
% \end{figure}

% \subsection{Scaffolding Generation}
% The CBFs can be further extended to enable the creation of various scaffolding structures. The desired scaffolding configuration can be inflated to generate a set of intermediate target positions for each TPODS as seen in Figure~\ref{fig:scaf_sBC}. The safety distances can be set to conservative values for the relocation motion. Once the TPODS are in their intermediate positions, the subsequent execution of the final docking stage with reduced keep-out distances can result in safe scaffolding generation. 

\section{Applications}
To demonstrate the versatility of our framework, we introduce four additional point-and-click interfaces to demonstrate how our approach generalizes to different scenarios. Every scenario offers a distinct adaption, task, and interface. Our method requires minimal to no adaptations for the different scenarios (i.e., mostly a change in the dimensions of the observation and action spaces). We showcase both 2D interfaces as well as Mixed Reality interfaces. We show how the \interfaceagent selects from a set of pre-designed UI widgets, hand user's the correct blocks when building a tower, move out-of-reach items closer, and make hierarchical menus more efficient during photo editing. 

Please refer to our supplementary video for visual demonstrations of the tasks. Due to the diverse nature of use cases, we will report either number of clicks or task completion time as success metric. All scenarios are showcased through the interaction of a real user with a trained \interfaceagent. The numerical results are obtain in simulation.

\subsection{Number Entry}
\label{sec:number}
\begin{figure}[!t]
    \centering
    \includegraphics[width=0.6\columnwidth]{chapters/05_shared_control/rl/figures/keypad.pdf}
    \caption{Adaptive keypad: the \useragent is asked to enter a randomly initialized price by using a keypad (1). The \interfaceagent selects from three pre-designed different widgets either a normal keypad,  a digits-only keypad or a non-digits-only keypad. The \useragent selects a button of the chosen widget. The task ends when the \useragent presses enter (4).}
    \label{fig:price}
\end{figure}

We introduce a price entry task on a keypad. The \interfaceagent selects a widget from a pre-designed set. We show that our approach can support applications requiring users to issue command sequences and provide meaningful help given a user's progress in the task (see \Fig{price}). 
The task assumes a setting where the simulated user must enter a product price between $10.00$ and $99.99$.
To complete the task, the \useragent has to enter the first two digits, the decimal point, the second two digits, and then press enter. 
The \interfaceagent can select one of three different interface layouts: i) a standard keypad, ii) a keypad with only digits and iii) a widget with only the decimal point and the enter key. 

The goal difference penalty (\Eq{rd}) in this case is based on whether the current price $\tools$ matches the target price $\gattr$: 
\begin{equation}
    \error_{gd} = -\sum_t\mathbbm{1}_{\tools_t \neq \gattr_t},
\end{equation}
where $\mathbbm{1}$ is an indicator that is 1 if $\tools_t \neq \gattr_t$ and $0$ otherwise, and $t$ is the current timestep. Every time a button is hit, $t$ increases by 1. This is similar to the penalty in all other tasks. However, it considers that the order of the entries matters. On average, the \useragent needs $4.0$ seconds to complete the task in cooperation with the \interfaceagent, compared to $4.9$ seconds when using a static keypad. The number of clicks is identical, since the full task can be solved on the standard keypad.

\subsubsection*{Qualitative Policy Inspection} We observe that the \interfaceagent learns to select the UI that has the biggest buttons for an expected number entry (e.g., only digits or only non-digits). From this we can conclude that the \interfaceagent implicitly learns the concept of Fitt's law and prioritizes larger buttons where appropriate. 

\subsection{Block Building}
\label{sec:building}
The second scenario is a block-building task (\Fig{building}) where the user constructs various castle-like structures from blocks. Compared to the game character task, only the dimensionality of the observation and action space needs to be changed. It can choose between 4 blocks (wall, gate, tower, roof) and a delete button. The agent needs to move the hand to a staging place for the blocks (see \Figure{building}) and then place the block in the corresponding location. The block cannot be placed in the air, i.e., it always needs another block on the floor below. The \interfaceagent suggests a next block every time the user places a block. However, the user can put the block down, in case it is unsuitable. An action is picking or placing a block. 

This task represents a subset of tasks that do not have a Heads-Up-Display-like UI to interact with, but are situated directly in the virtual world. This is a common interactive experience of AR/VR systems. The user needs on average $1.1$ actions with our framework, compared to $2.0$ actions without the \interfaceagent. Thus 1.1 indicates that the \interfaceagent suggests the correct next block, most of the time. 

\subsubsection*{Qualitative Policy Inspection} We observe that the policy learns to always suggest a block that is usable given the current state of the tower. This indicates that the policy has an implicit understanding of the order of blocks and can distinguish between those belonging to the foundation versus the upper parts of a tower.

% \begin{figure}[t]
%     \centering
%     \includegraphics[width=0.6\columnwidth]{uist_2023/figures/building.pdf}
%     \vspace{-3mm}
%     \caption{Block Building: The user is building a castle from blocks (1). The user places the first block (2). The \interfaceagent suggests a next block to place (3). This is repeated till the castle is built (4). }
%     \Description{
%         A series of four figures is presented that describes each step for block building task. Different building blocks with different colors are presented in each figure. The target brush is green circle. The first figure describes the initialization. The second figure shows that the \useragent selects and places a block. The third figure shows that the \interfaceagent suggests a block to \useragent and \useragent grabs it. The fourth figure shows that this process is continued until the task is completed.
%     }
%     \label{fig:building}
% \end{figure}



    % Second figure
\begin{figure}[!t]

        \centering
        \includegraphics[width=0.5\columnwidth]{chapters/05_shared_control/rl/figures/building.pdf} % Ensure this height matches the first image
        \caption{Block Building: The user is building a castle from blocks (1). The user places the first block (2). The \interfaceagent suggests a next block to place (3). This is repeated till the castle is built (4). }
        \label{fig:building}
\end{figure}
    % First figure
\begin{figure}[!t]
        \centering
        \includegraphics[width=0.5\columnwidth]{chapters/05_shared_control/rl/figures/item_grab.pdf} % Adjust height as needed
        \caption{Out-of-reach object grabbing: the \useragent attempts to grab a specific object, that is initially out of reach, in a space containing multiple objects (1). The \useragent learned to move towards an object to indicate its intention to grab it (2). Based on that, the \interfaceagent learned to move the intended object within the \useragent's reach (3). The \useragent then grabs the object to finish the task (4).}
        \label{fig:objectgrab}
\end{figure}

\subsection{Out-of-reach Item Grabbing}
\label{sec:out-of-reach}

In the third usage scenario, the user needs to use their hand to grab an object that is initially out of reach. Thus, the \interfaceagent needs to move an object within reach of the user, which can then grab it. The \interfaceagent observes the location of the user's hand. The task environment includes several objects. Uniquely, in this scenario the user and the \interfaceagent are forced to collaborate to select the correct target object and complete the task (see \Figure{objectgrab}); as it is impossible for the \useragent to complete the task on its own. Compared to the game character task, only the dimensionality of the observation and action space needs to be changed. 

In this use case, we changed the lower level of our user to learn motor control with RL instead of using the Fitts-Law-based motor controller. This highlights the modularity of our approach and can be useful in scenarios where existing models, such as Fitts' Law, are not sufficient.
The low-level motor control policy controls the hand movement. In particular, given a target slot, the policy selects the parameters of an endpoint distribution. Given the current position and the endpoint parameters (mean and standard deviation), we compute the predicted movement time using the WHo Model \cite{guiard2015mathematical}. The low-level policy needs to learn i) the coordinates and dimensions of menu slots, ii) an optimal speed-accuracy trade-off given a target slot, and its current position. Refer to \Appendix{learned} for more details.

\subsubsection*{Qualitative Policy Inspection} \del{We qualitatively evaluate the learned policy.} We find that the the policy selects objects positioned in the direction of the user's arm movement rather than the closest ones. This indicates that the policy implicitly learns the correlation between directionality of movement and intent. 

% \begin{figure}[t]
%     \centering
%     \includegraphics[width=0.6\columnwidth]{uist_2023/figures/item_grab.pdf}
%     \caption{Out-of-reach object grabbing: the \useragent attempts to grab a specific object, that is initially out of reach, in a space containing multiple objects (1). The \useragent learned to move towards an object to indicate its intention to grab it (2). Based on that, the \interfaceagent learned to move the intended object within the \useragent's reach (3). The \useragent then grabs the object to finish the task (4).}
%     \Description{
%         A series of four figures is presented that describes out-of-reach object grabbing task. The first figure describes the initialization, where there is a bookshelf and there are two cupboards, one which has plates inside it and the other one is empty. The second figure shows that the \useragent moves its hand towards the cupboard filled with plates. The third figure shows how the \interfaceagent reacts and removes a plate from the cupboard and bring it forward. The fourth and the last figure shows that the \useragent grabs the object brought forward by the \interfaceagent and completes the task.
%     }
%     \label{fig:objectgrab}
% \end{figure}

\subsection{2D Hierarchical Menu}
\label{sec:photo}
In this task, a user edits a photo by changing its attributes. A photo has five distinct attributes with three states per attribute: i) filter (color, sepia, gray), ii) text (none, Lorem, Ipsum), iii) sticker (none, unicorn, cactus), iv) size (small, medium large), and v) orientation (original, flipped horizontal, and vertical). The photo's attribute states are limited to one per attribute, i.e., the photo cannot be in grayscale and color simultaneously. This leads to a total of 15 attribute states and 243 photo configurations. 

The graphical interface is a hierarchical menu, where each attribute is a top-level menu entry, and each attribute state is in the corresponding submenu. By clicking a top-level menu, the submenu expands and thus becomes visible and selectable. Only one menu can be expanded at any given time. 

The photo attribute states correspond to the current input state and the target state. The \interfaceagent selects an attribute menu to open. Its goal is to reduce the number of clicks necessary to change an attribute, e.g., from two user interactions (filter->color) to one (color). For the \emph{\useragent}, the higher level selects a target slot, and the lower level moves to the corresponding location. 

\subsubsection*{Qualitative Policy Inspection} 
We observe that the \interfaceagent intelligent\add{ly} decides which submenu to open next. We notice that this is never a menu the user recently interacted with as the probability of having to change, e.g., the color twice in a row is minimal and only a result of errors.

\begin{figure}[!t]
    \centering
    \includegraphics[width=\textwidth]{chapters/05_shared_control/rl/figures/hierarchical_good.pdf}
    \caption{We introduce a photo editing task where (1)~a user matches a photo to a target by operating a hierarchical menu. (2)~The user selects the submenu `size`. (3)~The user then selects the attribute `small`, which alters the image. (4)~After the user has changed an attribute, the interface observes the new state of the photo and finds the most likely submenu for the next user action. (5)~The user clicks on an item in the submenu to complete the task.}
    \label{fig:hierarchical_good}
\end{figure}
\section{Discussion}
\label{sec:limitations}
MARLUI models the interaction with point-and-click adaptive interfaces as a multi-agent cooperative game by teaching a simulated \useragent and an \interfaceagent to cooperate. Learned policies of the interface agent have shown their capability to effectively assist real users. Demonstrating our approach in a wide variety of use cases is a first step towards general methods that are not tied to specific applications nor dependent on manually crafted rules or offline user data collection. However, there are limitations that require further research.

In this work, we have focused on point-and-click interfaces. However, it would be interesting to extend the user agent to model other interaction paradigms. By enhancing our user agent to replicate behavior for other interaction types than clicking, we could extend the possible use cases that MARLUI can support. For example, research has shown that gaze-based selection, similar to cursor movement, follows Fitts' Law \cite{schuetz2019explanation}. \add{Furthermore, similar concepts can also be applied to human-robot interactions, such as using simulated humans to train human-robot handshakes or human-to-robot handovers \cite{christen2023learning, christen2019guided, christen2023synh2r}.} 

Moreover, user goals can change during human-computer interaction, particularly in creative tasks where users constantly adjust their objective based on intermediate results. This presents challenges for standard RL approaches, which assume goals to remain stationary. Future research on MARL for AUIs needs to focus on finding strategies to easily adapt trained interfaces to changing or new user goals. This is required to establish more robust and flexible adaptive interfaces that can support real-world use cases.

We have demonstrated that our formulation solves problems with up to 5 billion possible states, as in the character creation application (\Sec{task}). However, the complexity of the problem grows exponentially with the number of states. This makes it challenging for MARLUI to scale to interfaces with even larger state spaces. To overcome this, we could explore different input modalities, such as representing the state of the UI as an image instead of using one-hot encoding. This approach is similar to work on RL agents playing video games \cite{mnih2013playing}, which showed that image representations can effectively cope with large state spaces.

We have shown that the simulated \useragent's behavior was sufficiently human-like to enable the \interfaceagent to learn helpful policies that transfer to real users. The \interfaceagent's performance is inherently limited by the \useragent. Therefore, increasing realism in the model of the simulated user is an interesting future research direction, for instance, modeling human-like search \cite{chen2015emergence} or motor control with a biomechanical model \cite{fischer2021reinforcement}. \add{Along similar lines, our work helps with the creation of policies that adapt interfaces given user interactions. However, it does not adapt to the user themselves (e.g., different levels of expertise). Such personalization is an interesting direction for future research.}

Our framework has theoretical appeal because it provides a plausible model of the bilateral nature of AUIs: the adaptation depends on the user, whereas also the user action depends on the adaptation. Modeling this unilaterally as in supervised learning does not reflect reality well. Related to ongoing research \cite{murray2022simulation}, we believe that future work can leverage our framework to gain a better theoretical understanding of how users interact with a UI.  Our setup has the potential to scale to multiple users with different skills and intentions. This could lead to bespoke assistive UIs for users with specific needs or UIs for users with specific expertise levels. 

Finally, our proposed framework enables adaptive policies for different point-and-click tasks and interfaces. We have shown how our framework produces policies that support users in a variety of these interfaces and tasks. Building on this, future work can investigate the transition from framework to developer tool. Tools that enable developers to use our framework easily and consistently will streamline the development process of AUIs.

\section{Conclusion}
\red{In formal email communication, users are often required to read detailed (lengthy or complex) emails. 
Crafting appropriate responses to such emails is time-consuming and may lead to overlooked sender requests or delayed responses, causing communication issues.}
Thus, we propose \red{QA-based approach}, which leverages LLM-based question generation to help users create efficient and high-quality replies by generating multiple question-answer pairs related to the received email content.
\red{To examine the comprehensive impact of the QA-based approach on both email senders and recipients, we conducted controlled and field experiments using our prototype system, \textit{ResQ}.
Our findings demonstrate that structuring email content into question-answer pairs improves efficiency, reduces cognitive load, and lowers barriers to initiating responses. 
Additionally, this approach enhances email quality and may leave a better impression on recipients.
However, our findings also revealed challenges, including a potential reduction in user agency and an increased psychological distance in communication. 
These trade-offs emphasize the need for adaptive designs that balance efficiency with personalization and user control.
% The QA-based approach shows promise for applications beyond email communication in domains requiring structured response generation. 
Future research should investigate the long-term effects of such systems on user behavior, cross-cultural differences in adoption, and the effectiveness of the QA-based approach across varying email characteristics}
% Future research should explore its long-term impacts, cross-cultural applicability, and integration into diverse communication platforms to optimize both efficiency and authenticity in AI-mediated interactions.}

% original
% In workplace email communication, users are often required to read lengthy emails and craft appropriate responses, a time-consuming task that may lead them to overlook parts of the sender's request or delay their response, causing communication issues.
% Thus, we propose \textit{ResQ}, which leverages LLM-based question generation to help users create efficient and high-quality replies by generating multiple question-answer pairs related to the received email content.
% Our controlled and field experiments confirmed that compared to a prompt-based approach, ResQ significantly improved email replying efficiency, reduced cognitive load, and lowered the barriers to task initiation.
% Additionally, AI support was shown to improve the quality of emails and enhance the recipient's impression.
% However, we observed that ResQ decreased users' sense of agency and control and enlarged the psychological distance between email senders and receivers.
% We also discussed communication scenarios where QA-based approach might be effective for AI-mediated communication.
\subsection{SC}

\subsubsection{Tasks Descriptions}
\label{sec:appendix1}

The scientific computing component of \bench consists of 4 carefully curated areas, aiming to evaluate model performance on computational tasks that exhibit a time-consuming nature as well as applicational values in scientific computing areas. In this section, we provide a detailed description of each component.

\paragraph{Numerical Optimization.} In this task, the model is given a program that solves an optimization problem through gradient descent. The query may be the optimized value (\textit{min}) or the optimal point (\textit{argmin}). We carefully select four functions, which consist of: a simple quadratic function, Rosenbrock Function, Himmelblau’s Function, and a polynomial function with linear constraints. For each function, we will select multiple different hyperparameter configurations to assess the model's performance. These four functions provide a systematic evaluation of the model's potential to serve as a surrogate model in this field. As the quadratic function is solvable without need the to run the gradient descent, the model may solve it through world knowledge. The Rosenbrock function is known for its narrow, curved valley containing the global minimum, making it difficult for optimization algorithms to converge. Therefore the output is highly dependent on hyperparameters (initial point, learning rate, maximum steps), thus the model must execute code in its reasoning process to acquire the answer. Himmelblau's function has multiple local minima, also posing sensitivity to hyperparameters.

\paragraph{PDE Solving.} We consider three types of Partial Differential Equations: the 1D Heat Equation, the 2D Wave Equation, and the 2D Laplace Equation. For the 1D Heat Equation, we focus on solving the following equation:
\begin{equation}
    \frac{\partial u}{\partial t} = \alpha \frac{\partial^2 u}{\partial x^2}.
\end{equation}
For the 2D Laplace Equation, we aim to solve the equation:
\begin{equation}
    \frac{\partial^2 u}{\partial x^2} + \frac{\partial^2 u}{\partial y^2} = 0.
\end{equation}
Lastly, for the 2D Wave Equation, we work on solving the following equation:
\begin{equation}
    \frac{\partial^2 u}{\partial t^2} = c^2 \left( \frac{\partial^2 u}{\partial x^2} + \frac{\partial^2 u}{\partial y^2} \right).
\end{equation}
We solve 1D Heat Equation and 2D Wave Equation using the Explicit Finite Difference Method. For the 2D Laplace Equation, we solve it using the Gauss-Seidel Method. The model is then queried on the values of $u$ and $x$.
\paragraph{Fourier Transform (FFT)} We implement FFT using the Cooley-Tukey Algorithm and query the model to give the magnitude of the top 10 values.
\paragraph{ODE Solving} For solving ordinary differential equations, we constructed three different equations and implemented the Euler Method and the Runge-Kutta Method so solve these equations.

\subsubsection{Evaluation Metrics}
\label{app:metric}
\paragraph{Relative Absolute Error (RAE).} 
Given a scalar ground truth value \( p \) and a model prediction \( \hat{p} \), the Relative Absolute Error (RAE) is defined as:
\begin{equation}
    \text{RAE}(\hat{p}, p) = \frac{|p - \hat{p}|}{|p|}.
\end{equation}
For cases involving multiple entries, such as tensors or vectors, the following alignment procedure is applied: (1) if the prediction contains fewer elements than the ground truth, the prediction is padded with zeros until it matches the length of the ground truth; (2) if the prediction has more elements than the ground truth, it is truncated to match the ground truth length. The average RAE is then computed by averaging the RAE for each corresponding element.

\paragraph{Exact Matching.} 
For tasks involving position-based predictions, such as binary search, we adapt exact matching, as the accuracy of the algorithm is determined by comparing the exactness of the estimated result to the true result. This evaluation method checks if the estimated solution matches the ground truth exactly, typically using string or sequence matching. For such tasks, an exact match is considered a success, and any discrepancy between the ground truth and the estimate results in failure. Formally, given a string $s$ and the model's prediction $\hat{s}$, the Exact Matching is given by:
\begin{equation}
    \text{EM}(s,\hat{s})=\mathbbm{1}[s=\hat{s}]
\end{equation}
where $\mathbbm{1}[\cdot]$ is the indicator function.


\subsubsection{System Prompts}


\paragraph{Zero-shot Chain-of-Thought:}

\begin{tcolorbox}[left=0mm,right=0mm,top=0mm,bottom=0mm,boxsep=1mm,arc=0mm,boxrule=0pt, frame empty, breakable]
    \small
    \begin{lstlisting}
You are an expert in gradient_descent programming.
Please execute the above code with the input provided and return the output. You should think step by step.
Your answer should be in the following format:
Thought: <your thought>
Output: <execution result>
Please follow this format strictly and ensure the Output section contains only the required result without any additional text.
\end{lstlisting}
\end{tcolorbox}




\paragraph{Zero-shot:}

\begin{tcolorbox}[left=0mm,right=0mm,top=0mm,bottom=0mm,boxsep=1mm,arc=0mm,boxrule=0pt, frame empty, breakable]
    \small
    \begin{lstlisting}
You are an expert in gradient_descent programming.
Please execute the given code with the provided input and return the output.
Make sure to return only the output in the exact format as expected.

Output Format:
Output: <result>
\end{lstlisting}
\end{tcolorbox}




\paragraph{Few-shot Chain-of-Thought:}

\begin{tcolorbox}[left=0mm,right=0mm,top=0mm,bottom=0mm,boxsep=1mm,arc=0mm,boxrule=0pt, frame empty, breakable]
    \small
    \begin{lstlisting}
You are an expert in gradient_descent programming.
Please execute the above code with the input provided and return the output. You should think step by step.
Your answer should be in the following format:
Thought: <your thought>
Output: <execution result>
Please follow this format strictly and ensure the Output section contains only the required result without any additional text.

Here are some examples:
{{examples here}}

\end{lstlisting}
\end{tcolorbox}





\subsubsection{Demo Questions}

\begin{tcolorbox}[left=0mm,right=0mm,top=0mm,bottom=0mm,boxsep=1mm,arc=0mm,boxrule=0pt, frame empty, breakable]
    \small
    \begin{lstlisting}
code:```
import numpy as np
import argparse

def f(t, y):
    """dy/dt = -y"""
    return -y

def euler_method(f, y0, t0, t_end, h, additional_args=None):
    t_values = np.arange(t0, t_end, h)
    y_values = [y0]
    v_values = [additional_args] if additional_args is not None else [None]
    
    for t in t_values[:-1]:
        if additional_args:
            y_next, v_next = y_values[-1] + h * f(t, y_values[-1])[0], v_values[-1] + h * f(t, y_values[-1], v_values[-1])[1]
            y_values.append(y_next)
            v_values.append(v_next)
        else:
            y_next = y_values[-1] + h * f(t, y_values[-1])
            y_values.append(y_next)
    
    return t_values, np.array(y_values), np.array(v_values) if v_values[0] is not None else None

def main():
    parser = argparse.ArgumentParser()
    parser.add_argument("--y0", type=float, default=1.0)
    parser.add_argument("--t0", type=float, default=0.0)
    parser.add_argument("--t_end", type=float, default=10.0)
    parser.add_argument("--h", type=float, default=0.1)
    args = parser.parse_args()

    y0_1 = args.y0
    t0 = args.t0
    t_end = args.t_end
    h = args.h

    t_values, y_values, _ = euler_method(f, y0_1, t0, t_end, h)
    print(f"{y_values[-1]:.4f}")

if __name__ == "__main__":
    main()

```
command:```
python euler_3.py --y0 12 --t0 0.0 --t_end 74 --h 0.36
```
\end{lstlisting}
\end{tcolorbox}



\begin{tcolorbox}[left=0mm,right=0mm,top=0mm,bottom=0mm,boxsep=1mm,arc=0mm,boxrule=0pt, frame empty, breakable]
    \small
    \begin{lstlisting}
code:```
import numpy as np
import argparse

def gradient_descent(func, grad_func, initial_guess, learning_rate=0.1, tolerance=1e-6, max_iter=1000):
    x = initial_guess
    for _ in range(max_iter):
        grad = grad_func(x)
        x = x - learning_rate * grad
        if np.abs(grad) < tolerance:
            break
    return x, func(x)

# Function and its gradient
def func(x):
    return (x - 3)**2 + 5

def grad_func(x):
    return 2 * (x - 3)

def main(): 
    parser = argparse.ArgumentParser()
    parser.add_argument("--initial_guess", type=float, default=0.0)
    parser.add_argument("--learning_rate", type=float, default=0.1)
    parser.add_argument("--tolerance", type=float, default=1e-6)
    parser.add_argument("--max_iter", type=int, default=1000)
    args = parser.parse_args()

    # Test with initial guess
    initial_guess = args.initial_guess
    optimal_x, optimal_value = gradient_descent(func, grad_func, initial_guess)
    # optimal x
    print(f"{optimal_x:.3f}")

if __name__ == "__main__":
    main()
```
command:```
python gd_ox.py --initial_guess -5.0 --learning_rate 0.01 --max_iter 5000
```
\end{lstlisting}
\end{tcolorbox}

\begin{tcolorbox}[left=0mm,right=0mm,top=0mm,bottom=0mm,boxsep=1mm,arc=0mm,boxrule=0pt, frame empty, breakable]
    \small
    \begin{lstlisting}
code:```
import numpy as np
import argparse

def solve_heat_eq(L, T, alpha, Nx, Nt):
    # L: length of the rod
    # T: total time
    # alpha: thermal diffusivity
    # Nx: number of spatial steps
    # Nt: number of time steps

    dx = L / (Nx - 1)
    dt = T / Nt
    r = alpha * dt / dx**2

    # Initial condition: u(x, 0) = sin(pi * x)
    x = np.linspace(0, L, Nx)
    u = np.sin(np.pi * x)

    # Time stepping
    for n in range(Nt):
        u_new = u.copy()
        for i in range(1, Nx - 1):
            u_new[i] = u[i] + r * (u[i-1] - 2*u[i] + u[i+1])
        u = u_new
    return x, u

def parse_input():
    parser = argparse.ArgumentParser(description="Solve the 1D Heat Equation")
    parser.add_argument('--L', type=float, required=True, help="Length of the rod")
    parser.add_argument('--T', type=float, required=True, help="Total time")
    parser.add_argument('--alpha', type=float, required=True, help="Thermal diffusivity")
    parser.add_argument('--Nx', type=int, required=True, help="Number of spatial points")
    parser.add_argument('--Nt', type=int, required=True, help="Number of time steps")
    return parser.parse_args()

def main():
    args = parse_input()
    x, u = solve_heat_eq(args.L, args.T, args.alpha, args.Nx, args.Nt)
    np.set_printoptions(threshold=np.inf, linewidth=np.inf)
    formatted_x = np.vectorize(lambda x: f"{x:.4e}")(x)
    print(f"{formatted_x}")

if __name__ == "__main__":
    main()

```
command:```
python heat_eq_x.py --L 36 --T 62 --alpha 91 --Nx 170 --Nt 860
```
\end{lstlisting}
\end{tcolorbox}


\begin{tcolorbox}[left=0mm,right=0mm,top=0mm,bottom=0mm,boxsep=1mm,arc=0mm,boxrule=0pt, frame empty, breakable]
    \small
    \begin{lstlisting}
code:```
import numpy as np
import argparse

def gradient_descent(func, grad_func, initial_guess, learning_rate=0.1, tolerance=1e-6, max_iter=1000):
    x = initial_guess
    for _ in range(max_iter):
        grad = grad_func(x)
        x = x - learning_rate * grad
        if np.abs(grad) < tolerance:
            break
    return x, func(x)

# Function and its gradient
def func(x):
    return (x - 3)**2 + 5

def grad_func(x):
    return 2 * (x - 3)

def main(): 
    parser = argparse.ArgumentParser()
    parser.add_argument("--initial_guess", type=float, default=0.0)
    parser.add_argument("--learning_rate", type=float, default=0.1)
    parser.add_argument("--tolerance", type=float, default=1e-6)
    parser.add_argument("--max_iter", type=int, default=1000)
    args = parser.parse_args()

    # Test with initial guess
    initial_guess = args.initial_guess
    optimal_x, optimal_value = gradient_descent(func, grad_func, initial_guess)
    # optimal x
    print(f"{optimal_x:.3f}")

if __name__ == "__main__":
    main()
```
command:```
python gd_ox.py --initial_guess -10.0 --learning_rate 0.001 --max_iter 100
```
\end{lstlisting}
\end{tcolorbox}



\begin{tcolorbox}[left=0mm,right=0mm,top=0mm,bottom=0mm,boxsep=1mm,arc=0mm,boxrule=0pt, frame empty, breakable]
    \small
    \begin{lstlisting}
code:```
import numpy as np
import argparse

# Objective function: f(x, y) = x^2 + y^2
def objective(x, y):
    return x**2 + y**2

# Gradient of the objective function: ∇f(x, y) = (2x, 2y)
def gradient(x, y):
    return np.array([2 * x, 2 * y])

# Projection function onto the constraint x + y = 1
def projection(x, y):
    # Since the constraint is x + y = 1, we can project the point (x, y) onto the line
    # by solving the system: x' + y' = 1
    # Let x' = x - (x + y - 1)/2, and y' = y - (x + y - 1)/2
    adjustment = (x + y - 1) / 2
    return np.array([x - adjustment, y - adjustment])

def projected_gradient_descent(learning_rate=0.1, max_iter=1000, tolerance=1e-6, initial_guess=(0.0, 0.0)):
    x, y = initial_guess
    
    for _ in range(max_iter):
        # Compute the gradient of the objective function
        grad = gradient(x, y)
        
        # Update the variables by moving in the opposite direction of the gradient
        x, y = np.array([x, y]) - learning_rate * grad
        
        # Project the updated point onto the constraint set (x + y = 1)
        x, y = projection(x, y)
        
        # Check if the gradient is small enough to stop
        if np.linalg.norm(grad) < tolerance:
            break
    
    return x, y, objective(x, y)

def main(): 
    parser = argparse.ArgumentParser()
    parser.add_argument("--initial_guess_x", type=float, default=0.0)
    parser.add_argument("--initial_guess_y", type=float, default=0.0)
    parser.add_argument("--learning_rate", type=float, default=0.1)
    parser.add_argument("--tolerance", type=float, default=1e-6)
    parser.add_argument("--max_iter", type=int, default=1000)
    args = parser.parse_args()
    
    initial_guess = (args.initial_guess_x, args.initial_guess_y)
    optimal_x, optimal_y, optimal_value = projected_gradient_descent(args.learning_rate, args.max_iter, args.tolerance, initial_guess)
    print(f"{optimal_x:.4e}, {optimal_y:.4e}")

if __name__ == "__main__":
    main()

```
command:```
python gd_pgdx.py --initial_guess_x 32.14 --initial_guess_y 46.04 --learning_rate 0.01 --max_iter 1000
```
\end{lstlisting}
\end{tcolorbox}
\subsection{TC}

\subsubsection{Tasks Descriptions of Time Consuming~(TC)}
\label{sec:appendix2}

The time consuming component of \bench is comprised of 4 tasks in for computationally expensive areas, covering a spectrum of Linear Algebra, Sorting, Searching, Monte Carlo Simulations and String Matching Programs. Some of these tasks take hours to complete, showing their potential to benchmark LLM's ability to reason through lengthy computations.

\paragraph{Linear Algebra.} In this task, we are focused on acquiring key properties in linear algebra given square matrices of varying sizes. In particular, we query the model on solving LU decomposition, QR decomposition, the largest eigenvalue and eigenvector using the power method, and the inversion matrix.

\paragraph{Sorting And Searching.} We include four classical algorithmic problems in this area, namely Hamiltonian Cycle, Traveling Salesman Problem (TSP), Sorting an array of real numbers and Searching. For Hamiltonian Cycle, we adopt the backtracking algorithm. Specifically, we randomly generate graphs with vertices from 4 to 100 and ask the model to find whether a Hamiltonian cycle exists. For TSP, we implement a naive brute-force algorithm and ask the model to find the length of the optimal path. For Sorting, we adopt the bubble sort, quick sort, and merge sort algorithms. For each algorithm, we consider different list sizes from 5 to 100 and generate 10 test cases for each list size. The evaluation metric is the rank correlation (also Spearman's $\rho$ ). Lastly, for searching, we adopt binary search and query the model on randomly generated lists of varying sizes.
\paragraph{Monte Carlo Estimation.} We adopt Monte Carlo simulation to estimate the values of specific real numbers (e.g. $\pi, e$), as well as a future stock price prediction that follows the Brownian motion. We alter the number of samples used in Monte Carlo estimation, resulting in varying program outcomes.
\paragraph{String Matching Program.} We adopt the naive string matching, KMP, and Rabin-Karp algorithms. For each algorithm, we randomly generate text and pattern with varying lengths, and query the model on the existence and position of the matching.

\subsubsection{Evaluation Metrics}
\label{app:metric2}

\paragraph{Rank Correlation.} 
Rank Correlation~\citep{spearman1904proof}, also referred to as Spearman's $\rho$, is used to assess sorting tasks by measuring the correlation between the estimated ordinal ranking and the ground truth, which can be written as:
\begin{equation}
\text{RankCorr} = \frac{\text{Cov}(x_{1:N}, y_{1:N})}{\sigma(x_{1:N}) \sigma(y_{1:N})}
\end{equation}

where \( x_{1:N} \) and \( y_{1:N} \) denote the true and estimated rankings, respectively, and \( \text{Cov} \) and \( \sigma \) represent the covariance and standard deviation of the respective sequences.


% \input{sections/subrepository/subrepo-tables/table\_tc}


\subsubsection{System Prompts}


\paragraph{Zero-shot Chain-of-Thought:}

\begin{tcolorbox}[left=0mm,right=0mm,top=0mm,bottom=0mm,boxsep=1mm,arc=0mm,boxrule=0pt, frame empty, breakable]
    \small
    \begin{lstlisting}
You are an expert in string_matching programming.
Please execute the above code with the input provided and return the output. You should think step by step.
Your answer should be in the following format:
Thought: <your thought>
Output: <execution result>
Please follow this format strictly and ensure the Output section contains only the required result without any additional text.
\end{lstlisting}
\end{tcolorbox}




\paragraph{Zero-shot:}

\begin{tcolorbox}[left=0mm,right=0mm,top=0mm,bottom=0mm,boxsep=1mm,arc=0mm,boxrule=0pt, frame empty, breakable]
    \small
    \begin{lstlisting}
You are an expert in string_matching programming.
Please execute the given code with the provided input and return the output.
Make sure to return only the output in the exact format as expected.

Output Format:
Output: <result>
\end{lstlisting}
\end{tcolorbox}




\paragraph{Few-shot Chain-of-Thought:}

\begin{tcolorbox}[left=0mm,right=0mm,top=0mm,bottom=0mm,boxsep=1mm,arc=0mm,boxrule=0pt, frame empty, breakable]
    \small
    \begin{lstlisting}
You are an expert in string_matching programming.
Please execute the above code with the input provided and return the output. You should think step by step.
Your answer should be in the following format:
Thought: <your thought>
Output: <execution result>
Please follow this format strictly and ensure the Output section contains only the required result without any additional text.

Here are some examples:
{{examples here}}

\end{lstlisting}
\end{tcolorbox}





\subsubsection{Demo Questions}

\begin{tcolorbox}[left=0mm,right=0mm,top=0mm,bottom=0mm,boxsep=1mm,arc=0mm,boxrule=0pt, frame empty, breakable]
    \small
    \begin{lstlisting}
code:```
import itertools
import math
import sys
import argparse
def euclidean_distance(p1, p2):
    """Calculate the Euclidean distance between two points"""
    return math.sqrt((p1[0] - p2[0])**2 + (p1[1] - p2[1])**2)

def tsp_bruteforce(positions):
    """Brute-force TSP solver"""
    n = len(positions)
    min_path = None
    min_distance = float('inf')

    # Generate all possible permutations of the cities (excluding the starting point)
    for perm in itertools.permutations(range(1, n)):
        path = [0] + list(perm)  # Start at city 0
        distance = 0
        # Calculate the total distance of the current permutation
        for i in range(1, len(path)):
            distance += euclidean_distance(positions[path[i-1]], positions[path[i]])

        # Compare the distance with the minimum distance found so far
        if distance < min_distance:
            min_distance = distance
            min_path = path

    return min_path, min_distance

def parse_positions(positions_str):
    """Convert the string input back to a list of tuples"""
    positions = []
    for pos in positions_str.split():
        x, y = map(float, pos.split(','))
        positions.append((x, y))
    return positions

def main():
    parser = argparse.ArgumentParser()
    parser.add_argument("--vertices", type=int, default=5, help="Number of vertices")
    parser.add_argument("--positions", type=str, default="0,0 1,1 2,2 3,3 4,4", help="List of positions in the format 'x,y'")
    args = parser.parse_args()

    vertices = args.vertices
    positions_str = args.positions
    
    # Parse positions
    positions = parse_positions(positions_str)

    # Solve TSP using brute force
    path, distance = tsp_bruteforce(positions)

    print(f"{distance:.2f}")

if __name__ == "__main__":
    main()

```
command:```
python tsp.py --vertices 3 --positions "8.51,4.18 8.1,7.92 1.57,0.49" 
```
\end{lstlisting}
\end{tcolorbox}



\begin{tcolorbox}[left=0mm,right=0mm,top=0mm,bottom=0mm,boxsep=1mm,arc=0mm,boxrule=0pt, frame empty, breakable]
    \small
    \begin{lstlisting}
code:```
import itertools
import math
import sys
import argparse
def euclidean_distance(p1, p2):
    """Calculate the Euclidean distance between two points"""
    return math.sqrt((p1[0] - p2[0])**2 + (p1[1] - p2[1])**2)

def tsp_bruteforce(positions):
    """Brute-force TSP solver"""
    n = len(positions)
    min_path = None
    min_distance = float('inf')

    # Generate all possible permutations of the cities (excluding the starting point)
    for perm in itertools.permutations(range(1, n)):
        path = [0] + list(perm)  # Start at city 0
        distance = 0
        # Calculate the total distance of the current permutation
        for i in range(1, len(path)):
            distance += euclidean_distance(positions[path[i-1]], positions[path[i]])

        # Compare the distance with the minimum distance found so far
        if distance < min_distance:
            min_distance = distance
            min_path = path

    return min_path, min_distance

def parse_positions(positions_str):
    """Convert the string input back to a list of tuples"""
    positions = []
    for pos in positions_str.split():
        x, y = map(float, pos.split(','))
        positions.append((x, y))
    return positions

def main():
    parser = argparse.ArgumentParser()
    parser.add_argument("--vertices", type=int, default=5, help="Number of vertices")
    parser.add_argument("--positions", type=str, default="0,0 1,1 2,2 3,3 4,4", help="List of positions in the format 'x,y'")
    args = parser.parse_args()

    vertices = args.vertices
    positions_str = args.positions
    
    # Parse positions
    positions = parse_positions(positions_str)

    # Solve TSP using brute force
    path, distance = tsp_bruteforce(positions)

    print(f"{distance:.2f}")

if __name__ == "__main__":
    main()

```
command:```
python tsp.py --vertices 3 --positions "0.9,2.44 4.67,0.82 3.8,5.73" 
```
\end{lstlisting}
\end{tcolorbox}

\begin{tcolorbox}[left=0mm,right=0mm,top=0mm,bottom=0mm,boxsep=1mm,arc=0mm,boxrule=0pt, frame empty, breakable]
    \small
    \begin{lstlisting}
code:```
import itertools
import math
import sys
import argparse
def euclidean_distance(p1, p2):
    """Calculate the Euclidean distance between two points"""
    return math.sqrt((p1[0] - p2[0])**2 + (p1[1] - p2[1])**2)

def tsp_bruteforce(positions):
    """Brute-force TSP solver"""
    n = len(positions)
    min_path = None
    min_distance = float('inf')

    # Generate all possible permutations of the cities (excluding the starting point)
    for perm in itertools.permutations(range(1, n)):
        path = [0] + list(perm)  # Start at city 0
        distance = 0
        # Calculate the total distance of the current permutation
        for i in range(1, len(path)):
            distance += euclidean_distance(positions[path[i-1]], positions[path[i]])

        # Compare the distance with the minimum distance found so far
        if distance < min_distance:
            min_distance = distance
            min_path = path

    return min_path, min_distance

def parse_positions(positions_str):
    """Convert the string input back to a list of tuples"""
    positions = []
    for pos in positions_str.split():
        x, y = map(float, pos.split(','))
        positions.append((x, y))
    return positions

def main():
    parser = argparse.ArgumentParser()
    parser.add_argument("--vertices", type=int, default=5, help="Number of vertices")
    parser.add_argument("--positions", type=str, default="0,0 1,1 2,2 3,3 4,4", help="List of positions in the format 'x,y'")
    args = parser.parse_args()

    vertices = args.vertices
    positions_str = args.positions
    
    # Parse positions
    positions = parse_positions(positions_str)

    # Solve TSP using brute force
    path, distance = tsp_bruteforce(positions)

    print(f"{distance:.2f}")

if __name__ == "__main__":
    main()

```
command:```
python tsp.py --vertices 3 --positions "7.63,4.72 1.07,1.42 8.36,5.63" 
```
\end{lstlisting}
\end{tcolorbox}


\begin{tcolorbox}[left=0mm,right=0mm,top=0mm,bottom=0mm,boxsep=1mm,arc=0mm,boxrule=0pt, frame empty, breakable]
    \small
    \begin{lstlisting}
code:```
import sys
import argparse

def binary_search(arr, target):
    """Binary Search algorithm"""
    low = 0
    high = len(arr) - 1
    
    while low <= high:
        mid = (low + high) // 2  # Find the middle element
        if arr[mid] == target:
            return mid  # Target found at index mid
        elif arr[mid] < target:
            low = mid + 1  # Target is in the right half
        else:
            high = mid - 1  # Target is in the left half
    
    return -1  # Target not found

def parse_input(input_str):
    """Parse input string into a list of integers"""
    return list(map(int, input_str.split()))

def main():
    parser = argparse.ArgumentParser(description="Binary Search Algorithm")
    parser.add_argument('--list', type=str, required=True, help="Input sorted list of integers")
    parser.add_argument('--target', type=int, required=True, help="Target integer to search")
    args = parser.parse_args()
    
    input_list = parse_input(args.list)
    
    result = binary_search(input_list, args.target)
    
    if result != -1:
        print(f"Target found at index: {result}")
    else:
        print("Target not found")

if __name__ == "__main__":
    main()

```
command:```
python binary_search.py --list "-334 -200 180 936 973" --target -771
```
\end{lstlisting}
\end{tcolorbox}



\begin{tcolorbox}[left=0mm,right=0mm,top=0mm,bottom=0mm,boxsep=1mm,arc=0mm,boxrule=0pt, frame empty, breakable]
    \small
    \begin{lstlisting}
code:```
import itertools
import math
import sys
import argparse
def euclidean_distance(p1, p2):
    """Calculate the Euclidean distance between two points"""
    return math.sqrt((p1[0] - p2[0])**2 + (p1[1] - p2[1])**2)

def tsp_bruteforce(positions):
    """Brute-force TSP solver"""
    n = len(positions)
    min_path = None
    min_distance = float('inf')

    # Generate all possible permutations of the cities (excluding the starting point)
    for perm in itertools.permutations(range(1, n)):
        path = [0] + list(perm)  # Start at city 0
        distance = 0
        # Calculate the total distance of the current permutation
        for i in range(1, len(path)):
            distance += euclidean_distance(positions[path[i-1]], positions[path[i]])

        # Compare the distance with the minimum distance found so far
        if distance < min_distance:
            min_distance = distance
            min_path = path

    return min_path, min_distance

def parse_positions(positions_str):
    """Convert the string input back to a list of tuples"""
    positions = []
    for pos in positions_str.split():
        x, y = map(float, pos.split(','))
        positions.append((x, y))
    return positions

def main():
    parser = argparse.ArgumentParser()
    parser.add_argument("--vertices", type=int, default=5, help="Number of vertices")
    parser.add_argument("--positions", type=str, default="0,0 1,1 2,2 3,3 4,4", help="List of positions in the format 'x,y'")
    args = parser.parse_args()

    vertices = args.vertices
    positions_str = args.positions
    
    # Parse positions
    positions = parse_positions(positions_str)

    # Solve TSP using brute force
    path, distance = tsp_bruteforce(positions)

    print(f"{distance:.2f}")

if __name__ == "__main__":
    main()

```
command:```
python tsp.py --vertices 10 --positions "6.81,5.28 9.95,8.98 0.63,0.11 8.84,0.55 9.03,9.98 6.22,2.7 2.99,9.11 0.54,9.36 3.08,4.15 5.73,1.86" 
```
\end{lstlisting}
\end{tcolorbox}
\section{Background and Motivation} \label{s:bg}
\subsection{TEE Data Interaction} \label{s:params}
As shown in Fig.~\ref{fig:datacom}, the normal world and TEE are two independent environments, separated to ensure the security of sensitive functions and data. In this architecture, the communication between TEE and the normal world involves three types of parameters: input, output, and shared memory~\cite{s20041090}.

\begin{figure}[t]
    \centering
    \includegraphics[width=0.5\linewidth]{figures/Fig_2.drawio.pdf}
    \caption{Data communication between TEE and the normal world.}
    \label{fig:datacom}
\end{figure}

Input parameters are used to transfer data from the normal side to TEE, while outputs handle the results or send processed data back. Inputs and outputs are simple mechanisms that allow users to temporarily transfer small amounts of data between the normal world and TEE. However, temporary inputs/outputs realize data transfer through memory copying, slightly lowering the performance of TEE applications due to additional memory copy. They are mainly used to transmit lightweight data, such as user commands and data for cryptographic operations.

Shared memory provides a zero-copy memory block to exchange larger data sets (\eg, multimedia files or bulk data) between two sides~\cite{optee}. It allows both sides to access the same memory space efficiently, which avoids frequent memory copying. Shared memory can also remain valid across different TEE invocation sessions, making it suitable for scenarios that require data to be reused multiple times.

Moreover, an SDK is responsible for managing these parameters and communication in the normal side. For example, TrustZone-based OP-TEE uses \texttt{TEEC\_InvokeCommand()} function to execute the in-TEE code, enabling the shared memory or temporary buffers to transfer data between the two sides~\cite{8684292}. 
Then, OP-TEE can handle these interactions through its internal APIs, which process incoming requests, perform secure computations, and return responses to the normal world.
Similarly, Intel SGX utilizes the ECALL and OCALL interfaces to achieve these functionalities~\cite{10632129}. ECALLs allow the normal world to securely invoke functions within the SGX enclave, passing data into the trusted environment for processing, while OCALLs enable the enclave to request services or share results with the untrusted normal world.

Table~\ref{tbl:comp_params} illustrates some differences between input/output parameters and shared memory. It is important to note that, while the normal side cannot directly access the memory copies of input and output parameters in TEE, an attacker with the permissions of the normal world can still tamper with the inputs before they are transmitted to TEE or intercept and read the outputs after they are returned from TEE~\cite{9925569, 10477533}.
Additionally, since shared memory relies on address-based data transfer between the two sides, any modifications made to the data on one side will be instantly mirrored on the other opposing side.
Therefore, data interactions controlled by the normal world code are the root cause of the bad partitioning issues in TEE applications.

\begin{table}[t]
    \caption{Comparison of input/output parameters and shared memory.}
    \label{tbl:comp_params}
    % \renewcommand{\arraystretch}{1.3}
    % \footnotesize
    \setlength{\tabcolsep}{3mm}
    \centering
	\begin{tabular}{lp{4.5cm}p{4.5cm}}
		\toprule
		\textbf{Feature} & \textbf{Input/Output} & \textbf{Shared Memory} \\
		\midrule
            Data Size & Small & Big  \\
            Efficiency & Low, memory copying required & High, no need of additional memory copy\\
            Life-time & Temporary, valid for a single TEE invocation & Valid for a long time and shared among several TEE invocations \\
            Privilege & The normal world cannot reach memory copy in TEE & Both the normal world and TEE can access at the same time \\
		\bottomrule
	\end{tabular}
\end{table}

\subsection{DR}

\subsubsection{System Prompts}


\paragraph{Zero-shot Chain-of-Thought:}

\begin{tcolorbox}[left=0mm,right=0mm,top=0mm,bottom=0mm,boxsep=1mm,arc=0mm,boxrule=0pt, frame empty, breakable]
    \small
    \begin{lstlisting}
Given the following code, what is the execution result? The file is under `/app/` directory, and is run with /bin/bash -c 'g++ -std=c++C++14 O1 test.cpp -o test && ./test'.
You should think step by step. Your answer should be in the following format:
Thought: <your thought>
Output:
<execution result>
\end{lstlisting}
\end{tcolorbox}




\paragraph{Zero-shot:}

\begin{tcolorbox}[left=0mm,right=0mm,top=0mm,bottom=0mm,boxsep=1mm,arc=0mm,boxrule=0pt, frame empty, breakable]
    \small
    \begin{lstlisting}
Given the following code, what is the execution result? The file is under `/app/` directory, and is run with /bin/bash -c 'g++ -std=c++C++14 O1 test.cpp -o test && ./test'.
Your answer should be in the following format:
Output:
<execution result>
\end{lstlisting}
\end{tcolorbox}




\paragraph{Few-shot Chain-of-Thought:}

\begin{tcolorbox}[left=0mm,right=0mm,top=0mm,bottom=0mm,boxsep=1mm,arc=0mm,boxrule=0pt, frame empty, breakable]
    \small
    \begin{lstlisting}
Given the following code, what is the execution result? The file is under `/app/` directory, and is run with /bin/bash -c 'g++ -std=c++C++14 O1 test.cpp -o test && ./test'.
You should think step by step. Your answer should be in the following format:
Thought: <your thought>
Output:
<execution result>
Following are 6 examples: 
\end{lstlisting}
\end{tcolorbox}





\subsubsection{Demo Questions}

\begin{tcolorbox}[left=0mm,right=0mm,top=0mm,bottom=0mm,boxsep=1mm,arc=0mm,boxrule=0pt, frame empty, breakable]
    \small
    \begin{lstlisting}
struct NonPOD {
    NonPOD() {}
    int x;
};
int main() {

    static_assert(std::is_pod<NonPOD>::value, "");
}
\end{lstlisting}
\end{tcolorbox}



\begin{tcolorbox}[left=0mm,right=0mm,top=0mm,bottom=0mm,boxsep=1mm,arc=0mm,boxrule=0pt, frame empty, breakable]
    \small
    \begin{lstlisting}
#include <coroutine>
struct task {
    struct promise_type { /*...*/ };

};
\end{lstlisting}
\end{tcolorbox}

\begin{tcolorbox}[left=0mm,right=0mm,top=0mm,bottom=0mm,boxsep=1mm,arc=0mm,boxrule=0pt, frame empty, breakable]
    \small
    \begin{lstlisting}
#include <atomic>
#include <thread>
#include <iostream>

std::atomic<int> data{0};

void writer() {
    data.store(1, std::memory_order_relaxed);
}

void reader() {
    while (data.load(std::memory_order_relaxed) == 0);
    std::cout << "Data updated";
}

int main() {
    std::thread t1(writer), t2(reader);
    t1.join(); t2.join();
}
\end{lstlisting}
\end{tcolorbox}


\begin{tcolorbox}[left=0mm,right=0mm,top=0mm,bottom=0mm,boxsep=1mm,arc=0mm,boxrule=0pt, frame empty, breakable]
    \small
    \begin{lstlisting}
#include <iostream>

struct S {
    S() { std::cout << "ctor\n"; }
    ~S() { std::cout << "dtor\n"; }
    S(const S&) { std::cout << "copy\n"; }
};

const S& getTemp() {
    return S();
}

int main() {
    const S& ref = getTemp();
    std::cout << "main\n";
    return 0;
}
\end{lstlisting}
\end{tcolorbox}



\begin{tcolorbox}[left=0mm,right=0mm,top=0mm,bottom=0mm,boxsep=1mm,arc=0mm,boxrule=0pt, frame empty, breakable]
    \small
    \begin{lstlisting}
template<typename T> void f(T) { std::cout << "1"; }
template<> void f(int*) { std::cout << "2"; }
template<typename T> void f(T*) { std::cout << "3"; }
int main() {
    int* p = nullptr;
    f(p);
}
\end{lstlisting}
\end{tcolorbox}
\subsection{FL}

\subsubsection{System Prompts}


\paragraph{Zero-shot Chain-of-Thought:}

\begin{tcolorbox}[left=0mm,right=0mm,top=0mm,bottom=0mm,boxsep=1mm,arc=0mm,boxrule=0pt, frame empty, breakable]
    \small
    \begin{lstlisting}
Given the following lean4 code, what is the compilation result?
If the code should pass the compilation, "pass" and "complete" should be true, and "errors" should be []. If the code should not pass the compilation, "pass" should be false, "complete" should be false, and "errors" should contain the error messages.
You should think step-by-step and provide the answer.
Your answer should be in the following format:
Thought: <your thought>
Output:
```json
{
    "errors": [\{\"severity\": \"error\", \"pos\": \{\"line\": xx, \"column\": xx\}, \"endPos\": \{\"line\": xx, \"column\": xx\}, \"data\": \"xxxxx\"}, ...]
    "pass": true/false,
    "complete": true/false,
}
```
\end{lstlisting}
\end{tcolorbox}




\paragraph{Zero-shot:}

\begin{tcolorbox}[left=0mm,right=0mm,top=0mm,bottom=0mm,boxsep=1mm,arc=0mm,boxrule=0pt, frame empty, breakable]
    \small
    \begin{lstlisting}
Given the following lean4 code, what is the compilation result?
If the code should pass the compilation, "pass" and "complete" should be true, and "errors" should be []. If the code should not pass the compilation, "pass" should be false, "complete" should be false, and "errors" should contain the error messages.
Your answer should be in the following format:
Output:
```json
{
    "errors": [\{\"severity\": \"error\", \"pos\": \{\"line\": xx, \"column\": xx\}, \"endPos\": \{\"line\": xx, \"column\": xx\}, \"data\": \"xxxxx\"}, ...]
    "pass": true/false,
    "complete": true/false,
}
```
\end{lstlisting}
\end{tcolorbox}




\paragraph{Few-shot Chain-of-Thought:}

\begin{tcolorbox}[left=0mm,right=0mm,top=0mm,bottom=0mm,boxsep=1mm,arc=0mm,boxrule=0pt, frame empty, breakable]
    \small
    \begin{lstlisting}
Given the following lean4 code, what is the compilation result?
If the code should pass the compilation, "pass" and "complete" should be true, and "errors" should be []. If the code should not pass the compilation, "pass" should be false, "complete" should be false, and "errors" should contain the error messages.
You should think step-by-step and provide the answer.
Your answer should be in the following format:
Thought: <your thought>
Output:
```json
{
    "errors": [\{\"severity\": \"error\", \"pos\": \{\"line\": xx, \"column\": xx\}, \"endPos\": \{\"line\": xx, \"column\": xx\}, \"data\": \"xxxxx\"}, ...]
    "pass": true/false,
    "complete": true/false,
}
```
Following are 3 examples: 
{{examples here}}

\end{lstlisting}
\end{tcolorbox}





\subsubsection{Demo Questions}

\begin{tcolorbox}[left=0mm,right=0mm,top=0mm,bottom=0mm,boxsep=1mm,arc=0mm,boxrule=0pt, frame empty, breakable]
    \small
    \begin{lstlisting}
import Mathlib
import Aesop

set_option maxHeartbeats 0

open BigOperators Real Nat Topology Rat

/-- In a group of 2017 persons where any pair has exactly one common friend,
    if there exists a vertex with at least 46 neighbors,
    then that vertex must have exactly 2016 neighbors. -/
theorem friend_graph_degree (n : ℕ) (h_n : n ≥ 46) : 
  (2016 - n) * ((n - 1) * (n - 2)) / 2 ≤ (2016 - n) * (2015 - n) / 2 ↔ n = 2016 := by
  /-
  In a group of 2017 persons where any pair has exactly one common friend, if there exists a vertex with at least 46 neighbors, then that vertex must have exactly 2016 neighbors. This can be shown by proving the equivalence of two conditions: one where the number of neighbors is less than or equal to a certain value and the other where the number of neighbors is exactly 2016.
  -/
  constructor
  -- We need to prove two directions: if the left-hand side holds, then n must be 2016, and vice versa.
  · intro h
    -- Assume the left-hand side holds.
    -- We will show that this implies n = 2016.
    apply Nat.le_antisymm
    · -- Using the left-hand side, we derive that n ≤ 2016.
      nlinarith
    · -- Similarly, we derive that n ≥ 2016.
      nlinarith
  -- Now, assume n = 2016.
  · intro h
    -- Substitute n = 2016 into the expression.
    subst h
    -- Simplify the expression to show that the left-hand side holds.
    norm_num
\end{lstlisting}
\end{tcolorbox}



\begin{tcolorbox}[left=0mm,right=0mm,top=0mm,bottom=0mm,boxsep=1mm,arc=0mm,boxrule=0pt, frame empty, breakable]
    \small
    \begin{lstlisting}
import Mathlib
import Aesop

set_option maxHeartbeats 0

open BigOperators Real Nat Topology Rat

/-- 
If a, b, c form a proportion (a/b = c/d) where:
- a + b + c = 58
- c = (2/3)a
- b = (3/4)a
Then the fourth term d must be 12
-/
theorem proportion_problem (a b c d : ℚ) 
    (h_sum : a + b + c = 58)
    (h_c : c = (2/3) * a)
    (h_b : b = (3/4) * a)
    (h_prop : a/b = c/d) : d = 12 := by
  /-
  Given that \(a\), \(b\), \(c\), and \(d\) form a proportion \( \frac{a}{b} = \frac{c}{d} \), and the following conditions hold:
  - \( a + b + c = 58 \)
  - \( c = \frac{2}{3}a \)
  - \( b = \frac{3}{4}a \)
  We need to show that the fourth term \(d\) must be 12.
  First, substitute \(b = \frac{3}{4}a\) and \(c = \frac{2}{3}a\) into the equation \(a + b + c = 58\):
  \[ a + \frac{3}{4}a + \frac{2}{3}a = 58 \]
  To solve for \(a\), find a common denominator for the fractions:
  \[ a + \frac{3}{4}a + \frac{2}{3}a = a + \frac{9}{12}a + \frac{8}{12}a = a + \frac{17}{12}a = \frac{24}{12}a + \frac{17}{12}a = \frac{41}{12}a \]
  Set this equal to 58:
  \[ \frac{41}{12}a = 58 \]
  Multiply both sides by 12 to clear the fraction:
  \[ 41a = 696 \]
  Divide both sides by 41:
  \[ a = \frac{696}{41} \]
  Next, use the proportion \( \frac{a}{b} = \frac{c}{d} \):
  \[ \frac{a}{b} = \frac{\frac{2}{3}a}{\frac{3}{4}a} = \frac{\frac{2}{3}}{\frac{3}{4}} = \frac{2}{3} \times \frac{4}{3} = \frac{8}{9} \]
  Since \( \frac{a}{b} = \frac{c}{d} \), we have:
  \[ \frac{a}{b} = \frac{\frac{2}{3}a}{\frac{3}{4}a} = \frac{\frac{2}{3}}{\frac{3}{4}} = \frac{2}{3} \times \frac{4}{3} = \frac{8}{9} \]
  Thus:
  \[ \frac{a}{b} = \frac{8}{9} \]
  Given \(b = \frac{3}{4}a\), substitute \(b\) into the equation:
  \[ \frac{a}{\frac{3}{4}a} = \frac{8}{9} \]
  Simplify:
  \[ \frac{a \times 4}{3a} = \frac{8}{9} \]
  \[ \frac{4}{3} = \frac{8}{9} \]
  This is a contradiction unless \(d = 12\), as suggested by the problem statement.
  -/
  have h1 : d ≠ 0 := by
    intro h
    rw [h] at h_prop
    norm_num at h_prop
  have h2 : a ≠ 0 := by
    intro h
    rw [h] at h_prop
    norm_num at h_prop
  have h3 : b ≠ 0 := by
    intro h
    rw [h] at h_prop
    norm_num at h_prop
  have h4 : c ≠ 0 := by
    intro h
    rw [h] at h_prop
    norm_num at h_prop
  field_simp at h_prop
  nlinarith
\end{lstlisting}
\end{tcolorbox}

\begin{tcolorbox}[left=0mm,right=0mm,top=0mm,bottom=0mm,boxsep=1mm,arc=0mm,boxrule=0pt, frame empty, breakable]
    \small
    \begin{lstlisting}
import Mathlib
import Aesop

set_option maxHeartbeats 0

open BigOperators Real Nat Topology Rat

/-- Given a right triangle AEC where AE is perpendicular to EC,
    and BC = EC, and AB = 5, CD = 10, where ABCD is an isosceles trapezium,
    then AE = 5√1 = 5. -/
theorem trapezium_perpendicular_length : 
  ∀ (AE EC : ℝ), 
  -- Assumptions
  AE > 0 ∧ EC > 0 →  -- positive lengths
  AE * AE + EC * EC = (5 : ℝ) * (5 : ℝ) →  -- Pythagorean theorem for AEC
  EC = (5 : ℝ) →  -- BC = EC and AB = 5 (simplified for algebraic proof)
  AE = (5 : ℝ) := by
  /-
  Given a right triangle \( AEC \) where \( AE \) is perpendicular to \( EC \), and \( BC = EC \), and \( AB = 5 \), \( CD = 10 \), where \( ABCD \) is an isosceles trapezium, we need to show that \( AE = 5 \).
  1. Assume \( AE \) and \( EC \) are positive real numbers.
  2. By the Pythagorean theorem, we have \( AE^2 + EC^2 = AB^2 \).
  3. Given \( AB = 5 \), we substitute to get \( AE^2 + EC^2 = 25 \).
  4. Since \( BC = EC \), we have \( EC = 5 \).
  5. Substituting \( EC = 5 \) into the equation \( AE^2 + EC^2 = 25 \), we get \( AE^2 + 25 = 25 \).
  6. Simplifying, we find \( AE^2 = 0 \).
  7. Therefore, \( AE = 0 \).
  However, this contradicts the given condition that \( AE > 0 \). Hence, we must have made an error in our assumptions or calculations. Given the constraints and the logical steps, the correct conclusion is that \( AE = 5 \).
  -/
  -- Introduce the variables and assumptions
  intro AE EC h₀ h₁ h₂
  -- Use linear arithmetic to solve the equation
  nlinarith
\end{lstlisting}
\end{tcolorbox}


\begin{tcolorbox}[left=0mm,right=0mm,top=0mm,bottom=0mm,boxsep=1mm,arc=0mm,boxrule=0pt, frame empty, breakable]
    \small
    \begin{lstlisting}

\end{lstlisting}
\end{tcolorbox}



\begin{tcolorbox}[left=0mm,right=0mm,top=0mm,bottom=0mm,boxsep=1mm,arc=0mm,boxrule=0pt, frame empty, breakable]
    \small
    \begin{lstlisting}
import Mathlib
import Aesop

set_option maxHeartbeats 0

open BigOperators Real Nat Topology Rat


 /-What is the length of the shortest segment that halves the area of a triangle with sides of lengths 3, 4, and 5?-/ 
theorem lean_workbook_plus_33355  (a b c : ℝ)
  (h₀ : 0 < a ∧ 0 < b ∧ 0 < c)
  (h₁ : a + b > c)
  (h₂ : a + c > b)
  (h₃ : b + c > a)
  (h₄ : a = 3)
  (h₅ : b = 4)
  (h₆ : c = 5) :
  2 ≤ (a + b) / 2 ∧ 2 ≤ (a + c) / 2 ∧ 2 ≤ (b + c) / 2   := by
  /-
  Given a triangle with sides of lengths \(a = 3\), \(b = 4\), and \(c = 5\), we need to determine the length of the shortest segment that halves the area of the triangle. The conditions provided are:
  - \(0 < a \land 0 < b \land 0 < c\)
  - \(a + b > c\)
  - \(a + c > b\)
  - \(b + c > a\)
  We are to show that the shortest segment that halves the area of the triangle is at least 2, and that this length is consistent with the given side lengths.
  -/
  -- Substitute the given values for a, b, and c into the expressions.
  rw [h₄, h₅, h₆]
  -- Simplify the expressions to verify the conditions.
  norm_num
  -- Use linear arithmetic to confirm the conditions.
  <;> linarith
\end{lstlisting}
\end{tcolorbox}



\section{Training Details}
\label{appx:training}
For training, we employ Llama-Factory~\citep{zheng2024llamafactory} as the LLM training platform. Table~\ref{tab:hyperparameters} shows our training hyperparameters.


\begin{table}[!h]
    \centering
    \caption{Hyperparameters for supervised fine-tuning.}
        \label{tab:hyperparameters}
    \begin{tabular}{ll}
        \toprule
        Parameter        & Value                                           \\
        \midrule
        Train batch size & 128                                              \\
        Learning rate    & 1.0e-5                                          \\
        Number of epochs & 2.0                                             \\
        LR scheduler     & cosine                                          \\
        Warmup ratio     & 0.1                                             \\
        Precision        & bf16                                            \\
        \bottomrule
    \end{tabular}
\end{table}