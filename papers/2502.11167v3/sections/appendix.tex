\newpage
\section*{Appendix}
\appendix
\label{sec:appendix}

\section{Prompts}

\subsection{Prompts for Dataset Refactoring}

\lstset{
    numbers=none,
    keywordstyle= \color{ blue!70},
    commentstyle= \color{red!50!green!50!blue!50},
    frame=none,
    rulesepcolor= \color{ red!20!green!20!blue!20} ,
    framexleftmargin=2em,
    columns=fullflexible,
    breaklines=true,
    breakindent=0pt,
    basicstyle=\ttfamily
}

\paragraph{ML:}

\begin{tcolorbox}[left=0mm,right=0mm,top=0mm,bottom=0mm,boxsep=1mm,arc=0mm,boxrule=0pt, frame empty, breakable]
    \small
    \begin{lstlisting}
I will provide you with a code problem with a solution. You need to generate a complete, executable code based on the raw json data, including all necessary package imports, the original code, the test cases, and the main function. 
You need to generate the executable code and expected result.
Please choose a test case according to the 'test' field from raw json data, and the code should print the answer of the test case.
The output should be json format, with code and expected_result fields.
Please only generate the number or string answer in 'expected_result' field without any extra description.
\end{lstlisting}
\end{tcolorbox}

\paragraph{CL:}

\begin{tcolorbox}[left=0mm,right=0mm,top=0mm,bottom=0mm,boxsep=1mm,arc=0mm,boxrule=0pt, frame empty, breakable]
    \small
    \begin{lstlisting}
I will provide you with the solution to a code problem in cpp, python, and javascript. You need to score according to the difficulty of the problem from 1 to 5, while 5 means the hardest. And generate topic keywords for the problem.
The output should only be json format, with difficulty and keywords fields.
difficulty: 1-5, integer
keywords: two or three words to best describe the problem, string list
\end{lstlisting}
\end{tcolorbox}

\paragraph{BG:}

\begin{tcolorbox}[left=0mm,right=0mm,top=0mm,bottom=0mm,boxsep=1mm,arc=0mm,boxrule=0pt, frame empty, breakable]
    \small
    \begin{lstlisting}
I will provide you with a piece of code and some test cases. You need to generate a complete, executable code based on these, including all necessary package imports, the original code, the test cases, and the main function. You should wrap the original code with ORIGINAL_CODE_START and ORIGINAL_CODE_END comments. Additionally, the program should output the results of the test cases. Do not include expected output in your answer.
\end{lstlisting}
\end{tcolorbox}

\section{Details of \bench}

\subsection{ML}


\begin{table}[!ht]
  \centering
  \caption{Language usage count across different categories in the ML subset.}
  \label{tab:ml_language_usage}
  \begin{tabular}{lccccccc}
      \toprule
      Java & C\# & Rust & Julia & Python & C++ & C \\
      \midrule
      25   & 20  & 20   & 26    & 18     & 21  & 20 \\
      \bottomrule
  \end{tabular}
\end{table}

In ML, the usage distribution of various programming languages is shown in Table \ref{tab:ml_language_usage}. We selected a variety of languages, including Java, C\#, Rust, Julia, Python, C++, and C, to evaluate the model's ability to handle multilingual code. This diverse selection helps to comprehensively assess the model's performance across different languages.

\subsubsection{System Prompts}


\paragraph{Zero-shot Chain-of-Thought:}

\begin{tcolorbox}[left=0mm,right=0mm,top=0mm,bottom=0mm,boxsep=1mm,arc=0mm,boxrule=0pt, frame empty, breakable]
    \small
    \begin{lstlisting}
Given the following code, what is the execution result?
You should think step by step.  Your answer should be in the following format:
Thought: <your thought>
Output:
<execution result>
\end{lstlisting}
\end{tcolorbox}




\paragraph{Zero-shot:}

\begin{tcolorbox}[left=0mm,right=0mm,top=0mm,bottom=0mm,boxsep=1mm,arc=0mm,boxrule=0pt, frame empty, breakable]
    \small
    \begin{lstlisting}
Given the following code, what is the execution result?
Your answer should be in the following format:
Output:
<execution result>
\end{lstlisting}
\end{tcolorbox}




\paragraph{Few-shot Chain-of-Thought:}

\begin{tcolorbox}[left=0mm,right=0mm,top=0mm,bottom=0mm,boxsep=1mm,arc=0mm,boxrule=0pt, frame empty, breakable]
    \small
    \begin{lstlisting}
Given the following code, what is the execution result?
You should think step by step.  Your answer should be in the following format:
Thought: <your thought>
Output:
<execution result>
Following are 3 examples: 
{{examples here}}

\end{lstlisting}
\end{tcolorbox}





\subsubsection{Demo Questions}

\begin{tcolorbox}[left=0mm,right=0mm,top=0mm,bottom=0mm,boxsep=1mm,arc=0mm,boxrule=0pt, frame empty, breakable]
    \small
    \begin{lstlisting}
def catalan_number(n: int) -> int:
    # Initialize an array to store the intermediate catalan numbers
    catalan = [0] * (n + 1)
    catalan[0] = 1  # Base case

    # Calculate catalan numbers using the recursive formula
    for i in range(1, n + 1):
        for j in range(i):
            catalan[i] += catalan[j] * catalan[i - j - 1]

    return catalan[n]

if __name__ == "__main__":
    # Run the test function and print the result of a specific test case
    print(catalan_number(3))
\end{lstlisting}
\end{tcolorbox}



\begin{tcolorbox}[left=0mm,right=0mm,top=0mm,bottom=0mm,boxsep=1mm,arc=0mm,boxrule=0pt, frame empty, breakable]
    \small
    \begin{lstlisting}
import java.util.*;

class Solution {
    public static int countPrefixWords(List<String> wordList, String prefix) {

        int count = 0;
        for (String word : wordList) {
            if (word.startsWith(prefix)) {
                count++;
            }
        }
        return count;
    }

    public static void main(String[] args) {
        System.out.println(countPrefixWords(Arrays.asList("dog", "dodge", "dot", "dough"), "do"));
    }
}
\end{lstlisting}
\end{tcolorbox}

\begin{tcolorbox}[left=0mm,right=0mm,top=0mm,bottom=0mm,boxsep=1mm,arc=0mm,boxrule=0pt, frame empty, breakable]
    \small
    \begin{lstlisting}
#include <assert.h>
#include <stdio.h>

long long minTotalCost(int n, int *C)
{
   return (long long)(C[n-2]) * (n - 1) + C[n-1];
}

int main() {
    int costs3[] = {5, 4, 3, 2};
    printf("%lld\n", minTotalCost(4, costs3));
    return 0;
}
\end{lstlisting}
\end{tcolorbox}


\begin{tcolorbox}[left=0mm,right=0mm,top=0mm,bottom=0mm,boxsep=1mm,arc=0mm,boxrule=0pt, frame empty, breakable]
    \small
    \begin{lstlisting}
function merge_sorted_arrays(nums1::Vector{Int}, m::Int, nums2::Vector{Int}, n::Int) :: Vector{Int}
    i = m 
    j = n 
    k = m + n
    
    while j > 0
        if i > 0 && nums1[i] > nums2[j]
            nums1[k] = nums1[i]
            i -= 1
        else
            nums1[k] = nums2[j]
            j -= 1
        end
        k -= 1
    end
    
    nums1
end

# Test case
result = merge_sorted_arrays([1, 3, 5, 0, 0, 0], 3, [2, 4, 6], 3)
println(result)
\end{lstlisting}
\end{tcolorbox}



\begin{tcolorbox}[left=0mm,right=0mm,top=0mm,bottom=0mm,boxsep=1mm,arc=0mm,boxrule=0pt, frame empty, breakable]
    \small
    \begin{lstlisting}
public class Solution {

  public static int findSmallestInteger(int n) {
    char[] characters = Integer.toString(n).toCharArray();
    int i = characters.length - 2;

    // Find the first digit that is smaller than the digit next to it.
    while (i >= 0 && characters[i] >= characters[i + 1]) {
      i--;
    }

    if (i == -1) {
      return -1; // Digits are in descending order, no greater number possible.
    }

    // Find the smallest digit on right side of (i) which is greater than characters[i]
    int j = characters.length - 1;
    while (characters[j] <= characters[i]) {
      j--;
    }

    // Swap the digits at indices i and j
    swap(characters, i, j);

    // Reverse the digits from index i+1 to the end of the array
    reverse(characters, i + 1);

    try {
      return Integer.parseInt(new String(characters));
    } catch (NumberFormatException e) {
      return -1; // The number formed is beyond the range of int.
    }
  }

  private static void swap(char[] arr, int i, int j) {
    char temp = arr[i];
    arr[i] = arr[j];
    arr[j] = temp;
  }

  private static void reverse(char[] arr, int start) {
    int end = arr.length - 1;
    while (start < end) {
      swap(arr, start, end);
      start++;
      end--;
    }
  }

  public static void main(String[] args) {
    System.out.println(findSmallestInteger(123));
  }
}
\end{lstlisting}
\end{tcolorbox}
\subsection{CL}

\begin{table}[!ht]
  \centering
  \caption{Language usage count across different categories in the CL subset.}
  \label{tab:cl_language_usage}
  \begin{tabular}{lccc}
      \toprule
      Python & C++ & JavaScript \\
      \midrule
       50     & 51  & 49         \\
      \bottomrule
  \end{tabular}
\end{table}

\begin{table}[!ht]
  \centering
  \caption{Details of problems in different languages and different difficulty levels.}
  \label{tab:stat:2}
      \begin{tabular}{cccc}
\toprule
\textbf{Difficulty} & JavaScript & CPP & Python \\
\midrule
1 & 10 & 11 & 11 \\
2 & 6 & 4 & 6 \\
3 & 12 & 14 & 12 \\
4 & 8 & 8 & 9 \\
5 & 13 & 14 & 12 \\
\bottomrule
\end{tabular}
\end{table}

In CL, we selected competition problems of varying difficulty, each with solutions in Python, C++, and JavaScript. You can see the distribution of language in Table \ref{tab:cl_language_usage}, and the distribution of problem difficulty in Table \ref{tab:stat:2}. This selection allows us to test the model's cross-language capabilities and its ability to handle problems of different difficulty levels.

% \input{sections/subrepository/subrepo-tables/table_cl}

\subsubsection{System Prompts}


\paragraph{Zero-shot Chain-of-Thought:}

\begin{tcolorbox}[left=0mm,right=0mm,top=0mm,bottom=0mm,boxsep=1mm,arc=0mm,boxrule=0pt, frame empty, breakable]
    \small
    \begin{lstlisting}
Given the following code, what is the execution result?
You should think step by step.  Your answer should be in the following format:
Thought: <your thought>
Output:
<execution result>
\end{lstlisting}
\end{tcolorbox}




\paragraph{Zero-shot:}

\begin{tcolorbox}[left=0mm,right=0mm,top=0mm,bottom=0mm,boxsep=1mm,arc=0mm,boxrule=0pt, frame empty, breakable]
    \small
    \begin{lstlisting}
Given the following code, what is the execution result?
Your answer should be in the following format:
Output:
<execution result>
\end{lstlisting}
\end{tcolorbox}




\paragraph{Few-shot Chain-of-Thought:}

\begin{tcolorbox}[left=0mm,right=0mm,top=0mm,bottom=0mm,boxsep=1mm,arc=0mm,boxrule=0pt, frame empty, breakable]
    \small
    \begin{lstlisting}
Given the following code, what is the execution result?
You should think step by step.  Your answer should be in the following format:
Thought: <your thought>
Output:
<execution result>
Following are 3 examples: 
{{examples here}}

\end{lstlisting}
\end{tcolorbox}





\subsubsection{Demo Questions}

\begin{tcolorbox}[left=0mm,right=0mm,top=0mm,bottom=0mm,boxsep=1mm,arc=0mm,boxrule=0pt, frame empty, breakable]
    \small
    \begin{lstlisting}
class TreeNode {
  constructor(val) {
    this.val = val;
    this.left = this.right = null;
  }
}

function maxDepth(root) {
  if (!root) return 0;
  const queue = [root, null];
  let depth = 1;

  while (queue.length > 0) {
    const node = queue.shift();
    if (node === null) {
      if (queue.length === 0) return depth;
      depth++;
      queue.push(null);
      continue;
    }
    if (node.left) queue.push(node.left);
    if (node.right) queue.push(node.right);
  }

  return depth;
}

// Test case
const root = new TreeNode(3);
root.left = new TreeNode(9);
root.right = new TreeNode(20);
root.right.left = new TreeNode(15);
root.right.right = new TreeNode(7);
console.log(maxDepth(root));
\end{lstlisting}
\end{tcolorbox}



\begin{tcolorbox}[left=0mm,right=0mm,top=0mm,bottom=0mm,boxsep=1mm,arc=0mm,boxrule=0pt, frame empty, breakable]
    \small
    \begin{lstlisting}
from collections import Counter
class Solution:
    def maxScoreWords(self, words, letters, score):
        self.ans = 0
        words_score = [sum(score[ord(c)-ord('a')] for c in word) for word in words]
        words_counter = [Counter(word) for word in words]

        def backtrack(start, cur, counter):
            if start > len(words):
                return
            self.ans = max(self.ans, cur)
            for j, w_counter in enumerate(words_counter[start:], start):
                if all(n <= counter.get(c,0) for c,n in w_counter.items()):
                    backtrack(j+1, cur+words_score[j], counter-w_counter)

        backtrack(0, 0, Counter(letters))
        return self.ans

solution = Solution()
print(solution.maxScoreWords(["dog","cat","dad","good"], ["a","a","c","d","d","d","g","o","o"], [1,0,9,5,0,0,3,0,0,0,0,0,0,0,2,0,0,0,0,0,0,0,0,0,0,0]))
\end{lstlisting}
\end{tcolorbox}

\begin{tcolorbox}[left=0mm,right=0mm,top=0mm,bottom=0mm,boxsep=1mm,arc=0mm,boxrule=0pt, frame empty, breakable]
    \small
    \begin{lstlisting}
#include <iostream>
#include <unordered_map>
#include <string>
using namespace std;

int findTheLongestSubstring(string s) {
    unordered_map<char, int> mapper = {{'a', 1}, {'e', 2}, {'i', 4}, {'o', 8}, {'u', 16}};
    unordered_map<int, int> seen;
    seen[0] = -1;
    int max_len = 0, cur = 0;

    for(int i = 0; i < s.size(); ++i){
        if(mapper.find(s[i]) != mapper.end()){
            cur ^= mapper[s[i]];
        }
        if(seen.find(cur) != seen.end()){
            max_len = max(max_len, i - seen[cur]);
        } else {
            seen[cur] = i;
        }
    }

    return max_len;
}

// Test case
class Solution {
public:
    void solve() {
        string input = "eleetminicoworoep";
        cout << findTheLongestSubstring(input) << endl; // Expected output: 13
    }
};

int main(){
    Solution sol;
    sol.solve();
    return 0;
}
\end{lstlisting}
\end{tcolorbox}


\begin{tcolorbox}[left=0mm,right=0mm,top=0mm,bottom=0mm,boxsep=1mm,arc=0mm,boxrule=0pt, frame empty, breakable]
    \small
    \begin{lstlisting}
class TreeNode {
    constructor(val) {
        this.val = val;
        this.left = this.right = null;
    }
}

function backtrack(root, sum, res, tempList) {
    if (root === null) return;
    if (root.left === null && root.right === null && sum === root.val)
        return res.push([...tempList, root.val]);

    tempList.push(root.val);
    backtrack(root.left, sum - root.val, res, tempList);
    backtrack(root.right, sum - root.val, res, tempList);
    tempList.pop();
}

function pathSum(root, sum) {
    if (root === null) return [];
    const res = [];
    backtrack(root, sum, res, []);
    return res;
}

// Test case setup
const root = new TreeNode(5);
root.left = new TreeNode(4);
root.right = new TreeNode(8);
root.left.left = new TreeNode(11);
root.right.left = new TreeNode(13);
root.right.right = new TreeNode(4);
root.left.left.left = new TreeNode(7);
root.left.left.right = new TreeNode(2);
root.right.right.left = new TreeNode(5);
root.right.right.right = new TreeNode(1);

console.log(pathSum(root, 22));
\end{lstlisting}
\end{tcolorbox}



\begin{tcolorbox}[left=0mm,right=0mm,top=0mm,bottom=0mm,boxsep=1mm,arc=0mm,boxrule=0pt, frame empty, breakable]
    \small
    \begin{lstlisting}
class TrieNode {
    constructor() {
        this.children = {};
        this.isEndOfWord = false;
    }
}

class Trie {
    constructor() {
        this.root = new TrieNode();
    }

    insert(word) {
        let node = this.root;
        for (let char of word) {
            if (!node.children[char]) {
                node.children[char] = new TrieNode();
            }
            node = node.children[char];
        }
        node.isEndOfWord = true;
    }

    search(stream) {
        let node = this.root;
        for (let char of stream) {
            if (!node.children[char]) {
                return false;
            }
            node = node.children[char];
            if (node.isEndOfWord) {
                return true;
            }
        }
        return false;
    }
}

class StreamChecker {
    constructor(words) {
        this.trie = new Trie();
        this.stream = [];

        for (let word of [...new Set(words)]) {
            this.trie.insert(word.split('').reverse().join(''));
        }
    }

    query(letter) {
        this.stream.unshift(letter);
        return this.trie.search(this.stream);
    }
}

// Test case
const streamChecker = new StreamChecker(["cd", "f", "kl"]);
console.log(streamChecker.query('a')); // false
console.log(streamChecker.query('b')); // false
console.log(streamChecker.query('c')); // false
console.log(streamChecker.query('d')); // true
console.log(streamChecker.query('e')); // false
console.log(streamChecker.query('f')); // true
console.log(streamChecker.query('g')); // false
console.log(streamChecker.query('h')); // false
console.log(streamChecker.query('i')); // false
console.log(streamChecker.query('j')); // false
console.log(streamChecker.query('k')); // false
console.log(streamChecker.query('l')); // true
\end{lstlisting}
\end{tcolorbox}
\subsection{RL}


\begin{table}[!ht]
  \centering
  \caption{Language usage count across different categories in the RL subset.}
  \label{tab:rl_language_usage}
  \begin{tabular}{lcc}
      \toprule
       Python & C++  \\
      \midrule
       24     & 36   \\
      \bottomrule
  \end{tabular}
\end{table}

In RL, the distribution of programming language usage is shown in Table \ref{tab:rl_language_usage}. We utilized five GitHub repositories for this study, consisting of two Python projects and three C++ projects. Each repository contains a set of ten or more test cases, providing a diverse set of data for evaluation across different programming languages.

\subsubsection{System Prompts}


\paragraph{Zero-shot Chain-of-Thought:}

\begin{tcolorbox}[left=0mm,right=0mm,top=0mm,bottom=0mm,boxsep=1mm,arc=0mm,boxrule=0pt, frame empty, breakable]
    \small
    \begin{lstlisting}
You will be given a github repository and a function that generates a latex file with this repo. Your task is to predict the content of the latex file generated by the function.
You should think step by step.  Your answer should be in the following format:
Thought: <your thought>
Output:
<file content>
\end{lstlisting}
\end{tcolorbox}




\paragraph{Zero-shot:}

\begin{tcolorbox}[left=0mm,right=0mm,top=0mm,bottom=0mm,boxsep=1mm,arc=0mm,boxrule=0pt, frame empty, breakable]
    \small
    \begin{lstlisting}
You will be given a github repository and a function that generates a latex file with this repo. Your task is to predict the content of the latex file generated by the function.
Your answer should be in the following format:
Output:
<file content>
\end{lstlisting}
\end{tcolorbox}




\paragraph{Few-shot Chain-of-Thought:}

\begin{tcolorbox}[left=0mm,right=0mm,top=0mm,bottom=0mm,boxsep=1mm,arc=0mm,boxrule=0pt, frame empty, breakable]
    \small
    \begin{lstlisting}
You will be given a github repository and a function that generates a latex file with this repo. Your task is to predict the content of the latex file generated by the function.
You should think step by step.  Your answer should be in the following format:
Thought: <your thought>
Output:
<file content>
Following is one example:  
{{examples here}}

\end{lstlisting}
\end{tcolorbox}





\subsubsection{Demo Questions}

\begin{tcolorbox}[left=0mm,right=0mm,top=0mm,bottom=0mm,boxsep=1mm,arc=0mm,boxrule=0pt, frame empty, breakable]
    \small
    \begin{lstlisting}
main.cpp:<start_file>#include <iostream>
#include <vector>
#include <utility>
#include <stdlib.h>
#include <time.h>
#include <unistd.h>
#include <fstream>
#include <map>
using namespace std;

typedef vector<vector<char> > Board;

const int N = 9;

class SudokuPlayer
{
private:
    // 使用位运算来表示某个数是否出现过
    int rowUsed[N];
    int columnUsed[N];
    int blockUsed[N];

public:
    vector<Board> result;
    vector<pair<int, int> > spaces;

public:
    SudokuPlayer()
    {
        initState();
    }

    void initState()
    {
        memset(rowUsed, 0, sizeof(rowUsed));
        memset(columnUsed, 0, sizeof(columnUsed));
        memset(blockUsed, 0, sizeof(blockUsed));
        spaces.clear();
        result.clear();
    }

    void addResult(Board &board)
    {
        vector<vector<char> > obj(board);
        result.push_back(obj);
    }

    void flip(int i, int j, int digit)
    {
        rowUsed[i] ^= (1 << digit);
        columnUsed[j] ^= (1 << digit);
        blockUsed[(i / 3) * 3 + j / 3] ^= (1 << digit);
    }

    vector<Board> solveSudoku(Board board)
    {
        initState();
        for (int i = 0; i < N; i++)
        {
            for (int j = 0; j < N; j++)
            {
                if (board[i][j] == '$')
                {
                    spaces.push_back(pair<int, int>(i, j));
                }
                else
                {
                    int digit = board[i][j] - '1';
                    flip(i, j, digit);
                }
            }
        }
        DFS(board, 0);
        return result;
    }

    void DFS(Board &board, int pos)
    {
        if (pos == spaces.size())
        {
            addResult(board);
            return;
        }
        int i = spaces[pos].first;
        int j = spaces[pos].second;
        int mask = ~(rowUsed[i] | columnUsed[j] | blockUsed[(i / 3) * 3 + j / 3]) & 0x1ff;
        int digit = 0;
        while (mask)
        {
            if (mask & 1)
            {
                flip(i, j, digit);
                board[i][j] = '1' + digit;
                DFS(board, pos + 1);
                flip(i, j, digit);
            }
            mask = mask >> 1;
            digit++;
        }
    }

    void getResult()
    {
        for (size_t i = 0; i < result.size(); i++)
        {
            Board board = result[i];
            printBoard(board);
        }
    }

    bool checkBoard(Board &board)
    {
        initState();
        for (int i = 0; i < 9; i++)
        {
            for (int j = 0; j < 9; j++)
            {
                if (board[i][j] != '$')
                {
                    int digit = board[i][j] - '1';
                    if ((rowUsed[i] | columnUsed[j] | blockUsed[(i / 3) * 3 + j / 3]) & (1 << digit))
                    {
                        return false;
                    }
                    flip(i, j, digit);
                }
            }
        }
        return true;
    }

    void printBoard(Board &board)
    {
        for (int i = 0; i < board.size(); i++)
        {
            for (int j = 0; j < board[i].size(); j++)
            {
                cout << board[i][j] << " ";
            }
            cout << "\n";
        }
    }

    Board generateBoard(int digCount)
    {
        vector<vector<char> > board(N, vector<char>(N, '$'));
        vector<int> row = getRand9();
        for (int i = 0; i < 3; i++)
        {
            board[3][i + 3] = row[i] + '1';
            board[4][i + 3] = row[i + 3] + '1';
            board[5][i + 3] = row[i + 6] + '1';
        }
        copySquare(board, 3, 3, true);
        copySquare(board, 3, 3, false);
        copySquare(board, 3, 0, false);
        copySquare(board, 3, 6, false);

        while (digCount)
        {
            int x = rand() % 9;
            int y = rand() % 9;
            if (board[x][y] == '$')
                continue;
            char tmp = board[x][y];
            board[x][y] = '$';

            solveSudoku(board);
            if (result.size() == 1)
            {
                digCount--;
            }
            else
            {
                board[x][y] = tmp;
            }
        }
        // printBoard(board);
        // cout << "spaces " << player.spaces.size() << "\n";
        if (!checkBoard(board))
        {
            cout << "wrong board" << endl;
        }

        return board;
    }

    vector<int> getRand9()
    {
        vector<int> result;
        int digit = 0;
        while (result.size() != 9)
        {
            int num = rand() % 9;
            if ((1 << num) & digit)
            {
                continue;
            }
            else
            {
                result.push_back(num);
                digit ^= (1 << num);
            }
        }
        return result;
    }

    void copySquare(Board &board, int src_x, int src_y, bool isRow)
    {
        int rand_tmp = rand() % 2 + 1;
        int order_first[3] = {1, 2, 0};
        int order_second[3] = {2, 0, 1};
        if (rand_tmp == 2)
        {
            order_first[0] = 2;
            order_first[1] = 0;
            order_first[2] = 1;
            order_second[0] = 1;
            order_second[1] = 2;
            order_second[2] = 0;
        }
        for (int i = 0; i < 3; i++)
        {
            if (isRow)
            {
                board[src_x][i] = board[src_x + order_first[0]][src_y + i];
                board[src_x + 1][i] = board[src_x + order_first[1]][src_y + i];
                board[src_x + 2][i] = board[src_x + order_first[2]][src_y + i];
                board[src_x][i + 6] = board[src_x + order_second[0]][src_y + i];
                board[src_x + 1][i + 6] = board[src_x + order_second[1]][src_y + i];
                board[src_x + 2][i + 6] = board[src_x + order_second[2]][src_y + i];
            }
            else
            {
                board[i][src_y] = board[src_x + i][src_y + order_first[0]];
                board[i][src_y + 1] = board[src_x + i][src_y + order_first[1]];
                board[i][src_y + 2] = board[src_x + i][src_y + order_first[2]];
                board[i + 6][src_y] = board[src_x + i][src_y + order_second[0]];
                board[i + 6][src_y + 1] = board[src_x + i][src_y + order_second[1]];
                board[i + 6][src_y + 2] = board[src_x + i][src_y + order_second[2]];
            }
        }
    }
};

char data[9][9] = {
    {'5', '3', '.', '.', '7', '.', '.', '.', '.'},
    {'6', '.', '.', '1', '9', '5', '.', '.', '.'},
    {'.', '9', '8', '.', '.', '.', '.', '6', '.'},
    {'8', '.', '.', '.', '6', '.', '.', '.', '3'},
    {'4', '.', '.', '8', '.', '3', '.', '.', '1'},
    {'7', '.', '.', '.', '2', '.', '.', '.', '6'},
    {'.', '6', '.', '.', '.', '.', '2', '8', '.'},
    {'.', '.', '.', '4', '1', '9', '.', '.', '5'},
    {'.', '.', '.', '.', '8', '.', '.', '7', '9'}};

void test()
{
    SudokuPlayer player;
    vector<vector<char> > board(N, vector<char>(N, '.'));

    for (int i = 0; i < board.size(); i++)
    {
        for (int j = 0; j < board[i].size(); j++)
        {
            board[i][j] = data[i][j];
        }
    }
    bool check = player.checkBoard(board);
    if (check)
        cout << "checked" << endl;

    player.solveSudoku(board);
    player.getResult();

    cout << endl;
}

vector<Board> readFile(string filePath)
{
    ifstream infile;
    vector<Board> boards;
    infile.open(filePath);
    char data[100];
    Board tmp;
    vector<char> row;
    while (!infile.eof())
    {
        infile.getline(data, 100);
        if (data[0] == '-')
        {
            boards.push_back(Board(tmp));
            tmp.clear();
            continue;
        }
        for (int i = 0; i < strlen(data); i++)
        {
            if (('1' <= data[i] && data[i] <= '9') || data[i] == '$')
            {
                row.push_back(data[i]);
            }
        }
        tmp.push_back(vector<char>(row));
        row.clear();
    }
    infile.close();
    return boards;
}

void writeFile(vector<Board> boards, ofstream &f)
{
    for (int k = 0; k < boards.size(); k++)
    {
        for (int i = 0; i < boards[k].size(); i++)
        {
            for (int j = 0; j < boards[k][i].size(); j++)
            {
                f << boards[k][i][j] << " ";
            }
            f << "\n";
        }
        f << "------- " << k << " -------" << endl;
    }
}

// 解析输入参数
map<char, string> parse(int argc, char *argv[])
{
    map<char, string> params;
    int compeleteBoardCount, gameNumber, gameLevel;
    vector<int> range;
    string inputFile;
    char opt = 0;
    while ((opt = getopt(argc, argv, "c:s:n:m:r:u")) != -1)
    {
        switch (opt)
        {
        case 'c':
            compeleteBoardCount = atoi(optarg);
            if (compeleteBoardCount < 1 || compeleteBoardCount > 1000000)
            {
                printf("生成数独终盘数量范围在1~1000000之间\n");
                exit(0);
            }
            params[opt] = string(optarg);
            break;
        case 's':
            inputFile = string(optarg);
            if (access(optarg, 0) == -1)
            {
                printf("file does not exist\n");
                exit(0);
            }
            params[opt] = string(optarg);
            break;
        case 'n':
            gameNumber = atoi(optarg);
            if (gameNumber < 1 || gameNumber > 10000)
            {
                printf("生成数独游戏数量范围在1~10000之间\n");
                exit(0);
            }
            params[opt] = string(optarg);
            break;
        case 'm':
            gameLevel = atoi(optarg);
            if (gameLevel < 1 || gameLevel > 3)
            {
                printf("生成游戏难度的范围在1~3之间\n");
                exit(0);
            }
            params[opt] = string(optarg);
            break;
        case 'r':
            char *p;
            p = strtok(optarg, "~");
            while (p)
            {
                range.push_back(atoi(p));
                p = strtok(NULL, "~");
            }
            if (range.size() != 2)
            {
                printf("请输入一个范围参数\n");
                exit(0);
            }
            if ((range[0] >= range[1]) || range[0] < 20 || range[1] > 55)
            {
                printf("请输入合法范围20~55\n");
                exit(0);
            }
            params[opt] = string(optarg);
            break;
        case 'u':
            params[opt] = string();
            break;
        default:
            printf("请输入合法参数\n");
            exit(0);
            break;
        }
    }
    return params;
}

void generateGame(int gameNumber, int gameLevel, vector<int> digCount, ofstream &outfile, SudokuPlayer &player)
{
    for (int i = 0; i < gameNumber; i++)
    {
        int cnt = 0;
        if (digCount.size() == 1)
        {
            cnt = digCount[0];
        }
        else
        {
            cnt = rand() % (digCount[1] - digCount[0] + 1) + digCount[0];
        }
        Board b = player.generateBoard(cnt);
        vector<Board> bs;
        bs.push_back(b);
        writeFile(bs, outfile);
    }
    outfile.close();
}

int main(int argc, char *argv[])
{
    srand((unsigned)time(NULL));
    SudokuPlayer player;

    map<char, string> params = parse(argc, argv);
    map<char, string>::iterator it, tmp;

    int opt = 0;

    vector<int> range;
    int gameNumber;
    int gameLevel = 0;
    int solution_count = 0;

    vector<Board> boards;
    ofstream outfile;

    it = params.begin();
    while (it != params.end())
    {
        switch (it->first)
        {
        case 'c':
            outfile.open("game.txt", ios::out | ios::trunc);
            range.push_back(0);
            generateGame(atoi(it->second.c_str()), 0, range, outfile, player);
            range.clear();
            break;

        case 's':
            outfile.open("sudoku.txt", ios::out | ios::trunc);
            boards = readFile(it->second);
            for (int i = 0; i < boards.size(); i++)
            {
                vector<Board> result = player.solveSudoku(boards[i]);
                writeFile(result, outfile);
            }
            outfile.close();
            break;

        case 'n':
        case 'm':
        case 'r':
        case 'u':
            tmp = params.find('n');
            if (tmp == params.end())
            {
                printf("缺少参数 n \n");
                exit(0);
            }

            gameNumber = atoi(tmp->second.c_str());

            tmp = params.find('u');
            if (tmp != params.end())
            {
                solution_count = 1;
            }

            tmp = params.find('m');
            if (tmp != params.end())
            {
                gameLevel = atoi(tmp->second.c_str());
            }

            tmp = params.find('r');
            if (tmp != params.end())
            {
                char *p;
                char *pc = new char[100];
                strcpy(pc, tmp->second.c_str());
                p = strtok(pc, "~");
                while (p)
                {
                    range.push_back(atoi(p));
                    p = strtok(NULL, "~");
                }
            }
            else
            {
                // 根据不同级别采取挖空数量不同
                if (gameLevel == 1)
                {
                    range.push_back(20);
                    range.push_back(30);
                }
                else if (gameLevel == 2)
                {
                    range.push_back(30);
                    range.push_back(40);
                }
                else if (gameLevel == 3)
                {
                    range.push_back(40);
                    range.push_back(55);
                }
                else
                {
                    range.push_back(20);
                    range.push_back(55);
                }
            }

            outfile.open("game.txt", ios::out | ios::trunc);
            generateGame(gameNumber, gameLevel, range, outfile, player);
            range.clear();
            break;
        }
        // cout << it->first << ' ' << it->second << endl;
        it++;
    }

    return 0;
}<end_file>;game.txt:<start_file><9 $ 5 $ 3 $ 7 1 2 
$ 1 2 $ $ 8 3 $ $ 
$ $ $ 2 7 $ 9 8 5 
8 $ 9 $ 6 $ 1 2 7 
1 $ $ $ 5 $ $ 6 3 
4 6 3 1 2 7 $ $ $ 
$ $ 8 3 4 6 2 7 1 
2 7 $ $ $ $ $ 3 $ 
$ 3 4 $ 1 $ $ $ 8 
------- 0 -------<endfile>
\end{lstlisting}
\end{tcolorbox}



\begin{tcolorbox}[left=0mm,right=0mm,top=0mm,bottom=0mm,boxsep=1mm,arc=0mm,boxrule=0pt, frame empty, breakable]
    \small
    \begin{lstlisting}
Here is the code repository:Cow.cpp:<start_file>#include "Cow.h"
Cow::Cow(std::string a,int b,int c,int d){
    name=a;
    l=b;
    u=c;
    m=d;
    in=0;
    state=0;
}<endfile>Cow.h:<start_file>#pragma once
#include <string>
class Cow{
    public:
    std::string name;
    int l,u,m;
    int in;
    int state;
    Cow(){}
    Cow(std::string a,int b,int c,int d);
};<endfile>Farm.cpp:<start_file>#include "Farm.h"
Farm::Farm(int a){
    n=a;
    num=0;
    cow=new Cow[a];
    milk=0;
}
void Farm::addCow(Cow a){
        cow[num]=a;
        num+=1;
    }
void Farm::supply(std::string a,int b){
    for(int i=0;i<n;i++){
        if(cow[i].name==a){
            cow[i].in+=b;
            break;
        }
    }
}
void Farm::startMeal(){
    for(int i=0;i<n;i++){
        if(cow[i].in==0)
        cow[i].state=0;
        if(cow[i].in>0&&cow[i].in<cow[i].l){
            cow[i].state=1;
            cow[i].in=0;
        }
        if(cow[i].in>=cow[i].l){
            cow[i].state=2;
            if(cow[i].in<=cow[i].u)
            cow[i].in=0;
            if(cow[i].in>cow[i].u)
            cow[i].in-=cow[i].u;
        }
    }
}
void Farm::produceMilk(){
    for(int i=0;i<n;i++){
        if(cow[i].state==0){
            milk+=0;
            continue;
        }
        if(cow[i].state==1){
            milk+=cow[i].m*0.5;
            continue;
        }
        if(cow[i].state==2){
            milk+=cow[i].m;
            continue;
        }
    }
}
float Farm::getMilkProduction(){
    return milk;
}<endfile>Farm.h:<start_file>#pragma once
#include"Cow.h"
class Farm{
    int n;
    int num;
    Cow* cow;
    public:
    float milk;
    Farm(int a);
    void addCow(Cow a);
    void supply(std::string a,int b);
    void startMeal();
    void produceMilk();
    float getMilkProduction();
    ~Farm(){
        delete[] cow;
    }
};<endfile>main.cpp:<start_file>#include <iostream>
#include <string>
#include "Cow.h"
#include "Farm.h"
using namespace std;

int main(){
    int n;
    cin >> n;
    Farm farm(n);
    string name;
    int l, u, m;
    for(int i = 0; i < n; ++i){
        cin >> name >> l >> u >> m;
        Cow cow(name, l, u, m);
        farm.addCow(cow);
    }

    int k;
    cin >> k;
    int t;
    int a;
    for(int i = 0; i < k; ++i){
        cin >> t;
        for(int j = 0; j < t; ++j){
            cin >> name >> a;
            farm.supply(name, a);
        }
        farm.startMeal();
        farm.produceMilk();
    }
    printf("%.1f", farm.getMilkProduction());
    return 0;
}<endfile>makefile:<start_file>main:main-3.o Farm.o Cow.o
	g++ main-3.o Farm.o Cow.o -o main

main-3.o:main-3.cpp Farm.h Cow.h
	g++ -c main-3.cpp -o main-3.o

Farm.o:Farm.cpp Farm.h Cow.h
	g++ -c Farm.cpp  -o Farm.o

Cow.o:Cow.cpp Cow.h
	g++ -c Cow.cpp  -o Cow.o

clean:
	rm *.o main<endfile>, and the input file is:./input/11.txt:<start_file>3
a 2 5 6
b 3 4 7
c 1 6 5
2
1 a 3
2 b 2 c 4<enfile>
\end{lstlisting}
\end{tcolorbox}

\begin{tcolorbox}[left=0mm,right=0mm,top=0mm,bottom=0mm,boxsep=1mm,arc=0mm,boxrule=0pt, frame empty, breakable]
    \small
    \begin{lstlisting}
Given the following code, what is the execution result? The file is under `/app/` directory, and is run with "python3 /app/test.py" if it is a python file, "g++ -std=c++11 /app/test.cpp -o /app/test
/app/test" if it is a cpp file, and "javac /app/\{class_name\}.java
java -cp /app \{class_name\}" if it is a java file.
You should think step by step.  Your answer should be in the following format:
Thought: <your thought>
Output:
<execution result>
\end{lstlisting}
\end{tcolorbox}


\begin{tcolorbox}[left=0mm,right=0mm,top=0mm,bottom=0mm,boxsep=1mm,arc=0mm,boxrule=0pt, frame empty, breakable]
    \small
    \begin{lstlisting}
Here is the code repository:car.cpp:<start_file>#include "car.h"
#include <iostream>
using namespace std;

Car::Car(int num,string eng):Vehicle(num,eng){}

void Car::describe(){
    cout<<"Finish building a car with "<<wheel.get_num()<<" wheels and a "<<engine.get_name()<<" engine."<<endl;
    cout<<"A car with "<<wheel.get_num()<<" wheels and a "<<engine.get_name()<<" engine."<<endl;
}

<endfile>car.h:<start_file>#pragma once
#include "vehicle.h"
using namespace std;

class Car: public Vehicle{
    public:
    Car(int num, string eng);
    void describe();
};<endfile>engine.cpp:<start_file>#include "engine.h"

Engine::Engine(string nam): name(nam) {
	cout << "Using "  << nam << " engine."<< endl;
}

string Engine::get_name() {
	return name;
}
<endfile>engine.h:<start_file>#pragma once
#include <iostream>
#include <string>
using namespace std;

class Engine {
	string name;
public:
	Engine(string);
	string get_name();
};<endfile>main.cpp:<start_file>
#include <iostream>
#include <string>
#include "wheel.h"
#include "engine.h"
#include "vehicle.h"
#include "motor.h"
#include "car.h"
using namespace std;

int main() {
	int n, type, num;
	string engine;

	cin >> n; 
	for (int i=0; i<n; i++) {
		cin >> type >> num >> engine;
		switch (type) {
			case 0: {
				Vehicle v = Vehicle(num, engine);
				v.describe();
				break;
			}
			case 1: {
				Motor m = Motor(num, engine);
				m.describe();
				m.sell();
				break;
			}
			case 2: {
				Car c = Car(num, engine);
				c.describe();
				break;
			}
		}
	}
	return 0;
}<endfile>motor.cpp:<start_file>#include "motor.h"
#include <iostream>
using namespace std;
Motor::Motor(int num,string eng):Vehicle(num,eng){}

void Motor::describe(){
    cout<<"Finish building a motor with "<<wheel.get_num()<<" wheels and a "<<engine.get_name()<<" engine."<<endl;
    cout<<"A motor with "<<wheel.get_num()<<" wheels and a "<<engine.get_name()<<" engine."<<endl;
}

void Motor::sell(){
    cout<<"A motor is sold!"<<endl;
}<endfile>motor.h:<start_file>#pragma once
#include "vehicle.h"
using namespace std;

class Motor: public Vehicle{
    public:
    Motor(int num, string eng);
    void describe();
    void sell();
};<endfile>vehicle.cpp:<start_file>#include "vehicle.h"
#include <iostream>
using namespace std;

Vehicle::Vehicle(int num,string eng):engine(eng),wheel(num){}

void Vehicle::describe(){
    cout<<"Finish building a vehicle with "<<wheel.get_num()<<" wheels and a "<<engine.get_name()<<" engine."<<endl;
    cout<<"A vehicle with "<<wheel.get_num()<<" wheels and a "<<engine.get_name()<<" engine."<<endl;
}<endfile>vehicle.h:<start_file>#pragma once
#include "wheel.h"
#include "engine.h"

using namespace std;

class Vehicle{
    public:
    Engine engine;
    Wheel wheel;
    Vehicle(int num, string eng);
    void describe();

};<endfile>wheel.cpp:<start_file>#include "wheel.h"

Wheel::Wheel(int num): number(num) {
	cout << "Building " << number << " wheels." << endl;
}

int Wheel::get_num() {
	return number;
}<endfile>wheel.h:<start_file>#pragma once
#include <iostream>
using namespace std;

class Wheel {
	int number;
public:
	Wheel(int);
	int get_num();
};<endfile>, and the input file is:./input/6.txt:<start_file>4
0 3 Gasoline
2 4 Hybrid
1 2 Electric
0 6 Magic<enfile>
\end{lstlisting}
\end{tcolorbox}



\begin{tcolorbox}[left=0mm,right=0mm,top=0mm,bottom=0mm,boxsep=1mm,arc=0mm,boxrule=0pt, frame empty, breakable]
    \small
    \begin{lstlisting}
Here is the code repository:24_game.py:<start_file>import itertools
import time
import math

# Operators
OP_CONST = 0  # Constant
OP_ADD = 1  # Addition
OP_SUB = 2  # Subtraction
OP_MUL = 3  # Multiplication
OP_DIV = 4  # Divition
OP_POW = 5  # Exponentiation

OP_SQRT = 6  # Squreroot
OP_FACT = 7  # Factorial
OP_LOG = 8  # Logarithm
OP_C = 9  # Combinations
OP_P = 10  # Permutations

# List of basic operators
operators = [OP_ADD,
             OP_SUB,
             OP_MUL,
             OP_DIV]

# List of advanced operators
advanced_operators = [OP_POW,
                      OP_LOG,
                      OP_C,
                      OP_P]

# List of unary operators
_unary_operators = [OP_SQRT,
                    OP_FACT]

# List of enabled unary operators
unary_operators = []

# Symbol of operators
symbol_of_operator = {OP_ADD: "%s+%s",
                      OP_SUB: "%s-%s",
                      OP_MUL: "%s*%s",
                      OP_DIV: "%s/%s",
                      OP_POW: "%s^%s",
                      OP_SQRT: "sqrt(%s)",
                      OP_FACT: "%s!",
                      OP_LOG: "log_%s(%s)",
                      OP_C: "C(%s, %s)",
                      OP_P: "P(%s, %s)"}

# Priority of operators
priority_of_operator = {OP_ADD: 0,
                        OP_SUB: 0,
                        OP_MUL: 1,
                        OP_DIV: 1,
                        OP_POW: 2,
                        OP_LOG: 3,
                        OP_C: 3,
                        OP_P: 3,
                        OP_SQRT: 3,
                        OP_FACT: 4,
                        OP_CONST: 5}

# Whether operator is commutative
is_operator_commutative = {OP_ADD: True,
                           OP_SUB: False,
                           OP_MUL: True,
                           OP_DIV: False,
                           OP_POW: False,
                           OP_LOG: False,
                           OP_C: False,
                           OP_P: False}

# Whether inside bracket is needed when rendering
need_brackets = {OP_ADD: True,
                 OP_SUB: True,
                 OP_MUL: True,
                 OP_DIV: True,
                 OP_POW: True,
                 OP_FACT: True,
                 OP_SQRT: False,
                 OP_LOG: False,
                 OP_C: False,
                 OP_P: False}


def permutation(n, k):
    return math.factorial(n)/math.factorial(k)


def combination(n, k):
    return permutation(n, k)/math.factorial(n-k)


def evaluate_operation(op, a, b=None):
    """
    Evaluate an operation on a and b.
    """
    if op == OP_ADD: return a + b
    if op == OP_SUB: return a - b
    if op == OP_MUL: return a * b

    try:
        if op == OP_POW and abs(a) < 20 and abs(b) < 20:
            return a ** b

        if op == OP_FACT and a < 10:
            return math.factorial(a)

        if op == OP_C and 0 < b <= a <= 13:
            return combination(a, b)

        if op == OP_P and 0 < b <= a <= 13:
            return permutation(a, b)

        if op == OP_SQRT and a < 1000000:
            return math.sqrt(a)

        if op == OP_DIV: return a / b
        if op == OP_LOG: return math.log(b, a)
    except (ZeroDivisionError, ValueError, TypeError):
        pass
    except OverflowError:
        print(a, b)

    return float("NaN")


def fit_to_int(x, eps=1e-9):
    """
    Convert x to int if x is close to an integer.
    """
    try:
        if abs(round(x) - x) <= eps:
            return round(x)
        else:
            return x
    except ValueError:
        return float("NaN")
    except TypeError:
        return float("NaN")


class Node:
    def __init__(self, value=None, left=None, right=None, op=OP_CONST):
        if op not in unary_operators \
                and op != OP_CONST and is_operator_commutative[op] \
                and str(left) > str(right):
            left, right = right, left

        self._value = value
        self._str_cache = None
        self.left = left
        self.right = right
        self.op = op

    @property
    def value(self):
        if self._value is None:
            assert self.op != OP_CONST

            if self.op in unary_operators:
                self._value = evaluate_operation(self.op, self.left.value)
            else:
                self._value = evaluate_operation(self.op, self.left.value, self.right.value)

            self._value = fit_to_int(self._value)
        return self._value

    def __str__(self):
        if self._str_cache is None:
            self._str_cache = self._str()
        return self._str_cache

    def _str(self):
        # Constant
        if self.op == OP_CONST:
            return str(self._value)

        # Unary operator
        elif self.op in unary_operators:
            str_left = str(self.left)

            if need_brackets[self.op] \
                    and priority_of_operator[self.left.op] < priority_of_operator[self.op]:
                str_left = "(" + str_left + ")"

            return symbol_of_operator[self.op] % str_left

        # Other operator
        else:
            str_left = str(self.left)
            str_right = str(self.right)

            # Add brackets inside
            if need_brackets[self.op] \
                    and priority_of_operator[self.left.op] < priority_of_operator[self.op]:
                str_left = "(" + str_left + ")"

            if need_brackets[self.op] \
                    and (priority_of_operator[self.right.op] < priority_of_operator[self.op]
                         or (priority_of_operator[self.right.op] == priority_of_operator[self.op]
                             and not is_operator_commutative[self.op])):
                str_right = "(" + str_right + ")"

            # Render
            return symbol_of_operator[self.op] % (str_left, str_right)


def enumerate_nodes(node_list, callback, max_depth):
    # Found an expression
    if len(node_list) == 1:
        callback(node_list[0])

    # Constrain maximum depth
    if max_depth == 0:
        return

    # Non-unary operators
    for left, right in itertools.permutations(node_list, 2):
        new_node_list = node_list.copy()
        new_node_list.remove(left)
        new_node_list.remove(right)

        for op in operators:
            enumerate_nodes(new_node_list + [Node(left=left, right=right, op=op)], callback, max_depth-1)

            if not is_operator_commutative[op] and str(left) != str(right):
                enumerate_nodes(new_node_list + [Node(left=right, right=left, op=op)], callback, max_depth-1)

    # Unary operators
    for number in node_list:
        new_node_list = node_list.copy()
        new_node_list.remove(number)

        for op in unary_operators:
            new_node = Node(left=number, op=op)
            if new_node.value == number.value:
                continue

            enumerate_nodes(new_node_list + [new_node], callback, max_depth-1)


class CallbackFindTarget:
    def __init__(self, target):
        self.target = target
        self.results = []
        self.duplication_count = 0
        self.enumeration_count = 0

    def __call__(self, node):
        if node.value == self.target and str(node) not in self.results:
            print(self.target, "=", node)
            self.results.append(str(node))
        elif node.value == self.target:
            self.duplication_count += 1

        self.enumeration_count += 1

    def show(self, execution_time):
        print()
        print("%d solution(s) in %.3f seconds" % (len(self.results), execution_time))
        print("%d duplication(s)" % self.duplication_count)
        print("%d combination(s)" % self.enumeration_count)


class CallbackAllTarget:
    def __init__(self):
        self.results = {}
        self.enumeration_count = 0

    def __call__(self, node):
        try:
            int(node.value)
        except ValueError:
            return

        if node.value not in self.results \
                and int(node.value) == node.value:
            self.results[node.value] = node

        self.enumeration_count += 1

    def __str__(self):
        string = ""
        for value in sorted(self.results.keys()):
            string += "%d = %s" % (value, str(self.results[value]))
            string += "\n"
        return string

    def show(self, execution_time):
        print(self)
        print()
        print("%d targets(s) in %.3f seconds" % (len(self.results), execution_time))
        print("%d combination(s)" % self.enumeration_count)


def select_yes_no(prompt, default=False):
    answer = input(prompt).strip().lower()
    if answer == "y":
        return True
    if answer == "n":
        return False
    return default


def select_int(prompt, default):
    try:
        return int(input(prompt).strip())
    except ValueError:
        return default


def main():
    global operators
    global unary_operators


    unary_operators_allowed = False
    enumerate_all = False


    if not enumerate_all:
        target = 24
        callback = CallbackFindTarget(target=target)

    if enumerate_all:
        callback = CallbackAllTarget()
    else:
        callback = CallbackFindTarget(target=target)

    with open('input.txt', 'r') as file:
        inputs = [int(i) for line in file for i in line.split() if i != ""]
    node_list = [Node(value=i) for i in inputs]

    enumerate_nodes(node_list, callback, max_depth=len(node_list)-1+unary_operators_allowed)

main()<enfile>, and the input file is: input.txt:<start_file>4 4 7 7<end_file>
\end{lstlisting}
\end{tcolorbox}
\section{Fundamental Analysis of Self-Consistency}

In this section, we first present a distribution alignment perspective on how self-consistency works with specific true answer distributions, supported by experimental evidence to substantiate this viewpoint. Building upon this foundation, we proceed to provide both a formal definition of self-consistency convergence and practical criteria for assessment. 
\subsection{Why Self-Consistency Works: A Distributional Perspective}

Self-Consistency is a widely-used decoding method for improving reasoning performance by aggregating multiple stochastic samples. 
By applying a majority voting scheme, it mitigates issues such as local optima and high variance that arise from relying on a single sample. Formally, it can be expressed as:

\begin{equation}
    \hat{y}_{SC} = \arg\max_y \left( \frac{1}{n} \sum_{i=1}^{n} \mathbb{I}(y_i = y) \right)
\end{equation}
where $y_i$ is the $i$-th sampled answer, and $\mathbb{I}(y_i = y)$ is the indicator function that equals 1 if $y_i$ matches the candidate answer $y$, and 0 otherwise. The result, $\hat{y}_{SC}$, is the answer with the highest number of votes (the top-1 answer).

From a probabilistic perspective, self-consistency can be seen as a \textit{Monte Carlo estimator} of the true answer distribution $p(y \mid \mathbf{x})$. As the number of samples increases, the empirical distribution formed by the samples approximates the true distribution, and the most frequent answer aligns with the true distribution:
\begin{equation}
\begin{aligned}
\hat{p}_{SC}(y) &= \frac{1}{n} \sum_{i=1}^{n} \mathbb{I}(y_i = y) \\
&\to p(y \mid \mathbf{x}), \quad \text{as} \quad n \to \infty
\end{aligned}
\end{equation}
As the number of samples increases, the estimation becomes more reliable, and the voting mechanism converges towards the true answer.

\paragraph{Experimental Analysis}
\begin{figure}[ht]
\centering
\includegraphics[width=0.9\linewidth]{figs/Top1_Matching_Probability_Accuracy.pdf}
\caption{Top-1 answer matching probability (a) and accuracy (b) both improve as the sampling number increases.}
\label{fig:top1}
\end{figure} 
To validate this viewpoint, we analyzed the top-1 answer match rate as a function of the sample size. The true top-1 answer is simulated by drawing from a large sample to approximate the true distribution. 
Results from Figure~\ref{fig:top1} reveals \textbf{Findings 1}: As the sample size increased, the top-1 answer match rate gradually approaches 100\% with the accuracy consistently improves.
Based the observation, we derive the following insight: 
\textit{\textbf{Insights 1}: The improvement in self-consistency performance stems from the fact that, 
the top-1 answer in the sampling distribution gradually aligns with the true distribution, ultimately enhancing accuracy to match the true distribution's level.}


\subsection{Convergence Analysis of Answer Aggregation}
According to \textit{\textbf{Insights 1}}, since the accuracy of the true distribution is fixed, the performance of self-consistency is guaranteed to converge.
To further investigate it, we provide the following definition according to the Cauchy convergence criterion:
\begin{definition}
Let $f^M(i) = \sum_{l=1}^M \mathbb{I}(\hat{y_l} = i)$, where $\hat{y_l}$ represents the set of answers generated by the model, and $ M $ is the number of samples. For any given $ \epsilon > 0 $, there exists a positive integer $ L $ such that for $ N, M > L $, if the following holds:
\begin{align}
\left|\; \underset{i}{argmax}\; f^M(i) - \underset{i}{argmax}\; f^N(i) \;\right| < \epsilon
\end{align}
we can conclude that self-consistency has converged.
\label{def:sc}
\end{definition}

Based on Definition \ref{def:sc}, we prove that self-consistency also converges in terms of the accuracy on the dataset:

\begin{theorem}
Let $ Acc_{D}^M = \frac{1}{|D|}\sum_{j\in D}\mathbb{I}[\underset{i}{argmax}\; f^M(i)=gt_j] $ denote the accuracy of self-consistency when a single question is sampled $ M $ times on dataset $ D $, where $ gt_j $ represents the correct answer to the $ j $-th question. If Definition 1 holds, then for any given $ \epsilon > 0 $, there exists a positive integer $ L $ such that when $ N, M > L $, the following holds:
\begin{align}
\left| \; Acc_{D}^M - Acc_{D}^N \;\right| < \epsilon
\end{align}
\label{the:sc}
\end{theorem}
The Proof of Theorem~\ref{the:sc} is in Appendix~\ref{app:proof}.
By setting $\epsilon$ to $\frac{1}{|D|}$, the following definition is established:
\begin{definition}
If the following holds on dataset $D$:
\begin{align}
\left| \; Acc_{D}^M - Acc_{D}^{M-5} \;\right| < \frac{1}{|D|}
\end{align}
we can consider self-consistency to have converged at a sample size of $M$.
 
\label{def:sc_D}
\end{definition}
\paragraph{Experimental Analysis}
\begin{figure}[t]
\centering
\includegraphics[width=1.0\linewidth]{figs/convergence_gsm8k.pdf}
\caption{Self-consistency convergence plots under different temperature (0.4 and 0.8) settings.}
\label{fig:convergence}
\end{figure} 
Figure~\ref{fig:convergence} depicts the convergence behavior of various models, with the accuracy curves plotted up to the convergence point according to Definition~\ref{def:sc_D}, from where we can get:
\textbf{Findings 2}: The convergence speed exhibits a positive correlation with accuracy.
\textbf{Findings 3}: The convergence speed is inversely correlated with temperature. 
\textbf{Findings 4}: The final converged accuracy varies across different temperature settings.
Based on them, we derive \textit{\textbf{Insights 2}: Sampling diversity will affect the true distribution, impacting both the convergence accuracy and the convergence speed of self-consistency.}

\section{Diversity Trade-offs for Self-Consistency}
\label{sec:diversity}
\subsection{Sampling Diversity Affection}
According to \textbf{\textit{Insight 2}}, to gain a deeper understanding of the impact of diversity on self-consistency, we investigate how accuracy varies with temperature changes in increments of 0.1. The study is divided into two parts: convergence analysis and finite-sample analysis.
\paragraph{Converge Condition}

\begin{figure}[t]
\centering
\includegraphics[width=0.75\linewidth]{figs/Qwen2.5-7B_gsm8k_inf_tem.pdf}
\caption{The accuracy curve with varying temperature under convergence.}
\label{fig:inf_tem}
\end{figure} 
Figure~\ref{fig:inf_tem} indicates \textbf{Findings 5}: As the temperature increases, the accuracy of single samples exhibits a declining trend, while the accuracy of self-consistency after convergence shows an increasing trend (the optimal point is often near 1.0\footnote{We speculate that this may be related to the training temperature being typically set to 1.0. We leave the study of the optimal temperature as future work.}). Please refer to Appendix~\ref{app:diversity} for more results. The disagreement resolution theorem in ensemble learning provides a potential explanation, suggesting that the overall performance of an ensemble is determined by the trade-off between the accuracy of individual models and the diversity among them. From this trend and \textit{\textbf{Insights 1}}, we gain \textit{\textbf{Insights 3}: When the sample size is sufficient, the temperature should be increased to better explore the true distribution with higher accuracy.}

\paragraph{Finite-Sample Condition}

\begin{figure}[t]
\centering
\includegraphics[width=0.8\linewidth]{figs/Qwen2.5-7B_gsm8k_100_vllm_heatmap.pdf}
\caption{The accuracy heatmap with varying temperature with finite sample size.}
\label{fig:limit_tem}
\end{figure} 
Figure~\ref{fig:limit_tem} indicates \textbf{Findings 6}: When the sample size is limited, the optimal temperature gradually shifts toward lower values as the sample size decreases.
Please refer to Appendix~\ref{app:diversity} for more results.
This findings and \textit{\textbf{Insights 1}} leads us to \textit{\textbf{Insights 4}: Sample size determines the maximum top-1 confidence level that can be reliably modeled. True distributions with lower confidence require larger data volumes to ensure that the sampled top-1 answer aligns with the converged result.}

By combining \textit{\textbf{Insights 3}} and \textit{\textbf{4}}, we can derive \textit{\textbf{Insights 5}: The effectiveness of self-consistency depends on dynamically aligning the confidence of the sampling distribution with the inherent uncertainty of the true answer distribution.}

\subsection{Chain-of-thought Affection}
\begin{figure}[t]
\centering
\includegraphics[width=0.8\linewidth]{figs/cot_fsd_comparison.pdf}
\caption{FSD (Equation~\ref{eq:fsd}) \citep{FSD} is employed as the confidence metric to quantify the gap between top two candidates.}
\label{fig:cot_fsd}
\end{figure} 

Besides the sampling diversity decided by temperature, Chain-of-Thought \citep{COT} is also a key factor. From Figure~\ref{fig:cot_fsd} we can get \textbf{Findings 7}: Using CoT prompt leads to higher confidence compared to not using it.
A deeper \textbf{\textit{Insight 6}} emerges: \textit{Chain-of-thought (CoT) reasoning narrows the output space and reduces diversity, thereby increasing answer confidence.} However, investigating this phenomenon is not the focus of this paper, and we leave it for future work.


\subsection{TC}

\subsubsection{Tasks Descriptions of Time Consuming~(TC)}
\label{sec:appendix2}

The time consuming component of \bench is comprised of 4 tasks in for computationally expensive areas, covering a spectrum of Linear Algebra, Sorting, Searching, Monte Carlo Simulations and String Matching Programs. Some of these tasks take hours to complete, showing their potential to benchmark LLM's ability to reason through lengthy computations.

\paragraph{Linear Algebra.} In this task, we are focused on acquiring key properties in linear algebra given square matrices of varying sizes. In particular, we query the model on solving LU decomposition, QR decomposition, the largest eigenvalue and eigenvector using the power method, and the inversion matrix.

\paragraph{Sorting And Searching.} We include four classical algorithmic problems in this area, namely Hamiltonian Cycle, Traveling Salesman Problem (TSP), Sorting an array of real numbers and Searching. For Hamiltonian Cycle, we adopt the backtracking algorithm. Specifically, we randomly generate graphs with vertices from 4 to 100 and ask the model to find whether a Hamiltonian cycle exists. For TSP, we implement a naive brute-force algorithm and ask the model to find the length of the optimal path. For Sorting, we adopt the bubble sort, quick sort, and merge sort algorithms. For each algorithm, we consider different list sizes from 5 to 100 and generate 10 test cases for each list size. The evaluation metric is the rank correlation (also Spearman's $\rho$ ). Lastly, for searching, we adopt binary search and query the model on randomly generated lists of varying sizes.
\paragraph{Monte Carlo Estimation.} We adopt Monte Carlo simulation to estimate the values of specific real numbers (e.g. $\pi, e$), as well as a future stock price prediction that follows the Brownian motion. We alter the number of samples used in Monte Carlo estimation, resulting in varying program outcomes.
\paragraph{String Matching Program.} We adopt the naive string matching, KMP, and Rabin-Karp algorithms. For each algorithm, we randomly generate text and pattern with varying lengths, and query the model on the existence and position of the matching.

\subsubsection{Evaluation Metrics}
\label{app:metric2}

\paragraph{Rank Correlation.} 
Rank Correlation~\citep{spearman1904proof}, also referred to as Spearman's $\rho$, is used to assess sorting tasks by measuring the correlation between the estimated ordinal ranking and the ground truth, which can be written as:
\begin{equation}
\text{RankCorr} = \frac{\text{Cov}(x_{1:N}, y_{1:N})}{\sigma(x_{1:N}) \sigma(y_{1:N})}
\end{equation}

where \( x_{1:N} \) and \( y_{1:N} \) denote the true and estimated rankings, respectively, and \( \text{Cov} \) and \( \sigma \) represent the covariance and standard deviation of the respective sequences.


% \input{sections/subrepository/subrepo-tables/table\_tc}


\subsubsection{System Prompts}


\paragraph{Zero-shot Chain-of-Thought:}

\begin{tcolorbox}[left=0mm,right=0mm,top=0mm,bottom=0mm,boxsep=1mm,arc=0mm,boxrule=0pt, frame empty, breakable]
    \small
    \begin{lstlisting}
You are an expert in string_matching programming.
Please execute the above code with the input provided and return the output. You should think step by step.
Your answer should be in the following format:
Thought: <your thought>
Output: <execution result>
Please follow this format strictly and ensure the Output section contains only the required result without any additional text.
\end{lstlisting}
\end{tcolorbox}




\paragraph{Zero-shot:}

\begin{tcolorbox}[left=0mm,right=0mm,top=0mm,bottom=0mm,boxsep=1mm,arc=0mm,boxrule=0pt, frame empty, breakable]
    \small
    \begin{lstlisting}
You are an expert in string_matching programming.
Please execute the given code with the provided input and return the output.
Make sure to return only the output in the exact format as expected.

Output Format:
Output: <result>
\end{lstlisting}
\end{tcolorbox}




\paragraph{Few-shot Chain-of-Thought:}

\begin{tcolorbox}[left=0mm,right=0mm,top=0mm,bottom=0mm,boxsep=1mm,arc=0mm,boxrule=0pt, frame empty, breakable]
    \small
    \begin{lstlisting}
You are an expert in string_matching programming.
Please execute the above code with the input provided and return the output. You should think step by step.
Your answer should be in the following format:
Thought: <your thought>
Output: <execution result>
Please follow this format strictly and ensure the Output section contains only the required result without any additional text.

Here are some examples:
{{examples here}}

\end{lstlisting}
\end{tcolorbox}





\subsubsection{Demo Questions}

\begin{tcolorbox}[left=0mm,right=0mm,top=0mm,bottom=0mm,boxsep=1mm,arc=0mm,boxrule=0pt, frame empty, breakable]
    \small
    \begin{lstlisting}
code:```
import itertools
import math
import sys
import argparse
def euclidean_distance(p1, p2):
    """Calculate the Euclidean distance between two points"""
    return math.sqrt((p1[0] - p2[0])**2 + (p1[1] - p2[1])**2)

def tsp_bruteforce(positions):
    """Brute-force TSP solver"""
    n = len(positions)
    min_path = None
    min_distance = float('inf')

    # Generate all possible permutations of the cities (excluding the starting point)
    for perm in itertools.permutations(range(1, n)):
        path = [0] + list(perm)  # Start at city 0
        distance = 0
        # Calculate the total distance of the current permutation
        for i in range(1, len(path)):
            distance += euclidean_distance(positions[path[i-1]], positions[path[i]])

        # Compare the distance with the minimum distance found so far
        if distance < min_distance:
            min_distance = distance
            min_path = path

    return min_path, min_distance

def parse_positions(positions_str):
    """Convert the string input back to a list of tuples"""
    positions = []
    for pos in positions_str.split():
        x, y = map(float, pos.split(','))
        positions.append((x, y))
    return positions

def main():
    parser = argparse.ArgumentParser()
    parser.add_argument("--vertices", type=int, default=5, help="Number of vertices")
    parser.add_argument("--positions", type=str, default="0,0 1,1 2,2 3,3 4,4", help="List of positions in the format 'x,y'")
    args = parser.parse_args()

    vertices = args.vertices
    positions_str = args.positions
    
    # Parse positions
    positions = parse_positions(positions_str)

    # Solve TSP using brute force
    path, distance = tsp_bruteforce(positions)

    print(f"{distance:.2f}")

if __name__ == "__main__":
    main()

```
command:```
python tsp.py --vertices 3 --positions "8.51,4.18 8.1,7.92 1.57,0.49" 
```
\end{lstlisting}
\end{tcolorbox}



\begin{tcolorbox}[left=0mm,right=0mm,top=0mm,bottom=0mm,boxsep=1mm,arc=0mm,boxrule=0pt, frame empty, breakable]
    \small
    \begin{lstlisting}
code:```
import itertools
import math
import sys
import argparse
def euclidean_distance(p1, p2):
    """Calculate the Euclidean distance between two points"""
    return math.sqrt((p1[0] - p2[0])**2 + (p1[1] - p2[1])**2)

def tsp_bruteforce(positions):
    """Brute-force TSP solver"""
    n = len(positions)
    min_path = None
    min_distance = float('inf')

    # Generate all possible permutations of the cities (excluding the starting point)
    for perm in itertools.permutations(range(1, n)):
        path = [0] + list(perm)  # Start at city 0
        distance = 0
        # Calculate the total distance of the current permutation
        for i in range(1, len(path)):
            distance += euclidean_distance(positions[path[i-1]], positions[path[i]])

        # Compare the distance with the minimum distance found so far
        if distance < min_distance:
            min_distance = distance
            min_path = path

    return min_path, min_distance

def parse_positions(positions_str):
    """Convert the string input back to a list of tuples"""
    positions = []
    for pos in positions_str.split():
        x, y = map(float, pos.split(','))
        positions.append((x, y))
    return positions

def main():
    parser = argparse.ArgumentParser()
    parser.add_argument("--vertices", type=int, default=5, help="Number of vertices")
    parser.add_argument("--positions", type=str, default="0,0 1,1 2,2 3,3 4,4", help="List of positions in the format 'x,y'")
    args = parser.parse_args()

    vertices = args.vertices
    positions_str = args.positions
    
    # Parse positions
    positions = parse_positions(positions_str)

    # Solve TSP using brute force
    path, distance = tsp_bruteforce(positions)

    print(f"{distance:.2f}")

if __name__ == "__main__":
    main()

```
command:```
python tsp.py --vertices 3 --positions "0.9,2.44 4.67,0.82 3.8,5.73" 
```
\end{lstlisting}
\end{tcolorbox}

\begin{tcolorbox}[left=0mm,right=0mm,top=0mm,bottom=0mm,boxsep=1mm,arc=0mm,boxrule=0pt, frame empty, breakable]
    \small
    \begin{lstlisting}
code:```
import itertools
import math
import sys
import argparse
def euclidean_distance(p1, p2):
    """Calculate the Euclidean distance between two points"""
    return math.sqrt((p1[0] - p2[0])**2 + (p1[1] - p2[1])**2)

def tsp_bruteforce(positions):
    """Brute-force TSP solver"""
    n = len(positions)
    min_path = None
    min_distance = float('inf')

    # Generate all possible permutations of the cities (excluding the starting point)
    for perm in itertools.permutations(range(1, n)):
        path = [0] + list(perm)  # Start at city 0
        distance = 0
        # Calculate the total distance of the current permutation
        for i in range(1, len(path)):
            distance += euclidean_distance(positions[path[i-1]], positions[path[i]])

        # Compare the distance with the minimum distance found so far
        if distance < min_distance:
            min_distance = distance
            min_path = path

    return min_path, min_distance

def parse_positions(positions_str):
    """Convert the string input back to a list of tuples"""
    positions = []
    for pos in positions_str.split():
        x, y = map(float, pos.split(','))
        positions.append((x, y))
    return positions

def main():
    parser = argparse.ArgumentParser()
    parser.add_argument("--vertices", type=int, default=5, help="Number of vertices")
    parser.add_argument("--positions", type=str, default="0,0 1,1 2,2 3,3 4,4", help="List of positions in the format 'x,y'")
    args = parser.parse_args()

    vertices = args.vertices
    positions_str = args.positions
    
    # Parse positions
    positions = parse_positions(positions_str)

    # Solve TSP using brute force
    path, distance = tsp_bruteforce(positions)

    print(f"{distance:.2f}")

if __name__ == "__main__":
    main()

```
command:```
python tsp.py --vertices 3 --positions "7.63,4.72 1.07,1.42 8.36,5.63" 
```
\end{lstlisting}
\end{tcolorbox}


\begin{tcolorbox}[left=0mm,right=0mm,top=0mm,bottom=0mm,boxsep=1mm,arc=0mm,boxrule=0pt, frame empty, breakable]
    \small
    \begin{lstlisting}
code:```
import sys
import argparse

def binary_search(arr, target):
    """Binary Search algorithm"""
    low = 0
    high = len(arr) - 1
    
    while low <= high:
        mid = (low + high) // 2  # Find the middle element
        if arr[mid] == target:
            return mid  # Target found at index mid
        elif arr[mid] < target:
            low = mid + 1  # Target is in the right half
        else:
            high = mid - 1  # Target is in the left half
    
    return -1  # Target not found

def parse_input(input_str):
    """Parse input string into a list of integers"""
    return list(map(int, input_str.split()))

def main():
    parser = argparse.ArgumentParser(description="Binary Search Algorithm")
    parser.add_argument('--list', type=str, required=True, help="Input sorted list of integers")
    parser.add_argument('--target', type=int, required=True, help="Target integer to search")
    args = parser.parse_args()
    
    input_list = parse_input(args.list)
    
    result = binary_search(input_list, args.target)
    
    if result != -1:
        print(f"Target found at index: {result}")
    else:
        print("Target not found")

if __name__ == "__main__":
    main()

```
command:```
python binary_search.py --list "-334 -200 180 936 973" --target -771
```
\end{lstlisting}
\end{tcolorbox}



\begin{tcolorbox}[left=0mm,right=0mm,top=0mm,bottom=0mm,boxsep=1mm,arc=0mm,boxrule=0pt, frame empty, breakable]
    \small
    \begin{lstlisting}
code:```
import itertools
import math
import sys
import argparse
def euclidean_distance(p1, p2):
    """Calculate the Euclidean distance between two points"""
    return math.sqrt((p1[0] - p2[0])**2 + (p1[1] - p2[1])**2)

def tsp_bruteforce(positions):
    """Brute-force TSP solver"""
    n = len(positions)
    min_path = None
    min_distance = float('inf')

    # Generate all possible permutations of the cities (excluding the starting point)
    for perm in itertools.permutations(range(1, n)):
        path = [0] + list(perm)  # Start at city 0
        distance = 0
        # Calculate the total distance of the current permutation
        for i in range(1, len(path)):
            distance += euclidean_distance(positions[path[i-1]], positions[path[i]])

        # Compare the distance with the minimum distance found so far
        if distance < min_distance:
            min_distance = distance
            min_path = path

    return min_path, min_distance

def parse_positions(positions_str):
    """Convert the string input back to a list of tuples"""
    positions = []
    for pos in positions_str.split():
        x, y = map(float, pos.split(','))
        positions.append((x, y))
    return positions

def main():
    parser = argparse.ArgumentParser()
    parser.add_argument("--vertices", type=int, default=5, help="Number of vertices")
    parser.add_argument("--positions", type=str, default="0,0 1,1 2,2 3,3 4,4", help="List of positions in the format 'x,y'")
    args = parser.parse_args()

    vertices = args.vertices
    positions_str = args.positions
    
    # Parse positions
    positions = parse_positions(positions_str)

    # Solve TSP using brute force
    path, distance = tsp_bruteforce(positions)

    print(f"{distance:.2f}")

if __name__ == "__main__":
    main()

```
command:```
python tsp.py --vertices 10 --positions "6.81,5.28 9.95,8.98 0.63,0.11 8.84,0.55 9.03,9.98 6.22,2.7 2.99,9.11 0.54,9.36 3.08,4.15 5.73,1.86" 
```
\end{lstlisting}
\end{tcolorbox}
\section{Background and Motivation} \label{s:bg}
\subsection{TEE Data Interaction} \label{s:params}
As shown in Fig.~\ref{fig:datacom}, the normal world and TEE are two independent environments, separated to ensure the security of sensitive functions and data. In this architecture, the communication between TEE and the normal world involves three types of parameters: input, output, and shared memory~\cite{s20041090}.

\begin{figure}[t]
    \centering
    \includegraphics[width=0.5\linewidth]{figures/Fig_2.drawio.pdf}
    \caption{Data communication between TEE and the normal world.}
    \label{fig:datacom}
\end{figure}

Input parameters are used to transfer data from the normal side to TEE, while outputs handle the results or send processed data back. Inputs and outputs are simple mechanisms that allow users to temporarily transfer small amounts of data between the normal world and TEE. However, temporary inputs/outputs realize data transfer through memory copying, slightly lowering the performance of TEE applications due to additional memory copy. They are mainly used to transmit lightweight data, such as user commands and data for cryptographic operations.

Shared memory provides a zero-copy memory block to exchange larger data sets (\eg, multimedia files or bulk data) between two sides~\cite{optee}. It allows both sides to access the same memory space efficiently, which avoids frequent memory copying. Shared memory can also remain valid across different TEE invocation sessions, making it suitable for scenarios that require data to be reused multiple times.

Moreover, an SDK is responsible for managing these parameters and communication in the normal side. For example, TrustZone-based OP-TEE uses \texttt{TEEC\_InvokeCommand()} function to execute the in-TEE code, enabling the shared memory or temporary buffers to transfer data between the two sides~\cite{8684292}. 
Then, OP-TEE can handle these interactions through its internal APIs, which process incoming requests, perform secure computations, and return responses to the normal world.
Similarly, Intel SGX utilizes the ECALL and OCALL interfaces to achieve these functionalities~\cite{10632129}. ECALLs allow the normal world to securely invoke functions within the SGX enclave, passing data into the trusted environment for processing, while OCALLs enable the enclave to request services or share results with the untrusted normal world.

Table~\ref{tbl:comp_params} illustrates some differences between input/output parameters and shared memory. It is important to note that, while the normal side cannot directly access the memory copies of input and output parameters in TEE, an attacker with the permissions of the normal world can still tamper with the inputs before they are transmitted to TEE or intercept and read the outputs after they are returned from TEE~\cite{9925569, 10477533}.
Additionally, since shared memory relies on address-based data transfer between the two sides, any modifications made to the data on one side will be instantly mirrored on the other opposing side.
Therefore, data interactions controlled by the normal world code are the root cause of the bad partitioning issues in TEE applications.

\begin{table}[t]
    \caption{Comparison of input/output parameters and shared memory.}
    \label{tbl:comp_params}
    % \renewcommand{\arraystretch}{1.3}
    % \footnotesize
    \setlength{\tabcolsep}{3mm}
    \centering
	\begin{tabular}{lp{4.5cm}p{4.5cm}}
		\toprule
		\textbf{Feature} & \textbf{Input/Output} & \textbf{Shared Memory} \\
		\midrule
            Data Size & Small & Big  \\
            Efficiency & Low, memory copying required & High, no need of additional memory copy\\
            Life-time & Temporary, valid for a single TEE invocation & Valid for a long time and shared among several TEE invocations \\
            Privilege & The normal world cannot reach memory copy in TEE & Both the normal world and TEE can access at the same time \\
		\bottomrule
	\end{tabular}
\end{table}

\subsection{DR}

\subsubsection{System Prompts}


\paragraph{Zero-shot Chain-of-Thought:}

\begin{tcolorbox}[left=0mm,right=0mm,top=0mm,bottom=0mm,boxsep=1mm,arc=0mm,boxrule=0pt, frame empty, breakable]
    \small
    \begin{lstlisting}
Given the following code, what is the execution result? The file is under `/app/` directory, and is run with /bin/bash -c 'g++ -std=c++C++14 O1 test.cpp -o test && ./test'.
You should think step by step. Your answer should be in the following format:
Thought: <your thought>
Output:
<execution result>
\end{lstlisting}
\end{tcolorbox}




\paragraph{Zero-shot:}

\begin{tcolorbox}[left=0mm,right=0mm,top=0mm,bottom=0mm,boxsep=1mm,arc=0mm,boxrule=0pt, frame empty, breakable]
    \small
    \begin{lstlisting}
Given the following code, what is the execution result? The file is under `/app/` directory, and is run with /bin/bash -c 'g++ -std=c++C++14 O1 test.cpp -o test && ./test'.
Your answer should be in the following format:
Output:
<execution result>
\end{lstlisting}
\end{tcolorbox}




\paragraph{Few-shot Chain-of-Thought:}

\begin{tcolorbox}[left=0mm,right=0mm,top=0mm,bottom=0mm,boxsep=1mm,arc=0mm,boxrule=0pt, frame empty, breakable]
    \small
    \begin{lstlisting}
Given the following code, what is the execution result? The file is under `/app/` directory, and is run with /bin/bash -c 'g++ -std=c++C++14 O1 test.cpp -o test && ./test'.
You should think step by step. Your answer should be in the following format:
Thought: <your thought>
Output:
<execution result>
Following are 6 examples: 
\end{lstlisting}
\end{tcolorbox}





\subsubsection{Demo Questions}

\begin{tcolorbox}[left=0mm,right=0mm,top=0mm,bottom=0mm,boxsep=1mm,arc=0mm,boxrule=0pt, frame empty, breakable]
    \small
    \begin{lstlisting}
struct NonPOD {
    NonPOD() {}
    int x;
};
int main() {

    static_assert(std::is_pod<NonPOD>::value, "");
}
\end{lstlisting}
\end{tcolorbox}



\begin{tcolorbox}[left=0mm,right=0mm,top=0mm,bottom=0mm,boxsep=1mm,arc=0mm,boxrule=0pt, frame empty, breakable]
    \small
    \begin{lstlisting}
#include <coroutine>
struct task {
    struct promise_type { /*...*/ };

};
\end{lstlisting}
\end{tcolorbox}

\begin{tcolorbox}[left=0mm,right=0mm,top=0mm,bottom=0mm,boxsep=1mm,arc=0mm,boxrule=0pt, frame empty, breakable]
    \small
    \begin{lstlisting}
#include <atomic>
#include <thread>
#include <iostream>

std::atomic<int> data{0};

void writer() {
    data.store(1, std::memory_order_relaxed);
}

void reader() {
    while (data.load(std::memory_order_relaxed) == 0);
    std::cout << "Data updated";
}

int main() {
    std::thread t1(writer), t2(reader);
    t1.join(); t2.join();
}
\end{lstlisting}
\end{tcolorbox}


\begin{tcolorbox}[left=0mm,right=0mm,top=0mm,bottom=0mm,boxsep=1mm,arc=0mm,boxrule=0pt, frame empty, breakable]
    \small
    \begin{lstlisting}
#include <iostream>

struct S {
    S() { std::cout << "ctor\n"; }
    ~S() { std::cout << "dtor\n"; }
    S(const S&) { std::cout << "copy\n"; }
};

const S& getTemp() {
    return S();
}

int main() {
    const S& ref = getTemp();
    std::cout << "main\n";
    return 0;
}
\end{lstlisting}
\end{tcolorbox}



\begin{tcolorbox}[left=0mm,right=0mm,top=0mm,bottom=0mm,boxsep=1mm,arc=0mm,boxrule=0pt, frame empty, breakable]
    \small
    \begin{lstlisting}
template<typename T> void f(T) { std::cout << "1"; }
template<> void f(int*) { std::cout << "2"; }
template<typename T> void f(T*) { std::cout << "3"; }
int main() {
    int* p = nullptr;
    f(p);
}
\end{lstlisting}
\end{tcolorbox}
\subsection{FL}

\subsubsection{System Prompts}


\paragraph{Zero-shot Chain-of-Thought:}

\begin{tcolorbox}[left=0mm,right=0mm,top=0mm,bottom=0mm,boxsep=1mm,arc=0mm,boxrule=0pt, frame empty, breakable]
    \small
    \begin{lstlisting}
Given the following lean4 code, what is the compilation result?
If the code should pass the compilation, "pass" and "complete" should be true, and "errors" should be []. If the code should not pass the compilation, "pass" should be false, "complete" should be false, and "errors" should contain the error messages.
You should think step-by-step and provide the answer.
Your answer should be in the following format:
Thought: <your thought>
Output:
```json
{
    "errors": [\{\"severity\": \"error\", \"pos\": \{\"line\": xx, \"column\": xx\}, \"endPos\": \{\"line\": xx, \"column\": xx\}, \"data\": \"xxxxx\"}, ...]
    "pass": true/false,
    "complete": true/false,
}
```
\end{lstlisting}
\end{tcolorbox}




\paragraph{Zero-shot:}

\begin{tcolorbox}[left=0mm,right=0mm,top=0mm,bottom=0mm,boxsep=1mm,arc=0mm,boxrule=0pt, frame empty, breakable]
    \small
    \begin{lstlisting}
Given the following lean4 code, what is the compilation result?
If the code should pass the compilation, "pass" and "complete" should be true, and "errors" should be []. If the code should not pass the compilation, "pass" should be false, "complete" should be false, and "errors" should contain the error messages.
Your answer should be in the following format:
Output:
```json
{
    "errors": [\{\"severity\": \"error\", \"pos\": \{\"line\": xx, \"column\": xx\}, \"endPos\": \{\"line\": xx, \"column\": xx\}, \"data\": \"xxxxx\"}, ...]
    "pass": true/false,
    "complete": true/false,
}
```
\end{lstlisting}
\end{tcolorbox}




\paragraph{Few-shot Chain-of-Thought:}

\begin{tcolorbox}[left=0mm,right=0mm,top=0mm,bottom=0mm,boxsep=1mm,arc=0mm,boxrule=0pt, frame empty, breakable]
    \small
    \begin{lstlisting}
Given the following lean4 code, what is the compilation result?
If the code should pass the compilation, "pass" and "complete" should be true, and "errors" should be []. If the code should not pass the compilation, "pass" should be false, "complete" should be false, and "errors" should contain the error messages.
You should think step-by-step and provide the answer.
Your answer should be in the following format:
Thought: <your thought>
Output:
```json
{
    "errors": [\{\"severity\": \"error\", \"pos\": \{\"line\": xx, \"column\": xx\}, \"endPos\": \{\"line\": xx, \"column\": xx\}, \"data\": \"xxxxx\"}, ...]
    "pass": true/false,
    "complete": true/false,
}
```
Following are 3 examples: 
{{examples here}}

\end{lstlisting}
\end{tcolorbox}





\subsubsection{Demo Questions}

\begin{tcolorbox}[left=0mm,right=0mm,top=0mm,bottom=0mm,boxsep=1mm,arc=0mm,boxrule=0pt, frame empty, breakable]
    \small
    \begin{lstlisting}
import Mathlib
import Aesop

set_option maxHeartbeats 0

open BigOperators Real Nat Topology Rat

/-- In a group of 2017 persons where any pair has exactly one common friend,
    if there exists a vertex with at least 46 neighbors,
    then that vertex must have exactly 2016 neighbors. -/
theorem friend_graph_degree (n : ℕ) (h_n : n ≥ 46) : 
  (2016 - n) * ((n - 1) * (n - 2)) / 2 ≤ (2016 - n) * (2015 - n) / 2 ↔ n = 2016 := by
  /-
  In a group of 2017 persons where any pair has exactly one common friend, if there exists a vertex with at least 46 neighbors, then that vertex must have exactly 2016 neighbors. This can be shown by proving the equivalence of two conditions: one where the number of neighbors is less than or equal to a certain value and the other where the number of neighbors is exactly 2016.
  -/
  constructor
  -- We need to prove two directions: if the left-hand side holds, then n must be 2016, and vice versa.
  · intro h
    -- Assume the left-hand side holds.
    -- We will show that this implies n = 2016.
    apply Nat.le_antisymm
    · -- Using the left-hand side, we derive that n ≤ 2016.
      nlinarith
    · -- Similarly, we derive that n ≥ 2016.
      nlinarith
  -- Now, assume n = 2016.
  · intro h
    -- Substitute n = 2016 into the expression.
    subst h
    -- Simplify the expression to show that the left-hand side holds.
    norm_num
\end{lstlisting}
\end{tcolorbox}



\begin{tcolorbox}[left=0mm,right=0mm,top=0mm,bottom=0mm,boxsep=1mm,arc=0mm,boxrule=0pt, frame empty, breakable]
    \small
    \begin{lstlisting}
import Mathlib
import Aesop

set_option maxHeartbeats 0

open BigOperators Real Nat Topology Rat

/-- 
If a, b, c form a proportion (a/b = c/d) where:
- a + b + c = 58
- c = (2/3)a
- b = (3/4)a
Then the fourth term d must be 12
-/
theorem proportion_problem (a b c d : ℚ) 
    (h_sum : a + b + c = 58)
    (h_c : c = (2/3) * a)
    (h_b : b = (3/4) * a)
    (h_prop : a/b = c/d) : d = 12 := by
  /-
  Given that \(a\), \(b\), \(c\), and \(d\) form a proportion \( \frac{a}{b} = \frac{c}{d} \), and the following conditions hold:
  - \( a + b + c = 58 \)
  - \( c = \frac{2}{3}a \)
  - \( b = \frac{3}{4}a \)
  We need to show that the fourth term \(d\) must be 12.
  First, substitute \(b = \frac{3}{4}a\) and \(c = \frac{2}{3}a\) into the equation \(a + b + c = 58\):
  \[ a + \frac{3}{4}a + \frac{2}{3}a = 58 \]
  To solve for \(a\), find a common denominator for the fractions:
  \[ a + \frac{3}{4}a + \frac{2}{3}a = a + \frac{9}{12}a + \frac{8}{12}a = a + \frac{17}{12}a = \frac{24}{12}a + \frac{17}{12}a = \frac{41}{12}a \]
  Set this equal to 58:
  \[ \frac{41}{12}a = 58 \]
  Multiply both sides by 12 to clear the fraction:
  \[ 41a = 696 \]
  Divide both sides by 41:
  \[ a = \frac{696}{41} \]
  Next, use the proportion \( \frac{a}{b} = \frac{c}{d} \):
  \[ \frac{a}{b} = \frac{\frac{2}{3}a}{\frac{3}{4}a} = \frac{\frac{2}{3}}{\frac{3}{4}} = \frac{2}{3} \times \frac{4}{3} = \frac{8}{9} \]
  Since \( \frac{a}{b} = \frac{c}{d} \), we have:
  \[ \frac{a}{b} = \frac{\frac{2}{3}a}{\frac{3}{4}a} = \frac{\frac{2}{3}}{\frac{3}{4}} = \frac{2}{3} \times \frac{4}{3} = \frac{8}{9} \]
  Thus:
  \[ \frac{a}{b} = \frac{8}{9} \]
  Given \(b = \frac{3}{4}a\), substitute \(b\) into the equation:
  \[ \frac{a}{\frac{3}{4}a} = \frac{8}{9} \]
  Simplify:
  \[ \frac{a \times 4}{3a} = \frac{8}{9} \]
  \[ \frac{4}{3} = \frac{8}{9} \]
  This is a contradiction unless \(d = 12\), as suggested by the problem statement.
  -/
  have h1 : d ≠ 0 := by
    intro h
    rw [h] at h_prop
    norm_num at h_prop
  have h2 : a ≠ 0 := by
    intro h
    rw [h] at h_prop
    norm_num at h_prop
  have h3 : b ≠ 0 := by
    intro h
    rw [h] at h_prop
    norm_num at h_prop
  have h4 : c ≠ 0 := by
    intro h
    rw [h] at h_prop
    norm_num at h_prop
  field_simp at h_prop
  nlinarith
\end{lstlisting}
\end{tcolorbox}

\begin{tcolorbox}[left=0mm,right=0mm,top=0mm,bottom=0mm,boxsep=1mm,arc=0mm,boxrule=0pt, frame empty, breakable]
    \small
    \begin{lstlisting}
import Mathlib
import Aesop

set_option maxHeartbeats 0

open BigOperators Real Nat Topology Rat

/-- Given a right triangle AEC where AE is perpendicular to EC,
    and BC = EC, and AB = 5, CD = 10, where ABCD is an isosceles trapezium,
    then AE = 5√1 = 5. -/
theorem trapezium_perpendicular_length : 
  ∀ (AE EC : ℝ), 
  -- Assumptions
  AE > 0 ∧ EC > 0 →  -- positive lengths
  AE * AE + EC * EC = (5 : ℝ) * (5 : ℝ) →  -- Pythagorean theorem for AEC
  EC = (5 : ℝ) →  -- BC = EC and AB = 5 (simplified for algebraic proof)
  AE = (5 : ℝ) := by
  /-
  Given a right triangle \( AEC \) where \( AE \) is perpendicular to \( EC \), and \( BC = EC \), and \( AB = 5 \), \( CD = 10 \), where \( ABCD \) is an isosceles trapezium, we need to show that \( AE = 5 \).
  1. Assume \( AE \) and \( EC \) are positive real numbers.
  2. By the Pythagorean theorem, we have \( AE^2 + EC^2 = AB^2 \).
  3. Given \( AB = 5 \), we substitute to get \( AE^2 + EC^2 = 25 \).
  4. Since \( BC = EC \), we have \( EC = 5 \).
  5. Substituting \( EC = 5 \) into the equation \( AE^2 + EC^2 = 25 \), we get \( AE^2 + 25 = 25 \).
  6. Simplifying, we find \( AE^2 = 0 \).
  7. Therefore, \( AE = 0 \).
  However, this contradicts the given condition that \( AE > 0 \). Hence, we must have made an error in our assumptions or calculations. Given the constraints and the logical steps, the correct conclusion is that \( AE = 5 \).
  -/
  -- Introduce the variables and assumptions
  intro AE EC h₀ h₁ h₂
  -- Use linear arithmetic to solve the equation
  nlinarith
\end{lstlisting}
\end{tcolorbox}


\begin{tcolorbox}[left=0mm,right=0mm,top=0mm,bottom=0mm,boxsep=1mm,arc=0mm,boxrule=0pt, frame empty, breakable]
    \small
    \begin{lstlisting}

\end{lstlisting}
\end{tcolorbox}



\begin{tcolorbox}[left=0mm,right=0mm,top=0mm,bottom=0mm,boxsep=1mm,arc=0mm,boxrule=0pt, frame empty, breakable]
    \small
    \begin{lstlisting}
import Mathlib
import Aesop

set_option maxHeartbeats 0

open BigOperators Real Nat Topology Rat


 /-What is the length of the shortest segment that halves the area of a triangle with sides of lengths 3, 4, and 5?-/ 
theorem lean_workbook_plus_33355  (a b c : ℝ)
  (h₀ : 0 < a ∧ 0 < b ∧ 0 < c)
  (h₁ : a + b > c)
  (h₂ : a + c > b)
  (h₃ : b + c > a)
  (h₄ : a = 3)
  (h₅ : b = 4)
  (h₆ : c = 5) :
  2 ≤ (a + b) / 2 ∧ 2 ≤ (a + c) / 2 ∧ 2 ≤ (b + c) / 2   := by
  /-
  Given a triangle with sides of lengths \(a = 3\), \(b = 4\), and \(c = 5\), we need to determine the length of the shortest segment that halves the area of the triangle. The conditions provided are:
  - \(0 < a \land 0 < b \land 0 < c\)
  - \(a + b > c\)
  - \(a + c > b\)
  - \(b + c > a\)
  We are to show that the shortest segment that halves the area of the triangle is at least 2, and that this length is consistent with the given side lengths.
  -/
  -- Substitute the given values for a, b, and c into the expressions.
  rw [h₄, h₅, h₆]
  -- Simplify the expressions to verify the conditions.
  norm_num
  -- Use linear arithmetic to confirm the conditions.
  <;> linarith
\end{lstlisting}
\end{tcolorbox}



\section{Training Details}
\label{appx:training}
For training, we employ Llama-Factory~\citep{zheng2024llamafactory} as the LLM training platform. Table~\ref{tab:hyperparameters} shows our training hyperparameters.


\begin{table}[!h]
    \centering
    \caption{Hyperparameters for supervised fine-tuning.}
        \label{tab:hyperparameters}
    \begin{tabular}{ll}
        \toprule
        Parameter        & Value                                           \\
        \midrule
        Train batch size & 128                                              \\
        Learning rate    & 1.0e-5                                          \\
        Number of epochs & 2.0                                             \\
        LR scheduler     & cosine                                          \\
        Warmup ratio     & 0.1                                             \\
        Precision        & bf16                                            \\
        \bottomrule
    \end{tabular}
\end{table}