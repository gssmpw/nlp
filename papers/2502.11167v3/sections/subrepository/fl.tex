\subsection{FL}

\subsubsection{System Prompts}


\paragraph{Zero-shot Chain-of-Thought:}

\begin{tcolorbox}[left=0mm,right=0mm,top=0mm,bottom=0mm,boxsep=1mm,arc=0mm,boxrule=0pt, frame empty, breakable]
    \small
    \begin{lstlisting}
Given the following lean4 code, what is the compilation result?
If the code should pass the compilation, "pass" and "complete" should be true, and "errors" should be []. If the code should not pass the compilation, "pass" should be false, "complete" should be false, and "errors" should contain the error messages.
You should think step-by-step and provide the answer.
Your answer should be in the following format:
Thought: <your thought>
Output:
```json
{
    "errors": [\{\"severity\": \"error\", \"pos\": \{\"line\": xx, \"column\": xx\}, \"endPos\": \{\"line\": xx, \"column\": xx\}, \"data\": \"xxxxx\"}, ...]
    "pass": true/false,
    "complete": true/false,
}
```
\end{lstlisting}
\end{tcolorbox}




\paragraph{Zero-shot:}

\begin{tcolorbox}[left=0mm,right=0mm,top=0mm,bottom=0mm,boxsep=1mm,arc=0mm,boxrule=0pt, frame empty, breakable]
    \small
    \begin{lstlisting}
Given the following lean4 code, what is the compilation result?
If the code should pass the compilation, "pass" and "complete" should be true, and "errors" should be []. If the code should not pass the compilation, "pass" should be false, "complete" should be false, and "errors" should contain the error messages.
Your answer should be in the following format:
Output:
```json
{
    "errors": [\{\"severity\": \"error\", \"pos\": \{\"line\": xx, \"column\": xx\}, \"endPos\": \{\"line\": xx, \"column\": xx\}, \"data\": \"xxxxx\"}, ...]
    "pass": true/false,
    "complete": true/false,
}
```
\end{lstlisting}
\end{tcolorbox}




\paragraph{Few-shot Chain-of-Thought:}

\begin{tcolorbox}[left=0mm,right=0mm,top=0mm,bottom=0mm,boxsep=1mm,arc=0mm,boxrule=0pt, frame empty, breakable]
    \small
    \begin{lstlisting}
Given the following lean4 code, what is the compilation result?
If the code should pass the compilation, "pass" and "complete" should be true, and "errors" should be []. If the code should not pass the compilation, "pass" should be false, "complete" should be false, and "errors" should contain the error messages.
You should think step-by-step and provide the answer.
Your answer should be in the following format:
Thought: <your thought>
Output:
```json
{
    "errors": [\{\"severity\": \"error\", \"pos\": \{\"line\": xx, \"column\": xx\}, \"endPos\": \{\"line\": xx, \"column\": xx\}, \"data\": \"xxxxx\"}, ...]
    "pass": true/false,
    "complete": true/false,
}
```
Following are 3 examples: 
{{examples here}}

\end{lstlisting}
\end{tcolorbox}





\subsubsection{Demo Questions}

\begin{tcolorbox}[left=0mm,right=0mm,top=0mm,bottom=0mm,boxsep=1mm,arc=0mm,boxrule=0pt, frame empty, breakable]
    \small
    \begin{lstlisting}
import Mathlib
import Aesop

set_option maxHeartbeats 0

open BigOperators Real Nat Topology Rat

/-- In a group of 2017 persons where any pair has exactly one common friend,
    if there exists a vertex with at least 46 neighbors,
    then that vertex must have exactly 2016 neighbors. -/
theorem friend_graph_degree (n : ℕ) (h_n : n ≥ 46) : 
  (2016 - n) * ((n - 1) * (n - 2)) / 2 ≤ (2016 - n) * (2015 - n) / 2 ↔ n = 2016 := by
  /-
  In a group of 2017 persons where any pair has exactly one common friend, if there exists a vertex with at least 46 neighbors, then that vertex must have exactly 2016 neighbors. This can be shown by proving the equivalence of two conditions: one where the number of neighbors is less than or equal to a certain value and the other where the number of neighbors is exactly 2016.
  -/
  constructor
  -- We need to prove two directions: if the left-hand side holds, then n must be 2016, and vice versa.
  · intro h
    -- Assume the left-hand side holds.
    -- We will show that this implies n = 2016.
    apply Nat.le_antisymm
    · -- Using the left-hand side, we derive that n ≤ 2016.
      nlinarith
    · -- Similarly, we derive that n ≥ 2016.
      nlinarith
  -- Now, assume n = 2016.
  · intro h
    -- Substitute n = 2016 into the expression.
    subst h
    -- Simplify the expression to show that the left-hand side holds.
    norm_num
\end{lstlisting}
\end{tcolorbox}



\begin{tcolorbox}[left=0mm,right=0mm,top=0mm,bottom=0mm,boxsep=1mm,arc=0mm,boxrule=0pt, frame empty, breakable]
    \small
    \begin{lstlisting}
import Mathlib
import Aesop

set_option maxHeartbeats 0

open BigOperators Real Nat Topology Rat

/-- 
If a, b, c form a proportion (a/b = c/d) where:
- a + b + c = 58
- c = (2/3)a
- b = (3/4)a
Then the fourth term d must be 12
-/
theorem proportion_problem (a b c d : ℚ) 
    (h_sum : a + b + c = 58)
    (h_c : c = (2/3) * a)
    (h_b : b = (3/4) * a)
    (h_prop : a/b = c/d) : d = 12 := by
  /-
  Given that \(a\), \(b\), \(c\), and \(d\) form a proportion \( \frac{a}{b} = \frac{c}{d} \), and the following conditions hold:
  - \( a + b + c = 58 \)
  - \( c = \frac{2}{3}a \)
  - \( b = \frac{3}{4}a \)
  We need to show that the fourth term \(d\) must be 12.
  First, substitute \(b = \frac{3}{4}a\) and \(c = \frac{2}{3}a\) into the equation \(a + b + c = 58\):
  \[ a + \frac{3}{4}a + \frac{2}{3}a = 58 \]
  To solve for \(a\), find a common denominator for the fractions:
  \[ a + \frac{3}{4}a + \frac{2}{3}a = a + \frac{9}{12}a + \frac{8}{12}a = a + \frac{17}{12}a = \frac{24}{12}a + \frac{17}{12}a = \frac{41}{12}a \]
  Set this equal to 58:
  \[ \frac{41}{12}a = 58 \]
  Multiply both sides by 12 to clear the fraction:
  \[ 41a = 696 \]
  Divide both sides by 41:
  \[ a = \frac{696}{41} \]
  Next, use the proportion \( \frac{a}{b} = \frac{c}{d} \):
  \[ \frac{a}{b} = \frac{\frac{2}{3}a}{\frac{3}{4}a} = \frac{\frac{2}{3}}{\frac{3}{4}} = \frac{2}{3} \times \frac{4}{3} = \frac{8}{9} \]
  Since \( \frac{a}{b} = \frac{c}{d} \), we have:
  \[ \frac{a}{b} = \frac{\frac{2}{3}a}{\frac{3}{4}a} = \frac{\frac{2}{3}}{\frac{3}{4}} = \frac{2}{3} \times \frac{4}{3} = \frac{8}{9} \]
  Thus:
  \[ \frac{a}{b} = \frac{8}{9} \]
  Given \(b = \frac{3}{4}a\), substitute \(b\) into the equation:
  \[ \frac{a}{\frac{3}{4}a} = \frac{8}{9} \]
  Simplify:
  \[ \frac{a \times 4}{3a} = \frac{8}{9} \]
  \[ \frac{4}{3} = \frac{8}{9} \]
  This is a contradiction unless \(d = 12\), as suggested by the problem statement.
  -/
  have h1 : d ≠ 0 := by
    intro h
    rw [h] at h_prop
    norm_num at h_prop
  have h2 : a ≠ 0 := by
    intro h
    rw [h] at h_prop
    norm_num at h_prop
  have h3 : b ≠ 0 := by
    intro h
    rw [h] at h_prop
    norm_num at h_prop
  have h4 : c ≠ 0 := by
    intro h
    rw [h] at h_prop
    norm_num at h_prop
  field_simp at h_prop
  nlinarith
\end{lstlisting}
\end{tcolorbox}

\begin{tcolorbox}[left=0mm,right=0mm,top=0mm,bottom=0mm,boxsep=1mm,arc=0mm,boxrule=0pt, frame empty, breakable]
    \small
    \begin{lstlisting}
import Mathlib
import Aesop

set_option maxHeartbeats 0

open BigOperators Real Nat Topology Rat

/-- Given a right triangle AEC where AE is perpendicular to EC,
    and BC = EC, and AB = 5, CD = 10, where ABCD is an isosceles trapezium,
    then AE = 5√1 = 5. -/
theorem trapezium_perpendicular_length : 
  ∀ (AE EC : ℝ), 
  -- Assumptions
  AE > 0 ∧ EC > 0 →  -- positive lengths
  AE * AE + EC * EC = (5 : ℝ) * (5 : ℝ) →  -- Pythagorean theorem for AEC
  EC = (5 : ℝ) →  -- BC = EC and AB = 5 (simplified for algebraic proof)
  AE = (5 : ℝ) := by
  /-
  Given a right triangle \( AEC \) where \( AE \) is perpendicular to \( EC \), and \( BC = EC \), and \( AB = 5 \), \( CD = 10 \), where \( ABCD \) is an isosceles trapezium, we need to show that \( AE = 5 \).
  1. Assume \( AE \) and \( EC \) are positive real numbers.
  2. By the Pythagorean theorem, we have \( AE^2 + EC^2 = AB^2 \).
  3. Given \( AB = 5 \), we substitute to get \( AE^2 + EC^2 = 25 \).
  4. Since \( BC = EC \), we have \( EC = 5 \).
  5. Substituting \( EC = 5 \) into the equation \( AE^2 + EC^2 = 25 \), we get \( AE^2 + 25 = 25 \).
  6. Simplifying, we find \( AE^2 = 0 \).
  7. Therefore, \( AE = 0 \).
  However, this contradicts the given condition that \( AE > 0 \). Hence, we must have made an error in our assumptions or calculations. Given the constraints and the logical steps, the correct conclusion is that \( AE = 5 \).
  -/
  -- Introduce the variables and assumptions
  intro AE EC h₀ h₁ h₂
  -- Use linear arithmetic to solve the equation
  nlinarith
\end{lstlisting}
\end{tcolorbox}


\begin{tcolorbox}[left=0mm,right=0mm,top=0mm,bottom=0mm,boxsep=1mm,arc=0mm,boxrule=0pt, frame empty, breakable]
    \small
    \begin{lstlisting}

\end{lstlisting}
\end{tcolorbox}



\begin{tcolorbox}[left=0mm,right=0mm,top=0mm,bottom=0mm,boxsep=1mm,arc=0mm,boxrule=0pt, frame empty, breakable]
    \small
    \begin{lstlisting}
import Mathlib
import Aesop

set_option maxHeartbeats 0

open BigOperators Real Nat Topology Rat


 /-What is the length of the shortest segment that halves the area of a triangle with sides of lengths 3, 4, and 5?-/ 
theorem lean_workbook_plus_33355  (a b c : ℝ)
  (h₀ : 0 < a ∧ 0 < b ∧ 0 < c)
  (h₁ : a + b > c)
  (h₂ : a + c > b)
  (h₃ : b + c > a)
  (h₄ : a = 3)
  (h₅ : b = 4)
  (h₆ : c = 5) :
  2 ≤ (a + b) / 2 ∧ 2 ≤ (a + c) / 2 ∧ 2 ≤ (b + c) / 2   := by
  /-
  Given a triangle with sides of lengths \(a = 3\), \(b = 4\), and \(c = 5\), we need to determine the length of the shortest segment that halves the area of the triangle. The conditions provided are:
  - \(0 < a \land 0 < b \land 0 < c\)
  - \(a + b > c\)
  - \(a + c > b\)
  - \(b + c > a\)
  We are to show that the shortest segment that halves the area of the triangle is at least 2, and that this length is consistent with the given side lengths.
  -/
  -- Substitute the given values for a, b, and c into the expressions.
  rw [h₄, h₅, h₆]
  -- Simplify the expressions to verify the conditions.
  norm_num
  -- Use linear arithmetic to confirm the conditions.
  <;> linarith
\end{lstlisting}
\end{tcolorbox}