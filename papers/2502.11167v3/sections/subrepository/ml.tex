\subsection{ML}


\begin{table}[!ht]
  \centering
  \caption{Language usage count across different categories in the ML subset.}
  \label{tab:ml_language_usage}
  \begin{tabular}{lccccccc}
      \toprule
      Java & C\# & Rust & Julia & Python & C++ & C \\
      \midrule
      25   & 20  & 20   & 26    & 18     & 21  & 20 \\
      \bottomrule
  \end{tabular}
\end{table}

In ML, the usage distribution of various programming languages is shown in Table \ref{tab:ml_language_usage}. We selected a variety of languages, including Java, C\#, Rust, Julia, Python, C++, and C, to evaluate the model's ability to handle multilingual code. This diverse selection helps to comprehensively assess the model's performance across different languages.

\subsubsection{System Prompts}


\paragraph{Zero-shot Chain-of-Thought:}

\begin{tcolorbox}[left=0mm,right=0mm,top=0mm,bottom=0mm,boxsep=1mm,arc=0mm,boxrule=0pt, frame empty, breakable]
    \small
    \begin{lstlisting}
Given the following code, what is the execution result?
You should think step by step.  Your answer should be in the following format:
Thought: <your thought>
Output:
<execution result>
\end{lstlisting}
\end{tcolorbox}




\paragraph{Zero-shot:}

\begin{tcolorbox}[left=0mm,right=0mm,top=0mm,bottom=0mm,boxsep=1mm,arc=0mm,boxrule=0pt, frame empty, breakable]
    \small
    \begin{lstlisting}
Given the following code, what is the execution result?
Your answer should be in the following format:
Output:
<execution result>
\end{lstlisting}
\end{tcolorbox}




\paragraph{Few-shot Chain-of-Thought:}

\begin{tcolorbox}[left=0mm,right=0mm,top=0mm,bottom=0mm,boxsep=1mm,arc=0mm,boxrule=0pt, frame empty, breakable]
    \small
    \begin{lstlisting}
Given the following code, what is the execution result?
You should think step by step.  Your answer should be in the following format:
Thought: <your thought>
Output:
<execution result>
Following are 3 examples: 
{{examples here}}

\end{lstlisting}
\end{tcolorbox}





\subsubsection{Demo Questions}

\begin{tcolorbox}[left=0mm,right=0mm,top=0mm,bottom=0mm,boxsep=1mm,arc=0mm,boxrule=0pt, frame empty, breakable]
    \small
    \begin{lstlisting}
def catalan_number(n: int) -> int:
    # Initialize an array to store the intermediate catalan numbers
    catalan = [0] * (n + 1)
    catalan[0] = 1  # Base case

    # Calculate catalan numbers using the recursive formula
    for i in range(1, n + 1):
        for j in range(i):
            catalan[i] += catalan[j] * catalan[i - j - 1]

    return catalan[n]

if __name__ == "__main__":
    # Run the test function and print the result of a specific test case
    print(catalan_number(3))
\end{lstlisting}
\end{tcolorbox}



\begin{tcolorbox}[left=0mm,right=0mm,top=0mm,bottom=0mm,boxsep=1mm,arc=0mm,boxrule=0pt, frame empty, breakable]
    \small
    \begin{lstlisting}
import java.util.*;

class Solution {
    public static int countPrefixWords(List<String> wordList, String prefix) {

        int count = 0;
        for (String word : wordList) {
            if (word.startsWith(prefix)) {
                count++;
            }
        }
        return count;
    }

    public static void main(String[] args) {
        System.out.println(countPrefixWords(Arrays.asList("dog", "dodge", "dot", "dough"), "do"));
    }
}
\end{lstlisting}
\end{tcolorbox}

\begin{tcolorbox}[left=0mm,right=0mm,top=0mm,bottom=0mm,boxsep=1mm,arc=0mm,boxrule=0pt, frame empty, breakable]
    \small
    \begin{lstlisting}
#include <assert.h>
#include <stdio.h>

long long minTotalCost(int n, int *C)
{
   return (long long)(C[n-2]) * (n - 1) + C[n-1];
}

int main() {
    int costs3[] = {5, 4, 3, 2};
    printf("%lld\n", minTotalCost(4, costs3));
    return 0;
}
\end{lstlisting}
\end{tcolorbox}


\begin{tcolorbox}[left=0mm,right=0mm,top=0mm,bottom=0mm,boxsep=1mm,arc=0mm,boxrule=0pt, frame empty, breakable]
    \small
    \begin{lstlisting}
function merge_sorted_arrays(nums1::Vector{Int}, m::Int, nums2::Vector{Int}, n::Int) :: Vector{Int}
    i = m 
    j = n 
    k = m + n
    
    while j > 0
        if i > 0 && nums1[i] > nums2[j]
            nums1[k] = nums1[i]
            i -= 1
        else
            nums1[k] = nums2[j]
            j -= 1
        end
        k -= 1
    end
    
    nums1
end

# Test case
result = merge_sorted_arrays([1, 3, 5, 0, 0, 0], 3, [2, 4, 6], 3)
println(result)
\end{lstlisting}
\end{tcolorbox}



\begin{tcolorbox}[left=0mm,right=0mm,top=0mm,bottom=0mm,boxsep=1mm,arc=0mm,boxrule=0pt, frame empty, breakable]
    \small
    \begin{lstlisting}
public class Solution {

  public static int findSmallestInteger(int n) {
    char[] characters = Integer.toString(n).toCharArray();
    int i = characters.length - 2;

    // Find the first digit that is smaller than the digit next to it.
    while (i >= 0 && characters[i] >= characters[i + 1]) {
      i--;
    }

    if (i == -1) {
      return -1; // Digits are in descending order, no greater number possible.
    }

    // Find the smallest digit on right side of (i) which is greater than characters[i]
    int j = characters.length - 1;
    while (characters[j] <= characters[i]) {
      j--;
    }

    // Swap the digits at indices i and j
    swap(characters, i, j);

    // Reverse the digits from index i+1 to the end of the array
    reverse(characters, i + 1);

    try {
      return Integer.parseInt(new String(characters));
    } catch (NumberFormatException e) {
      return -1; // The number formed is beyond the range of int.
    }
  }

  private static void swap(char[] arr, int i, int j) {
    char temp = arr[i];
    arr[i] = arr[j];
    arr[j] = temp;
  }

  private static void reverse(char[] arr, int start) {
    int end = arr.length - 1;
    while (start < end) {
      swap(arr, start, end);
      start++;
      end--;
    }
  }

  public static void main(String[] args) {
    System.out.println(findSmallestInteger(123));
  }
}
\end{lstlisting}
\end{tcolorbox}