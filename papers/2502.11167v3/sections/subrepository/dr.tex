\subsection{DR}

\subsubsection{System Prompts}


\paragraph{Zero-shot Chain-of-Thought:}

\begin{tcolorbox}[left=0mm,right=0mm,top=0mm,bottom=0mm,boxsep=1mm,arc=0mm,boxrule=0pt, frame empty, breakable]
    \small
    \begin{lstlisting}
Given the following code, what is the execution result? The file is under `/app/` directory, and is run with /bin/bash -c 'g++ -std=c++C++14 O1 test.cpp -o test && ./test'.
You should think step by step. Your answer should be in the following format:
Thought: <your thought>
Output:
<execution result>
\end{lstlisting}
\end{tcolorbox}




\paragraph{Zero-shot:}

\begin{tcolorbox}[left=0mm,right=0mm,top=0mm,bottom=0mm,boxsep=1mm,arc=0mm,boxrule=0pt, frame empty, breakable]
    \small
    \begin{lstlisting}
Given the following code, what is the execution result? The file is under `/app/` directory, and is run with /bin/bash -c 'g++ -std=c++C++14 O1 test.cpp -o test && ./test'.
Your answer should be in the following format:
Output:
<execution result>
\end{lstlisting}
\end{tcolorbox}




\paragraph{Few-shot Chain-of-Thought:}

\begin{tcolorbox}[left=0mm,right=0mm,top=0mm,bottom=0mm,boxsep=1mm,arc=0mm,boxrule=0pt, frame empty, breakable]
    \small
    \begin{lstlisting}
Given the following code, what is the execution result? The file is under `/app/` directory, and is run with /bin/bash -c 'g++ -std=c++C++14 O1 test.cpp -o test && ./test'.
You should think step by step. Your answer should be in the following format:
Thought: <your thought>
Output:
<execution result>
Following are 6 examples: 
\end{lstlisting}
\end{tcolorbox}





\subsubsection{Demo Questions}

\begin{tcolorbox}[left=0mm,right=0mm,top=0mm,bottom=0mm,boxsep=1mm,arc=0mm,boxrule=0pt, frame empty, breakable]
    \small
    \begin{lstlisting}
struct NonPOD {
    NonPOD() {}
    int x;
};
int main() {

    static_assert(std::is_pod<NonPOD>::value, "");
}
\end{lstlisting}
\end{tcolorbox}



\begin{tcolorbox}[left=0mm,right=0mm,top=0mm,bottom=0mm,boxsep=1mm,arc=0mm,boxrule=0pt, frame empty, breakable]
    \small
    \begin{lstlisting}
#include <coroutine>
struct task {
    struct promise_type { /*...*/ };

};
\end{lstlisting}
\end{tcolorbox}

\begin{tcolorbox}[left=0mm,right=0mm,top=0mm,bottom=0mm,boxsep=1mm,arc=0mm,boxrule=0pt, frame empty, breakable]
    \small
    \begin{lstlisting}
#include <atomic>
#include <thread>
#include <iostream>

std::atomic<int> data{0};

void writer() {
    data.store(1, std::memory_order_relaxed);
}

void reader() {
    while (data.load(std::memory_order_relaxed) == 0);
    std::cout << "Data updated";
}

int main() {
    std::thread t1(writer), t2(reader);
    t1.join(); t2.join();
}
\end{lstlisting}
\end{tcolorbox}


\begin{tcolorbox}[left=0mm,right=0mm,top=0mm,bottom=0mm,boxsep=1mm,arc=0mm,boxrule=0pt, frame empty, breakable]
    \small
    \begin{lstlisting}
#include <iostream>

struct S {
    S() { std::cout << "ctor\n"; }
    ~S() { std::cout << "dtor\n"; }
    S(const S&) { std::cout << "copy\n"; }
};

const S& getTemp() {
    return S();
}

int main() {
    const S& ref = getTemp();
    std::cout << "main\n";
    return 0;
}
\end{lstlisting}
\end{tcolorbox}



\begin{tcolorbox}[left=0mm,right=0mm,top=0mm,bottom=0mm,boxsep=1mm,arc=0mm,boxrule=0pt, frame empty, breakable]
    \small
    \begin{lstlisting}
template<typename T> void f(T) { std::cout << "1"; }
template<> void f(int*) { std::cout << "2"; }
template<typename T> void f(T*) { std::cout << "3"; }
int main() {
    int* p = nullptr;
    f(p);
}
\end{lstlisting}
\end{tcolorbox}