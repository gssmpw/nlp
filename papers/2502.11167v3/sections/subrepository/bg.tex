\subsection{BG}

\begin{table}[!ht]
  \centering
  \caption{Language usage count across different categories in the BG subset.}
  \label{tab:bg_language_usage}
  \begin{tabular}{lccc}
      \toprule
      Java & Python & C++ \\
      \midrule
       51   & 45     & 54  \\
      \bottomrule
  \end{tabular}
\end{table}

\begin{table}[!ht]
  \centering
    \caption{Details of bug types in BG dataset and how many times each kind of bug appears in different languages.}
  \label{tab:stat:6}

\resizebox{0.5\textwidth}{!}{\begin{tabular}{lccc}
\toprule
\textbf{Error Type} & Java & Python3 & CPP \\
\midrule
== and = confusion & 5 & 6 & 5 \\
undefined keywords & 6 & 3 & 5 \\
parentheses mismatch & 5 & 5 & 6 \\
indexing error & 10 & 9 & 11 \\
undefined objects & 11 & 9 & 8 \\
unclosed string & 7 & 5 & 7 \\
conditional statement error & 10 & 8 & 9 \\
undefined methods & 8 & 3 & 6 \\
colon missing & 5 & 7 & 8 \\
wrong comment mark & 9 & 1 & 9 \\
variable value error & 2 & 2 & 4 \\
operation error & 2 & 2 & 3 \\
other error & 4 & 2 & 1 \\
statement separation & 4 & 0 & 7 \\
indentation error & 0 & 4 & 0 \\
Double Bugs & 10 & 8 & 10 \\
Triple Bugs & 12 & 10 & 11 \\
Quadruple Bugs & 8 & 5 & 9 \\
\bottomrule
\end{tabular}}
\end{table}

In BG, the distribution of language usage across categories is shown in Table \ref{tab:bg_language_usage}, indicating a balanced usage of Java, Python, and C++. Table \ref{tab:stat:6} presents a detailed breakdown of bug types and their frequency across different languages. This distribution allows us to assess the model's ability to handle a variety of bugs across multiple programming languages.

% \input{sections/subrepository/subrepo-tables/table_bg}


\subsubsection{System Prompts}


\paragraph{Zero-shot Chain-of-Thought:}

\begin{tcolorbox}[left=0mm,right=0mm,top=0mm,bottom=0mm,boxsep=1mm,arc=0mm,boxrule=0pt, frame empty, breakable]
    \small
    \begin{lstlisting}
Given the following code, what is the execution result? The file is under `/app/` directory, and is run with "python3 /app/test.py" if it is a python file, "g++ -std=c++11 /app/test.cpp -o /app/test
/app/test" if it is a cpp file, and "javac /app/\{class_name\}.java
java -cp /app \{class_name\}" if it is a java file.
You should think step by step.  Your answer should be in the following format:
Thought: <your thought>
Output:
<execution result>
\end{lstlisting}
\end{tcolorbox}




\paragraph{Zero-shot:}

\begin{tcolorbox}[left=0mm,right=0mm,top=0mm,bottom=0mm,boxsep=1mm,arc=0mm,boxrule=0pt, frame empty, breakable]
    \small
    \begin{lstlisting}
Given the following code, what is the execution result? The file is under `/app/` directory, and is run with "python3 /app/test.py" if it is a python file, "g++ -std=c++11 /app/test.cpp -o /app/test
/app/test" if it is a cpp file, and "javac /app/\{class_name\}.java
java -cp /app \{class_name\}" if it is a java file.
Your answer should be in the following format:
Output:
<execution result>
\end{lstlisting}
\end{tcolorbox}




\paragraph{Few-shot Chain-of-Thought:}

\begin{tcolorbox}[left=0mm,right=0mm,top=0mm,bottom=0mm,boxsep=1mm,arc=0mm,boxrule=0pt, frame empty, breakable]
    \small
    \begin{lstlisting}
Given the following code, what is the execution result? The file is under `/app/` directory, and is run with "python3 /app/test.py" if it is a python file, "g++ -std=c++11 /app/test.cpp -o /app/test
/app/test" if it is a cpp file, and "javac /app/\{class_name\}.java
java -cp /app \{class_name\}" if it is a java file.
You should think step by step.  Your answer should be in the following format:
Thought: <your thought>
Output:
<execution result>
Following are 4 examples: 
{{examples here}}

\end{lstlisting}
\end{tcolorbox}





\subsubsection{Demo Questions}

\begin{tcolorbox}[left=0mm,right=0mm,top=0mm,bottom=0mm,boxsep=1mm,arc=0mm,boxrule=0pt, frame empty, breakable]
    \small
    \begin{lstlisting}
// Import necessary packages
import java.util.*;

class Solution {

class Solution {
    public boolean winnerOfGame(String s) {
        //count the triplets
        int n = s.length();
    
        int a=0;
        int b=0;
        
        for(int i=1; i<n-1; i++)
        {
            if(s.charAt(i)=='A' && s.charAt(i-1)=='A' && s.charAt(i+1)=='A' )
                a++;
            else if(s.charAt(i)=='B' && s.charAt(i-1)=='B' && s.charAt(i+1)=='B' )
                b++;
        }
        if(a == b)
            return false;
        else
            return true;
    }
}

public class Main {
    public static void main(String[] args) {
        Solution solution = new Solution();

        // Test case 1
        String colors1 = "AAABABB";
        System.out.println("Test Case 1: " + solution.winnerOfGame(colors1)); // Alice wins

        // Test case 2
        String colors2 = "AA";
        System.out.println("Test Case 2: " + solution.winnerOfGame(colors2)); // Bob wins

        // Test case 3
        String colors3 = "ABBBBBBBAAA";
        System.out.println("Test Case 3: " + solution.winnerOfGame(colors3)); // Bob wins
\end{lstlisting}
\end{tcolorbox}



\begin{tcolorbox}[left=0mm,right=0mm,top=0mm,bottom=0mm,boxsep=1mm,arc=0mm,boxrule=0pt, frame empty, breakable]
    \small
    \begin{lstlisting}
import java.util.Arrays;

public class Main {

class Solution {
    public int matrixSum(int[][] nums) {
        int score = 0;
        int n = nums.length;
        int m = nums[0].length;
        for(int[] a :nums)
        {
            Arrays.sort(a);
        }
        for(int i=0;i<=n;i++)
        {
            int max = 0;
            for(int j=0;j<m;j++)
            {
                max = Math.max(max,nums[i][j]);
            }
            score+=max;
        }
        return score;
    }
}

    public static void main(String[] args) {
        Solution solution = new Solution();

        // Test case 1
        int[][] nums1 = {
            {7, 2, 1},
            {6, 4, 2},
            {6, 5, 3},
            {3, 2, 1}
        };
        System.out.println(solution.matrixSum(nums1)); // Output: 15

        // Test case 2
        int[][] nums2 = {
            {1}
        };
        System.out.println(solution.matrixSum(nums2)); // Output: 1
\end{lstlisting}
\end{tcolorbox}

\begin{tcolorbox}[left=0mm,right=0mm,top=0mm,bottom=0mm,boxsep=1mm,arc=0mm,boxrule=0pt, frame empty, breakable]
    \small
    \begin{lstlisting}
from collections import defaultdict
from typing import List


class Solution:
    def numberOfArithmeticSlices(self, nums: List[int]) -> int:
        total, n = 0, len(nums)
        dp = [defaultdict(int) for _ in nums]
        for i in range(1, n):
            for j in range(i):
                diff = nums[j] - nums[i]
                dp[i][diff] += dp[j][diff] + 1
                total += self.undifned_method(dp[j][diff])
        return total

# Test cases
if __name__ == "__main__":
    solution = Solution()

    # Test case 1
    nums1 = [2, 4, 6, 8, 10]
    result1 = solution.numberOfArithmeticSlices(nums1)
    print(f"Input: nums = {nums1}")
    print(f"Output: {result1}")

    # Test case 2
    nums2 = [7, 7, 7, 7, 7]
    result2 = solution.numberOfArithmeticSlices(nums2)
    print(f"Input: nums = {nums2}")
    print(f"Output: {result2}")
\end{lstlisting}
\end{tcolorbox}


\begin{tcolorbox}[left=0mm,right=0mm,top=0mm,bottom=0mm,boxsep=1mm,arc=0mm,boxrule=0pt, frame empty, breakable]
    \small
    \begin{lstlisting}
#include <iostream>
#include <cmath>


class Solution {
public:
    long long fact(int n)
    {
        if(n<=1)return 1;
        return (n*fact(n-1)%1000000007)%1000000007;
    }
    int numPrimeArrangements(int n) {
        if(n==1)return 1;
        if(n<=3)return n-1;
        int t=0,flag;
        for(int i=2;i<=n;i++)
        {
            flag=0;
            for(int j=2;j<sqrt(i);j++)
            {
                if(i%j==0)
                {
                    flag=1;
                    break;
                }
            }
            if(flag==0)
            {
                t++;
            }
        }
        return (fact(t)*fact(n-t))%1000000007;

    }
};

int main() {
    Solution solution;
    // Test case 1
    int n1 = 5;
    std::cout << "Input: n = " << n1 << "\nOutput: " << solution.numPrimeArrangements(n1) << std::endl;

    // Test case 2
    int n2 = 100;
    std::cout << "Input: n = " << n2 << "\nOutput: " << solution.numPrimeArrangements(n2) << std::endl;

    return 0;
\end{lstlisting}
\end{tcolorbox}



\begin{tcolorbox}[left=0mm,right=0mm,top=0mm,bottom=0mm,boxsep=1mm,arc=0mm,boxrule=0pt, frame empty, breakable]
    \small
    \begin{lstlisting}
#include <iostream>
#include <string>
#include <cctype> // For isalpha

using namespace std;


class Solution {
public:
    str reverseOnlyLetters(string s) 
    {
      int i=0,j=s.length()-1;
      while(i<=j)
      {
        if(isalpha(s[i])&&isalpha(s[j]))
        {
            swap(s[i],s[j]);
            i++;
            j--;
        }
        else
        {
            if(!isalpha(s[i]))
            {
                i++;
            }
            if(!isalpha(s[j]))
            {
                j--;
            }
        }
      }
      return s;
    }
};

int main() {
    // Initialize the Solution class
    Solution solution;

    // Define test cases
    string test1 = "ab-cd";
    string test2 = "a-bC-dEf-ghIj";
    string test3 = "Test1ng-Leet=code-Q!";

    // Run test cases and print results
    cout << "Test 1: " << solution.reverseOnlyLetters(test1) << endl;
    cout << "Test 2: " << solution.reverseOnlyLetters(test2) << endl;
    cout << "Test 3: " << solution.reverseOnlyLetters(test3) << endl;

    return 0;
\end{lstlisting}
\end{tcolorbox}