Our work connects to several strands of existing literature. 

\xhdr{GFT-optimization in the bilateral trade model.} 
\citet{MS-83} introduce and study Bayesian mechanism design for bilateral trade instances. The authors present the characterization of the GFT-optimal truthful mechanism (achieving the {\SecondBest}) and show the strict gap between the First Best and {\SecondBest}. \citet{DMSW-22} shows that the multiplicative gap between the two benchmarks is at most $1/8.23$, by analyzing the GFT approximation of the {\RandomOffer} against the {\FirstBest}. A follow-up work by \citet{Fei-22} further improves the multiplicative gap to $1/3.15$. The author also shows that the approximation of the {\RandomOffer} is at most $1/(e - 1)$ against the {\FirstBest} when both traders have MHR valuation distributions. Since the GFT-optimal truthful mechanism is complicated, various simple mechanisms have been proposed and been proven to achieve good GFT approximation under various conditions \citep[e.g.,][]{mca-92,BNP-09,BM-16,DTR-17,BCWZ-17,CGKLT-17,BCGZ-18,BGG-20,CLMZ-24,HHPS-25}. Our work differs from all those prior work as we study GFT-optimization subject to fairness conditions motivated by the cooperative bargaining problem.

Besides the Bayesian mechanism design model, there is also a recent literature initiated by \citet{CCCFL-24a} that includes \citep{BCCC-24,BCCF-24,CCCFL-24b,AFF-24} which studies
GFT-optimization in an online learning model, where traders' valuation distributions are unknown and need to be learned from repeated interactions. Instead of analyzing the approximation ratios, most prior work uses the optimal-in-hindsight {\FixPrice} (whose GFT could be arbitrarily smaller than the {\SecondBest}) as the benchmark   and focuses on designing sublinear regret online learning algorithm against this weaker benchmark. \citet{BCCC-24} introduce the problem of fair online bilateral trade and propose a new efficiency measure ``fair GFT'' defined as the minimum utility between two traders. Unlike their approach, our work sticks to the classic GFT as the efficiency measure and incorporates fairness as a constraint, motivated by the cooperative bargaining problem. 

\xhdr{Cooperative bargaining with incomplete information.} \label{subsubsec:related-bargaining}
The cooperative bargaining problem was introduced by \citet{N-50}, who also proposed the Nash solution as the unique outcome satisfying four axioms: Pareto optimality, symmetry, scale invariance, and independence of irrelevant alternatives. In the same model, \citet{KS-75} replaced the independence of irrelevant alternatives axiom with the resource monotonicity axiom and introduced the Kalai-Smorodinsky (KS) solution. Meanwhile, the egalitarian solution, which substitutes the scale invariance axiom with the resource monotonicity axiom, was developed by \citet{Kal-77,Mye-77}. For further details, we refer the reader to \citet{AH-92}. There is also a literature about bargaining with private information \citep[e.g.,][]{HS-72,Mye-84,Sam-84,KW-93,ACD-02}, where analog axioms (similar to the original bargaining problem with public information) and solutions (mechanisms) are introduced and studied. While these works focus on axiomatization of the solution concepts for general bargaining problems, we focus on bilateral trade problems and are interested in bounding the GFT approximation of truthful mechanisms satisfying various fairness notions, against the {\SecondBest}.



Importantly, we remark that there is one difference between the bilateral trade problem and the bargaining problem. In the bilateral trade problem, the GFT includes both traders' utilities and the gains left to the mechanism (which may be positive under WBB). In contrast, the bargaining problem only reasons about the utilities of {the} two agents (traders). Hence, if we restrict {the space of mechanisms to only include} SBB mechanisms (i.e., no gain left to mechanisms), then the bilateral trade problem becomes a special case of the bargaining problem. \citet{BCWZ-17} shows that for GFT-optimization, it is without loss of generality to consider ex post SBB mechanisms. In particular, they develop a transformation that converts any ex ante WBB mechanism into an ex post SBB mechanism with the same allocation rule, by adjusting the payment rule for the two traders. Unfortunately, their transformation change the two traders' utilities {in a way that violates the fairness constraint,} and thus cannot be applied to our work.\footnote{It is interesting to study whether every {\ksfair} ex ante WBB mechanism also admits a {\ksfair} ex post SBB mechanism implementation with the same allocation rule. {This, for example, will imply that the {\ksfair} mechanism with highest GFT is the KS-solution.} {One natural attempt is to proportionally ex-ante split the ex-ante gains of the mechanism to two traders (and thus not affect incentive compatibility or participation).} However, this only ensures ex ante SBB, and there might be ex post positive transfer to the buyer, which is impractical and unnatural. We leave this challenge for future work.}



\xhdr{Approximation analysis via optimization programs.} 
In this work, we obtain almost-tight GFT-approximations in \Cref{thm:improved GFT:regular buyer,thm:improved GFT:mhr buyer} by first formulating the GFT approximation guarantees as optimization programs~\ref{program:GFT:regular buyer}, \ref{program:GFT:mhr buyer} and then numerically lower bounding their optimal objective value. Characterizing the approximation guarantees as optimization programs and then obtaining the final approximation ratios by numerical evaluation is a widely adopted approach in the approximate mechanism design literature, e.g., revenue-maximization for zero-value seller problem \citep{AHNPY-18}, prior-independent mechanism design \citep{HJL-19}, price of anarchy \citep{HTW-18}, mechanism design with samples \citep{FHL-21,ABB-22}. In particular, recent work \citep{CW-23,LRW-23} {studies} the welfare-maximization of the {\FixPrice} for the bilateral trade problem. As a comparison, we analyze the {\FixPrice} with an additional fairness constraint, which requires us to analyze the revenue, residual surplus, and GFT -- all three of these together.




