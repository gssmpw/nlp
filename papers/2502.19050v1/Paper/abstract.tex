
We consider the impact of fairness requirements on the social efficiency of truthful mechanisms for trade, focusing on Bayesian bilateral-trade settings. 
Unlike the full information case in which all gains-from-trade can be realized and equally split between the two parties, in the private information setting, equitability has devastating welfare implications (even if only required to hold ex-ante). 
We thus search for an alternative fairness notion and suggest requiring the mechanism to be {\ksfair}: it must ex-ante equalize the fraction of the ideal utilities of the two traders.  
We show that there is always a {\ksfair} (simple) truthful mechanism with expected gains-from-trade that are half the optimum, but always ensuring any better fraction is impossible (even when the seller value is zero). 
We then restrict our attention to trade settings with a zero-value seller and a buyer with valuation distribution that is Regular or MHR, proving that much better fractions can be obtained under these conditions, with simple posted-price mechanisms. 