In this section, we provide additional details for our numerical evaluation of program~\ref{program:GFT:regular buyer}.
As the first step, we prove that it suffices to assume $H = 1$ and $L = \LOverBar$.
\begin{lemma}
\label{lem:regular buyer program:reducing variables}
    For program~\ref{program:GFT:regular buyer}, there exists an optimal solution with $H = 1$ and $L = \LOverBar$. Namely, the optimal objective value of program~\ref{program:GFT:regular buyer} is the same as the optimal objective value of program~\ref{program:GFT:regular buyer} with additional constraints (for the minimization) that $H = 1$ and $L = \LOverBar$
\end{lemma}
\begin{proof}
    We first argue that forcing $L = \LOverBar$ does not change the optimal objective value of the program. To see this, note that $L$ only appears in the objective function, where auxiliary variable $\exanteutilratio$ depends on $L$. By algebra, it can be verified that auxiliary variable $\exanteutilratio$ decreases as $L$ increases. Hence, the objective function is decreasing in $L$ and thus there exists an optimal solution such that $L = \LOverBar$.

    We next argue that forcing $H = 1$ does not change the optimal objective value of the program. Suppose there exists an optimal solution $(\optquant, H, \revratio, \quant, \val_0, M, L)$ of the program with $H > 1$ and $L = 1$. Consider an alternative solution $(\optquant\primed, H\primed, \revratio\primed, \quant\primed, \val_0\primed, M\primed, L\primed)$ where we set 
    \begin{align*}
        \optquant\primed \gets e^{1-H}\cdot \optquant,~
        H\primed \gets 1,~
        \revratio\primed \gets \revratio,~
        \quant\primed \gets \quant,~
        \val_0\primed \gets \val_0,~
        M\primed \gets M + (H - 1),~
        L\primed \gets L
    \end{align*} 
    It can be verified that the objective values of two solutions are identical. Moreover, the alternative solution is feasible. To see this, note that $\optquant\primed < \optquant$ since $H > 1$. Consequently, by algebra, the feasible regions of $\quant\primed, \val\primed$, and $M\primed$ becomes larger. This also ensures that there does not exists $\revratio\doubleprimed$ which can improve the objective value while fixing $\optquant\primed$ and $H\primed$. This finishes the proof the lemma.
\end{proof}
After simplifying program~\ref{program:GFT:regular buyer} by fixing $H = 1$ and $L = \LOverBar$, there are still five variables $(\optquant, \revratio,\quant,\val_0, M)$, where $\optquant$ appears in the outer minimization, $\revratio$ appears in the middle maximization, and $\quant,\val_0, M$ appear in the inner minimization. Next, we partition the feasible region of $\optquant$ (i.e., $[0, 1]$) into subintervals. For each subinterval, we manually set the value of $\revratio$. In this way, for each subinterval $[s, \ell]$, we obtain a pure minimization program~\ref{program:GFT:regular buyer:decompose} parameterized by $(s,\ell, \revratio)$ with variables $(\optquant,\quant,\val_0, M)$ defined as follows:
\begin{align}
\label{program:GFT:regular buyer:decompose}
\tag{$\mathcal{P}_{\mathrm{REG}}[s,\ell,\revratio]$}
\arraycolsep=5.4pt\def\arraystretch{1}
    \begin{array}{llll}
     &\min\limits_{\optquant, \quant, \val_0, M}   & 
      \displaystyle\exanteutilratio + \frac{\exanteutilratio}{1 + M + \LOverBar} &
      \vspace{10pt}
      \\
      \vspace{10pt}
      &\text{s.t.}
      & \optquant\in[s, \ell],~  
      % & 
      % \\
      % \vspace{10pt}
      % && 
      \quant\in \left[\optquant + (1 - \revratio)(1 - \optquant), 1\right],~ 
      & 
      \\
      \vspace{10pt}
      && 
      \val_0\in \left[0, 1 - \displaystyle\frac{1 - \revratio}{\quant - \optquant}(1 - \optquant)\right],~ 
      % & 
      % \\
      % \vspace{10pt}
      % && 
      M\in\left[\MUnderBar, \MOverBar\right]& 
    \end{array}
\end{align}
where auxiliary variables $\exanteutilratio$, $\MUnderBar, \MOverBar, \LOverBar$ are follows the same definitions as program~\ref{program:GFT:regular buyer}. 
By construction, program~\ref{program:GFT:regular buyer} can be lower bounded by program~\ref{program:GFT:regular buyer:decompose} as follows
\begin{lemma}
    Fix any $K\in\naturals$, any partition $\{[s_k, \ell_k]\}_{k\in[K]}$ of $[0, 1]$, and any assignment $\{\revratio_k\}_{k\in[K]} \in[0, 1]^K$. The optimal objective value $\Obj{\text{\ref{program:GFT:regular buyer}}}$ of program~\ref{program:GFT:regular buyer} can be lower bounded as
    \begin{align*}
        \Obj{\text{\ref{program:GFT:regular buyer}}} 
        \geq 
        \min_{k\in[K]} 
        \Obj{\text{\hyperref[program:GFT:regular buyer:decompose]{$\mathcal{P}_{\mathrm{REG}}[s_k,\ell_k,\revratio_k]$}}}
    \end{align*}
\end{lemma}
\begin{proof}
    Suppose $(\optquant, H, \revratio, \quant, \val_0, M, L)$ is an optimal solution of program~\ref{program:GFT:regular buyer}. Invoking \Cref{lem:regular buyer program:reducing variables}, it is without loss of generality to assume $H = 1$ and $L = \LOverBar$. Let $k\primed$ be the index such that $\optquant\in [s_{k\primed}, \ell_{k\primed}]$. By definition, we have 
    \begin{align*}
        \Obj{\text{\ref{program:GFT:regular buyer}}} 
        =
        \Obj{\text{\hyperref[program:GFT:regular buyer:decompose]{$\mathcal{P}_{\mathrm{REG}}[s_{k\primed},\ell_{k\primed},\revratio]$}}}
        \geq 
        \Obj{\text{\hyperref[program:GFT:regular buyer:decompose]{$\mathcal{P}_{\mathrm{REG}}[s_{k\primed},\ell_{k\primed},\revratio_k]$}}}
        \geq 
        \min_{k\in[K]} 
        \Obj{\text{\hyperref[program:GFT:regular buyer:decompose]{$\mathcal{P}_{\mathrm{REG}}[s_k,\ell_k,\revratio_k]$}}}
    \end{align*}
    where the first inequality holds since $\revratio$ belong to the optimal solution of program~\ref{program:GFT:regular buyer}.
\end{proof}
We report the empirical choice of partition $\{[s_k, \ell_k]\}$ and assignment $\{\revratio_k\}$ in \Cref{tab:GFT:regular buyer:decompose}. For each $k$, we uniformly discretize the feasible region of each variable into $500$ points and then numerically evaluate program~\ref{program:GFT:regular buyer:decompose}, which leads to the numerical lower bound of $\fixedPriceGFTPercentageRegular$.


\begin{table}
    \centering
    \begin{tabular}{|c|c|c|c|c|}
    \hline
    $[s, \ell]$ &
       [0, 0.002] & [0.002, 0.008]  & [0.008, 0.018]  & [0.018, 0.034] \\
       \hline
       $\revratio$ & 0.8 & 0.78 & 0.76 & 0.74
       \\
       \hline
       \multicolumn{5}{c}{}
       \\
       \hline
      $[s, \ell]$  & [0.034, 0.044]  & [0.044, 0.078] & [0.078, 0.1] & [0.1, 1] \\
       \hline
       $\revratio$ & 0.72 & 0.7 & 0.68 & 0.66
       \\
       \hline
    \end{tabular}
    \caption{Parameters used in program~\ref{program:GFT:regular buyer:decompose}.}
    \label{tab:GFT:regular buyer:decompose}
\end{table}