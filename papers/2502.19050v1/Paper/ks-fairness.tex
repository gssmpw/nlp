In this section, we define {\ksfairness}, the main notion of fairness for bilateral trade that we study in this paper. Before presenting the formal definition, we introduce $\sellerbenchmark$ and $\buyerbenchmark$, which are the seller and buyer's ideal (ex ante)  utilities (aka., the seller and buyer benchmarks) for instance $\btinstance$, computed as\footnote{In defining the ideal utilities $\sellerbenchmark$, $\buyerbenchmark$, as well as in some other definitions later in the paper, we use $\max$ instead of $\sup$ because the set $\mechfam$ is compact, ensuring that the maximum is always attainable.}
\begin{align*}
{\sellerbenchmark= 
    \sellerbenchmark\btinstance} \triangleq \max\limits_{{\mech\in\mechfamily}}\sellerexanteutil(\mech) 
    \;\;
    \mbox{and}
    \;\; 
    {\buyerbenchmark = \buyerbenchmark\btinstance} \triangleq \max\limits_{\mech\in\mechfamily}\buyerexanteutil(\mech)
\end{align*}
As a {reminder}, for {every} bilateral trade instance $\btinstance$, the ideal utilities $\sellerbenchmark$ and $\buyerbenchmark$ are achievable by the {\SellerOffer} and {\BuyerOffer} \citep{mye-81}, respectively. For zero-value seller instances, the seller and buyer's ideal utilities $\sellerbenchmark$ and $\buyerbenchmark$ are achieved by posting a deterministic take-it-or-leave it price at the monopoly reserve and at zero, respectively \citep{mye-81}.

In the spirit of the Kalai-Smorodinsky (KS) solution to bargaining problems \citep{KS-75}, we study \emph{{\ksfairness}} for bilateral trade, defined as follows: 
\begin{definition}[{\ksfairness}]
\label{def:ks fairness} {Fix any bilateral trade instance $\btinstance$.} 
    A mechanism $\mech\in\mechfamily$ is \emph{{\ksfair}} if two players' ex ante utilities achieve the same fraction of
    each player's own ideal utility, i.e.,
    \begin{align*}
        \frac{\sellerexanteutil(\mech,\buyerdist,\sellerdist)}{\sellerbenchmark\btinstance}
        = \frac{\buyerexanteutil(\mech, \buyerdist,\sellerdist)}{\buyerbenchmark\btinstance}
    \end{align*}
\end{definition}

We remark that for every instance, there is at least one mechanism that is (trivially) {\ksfair}: the {\NoTrade} (one in which there is never trade) satisfies {\ksfairness}, since both players receive zero ex ante utility. However, its GFT is also zero, which leads to a zero-approximation to the {\SecondBest} $\OPTSB$ for any non-trivial instance.

In \Cref{appendix:bargaining-and-trade} we discuss a way to view the bilateral trade problem as a bargaining problem, where each mechanism corresponds to a point in the two dimensional plane of ex ante utilities obtained by the two traders. With that perspective, the Kalai-Smorodinsky solution to the bargaining problem that corresponds to the bilateral trade instance, is the {\ksfair} mechanism which is Pareto optimal (maximizes $\buyerexanteutil(\mech)+\sellerexanteutil(\mech)$). In \Cref{appendix:bargaining-and-trade} we also discuss other bargaining solutions (as the Nash solution and the Egalitarian solution) and their applicability to the bilateral trade problem. 


Note that {\ksfairness} is defined on the \emph{ex ante} utilities of the traders. In \Cref{subsec:interim ks fairness} we justify the focus on ex ante utilities, by demonstrating that the corresponding interim and ex post fairness notions are too stringent, as both imply that trade must never happen.

For a given bilateral trade instance $\btinstance$, a {\ksfair} mechanism with the highest gains-from-trade $\mech$ is a solution to a linear optimization problem.\footnote{Note that for non-discrete distributions that program is infinite.} The variables of the linear optimization problem are  the {ex post} allocation probability, buyer payment and seller payment $\alloc(\val,\cost),\price(\val,\cost),\sellerprice(\val,\cost)$ for every $\val\in \supp(\buyerdist), \cost\in\supp(\sellerdist)$. It is well known that the requirements of BIC, IIR and ex ante WBB can all be enforced with linear constraints. Additionally, {\ksfairness} can also be enforced {as linear constraints}, as the ex ante utilities $\buyerexanteutil(\mech), \sellerexanteutil(\mech)$ of both traders are linear functions in the variables and the benchmark values $\buyerbenchmark, \sellerbenchmark$ can be computed prior to running this linear optimization problem. Finally, the program maximizes the (expected) GFT obtained by the mechanism, which is a linear objective in the variables.


Notice that maximizing the GFT is not the only option; by changing the objective, we can obtain a  {\ksfair} mechanism which maximizes any linear objective. For example:
\begin{enumerate}
    \item a seller-optimal (resp.\ buyer-optimal) {\ksfair} mechanism can be obtained by maximizing $\sellerexanteutil(\mech)$ (resp. $\buyerexanteutil(\mech)$).
     \item a KS-solution\footnote{\label{footnote:ref-to-ks-solution}See \Cref{subsec:ks-solution} for the formal definition of the KS-solution.} can be obtained by maximizing the objective $\sellerexanteutil(\mech)+\buyerexanteutil(\mech)$.%\footnote{\nmme{Note that this not equivalent to maximizing the GFT since the mechanism may be WBB, i.e. the buyer may pay more then the seller receives. See \Cref{subsec:ks-solution} for a deeper discussion. }}  %\nmmc{There may be more than one, with different GFTs} MB: this is true for the seller optimal as well. I have made some edits. 
\end{enumerate}
We remark that the requirement of {\ksfairness} imposes a constant ratio between $\buyerexanteutil(\mech)$ and $\sellerexanteutil(\mech)$. Thus, the KS-solution, the seller-optimal {\ksfair} mechanism, and the buyer-optimal {\ksfair} mechanism, all have the same seller ex ante utility (and also the same buyer ex ante utility).



A significant issue is that the solution to the linear optimization problem might be a very complex mechanism, that is hard to describe, and  thus not practical. Another issue is that it solves a particular instance and does not give us any insights about the guarantees on the approximation to the {\SecondBest} achieved by such mechanisms for particular sets of instances that are of special interest (as when the buyer distribution is regular). Throughout the paper we overcome both issues by presenting {\ksfair} mechanisms which are simple (ex post SBB, ex post IIR, and sometimes even DSIC) and yet achieve some good approximation to the second best GFT. The following lemma formalizes how any such result implies a lower bound for the fraction of the second best GFT achievable by the above linear optimization problems. 

\begin{restatable}{lemma}{lemsbbimplication}
\label{lem:SBB implication}
    Fix any bilateral trade instance $\btinstance$, if there exists a mechanism $\mech\in \mechfamily$ that is ex ante SBB, {\ksfair}, and guarantees a GFT of at least $\calC$ fraction of the {\SecondBest} $\OPTSB$, then each of the following mechanisms also obtains a GFT that is at least $\calC$ fraction of the {\SecondBest}:
\begin{enumerate}
    \item any GFT-maximizing {\ksfair} mechanism $\mech\primed\in\mechfamily$,
    \item {any mechanism $\mech\doubleprimed\in\mechfamily$ that is a} KS-solution\textsuperscript{\ref{footnote:ref-to-ks-solution}}{(and thus also any seller-optimal {\ksfair} mechanism, and any buyer-optimal {\ksfair} mechanism)}. 
\end{enumerate}
\end{restatable}
\begin{proof}
    We show the GFT approximation for each mechanism separately. For a GFT-maximizing {\ksfair} mechanism $\mech\primed$, its GFT is at least the GFT of mechanism $\mech$ by definition, and thus is a $\calC$-approximation {to the {\SecondBest}}. 
    
    For a KS-solution mechanism $\mech\doubleprimed$, since it is also a seller-optimal {\ksfair} mechanism, the seller's ex ante utility satisfies $\sellerexanteutil(
    {\mech\doubleprimed}) \geq \sellerexanteutil(\mech)$. Combining with the fact that both mechanisms $\mech$ and $\mech\doubleprimed$ are {\ksfair}, we also obtain $\buyerexanteutil(
    {\mech\doubleprimed}) \geq \buyerexanteutil(\mech)$. Thus,
    \begin{align*}
        \GFT{
        {\mech\doubleprimed}} \geq \buyerexanteutil(
        {\mech\doubleprimed}) + \sellerexanteutil(
        {\mech\doubleprimed}) \geq \buyerexanteutil(\mech) + \sellerexanteutil(\mech)
        = \GFT{\mech} \geq \calC \cdot \OPTSB
    \end{align*}
    where the first inequality holds since mechanism $\mech\doubleprimed$ is ex ante WBB, and the equality holds since mechanism $\mech$ is ex ante SBB. 
\end{proof}



As all our positive results prove the existence of ex post SBB {(and thus ex ante SBB as well)} mechanisms $\mech\in \mechfamily$ that {are} {\ksfair} and achieve a good approximation to the {\SecondBest} $\OPTSB$, the lemma applies to all our positive results (\Cref{thm:optimal GFT:general instance,thm:improved GFT:regular buyer,thm:improved GFT:mhr buyer} and \Cref{thm:improved GFT:mhr traders}). Moreover, since all these results also include upper bounds on the fraction of the second best GFT achievable by \emph{any} {\ksfair} truthful mechanism in different settings, these upper bounds also hold for each of the above mechanisms.
