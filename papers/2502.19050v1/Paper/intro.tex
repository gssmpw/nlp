The field of \emph{Mechanism Design} studies the design of mechanisms that obtain good outcomes (as high social welfare or revenue) in the presence of strategic traders. In this paper, we consider imposing a fairness requirement on truthful trade mechanisms\footnote{A trade mechanism must also encourage participation even after an agent knows her value (interim individually rational), and not run a deficit.}, and the impact of such a requirement on the economic efficiency (social welfare) of the mechanism.  That is, a mechanism results in trade which generates gains, and (some of these) gains are allocated to the traders. We take a normative approach and search for simple truthful mechanisms that are constrained to distribute these gains ``fairly'', 
studying their economic efficiency. This problem can be viewed as searching for a fair solution to the bargaining problem between the two traders, but in settings with private valuations. Thus, for trade settings, our work focuses on \emph{the implications of the combination of strategic behavior and fairness requirements on the efficiency of the outcome}. 


We focus on the fundamental mechanism-design problem of bilateral trade in the Bayesian setting, that is, a seller that sells a single good to a buyer, where traders' values for the good are private, but are sampled independently from a known Bayesian prior.\footnote{Note that the seller is not the mechanism designer, but rather a trader in the mechanism.} The most basic of these problems is the setting where the seller has no value for the good.\footnote{This model is equivalent to a seller with known value for keeping the item: we can assume that value is $0$ by applying a simple normalization. Thus, we assume a value of $0$ for simplicity. The gains-from-trade (GFT) is invariant to the normalization (but not the welfare), and is harder to approximate than the welfare (so positive results for GFT are better).} We call this basic setting the \emph{zero-value seller setting}.

As an example, consider a zero-value seller setting with a buyer that has a value of $2$ for the good. In this simple setting there is no uncertainty about the values (a full information setting). If the seller can dictate the mechanism, she will use the seller's optimal mechanism ({\SellerOffer}) and post an optimal price of $2$. She will always sell the item,\footnote{Here the buyer is indifferent between buying or not, but for any $\varepsilon>0$, the seller can get revenue of $2-\varepsilon$ which giving a strict incentive to buy. Throughout the paper we ignore this issue and allow ties to be broken as needed.} obtaining utility of $2$, leaving the buyer with utility of $0$. This outcome seems highly unfair to the buyer, since the gains-from-trade (GFT) of $2$ are the result of a trade that cannot take place without the buyer, yet the buyer receives none of these gains. The alternative mechanism in which trade happens at a price of $1$, results with an efficient trade in which each trader gets utility of $1$, and that mechanism seems much more fair. More generally, when the value of the buyer is known to be $\val > 0$ (no private information), the mechanism in which trade happens at the price $\val/2$ maximizes the gains and seems to be perfectly fair to both traders, as it is \emph{equitable} (both get the same utility).\footnote{In fact, in such a full information setting, any symmetric solution to the bargaining problem (e.g., the Nash solution, or the egalitarian solution) will result with the same outcome where the two parties equally split the gains.} We thus see that when there is no private information, there is a mechanism that maximizes the GFT and is ``perfectly fair''.  

This work focuses on the more involved situation where traders do have private information, and are acting strategically. As an example, consider a seller of a digital good (zero cost/value for the good) to a population of buyers with values distributed according to some known distribution $\buyerdist$. For example, $\buyerdist$ might be the uniform distribution over $[0,1]$. As it is a digital good, the setting is equivalent to a single-buyer setting with value sampled from the valuation distribution. Let us consider the following three properties of a trade mechanism: being \emph{truthful}, being \emph{fair}, and \emph{maximizing the gains-from-trade (GFT)}. Without any fairness considerations, all GFT can be realized by a truthful mechanism that always trades at price 0. If we do not care about gains, the mechanism that never trades is truthful and can be considered fair under various definitions (e.g., it equalizes the utilities). Finally, disregarding the issue of strategic behavior (assuming access to the true values), the problem reduces to the full information case, for which we saw that it is possible to realize all gains and split them equally. We thus see that any two out of three properties can be achieved. Can all three be achieved together? Unfortunately, it seems that the answer is no. For example, in the zero-value seller setting and a buyer with value uniformly distributed over $[0,1]$, the only way to truthfully maximize the gains is to trade at 0, but that mechanism leaves the seller with no profit at all, which intuitively seems unfair (and is also formally unfair under essentially any fairness definition we can conceive). 

We thus relax the goal of exactly maximizing the GFT and only ask to approximate it in expectation, where approximation is with respect to {the {\SecondBest}, that is, the} maximum expected GFT {achievable by a mechanism} without fairness constraints, but with incentive constraints as well as 
participation constraints and the constraint of the mechanism not running a deficit. We ask:   
\emph{For which fairness notions can the truthful trade mechanism guarantee a good GFT approximation? For those fairness notions where a good approximation is possible, how good can the approximation be?}

We move to consider the problem {of picking a fair truthful mechanism} from an ex ante perspective (in expectation, before any values are realized). If the seller is a monopolist in the market, she can price at a Myerson price and maximize her own expected revenue, but this might result in very low expected utility for the buyer (such as in the case of full information), an outcome that can be considered very unfair. A market regulator (or a court) might impose the constraint that the mechanism used for trade be fair to both parties (at least ex ante). But, what kind of fairness can be required? Possibly the most natural would be to require equitability (at least ex ante). Unfortunately, such a definition is too stringent and does not allow for good GFT approximation, as we illustrate next.     

Consider a zero-value seller setting with a buyer that has a private value $\val$, sampled from the Equal-Revenue (ER) distribution with support $[1,\constantH]$ (where $\constantH>1$ is some constant).\footnote{The Equal-Revenue distribution (with support $[1,\constantH]$) has probability $1/\price$ of having value at least $\price\in [1,\constantH]$. Thus the expected revenue from any fixed trading price $\price\in [1,\constantH]$ is $1$.} As the buyer value is private information, we now consider the utilities of the parties in mechanisms that are truthful. It is well known \citep{mye-81} that the seller's ex ante utility in any such mechanism is at most $1$, and thus an ex ante equitable mechanism can yield total gains for both traders \footnote{Here we only consider the utilities of the two traders (disregarding any money kept by the mechanism). Our results in the paper allow for weak budget-balanced mechanisms, and our positive results hold with respect to the more challenging benchmark of the entire GFT, including the gains kept by the mechanism.} of at most 2. This is a negligible fraction of the maximal GFT of $\Theta(\log \constantH)$ (obtainable by the mechanism that always trades at price 0, which is very unfair to the seller).

We conclude that the fairness notion of ex ante equitability is too stringent for the setting of trade where the buyer has private information: imposing ex ante equitability in settings where the buyer and seller differ significantly from each other (ex ante) results in devastating implications on the GFT. Therefore, we seek an alternative fairness notion that is better suited for trade settings where the buyer is ex ante very different from the seller and the traders have private information, yet allows for a good GFT approximation.


In spirit, this problem is a cooperative bargaining problem, where the traders need to ex ante agree on a truthful mechanism that is fair. If they fail to reach an agreement, the default outcome of no trade occurs. Yet, the set over which the traders are bargaining is not explicitly given, but rather induced by the traders' valuation distributions. Moreover, in bilateral-trade settings, where the seller also has a non-trivial valuation distribution, it is known that some of these mechanisms (e.g., the one that maximizes GFT subject to truthfulness) are very complicated and unintuitive, even for simple distributions. 

Two prominent solutions to cooperative bargaining problems are the Kalai-Smorodinsky (KS) solution \citep{KS-75}, and the Nash solution \citep{nash-51}. We first ask:
\begin{displayquote}
\emph{How large is the fraction of GFT guaranteed by the Kalai-Smorodinsky solution? By the Nash solution?} 
\end{displayquote}
Our work mainly focuses on studying the GFT of mechanisms that are \emph{\ksfair}, satisfying the ``Kalai-Smorodinsky condition'': mechanisms that ex ante equalize the fraction of the ideal utilities of the two traders. 
We explain {\ksfairness} using the following simple example: Consider the zero-value seller setting where the buyer's value is uniformly distributed over $[0, 1]$. The buyer's ideal ex ante utility is $0.5$, obtained by trading at a price of $0$, while the seller's ideal ex ante utility is $0.25$, obtained by trading at the monopoly reserve (i.e., Myerson price) of $0.5$. If the two traders trade at a price of $0.2$, the seller receives an expected utility of $0.16$, while the buyer receives $0.32$. Since both utilities achieve the same fraction (64\%) of their respective optima, this mechanism is {\ksfair}. Moreover, its GFT is $0.16+0.32 = 0.48$, which is 96\% fraction of the optimal GFT of $0.5$, achieved by trading at price of $0$ (which is unfair to the seller). 


Although the Kalai-Smorodinsky solution satisfies {\ksfairness}, the mechanism corresponding to that solution may be complex, making both theoretical analysis and practical implementation challenging. This motivates us to design \emph{simple} mechanisms that are {\ksfair} and guarantee good GFT. Therefore, we pose the following question:
\begin{displayquote}
\emph{How large is the fraction of the {\SecondBest} that can be guaranteed by a simple mechanism satisfying {\ksfairness} (aka., the Kalai-Smorodinsky condition)?} 
\end{displayquote}
As we will explain in detail in the following section, our work presents simple mechanisms that are {\ksfair} and give nearly the best fraction of the {\SecondBest} we can hope for {from any {\ksfair} mechanism} (and split all the gains between the two traders). Furthermore, while these mechanisms may not always be Pareto-optimal (with the Pareto-optimal solution being the Kalai-Smorodinsky solution), they imply that the KS solution is also guaranteed to obtain {at least} the same GFT approximation.




\subsection{Our Contributions} 
\label{sec:intro-our}
\label{subsec:contribution}
In this work, we study the efficiency of fair and truthful trade mechanisms. Below, we present
an overview of our contributions.

We focus on direct-revelation mechanisms, which specify an allocation (possibly randomized) and payments for each trader, for every valuation profile of the traders.\footnote{This is without loss of generality, by the revelation principle \citep{mye-81}.}
We restrict our attention to mechanisms that are interim individually rational (IIR), Bayesian incentive compatible (BIC) and ex-ante weak budget balance (ex-ante WBB),
\footnote{\label{footnote:reference to prelim}See \Cref{sec:prelim} for formal definitions.} and refer to such mechanisms as \emph{truthful mechanisms}.   
For general bilateral trade instances where both traders have private values sampled from overlapping distributions, the optimal expected GFT (also known as the {\FirstBest}) may not be achievable \citep{MS-83}. Thus, to understand the impact of fairness on truthful mechanisms, we analyze the GFT approximation of our proposed mechanism with respect to the {\SecondBest}, defined as the maximum GFT achievable by any truthful mechanism.\footnote{For zero-value seller instance, the {\FirstBest} and the {\SecondBest} are clearly equal. For general bilateral trade instance, it is known that the {\SecondBest} is at least ${1}/{3.15}$ fraction of the {\FirstBest} \citep{DMSW-22,Fei-22}.}
Truthful mechanisms that maximize the GFT might be complicated, even for simple distributions \citep{MS-83}. 
In contrast, all truthful mechanisms proposed in this work not only satisfy {\ksfairness} and achieve good GFT approximation to the {\SecondBest}, but they are also simple and easily implementable.


\xhdr{Optimal GFT approximation under {\ksfairness}.} 
As the first result of this work, we establish that the optimal GFT approximation of truthful mechanisms under {\ksfairness} is 50\%.

\begin{informal}[\Cref{thm:optimal GFT:general instance} and \Cref{lem:optimal GFT upper bound:irregular}]
\label{infmthm:general}
For every bilateral trade instance (i.e., any pair of seller and buyer distributions), there exists a truthful mechanism that is {\ksfair} and guarantees a GFT of at least 50\% of the {\SecondBest}.

Moreover, for any $\calC > 50\%$, there exists a zero-value seller instance in which no {\ksfair} truthful mechanism can achieve a 
$\calC$-fraction of the {\SecondBest}. 
\end{informal}
To obtain the positive approximation result, we develop a black-box reduction (\Cref{thm:blackbox reduction}) that converts any mechanism (possibly not {\ksfair}) into a {\ksfair} mechanism whose GFT is at least $\calC$-fraction of the sum of the traders’ ideal utilities, where $\calC$ is the smaller ratio between each trader's ex ante utility in the original mechanism, and her own ideal utility.
Our black-box reduction framework is both simple, general, and thus might be of independent interest.\footnote{In \Cref{appendix:bargaining-and-trade}, we generalize this framework to the cooperative bargaining problem.}
 
We apply our black-box reduction to analyze the {\BiasedRandomOffer}. 
This mechanism, for a given parameter $\mixprob\in[0,1]$, runs 
the {\SellerOffer} with probability $\mixprob$, and the {\BuyerOffer}\textsuperscript{\ref{footnote:reference to prelim}} with probability $1 - \mixprob$.
We show that, with an appropriately chosen 
$\mixprob$, this mechanism guarantees at least 50\% of the {\SecondBest} and is {\ksfair}. Notably, the proposed {\BiasedRandomOffer} is both simple and easy to implement. 
In particular, this mechanism is ex post IR and ex post strong budget balance (SBB), ensuring that all gains of the trade are split between the buyer and seller, leaving nothing to the mechanism. The (unbiased) {\RandomOffer}, which sets $\mixprob = 0.5$, already achieves a $\frac{1}{2}$-approximation to the {\SecondBest} \citep{BCWZ-17}. However, the {\RandomOffer} is generally not {\ksfair}. Our result shows that by carefully selecting $\mixprob$, we can preserve the same GFT approximation ratio while ensuring {\ksfairness}. 

To complete the picture, we also show that this GFT approximation ratio of $\frac{1}{2}$ is optimal among all {\ksfair} truthful mechanisms. To {prove} 
this, we present an explicit construction of an example
(\Cref{example:all fair mech:irregular}) with
zero-value seller and a buyer with value sampled from a distribution that we have carefully constructed to obtain
this tight bound.
Notably, this buyer distribution does not satisfy the regularity condition.
Regularity is a common assumption in the mechanism design literature \citep{mye-81,BR-89}, which holds for many classic distributions (e.g., Gaussian, exponential, uniform). In contrast to this bound of 50\%, as we have illustrated above, when the seller has {no value for the item} 
and the buyer's distribution is uniform between $[0, 1]$, posting a fixed trading price of $0.2$ is {\ksfair} and achieves 96\% of the {\SecondBest}. Motivated by this, we next consider settings where additional assumptions are imposed on the traders' distributions, and show that there are simple truthful mechanisms which are {\ksfair} and obtain much better GFT approximations. 

As a starting point, we study the bilateral trade instances where both traders' distributions satisfy the monotone hazard rate (MHR) condition.\footnote{The MHR condition, which is a {strengthening} of regularity, is also widely adopted in the mechanism design literature and satisfied by classic distributions such as exponential or uniform. See \Cref{sec:prelim} for the formal definition.} In this setting, we prove a stronger guarantee of $\frac{1}{e - 1} \geq 58.1\%$ for the {\ksfair} {\BiasedRandomOffer}, based on the results of \citep{Fei-22}. We next move to focus on the case of $0$-value seller, and prove stronger GFT approximation results.

\xhdr{Zero-value seller instances with regular or MHR distributions.}
In the second part of this work, we focus on the special case of the bilateral trade model where the seller has zero value for the item. This case is of particular interest as, while all gains can be realized by the simple and truthful mechanism which always trades at price $0$, such a mechanism is very unfair to the seller. On the other hand, letting the seller set the mechanism may result in arbitrarily small GFT (i.e., in the case of an equal-revenue distribution) or very unfair allocation (in the case of constant-value buyer).

In \Cref{infmthm:general} (\Cref{lem:optimal GFT upper bound:irregular}) we have shown that, even when the seller has no value for the item, the GFT approximation of $\frac{1}{2}$ cannot be improved when the buyer distribution is not regular. We next study the zero-value seller settings where the buyer's valuation distributions are regular or MHR.


We first consider the case that the buyer's valuation distribution is regular, and significantly improve the approximation of $50\%$ to more than $85\%$: 
\begin{informal}[\Cref{thm:improved GFT:regular buyer}]
\label{infmthm:regular buyer}
    For every zero-value seller instance where the buyer has a regular distribution, there exists a {\FixPrice} (which is truthful) that is {\ksfair}, and whose GFT is at least $\fixedPriceGFTPercentageRegular$ of the {\SecondBest}.

    Moreover, there exists a zero-value seller instance in which the buyer has a regular distribution and no {\ksfair} truthful mechanism obtains more than $\fixedPriceGFTPercentageUBRegular$ of the {\SecondBest}. 
\end{informal}



We remark that while the positive result is established by a {\FixPrice} (i.e., posting a trading price to two traders ex-ante),\footnote{{Interestingly, such a result cannot be obtained with a {\BiasedRandomOffer}. We show that} there exists a zero-value seller instance where the buyer has a regular distribution and yet, the {\ksfair} {\BiasedRandomOffer} only obtains 50\% of the {\SecondBest}, and not more.} the negative result holds for all {\ksfair} truthful mechanisms. Note that {\FixPrices} are the most simple mechanisms, and enjoy some excellent properties: not only they deterministic, they are also dominant strategy incentive compatible (DSIC), ex post IR and ex post SBB (so all GFT is split between the two traders, leaving nothing to the mechanism).

The almost-tight negative result of at most $\fixedPriceGFTPercentageUBRegular$ approximation is proven by presenting an explicit construction of an example (\Cref{example:BROM:regular}) with a zero-value seller and a buyer with value sampled from a regular distribution, and analyzing it (\Cref{lem:GFT UB:regular buyer}). For the positive result, we develop a novel argument based on a \emph{revenue curve reduction} analysis (\Cref{lem:GFT program:regular buyer}). As this analysis is our most significant technical contribution we discuss it in more details after presenting the rest of our results.

We next study the zero-value seller instance where the buyer's valuation distribution is MHR.  
\begin{informal}[\Cref{thm:improved GFT:mhr buyer}]
\label{infmthm:mhr buyer}
    For every zero-value seller instance where the buyer has an MHR distribution, there exists a {\FixPrice} (which is  truthful) that is {\ksfair}, and whose GFT is at least $\fixedPriceGFTPercentageMHR$ of the {\SecondBest}.

    Moreover, there exists a zero-value seller instance in which the buyer has an MHR distribution and no {\ksfair} truthful mechanism obtains more than $\fixedPriceGFTPercentageUBMHR$ of the {\SecondBest}. 
\end{informal}
The almost-tight negative result of at most $\fixedPriceGFTPercentageUBMHR$ approximation is proven by presenting an explicit construction of an example (\Cref{example:all fair:mhr buyer}) with a zero-value seller and a buyer with value sampled from an MHR distribution, and analyzing it (\Cref{lem:GFT UB:mhr buyer}). For the positive result, we conduct a similar argument (\Cref{lem:GFT program:mhr buyer}) as the one used for regular distributions, but now the argument centers on the cumulative hazard rate function instead of the revenue curve. 



\xhdr{Implication for the KS-solution.}
While all of our results above (Informal Theorems~\labelcref{infmthm:general,infmthm:regular buyer,infmthm:mhr buyer}) are stated for {\ksfair} truthful mechanisms, the negative results trivially apply to the KS-solution as well, since it satisfies {\ksfairness}. Importantly, it is worth noting that all the positive results also hold for the KS-solution. This follows from the fact that the simple {\ksfair} mechanisms we proposed ({\BiasedRandomOffer} and {\FixPrice}) are ex post SBB, and therefore their GFT is at most the GFT of the KS-solution. See \Cref{lem:SBB implication} for the formal statement.\footnote{Though the KS-solution maximizes the sum of the two traders' utilities among all {\ksfair} mechanisms, it does not directly imply that the KS-solution also maximizes the GFT among all {\ksfair} mechanisms, since the GFT {not only includes the utilities of the two  traders, but also includes} the gains left to the mechanism (which could be positive under WBB).}


\xhdr{Implication for market regulation.} 
Our results also shed light on the following connection between efficiency and fairness regulation. Suppose the seller (resp.\ buyer) is a monopolist and can freely decide on the truthful mechanism to maximize her own utility. In this case, the GFT from the mechanism picked by the monopolist can be arbitrarily smaller than the {\SecondBest}. In contrast, consider an alternative scenario where a regulator imposes a regulation so that the monopolist may only choose a truthful mechanism that is {\ksfair}. In \Cref{lem:SBB implication}, we show that the seller-optimal (resp.\ buyer-optimal) {\ksfair} mechanism achieves the same GFT approximation as the ones stated in Informal Theorems~\labelcref{infmthm:general,infmthm:regular buyer,infmthm:mhr buyer}, so high GFT is guaranteed when fairness is imposed.


\xhdr{Alternative fairness definitions.} Besides {\ksfairness}, we also explore alternative fairness definitions for the bilateral trade model.

We first explore another solution concept for the bargaining problem -- the Nash solution \citep{nash-51}. In the context of the bilateral trade model, the Nash solution corresponds to a {\NashSocialWelfareMaximizer}: a truthful mechanism that maximizes the Nash social welfare (NSW), i.e., the product of the two traders' ex ante utilities. Following an argument that is conceptually similar to our black-box reduction framework (for {\ksfair} mechanisms), we obtain a tight bound on the GFT approximation of any {\NashSocialWelfareMaximizer}.

\begin{informal}
    For every bilateral trade instance, a {\NashSocialWelfareMaximizer} guarantees a GFT of at least 50\% of the {\SecondBest}.

    Moreover, for any $\calC > 50\%$, there exists a zero-value seller instance in which no {\NashSocialWelfareMaximizer} can achieve a
    $\calC$-fraction of the {\SecondBest}.
\end{informal}
While our definition of {\ksfairness} aims to explicitly define fairness for bilateral trade, the  fairness properties of the Nash solution are rather implicit (it is more about subscribing a way to trade-off the traders' utilities). We also remark that a {\NashSocialWelfareMaximizer} could have complicated allocation and payment rules. In contrast, both the {\BiasedRandomOffer} and {\FixPrice} which we proposed and analyzed for {\ksfairness} are simple and easy to be implemented. Hence, we believe {\ksfairness} may be more suitable for the bilateral trade problem, and our paper mainly focuses on it.


As we mentioned earlier, we also establish negative results showing that equitability (motivated by the egalitarian solution \citep{Kal-77,Mye-77} to the bargaining problem) and interim or ex post {\ksfairness} (\Cref{def:interim ks fairness,def:ex post ks fairness}) may not be appropriate. Specifically, we show that the GFT of truthful mechanisms that satisfy any of those alternative fairness definitions, can be arbitrary lower than the {\SecondBest} (\Cref{lem:equitable GFT UB}) or imply no trade and thus zero GFT (\Cref{prop:interim/ex post fairness:no trade}), even in settings with a zero-value seller.

\subsection{Our Techniques} 
\label{subsec:intro:techniques}
We next describe the technical framework we put forward for proving our positive results for instances with a zero-value seller and a buyer with either a regular or an MHR valuation distribution (Informal Theorems~\labelcref{infmthm:regular buyer,infmthm:mhr buyer}, respectively), {and the novelty of our technique}. We start by explaining the high-level proof idea behind \Cref{infmthm:regular buyer}. To obtain the almost-tight positive result of at least $\fixedPriceGFTPercentageRegular$, we aim to directly compare a {\ksfair} {\FixPrice} with the {\SecondBest}.\footnote{As a comparison, in the analysis of the positive result in \Cref{infmthm:general}, we compare the GFT of {\ksfair} mechanisms with the summation of two traders' ideal utilities (which upper bounds the {\SecondBest}). Consider a simple example, where the seller and buyer have deterministic value of zero and one, respectively. In this example, the summation of two traders' ideal utilities could  be twice the {\FirstBest} (and thus at least twice the GFT of any truthful mechanism)}. Hence, by comparing with the summation of two traders' ideal utilities, it is impossible to obtain an approximation ratio strictly better than $\frac{1}{2}$. To do this, we develop a novel revenue curve reduction analysis. 

In the mechanism design literature, a revenue-curve-based analysis is a common approach to deriving approximation guarantees \citep[e.g.,][]{FILS-15,DRY-15,AB-18,AHNPY-18,JLQTX-19,BCW-22,FHL-21,ABB-22,JL-23}. Revenue curves \citep{BR-89} give an equivalent representation of the buyer's valuation distribution and enable clean characterization of the revenue of a given mechanism. Though there does not exist an automatic way to conduct a revenue-curve-based analysis, the common high-level idea is to argue that the worst approximation ratio among the class of regular distributions belongs to a subclass of distributions that can be characterized with a small set of parameters (e.g., monopoly reserve, quantile, or revenue). Most prior work focuses on revenue approximation and allows consideration of a class of mechanisms (e.g., anonymous pricing with all possible prices). In this way, the approximation ratio usually depends on a small set of parameters of the revenue curve, and thus the reduction argument can be established.

In contrast, our analysis requires studying the seller's ex-ante utility (revenue), the buyer's ex-ante utility (residual surplus) and the GFT (expected value) -- all three of these \emph{together}. Furthermore, we focus on a single mechanism ({\ksfair} {\FixPrice}). Hence, both the GFT benchmark and the GFT of our mechanism are highly sensitive to the \emph{entire} revenue curve {(e.g., parameters like the monopoly reserve and revenue are not enough)}.

To overcome this challenge, we conduct a three-step argument (see \Cref{lem:GFT program:regular buyer}). We first consider a {\FixPrice} (that is possibly not {\ksfair})  whose trading price $\price$ is selected based on a \emph{small set} of parameters of the revenue curve. 
In the second step, we argue that depending on buyer's ex-ante utility under trading price $\price$, we can modify this price to obtain a {\ksfair} trading price $\fprice$. By comparing the two traders' utility changes from trading price $\price$ to $\fprice$, we obtain a lower bound of the GFT approximation under trading price $\fprice$, which is {\ksfair}. Importantly, this lower bound is a {function of the trading price $\price$ chosen in the first step}. Loosely speaking, using the first and second step, we show that the worst GFT approximation lower bound belongs to a \emph{subclass of regular distributions} parameterized by trading price $\price$ and several other parameters including monopoly quantile and monopoly revenue. Finally, since this GFT lower bound depends on the choice of $\price$ determined in the first step, we formulate the problem of selecting the best price~$\price$ (to maximize the GFT approximation lower bound) as a \emph{two-player zero-sum game}. In this game, the ``min player'' (adversary) selects the revenue curve (represented by finitely many quantities on the revenue curve), while the ``max player'' (ourselves) selects the price $\price$. The game's payoff is the GFT approximation lower bound derived in the second step. Finally, we capture this two-player zero-sum game as an \emph{optimization program~\ref{program:GFT:regular buyer}} {(a minimization problem)} and numerically lower bound the optimal payoff of this game (i.e., optimal objective value of \ref{program:GFT:regular buyer}) by $\fixedPriceGFTPercentageRegular$.\footnote{Formulating the approximation guarantees as optimization program and then solve it numerically is common in the mechanism design literature. See the related work section (\Cref{subsec:related work}) for more discussion.} Also see \Cref{fig:GFT program:regular buyer} for a graphical illustration. 

The positive result of \Cref{infmthm:mhr buyer} relies on a similar reduction argument. However, instead of conducting the reduction on the revenue curve, we analyze the \emph{cumulative hazard rate function}, which is another equivalent representation of the buyer's valuation distribution. In particular, the buyer's valuation distribution is MHR if and only if the induced cumulative hazard rate function is convex. In our analysis (\Cref{lem:GFT program:mhr buyer} and \Cref{fig:GFT program:mhr buyer}), we formulate the GFT approximation of a {\ksfair} {\FixPrice} as a %n
minimization program~\ref{program:GFT:mhr buyer}. We then numerically lower bound its optimal objective value by $\fixedPriceGFTPercentageMHR$.


\subsection{Related Work}
\label{subsec:related work}

%\section{Related Work}
%\label{sec:related-work}

%\subsection{Background}

%Defect detection is critical to ensure the yield of integrated circuit manufacturing lines and reduce faults. Previous research has primarily focused on wafer map data, which engineers produce by marking faulty chips with different colors based on test results. The specific spatial distribution of defects on a wafer can provide insights into the causes, thereby helping to determine which stage of the manufacturing process is responsible for the issues. Although such research is relatively mature, the continual miniaturization of integrated circuits and the increasing complexity and density of chip components have made chip-level detection more challenging, leading to potential risks\cite{ma2023review}. Consequently, there is a need to combine this approach with magnified imaging of the wafer surface using scanning electron microscopes (SEMs) to detect, classify, and analyze specific microscopic defects, thus helping to identify the particular process steps where defects originate.

%Previously, wafer surface defect classification and detection were primarily conducted by experienced engineers. However, this method relies heavily on the engineers' expertise and involves significant time expenditure and subjectivity, lacking uniform standards. With the ongoing development of artificial intelligence, deep learning methods using multi-layer neural networks to extract and learn target features have proven highly effective for this task\cite{gao2022review}.

%In the task of defect classification, it is typical to use a model structure that initially extracts features through convolutional and pooling layers, followed by classification via fully connected layers. Researchers have recently developed numerous classification model structures tailored to specific problems. These models primarily focus on how to extract defect features effectively. For instance, Chen et al. presented a defect recognition and classification algorithm rooted in PCA and classification SVM\cite{chen2008defect}. Chang et al. utilized SVM, drawing on features like smoothness and texture intricacy, for classifying high-intensity defect images\cite{chang2013hybrid}. The classification of defect images requires the formulation of numerous classifiers tailored for myriad inspection steps and an Abundance of accurately labeled data, making data acquisition challenging. Cheon et al. proposed a single CNN model adept at feature extraction\cite{cheon2019convolutional}. They achieved a granular classification of wafer surface defects by recognizing misclassified images and employing a k-nearest neighbors (k-NN) classifier algorithm to gauge the aggregate squared distance between each image feature vector and its k-neighbors within the same category. However, when applied to new or unseen defects, such models necessitate retraining, incurring computational overheads. Moreover, with escalating CNN complexity, the computational demands surge.

%Segmentation of defects is necessary to locate defect positions and gather information such as the size of defects. Unlike classification networks, segmentation networks often use classic encoder-decoder structures such as UNet\cite{ronneberger2015u} and SegNet\cite{badrinarayanan2017segnet}, which focus on effectively leveraging both local and global feature information. Han Hui et al. proposed integrating a Region Proposal Network (RPN) with a UNet architecture to suggest defect areas before conducting defect segmentation \cite{han2020polycrystalline}. This approach enables the segmentation of various defects in wafers with only a limited set of roughly labeled images, enhancing the efficiency of training and application in environments where detailed annotations are scarce. Subhrajit Nag et al. introduced a new network structure, WaferSegClassNet, which extracts multi-scale local features in the encoder and performs classification and segmentation tasks in the decoder \cite{nag2022wafersegclassnet}. This model represents the first detection system capable of simultaneously classifying and segmenting surface defects on wafers. However, it relies on extensive data training and annotation for high accuracy and reliability. 

%Recently, Vic De Ridder et al. introduced a novel approach for defect segmentation using diffusion models\cite{de2023semi}. This approach treats the instance segmentation task as a denoising process from noise to a filter, utilizing diffusion models to predict and reconstruct instance masks for semiconductor defects. This method achieves high precision and improved defect classification and segmentation detection performance. However, the complex network structure and the computational process of the diffusion model require substantial computational resources. Moreover, the performance of this model heavily relies on high-quality and large amounts of training data. These issues make it less suitable for industrial applications. Additionally, the model has only been applied to detecting and segmenting a single type of defect(bridges) following a specific manufacturing process step, limiting its practical utility in diverse industrial scenarios.

%\subsection{Few-shot Anomaly Detection}
%Traditional anomaly detection techniques typically rely on extensive training data to train models for identifying and locating anomalies. However, these methods often face limitations in rapidly changing production environments and diverse anomaly types. Recent research has started exploring effective anomaly detection using few or zero samples to address these challenges.

%Huang et al. developed the anomaly detection method RegAD, based on image registration technology. This method pre-trains an object-agnostic registration network with various images to establish the normality of unseen objects. It identifies anomalies by aligning image features and has achieved promising results. Despite these advancements, implementing few-shot settings in anomaly detection remains an area ripe for further exploration. Recent studies show that pre-trained vision-language models such as CLIP and MiniGPT can significantly enhance performance in anomaly detection tasks.

%Dong et al. introduced the MaskCLIP framework, which employs masked self-distillation to enhance contrastive language-image pretraining\cite{zhou2022maskclip}. This approach strengthens the visual encoder's learning of local image patches and uses indirect language supervision to enhance semantic understanding. It significantly improves transferability and pretraining outcomes across various visual tasks, although it requires substantial computational resources.
%Jeong et al. crafted the WinCLIP framework by integrating state words and prompt templates to characterize normal and anomalous states more accurately\cite{Jeong_2023_CVPR}. This framework introduces a novel window-based technique for extracting and aggregating multi-scale spatial features, significantly boosting the anomaly detection performance of the pre-trained CLIP model.
%Subsequently, Li et al. have further contributed to the field by creating a new expansive multimodal model named Myriad\cite{li2023myriad}. This model, which incorporates a pre-trained Industrial Anomaly Detection (IAD) model to act as a vision expert, embeds anomaly images as tokens interpretable by the language model, thus providing both detailed descriptions and accurate anomaly detection capabilities.
%Recently, Chen et al. introduced CLIP-AD\cite{chen2023clip}, and Li et al. proposed PromptAD\cite{li2024promptad}, both employing language-guided, tiered dual-path model structures and feature manipulation strategies. These approaches effectively address issues encountered when directly calculating anomaly maps using the CLIP model, such as reversed predictions and highlighting irrelevant areas. Specifically, CLIP-AD optimizes the utilization of multi-layer features, corrects feature misalignment, and enhances model performance through additional linear layer fine-tuning. PromptAD connects normal prompts with anomaly suffixes to form anomaly prompts, enabling contrastive learning in a single-class setting.

%These studies extend the boundaries of traditional anomaly detection techniques and demonstrate how to effectively address rapidly changing and sample-scarce production environments through the synergy of few-shot learning and deep learning models. Building on this foundation, our research further explores wafer surface defect detection based on the CLIP model, especially focusing on achieving efficient and accurate anomaly detection in the highly specialized and variable semiconductor manufacturing process using a minimal amount of labeled data.


