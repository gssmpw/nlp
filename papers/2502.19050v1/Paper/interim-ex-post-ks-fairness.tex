Recall that the definition of {\ksfairness} (\Cref{def:ks fairness}) imposes a condition on the ex-ante utilities of the two traders. Specifically, it requires that the ex-ante utilities of both traders achieve the same fraction of their respective benchmarks. In this section, we explore two variants of {\ksfairness}, where the condition is modified to apply at the interim stage (\Cref{def:interim ks fairness}) or the ex post stage (\Cref{def:ex post ks fairness}), as defined below.

As demonstrated in \Cref{prop:interim/ex post fairness:no trade}, there exist simple instances in which any BIC, IIR, and ex-ante WBB mechanism that satisfies either interim {\ksfairness} or ex post {\ksfairness} results in the {\NoTrade} mechanism (i.e., one where trade never occurs). 
Thus, these definitions are too stringent, justifying our focus on the ex ante definition of fairness ({\ksfairness}, as defined in \Cref{def:ks fairness}).


\begin{definition}[Interim {\ksfairness}]
\label{def:interim ks fairness}
    For bilateral trade instance $\btinstance$, mechanism $\mech\in \mechfam$ is \emph{interim {\ksfair}}, if the two traders' interim utilities for any pair of value realizations $(\val, \cost)$ achieve the same {fraction of} each trader's own interim benchmark $\sellerbenchmark(\cost)$ and $\buyerbenchmark(\val)$, i.e.,
    \begin{align*}
        \frac{\sellerutil(\cost)}{\sellerbenchmark(\cost)}
        = \frac{\buyerutil(\val)}{\buyerbenchmark(\val)}
    \end{align*}
    where $\sellerutil(\cost)$ (resp.\ $\buyerutil(\val)$) is the seller's (resp.\ buyer's) interim utility given value realization $\cost$ (resp.\ $\val)$ in mechanism $\mech$. Benchmark $\sellerbenchmark(\cost)$ is defined as the seller's interim utility in the BIC, IIR, and ex ante WBB mechanism that maximizes that utility (when she has a deterministic value $\cost$), when facing a buyer with distribution $\buyerdist$. Similarly, benchmark $\buyerbenchmark(\val)$ is defined as the buyer's interim utility in the BIC, IIR, and ex ante WBB mechanism that maximizes that utility (when she has a deterministic value $\val$), when facing a seller with distribution $\sellerdist$.
\end{definition}

\begin{definition}[Ex post {\ksfairness}]
\label{def:ex post ks fairness}
    For bilateral trade instance $\btinstance$, mechanism $\mech\in \mechfam$ is \emph{ex post {\ksfair}}, if the two traders' ex post utilities for any pair of value realizations $(\val, \cost)$ achieve the same {fraction of} each trader's own ex post benchmark $\sellerbenchmark(\val, \cost)$ and $\buyerbenchmark(\val, \cost)$, i.e.,
    \begin{align*}
        \frac{\sellerutil(\val, \cost)}{\sellerbenchmark(\val, \cost)}
        = \frac{\buyerutil(\val, \cost)}{\buyerbenchmark(\val,\cost)}
    \end{align*}
    where $\sellerutil(\val, \cost)$ (resp.\ $\buyerutil(\val, \cost)$) is the seller's (resp.\ buyer's) ex post utility given value realization $\val$ and $\cost$ in mechanism $\mech$. Benchmark $\sellerbenchmark(\val, \cost)$ is defined as the seller's ex post in the BIC, IIR, and ex ante WBB mechanism that maximizes that utility (when she has a deterministic value $\cost$), when facing a buyer with a deterministic value $\val$. Similarly, benchmark $\buyerbenchmark(\val,\cost)$ is defined as the buyer's ex post utility in the BIC, IIR, and ex ante WBB mechanism that maximizes that utility (when she has a deterministic value $\val$), when facing a seller with a deterministic value $\cost$.
\end{definition}

We remark that the interim {\ksfairness} (\Cref{def:interim ks fairness}) implies the ex ante {\ksfairness} (\Cref{def:ks fairness}). To see this, note that in both fairness definition, the seller's ex ante benchmark $\sellerbenchmark$ and interim benchmark $\sellerbenchmark(\cost)$ are consistent, i.e., $\sellerbenchmark = \expect[\cost]{\sellerbenchmark(\cost)}$, and the same consistency holds for the buyer's benchmarks, i.e., $\buyerbenchmark = \expect[\val]{\buyerbenchmark(\val)}$. Following the same reason, we observe that the ex post {\ksfairness} does not imply interim or ex ante {\ksfairness}, since its benchmark may be different, i.e., $\sellerbenchmark \not= \expect[\val,\cost]{\sellerbenchmark(\val,\cost)}$. 

We next present our result for these fairness notions, showing that for non-degenerate instances, only the {\NoTrade} satisfy the interim or ex post {\ksfairness}. The proof is based on studying the following instance.


\begin{example}
\label{example:interim fairness:zero-value seller uniform buyer}
The buyer has valuation distribution $\buyerdist$ with support $\supp(\buyerdist) = [\lval, \hval]$, where $0\leq \lval < \hval$. The seller has a deterministic value of zero.
\end{example}

\begin{lemma}
\label{prop:interim/ex post fairness:no trade}
    In \Cref{example:interim fairness:zero-value seller uniform buyer}, if a BIC, IIR, ex ante WBB mechanism $\mech$ is interim {\ksfair} or ex post {\ksfair}, then it must be the {\NoTrade} (which has zero GFT). 
\end{lemma}
\begin{proof}
    We first consider the case where mechanism $\mech = (\alloc,\price,\sellerprice)$ is interim {\ksfair}. 
    Since mechanism $\mech$ is BIC and IIR, the buyer's interim utility $\buyerutil(\val)$ for every value $\val\in[\lval,\hval]$ can be expressed as 
    $\buyerutil(\val) = \buyerutil(\lval) + \int_{\lval}^\val \alloc(t)\cdot \d t$.
    Meanwhile, since the seller has deterministic value of zero, the buyer's interim benchmark $\buyerbenchmark(\val) = \val$.
    Invoking the assumption that mechanism $\mech$ is interim {\ksfair}, for every value $\val\in[\lval,\hval]$ of the buyer, we know 
    $\frac{\buyerutil(\val)}{\buyerbenchmark(\val)}
        =
        \frac{\sellerutil(0)}{\sellerbenchmark(0)}$.
    Putting all the three pieces together, we know that for every value $\val\in[\lval,\hval]$,
    \begin{align*}
        \frac{1}{\val}\cdot \left(\buyerutil(\lval) + \displaystyle\int_{\lval}^\val \alloc(t)\cdot \d t \right)= 
        \frac{\sellerutil(0)}{\sellerbenchmark(0)} 
        \;\;
        \Longleftrightarrow
        \;\;
        \left(\buyerutil(\lval) + \displaystyle\int_{\lval}^\val \alloc(t)\cdot \d t \right)= 
        \frac{\sellerutil(0)}{\sellerbenchmark(0)} 
        \cdot \val
    \end{align*}
    Taking the derivative on both sides, we obtain that 
    $\alloc(\val) = \frac{\sellerutil(0)}{\sellerbenchmark(0)}$ for every value $\val\in[\lval,\hval]$ and $\buyerutil(\lval) = \frac{\sellerutil(0)}{\sellerbenchmark(0)}\cdot \lval$.
    Consequently, we obtain $\buyerutil(\val) = \frac{\sellerutil(0)}{\sellerbenchmark(0)}\cdot \val$ and $\price(\val) = 0$ for every value $\val\in[\lval,\hval]$.
    Now consider the seller's (interim) utility $\sellerutil(0)$, which can be upper bounded by
    \begin{align*}
        \sellerutil(0) = \sellerprice(0)\leq \expect[\val]{\price(\val)} 
        = 0
    \end{align*}
    where the inequality holds due to ex ante WBB, the second equality holds due to $\price(\val) = 0$ for every value $\val$ argued above.
    Finally, since $\hval > \lval \geq 0$, the seller's (interim) benchmark $\sellerbenchmark(0) > 0$, and thus for every value $\val \in[\lval, \hval]$:
    \begin{align*}
        \alloc(\val) = \frac{\sellerutil(0)}{\sellerbenchmark(0)} = 0 
    \end{align*}
    which implies mechanism $\mech$ is the {\NoTrade}.
    
    Next we assume mechanism $\mech$ is ex post {\ksfair}. Since the seller has a deterministic value of zero, the buyer's interim allocation is equivalent to her ex post allocation. Hence, following the same argument, we obtain $\alloc(\val, 0) = \frac{\sellerutil(\hval,0)}{\sellerbenchmark(\hval,0)} = 0$ (due to the ex post {\ksfairness}), and thus mechanism $\mech$ is the {\NoTrade}. This completes the proof of \Cref{prop:interim/ex post fairness:no trade}.
\end{proof}