
In this section we present tight results regarding the fraction of the {\SecondBest} that is obtained when a mechanism (that is  BIC, IIR, and ex ante WBB) maximizes the ex ante \emph{Nash social welfare (NSW)}. The NSW is defined as the product of the two traders' ex-ante utilities. We refer to such a mechanism as a \emph{\NashSocialWelfareMaximizer} ({\NSWM}).

\begin{definition}
    For bilateral trade instance $\btinstance$, mechanism $\mech\in \mechfam$ is called an {\NashSocialWelfareMaximizer} if it maximizes the ex ante Nash social welfare among all BIC, IIR, ex-ante WBB mechanisms, i.e.,
    \begin{align*}
         \mech
         \in \argmax\limits_{\mech\primed\in\mechfam}~\sellerexanteutil(\mech\primed)\cdot \buyerexanteutil(\mech\primed)
    \end{align*}
\end{definition}

There might be different {\NashSocialWelfareMaximizers} (depending on how ties are broken) --- our results will hold for all of them. 

\begin{theorem}
\label{thm:NSWM GFT}
For every bilateral trade instance, the GFT of every {\NashSocialWelfareMaximizer} is at least $\frac{1}{2}$ fraction of the {\SecondBest} $\OPTSB$.
    
Moreover, for any $\eps>0$ there exists a distribution $\buyerdist$ (that is not regular) such that in the bilateral setting with a zero-value seller and a buyer with valuation distribution $\buyerdist$, every {\NashSocialWelfareMaximizer} obtains at most $\frac{1}{2}+\eps$ fraction of the {\SecondBest} $\OPTSB$.
\end{theorem}

\Cref{thm:NSWM GFT} directly follows \Cref{lem:NSWM trader approx,lem:NSWM GFT UB:general instance} which prove the positive result and negative result in the theorem statement, respectively. 

\subsection{GFT-Approximation by {\NashSocialWelfareMaximizer}}

We characterize the GFT approximation of the {\NashSocialWelfareMaximizer} as follows.


\begin{lemma}
\label{lem:NSWM trader approx}
    For every bilateral trade instance, the ex ante utility of each trader in every {\NashSocialWelfareMaximizer} $\mech$ is at least $\frac{1}{2}$ fraction of her benchmark, i.e.,
    \begin{align*}
        \sellerexanteutil(\mech) \geq \frac{1}{2} \cdot \sellerbenchmark
        \;\;\mbox{and}\;\;
        \buyerexanteutil(\mech) \geq \frac{1}{2} \cdot \buyerbenchmark  
    \end{align*}
    Therefore, the GFT of {\NashSocialWelfareMaximizer} $\mech$ is at least $\frac{1}{2}$ fraction of th {\SecondBest}. 
\end{lemma}
\begin{proof}
    Fix any {\NashSocialWelfareMaximizer} $\mech$. 
    We prove the lemma statement by a contradiction argument. 
    Suppose $\frac{\sellerexanteutil(\mech)}{\sellerbenchmark} < \frac{1}{2}$ for the seller. (The other case for the buyer follows a symmetric argument.) Define auxiliary notation $\revratio \triangleq \frac{\sellerexanteutil(\mech)}{\sellerbenchmark} < \frac{1}{2}$.

    Consider a new mechanism $\mech\primed$ that runs the {\SellerOffer} with probability $\frac{1}{2} - \frac{\revratio}{2 - 2 \revratio}$, and runs mechanism $\mech$ with probability $\frac{1}{2} + \frac{\revratio}{2 - 2 \revratio}$. Since $\revratio < \frac{1}{2}$, mechanism $\mech\primed$ is well-defined. By construction, the ex ante utility of each trader in mechanism $\mech\primed$ can be computed as  
    \begin{align*}
        \sellerexanteutil(\mech\primed) &{} = \left(\frac{1}{2} - \frac{\revratio}{2 - 2 \revratio}\right) \cdot \sellerbenchmark + \left(\frac{1}{2} + \frac{\revratio}{2 - 2 \revratio}\right) \cdot \sellerexanteutil(\mech)
        \\
        &{} \overset{(a)}{=} 
        \left(\frac{1}{2} - \frac{\revratio}{2 - 2 \revratio}\right) \cdot \frac{1}{\revratio}\cdot \sellerexanteutil(\mech) + \left(\frac{1}{2} + \frac{\revratio}{2 - 2 \revratio}\right) \cdot \sellerexanteutil(\mech)    
        \\
        \buyerexanteutil(\mech\primed) &{} \geq 
        \left(\frac{1}{2} + \frac{\revratio}{2 - 2 \revratio}\right) \cdot \buyerexanteutil(\mech)   
    \end{align*}
    where equality~(a) holds since $\revratio = \frac{\sellerexanteutil(\mech)}{\sellerbenchmark}$. Thus, the Nash social welfare of mechanism $\mech\primed$ is 
    \begin{align*}
        &\left(\left(\frac{1}{2} - \frac{\revratio}{2 - 2 \revratio}\right) \cdot \frac{1}{\revratio} + \left(\frac{1}{2} + \frac{\revratio}{2 - 2 \revratio}\right)
        \right)
        \cdot 
        \sellerexanteutil(\mech)
        \cdot 
        \left(\frac{1}{2} + \frac{\revratio}{2 - 2 \revratio}\right) \cdot \buyerexanteutil(\mech)
        \\
        ={} &
        \frac{1}{4(1-\revratio)\revratio}
        \cdot \sellerexanteutil(\mech) \cdot 
        \buyerexanteutil(\mech)
        > {} 
        \sellerexanteutil(\mech) \cdot 
        \buyerexanteutil(\mech)
    \end{align*}
    where the equality holds by algebra and the strict inequality holds since $\revratio < \frac{1}{2}$. The Nash social welfare of mechanism $\mech\primed$ is strictly higher than the {\NashSocialWelfareMaximizer} $\mech$, which is a contradiction. This completes the proof of the first claim about traders' ex ante utility in the lemma statement. Invoking the fact that the {\SecondBest} $\OPTSB$ is upper bounded by $\sellerbenchmark + \buyerbenchmark$, we finish the whole analysis of \Cref{lem:NSWM trader approx} as desired.
\end{proof}

\subsection{Tight Negative Result of {\NashSocialWelfareMaximizer} for Zero-Value Seller}
We next show that even for the simple setting where the seller has zero value,
there exists an instance (\Cref{example:all fair mech:irregular}) in which no {\NashSocialWelfareMaximizer} can obtain more than half of the {\SecondBest} $\OPTSB$. We note that this is the same example which shows the tight negative result for all BIC, IIR, ex ante WBB, {\ksfair} mechanisms in \Cref{thm:optimal GFT:general instance} and \Cref{lem:optimal GFT upper bound:irregular}.

\begin{lemma}
\label{lem:NSWM GFT UB:general instance}
    Fix any $\eps > 0$.
    In \Cref{example:all fair mech:irregular} with sufficiently large $\constantH$ (as a function of $\eps$), every {\NashSocialWelfareMaximizer} $\mech$ obtains at most $(\frac{1}{2} + \eps)$ fraction of the {\SecondBest} $\OPTSB$, i.e., $\GFT{\mech} \leq (\frac{1}{2} + \eps) \cdot \OPTSB$. 
\end{lemma}

\begin{proof}
    In \Cref{example:all fair mech:irregular}, the {\SecondBest} $\OPTSB$ can be computed as
    \begin{align*}
        \OPTSB = \expect[\val]{\val} =
        \displaystyle\int_1^{\constantH} \val\cdot \d \buyercdf(\val)
        + \constantH\cdot (1 - \buyercdf(\constantH))
        =
        (1 - o_{}(1))\cdot \ln \constantH
    \end{align*}
    Moreover, the seller's benchmark can be computed as $
        \sellerbenchmark = \sqrt{\ln \constantH}$.
    Fix any {\NashSocialWelfareMaximizer} $\mech = (\alloc,\price,\sellerprice)$. Invoking \Cref{lem:NSWM trader approx}, we know $
        \frac{\sellerexanteutil(\mech)}{\sellerbenchmark} \geq \frac{1}{2}$.
    We next upper bound the seller's ex ante utility $\sellerexanteutil(\mech)$ as follows
    \begin{align*}
        \sellerexanteutil(\mech) &{} = 
        \sellerprice(0) \overset{(a)}{\leq} 
        \expect[\val]{\price(\val)}
        \overset{(b)}{=}
        \expect[\val]{\virtualval(\val)\cdot \alloc(\val)}
        \\
        &{} 
        \overset{(c)}{\leq} 
        \virtualval(\constantH)\cdot (1 - \buyercdf(\constantH))
        +
        \virtualval(\val\primed)\cdot (\buyercdf(\constantH) - \buyercdf(\val\primed)) \cdot \alloc(\val\primed)
        % \\
        % &{}
        =
        \left(1 - \alloc(\val\primed) + o_{}(1)\right)\cdot \sqrt{\ln\constantH}
        % \\
        % &{} 
    \end{align*}
    where inequality~(a) holds since mechanism $\mech$ is ex ante WBB, 
    equality~(b) holds due to \Cref{prop:revenue equivalence}, and
    inequality~(c) holds since the virtual value function $\virtualval$ satisfies $\virtualval(\constantH) = \constantH \geq 0$, $\virtualval(\val) = \virtualval(\val\primed) < 0$ for $\val\in[\val\primed,\constantH)$, $\virtualval(\val) = 0$ for $\val\in[1, \val\primed)$, and the buyer's interim allocation $\alloc$ is weakly increasing due to BIC. Putting the two pieces together, we know 
    \begin{align*}
        \frac{1}{2} \leq \frac{\sellerexanteutil(\mech)}{\sellerbenchmark}
        \leq \frac{        \left(1 - \alloc(\val\primed) + o_{}(1)\right)\cdot \sqrt{\ln\constantH}}{\sqrt{\ln \constantH}}
    \end{align*}
    which implies $\alloc(\val\primed) \leq \frac{1}{2} + o_{}(1)$. Thus, the GFT of mechanism $\mech$ can be upper bounded as 
    \begin{align*}
        \GFT{\mech} & = \expect[\val]{\val\cdot \alloc(\val)} 
        \leq
        \displaystyle\int_1^{\val\primed} \val\cdot \alloc(\val)\cdot \d \buyercdf(\val)
        +
        \displaystyle\int_{\val\primed}^{\constantH} \val\cdot \d \buyercdf(\val)
        + \constantH\cdot (1 - \buyercdf(\constantH))
        \\
        &{}=
        \displaystyle\int_1^{\val\primed} \val\alloc(\val)\cdot \d \buyercdf(\val)
        +
        o_{}(\ln\constantH)
        \overset{(a)}{\leq} 
        \displaystyle\int_1^{\val\primed} \val\cdot \alloc(\val\primed)\cdot \d \buyercdf(\val)
        +
        o_{}(\ln\constantH)
        \\
        & {} =
        (\alloc(\val\primed) - o_{}(1))\cdot \ln\constantH
        \leq 
        \left(\frac{1}{2} + o_{}(1)\right)\cdot \ln\constantH
    \end{align*}
    where inequality~(a) holds since mechanism $\mech$ is BIC and thus interim allocation $\alloc(\val)$ is weakly increasing in $\val$.
    Combining with the fact that the {\SecondBest} $\OPTSB = (1 - o_{}(1))\cdot \ln\constantH$ argued above, we finish the proof of \Cref{lem:NSWM GFT UB:general instance}.
\end{proof}
