
In this section, we show that for general bilateral instances, there exists a simple mechanism (the {\BiasedRandomOffer}) that is {\ksfair} and has GFT approximation of $\frac{1}{2}$, and that fraction can be improved to $\frac{1}{e-1}$ when the distributions are MHR. The {\BiasedRandomOffer} that we use is BIC, ex post IR, and ex post SBB. 
\begin{definition}[{\BROM}]
    \label{def:biased random offer}
    In the {\BiasedRandomOffer}, the social planner {fixes} a probability $\mixprob\in[0, 1]$ {ex ante}. Then, the {\SellerOffer} is implemented with probability $\mixprob$, and the {\BuyerOffer} is implemented with probability $1 - \mixprob$.\footnote{As a sanity check, with $\mixprob= 0, 0.5, 1$, the {\BiasedRandomOffer} recovers the classic {\BuyerOffer}, (unbiased) {\RandomOffer}, and {\SellerOffer}, respectively.}
\end{definition}
With start for our main result in this section, the results for general bilateral instances.


\begin{theorem}
\label{thm:optimal GFT:general instance}
\label{cor:biased random offer}
    For every bilateral trade instance $\btinstance$, there exists probability $\mixprob\in[0, 1]$ such that the {\BiasedRandomOffer} is {\ksfair} and its GFT is at least at least $\frac{1}{2}$ fraction of the {\SecondBest} $\OPTSB$. 
\end{theorem}
We remark that GFT approximation of $\frac{1}{2}$ in the above theorem does not require any assumption on the traders' valuation distributions. Moreover, this GFT approximation is \emph{optimal}: we show in \Cref{sec:improved GFT} that there exists a zero-value seller instance (\Cref{example:all fair mech:irregular}) where every BIC, IIR, ex ante WBB mechanism that is {\ksfair}, cannot obtain more than half of the {\SecondBest}.

To prove \Cref{thm:optimal GFT:general instance}, we develop a black-box reduction framework: it converts an arbitrary mechanism $\mech$ into a {\ksfair} mechanism $\mech\primed$ with provable GFT approximation guarantee, which is equal to the smaller ratio between each trader's ex ante utility in mechanism $\mech$ with her own benchmark. We then apply this framework on the (unbiased) {\RandomOffer} to prove the GFT approximation of $\frac{1}{2}$ 
as claimed.

\begin{restatable}[Black-box reduction]{theorem}{thmblackboxreduction}
    \label{thm:blackbox reduction}
    {Fix any bilateral trade instance $\btinstance$.}
    Fix any mechanism $\mech$ {that is BIC, IIR and ex ante WBB} (possibly not {\ksfair}). 
    Define the constant $\GFTapprox\in[0,1]$ as
    \begin{align*}
        \GFTapprox \triangleq \min\left\{
        \frac{\sellerexanteutil(\mech)}{\sellerbenchmark},
        \frac{\buyerexanteutil(\mech)}{\buyerbenchmark}
        \right\}
    \end{align*}
    Then, there exists BIC, IIR and ex ante WBB mechanism $\mech\primed$ that is {\ksfair} and guarantees a GFT of at least $\GFTapprox$ fraction of the {\SecondBest} $\OPTSB$. Specifically, mechanism $\mech\primed$ is constructed as follows:
    \begin{itemize}
        \item If $\frac{\buyerexanteutil(\mech)}{\buyerbenchmark} \geq \frac{\sellerexanteutil(\mech)}{\sellerbenchmark}$, mechanism $\mech$ is implemented with probability $\mixprob$, and the {\SellerOffer} is implemented with probability $1-\mixprob$, for some  $\mixprob\in[0, 1]$.
        \item If $\frac{\buyerexanteutil(\mech)}{\buyerbenchmark} \leq \frac{\sellerexanteutil(\mech)}{\sellerbenchmark}$, mechanism $\mech$ is implemented with probability $\mixprob$, and the {\BuyerOffer} is implemented with probability $1-\mixprob$, for some  $\mixprob\in[0, 1]$.
    \end{itemize}
\end{restatable}
We remark that if the original mechanism $\mech$ is ex post IR (resp.\ ex post SBB), the constructed mechanism $\mech\primed$ from \Cref{thm:blackbox reduction} is also ex post IR (resp.\ ex post SBB). Moreover, our black-box reduction (\Cref{thm:blackbox reduction}) and the positive result from \Cref{thm:optimal GFT:general instance} can be extrapolated to obtain similar results for the more abstract model of cooperative bargaining, with an almost identical proof. In \Cref{appendix:general-bargaining-results} we detail the formulations of these results in the bargaining model. Therefore, though the black-box reduction framework is conceptually simple, it is general and could be of independent interest. 
\begin{proof}[Proof of \Cref{thm:blackbox reduction}]
    Without loss of generality, we assume $\frac{\buyerexanteutil(\mech)}{\buyerbenchmark} \geq \frac{\sellerexanteutil(\mech)}{\sellerbenchmark} = %\geq
    \GFTapprox$. (The other case follows a symmetric argument.)
    For every $\mixprob\in[0, 1]$, consider $\mech\primed_{\mixprob}$ constructed as follows: original mechanism $\mech$ is implemented with probability $\mixprob$ and the {\SellerOffer} is implemented with probability $1-\mixprob$. By construction, we have
    \begin{align*}
        \frac{\sellerexanteutil(\mech\primed_\mixprob)}{\sellerbenchmark} 
        =
        \frac{\mixprob\cdot \sellerexanteutil(\mech) + (1-\mixprob)\cdot \sellerexanteutil(\SOM)}{\sellerbenchmark} 
        =
        \frac{\mixprob\cdot \sellerexanteutil(\mech) + (1-\mixprob)\cdot \sellerbenchmark}{\sellerbenchmark} 
    \end{align*}
    which is weakly decreasing linearly in $\mixprob\in[0, 1]$, since $\sellerbenchmark(\mech) \leq \sellerbenchmark$ by definition. Similarly, 
    \begin{align*}
        \frac{\buyerexanteutil(\mech\primed_\mixprob)}{\buyerbenchmark} 
        =
        \frac{\mixprob\cdot \buyerexanteutil(\mech) + (1-\mixprob)\cdot \buyerexanteutil(\SOM)}{\buyerbenchmark} 
    \end{align*}
    which is also linear in $\mixprob\in[0, 1]$. Moreover, due to the case assumption, we know 
    \begin{align*}
        \frac{\sellerexanteutil(\mech\primed_1)}{\sellerbenchmark} 
        \leq 
        \frac{\buyerexanteutil(\mech\primed_1)}{\buyerbenchmark}
        \;\;
        \mbox{and}
        \;\;
        \frac{\sellerexanteutil(\mech\primed_0)}{\sellerbenchmark} = 1
        \geq 
        \frac{\buyerexanteutil(\mech\primed_0)}{\buyerbenchmark}
    \end{align*}
    Thus, there exists $\mixprob^*\in[0, 1]$ such that 
    \begin{align*}
        \frac{\buyerexanteutil(\mech\primed_{\mixprob^*})}{\buyerbenchmark} 
        \overset{(a)}{=} 
        \frac{\sellerexanteutil(\mech\primed_{\mixprob^*})}{\sellerbenchmark} 
        \overset{(b)}{\geq}
        \frac{\sellerexanteutil(\mech\primed_{1})}{\sellerbenchmark} 
        \overset{(c)}{\geq}
        \GFTapprox
    \end{align*}
    which implies that mechanism $\mech\primed_{\mixprob^*}$ is {\ksfair} and its GFT is a $\GFTapprox$-approximation to the {\SecondBest} $\OPTSB$. (Recall that the {\SecondBest} $\OPTSB$ is upper bounded by $\sellerbenchmark + \buyerbenchmark$.) Here equality~(a) holds due to the intermediate value theorem, inequality~(b) holds due to the monotonicity of $
    {\sellerexanteutil(\mech\primed_{\mixprob})}/{\sellerbenchmark}$ as a function of $\mixprob$ argued above,
    and 
    inequality~(c) holds since mechanism $\mech\primed_1$ is equivalent to original mechanism $\mech$.
\end{proof}



Utilizing this black-box reduction,  we immediately prove \Cref{thm:optimal GFT:general instance} as follows. 
\begin{proof}[Proof of \Cref{thm:optimal GFT:general instance}]
    Consider the (unbiased) {\RandomOffer}. Note that both traders' ex ante utilities satisfy
    \begin{align*}
        \frac{\buyerexanteutil(\ROM)}{\buyerbenchmark}
        =
        \frac{1}{2} \left(\frac{\buyerexanteutil(\SOM)}{\buyerbenchmark}
        +
        \frac{\buyerexanteutil(\BOM)}{\buyerbenchmark}\right)
        \geq 
        \frac{1}{2}
        \;
        \mbox{and}
        \;
        \frac{\sellerexanteutil(\ROM)}{\sellerbenchmark}
        =
        \frac{1}{2} \left(\frac{\sellerexanteutil(\SOM)}{\sellerbenchmark}
        +\frac{\sellerexanteutil(\BOM)}{\sellerbenchmark}
        \right)
        \geq 
        \frac{1}{2}
    \end{align*}    
    where we use the fact that the buyer's (resp.\ seller's) benchmark $\buyerbenchmark$ ($\sellerbenchmark$) is achieved by the {\BuyerOffer} (resp.\ {\SellerOffer}), i.e., $\buyerexanteutil(\BOM) = \buyerbenchmark$ (resp.\ $\sellerexanteutil(\SOM) = \sellerbenchmark$). Invoking \Cref{thm:blackbox reduction} for the {\RandomOffer}, we obtain {\ksfair} mechanism $\mech\primed$ whose GFT is at least $\frac{1}{2}$ fraction of the {\SecondBest} $\OPTSB$. Moreover, it can be verified by the construction in \Cref{thm:blackbox reduction}, mechanism $\mech\primed$ is the {\BiasedRandomOffer}. 
\end{proof}


We next consider the bilateral trade instances where both traders' valuation distributions are MHR. In this setting, prior work \citep{Fei-22} shows that both the {\BuyerOffer} and {\SellerOffer} (which might not be {\ksfair}) guarantee a GFT approximation of $\frac{1}{e - 1}\approx 58.1\%$. Combining this with our black-box reduction, an improved GFT approximation of $\frac{1}{e - 1}$ for the {\ksfair} {\BiasedRandomOffer} is obtained. For completeness, we also include its proof below.

\begin{theorem}[Theorem~3.1 in \citealp{Fei-22}]
\label{thm:BO SO GFT:mhr traders}
    For every bilateral trade instance, if the buyer (resp.\ seller) has MHR valuation distribution, then the {\SellerOffer} (resp.\ {\BuyerOffer}) obtains at least $\frac{1}{e - 1}$ fraction of the {\FirstBest} $\OPTFB$.
\end{theorem}
\begin{restatable}{corollary}{thmmhrtraders}
\label{thm:improved GFT:mhr traders}
For every bilateral trade instance $\btinstance$ where both traders have MHR valuation distributions, there exists probability $\mixprob\in[0, 1]$ such that the {\BiasedRandomOffer} is {\ksfair} and its GFT is at least at least $\frac{1}{e-1}$ fraction of the {\SecondBest} $\OPTSB$.\footnote{The $\frac{1}{e - 1}$ approximation in this theorem also holds for the {\FirstBest} $\OPTFB$, i.e., $\GFT{\mech} \geq \frac{1}{e - 1}\cdot \OPTFB$.}
\end{restatable}
\begin{proof}[Proof of \Cref{thm:improved GFT:mhr traders}]
    Applying \Cref{thm:blackbox reduction} with the (unbiased) {\RandomOffer}, there exists $\mixprob\in[0, 1]$ such that the {\BiasedRandomOffer} is {\ksfair}. Since in the {\BiasedRandomOffer}, either {\BuyerOffer} or {\SellerOffer} are implemented in every realized execution, invoking \Cref{thm:BO SO GFT:mhr traders} finishes the proof.
\end{proof}