\label{sec:prelim}

In this section, we first present the bilateral trade model, and then review some important concepts in Bayesian mechanism design that will be used in our analysis. 

\subsection{Bilateral Trade Model}
There is a single seller and a single buyer, and we refer to them as traders (or agents). 
The seller holds one item, which is demanded by the buyer. 
{Trade will determine an allocation of $\alloc\in[0, 1]$ fraction of the item (or that entire item with probability $\alloc$) that will be traded and transferred to the buyer.} 
The value of each trader for the item is private information, known only to her. 

Specifically, we denote the private value of the buyer by $\val\in\reals_+$, and given an allocation $\alloc\in[0, 1]$ and the payment she pays (monetary transfer from her) $\price\in\reals_+$, her utility $\buyerutil(\alloc,\price)$ is defined to be $\buyerutil(\alloc,\price) = \val\cdot \alloc - \price$.
Similarly, we denote the private value of the seller by $\cost\in\reals_+$, and given an allocation $\alloc\in[0, 1]$ and payment she receives (monetary transfer to her) $\sellerprice\in \reals_+$,\footnote{As it is more convenient  to assume that the seller receives money (rather than paying negative amount), we use $\sellerprice$ to denote the payment the seller receives.} her utility $\sellerutil(\alloc, \sellerprice)$ is defined to be $\sellerutil(\alloc, \sellerprice) = \sellerprice - \cost \cdot \alloc$.

Finally, given allocation $\alloc\in[0, 1]$, the \emph{gains-from-trade} (\emph{GFT}) when there is trade between a seller with value $\cost$ and a buyer with value $\val$ is $(\val - \cost)\cdot \alloc$.

\xhdr{The Bayesian setting.} There is benevolent social planner (mechanism designer) whose goal is to design mechanisms to facilitate trade between the seller and buyer. Though both traders' values are private, we assume each trader's value is independently sampled from some prior distribution (the distributions are not necessarily identical), and the distributions are known to all parties (seller, buyer and social planner). We use $\buyerdist$ to denote the buyer's valuation distribution, and use $\sellerdist$ to denote the seller's valuation distribution, so a bilateral trade instance is a pair $\btinstance$. To simplify the presentation and analysis, we impose the following assumptions on the traders' valuation distributions.
\begin{itemize}
    \item 
    The buyer's valuation distribution $\buyerdist$ has a bounded support $\supp(\buyerdist)$, 
    so it holds that ${\supp(\buyerdist)} \subseteq [\lval, \hval]$ for some $\hval\geq \lval\geq 0$.    
    We assume that $\buyerdist$ has no atoms, except possibly at $\hval$. 
    The cumulative density function (denoted by $\buyercdf$) is left-continuous,\footnote{Following the convention in auction design literature, we define cumulative density function $\buyercdf(t)\triangleq \prob[\val\sim\buyercdf]{\val < t}$ for the buyer's valuation distribution, and $\sellercdf(t) \triangleq \prob[\cost\sim\sellerdist]{\cost \leq t}$ for the seller's valuation distribution.} is differentiable within the interior of the support with measure 1. We denote the corresponding probability density function by $\buyerpdf$.
    
    \item The seller's valuation distribution $\sellerdist$ has a bounded support $\supp(\sellerdist)$, 
    and thus ${\supp(\sellerdist)} \subseteq [\lcost, \hcost]$ for some $\hcost\geq \lcost\geq 0$.    
    We assume that $\sellerdist$ has no atoms, except possibly at $\lcost$. 
    The cumulative density function (denoted by $\sellercdf$) is right-continuous, is differentiable within the interior of the support with measure 1. We denote the corresponding probability density function by $\sellerpdf$.
\end{itemize}
Given a bilateral trade instance $\btinstance$, the maximum (optimal) expected GFT for the instance, 
$\OPTFB\btinstance$, is defined as follows  
\begin{align*}
    \OPTFB\btinstance \triangleq \expect[\val\sim\buyerdist,\cost\sim\sellerdist]{\plus{\val - \cost}}
\end{align*}
where operator $\plus{a} \triangleq \max\{0, a\}$.
We refer to $\OPTFB\btinstance$ as the {\sf First-Best (FB) GFT benchmark}.

\xhdr{Bayesian mechanism design for GFT maximization.} 
A (direct-revelation) mechanism $\mech = (\alloc,\price, \sellerprice)$ first solicits bids from both traders and then decides the allocation and payment based on allocation rule $\alloc$, buyer payment rule $\price$, and seller payment rule $\sellerprice$. Allocation $\alloc:\reals_+^2\rightarrow [0, 1]$ and buyer/seller payments $\price,\sellerprice:\reals_+^{2}\rightarrow\reals_+$ are mappings from the bid profile of the traders to the fraction of item allocated and monetary transfer from the buyer/to the seller, respectively. 

We are interested in the GFT obtainable in some Bayesian Nash equilibrium (BNE) of the designed mechanism. Applying the standard revelation principle \citep{mye-81}, without loss of generality, we focus on mechanisms that satisfy \emph{Bayesian Inventive Compatibility (BIC)}, i.e., reporting private value truthfully forms a BNE. By imposing BIC, we assume that traders report their private values truthfully and view the allocation and payment rules $(\alloc, \price, \sellerprice)$ of a given mechanism as mappings from traders' (true) valuation profile $(\val, \cost)$, to their allocation and payments, respectively. 

With slight abuse of notation, we further define the interim allocation, payment, and utility of a trader in a given mechanism $\mech = (\alloc, \price, \sellerprice)$ as the expected allocation, payment, and utility over the randomness of the other trader's value. Namely,
\begin{align*}
    \text{for the seller:}
    \quad
    &\selleralloc(\cost) = \expect[\val\sim\buyerdist]{\alloc(\val, \cost)},~
    \sellerprice(\cost) = \expect[\val\sim\buyerdist]{\sellerprice(\val, \cost)},~
    \sellerutil(\cost) = \expect[\val\sim\buyerdist]{\sellerprice(\val,\cost) - \cost \cdot \alloc(\val,\cost)}
    \\
    \text{for the buyer:}
    \quad
    &\alloc(\val) = \expect[\cost\sim\sellerdist]{\alloc(\val,\cost)},~
    \price(\val) = \expect[\cost\sim\sellerdist]{\price(\val, \cost)},~
    \buyerutil(\val) = \expect[\cost\sim\sellerdist]{\val\cdot \alloc(\val,\cost) - \price(\val,\cost)}
\end{align*}
Fix any bilateral trade instance $\btinstance$ and any mechanism $\mech = (\alloc,\price,\sellerprice)$, we define the ex ante seller (resp.\ buyer) utility $\sellerexanteutil(\mech,\buyerdist,\sellerdist)$ (resp.\ $\buyerexanteutil(\mech,\buyerdist,\sellerdist)$) as the expected utilities of the seller (resp.\ buyer) over the randomness of the other trader's value. Namely,
\begin{align*}
    \sellerexanteutil(\mech,\buyerdist,\sellerdist) = \expect[\cost\sim\sellerdist]{\sellerutil(\cost)}
    \;\;\mbox{and}\;\;
    \buyerexanteutil(\mech,\buyerdist,\sellerdist) =  \expect[\val\sim\buyerdist]{\buyerutil(\val)}
\end{align*}
Besides Bayesian incentive compatibility, any mechanism under consideration must also satisfy \emph{interim individual rationality} and \emph{ex ante weak budget balance}, formally defined as follows:
\begin{itemize}
    \item \emph{Interim individual rationality (IIR)}: For each trader and each value realization {of that trader}, her interim utility is non-negative.
    \item \emph{Ex ante weak budget balance (ex ante WBB)}: The expected payment collected from the buyer is weakly larger than the expected payment given to the seller, i.e., $\expect[\val\sim\buyerdist]{\price(\val)} \geq \expect[\cost\sim\sellerdist]{\sellerprice(\cost)}$.
\end{itemize}
For every bilateral instance $\btinstance$, we let $\mechfamily$ denote the family of all BIC, IIR  and ex ante WBB mechanisms for these distributions. 


The social planner measures the efficiency of mechanisms by their GFT. Specifically, for a bilateral trade instance $\btinstance$ the GFT of mechanism $\mech = (\alloc,\price, \sellerprice)$ is defined as 
\begin{align*}
    \GFT{\mech,\buyerdist,\sellerdist} = \expect[\val\sim\buyerdist,\cost\sim\sellerdist]{(\val - \cost)\cdot \alloc(\val,\cost)}
\end{align*}
We use $\OPTSB$ to denote the maximum (optimal) expected GFT obtainable by any mechanism that is BIC, IIR and ex-ante WBB, i.e.,
\begin{align*}
    \OPTSB\btinstance \triangleq \max_{{\mech\in\mechfamily}}
    \GFT{\mech,\buyerdist,\sellerdist}
\end{align*} 
We also refer to $\OPTSB\btinstance$ as the {\sf Second-Best (SB) GFT benchmark}. It is known that for ``non-degenerate'' bilateral trade instances $\btinstance$, the {\FirstBest} $\OPTFB\btinstance$ is strictly larger than the {{\SecondBest}} $\OPTSB\btinstance$ \citep{MS-83}, but the latter is also a constant approximation to the former, i.e., $\OPTFB\btinstance \leq 3.15 \cdot \OPTSB\btinstance$ \citep{DMSW-22,Fei-22}. 

Throughout the paper, we mainly compare the GFT of a given mechanism $\mech\in\mechfamily$ with the second best benchmark $\OPTSB$. For any $\GFTapprox\in[0, 1]$, we say the GFT of a given mechanism ${\mech\in\mechfamily}$ is a $\GFTapprox$-approximation (to the {{\SecondBest}} $\OPTSB$) if $\GFT{\mech,\buyerdist,\sellerdist} \geq \GFTapprox \cdot \OPTSB\btinstance$.\footnote{Nonetheless, due to the bounded gap between two benchmarks, i.e., $\OPTFB\leq 3.15 \cdot \OPTSB$ for every bilateral trade instances, our results also imply GFT approximations against the {\FirstBest} $\OPTFB$.}


While the {\SecondBest} $\OPTSB$ is with the most relaxed properties (BIC, IIR, and ex ante WBB), all the positive results in this work are obtained with the stronger properties of \emph{ex post individual rationality (ex post IR)} where every trader's ex post utility is non-negative for every realized valuation profile, and \emph{ex post strong budget balance (ex post SBB)} where the buyer's ex post payment is equal to the seller's ex post payment for every realized valuation profile. Some of the positive results even replace BIC with \emph{dominate strategy incentive compatibility (DSIC)} where reporting value truthfully is a dominant strategy. 



We make use of four classic mechanisms studied extensively in the bilateral trade model:
\begin{itemize}
    \item {\FixPrice} ({\FPM}): the social planner offers a price $\price$ to both traders ex ante, who individually chooses whether to accept it or not. If both traders accept the price, a trade occurs at price $\price$. Otherwise, no trade occurs and no payments are transferred.\footnote{{We always assume that ties are broken towards trade.}}
     \item {\SellerOffer} ({\SOM}): given her realized value $\cost$, the seller optimally picks a take-it-or-leave-it price $\price$ and offers to sell the item at that price.\footnote{Specifically, the seller picks a monopoly reserve $\optreserve \in\argmax_{\price}(\price -\cost)\cdot (1 - \buyercdf(\price))$ given her realized value $\cost$.} Trade occurs at price $\price$ if the buyer accepts the offer (Otherwise, no trade occurs and no payments are transferred).
    \item {\BuyerOffer} ({\BOM}): given her realized value $\val$, the buyer optimally picks a take-it-or-leave-it price $\price$ and offers to buy the item at that price. Trade occurs at price $\price$ if the seller accepts the offer (Otherwise, no trade occurs and no payments are transferred).
    \item {\RandomOffer} ({\ROM}): 
    the social planner implements the {\RandomOffer} and {\BuyerOffer} with identical ex ante probability $\frac{1}{2}$.
\end{itemize}
All four mechanisms are ex post IR and ex post SBB. The {\FixPrice} is DSIC, while the other three mechanisms are BIC.\footnote{In fact, 
in the {\SellerOffer} (resp.\ {\BuyerOffer}), the buyer (resp.\ seller) has a dominant strategy.} By \citet{mye-81}, the {\SellerOffer} (resp.\ {\BuyerOffer}) maximizes the seller's (resp.\ buyer's) ex ante utility among all BIC, IIR, and ex ante WBB mechanisms.


When traders' distributions $\buyerdist,\sellerdist$ are clear from the context, we sometimes simplify notations and omit them, e.g., writing $\GFT{\mech}$ instead of $\GFT{\mech,\buyerdist,\sellerdist}$. This includes the notations $\mechfam$, $\GFT{\mech}$, $\sellerexanteutil(\mech)$, $\buyerexanteutil(\mech)$, $\OPTFB$, and $\OPTSB$, among others.


\subsection{Necessary Mechanism Design Notations and Concepts}

We next establish some notations and standard properties that we use in the paper. 

\xhdr{Regularity and monotone hazard rate.}
Two important distribution subclasses have been introduced and commonly studied in the mechanism design literature \citep{mye-81,MS-83}.
\begin{definition}[Regularity and MHR for the buyer]
\label{def:regularity buyer}
    A buyer's valuation distribution $\buyerdist$ is \emph{regular} if its virtual value function 
    $
        \virtualval(\val) \triangleq \val -
        \frac{1 - \buyercdf(\val)}{\buyerpdf(\val)}
    $
    is non-decreasing in $\val$. 
    
    A buyer's valuation distribution $\buyerdist$ satisfies the \emph{monotone hazard rate (MHR) condition} if its hazard rate function 
    $
        \buyerhazardrate(\val) \triangleq
        \frac{\buyerpdf(\val)}{1 - \buyercdf(\val)}
        $
    is non-decreasing in $\val$.
\end{definition}
MHR is a stronger condition: any distribution that is MHR is also regular. Symmetrically, we say the seller's valuation distribution $\sellerdist$ is regular (resp.\ MHR) if the random variable $\hcost - \cost$ satisfies the regularity (resp.\ MHR) condition defined for the buyer in \Cref{def:regularity buyer}, where $\hcost$ is the largest value in support $\supp(\sellerdist)$. 

\xhdr{Revenue curve and cumulative hazard rate.}
We first present the revenue curve, which is useful for the revenue analysis.

\begin{definition}[Revenue curve]
    Fix a valuation distribution $\buyerdist$ of the buyer. The \emph{revenue curve} $\revcurve:[0, 1]\rightarrow\reals_+$ is a mapping from a quantile $\quant\in[0, 1]$ to the expected revenue from posting a price equal to $\val(\quant)= \buyercdf^{-1}(1 - \quant)  \triangleq \sup\{\val:\buyercdf(\val) \leq 1 - \quant\}$, the price for which sell happens with probability~$\quant$. Namely, for every quantile $\quant\in[0, 1]$, $\revcurve(\quant) \triangleq \quant\cdot \buyercdf^{-1}(1 - \quant)$.
\end{definition}
We define the \emph{monopoly revenue} as the largest revenue $\max_{\quant\in[0, 1]}\revcurve(\quant)$ and denote every quantile (resp.\ price) attaining this largest revenue as a \emph{monopoly quantile} $\optquant$ (resp.\ \emph{monopoly reserve $\optreserve$}).
The regularity of a valuation distribution has the following nice geometric interpretation on the induced revenue curve. 
\begin{lemma}[\citealp{BR-89}]
\label{lem:concave revenue curve}
    A valuation distribution $\buyerdist$ of the buyer is regular if and only if the induced revenue curve $\revcurve$ is weakly concave (i.e., $\revcurve \equiv \ironrevcurve$). Moreover, for every value $\val$, the virtual value $\virtualval(\val)$ is equal to the right derivative of revenue curve $\revcurve$ at quantile $\quant = 1  - \buyercdf(\val)$, i.e., $\virtualval(\val) = \partial_+\revcurve(\quant)$. 
\end{lemma} 

Similar to the connection between the virtual value and revenue curve, we also introduce the \emph{cumulative hazard rate function} $\cumhazard:\supp(\buyerdist)\rightarrow \reals_+$ where $\cumhazard(\val) = \int_{0}^\val \buyerhazardrate(t)\cdot \d t \equiv -\ln(1 - \buyercdf(\val))$. Note that buyer distribution $\buyerdist$ is MHR if and only if the induced cumulative hazard rate function $\cumhazard$ is weakly convex. 

The virtual value and hazard rate admit the following characterization on the buyer's ex ante utility and payment for the zero-value seller instances.
\begin{proposition}[{\citealp{mye-81,HR-08}}]
\label{prop:revenue equivalence}
\label{prop:buyer surplus equivalence}
\begin{flushleft}
{Fix any zero-value seller instance where the buyer has a valuation distribution $\buyerdist$. For every any BIC, IIR, ex ante WBB mechanism $\mech \in\mechfam$,} the buyer's ex ante utility and payment satisfy 
$\buyerexanteutil(\mech)
= 
\expect[\val\sim\buyerdist]{\buyerhazardrate(\val)\cdot \alloc(\val)}$
and 
$\expect[\val\sim\buyerdist]{\price(\val)}
= 
\expect[\val\sim\buyerdist]{\virtualval(\val)\cdot \alloc(\val)}$,
where $\alloc(\val)$ is the interim allocation of the buyer with value $\val$.
\end{flushleft}
\end{proposition}