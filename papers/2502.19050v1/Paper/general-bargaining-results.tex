
In this section we extrapolate the results of \Cref{sec:general-gft-approx} to the more general model of cooperative bargaining.
% \footnote{For formal definitions of the bargaining model it's relation to bilateral trade, see \Cref{appendix:bargaining-and-trade}.} 
We show that both \Cref{thm:blackbox reduction} and \Cref{thm:optimal GFT:general instance} can be generalized to the cooperative bargaining model. 

The following known result for bilateral trade helps bridge the gap between the models.

\begin{restatable}[\citealp{BCWZ-17}]{lemma}{SBupperbound}
 \label{lemma:SB-upper-bound}
     For every bilateral trade instance $\btinstance$, the {\SecondBest} $\OPTSB$ is at most the sum of two players' ideal utilities $\sellerbenchmark,\buyerbenchmark$, i.e., $\OPTSB \leq \sellerbenchmark + \buyerbenchmark$.
\end{restatable}

This result motivates the following definition for the bargaining model:

\begin{definition}
    The \emph{ideal {total} utility} of a given bargaining problem $\bargainprob$ is $\idealutility{\bargainprob} = \idealbuyer{\bargainprob} + \idealseller{\bargainprob}$. We write that point $\pnta\in\bargainprob$ has a $\GFTapprox$ fraction of the ideal utility if $\buyercoord{\pnta} + \sellercoord{\pnta} \geq \GFTapprox\cdot\idealutility{\bargainprob}$.
\end{definition}

Recall that {\ksfair} mechanisms in the bilateral trade model correspond to points on the {\ksline} in the bargaining model. Combining this with the above definition and \Cref{lemma:SB-upper-bound} implies a reduction of the bilateral trade model to the bargaining problem: {any positive result shown for the fraction of the ideal {total} utility attainable by a point on the {\ksline} for the general bargaining problem implies that same fraction of the {\SecondBest} can always be obtained by a truthful {\ksfair} mechanism in the bilateral trade model}. This reduction shows that the general results depicted in this section imply their corresponding results from \Cref{sec:general-gft-approx}.

\subsubsection{Generalization of \texorpdfstring{\Cref{thm:blackbox reduction}}{Theorem~\ref{thm:blackbox reduction}}}

%In this section we 
{We first} generalize the black-box reduction framework of \Cref{thm:blackbox reduction} to the bargaining model. Given a bargaining problem $\bargainprob$, the framework converts an arbitrary point $\pnta\in\bargainprob$ into a point $\pntb$ along the {\ksline} of $\bargainprob$, with a provable approximation guarantee of the ideal {total} utility of $\bargainprob$. The proof is near identical to the proof of \Cref{thm:blackbox reduction}, we write the proof explicitly for the sake of formality.

\begin{theorem}[Black-box reduction]
    \label{lemma:bargaining:blackbox reduction}
    Fix some bargaining problem $\bargainprob$ and point $\pnta\in \bargainprob$. Define constant $\GFTapprox\in[0,1]$ as
    \begin{align*}
        \GFTapprox \triangleq \min\left\{
        \frac{\buyercoord{x}}{\idealbuyer{\bargainprob}},
        \frac{\sellercoord{x}}{\idealseller{\bargainprob}}
        \right\}
    \end{align*}
    Then, there exists a point $\pntb\in S$ on the {\ksline} which is has a $\GFTapprox$ fraction of the ideal utility of $\bargainprob$, that is $\buyercoord{\pntb} + \sellercoord{\pntb} \geq \GFTapprox\cdot \idealutility{\bargainprob}$. Specifically, the solution $\pntb$ is a convex combination $\pntb=\mixprob \cdot \pnta + (1-\mixprob) \cdot \pntc$ for some appropriate $\mixprob\in[0, 1]$, where $\pntc$ is chosen as follows:
    \begin{itemize}
        \item If $\frac{\buyercoord{\pnta}}{\idealbuyer{\bargainprob}} \geq \frac{\sellercoord{\pnta}}{\idealseller{\bargainprob}}$ then let $\pntc$ be some arbitrary point satisfying $\sellercoord{\pntc} = \idealseller{\bargainprob}$.
        \item If $\frac{\buyercoord{\pnta}}{\idealbuyer{\bargainprob}} < \frac{\sellercoord{\pnta}}{\idealseller{\bargainprob}}$ then let $\pntc$ be some arbitrary point satisfying $\buyercoord{\pntc} = \idealbuyer{\bargainprob}$.
    \end{itemize}
\end{theorem}

\begin{proof}
    Without loss of generality, we assume $\frac{\buyercoord{\pnta}}{\idealbuyer{\bargainprob}} \geq \frac{\sellercoord{\pnta}}{\idealseller{\bargainprob}} = \GFTapprox$. The other case follows a symmetric argument. Fix an arbitrary point $\pntc\in {\bargainprob}$ such that $\sellercoord{\pntc}=\idealseller{\bargainprob}$. For every $\mixprob\in[0, 1]$, consider the point $\pntb_{\mixprob}$ defined as the convex combination $\pntb_{\mixprob}=\mixprob \cdot \pnta + (1-\mixprob)\cdot \pntc$. By construction, we have
    \begin{align*}
        \frac{\sellercoord{\pntb_{\mixprob}}}{\idealseller{\bargainprob}} 
        =
        \frac{\mixprob\cdot \sellercoord{\pnta} + (1-\mixprob)\cdot \sellercoord{z}}{\idealseller{\bargainprob}} 
        =
        \frac{\mixprob\cdot \sellercoord{\pnta} + (1-\mixprob)\cdot \idealseller{\bargainprob}}{\idealseller{\bargainprob}} 
    \end{align*}
    which is weakly decreasing linearly in $\mixprob\in[0, 1]$, since $\sellercoord{\pnta} \leq \idealseller{\bargainprob}$ by definition. Similarly, 
    \begin{align*}
        \frac{\buyercoord{\pntb_{\mixprob}}}{\idealbuyer{\bargainprob}} 
        =
        \frac{\mixprob\cdot \buyercoord{\pnta} + (1-\mixprob)\cdot \buyercoord{z}}{\idealbuyer{\bargainprob}} 
    \end{align*}
    which is also linear in $\mixprob\in[0, 1]$. Moreover, due to the case assumption, we know 
    \begin{align*}
        \frac{\sellercoord{\pnta}}{\idealseller{\bargainprob}} 
        \leq 
        \frac{\buyercoord{\pnta}}{\idealbuyer{\bargainprob}}
        \;\;
        \mbox{and}
        \;\;
        \frac{\pntb^{0}}{\idealseller{\bargainprob}} = 1
        \geq 
        \frac{\pntb^{0}}{\idealbuyer{\bargainprob}}
    \end{align*}
    Thus, there exists $\mixprob^*\in[0, 1]$ such that 
    \begin{align*}
        \frac{\buyercoord{\pntb_{\mixprob^*}}}{\idealbuyer{\bargainprob}} 
        \overset{(a)}{=} 
        \frac{\sellercoord{\pntb_{\mixprob^*}}}{\idealseller{\bargainprob}} 
        \overset{(b)}{\geq}
        \frac{\sellercoord{\pntb_{1}}}{\idealseller{\bargainprob}} 
        \overset{(c)}{\geq}
        \GFTapprox
    \end{align*}
    which implies that the point $\pntb_{\mixprob^*}$ is on the {\ksline}. Here, equality~(a) holds due to the intermediate value theorem, inequality~(b) holds due to the monotonicity of $
    {\pntb_{\mixprob}}/{\idealseller{\bargainprob}}$ as a function of $\mixprob$ argued above,
    and 
    inequality~(c) holds since the point $\pntb^{1}$ is equivalent to original point $\pnta$. Finally,
    \begin{align*}
        \pntb_{\mixprob^*} + \pntb_{\mixprob^*} \geq \GFTapprox \cdot (\idealbuyer{\bargainprob} + \idealseller{\bargainprob}) = \GFTapprox \cdot \idealutility{\bargainprob}
    \end{align*}
    which completes the proof. 
\end{proof}

% I deleted this figure because I feel it is too complicated.
%    \usefigures{        \begin{figure}            \centering            \subfloat[]{        \input{Figures/fig-bargaining-blackbox}        \label{fig:bargaining:black-box}        }\caption{            A visual proof of \Cref{lemma:bargaining:blackbox reduction}: We begin from the point $\pnta=(1.2,0.6)$. This point has $\frac{4}{5}$ of the ideal value on the horizontal axis and $\frac{2}{3}$ of the ideal value on the vertical axis, so $\GFTapprox=\min(\frac{4}{5},\frac{2}{3})=\frac{2}{3}$. We connect our point to an arbitrary optimal point of the vertical axis, and find the intersection with the {\ksline}. The green line represents all points where the total utility is $\frac{2}{3}$ of the ideal {total} utility: we wish to show that the red point is above this line.  \Cref{fig:bargaining:half-approx} shows a visual proof of \Cref{thm:bargaining:half-approx-ksline}. Explanation: the red line is the {\ksline}. The black points are the optimal solutions, and the black line connects them. The green line is the line of all points with... \textbf{TBD}}\end{figure}    }



\subsubsection{Generalization of \texorpdfstring{\Cref{thm:optimal GFT:general instance}}{Theorem~\ref{thm:optimal GFT:general instance}}}

%In this section we 
{We now present a generalization of} \Cref{thm:optimal GFT:general instance} to the bargaining model. We show that for any bargaining problem $\bargainprob$, there always exists a point $\pnta\in \bargainprob$ on the {\ksline} such that $\pnta$ is a $\frac{1}{2}$ approximation of the ideal {total} utility of $\bargainprob$. The proof is near identical to the proof of \Cref{thm:optimal GFT:general instance}, we write the proof explicitly for the sake of formality. We make use of the generalized black-box reduction  \Cref{lemma:bargaining:blackbox reduction} proven in the previous section.

\begin{theorem}[Ideal {Total} Utility Approximation on the {\ksline}]
\label{lemma:bargaining:half-approx-ksline}
    Fix a bargaining problem $\bargainprob$ and points $\buyerbenchmarkpoint,\sellerbenchmarkpoint\in\bargainprob$ such that $\buyerbenchmarkpoint$ is an ideal point for the first agent ($\buyercoord{\buyerbenchmarkpoint}=\idealbuyer{\bargainprob}$ and $\sellerbenchmarkpoint$ is an ideal point for the second agent ($\sellercoord{\sellerbenchmarkpoint}=\idealseller{\bargainprob}$). Then the line connecting $\buyerbenchmarkpoint$ and $\sellerbenchmarkpoint$ intersects the {\ksline} at a point $\pntb\in \bargainprob$ which satisfies $\buyercoord{\pntb}+\sellercoord{\pntb}\geq \frac{1}{2} \cdot \idealutility{\bargainprob}$.
\end{theorem}

\begin{proof}
    Consider the point $\pnta=\frac{1}{2}\cdot \buyerbenchmarkpoint + \frac{1}{2}\cdot \sellerbenchmarkpoint$. Notice that both players' ex ante utilities satisfy
    \begin{align*}
        \frac{\buyercoord{\pnta}}{\idealbuyer{\bargainprob}}
        =
        \frac{\buyercoord{\buyerbenchmarkpoint}+\buyercoord{\sellerbenchmarkpoint}}{2\cdot \idealbuyer{\bargainprob}}
        %=\frac{\idealbuyer{\bargainprob}+\buyercoord{\sellerbenchmarkpoint}}{2\cdot \idealbuyer{\bargainprob}}
        \geq 
        \frac{1}{2}
        \;
        \mbox{ and }
        \;
        \frac{\sellercoord{\pnta}}{\idealbuyer{\bargainprob}}
        =
        \frac{\sellercoord{\buyerbenchmarkpoint}+\sellercoord{\sellerbenchmarkpoint}}{2\cdot \idealseller{\bargainprob}}
        %=\frac{\sellercoord{\buyerbenchmarkpoint}+\idealseller{\bargainprob}}{2\cdot \idealseller{\bargainprob}}
        \geq 
        \frac{1}{2}
    \end{align*}    
    Invoking \Cref{lemma:bargaining:blackbox reduction} with $\GFTapprox=\frac{1}{2}$ for the point $\pnta$, we obtain a point $\pntb\in S$ on the intersection between the {\ksline} and the line connecting $\buyerbenchmarkpoint, \sellerbenchmarkpoint$ such that $\buyercoord{\pntb}+\sellercoord{\pntb}\geq \frac{1}{2} \cdot (\idealbuyer{\bargainprob} +\idealseller{\bargainprob})$, completing the proof.
\end{proof}

% I decided to remove this figure as well since it is strange to try to present an intuitive proof this far into the appendices.
%\usefigures{    \begin{figure}        \centering        \subfloat[]{    \input{Figures/fig-bargaining-half-approx}    \label{fig:bargaining:half-approx}    }    \caption{ A visual proof of \Cref{thm:bargaining:half-approx}: The black points are the optimal solutions of each axis, and the black line connects them. The red line is the {\ksline}, and the red point on the intersection between these lines is $\pnta$. The green line is the of all points $\pntc$ with $\buyercoord{\pntc}+\sellercoord{\pntc}=\frac{1}{2}\cdot\idealutility{\bargainprob}$ ($=8$). Finally, the orange line is the secondary axis of the rectangle. The visual proof is as follows: note that the red, green and orange lines must intersect at the center of the rectangle. Since the optimal points (in black) must be above the orange line, the black line must intersect the red line above the green line.}    \end{figure}}
