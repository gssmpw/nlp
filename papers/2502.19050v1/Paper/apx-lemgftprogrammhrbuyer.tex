\label{apx:lemgftprogrammhrbuyer}
In this section, we prove \Cref{lem:GFT program:mhr buyer}. Its analysis is similar to the one for \Cref{lem:GFT program:regular buyer} in \Cref{subsec:improved GFT:regular buyer}. Specifically, we search over all MHR distributions and argue that the worst GFT can only be induced by a subclass of them, whose \emph{cumulative hazard rate function} (rather than the revenue curve, as was the case for regular buyer distributions) can be characterized by finite many parameters (see \Cref{fig:GFT program:mhr buyer}).

\lemgftprogrammhrbuyer*

\begin{proof}
    Let $\cumhazard$ be the cumulative hazard rate function of the buyer and $\optreserve$ be the monopoly reserve for $\buyerdist$. (Since the buyer's valuation distribution is MHR, the monopoly reserve is unique.) Without loss of generality, we normalize the monopoly revenue to be equal to one. Since the buyer's valuation distribution is MHR, cumulative hazard rate function $\cumhazard$ is convex.
    
    \xhdr{Step 0- Introducing necessary notations.} We introduce $\revratio\in(0, 1)$ as a constant whose value will be pined down at the end of this analysis. Given constant $\revratio$, let $\price\in[0, \optreserve]$ be the smallest price such that $\price\cdot (1 - \buyercdf(\price)) \geq \revratio$. 
    Moreover, let $\val_0 \triangleq \price - \cumhazard(\price) / \cumhazard'(\price)$ and $\val_1 \triangleq (\cumhazard(\price) - \cumhazard(\optreserve) + \optreserve\cdot \cumhazard'(\optreserve) - \price\cdot \cumhazard'(\price))/(\cumhazard'(\optreserve) - \cumhazard'(\price))$. Finally, we also define 
    \begin{align*}
        H &\triangleq \expect[\val\sim\buyerdist]{\val\cdot \indicator{\val \geq \optreserve}}~,
        \;\;
        M \triangleq \expect[\val\sim\buyerdist]{\val\cdot \indicator{\price \leq \val < \optreserve}}~,
        \;\;
        L \triangleq \expect[\val\sim\buyerdist]{\val\cdot \indicator{ \val < \price}}~.
    \end{align*}
    which partition the {\SecondBest} into three pieces, i.e., $\OPTSB = \expect[\val]{\val} = H + M + L$.
    All notations are illustrated in \Cref{fig:GFT program:mhr buyer}.

    \xhdr{Step 1- Characterizing of a (possibly) not {\ksfair} {\FixPrice}.}
    First, we consider a (possibly not {\ksfair}) {\FixPrice} ({$\FPM_{\price}$}) with trading price $\price$:
    \begin{itemize}
        \item For the seller, her ex ante utility (aka., revenue) is $\sellerexanteutil(\FPM_{\price}) = \price\cdot (1 - \buyercdf(\price)) = \revratio$. Since her benchmark $\sellerbenchmark$ (aka., monopoly revenue) is normalized to one, her ex ante utility is an $\revratio$ fraction of her benchmark $\sellerbenchmark$.
        \item For the buyer, her ex ante utility can be computed as 
        \begin{align*}
            \buyerexanteutil(\FPM_{\price}) &= 
            \expect[\val]{\plus{\val - \price}} = 
            H + M - \revratio
        \end{align*}
        Meanwhile, the buyer's benchmark $\buyerbenchmark$ is 
        \begin{align*}
            \buyerbenchmark = H + M + L
        \end{align*}
    \end{itemize}
    Putting the two pieces together, we conclude that in this (possibly not {\ksfair}) {\FixPrice} ({$\FPM_{\price}$}), both traders' ex ante utilities satisfy
    \begin{align*}
        \frac{\sellerexanteutil(\FPM_{\price})}{\sellerbenchmark} = \revratio
        \;\;
        \mbox{and}
        \;\;
        \frac{\buyerexanteutil(\FPM_{\price})}{\buyerbenchmark} = \frac{H + M - \revratio}{H + M + L}
    \end{align*}
    
    \xhdr{Step 2- Characterizing of {\ksfair} {\FixPrice}.} Define auxiliary notation $\exanteutilratio\in[0, 1]$ as 
    \begin{align*}
        \exanteutilratio \triangleq \displaystyle
        \revratio - \plus{\frac{H + M + L}{H + M + L + 1}\left(\revratio - \frac{H + M - \revratio}{H + M + L}\right)}
    \end{align*}
    We next show that there exists price $\fprice\in[0,\optreserve]$ such that the {\FixPrice} with trading price $\fprice$ is {\ksfair} and both traders' ex ante utilities are at least $\exanteutilratio$ fraction of their benchmarks $\sellerbenchmark,\buyerbenchmark$, respectively. To see this, consider the following two cases separately.
    \begin{itemize}
        \item Suppose that $\revratio < \frac{H + M - \revratio}{H + M + L}$ and thus $\exanteutilratio = \revratio$. In this case, by increasing trading price $\price$ in the {\FixPrice}, the seller's ex ante utility increases continuously (due to the concavity of revenue curve $\revcurve$) and the buyer's ex ante utility decreases continuously. Invoking the intermediate value theorem, there exists price $\fprice \in (\price, \optreserve)$ such that both traders' ex ante utilities is at least $\exanteutilratio$ fraction of their benchmarks $\sellerbenchmark,\buyerbenchmark$, respectively.
        \item Suppose that $\revratio \geq \frac{H + M - \revratio}{H + M + L}$ and thus $\exanteutilratio = \revratio - {\frac{H + M + L}{H + M + L + 1}\left(\revratio - \frac{H + M - \revratio}{H + M + L}\right)}$. In this case, let $\Delta \triangleq {\frac{H + M + L}{H + M + L + 1}\left(\revratio - \frac{H + M - \revratio}{H + M + L}\right)}\geq 0$. By decreases trading price $\price$ in the {\FixPrice}, the seller's ex ante utility decreases continuously (due to the concavity of revenue curve $\revcurve$). Let $\price\primed < \price$ be the trading price such that the seller's ex ante   utility (aka., revenue) is equal to $\revratio - \Delta$. Under trading price $\price\primed$, the buyer's ex ante utility is at least $H + M - \revratio + \Delta$. (To see this, note that by decreasing trading price from $\price$ to $\price\primed$, the GFT weakly increases and the seller's ex ante utility decreases by $\Delta$. Thus, the buyer's ex ante utility increases by at least $\Delta$.) Due to the definition of $\Delta$, two traders' ex ante utilities in the {\FixPrice} ($\FPM_{\price\primed}$) with trading price $\price\primed$ satisfy
        \begin{align*}
            \frac{\sellerexanteutil(\FPM_{\price\primed})}{\sellerbenchmark} = \revratio - \Delta = \frac{H + M - \revratio + \Delta}{H + M + L}
            \leq 
            \frac{\buyerexanteutil(\FPM_{\price\primed})}{\buyerbenchmark}
        \end{align*}
        If the inequality above holds with equality, the {\FixPrice} with trading price $\price\primed$ is {\ksfair} and both traders' ex ante utilities are at least $\exanteutilratio$ fraction of their benchmarks $\sellerbenchmark,\buyerbenchmark$, respectively. Otherwise, we can invoke argument in the previous case.
    \end{itemize}
    Summarizing the analysis above, the GFT-approximation of the {\ksfair} {\FixPrice} ($\FPM_{\fprice}$) with trading price $\fprice$ can be computed as 
    \begin{align*}
        \frac{\GFT{\FPM_{\fprice}}}{\OPTSB}
        \geq
        \frac{\exanteutilratio \cdot (\sellerbenchmark + \buyerbenchmark)}{\expect[\val]{\val}}
        =
        \exanteutilratio + \frac{\exanteutilratio}{H + M + L}
    \end{align*}

    \xhdr{Step 3- Formulating GFT approximation as two-player game.} Putting all the pieces together, the optimization program in the lemma statement can be viewed as a two-player zero-sum game between a min player (adversary) and a max player (ourself as GFT-approximation prover). The payoff in this game is the GFT approximation lower bound $\exanteutilratio + \frac{\exanteutilratio}{H + M + L}$ shown above. As a reminder
    , quantities $\exanteutilratio, H, M, L$ depend on both the buyer's valuation distribution and constant $\revratio$ used in the analysis. The min player chooses the worst MHR distribution (equivalently, convex cumulative hazard rate function) of the buyer, and the max player chooses constant $\revratio$. Importantly, the choice of constant $\revratio$ can depend on the buyer's valuation distribution. To capture this, we formulate this two-player zero-sum game in three stages: 
    \begin{itemize}
        \item (Stage 1) The min player (adversary) chooses monopoly quantile $\optreserve$ and $H$.
        \item (Stage 2) The max player (ourself as GFT-approximation prover) chooses constant $\revratio$ for the analysis.
        \item (Stage 3) The min player (adversary) chooses $\val_0, \price, M, L$.
    \end{itemize}
    It remains to verify that all constraints in the optimization program capture the feasibility condition for the both min player and max player's actions. We next verify the non-trivial constraints individually. 
    \begin{itemize}
        \item (Bounds for monopoly reserve $\optreserve$) Recall that we normalize the monopoly revenue to be one. Hence, the lower bound that $\optreserve \geq 1$ is trivial. Meanwhile, the upper bound that $\optreserve \leq e$ is due to the MHR condition \citep[Lemma~1 in][]{AGM-09}.
        \item (Bounds for price $\price$) Recall that $\price\in[0, \optreserve]$ is the smallest price such that $\price\cdot (1 - \buyercdf(\price)) \geq \revratio$. Hence, $\price \geq \revratio$.
        Moreover, the convexity of cumulative hazard rate function $\cumhazard$ implies that 
        \begin{align*}
            \cumhazard(\optreserve) + (\price - \optreserve)\cdot \cumhazard'(\optreserve) \leq \cumhazard(\price) \leq \cumhazard(\optreserve) \cdot \frac{\price}{\optreserve}
        \end{align*}
        By the definition of cumulative hazard rate function $\cumhazard$ and our normalization of the monopoly revenue to be one, we have $\cumhazard(\optreserve) = \ln(\optreserve)$, $\cumhazard(\price) = \ln(\price/\revratio)$, and $\cumhazard'(\optreserve) = 1/{\optreserve}$.\footnote{To see $\cumhazard'(\optreserve) = 1/{\optreserve}$, consider the equal-revenue distribution (with monopoly revenue equal to one). The cumulative hazard rate function of the equal-revenue distribution is $\ln(\val)$ for $\val\in[1, \infty)$. Since buyer's valuation distribution $\buyerdist$ is MHR and we normalize its monopoly revenue to one, its induced cumulative hazard rate function $\cumhazard$ is convex and has at most one intersection with $\ln(\val)$ at $\val=\optreserve$. This implies that the derivatives of two curves are identical at $\optreserve$, i.e., $\cumhazard'(\optreserve) = 1/\optreserve$. See the black dashed line in \Cref{fig:GFT program:mhr buyer} for an illustration.} Hence, the inequality above is equivalent to  
        \begin{align*}
            -\optreserve\cdot \LambertFunc\left(-\frac{\revratio}{e}\right) 
            \leq \price \leq  
            - \optreserve\cdot \LambertFunc\left(-\frac{\revratio\ln(\optreserve)}{\optreserve}\right)\cdot \frac{1}{\ln(\optreserve)}
        \end{align*}
        \item (Upper bound for value $\val_0$) Recall that $\val_0 \triangleq \price - \cumhazard(\price) / \cumhazard'(\price)$. The convexity of cumulative hazard rate function $\cumhazard$ implies that 
        \begin{align*}
            \cumhazard'(\quant) \leq \frac{\cumhazard(\optreserve) - \cumhazard(\price)}{\optreserve - \price} = \frac{\ln(\optreserve) - \ln(\price/\revratio)}{\optreserve - \price}
        \end{align*}
        After rearranging, we obtain 
        \begin{align*}
            \val_0 \leq \optreserve - \frac{\ln(\optreserve)}{\ln(\optreserve) - \ln(\frac{\price}{\revratio})}\cdot (\optreserve - \price)
        \end{align*}
        as stated in the constraint. 
        
        \item (Bounds for truncated GFT $L$) Recall that $L \triangleq \expect[\val]{\val\cdot \indicator{\val < \price}}$. The convexity of the cumulative hazard rate function implies that for every value $\val \in [0, \price]$,
        \begin{align*}
            \plus{
            \frac{\val - \val_0}{\price - \val_0}
            \cdot \cumhazard\left(\price\right)
            }
            \leq 
            \cumhazard(\val) 
            \leq
            \frac{\val}{\price}\cdot \cumhazard\left(\price\right)
        \end{align*}
        where both inequalities bind at $\val = \price$ and $\val = 0$. Moreover, $\cumhazard(\price) = \ln(\price/\revratio)$. The left-hand side and right-hand side can be viewed as two cumulative hazard rate functions $\cumhazard_1, \cumhazard_2$ that sandwich the original cumulative hazard rate function $\cumhazard$ (see blue and red cumulative hazard rate functions illustrated in \Cref{fig:GFT program:mhr buyer}). It can be verified that $\LOverBar$ and $\LUnderBar$ are $\expect[\val]{\val\cdot \indicator{ \val < \price}}$ where the random value $\val$ is realized from valuation distribution induced by those two cumulative hazard rate functions $\cumhazard_1, \cumhazard_2$, respectively. Invoking \Cref{lem:cum hazard rate monotonicity}, we obtain $\LUnderBar\leq L \leq \LOverBar$ as stated in the constraint, since these two cumulative hazard rate functions $\cumhazard_1, \cumhazard_2$ sandwich the original cumulative hazard rate function $\cumhazard$.
        \item (Bounds for truncated GFT $M$) The argument is similar to the argument above for truncated GFT $L$. Recall that $M \triangleq \expect[\val]{\val\cdot \indicator{ \price \leq \val <\optreserve }}$. The convexity of the cumulative hazard rate function implies that for every quantile $\val \in [\price, \optreserve]$,
        \begin{align*}
            \max\left\{
            \frac{\val - \val_0}{\price - \val_0}
            \cdot 
            \cumhazard\left(\price\right)
            ,
            \cumhazard(\optreserve) - (\optreserve - \val)\cdot \cumhazard'(\optreserve) 
            \right\}
            \leq 
            \cumhazard(\val) 
            \leq
            \frac{\optreserve - \val}{\optreserve - \price}
            \cdot \left(\cumhazard(\optreserve) - \cumhazard\left(\price\right)\right)
            +
            \cumhazard\left(\price\right)
        \end{align*}
        where both inequalities bind at $\val = \price$ and $\val = \optreserve$. Moreover, $\cumhazard(\price) = \ln(\price/\revratio)$, $\cumhazard(\optreserve) = \ln(\optreserve)$, and $\cumhazard'(\optreserve) = 1/\optreserve$. The left-hand side and right-hand side can be viewed as two cumulative hazard rate functions $\cumhazard_1, \cumhazard_2$ that sandwich the original cumulative hazard rate function $\cumhazard$ (see blue and red cumulative hazard rate functions illustrated in \Cref{fig:GFT program:mhr buyer}). It can be verified that $\MOverBar$ and $\MUnderBar$ are $\expect[\val]{\val\cdot \indicator{ \price \leq \val <\optreserve }}$ where the random value $\val$ is realized from valuation distribution induced by those two cumulative hazard rate functions $\cumhazard_1, \cumhazard_2$, respectively. Invoking \Cref{lem:cum hazard rate monotonicity}, we obtain $\MUnderBar\leq M \leq \MOverBar$ as stated in the constraint, since these two cumulative hazard rate functions $\cumhazard_1, \cumhazard_2$ sandwich the original cumulative hazard rate function $\cumhazard$.
        \item (Bounds for truncated GFT $H$) The argument is similar to the argument above for truncated GFT $L$. Recall that $H \triangleq \expect[\val]{\val\cdot \indicator{\val \geq \optreserve}}$. The convexity of the cumulative hazard rate function implies that for every value $\val \in [\optreserve, \infty)$,
        \begin{align*}
            \cumhazard(\optreserve) + (\val - \optreserve) \cdot 
            \cumhazard'(\optreserve)
            \leq 
            \cumhazard(\val) 
        \end{align*}
        where inequality binds at $\val = \optreserve$. The left-hand side can be viewed as cumulative hazard rate functions $\cumhazard_1$. Moreover, consider another cumulative hazard function $\cumhazard_2$ such that $\cumhazard_2(\optreserve) = \cumhazard(\optreserve)$ and $\cumhazard_2(\val) = \infty$ for every $\val > \optreserve$. Note that $\cumhazard_1$ and $\cumhazard_2$ sandwich the original cumulative hazard rate function $\cumhazard$ (see blue and red cumulative hazard rate functions illustrated in \Cref{fig:GFT program:mhr buyer}). It can be verified that $\HOverBar$ and $\HUnderBar$ are $\expect[\val]{\val\cdot \indicator{  \val \geq\optreserve }}$ where the random value $\val$ is realized from valuation distribution induced by those two cumulative hazard rate functions $\cumhazard_1, \cumhazard_2$, respectively. Invoking \Cref{lem:cum hazard rate monotonicity}, we obtain $\HUnderBar\leq H \leq \HOverBar$ as stated in the constraint, since these two cumulative hazard rate functions $\cumhazard_1, \cumhazard_2$ sandwich the original cumulative hazard rate function $\cumhazard$.
        \item (Equation for value $\val_1$) Recall that  $\val_0 \triangleq \price - \cumhazard(\price) / \cumhazard'(\price)$ and $\val_1 \triangleq (\cumhazard(\price) - \cumhazard(\optreserve) + \optreserve\cdot \cumhazard'(\optreserve) - \price\cdot \cumhazard'(\price))/(\cumhazard'(\optreserve) - \cumhazard'(\price))$. Combining both equations with $\cumhazard(\price) = \ln(\price/\revratio)$, $\cumhazard(\optreserve) = \ln(\optreserve)$, and $\cumhazard'(\optreserve) = 1/\optreserve$, we obtain 
        \begin{align*}
            \val_1 = \left(\frac{1}{\optreserve} - \frac{1}{\price - \val_0}\ln\left(\frac{\price}{\revratio}\right)\right)^{-1}\left(
        \ln\left(\frac{\price}{\revratio}\right) - \ln(\optreserve)
        + 1 
        - \frac{\price}{\price-\val_0}\ln\left(\frac{\price}{\revratio}\right) 
        \right)~,
        \end{align*}
        as stated in the optimization program.
    \end{itemize}
    Finally, we numerically evaluation the optimization program and obtain $\fixedPriceGFTPercentageMHR$. We present more details of this numerical evaluation in \Cref{apx:numerical evaluation:mhr buyer}.
    This completes the proof of \Cref{lem:GFT program:mhr buyer}.
\end{proof}

\begin{lemma}
\label{lem:cum hazard rate monotonicity}
    Given any two distributions $\buyerdist_1,\buyerdist_2$ and any two value $\val\primed, \val\doubleprimed$ with $\val\primed \leq \val\doubleprimed$. 
    Suppose $\buyercdf_1(\val\primed) = \buyercdf_2(\val\primed)$ and $\buyercdf_1(\val\doubleprimed) = \buyercdf_2(\val\doubleprimed)$. If the induced cumulative hazard rate functions $\cumhazard_1,\cumhazard_2$ satisfy that for every value $\val \in[\val\primed,\val\doubleprimed]$,
    $\cumhazard_1(\val) \geq \cumhazard_2(\val)$,
    then 
    \begin{align*}
        \expect[\val\sim\buyerdist_1]{\val\cdot \indicator{\val\primed \leq \val\leq \val\doubleprimed}}
        \leq 
        \expect[\val\sim\buyerdist_2]{\val\cdot \indicator{\val\primed \leq \val\leq \val\doubleprimed}}
    \end{align*}
\end{lemma}
\begin{proof}
    Since for every value $\val \in[\val\primed,\val\doubleprimed]$,
    $\cumhazard_1(\val) \geq \cumhazard_2(\val)$ and the inequality is binding at $\val = \val\primed$ and $\val = \val\doubleprimed$, it is guaranteed that for every value $\val\in[\val\primed,\val\doubleprimed]$, $\buyercdf_1(\val) \geq \buyercdf_2(\val)$
    and the inequality is binding at $\val = \val\primed$ and $\val = \val\doubleprimed$. This is sufficient to prove the lemma statement:
    \begin{align*}
        &\expect[\val\sim\buyerdist_1]{\val\cdot \indicator{\val\primed \leq \val\leq \val\doubleprimed}} 
        =
        \displaystyle\int_{\val\primed}^{\val\doubleprimed}
        \val\cdot \d \buyercdf_1(\val) 
        \overset{(a)}{=}
        \val\doubleprimed \buyercdf_1(\val\doubleprimed)
        -
        \val\primed\buyercdf_1(\val\primed)
        -
        \displaystyle\int_{\val\primed}^{\val\doubleprimed}
        \buyercdf_1(\val) \cdot \d \val
        \\
        &\qquad\overset{(b)}{\leq}
        \val\doubleprimed \buyercdf_2(\val\doubleprimed)
        -
        \val\primed\buyercdf_2(\val\primed)
        -
        \displaystyle\int_{\val\primed}^{\val\doubleprimed}
        \buyercdf_2(\val) \cdot \d \val
        \overset{(c)}{=}  
        \displaystyle\int_{\val\primed}^{\val\doubleprimed}
        \val\cdot \d \buyercdf_2(\val) 
        =
        \expect[\val\sim\buyerdist_2]{\val\cdot \indicator{\val\primed \leq \val\leq \val\doubleprimed}} 
    \end{align*}
    where equalities~(a) (c) hold due to the integration by parts, and inequality~(b) holds as we argued above. This complete the proof of \Cref{lem:cum hazard rate monotonicity}.
\end{proof}
