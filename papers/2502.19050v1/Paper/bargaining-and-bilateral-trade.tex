In this section, we discuss the relationship between fair bilateral trade and cooperative bargaining. First, we discuss an abstract model of cooperative bargaining, and discuss a reduction from picking a truthful mechanism for a bilateral trade instance, to picking a solution to a bargaining problem. Next, we discuss several prominent solutions for the cooperative bargaining problem and reexamine them through the lens of bilateral trade. We show a connection between fairness notions discussed in this paper for bilateral trade, and 
solutions to bargaining problems.

Consider a bilateral instance $\btinstance$. Recall that we denote by $\mechfamily$ the family of all BIC, IIR  and ex ante WBB mechanisms for $\btinstance$. For each such mechanism $\mech\in \mechfamily$, we can compute the ex ante utilities of the buyer and the seller, which are  $\buyerexanteutil(\mech)$ and $ \sellerexanteutil(\mech)$, respectively. We are interested in studying fairness notions for bilateral trade mechanisms. In essence, this amounts to selecting which pairs of utility values $(\buyerexanteutil(\mech), \sellerexanteutil(\mech))$ are considered fair (ex ante). Thus, instead of thinking of a trading problem over an item, one can  take an ex ante view and think of the problem of picking a fair truthful mechanism as a cooperative bargaining problem over ex ante utilities of truthful mechanisms.

The general study of cooperative bargaining problems has a rich history, originating from the seminal work of Nash \cite{N-50}. We begin by presenting an abstract model of cooperative bargaining.

\begin{definition} [Bargaining Problem]
    A \emph{bargaining problem} for two agents is a pair $(\bargainprob,\dispnt)$ where $\bargainprob$ is a bounded, closed and convex subset of $\RR\times\RR$ and $\dispnt\in \bargainprob$ is the disagreement point.

    A \emph{bargaining solution} to the bargaining problem is a function $\bargainsolution$ that given a pair $(\bargainprob,\dispnt)$ outputs an agreement point $\bargainsolution(\bargainprob,\dispnt) \in \bargainprob$.
\end{definition}

Intuitively, the coordinates of each point in $\bargainprob$ represent the utilities of the agents when agreeing on that outcome. {Throughout the paper, for a given point $\pnta\in \RR\times\RR$ we will denote the utilities of agents by $\buyercoord{\pnta}=x_1$ and $\sellercoord{\pnta}=x_2$. The disagreement point $\dispnt$ represents the utilities that each agent gets when no agreement is reached. Note that the agents will never agree on a solution $\pnta\in \bargainprob$ if $x_i < d_i$ for some agent $i$. For our purposes we will always assume that $\dispnt=(0,0)$ and omit $\dispnt$ entirely.

For a bilateral trade instance $\btinstance$ we will define a bargaining problem $\bargainprobBT$ where each point corresponds to the ex ante utilities of the agents for some BIC, IIR and ex ante WBB mechanism. Formally:
\begin{definition} [Bilateral Trade Bargaining] 
The bargaining set that corresponds to the bilateral trade instance $\btinstance$ is:
\begin{align*}
    \bargainprobBT=\left\{(\buyerexanteutil(\mech),\sellerexanteutil(\mech)) \mid \mech \in \mechfamily\right\}
\end{align*}
\end{definition}


Note that the set $\bargainprobBT$ is indeed convex (and therefore compact): for any two mechanisms $\mech_1,\mech_2\in \mechfamily$ and constant probability $\mixprob$, we can define a mechanism $\mech$ which runs $\mech_1$ with probability $\mixprob$ and $\mech_2$ with probability $1-\mixprob$, and the ex ante utilities of this new mechanism will be the convex combination of the utilities of each mechanism. Note that this new mechanism is also in $\mechfamily$ since randomizing between two BIC, IIR and WBB mechanisms maintains these properties. Additionally, the set $\bargainprobBT$ is closed since the set of possible utility outcomes from the set of BIC, IIR and ex ante WBB mechanisms is closed, as these requirements all correspond to weak inequalities. Finally, the disagreement point for the bilateral trade bargaining problem is always $\dispnt=(0,0)$, since if the agents cannot reach an agreement then no trade occurs and neither gains or loses utility.


An important aspect of the bilateral trade problem that is not captured by the corresponding bargaining problem is the profit of the mechanism. WBB mechanisms allow the buyer to pay a higher price than the seller receives, and this additional payment becomes the profit of the mechanism. Importantly, this payment is considered as part of the GFT of the mechanism, yet it is not visible in the bargaining problem. Thus, given a bilateral trade instance $\btinstance$, each point $\pnta\in\bargainprobBT$ corresponds to a BIC, IIR and ex post WBB mechanism $\mech\in\mechfamily$ with $\GFT{\mech} \geq \buyercoord{\pnta}+\sellercoord{\pnta}$, {and the inequality might be strict}. 

A consequence of this difference is that given a bilateral trade instance $\btinstance$, the set of Second-Best mechanisms (those that maximize the GFT) is not necessarily on the Pareto front of the bargaining problem $\bargainprobBT$ (the set of Pareto efficient points). In \cite{BCWZ-17} it is shown that any BIC, IIR and ex ante WBB mechanism can be converted to a BIC, IIR and ex post SBB mechanism with the same GFT, which implies that at least one of the Second-Best mechanisms should be on the Pareto front, but this reduction may change the relative utilities of the traders. Thus the KS solution (see \Cref{def:ks-solution}) to the bargaining problem $\bargainprobBT$ may not correspond to the {\ksfair} mechanism with the highest GFT. To show that this is not the case, it is sufficient to convert a BIC, IIR, ex ante WBB and {\ksfair} mechanism into a BIC, IIR, ex ante \emph{SBB} and {\ksfair} mechanism with the same GFT. A natural attempt to accomplish this is to proportionally ex ante split the ex ante gains of the mechanism between the two traders (an ex ante split does not affect incentive compatibility or participation). However, using this method there might be ex post positive transfer to the buyer, which is impractical and unnatural. We leave this challenge for future work. 



\subsection{Bargaining Solutions}

There are many proposed solutions to the bargaining problem, each corresponding to different notions of fairness\footnote{{Each of the solutions can be proven to match a different set of axiomatic assumptions on the behavior of the solution. We will not discuss this axiomatic analysis in this paper. See \cite{T-94} for a general overview.}}. In this paper we will discuss the three most prominent solutions:
\begin{enumerate}
    \item \emph{The Egalitarian solution:} the maximal\footnote{{Throughout this paper we will use the term "maximal" to refer to the "north-east" point on a closed line with positive slope. Formally, this will be the point $\pnta$ on the line which maximizes $\buyercoord{\pnta} + \sellercoord{\pnta}$. Since we will only consider lines with positive slope, this will be equivalent to maximizing $\buyercoord{\pnta}$ or $\sellercoord{\pnta}$ separately.}} outcome in which both agents receive the same utility.
    \item \emph{The Kalai-Smorodinsky solution:} the maximal outcome in which both agents receive the same portion of their optimal utility.
    \item \emph{The Nash solution:} the outcome which maximizes the product of both agents' utilities.
\end{enumerate}

We will now use the reduction from bilateral-trade to the bargaining problem defined above to adapt these solutions into fairness notions for bilateral trade.

\subsubsection{Egalitarian Solution}
One solution to the bargaining problem is the egalitarian solution, originally presented in \citet{Kal-77,Mye-77}:

\begin{definition}
    The \emph{egalitarian solution} $\pnta=\soleg(\bargainprob)$ is the maximal point $\pnta\in\bargainprob$ of equal coordinates $\buyercoord{x}=\sellercoord{x}$. 
\end{definition}

A useful intuition is to think of the egalitarian solution as intersection between the {north-east} (Pareto) boundary of $\bargainprob$ and the 45-degree ray from the origin.

Consider now a bilateral-trade instance $\btinstance$ and the corresponding bargaining problem $\bargainprobBT$. The egalitarian solution $\soleg(\bargainprobBT)$ corresponds to the mechanism $\mech\in \mechfamily$ that satisfies $\buyerexanteutil(\mech) = \sellerexanteutil(\mech)$ {with the highest ex ante utility for both agent's}. This incentivizes the following fairness notion:

\begin{restatable}[Equitability]{definition}{defequitability}
\label{def:equitable}
    {For a given bilateral trade instance $\btinstance$, mechanism $\mech\in \mechfamily$} is \emph{\equitable} if two players achieve the same ex ante utilities, i.e., $\sellerexanteutil(\mech) = \buyerexanteutil(\mech)$.
\end{restatable}

Thus, in bilateral trade the egalitarian solution corresponds to the equitable mechanism $\mech$ {that maximizes $\buyerexanteutil(\mech) + \sellerexanteutil(\mech)$} (or equivalently, either $\buyerexanteutil(\mech)$ or $\sellerexanteutil(\mech)$ individually). Equitability is certainly a desirable fairness notion, but, unfortunately, it also too stringent a requirement in many cases: equitable mechanisms cannot guarantee any constant fraction of the {\SecondBest}, even when limited to a zero-cost seller any buyer with regular distribution. An analysis of equitable mechanisms can be found in \Cref{appendix:equitablity}.

\subsubsection{Kalai-Smorodinsky Solution}

\label{subsec:ks-solution}

Throughout this paper we study the fairness notion of {\ksfairness} for bilateral trade.\footnote{See \Cref{sec:ksfairness} for a formal definition.} This notion is related to the Kalai-Smorodinsky solution to the bargaining problem, originally presented in \citep{KS-75}.
This solution picks the Pareto (maximal) point that equalizes the ratio of each agent's utility to her ideal utility.



\begin{definition}
\label{def:ks-solution}
    For a given bargaining problem $\bargainprob$, the \emph{ideal point} $\idealp{\bargainprob}$ of $\bargainprob$ is defined $\idealbuyer{\bargainprob}=\max\left\{\buyercoord{\pnta}\mid\pnta\in \bargainprob\right\}$ and $\idealseller{\bargainprob}=\max\left\{\sellercoord{\pnta}\mid \pnta\in \bargainprob\right\}$.
    
    The \emph{Kalai-Smorodinsky solution} $\pnta=\solks(\bargainprob)$ is the point $\pnta\in \bargainprob$ satisfying $\frac{\buyercoord{\pnta}}{\idealbuyer{\bargainprob}}=\frac{\sellercoord{\pnta}}{\idealseller{\bargainprob}}$ which maximizes $\buyercoord{\pnta}+\sellercoord{\pnta}$. Alternatively, it is the maximal point of $\bargainprob$ on the line connecting the origin to $\idealp{\bargainprob}$. {We refer to the line connecting the ideal point and the origin as the \emph{\ksline}}.
\end{definition}


For a bilateral trade instance $\btinstance$ we consider the corresponding bargaining problem $\bargainprobBT$. The ideal point of this problem is
\begin{align*}
\idealp{\bargainprobBT}= \left(\max_{\mech\in\mechfamily}\buyerexanteutil(\mech),\max_{\mech\in\mechfamily}\sellerexanteutil(\mech)\right)   =  (\buyerbenchmark, \sellerbenchmark)
\end{align*}

Thus, the Kalai-Smorodinsky KS solution corresponds to the mechanism $\mech\in \mechfamily$ {which maximizes $\buyerexanteutil(\mech)$ among all those satisfying} {\ksfairness}.


We note that given a bilateral trade instance $\btinstance$, it is possible to explicitly cast the problem of finding a KS solution mechanism $\mech$ as a linear optimization problem  (See \Cref{sec:ksfairness} for a detailed description of this program). Moreover, \Cref{lem:SBB implication} implies that all the positive results of this paper (\Cref{thm:optimal GFT:general instance,thm:improved GFT:mhr traders,thm:improved GFT:regular buyer,thm:improved GFT:mhr buyer}) for the approximation to the {\SecondBest} achievable by a {\ksfair} and truthful mechanisms also apply to the KS solution. Additionally, the upper bounds presented in these results also hold for the KS solution in their corresponding settings.



\subsubsection{Nash Solution}
The Nash solution is a prominent solution to the bargaining problem, originally presented in \cite{N-50}. 

\begin{definition}
    The \emph{Nash solution} $x=\solnash(\bargainprob)$ is the maximizer of the product $x_1 \cdot x_2$.
\end{definition}

For the bilateral trade mechanism, the Nash solution will be the mechanism $
\mech \in \mechfamily$ that maximizes $\buyerexanteutil(\mech)\cdot \sellerexanteutil(\mech)$. An examination of Nash Social Welfare Maximization for bilateral trade can be found in \Cref{sec:nash fairness}. In particular, we show a tight bound of $\frac{1}{2}$ on the approximation of the {\SecondBest} obtainable by a mechanism which maximizes the (ex ante) Nash Social Welfare.


