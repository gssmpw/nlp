



In light of the established similarities between MSIP and Lloyd's algorithm, this work opens the door to further theoretical developments for MMD minimization which bypass mean-field analysis. In particular, \Cref{thm:gradient_opt_mmd} suggests that the extensive theory developed for Lloyd’s algorithm may be leveraged for MSIP. For instance, a natural question is whether this analogy extends further through an analysis of the non-degeneracy of the map $\bm{\Psi}_{\MMS}$ \cite{EmJuRa08}, which would allow us to prove that MSIP converges to critical points of the $\MMD$. In this regard, a promising direction would be to study the monotonicity of the MMD for the iterates of MSIP, where \Cref{thm:gradient_opt_mmd} might be useful. For this, the rich literature on quasi-Newton methods could further our understanding of MSIP and WFR \cite{NoWr99}. Proving a descent property for MSIP would involve studying technical properties of the method, including analyzing second order properties of the function $F_M$ as was done for Lloyd's map~\cite{DuFaGu99}. 
Beyond convergence guarantees, an interesting direction for future research is to explore the impact of this approach on the understanding of neural network training, as studied in \cite{ArKoSaGr19,RoJeBrVa19}, and on sparse measure reconstructions \cite{DeGa12,Chi22,BeGr24}. Generally, our methods show promising results for mode-seeking using interacting particle systems, which we hope will better tackle general non-convex high-dimensional approximation problems.

\section{Acknowledgements}
AB, DS, and YM acknowledge support from the US Department of Energy (DOE), Office of Advanced Scientific Computing Research, under grants DE-SC0021226 (FASTMath SciDAC Institute) and DE-SC0023188. AB and YM also acknowledge support from the ExxonMobil Technology and Engineering Company. The authors also acknowledge the MIT SuperCloud and Lincoln Laboratory Supercomputing Center for providing HPC resources that have contributed to the research results reported within this paper.















