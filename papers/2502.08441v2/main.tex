% This must be in the first 5 lines to tell arXiv to use pdfLaTeX, which is strongly recommended.
\pdfoutput=1
% In particular, the hyperref package requires pdfLaTeX in order to break URLs across lines.

\documentclass[11pt]{article}

\usepackage[dvipsnames]{xcolor} 
% Change "review" to "final" to generate the final (sometimes called camera-ready) version.
% Change to "preprint" to generate a non-anonymous version with page numbers.
\usepackage[preprint]{acl}

\usepackage[utf8]{inputenc}

%% full size page
\usepackage{fullpage}
\usepackage[margin = 2.5cm]{geometry}

%% AMS packages
\usepackage{amsmath, amsthm, amssymb}
\usepackage{thmtools}
\usepackage{thm-restate}
\usepackage{mathtools}  
\usepackage{xfrac} 
%\usepackage[normalem]{ulem}

\usepackage{placeins} 
\usepackage{times}

%%bibliography
% \usepackage{amsthm}
\usepackage{url}
\usepackage{array}


%% algorithm
%\usepackage{algorithm}
%\usepackage{algpseudocode}
\usepackage[ruled,vlined]{algorithm2e}
% Create a new environment named "algorithm2e" that behaves like "algorithm"
\newenvironment{algorithm2e}[1][]{%
    \begin{algorithm}[#1]%
}{%
    \end{algorithm}
}

%% graphics and colors
\usepackage{graphicx}
\usepackage{color}
%\usepackage[dvipsnames]{xcolor}

%%bibliography
\usepackage[round]{natbib}
\usepackage[hyperindex,breaklinks]{hyperref}
\usepackage{url}

%%color box figure
\usepackage{tcolorbox}

%%draw figure
\usepackage{tikz}

%%misc
\usepackage{nicefrac}

%\citestyle{acmauthoryear}

% \declaretheorem[name=Theorem]{theorem}

%% define theorems, lemmas, claims
\newtheorem{theorem}{Theorem}[section]
\newtheorem{claim}[theorem]{Claim}
\newtheorem{corollary}[theorem]{Corollary}
\newtheorem{proposition}[theorem]{Proposition}
\newtheorem{lemma}[theorem]{Lemma}
\newtheorem{definition}[theorem]{Definition}
\newtheorem{observation}[theorem]{Observation}
\newtheorem{question}[theorem]{Question}
\newtheorem{assumption}[theorem]{Assumption}
\newtheorem*{remark*}{Remark}


% \newenvironment{numberedtheorem}[1]{%
% \renewcommand{\thetheorem}{#1}%
% \begin{theorem}}{\end{theorem}\addtocounter{theorem}{-1}}

% \newenvironment{numberedlemma}[1]{%
% \renewcommand{\thetheorem}{#1}%
% \begin{lemma}}{\end{lemma}\addtocounter{theorem}{-1}}

% \newenvironment{oneshot}[1]{\@begintheorem{#1}{\unskip}}{\@endtheorem}

% \makeatletter
% \newtheorem*{rep@theorem}{\rep@title}
% \newcommand{\newreptheorem}[2]{%
% \newenvironment{rep#1}[1]{%
%  \def\rep@title{#2 \ref{##1}}%
%  \begin{rep@theorem}}%
%  {\end{rep@theorem}}}
% \makeatother


%cleveref package loaded at the end
\usepackage{cleveref}
\crefname{theorem}{theorem}{theorems}
\Crefname{theorem}{Theorem}{Theorems}

\crefname{lemma}{lemma}{lemmas}
\Crefname{lemma}{Lemma}{Lemmas}

\crefname{claim}{claim}{claims}
\Crefname{claim}{Claim}{Claims}

\crefname{corollary}{corollary}{corollaries}
\Crefname{corollary}{Corollary}{Corollaries}

\crefname{proposition}{proposition}{propositions}
\Crefname{proposition}{Proposition}{Propositions}

\crefname{definition}{definition}{definitions}  % Explicitly set "Definition"
\Crefname{definition}{Definition}{Definitions}

\crefname{observation}{observation}{observations}
\Crefname{observation}{Observation}{Observations}

\crefname{question}{question}{questions}
\Crefname{question}{Question}{Questions}

\crefname{assumption}{assumption}{assumptions}
\Crefname{assumption}{Assumption}{Assumptions}

\crefname{algorithm}{algorithm}{algorithms}
\Crefname{algorithm}{Algorithm}{Algorithms}

\crefname{AlgoLine}{line}{lines}  % If using `algorithm2e` with line numbers
\Crefname{AlgoLine}{Line}{Lines}


%% probability notation
\DeclareMathOperator{\cov}{cov}
\DeclareMathOperator{\sgn}{\mathbf{sgn}}
\DeclareMathOperator{\E}{\mathbf{E}}
\DeclareMathOperator{\Var}{\mathbf{Var}}
\DeclareMathOperator{\one}{\mathbf{1}}
\newcommand{\given}{\mid}
\DeclareMathOperator{\Ball}{Ball}
\DeclareMathOperator{\tr}{tr}

% rounding up and down
\newcommand {\roundup}   [1] {{\lceil {#1} \rceil}}
\newcommand {\rounddown} [1] {{\lfloor {#1} \rfloor}}

%% black board letters
\newcommand{\bbB}{\mathbb{B}}
\newcommand{\bbC}{\mathbb{C}}
\newcommand{\bbR}{\mathbb{R}}
\newcommand{\bbZ}{\mathbb{Z}}

%% calligraphic letters 
\newcommand{\calA}{\mathcal{A}}
\newcommand{\calB}{\mathcal{B}}
\newcommand{\calC}{\mathcal{C}}
\newcommand{\calD}{\mathcal{D}}
\newcommand{\calE}{\mathcal{E}}
\newcommand{\calF}{\mathcal{F}}
\newcommand{\calG}{\mathcal{G}}
\newcommand{\calH}{\mathcal{H}}
\newcommand{\calI}{\mathcal{I}}
\newcommand{\calJ}{\mathcal{J}}
\newcommand{\calK}{\mathcal{K}}
\newcommand{\calL}{\mathcal{L}}
\newcommand{\calM}{\mathcal{M}}
\newcommand{\calN}{\mathcal{N}}
\newcommand{\calO}{\mathcal{O}}
\newcommand{\calP}{\mathcal{P}}
\newcommand{\calQ}{\mathcal{Q}}
\newcommand{\calR}{\mathcal{R}}
\newcommand{\calS}{\mathcal{S}}
\newcommand{\calT}{\mathcal{T}}
\newcommand{\calU}{\mathcal{U}}
\newcommand{\calV}{\mathcal{V}}
\newcommand{\calW}{\mathcal{W}}
\newcommand{\calX}{\mathcal{X}}
\newcommand{\calY}{\mathcal{Y}}
\newcommand{\calZ}{\mathcal{Z}}

\newcommand{\N}{\mathbb{N}}
\newcommand{\R}{\mathbb{R}}
\DeclareMathOperator*{\argmin}{arg\,min}
\DeclareMathOperator*{\argmax}{arg\,max}




\newcommand{\err}{\mathrm{err}}
\newcommand{\errstar}{\mathrm{err}^*}

\DeclareMathOperator{\REC}{REC}
\DeclareMathOperator{\NRD}{NRD}
\DeclareMathOperator{\FN}{FN}
\DeclareMathOperator{\FP}{FP}

\DeclareMathOperator{\Tr}{Tr}

\newcommand{\conv}{\mathrm{conv}}
\newcommand{\cone}{\mathrm{cone}}
\newcommand{\inner}[2]{\langle #1, #2\rangle}
\section{Problem Studied}\label{sec:def}
We first present Fixed-Radius Near Neighbor (FRNN) queries and then formalize Aggregation Queries over Nearest Neighbors (AQNNs) that build on them. We then state our problem.

\subsection{Nearest Neighbor Queries}\label{subsec:FRNN}
We build on generalized Fixed-Radius Near Neighbor (FRNN) queries \cite{FRNNSurvey}. Given a dataset \( D \), a query object \( q \), a radius \( r \), and a distance function \( dist \), a generalized FRNN query retrieves all nearest neighbors of \( q \) within radius \( r \). More formally:
\[
NN_D(q, r) = \{x \in D \mid dist(x, q) \leq r\},
\]
where \(x\) is any data point in \(D\) and \(dist(x, q)\) denotes the distance between them. We use \(|NN_D(q,r)|\) to denote the neighborhood size of \(q\). As shown in Fig. \ref{fig:framework}, given a radius \(r\) and a target patient \(q\), patients in the dotted circle are nearest neighbors, and the neighborhood size is 6.

\subsection{Aggregation Queries over Nearest Neighbors}\label{subsec:AQNN} 
Given an FRNN query object \(q\) in dataset \(D\), a radius \(r\), and an attribute \(\texttt{attr}\), an Aggregation Query over Nearest Neighbors (AQNN) is defined as:
\[ \text{agg}(NN_D(q,r)[\texttt{attr}]) \]
where agg is an aggregation function, such as $\mathtt{AVG}$, $\mathtt{SUM}$, and $\mathtt{PCT}$, and \(NN_D(q,r)[\texttt{attr}]\) denotes the bag of values of attribute \texttt{attr} of all FRNN results of \(q\) within radius \(r\). 
% \end{definition}

An AQNN expresses aggregation operations to capture key insights about the neighborhood of a query object. For example, \(\mathtt{AVG}\) can be used to reflect the average heart rate or systolic blood pressure of patients in the neighborhood, providing a measure of typical health conditions. \(\mathtt{SUM}\) is useful for assessing cumulative effects, such as the total cost of treatments in the neighborhood that instructs public policy in terms of health. Similarly, $\mathtt{PCT}$ can be used to find the proportion of patients in the neighborhood of a patient of interest, relative to the population in the dataset.
%\laks{Why is finding the total \#meds to NNs or the total treatment cost of everyone in the NN interesting?}

% \texttt{MIN} and \texttt{MAX} are not included in the aggregation functions because they only capture extreme values, which may not represent the typical characteristics of the nearest neighbors and are more sensitive to outliers. 
% \laks{AVG is also sensitive to outliers, but we still allow it. isn't the real reason we don't consider MIN/MAX because they are amenable to estimation via sampling?} We choose \texttt{PCT} instead of \texttt{COUNT} in order to provide a normalized measure that remains comparable across different neighborhood sizes. It allows for more consistent interpretation of relative popularity \cite{moore1989introduction}.


Fig. \ref{fig:framework} illustrates an example of an AQNN: ``\textit{Find the average systolic blood pressure of patients similar to an insomnia patient \(q\)}''. The aggregation function is \(\mathtt{AVG}\) and the target attribute of interest is systolic blood pressure. Exact query evaluation requires consulting physicians (or predicting embeddings by an expensive machine learning model) for all 500 patients in \(D\) and calculate \(q\)'s nearest neighbors wrt \(r\) \cite{DBLP:journals/isci/RodriguesGSBA21}. We refer to such highly accurate but computationally expensive models as \textit{oracle models}, denoted as \(O\), including deep learning models trained on domain-specific data or human expert annotations \cite{DBLP:conf/sigmod/LuCKC18}. Using oracle models is very expensive \cite{sze2017efficient, DujianPQA, DBLP:journals/pvldb/KangGBHZ20}. To address that, we seek an approximate solution by \textit{proxy models}, denoted as \(P\), that are at least one order of magnitude cheaper than oracle models. In the example, if consulting physicians for one patient incurs one cost unit, calling a cheap machine learning model instead incurs at most \(0.1\) cost unit. Once the similar patients are identified, their systolic blood pressure values are averaged and returned as  output. The use of a proxy model may reduce the accuracy of the neighborhood prediction and hence, we should judiciously call oracle and proxy models to minimize the error of aggregate results.

Note that the values of the target attribute \texttt{attr} are \textit{not} predicted but are instead known quantities.

\subsection{Problem Statement}
Given an AQNN, our goal is to return an approximate aggregate result by leveraging both oracle and proxy models while reducing error and cost.


\pagecolor{anthracite}\afterpage{\nopagecolor}


\definecolor{poste_color}{RGB}{222, 184, 65}

\begin{center}
\color{white}
\begin{minipage}{0.99\linewidth}
    \thispagestyle{empty}
    \centering
    \tikz[remember picture,overlay] \node[opacity=0.3,inner sep=0pt] at (current page.center){\includegraphics[width=\paperwidth,height=\paperheight]{assets/topographic2.png}};

    \vspace{-1cm}
    

    \vspace{1cm}
    
    {\Huge {\textbf{Sparks of Explainability}}\\ Recent Advancements in Explaining Large Vision Models\\[0.5em]}

    \vspace{2cm}
    
    {
    Presented by\\
    {\textbf{\Large Thomas Fel}}\\[1em]

    
    Supervised by\\
    \textbf{Prof. Thomas Serre}\\[4em]
    }

    \vspace{1cm}
    
    {
    \raggedright
    Presented and publicly defended on July 25, 2024\\
    }
    \vspace{0em}
    \begin{tabbing}
        \hspace{8cm} \= \kill
        Prof. \textbf{George A. Alvarez} \> Reviewer\\
        \textit{\textcolor{poste_color}{Professor, Harvard University}} \\[0.5em]
        
        Prof. \textbf{Céline Hudelot} \> Reviewer\\
        \textit{\textcolor{poste_color}{Professor, Centrale Paris}} \\[0.5em]

        
        Dr. \textbf{Robert Geirhos} \> Examiner\\
        \textit{\textcolor{poste_color}{Research Scientist, Google}} \\[0.5em]
        
        Prof. \textbf{Ruth Fong} \> Examiner\\
        \textit{\textcolor{poste_color}{Professor, Princeton University}} \\[0.5em]
        
        Prof. \textbf{Rufin Van Rullen} \> Examiner\\
        \textit{\textcolor{poste_color}{Professor, Cerco, CNRS}} \\[0.5em]

        \\
        
        \textbf{Prof. Thomas Serre} \> Thesis Director\\
        \textit{\textcolor{poste_color}{Professor, Brown University \& ANITI}} \\[0.5em]
    \end{tabbing}

    \vspace{1.5cm}
    
    {\Large\textbf{DOCTORAL THESIS}\\[0.5em]}
    {\large Doctoral School of Mathematics, Computer Science, and Telecommunications of Toulouse}\\[2em]
    
\end{minipage}
\end{center}


\begin{document}
\maketitle
\begin{abstract}

% Recent works to jointly reconstruct 3D human and object from a single RGB image, are mostly model-based, that fail to capture the fine details of the clothed human body and object surface. In this paper, we introduce ReCHOR, a novel, model-free, first-method to produce realistic clothed human-object reconstructions from a monocular view. This is extremely challenging due to human-object occlusions, diverse interactions and depth ambiguity, as it needs to infer both 3D spatial awareness and high resolution details. Our core idea is based on estimating neural implicit representations for human and object respectively by an attention-based neural implicit model that attends to pixel-aligned features from both the global human-object image for spatial awareness and  the local separate view of human and object images for high quality details. Additionally, the network is conditioned on semantic features from an initial estimated human-object pose prior and a generative diffusion model that inpaints occluded regions, thus enabling the retrieval of details from them.
% We also propose a synthetic dataset with rendered scenes of diverse, inter-occluded 3D human and object scans, to train our network. We evaluate our method on the synthetic and real world BEHAVE dataset. Our experiments show that our method outperforms the SOTA in achieving realistic clothed human-object reconstructions.
Recent approaches to jointly reconstruct 3D humans and objects from a single RGB image represent 3D shapes with template-based or coarse models, which fail to capture details of loose clothing on human bodies. In this paper, we introduce a novel implicit approach for jointly reconstructing realistic 3D clothed humans and objects from a monocular view. For the first time, we model both the human and the object with an implicit representation, allowing to capture more realistic details such as clothing. This task is extremely challenging due to human-object occlusions and the lack of 3D information in 2D images, often leading to poor detail reconstruction and depth ambiguity. To address these problems, we propose a novel attention-based neural implicit model that leverages image pixel alignment from both the input human-object image for a global understanding of the human-object scene and from local separate views of the human and object images to improve realism with, for example, clothing details. Additionally, the network is conditioned on semantic features derived from an estimated human-object pose prior, which provides 3D spatial information about the shared space of humans and objects. To handle human occlusion caused by objects, we use a generative diffusion model that inpaints the occluded regions, recovering otherwise lost details. For training and evaluation, we introduce a synthetic dataset featuring rendered scenes of inter-occluded 3D human scans and diverse objects. Extensive evaluation on both synthetic and real-world datasets demonstrates the superior quality of the proposed human-object reconstructions over competitive methods.
\end{abstract}
\section{Introduction}\label{sec:intro}

In computational finance, Monte Carlo simulations are used extensively to estimate the expected value of financial payoffs based on the solution of stochastic differential equations (SDEs) which model the evolution of stock prices, interest rates, exchange rates and other quantities \cite{glasserman04}.  Monte Carlo methods are very general and flexible, but for high accuracy it requires generating a large number of costly SDE path approximations, which has motivated research into a number of variance reduction or, equivalently, cost reduction techniques. One such method is
Multilevel Monte Carlo (MLMC), which was proposed in \cite{GILES2008} and was adapted for various applications that are summarised in \cite{Giles_overview17} and successfully combined with other methods such as quasi-Monte Carlo methods. The main idea of MLMC is to approximate the payoff using different time stepping resolutions when numerically solving the underlying SDE and to generate an optimal number of samples on each level, such that the overall computational cost is minimised subject to the desired bound on the variance. %, such that the total computational cost is minimised. 
The computational savings come from the fact that most samples are computed on the coarser levels and hence are less expensive while only a few samples from the finest levels are required \cite{GILES2008}.


Among the directions in which the computational cost 
of MLMC methods could further be reduced, an important avenue is the use of lower precision calculations, especially for the first Monte Carlo levels where the targeted accuracy is relatively low. 
 An overview of the research on mixed precision for the standard Monte Carlo (MC) framework is provided in \cite{ChowMixedPrecisionStandardMC} but only a few references study the potential of low precision computation in the MLMC framework \cite{Rounding_error_oliver}. To the best of our knowledge, the only MLMC framework with customised precision in the literature is \cite{brugger2014mixed}, but they use a uniform precision for all operations on each Monte Carlo level instead of optimising 
 the precision of each intermediary variable to reduce as much as possible the cost of path generation.
 
An important motivation for an MLMC framework with variable precision would be performing the low precision computations on reconfigurable hardware devices such as Field Programmable Gate Arrays (FPGAs). FPGAs contain customizable logic blocks and connectors that make it easy to adapt the digital circuit architecture for a specific application, leading to a highly parallel and optimised implementation. Therefore they are successfully exploited in applications that require high speed and have high computational workload, such as signal processing \cite{woods2008fpga}, and real time applications like high frequency trading \cite{HFT1,HFT2}. That is why a number of previous works in hardware architecture design implemented the MLMC algorithm to price financial options using FPGAs as accelerators, which resulted in improved speed and power efficiency compared to full CPU architectures \cite{Schryver2013AMM}. The paper \cite{lindsey2016domain} also proposed 
a Domain Specific Language to automate the configuration of FPGAs for this specific application. However, only \cite{brugger2014mixed} proposed a heuristic to reduce the precision in calculations.

In addition, all aforementioned works considered that the random number generation (RNG) is performed in single or double precision. Yet in most cases an important portion of the workload in the overall MLMC simulation comes from the RNG and in \cite{brugger2014mixed} this limited the total computational savings.
To reduce the cost of MLMC simulations in particular those based on the Geometric Brownian Motion (GBM), \cite{approximateICDF_Oliver, NestedOliver} have proposed to use approximate random numbers that are generated by applying an approximation of the inverse CDF to uniform random numbers. In \cite{NestedOliver}, the authors proposed a way to integrate these lower precision random variables into a \textit{nested} MLMC framework and completed a numerical analysis to bound the resulting error at each MC level by a product of the time step and the error in the random number approximation. The same authors show in \cite{approximateICDF_Oliver} that using approximate random variables reduces the cost of path generation by a factor 7.


In this paper we propose a nested MLMC framework that combines the use of approximate random normal variables and lower precision calculations to reduce the computational cost of MLMC even further than \cite{brugger2014mixed,NestedOliver}. We illustrate the efficiency of our framework in Matlab, after making several assumptions on the cost of operations and size of the errors that we carefully justify. We focus on the case of GBM and use the approximate RNG methods presented in \cite{approximateICDF_Oliver} as well as a new slightly modified method that combines CDF inversion and the central limit theorem. To choose the precision of the variables in the low precision path generation, we introduce a novel method to optimise the bit-widths. This optimisation is performed before the main path generation loop is executed and is based on a linear model of the payoff error  
due to rounding when computing in low precision. The error model relies on algorithmic differentiation in a similar manner to \cite{unifying-bwoptim,bitwidth-AD,ADAPT}. The bit-width optimisation procedure can be performed off-line, so this stage can be excluded from the on-line time complexity of our framework. The user specified desired accuracy is then enforced by calculating on-line the number of samples that need to be generated.

In terms of hardware design, we suggest implementing the low precision path generation on FPGAs and the full-precision ones on a CPU or GPU. 
The FPGA offers enough flexibility to define a separate bit-width for every variable in the low precision path generation, and can be reconfigured periodically to update the bit-widths when the market parameters have changed considerably. 


The paper is organized as follows : \Cref{sec:MLMC} introduces MLMC and nested MLMC to make clear the estimator that is implemented in our framework. Then in \Cref{sec:RNG} we detail the methods that could be used to obtain approximate random normally distributed numbers very cheaply for the low precision path generation. In \Cref{sec:error_model} and \Cref{sec:costModel} we propose an error model and a cost model (resp.) that we then use to formulate the optimisation problem that is solved to obtain the optimal bit-widths of fixed point variables in \Cref{sec:optimisation}. Finally we summarise our results and future directions in \Cref{sec:conclusion}.




	\vspace{-5pt}
	\section{Problem Setup and Motivation}\label{sec:setup}
	
	In this section, we introduce the setup for the problem studied in this work. In \Cref{sec:intro_alignment}, we present the basic framework for aligning language models with human preferences. In \Cref{sec:intro_DPO}, we provide an overview of the widely-used Direct Preference Optimization (DPO) method. Finally, in \Cref{sec:intro_goal}, we introduce the core problem investigated in this work: designing an optimal sampling scheme for response generation.
    % \yaqi{Delete this to avoid redundancy.}
	\vspace{-3pt}
	\subsection{Aligning LMs with Human Preferences}
	\label{sec:intro_alignment}
	
	% We break down the alignment process into three key components: the language model, preference data, and policy optimization.
	
	\paragraph{Language Model (LM).}
	At the core of RLHF is a language model that processes prompts~\mbox{$\prompt \in \PromptSp$} and generates responses $ \response \in \ResponseSp $. 
	Each response is represented as a sequence of tokens $\response = (\tokent{1}, \tokent{2}, \ldots, \tokent{\numtok}).$ The primary goal of RLHF is to guide the model to generate responses that align with human preferences. This translates to designing a policy $\policy$ (parameterized as a LM) that maps prompts to responses, maximizing a reward that reflects human preferences (with a KL regularization).
	
	
	\paragraph{Preference Data.} The oracle reward for human values is inherently inaccessible. Instead, the alignment process approximates the reward using a dataset of human-labeled preferences,
	\begin{align*}
		\Data = \big\{ (\prompti{i}, \responsewini{i}, \responselosei{i}) \big\}_{i=1}^{\numobs} \, ,
	\end{align*}
	where each sample contains: (i) a prompt $ \prompti{i}$, independently drawn from a distribution $ \promptdistr$, and (ii) a pair of responses $(\responsewini{i}, \responselosei{i})$, where $\responsewini{i}$ is preferred over $\responselosei{i}$ in human labeling. The response pair~\mbox{$(\responsewini{i}, \responselosei{i})$} is first generated from a joint distribution $\responsedistr(\cdot \mid \prompt)$ 
    %\yaqi{$\Leftarrow$ $\responsedistr$ first appears here. It is indeed quite far from \Cref{sec:theory}.} 
    and then presented to human labelers for preference annotation. Human preferences are commonly modeled using the \emph{Bradley–Terry (BT)} model, which assumes: \begin{align}
			\label{eq:BT}
			\Prob\big( \responseone \succ \responsetwo \bigm| \prompt \big)
			\; = \; \sigmoid\big( \rewardstar(\prompt, \responseone) - \rewardstar(\prompt, \responsetwo) \big) \, ,
		\end{align}
		where $\rewardstar(\prompt, \response)$ represents the (unknown) oracle reward of a response given a prompt, and $\sigmoid(z) = \{ 1 + \exp(-z) \}^{-1}$ is the sigmoid function, mapping differences in rewards to probabilities. We adopt the BT model throughout this paper.

    \paragraph{Reward Modeling.} The preference data, encoding human judgment, is then used to train a reward model, $r_\theta$, which serves as a measurable objective for training the policy model. $r_\theta$ is trained by solving a MLE objective:
\begin{align}
		\label{eq:RM_objective}
		\min_{\paratheta} \
        % \quad \Losshat(\paratheta)  \\
        % \notag
        \Losshat(\paratheta) :=
		- \frac{1}{\numobs} \sum_{i=1}^{\numobs} \log \sigmoid \Big( \rewardtheta\big(\prompti{i}, \responsewini{i}\big) - \rewardtheta\big(\prompti{i}, \responselosei{i}\big) \Big).
	\end{align}
%
    This empirical loss approximates the expected negative log-likelihood
	\begin{align}
		\label{eq:def_Loss}
        \Loss(\paratheta) \; := \;
		\Exp_{\prompt \sim \promptdistr, \,(\responseone, \responsetwo) \sim \responsedistr(\cdot \mid \prompt)} \Big[ - \log \sigmoid \big( \rewardtheta(\prompt, \responsewin) - \rewardtheta(\prompt, \responselose ) \big) \Big] \, .
	\end{align}
    
    \paragraph{Policy Optimization.} To align a language model $\phi$ with human preferences, we optimize it to maximize the learned rewards $\rewardtheta$ while staying close to a reference policy $\policyref$. The objective is
	\begin{equation}\label{eq:policy_loss_with_rm}
    \max\nolimits_{{\phi}} \ 
    \Exp_{\prompt \sim \promptdistr, \response \sim \policy_{\phi}(\cdot \mid \prompt)}
    \big[ \rewardtheta (\context, \response) \big]
    - \parabeta \kull{\policy_{\phi}}{\policyref}.
\end{equation}
	It consists of two parts: 
    % \vspace{-0.8em}
    \begin{itemize}
	\item[(i)] The \emph{reward} term $\Exp_{\prompt \sim \promptdistr, \, \response \sim \policy(\cdot \mid \prompt)} [ \rewardtheta(\context, \response) ]$ encourages the policy to generate high-quality responses.
	\item[(ii)] The \emph{regularization} term \mbox{$\kull{\policy}{\policyref}$} penalizes deviations from the reference policy~$\policyref$ and is defined as \mbox{$\Exp_{\prompt \sim \promptdistr} \big[ \kull[\big]{\policy(\cdot \mid \prompt)}{\policyref(\cdot \mid \prompt)} \big]$}.
    \end{itemize}
    % \vspace{-0.8em}
    Here, $\parabeta$ is a regularization parameter that balances the trade-off between reward maximization and adherence to the reference policy. 
    % The reference policy $\policyref$, often obtained via supervised fine-tuning (SFT), serves as the starting point for optimization. 
    We assume $\parabeta$ is fixed and practitioner-specified.


    
	% \begin{enumerate}
	% 	\item[(i)] Two candidate responses $\responseonei{i}$ and $\responsetwoi{i}$ are drawn from a joint distribution $\responsedistr(\cdot \mid \prompt)$.
	% 	\item[(ii)] The preference between them is modeled using the \emph{Bradley–Terry (BT)} model:
	% 	\begin{align}
	% 		\label{eq:BT}
	% 		\Prob\big( \responseone \succ \responsetwo \bigm| \prompt \big)
	% 		\; = \; \sigmoid\big( \rewardstar(\prompt, \responseone) - \rewardstar(\prompt, \responsetwo) \big) \, ,
	% 	\end{align}
	% 	where $\rewardstar(\prompt, \response)$ represents the (unknown) quality or reward of a response given a prompt, and $\sigmoid(z) = \{ 1 + \exp(-z) \}^{-1}$ is the sigmoid function, mapping differences in rewards to probabilities.
	% \end{enumerate}

	
	    
	
	

%%%%%%%%%%%%%%%%%%%%%%%%%%%%%%%%%%%%%%%%%%%%%%%%%%%%%%%%%%%
	
	\subsection{Direct Preference Optimization}
	\label{sec:intro_DPO}

    The above-described RLHF pipeline typically leverages the Proximal Policy Optimization (PPO) algorithm \citep{schulman2017proximal} to perform policy optimization. This approach requires loading the policy network, reward model, reference model, and a value model onto the GPU during training, making it highly resource-intensive. To improve computational efficiency and practicality, Direct Preference Optimization (DPO) \citep{rafailov2023direct} has been proposed, enabling direct alignment without the need for a reward model or a value model.
	
    A key insight of DPO is that any policy~$\policytheta$ can be viewed as the optimal solution to problem~\eqref{eq:policy_loss_with_rm} if the reward~$\rewardtheta$ is 
	\vspace{-3mm}
    \begin{align}
		\label{eq:def_reward}
		\rewardtheta(\prompt, \response)
		\; \defn \; \parabeta \cdot \log \bigg( \frac{\policytheta(\response \mid \prompt)}{\policyref(\response \mid \prompt)} \bigg).
	\end{align} 
	% This suggests that if the reward function $\rewardtheta$ approximates the true $\rewardstar$ closely, the associated policy $\policytheta$ is expected to maximize the expected value.
%
    % DPO directly operates within a parameterized policy class \mbox{$\PolicySp = \{ \policytheta \mid \paratheta \in \Real^{\Dim} \}$}.
%
    Thus, DPO can directly optimize the policy $\policytheta$ using $\Losshat(\paratheta)$ in \cref{eq:RM_objective}, where $\rewardtheta$ is replaced by $\policytheta$ as defined in \cref{eq:def_reward}. This reformulation makes the objective dependent solely on $\theta$, with the reward being implicitly learned through the policy itself. As a result, the optimization process becomes significantly more efficient.
	
	% To estimate $\rewardstar$, DPO employs maximum likelihood estimation (MLE) by solving
	% \begin{align}
	% 	% \label{eq:RM_objective}
	% 	& {\rm minimize}_{\paratheta} \quad \Losshat(\paratheta)  \\
 %        \notag
 %        & \quad \Losshat(\paratheta) \; \defn \;
	% 	- \frac{1}{\numobs} \sum_{i=1}^{\numobs} \, \log \sigmoid \Big( \rewardtheta\big(\prompti{i}, \responsewini{i}\big) - \rewardtheta\big(\prompti{i}, \responselosei{i}\big) \Big) \, .
	% \end{align}
	
	
	
	\subsection{Motivation: Realigning Oracle Reward Maximization}
	\label{sec:intro_goal}

    To fully align with human values, RLHF should, in principle, train the policy to maximize the oracle reward, $\rewardstar$, as defined in the BT model. The corresponding oracle objective is then: 
    % \vspace{-.7em}
    \begin{equation} 
		%\max_{\policy}  \quad 
		\scalarvalue(\policy) \defn \; \Exp_{\prompt \sim \promptdistr, \, \response \sim \policy(\cdot \mid \prompt)} \big[ \rewardstar(\context, \response) \big] \notag \; - \; \parabeta \, \kull{\policy}{\policyref} \, .
        \label{eq:objective}
	\end{equation}
    %\ariel{The current equation seems to read as the definition of $\max_\pi J(\pi)$, but I think it's meant to be a definition of $J(\pi)$?}
    %\yaqi{Reform this sentence since it appeared once in intro.} 
    Since direct access to $\rewardstar$ is unavailable, RLHF instead relies on preference data, either through MLE-based reward modeling or methods like DPO. 
    However, these processes are not inherently designed to train the policy to directly maximize the oracle objective, $\scalarvalue(\policy)$.%\ariel{What do they directly maximize instead?}. 
    
    In this work, we aim to design an optimal sampling distribution $\responsedistr$ to realign DPO with the maximization of $\scalarvalue(\policy)$. Such a sampling strategy will improve the quality of the preference dataset, maximize the utility of limited data, and enhance both performance and efficiency.

    This focus is particularly crucial in scenarios where additional data is collected during mid-training—a key phase in the iterative fine-tuning of LMs
    \citep{touvron2023llama,bai2022training, xiong2024iterative, guo2024direct}.
    %\yaqi{Reform this sentence since it appeared once in intro.} 
    At this stage, a preliminary policy $\policytheta$ (distinct from $\policyref$) is already in place, but its performance may fall short of expectations. It is thus necessary to gather additional preference data, ideally on-policy data that target areas where the current policy shows room for improvement. An effective sampling design can significantly enhance the efficiency of leveraging human feedback in this process.
    
    % By strategically leveraging human feedback, this focused data collection ensures that the newly acquired information can enhance the model.
	
	% A closer examination of the DPO optimization process reveals that the choice of the sampling distribution $\responsedistr$ for response selection is a critical factor in determining training efficiency. In this work, we aim to design an optimal sampling distribution $\responsedistr$ that enables the DPO method to extract greater value from limited data while improving performance (as measured by $\scalarvalue(\policy)$).
	

	

    \section{T-PILAF: Theoretical Sampling Scheme }
% to align reward modeling with value maximization?
 %       \subsection{Mismatch between reward modeling and optimization objective}
        
%          \begin{align*}
%        \gradtheta \Loss(\paratheta) 
%        \qquad
%	- \gradtheta \scalarvalue(\policytheta)
%    \end{align*}
	% \subsection{Sampling scheme}
	\label{sec:sampling}

	
	We now present T-PILAF - {\em theoretically grounded policy interpolation for aligned feedback} - our sampling scheme for generating responses in data collection\footnote{The T in T-PILAF serves to distinguish the theoretical scheme from the derived, simplified, efficiently implementable PILAF.}. The scheme is shown (in \Cref{sec:theory}) to be optimal from both optimization and statistical perspectives. 
    % However, it is computationally impractical for large-scale language models. An experimental design is then proposed in \cref{sec:sampling_exp} ensuring scalability and enabling efficient inference. 
    
	Consider we have an {initial} policy $\policytheta$ and aim to collect preference data to further refine its performance.
	We propose two complementary variants of policy $\policytheta$: one that encourages exploration in regions {more} preferred by~$\policytheta$, reflecting an optimistic perspective, and another that focuses on areas {less favored by~$\policytheta$}, reflecting a conservative adjustment.
%	By generating response pairs using these two policies, we aim to enable efficient exploration and refinement of the current model $\policytheta$.
	
	Specifically, we define policies $\policythetapos$ and $\policythetaneg$ around $\policytheta$ as
    \begin{subequations}
	\begin{align}
		\label{eq:def_policythetapos}
		\policythetapos(\response \mid \prompt)
		& := \frac{1}{\Partitionthetapos(\prompt)} \; \policytheta(\response \mid \prompt)
		\exp \big\{ \rewardtheta(\prompt, \response) \big\} \, ,  \\[-1pt]
		\label{eq:def_policythetaneg}
		\policythetaneg(\response \mid \prompt)
		& := \frac{1}{\Partitionthetaneg(\prompt)} \policytheta(\response \mid \prompt)
		\exp \big\{ - \rewardtheta(\prompt, \response) \big\},
	\end{align}
	\end{subequations}
	where the reward function $\rewardtheta$ is defined in equation~\eqref{eq:def_reward}.
	The partition function $\Partitionthetapos(\prompt)$ (or $\Partitionthetaneg(\prompt)$) is given by
	\mbox{$\Partitionthetapos(\prompt)
	\defn \int_{\ResponseSp} \policytheta(\response \mid \prompt) \exp \{ \rewardtheta(\prompt, \response) \} \, \diff \response $}. %\yaqi{$\policythetapos$??? typo?}
    %\yaqi{Here it should be $\policytheta$ in the integration, consistent with the right-hand side in \Cref{eq:def_policythetapos}.}
	
	For any prompt $\prompt \in \PromptSp$, our sampling procedure involves the following steps:
	\begin{enumerate}  % \itemsep = -.2em
		\item[(i)] Draw a random variable $\xi$ from ${\rm Bernoulli}(\sampleprob(\prompt))$, where
%		\begin{align}
%			\label{eq:def_sampleprob}
        \vspace{-1em}
			$$\sampleprob(\prompt) \defn {\Partitionthetapos(\prompt) \, \Partitionthetaneg(\prompt)}/ \{1 + \Partitionthetapos(\prompt) \, \Partitionthetaneg(\prompt) \}. $$ ~ \\ \vspace{-4em}
%		\end{align}
		\item[(ii)] If $\xi = 1$, independently draw responses $\responseone, \responsetwo \in \ResponseSp$ according to
%	\begin{align*}
		$\responseone \sim \policythetapos(\cdot \mid \prompt)$ and $\responsetwo \sim \policythetaneg(\cdot \mid \prompt)$. 
%	\end{align*}
		If $\xi = 0$, draw responses as $\responseone, \responsetwo \sim \policytheta(\cdot \mid \prompt)$.
	\end{enumerate}
	
	In the next section, we will theoretically analyze T-PILAF. To account for the changes in sampling, we adopt a slightly modified loss function in the theoretical framework:
     \begin{align*}
%     	\label{eq:def_Losshat_adj}
		\Losshat(\paratheta) \defn
     	& - \frac{1}{\numobs} \sum_{i=1}^{\numobs} \weight(\prompti{i}) \cdot \log \sigmoid \Big( \rewardtheta\big(\prompti{i}, \responsewini{i}\big) - \rewardtheta\big(\prompti{i}, \responselosei{i}\big) \Big) .
     \end{align*}
	The newly introduced weight function $\weight$ is defined as
	\begin{align}
		\weight(\prompt)
		& \label{eq:weight}
        \; \defn \; \big\{ 1 + \Partitionthetapos(\prompt) \, \Partitionthetaneg(\prompt) \big\} / \, \Partitionthetabar \, ,
	\end{align}
	where the normalization constant~$\Partitionthetabar > 0$ is given by \mbox{$\Partitionthetabar \defn 1 + \int_{\PromptSp} \Partitionthetapos(\prompt) \, \Partitionthetaneg(\prompt) \, \promptdistr(\prompt) \, \diff \prompt$}. We also modify the population loss $\mathcal{L}$ in \cref{eq:def_Loss} with the weight function.
	
	
	

%%%%%%%%%%%%%%%%%%%%%%%%%%%%%%%%%%%%%%%%%%%%%%%%%%%%%%%%%%%%%%%%%%%%%%%%%%%%%%%
\section{Theoretical Analysis}\label{sec:theory}
This section provides the theoretical grounding and analysis of our proposed sampling scheme from two perspectives. In the {\em optimization} analysis (\Cref{sec:theory_opt}) we show that T-PILAF {\em aligns two objectives (gradient alignment property)}: maximizing the likelihood function (\cref{eq:def_Loss}) becomes equivalent to gradient ascent on the value function $\scalarvalue(\policytheta)$ (\cref{eq:objective}). Consequently, policy updates on $\pi_\theta$ move the parameters in the direction of steepest increase of $J$. T-PILAF thus provides the potential to accelerate training and improve generalization, compared to vanilla (uniform)  sampling. In the {\em statistical} analysis (\Cref{sec:theory_stat}) we focus on statistical error and show that {the asymptotic covariance} of the estimated parameter~$\parathetahat$ (inversely) aligns with the Hessian of the objective function~$\scalarvalue$ when sampling with T-PILAF. As a result, T-PILAF makes the sampled comparisons more informative, as they align with directions where~$\scalarvalue$ is most sensitive. The net outcome is reduced statistical variance of our method through tighter concentration of estimates in directions that matter most for performance.


\subsection{Optimization Considerations}
            \label{sec:theory_opt}



            
We begin by analyzing the DPO algorithm from an optimization perspective.
{\Cref{thm:grad} below formally illustrates how T-PILAF ensures alignment between the MLE gradient, $\gradtheta \Loss(\paratheta)$, and the oracle objective gradient, $\gradtheta \scalarvalue(\policytheta)$.}

%
%
\begin{theorem}[Gradient structure in DPO training]
\label{thm:grad}
  Using data collected from our proposed response sampling scheme T-PILAF, the gradient of $ \Loss(\paratheta) $ satisfies
\begin{align*}
    \gradtheta \Loss(\paratheta) \; = \;
    - \, \frac{\parabeta}{\Partitionthetabar} \, \gradtheta \scalarvalue(\policytheta) \, + \, \Term_2 \, ,
\end{align*}
where the constant $ \Partitionthetabar $ is defined in equation~\eqref{eq:weight}, and the term $ \Term_2% = \bigO( \norm{\rewardtheta - \rewardstar}^2 ) 
$ represents a second-order error.
\end{theorem}
The detailed proof of \Cref{thm:grad} is deferred to \Cref{sec:proof:thm:grad}. 
It involves calculation of explicit forms of the gradients $\gradtheta \Loss(\paratheta)$ and $\gradtheta \scalarvalue(\policytheta)$; the most notable technical contribution is showing how to leverage our sampling scheme to approximate the derivative $\divsigmoid$ of the sigmoid function. By using T-PILAF sampling, we can transform a difference term of the form $\sigmoid ( \Delta \rewardstar ) - \sigmoid ( \Delta \rewardtheta )$ in $\gradtheta \Loss(\paratheta)$ into a linear difference $\Delta \rewardstar - \Delta \rewardtheta$ in $\gradtheta \scalarvalue(\policytheta)$.

\Cref{thm:grad} establishes the \emph{gradient alignment} property, demonstrating that minimizing the likelihood-based loss function~$\Loss$ closely aligns with maximizing the oracle objective function~$\scalarvalue$, with only a minor second-order error. It highlights how the proposed sampling scheme enables the DPO framework to effectively guide the policy toward optimizing the expected reward.
%
Beyond DPO, in \Cref{app:extension}, we show how the same principle can be applied to PPO-based RLHF algorithms to help improve the sampling. %\yaqi{I will work on adding more details in the appendix tomorrow to address the generalization of our approach. Depending on the progress, if I can include sufficient detail in the appendix discussion, we might consider adding a brief sentence at the end of this subsection to mention this generalization.} 
		

		
		
		
		
%%%%%%%%%%%%%%%%%%%%%%%%%%%%%%%%%%%%%%%%%%%%%%%%%%%%%%%%%%%%%%%%%%%%%%%%%%

\subsection{Statistical Considerations \yaqidone}
    \label{sec:theory_stat}

From a statistical standpoint, we first examine the asymptotic distribution of the estimated parameter $\parathetahat$ when it (approximately) solves the optimization problem~\eqref{eq:RM_objective}. In \Cref{thm:asymp}, we formally characterize the randomness or statistical error inherent in $\parathetahat$ under this idealized scenario.
The detailed proof of \Cref{thm:asymp} is provided in \Cref{sec:proof:thm:asymp}.
\begin{theorem}
    \label{thm:asymp}
%			We take $\weight(\prompt) \equiv 1$.
    Assume the reward model $\rewardstar$ in the BT model~\eqref{eq:BT} satisfies $\rewardstar = \reward_{\parathetastar}$ for some parameter $\parathetastar$.
    Under mild regularity conditions, the estimate $\parathetahat$ asymptotically follows a Gaussian distribution
    \begin{align*}
        \sqrt{\numobs} \; ( \parathetahat - \parathetastar)
        \; \stackrel{d}{\longrightarrow} \; \Gauss( \veczero, \CovOmega )
        \qquad \mbox{as $\numobs \rightarrow \infty$} \, .
    \end{align*}
    We have an estimate of the covariance matrix $\CovOmega$:
    \begin{align*}
        \CovOmega \; \preceq \; \Const_{1} \cdot \CovOpstar^{-1} \, ,
    \end{align*}
    where $\Const_{1} > 0$ is a universal constant. 
    When using T-PILAF, the matrix~$\CovOpstar$ is given by
    \begin{align}
        \label{eq:def_CovOpstar_simple}
        \CovOpstar \defn \; \Exp_{\prompt \sim \promptdistr} \Big[ \Cov_{\response \sim \policystar(\cdot \mid \prompt)} \big[ \gradtheta \rewardstar(\prompt, \response) \bigm| \prompt \big] \Big] \, .
    \end{align}
\end{theorem}

Next we analyze the performance of the output policy \mbox{$\policyhat = \policy_{\parathetahat}$} from \Cref{thm:asymp} in terms of the expected value~$\scalarvalue(\policy)$. In \Cref{lemma:hess_scalarvalue}, we show that our proposed sampling method guarantees that the covariance of the statistical error in~$\parathetahat$ aligns inversely with the Hessian of~$\scalarvalue$ at the optimal policy~$\policystar$. This alignment prioritizes convergence efficiency along directions where the Hessian has large eigenvalues, adapting to the geometry of the optimization landscape. It highlights the efficiency of our sampling scheme in reducing statistical error.
For the detailed proof, see \Cref{sec:proof:lemma:hess_scalarvalue}.
\begin{theorem}
        \label{lemma:hess_scalarvalue}
    The value function $\scalarvalue(\policy)$ we define in equation~\eqref{eq:objective} satisfies $\gradtheta \scalarvalue(\policystar) = \veczero$ and % $\hesstheta \scalarvalue(\policystar) =$
    %\vspace{-.6em}
    \begin{align}
        \label{eq:hessscalarvalue}
        \hesstheta \scalarvalue(\policystar) \; = \;
%				- \frac{1}{\parabeta} \, \Exp_{\prompt \sim \promptdistr} \Big[ \Cov_{\response \sim \policystar(\cdot \mid \prompt)} \big[ \gradtheta \rewardstar(\prompt, \response) \bigm| \prompt \big] \Big]
%				= 
        - \frac{1}{\parabeta} \, \CovOpstar
    \end{align} %~ \vspace{-1.2em} \\
    for matrix $\CovOpstar$ defined in equation~\eqref{eq:def_CovOpstar_simple}.
    As a corollary, suppose $\CovOpstar$ is nonsingular, then there exists a constant $\Const_{2} > 0$ such that for any $\varepsilon > 0$, 
    % \vspace{-.3em}
%			with high probability,
%			\begin{align*}
%				\scalarvalue(\policyhat)
%				\; \geq \; \scalarvalue(\policystar) \, - \, \frac{\Const{} \, \big\{ 1 + \supnorm{\Partitionthetapos \Partitionthetaneg} \big\}}{\parabeta} \cdot \frac{\Dim \log \numobs}{\numobs}
%			\end{align*}
%			for some universal constant $\Const{} > 0$.
    \begin{align}
        & \limsup_{\numobs \rightarrow \infty} \Prob \bigg\{ \scalarvalue(\policyhat) < \scalarvalue(\policystar) - \Const_{2} \cdot \frac{\Dim \, (1 + \varepsilon)}{\numobs} \bigg\} \notag  \\
        & \qquad \leq \; \Prob\big\{ \chisquare_{\Dim} > (1 + \varepsilon) \, \Dim \big\}
        \leq \exp\Big\{  -\frac{\Dim}{2} \bigl(\varepsilon - \log(1 + \varepsilon)\bigr)  \Big\} . \label{eq:gap_bd}
    \end{align}
    % \vspace{-1.5em}
\end{theorem}		

    Our proposed sampling distribution $\responsedistr$ ensures that the output policy $\policyhat$ performs predictably and reliably. The value gap $\scalarvalue(\policystar) - \scalarvalue(\policyhat)$ asymptotically follows a chi-square distribution, irrespective of the problem instance details, such as the underlying reward model $\rewardstar$. 
    This \emph{structure-invariant statistical efficiency} allows the method to achieve asymptotically efficient estimates without requiring explicit knowledge of the model structure. % \yaqi{maybe cut from here}


    %\iffalse
    In addition to our analysis of the proposed sampling scheme in \Cref{sec:sampling}, we present a generalized version of \Cref{thm:asymp} that applies to any response sampling distribution~$\responsedistr$. While not directly tied to the main focus of this work, this broader result may be of independent interest to readers.
    The proof of \Cref{thm:asymp_full} is provided in \Cref{sec:proof:thm:asymp_full}.
    \begin{lemma}
        \label{thm:asymp_full}
        For a general sampling distribution $\responsedistr$, the statement in \Cref{thm:asymp} remains valid with the matrix $\CovOpstar$ redefined as % $\CovOpstar \defn$
        \begin{align}
            \CovOpstar \defn
            \Exp_{\prompt \sim \promptdistr,(\responseone, \, \responsetwo) \sim \responsedistravg(\cdot \mid \prompt)}
            \Big[ \, \weight(\prompt) \cdot \Var\big(\indicator\{\responseone = \responsewin\} \bigm| \prompt, \responseone, \responsetwo \big) \cdot \grad \, \grad^{\top} \Big] \, ,
            \label{eq:def_CovOpstar}
        \end{align} % ~ \vspace{-1.8em} \\
        where the expectation is taken over the distribution
        % \vspace{-.3em}
        \begin{subequations}
            \begin{align}
                \label{eq:def_responsedistravg}
                \responsedistravg(\responseone, \responsetwo \mid \prompt) 
                \defn \frac{1}{2} \, \big\{ \responsedistr(\responseone, \responsetwo \mid \prompt) + \responsedistr(\responsetwo, \responseone \mid \prompt) \big\} \, .
            \end{align} % ~ \vspace{-1.8em} \\
        The variance term is specified as
            \begin{align}
                & \Var\big(\indicator\{\responseone = \responsewin\} \mid \prompt, \responseone, \responsetwo \big)
                \label{eq:def_var}
                = \sigmoid\big( \rewardstar(\prompt, \responseone) - \rewardstar(\prompt, \responsetwo) \big) \, \sigmoid\big( \rewardstar(\prompt, \responsetwo) - \rewardstar(\prompt, \responseone) \big)
            \end{align}
        and the gradient difference $\grad$ is defined as
            \begin{align}
                \label{eq:def_grad}
                \grad \defn \gradtheta \rewardstar(\prompt, \responseone) - \gradtheta \rewardstar(\prompt, \responsetwo) \, .
            \end{align}
        \end{subequations}
    \end{lemma}
    
    The general form of the matrix $\CovOpstar$ offers valuable insights for designing a sampling scheme. To ensure $\CovOpstar$ is well-conditioned (less singular), we must balance two key factors when selecting responses $\responseone$ and $\responsetwo$:
    % \vspace{-.8em}
    \begin{description} \itemsep = -.05em
        \item \emph{Large variance:} The variance in definition~\eqref{eq:def_var} should be maximized. This occurs when $\rewardstar(\prompt, \responseone) \approx \rewardstar(\prompt, \responsetwo)$. Intuitively, preference feedback is most informative when annotators compare responses of similar quality.
        \item \emph{Large gradient difference:} The gradient difference $\grad$ from definition~\eqref{eq:def_grad} should also be large. This requires responses with significantly different gradients. Only then can the comparison provide a clear and meaningful direction for model training.
    \end{description}
    %\fi

		
		
		
		


%%%%%%%%%%%%%%%%%%%%%%%%%%%%%%%%%%%%%%%%%%%%%%%%%%%%%%%%%%%%%%%%%%%%%%%%%%%
		
	
\section{Coupled Adam}
\label{sec:lmwap}

In the previous section, we have identified the individual scales of the second moments $v_i$ for different embedding vectors $e_i$ as the root cause of the anisotropy problem. This implies that a solution to the problem is to enforce that the second moments are the same for every $i$.
The question arises whether and how this can be done in the best way, without harming the performance of the model.
To answer this, we note that the normalization of the embedding update vector by the Adam second moment can be split into two parts:
\begin{equation}
\E \left[ \secondmomentshort \right] 
\stackrel{(\ref{eq:second_moment_proportionality_constant})}{=} A \cdot \widetilde p_i
= \frac{A}{V} \cdot \left( \widetilde p_i V \right)
\label{eq:second_moment_factorization}
\end{equation}
The first factor introduces a global scale to all update vectors simultaneously:
\begin{equation}
    \frac{A}{V} \stackrel{(\ref{eq:second_moment_proportionality_constant})}{\approx} \frac{10^{-4}}{5 \cdot 10^{4}} = 2 \cdot 10^{-9} \; ,
\label{eq:second_moment_global_factor}
\end{equation}
where the numbers correspond to our experiments from the previous section with $V \approx 50000$.
The second factor scales the update vectors individually. It is one on average:
\begin{equation}
\frac{1}{V} \sum_{i=1}^{V} \left( \widetilde p_i V \right) = 1
\label{eq:second_moment_individual_factor}
\end{equation}
Our goal is to retain the first, global factor and get rid of the second, individual factor. 
The canonical way to do this is to simply take the average of the second moment over the vocabulary items $i$:
\begin{equation}
\frac{1}{V} \sum_{i=1}^{V} \E \left[ \secondmomentshort \right]
\stackrel{(\ref{eq:second_moment_factorization}, \ref{eq:second_moment_individual_factor})}{=} \frac{A}{V}
\label{eq:optimizer_update_second_moment_avg_canonical}
\end{equation}
In practice, the exponentially averaged second moments $\secondmoment$ as they appear in Eq.~(\ref{eq:adam_learning_rate}) are replaced by their average:
\begin{align}
\secondmomentavg \: &:= \: \frac{1}{V} \sum_{i=1}^V \secondmoment
\label{eq:optimizer_update_second_moment_avg}
\end{align}
We call the resulting algorithm \textit{Coupled Adam}, as it couples the second moments of the embedding vectors via Eq.~(\ref{eq:optimizer_update_second_moment_avg}). 
It is displayed in Algorithm~\ref{alg:algorithm_adam}.
Evidently, with Coupled Adam, the effective learning rate in Eq.~(\ref{eq:adam_learning_rate}) that enters the update vector in Eq.~(\ref{eq:update_vector_definition_Adam}) becomes independent of $i$. Hence, like SGD but unlike standard Adam, the sum of embedding updates vanishes.
However, like standard Adam but unlike SGD, Coupled Adam uses a second moment to normalize the embedding update vectors. 

\section{Experiments}
\label{sec:experiments}

\begin{figure*}[t]
\vspace{-6mm}
    \centering
    \includegraphics[width=0.8\linewidth]{figs/compare.pdf}
    \vspace{-4mm}
    \caption{\textbf{Qualitative comparison} with the baseline for generating a sequence of novel view images.  
    The results demonstrate that our method synthesizes more consistent multi-view images compared to our baseline model (Zero123). In addition, compared to SyncDreamer, our method visually maintains better similarity to the conditioned image and appears more natural.}
    \label{fig:sota_compare}
\vspace{-5mm}
\end{figure*}

\subsection{Experimental Setups}
\textbf{Dataset.}
Following previous work~\cite{zero123, SyncDreamer}, we evaluate our work on the Google Scanned Object (GSO)~\cite{GSO} dataset to verify the zero-shot novel view image synthesis capability. 
We also provide results for additional datasets in the Supplementary Material.
Specifically, we randomly select 30 objects from the GSO dataset with various object categories. 
Unlike recent approaches~\cite{mvdream, SyncDreamer} that aim to enhance the consistency of novel view synthesis models by generating multiple fixed-view images, our method can generate images from any camera pose and any number of views. Therefore, we conduct experiments under different camera pose settings to validate our approach:
specifically, 
1) \textit{16-views with free camera pose}: for each object, we circularly render 16 views with the elevation angles ranging in $[-10\degree, 40\degree]$ and the azimuth angles are evenly distributed in $[0\degree, 360\degree]$. 
2) \textit{16-views with fixed camera pose}: We maintain a constant elevation angle of $30\degree$ and uniformly sample azimuth angles (same as SyncDreamer~\cite{SyncDreamer}).
3) \textit{32-views with free camera pose}: Similar to the first setting, but we sample 32 views.
It's important to note that our method does not require additional training or fine-tuning on any datasets.

\noindent\textbf{Metrics.}
To validate the effectiveness of our method, we mainly evaluate it based on three criteria:
1) \textit{Quality Score}. We evaluate the image quality of synthesized multi-view images by measuring their similarity with ground truth images. Following prior research~\cite{zero123, sparsefusion}, we report the similarity between the synthesized images and the ground truth images with standard metrics: PSNR, SSIM~\cite{ssim}, and LPIPS~\cite{lpips}.
2) \textit{Multi-view Consistency Score}. As the primary goal of our work is to improve the consistency of generated images, we also employ the 3D consistency score~\cite{3dim} to verify the consistency among the synthesized images. Specifically, we train an Instant-NGP~\cite{instant_ngp} with the input image and part of the synthesized novel view images of our model and evaluate the similarity between the remaining synthesized images and the rendered images of Instant-NGP. For the synthesized multi-view images of each object, we allocate $3/4$ for training and reserve the remaining $1/4$ for validation.
Intuitively, if the consistency of synthesized images is improved, the NeRF-like model will train a better object representation, and the re-rendered images will agree more with the validation images.
3) \textit{Input Consistency Score}. To assess the faithfulness of synthesized images in preserving the identity of the input condition image, we introduce the input consistency score. This score calculates the similarity of each synthesized image with the input condition image, utilizing the LPIPS metric.

In addition, we use synthesized multi-view images to train a neural 3D reconstruction model (NeuS~\cite{neus}) and report commonly used Chamfer Distances (CD) and Volume IoUs between the trained 3D model and the ground truth.

\noindent\textbf{Baselines.}
Given that our main goal is to improve the consistency of the trained baseline model without further fine-tuning, we mainly compare our approach with the used baseline model Zero123~\cite{zero123}. Additionally, we compare our method to the SOTA approaches such as PGD~\cite{tseng2023consistent} and SyncDreamer~\cite{SyncDreamer} using the same Zero123 base model.

\noindent\textbf{Implementation Details.}
We use the official checkpoint provided by Zero123~\cite{zero123}, which is trained on objaverse~\cite{objaverse} for 165,000 steps. We inject our epipolar attention layer after step $T=4$ and layer $L=10$ by default. We find that feature fusion weight $\alpha=0.5$, and the number of context views $M=2$ work better.

\begin{table}[t]
\centering
\caption{Comparison of multi-view consistency, image quality, and input consistency of synthesized multi-view images at the 16-view setting with free camera pose.}
\label{tab:view16_free_compare}
\vspace{-2mm}
\scalebox{0.6}{
\begin{tabular}{c ccc ccc c}
\toprule
              & \multicolumn{3}{c}{Multi-view Consistency} & \multicolumn{3}{c}{Quality Score} & \multicolumn{1}{c}{Input Consis.} \\
              \cmidrule(lr){2-4} \cmidrule(lr){5-7} \cmidrule(lr){8-8}
              & PSNR$\uparrow$  & SSIM$\uparrow$ & LPIPS$\downarrow$ 
              & PSNR$\uparrow$  & SSIM$\uparrow$ & LPIPS$\downarrow$ 
              & LPIPS$\downarrow$ 
              \\ \midrule

Zero123
& 15.225        & 0.645       & 0.408
& 14.255        & 0.747       &	0.208
& 0.303         
\\
SyncDreamer
& 14.830        & 0.626       & 0.434
& 12.650        & 0.713       &	0.254
& 0.317         
\\
Ours 
& \best{18.300}	& \best{0.734}	& \best{0.355}
& \best{14.947}	& \best{0.763}	& \best{0.191}
& \best{0.282}
\\

\bottomrule
\end{tabular}
}
\end{table}

\begin{table}[t]
\vspace{-1mm}
\centering
\caption{Comparison of multi-view consistency, image quality, and input consistency at the 16-view setting with fixed camera pose as SyncDreamer~\cite{SyncDreamer}.}
\label{tab:view16_fxied_compare}
\vspace{-3mm}
\scalebox{0.6}{
\begin{tabular}{c ccc ccc c}
\toprule
              & \multicolumn{3}{c}{Multi-view Consistency} & \multicolumn{3}{c}{Quality Score} & \multicolumn{1}{c}{Input Consis.} \\
              \cmidrule(lr){2-4} \cmidrule(lr){5-7} \cmidrule(lr){8-8}
              & PSNR$\uparrow$  & SSIM$\uparrow$ & LPIPS$\downarrow$ 
              & PSNR$\uparrow$  & SSIM$\uparrow$ & LPIPS$\downarrow$ 
              & LPIPS$\downarrow$ 
              \\ \midrule

Zero123
& 16.556        & 0.682       & 0.378
& 14.592        & 0.750       &	0.207
& 0.305         
\\
SyncDreamer
& \best{22.424}        & \best{0.812}       & \best{0.268}
& 15.269        & 0.749       &	0.196
& 0.300         
\\
Ours 
& 21.151	& 0.780	& 0.302
& \best{15.293}	& \best{0.764}	& \best{0.184}
& \best{0.287}
\\

\bottomrule
\end{tabular}
}
\vspace{-4mm}
\end{table}


\subsection{Comparison With Baseline Models}
The quantitative comparison on three settings are shown in Tab.~\ref{tab:view16_free_compare}, Tab.~\ref{tab:view16_fxied_compare}, and Tab.~\ref{tab:view32_free_compare}. The qualitative comparison is shown in Fig.~\ref{fig:sota_compare}.

\begin{table}[t]
\centering
\caption{Comparison of multi-view consistency and image quality scores of synthesized multi-view images at the 32-view setting with free camera pose.}
\vspace{-3mm}
\label{tab:view32_free_compare}
\scalebox{0.7}{
\begin{tabular}{c ccc ccc}
\toprule
              & \multicolumn{3}{c}{Multi-view Consistency} & \multicolumn{3}{c}{Quality Score} \\
              \cmidrule(lr){2-4} \cmidrule(lr){5-7}
              & PSNR$\uparrow$  & SSIM$\uparrow$ & LPIPS$\downarrow$ 
              & PSNR$\uparrow$  & SSIM$\uparrow$ & LPIPS$\downarrow$ 
              \\ \midrule

Zero123
& 16.515        & 0.694       & 0.378
& 15.142        & 0.733       &	0.211
\\
PGD~\cite{tseng2023consistent}
& 18.481        & 0.720       & 0.343
& 15.281        & 0.739       &	0.205
\\
Ours 
& \best{20.655}	& \best{0.792}	& \best{0.305}
& \best{15.268}	& \best{0.742}	& \best{0.203}
\\

\bottomrule
\end{tabular}
}
\vspace{-3mm}
\end{table}

\begin{table*}
  [t]
  \centering
  \resizebox{\textwidth}{!}{%
  \begin{tabular}{cccccccccccc}
    \toprule \multicolumn{2}{c}{Components}                                                             & \multicolumn{5}{c}{Re-executability Rate (\%)} & \multicolumn{5}{c}{Readability (\#)} \\
    \cmidrule(lr){1-2} \cmidrule(lr){3-7} \cmidrule(lr){8-12}        \hspace{8pt}\labelemoji\hspace{8pt}                                                                & \hspace{8pt}\toolemoji\hspace{8pt}                                      & O0                                 & O1             & O2             & O3             & AVG            & O0             & O1             & O2             & O3             & AVG            \\
    \hline
    \rowcolor[rgb]{0.93,0.93,0.93}\multicolumn{12}{c}{\textbf{Initialize with LLM4Decompile-End-6.7B~\citep{llm4decompile}}}   \\
    \xmark                                                                                              & \xmark                                    & 69.51                              & 46.95          & 50.61          & 46.34          & 53.35          & 3.98 & 3.41 & 3.44 & 3.38 & 3.55 \\
    \cmark                                                                                              & \xmark                                    & 75.61                              & 50.61          & 50.00          & 50.00          & 56.55          & 4.01 & 3.44 & 3.39 & \textbf{3.49} & 3.58 \\
    \xmark                                                                                              & \cmark                                    & 83.54                     & \textbf{56.10}          & 51.22          & 50.61 & 60.37 & 4.05 & 3.51 & 3.51 & 3.42 & 3.62 \\
    \cmark                                                                                              & \cmark                                    & \textbf{85.37}                            & \textbf{56.10}                     & \textbf{51.83} & \textbf{52.43}          & \textbf{61.43} & \textbf{4.13} & \textbf{3.60} & \textbf{3.54} & \textbf{3.49} & \textbf{3.69} \\

    \rowcolor[rgb]{0.93,0.93,0.93}\multicolumn{12}{c}{\textbf{Initialize with Deepseek-Coder-6.7B-base~\citep{deepseekcoder}}} \\
    \xmark                                                                                              & \xmark                                    & 59.15                              & 35.98          & 39.02          & 37.80          & 42.99          & 3.71 & 3.05 & 3.16 & 3.05 & 3.24 \\
    \cmark                                                                                              & \xmark                                    & 66.46                              & 41.46          & 38.41          & 36.59          & 45.73          & 3.76 & 3.17 & \textbf{3.21} & 3.08 & 3.31 \\
    \xmark                                                                                              & \cmark                                    & 70.73                              & 39.63          & 39.02          & 40.24          & 47.41          & 3.90 & 3.17 & 3.08 & 3.11 & 3.31 \\
    \cmark                                                                                              & \cmark                                    & \textbf{79.88}                     & \textbf{45.73} & \textbf{43.90} & \textbf{42.68} & \textbf{53.05} & \textbf{3.96} & \textbf{3.21} & 3.18 & \textbf{3.19} & \textbf{3.38} \\
    \bottomrule
  \end{tabular}%
  }
  \caption{The ablation study of different methods across four optimization levels
  (O0, O1, O2, O3), as well as their average scores (AVG). The results in bold represent the optimal performance. The ~\labelemoji~ and ~\toolemoji~ means Relabedling and Function Call. \textbf{Bold} denotes the best performance.}
  \label{tab:ablation}
\end{table*}



\begin{figure*}[ht]
    \centering
    \begin{minipage}{0.65\textwidth}
        \centering
        \includegraphics[width=0.95\linewidth]{figs/ablation.pdf}
        \vspace{-2mm}
        \captionof{figure}{Qualitative Comparison for different design choices. Our method, employing multi-view epipolar attention, demonstrates the best consistency.}
        \label{fig:ablation}
    \end{minipage}\hfill
    \begin{minipage}{0.33\textwidth}
        \centering
        \includegraphics[width=0.8\linewidth]{figs/neus_ver.pdf}
        \vspace{-3mm}
        \caption{Our method shows better direct 3D reconstruction~\cite{neus}.}
        \label{fig:neus}
    \end{minipage}
    \vspace{-5mm}
\end{figure*}

\noindent\textbf{Multi-view Consistency.}
Tab.~\ref{tab:view16_fxied_compare} presents the 3D consistency scores compared to our baseline model (Zero123) and SyncDreamer. The results indicate a significant improvement across all three metrics achieved by our method when compared with Zero123.
While our method exhibits a marginally lower numerical consistency score compared to SyncDreamer, it enables the synthesis of images with arbitrary camera poses.	
This capability is illustrated in Tab.~\ref{tab:view16_free_compare}, where our method consistently enhances consistency with changes in camera pose settings, whereas SyncDreamer fails to do so and exhibits inferior results compared to Zero123.
Furthermore, our method facilitates the synthesis of multi-view images with any number of camera views. This versatility is demonstrated in Tab.~\ref{tab:view32_free_compare}, where our method continues to achieve significant improvements in consistency scores, while SyncDreamer is unable to operate under such conditions.	

Meanwhile, Fig.~\ref{fig:sota_compare} provides a qualitative comparison with the baseline. While both our method and SyncDreamer enhance consistency, our method visually preserves better similarity to the input image, including color and texture details. The input consistency score further corroborates this.

\noindent\textbf{Image Quality.}
While our primary goal centers around enhancing the consistency of synthesized multi-view images, we also evaluate the image quality by comparing the similarity with the ground truth images. The results shown in Tab.~\ref{tab:view16_free_compare}, Tab.~\ref{tab:view16_fxied_compare}, and Tab.~\ref{tab:view32_free_compare} indicate that our method also enhances the image quality under different settings besides improving the consistency.
Moreover, our method shows better image quality compared with SyncDreamer even in the 16-view setting with fixed camera pose.

\noindent\textbf{Input Consistency.}
Input consistency terms whether the results align with the input image.
Fig.~\ref{fig:sota_compare} illustrates that both our method and SyncDreamer enhance multi-view consistency. However, the color and texture details of SyncDreamer's results diverge from the input image and appear visually unnatural.
This discrepancy is evident in the input consistency score presented in Tab.~\ref{tab:view16_fxied_compare}, indicating lower similarity with the condition image in the SyncDreamer results.	

\subsection{Ablation Study}
The overall quantitative results are shown in Tab.~\ref{tab:ablation}, and the qualitative comparisons are shown in Fig.~\ref{fig:ablation}.

\noindent \textbf{Full Attention \vs Epipolar Attention.}
The results presented in Tab.\ref{tab:ablation} and Fig.\ref{fig:ablation} demonstrate that our epipolar attention mechanism can synthesize more consistent multi-view images compared with full attention. Furthermore, our epipolar attention achieves a greater performance improvement compared to full attention when using multiple reference images. This could be attributed to the fact that our epipolar attention more effectively localizes target information, as depicted in Fig.~\ref{fig:full_attn_compare}, thereby reducing noise from the reference images. In the multi-view setting, where multiple reference images are utilized, this noise reduction becomes particularly crucial.
Moreover, it is noteworthy that the epipolar attention mechanism consumes less GPU memory compared to our baseline, as discussed in Sec.~\ref{sec:attn_analysis}.

\noindent \textbf{Attending Single-View \vs Multi-View.}
Applying the epipolar attention significantly improves the consistency between the input and target views. However, the consistency between different views in the unobserved regions of the input view is not well preserved.
After implementing our epipolar attention in the multi-view setting, the consistency across the generated multi-view images is further improved. The last row in Tab.~\ref{tab:ablation} shows that after applying our multi-view epipolar attention, the consistency score is further improved compared with the single-view setting. Besides, the qualitative result in Fig.~\ref{fig:ablation} also shows better consistency among different target views.



\begin{table}[t]
\centering
\vspace{-1mm}
\caption{Comparison of 3D reconstruction results. Our method significantly improves the reconstruction quality.}
\vspace{-3mm}
\label{tab:neus}
\scalebox{0.7}{
\begin{tabular}{c cc}
\toprule
              &  Chamfer Dist.$\downarrow$  & Volume IoU$\uparrow$
\\ \midrule

            Zero123         & 0.017         & 0.819    \\
            SyncDreamer     & \best{0.013}         & \best{0.847}    \\
            Ours            & 0.014	& 0.842 \\

\bottomrule
\end{tabular}
}
\vspace{-5mm}
\end{table}


\vspace{-2mm}
\subsection{Downstream Application}
\vspace{-2mm}
To demonstrate the effectiveness of our method, we also applied it to the downstream 3D reconstruction task. Specifically, we trained the NeuS model~\cite{neus} directly using images synthesized by our method, Zero123, and SyncDreamer, respectively.
The quantitative results in Tab.~\ref{tab:neus} show that the consistent multi-view images synthesized by our method can significantly improve the 3D reconstruction quality.
Additionally, our method exhibits similar performance to SyncDreamer which requires time-consuming re-training.
The qualitative results in Fig.~\ref{fig:neus} show that it is challenging to train the NeuS model directly due to the lack of consistency in the images generated by Zero123. In contrast, our method generates more consistent multi-view images and, therefore, better reconstructs the geometry and texture details.
We show improvements on other downstream applications such as image-to-3D in the Supplementary Material.


% \section{Simulation Evaluation \& Results}\label{sec:results}

\subsection{Baseline Planners}

To evaluate the performance of \PlannerName, we compare it against several baseline methods. In the following section, we describe these baselines, their implementation details, and their respective advantages and limitations, particularly in the context of information gathering in large, high-dimensional search spaces. The simulation framework and vehicle parameters remain consistent across all planners, and each method is allowed to replan during testing.

\subsubsection{Monte-Carlo Tree Search}

Monte Carlo Tree Search (MCTS) can be a powerful technique for finding feasible and optimal paths in complex environments. It is a heuristic search algorithm that builds a search tree incrementally through repeated simulations. At each iteration, it selects a node to explore based on a selection policy (often the Upper Confidence Bound or UCB1 algorithm), expands the tree by adding possible actions from that node, runs a simulation from the newly added node, and updates the statistics of nodes along the path traversed during the simulation. 

The UCB1 (Upper Confidence Bound) algorithm is a technique commonly used in the context of multi-armed bandit problems and Monte Carlo Tree Search (MCTS) for balancing exploration and exploitation. It helps in selecting actions or nodes that are likely to yield high rewards while also exploring less-frequented options to gather more information about their potential rewards. 

We formulate our UCB score in the following manner, \\
\begin{equation*}
    UCB_\text{node} = \frac{I(X_{\text{node}})}{\alpha} + C \times \sqrt{\frac{\ln(N_\text{tree})}{N_\text{node}}}
\end{equation*}
%  $
% UCB_\text{node} = \frac{\overline{X_\text{node}}}{\alpha} + C \times \sqrt{\frac{\ln(N_\text{tree})}{N_\text{node}}}
% $ \\
Here $I(X_{\text{node}})$ denotes the estimated information gain from the node, $\alpha$ denotes the normalization factor which is given by $\frac{B}{v_\text{desired}}$, $B$ being the maximum planning budget and $v_\text{desired}$ being the desired speed of our UAV. $C$ denotes the exploration weight, and $N_\text{tree}$ denotes the number of visits to the tree root node while $N_\text{node}$ denotes the number of times the present node has been visited.

After selecting a candidate node, if it has been visited before, it is expanded by applying motion primitives to generate child nodes, growing the tree. Unvisited nodes skip this step. Following expansion, either the unvisited candidate node or one of its children is selected for the simulation phase, where the future values of nodes along the path are estimated to update the total potential information gain. This informs the selection policy in subsequent iterations. Once planning time is exhausted, the path with the highest information gain is returned.

% with authors goes here
\begin{figure}[t]
\centering
\includegraphics[trim={.7cm 0cm .5cm 1.4cm},clip,width=\columnwidth]{figs/5_/Results1v3.pdf}
\caption{The Monte Carlo simulation results for the planners. The plots show the average percent reduction in entropy over the course of the simulations, and the shading shows the 95\% confidence intervals. IA-TIGRIS outperforms all of the baselines.}
\label{fig:mc_results}
\end{figure}

While MCTS is probabilistically guaranteed to converge to the optimal path \cite{mcts_ref_1}, it is constrained to actions within a predefined set of motion primitives. Its reliance on random sampling to estimate the future value of nodes can result in poor approximations, particularly in environments with sparse, localized pockets of high information gain. This limitation is especially pronounced in large search areas or scenarios with large budgets constraints, where estimating future node values becomes increasingly expensive. As a result, in such scenarios, MCTS is often implemented with a finite planning horizon, which can restrict its ability to account for long-term consequences or dependencies in the environment.

% This property of MCTS, which causes unguided exploration of the environment, leads to increased convergence times on the optimal path, as a result of a lot of budget being spent in exploring information sparse areas of the map. 
% Also, the computation time of MCTS increases exponentially with the depth of the search tree. The time complexity of MCTS is given by $\mathcal{O}(\frac{T}{t_\text{iter}} \cdot |A|^d)$. Here, $T$ is the total planning time and $t_\text{iter}$ is the time taken per iteration of the planning loop. $|A|$ is the number of actions and $d$ represents the average depth of the search tree. 

% The above limitations are not inconsequential in the context of performing informative path planning in large high-dimensional search spaces. We compare MCTS with \PlannerName, in \ref{}, and empirically demonstrate its drawbacks and how \PlannerName, is able to outperform MCTS in the context of the mission parameters we examine in this work.  

\subsubsection{Greedy}

For the greedy planner, we iterated through each cell within the search bounds and calculated the reward for a given cell $i$ as $g_i = R(X_i) / d_i$ where $R(X_i)$ is given through \eqref{equ:reward} and $d_i$ represents the Euclidean distance between the current position the robot at the current time $t$ and the closest viewpoint to the cell. To compute this viewpoint, the yaw between the current pose of the robot and the intersected cell is first calculated. Using the robot's sensor configuration and this yaw, $x$ and $y$ coordinates are calculated that view the cell at the desired flight altitude. With this formulation, the planner prioritizes regions with a high ratio of entropy to distance. This can lead to locally optimal choices that contradict with paths that lead to higher information gain over the entire trajectory. 

% without authors goes here
% \begin{figure}[t]
% \centering
% \includegraphics[trim={.7cm 0cm .5cm 1.4cm},clip,width=\columnwidth]{figs/5_/Results1v3.pdf}
% \caption{The Monte Carlo simulation results for the planners. The plots show the average percent reduction in entropy over the course of the simulations, and the shading shows the 95\% confidence intervals. IA-TIGRIS outperforms all of the baselines.}
% \label{fig:mc_results}
% \end{figure}


\begin{figure*}[t]
    \centering
    \begin{subfigure}[b]{0.99\textwidth}
        \centering
        \includegraphics[trim={0cm 0.3cm 0cm 0cm},clip,width=\textwidth]{figs/5_/Fig2v1_target.png}
        % \caption{Slice by targets}
        % \vspace{.1cm}
    \end{subfigure}
    
    \begin{subfigure}[b]{0.99\textwidth}
        \centering
        \includegraphics[trim={0cm 0cm 0cm 0cm},clip,width=\textwidth]{figs/5_/Fig2v1_sigma.png}
        % \caption{Slice by sigma }
    \end{subfigure}
    \caption{A comparison of the methods based on the number of sampled prior clusters and the standard deviation of sampled prior clusters. IA-TIGRIS is most effective compared to the baselines when there is high variation in the search space. As the search space prior information becomes more evenly spread out, the performance gap between the methods tends to decrease.}
    \label{fig:targets_sigmas}
\end{figure*}

\subsubsection{Random}

The random planner operates by iteratively sampling points within the defined search bounds and calculating the minimum-cost path to observe each sampled point. This process is repeated until the available budget is fully expended. The random planner does not utilize any prior information about the environment or target distribution. Additionally, it does not optimize the sequence of actions, instead treating each sampled point independently without considering the global structure of the search problem. This simplicity allows the random planner to highlight the performance benefits of more sophisticated methods by providing a lower-bound comparison for evaluation.

\subsubsection{Coverage}

The coverage planner generates a plan that systematically covers the entire search space using a straightforward lawn-mower pattern. The spacing between each pass is set to match the width of the projected observation footprint at 20\% from the bottom, ensuring that no grid cells are missed. This spacing also maintains a distance that enables high-quality sensor measurements. However, due to the size of the search spaces considered, the coverage planner spends significant time surveying empty regions. This approach results in inefficient use of the budget, as it prioritizes full coverage with safe sensor overlap, even in areas with little or no valuable information. While simple and robust, this method highlights the tradeoff between exhaustive coverage and efficient, targeted exploration.

% \subsubsection{Branch and Bound}
% The branch and bound baseline is based on motion primitive planning. In each future step the drone has a set of motion primitives with future states and each of these future states also has a set of motion primitives. In this way, a tree can be built with multiple path candidates. The path candidate with the highest information gain will be selected and form the output. 

% By adding branch and bound, there will be an estimation of a node's upper bound information reward, using the node's current information reward, updated information map and the remaining budget. If this upper bound is already lower than the information reward of any other node in the tree, the corresponding node will be closed and not expanded in the future to accelerate the expansion of the tree. 



\subsection{Tests and Analysis}
% To evaluate the efficacy of IA-TIGRIS compared to the baseline methods, we conduct Monte Carlo testing as well as analyze how the prior and budget affect the performance of each method. In all of these test cases, there are no time-based or priority rewards and have horizon lengths set to the full budget. All tests were performed using an Intel Xeon CPU E5-2620 v4 @ 2.10GHz.
To evaluate the efficacy of IA-TIGRIS against baseline methods, we perform Monte Carlo testing and analyze the impact of the prior and budget on the performance of each method. In all test cases, rewards are calculated using \eqref{equ:reward}, and horizon lengths are set to match the full budget. The tests are conducted on an Intel Xeon CPU E5-2620 v4 @ 2.10GHz, ensuring consistent computational conditions across all evaluations.

% Random sample across which parameters.

% Quantitative ideas. Look into number and std of prior (metric for this? std of grid cell values, mediuan, mean,). 
% Uniform prior? 
% Split distinct regions, not smooth. 
% Compare to coverage and amount of time to reach specific amount. 
% Compare with different budgets. 
% Repeatability test. 
% Graph size vs time. 
% Look at coverage with different altitudes or widths. Something that shows long horizon vs not nature of things?
% Shape of search space?
% Time/budget to get x\% of all info gain. Have to do moving horizon. 
% Targets detected? 

% Key thought for results where I show time, our optimization does not optimize for time, only final value. Key thing to show across the different budgets. 

% \BM{Qualitative. Nayana idea of plot with example sampled case. Should add one here.} 



\subsubsection{Monte Carlo Testing}
Our simulated testing environment is a $5000\times5000$ m square with Gaussian-distributed prior information randomly placed throughout the search space. The number of prior clusters was sampled uniformly between $[4,20]$, with standard deviations between $[60,450]$, and maximum value between $[0.05,0.5]$. 

The results of $100$ Monte Carlo tests are shown in Fig.~\ref{fig:mc_results}. IA-TIGRIS clearly outperforms the other methods, achieving nearly a $40\%$ greater reduction in entropy than the next best method. Early in the simulation, the greedy method initially gains information more quickly, as expected, but this does not translate to better long-term performance. Since our method optimizes for total information gain, it generates paths that maximize information collection over the entire budget. MCTS performed slightly worse than the greedy approach.

The random paths slightly outperformed the coverage paths. This is likely because the lawnmower strategy requires sufficient overlap between passes to avoid missing areas, and its long straight paths often lead to redundant observations due to the UAV’s forward-facing camera. Changing the heading of the UAV is beneficial to viewing more of the search space, which may explain why random paths performed better.

We also conducted Monte Carlo tests where either the number of prior clusters or their standard deviation was held constant to analyze how variations in the information map affect planner performance. The results, shown in Fig.~\ref{fig:targets_sigmas}, include two cases: the upper figure fixes the number of priors, while the lower figure fixes their standard deviation. All other agent and simulation parameters remained unchanged.


% The first thing to note from these results is that for all tests the proportional performance gap between IA-TIGRIS and the baselines increases as the number and standard deviation of the Gaussian priors decreases. As the search space becomes more uniformly filled with entropy in the information map, the need for longer-horizon planning decreases and other simple or random approaches can perform satisfactorily given the testing budget. As the information becomes more sparsely distribution in the space, such as when the information is contained in separated pockets of areas, there is a greater need to plan longer-horizon paths that reason about the given budget.
% \BM{Could have figures here or refer to others}

Across these tests, the performance gap between IA-TIGRIS and the baselines widens as the number and standard deviation of the Gaussian priors decrease. When entropy is more uniformly distributed across the search space, simpler methods perform reasonably well within the given budget. However, when information is concentrated in sparse, distinct regions, longer-horizon planning becomes essential. In such cases, IA-TIGRIS demonstrates a significant advantage by effectively reasoning about the budget and prioritizing high-value regions.

% Show plot of first plans expected info gain versus planning time. (plans not executed)


\subsubsection{Budget Analysis}
To evaluate the impact of budget constraints on performance, we conducted additional tests beyond our initial Monte Carlo experiments, evaluating budgets of $5000$ m, $10000$ m, $30000$ m, and $60000$ m. Table~\ref{tab:budgets} summarizes the average entropy reduction across these budgets.

\definecolor{tabfirst}{rgb}{1, 0.7, 0.7} % red
\definecolor{tabsecond}{rgb}{1, 0.85, 0.7} % orange
\definecolor{tabthird}{rgb}{1, 1, 0.7} % yellow
\begin{table}[t]
    \centering
    \resizebox{\linewidth}{!}{
    \begin{tabular}{l|ccccc}
    & $5000$ m & 10000 m  & 15000 m& 30000 m& 60000 m\\ \hline

    % \hline
    IA-TIGRIS  &  \cellcolor{tabfirst}$9.41\pm1.0$ &  \cellcolor{tabfirst}$18.28\pm1.8$ & \cellcolor{tabfirst}$25.36\pm2.3$ & \cellcolor{tabfirst}$41.08\pm2.9$ & \cellcolor{tabfirst}$58.85\pm2.9$ \\
    Greedy  &  \cellcolor{tabsecond}$6.99\pm0.8$ &  \cellcolor{tabsecond}$13.10\pm1.5$ & \cellcolor{tabsecond}$17.97\pm2.0$ & \cellcolor{tabthird}$30.00\pm2.3$ & \cellcolor{tabsecond}$49.38\pm3.5$ \\
    MCTS  &  \cellcolor{tabthird}$6.06\pm0.7$ &  \cellcolor{tabthird}$11.80\pm1.1$ & \cellcolor{tabthird}$17.11\pm1.4$ & \cellcolor{tabsecond}$30.21\pm2.2$ & \cellcolor{tabthird}$48.68\pm2.7$ \\
    Random  &  $2.19\pm0.3$ & $4.29\pm0.7$ & $6.61\pm0.6$ & $17.50\pm1.2$ & $22.47\pm1.4$ \\
    Coverage  &  $1.58\pm0.3$ &  $2.82\pm0.4$ & $4.09\pm0.7$ & $12.04\pm1.9$ & $16.77\pm2.4$ \\

    \end{tabular}
    }
    \caption{Monte Carlo testing results given different budgets. The values are the average percent reduction in entropy and the 95\% confidence bounds. \mbox{IA-TIGRIS} had the best performance for all budgets.}
    \label{tab:budgets}
\end{table}
%$\uparrow$ 

IA-TIGRIS consistently achieved the highest entropy reduction across all budget constraints, with a statistically significant margin over alternative methods. Greedy generally ranked second but was slightly outperformed by MCTS at the $30000$ m budget level. Greedy and MCTS exhibited comparable performance throughout the tests, with their results closely tracking each other. Consistent with our previous findings, Random and Coverage methods yielded the lowest results.


Among the tested methods, only IA-TIGRIS and MCTS explicitly incorporate budget constraints into their planning algorithms. Notably, at lower budgets ($5000$ m and $10000$ m), these methods achieved higher entropy reduction compared to the equivalent time steps ($200$ s and $400$ s) in the $15000$ m budget scenario shown in Fig.~\ref{fig:mc_results}. This improved performance stems from IA-TIGRIS's optimization of total path reward under budget constraints, contrasting with the myopic next-best-action approach of the greedy method. The remaining methods---Greedy, Random, and Coverage---maintain consistent behavior regardless of budget constraints, as their planning strategies do not account for resource limitations.


The performance gap between IA-TIGRIS and the next-best method varied with budget size, showing margins of $34.6\%$, $39.5\%$, $41.1\%$, $36.0\%$, and $19.2\%$ in ascending budget order. This gap widened through the first three budget levels as problem complexity increased, before declining significantly at higher budgets. This performance pattern suggests that implementing a planning horizon could enhance efficiency by limiting tree search depth, enabling the planner to prioritize path quality optimization over exhaustive space exploration.


% percent improved from next best
% 34.6, 39.5, 41.1, 36.0, 19.2
% reasons, too long horizon is a larger search space, so less quality paths closer. Or larger horizon, more packing in


% with authors goes here
\begin{figure}[t] 
    \centering
    \renewcommand\arraystretch{0} % Adjust the height between rows here
    \setlength{\tabcolsep}{1pt} % Adjust the column separation here
    \begin{tabular}{c}
        \begin{tikzpicture}
            \node[anchor=south west, inner sep=0] (image) at (0,0) {
                \includegraphics[width=0.9\linewidth]{figs/5_/google_earth_prior.png}
            };
            \begin{scope}[x={(image.south east)},y={(image.north west)}]
                % \fill[OrangeRed] (0.02, 0.03) circle (2pt); 
                % \fill[OrangeRed] (0.51, 0.04) circle (2pt); 
                % \fill[OrangeRed] (0.61, 0.04) arc (0:90:2pt); 
                \fill[Orange, opacity=0.8] (0.74, 0.45) circle (3pt); % Adjust 
                \fill[Orange, opacity=0.8] (0.27, 0.42) circle (3pt); % Adjust 
                \fill[Orange, opacity=0.8] (0.39, 0.63) circle (3pt); % Adjust 
            \end{scope}
        \end{tikzpicture} \\
        % \includegraphics[width=0.9\linewidth]{figs/5_/google_earth_prior.png} \\
        \\
        \includegraphics[width=0.9\linewidth]{figs/5_/google_earth_path.png} 
    \end{tabular}
    \caption{Google Earth screenshots illustrating the mission planning process and execution. Top: Areas of high entropy targeted for search are highlighted in red, representing regions with a binary occupied/unoccupied probability of 0.2. Three points of particular interest, each assigned a 0.5 probability, are marked in orange. Bottom: The executed drone flight path (yellow) shows the optimized path for maximum information gain across the search space.} 
    \label{fig:google_earth}
\end{figure}
\begin{figure}[t]
\centering
% https://docs.google.com/presentation/d/1RjI-QqHpBRLHN60UAxzmQYs4EaWaVCOoSBkEkA39kk0/edit?usp=sharing
\includegraphics[width=\columnwidth]{figs/5_/m600_labeled.jpg}
\caption{Hexarotor system (DJI M600 Pro) with onboard compute and camera. Left image shows drone on the ground, right image shows drone in flight.}
\label{fig:m600}
\end{figure}


\section{Field Deployments}\label{sec:field}


\subsection{Hexarotor Deployment}
The first field experiment that we present uses a hexarotor drone to cover an urban area shown in Fig.~\ref{fig:fig1}.
We designed this field experiment to simulate classifying where cars are within a search area.  
Hence, we set the plan request to focus on parking lots at the field test site (Fig.~\ref{fig:google_earth}, top), with the addition of three chosen grid cells within the parking lots being marked as having a higher uncertainty. The plan request boundaries and priors were created with GPS coordinates in Google Earth, exported as kml files, and then converted into our plan request message format. 

The following sections details the hardware, autonomy, and experimental results for our hexarotor deployments.

% without the authors goes here
% \begin{figure}[t] 
%     \centering
%     \renewcommand\arraystretch{0} % Adjust the height between rows here
%     \setlength{\tabcolsep}{1pt} % Adjust the column separation here
%     \begin{tabular}{c}
%         \begin{tikzpicture}
%             \node[anchor=south west, inner sep=0] (image) at (0,0) {
%                 \includegraphics[width=0.9\linewidth]{figs/5_/google_earth_prior.png}
%             };
%             \begin{scope}[x={(image.south east)},y={(image.north west)}]
%                 % \fill[OrangeRed] (0.02, 0.03) circle (2pt); 
%                 % \fill[OrangeRed] (0.51, 0.04) circle (2pt); 
%                 % \fill[OrangeRed] (0.61, 0.04) arc (0:90:2pt); 
%                 \fill[Orange, opacity=0.8] (0.74, 0.45) circle (3pt); % Adjust 
%                 \fill[Orange, opacity=0.8] (0.27, 0.42) circle (3pt); % Adjust 
%                 \fill[Orange, opacity=0.8] (0.39, 0.63) circle (3pt); % Adjust 
%             \end{scope}
%         \end{tikzpicture} \\
%         % \includegraphics[width=0.9\linewidth]{figs/5_/google_earth_prior.png} \\
%         \\
%         \includegraphics[width=0.9\linewidth]{figs/5_/google_earth_path.png} 
%     \end{tabular}
%     \caption{Google Earth screenshots illustrating the mission planning process and execution. Top: Areas of high entropy targeted for search are highlighted in red, representing regions with a binary occupied/unoccupied probability of 0.2. Three points of particular interest, each assigned a 0.5 probability, are marked in orange. Bottom: The executed drone flight path (yellow) shows the optimized path for maximum information gain across the search space.} 
%     \label{fig:google_earth}
% \end{figure}
% \begin{figure}[t]
% \centering
% % https://docs.google.com/presentation/d/1RjI-QqHpBRLHN60UAxzmQYs4EaWaVCOoSBkEkA39kk0/edit?usp=sharing
% \includegraphics[width=\columnwidth]{figs/5_/m600_labeled.jpg}
% \caption{Hexarotor system (DJI M600 Pro) with onboard compute and camera. Left image shows drone on the ground, right image shows drone in flight.}
% \label{fig:m600}
% \end{figure}

\subsubsection{Hardware System}
The hardware consists of the DJI M600 Pro, shown in Fig.~\ref{fig:m600}, along with the physical sensing and onboard computer payload. The DJI M600 Pro contains a flight controller that handles pose estimation and position-based control. The DJI M600 Pro’s flight controller also handles teleloperation if human intervention is necessary. Beneath the drone's base, we mount a custom hardware payload.
That payload consists of an onboard computer, a Jetson Xavier, to run the autonomy software shown in Fig.~\ref{fig:functional_diagram}.
The payload also contains a downward-facing a camera for sensing the environment. The camera is a Seek S304SP thermal camera.
The camera intrinsics are used to calculate the frustum's intersection with the search map's cells in IA-TIGRIS.

% without authors goes here
\begin{figure}[t]
\centering
% https://lucid.app/lucidchart/f750ddb4-2809-4773-8361-d5fbb1ba49eb/edit?viewport_loc=-257%2C-116%2C2219%2C1140%2C0_0&invitationId=inv_56e8a3a9-e8cf-4cad-a280-48bd967ff651
\includegraphics[trim={0cm 0cm 0cm 0cm},clip,width=\columnwidth]{figs/5_/functional_diagram.jpeg}
\caption{Functional diagram of the DJI M600 Pro autonomy software.}
\label{fig:functional_diagram}
\end{figure}
\begin{figure}[b]
    \centering
    \begin{subfigure}[b]{0.48\columnwidth}
        \centering
        \includegraphics[width=1.0\linewidth]{figs/5_/field_test_altitude_over_time.png}
        \caption{}
        \label{fig:m600_altitude_over_time}
    \end{subfigure}
    \begin{subfigure}[b]{0.48\columnwidth}
        \centering
        \includegraphics[width=1.0\linewidth]{figs/5_/field_test_entropy_over_time.png}
        \caption{}
        \label{fig:m600_entropy_over_time}
    \end{subfigure}
    \caption{The results for our hexarotor field deployment. (a) Plot of flown altitude over time, showing large variation throughout the experiment. (b) Reduction in entropy percentage over time of field experiment.}
\end{figure}

\subsubsection{Autonomy System}
Fig.~\ref{fig:functional_diagram} illustrates the functional system diagram for the real world field test on the DJI M600. The user specifies the initial plan request prior to takeoff. The TIGRIS planner makes an initial plan on that plan request and sends a global path to the waypoint manager. The waypoint manager tracks the current waypoint within the plan and sends the next waypoint to the DJI software development kit, which then sends actuation commands to the motors. The position of the drone is used to calculate the distance from the drone to the ground and sends that distance parameter to the sensor model. The sensor model's true positive and false positive rate is used to calculate the per-cell entropy updates in the search map manager. The search map manager publishes the current information map, and the replanning node sends an updated plan request to the IA-TIGRIS planner every ten seconds.

The drone started at an altitude of $50$ m above the origin of the reference frame. The informed sampler in IA-TIGRIS was set to add states at altitudes of either $30$ m or $60$ m, creating a trade-off between observation area and detector accuracy. The budget was $2000$ m, the planning horizon was $600$ m, and the planning time was $10$ seconds. 

% % without authors goes here
% \begin{figure}[t]
% \centering
% % https://lucid.app/lucidchart/f750ddb4-2809-4773-8361-d5fbb1ba49eb/edit?viewport_loc=-257%2C-116%2C2219%2C1140%2C0_0&invitationId=inv_56e8a3a9-e8cf-4cad-a280-48bd967ff651
% \includegraphics[trim={0cm 0cm 0cm 0cm},clip,width=\columnwidth]{figs/5_/functional_diagram.jpeg}
% \caption{Functional diagram of the DJI M600 Pro autonomy software.}
% \label{fig:functional_diagram}
% \end{figure}
% \begin{figure}[b]
%     \centering
%     \begin{subfigure}[b]{0.48\columnwidth}
%         \centering
%         \includegraphics[width=1.0\linewidth]{figs/5_/field_test_altitude_over_time.png}
%         \caption{}
%         \label{fig:m600_altitude_over_time}
%     \end{subfigure}
%     \begin{subfigure}[b]{0.48\columnwidth}
%         \centering
%         \includegraphics[width=1.0\linewidth]{figs/5_/field_test_entropy_over_time.png}
%         \caption{}
%         \label{fig:m600_entropy_over_time}
%     \end{subfigure}
%     \caption{The results for our hexarotor field deployment. (a) Plot of flown altitude over time, showing large variation throughout the experiment. (b) Reduction in entropy percentage over time of field experiment.}
% \end{figure}

\subsubsection{Experimental Results}


The bottom image of Fig.~\ref{fig:google_earth} shows the path selected by IA-TIGRIS in the search area. The figure highlights how the planner dynamically adjusts altitudes over time to balance coverage and sensing resolution, maximizing information gain. Higher altitudes allow for broader area coverage, while lower altitudes provide more detailed observations where needed. Additionally, the planner prioritizes revisiting the three regions of higher uncertainty, recognizing the need for repeated observations reduce entropy. This adaptive strategy ensures that uncertain areas receive sufficient attention to improve the belief map. As a result, the entropy of the information map decreases to near zero by the end of the mission, as shown in Fig.~\ref{fig:m600_entropy_over_time}, indicating that the planner has effectively gathered the necessary information. This behavior demonstrates the planner’s ability to optimize sensing actions, balancing altitude selection, revisit frequency, and exploration to maximize mission success.

\begin{figure}[t]
\centering
% \includegraphics[width=2.5in]{fig1}
\includegraphics[trim={4cm 4cm 0cm 4cm},clip,width=\columnwidth]{figs/5_/TL1.jpg}
\caption{Fixed-wing platform used for autonomous flights with an onboard camera pitched at 10 degrees\cite{alarewebsite}}
\label{fig:tl1}
\end{figure}






\subsection{Fixed-wing Deployments}

Our proposed approach was extensively tested on the fixed-wing AlareTech TL-1 UAV, shown in Fig.~\ref{fig:tl1}. The UAV is equipped with an onboard camera pitched at 10 degrees, which introduces a more challenging planning problem due to the non-holonomic motion model and the camera's field of view. Over more than 20 flight hours and 100 flights running IA-TIGRIS, we validated our approach with the objective to search for objects of interest in a large search space across a variety of test scenarios, including different terrain types, varying environmental conditions, and diverse target distributions. An example mission from these tests is shown in Fig.~\ref{fig:fwd}. In this scenario, the planner was given the search bounds and a designated high-priority region. The resulting flight path prioritized revisiting the high-priority area twice, optimizing sensor use and ensuring maximum information gain. This strategy led to the successful detection of the object of interest, with its estimated position marked by the red dot in the figure. 

The map on the upper right in Fig.~\ref{fig:fwd} shows the information map after plan execution was complete. Due to the UAV's limited budget, the upper right and lower left corners of the map are not searched by the agent. The budget is instead utilized to search over the area of higher priority two times. Compared to the paths in Fig.~\ref{fig:google_earth}, we observe that the paths for the fixed wing are smoother and have a larger turning radius, demonstrating how IA-TIGRIS respects the motion constraints of the vehicle. We can also see the effect of wind on the path execution, where the flown path shown in green deviates from the planned path shown in yellow. This illustrates the importance of online planning in the cases where this deviation is large or would accumulate over the course of a longer mission and cause the expected observed area to be much different than actual observed area. 

\begin{figure}[t]
\centering
% \includegraphics[width=2.5in]{fig1}
% [trim={left bottom right top},clip]
\includegraphics[trim={3.0cm, 1.0cm, 3.0cm, 1.0cm},clip,width=\columnwidth]{figs/5_/ONRFig_v3.pdf}
\caption{An example path generated for the fixed-wing platform conducting a large-area search for an object of interest. The larger black rectangle denotes the search bounds, while the smaller black rectangle highlights a region of higher uncertainty. The red dot marks the estimated position of the detected object based on image detections. The upper-right map displays the information state after planning is complete, while the middle plot shows the percent change in entropy over mission time. The flown path illustrates a balance between allocating resources to the high-priority region and exploring other areas within the search space.}
\label{fig:fwd}
\end{figure}

% Also tested extensively on the AlareTech TL-1 (citation?) tube launched UAV seen in Fig.~\ref{fig:tl1}.

% Talk about amount of flights, hours. Platform. Compute. Show visualization fo example flight. Talk about objects of interest in a broad sense (no mention of water/ocean/land for targets). Follow similar figure format as previous section. Main thing we want to highlight is the differences introduced in plans by having a fixed-wing platform compared to a drone. Include image of Alare TL-1 somewhere.

% One big figure showing all the info we want to convey. 

% \BM{Pitch 10 degrees, onboard computer type, etc}


% \subsection{VTOL?}
% what would it bring?


\section{Ablations}

\subsection{Number and Type of Calibration Images}

\begin{figure}[htbp]
    \vspace{-3mm}
    \centering
    \includegraphics[width=1.0\linewidth]{fig/ablations1.png}
    \caption{Ablation study on the number and type of calibration images used in SITR, showing their effect on (i) Classification accuracy for inter-sensor transfer, (ii) Classification accuracy for intra-sensor transfer, and (iii) Pose estimation error for inter-sensor transfer. }
    \label{fig:ablations_calibration}
    \vspace{-3mm}
\end{figure}

We conduct an ablation study to investigate the impact of the number and type of calibration images on the performance of \modelname. 
In the standard \modelname~setup, we press two objects—a ball and a cube corner—at nine locations roughly arranged in a 3x3 grid pattern across the sensor surface. 
To explore variations, we retrained \modelname~using different subsets of these calibration images and evaluated performance across all downstream tasks.

We test on five calibration configurations: No calibration images (0); Ball pressed at 4 corners (4); Ball pressed in a 3x3 grid (9); Ball and cube pressed at 4 corners ($8^*$); Ball and cube pressed in a 3x3 grid, which is the standard setup ($18^*$).
Fig. \ref{fig:ablations_calibration} illustrates how different numbers and types of calibration images impact \modelname's performance. We observe that increasing the number of calibration images increases performance across all tasks. However, the performance gains diminish as more images of the same object are added (as seen in the progression from cases (0) to (4) to (9)). Introducing a second calibration object with a distinct geometry, such as the cube (cases (4) to (8*)), results in a larger performance boost compared to simply adding more images of the same object (cases (4) to (9)). The effect of calibration images is particularly notable in the inter-sensor setting, where we see upwards of a 20\% increase in classification accuracy from case (0) to (18*). We choose case (18*) for SITR since increasing the number of calibration images does not incur additional inference costs, as calibration tokens are computed only once per sensor.

\subsection{Contrastive loss and temperature}

\begin{figure}[htbp]
    \vspace{-3mm}
    \centering
    \includegraphics[width=1.0\linewidth]{fig/ablations2.png}
    \caption{Ablation study examining the impact of SCL and varying contrastive temperature $\tau$ on SITR’s performance. Subplots (i) and (ii) show classification accuracy in inter-sensor and intra-sensor settings, respectively, while (iii) shows the effect on pose estimation RMSE.}
    \label{fig:ablations_temperature}
    \vspace{-3mm}
\end{figure}

We conduct an ablation study to assess the effect of SCL and varying contrastive temperatures $\tau$ on \modelname's performance. Specifically, we compared models with and without the SCL term and tested five contrastive temperatures: 0.25, 0.10, 0.07, 0.03, and 0.01.
No SCL corresponds to using only the normal map reconstruction loss during pre-training.
Results in Fig. \ref{fig:ablations_temperature} show that a contrastive temperature of 0.07 achieves the best classification performance in the intra-sensor setting, while 0.03 performs best for the inter-sensor setting. Lower or higher temperatures lead to reduced performance in both cases.
For the pose estimation task, the addition of SCL has a negligible impact on the RMSE. 
These results suggest that contrastive learning helps align features across sensors in classification tasks. However, in the pose estimation task, the model's performance is more dependent on the fine-grained geometry information from the contact surface. For SITR, we choose a temperature of 0.07 for its strong performance in the classification task. 
\section{Related Work}
\label{sec:related_work}

\citet{gao2019representationdegenerationproblemtraining}
first described the anisotropy issue, which they referred to as \textit{representation degeneration problem}, and suggested cosine regularization as a mitigation strategy.
Alternative techniques to address the problem have been developed, including adversarial noise \cite{pmlr-v97-wang19f}, spectrum control \cite{Wang2020ImprovingNL} and Laplacian regularization \cite{zhang-etal-2020-revisiting}.  
\citet{bis2021tmic} have shown that the anisotropy of embeddings can for the most part be traced back to a common shift of the embeddings in a dominant direction. They called this phenomenon \textit{common enemy effect}, and provided a semi-quantitative explanation (Eq.~(\ref{eq:chain_rule_e})), which we developed further in the present work by including the optimizer in the analysis.
In \citet{yu-etal-2022-rare}, Adaptive Gradient Gating is proposed, based on the empirical observation that it is the gradients for embeddings of rare tokens that cause anisotropy. Our analysis conforms to this finding and attributes it to a massive up-scaling of the gradients for rare embeddings with Adam, cf.~Fig.~\ref{fig:gradients_example}.
\citet{machina-mercer-2024-anisotropy} have demonstrated that large Pythia models \cite{pmlr-v202-biderman23a} show improved isotropy compared to similar models, and attribute this to the absence of weight tying. This is in accordance with our analysis of the unembedding gradients in conjunction with Adam, Sec.~\ref{sec:theory}.
While all the previously mentioned papers use average cosine similarity \cite{ethayarajh-2019-contextual} or $\ISO$ from Eq.~(\ref{eq:isotropy}) to quantify the geometry of embedding vectors, \citet{rudman-etal-2022-isoscore} deviate from this. 
Their notion of isotropy is based solely on the embeddings' covariance matrix and embodied by the metric IsoScore. 
In particular, IsoScore is mean-agnostic, while $\ISO$ strongly correlates with the mean embedding (see e.g. Tab.~\ref{tab:results_S}).
Finally, concurrent to our work, \citet{zhao2024deconstructingmakesgoodoptimizer} have investigated the importance of using the second moment in Adam with regard to performance and stability. They found that simplified variants of Adam that use the same effective learning rate either for the whole embedding matrix (Adalayer) or each embedding vector (Adalayer*) are slightly worse than Adam but better than SGD. Adalayer* is similar to Coupled Adam, but corresponds to the second moment averaged over hidden space instead of vocabulary space.

\section{Conclusions}

Our work addresses the well-known anisotropy problem for LLM embeddings. 
We have advanced the theoretical understanding of the phenomenon by showing that it is a combination of the common enemy effect and the individual second moments in Adam that causes a collective shift of the embedding vectors away from the origin.
To mitigate the problem, we have introduced Coupled Adam, which enforces the same effective learning rate for every embedding vector, and thus suppresses the collective shift of the embeddings.
We have found that Coupled Adam consistently improves embedding-specific metrics across all experiments, while also achieving better downstream and upstream performance for large datasets, as they are typically used in LLM training.
The code to reproduce our results is available at \href{https://github.com/flxst/coupled-adam}{\nolinkurl{github.com/flxst/coupled-adam}}~.

\section{Limitations}

Although our method is generally applicable to all common LLM architectures, as they share the same language modeling head and embeddings, only dense decoders were used in our experiments. 
In addition, only models with up to $N=2.6\B$ parameters have been tested.
The cosine decay learning rate schedule was applied throughout all experiments (App.~\ref{app:hyperparameters}). Alternatives such as an infinite learning rate schedule are not incorporated in our study.
Furthermore, as mentioned at the end of Sec.~\ref{sec:results}, we have not explicitly verified that the slight residual shift of the mean embedding, which is observed even for Coupled Adam, is caused by weight tying.
Finally, we have used a straightforward implementation of Coupled Adam, closely following Algorithm~\ref{alg:algorithm_adam}. More sophisticated implementations might lead to increased efficiency and further improvements; we leave it for future work to investigate this.

\section*{Acknowledgements}
Our computational experiments used around 20000 GPU hours. They were partly run on the EuroHPC supercomputers MeluXina and MareNostrum5 in conjunction with the grants EHPC-DEV-2023D10-032 and EHPC-EXT-2023E02-038.


\bibliography{custom}

\appendix
\section{Unigram Probability Distribution}
\label{app:unigram_probability_example}

Fig.~\ref{fig:prior_example_gpt2_openwebtext} shows the unigram probability distribution for the example of the OpenWebText Corpus dataset and the GPT-2 tokenizer.
\begin{figure}[H]
    \centering
    \includegraphics[scale=0.5]{figs/global_probability_distribution_openwebtext_log.png}
    \caption{Logarithm $\log(\widetilde{p_i})$ of the unigram probability distribution for the OpenWebText Corpus and the GPT-2 tokenizer. The maximum probability is $\max_{i} \widetilde{p_i} \approx 0.037$ or $\max_{i} \log(\widetilde{p_i}) \approx -3.30$. The minimum probability (not shown) is $\min_{i} \widetilde{p_i} = 0$ or $\min_{i} \log(\widetilde{p_i}) = -\infty$.}
    \label{fig:prior_example_gpt2_openwebtext}
\end{figure}

\pagebreak
\section{Embedding Gradients}
\label{app:chain_rule_e}

We explicitly derive Eq.~(\ref{eq:chain_rule_e}), which we recall here for convenience:
\begin{align}
g_i :=~ &\frac{\partial \mathcal{L}}{\partial e_i} 
= - \left( \delta_{it} - p_i \right) \cdot h \tag{\ref{eq:chain_rule_e}}
\end{align}
The chain rule yields
\begin{align}
\frac{\partial \mathcal{L}}{\partial e_i} 
&= \sum_{k=1}^{V} \frac{\partial \mathcal{L}}{\partial p_t} 
\cdot \frac{\partial p_t}{\partial l_k}  
\cdot \frac{\partial l_k}{\partial e_i} \; ,
\label{eq:chain_rule_basis}
\end{align}
where the individual factors can directly be obtained from Eqs.~(\ref{eq:forward_loss})-(\ref{eq:gradient_function_new}):
\begin{align}
\frac{\partial \mathcal{L}}{\partial p_t} &= - \frac{1}{p_t} \label{eq:backward_loss} \\
\frac{\partial p_t}{\partial l_k} 
&= \frac{\delta_{kt} \exp{(l_t)} \cdot \Sigma - \exp{(l_t)} \exp{(l_k)}}{\Sigma^2} \nonumber \\ 
&= \delta_{kt} p_t - p_t p_k \nonumber \\
&= p_t ( \delta_{kt} - p_k ) 
\label{eq:backward_loss_2} \\
\frac{\partial l_k}{\partial e_i} &= \delta_{ki} h \label{eq:backward_e}
\end{align}
Note that in the first line of Eq.~(\ref{eq:backward_loss_2}), we use the abbreviation $\Sigma = \Big( \sum_{j=1}^V \exp{(l_j)} \Big)$.
Inserting Eqs.~(\ref{eq:backward_loss}), (\ref{eq:backward_loss_2}) and (\ref{eq:backward_e}) into Eq.~(\ref{eq:chain_rule_basis}) directly leads to Eq.~(\ref{eq:chain_rule_e}):
\begin{align}
\frac{\partial \mathcal{L}}{\partial e_i} 
&= - \sum_{k=1}^{V} \frac{1}{p_t}
\cdot p_t ( \delta_{kt} - p_k )  
\cdot \delta_{ki} h \nonumber \\
&= - \sum_{k=1}^{V} ( \delta_{kt} - p_k )  
\cdot \delta_{ki} h \nonumber \\
&= - ( \delta_{it} - p_i )  
\cdot h \nonumber
\end{align}

\section{SGD Algorithm}
\label{app:sgd_algorithm}

For completeness and comparison to (Coupled) Adam as displayed in Algorithm~\ref{alg:algorithm_adam}, we summarize the SGD algorithm in Algorithm~\ref{alg:algorithm_sgd}.

\begin{algorithm}[H]
    \small
    \textbf{Input:}
    $\eta$ (lr), $e_i^{(0)}$ (initial embeddings),
    $\mathcal{L}(e_i)$ (objective), $\gamma$ (momentum), $T$ (number of time steps) \\
    \textbf{Output}: $e^{(T)}$ (final embeddings)
    \begin{algorithmic}[1]
        \For{$\tm=1 \dots T$}
            \For{$i=1 \dots V$}
                \State $g_i^{(\tm)}$ $\gets$ $\nabla_{e_i} \mathcal{L}^{(\tm)} (e_i^{(\tm-1)})$
                \If{$t>1$}
                    \State $\mathbf{b}_i^{(\tm)}$ $\gets$ $\gamma \mathbf{b}_i^{(\tm-1)} + g_i^{(\tm)}$
                \Else
                    \State $\mathbf{b}_i^{(\tm)}$ $\gets$ $g_i^{(\tm)}$
                \EndIf
                \State $e_i^{(\tm)}$ $\gets$ $e_i^{(\tm-1)} - \eta \mathbf{b}_i^{(\tm)}$
            \EndFor
        \EndFor
        \vspace*{1.0ex}
        \State \Return $e^{(T)}$
    \end{algorithmic}
    \caption{Pseudocode for the SGD algorithm with optional momentum, applied to the embedding vectors $e_i$.}
    \label{alg:algorithm_sgd}
\end{algorithm}


\section{Magnitude of the Second Moment in Adam}

In this appendix, the validity of 
\begin{equation}
    \E \left[ \secondmomentshort \right] \propto \widetilde p_i
    \tag{\ref{eq:second_moment_linear_in_unigram_prob}}
\end{equation}
is verified.
Due to the linearity of lines 5 and 7 in Algorithm 2, it suffices to show that the squared gradient has the property in question:
\begin{align}
\E \left[ g_i^2 \right] &\propto \widetilde p_i
\label{eq:squared_gradient_linear_in_unigram_prob}
\end{align}
We do this in two different ways.
First, we derive Eq.~(\ref{eq:squared_gradient_linear_in_unigram_prob}) using a semi-theoretical approach with minimal experimental input.
Afterwards, we confirm the relationship in a purely experimental manner. %

\subsection{Semi-theoretical Derivation}
\label{app:second_moment_theory}

Here, we derive an expression for the expectation value of the squared gradient in terms of simple observables (Theorem~\ref{theorem}). Subsequently, the dependency of those observables on $\widetilde p_i$ is determined experimentally. Together, this will yield the proportionality expressed by Eq.~(\ref{eq:squared_gradient_linear_in_unigram_prob}).
We begin our reasoning with a lemma.

\begin{lemma}[Expectation Value Decomposition]\label{lemma}
The expectation value of the squared gradient can be decomposed into conditional expectation values as follows:
\begin{align}
\E \left[ g_i^2 \right] =~ 
&\widetilde p_i \cdot \E \left[ g_i^2 ~\big|~ i=t \right] \nonumber \\
&+ (1 - \widetilde p_i) \cdot \E \left[ g_i^2 ~\big|~ i \neq t \right]
\label{eq:lemma1}
\end{align}
\end{lemma}

\begin{proof}
Our starting point is the definition of the expectation value for the continuous random variable $g_i^2$:
\begin{align}
\E \left[ g_i^2 \right] = \int g_i^2 ~p(g_i) ~dg_i \; ,
\label{eq:aux_E_definition}
\end{align}
where $p$ denotes the probability distribution of $g_i$. Since the vocabulary item $i$ can only be either the true token $t$ or not, we can decompose $p$ into a sum of joint probability distributions (using the {\em law of total probabilities}), each of which can be expressed in terms of conditional probabilities like so:
\begin{align}
p(g_i) 
&= p(g_i, i=t) + p(g_i, i\neq t) \nonumber \\
&= p(g_i ~|~ i=t) \cdot p(i=t) \nonumber \\
&\quad + p(g_i ~|~ i\neq t) \cdot p(i\neq t)
\end{align}
Using the unigram probability $\widetilde p_i = p(i = t)$, this can also be written as
\begin{align}
p(g_i) 
&= \widetilde p_i \cdot p(g_i ~|~ i=t) \nonumber \\
&\quad + (1 - \widetilde p_i) \cdot p(g_i ~|~ i\neq t)
\label{eq:aux_p_decomposition}
\end{align}
If we insert Eq.~(\ref{eq:aux_p_decomposition}) back into Eq.~(\ref{eq:aux_E_definition}), the expectation value becomes
\begin{align}
\E \left[ g_i^2 \right] 
&= \widetilde p_i \cdot \int g_i^2 ~p(g_i ~|~ i=t) ~dg_i \nonumber \\
&\quad + (1 - \widetilde p_i) \cdot \int g_i^2 ~p(g_i ~|~ i\neq t) ~dg_i \;, 
\label{eq:aux_E_decomposition} %
\end{align}
which by definition of the (conditional) expectation value, Eq.~(\ref{eq:aux_E_definition}), is equivalent to Eq.~(\ref{eq:lemma1}).
\end{proof}
\begin{theorem}[Expectation Value Squared Gradient] \label{theorem}
Given that the squared hidden state vector $h^2$ is independent of $p_i$ and whether $i$ is the true token or not, the expectation value of the squared gradient $g_i^2$ is given by
\begin{align}
\E \left[ g_i^2 \right] 
&= S \cdot \left[ \widetilde p_i \cdot X_i^{(i=t)} + (1 - \widetilde p_i) \cdot X_i^{(i \neq t)} \right] \; ,
\label{eq:theorem}
\end{align}
with
\begin{align}
S &:= \E \left[ h^2 \right] \label{eq:optimizer_second_moment_expansion_S} \\
X_i^{(i=t)} &:= \E \left[ (1 - p_i)^2 ~\big|~ i=t \right]
\label{eq:optimizer_second_moment_expansion_true_2} \\
X_i^{(i \neq t)} &:= \E \left[ p_i^2 ~\big|~ i \neq t \right]
\label{eq:optimizer_second_moment_expansion_false_2}
\end{align}
\end{theorem}

\begin{proof}
We start from Lemma~\ref{lemma} and the square of the gradient,
\begin{align}
g_i^2 
&\stackrel{(\ref{eq:chain_rule_e})}{=} \left( \delta_{it} - p_i \right)^2 h^2 
\label{eq:optimizer_second_moment}
\end{align}
Note that squared variables of vectors in $\mathbb{R}^H$ always denote the elementwise (Hadamard) product, e.g.
\begin{align}
g_i^2 &\equiv g_i \odot g_i \in \mathbb{R}_{\geq 0}^H \; ,
\label{eq:hadamard_product}
\end{align}
with strictly non-negative elements.
Using Eq.~(\ref{eq:optimizer_second_moment}), the expectation values on the right side of Eq.~(\ref{eq:lemma1}) can be expressed as
\begin{align}
\E \left[ g_i^2 ~\big|~ i=t \right]
&= \E \left[ \left( 1 - p_i \right)^2 \cdot h^2 ~\big|~ i=t \right] \\
\E \left[ g_i^2 ~\big|~ i\neq t \right]
&= \E \left[ p_i^2 \cdot h^2 ~\big|~ i\neq t \right]
\end{align}
Given our assumptions regarding $h^2$, its expectation value can be factored out:
\begin{align}
\E \left[ g_i^2 ~\big|~ i=t \right]
&= S \cdot X_i^{(i=t)} \label{eq:Etrue} \\
\E \left[ g_i^2 ~\big|~ i\neq t \right]
&= S \cdot X_i^{(i \neq t)} \label{eq:Efalse} 
\end{align}
Inserting Eqs.~(\ref{eq:Etrue}) and (\ref{eq:Efalse}) into Eq.~(\ref{eq:lemma1}) yields Eq.~(\ref{eq:theorem}).
\end{proof}

Note that Eq.~(\ref{eq:theorem}) is a vector equation, with $\E \left[ g_i^2 \right], S \in \mathbb{R}_{\geq 0}^H$ and $\widetilde p_i, X_i^{(i=t)}, X_i^{(i \neq t)} \in \mathbb{R}_{\geq 0}$.
It states that the expectation value of $g_i^2$ factorizes into a global constant $S$ that is $i$-independent, and a factor that is $i$-dependent. The latter is a specific combination of the unigram probability $\widetilde p_i$, determined by the data, and the conditional expectation values $X_i^{(i=t)}$ and $X_i^{(i\neq t)}$, determined by the model.

\paragraph{Experimental Input}

Regarding the unigram probability, we know that
\begin{enumerate}
\item $\widetilde p_i \ll 1$. \\
This is the case for virtually all natural language datasets with a common vocabulary size of $V > 10000$, according to Zipf's law.
\end{enumerate}
The conditional expectation values $X_i^{(i=t)}$ and $X_i^{(i\neq t)}$ can be empirically estimated by applying training data to different checkpoints. We consider the three small-scale experiments of Sec.~\ref{sec:experiments_S} with $N \in \{ 125\M, 355\M, 760\M \}$ and $D=20\B$, and take ten equidistant checkpoints after $D^\prime \in \{ 2\B, 4\B, \ldots, 20\B\}$ seen tokens for each of them. We then continue pseudo-training on 20 batches ($\approx$ 2k samples or 2M tokens, see Tab.~\ref{tab:model_architecture}) of data using a zero learning rate, and measure the conditional probabilities in Eqs.~(\ref{eq:optimizer_second_moment_expansion_true_2}, \ref{eq:optimizer_second_moment_expansion_false_2}) from which our target quantities can be estimated.  
Subsequently, linear fits of the form 
\begin{align}
    X_i^{(i = t)} &= A^{(i = t)} \cdot \widetilde p_i \\
    X_i^{(i \neq t)} &= A^{(i \neq t)} \cdot \widetilde p_i \; ,
\end{align}
with fit parameters $A^{(i = t)}$ and $A^{(i \neq t)}$ are performed. $R^2$ is used to assess the quality of the fits. In addition, the mutual information $\I$ between the response and the explanatory variable is computed. 
Since we observe only a very weak dependence of the results for $R^2$ and $\I$ on $N$ and $D^\prime$, we specify the mean and standard deviation over all experiments for them.
Our findings are:
\begin{enumerate}
\item[2.] $X_i^{(i = t)}$ is independent of $\widetilde p_i$. \\
The linear fits yield $R^2 = 0.003(1)$, and the mutual information is $\I \left(X_i^{(i = t)}; \widetilde p_i \right) = 0.14(2)$.
\item[3.] $X_i^{(i \neq t)}$ is proportional to $\widetilde p_i$. \\
The linear fits yield $R^2 = 0.92(1)$, and the mutual information is $\I \left( X_i^{(i \neq t)}; \widetilde p_i \right) = 0.50(2)$.
\end{enumerate}

The three empirical results above, together with Theorem~\ref{theorem}, immediately lead to Eq.~(\ref{eq:squared_gradient_linear_in_unigram_prob}).

\subsection{Experimental Confirmation}
\label{app:second_moment_empirical}

We reuse the experiments from the previous section to measure the second moment $\secondmomentshort$ directly, in order to estimate $\E \left[ \secondmomentshort \right]$. Again, linear fits of the form
\begin{align}
    \E \left[ \secondmomentshort \right] = A \cdot \widetilde p_i
\end{align}
are performed and the mutual information is computed. 
We find
\begin{enumerate}
\item[4.] $\E \left[ \secondmomentshort \right]$ is proportional to $\widetilde p_i$. \\
The linear fits yield $R^2 = 0.85(7)$, and the mutual information is $\I \left( \E \left[ \secondmomentshort \right]; \widetilde p_i \right) = 1.18(9)$.
\end{enumerate}
The results for $N=125\M$ and $D=D^\prime=20\B$ are depicted in Fig.~\ref{fig:experimental_results_E_p}, as an example.
\begin{figure}[h!]
    \centering
    \includegraphics[scale=0.5]{figs/experimental_results_E_p_125M_20B.png}
    \caption{Experimental results for $\E \left[ \secondmomentshort \right]$ (vertical axis) vs. $\widetilde p_i$ (horizontal axis) for $N=125\M$ and $D=D^\prime=20\B$. The blue line shows the linear fit with $R^2 = 0.91$.}
    \label{fig:experimental_results_E_p}
\end{figure}

Note that while $R^2$ and $\I$ are again virtually independent of $N$ and $D^\prime$, the fit parameter $A$ is not. Instead, it seems to increase with $D^\prime$, as shown in Fig.~\ref{fig:experimental_results_A}.
\begin{figure}[h!]
    \centering
    \includegraphics[scale=0.5]{figs/experimental_results_A.png}
    \caption{Experimental results for the linear fit parameter $A$ as a function of $N$ and $D^\prime$.}
    \label{fig:experimental_results_A}
\end{figure}
However, as stated in Eq.~(\ref{eq:second_moment_proportionality_constant}), the order of magnitude is $A \approx 10^{-4}$ throughout our experiments.

\section{Experimental Details}
\subsection{Model and Dataset Sizes}
\label{app:experiments_overview}

The model sizes $N$ and dataset sizes $D$ employed in our experiments are depicted in Fig.~\ref{fig:experiments}.
\begin{figure}[h!]
    \centering
    \includegraphics[scale=0.5]{figs/experiments_log.png}
    \caption{Overview of the dataset (horizontal axis) and model sizes (vertical axis) involved in our small-scale (blue, green and orange circles) and large-scale (red squares) experiments. The dashed, black line shows $N = D / 20$, which is approximately the compute-optimal trajectory according to \citet{hoffmann2022trainingcomputeoptimallargelanguage}.}
    \label{fig:experiments}
\end{figure}


\subsection{Training Hyperparameters}
\label{app:hyperparameters}

In Tab.~\ref{tab:model_architecture}, we list the general hyperparameters used in our small-scale (Sec.~\ref{sec:experiments_S}) and large-scale (Sec.~\ref{sec:experiments_L}) experiments. 
\begin{table}[ht]
\centering
\scriptsize
\begin{tabular}{l|cc}
\toprule
Description & Small-scale & Large-scale \\ 
\midrule
optimizer & \multicolumn{2}{c}{AdamW} \\ 
$\beta_1$ & \multicolumn{2}{c}{0.9} \\
$\beta_2$ & \multicolumn{2}{c}{0.95} \\
$\epsilon$ & \multicolumn{2}{c}{1e-8} \\
weight decay & \multicolumn{2}{c}{0.1} \\
gradient clipping & \multicolumn{2}{c}{1.0} \\
dropout & \multicolumn{2}{c}{0.0} \\
weight tying & \multicolumn{2}{c}{true} \\
vocab size & \multicolumn{2}{c}{50304} \\
learning rate schedule & \multicolumn{2}{c}{cosine decay} \\
layer normalization & \multicolumn{2}{c}{LayerNorm} \\
precision & \multicolumn{2}{c}{BF16} \\
\midrule
hidden activation & GeLU & SwiGLU \\
positional embedding & absolute (learned) & RoPE \\ 
sequence length & 1024 & 2048 \\
batch size (samples) & 96 & 256 \\
batch size (tokens) & $\sim$100k & $\sim$500k \\
warmup & 100 steps & $1\%$ of steps \\
training framework & nanoGPT & Modalities \\
training parallelism & DDP & FSDP \\
\bottomrule 
\end{tabular}
\caption{General hyperparameters used in our two sets of experiments.}
\label{tab:model_architecture}
\end{table}
During warm-up, the learning rate is increased from zero to the maximum learning rate. This is followed by a cosine decay which reduces the learning rate to $10\%$ of the maximum at the end of training. Note that weight decay is applied only to linear layers, not layer norms or embeddings.
Tab.~\ref{tab:model_sizes} shows the hyperparameters related to model size, following GPT-3 \cite{brown2020languagemodelsfewshotlearners}.
\begin{table}[ht]
\centering
\scriptsize
\begin{tabular}{c|cccc}
\toprule
$N$ & lr & heads & layers & emb. dim. \\ 
\midrule
124M & 6.0e-4 & 12 & 12 & 768 \\
350M & 3.0e-4 & 16 & 24 & 1024 \\
760M & 2.5e-4 & 16 & 24 & 1536 \\
1.3B & 2.0e-4 & 32 & 24 & 2048 \\
2.6B & 1.6e-4 & 32 & 32 & 2560 \\
\bottomrule 
\end{tabular}
\caption{Model-size dependent hyperparameter used in our experiments. $N$ denotes the model size in terms of parameters, while lr corresponds to the maximum learning rate.}
\label{tab:model_sizes}
\end{table}

\section{Error Analysis and Statistical Significance}
\label{app:error}

For the error analysis, two separate random variables, $X_0$ and $X_1$, are considered. The symbol $X$ represents one of the metrics discussed in Sec.~\ref{sec:experiments_evaluation}, while $0$ and $1$ stand for two approaches that are to be compared, like standard Adam and Coupled Adam, for instance.
For each of the two random variables $i = \{ 0, 1 \}$, we conduct and evaluate $S$ training runs with different seeds, yielding results 
\begin{align}
\{ X_i^{(1)}, \ldots, X_i^{(S)} \}
\end{align}
While it is desirable to have a large sample size $S$, it is  prohibitively expensive for large model and dataset sizes to repeat training runs. We use
\begin{align}
S &= 3
\label{eq:error_S3}
\end{align}
except for the large-scale experiments (Sec.~\ref{sec:experiments_L}), where we restrict ourselves to
\begin{align}
S &= 1
\label{eq:error_S1}
\end{align}
We are interested in the difference 
\begin{align}
d = X_1 - X_0
\label{eq:error_d}
\end{align}
For $S=1$, it can be computed straight forwardly. However, no statement about the statistical uncertainty or significance of $d$ can be made.
In the case of $S = 3$, we apply a one-sided Student's t-test with a confidence level of 
\begin{align}
\alpha = 95\%
\label{eq:error_confidence_level_alpha}
\end{align}
First, the sample means
\begin{align}
\bar X_i &= \frac{1}{S} \sum_{s=1}^S X_i^{(s)}
\label{eq:student_mean_i}
\end{align}
and the corrected sample standard deviations
\begin{align}
\hat \sigma_i^2 &= \frac{1}{S-1} \sum_{s=1}^S \left( X_i^{(s)} - \bar X_i \right)^2
\label{eq:student_std_i}
\end{align}
for the two samples $i \in \{0, 1\}$ are estimated.
The sample means from Eq.~(\ref{eq:student_mean_i}) are combined to an estimate for their difference,
\begin{align}
\bar d &= \bar X_1 - \bar X_0
\label{eq:student_mean_d}
\end{align}
and the sample standard deviations from Eq.~(\ref{eq:student_std_i}) are propagated to the sample standard deviation of $d$ via Gaussian error propagation:
\begin{align}
\hat \sigma_d &= 
\sqrt{\left( \frac{\partial d}{\partial X_0} \cdot \hat \sigma_0 \right)^2 + \left( \frac{\partial d}{\partial X_1} \cdot \hat \sigma_1 \right)^2} \nonumber \\
&\stackrel{(\ref{eq:error_d})}{=} \sqrt{\hat \sigma_0^2 + \hat \sigma_1^2}
\label{eq:sigmad}
\end{align}

Student's t-distribution for the chosen confidence level $\alpha$ (see Eq.~(\ref{eq:error_confidence_level_alpha})) and the
\begin{align}
\nu &= S - 1 \stackrel{(\ref{eq:error_S3})}{=} 2 
\end{align}
degrees of freedom yields
\begin{align}
t_{\alpha, \nu} = 2.92
\label{eq:talphanu}
\end{align}
With $S$, $\sigma_d$ and $t_{\alpha, \nu}$ from Eqs.~(\ref{eq:error_S3}), (\ref{eq:sigmad}) and (\ref{eq:talphanu}) as ingredients, the one-sided confidence threshold for the difference can be computed as
\begin{align}
d_{\rm significance}
&= t_{\alpha, \nu} \cdot \frac{\hat \sigma_d}{\sqrt{S}}
\label{eq:studentonesidedconfidencethreshold}
\end{align}
Hence, the estimate $\bar d$ from Eq.~(\ref{eq:student_mean_d}) is considered a statistically significant improvement of approach $i=1$ over approach $i=0$ if
\begin{align}
\bar d < - d_{\rm significance}
\label{eq:student_signficance_loss}
\end{align}
for metrics where smaller values are desirable (e.g. $\Loss$), and 
\begin{align}
\bar d > d_{\rm significance}
\label{eq:student_signficance_other}
\end{align}
for metrics where larger values are better (e.g. $\Acc$).

\CatchFileDef{\resultsAblationsSGDExpFive}{tables/results_ablations_sgd_only_exp12.tex}{}
\CatchFileDef{\resultsAblationsSGDExpTen}{tables/results_ablations_sgd_only_exp13.tex}{}
\CatchFileDef{\resultsAblationsSGDExpTwenty}{tables/results_ablations_sgd_only_exp15.tex}{}


\section{Additional Results}

\subsection{Small-scale Experiments}
\label{app:additional_results_S}

In Fig.~\ref{fig:results_S2}, we visualize the results of our small-scale experiments (Sec.~\ref{sec:results_S}) for the loss $\Loss$ and the average downstream task accuracy $\Acc$, as listed in Tab.~\ref{tab:results_S}.
\begin{figure*}[t]
    \centering
    \includegraphics[scale=0.5]{figs/diff-loss-error.png} \qquad
    \includegraphics[scale=0.5]{figs/diff-lm-eval-error.png}
    \caption{Difference in loss (left) and average downstream task accuracy (right) between Coupled Adam and standard Adam, for the different dataset sizes $D$ (horizontal axis) and model sizes $N$ (colors) of the small-scale experiments. The vertical bars indicate the one-sided $95\%$ confidence interval for the difference to be significant. In order to avoid overlaps, the data points for $N=125\M$ and $N=760\M$ are slightly shifted to the left and right, respectively. }
    \label{fig:results_S2}
\end{figure*}

\subsection{Scaled Coupled Adam}
\label{app:additional_results_scale}

Tab.~\ref{tab:results_ablations_scale} of 
Sec.~\ref{sec:ablation_different_learning_rate} shows the results of varying the scaling exponent $n$ (see Eq.~(\ref{eq:optimizer_update_second_moment_avg_scaled})) for $D = 20\B$. The dependency of the loss is visualized in Fig.~\ref{fig:ablation_scale}.
Here, in Fig.~\ref{fig:ablation_scale_complete}, we extend the visualization of the results to $D \in \{ 5\B, 10\B, 20\B \}$ and the other evaluation metrics.
\begin{figure*}[ht]
    \centering
    \includegraphics[scale=0.5]{figs/ablation_1_loss.png}
    \includegraphics[scale=0.5]{figs/ablation_1_lmeval.png}
    \includegraphics[scale=0.5]{figs/ablation_1_isotropy.png}
    \includegraphics[scale=0.5]{figs/ablation_1_mu_rel.png}
    \includegraphics[scale=0.5]{figs/ablation_1_cos.png}
    \caption{Dependency of different metrics on the scaling exponent $n$, see Eq.~(\ref{eq:optimizer_update_second_moment_avg_scaled}). From top to bottom: loss (upstream performance), average accuracy (downstream performance), isotropy, mean embedding norm ratio and $\rcos$. Each plot shows the difference to the respective metric obtained for $n = 0$. The arrows indicate whether larger ($\uparrow$) or smaller ($\downarrow$) values are desirable.}
    \label{fig:ablation_scale_complete}
\end{figure*}

\subsection{SGD}
\label{app:additional_results_sgd}

In Tab.~\ref{tab:results_ablations_sgd_all} of Sec.~\ref{sec:ablation_sgd}, we showed results for SGD using the best hyperparameter $f$. 
Detailed results of the corresponding hyperparameter searches can be found in Tab.~\ref{tab:results_ablations_sgd}.
\begin{table*}[ht]
\centering
\scriptsize
\begin{tabular}{ccc|rrrrrrrr}
\toprule
$D$ & $N$ & Optimizer & $\Loss$ ($\downarrow$) & $\Acc$ ($\uparrow$) & $\ISO$ ($\uparrow$) & $\munorm$ ($\downarrow$) & $\munormrel$ ($\downarrow$) & $\rcos$ ($\uparrow$) & $\rho$ ($\uparrow$) & $\kappa$ ($\uparrow$) \\ 
\midrule
\resultsAblationsSGDExpFive
\bottomrule 
\\
\\
\end{tabular}
\begin{tabular}{ccc|rrrrrrrr}
\toprule
$D$ & $N$ & Optimizer & $\Loss$ ($\downarrow$) & $\Acc$ ($\uparrow$) & $\ISO$ ($\uparrow$) & $\munorm$ ($\downarrow$) & $\munormrel$ ($\downarrow$) & $\rcos$ ($\uparrow$) & $\rho$ ($\uparrow$) & $\kappa$ ($\uparrow$) \\ 
\midrule
\resultsAblationsSGDExpTen
\bottomrule 
\\
\\
\end{tabular}
\begin{tabular}{ccc|rrrrrrrr}
\toprule
$D$ & $N$ & Optimizer & $\Loss$ ($\downarrow$) & $\Acc$ ($\uparrow$) & $\ISO$ ($\uparrow$) & $\munorm$ ($\downarrow$) & $\munormrel$ ($\downarrow$) & $\rcos$ ($\uparrow$) & $\rho$ ($\uparrow$) & $\kappa$ ($\uparrow$) \\ 
\midrule
\resultsAblationsSGDExpTwenty
\bottomrule 
\end{tabular}
\caption{Results of our experiments with SGD. Values are highlighted in bold if they are significantly better than all the other values in the same column.}
\label{tab:results_ablations_sgd}
\end{table*}


\end{document}
