% This must be in the first 5 lines to tell arXiv to use pdfLaTeX, which is strongly recommended.
\pdfoutput=1
% In particular, the hyperref package requires pdfLaTeX in order to break URLs across lines.

\documentclass[11pt]{article}

\usepackage[dvipsnames]{xcolor} 
% Change "review" to "final" to generate the final (sometimes called camera-ready) version.
% Change to "preprint" to generate a non-anonymous version with page numbers.
\usepackage[preprint]{acl}


\usepackage{times}
\usepackage{latexsym}

\usepackage[T1]{fontenc}

\usepackage[utf8]{inputenc}

\usepackage{microtype}

\usepackage{inconsolata}

\usepackage{graphicx}


\usepackage{amsmath}
\usepackage{amssymb}
\usepackage{multirow}
\usepackage{booktabs}
\usepackage{catchfile}

\usepackage[boxed]{algorithm}
\usepackage{varwidth}
\usepackage[noEnd=true,indLines=false]{algpseudocodex}
\usepackage{cleveref}
\makeatletter
\@addtoreset{ALG@line}{algorithm}
\renewcommand{\ALG@beginalgorithmic}{\small}
\algrenewcommand\alglinenumber[1]{\small #1:}
\makeatother

\usepackage[normalem]{ulem}
\usepackage{todonotes}

\usepackage{lipsum}    %
\usepackage{comment}   %
\usepackage{graphicx}  %
\usepackage{pifont}    %

\usepackage[font=small,labelfont=bf]{caption}
\usepackage{float}     %
\usepackage{booktabs}  %
\usepackage{subcaption}  %

\usepackage{listings}

\usepackage{amsthm}  %

\section{Problem Studied}\label{sec:def}
We first present Fixed-Radius Near Neighbor (FRNN) queries and then formalize Aggregation Queries over Nearest Neighbors (AQNNs) that build on them. We then state our problem.

\subsection{Nearest Neighbor Queries}\label{subsec:FRNN}
We build on generalized Fixed-Radius Near Neighbor (FRNN) queries \cite{FRNNSurvey}. Given a dataset \( D \), a query object \( q \), a radius \( r \), and a distance function \( dist \), a generalized FRNN query retrieves all nearest neighbors of \( q \) within radius \( r \). More formally:
\[
NN_D(q, r) = \{x \in D \mid dist(x, q) \leq r\},
\]
where \(x\) is any data point in \(D\) and \(dist(x, q)\) denotes the distance between them. We use \(|NN_D(q,r)|\) to denote the neighborhood size of \(q\). As shown in Fig. \ref{fig:framework}, given a radius \(r\) and a target patient \(q\), patients in the dotted circle are nearest neighbors, and the neighborhood size is 6.

\subsection{Aggregation Queries over Nearest Neighbors}\label{subsec:AQNN} 
Given an FRNN query object \(q\) in dataset \(D\), a radius \(r\), and an attribute \(\texttt{attr}\), an Aggregation Query over Nearest Neighbors (AQNN) is defined as:
\[ \text{agg}(NN_D(q,r)[\texttt{attr}]) \]
where agg is an aggregation function, such as $\mathtt{AVG}$, $\mathtt{SUM}$, and $\mathtt{PCT}$, and \(NN_D(q,r)[\texttt{attr}]\) denotes the bag of values of attribute \texttt{attr} of all FRNN results of \(q\) within radius \(r\). 
% \end{definition}

An AQNN expresses aggregation operations to capture key insights about the neighborhood of a query object. For example, \(\mathtt{AVG}\) can be used to reflect the average heart rate or systolic blood pressure of patients in the neighborhood, providing a measure of typical health conditions. \(\mathtt{SUM}\) is useful for assessing cumulative effects, such as the total cost of treatments in the neighborhood that instructs public policy in terms of health. Similarly, $\mathtt{PCT}$ can be used to find the proportion of patients in the neighborhood of a patient of interest, relative to the population in the dataset.
%\laks{Why is finding the total \#meds to NNs or the total treatment cost of everyone in the NN interesting?}

% \texttt{MIN} and \texttt{MAX} are not included in the aggregation functions because they only capture extreme values, which may not represent the typical characteristics of the nearest neighbors and are more sensitive to outliers. 
% \laks{AVG is also sensitive to outliers, but we still allow it. isn't the real reason we don't consider MIN/MAX because they are amenable to estimation via sampling?} We choose \texttt{PCT} instead of \texttt{COUNT} in order to provide a normalized measure that remains comparable across different neighborhood sizes. It allows for more consistent interpretation of relative popularity \cite{moore1989introduction}.


Fig. \ref{fig:framework} illustrates an example of an AQNN: ``\textit{Find the average systolic blood pressure of patients similar to an insomnia patient \(q\)}''. The aggregation function is \(\mathtt{AVG}\) and the target attribute of interest is systolic blood pressure. Exact query evaluation requires consulting physicians (or predicting embeddings by an expensive machine learning model) for all 500 patients in \(D\) and calculate \(q\)'s nearest neighbors wrt \(r\) \cite{DBLP:journals/isci/RodriguesGSBA21}. We refer to such highly accurate but computationally expensive models as \textit{oracle models}, denoted as \(O\), including deep learning models trained on domain-specific data or human expert annotations \cite{DBLP:conf/sigmod/LuCKC18}. Using oracle models is very expensive \cite{sze2017efficient, DujianPQA, DBLP:journals/pvldb/KangGBHZ20}. To address that, we seek an approximate solution by \textit{proxy models}, denoted as \(P\), that are at least one order of magnitude cheaper than oracle models. In the example, if consulting physicians for one patient incurs one cost unit, calling a cheap machine learning model instead incurs at most \(0.1\) cost unit. Once the similar patients are identified, their systolic blood pressure values are averaged and returned as  output. The use of a proxy model may reduce the accuracy of the neighborhood prediction and hence, we should judiciously call oracle and proxy models to minimize the error of aggregate results.

Note that the values of the target attribute \texttt{attr} are \textit{not} predicted but are instead known quantities.

\subsection{Problem Statement}
Given an AQNN, our goal is to return an approximate aggregate result by leveraging both oracle and proxy models while reducing error and cost.


%% ----------------------------------------------------------------------------
% BIWI SA/MA thesis template
%
% Created 09/29/2006 by Andreas Ess
% Extended 13/02/2009 by Jan Lesniak - jlesniak@vision.ee.ethz.ch
% Updated 16/03/2023 by Danda Pani Paudel - paudel@vision.ee.ethz.ch
%% ----------------------------------------------------------------------------

\begin{titlepage}

\thispagestyle{empty}

\fancypagestyle{empty}{
\lhead{\includegraphics[height=1.5cm]{images/ethlogo_black}}
\renewcommand{\headrulewidth}{0.0pt}
\rhead{\vspace*{-0.2cm}\includegraphics[height=1.4cm]{images/cvl_logo}}
\fancyfoot{}
}



\vspace*{2cm}
\begin{center}
\LARGE{\textbf{NPSim: Nighttime Photorealistic Simulation From Daytime Images With Monocular Inverse Rendering and Ray Tracing
}\\}
% NPSim: Nighttime Photorealistic Simulation From Daytime Images With Monocular Inverse Rendering and Ray Tracing
% \LARGE{\textbf{Subtitle Subtitle Subtitle}\\[1cm]}
\vspace{5pt}
\large{Project Thesis\\[0.8cm]}
\LARGE{Shutong Zhang\\}
\normalsize{Department of Information Technology and Electrical Engineering}
\end{center}

\begin{center}
 


% \begin{center}
% \begin{tabular}{ll}
% \multirow{2}{*}{\includegraphics[height=1cm]{images/biwi_logo}} & Computer Vision Laboratory\\ 
% & ETH Zurich
% \end{tabular}
%  \end{center}

\end{center}


\vfill
\begin{center}
\begin{tabular}{ll}
\Large{\textbf Advisor:} & \Large{Dr.~Christos Sakaridis}\\
\Large{\textbf Supervisor:} & \Large{Prof.~Dr.~Luc Van Gool}\\
% 			    & \small{Computer Vision Laboratory}\\
% 			    & \small{Department of Information Technology and Electrical Engineering}\\
\end{tabular}
\end{center}

\begin{center}
% \today\\
August 18, 2023
\end{center}


\end{titlepage}



\begin{document}
\maketitle
\begin{abstract}


The choice of representation for geographic location significantly impacts the accuracy of models for a broad range of geospatial tasks, including fine-grained species classification, population density estimation, and biome classification. Recent works like SatCLIP and GeoCLIP learn such representations by contrastively aligning geolocation with co-located images. While these methods work exceptionally well, in this paper, we posit that the current training strategies fail to fully capture the important visual features. We provide an information theoretic perspective on why the resulting embeddings from these methods discard crucial visual information that is important for many downstream tasks. To solve this problem, we propose a novel retrieval-augmented strategy called RANGE. We build our method on the intuition that the visual features of a location can be estimated by combining the visual features from multiple similar-looking locations. We evaluate our method across a wide variety of tasks. Our results show that RANGE outperforms the existing state-of-the-art models with significant margins in most tasks. We show gains of up to 13.1\% on classification tasks and 0.145 $R^2$ on regression tasks. All our code and models will be made available at: \href{https://github.com/mvrl/RANGE}{https://github.com/mvrl/RANGE}.

\end{abstract}


\section{Introduction}

Video generation has garnered significant attention owing to its transformative potential across a wide range of applications, such media content creation~\citep{polyak2024movie}, advertising~\citep{zhang2024virbo,bacher2021advert}, video games~\citep{yang2024playable,valevski2024diffusion, oasis2024}, and world model simulators~\citep{ha2018world, videoworldsimulators2024, agarwal2025cosmos}. Benefiting from advanced generative algorithms~\citep{goodfellow2014generative, ho2020denoising, liu2023flow, lipman2023flow}, scalable model architectures~\citep{vaswani2017attention, peebles2023scalable}, vast amounts of internet-sourced data~\citep{chen2024panda, nan2024openvid, ju2024miradata}, and ongoing expansion of computing capabilities~\citep{nvidia2022h100, nvidia2023dgxgh200, nvidia2024h200nvl}, remarkable advancements have been achieved in the field of video generation~\citep{ho2022video, ho2022imagen, singer2023makeavideo, blattmann2023align, videoworldsimulators2024, kuaishou2024klingai, yang2024cogvideox, jin2024pyramidal, polyak2024movie, kong2024hunyuanvideo, ji2024prompt}.


In this work, we present \textbf{\ours}, a family of rectified flow~\citep{lipman2023flow, liu2023flow} transformer models designed for joint image and video generation, establishing a pathway toward industry-grade performance. This report centers on four key components: data curation, model architecture design, flow formulation, and training infrastructure optimization—each rigorously refined to meet the demands of high-quality, large-scale video generation.


\begin{figure}[ht]
    \centering
    \begin{subfigure}[b]{0.82\linewidth}
        \centering
        \includegraphics[width=\linewidth]{figures/t2i_1024.pdf}
        \caption{Text-to-Image Samples}\label{fig:main-demo-t2i}
    \end{subfigure}
    \vfill
    \begin{subfigure}[b]{0.82\linewidth}
        \centering
        \includegraphics[width=\linewidth]{figures/t2v_samples.pdf}
        \caption{Text-to-Video Samples}\label{fig:main-demo-t2v}
    \end{subfigure}
\caption{\textbf{Generated samples from \ours.} Key components are highlighted in \textcolor{red}{\textbf{RED}}.}\label{fig:main-demo}
\end{figure}


First, we present a comprehensive data processing pipeline designed to construct large-scale, high-quality image and video-text datasets. The pipeline integrates multiple advanced techniques, including video and image filtering based on aesthetic scores, OCR-driven content analysis, and subjective evaluations, to ensure exceptional visual and contextual quality. Furthermore, we employ multimodal large language models~(MLLMs)~\citep{yuan2025tarsier2} to generate dense and contextually aligned captions, which are subsequently refined using an additional large language model~(LLM)~\citep{yang2024qwen2} to enhance their accuracy, fluency, and descriptive richness. As a result, we have curated a robust training dataset comprising approximately 36M video-text pairs and 160M image-text pairs, which are proven sufficient for training industry-level generative models.

Secondly, we take a pioneering step by applying rectified flow formulation~\citep{lipman2023flow} for joint image and video generation, implemented through the \ours model family, which comprises Transformer architectures with 2B and 8B parameters. At its core, the \ours framework employs a 3D joint image-video variational autoencoder (VAE) to compress image and video inputs into a shared latent space, facilitating unified representation. This shared latent space is coupled with a full-attention~\citep{vaswani2017attention} mechanism, enabling seamless joint training of image and video. This architecture delivers high-quality, coherent outputs across both images and videos, establishing a unified framework for visual generation tasks.


Furthermore, to support the training of \ours at scale, we have developed a robust infrastructure tailored for large-scale model training. Our approach incorporates advanced parallelism strategies~\citep{jacobs2023deepspeed, pytorch_fsdp} to manage memory efficiently during long-context training. Additionally, we employ ByteCheckpoint~\citep{wan2024bytecheckpoint} for high-performance checkpointing and integrate fault-tolerant mechanisms from MegaScale~\citep{jiang2024megascale} to ensure stability and scalability across large GPU clusters. These optimizations enable \ours to handle the computational and data challenges of generative modeling with exceptional efficiency and reliability.


We evaluate \ours on both text-to-image and text-to-video benchmarks to highlight its competitive advantages. For text-to-image generation, \ours-T2I demonstrates strong performance across multiple benchmarks, including T2I-CompBench~\citep{huang2023t2i-compbench}, GenEval~\citep{ghosh2024geneval}, and DPG-Bench~\citep{hu2024ella_dbgbench}, excelling in both visual quality and text-image alignment. In text-to-video benchmarks, \ours-T2V achieves state-of-the-art performance on the UCF-101~\citep{ucf101} zero-shot generation task. Additionally, \ours-T2V attains an impressive score of \textbf{84.85} on VBench~\citep{huang2024vbench}, securing the top position on the leaderboard (as of 2025-01-25) and surpassing several leading commercial text-to-video models. Qualitative results, illustrated in \Cref{fig:main-demo}, further demonstrate the superior quality of the generated media samples. These findings underscore \ours's effectiveness in multi-modal generation and its potential as a high-performing solution for both research and commercial applications.
\section{On the Root Cause of Anisotropic Embeddings}%
\label{sec:theory}

We study the collective shift of the embeddings (that underlies the anisotropy problem), by analyzing their vector updates based on the optimization algorithms SGD and Adam. Weight tying is assumed, but only contributions from the output layer are considered, following \citet{bis2021tmic}. 
Our results apply to all model architectures with a standard language modeling head.

\subsection{Language Modeling Head}

The equations for the standard language modeling head read
\begin{align}
\mathcal{L} &= - \log{(p_t)} \label{eq:forward_loss} \\
    p_t &= \frac{\exp{(l_t)}}{\sum_{j=1}^V \exp{(l_j)}} \label{eq:forward_probability}\\ %
l_i &= e_i \bigcdot h \label{eq:gradient_function_new} \; ,
\end{align}
where $\mathcal{L} \in \mathbb{R}_{\geq 0}$ is the loss for next token prediction, and $p_t \in [0, 1]$ is the predicted probability of the true token $t \in \V$. $l_i \in \mathbb{R}$ and $e_i \in \mathbb{R}^H$ denote the logits and embeddings for each token $i \in \V$, respectively. $h \in \mathbb{R}^H$ is the final hidden state provided by the model for a single token. Note that the operation in Eq.~(\ref{eq:gradient_function_new}) is the dot product of two vectors in $\mathbb{R}^H$.
Backward propagation yields the following gradients with respect to the input vectors $e_i$ and $h$ of Eq.~(\ref{eq:gradient_function_new}):
\begin{align}
g_i :=~ &\frac{\partial \mathcal{L}}{\partial e_i} 
= - \left( \delta_{it} - p_i \right) \cdot h \label{eq:chain_rule_e} 
\end{align}
This result was first reported using a different notation in \citet{bis2021tmic}, and is rederived in App.~\ref{app:chain_rule_e} for the reader's convenience.

\subsection{Vanishing Sum of Embedding Gradients}

Optimization algorithms for neural networks usually update the model parameters iteratively, using an additive update vector that points in direction opposite to the gradient of the loss with respect to the parameters. In the case of embedding vectors, this can be expressed by
\begin{equation}
e_i^{(\tm)} \: = \: e_i^{(\tm-1)} + u_i^{(\tm)}  \; ,
\label{eq:update_general}
\end{equation}
with
\begin{equation}
u_i^{(\tm)} \: \propto \: - g_i^{(\tm)} \; ,
\label{eq:update_vector_definition}
\end{equation}
where $u_i^{(\tm)}$ is the update vector for $e_i^{(\tm)}$ at time step $\tm$.
Eq.~(\ref{eq:chain_rule_e}) implies that the embedding vector $e_t$ of the true token is updated in direction $+h$, while the update vectors $u_i$ for all the other embedding vectors $e_i$ with $i \neq t$ are proportional to $-h$, see Fig.~\ref{fig:gradients_example}. 
\begin{figure}[t]
\centering
\includegraphics[scale=0.5]{figs/toy_example.png}
\caption{Toy example of a hidden state vector $h$ (shown in blue) and three embedding vectors $e_i$ (shown in red) in $H = 2$ dimensions. The gray vectors represent the embedding update vectors, for the SGD (dark) and the Adam (light) optimizer. The update vector of the true token is aligned with $h$, while the others point in the opposite direction, see Eq.~(\ref{eq:chain_rule_e}). Note that the {\em sum of embedding update vectors} vanishes for SGD, while this is not necessarily the case for Adam, cf.~Eqs.~(\ref{eq:vanishing_updates_SGD}) and (\ref{eq:non_vanishing_updates_Adam}).}
\label{fig:gradients_example}
\end{figure}
This circumstance is referred to in the literature as the "common enemy effect" \cite{bis2021tmic}, and regarded as the cause of the representation degeneration problem. 
However, as we will see in the following sections, this explanation is incomplete, as it does not take into account the scaling of the gradients with the predicted probabilities $p_i$, see Eq.~(\ref{eq:chain_rule_e}). The basis for our argumentation is the observation that the {\em sum of embedding gradients vanishes}, as the following simple calculation shows:
\begin{align}
\sum_{i=1}^V g_i^{(\tm)} 
&\stackrel{(\ref{eq:chain_rule_e})}{=} 
- \sum_{i=1}^V  \left( \delta_{it}^{(\tm)} - p_i^{(\tm)} \right) \cdot h^{(\tm)} \nonumber \\
&= - \left( 1 - \sum_{i=1}^V p_i^{(\tm)} \right) \cdot h^{(\tm)} = 0 
\label{eq:optimizer_momentum_conservation}
\end{align}
Next, we will study how Eq.~(\ref{eq:optimizer_momentum_conservation}) translates to the sum $\sum_{i=1}^V u_i^{(\tm)}$
of embedding update vectors, as well as the mean embedding vector
\begin{equation}
\mu^{(\tm)} = \frac{1}{V} \sum_{i=1}^V e_i^{(\tm)}
\label{eq:mu}
\end{equation}
Since the exact definition of the embedding update vector $u_i$, i.e. the proportionality factor in Eq.~(\ref{eq:update_vector_definition}), depends on the optimization algorithm, we discuss SGD and Adam separately.

\subsection{Invariant Mean Embedding with SGD}

We consider the application of the SGD optimization algorithm on the embedding vectors\footnote{Details are given in App.~\ref{app:sgd_algorithm}.}.
At each training step, an embedding vector is simply updated by adding the associated negative gradient $-g_i$, multiplied by a global learning rate $\eta$. Hence, Eq.~(\ref{eq:update_vector_definition}) becomes
\begin{equation}
u_i^{(\tm)} = - \eta \cdot g_i^{(\tm)}
\label{eq:update_vector_definition_SGD}
\end{equation}
Together with Eq.~(\ref{eq:optimizer_momentum_conservation}), this implies that the {\em sum of embedding update vectors vanishes} at any time step $\tm$:
\begin{equation}
\sum_{i=1}^V  u_i^{(\tm)}
\stackrel{(\ref{eq:update_vector_definition_SGD})}{=} - \eta \sum_{i=1}^V g_i^{(\tm)} 
\stackrel{(\ref{eq:optimizer_momentum_conservation})}{=} 0
\label{eq:vanishing_updates_SGD}
\end{equation}
Consequently, the mean embedding vector will stay invariant during the training process:
\begin{equation}
    \mu^{(\tm)} - \mu^{(\tm-1)} 
    \stackrel{(\ref{eq:mu},\ref{eq:update_general})}{=} 
    \frac{1}{V} \sum_{i=1}^V u_i^{(\tm)} 
    \stackrel{(\ref{eq:vanishing_updates_SGD})}{=} 0
    \label{eq:invariant_mean_embedding_SGD}
\end{equation}
This holds even though the different embeddings $e_i$ will be individually updated in different directions with different magnitudes. 
Moreover, all of the above is true also in the case of SGD with momentum,
which follows from linearity and mathematical induction.
Eq.~(\ref{eq:invariant_mean_embedding_SGD}) has far-reaching implications with regard to the anisotropy problem. It entails that the embedding vectors do not collectively shift away from the origin if SGD (with or without momentum) is used. 

\subsection{Shifted Mean Embedding with Adam}

In this section, we analyze the behavior of the mean embedding during optimization with Adam~\cite{adam}, see Algorithm~\ref{alg:algorithm_adam}.
\begin{algorithm}[t]
    \small
    \textbf{Input:}
    $\eta$ (lr), $e_i^{(0)}$ (initial embeddings),
    $\mathcal{L}(e_i)$ (objective), $\beta_1, \beta_2$ (betas), $T$ (number of time steps)
    \\
    \textbf{Initialize:}
    $m_i^{(0)} \leftarrow 0$ (1st moment),
    $v_i^{(0)} \leftarrow 0$ (2nd moment)
    \\
    \textbf{Output}: $e^{(T)}$ (final embeddings)
    \begin{algorithmic}[1]
        \For{$\tm=1 \dots T$}
            \For{$i=1 \dots V$}
                \State $g_i^{(\tm)}$ $\gets$ $\nabla_{e_i} \mathcal{L}^{(\tm)} (e_i^{(\tm-1)})$
                \State $m_i^{(\tm)}$ $\gets$ $\beta_1 m_i^{(\tm-1)} + (1 - \beta_1) g_i^{(\tm)}$%
                \label{alg:line:adam_first_moment_definition}
                \State $v_i^{(\tm)}$ $\gets$ $\beta_2 v_i^{(\tm-1)} + (1-\beta_2) \left(g_i^{(\tm)}\right)^2$%
                \label{alg:line:adam_second_moment_definition}
                \State $\firstmoment$ $\gets$ $m_i^{(\tm)}/\big(1-\beta_1^\tm \big)$%
                \label{alg:line:adam_first_moment_exp_averages}
                \State $\secondmoment$ $\gets$ $v_i^{(\tm)}/\big(1-\beta_2^\tm \big)$%
              \label{alg:line:adam_second_moment_exp_averages}
            \EndFor
            \BeginBox[fill=\colhighlight!10!White, xshift=0.6em, inner xsep=-0.7em]
            \If{\highlight{coupled}}
                \State $\highlight{\secondmomentavg}$ $\gets$ $\highlight{\frac{1}{V} \sum_{i=1}^V \secondmoment}$ %
                \label{alg:line:coupled_adam_second_moment}%
            \EndIf
            \For{$i=1 \dots V$}
                \If{\highlight{coupled}}
                    \State $\highlight{\secondmoment}$ $\gets$ $\highlight{\secondmomentavg}$%
                    \label{alg:line:coupled_adam_second_moment_2}
                \EndIf
            \EndBox
                \State $e_i^{(\tm)}$ $\gets$ $e_i^{(\tm-1)} - \eta \frac{\firstmoment}{\sqrt{\secondmoment} + \epsilon}$%
                \label{alg:line:adam_update}
            \EndFor
        \EndFor
        \State \Return $e^{(T)}$
    \end{algorithmic}
    \caption{Pseudocode for the Adam algorithm and our extension, the \highlight{Coupled Adam algorithm (highlighted)}, applied to the embedding vectors $e_i$. Note that weight decay is not applied.}
    \label{alg:algorithm_adam}
\end{algorithm}

The update vector~Eq.~(\ref{eq:update_vector_definition}) for the Adam algorithm is given by
\begin{equation}
    u_i^{(\tm)} 
    = 
    - \eta^{(\tm)}_i \cdot \firstmoment \label{eq:update_vector_definition_Adam}  \; ,
\end{equation}
where we have introduced an $i$-dependent effective learning rate 
\begin{equation}
    \eta^{(\tm)}_i
    := 
    \frac{\eta}{\sqrt{\secondmoment} + \epsilon} \label{eq:adam_learning_rate} 
\end{equation}
Note that $\firstmoment$ and $\secondmoment$ denote the exponentially averaged first and second moments, respectively, defined according to lines~\ref{alg:line:adam_first_moment_definition}-\ref{alg:line:adam_second_moment_exp_averages}
in Algorithm~\ref{alg:algorithm_adam}.
The $i$-dependent learning rate serves the purpose of individually normalizing the update vectors for different parameters in the Adam optimizer.
However, it also has an unwanted effect specifically on the embedding vectors. While we know from Eq.~\eqref{eq:optimizer_momentum_conservation} and Algorithm~\ref{alg:algorithm_adam} (lines~\ref{alg:line:adam_first_moment_definition},\ref{alg:line:adam_first_moment_exp_averages}) that the {\em unweighted} sum over the first moments vanishes, 
$\sum_{i=1}^V \firstmoment = 0$,
this is not true for the {\em weighted} sum,
\begin{equation}
\sum_{i=1}^V \eta^{(\tm)}_i \firstmoment \neq 0 \; ,
\label{eq:non_vanishing_weighted_sum_of_first_moments_adam}
\end{equation}
unless $\eta^{(\tm)}_i = \eta^{(\tm)}_j$ for all $i, j \in \V$.
Hence, the \textit{sum of embedding update vectors does not vanish} in general,
\begin{equation}
\sum_{i=1}^V  u_i^{(\tm)}
\stackrel{(\ref{eq:update_vector_definition_Adam})}{=} - \sum_{i=1}^V \eta^{(\tm)}_i \cdot \firstmoment 
\stackrel{(\ref{eq:non_vanishing_weighted_sum_of_first_moments_adam})}{\neq} 0
\label{eq:non_vanishing_updates_Adam}
\end{equation}
This, in turn, causes the mean embedding to change during training,
\begin{equation} 
    \mu^{(\tm)} - \mu^{(\tm-1)} 
    \stackrel{(\ref{eq:mu},\ref{eq:update_general})}{=} 
    \frac{1}{V} \sum_{i=1}^V u_i^{(\tm)}
    \stackrel{(\ref{eq:non_vanishing_updates_Adam})}{\neq} 0 \; ,
    \label{eq:mean_embedding_change_adam}
\end{equation}
which is in stark contrast to the case of SGD (cf.~Eq.~\eqref{eq:invariant_mean_embedding_SGD}).
We have thus identified that an $i$-dependency of the second moment $\secondmoment$ of the Adam optimizer leads to the observed collective shift of the embedding vectors away from the origin.
Next, we will show that the second moment indeed depends on $i$. More concretely, we will argue that its expectation value is proportional to the unigram probabilitity\footnote{Note that from here until Eq.~(\ref{eq:optimizer_update_second_moment_avg_canonical}), the time index ($\tau$) is dropped for the sake of readability.} (see Eq.~(\ref{eq:unigram_probability})),
\begin{equation}
    \E \left[ \secondmomentshort \right] \propto \widetilde p_i 
    \label{eq:second_moment_linear_in_unigram_prob}
\end{equation}
In App.~\ref{app:second_moment_theory}, Eq.~(\ref{eq:second_moment_linear_in_unigram_prob}) is derived using minimal assumptions and experimental input.
Here, we restrict ourselves to confirming the relationship in a purely experimental manner. 
$\E \left[ \secondmomentshort \right]$ is estimated directly by measuring $\secondmomentshort$ multiple times during training, using different models. 
We then perform linear fits of $\E \left[ \secondmomentshort \right]$ as a function of $\widetilde p_i$. 
Indeed, the fits yield a high coefficient of determination, on average $R^2 = 0.85(7)$, and a proportionality constant of 
\begin{equation}
A := \frac{\E \left[ \secondmomentshort \right]}{\widetilde p_i} \approx 10^{-4}
\label{eq:second_moment_proportionality_constant}
\end{equation}
Details about the exact procedure and plots showing the data and linear fits can be found in App.~\ref{app:second_moment_empirical}.

\section{Coupled Adam}
\label{sec:lmwap}

In the previous section, we have identified the individual scales of the second moments $v_i$ for different embedding vectors $e_i$ as the root cause of the anisotropy problem. This implies that a solution to the problem is to enforce that the second moments are the same for every $i$.
The question arises whether and how this can be done in the best way, without harming the performance of the model.
To answer this, we note that the normalization of the embedding update vector by the Adam second moment can be split into two parts:
\begin{equation}
\E \left[ \secondmomentshort \right] 
\stackrel{(\ref{eq:second_moment_proportionality_constant})}{=} A \cdot \widetilde p_i
= \frac{A}{V} \cdot \left( \widetilde p_i V \right)
\label{eq:second_moment_factorization}
\end{equation}
The first factor introduces a global scale to all update vectors simultaneously:
\begin{equation}
    \frac{A}{V} \stackrel{(\ref{eq:second_moment_proportionality_constant})}{\approx} \frac{10^{-4}}{5 \cdot 10^{4}} = 2 \cdot 10^{-9} \; ,
\label{eq:second_moment_global_factor}
\end{equation}
where the numbers correspond to our experiments from the previous section with $V \approx 50000$.
The second factor scales the update vectors individually. It is one on average:
\begin{equation}
\frac{1}{V} \sum_{i=1}^{V} \left( \widetilde p_i V \right) = 1
\label{eq:second_moment_individual_factor}
\end{equation}
Our goal is to retain the first, global factor and get rid of the second, individual factor. 
The canonical way to do this is to simply take the average of the second moment over the vocabulary items $i$:
\begin{equation}
\frac{1}{V} \sum_{i=1}^{V} \E \left[ \secondmomentshort \right]
\stackrel{(\ref{eq:second_moment_factorization}, \ref{eq:second_moment_individual_factor})}{=} \frac{A}{V}
\label{eq:optimizer_update_second_moment_avg_canonical}
\end{equation}
In practice, the exponentially averaged second moments $\secondmoment$ as they appear in Eq.~(\ref{eq:adam_learning_rate}) are replaced by their average:
\begin{align}
\secondmomentavg \: &:= \: \frac{1}{V} \sum_{i=1}^V \secondmoment
\label{eq:optimizer_update_second_moment_avg}
\end{align}
We call the resulting algorithm \textit{Coupled Adam}, as it couples the second moments of the embedding vectors via Eq.~(\ref{eq:optimizer_update_second_moment_avg}). 
It is displayed in Algorithm~\ref{alg:algorithm_adam}.
Evidently, with Coupled Adam, the effective learning rate in Eq.~(\ref{eq:adam_learning_rate}) that enters the update vector in Eq.~(\ref{eq:update_vector_definition_Adam}) becomes independent of $i$. Hence, like SGD but unlike standard Adam, the sum of embedding updates vanishes.
However, like standard Adam but unlike SGD, Coupled Adam uses a second moment to normalize the embedding update vectors. 

\section{Experiments}

\subsection{Setups}
\subsubsection{Implementation Details}
We apply our FDS method to two types of 3DGS: 
the original 3DGS, and 2DGS~\citep{huang20242d}. 
%
The number of iterations in our optimization 
process is 35,000.
We follow the default training configuration 
and apply our FDS method after 15,000 iterations,
then we add normal consistency loss for both
3DGS and 2DGS after 25000 iterations.
%
The weight for FDS, $\lambda_{fds}$, is set to 0.015,
the $\sigma$ is set to 23,
and the weight for normal consistency is set to 0.15
for all experiments. 
We removed the depth distortion loss in 2DGS 
because we found that it degrades its results in indoor scenes.
%
The Gaussian point cloud is initialized using Colmap
for all datasets.
%
%
We tested the impact of 
using Sea Raft~\citep{wang2025sea} and 
Raft\citep{teed2020raft} on FDS performance.
%
Due to the blurriness of the ScanNet dataset, 
additional prior constraints are required.
Thus, we incorporate normal prior supervision 
on the rendered normals 
in ScanNet (V2) dataset by default.
The normal prior is predicted by the Stable Normal 
model~\citep{ye2024stablenormal}
across all types of 3DGS.
%
The entire framework is implemented in 
PyTorch~\citep{paszke2019pytorch}, 
and all experiments are conducted on 
a single NVIDIA 4090D GPU.

\begin{figure}[t] \centering
    \makebox[0.16\textwidth]{\scriptsize Input}
    \makebox[0.16\textwidth]{\scriptsize 3DGS}
    \makebox[0.16\textwidth]{\scriptsize 2DGS}
    \makebox[0.16\textwidth]{\scriptsize 3DGS + FDS}
    \makebox[0.16\textwidth]{\scriptsize 2DGS + FDS}
    \makebox[0.16\textwidth]{\scriptsize GT (Depth)}

    \includegraphics[width=0.16\textwidth]{figure/fig3_img/compare3/gt_rgb/frame_00522.jpg}
    \includegraphics[width=0.16\textwidth]{figure/fig3_img/compare3/3DGS/frame_00522.jpg}
    \includegraphics[width=0.16\textwidth]{figure/fig3_img/compare3/2DGS/frame_00522.jpg}
    \includegraphics[width=0.16\textwidth]{figure/fig3_img/compare3/3DGS+FDS/frame_00522.jpg}
    \includegraphics[width=0.16\textwidth]{figure/fig3_img/compare3/2DGS+FDS/frame_00522.jpg}
    \includegraphics[width=0.16\textwidth]{figure/fig3_img/compare3/gt_depth/frame_00522.jpg} \\

    % \includegraphics[width=0.16\textwidth]{figure/fig3_img/compare1/gt_rgb/frame_00137.jpg}
    % \includegraphics[width=0.16\textwidth]{figure/fig3_img/compare1/3DGS/frame_00137.jpg}
    % \includegraphics[width=0.16\textwidth]{figure/fig3_img/compare1/2DGS/frame_00137.jpg}
    % \includegraphics[width=0.16\textwidth]{figure/fig3_img/compare1/3DGS+FDS/frame_00137.jpg}
    % \includegraphics[width=0.16\textwidth]{figure/fig3_img/compare1/2DGS+FDS/frame_00137.jpg}
    % \includegraphics[width=0.16\textwidth]{figure/fig3_img/compare1/gt_depth/frame_00137.jpg} \\

     \includegraphics[width=0.16\textwidth]{figure/fig3_img/compare2/gt_rgb/frame_00262.jpg}
    \includegraphics[width=0.16\textwidth]{figure/fig3_img/compare2/3DGS/frame_00262.jpg}
    \includegraphics[width=0.16\textwidth]{figure/fig3_img/compare2/2DGS/frame_00262.jpg}
    \includegraphics[width=0.16\textwidth]{figure/fig3_img/compare2/3DGS+FDS/frame_00262.jpg}
    \includegraphics[width=0.16\textwidth]{figure/fig3_img/compare2/2DGS+FDS/frame_00262.jpg}
    \includegraphics[width=0.16\textwidth]{figure/fig3_img/compare2/gt_depth/frame_00262.jpg} \\

    \includegraphics[width=0.16\textwidth]{figure/fig3_img/compare4/gt_rgb/frame00000.png}
    \includegraphics[width=0.16\textwidth]{figure/fig3_img/compare4/3DGS/frame00000.png}
    \includegraphics[width=0.16\textwidth]{figure/fig3_img/compare4/2DGS/frame00000.png}
    \includegraphics[width=0.16\textwidth]{figure/fig3_img/compare4/3DGS+FDS/frame00000.png}
    \includegraphics[width=0.16\textwidth]{figure/fig3_img/compare4/2DGS+FDS/frame00000.png}
    \includegraphics[width=0.16\textwidth]{figure/fig3_img/compare4/gt_depth/frame00000.png} \\

    \includegraphics[width=0.16\textwidth]{figure/fig3_img/compare5/gt_rgb/frame00080.png}
    \includegraphics[width=0.16\textwidth]{figure/fig3_img/compare5/3DGS/frame00080.png}
    \includegraphics[width=0.16\textwidth]{figure/fig3_img/compare5/2DGS/frame00080.png}
    \includegraphics[width=0.16\textwidth]{figure/fig3_img/compare5/3DGS+FDS/frame00080.png}
    \includegraphics[width=0.16\textwidth]{figure/fig3_img/compare5/2DGS+FDS/frame00080.png}
    \includegraphics[width=0.16\textwidth]{figure/fig3_img/compare5/gt_depth/frame00080.png} \\



    \caption{\textbf{Comparison of depth reconstruction on Mushroom and ScanNet datasets.} The original
    3DGS or 2DGS model equipped with FDS can remove unwanted floaters and reconstruct
    geometry more preciously.}
    \label{fig:compare}
\end{figure}


\subsubsection{Datasets and Metrics}

We evaluate our method for 3D reconstruction 
and novel view synthesis tasks on
\textbf{Mushroom}~\citep{ren2024mushroom},
\textbf{ScanNet (v2)}~\citep{dai2017scannet}, and 
\textbf{Replica}~\citep{replica19arxiv}
datasets,
which feature challenging indoor scenes with both 
sparse and dense image sampling.
%
The Mushroom dataset is an indoor dataset 
with sparse image sampling and two distinct 
camera trajectories. 
%
We train our model on the training split of 
the long capture sequence and evaluate 
novel view synthesis on the test split 
of the long capture sequences.
%
Five scenes are selected to evaluate our FDS, 
including "coffee room", "honka", "kokko", 
"sauna", and "vr room". 
%
ScanNet(V2)~\citep{dai2017scannet}  consists of 1,613 indoor scenes
with annotated camera poses and depth maps. 
%
We select 5 scenes from the ScanNet (V2) dataset, 
uniformly sampling one-tenth of the views,
following the approach in ~\citep{guo2022manhattan}.
To further improve the geometry rendering quality of 3DGS, 
%
Replica~\citep{replica19arxiv} contains small-scale 
real-world indoor scans. 
We evaluate our FDS on five scenes from 
Replica: office0, office1, office2, office3 and office4,
selecting one-tenth of the views for training.
%
The results for Replica are provided in the 
supplementary materials.
To evaluate the rendering quality and geometry 
of 3DGS, we report PSNR, SSIM, and LPIPS for 
rendering quality, along with Absolute Relative Distance 
(Abs Rel) as a depth quality metrics.
%
Additionally, for mesh evaluation, 
we use metrics including Accuracy, Completion, 
Chamfer-L1 distance, Normal Consistency, 
and F-scores.




\subsection{Results}
\subsubsection{Depth rendering and novel view synthesis}
The comparison results on Mushroom and 
ScanNet are presented in \tabref{tab:mushroom} 
and \tabref{tab:scannet}, respectively. 
%
Due to the sparsity of sampling 
in the Mushroom dataset,
challenges are posed for both GOF~\citep{yu2024gaussian} 
and PGSR~\citep{chen2024pgsr}, 
leading to their relative poor performance 
on the Mushroom dataset.
%
Our approach achieves the best performance 
with the FDS method applied during the training process.
The FDS significantly enhances the 
geometric quality of 3DGS on the Mushroom dataset, 
improving the "abs rel" metric by more than 50\%.
%
We found that Sea Raft~\citep{wang2025sea}
outperforms Raft~\citep{teed2020raft} on FDS, 
indicating that a better optical flow model 
can lead to more significant improvements.
%
Additionally, the render quality of RGB 
images shows a slight improvement, 
by 0.58 in 3DGS and 0.50 in 2DGS, 
benefiting from the incorporation of cross-view consistency in FDS. 
%
On the Mushroom
dataset, adding the FDS loss increases 
the training time by half an hour, which maintains the same
level as baseline.
%
Similarly, our method shows a notable improvement on the ScanNet dataset as well using Sea Raft~\citep{wang2025sea} Model. The "abs rel" metric in 2DGS is improved nearly 50\%. This demonstrates the robustness and effectiveness of the FDS method across different datasets.
%


% \begin{wraptable}{r}{0.6\linewidth} \centering
% \caption{\textbf{Ablation study on geometry priors.}} 
%         \label{tab:analysis_prior}
%         \resizebox{\textwidth}{!}{
\begin{tabular}{c| c c c c c | c c c c}

    \hline
     Method &  Acc$\downarrow$ & Comp $\downarrow$ & C-L1 $\downarrow$ & NC $\uparrow$ & F-Score $\uparrow$ &  Abs Rel $\downarrow$ &  PSNR $\uparrow$  & SSIM  $\uparrow$ & LPIPS $\downarrow$ \\ \hline
    2DGS&   0.1078&  0.0850&  0.0964&  0.7835&  0.5170&  0.1002&  23.56&  0.8166& 0.2730\\
    2DGS+Depth&   0.0862&  0.0702&  0.0782&  0.8153&  0.5965&  0.0672&  23.92&  0.8227& 0.2619 \\
    2DGS+MVDepth&   0.2065&  0.0917&  0.1491&  0.7832&  0.3178&  0.0792&  23.74&  0.8193& 0.2692 \\
    2DGS+Normal&   0.0939&  0.0637&  0.0788&  \textbf{0.8359}&  0.5782&  0.0768&  23.78&  0.8197& 0.2676 \\
    2DGS+FDS &  \textbf{0.0615} & \textbf{ 0.0534}& \textbf{0.0574}& 0.8151& \textbf{0.6974}&  \textbf{0.0561}&  \textbf{24.06}&  \textbf{0.8271}&\textbf{0.2610} \\ \hline
    2DGS+Depth+FDS &  0.0561 &  0.0519& 0.0540& 0.8295& 0.7282&  0.0454&  \textbf{24.22}& \textbf{0.8291}&\textbf{0.2570} \\
    2DGS+Normal+FDS &  \textbf{0.0529} & \textbf{ 0.0450}& \textbf{0.0490}& \textbf{0.8477}& \textbf{0.7430}&  \textbf{0.0443}&  24.10&  0.8283& 0.2590 \\
    2DGS+Depth+Normal &  0.0695 & 0.0513& 0.0604& 0.8540&0.6723&  0.0523&  24.09&  0.8264&0.2575\\ \hline
    2DGS+Depth+Normal+FDS &  \textbf{0.0506} & \textbf{0.0423}& \textbf{0.0464}& \textbf{0.8598}&\textbf{0.7613}&  \textbf{0.0403}&  \textbf{24.22}& 
    \textbf{0.8300}&\textbf{0.0403}\\
    
\bottomrule
\end{tabular}
}
% \end{wraptable}



The qualitative comparisons on the Mushroom and ScanNet dataset 
are illustrated in \figref{fig:compare}. 
%
%
As seen in the first row of \figref{fig:compare}, 
both the original 3DGS and 2DGS suffer from overfitting, 
leading to corrupted geometry generation. 
%
Our FDS effectively mitigates this issue by 
supervising the matching relationship between 
the input and sampled views, 
helping to recover the geometry.
%
FDS also improves the refinement of geometric details, 
as shown in other rows. 
By incorporating the matching prior through FDS, 
the quality of the rendered depth is significantly improved.
%

\begin{table}[t] \centering
\begin{minipage}[t]{0.96\linewidth}
        \captionof{table}{\textbf{3D Reconstruction 
        and novel view synthesis results on Mushroom dataset. * 
        Represents that FDS uses the Raft model.
        }}
        \label{tab:mushroom}
        \resizebox{\textwidth}{!}{
\begin{tabular}{c| c c c c c | c c c c c}
    \hline
     Method &  Acc$\downarrow$ & Comp $\downarrow$ & C-L1 $\downarrow$ & NC $\uparrow$ & F-Score $\uparrow$ &  Abs Rel $\downarrow$ &  PSNR $\uparrow$  & SSIM  $\uparrow$ & LPIPS $\downarrow$ & Time  $\downarrow$ \\ \hline

    % DN-splatter &   &  &  &  &  &  &  &  & \\
    GOF &  0.1812 & 0.1093 & 0.1453 & 0.6292 & 0.3665 & 0.2380  & 21.37  &  0.7762  & 0.3132  & $\approx$1.4h\\ 
    PGSR &  0.0971 & 0.1420 & 0.1196 & 0.7193 & 0.5105 & 0.1723  & 22.13  & 0.7773  & 0.2918  & $\approx$1.2h \\ \hline
    3DGS &   0.1167 &  0.1033&  0.1100&  0.7954&  0.3739&  0.1214&  24.18&  0.8392& 0.2511 &$\approx$0.8h \\
    3DGS + FDS$^*$ & 0.0569  & 0.0676 & 0.0623 & 0.8105 & 0.6573 & 0.0603 & 24.72  & 0.8489 & 0.2379 &$\approx$1.3h \\
    3DGS + FDS & \textbf{0.0527}  & \textbf{0.0565} & \textbf{0.0546} & \textbf{0.8178} & \textbf{0.6958} & \textbf{0.0568} & \textbf{24.76}  & \textbf{0.8486} & \textbf{0.2381} &$\approx$1.3h \\ \hline
    2DGS&   0.1078&  0.0850&  0.0964&  0.7835&  0.5170&  0.1002&  23.56&  0.8166& 0.2730 &$\approx$0.8h\\
    2DGS + FDS$^*$ &  0.0689 &  0.0646& 0.0667& 0.8042& 0.6582& 0.0589& 23.98&  0.8255&0.2621 &$\approx$1.3h\\
    2DGS + FDS &  \textbf{0.0615} & \textbf{ 0.0534}& \textbf{0.0574}& \textbf{0.8151}& \textbf{0.6974}&  \textbf{0.0561}&  \textbf{24.06}&  \textbf{0.8271}&\textbf{0.2610} &$\approx$1.3h \\ \hline
\end{tabular}
}
\end{minipage}\hfill
\end{table}

\begin{table}[t] \centering
\begin{minipage}[t]{0.96\linewidth}
        \captionof{table}{\textbf{3D Reconstruction 
        and novel view synthesis results on ScanNet dataset.}}
        \label{tab:scannet}
        \resizebox{\textwidth}{!}{
\begin{tabular}{c| c c c c c | c c c c }
    \hline
     Method &  Acc $\downarrow$ & Comp $\downarrow$ & C-L1 $\downarrow$ & NC $\uparrow$ & F-Score $\uparrow$ &  Abs Rel $\downarrow$ &  PSNR $\uparrow$  & SSIM  $\uparrow$ & LPIPS $\downarrow$ \\ \hline
    GOF & 1.8671  & 0.0805 & 0.9738 & 0.5622 & 0.2526 & 0.1597  & 21.55  & 0.7575  & 0.3881 \\
    PGSR &  0.2928 & 0.5103 & 0.4015 & 0.5567 & 0.1926 & 0.1661  & 21.71 & 0.7699  & 0.3899 \\ \hline

    3DGS &  0.4867 & 0.1211 & 0.3039 & 0.7342& 0.3059 & 0.1227 & 22.19& 0.7837 & 0.3907\\
    3DGS + FDS &  \textbf{0.2458} & \textbf{0.0787} & \textbf{0.1622} & \textbf{0.7831} & 
    \textbf{0.4482} & \textbf{0.0573} & \textbf{22.83} & \textbf{0.7911} & \textbf{0.3826} \\ \hline
    2DGS &  0.2658 & 0.0845 & 0.1752 & 0.7504& 0.4464 & 0.0831 & 22.59& 0.7881 & 0.3854\\
    2DGS + FDS &  \textbf{0.1457} & \textbf{0.0679} & \textbf{0.1068} & \textbf{0.7883} & 
    \textbf{0.5459} & \textbf{0.0432} & \textbf{22.91} & \textbf{0.7928} & \textbf{0.3800} \\ \hline
\end{tabular}
}
\end{minipage}\hfill
\end{table}


\begin{table}[t] \centering
\begin{minipage}[t]{0.96\linewidth}
        \captionof{table}{\textbf{Ablation study on geometry priors.}}
        \label{tab:analysis_prior}
        \resizebox{\textwidth}{!}{
\begin{tabular}{c| c c c c c | c c c c}

    \hline
     Method &  Acc$\downarrow$ & Comp $\downarrow$ & C-L1 $\downarrow$ & NC $\uparrow$ & F-Score $\uparrow$ &  Abs Rel $\downarrow$ &  PSNR $\uparrow$  & SSIM  $\uparrow$ & LPIPS $\downarrow$ \\ \hline
    2DGS&   0.1078&  0.0850&  0.0964&  0.7835&  0.5170&  0.1002&  23.56&  0.8166& 0.2730\\
    2DGS+Depth&   0.0862&  0.0702&  0.0782&  0.8153&  0.5965&  0.0672&  23.92&  0.8227& 0.2619 \\
    2DGS+MVDepth&   0.2065&  0.0917&  0.1491&  0.7832&  0.3178&  0.0792&  23.74&  0.8193& 0.2692 \\
    2DGS+Normal&   0.0939&  0.0637&  0.0788&  \textbf{0.8359}&  0.5782&  0.0768&  23.78&  0.8197& 0.2676 \\
    2DGS+FDS &  \textbf{0.0615} & \textbf{ 0.0534}& \textbf{0.0574}& 0.8151& \textbf{0.6974}&  \textbf{0.0561}&  \textbf{24.06}&  \textbf{0.8271}&\textbf{0.2610} \\ \hline
    2DGS+Depth+FDS &  0.0561 &  0.0519& 0.0540& 0.8295& 0.7282&  0.0454&  \textbf{24.22}& \textbf{0.8291}&\textbf{0.2570} \\
    2DGS+Normal+FDS &  \textbf{0.0529} & \textbf{ 0.0450}& \textbf{0.0490}& \textbf{0.8477}& \textbf{0.7430}&  \textbf{0.0443}&  24.10&  0.8283& 0.2590 \\
    2DGS+Depth+Normal &  0.0695 & 0.0513& 0.0604& 0.8540&0.6723&  0.0523&  24.09&  0.8264&0.2575\\ \hline
    2DGS+Depth+Normal+FDS &  \textbf{0.0506} & \textbf{0.0423}& \textbf{0.0464}& \textbf{0.8598}&\textbf{0.7613}&  \textbf{0.0403}&  \textbf{24.22}& 
    \textbf{0.8300}&\textbf{0.0403}\\
    
\bottomrule
\end{tabular}
}
\end{minipage}\hfill
\end{table}




\subsubsection{Mesh extraction}
To further demonstrate the improvement in geometry quality, 
we applied methods used in ~\citep{turkulainen2024dnsplatter} 
to extract meshes from the input views of optimized 3DGS. 
The comparison results are presented  
in \tabref{tab:mushroom}. 
With the integration of FDS, the mesh quality is significantly enhanced compared to the baseline, featuring fewer floaters and more well-defined shapes.
 %
% Following the incorporation of FDS, the reconstruction 
% results exhibit fewer floaters and more well-defined 
% shapes in the meshes. 
% Visualized comparisons
% are provided in the supplementary material.

% \begin{figure}[t] \centering
%     \makebox[0.19\textwidth]{\scriptsize GT}
%     \makebox[0.19\textwidth]{\scriptsize 3DGS}
%     \makebox[0.19\textwidth]{\scriptsize 3DGS+FDS}
%     \makebox[0.19\textwidth]{\scriptsize 2DGS}
%     \makebox[0.19\textwidth]{\scriptsize 2DGS+FDS} \\

%     \includegraphics[width=0.19\textwidth]{figure/fig4_img/compare1/gt02.png}
%     \includegraphics[width=0.19\textwidth]{figure/fig4_img/compare1/baseline06.png}
%     \includegraphics[width=0.19\textwidth]{figure/fig4_img/compare1/baseline_fds05.png}
%     \includegraphics[width=0.19\textwidth]{figure/fig4_img/compare1/2dgs04.png}
%     \includegraphics[width=0.19\textwidth]{figure/fig4_img/compare1/2dgs_fds03.png} \\

%     \includegraphics[width=0.19\textwidth]{figure/fig4_img/compare2/gt00.png}
%     \includegraphics[width=0.19\textwidth]{figure/fig4_img/compare2/baseline02.png}
%     \includegraphics[width=0.19\textwidth]{figure/fig4_img/compare2/baseline_fds01.png}
%     \includegraphics[width=0.19\textwidth]{figure/fig4_img/compare2/2dgs04.png}
%     \includegraphics[width=0.19\textwidth]{figure/fig4_img/compare2/2dgs_fds03.png} \\
      
%     \includegraphics[width=0.19\textwidth]{figure/fig4_img/compare3/gt05.png}
%     \includegraphics[width=0.19\textwidth]{figure/fig4_img/compare3/3dgs03.png}
%     \includegraphics[width=0.19\textwidth]{figure/fig4_img/compare3/3dgs_fds04.png}
%     \includegraphics[width=0.19\textwidth]{figure/fig4_img/compare3/2dgs02.png}
%     \includegraphics[width=0.19\textwidth]{figure/fig4_img/compare3/2dgs_fds01.png} \\

%     \caption{\textbf{Qualitative comparison of extracted mesh 
%     on Mushroom and ScanNet datasets.}}
%     \label{fig:mesh}
% \end{figure}












\subsection{Ablation study}


\textbf{Ablation study on geometry priors:} 
To highlight the advantage of incorporating matching priors, 
we incorporated various types of priors generated by different 
models into 2DGS. These include a monocular depth estimation
model (Depth Anything v2)~\citep{yang2024depth}, a two-view depth estimation 
model (Unimatch)~\citep{xu2023unifying}, 
and a monocular normal estimation model (DSINE)~\citep{bae2024rethinking}.
We adapt the scale and shift-invariant loss in Midas~\citep{birkl2023midas} for
monocular depth supervision and L1 loss for two-view depth supervison.
%
We use Sea Raft~\citep{wang2025sea} as our default optical flow model.
%
The comparison results on Mushroom dataset 
are shown in ~\tabref{tab:analysis_prior}.
We observe that the normal prior provides accurate shape information, 
enhancing the geometric quality of the radiance field. 
%
% In contrast, the monocular depth prior slightly increases 
% the 'Abs Rel' due to its ambiguous scale and inaccurate depth ordering.
% Moreover, the performance of monocular depth estimation 
% in the sauna scene is particularly poor, 
% primarily due to the presence of numerous reflective 
% surfaces and textureless walls, which limits the accuracy of monocular depth estimation.
%
The multi-view depth prior, hindered by the limited feature overlap 
between input views, fails to offer reliable geometric 
information. We test average "Abs Rel" of multi-view depth prior
, and the result is 0.19, which performs worse than the "Abs Rel" results 
rendered by original 2DGS.
From the results, it can be seen that depth order information provided by monocular depth improves
reconstruction accuracy. Meanwhile, our FDS achieves the best performance among all the priors, 
and by integrating all
three components, we obtained the optimal results.
%
%
\begin{figure}[t] \centering
    \makebox[0.16\textwidth]{\scriptsize RF (16000 iters)}
    \makebox[0.16\textwidth]{\scriptsize RF* (20000 iters)}
    \makebox[0.16\textwidth]{\scriptsize RF (20000 iters)  }
    \makebox[0.16\textwidth]{\scriptsize PF (16000 iters)}
    \makebox[0.16\textwidth]{\scriptsize PF (20000 iters)}


    % \includegraphics[width=0.16\textwidth]{figure/fig5_img/compare1/16000.png}
    % \includegraphics[width=0.16\textwidth]{figure/fig5_img/compare1/20000_wo_flow_loss.png}
    % \includegraphics[width=0.16\textwidth]{figure/fig5_img/compare1/20000.png}
    % \includegraphics[width=0.16\textwidth]{figure/fig5_img/compare1/16000_prior.png}
    % \includegraphics[width=0.16\textwidth]{figure/fig5_img/compare1/20000_prior.png}\\

    % \includegraphics[width=0.16\textwidth]{figure/fig5_img/compare2/16000.png}
    % \includegraphics[width=0.16\textwidth]{figure/fig5_img/compare2/20000_wo_flow_loss.png}
    % \includegraphics[width=0.16\textwidth]{figure/fig5_img/compare2/20000.png}
    % \includegraphics[width=0.16\textwidth]{figure/fig5_img/compare2/16000_prior.png}
    % \includegraphics[width=0.16\textwidth]{figure/fig5_img/compare2/20000_prior.png}\\

    \includegraphics[width=0.16\textwidth]{figure/fig5_img/compare3/16000.png}
    \includegraphics[width=0.16\textwidth]{figure/fig5_img/compare3/20000_wo_flow_loss.png}
    \includegraphics[width=0.16\textwidth]{figure/fig5_img/compare3/20000.png}
    \includegraphics[width=0.16\textwidth]{figure/fig5_img/compare3/16000_prior.png}
    \includegraphics[width=0.16\textwidth]{figure/fig5_img/compare3/20000_prior.png}\\
    
    \includegraphics[width=0.16\textwidth]{figure/fig5_img/compare4/16000.png}
    \includegraphics[width=0.16\textwidth]{figure/fig5_img/compare4/20000_wo_flow_loss.png}
    \includegraphics[width=0.16\textwidth]{figure/fig5_img/compare4/20000.png}
    \includegraphics[width=0.16\textwidth]{figure/fig5_img/compare4/16000_prior.png}
    \includegraphics[width=0.16\textwidth]{figure/fig5_img/compare4/20000_prior.png}\\

    \includegraphics[width=0.30\textwidth]{figure/fig5_img/bar.png}

    \caption{\textbf{The error map of Radiance Flow and Prior Flow.} RF: Radiance Flow, PF: Prior Flow, * means that there is no FDS loss supervision during optimization.}
    \label{fig:error_map}
\end{figure}




\textbf{Ablation study on FDS: }
In this section, we present the design of our FDS 
method through an ablation study on the 
Mushroom dataset to validate its effectiveness.
%
The optional configurations of FDS are outlined in ~\tabref{tab:ablation_fds}.
Our base model is the 2DGS equipped with FDS,
and its results are shown 
in the first row. The goal of this analysis 
is to evaluate the impact 
of various strategies on FDS sampling and loss design.
%
We observe that when we 
replace $I_i$ in \eqref{equ:mflow} with $C_i$, 
as shown in the second row, the geometric quality 
of 2DGS deteriorates. Using $I_i$ instead of $C_i$ 
help us to remove the floaters in $\bm{C^s}$, which are also 
remained in $\bm{C^i}$.
We also experiment with modifying the FDS loss. For example, 
in the third row, we use the neighbor 
input view as the sampling view, and replace the 
render result of neighbor view with ground truth image of its input view.
%
Due to the significant movement between images, the Prior Flow fails to accurately 
match the pixel between them, leading to a further degradation in geometric quality.
%
Finally, we attempt to fix the sampling view 
and found that this severely damaged the geometric quality, 
indicating that random sampling is essential for the stability 
of the mean error in the Prior flow.



\begin{table}[t] \centering

\begin{minipage}[t]{1.0\linewidth}
        \captionof{table}{\textbf{Ablation study on FDS strategies.}}
        \label{tab:ablation_fds}
        \resizebox{\textwidth}{!}{
\begin{tabular}{c|c|c|c|c|c|c|c}
    \hline
    \multicolumn{2}{c|}{$\mathcal{M}_{\theta}(X, \bm{C^s})$} & \multicolumn{3}{c|}{Loss} & \multicolumn{3}{c}{Metric}  \\
    \hline
    $X=C^i$ & $X=I^i$  & Input view & Sampled view     & Fixed Sampled view        & Abs Rel $\downarrow$ & F-score $\uparrow$ & NC $\uparrow$ \\
    \hline
    & \ding{51} &     &\ding{51}    &    &    \textbf{0.0561}        &  \textbf{0.6974}         & \textbf{0.8151}\\
    \hline
     \ding{51} &           &     &\ding{51}    &    &    0.0839        &  0.6242         &0.8030\\
     &  \ding{51} &   \ding{51}  &    &    &    0.0877       & 0.6091        & 0.7614 \\
      &  \ding{51} &    &    & \ding{51}    &    0.0724           & 0.6312          & 0.8015 \\
\bottomrule
\end{tabular}
}
\end{minipage}
\end{table}




\begin{figure}[htbp] \centering
    \makebox[0.22\textwidth]{}
    \makebox[0.22\textwidth]{}
    \makebox[0.22\textwidth]{}
    \makebox[0.22\textwidth]{}
    \\

    \includegraphics[width=0.22\textwidth]{figure/fig6_img/l1/rgb/frame00096.png}
    \includegraphics[width=0.22\textwidth]{figure/fig6_img/l1/render_rgb/frame00096.png}
    \includegraphics[width=0.22\textwidth]{figure/fig6_img/l1/render_depth/frame00096.png}
    \includegraphics[width=0.22\textwidth]{figure/fig6_img/l1/depth/frame00096.png}

    % \includegraphics[width=0.22\textwidth]{figure/fig6_img/l2/rgb/frame00112.png}
    % \includegraphics[width=0.22\textwidth]{figure/fig6_img/l2/render_rgb/frame00112.png}
    % \includegraphics[width=0.22\textwidth]{figure/fig6_img/l2/render_depth/frame00112.png}
    % \includegraphics[width=0.22\textwidth]{figure/fig6_img/l2/depth/frame00112.png}

    \caption{\textbf{Limitation of FDS.} }
    \label{fig:limitation}
\end{figure}


% \begin{figure}[t] \centering
%     \makebox[0.48\textwidth]{}
%     \makebox[0.48\textwidth]{}
%     \\
%     \includegraphics[width=0.48\textwidth]{figure/loss_Ignatius.pdf}
%     \includegraphics[width=0.48\textwidth]{figure/loss_family.pdf}
%     \caption{\textbf{Comparison the photometric error of Radiance Flow and Prior Flow:} 
%     We add FDS method after 2k iteration during training.
%     The results show
%     that:  1) The Prior Flow is more precise and 
%     robust than Radiance Flow during the radiance 
%     optimization; 2) After adding the FDS loss 
%     which utilize Prior 
%     flow to supervise the Radiance Flow at 2k iterations, 
%     both flow are more accurate, which lead to
%     a mutually reinforcing effects.(TODO fix it)} 
%     \label{fig:flowcompare}
% \end{figure}






\textbf{Interpretive Experiments: }
To demonstrate the mutual refinement of two flows in our FDS, 
For each view, we sample the unobserved 
views multiple times to compute the mean error 
of both Radiance Flow and Prior Flow. 
We use Raft~\citep{teed2020raft} as our default optical flow model
for visualization.
The ground truth flow is calculated based on 
~\eref{equ:flow_pose} and ~\eref{equ:flow} 
utilizing ground truth depth in dataset.
We introduce our FDS loss after 16000 iterations during 
optimization of 2DGS.
The error maps are shown in ~\figref{fig:error_map}.
Our analysis reveals that Radiance Flow tends to 
exhibit significant geometric errors, 
whereas Prior Flow can more accurately estimate the geometry,
effectively disregarding errors introduced by floating Gaussian points. 

%





\subsection{Limitation and further work}

Firstly, our FDS faces challenges in scenes with 
significant lighting variations between different 
views, as shown in the lamp of first row in ~\figref{fig:limitation}. 
%
Incorporating exposure compensation into FDS could help address this issue. 
%
 Additionally, our method struggles with 
 reflective surfaces and motion blur,
 leading to incorrect matching. 
 %
 In the future, we plan to explore the potential 
 of FDS in monocular video reconstruction tasks, 
 using only a single input image at each time step.
 


\section{Conclusions}
In this paper, we propose Flow Distillation Sampling (FDS), which
leverages the matching prior between input views and 
sampled unobserved views from the pretrained optical flow model, to improve the geometry quality
of Gaussian radiance field. 
Our method can be applied to different approaches (3DGS and 2DGS) to enhance the geometric rendering quality of the corresponding neural radiance fields.
We apply our method to the 3DGS-based framework, 
and the geometry is enhanced on the Mushroom, ScanNet, and Replica datasets.

\section*{Acknowledgements} This work was supported by 
National Key R\&D Program of China (2023YFB3209702), 
the National Natural Science Foundation of 
China (62441204, 62472213), and Gusu 
Innovation \& Entrepreneurship Leading Talents Program (ZXL2024361)
% \begin{table}[!t]
% \centering
% \scalebox{0.68}{
%     \begin{tabular}{ll cccc}
%       \toprule
%       & \multicolumn{4}{c}{\textbf{Intellipro Dataset}}\\
%       & \multicolumn{2}{c}{Rank Resume} & \multicolumn{2}{c}{Rank Job} \\
%       \cmidrule(lr){2-3} \cmidrule(lr){4-5} 
%       \textbf{Method}
%       &  Recall@100 & nDCG@100 & Recall@10 & nDCG@10 \\
%       \midrule
%       \confitold{}
%       & 71.28 &34.79 &76.50 &52.57 
%       \\
%       \cmidrule{2-5}
%       \confitsimple{}
%     & 82.53 &48.17
%        & 85.58 &64.91
     
%        \\
%        +\RunnerUpMiningShort{}
%     &85.43 &50.99 &91.38 &71.34 
%       \\
%       +\HyReShort
%         &- & -
%        &-&-\\
       
%       \bottomrule

%     \end{tabular}
%   }
% \caption{Ablation studies using Jina-v2-base as the encoder. ``\confitsimple{}'' refers using a simplified encoder architecture. \framework{} trains \confitsimple{} with \RunnerUpMiningShort{} and \HyReShort{}.}
% \label{tbl:ablation}
% \end{table}
\begin{table*}[!t]
\centering
\scalebox{0.75}{
    \begin{tabular}{l cccc cccc}
      \toprule
      & \multicolumn{4}{c}{\textbf{Recruiting Dataset}}
      & \multicolumn{4}{c}{\textbf{AliYun Dataset}}\\
      & \multicolumn{2}{c}{Rank Resume} & \multicolumn{2}{c}{Rank Job} 
      & \multicolumn{2}{c}{Rank Resume} & \multicolumn{2}{c}{Rank Job}\\
      \cmidrule(lr){2-3} \cmidrule(lr){4-5} 
      \cmidrule(lr){6-7} \cmidrule(lr){8-9} 
      \textbf{Method}
      & Recall@100 & nDCG@100 & Recall@10 & nDCG@10
      & Recall@100 & nDCG@100 & Recall@10 & nDCG@10\\
      \midrule
      \confitold{}
      & 71.28 & 34.79 & 76.50 & 52.57 
      & 87.81 & 65.06 & 72.39 & 56.12
      \\
      \cmidrule{2-9}
      \confitsimple{}
      & 82.53 & 48.17 & 85.58 & 64.91
      & 94.90&78.40 & 78.70& 65.45
       \\
      +\HyReShort{}
       &85.28 & 49.50
       &90.25 & 70.22
       & 96.62&81.99 & \textbf{81.16}& 67.63
       \\
      +\RunnerUpMiningShort{}
       % & 85.14& 49.82
       % &90.75&72.51
       & \textbf{86.13}&\textbf{51.90} & \textbf{94.25}&\textbf{73.32}
       & \textbf{97.07}&\textbf{83.11} & 80.49& \textbf{68.02}
       \\
   %     +\RunnerUpMiningShort{}
   %    & 85.43 & 50.99 & 91.38 & 71.34 
   %    & 96.24 & 82.95 & 80.12 & 66.96
   %    \\
   %    +\HyReShort{} old
   %     &85.28 & 49.50
   %     &90.25 & 70.22
   %     & 96.62&81.99 & 81.16& 67.63
   %     \\
   % +\HyReShort{} 
   %     % & 85.14& 49.82
   %     % &90.75&72.51
   %     & 86.83&51.77 &92.00 &72.04
   %     & 97.07&83.11 & 80.49& 68.02
   %     \\
      \bottomrule

    \end{tabular}
  }
\caption{\framework{} ablation studies. ``\confitsimple{}'' refers using a simplified encoder architecture. \framework{} trains \confitsimple{} with \RunnerUpMiningShort{} and \HyReShort{}. We use Jina-v2-base as the encoder due to its better performance.
}
\label{tbl:ablation}
\end{table*}

\section{Results}
\label{sec:results}

In this section, we present detailed results demonstrating \emph{CellFlow}'s state-of-the-art performance in cellular morphology prediction under perturbations, outperforming existing methods across multiple datasets and evaluation metrics.

\subsection{Datasets}

Our experiments were conducted using three cell imaging perturbation datasets: BBBC021 (chemical perturbation)~\cite{caie2010high}, RxRx1 (genetic perturbation)~\cite{sypetkowski2023rxrx1}, and the JUMP dataset (combined perturbation)~\cite{chandrasekaran2023jump}. We followed the preprocessing protocol from IMPA~\cite{palma2023predicting}, which involves correcting illumination, cropping images centered on nuclei to a resolution of 96×96, and filtering out low-quality images. The resulting datasets include 98K, 171K, and 424K images with 3, 5, and 6 channels, respectively, from 26, 1,042, and 747 perturbation types. Examples of these images are provided in Figure~\ref{fig:comparison}. Details of datasets are provided in \S\ref{sec:data}.

\subsection{Experimental Setup}

\textbf{Evaluation metrics.} We evaluate methods using two types of metrics: (1) FID and KID, which measure image distribution similarity via Fréchet and kernel-based distances, computed on 5K generated images for BBBC021 and 100 randomly selected perturbation classes for RxRx1 and JUMP; we report both overall scores across all samples and conditional scores per perturbation class. (2) Mode of Action (MoA) classification accuracy, which assesses biological fidelity by using a trained classifier to predict a drug’s effect from perturbed images and comparing it to its known MoA from the literature.

\textbf{Baselines.} We compare our approach against two baselines, PhenDiff~\cite{bourou2024phendiff} and IMPA~\cite{palma2023predicting}, the only two baselines that incorporate control images into their model design --- a crucial setup for distinguishing true perturbation effects from artifacts such as batch effects. PhenDiff uses diffusion models to first map control images to noise and then transform the noise into target images. In contrast, IMPA employs GANs with an AdaIN layer to transfer the style of control images to target images, specifically designed for paired image-to-image mappings. Our method uses flow matching, which is tailored for distribution-to-distribution mapping, providing a more suitable solution for our problem. We reproduce these baselines with official codes.

\textbf{Training details.} \emph{CellFlow} employs a UNet-based velocity field with a four-stage design. Perturbations are encoded following IMPA~\cite{palma2023predicting}. Training is conducted for 100 epochs on 4 A100 GPUs. Details are in \S\ref{sec:experimental}.


\begin{table*}[ht!]
\centering
\footnotesize
\renewcommand{\arraystretch}{1.2}
\begin{tabular}{p{2.15cm} p{14.05cm}}
\toprule
 & \includegraphics[height=2.6cm]{figure/qualitative_example.pdf}\\
\midrule
Original \phantom{00000} Description & Bromazepam binds to the GABA receptor GABA-A, causing a conformational change and \textbf{increasing inhibitory effects of GABA}. Other neurotransmitters are not influenced. One of the \textbf{benzodiazepines} that is \textbf{used in the treatment of anxiety disorders.} \\
\midrule
User & Explain the features of the given molecule that are most distinct from those of other molecules. \\
\midrule
Mol-LLaMA (Ours) & The given molecule is a member of the \textcolor{tabblue}{benzodiazepine class}, which is characterized by a fusion of a benzene ring and a seven-membered diazepine ring. This structure is crucial for its biological activity, particularly its role as \textcolor{tabblue}{a sedative and hypnotic agent}. ...$<$omitted$>$...\\

& The given molecule is also characterized by a ketone group within the diazepine ring, which is a common feature in benzodiazepines. \textcolor{tabblue}{This functional group is important for the interaction with the GABA-A receptor, enhancing the inhibitory effects of GABA}. ...$<$omitted$>$...\\

\midrule
GPT-4o & 1. **Quinazoline Core**: The molecule contains a \textcolor{tabred}{quinazoline core}, ...$<$omitted$>$... \\
 & Overall, the combination of a \textcolor{tabred}{quinazoline} structure with unique substitution patterns, including bromine halogenation and the presence of a pyridinyl group, ...$<$omitted$>$... . \\
\midrule
LLaMo & The molecule has a \textcolor{tabblue}{benzodiazepine} structure with a bromo substituent at the 7-position and \textcolor{tabred}{a methyl group at the 1-position}. It is distinct from other molecules in that it contains a bromine atom, a nitrogen atom, and a methyl group, which are not present in the other molecules. \\
\midrule
3D-MoLM & ...$<$omitted$>$... It belongs to the class of compounds called \textcolor{tabred}{quinazolines}, which are characterized by a bicyclic structure consisting of a benzene ring fused to a pyrimidine ring. ...$<$omitted$>$...\\
& The molecule's structure suggests potential applications in medicinal chemistry, as quinazolines have been found to possess various biological activities, including \textcolor{tabred}{antitumor, antimicrobial, and anti-inflammatory properties.} \\
\midrule
Mol-Instructions & The molecule is a \textcolor{tabred}{quinoxaline derivative}. \\
\bottomrule
\end{tabular}
\vspace{-0.1in}
\caption{\small Case study to compare molecular understanding and reasoning ability. Mol-LLaMA accurately understands the molecular features, answering a correct molecular taxonomy and providing its distinct properties that are relevant to the given molecule.}
\label{tab:qualitative}
\vspace{-0.1in}
\end{table*}

\subsection{Main Results}

\textbf{\emph{CellFlow} generates highly realistic cell images.}  
\emph{CellFlow} outperforms existing methods in capturing cellular morphology across all datasets (Table~\ref{tab:results}a), achieving overall FID scores of 18.7, 33.0, and 9.0 on BBBC021, RxRx1, and JUMP, respectively --- improving FID by 21\%–45\% compared to previous methods. These gains in both FID and KID metrics confirm that \emph{CellFlow} produces significantly more realistic cell images than prior approaches.

\textbf{\emph{CellFlow} accurately captures perturbation-specific morphological changes.}  
As shown in Table~\ref{tab:results}a, \emph{CellFlow} achieves conditional FID scores of 56.8 (a 26\% improvement), 163.5, and 84.4 (a 16\% improvement) on BBBC021, RxRx1, and JUMP, respectively. These scores are computed by measuring the distribution distance for each specific perturbation and averaging across all perturbations.   
Table~\ref{tab:results}b further highlights \emph{CellFlow}’s performance on six representative chemical and three genetic perturbations. For chemical perturbations, \emph{CellFlow} reduces FID scores by 14–55\% compared to prior methods.
The smaller improvement (5–12\% improvements) on RxRx1 is likely due to the limited number of images per perturbation type.

\textbf{\emph{CellFlow} preserves biological fidelity across perturbation conditions.} 
Table~\ref{tab:ablation}a presents mode of action (MoA) classification accuracy on the BBBC021 dataset using generated cell images. MoA describes how a drug affects cellular function and can be inferred from morphology. To assess this, we train an image classifier on real perturbed images and test it on generated ones. \emph{CellFlow} achieves 71.1\% MoA accuracy, closely matching real images (72.4\%) and significantly surpassing other methods (best: 63.7\%), demonstrating its ability to maintain biological fidelity across perturbations. Qualitative comparisons in Figure~\ref{fig:comparison} further highlight \emph{CellFlow}’s accuracy in capturing key biological effects. For example, demecolcine produces smaller, fragmented nuclei, which other methods fail to reproduce accurately.

\textbf{\emph{CellFlow} generalizes to out-of-distribution (OOD) perturbations.}  
On BBBC021, \emph{CellFlow} demonstrates strong generalization to novel chemical perturbations never seen during training (Table~\ref{tab:ablation}b). It achieves 6\% and 28\% improvements in overall and conditional FID over the best baseline. This OOD generalization is critical for biological research, enabling the exploration of previously untested interventions and the design of new drugs.

\textbf{Ablations highlight the importance of each component in \emph{CellFlow}.}  
Table~\ref{tab:ablation}c shows that removing conditional information, classifier-free guidance, or noise augmentation significantly degrades performance, leading to higher FID scores. These underscore the critical role of each component in enabling \emph{CellFlow}’s state-of-the-art performance.  

\begin{figure*}[!tb]
    \centering
     \includegraphics[width=\linewidth]{imgs/interpolation.pdf}
     \vspace{-2em}
    \caption{
    \textbf{\emph{CellFlow} enables new capabilities.} 
\textit{(a.1) Batch effect calibration.}  
\emph{CellFlow} initializes with control images, enabling batch-specific predictions. Comparing predictions from different batches highlights actual perturbation effects (smaller cell size) while filtering out spurious batch effects (cell density variations).  
\textit{(a.2) Interpolation trajectory.}  
\emph{CellFlow}'s learned velocity field supports interpolation between cell states, which might provide insights into the dynamic cell trajectory. 
\textit{(b) Diffusion model comparison.}  
Unlike flow matching, diffusion models that start from noise cannot calibrate batch effects or support interpolation.  
\textit{(c) Reverse trajectory.}  
\emph{CellFlow}'s reversible velocity field can predict prior cell states from perturbed images, offering potential applications such as restoring damaged cells.
    }
    \label{fig:interpolation}
    \vspace{-1em}
\end{figure*}

\subsection{New Capabilities}

\textbf{\emph{CellFlow} addresses batch effects and reveals true perturbation effects.}  
\emph{CellFlow}’s distribution-to-distribution approach effectively addresses batch effects, a significant challenge in biological experimental data collection. As shown in Figure~\ref{fig:interpolation}a, when conditioned on two distinct control images with varying cell densities from different batches, \emph{CellFlow} consistently generates the expected perturbation effect (cell shrinkage due to mevinolin) while recapitulating batch-specific artifacts, revealing the true perturbation effect. Table~\ref{tab:ablation}d further quantifies the importance of conditioning on the same batch. By comparing generated images conditioned on control images from the same or different batches against the target perturbation images, we find that same-batch conditioning reduces overall and conditional FID by 21\%. This highlights the importance of modeling control images to more accurately capture true perturbation effects—an aspect often overlooked by prior approaches, such as diffusion models that initialize from noise (Figure~\ref{fig:interpolation}b).

\textbf{\emph{CellFlow} has the potential to model cellular morphological change trajectories.}
Cell trajectories could offer valuable information about perturbation mechanisms, but capturing them with current imaging technologies remains challenging due to their destructive nature. Since \emph{CellFlow} continuously transforms the source distribution into the target distribution, it can generate smooth interpolation paths between initial and final predicted cell states, producing video-like sequences of cellular transformation based on given source images (Figure~\ref{fig:interpolation}a). This suggests a possible approach for simulating morphological trajectories during perturbation response, which diffusion methods cannot achieve (Figure~\ref{fig:interpolation}b). Additionally, the reversible distribution transformation learned through flow matching enables \emph{CellFlow} to model backward cell state reversion (Figure~\ref{fig:interpolation}c), which could be useful for studying recovery dynamics and predicting potential treatment outcomes.

\CatchFileDef{\resultsAblationsScaleSmall}{tables/results_ablations_scale_small.tex}{}
\CatchFileDef{\resultsAblationsSGDAll}{tables/results_ablations_sgd_all.tex}{}

\section{Ablations}
\label{sec:ablations}

We perform some additional experiments to shed further light on how Coupled Adam works. A model size of $N = 125\M$ and the dataset sizes $D \in \{ 5\B, 10\B, 20\B \}$ from the small-scale experiments (Sec.~\ref{sec:experiments_S}) are used, and each experiment is repeated $S = 3$ times with different seeds. 

\subsection{Scaled Coupled Adam}
\label{sec:ablation_different_learning_rate}

While coupling the second moment of the embedding gradients using the average in Eq.~(\ref{eq:optimizer_update_second_moment_avg}) is the canonical choice, one could also use a multiple of the average. We conduct additional experiments where the coupled second moment is scaled by powers of $2$:
\begin{equation}
\secondmomentavg \: \to \: 2^{-n} \cdot \secondmomentavg  \; ,
\label{eq:optimizer_update_second_moment_avg_scaled}
\end{equation}
with scaling exponents
$n \in \{ z \in \mathbb{Z} ~| -5 \leq z \leq 5 \}$.
Note that using a scaling exponent $n \neq 0$ is equivalent to using a different effective learning rate for the embeddings than for all the other parameters, via Eqs.~(\ref{eq:optimizer_update_second_moment_avg}) and (\ref{eq:adam_learning_rate}). In particular, a smaller scaling exponent $n$ corresponds to a smaller effective learning rate and vice versa. 
The results for $D=20\B$ are shown in Tab.~\ref{tab:results_ablations_scale}, 
and the dependency of the loss on the scaling exponent $n$ for that very dataset size is visualized in Fig.~\ref{fig:ablation_scale}.
\begin{figure*}[t]
\begin{minipage}{\textwidth}
  \begin{minipage}[b]{0.63\textwidth}
    \centering
    \scriptsize
    \begin{tabular}{c|rrrrrrrr}
    \toprule
    $n$ & $\Loss$ ($\downarrow$) & $\Acc$ ($\uparrow$) & $\ISO$ ($\uparrow$) & $\munorm$ ($\downarrow$) & $\munormrel$ ($\downarrow$) & $\rcos$ ($\uparrow$) & $\rho$ ($\uparrow$) & $\kappa$ ($\uparrow$) \\ 
    \midrule
    \resultsAblationsScaleSmall
    \bottomrule 
    \end{tabular}
    \captionof{table}{Results of our experiments with Scaled Coupled Adam, for $N=125\M$ and $D=20\B$. Values are highlighted in bold if they are significantly better than \textit{all} the other values in the same column, see the caption of Tab.~\ref{tab:results_S} for more details.}
    \label{tab:results_ablations_scale}
  \end{minipage}
  \hfill
  \begin{minipage}[b]{0.35\textwidth}
    \centering
    \includegraphics[scale=0.40]{figs/ablation_1_loss_20B.png}
    \captionof{figure}{Dependency of the loss on the scaling exponent $n$, see Eq.~(\ref{eq:optimizer_update_second_moment_avg_scaled}), for $N=125\M$ and $D=20\B$. The plot shows the difference to the loss obtained for $n = 0$.}
    \label{fig:ablation_scale}
  \end{minipage}
\end{minipage}
\end{figure*}
Results for other dataset sizes and plots for the other evaluation metrics can be found in App.~\ref{app:additional_results_scale}.
Our data shows that the loss reaches a minimum close to $n=0$, with a rather weak dependence on the scaling exponent in its vicinity. Nevertheless, for the smallest and largest scaling exponents studied, we find that the loss gets significantly worse.
Regarding downstream performance, we see indications of a similar pattern, although the statistical uncertainties are too large to draw definite conclusions.  
The semantic usefulness of the embedding vectors as measured by $\rcos$ seems to suffer from a scaling exponent $n < 0$. For the isotropy and the mean embedding, we observe the opposite behavior. They benefit from a smaller scaling exponent $n$ and the associated smaller embedding updates, with the effect being more pronounced the larger the training dataset size $D$.  
However, this also negatively affects the model performance. 
Hence, we conclude that, at least within the range of our experiments, the optimal setting is to have the same learning rate for the embedding parameters as for all the other model parameters, as implied by $n=0$ and Eq.~(\ref{eq:optimizer_update_second_moment_avg}).

\subsection{SGD}
\label{sec:ablation_sgd}

We train several models using SGD with momentum $\gamma = 0.9$ as the optimizer for the embeddings. Since Adam via the inverse square root of its second moment effectively scales the learning rate up by a factor comprising orders of magnitude (see Eq.~(\ref{eq:second_moment_global_factor})), we explicitly multiply the learning rate in SGD by a factor $f$ of comparable size\footnote{Note that the difference between momentum in SGD and the first moment in Adam also plays a role here.}. A hyperparameter search using 
$f \in \{ 100, 200, 300, 400, 500, 600 \}$
is performed to search for the optimum with respect to upstream performance (loss), see App.~\ref{app:additional_results_sgd} for details. It is found at $f = 300$ for $D \in \{ 5\B, 10\B \}$ and $f = 400$ for $D = 20\B$.
The respective optimal model is compared to its counterpart trained with Coupled Adam in Tab.~\ref{tab:results_ablations_sgd_all}.
\begin{table*}[htb]
\centering
\scriptsize
\begin{tabular}{ccc|rrrrrrrr}
\toprule
$D$ & $N$ & Optimizer & $\Loss$ ($\downarrow$) & $\Acc$ ($\uparrow$) & $\ISO$ ($\uparrow$) & $\munorm$ ($\downarrow$) & $\munormrel$ ($\downarrow$) & $\rcos$ ($\uparrow$) & $\rho$ ($\uparrow$) & $\kappa$ ($\uparrow$) \\ 
\midrule
\resultsAblationsSGDAll
\bottomrule 
\end{tabular}
\caption{Comparison of models whose embeddings were trained with SGD and Coupled Adam. The SGD models were obtained after hyperparameter search for the learning rate. The associated factor $f$ is specified in parentheses in the Optimizer column. Bold values indicate better results with statistical significance, see the caption of Tab.~\ref{tab:results_S} for more details.}
\label{tab:results_ablations_sgd_all}
\end{table*}
The results show that, although SGD is advantageous with respect to isotropy, the mean embedding shift and the condition number, Coupled Adam consistently achieves better results on all upstream and downstream task metrics, while having one less hyperparameter to fine-tune.

\section{Related Work}
\label{sec:related_work}

\citet{gao2019representationdegenerationproblemtraining}
first described the anisotropy issue, which they referred to as \textit{representation degeneration problem}, and suggested cosine regularization as a mitigation strategy.
Alternative techniques to address the problem have been developed, including adversarial noise \cite{pmlr-v97-wang19f}, spectrum control \cite{Wang2020ImprovingNL} and Laplacian regularization \cite{zhang-etal-2020-revisiting}.  
\citet{bis2021tmic} have shown that the anisotropy of embeddings can for the most part be traced back to a common shift of the embeddings in a dominant direction. They called this phenomenon \textit{common enemy effect}, and provided a semi-quantitative explanation (Eq.~(\ref{eq:chain_rule_e})), which we developed further in the present work by including the optimizer in the analysis.
In \citet{yu-etal-2022-rare}, Adaptive Gradient Gating is proposed, based on the empirical observation that it is the gradients for embeddings of rare tokens that cause anisotropy. Our analysis conforms to this finding and attributes it to a massive up-scaling of the gradients for rare embeddings with Adam, cf.~Fig.~\ref{fig:gradients_example}.
\citet{machina-mercer-2024-anisotropy} have demonstrated that large Pythia models \cite{pmlr-v202-biderman23a} show improved isotropy compared to similar models, and attribute this to the absence of weight tying. This is in accordance with our analysis of the unembedding gradients in conjunction with Adam, Sec.~\ref{sec:theory}.
While all the previously mentioned papers use average cosine similarity \cite{ethayarajh-2019-contextual} or $\ISO$ from Eq.~(\ref{eq:isotropy}) to quantify the geometry of embedding vectors, \citet{rudman-etal-2022-isoscore} deviate from this. 
Their notion of isotropy is based solely on the embeddings' covariance matrix and embodied by the metric IsoScore. 
In particular, IsoScore is mean-agnostic, while $\ISO$ strongly correlates with the mean embedding (see e.g. Tab.~\ref{tab:results_S}).
Finally, concurrent to our work, \citet{zhao2024deconstructingmakesgoodoptimizer} have investigated the importance of using the second moment in Adam with regard to performance and stability. They found that simplified variants of Adam that use the same effective learning rate either for the whole embedding matrix (Adalayer) or each embedding vector (Adalayer*) are slightly worse than Adam but better than SGD. Adalayer* is similar to Coupled Adam, but corresponds to the second moment averaged over hidden space instead of vocabulary space.

\section{Conclusion}

In this work, we present a general framework for in-bed human shape estimation with pressure images, bridging from pseudo-label generation to algorithm design. For label generation, we present SMPLify-IB, a low-cost monocular optimization approach to generate SMPL p-GTs for in-bed scenes. By introducing gravity constraints and a lightweight but efficient self-penetration detection module, we regenerate higher-quality SMPL labels for a public dataset TIP. For model design, we introduce PI-HMR, a pressure-based HPS network to predict in-bed motions from pressure sequences. By fusing pressure distribution and spatial priors, accompanied with KD and TTO exploration, PI-HMR outperforms previous methods. Results verify the feasibility of enhancing model's performance by exploiting pressure's nature.


% This work would provide a whole tool-chain and baseline to support the development of in-bed HPS with pressure images and other modalities.


% In this paper, we propose PIMesh, a temporal human shape estimation network to directly generate human meshes from input pressure image sequences that are collected by a pressure-sensing bedsheet. Moreover, to overcome the dataset bottleneck for the human shape estimation task in in-bed scenarios, we present TIP, the first-of-its-kind multi-modal temporal human in-bed pose dataset with multiple human representation labels including posture, 2D joints, and 3D meshes. TIP contains 156K synchronized temporal images from 9 volunteers in three modalities~(RGB, depth, and pressure images). To generate reliable 3D mesh ground truths, we leverage state-of-the-art RGB-based human shape estimators and propose a 3D mesh label generation pipeline for in-bed scenarios. By deploying a SMPLify-based optimizer with strong human-scene penetration penalty terms, our optimized 3D shape ground truths present 25mm joint prediction errors compared with the manually annotated 2D joint ground truths. Finally, PIMesh achieves 79.17mm of MPJPE, 91.01mm of MPVE, and a minimum of 5.8mm/s$^2$ acceleration errors on the TIP dataset. This work provides a privacy-preserving approach to estimating in-bed human meshes in non-line-of-sight environments and demonstrates more potential application scenarios of pressure-sensing bedsheets
\section*{Limitations}
Our study focuses on entity unlearning, leaving hazardous knowledge and copyrighted content unlearning unexplored. These cases may require different evaluation strategies.

Additionally, our experiments use mid-sized models (LLaMA-2-7B-Chat, LLaMA-3-8B-Instruct). Larger models, with their computational demands and structural differences, may respond differently. Future research should assess their applicability to such models.


\begin{acks}
This work was supported in part by Advanced Micro Devices, Inc. under the ``Funding Academic Research" and the AMD AI \& HPC Cluster Program.
\end{acks}

\bibliography{custom}

\appendix
\section{Unigram Probability Distribution}
\label{app:unigram_probability_example}

Fig.~\ref{fig:prior_example_gpt2_openwebtext} shows the unigram probability distribution for the example of the OpenWebText Corpus dataset and the GPT-2 tokenizer.
\begin{figure}[H]
    \centering
    \includegraphics[scale=0.5]{figs/global_probability_distribution_openwebtext_log.png}
    \caption{Logarithm $\log(\widetilde{p_i})$ of the unigram probability distribution for the OpenWebText Corpus and the GPT-2 tokenizer. The maximum probability is $\max_{i} \widetilde{p_i} \approx 0.037$ or $\max_{i} \log(\widetilde{p_i}) \approx -3.30$. The minimum probability (not shown) is $\min_{i} \widetilde{p_i} = 0$ or $\min_{i} \log(\widetilde{p_i}) = -\infty$.}
    \label{fig:prior_example_gpt2_openwebtext}
\end{figure}

\pagebreak
\section{Embedding Gradients}
\label{app:chain_rule_e}

We explicitly derive Eq.~(\ref{eq:chain_rule_e}), which we recall here for convenience:
\begin{align}
g_i :=~ &\frac{\partial \mathcal{L}}{\partial e_i} 
= - \left( \delta_{it} - p_i \right) \cdot h \tag{\ref{eq:chain_rule_e}}
\end{align}
The chain rule yields
\begin{align}
\frac{\partial \mathcal{L}}{\partial e_i} 
&= \sum_{k=1}^{V} \frac{\partial \mathcal{L}}{\partial p_t} 
\cdot \frac{\partial p_t}{\partial l_k}  
\cdot \frac{\partial l_k}{\partial e_i} \; ,
\label{eq:chain_rule_basis}
\end{align}
where the individual factors can directly be obtained from Eqs.~(\ref{eq:forward_loss})-(\ref{eq:gradient_function_new}):
\begin{align}
\frac{\partial \mathcal{L}}{\partial p_t} &= - \frac{1}{p_t} \label{eq:backward_loss} \\
\frac{\partial p_t}{\partial l_k} 
&= \frac{\delta_{kt} \exp{(l_t)} \cdot \Sigma - \exp{(l_t)} \exp{(l_k)}}{\Sigma^2} \nonumber \\ 
&= \delta_{kt} p_t - p_t p_k \nonumber \\
&= p_t ( \delta_{kt} - p_k ) 
\label{eq:backward_loss_2} \\
\frac{\partial l_k}{\partial e_i} &= \delta_{ki} h \label{eq:backward_e}
\end{align}
Note that in the first line of Eq.~(\ref{eq:backward_loss_2}), we use the abbreviation $\Sigma = \Big( \sum_{j=1}^V \exp{(l_j)} \Big)$.
Inserting Eqs.~(\ref{eq:backward_loss}), (\ref{eq:backward_loss_2}) and (\ref{eq:backward_e}) into Eq.~(\ref{eq:chain_rule_basis}) directly leads to Eq.~(\ref{eq:chain_rule_e}):
\begin{align}
\frac{\partial \mathcal{L}}{\partial e_i} 
&= - \sum_{k=1}^{V} \frac{1}{p_t}
\cdot p_t ( \delta_{kt} - p_k )  
\cdot \delta_{ki} h \nonumber \\
&= - \sum_{k=1}^{V} ( \delta_{kt} - p_k )  
\cdot \delta_{ki} h \nonumber \\
&= - ( \delta_{it} - p_i )  
\cdot h \nonumber
\end{align}

\section{SGD Algorithm}
\label{app:sgd_algorithm}

For completeness and comparison to (Coupled) Adam as displayed in Algorithm~\ref{alg:algorithm_adam}, we summarize the SGD algorithm in Algorithm~\ref{alg:algorithm_sgd}.

\begin{algorithm}[H]
    \small
    \textbf{Input:}
    $\eta$ (lr), $e_i^{(0)}$ (initial embeddings),
    $\mathcal{L}(e_i)$ (objective), $\gamma$ (momentum), $T$ (number of time steps) \\
    \textbf{Output}: $e^{(T)}$ (final embeddings)
    \begin{algorithmic}[1]
        \For{$\tm=1 \dots T$}
            \For{$i=1 \dots V$}
                \State $g_i^{(\tm)}$ $\gets$ $\nabla_{e_i} \mathcal{L}^{(\tm)} (e_i^{(\tm-1)})$
                \If{$t>1$}
                    \State $\mathbf{b}_i^{(\tm)}$ $\gets$ $\gamma \mathbf{b}_i^{(\tm-1)} + g_i^{(\tm)}$
                \Else
                    \State $\mathbf{b}_i^{(\tm)}$ $\gets$ $g_i^{(\tm)}$
                \EndIf
                \State $e_i^{(\tm)}$ $\gets$ $e_i^{(\tm-1)} - \eta \mathbf{b}_i^{(\tm)}$
            \EndFor
        \EndFor
        \vspace*{1.0ex}
        \State \Return $e^{(T)}$
    \end{algorithmic}
    \caption{Pseudocode for the SGD algorithm with optional momentum, applied to the embedding vectors $e_i$.}
    \label{alg:algorithm_sgd}
\end{algorithm}


\section{Magnitude of the Second Moment in Adam}

In this appendix, the validity of 
\begin{equation}
    \E \left[ \secondmomentshort \right] \propto \widetilde p_i
    \tag{\ref{eq:second_moment_linear_in_unigram_prob}}
\end{equation}
is verified.
Due to the linearity of lines 5 and 7 in Algorithm 2, it suffices to show that the squared gradient has the property in question:
\begin{align}
\E \left[ g_i^2 \right] &\propto \widetilde p_i
\label{eq:squared_gradient_linear_in_unigram_prob}
\end{align}
We do this in two different ways.
First, we derive Eq.~(\ref{eq:squared_gradient_linear_in_unigram_prob}) using a semi-theoretical approach with minimal experimental input.
Afterwards, we confirm the relationship in a purely experimental manner. %

\subsection{Semi-theoretical Derivation}
\label{app:second_moment_theory}

Here, we derive an expression for the expectation value of the squared gradient in terms of simple observables (Theorem~\ref{theorem}). Subsequently, the dependency of those observables on $\widetilde p_i$ is determined experimentally. Together, this will yield the proportionality expressed by Eq.~(\ref{eq:squared_gradient_linear_in_unigram_prob}).
We begin our reasoning with a lemma.

\begin{lemma}[Expectation Value Decomposition]\label{lemma}
The expectation value of the squared gradient can be decomposed into conditional expectation values as follows:
\begin{align}
\E \left[ g_i^2 \right] =~ 
&\widetilde p_i \cdot \E \left[ g_i^2 ~\big|~ i=t \right] \nonumber \\
&+ (1 - \widetilde p_i) \cdot \E \left[ g_i^2 ~\big|~ i \neq t \right]
\label{eq:lemma1}
\end{align}
\end{lemma}

\begin{proof}
Our starting point is the definition of the expectation value for the continuous random variable $g_i^2$:
\begin{align}
\E \left[ g_i^2 \right] = \int g_i^2 ~p(g_i) ~dg_i \; ,
\label{eq:aux_E_definition}
\end{align}
where $p$ denotes the probability distribution of $g_i$. Since the vocabulary item $i$ can only be either the true token $t$ or not, we can decompose $p$ into a sum of joint probability distributions (using the {\em law of total probabilities}), each of which can be expressed in terms of conditional probabilities like so:
\begin{align}
p(g_i) 
&= p(g_i, i=t) + p(g_i, i\neq t) \nonumber \\
&= p(g_i ~|~ i=t) \cdot p(i=t) \nonumber \\
&\quad + p(g_i ~|~ i\neq t) \cdot p(i\neq t)
\end{align}
Using the unigram probability $\widetilde p_i = p(i = t)$, this can also be written as
\begin{align}
p(g_i) 
&= \widetilde p_i \cdot p(g_i ~|~ i=t) \nonumber \\
&\quad + (1 - \widetilde p_i) \cdot p(g_i ~|~ i\neq t)
\label{eq:aux_p_decomposition}
\end{align}
If we insert Eq.~(\ref{eq:aux_p_decomposition}) back into Eq.~(\ref{eq:aux_E_definition}), the expectation value becomes
\begin{align}
\E \left[ g_i^2 \right] 
&= \widetilde p_i \cdot \int g_i^2 ~p(g_i ~|~ i=t) ~dg_i \nonumber \\
&\quad + (1 - \widetilde p_i) \cdot \int g_i^2 ~p(g_i ~|~ i\neq t) ~dg_i \;, 
\label{eq:aux_E_decomposition} %
\end{align}
which by definition of the (conditional) expectation value, Eq.~(\ref{eq:aux_E_definition}), is equivalent to Eq.~(\ref{eq:lemma1}).
\end{proof}
\begin{theorem}[Expectation Value Squared Gradient] \label{theorem}
Given that the squared hidden state vector $h^2$ is independent of $p_i$ and whether $i$ is the true token or not, the expectation value of the squared gradient $g_i^2$ is given by
\begin{align}
\E \left[ g_i^2 \right] 
&= S \cdot \left[ \widetilde p_i \cdot X_i^{(i=t)} + (1 - \widetilde p_i) \cdot X_i^{(i \neq t)} \right] \; ,
\label{eq:theorem}
\end{align}
with
\begin{align}
S &:= \E \left[ h^2 \right] \label{eq:optimizer_second_moment_expansion_S} \\
X_i^{(i=t)} &:= \E \left[ (1 - p_i)^2 ~\big|~ i=t \right]
\label{eq:optimizer_second_moment_expansion_true_2} \\
X_i^{(i \neq t)} &:= \E \left[ p_i^2 ~\big|~ i \neq t \right]
\label{eq:optimizer_second_moment_expansion_false_2}
\end{align}
\end{theorem}

\begin{proof}
We start from Lemma~\ref{lemma} and the square of the gradient,
\begin{align}
g_i^2 
&\stackrel{(\ref{eq:chain_rule_e})}{=} \left( \delta_{it} - p_i \right)^2 h^2 
\label{eq:optimizer_second_moment}
\end{align}
Note that squared variables of vectors in $\mathbb{R}^H$ always denote the elementwise (Hadamard) product, e.g.
\begin{align}
g_i^2 &\equiv g_i \odot g_i \in \mathbb{R}_{\geq 0}^H \; ,
\label{eq:hadamard_product}
\end{align}
with strictly non-negative elements.
Using Eq.~(\ref{eq:optimizer_second_moment}), the expectation values on the right side of Eq.~(\ref{eq:lemma1}) can be expressed as
\begin{align}
\E \left[ g_i^2 ~\big|~ i=t \right]
&= \E \left[ \left( 1 - p_i \right)^2 \cdot h^2 ~\big|~ i=t \right] \\
\E \left[ g_i^2 ~\big|~ i\neq t \right]
&= \E \left[ p_i^2 \cdot h^2 ~\big|~ i\neq t \right]
\end{align}
Given our assumptions regarding $h^2$, its expectation value can be factored out:
\begin{align}
\E \left[ g_i^2 ~\big|~ i=t \right]
&= S \cdot X_i^{(i=t)} \label{eq:Etrue} \\
\E \left[ g_i^2 ~\big|~ i\neq t \right]
&= S \cdot X_i^{(i \neq t)} \label{eq:Efalse} 
\end{align}
Inserting Eqs.~(\ref{eq:Etrue}) and (\ref{eq:Efalse}) into Eq.~(\ref{eq:lemma1}) yields Eq.~(\ref{eq:theorem}).
\end{proof}

Note that Eq.~(\ref{eq:theorem}) is a vector equation, with $\E \left[ g_i^2 \right], S \in \mathbb{R}_{\geq 0}^H$ and $\widetilde p_i, X_i^{(i=t)}, X_i^{(i \neq t)} \in \mathbb{R}_{\geq 0}$.
It states that the expectation value of $g_i^2$ factorizes into a global constant $S$ that is $i$-independent, and a factor that is $i$-dependent. The latter is a specific combination of the unigram probability $\widetilde p_i$, determined by the data, and the conditional expectation values $X_i^{(i=t)}$ and $X_i^{(i\neq t)}$, determined by the model.

\paragraph{Experimental Input}

Regarding the unigram probability, we know that
\begin{enumerate}
\item $\widetilde p_i \ll 1$. \\
This is the case for virtually all natural language datasets with a common vocabulary size of $V > 10000$, according to Zipf's law.
\end{enumerate}
The conditional expectation values $X_i^{(i=t)}$ and $X_i^{(i\neq t)}$ can be empirically estimated by applying training data to different checkpoints. We consider the three small-scale experiments of Sec.~\ref{sec:experiments_S} with $N \in \{ 125\M, 355\M, 760\M \}$ and $D=20\B$, and take ten equidistant checkpoints after $D^\prime \in \{ 2\B, 4\B, \ldots, 20\B\}$ seen tokens for each of them. We then continue pseudo-training on 20 batches ($\approx$ 2k samples or 2M tokens, see Tab.~\ref{tab:model_architecture}) of data using a zero learning rate, and measure the conditional probabilities in Eqs.~(\ref{eq:optimizer_second_moment_expansion_true_2}, \ref{eq:optimizer_second_moment_expansion_false_2}) from which our target quantities can be estimated.  
Subsequently, linear fits of the form 
\begin{align}
    X_i^{(i = t)} &= A^{(i = t)} \cdot \widetilde p_i \\
    X_i^{(i \neq t)} &= A^{(i \neq t)} \cdot \widetilde p_i \; ,
\end{align}
with fit parameters $A^{(i = t)}$ and $A^{(i \neq t)}$ are performed. $R^2$ is used to assess the quality of the fits. In addition, the mutual information $\I$ between the response and the explanatory variable is computed. 
Since we observe only a very weak dependence of the results for $R^2$ and $\I$ on $N$ and $D^\prime$, we specify the mean and standard deviation over all experiments for them.
Our findings are:
\begin{enumerate}
\item[2.] $X_i^{(i = t)}$ is independent of $\widetilde p_i$. \\
The linear fits yield $R^2 = 0.003(1)$, and the mutual information is $\I \left(X_i^{(i = t)}; \widetilde p_i \right) = 0.14(2)$.
\item[3.] $X_i^{(i \neq t)}$ is proportional to $\widetilde p_i$. \\
The linear fits yield $R^2 = 0.92(1)$, and the mutual information is $\I \left( X_i^{(i \neq t)}; \widetilde p_i \right) = 0.50(2)$.
\end{enumerate}

The three empirical results above, together with Theorem~\ref{theorem}, immediately lead to Eq.~(\ref{eq:squared_gradient_linear_in_unigram_prob}).

\subsection{Experimental Confirmation}
\label{app:second_moment_empirical}

We reuse the experiments from the previous section to measure the second moment $\secondmomentshort$ directly, in order to estimate $\E \left[ \secondmomentshort \right]$. Again, linear fits of the form
\begin{align}
    \E \left[ \secondmomentshort \right] = A \cdot \widetilde p_i
\end{align}
are performed and the mutual information is computed. 
We find
\begin{enumerate}
\item[4.] $\E \left[ \secondmomentshort \right]$ is proportional to $\widetilde p_i$. \\
The linear fits yield $R^2 = 0.85(7)$, and the mutual information is $\I \left( \E \left[ \secondmomentshort \right]; \widetilde p_i \right) = 1.18(9)$.
\end{enumerate}
The results for $N=125\M$ and $D=D^\prime=20\B$ are depicted in Fig.~\ref{fig:experimental_results_E_p}, as an example.
\begin{figure}[h!]
    \centering
    \includegraphics[scale=0.5]{figs/experimental_results_E_p_125M_20B.png}
    \caption{Experimental results for $\E \left[ \secondmomentshort \right]$ (vertical axis) vs. $\widetilde p_i$ (horizontal axis) for $N=125\M$ and $D=D^\prime=20\B$. The blue line shows the linear fit with $R^2 = 0.91$.}
    \label{fig:experimental_results_E_p}
\end{figure}

Note that while $R^2$ and $\I$ are again virtually independent of $N$ and $D^\prime$, the fit parameter $A$ is not. Instead, it seems to increase with $D^\prime$, as shown in Fig.~\ref{fig:experimental_results_A}.
\begin{figure}[h!]
    \centering
    \includegraphics[scale=0.5]{figs/experimental_results_A.png}
    \caption{Experimental results for the linear fit parameter $A$ as a function of $N$ and $D^\prime$.}
    \label{fig:experimental_results_A}
\end{figure}
However, as stated in Eq.~(\ref{eq:second_moment_proportionality_constant}), the order of magnitude is $A \approx 10^{-4}$ throughout our experiments.

\section{Experimental Details}
\subsection{Model and Dataset Sizes}
\label{app:experiments_overview}

The model sizes $N$ and dataset sizes $D$ employed in our experiments are depicted in Fig.~\ref{fig:experiments}.
\begin{figure}[h!]
    \centering
    \includegraphics[scale=0.5]{figs/experiments_log.png}
    \caption{Overview of the dataset (horizontal axis) and model sizes (vertical axis) involved in our small-scale (blue, green and orange circles) and large-scale (red squares) experiments. The dashed, black line shows $N = D / 20$, which is approximately the compute-optimal trajectory according to \citet{hoffmann2022trainingcomputeoptimallargelanguage}.}
    \label{fig:experiments}
\end{figure}


\subsection{Training Hyperparameters}
\label{app:hyperparameters}

In Tab.~\ref{tab:model_architecture}, we list the general hyperparameters used in our small-scale (Sec.~\ref{sec:experiments_S}) and large-scale (Sec.~\ref{sec:experiments_L}) experiments. 
\begin{table}[ht]
\centering
\scriptsize
\begin{tabular}{l|cc}
\toprule
Description & Small-scale & Large-scale \\ 
\midrule
optimizer & \multicolumn{2}{c}{AdamW} \\ 
$\beta_1$ & \multicolumn{2}{c}{0.9} \\
$\beta_2$ & \multicolumn{2}{c}{0.95} \\
$\epsilon$ & \multicolumn{2}{c}{1e-8} \\
weight decay & \multicolumn{2}{c}{0.1} \\
gradient clipping & \multicolumn{2}{c}{1.0} \\
dropout & \multicolumn{2}{c}{0.0} \\
weight tying & \multicolumn{2}{c}{true} \\
vocab size & \multicolumn{2}{c}{50304} \\
learning rate schedule & \multicolumn{2}{c}{cosine decay} \\
layer normalization & \multicolumn{2}{c}{LayerNorm} \\
precision & \multicolumn{2}{c}{BF16} \\
\midrule
hidden activation & GeLU & SwiGLU \\
positional embedding & absolute (learned) & RoPE \\ 
sequence length & 1024 & 2048 \\
batch size (samples) & 96 & 256 \\
batch size (tokens) & $\sim$100k & $\sim$500k \\
warmup & 100 steps & $1\%$ of steps \\
training framework & nanoGPT & Modalities \\
training parallelism & DDP & FSDP \\
\bottomrule 
\end{tabular}
\caption{General hyperparameters used in our two sets of experiments.}
\label{tab:model_architecture}
\end{table}
During warm-up, the learning rate is increased from zero to the maximum learning rate. This is followed by a cosine decay which reduces the learning rate to $10\%$ of the maximum at the end of training. Note that weight decay is applied only to linear layers, not layer norms or embeddings.
Tab.~\ref{tab:model_sizes} shows the hyperparameters related to model size, following GPT-3 \cite{brown2020languagemodelsfewshotlearners}.
\begin{table}[ht]
\centering
\scriptsize
\begin{tabular}{c|cccc}
\toprule
$N$ & lr & heads & layers & emb. dim. \\ 
\midrule
124M & 6.0e-4 & 12 & 12 & 768 \\
350M & 3.0e-4 & 16 & 24 & 1024 \\
760M & 2.5e-4 & 16 & 24 & 1536 \\
1.3B & 2.0e-4 & 32 & 24 & 2048 \\
2.6B & 1.6e-4 & 32 & 32 & 2560 \\
\bottomrule 
\end{tabular}
\caption{Model-size dependent hyperparameter used in our experiments. $N$ denotes the model size in terms of parameters, while lr corresponds to the maximum learning rate.}
\label{tab:model_sizes}
\end{table}

\section{Error Analysis and Statistical Significance}
\label{app:error}

For the error analysis, two separate random variables, $X_0$ and $X_1$, are considered. The symbol $X$ represents one of the metrics discussed in Sec.~\ref{sec:experiments_evaluation}, while $0$ and $1$ stand for two approaches that are to be compared, like standard Adam and Coupled Adam, for instance.
For each of the two random variables $i = \{ 0, 1 \}$, we conduct and evaluate $S$ training runs with different seeds, yielding results 
\begin{align}
\{ X_i^{(1)}, \ldots, X_i^{(S)} \}
\end{align}
While it is desirable to have a large sample size $S$, it is  prohibitively expensive for large model and dataset sizes to repeat training runs. We use
\begin{align}
S &= 3
\label{eq:error_S3}
\end{align}
except for the large-scale experiments (Sec.~\ref{sec:experiments_L}), where we restrict ourselves to
\begin{align}
S &= 1
\label{eq:error_S1}
\end{align}
We are interested in the difference 
\begin{align}
d = X_1 - X_0
\label{eq:error_d}
\end{align}
For $S=1$, it can be computed straight forwardly. However, no statement about the statistical uncertainty or significance of $d$ can be made.
In the case of $S = 3$, we apply a one-sided Student's t-test with a confidence level of 
\begin{align}
\alpha = 95\%
\label{eq:error_confidence_level_alpha}
\end{align}
First, the sample means
\begin{align}
\bar X_i &= \frac{1}{S} \sum_{s=1}^S X_i^{(s)}
\label{eq:student_mean_i}
\end{align}
and the corrected sample standard deviations
\begin{align}
\hat \sigma_i^2 &= \frac{1}{S-1} \sum_{s=1}^S \left( X_i^{(s)} - \bar X_i \right)^2
\label{eq:student_std_i}
\end{align}
for the two samples $i \in \{0, 1\}$ are estimated.
The sample means from Eq.~(\ref{eq:student_mean_i}) are combined to an estimate for their difference,
\begin{align}
\bar d &= \bar X_1 - \bar X_0
\label{eq:student_mean_d}
\end{align}
and the sample standard deviations from Eq.~(\ref{eq:student_std_i}) are propagated to the sample standard deviation of $d$ via Gaussian error propagation:
\begin{align}
\hat \sigma_d &= 
\sqrt{\left( \frac{\partial d}{\partial X_0} \cdot \hat \sigma_0 \right)^2 + \left( \frac{\partial d}{\partial X_1} \cdot \hat \sigma_1 \right)^2} \nonumber \\
&\stackrel{(\ref{eq:error_d})}{=} \sqrt{\hat \sigma_0^2 + \hat \sigma_1^2}
\label{eq:sigmad}
\end{align}

Student's t-distribution for the chosen confidence level $\alpha$ (see Eq.~(\ref{eq:error_confidence_level_alpha})) and the
\begin{align}
\nu &= S - 1 \stackrel{(\ref{eq:error_S3})}{=} 2 
\end{align}
degrees of freedom yields
\begin{align}
t_{\alpha, \nu} = 2.92
\label{eq:talphanu}
\end{align}
With $S$, $\sigma_d$ and $t_{\alpha, \nu}$ from Eqs.~(\ref{eq:error_S3}), (\ref{eq:sigmad}) and (\ref{eq:talphanu}) as ingredients, the one-sided confidence threshold for the difference can be computed as
\begin{align}
d_{\rm significance}
&= t_{\alpha, \nu} \cdot \frac{\hat \sigma_d}{\sqrt{S}}
\label{eq:studentonesidedconfidencethreshold}
\end{align}
Hence, the estimate $\bar d$ from Eq.~(\ref{eq:student_mean_d}) is considered a statistically significant improvement of approach $i=1$ over approach $i=0$ if
\begin{align}
\bar d < - d_{\rm significance}
\label{eq:student_signficance_loss}
\end{align}
for metrics where smaller values are desirable (e.g. $\Loss$), and 
\begin{align}
\bar d > d_{\rm significance}
\label{eq:student_signficance_other}
\end{align}
for metrics where larger values are better (e.g. $\Acc$).

\CatchFileDef{\resultsAblationsSGDExpFive}{tables/results_ablations_sgd_only_exp12.tex}{}
\CatchFileDef{\resultsAblationsSGDExpTen}{tables/results_ablations_sgd_only_exp13.tex}{}
\CatchFileDef{\resultsAblationsSGDExpTwenty}{tables/results_ablations_sgd_only_exp15.tex}{}


\section{Additional Results}

\subsection{Small-scale Experiments}
\label{app:additional_results_S}

In Fig.~\ref{fig:results_S2}, we visualize the results of our small-scale experiments (Sec.~\ref{sec:results_S}) for the loss $\Loss$ and the average downstream task accuracy $\Acc$, as listed in Tab.~\ref{tab:results_S}.
\begin{figure*}[t]
    \centering
    \includegraphics[scale=0.5]{figs/diff-loss-error.png} \qquad
    \includegraphics[scale=0.5]{figs/diff-lm-eval-error.png}
    \caption{Difference in loss (left) and average downstream task accuracy (right) between Coupled Adam and standard Adam, for the different dataset sizes $D$ (horizontal axis) and model sizes $N$ (colors) of the small-scale experiments. The vertical bars indicate the one-sided $95\%$ confidence interval for the difference to be significant. In order to avoid overlaps, the data points for $N=125\M$ and $N=760\M$ are slightly shifted to the left and right, respectively. }
    \label{fig:results_S2}
\end{figure*}

\subsection{Scaled Coupled Adam}
\label{app:additional_results_scale}

Tab.~\ref{tab:results_ablations_scale} of 
Sec.~\ref{sec:ablation_different_learning_rate} shows the results of varying the scaling exponent $n$ (see Eq.~(\ref{eq:optimizer_update_second_moment_avg_scaled})) for $D = 20\B$. The dependency of the loss is visualized in Fig.~\ref{fig:ablation_scale}.
Here, in Fig.~\ref{fig:ablation_scale_complete}, we extend the visualization of the results to $D \in \{ 5\B, 10\B, 20\B \}$ and the other evaluation metrics.
\begin{figure*}[ht]
    \centering
    \includegraphics[scale=0.5]{figs/ablation_1_loss.png}
    \includegraphics[scale=0.5]{figs/ablation_1_lmeval.png}
    \includegraphics[scale=0.5]{figs/ablation_1_isotropy.png}
    \includegraphics[scale=0.5]{figs/ablation_1_mu_rel.png}
    \includegraphics[scale=0.5]{figs/ablation_1_cos.png}
    \caption{Dependency of different metrics on the scaling exponent $n$, see Eq.~(\ref{eq:optimizer_update_second_moment_avg_scaled}). From top to bottom: loss (upstream performance), average accuracy (downstream performance), isotropy, mean embedding norm ratio and $\rcos$. Each plot shows the difference to the respective metric obtained for $n = 0$. The arrows indicate whether larger ($\uparrow$) or smaller ($\downarrow$) values are desirable.}
    \label{fig:ablation_scale_complete}
\end{figure*}

\subsection{SGD}
\label{app:additional_results_sgd}

In Tab.~\ref{tab:results_ablations_sgd_all} of Sec.~\ref{sec:ablation_sgd}, we showed results for SGD using the best hyperparameter $f$. 
Detailed results of the corresponding hyperparameter searches can be found in Tab.~\ref{tab:results_ablations_sgd}.
\begin{table*}[ht]
\centering
\scriptsize
\begin{tabular}{ccc|rrrrrrrr}
\toprule
$D$ & $N$ & Optimizer & $\Loss$ ($\downarrow$) & $\Acc$ ($\uparrow$) & $\ISO$ ($\uparrow$) & $\munorm$ ($\downarrow$) & $\munormrel$ ($\downarrow$) & $\rcos$ ($\uparrow$) & $\rho$ ($\uparrow$) & $\kappa$ ($\uparrow$) \\ 
\midrule
\resultsAblationsSGDExpFive
\bottomrule 
\\
\\
\end{tabular}
\begin{tabular}{ccc|rrrrrrrr}
\toprule
$D$ & $N$ & Optimizer & $\Loss$ ($\downarrow$) & $\Acc$ ($\uparrow$) & $\ISO$ ($\uparrow$) & $\munorm$ ($\downarrow$) & $\munormrel$ ($\downarrow$) & $\rcos$ ($\uparrow$) & $\rho$ ($\uparrow$) & $\kappa$ ($\uparrow$) \\ 
\midrule
\resultsAblationsSGDExpTen
\bottomrule 
\\
\\
\end{tabular}
\begin{tabular}{ccc|rrrrrrrr}
\toprule
$D$ & $N$ & Optimizer & $\Loss$ ($\downarrow$) & $\Acc$ ($\uparrow$) & $\ISO$ ($\uparrow$) & $\munorm$ ($\downarrow$) & $\munormrel$ ($\downarrow$) & $\rcos$ ($\uparrow$) & $\rho$ ($\uparrow$) & $\kappa$ ($\uparrow$) \\ 
\midrule
\resultsAblationsSGDExpTwenty
\bottomrule 
\end{tabular}
\caption{Results of our experiments with SGD. Values are highlighted in bold if they are significantly better than all the other values in the same column.}
\label{tab:results_ablations_sgd}
\end{table*}


\end{document}
