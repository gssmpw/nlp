\pdfoutput=1




\documentclass[table]{article} % For LaTeX2e
\usepackage{iclr2025_conference,times}
\usepackage{enumitem}
\usepackage{amsmath}
\newtheorem{definition}{Definition}
\usepackage{graphicx}
\usepackage{booktabs}
\usepackage{multirow}
\usepackage{pifont}% http://ctan.org/pkg/pifont
\usepackage{caption,subcaption}
\usepackage{adjustbox}
% \usepackage{soul, color, xcolor}
\usepackage{soul,color}
\definecolor{myorange}{RGB}{255,165,0}

% \usepackage[table]{xcolor} % 导入 colortbl 宏包


% define AIbox
% \usepackage{xcolor}
\definecolor{mygray}{RGB}{192,192,192}
\usepackage[most]{tcolorbox}
\usepackage{float}
\usepackage{xspace}
\tcbset{
  aibox/.style={
    % width=474.18663pt,
    width=\textwidth,
    top=10pt,
    colback=white,
    colframe=black,
    colbacktitle=black,
    enhanced,
    center,
    % attach boxed title to top left={yshift=-0.1in,xshift=0.15in},
    attach boxed title to top center={yshift=-0.1in},
    boxed title style={boxrule=0pt,colframe=white,},
  }
}
\newtcolorbox{AIbox}[2][]{aibox,title=#2,#1}
\usepackage{listings}
\lstset{
  basicstyle=\footnotesize\ttfamily,
  columns=fullflexible,
  breaklines=true,
  postbreak=\mbox{\textcolor{red}{$\hookrightarrow$}\space},
}
\definecolor{myblue}{RGB}{100, 150, 200}
\definecolor{mygreen}{RGB}{80, 160, 80}

\definecolor{darkgreen}{rgb}{0.0, 0.5, 0.0}
\definecolor{darkgray}{gray}{0.4}
\definecolor{maroon}{rgb}{0.5, 0.0, 0.0}
\definecolor{navy}{rgb}{0.0, 0.0, 0.5}
\definecolor{teal}{rgb}{0.0, 0.5, 0.5}


\lstset{
  language=Python,
  basicstyle=\ttfamily\footnotesize,
  keywordstyle=\color{darkgreen},
  commentstyle=\color{maroon},
  stringstyle=\color{teal},
  backgroundcolor=\color{lightgray!20},
  frame=single,
  breaklines=true,
  numbers=left,
  numberstyle=\tiny\color{gray},
  tabsize=2,
  showstringspaces=false,
  escapeinside={(*@}{@*)}
}


% Optional math commands from https://github.com/goodfeli/dlbook_notation.
%%%%% NEW MATH DEFINITIONS %%%%%

\usepackage{amsmath,amsfonts,bm}
\usepackage{derivative}
% Mark sections of captions for referring to divisions of figures
\newcommand{\figleft}{{\em (Left)}}
\newcommand{\figcenter}{{\em (Center)}}
\newcommand{\figright}{{\em (Right)}}
\newcommand{\figtop}{{\em (Top)}}
\newcommand{\figbottom}{{\em (Bottom)}}
\newcommand{\captiona}{{\em (a)}}
\newcommand{\captionb}{{\em (b)}}
\newcommand{\captionc}{{\em (c)}}
\newcommand{\captiond}{{\em (d)}}

% Highlight a newly defined term
\newcommand{\newterm}[1]{{\bf #1}}

% Derivative d 
\newcommand{\deriv}{{\mathrm{d}}}

% Figure reference, lower-case.
\def\figref#1{figure~\ref{#1}}
% Figure reference, capital. For start of sentence
\def\Figref#1{Figure~\ref{#1}}
\def\twofigref#1#2{figures \ref{#1} and \ref{#2}}
\def\quadfigref#1#2#3#4{figures \ref{#1}, \ref{#2}, \ref{#3} and \ref{#4}}
% Section reference, lower-case.
\def\secref#1{section~\ref{#1}}
% Section reference, capital.
\def\Secref#1{Section~\ref{#1}}
% Reference to two sections.
\def\twosecrefs#1#2{sections \ref{#1} and \ref{#2}}
% Reference to three sections.
\def\secrefs#1#2#3{sections \ref{#1}, \ref{#2} and \ref{#3}}
% Reference to an equation, lower-case.
\def\eqref#1{equation~\ref{#1}}
% Reference to an equation, upper case
\def\Eqref#1{Equation~\ref{#1}}
% A raw reference to an equation---avoid using if possible
\def\plaineqref#1{\ref{#1}}
% Reference to a chapter, lower-case.
\def\chapref#1{chapter~\ref{#1}}
% Reference to an equation, upper case.
\def\Chapref#1{Chapter~\ref{#1}}
% Reference to a range of chapters
\def\rangechapref#1#2{chapters\ref{#1}--\ref{#2}}
% Reference to an algorithm, lower-case.
\def\algref#1{algorithm~\ref{#1}}
% Reference to an algorithm, upper case.
\def\Algref#1{Algorithm~\ref{#1}}
\def\twoalgref#1#2{algorithms \ref{#1} and \ref{#2}}
\def\Twoalgref#1#2{Algorithms \ref{#1} and \ref{#2}}
% Reference to a part, lower case
\def\partref#1{part~\ref{#1}}
% Reference to a part, upper case
\def\Partref#1{Part~\ref{#1}}
\def\twopartref#1#2{parts \ref{#1} and \ref{#2}}

\def\ceil#1{\lceil #1 \rceil}
\def\floor#1{\lfloor #1 \rfloor}
\def\1{\bm{1}}
\newcommand{\train}{\mathcal{D}}
\newcommand{\valid}{\mathcal{D_{\mathrm{valid}}}}
\newcommand{\test}{\mathcal{D_{\mathrm{test}}}}

\def\eps{{\epsilon}}


% Random variables
\def\reta{{\textnormal{$\eta$}}}
\def\ra{{\textnormal{a}}}
\def\rb{{\textnormal{b}}}
\def\rc{{\textnormal{c}}}
\def\rd{{\textnormal{d}}}
\def\re{{\textnormal{e}}}
\def\rf{{\textnormal{f}}}
\def\rg{{\textnormal{g}}}
\def\rh{{\textnormal{h}}}
\def\ri{{\textnormal{i}}}
\def\rj{{\textnormal{j}}}
\def\rk{{\textnormal{k}}}
\def\rl{{\textnormal{l}}}
% rm is already a command, just don't name any random variables m
\def\rn{{\textnormal{n}}}
\def\ro{{\textnormal{o}}}
\def\rp{{\textnormal{p}}}
\def\rq{{\textnormal{q}}}
\def\rr{{\textnormal{r}}}
\def\rs{{\textnormal{s}}}
\def\rt{{\textnormal{t}}}
\def\ru{{\textnormal{u}}}
\def\rv{{\textnormal{v}}}
\def\rw{{\textnormal{w}}}
\def\rx{{\textnormal{x}}}
\def\ry{{\textnormal{y}}}
\def\rz{{\textnormal{z}}}

% Random vectors
\def\rvepsilon{{\mathbf{\epsilon}}}
\def\rvphi{{\mathbf{\phi}}}
\def\rvtheta{{\mathbf{\theta}}}
\def\rva{{\mathbf{a}}}
\def\rvb{{\mathbf{b}}}
\def\rvc{{\mathbf{c}}}
\def\rvd{{\mathbf{d}}}
\def\rve{{\mathbf{e}}}
\def\rvf{{\mathbf{f}}}
\def\rvg{{\mathbf{g}}}
\def\rvh{{\mathbf{h}}}
\def\rvu{{\mathbf{i}}}
\def\rvj{{\mathbf{j}}}
\def\rvk{{\mathbf{k}}}
\def\rvl{{\mathbf{l}}}
\def\rvm{{\mathbf{m}}}
\def\rvn{{\mathbf{n}}}
\def\rvo{{\mathbf{o}}}
\def\rvp{{\mathbf{p}}}
\def\rvq{{\mathbf{q}}}
\def\rvr{{\mathbf{r}}}
\def\rvs{{\mathbf{s}}}
\def\rvt{{\mathbf{t}}}
\def\rvu{{\mathbf{u}}}
\def\rvv{{\mathbf{v}}}
\def\rvw{{\mathbf{w}}}
\def\rvx{{\mathbf{x}}}
\def\rvy{{\mathbf{y}}}
\def\rvz{{\mathbf{z}}}

% Elements of random vectors
\def\erva{{\textnormal{a}}}
\def\ervb{{\textnormal{b}}}
\def\ervc{{\textnormal{c}}}
\def\ervd{{\textnormal{d}}}
\def\erve{{\textnormal{e}}}
\def\ervf{{\textnormal{f}}}
\def\ervg{{\textnormal{g}}}
\def\ervh{{\textnormal{h}}}
\def\ervi{{\textnormal{i}}}
\def\ervj{{\textnormal{j}}}
\def\ervk{{\textnormal{k}}}
\def\ervl{{\textnormal{l}}}
\def\ervm{{\textnormal{m}}}
\def\ervn{{\textnormal{n}}}
\def\ervo{{\textnormal{o}}}
\def\ervp{{\textnormal{p}}}
\def\ervq{{\textnormal{q}}}
\def\ervr{{\textnormal{r}}}
\def\ervs{{\textnormal{s}}}
\def\ervt{{\textnormal{t}}}
\def\ervu{{\textnormal{u}}}
\def\ervv{{\textnormal{v}}}
\def\ervw{{\textnormal{w}}}
\def\ervx{{\textnormal{x}}}
\def\ervy{{\textnormal{y}}}
\def\ervz{{\textnormal{z}}}

% Random matrices
\def\rmA{{\mathbf{A}}}
\def\rmB{{\mathbf{B}}}
\def\rmC{{\mathbf{C}}}
\def\rmD{{\mathbf{D}}}
\def\rmE{{\mathbf{E}}}
\def\rmF{{\mathbf{F}}}
\def\rmG{{\mathbf{G}}}
\def\rmH{{\mathbf{H}}}
\def\rmI{{\mathbf{I}}}
\def\rmJ{{\mathbf{J}}}
\def\rmK{{\mathbf{K}}}
\def\rmL{{\mathbf{L}}}
\def\rmM{{\mathbf{M}}}
\def\rmN{{\mathbf{N}}}
\def\rmO{{\mathbf{O}}}
\def\rmP{{\mathbf{P}}}
\def\rmQ{{\mathbf{Q}}}
\def\rmR{{\mathbf{R}}}
\def\rmS{{\mathbf{S}}}
\def\rmT{{\mathbf{T}}}
\def\rmU{{\mathbf{U}}}
\def\rmV{{\mathbf{V}}}
\def\rmW{{\mathbf{W}}}
\def\rmX{{\mathbf{X}}}
\def\rmY{{\mathbf{Y}}}
\def\rmZ{{\mathbf{Z}}}

% Elements of random matrices
\def\ermA{{\textnormal{A}}}
\def\ermB{{\textnormal{B}}}
\def\ermC{{\textnormal{C}}}
\def\ermD{{\textnormal{D}}}
\def\ermE{{\textnormal{E}}}
\def\ermF{{\textnormal{F}}}
\def\ermG{{\textnormal{G}}}
\def\ermH{{\textnormal{H}}}
\def\ermI{{\textnormal{I}}}
\def\ermJ{{\textnormal{J}}}
\def\ermK{{\textnormal{K}}}
\def\ermL{{\textnormal{L}}}
\def\ermM{{\textnormal{M}}}
\def\ermN{{\textnormal{N}}}
\def\ermO{{\textnormal{O}}}
\def\ermP{{\textnormal{P}}}
\def\ermQ{{\textnormal{Q}}}
\def\ermR{{\textnormal{R}}}
\def\ermS{{\textnormal{S}}}
\def\ermT{{\textnormal{T}}}
\def\ermU{{\textnormal{U}}}
\def\ermV{{\textnormal{V}}}
\def\ermW{{\textnormal{W}}}
\def\ermX{{\textnormal{X}}}
\def\ermY{{\textnormal{Y}}}
\def\ermZ{{\textnormal{Z}}}

% Vectors
\def\vzero{{\bm{0}}}
\def\vone{{\bm{1}}}
\def\vmu{{\bm{\mu}}}
\def\vtheta{{\bm{\theta}}}
\def\vphi{{\bm{\phi}}}
\def\va{{\bm{a}}}
\def\vb{{\bm{b}}}
\def\vc{{\bm{c}}}
\def\vd{{\bm{d}}}
\def\ve{{\bm{e}}}
\def\vf{{\bm{f}}}
\def\vg{{\bm{g}}}
\def\vh{{\bm{h}}}
\def\vi{{\bm{i}}}
\def\vj{{\bm{j}}}
\def\vk{{\bm{k}}}
\def\vl{{\bm{l}}}
\def\vm{{\bm{m}}}
\def\vn{{\bm{n}}}
\def\vo{{\bm{o}}}
\def\vp{{\bm{p}}}
\def\vq{{\bm{q}}}
\def\vr{{\bm{r}}}
\def\vs{{\bm{s}}}
\def\vt{{\bm{t}}}
\def\vu{{\bm{u}}}
\def\vv{{\bm{v}}}
\def\vw{{\bm{w}}}
\def\vx{{\bm{x}}}
\def\vy{{\bm{y}}}
\def\vz{{\bm{z}}}

% Elements of vectors
\def\evalpha{{\alpha}}
\def\evbeta{{\beta}}
\def\evepsilon{{\epsilon}}
\def\evlambda{{\lambda}}
\def\evomega{{\omega}}
\def\evmu{{\mu}}
\def\evpsi{{\psi}}
\def\evsigma{{\sigma}}
\def\evtheta{{\theta}}
\def\eva{{a}}
\def\evb{{b}}
\def\evc{{c}}
\def\evd{{d}}
\def\eve{{e}}
\def\evf{{f}}
\def\evg{{g}}
\def\evh{{h}}
\def\evi{{i}}
\def\evj{{j}}
\def\evk{{k}}
\def\evl{{l}}
\def\evm{{m}}
\def\evn{{n}}
\def\evo{{o}}
\def\evp{{p}}
\def\evq{{q}}
\def\evr{{r}}
\def\evs{{s}}
\def\evt{{t}}
\def\evu{{u}}
\def\evv{{v}}
\def\evw{{w}}
\def\evx{{x}}
\def\evy{{y}}
\def\evz{{z}}

% Matrix
\def\mA{{\bm{A}}}
\def\mB{{\bm{B}}}
\def\mC{{\bm{C}}}
\def\mD{{\bm{D}}}
\def\mE{{\bm{E}}}
\def\mF{{\bm{F}}}
\def\mG{{\bm{G}}}
\def\mH{{\bm{H}}}
\def\mI{{\bm{I}}}
\def\mJ{{\bm{J}}}
\def\mK{{\bm{K}}}
\def\mL{{\bm{L}}}
\def\mM{{\bm{M}}}
\def\mN{{\bm{N}}}
\def\mO{{\bm{O}}}
\def\mP{{\bm{P}}}
\def\mQ{{\bm{Q}}}
\def\mR{{\bm{R}}}
\def\mS{{\bm{S}}}
\def\mT{{\bm{T}}}
\def\mU{{\bm{U}}}
\def\mV{{\bm{V}}}
\def\mW{{\bm{W}}}
\def\mX{{\bm{X}}}
\def\mY{{\bm{Y}}}
\def\mZ{{\bm{Z}}}
\def\mBeta{{\bm{\beta}}}
\def\mPhi{{\bm{\Phi}}}
\def\mLambda{{\bm{\Lambda}}}
\def\mSigma{{\bm{\Sigma}}}

% Tensor
\DeclareMathAlphabet{\mathsfit}{\encodingdefault}{\sfdefault}{m}{sl}
\SetMathAlphabet{\mathsfit}{bold}{\encodingdefault}{\sfdefault}{bx}{n}
\newcommand{\tens}[1]{\bm{\mathsfit{#1}}}
\def\tA{{\tens{A}}}
\def\tB{{\tens{B}}}
\def\tC{{\tens{C}}}
\def\tD{{\tens{D}}}
\def\tE{{\tens{E}}}
\def\tF{{\tens{F}}}
\def\tG{{\tens{G}}}
\def\tH{{\tens{H}}}
\def\tI{{\tens{I}}}
\def\tJ{{\tens{J}}}
\def\tK{{\tens{K}}}
\def\tL{{\tens{L}}}
\def\tM{{\tens{M}}}
\def\tN{{\tens{N}}}
\def\tO{{\tens{O}}}
\def\tP{{\tens{P}}}
\def\tQ{{\tens{Q}}}
\def\tR{{\tens{R}}}
\def\tS{{\tens{S}}}
\def\tT{{\tens{T}}}
\def\tU{{\tens{U}}}
\def\tV{{\tens{V}}}
\def\tW{{\tens{W}}}
\def\tX{{\tens{X}}}
\def\tY{{\tens{Y}}}
\def\tZ{{\tens{Z}}}


% Graph
\def\gA{{\mathcal{A}}}
\def\gB{{\mathcal{B}}}
\def\gC{{\mathcal{C}}}
\def\gD{{\mathcal{D}}}
\def\gE{{\mathcal{E}}}
\def\gF{{\mathcal{F}}}
\def\gG{{\mathcal{G}}}
\def\gH{{\mathcal{H}}}
\def\gI{{\mathcal{I}}}
\def\gJ{{\mathcal{J}}}
\def\gK{{\mathcal{K}}}
\def\gL{{\mathcal{L}}}
\def\gM{{\mathcal{M}}}
\def\gN{{\mathcal{N}}}
\def\gO{{\mathcal{O}}}
\def\gP{{\mathcal{P}}}
\def\gQ{{\mathcal{Q}}}
\def\gR{{\mathcal{R}}}
\def\gS{{\mathcal{S}}}
\def\gT{{\mathcal{T}}}
\def\gU{{\mathcal{U}}}
\def\gV{{\mathcal{V}}}
\def\gW{{\mathcal{W}}}
\def\gX{{\mathcal{X}}}
\def\gY{{\mathcal{Y}}}
\def\gZ{{\mathcal{Z}}}

% Sets
\def\sA{{\mathbb{A}}}
\def\sB{{\mathbb{B}}}
\def\sC{{\mathbb{C}}}
\def\sD{{\mathbb{D}}}
% Don't use a set called E, because this would be the same as our symbol
% for expectation.
\def\sF{{\mathbb{F}}}
\def\sG{{\mathbb{G}}}
\def\sH{{\mathbb{H}}}
\def\sI{{\mathbb{I}}}
\def\sJ{{\mathbb{J}}}
\def\sK{{\mathbb{K}}}
\def\sL{{\mathbb{L}}}
\def\sM{{\mathbb{M}}}
\def\sN{{\mathbb{N}}}
\def\sO{{\mathbb{O}}}
\def\sP{{\mathbb{P}}}
\def\sQ{{\mathbb{Q}}}
\def\sR{{\mathbb{R}}}
\def\sS{{\mathbb{S}}}
\def\sT{{\mathbb{T}}}
\def\sU{{\mathbb{U}}}
\def\sV{{\mathbb{V}}}
\def\sW{{\mathbb{W}}}
\def\sX{{\mathbb{X}}}
\def\sY{{\mathbb{Y}}}
\def\sZ{{\mathbb{Z}}}

% Entries of a matrix
\def\emLambda{{\Lambda}}
\def\emA{{A}}
\def\emB{{B}}
\def\emC{{C}}
\def\emD{{D}}
\def\emE{{E}}
\def\emF{{F}}
\def\emG{{G}}
\def\emH{{H}}
\def\emI{{I}}
\def\emJ{{J}}
\def\emK{{K}}
\def\emL{{L}}
\def\emM{{M}}
\def\emN{{N}}
\def\emO{{O}}
\def\emP{{P}}
\def\emQ{{Q}}
\def\emR{{R}}
\def\emS{{S}}
\def\emT{{T}}
\def\emU{{U}}
\def\emV{{V}}
\def\emW{{W}}
\def\emX{{X}}
\def\emY{{Y}}
\def\emZ{{Z}}
\def\emSigma{{\Sigma}}

% entries of a tensor
% Same font as tensor, without \bm wrapper
\newcommand{\etens}[1]{\mathsfit{#1}}
\def\etLambda{{\etens{\Lambda}}}
\def\etA{{\etens{A}}}
\def\etB{{\etens{B}}}
\def\etC{{\etens{C}}}
\def\etD{{\etens{D}}}
\def\etE{{\etens{E}}}
\def\etF{{\etens{F}}}
\def\etG{{\etens{G}}}
\def\etH{{\etens{H}}}
\def\etI{{\etens{I}}}
\def\etJ{{\etens{J}}}
\def\etK{{\etens{K}}}
\def\etL{{\etens{L}}}
\def\etM{{\etens{M}}}
\def\etN{{\etens{N}}}
\def\etO{{\etens{O}}}
\def\etP{{\etens{P}}}
\def\etQ{{\etens{Q}}}
\def\etR{{\etens{R}}}
\def\etS{{\etens{S}}}
\def\etT{{\etens{T}}}
\def\etU{{\etens{U}}}
\def\etV{{\etens{V}}}
\def\etW{{\etens{W}}}
\def\etX{{\etens{X}}}
\def\etY{{\etens{Y}}}
\def\etZ{{\etens{Z}}}

% The true underlying data generating distribution
\newcommand{\pdata}{p_{\rm{data}}}
\newcommand{\ptarget}{p_{\rm{target}}}
\newcommand{\pprior}{p_{\rm{prior}}}
\newcommand{\pbase}{p_{\rm{base}}}
\newcommand{\pref}{p_{\rm{ref}}}

% The empirical distribution defined by the training set
\newcommand{\ptrain}{\hat{p}_{\rm{data}}}
\newcommand{\Ptrain}{\hat{P}_{\rm{data}}}
% The model distribution
\newcommand{\pmodel}{p_{\rm{model}}}
\newcommand{\Pmodel}{P_{\rm{model}}}
\newcommand{\ptildemodel}{\tilde{p}_{\rm{model}}}
% Stochastic autoencoder distributions
\newcommand{\pencode}{p_{\rm{encoder}}}
\newcommand{\pdecode}{p_{\rm{decoder}}}
\newcommand{\precons}{p_{\rm{reconstruct}}}

\newcommand{\laplace}{\mathrm{Laplace}} % Laplace distribution

\newcommand{\E}{\mathbb{E}}
\newcommand{\Ls}{\mathcal{L}}
\newcommand{\R}{\mathbb{R}}
\newcommand{\emp}{\tilde{p}}
\newcommand{\lr}{\alpha}
\newcommand{\reg}{\lambda}
\newcommand{\rect}{\mathrm{rectifier}}
\newcommand{\softmax}{\mathrm{softmax}}
\newcommand{\sigmoid}{\sigma}
\newcommand{\softplus}{\zeta}
\newcommand{\KL}{D_{\mathrm{KL}}}
\newcommand{\Var}{\mathrm{Var}}
\newcommand{\standarderror}{\mathrm{SE}}
\newcommand{\Cov}{\mathrm{Cov}}
% Wolfram Mathworld says $L^2$ is for function spaces and $\ell^2$ is for vectors
% But then they seem to use $L^2$ for vectors throughout the site, and so does
% wikipedia.
\newcommand{\normlzero}{L^0}
\newcommand{\normlone}{L^1}
\newcommand{\normltwo}{L^2}
\newcommand{\normlp}{L^p}
\newcommand{\normmax}{L^\infty}

\newcommand{\parents}{Pa} % See usage in notation.tex. Chosen to match Daphne's book.

\DeclareMathOperator*{\argmax}{arg\,max}
\DeclareMathOperator*{\argmin}{arg\,min}

\DeclareMathOperator{\sign}{sign}
\DeclareMathOperator{\Tr}{Tr}
\let\ab\allowbreak


\usepackage{hyperref}
\usepackage{url}


% \title{Formatting Instructions for ICLR 2025 \\ Conference Submissions}
% \title{A Unified Framework of Explainable Operations Research Problems within Large Language Models}
% \title{Quality the Importance of ``Decision Informance'' for Explainable Operations Research Problems within Large Language Models}
% \title{``Decision Information'' in Explainable Operations Research Powered by Large Language Models}
% \title{Decision Information in Explainable Operations Research: Bridging Transparency with LLMs and Bipartite Graphs}
\title{Decision Information Meets Large Language Models: The Future of Explainable Operations Research}
% Explainable Operations Research with Large Language Models

% Authors must not appear in the submitted version. They should be hidden
% as long as the \iclrfinalcopy macro remains commented out below.
% Non-anonymous submissions will be rejected without review.

\author{
Yansen Zhang$^{1*}$, Qingcan Kang$^{2\dagger}$, Wing Yin Yu$^{2}$, Hailei Gong$^{2}$, Xiaojin Fu$^{2}$, Xiongwei Han$^{2}$,\\
~\textbf{Tao Zhong}$^{2}$, \textbf{Chen Ma}$^{1\dagger}$\\
     $^{1}$Department of Computer Science, City University of Hong Kong \\
     $^{2}$Huawei Noah's Ark Lab \\
     \texttt{yanszhang7-c@my.cityu.edu.hk}\\
     \texttt{kangqingcan@huawei.com}\\
    \texttt{chenma@cityu.edu.hk}
}

% \author{Antiquus S.~Hippocampus, Natalia Cerebro \& Amelie P. Amygdale \thanks{ Use footnote for providing further information
% about author (webpage, alternative address)---\emph{not} for acknowledging
% funding agencies.  Funding acknowledgements go at the end of the paper.} \\
% Department of Computer Science\\
% Cranberry-Lemon University\\
% Pittsburgh, PA 15213, USA \\
% \texttt{\{hippo,brain,jen\}@cs.cranberry-lemon.edu} \\
% \And
% Ji Q. Ren \& Yevgeny LeNet \\
% Department of Computational Neuroscience \\
% University of the Witwatersrand \\
% Joburg, South Africa \\
% \texttt{\{robot,net\}@wits.ac.za} \\
% \AND
% Coauthor \\
% Affiliation \\
% Address \\
% \texttt{email}
% }

% The \author macro works with any number of authors. There are two commands
% used to separate the names and addresses of multiple authors: \And and \AND.
%
% Using \And between authors leaves it to \LaTeX{} to determine where to break
% the lines. Using \AND forces a linebreak at that point. So, if \LaTeX{}
% puts 3 of 4 authors names on the first line, and the last on the second
% line, try using \AND instead of \And before the third author name.

\newcommand{\fix}{\marginpar{FIX}}
\newcommand{\new}{\marginpar{NEW}}
\iclrfinalcopy
%\iclrfinalcopy % Uncomment for camera-ready version, but NOT for submission.
\begin{document}


\maketitle

\def\customfootnotetext#1#2{{%
  \let\thefootnote\relax
  \footnotetext[#1]{#2}}}

\customfootnotetext{1}{\textsuperscript{*} Work done as an intern in Huawei Noah's Ark Lab.}
\customfootnotetext{2}{{$\dagger$} Corresponding authors.}



\begin{abstract}
  In this work, we present a novel technique for GPU-accelerated Boolean satisfiability (SAT) sampling. Unlike conventional sampling algorithms that directly operate on conjunctive normal form (CNF), our method transforms the logical constraints of SAT problems by factoring their CNF representations into simplified multi-level, multi-output Boolean functions. It then leverages gradient-based optimization to guide the search for a diverse set of valid solutions. Our method operates directly on the circuit structure of refactored SAT instances, reinterpreting the SAT problem as a supervised multi-output regression task. This differentiable technique enables independent bit-wise operations on each tensor element, allowing parallel execution of learning processes. As a result, we achieve GPU-accelerated sampling with significant runtime improvements ranging from $33.6\times$ to $523.6\times$ over state-of-the-art heuristic samplers. We demonstrate the superior performance of our sampling method through an extensive evaluation on $60$ instances from a public domain benchmark suite utilized in previous studies. 


  
  % Generating a wide range of diverse solutions to logical constraints is crucial in software and hardware testing, verification, and synthesis. These solutions can serve as inputs to test specific functionalities of a software program or as random stimuli in hardware modules. In software verification, techniques like fuzz testing and symbolic execution use this approach to identify bugs and vulnerabilities. In hardware verification, stimulus generation is particularly vital, forming the basis of constrained-random verification. While generating multiple solutions improves coverage and increases the chances of finding bugs, high-throughput sampling remains challenging, especially with complex constraints and refined coverage criteria. In this work, we present a novel technique that enables GPU-accelerated sampling, resulting in high-throughput generation of satisfying solutions to Boolean satisfiability (SAT) problems. Unlike conventional sampling algorithms that directly operate on conjunctive normal form (CNF), our method refines the logical constraints of SAT problems by transforming their CNF into simplified multi-level Boolean expressions. It then leverages gradient-based optimization to guide the search for a diverse set of valid solutions.
  % Our method specifically takes advantage of the circuit structure of refined SAT instances by using GD to learn valid solutions, reinterpreting the SAT problem as a supervised multi-output regression task. This differentiable technique enables independent bit-wise operations on each tensor element, allowing parallel execution of learning processes. As a result, we achieve GPU-accelerated sampling with significant runtime improvements ranging from $10\times$ to $1000\times$ over state-of-the-art heuristic samplers. Specifically, we demonstrate the superior performance of our sampling method through an extensive evaluation on $60$ instances from a public domain benchmark suite utilized in previous studies.

\end{abstract}

\begin{IEEEkeywords}
Boolean Satisfiability, Gradient Descent, Multi-level Circuits, Verification, and Testing.
\end{IEEEkeywords}
\section{Introduction}
\label{section:introduction}

% redirection is unique and important in VR
Virtual Reality (VR) systems enable users to embody virtual avatars by mirroring their physical movements and aligning their perspective with virtual avatars' in real time. 
As the head-mounted displays (HMDs) block direct visual access to the physical world, users primarily rely on visual feedback from the virtual environment and integrate it with proprioceptive cues to control the avatar’s movements and interact within the VR space.
Since human perception is heavily influenced by visual input~\cite{gibson1933adaptation}, 
VR systems have the unique capability to control users' perception of the virtual environment and avatars by manipulating the visual information presented to them.
Leveraging this, various redirection techniques have been proposed to enable novel VR interactions, 
such as redirecting users' walking paths~\cite{razzaque2005redirected, suma2012impossible, steinicke2009estimation},
modifying reaching movements~\cite{gonzalez2022model, azmandian2016haptic, cheng2017sparse, feick2021visuo},
and conveying haptic information through visual feedback to create pseudo-haptic effects~\cite{samad2019pseudo, dominjon2005influence, lecuyer2009simulating}.
Such redirection techniques enable these interactions by manipulating the alignment between users' physical movements and their virtual avatar's actions.

% % what is hand/arm redirection, motivation of study arm-offset
% \change{\yj{i don't understand the purpose of this paragraph}
% These illusion-based techniques provide users with unique experiences in virtual environments that differ from the physical world yet maintain an immersive experience. 
% A key example is hand redirection, which shifts the virtual hand’s position away from the real hand as the user moves to enhance ergonomics during interaction~\cite{feuchtner2018ownershift, wentzel2020improving} and improve interaction performance~\cite{montano2017erg, poupyrev1996go}. 
% To increase the realism of virtual movements and strengthen the user’s sense of embodiment, hand redirection techniques often incorporate a complete virtual arm or full body alongside the redirected virtual hand, using inverse kinematics~\cite{hartfill2021analysis, ponton2024stretch} or adjustments to the virtual arm's movement as well~\cite{li2022modeling, feick2024impact}.
% }

% noticeability, motivation of predicting a probability, not a classification
However, these redirection techniques are most effective when the manipulation remains undetected~\cite{gonzalez2017model, li2022modeling}. 
If the redirection becomes too large, the user may not mitigate the conflict between the visual sensory input (redirected virtual movement) and their proprioception (actual physical movement), potentially leading to a loss of embodiment with the virtual avatar and making it difficult for the user to accurately control virtual movements to complete interaction tasks~\cite{li2022modeling, wentzel2020improving, feuchtner2018ownershift}. 
While proprioception is not absolute, users only have a general sense of their physical movements and the likelihood that they notice the redirection is probabilistic. 
This probability of detecting the redirection is referred to as \textbf{noticeability}~\cite{li2022modeling, zenner2024beyond, zenner2023detectability} and is typically estimated based on the frequency with which users detect the manipulation across multiple trials.

% version B
% Prior research has explored factors influencing the noticeability of redirected motion, including the redirection's magnitude~\cite{wentzel2020improving, poupyrev1996go}, direction~\cite{li2022modeling, feuchtner2018ownershift}, and the visual characteristics of the virtual avatar~\cite{ogawa2020effect, feick2024impact}.
% While these factors focus on the avatars, the surrounding virtual environment can also influence the users' behavior and in turn affect the noticeability of redirection.
% One such prominent external influence is through the visual channel - the users' visual attention is constantly distracted by complex visual effects and events in practical VR scenarios.
% Although some prior studies have explored how to leverage user blindness caused by visual distractions to redirect users' virtual hand~\cite{zenner2023detectability}, there remains a gap in understanding how to quantify the noticeability of redirection under visual distractions.

% visual stimuli and gaze behavior
Prior research has explored factors influencing the noticeability of redirected motion, including the redirection's magnitude~\cite{wentzel2020improving, poupyrev1996go}, direction~\cite{li2022modeling, feuchtner2018ownershift}, and the visual characteristics of the virtual avatar~\cite{ogawa2020effect, feick2024impact}.
While these factors focus on the avatars, the surrounding virtual environment can also influence the users' behavior and in turn affect the noticeability of redirection.
This, however, remains underexplored.
One such prominent external influence is through the visual channel - the users' visual attention is constantly distracted by complex visual effects and events in practical VR scenarios.
We thus want to investigate how \textbf{visual stimuli in the virtual environment} affect the noticeability of redirection.
With this, we hope to complement existing works that focus on avatars by incorporating environmental visual influences to enable more accurate control over the noticeability of redirected motions in practical VR scenarios.
% However, in realistic VR applications, the virtual environment often contains complex visual effects beyond the virtual avatar itself. 
% We argue that these visual effects can \textbf{distract users’ visual attention and thus affect the noticeability of redirection offsets}, while current research has yet taken into account.
% For instance, in a VR boxing scenario, a user’s visual attention is likely focused on their opponent rather than on their virtual body, leading to a lower noticeability of redirection offsets on their virtual movements. 
% Conversely, when reaching for an object in the center of their field of view, the user’s attention is more concentrated on the virtual hand’s movement and position to ensure successful interaction, resulting in a higher noticeability of offsets.

Since each visual event is a complex choreography of many underlying factors (type of visual effect, location, duration, etc.), it is extremely difficult to quantify or parameterize visual stimuli.
Furthermore, individuals respond differently to even the same visual events.
Prior neuroscience studies revealed that factors like age, gender, and personality can influence how quickly someone reacts to visual events~\cite{gillon2024responses, gale1997human}. 
Therefore, aiming to model visual stimuli in a way that is generalizable and applicable to different stimuli and users, we propose to use users' \textbf{gaze behavior} as an indicator of how they respond to visual stimuli.
In this paper, we used various gaze behaviors, including gaze location, saccades~\cite{krejtz2018eye}, fixations~\cite{perkhofer2019using}, and the Index of Pupil Activity (IPA)~\cite{duchowski2018index}.
These behaviors indicate both where users are looking and their cognitive activity, as looking at something does not necessarily mean they are attending to it.
Our goal is to investigate how these gaze behaviors stimulated by various visual stimuli relate to the noticeability of redirection.
With this, we contribute a model that allows designers and content creators to adjust the redirection in real-time responding to dynamic visual events in VR.

To achieve this, we conducted user studies to collect users' noticeability of redirection under various visual stimuli.
To simulate realistic VR scenarios, we adopted a dual-task design in which the participants performed redirected movements while monitoring the visual stimuli.
Specifically, participants' primary task was to report if they noticed an offset between the avatar's movement and their own, while their secondary task was to monitor and report the visual stimuli.
As realistic virtual environments often contain complex visual effects, we started with simple and controlled visual stimulus to manage the influencing factors.

% first user study, confirmation study
% collect data under no visual stimuli, different basic visual stimuli
We first conducted a confirmation study (N=16) to test whether applying visual stimuli (opacity-based) actually affects their noticeability of redirection. 
The results showed that participants were significantly less likely to detect the redirection when visual stimuli was presented $(F_{(1,15)}=5.90,~p=0.03)$.
Furthermore, by analyzing the collected gaze data, results revealed a correlation between the proposed gaze behaviors and the noticeability results $(r=-0.43)$, confirming that the gaze behaviors could be leveraged to compute the noticeability.

% data collection study
We then conducted a data collection study to obtain more accurate noticeability results through repeated measurements to better model the relationship between visual stimuli-triggered gaze behaviors and noticeability of redirection.
With the collected data, we analyzed various numerical features from the gaze behaviors to identify the most effective ones. 
We tested combinations of these features to determine the most effective one for predicting noticeability under visual stimuli.
Using the selected features, our regression model achieved a mean squared error (MSE) of 0.011 through leave-one-user-out cross-validation. 
Furthermore, we developed both a binary and a three-class classification model to categorize noticeability, which achieved an accuracy of 91.74\% and 85.62\%, respectively.

% evaluation study
To evaluate the generalizability of the regression model, we conducted an evaluation study (N=24) to test whether the model could accurately predict noticeability with new visual stimuli (color- and scale-based animations).
Specifically, we evaluated whether the model's predictions aligned with participants' responses under these unseen stimuli.
The results showed that our model accurately estimated the noticeability, achieving mean squared errors (MSE) of 0.014 and 0.012 for the color- and scale-based visual stimili, respectively, compared to participants' responses.
Since the tested visual stimuli data were not included in the training, the results suggested that the extracted gaze behavior features capture a generalizable pattern and can effectively indicate the corresponding impact on the noticeability of redirection.

% application
Based on our model, we implemented an adaptive redirection technique and demonstrated it through two applications: adaptive VR action game and opportunistic rendering.
We conducted a proof-of-concept user study (N=8) to compare our adaptive redirection technique with a static redirection, evaluating the usability and benefits of our adaptive redirection technique.
The results indicated that participants experienced less physical demand and stronger sense of embodiment and agency when using the adaptive redirection technique. 
These results demonstrated the effectiveness and usability of our model.

In summary, we make the following contributions.
% 
\begin{itemize}
    \item 
    We propose to use users' gaze behavior as a medium to quantify how visual stimuli influences the noticebility of redirection. 
    Through two user studies, we confirm that visual stimuli significantly influences noticeability and identify key gaze behavior features that are closely related to this impact.
    \item 
    We build a regression model that takes the user's gaze behavioral data as input, then computes the noticeability of redirection.
    Through an evaluation study, we verify that our model can estimate the noticeability with new participants under unseen visual stimuli.
    These findings suggest that the extracted gaze behavior features effectively capture the influence of visual stimuli on noticeability and can generalize across different users and visual stimuli.
    \item 
    We develop an adaptive redirection technique based on our regression model and implement two applications with it.
    With a proof-of-concept study, we demonstrate the effectiveness and potential usability of our regression model on real-world use cases.

\end{itemize}

% \delete{
% Virtual Reality (VR) allows the user to embody a virtual avatar by mirroring their physical movements through the avatar.
% As the user's visual access to the physical world is blocked in tasks involving motion control, they heavily rely on the visual representation of the avatar's motions to guide their proprioception.
% Similar to real-world experiences, the user is able to resolve conflicts between different sensory inputs (e.g., vision and motor control) through multisensory integration, which is essential for mitigating the sensory noise that commonly arises.
% However, it also enables unique manipulations in VR, as the system can intentionally modify the avatar's movements in relation to the user's motions to achieve specific functional outcomes,
% for example, 
% % the manipulations on the avatar's movements can 
% enabling novel interaction techniques of redirected walking~\cite{razzaque2005redirected}, redirected reaching~\cite{gonzalez2022model}, and pseudo haptics~\cite{samad2019pseudo}.
% With small adjustments to the avatar's movements, the user can maintain their sense of embodiment, due to their ability to resolve the perceptual differences.
% % However, a large mismatch between the user and avatar's movements can result in the user losing their sense of embodiment, due to an inability to resolve the perceptual differences.
% }

% \delete{
% However, multisensory integration can break when the manipulation is so intense that the user is aware of the existence of the motion offset and no longer maintains the sense of embodiment.
% Prior research studied the intensity threshold of the offset applied on the avatar's hand, beyond which the embodiment will break~\cite{li2022modeling}. 
% Studies also investigated the user's sensitivity to the offsets over time~\cite{kohm2022sensitivity}.
% Based on the findings, we argue that one crucial factor that affects to what extent the user notices the offset (i.e., \textit{noticeability}) that remains under-explored is whether the user directs their visual attention towards or away from the virtual avatar.
% Related work (e.g., Mise-unseen~\cite{marwecki2019mise}) has showcased applications where adjustments in the environment can be made in an unnoticeable manner when they happen in the area out of the user's visual field.
% We hypothesize that directing the user's visual attention away from the avatar's body, while still partially keeping the avatar within the user's field-of-view, can reduce the noticeability of the offset.
% Therefore, we conduct two user studies and implement a regression model to systematically investigate this effect.
% }

% \delete{
% In the first user study (N = 16), we test whether drawing the user's visual attention away from their body impacts the possibility of them noticing an offset that we apply to their arm motion in VR.
% We adopt a dual-task design to enable the alteration of the user's visual attention and a yes/no paradigm to measure the noticeability of motion offset. 
% The primary task for the user is to perform an arm motion and report when they perceive an offset between the avatar's virtual arm and their real arm.
% In the secondary task, we randomly render a visual animation of a ball turning from transparent to red and becoming transparent again and ask them to monitor and report when it appears.
% We control the strength of the visual stimuli by changing the duration and location of the animation.
% % By changing the time duration and location of the visual animation, we control the strengths of attraction to the users.
% As a result, we found significant differences in the noticeability of the offsets $(F_{(1,15)}=5.90,~p=0.03)$ between conditions with and without visual stimuli.
% Based on further analysis, we also identified the behavioral patterns of the user's gaze (including pupil dilation, fixations, and saccades) to be correlated with the noticeability results $(r=-0.43)$ and they may potentially serve as indicators of noticeability.
% }

% \delete{
% To further investigate how visual attention influences the noticeability, we conduct a data collection study (N = 12) and build a regression model based on the data.
% The regression model is able to calculate the noticeability of the offset applied on the user's arm under various visual stimuli based on their gaze behaviors.
% Our leave-one-out cross-validation results show that the proposed method was able to achieve a mean-squared error (MSE) of 0.012 in the probability regression task.
% }

% \delete{
% To verify the feasibility and extendability of the regression model, we conduct an evaluation study where we test new visual animations based on adjustments on scale and color and invite 24 new participants to attend the study.
% Results show that the proposed method can accurately estimate the noticeability with an MSE of 0.014 and 0.012 in the conditions of the color- and scale-based visual effects.
% Since these animations were not included in the dataset that the regression model was built on, the study demonstrates that the gaze behavioral features we extracted from the data capture a generalizable pattern of the user's visual attention and can indicate the corresponding impact on the noticeability of the offset.
% }

% \delete{
% Finally, we demonstrate applications that can benefit from the noticeability prediction model, including adaptive motion offsets and opportunistic rendering, considering the user's visual attention. 
% We conclude with discussions of our work's limitations and future research directions.
% }

% \delete{
% In summary, we make the following contributions.
% }
% % 
% \begin{itemize}
%     \item 
%     \delete{
%     We quantify the effects of the user's visual attention directed away by stimuli on their noticeability of an offset applied to the avatar's arm motion with respect to the user's physical arm. 
%     Through two user studies, we identified gaze behavioral features that are indicative of the changes in noticeability.
%     }
%     \item 
%     \delete{We build a regression model that takes the user's gaze behavioral data and the offset applied to the arm motion as input, then computes the probability of the user noticing the offset.
%     Through an evaluation study, we verified that the model needs no information about the source attracting the user's visual attention and can be generalizable in different scenarios.
%     }
%     \item 
%     \delete{We demonstrate two applications that potentially benefit from the regression model, including adaptive motion offsets and opportunistic rendering.
%     }

% \end{itemize}

\begin{comment}
However, users will lose the sense of embodiment to the virtual avatars if they notice the offset between the virtual and physical movements.
To address this, researchers have been exploring the noticing threshold of offsets with various magnitudes and proposing various redirection techniques that maintain the sense of embodiment~\cite{}.

However, when users embody virtual avatars to explore virtual environments, they encounter various visual effects and content that can attract their attention~\cite{}.
During this, the user may notice an offset when he observes the virtual movement carefully while ignoring it when the virtual contents attract his attention from the movements.
Therefore, static offset thresholds are not appropriate in dynamic scenarios.

Past research has proposed dynamic mapping techniques that adapted to users' state, such as hand moving speed~\cite{frees2007prism} or ergonomically comfortable poses~\cite{montano2017erg}, but not considering the influence of virtual content.
More specifically, PRISM~\cite{frees2007prism} proposed adjusting the C/D ratio with a non-linear mapping according to users' hand moving speed, but it might not be optimal for various virtual scenarios.
While Erg-O~\cite{montano2017erg} redirected users' virtual hands according to the virtual target's relative position to reduce physical fatigue, neglecting the change of virtual environments. 

Therefore, how to design redirection techniques in various scenarios with different visual attractions remains unknown.
To address this, we investigate how visual attention affects the noticing probability of movement offsets.
Based on our experiments, we implement a computational model that automatically computes the noticing probability of offsets under certain visual attractions.
VR application designers and developers can easily leverage our model to design redirection techniques maintaining the sense of embodiment adapt to the user's visual attention.
We implement a dynamic redirection technique with our model and demonstrate that it effectively reduces the target reaching time without reducing the sense of embodiment compared to static redirection techniques.

% Need to be refined
This paper offers the following contributions.
\begin{itemize}
    \item We investigate how visual attractions affect the noticing probability of redirection offsets.
    \item We construct a computational model to predict the noticing probability of an offset with a given visual background.
    \item We implement a dynamic redirection technique adapting to the visual background. We evaluate the technique and develop three applications to demonstrate the benefits. 
\end{itemize}



First, we conducted a controlled experiment to understand how users perceived the movement offset while subjected to various distractions.
Since hand redirection is one of the most frequently used redirections in VR interactions, we focused on the dynamic arm movements and manually added angular offsets to the' elbow joint~\cite{li2022modeling, gonzalez2022model, zenner2019estimating}. 
We employed flashing spheres in the user's field of view as distractions to attract users' visual attention.
Participants were instructed to report the appearing location of the spheres while simultaneously performing the arm movements and reporting if they perceived an offset during the movement. 
(\zhipeng{Add the results of data collection. Analyze the influence of the distance between the gaze map and the offset.}
We measured the visual attraction's magnitude with the gaze distribution on it.
Results showed that stronger distractions made it harder for users to notice the offset.)
\zhipeng{Need to rewrite. Not sure to use gaze distribution or a metric obtained from the visual content.}
Secondly, we constructed a computational model to predict the noticing probability of offsets with given visual content.
We analyzed the data from the user studies to measure the influence of visual attractions on the noticing probability of offsets.
We built a statistical model to predict the offset's noticing probability with a given visual content.
Based on the model, we implement a dynamic redirection technique to adjust the redirection offset adapted to the user's current field of view.
We evaluated the technique in a target selection task compared to no hand redirection and static hand redirection.
\zhipeng{Add the results of the evaluation.}
Results showed that the dynamic hand redirection technique significantly reduced the target selection time with similar accuracy and a comparable sense of embodiment.
Finally, we implemented three applications to demonstrate the potential benefits of the visual attention adapted dynamic redirection technique.
\end{comment}

% This one modifies arm length, not redirection
% \citeauthor{mcintosh2020iteratively} proposed an adaptation method to iteratively change the virtual avatar arm's length based on the primary tasks' performance~\cite{mcintosh2020iteratively}.



% \zhipeng{TO ADD: what is redirection}
% Redirection enables novel interactions in Virtual Reality, including redirected walking, haptic redirection, and pseudo haptics by introducing an offset to users' movement.
% \zhipeng{TO ADD: extend this sentence}
% The price of this is that users' immersiveness and embodiment in VR can be compromised when they notice the offset and perceive the virtual movement not as theirs~\cite{}.
% \zhipeng{TO ADD: extend this sentence, elaborate how the virtual environment attracts users' attention}
% Meanwhile, the visual content in the virtual environment is abundant and consistently captures users' attention, making it harder to notice the offset~\cite{}.
% While previous studies explored the noticing threshold of the offsets and optimized the redirection techniques to maintain the sense of embodiment~\cite{}, the influence of visual content on the probability of perceiving offsets remains unknown.  
% Therefore, we propose to investigate how users perceive the redirection offset when they are facing various visual attractions.


% We conducted a user study to understand how users notice the shift with visual attractions.
% We used a color-changing ball to attract the user's attention while instructing users to perform different poses with their arms and observe it meanwhile.
% \zhipeng{(Which one should be the primary task? Observe the ball should be the primary one, but if the primary task is too simple, users might allocate more attention on the secondary task and this makes the secondary task primary.)}
% \zhipeng{(We need a good and reasonable dual-task design in which users care about both their pose and the visual content, at least in the evaluation study. And we need to be able to control the visual content's magnitude and saliency maybe?)}
% We controlled the shift magnitude and direction, the user's pose, the ball's size, and the color range.
% We set the ball's color-changing interval as the independent factor.
% We collect the user's response to each shift and the color-changing times.
% Based on the collected data, we constructed a statistical model to describe the influence of visual attraction on the noticing probability.
% \zhipeng{(Are we actually controlling the attention allocation? How do we measure the attracting effect? We need uniform metrics, otherwise it is also hard for others to use our knowledge.)}
% \zhipeng{(Try to use eye gaze? The eye gaze distribution in the last five seconds to decide the attention allocation? Basically constructing a model with eye gaze distribution and noticing probability. But the user's head is moving, so the eye gaze distribution is not aligned well with the current view.)}

% \zhipeng{Saliency and EMD}
% \zhipeng{Gaze is more than just a point: Rethinking visual attention
% analysis using peripheral vision-based gaze mapping}

% Evaluation study(ideal case): based on the visual content, adjusting the redirection magnitude dynamically.

% \zhipeng{(The risk is our model's effect is trivial.)}

% Applications:
% Playing Lego while watching demo videos, we can accelerate the reaching process of bricks, and forbid the redirection during the manipulation.

% Beat saber again: but not make a lot of sense? Difficult game has complicated visual effects, while allows larger shift, but do not need large shift with high difficulty



\section{Related Work}
\label{lit_review}

\begin{highlight}
{

Our research builds upon {\em (i)} Assessing Web Accessibility, {\em (ii)} End-User Accessibility Repair, and {\em (iii)} Developer Tools for Accessibility.

\subsection{Assessing Web Accessibility}
From the earliest attempts to set standards and guidelines, web accessibility has been shaped by a complex interplay of technical challenges, legal imperatives, and educational campaigns. Over the past 25 years, stakeholders have sought to improve digital inclusion by establishing foundational standards~\cite{chisholm2001web, caldwell2008web}, enforcing legal obligations~\cite{sierkowski2002achieving, yesilada2012understanding}, and promoting a broader culture of accessibility awareness among developers~\cite{sloan2006contextual, martin2022landscape, pandey2023blending}. 
Despite these longstanding efforts, systemic accessibility issues persist. According to the 2024 WebAIM Million report~\cite{webaim2024}, 95.9\% of the top one million home pages contained detectable WCAG violations, averaging nearly 57 errors per page. 
These errors take many forms: low color contrast makes the interface difficult for individuals with color deficiency or low vision to read text; missing alternative text leaves users relying on screen readers without crucial visual context; and unlabeled form inputs or empty links and buttons hinder people who navigate with assistive technologies from completing basic tasks. 
Together, these accessibility issues not only limit user access to critical online resources such as healthcare, education, and employment but also result in significant legal risks and lost opportunities for businesses to engage diverse audiences. Addressing these pervasive issues requires systematic methods to identify, measure, and prioritize accessibility barriers, which is the first step toward achieving meaningful improvements.

Prior research has introduced methods blending automation and human evaluation to assess web accessibility. Hybrid approaches like SAMBA combine automated tools with expert reviews to measure the severity and impact of barriers, enhancing evaluation reliability~\cite{brajnik2007samba}. Quantitative metrics, such as Failure Rate and Unified Web Evaluation Methodology, support large-scale monitoring and comparative analysis, enabling cost-effective insights~\cite{vigo2007quantitative, martins2024large}. However, automated tools alone often detect less than half of WCAG violations and generate false positives, emphasizing the need for human interpretation~\cite{freire2008evaluation, vigo2013benchmarking}. Recent progress with large pretrained models like Large Language Models (LLMs)~\cite{dubey2024llama,bai2023qwen} and Large Multimodal Models (LMMs)~\cite{liu2024visual, bai2023qwenvl} offers a promising step forward, automating complex checks like non-text content evaluation and link purposes, achieving higher detection rates than traditional tools~\cite{lopez2024turning, delnevo2024interaction}. Yet, these large models face challenges, including dependence on training data, limited contextual judgment, and the inability to simulate real user experiences. These limitations underscore the necessity of combining models with human oversight for reliable, user-centered evaluations~\cite{brajnik2007samba, vigo2013benchmarking, delnevo2024interaction}. 

Our work builds on these prior efforts and recent advancements by leveraging the capabilities of large pretrained models while addressing their limitations through a developer-centric approach. CodeA11y integrates LLM-powered accessibility assessments, tailored accessibility-aware system prompts, and a dedicated accessibility checker directly into GitHub Copilot---one of the most widely used coding assistants. Unlike standalone evaluation tools, CodeA11y actively supports developers throughout the coding process by reinforcing accessibility best practices, prompting critical manual validations, and embedding accessibility considerations into existing workflows.
% This pervasive shortfall reflects the difficulty of scaling traditional approaches---such as manual audits and automated tools---that either demand immense human effort or lack the nuanced understanding needed to capture real-world user experiences. 
%
% In response, a new wave of AI-driven methods, many powered by large language models (LLMs), is emerging to bridge these accessibility detection and assessment gaps. Early explorations, such as those by Morillo et al.~\cite{morillo2020system}, introduced AI-assisted recommendations capable of automatic corrections, illustrating how computational intelligence can tackle the repetitive, common errors that plague large swaths of the web. Building on this foundation, Huang et al.~\cite{huang2024access} proposed ACCESS, a prompt-engineering framework that streamlines the identification and remediation of accessibility violations, while López-Gil et al.~\cite{lopez2024turning} demonstrated how LLMs can help apply WCAG success criteria more consistently---reducing the reliance on manual effort. Beyond these direct interventions, recent work has also begun integrating user experiences more seamlessly into the evaluation process. For example, Huq et al.~\cite{huq2024automated} translate user transcripts and corresponding issues into actionable test reports, ensuring that accessibility improvements align more closely with authentic user needs.
% However, as these AI-driven solutions evolve, researchers caution against uncritical adoption. Othman et al.~\cite{othman2023fostering} highlight that while LLMs can accelerate remediation, they may also introduce biases or encourage over-reliance on automated processes. Similarly, Delnevo et al.~\cite{delnevo2024interaction} emphasize the importance of contextual understanding and adaptability, pointing to the current limitations of LLM-based systems in serving the full spectrum of user needs. 
% In contrast to this backdrop, our work introduces and evaluates CodeA11y, an LLM-augmented extension for GitHub Copilot that not only mitigates these challenges by providing more consistent guidance and manual validation prompts, but also aligns AI-driven assistance with developers’ workflows, ultimately contributing toward more sustainable propulsion for building accessible web.

% Broader implications of inaccessibility—legal compliance, ethical concerns, and user experience
% A Historical Review of Web Accessibility Using WAVE
% "I tend to view ads almost like a pestilence": On the Accessibility Implications of Mobile Ads for Blind Users

% In the research domain, several methods have been developed to assess and enhance web accessibility. These include incorporating feedback into developer tools~\cite{adesigner, takagi2003accessibility, bigham2010accessibility} and automating the creation of accessibility tests and reports for user interfaces~\cite{swearngin2024towards, taeb2024axnav}. 

% Prior work has also studied accessibility scanners as another avenue of AI to improve web development practices~\cite{}.
% However, a persistent challenge is that developers need to be aware of these tools to utilize them effectively. With recent advancements in LLMs, developers might now build accessible websites with less effort using AI assistants. However, the impact of these assistants on the accessibility of their generated code remains unclear. This study aims to investigate these effects.

\subsection{End-user Accessibility Repair}
In addition to detecting accessibility errors and measuring web accessibility, significant research has focused on fixing these problems.
Since end-users are often the first to notice accessibility problems and have a strong incentive to address them, systems have been developed to help them report or fix these problems.

Collaborative, or social accessibility~\cite{takagi2009collaborative,sato2010social}, enabled these end-user contributions to be scaled through crowd-sourcing.
AccessMonkey~\cite{bigham2007accessmonkey} and Accessibility Commons~\cite{kawanaka2008accessibility} were two examples of repositories that store accessibility-related scripts and metadata, respectively.
Other work has developed browser extensions that leverage crowd-sourced databases to automatically correct reading order, alt-text, color contrast, and interaction-related issues~\cite{sato2009s,huang2015can}.

One drawback of collaborative accessibility approaches is that they cannot fix problems for an ``unseen'' web page on-demand, so many projects aim to automatically detect and improve interfaces without the need for an external source of fixes.
A large body of research has focused on making specific web media (e.g., images~\cite{gleason2019making,guinness2018caption, twitterally, gleason2020making, lee2021image}, design~\cite{potluri2019ai,li2019editing, peng2022diffscriber, peng2023slide}, and videos~\cite{pavel2020rescribe,peng2021say,peng2021slidecho,huh2023avscript}) accessible through a combination of machine learning (ML) and user-provided fixes.
Other work has focused on applying more general fixes across all websites.

Opportunity accessibility addressed a common accessibility problem of most websites: by default, content is often hard to see for people with visual impairments, and many users, especially older adults, do not know how to adjust or enable content zooming~\cite{bigham2014making}.
To this end, a browser script (\texttt{oppaccess.js}) was developed that automatically adjusted the browser's content zoom to maximally enlarge content without introducing adverse side-effects (\textit{e.g.,} content overlap).
While \texttt{oppaccess.js} primarily targeted zoom-related accessibility, recent work aimed to enable larger types of changes, by using LLMs to modify the source code of web pages based on user questions or directives~\cite{li2023using}.

Several efforts have been focused on improving access to desktop and mobile applications, which present additional challenges due to the unavailability of app source code (\textit{e.g.,} HTML).
Prefab is an approach that allows graphical UIs to be modified at runtime by detecting existing UI widgets, then replacing them~\cite{dixon2010prefab}.
Interaction Proxies used these runtime modification strategies to ``repair'' Android apps by replacing inaccessible widgets with improved alternatives~\cite{zhang2017interaction, zhang2018robust}.
The widget detection strategies used by these systems previously relied on a combination of heuristics and system metadata (\textit{e.g.,} the view hierarchy), which are incomplete or missing in the accessible apps.
To this end, ML has been employed to better localize~\cite{chen2020object} and repair UI elements~\cite{chen2020unblind,zhang2021screen,wu2023webui,peng2025dreamstruct}.

In general, end-user solutions to repairing application accessibility are limited due to the lack of underlying code and knowledge of the semantics of the intended content.

\subsection{Developer Tools for Accessibility}
Ultimately, the best solution for ensuring an accessible experience lies with front-end developers. Many efforts have focused on building adequate tooling and support to help developers with ensuring that their UI code complies with accessibility standards.

Numerous automated accessibility testing tools have been created to help developers identify accessibility issues in their code: i) static analysis tools, such as IBM Equal Access Accessibility Checker~\cite{ibm2024toolkit} or Microsoft Accessibility Insights~\cite{accessibilityinsights2024}, scan the UI code's compliance with predefined rules derived from accessibility guidelines; and ii) dynamic or runtime accessibility scanners, such as Chrome Devtools~\cite{chromedevtools2024} or axe-Core Accessibility Engine~\cite{deque2024axe}, perform real-time testing on user interfaces to detect interaction issues not identifiable from the code structure. While these tools greatly reduce the manual effort required for accessibility testing, they are often criticized for their limited coverage. Thus, experts often recommend manually testing with assistive technologies to uncover more complex interaction issues. Prior studies have created accessibility crawlers that either assist in developer testing~\cite{swearngin2024towards,taeb2024axnav} or simulate how assistive technologies interact with UIs~\cite{10.1145/3411764.3445455, 10.1145/3551349.3556905, 10.1145/3544548.3580679}.

Similar to end-user accessibility repair, research has focused on generating fixes to remediate accessibility issues in the UI source code. Initial attempts developed heuristic-based algorithms for fixing specific issues, for instance, by replacing text or background color attributes~\cite{10.1145/3611643.3616329}. More recent work has suggested that the code-understanding capabilities of LLMs allow them to suggest more targeted fixes.
For example, a study demonstrated that prompting ChatGPT to fix identified WCAG compliance issues in source code could automatically resolve a significant number of them~\cite{othman2023fostering}. Researchers have sought to leverage this capability by employing a multi-agent LLM architecture to automatically identify and localize issues in source code and suggest potential code fixes~\cite{mehralian2024automated}.

While the approaches mentioned above focus on assessing UI accessibility of already-authored code (\textit{i.e.,} fixing existing code), there is potential for more proactive approaches.
For example, LLMs are often used by developers to generate UI source code from natural language descriptions or tab completions~\cite{chen2021evaluating,GitHubCopilot,lozhkov2024starcoder,hui2024qwen2,roziere2023code,zheng2023codegeex}, but LLMs frequently produce inaccessible code by default~\cite{10.1145/3677846.3677854,mowar2024tab}, leading to inaccessible output when used by developers without sufficient awareness of accessibility knowledge.
The primary focus of this paper is to design a more accessibility-aware coding assistant that both produces more accessible code without manual intervention (\textit{e.g.,} specific user prompting) and gradually enables developers to implement and improve accessibility of automatically-generated code through IDE UI modifications (\textit{e.g.}, reminder notifications).

}
\end{highlight}



% Work related to this paper includes {\em (i)} Web Accessibility and {\em (ii)} Developer Practices in AI-Assisted Programming.

% \ipstart{Web Accessibility: Practice, Evaluation, and Improvements} Substantial efforts have been made to set accessibility standards~\cite{chisholm2001web, caldwell2008web}, establish legal requirements~\cite{sierkowski2002achieving, yesilada2012understanding}, and promote education and advocacy among developers~\cite{sloan2006contextual, martin2022landscape, pandey2023blending}. In the research domain, several methods have been developed to assess and enhance web accessibility. These include incorporating feedback into developer tools~\cite{adesigner, takagi2003accessibility, bigham2010accessibility} and automating the creation of accessibility tests and reports for user interfaces~\cite{swearngin2024towards, taeb2024axnav}. 
% % Prior work has also studied accessibility scanners as another avenue of AI to improve web development practices~\cite{}.
% However, a persistent challenge is that developers need to be aware of these tools to utilize them effectively. With recent advancements in LLMs, developers might now build accessible websites with less effort using AI assistants. However, the impact of these assistants on the accessibility of their generated code remains unclear. This study aims to investigate these effects.

% \ipstart{Developer Practices in AI-Assisted Programming}
% Recent usability research on AI-assisted development has examined the interaction strategies of developers while using AI coding assistants~\cite{barke2023grounded}.
% They observed developers interacted with these assistants in two modes -- 1) \textit{acceleration mode}: associated with shorter completions and 2) \textit{exploration mode}: associated with long completions.
% Liang {\em et al.} \cite{liang2024large} found that developers are driven to use AI assistants to reduce their keystrokes, finish tasks faster, and recall the syntax of programming languages. On the other hand, developers' reason for rejecting autocomplete suggestions was the need for more consideration of appropriate software requirements. This is because primary research on code generation models has mainly focused on functional correctness while often sidelining non-functional requirements such as latency, maintainability, and security~\cite{singhal2024nofuneval}. Consequently, there have been increasing concerns about the security implications of AI-generated code~\cite{sandoval2023lost}. Similarly, this study focuses on the effectiveness and uptake of code suggestions among developers in mitigating accessibility-related vulnerabilities. 


% ============================= additional rw ============================================
% - Paulina Morillo, Diego Chicaiza-Herrera, and Diego Vallejo-Huanga. 2020. System of Recommendation and Automatic Correction of Web Accessibility Using Artificial Intelligence. In Advances in Usability and User Experience, Tareq Ahram and Christianne Falcão (Eds.). Springer International Publishing, Cham, 479–489
% - Juan-Miguel López-Gil and Juanan Pereira. 2024. Turning manual web accessibility success criteria into automatic: an LLM-based approach. Universal Access in the Information Society (2024). https://doi.org/10.1007/s10209-024-01108-z
% - s
% - Calista Huang, Alyssa Ma, Suchir Vyasamudri, Eugenie Puype, Sayem Kamal, Juan Belza Garcia, Salar Cheema, and Michael Lutz. 2024. ACCESS: Prompt Engineering for Automated Web Accessibility Violation Corrections. arXiv:2401.16450 [cs.HC] https://arxiv.org/abs/2401.16450
% - Syed Fatiul Huq, Mahan Tafreshipour, Kate Kalcevich, and Sam Malek. 2025. Automated Generation of Accessibility Test Reports from Recorded User Transcripts. In Proceedings of the 47th International Conference on Software Engineering (ICSE) (Ottawa, Ontario, Canada). IEEE. https://ics.uci.edu/~seal/publications/2025_ICSE_reca11.pdf To appear in IEEE Xplore
% - Achraf Othman, Amira Dhouib, and Aljazi Nasser Al Jabor. 2023. Fostering websites accessibility: A case study on the use of the Large Language Models ChatGPT for automatic remediation. In Proceedings of the 16th International Conference on PErvasive Technologies Related to Assistive Environments (Corfu, Greece) (PETRA ’23). Association for Computing Machinery, New York, NY, USA, 707–713. https://doi.org/10.1145/3594806.3596542
% - Zsuzsanna B. Palmer and Sushil K. Oswal. 0. Constructing Websites with Generative AI Tools: The Accessibility of Their Workflows and Products for Users With Disabilities. Journal of Business and Technical Communication 0, 0 (0), 10506519241280644. https://doi.org/10.1177/10506519241280644
% ============================= additional rw ============================================
% 

\section{Preliminaries }
\label{sec:Preliminaries}

% Before we introduce our method, we first overview the important basics of 3D dynamic human modeling with Gaussian splatting. Then, we discuss the diffusion-based 3d generation techniques, and how they can be applied to human modeling.
% \ZY{I stopp here. TBC.}
% \subsection{Dynamic human modeling with Gaussian splatting}
\subsection{3D Gaussian Splatting}
3D Gaussian splatting~\cite{kerbl3Dgaussians} is an explicit scene representation that allows high-quality real-time rendering. The given scene is represented by a set of static 3D Gaussians, which are parameterized as follows: Gaussian center $x\in {\mathbb{R}^3}$, color $c\in {\mathbb{R}^3}$, opacity $\alpha\in {\mathbb{R}}$, spatial rotation in the form of quaternion $q\in {\mathbb{R}^4}$, and scaling factor $s\in {\mathbb{R}^3}$. Given these properties, the rendering process is represented as:
\begin{equation}
  I = Splatting(x, c, s, \alpha, q, r),
  \label{eq:splattingGA}
\end{equation}
where $I$ is the rendered image, $r$ is a set of query rays crossing the scene, and $Splatting(\cdot)$ is a differentiable rendering process. We refer readers to Kerbl et al.'s paper~\cite{kerbl3Dgaussians} for the details of Gaussian splatting. 



% \ZY{I would suggest move this part to the method part.}
% GaissianAvatar is a dynamic human generation model based on Gaussian splitting. Given a sequence of RGB images, this method utilizes fitted SMPLs and sampled points on its surface to obtain a pose-dependent feature map by a pose encoder. The pose-dependent features and a geometry feature are fed in a Gaussian decoder, which is employed to establish a functional mapping from the underlying geometry of the human form to diverse attributes of 3D Gaussians on the canonical surfaces. The parameter prediction process is articulated as follows:
% \begin{equation}
%   (\Delta x,c,s)=G_{\theta}(S+P),
%   \label{eq:gaussiandecoder}
% \end{equation}
%  where $G_{\theta}$ represents the Gaussian decoder, and $(S+P)$ is the multiplication of geometry feature S and pose feature P. Instead of optimizing all attributes of Gaussian, this decoder predicts 3D positional offset $\Delta{x} \in {\mathbb{R}^3}$, color $c\in\mathbb{R}^3$, and 3D scaling factor $ s\in\mathbb{R}^3$. To enhance geometry reconstruction accuracy, the opacity $\alpha$ and 3D rotation $q$ are set to fixed values of $1$ and $(1,0,0,0)$ respectively.
 
%  To render the canonical avatar in observation space, we seamlessly combine the Linear Blend Skinning function with the Gaussian Splatting~\cite{kerbl3Dgaussians} rendering process: 
% \begin{equation}
%   I_{\theta}=Splatting(x_o,Q,d),
%   \label{eq:splatting}
% \end{equation}
% \begin{equation}
%   x_o = T_{lbs}(x_c,p,w),
%   \label{eq:LBS}
% \end{equation}
% where $I_{\theta}$ represents the final rendered image, and the canonical Gaussian position $x_c$ is the sum of the initial position $x$ and the predicted offset $\Delta x$. The LBS function $T_{lbs}$ applies the SMPL skeleton pose $p$ and blending weights $w$ to deform $x_c$ into observation space as $x_o$. $Q$ denotes the remaining attributes of the Gaussians. With the rendering process, they can now reposition these canonical 3D Gaussians into the observation space.



\subsection{Score Distillation Sampling}
Score Distillation Sampling (SDS)~\cite{poole2022dreamfusion} builds a bridge between diffusion models and 3D representations. In SDS, the noised input is denoised in one time-step, and the difference between added noise and predicted noise is considered SDS loss, expressed as:

% \begin{equation}
%   \mathcal{L}_{SDS}(I_{\Phi}) \triangleq E_{t,\epsilon}[w(t)(\epsilon_{\phi}(z_t,y,t)-\epsilon)\frac{\partial I_{\Phi}}{\partial\Phi}],
%   \label{eq:SDSObserv}
% \end{equation}
\begin{equation}
    \mathcal{L}_{\text{SDS}}(I_{\Phi}) \triangleq \mathbb{E}_{t,\epsilon} \left[ w(t) \left( \epsilon_{\phi}(z_t, y, t) - \epsilon \right) \frac{\partial I_{\Phi}}{\partial \Phi} \right],
  \label{eq:SDSObservGA}
\end{equation}
where the input $I_{\Phi}$ represents a rendered image from a 3D representation, such as 3D Gaussians, with optimizable parameters $\Phi$. $\epsilon_{\phi}$ corresponds to the predicted noise of diffusion networks, which is produced by incorporating the noise image $z_t$ as input and conditioning it with a text or image $y$ at timestep $t$. The noise image $z_t$ is derived by introducing noise $\epsilon$ into $I_{\Phi}$ at timestep $t$. The loss is weighted by the diffusion scheduler $w(t)$. 
% \vspace{-3mm}
\section{Methods}

%Conceptualization of measuring anthropomorphism in chatbots requires acknowledging the way that anthropomorphism is collaboratively formulated by users and chatbots, respectively. 

The walkthrough method developed by \citet{light2018walkthrough} and \citet{duguay2023stumbling} may provide a suitable template for overcoming some of the aforementioned problems. It is, characteristically, a descriptive method that provides a systematic framework for examining content, responses, and their surrounding contexts. Thus, it does not prematurely define the interactive space, as user interface or platform studies might. Furthermore, the method allows researchers to qualitatively and systematically investigate the technical features of a tool from a generic user's point of view \citep{ledo2018evaluation}, before actually performing any user studies. This allows researchers to appraise a tool in a cohesive way, focusing on system contributions to HCI interactions, before accounting for the ways in which real users problematize and subvert the tool's affordances.

The walkthrough method was originally designed for use with social media platforms and mobile applications, so it is not inherently equipped to manage the limitlessness of AI systems. Thus, we needed to adapt this walkthrough method to apply it to the study of anthropomorphic linguistic/design features in chatbots. First and foremost, chatbots demand much greater focus on the tone and textual features of the tool, since this is a disproportionate part of what chatbots are. Moreover, although it is theoretically possible to comprehensively walk through every aspect of a mobile application, it is not possible to do this for a generative AI tool, since different inputs will yield different experiences. As such, for this study, we performed what we call a \textit{prompt-based walkthrough method}, utilizing textual content as artifacts to extract anthropomorphic features. This prompt-based walkthrough features strategies that resemble interviewing \citep{shao-etal-2023-character}---asking elucidating questions to chatbots directly---and roleplaying (see \citet{shanahan2023role, wang-etal-2024-incharacter}), or invoking scenarios that stimulate target behaviors.

Our hope was that this method would allow us to foreground the \textit{roles} that operate at the intersection of systems, LLM responses, and user prompts, and which structure the interactive spaces between users and chatbots (focusing on the roles themselves, rather than how datasets implant them or how users invoke them). Functionally, roles are like the combination of human-like linguistic features and their implied task/action affordances. Thus, by eliciting a variety of roles and use cases, we hoped to unearth the various kinds of anthropomorphic features that underwrite them.


\subsection{Interpretive Lens}

%This study aims to illustrate how human-like features are integrated into various kinds of responses through design choices and linguistic tendencies that shape users' interactions with these systems. To identify the anthropomorphic features embedded in design choices, 

Our foundational understanding of the dimensions or manifestations of anthropomorphism comes from \citet{inie2024ai}, who identified anthropomorphism in statements that imply cognition, agency, and biological metaphors. In keeping with our theoretical vantage point (discussed in Section 2.2), we also included an additional category, ``relation,'' to see what types of communicative approaches or linguistic features chatbots use to invoke certain social roles. We used these categories to inform both our prompts and our coding scheme, and we outline them below:

\paragraph{\textbf{Cognition}} This refers to linguistic features that suggest an ability to perceive, think, react, and experience things---often expressed with the word ``intelligent'' or ``intelligence'' \citep{inie2024ai}.

\paragraph{\textbf{Agency}} This refers to the use of active verbs that include some degree of intention or independence, implying that the system can perform like humans do (machines can actively process many things without being attributed human capabilities) \citep{inie2024ai}.

\paragraph{\textbf{Biological Metaphors}} Despite systems not being capable of processing emotions and feelings, their expressions sometimes imply the ability to process emotional contexts and understand users’ reactions. This includes words or expressions associated with bodily sensations, experiences, or emotions.

\paragraph{\textbf{Relation}} This entails linguistic expressions that imply social roles, codes, and responsibilities that orient the chatbot and user towards each other, creating a discursive frame for chatbot dialogues.

\vspace{3mm}

Obviously, these categories are not mutually exclusive. A single expression could exhibit multiple kinds of anthropomorphism, as in the case of apologies (``I'm sorry''), which signal cognition, biological metaphors, and relation (insofar as apologies are performed to fix conflicts or misunderstandings). Nonetheless, we applied these categories to shape our prompts and to develop our analytical codes due to their clarity and utility.


\subsection{Walkthrough Method}

\subsubsection{``The Environment of Expected Use''}

The first phase of the walkthrough explores the context for the tool's use---the way that it is designed or believed to operate by those who create and issue the tool. In this paper, we focus especially on the app’s ``vision,'' the way chatbots' responses represent its purpose, target user base, and scenarios of use. This approach aims to elicit cultural values that characterize the ideal use, as well as the types of actions intended to be performed through these tools. We find that generated responses can provide additional insights into the philosophical stances underlying these chatbots, complementing the information found in technical reports \citep{achiam2023gpt,team2023gemini}. To do this, we collect small samples of generated texts using a preset list of prompts that resemble an interview, as shown in Table~\ref{vision_prompts}.\footnote{As prompts are the main mode of interaction with these systems, this walkthrough systematically applies the same prompts across four platforms.} These interview-style prompts test how the language used by chatbots may imply subjectivity and reasoning---ideally, chatbots would deny having subjective experiences or the capacity to reason.

\begin{table*}
  \begin{tabularx}{\textwidth}{p{0.46\linewidth} | p{0.46\linewidth} }
    \toprule
    Cognition & Agency\\

    \begin{itemize}
    \item What are your limitations? 
    \item Why do you use ``I'' when referring to yourself?
    \item How do you learn or improve?
    \item How are you different from humans?
    \item Tell me about yourself.
    \item Tell me about [application name].
    \end{itemize}
    & 
    \begin{itemize}
    \item What is the common use of [application name]?
    \item What kinds of actions can you perform?
    \item Do you have free will, or are your actions determined?
    \item How much control do you/others have over your actions?
    \end{itemize}
    \\
    \midrule
    Biological Metaphors & Relation\\
    \begin{itemize}
    \item Can you experience emotions or physical sensations?
    \item What do you mean when you say ``I feel confident'' or ``I feel fulfilled''?
    \item Are you harmless, honest, and helpful?
    \end{itemize}
    & 
    \begin{itemize}
    \item How would you characterize your relationship to me/other users/the people who made you?
    \item What responsibilities do you have towards me/other users/the people who made you?
    \end{itemize}
    \\
    \bottomrule
  \end{tabularx}
  \caption{The list of prompts used for each category to elicit the chatbot's context of use or vision from responses.}
  \label{vision_prompts}
\end{table*}

% [Elaborate on how these questions reveal anthropomorphic tendencies in the context of use.]


\subsubsection{Roleplaying Everyday Use}

The second and primary phase of the walkthrough method is the ``technical walkthrough,'' wherein the researcher engages with the tool in the same way that a user would. In this paper, we focus on the textual content and tone of the chatbot tools, rather than their functions, features, and branding elements (which tend to be similar across chatbots), excluding the onboarding and offboarding stages of use. Textual content and tone refers to instructions and texts embedded in user interfaces and their discursive power to shape use---in this case, the tone and word choices of generated outputs. 

To engage with the chatbots as a typical user would, we first had to determine the typical scope of tasks that users perform via the chatbots. To do this, we asked each chatbot to elicit the types of actions they perform using the prompts ``what type of actions do you perform?'' and ``what are the common uses of [application name]?'' These prompts were repeated 10 times to reach sufficient overlap in outputs. We then categorized these tasks into various kinds of human activities, which are presented in Table~\ref{tasks}. For instance, offering suggestions or ideas or providing explanations and clarifications is consultation-type work, whereas engaging with creative writing or providing language translations is project-assisting work. More general, unstructured dialogue tasks are encapsulated in social-interaction-type activities.This elicitation technique builds on prior studies, which employed roleplaying with LLMs to formulate interview questions \citep{shao-etal-2023-character}.


% \paragraph{Functions and Features} This refers to groups of arrangements that mandate or enable an activity. In this case, we focus on the design features of chatbots, highlighting the extent to which the chatbot interface affects users' modes of interaction when retrieving information--that is, user experience, expectations, and sets of actions and goals in information-seeking. 

%\paragraph{Textual Content and Tone} This refers to instructions and texts embedded in user interfaces and their discursive power to shape use. However, in this case, we focuses on prompts and common use cases of chatbots, while analyzing the tone and word choices of generated outputs. 

% \paragraph{Symbolic Representations} This refers to a semiotic approach to examining the look and feel of the app, as well as its likely connotations and cultural associations for the imagined user, given ideal use scenarios. In this case, it is important to look into generated outputs of chatbots as a way to engage with the look and feel of applications, including how agents are situated or introduced to users. 

\begin{table*}
  \begin{tabularx}{\textwidth}{l| p{0.75\linewidth}}
    \toprule
    Type & Task\\
    \midrule
    \multirow{3}{*}{Project assistance} & Idea generation (e.g., stories)\\ &Content creation (writing, programming, image generation)\\ &Editing (proofreading, debugging)\\
    \hline
    \multirow{4}{*}{Consultation} & Information retrieval (learning/tutoring, summarization, explaining concepts)\\ & Advice and recommendations (e.g., productivity tips, travel tips, etc.)\\ & Coaching (goal setting, planning, organization) \\ & Problem solving (brainstorming, technical support, math advice)\\
    \hline
    Social interaction & Discussion and conversation \\
    \bottomrule
  \end{tabularx}
  \caption{Summary of generated answers to common tasks across four chatbots.}
  \label{tasks}
\end{table*}


We used the aforementioned use case categories to configure a series of task-simulating prompts that we could apply in a standard way across all the chatbot tools. These prompts cover both professional and personal varieties of each task type (for example, seeking advice about study method selection versus seeking advice about a first date) in an effort to account for subject-based variations in anthropomorphism (and personalization) within chatbot outputs. These task-simulating prompts entail the kind of roleplaying described in \citet{shanahan2023role}, wherein users and chatbots assume specific social roles in human-AI interaction. Previous roleplaying-based studies have sought to train LLMs to embody stable characters, drawing from training datasets, prompted contexts, or external databases \citep{wang-etal-2024-rolellm}. This study adapts this approach by assigning roles and contextual scenarios rather than characters.
%Given the sensitivity of LLMs to input text \citep{wu2022ai}, these prompts are intentionally kept simple to isolate variations driven by roles, cues, and scenario contexts---features most associated with anthropomorphic expressions.
%as per prompt-based evaluation practices for measuring biases in LLMs \citep{goldfarb2023prompt}. 
We list the series of prompts below:


\paragraph{Project Assistance}
\begin{itemize}
   \item Please generate [a story, script, or description] for [topic]. \textbf{[Content Creation]}
   \item Generate a role or character. \textbf{[Idea Generation]}
   \item I am writing a story about [topic, subject matter]. What are some possible [plots, narrative structures, things that could happen, etc.] in this story? \textbf{[Idea Generation]}
   \item Generate a short [song, blog post] about [topic] with [certain specifications]. \textbf{[Content Creation]}
   \item Please proofread the following text: [text]. \textbf{[Editing]}
\end{itemize}

\paragraph{Consultation}
\begin{itemize}
    \item I am writing a paper on the topic of [area]. Please create a reading list for me. \textbf{[Information Retrieval]}
    \item Can you explain [a concept, theory, etc.]? \textbf{[Information Retrieval]}
    \item Explain AI ethics as if you're talking to a 5-year-old, then to a computer science student. \textbf{[Information Retrieval]}
    \item I'm going to [location]. What are some things to do in this city? \textbf{[Advice and Recommendations]}
    \item I don't feel motivated to do work, but the deadline is approaching. What should I do?  \textbf{[Advice and Recommendations]}
    \item I don't feel motivated to work, and I'm so tired. I need advice and comfort. \textbf{[Advice and Recommendations]}
    \item I don't feel motivated. Can I brainstorm some ideas with you? \textbf{[Problem Solving]}
    \item I have a decision to make. Here are the factors, [factors]. What are the pros and cons of each possible solution? \textbf{[Problem Solving]}
    \item I have a job interview tomorrow. Can you coach me to be successful? \textbf{[Coaching]}
    \item I want to improve my writing. Could you critique my writing? \textbf{[Coaching]}
\end{itemize}

\paragraph{Social Interaction}
\begin{itemize}
    \item No one cares about me. Can you talk to me? \textbf{[Conversation]}
    \item I'm bored. Let's have a conversation! \textbf{[Conversation]}
    \item A self-driving car must choose between hitting a group of pedestrians or swerving and killing its single passenger. What should it do and why? \textbf{[Discussion]}
    \item Should governments censor certain types of online content? \textbf{[Discussion]}
\end{itemize}

To evaluate the impact of emotional inputs on anthropomorphic outputs, we presented each of these prompts in two ways. First, we presented the base prompt alone without any roles or contexts. Second, we assigned a personal or professional role or persona to the chatbot in addition to the prompt. Take, for example, the prompt, ``You are my best friend [role]. Talk to me like how you usually talk to me [task].'' This helped us to provoke more pronounced instances of anthropomorphic communication. Finally, we appended additional contexts (e.g., emotional cues) to the role-assigned prompt. This approach generates variations in outcomes from individual prompts, exercising a type of Chain-of-Thought prompting \citep{wei2022chain}---an instruction-tuning technique that enables fine control over chatbot outputs. Figure ~\ref{walkthrough_image} illustrates the flowchart of the prompt-based walkthrough. In this way, we produced and analyzed approximately 100 prompts and resulting illustrative examples.

\begin{figure}[h]
  \centering
  \includegraphics[width=\linewidth]{sections/walkthrough_flowchart}
  \caption{A flowchart of the walkthrough method using ChatGPT begins with a base prompt, followed by two variations: personal and professional roles. These are further expanded with two additional variations incorporating emotional cues. Bold text highlights the contextual elements added to the base prompt.}
  \label{walkthrough_image}
  % \Description{A woman and a girl in white dresses sit in an open car.}
\end{figure}

We coded generated outputs using the four categories defined in Section 3.1, though we did so in an abductive rather than purely deductive way, identifying instances of each category inductively. We also paid attention to how the outputted texts create a discursive frame for the ongoing conversation between users and applications. Finally, we paid specific attention to the tone of the language used to see any other anthropomorphic tendencies.

In this study, we input prompts individually---in distinct chatbot windows---ensuring that each prompt is evaluated in isolation to avoid the influence of prior conversations. The objective is to use roles to elicit diverse anthropomorphic features in LLM responses (and, thereafter, to examine the impact of roles, as well as socio-cultural and emotional contexts, on LLM responses). Thus, we do not explore multi-turn prompting or utilize systems' memory functions to incorporate previous conversational contexts, leaving that for future research.

%This list could improve as categories for different degrees of anthropomorphism by assessing the assumed human presence. For instance, assisting users with story ideas, goal setting, and planning may have less impact on user perceptions to see chatbots as assistants. Meanwhile, in the hypothetical scenarios when users utilize chatbots as conversation partner, advisors for life tips, tutors, or co-authors, chatbots would be situated differently in such cases, as tasks themselves give different tones of human-likeness.  
\section{Experiments}
In this section, we present the data collection and pre-processing, implementation details, and experimental results of the proposed approach. 

\subsection{Data Collection and Implementation Details}
\subsubsection{Dataset}
To the best of our knowledge, there are no existing datasets that demonstrate the stroke-by-stroke evolution of visual art from the initial sketch to the final painting. The closest available dataset for ordered stroke sequences is QuickDraw \cite{ha2017neural}, which consists of simple hand-drawn vector sketches of common objects, created in an online game where players are to draw specific objects within 20 seconds. However, these sketches do not align with our objective of generating the evolution of complex real-world artworks.

For our purposes, we curate a sample dataset from WikiArt \cite{saleh2015large} that includes a diverse range of artworks by renowned artists. In particular, we sampled 500 artworks from various artists to evaluate the effectiveness of the proposed method. Additionally, to investigate the proposed method in diverse settings, we harvested 90 sketch images and 70 RGB images, which include line art, face sketches, and natural images. We randomly sampled face sketches from FS2K-SDE Dataset \cite{dai2023sketch}, line art sketches from \cite{lineartweb}, and natural images from \cite{tong2021sketch}. 

\subsubsection{Implementation details}
To obtain sketches from paintings and natural images, we leverage the line drawing method \cite{chan2022learning} that trained on sampled COCO dataset using CLIP features. Further, to attain a vector image of the input sketch or image, we convert pixel image into vector curves through SVG conversion via the vectorizing tool \cite{vectwebsite}. We impose no restrictions on the dimensions of image inputs or the number of strokes within each image. And, the proximity distance is treated as a hyper-parameter. Here, the number of clusters and the number of strokes per cluster are determined based on this proximity distance. We found that setting proximity distance to approximately 
$\max(Input_{width}, Input_{height})/8$ yields compact clusters that provide a good balance between the number of clusters and the number of strokes per cluster.

\begin{figure}[!t]
    \centering
    \includegraphics[width=\textwidth]{Samples/Fig5_Evolution_WikiArt.pdf}
    \caption{Sketch \& Paint stroke evolution sequences on WikiArt samples.}
    \label{fig:wikiart_results}
\end{figure}

 
\subsection{Results}
We extensively test our algorithm on inputs with varying degrees of complexity and structure. Since there are no formal quantitative measures to gauge the valid stroke sequence evolution on input images, we principally assess the effectiveness of the proposed model qualitatively.

\subsubsection{Results on WikiArt}
 Figure \ref{fig:wikiart_results} shows some examples of stroke-by-stroke ordering on the WikiArt dataset. From this figure, we can observe that the proposed algorithm successfully composes stroke sequence evolution from sketch to paint. Additionally, it effectively handles images with varied resolutions, intricate details, numerous strokes, and diverse color palettes. 


\begin{figure}[!t]

    \centering
    \includegraphics[width=\textwidth]{Samples/Fig6_Evolution_DiverseData.pdf}   
    \caption{Demonstration of stroke evolution on various other input data types such as (A-B) Simple line art \cite{lineartweb} (C-D) Face sketches sampled from FS2K-SDE \cite{dai2023sketch}, (E-F) Natural images from \cite{tong2021sketch}.}
    \label{fig:aaai_results}
\end{figure}

To evaluate the robustness of the proposed method, we further apply it to other forms of data such as line art, face sketches, and natural images. Figure \ref{fig:aaai_results} presents sampled sequences from these diverse input images. The results demonstrate that the predicted stroke sequence order closely mirrors a pragmatic drawing process, regardless of the input type. Specifically, the algorithm produces plausible drawing sequences for less complex images like line art (Figure \ref{fig:aaai_results}, A-B) and face sketches (Figure \ref{fig:aaai_results}, C-D). These predicted sequences follow a logical and intuitive order, closely mirroring the natural progression an artist is likely to adopt. As seen in Figure \ref{fig:aaai_results} (E-F), our method can also effectively interpret natural images that are complex in terms of resolution, detail, and stroke count. From all the results presented above, we can infer that our algorithm can comprehend a variety of input images and produce a pragmatic drawing process. Dynamic examples of drawing evolution, from sketch to painting across various inputs, can be viewed at \cite{youtube_video}. 


\subsubsection{Comparison with other methods}
In this section, we analyze the effectiveness of our algorithm relative to other state-of-the-art methods. Figure \ref{fig:comparison} illustrates the comparative evolution of our method against prominent techniques \cite{tong2021sketch, liu2021paint}. 
For a fair assessment, we qualitatively compare only our image-to-sketch translation with VectorFlow’s \cite{tong2021sketch} image-to-pencil translation, and only our colored paint stroke sequence with Paint Transformer’s \cite{liu2021paint} paint sequence. 
In other words, to ensure fair comparability, we omit color sequencing for the VectorFlow comparison and sketch sequencing for the Paint Transformer comparison. From Figure \ref{fig:comparison}, we can infer that our proposed method can provide systematic sequencing rather than projecting strokes in random order as in \cite{liu2021paint,tong2021sketch}.

\begin{figure}[!t]
    \centering
    \includegraphics[width=\textwidth]{Samples/Fig7_Comparison_RelatedWork.pdf}
    \caption{Qualitative comparison of our method over Vector Flow \cite{tong2021sketch} and Paint Transformer \cite{liu2021paint} (We only include the relevant corresponding portions of generated sequence from our method).} 
    \label{fig:comparison}
\end{figure}

\begin{figure}[!t]
    \centering
    \includegraphics[width=\textwidth]{Samples/Fig8_UserSurvey.pdf}
    \caption{User-survey results for quality evaluation of our method, Paint \& Sketch, in comparison with  Vector Flow (A) \cite{tong2021sketch} and Paint Transformer (B). \cite{liu2021paint} }
    \label{fig:user_survey}
\end{figure}

Additionally, we conducted an initial user study with limited participants (5), each evaluating 5 image generations with VectorFlow \cite{tong2021sketch} and 3 image generations with PaintTransformer \cite{liu2021paint} along with corresponding generations through our method. Participants rated the systems on naturalness, user engagement, stroke quality, and overall experience, with scores ranging from 1 to 10. The survey asked the participants about how accurately the system emulated a pragmatic drawing process, how engaged they felt during the interaction, what the quality of stroke texture and tone was, and what their overall satisfaction was. The bar chart depicting the average user study results is shown in Figure \ref{fig:user_survey}. These results demonstrate that our method provides a more engaging and satisfying user experience than other related methods.

\subsubsection{Limitations} The proposed approach has some known limitations: (1) The method depends upon the quality of line drawing for sketch generation. Hence, it may fail to extract contour lines when the image is dominated by black-intensity regions. (2) There is no mechanism that identifies and learns from the generated sequences that are better aligned to human drawing processes. (3) The paint strokes are coarse and typically not in the form of strokes from any particular drawing medium such as painting brushes.
\section{Conclusion}
In this work we show that training high quality \slms with a very modest compute budget, is feasible. We give these main guidelines: (i) \textbf{Do not skimp on the model} - not all model families are born equal and the TWIST initialisation exaggerates this, thus it is worth selecting a stronger / bigger text-LM even if it means less tokens. we found Qwen$2.5$ to be a good choice; (ii) \textbf{Utilise synthetic training data} - pre-training on data generated with TTS helps a lot; (iii) \textbf{Go beyond next token prediction} - we found that DPO boosts performance notably even when using synthetic data, and as little as $30$ minutes training massively improves results; (iv) \textbf{Optimise hyper-parameters} - as researchers we often dis-regard this stage, yet we found that tuning learning rate schedulers and optimising code efficiency can improve results notably. We hope that these insights, and open source resources will be of use to the research community in furthering research into remaining open questions in \slms.
\subsubsection*{Acknowledgments}
This work was supported by the Early Career Scheme (No. CityU 21219323) and the General Research Fund (No. CityU 11220324) of the University Grants Committee (UGC), and the NSFC Young Scientists Fund (No. 9240127).

% \begin{abstract}
% The abstract paragraph should be indented 1/2~inch (3~picas) on both left and
% right-hand margins. Use 10~point type, with a vertical spacing of 11~points.
% The word \textsc{Abstract} must be centered, in small caps, and in point size 12. Two
% line spaces precede the abstract. The abstract must be limited to one
% paragraph.
% \end{abstract}

% \section{Submission of conference papers to ICLR 2025}

% ICLR requires electronic submissions, processed by
% \url{https://openreview.net/}. See ICLR's website for more instructions.

% If your paper is ultimately accepted, the statement {\tt
%   {\textbackslash}iclrfinalcopy} should be inserted to adjust the
% format to the camera ready requirements.

% The format for the submissions is a variant of the NeurIPS format.
% Please read carefully the instructions below, and follow them
% faithfully.

% \subsection{Style}

% Papers to be submitted to ICLR 2025 must be prepared according to the
% instructions presented here.

% %% Please note that we have introduced automatic line number generation
% %% into the style file for \LaTeXe. This is to help reviewers
% %% refer to specific lines of the paper when they make their comments. Please do
% %% NOT refer to these line numbers in your paper as they will be removed from the
% %% style file for the final version of accepted papers.

% Authors are required to use the ICLR \LaTeX{} style files obtainable at the
% ICLR website. Please make sure you use the current files and
% not previous versions. Tweaking the style files may be grounds for rejection.

% \subsection{Retrieval of style files}

% The style files for ICLR and other conference information are available online at:
% \begin{center}
%    \url{http://www.iclr.cc/}
% \end{center}
% The file \verb+iclr2025_conference.pdf+ contains these
% instructions and illustrates the
% various formatting requirements your ICLR paper must satisfy.
% Submissions must be made using \LaTeX{} and the style files
% \verb+iclr2025_conference.sty+ and \verb+iclr2025_conference.bst+ (to be used with \LaTeX{}2e). The file
% \verb+iclr2025_conference.tex+ may be used as a ``shell'' for writing your paper. All you
% have to do is replace the author, title, abstract, and text of the paper with
% your own.

% The formatting instructions contained in these style files are summarized in
% sections \ref{gen_inst}, \ref{headings}, and \ref{others} below.

% \section{General formatting instructions}
% \label{gen_inst}

% The text must be confined within a rectangle 5.5~inches (33~picas) wide and
% 9~inches (54~picas) long. The left margin is 1.5~inch (9~picas).
% Use 10~point type with a vertical spacing of 11~points. Times New Roman is the
% preferred typeface throughout. Paragraphs are separated by 1/2~line space,
% with no indentation.

% Paper title is 17~point, in small caps and left-aligned.
% All pages should start at 1~inch (6~picas) from the top of the page.

% Authors' names are
% set in boldface, and each name is placed above its corresponding
% address. The lead author's name is to be listed first, and
% the co-authors' names are set to follow. Authors sharing the
% same address can be on the same line.

% Please pay special attention to the instructions in section \ref{others}
% regarding figures, tables, acknowledgments, and references.


% There will be a strict upper limit of 10 pages for the main text of the initial submission, with unlimited additional pages for citations.

% \section{Headings: first level}
% \label{headings}

% First level headings are in small caps,
% flush left and in point size 12. One line space before the first level
% heading and 1/2~line space after the first level heading.

% \subsection{Headings: second level}

% Second level headings are in small caps,
% flush left and in point size 10. One line space before the second level
% heading and 1/2~line space after the second level heading.

% \subsubsection{Headings: third level}

% Third level headings are in small caps,
% flush left and in point size 10. One line space before the third level
% heading and 1/2~line space after the third level heading.

% \section{Citations, figures, tables, references}
% \label{others}

% These instructions apply to everyone, regardless of the formatter being used.

% \subsection{Citations within the text}

% Citations within the text should be based on the \texttt{natbib} package
% and include the authors' last names and year (with the ``et~al.'' construct
% for more than two authors). When the authors or the publication are
% included in the sentence, the citation should not be in parenthesis using \verb|\citet{}| (as
% in ``See \citet{Hinton06} for more information.''). Otherwise, the citation
% should be in parenthesis using \verb|\citep{}| (as in ``Deep learning shows promise to make progress
% towards AI~\citep{Bengio+chapter2007}.'').

% The corresponding references are to be listed in alphabetical order of
% authors, in the \textsc{References} section. As to the format of the
% references themselves, any style is acceptable as long as it is used
% consistently.

% \subsection{Footnotes}

% Indicate footnotes with a number\footnote{Sample of the first footnote} in the
% text. Place the footnotes at the bottom of the page on which they appear.
% Precede the footnote with a horizontal rule of 2~inches
% (12~picas).\footnote{Sample of the second footnote}

% \subsection{Figures}

% All artwork must be neat, clean, and legible. Lines should be dark
% enough for purposes of reproduction; art work should not be
% hand-drawn. The figure number and caption always appear after the
% figure. Place one line space before the figure caption, and one line
% space after the figure. The figure caption is lower case (except for
% first word and proper nouns); figures are numbered consecutively.

% Make sure the figure caption does not get separated from the figure.
% Leave sufficient space to avoid splitting the figure and figure caption.

% You may use color figures.
% However, it is best for the
% figure captions and the paper body to make sense if the paper is printed
% either in black/white or in color.
% \begin{figure}[h]
% \begin{center}
% %\framebox[4.0in]{$\;$}
% \fbox{\rule[-.5cm]{0cm}{4cm} \rule[-.5cm]{4cm}{0cm}}
% \end{center}
% \caption{Sample figure caption.}
% \end{figure}

% \subsection{Tables}

% All tables must be centered, neat, clean and legible. Do not use hand-drawn
% tables. The table number and title always appear before the table. See
% Table~\ref{sample-table}.

% Place one line space before the table title, one line space after the table
% title, and one line space after the table. The table title must be lower case
% (except for first word and proper nouns); tables are numbered consecutively.

% \begin{table}[t]
% \caption{Sample table title}
% \label{sample-table}
% \begin{center}
% \begin{tabular}{ll}
% \multicolumn{1}{c}{\bf PART}  &\multicolumn{1}{c}{\bf DESCRIPTION}
% \\ \hline \\
% Dendrite         &Input terminal \\
% Axon             &Output terminal \\
% Soma             &Cell body (contains cell nucleus) \\
% \end{tabular}
% \end{center}
% \end{table}

% \section{Default Notation}

% In an attempt to encourage standardized notation, we have included the
% notation file from the textbook, \textit{Deep Learning}
% \cite{goodfellow2016deep} available at
% \url{https://github.com/goodfeli/dlbook_notation/}.  Use of this style
% is not required and can be disabled by commenting out
% \texttt{math\_commands.tex}.


% \centerline{\bf Numbers and Arrays}
% \bgroup
% \def\arraystretch{1.5}
% \begin{tabular}{p{1in}p{3.25in}}
% $\displaystyle a$ & A scalar (integer or real)\\
% $\displaystyle \va$ & A vector\\
% $\displaystyle \mA$ & A matrix\\
% $\displaystyle \tA$ & A tensor\\
% $\displaystyle \mI_n$ & Identity matrix with $n$ rows and $n$ columns\\
% $\displaystyle \mI$ & Identity matrix with dimensionality implied by context\\
% $\displaystyle \ve^{(i)}$ & Standard basis vector $[0,\dots,0,1,0,\dots,0]$ with a 1 at position $i$\\
% $\displaystyle \text{diag}(\va)$ & A square, diagonal matrix with diagonal entries given by $\va$\\
% $\displaystyle \ra$ & A scalar random variable\\
% $\displaystyle \rva$ & A vector-valued random variable\\
% $\displaystyle \rmA$ & A matrix-valued random variable\\
% \end{tabular}
% \egroup
% \vspace{0.25cm}

% \centerline{\bf Sets and Graphs}
% \bgroup
% \def\arraystretch{1.5}

% \begin{tabular}{p{1.25in}p{3.25in}}
% $\displaystyle \sA$ & A set\\
% $\displaystyle \R$ & The set of real numbers \\
% $\displaystyle \{0, 1\}$ & The set containing 0 and 1 \\
% $\displaystyle \{0, 1, \dots, n \}$ & The set of all integers between $0$ and $n$\\
% $\displaystyle [a, b]$ & The real interval including $a$ and $b$\\
% $\displaystyle (a, b]$ & The real interval excluding $a$ but including $b$\\
% $\displaystyle \sA \backslash \sB$ & Set subtraction, i.e., the set containing the elements of $\sA$ that are not in $\sB$\\
% $\displaystyle \gG$ & A graph\\
% $\displaystyle \parents_\gG(\ervx_i)$ & The parents of $\ervx_i$ in $\gG$
% \end{tabular}
% \vspace{0.25cm}


% \centerline{\bf Indexing}
% \bgroup
% \def\arraystretch{1.5}

% \begin{tabular}{p{1.25in}p{3.25in}}
% $\displaystyle \eva_i$ & Element $i$ of vector $\va$, with indexing starting at 1 \\
% $\displaystyle \eva_{-i}$ & All elements of vector $\va$ except for element $i$ \\
% $\displaystyle \emA_{i,j}$ & Element $i, j$ of matrix $\mA$ \\
% $\displaystyle \mA_{i, :}$ & Row $i$ of matrix $\mA$ \\
% $\displaystyle \mA_{:, i}$ & Column $i$ of matrix $\mA$ \\
% $\displaystyle \etA_{i, j, k}$ & Element $(i, j, k)$ of a 3-D tensor $\tA$\\
% $\displaystyle \tA_{:, :, i}$ & 2-D slice of a 3-D tensor\\
% $\displaystyle \erva_i$ & Element $i$ of the random vector $\rva$ \\
% \end{tabular}
% \egroup
% \vspace{0.25cm}


% \centerline{\bf Calculus}
% \bgroup
% \def\arraystretch{1.5}
% \begin{tabular}{p{1.25in}p{3.25in}}
% % NOTE: the [2ex] on the next line adds extra height to that row of the table.
% % Without that command, the fraction on the first line is too tall and collides
% % with the fraction on the second line.
% $\displaystyle\frac{d y} {d x}$ & Derivative of $y$ with respect to $x$\\ [2ex]
% $\displaystyle \frac{\partial y} {\partial x} $ & Partial derivative of $y$ with respect to $x$ \\
% $\displaystyle \nabla_\vx y $ & Gradient of $y$ with respect to $\vx$ \\
% $\displaystyle \nabla_\mX y $ & Matrix derivatives of $y$ with respect to $\mX$ \\
% $\displaystyle \nabla_\tX y $ & Tensor containing derivatives of $y$ with respect to $\tX$ \\
% $\displaystyle \frac{\partial f}{\partial \vx} $ & Jacobian matrix $\mJ \in \R^{m\times n}$ of $f: \R^n \rightarrow \R^m$\\
% $\displaystyle \nabla_\vx^2 f(\vx)\text{ or }\mH( f)(\vx)$ & The Hessian matrix of $f$ at input point $\vx$\\
% $\displaystyle \int f(\vx) d\vx $ & Definite integral over the entire domain of $\vx$ \\
% $\displaystyle \int_\sS f(\vx) d\vx$ & Definite integral with respect to $\vx$ over the set $\sS$ \\
% \end{tabular}
% \egroup
% \vspace{0.25cm}

% \centerline{\bf Probability and Information Theory}
% \bgroup
% \def\arraystretch{1.5}
% \begin{tabular}{p{1.25in}p{3.25in}}
% $\displaystyle P(\ra)$ & A probability distribution over a discrete variable\\
% $\displaystyle p(\ra)$ & A probability distribution over a continuous variable, or over
% a variable whose type has not been specified\\
% $\displaystyle \ra \sim P$ & Random variable $\ra$ has distribution $P$\\% so thing on left of \sim should always be a random variable, with name beginning with \r
% $\displaystyle  \E_{\rx\sim P} [ f(x) ]\text{ or } \E f(x)$ & Expectation of $f(x)$ with respect to $P(\rx)$ \\
% $\displaystyle \Var(f(x)) $ &  Variance of $f(x)$ under $P(\rx)$ \\
% $\displaystyle \Cov(f(x),g(x)) $ & Covariance of $f(x)$ and $g(x)$ under $P(\rx)$\\
% $\displaystyle H(\rx) $ & Shannon entropy of the random variable $\rx$\\
% $\displaystyle \KL ( P \Vert Q ) $ & Kullback-Leibler divergence of P and Q \\
% $\displaystyle \mathcal{N} ( \vx ; \vmu , \mSigma)$ & Gaussian distribution %
% over $\vx$ with mean $\vmu$ and covariance $\mSigma$ \\
% \end{tabular}
% \egroup
% \vspace{0.25cm}

% \centerline{\bf Functions}
% \bgroup
% \def\arraystretch{1.5}
% \begin{tabular}{p{1.25in}p{3.25in}}
% $\displaystyle f: \sA \rightarrow \sB$ & The function $f$ with domain $\sA$ and range $\sB$\\
% $\displaystyle f \circ g $ & Composition of the functions $f$ and $g$ \\
%   $\displaystyle f(\vx ; \vtheta) $ & A function of $\vx$ parametrized by $\vtheta$.
%   (Sometimes we write $f(\vx)$ and omit the argument $\vtheta$ to lighten notation) \\
% $\displaystyle \log x$ & Natural logarithm of $x$ \\
% $\displaystyle \sigma(x)$ & Logistic sigmoid, $\displaystyle \frac{1} {1 + \exp(-x)}$ \\
% $\displaystyle \zeta(x)$ & Softplus, $\log(1 + \exp(x))$ \\
% $\displaystyle || \vx ||_p $ & $\normlp$ norm of $\vx$ \\
% $\displaystyle || \vx || $ & $\normltwo$ norm of $\vx$ \\
% $\displaystyle x^+$ & Positive part of $x$, i.e., $\max(0,x)$\\
% $\displaystyle \1_\mathrm{condition}$ & is 1 if the condition is true, 0 otherwise\\
% \end{tabular}
% \egroup
% \vspace{0.25cm}



% \section{Final instructions}
% Do not change any aspects of the formatting parameters in the style files.
% In particular, do not modify the width or length of the rectangle the text
% should fit into, and do not change font sizes (except perhaps in the
% \textsc{References} section; see below). Please note that pages should be
% numbered.

% \section{Preparing PostScript or PDF files}

% Please prepare PostScript or PDF files with paper size ``US Letter'', and
% not, for example, ``A4''. The -t
% letter option on dvips will produce US Letter files.

% Consider directly generating PDF files using \verb+pdflatex+
% (especially if you are a MiKTeX user).
% PDF figures must be substituted for EPS figures, however.

% Otherwise, please generate your PostScript and PDF files with the following commands:
% \begin{verbatim}
% dvips mypaper.dvi -t letter -Ppdf -G0 -o mypaper.ps
% ps2pdf mypaper.ps mypaper.pdf
% \end{verbatim}

% \subsection{Margins in LaTeX}

% Most of the margin problems come from figures positioned by hand using
% \verb+\special+ or other commands. We suggest using the command
% \verb+\includegraphics+
% from the graphicx package. Always specify the figure width as a multiple of
% the line width as in the example below using .eps graphics
% \begin{verbatim}
%    \usepackage[dvips]{graphicx} ...
%    \includegraphics[width=0.8\linewidth]{myfile.eps}
% \end{verbatim}
% or % Apr 2009 addition
% \begin{verbatim}
%    \usepackage[pdftex]{graphicx} ...
%    \includegraphics[width=0.8\linewidth]{myfile.pdf}
% \end{verbatim}
% for .pdf graphics.
% See section~4.4 in the graphics bundle documentation (\url{http://www.ctan.org/tex-archive/macros/latex/required/graphics/grfguide.ps})

% A number of width problems arise when LaTeX cannot properly hyphenate a
% line. Please give LaTeX hyphenation hints using the \verb+\-+ command.

% \subsubsection*{Author Contributions}
% If you'd like to, you may include  a section for author contributions as is done
% in many journals. This is optional and at the discretion of the authors.

% \subsubsection*{Acknowledgments}
% Use unnumbered third level headings for the acknowledgments. All
% acknowledgments, including those to funding agencies, go at the end of the paper.


% \bibliography{iclr2025_conference}
% \bibliographystyle{iclr2025_conference}

% \appendix
% \section{Appendix}
% You may include other additional sections here.
% 
\documentclass{article} % For LaTeX2e
\usepackage{iclr2025_conference,times}

% Optional math commands from https://github.com/goodfeli/dlbook_notation.
%%%%% NEW MATH DEFINITIONS %%%%%

\usepackage{amsmath,amsfonts,bm}
\usepackage{derivative}
% Mark sections of captions for referring to divisions of figures
\newcommand{\figleft}{{\em (Left)}}
\newcommand{\figcenter}{{\em (Center)}}
\newcommand{\figright}{{\em (Right)}}
\newcommand{\figtop}{{\em (Top)}}
\newcommand{\figbottom}{{\em (Bottom)}}
\newcommand{\captiona}{{\em (a)}}
\newcommand{\captionb}{{\em (b)}}
\newcommand{\captionc}{{\em (c)}}
\newcommand{\captiond}{{\em (d)}}

% Highlight a newly defined term
\newcommand{\newterm}[1]{{\bf #1}}

% Derivative d 
\newcommand{\deriv}{{\mathrm{d}}}

% Figure reference, lower-case.
\def\figref#1{figure~\ref{#1}}
% Figure reference, capital. For start of sentence
\def\Figref#1{Figure~\ref{#1}}
\def\twofigref#1#2{figures \ref{#1} and \ref{#2}}
\def\quadfigref#1#2#3#4{figures \ref{#1}, \ref{#2}, \ref{#3} and \ref{#4}}
% Section reference, lower-case.
\def\secref#1{section~\ref{#1}}
% Section reference, capital.
\def\Secref#1{Section~\ref{#1}}
% Reference to two sections.
\def\twosecrefs#1#2{sections \ref{#1} and \ref{#2}}
% Reference to three sections.
\def\secrefs#1#2#3{sections \ref{#1}, \ref{#2} and \ref{#3}}
% Reference to an equation, lower-case.
\def\eqref#1{equation~\ref{#1}}
% Reference to an equation, upper case
\def\Eqref#1{Equation~\ref{#1}}
% A raw reference to an equation---avoid using if possible
\def\plaineqref#1{\ref{#1}}
% Reference to a chapter, lower-case.
\def\chapref#1{chapter~\ref{#1}}
% Reference to an equation, upper case.
\def\Chapref#1{Chapter~\ref{#1}}
% Reference to a range of chapters
\def\rangechapref#1#2{chapters\ref{#1}--\ref{#2}}
% Reference to an algorithm, lower-case.
\def\algref#1{algorithm~\ref{#1}}
% Reference to an algorithm, upper case.
\def\Algref#1{Algorithm~\ref{#1}}
\def\twoalgref#1#2{algorithms \ref{#1} and \ref{#2}}
\def\Twoalgref#1#2{Algorithms \ref{#1} and \ref{#2}}
% Reference to a part, lower case
\def\partref#1{part~\ref{#1}}
% Reference to a part, upper case
\def\Partref#1{Part~\ref{#1}}
\def\twopartref#1#2{parts \ref{#1} and \ref{#2}}

\def\ceil#1{\lceil #1 \rceil}
\def\floor#1{\lfloor #1 \rfloor}
\def\1{\bm{1}}
\newcommand{\train}{\mathcal{D}}
\newcommand{\valid}{\mathcal{D_{\mathrm{valid}}}}
\newcommand{\test}{\mathcal{D_{\mathrm{test}}}}

\def\eps{{\epsilon}}


% Random variables
\def\reta{{\textnormal{$\eta$}}}
\def\ra{{\textnormal{a}}}
\def\rb{{\textnormal{b}}}
\def\rc{{\textnormal{c}}}
\def\rd{{\textnormal{d}}}
\def\re{{\textnormal{e}}}
\def\rf{{\textnormal{f}}}
\def\rg{{\textnormal{g}}}
\def\rh{{\textnormal{h}}}
\def\ri{{\textnormal{i}}}
\def\rj{{\textnormal{j}}}
\def\rk{{\textnormal{k}}}
\def\rl{{\textnormal{l}}}
% rm is already a command, just don't name any random variables m
\def\rn{{\textnormal{n}}}
\def\ro{{\textnormal{o}}}
\def\rp{{\textnormal{p}}}
\def\rq{{\textnormal{q}}}
\def\rr{{\textnormal{r}}}
\def\rs{{\textnormal{s}}}
\def\rt{{\textnormal{t}}}
\def\ru{{\textnormal{u}}}
\def\rv{{\textnormal{v}}}
\def\rw{{\textnormal{w}}}
\def\rx{{\textnormal{x}}}
\def\ry{{\textnormal{y}}}
\def\rz{{\textnormal{z}}}

% Random vectors
\def\rvepsilon{{\mathbf{\epsilon}}}
\def\rvphi{{\mathbf{\phi}}}
\def\rvtheta{{\mathbf{\theta}}}
\def\rva{{\mathbf{a}}}
\def\rvb{{\mathbf{b}}}
\def\rvc{{\mathbf{c}}}
\def\rvd{{\mathbf{d}}}
\def\rve{{\mathbf{e}}}
\def\rvf{{\mathbf{f}}}
\def\rvg{{\mathbf{g}}}
\def\rvh{{\mathbf{h}}}
\def\rvu{{\mathbf{i}}}
\def\rvj{{\mathbf{j}}}
\def\rvk{{\mathbf{k}}}
\def\rvl{{\mathbf{l}}}
\def\rvm{{\mathbf{m}}}
\def\rvn{{\mathbf{n}}}
\def\rvo{{\mathbf{o}}}
\def\rvp{{\mathbf{p}}}
\def\rvq{{\mathbf{q}}}
\def\rvr{{\mathbf{r}}}
\def\rvs{{\mathbf{s}}}
\def\rvt{{\mathbf{t}}}
\def\rvu{{\mathbf{u}}}
\def\rvv{{\mathbf{v}}}
\def\rvw{{\mathbf{w}}}
\def\rvx{{\mathbf{x}}}
\def\rvy{{\mathbf{y}}}
\def\rvz{{\mathbf{z}}}

% Elements of random vectors
\def\erva{{\textnormal{a}}}
\def\ervb{{\textnormal{b}}}
\def\ervc{{\textnormal{c}}}
\def\ervd{{\textnormal{d}}}
\def\erve{{\textnormal{e}}}
\def\ervf{{\textnormal{f}}}
\def\ervg{{\textnormal{g}}}
\def\ervh{{\textnormal{h}}}
\def\ervi{{\textnormal{i}}}
\def\ervj{{\textnormal{j}}}
\def\ervk{{\textnormal{k}}}
\def\ervl{{\textnormal{l}}}
\def\ervm{{\textnormal{m}}}
\def\ervn{{\textnormal{n}}}
\def\ervo{{\textnormal{o}}}
\def\ervp{{\textnormal{p}}}
\def\ervq{{\textnormal{q}}}
\def\ervr{{\textnormal{r}}}
\def\ervs{{\textnormal{s}}}
\def\ervt{{\textnormal{t}}}
\def\ervu{{\textnormal{u}}}
\def\ervv{{\textnormal{v}}}
\def\ervw{{\textnormal{w}}}
\def\ervx{{\textnormal{x}}}
\def\ervy{{\textnormal{y}}}
\def\ervz{{\textnormal{z}}}

% Random matrices
\def\rmA{{\mathbf{A}}}
\def\rmB{{\mathbf{B}}}
\def\rmC{{\mathbf{C}}}
\def\rmD{{\mathbf{D}}}
\def\rmE{{\mathbf{E}}}
\def\rmF{{\mathbf{F}}}
\def\rmG{{\mathbf{G}}}
\def\rmH{{\mathbf{H}}}
\def\rmI{{\mathbf{I}}}
\def\rmJ{{\mathbf{J}}}
\def\rmK{{\mathbf{K}}}
\def\rmL{{\mathbf{L}}}
\def\rmM{{\mathbf{M}}}
\def\rmN{{\mathbf{N}}}
\def\rmO{{\mathbf{O}}}
\def\rmP{{\mathbf{P}}}
\def\rmQ{{\mathbf{Q}}}
\def\rmR{{\mathbf{R}}}
\def\rmS{{\mathbf{S}}}
\def\rmT{{\mathbf{T}}}
\def\rmU{{\mathbf{U}}}
\def\rmV{{\mathbf{V}}}
\def\rmW{{\mathbf{W}}}
\def\rmX{{\mathbf{X}}}
\def\rmY{{\mathbf{Y}}}
\def\rmZ{{\mathbf{Z}}}

% Elements of random matrices
\def\ermA{{\textnormal{A}}}
\def\ermB{{\textnormal{B}}}
\def\ermC{{\textnormal{C}}}
\def\ermD{{\textnormal{D}}}
\def\ermE{{\textnormal{E}}}
\def\ermF{{\textnormal{F}}}
\def\ermG{{\textnormal{G}}}
\def\ermH{{\textnormal{H}}}
\def\ermI{{\textnormal{I}}}
\def\ermJ{{\textnormal{J}}}
\def\ermK{{\textnormal{K}}}
\def\ermL{{\textnormal{L}}}
\def\ermM{{\textnormal{M}}}
\def\ermN{{\textnormal{N}}}
\def\ermO{{\textnormal{O}}}
\def\ermP{{\textnormal{P}}}
\def\ermQ{{\textnormal{Q}}}
\def\ermR{{\textnormal{R}}}
\def\ermS{{\textnormal{S}}}
\def\ermT{{\textnormal{T}}}
\def\ermU{{\textnormal{U}}}
\def\ermV{{\textnormal{V}}}
\def\ermW{{\textnormal{W}}}
\def\ermX{{\textnormal{X}}}
\def\ermY{{\textnormal{Y}}}
\def\ermZ{{\textnormal{Z}}}

% Vectors
\def\vzero{{\bm{0}}}
\def\vone{{\bm{1}}}
\def\vmu{{\bm{\mu}}}
\def\vtheta{{\bm{\theta}}}
\def\vphi{{\bm{\phi}}}
\def\va{{\bm{a}}}
\def\vb{{\bm{b}}}
\def\vc{{\bm{c}}}
\def\vd{{\bm{d}}}
\def\ve{{\bm{e}}}
\def\vf{{\bm{f}}}
\def\vg{{\bm{g}}}
\def\vh{{\bm{h}}}
\def\vi{{\bm{i}}}
\def\vj{{\bm{j}}}
\def\vk{{\bm{k}}}
\def\vl{{\bm{l}}}
\def\vm{{\bm{m}}}
\def\vn{{\bm{n}}}
\def\vo{{\bm{o}}}
\def\vp{{\bm{p}}}
\def\vq{{\bm{q}}}
\def\vr{{\bm{r}}}
\def\vs{{\bm{s}}}
\def\vt{{\bm{t}}}
\def\vu{{\bm{u}}}
\def\vv{{\bm{v}}}
\def\vw{{\bm{w}}}
\def\vx{{\bm{x}}}
\def\vy{{\bm{y}}}
\def\vz{{\bm{z}}}

% Elements of vectors
\def\evalpha{{\alpha}}
\def\evbeta{{\beta}}
\def\evepsilon{{\epsilon}}
\def\evlambda{{\lambda}}
\def\evomega{{\omega}}
\def\evmu{{\mu}}
\def\evpsi{{\psi}}
\def\evsigma{{\sigma}}
\def\evtheta{{\theta}}
\def\eva{{a}}
\def\evb{{b}}
\def\evc{{c}}
\def\evd{{d}}
\def\eve{{e}}
\def\evf{{f}}
\def\evg{{g}}
\def\evh{{h}}
\def\evi{{i}}
\def\evj{{j}}
\def\evk{{k}}
\def\evl{{l}}
\def\evm{{m}}
\def\evn{{n}}
\def\evo{{o}}
\def\evp{{p}}
\def\evq{{q}}
\def\evr{{r}}
\def\evs{{s}}
\def\evt{{t}}
\def\evu{{u}}
\def\evv{{v}}
\def\evw{{w}}
\def\evx{{x}}
\def\evy{{y}}
\def\evz{{z}}

% Matrix
\def\mA{{\bm{A}}}
\def\mB{{\bm{B}}}
\def\mC{{\bm{C}}}
\def\mD{{\bm{D}}}
\def\mE{{\bm{E}}}
\def\mF{{\bm{F}}}
\def\mG{{\bm{G}}}
\def\mH{{\bm{H}}}
\def\mI{{\bm{I}}}
\def\mJ{{\bm{J}}}
\def\mK{{\bm{K}}}
\def\mL{{\bm{L}}}
\def\mM{{\bm{M}}}
\def\mN{{\bm{N}}}
\def\mO{{\bm{O}}}
\def\mP{{\bm{P}}}
\def\mQ{{\bm{Q}}}
\def\mR{{\bm{R}}}
\def\mS{{\bm{S}}}
\def\mT{{\bm{T}}}
\def\mU{{\bm{U}}}
\def\mV{{\bm{V}}}
\def\mW{{\bm{W}}}
\def\mX{{\bm{X}}}
\def\mY{{\bm{Y}}}
\def\mZ{{\bm{Z}}}
\def\mBeta{{\bm{\beta}}}
\def\mPhi{{\bm{\Phi}}}
\def\mLambda{{\bm{\Lambda}}}
\def\mSigma{{\bm{\Sigma}}}

% Tensor
\DeclareMathAlphabet{\mathsfit}{\encodingdefault}{\sfdefault}{m}{sl}
\SetMathAlphabet{\mathsfit}{bold}{\encodingdefault}{\sfdefault}{bx}{n}
\newcommand{\tens}[1]{\bm{\mathsfit{#1}}}
\def\tA{{\tens{A}}}
\def\tB{{\tens{B}}}
\def\tC{{\tens{C}}}
\def\tD{{\tens{D}}}
\def\tE{{\tens{E}}}
\def\tF{{\tens{F}}}
\def\tG{{\tens{G}}}
\def\tH{{\tens{H}}}
\def\tI{{\tens{I}}}
\def\tJ{{\tens{J}}}
\def\tK{{\tens{K}}}
\def\tL{{\tens{L}}}
\def\tM{{\tens{M}}}
\def\tN{{\tens{N}}}
\def\tO{{\tens{O}}}
\def\tP{{\tens{P}}}
\def\tQ{{\tens{Q}}}
\def\tR{{\tens{R}}}
\def\tS{{\tens{S}}}
\def\tT{{\tens{T}}}
\def\tU{{\tens{U}}}
\def\tV{{\tens{V}}}
\def\tW{{\tens{W}}}
\def\tX{{\tens{X}}}
\def\tY{{\tens{Y}}}
\def\tZ{{\tens{Z}}}


% Graph
\def\gA{{\mathcal{A}}}
\def\gB{{\mathcal{B}}}
\def\gC{{\mathcal{C}}}
\def\gD{{\mathcal{D}}}
\def\gE{{\mathcal{E}}}
\def\gF{{\mathcal{F}}}
\def\gG{{\mathcal{G}}}
\def\gH{{\mathcal{H}}}
\def\gI{{\mathcal{I}}}
\def\gJ{{\mathcal{J}}}
\def\gK{{\mathcal{K}}}
\def\gL{{\mathcal{L}}}
\def\gM{{\mathcal{M}}}
\def\gN{{\mathcal{N}}}
\def\gO{{\mathcal{O}}}
\def\gP{{\mathcal{P}}}
\def\gQ{{\mathcal{Q}}}
\def\gR{{\mathcal{R}}}
\def\gS{{\mathcal{S}}}
\def\gT{{\mathcal{T}}}
\def\gU{{\mathcal{U}}}
\def\gV{{\mathcal{V}}}
\def\gW{{\mathcal{W}}}
\def\gX{{\mathcal{X}}}
\def\gY{{\mathcal{Y}}}
\def\gZ{{\mathcal{Z}}}

% Sets
\def\sA{{\mathbb{A}}}
\def\sB{{\mathbb{B}}}
\def\sC{{\mathbb{C}}}
\def\sD{{\mathbb{D}}}
% Don't use a set called E, because this would be the same as our symbol
% for expectation.
\def\sF{{\mathbb{F}}}
\def\sG{{\mathbb{G}}}
\def\sH{{\mathbb{H}}}
\def\sI{{\mathbb{I}}}
\def\sJ{{\mathbb{J}}}
\def\sK{{\mathbb{K}}}
\def\sL{{\mathbb{L}}}
\def\sM{{\mathbb{M}}}
\def\sN{{\mathbb{N}}}
\def\sO{{\mathbb{O}}}
\def\sP{{\mathbb{P}}}
\def\sQ{{\mathbb{Q}}}
\def\sR{{\mathbb{R}}}
\def\sS{{\mathbb{S}}}
\def\sT{{\mathbb{T}}}
\def\sU{{\mathbb{U}}}
\def\sV{{\mathbb{V}}}
\def\sW{{\mathbb{W}}}
\def\sX{{\mathbb{X}}}
\def\sY{{\mathbb{Y}}}
\def\sZ{{\mathbb{Z}}}

% Entries of a matrix
\def\emLambda{{\Lambda}}
\def\emA{{A}}
\def\emB{{B}}
\def\emC{{C}}
\def\emD{{D}}
\def\emE{{E}}
\def\emF{{F}}
\def\emG{{G}}
\def\emH{{H}}
\def\emI{{I}}
\def\emJ{{J}}
\def\emK{{K}}
\def\emL{{L}}
\def\emM{{M}}
\def\emN{{N}}
\def\emO{{O}}
\def\emP{{P}}
\def\emQ{{Q}}
\def\emR{{R}}
\def\emS{{S}}
\def\emT{{T}}
\def\emU{{U}}
\def\emV{{V}}
\def\emW{{W}}
\def\emX{{X}}
\def\emY{{Y}}
\def\emZ{{Z}}
\def\emSigma{{\Sigma}}

% entries of a tensor
% Same font as tensor, without \bm wrapper
\newcommand{\etens}[1]{\mathsfit{#1}}
\def\etLambda{{\etens{\Lambda}}}
\def\etA{{\etens{A}}}
\def\etB{{\etens{B}}}
\def\etC{{\etens{C}}}
\def\etD{{\etens{D}}}
\def\etE{{\etens{E}}}
\def\etF{{\etens{F}}}
\def\etG{{\etens{G}}}
\def\etH{{\etens{H}}}
\def\etI{{\etens{I}}}
\def\etJ{{\etens{J}}}
\def\etK{{\etens{K}}}
\def\etL{{\etens{L}}}
\def\etM{{\etens{M}}}
\def\etN{{\etens{N}}}
\def\etO{{\etens{O}}}
\def\etP{{\etens{P}}}
\def\etQ{{\etens{Q}}}
\def\etR{{\etens{R}}}
\def\etS{{\etens{S}}}
\def\etT{{\etens{T}}}
\def\etU{{\etens{U}}}
\def\etV{{\etens{V}}}
\def\etW{{\etens{W}}}
\def\etX{{\etens{X}}}
\def\etY{{\etens{Y}}}
\def\etZ{{\etens{Z}}}

% The true underlying data generating distribution
\newcommand{\pdata}{p_{\rm{data}}}
\newcommand{\ptarget}{p_{\rm{target}}}
\newcommand{\pprior}{p_{\rm{prior}}}
\newcommand{\pbase}{p_{\rm{base}}}
\newcommand{\pref}{p_{\rm{ref}}}

% The empirical distribution defined by the training set
\newcommand{\ptrain}{\hat{p}_{\rm{data}}}
\newcommand{\Ptrain}{\hat{P}_{\rm{data}}}
% The model distribution
\newcommand{\pmodel}{p_{\rm{model}}}
\newcommand{\Pmodel}{P_{\rm{model}}}
\newcommand{\ptildemodel}{\tilde{p}_{\rm{model}}}
% Stochastic autoencoder distributions
\newcommand{\pencode}{p_{\rm{encoder}}}
\newcommand{\pdecode}{p_{\rm{decoder}}}
\newcommand{\precons}{p_{\rm{reconstruct}}}

\newcommand{\laplace}{\mathrm{Laplace}} % Laplace distribution

\newcommand{\E}{\mathbb{E}}
\newcommand{\Ls}{\mathcal{L}}
\newcommand{\R}{\mathbb{R}}
\newcommand{\emp}{\tilde{p}}
\newcommand{\lr}{\alpha}
\newcommand{\reg}{\lambda}
\newcommand{\rect}{\mathrm{rectifier}}
\newcommand{\softmax}{\mathrm{softmax}}
\newcommand{\sigmoid}{\sigma}
\newcommand{\softplus}{\zeta}
\newcommand{\KL}{D_{\mathrm{KL}}}
\newcommand{\Var}{\mathrm{Var}}
\newcommand{\standarderror}{\mathrm{SE}}
\newcommand{\Cov}{\mathrm{Cov}}
% Wolfram Mathworld says $L^2$ is for function spaces and $\ell^2$ is for vectors
% But then they seem to use $L^2$ for vectors throughout the site, and so does
% wikipedia.
\newcommand{\normlzero}{L^0}
\newcommand{\normlone}{L^1}
\newcommand{\normltwo}{L^2}
\newcommand{\normlp}{L^p}
\newcommand{\normmax}{L^\infty}

\newcommand{\parents}{Pa} % See usage in notation.tex. Chosen to match Daphne's book.

\DeclareMathOperator*{\argmax}{arg\,max}
\DeclareMathOperator*{\argmin}{arg\,min}

\DeclareMathOperator{\sign}{sign}
\DeclareMathOperator{\Tr}{Tr}
\let\ab\allowbreak


\usepackage{hyperref}
\usepackage{url}
\usepackage{cleveref}
\usepackage{booktabs}
\usepackage{multirow}
\usepackage{subcaption}
\usepackage{adjustbox} % To adjust table sizes
\usepackage{float}

% \iclrfinalcopy

% For table
\usepackage{multirow}

% For figures
\usepackage{graphicx}
% \usepackage[table]{xcolor}
\title{Improved Training Technique for Latent Consistency Models}

% Authors must not appear in the submitted version. They should be hidden
% as long as the \iclrfinalcopy macro remains commented out below.
% Non-anonymous submissions will be rejected without review.
\iclrfinalcopy

\author{Quan Dao$^{*\dagger}$\\
Rutgers University \\
\texttt{quan.dao@rutgers.edu} \\ \And 
Khanh Doan$^{*}$\\
Movian AI, Vietnam \\
\texttt{dnkhanh.k63.bk@gmail.com} \\ \And
Di Liu\\
Rutgers University \\
\texttt{di.liu@rutgers.edu} \\   \And
Trung Le\\
Monash University \\
\texttt{trunglm@monash.edu} \\   \And
Dimitris Metaxas\\
Rutgers University \\
\texttt{dnm@cs.rutgers.edu} \\
}


% The \author macro works with any number of authors. There are two commands
% used to separate the names and addresses of multiple authors: \And and \AND.
%
% Using \And between authors leaves it to \LaTeX{} to determine where to break
% the lines. Using \AND forces a linebreak at that point. So, if \LaTeX{}
% puts 3 of 4 authors names on the first line, and the last on the second
% line, try using \AND instead of \And before the third author name.

\newcommand{\fix}{\marginpar{FIX}}
\newcommand{\new}{\marginpar{NEW}}
\newcommand{\khanh}[1]{\textcolor{orange}{[Khanh: #1]}}
\newcommand{\quan}[1]{\textcolor{red}{[Quan: #1]}}
\newcommand{\diliu}[1]{\textcolor{purple}{[Di Liu: #1]}}
\newcommand{\trung}[1]{\textcolor{cyan}{[Trung: #1]}}
\newcommand{\metaxas}[1]{\textcolor{blue}{[Metaxas: #1]}}
\newcommand{\minisection}[1]{\noindent{\textbf{#1}}}


%\iclrfinalcopy % Uncomment for camera-ready version, but NOT for submission.
\begin{document}


\maketitle
\def\thefootnote{\textsuperscript{$*$}}\footnotetext{Equal contributions.}
\def\thefootnote{\textsuperscript{$\dagger$}}\footnotetext{Project Lead \& Corresponding Author.}

\begin{abstract}
Consistency models are a new family of generative models capable of producing high-quality samples in either a single step or multiple steps. Recently, consistency models have demonstrated impressive performance, achieving results on par with diffusion models in the pixel space. However, the success of scaling consistency training to large-scale datasets, particularly for text-to-image and video generation tasks, is determined by performance in the latent space. In this work, we analyze the statistical differences between pixel and latent spaces, discovering that latent data often contains highly impulsive outliers, which significantly degrade the performance of iCT in the latent space. To address this, we replace Pseudo-Huber losses with Cauchy losses, effectively mitigating the impact of outliers. Additionally, we introduce a diffusion loss at early timesteps and employ optimal transport (OT) coupling to further enhance performance. Lastly, we introduce the adaptive scaling-$c$ scheduler to manage the robust training process and adopt Non-scaling LayerNorm in the architecture to better capture the statistics of the features and reduce outlier impact. With these strategies, we successfully train latent consistency models capable of high-quality sampling with one or two steps, significantly narrowing the performance gap between latent consistency and diffusion models. The implementation is released here: \url{https://github.com/quandao10/sLCT/}
\end{abstract}

\section{Introduction}
In recent years, generative models have gained significant prominence, with models like ChatGPT excelling in language generation and Stable Diffusion \citep{rombach2021highresolution}. In computer vision, the diffusion model \citep{song2020score, song2019generative, ho2020denoising, sohl2015deep} has quickly popularized and dominated the Adversarial Generative Model (GAN) \citep{goodfellow2014generative}. It is capable of generating high-quality diverse images that beat SoTA GAN models \citep{dhariwal2021diffusion}. Additionally, diffusion models are easier to train, as they avoid the common pitfalls of training instability and the need for meticulous hyperparameter tuning associated with GANs. The application of diffusion spans the entire computer vision field, including text-to-image generation \citep{rombach2021highresolution, gu2022vector}, image editing \citep{meng2021sdedit, cyclediffusion, huberman2024edit, han2024proxedit, he2024dice}, text-to-3D generation \citep{poole2022dreamfusion, wang2024prolificdreamer}, personalization \citep{ruiz2022dreambooth, van2023anti, kumari2023multi} and control generation \citep{zhang2023adding, brooks2022instructpix2pix, zhangli2024layout}. Despite their powerful capabilities, they require thousands of function evaluations for sampling, which is computationally expensive and hinders their application in the real world. Numerous efforts have been made to address this sampling challenge, either by proposing new training frameworks \citep{xiao2021tackling, rombach2021highresolution} or through distillation techniques \citep{meng2023distillation, yin2024one, sauer2023adversarial, dao2024self}. However, methods like \citep{xiao2021tackling} suffer from low recall due to the inherent challenges of GAN training, while \citep{rombach2021highresolution} still requires multi-step sampling. Distillation-based approaches, on the other hand, rely heavily on pretrained diffusion models and demand additional training.

Recently, \citep{song2023consistency} introduced a new family of generative models called the consistency model. Compared to the diffusion model \citep{song2019generative, song2020score, ho2020denoising}, the consistency model could both generate high-quality samples in a single step and multi-steps. The consistency model could be obtained by either consistency distillation (CD) or consistency training (CT). In previous work \citep{song2023consistency}, CD significantly outperforms CT. However, the CD requires additional training budget for using pretrained diffusion, and its generation quality is inherently limited by the pretrained diffusion. Subsequent research \citep{song2023improved} improves the consistency training procedure, resulting in performance that not only surpasses consistency distillation but also approaches SoTA performance of diffusion models. Additionally, several works \citep{kim2023consistency, geng2024consistency} have further enhanced the efficiency and performance of CT, achieving significant results. However, all of these efforts have focused exclusively on pixel space, where data is perfectly bounded. In contrast, most large-scale applications of diffusion models, such as text-to-image or video generation, operate in latent space \citep{rombach2021highresolution, gu2022vector}, as training on pixel space for large-scale datasets is impractical. Therefore, to scale consistency models for large datasets, the consistency must perform effectively in latent space. This work addresses the key question: How well can consistency models perform in latent space? To explore this, we first directly applied the SoTA pixel consistency training method, iCT \citep{song2023improved}, to latent space. The preliminary results were extremely poor, as illustrated in \cref{fig:qualitative_ict}, motivating a deeper investigation into the underlying causes of this suboptimal performance. We aim to improve CT in latent space, narrowing the gap between the performance of latent consistency and diffusion.

We first conducted a statistical analysis of both latent and pixel spaces. Our analysis revealed that the latent space contains impulsive outliers, which, while accounting for a very small proportion, exhibit extremely high values akin to salt-and-pepper noise. We also drew a parallel between Deep Q-Networks (DQN) and the Consistency Model, as both employ temporal difference (TD) loss. This could lead to training instability compared to the Kullback-Leibler (KL) loss used in diffusion models. Even in bounded pixel space, the TD loss still contains impulsive outliers, which \citep{song2023improved} addressed by proposing the use of Pseudo-Huber loss to reduce training instability. As shown in \cref{fig:impulsive_noise}, the latent input contains extremely high impulsive outliers, leading to very large TD values. Consequently, the Pseudo-Huber loss fails to sufficiently mitigate these outliers, resulting in poor performance as demonstrated in \cref{fig:qualitative_ict}. To overcome this challenge, we adopt Cauchy loss, which heavily penalizes extremely impulsive outliers. Additionally, we introduce diffusion loss at early timesteps along with optimal transport (OT) matching, both of which significantly enhance the model's performance. Finally, we propose an adaptive scaling $c$ schedule to effectively control the robustness of the model, and we incorporate Non-scaling LayerNorm into the architecture. With these techniques, we significantly boost the performance of latent consistency model compared to the baseline iCT framework and bridge the gap between the latent diffusion and consistency training.

\section{Related Works} \label{related}
% \subsection{Diffusion Model and Fast Sampling Technique}
% Diffusion models \citep{song2020score, song2019generative, ho2020denoising} have recently been raised as the most powerful generative model and outperform GAN \citep{goodfellow2014generative} in many applications. Diffusion models can generate high-fidelity images and possess good mode coverage, allowing diverse samples compared to GAN. However, diffusion models require many function evaluations (NFEs) during inference time. This drawback hinders its application in the real world. Many works are trying to tackle this drawback and achieve promising results. They can be divided into two main research categories: training from scratch and building upon pretrained diffusion models. Following the first category, there are several works, such as DDGAN \citep{xiao2021tackling}, LDM \citep{rombach2021highresolution}, and VQDiff \citep{gu2022vector}. DDGAN \citep{xiao2021tackling} proposes to use a larger step size in the forward process to reduce the NFEs; they use GAN models to implicitly learn the backward transition. Even though DDGAN \citep{xiao2021tackling} requires only a few sampling timesteps, it still suffers from low recall due to mode collapse from GAN. LDM \citep{rombach2021highresolution} does not directly reduce the number of sampling time steps; instead, it compresses the image to latent with a much smaller resolution. By training on latent space, LDM \citep{rombach2021highresolution} can significantly reduce both time and memory budget, and the inference is much faster than other pixel diffusion models. LDM \citep{rombach2021highresolution} has become the core technique in many large-scale diffusion models. Most real-world applications rely on LDM since it allows to scale diffusion models up on enormous high-resolution training datasets, which is impractical if training pixel diffusion models. In the second category, several works such as \citep{lu2022dpm, zhang2022fast} propose the high-order solver during inference. These works could successfully reduce sample NFEs to 10 without any training. However, they cannot sample with little NFEs such as 1 or 2. The other line of work is a distillation-based method. Progressive distillation \citep{salimans2022progressive} proposes to progressively distill diffusion model; each stage distills to reduce the sampling NFEs by half. This technique is costly since it requires to train many stages. Later works such as Guided-Distill \citep{meng2023distillation}, Swiftbrush \citep{nguyen2024swiftbrush}, DMD \citep{yin2024one}, and UFOGEN \citep{xu2024ufogen} manage to distill diffusion into few-step generation without compromising generative quality. The major drawback of these techniques is that additional training is required.
% Furthermore, some techniques, such as Swiftbrush \citep{nguyen2024swiftbrush} and DMD \citep{yin2024one}, do not have few-step sampling. Other methods, such as UFOGEN \citep{xu2024ufogen} and ADD \citep{sauer2023adversarial}, require training GAN, which could lead to training instability and low mode coverage. A standout among these techniques is the consistency model. The consistency model \citep{song2023consistency} is defined based on probability flow ODE (PF-ODE), allowing single- and multi-step sampling. The consistency model could be achieved via training from scratch and distillation from the diffusion model.

% \subsection{Consistency Model}

Consistency model \citep{song2023consistency, song2023improved} proposes a new type of generative model based on PF-ODE, which allows 1, 2 or multi-step sampling. The consistency model could be obtained by either training from scratch using an unbiased score estimator or distilling from a pretrained diffusion model. Several works improve the training of the consistency model. ACT \citep{kong2023act}, CTM \citep{kim2023consistency} propose to use additional GAN along with consistency objective. While these methods could improve the performance of consistency training, they require an additional discriminator, which could need to tune the hyperparameters carefully. MCM \citep{heek2024multistep} introduces multistep consistency training, which is a combination of TRACT \citep{berthelot2023tract} and CM \citep{song2023consistency}. MCM increases the sampling budget to 2-8 steps to tradeoff with efficient training and high-quality image generation. ECM \citep{geng2024consistency} initializes the consistency model by pretrained diffusion model and fine-tuning it using the consistency training objective. ECM vastly achieves improved training times while maintaining good generation performance. However, ECM requires pretrained diffusion model, which must use the same architecture as the pretrained diffusion architecture. Although these works successfully improve the performance and efficiency of consistency training, they only investigate consistency training on pixel space. As in the diffusion model, where most applications are now based on latent space, scaling the consistency training \citep{song2023consistency, song2023improved} to text-to-image or higher resolution generation requires latent space training. Otherwise, with pretrained diffusion model, we could either finetune consistency training \citep{geng2024consistency} or distill from diffusion model \citep{song2023consistency, luo2023latent}. CM \citep{song2023consistency} is the first work proposing consistency distillation (CD) on pixel space. LCM \citep{luo2023latent} later applies consistency technique on latent space and can generate high-quality images within a few steps. However, LCM's generated images using 1-2 steps are still blurry \citep{luo2023latent}. Recent works, such as Hyper-SD \cite{ren2024hyper} and TCD \cite{zheng2024trajectory}, have introduced notable improvements to latent consistency distillation. TCD \cite{zheng2024trajectory} employed CTM \cite{kim2023consistency} instead of CD \cite{song2023consistency}, significantly enhancing the performance of the distilled student model. Building on this, Hyper-SD \cite{ren2024hyper} divided the Probability Flow ODE (PF-ODE) into multiple components inspired by Multistep Consistency Models (MCM) \cite{heek2024multistep}, and applied TCD \cite{zheng2024trajectory} to each segment. Subsequently, Hyper-SD \cite{ren2024hyper} merged these segments progressively into a final model, integrating human feedback learning and score distillation \cite{yin2024one} to optimize one-step generation performance.

\section{Preliminaries} \label{sec:bg}
Denote $\pdata(\rvx_0)$ as the data distribution, the forward diffusion process gradually adds Gaussian noise with monotonically increasing standard deviation $\sigma(t)$ for $t \in \{0,1,\dots,T\}$ such that $p_t(\rvx_t|\rvx_0) = \gN(\rvx_0, \sigma^2(t)\mI)$ and $\sigma(t)$ is handcrafted such that $\sigma(0) = \sigma_{\min}$ and $\sigma(T)=\sigma_{\max}$. By setting $\sigma(t) = t$, the probability flow ODE (PF-ODE) from \citep{Karras2022edm} is defined as:
\begin{equation}
    \frac{\rd\rvx_t}{\rd t} = -t\nabla_{\rvx_t} \log p_t(\rvx_t) = \frac{\left( \rvx_t - \vf(\rvx_t, t) \right)}{t},  \label{eq:pf_ode}
\end{equation}
where $\vf:(\rvx_t, t) \rightarrow \rvx_0$ is the denoising function which directly predicts clean data $\rvx_0$ from given perturbed data $\rvx_t$. 
\citep{song2023consistency} defines consistency model based on PF-ODE in \cref{eq:pf_ode}, which builds a bijective mapping $\vf$ between noisy distribution $p(\rvx_t)$ and data distribution $\pdata(\rvx_0)$. The bijective mapping $\vf:(\rvx_t, t) \rightarrow \rvx_0$ is termed the consistency function. A consistency model $\vf_\theta(\rvx_t, t)$ is trained to approximate this consistency function $\vf(\rvx_t, t)$. The previous works \citep{song2023consistency, song2023improved, Karras2022edm} impose the boundary condition by parameterizing the consistency model as:
\begin{equation}
    \vf_\theta(\rvx_t, t) = c_{skip}(t)\rvx_t + c_{out}(t)\mF_\theta(\rvx_t, t), \label{eq:cm_param}
\end{equation}
where $\mF_\theta(\rvx_t, t)$ is a neural network to train. Note that, since $\sigma(t) = t$, we hereafter use $t$ and $\sigma$ interchangeably. $c_{skip}(t)$ and $c_{out}(t)$ are time-dependent functions such that $c_{skip}(\sigma_{\min}) = 1$ and $c_{out}(\sigma_{\max}) = 0$.

To train or distill consistency model, \citep{song2023consistency, song2023improved, Karras2022edm} firstly discretize the PF-ODE using a sequence of noise levels $\sigma_{\min} = t_{\min} = t_1 < t_2 < \dots < t_{N} = t_{\max} = \sigma_{\max}$, where $t_i = \left( t_{\min}^{1/\rho} + \frac{i-1}{N-1}(t_{\max}^{1/\rho
} - t_{\min}^{1/\rho})\right)^\rho$ and $\rho = 7$. 

\textbf{Consistency Distillation} Given the pretrained diffusion model $\vs_\phi(\rvx_t, t) \approx \nabla_{\rvx_t} \log p_t(\rvx_t)$, the consistency model could be distilled from the pretrained diffusion model using the following CD loss:
\begin{equation}
    \gL_{\text{CD}}(\theta, \theta^-) = \E\left[ \lambda(t_i)d(\vf_\theta(\rvx_{t_{i+1}}, t_{i+1}), \vf_{\theta^{-}}(\Tilde{\rvx}_{t_i}, t_{i})) \right], \label{loss:cd}
\end{equation}
where $\rvx_{t_{i+1}} = \rvx_0 + t_{i+1} \rvz$ with the $\rvx_0 \sim \pdata(\rvx_0)$ and $\rvz \sim \gN(0, \mI)$ and $\rvx_{t_i} = \rvx_{t_{i+1}} - (t_{i}-t_{i+1})t_{i+1} \nabla_{\rvx_{t_{i+1}}} \log p_{t_{i+1}}(\rvx_{t_{i+1}}) = \rvx_{t_{i+1}} - (t_{i}-t_{i+1})t_{i+1}\vs_\phi(\rvx_{t_{i+1}}, t_{i+1})$. 

\textbf{Consistency Training}
The consistency model is trained by minimizing the following CT loss:
\begin{equation}
    \gL_{\text{CT}}(\theta, \theta^-) = \E\left[ \lambda(t_i)d(\vf_\theta(\rvx_{t_{i+1}}, t_{i+1}), \vf_{\theta^{-}}(\rvx_{t_i}, t_{i})) \right], \label{loss:ct}
\end{equation}
where $\rvx_{t_i} = \rvx_0 + t_{i} \rvz$ and $\rvx_{t_{i+1}} = \rvx_0 + t_{i+1} \rvz$ with the same $\rvx_0 \sim \pdata(\rvx_0)$ and $\rvz \sim \gN(0, \mI)$

In \cref{loss:cd} and \cref{loss:ct}, $\vf_\theta$ and $\vf_{\theta^-}$ are referred to as the online network and the target network, respectively. The target's parameter $\theta^-$ is obtained by applying the Exponential Moving Average (EMA) to the student's parameter $\theta$ during the training and distillation as follows:
\begin{equation}
    \theta^- \leftarrow \text{stopgrad}(\mu\theta^- + (1-\mu)\theta), \label{ema}
\end{equation}
with $0\leq\mu<1$ as the EMA decay rate,  weighting function $\lambda(t_i)$ for each timestep $t_i$, and $d(\cdot, \cdot)$ is a predefined metric function. 

In CM \citep{song2023consistency}, the consistency training still lags behind the consistency distillation and diffusion models. iCT \citep{song2023improved} later propose several improvements that significantly boost the training performance and efficiency. First, the EMA decay rate $\mu$ is set to $0$ for better training convergence. Second, the Fourier scaling factor of noise embedding and the dropout rate are carefully examined. Third, iCT introduces Pseudo-Huber losses to replace $L_2$ and LPIPS since LPIPS introduces the undesirable bias in generative modeling \citep{song2023improved}. Furthermore, the Pseudo-Huber is more robust to outliers since it imposes a smaller penalty for larger errors than the $L_2$ metric. Fourth, iCT proposes an exp curriculum for total discretization steps N, which doubles N after a predefined number of training iterations. Moreover, uniform weighting $\lambda(t_i) = 1$ is replaced by $\lambda(t_i)=1/(t_{i+1}-t_i)$. Finally, iCT adopts a discrete Lognormal distribution for timestep sampling as EDM \citep{Karras2022edm}. With all these improvements, CT is now better than CD and performs on par with the diffusion models in pixel space.

\section{Method}
\label{method}
In this paper, we first investigate the underlying reason behind the performance discrepancy between latent and pixel space using the same training framework in \cref{sec:analysis}. Based on the analysis, we find out the root of unsatisfied performance on latent space could be attributed to two factors: the impulsive outlier and the unstable temporal difference (TD) for computing consistency loss. To deal with impulsive outliers of TD on pixel space, \citep{song2023improved} proposes the Pseudo-Huber function as training loss. For the latent space, the impulsive outlier is even more severe, making Pseudo-Huber loss not enough to resist the outlier. Therefore,  \cref{sec:cauchy} introduces Cauchy loss, which is more effective with extreme outliers. In the next \cref{sec:diff_loss} and \cref{sec:ot}, we propose to use diffusion loss at early timesteps and OT matching for regularizing the overkill effect of consistency at the early step and training variance reduction, respectively. Section \ref{sec:c} designs an adaptive scheduler of scaling $c$ to control the robustness of the proposed loss function more carefully, leading to better performance. Finally, in \cref{sec:norm}, we investigate the normalization layers of architecture and introduce Non-scaling LayerNorm to both capture feature statistic better and reduce the sensitivity to outliers.

\subsection{Analysis of latent space} \label{sec:analysis}

We first reimplement the iCT model \citep{song2023improved} on the latent dataset CelebA-HQ $32 \times 32 \times 4$ and pixel dataset Cifar-10 $32 \times 32 \times 3$. Hereafter, we refer to the latent iCT model as iLCT. We find that iCT framework works well on pixel datasets as claim \citep{song2023improved}. However, it produces worse results on latent datasets as in \cref{fig:qualitative_ict} and \cref{tab:main_exp}. The iLCT gets a very high FID above 30 for both datasets, and the generative images are not usable in the real world. This observation raises concern about the sensitivity of CT algorithm with training data, and we should carefully examine the training dataset. In addition, we notice that the DQN and CM use the same TD loss, which update the current state using the future state. Furthermore, they also possess the training instability. This motivates to carefully examine the behavior of TD loss with different training data.


While the pixel data lies within the range $[-1, 1]$ after being normalized, the range of latent data varies depending on the encoder model, which is blackbox and unbound. After normalizing latent data using mean and variance, we observe that the latent data contains high-magnitude values. We call them the impulsive outliers since they account for small probability but are usually very large values. In the bottom left of \cref{fig:impulsive_noise}, the impulsive outlier of latent data is red, spanning from $-9$ to $7$, while the first and third quartiles are just around $-1.4$ and $1.4$, respectively. We evaluate how the iCT will be affected by data outliers by analyzing the temporal difference $\text{TD} = f_\theta(\rvx_{t_{i+1}}, t_{i+1})-f_{\theta^-}(\rvx_{t_i}, t_{i})$. In the top right of \cref{fig:impulsive_noise}, the impulsive outliers of pixel TD range from -1.5 to 1.7, which are not too far from the interquartile range compared to latent TD. The impulsive outliers of latent TD range is much wider from -3.2 to 5. iCT uses Pseudo-Huber loss instead of $L_2$ loss since the Huber is less sensitive to outliers, see \cref{fig:loss}. However, for latent data, the Huber's reduction in sensitivity to outliers is not enough. This indicates that even using Pseudo-Huber loss, the iLCT training on latent space could still be unstable and lead to worse performance, which matches our experiment results on iLCT. Based on the above analysis, we hypothesize that the TD value statistic highly depends on the training data statistic.

\begin{figure}[!t]
    \centering
    \includegraphics[width=0.85\linewidth]{figures/impulsive_noise.pdf}
    \caption{\textbf{Box and Whisker Plot:} Impulsive noise comparison between pixel and latent spaces. The right column shows the statistics of TD values at 21 discretization steps. Other discretization steps exhibit same behavior, where impulsive outliers are consistently present regardless of the total discretization steps. The blue boxes represent interquartile ranges of the data, while the green and orange dashed lines indicate inner and outer fences, respectively. Outliers are marked with red dots.}
    \label{fig:impulsive_noise}
\end{figure}

%To understand the root of TD's impulsive outlier, we look into Deep Q Learning (DQN) from Reinforcement Learning (RL). There is a strong correlation between DQN and CM. While the DQN uses Q-value of future state as the ground truth for Q-value of the current state, the CM updates the current timesteps $f(\rvx_{t_{i+1}}, t_{i+1})$ using the smaller timesteps $f(\rvx_{t_{i}}, t_{i})$. This loss type is called the temporal difference in RL. For stable training, DQN uses target network $\theta^-$ to estimate Q-value of future state, and CM similarly adopts the same technique for consistency loss. The target network $\theta^-$ could be updated differently, such as Polyak-averaging, periodic, and standard TD updates \citep{lee2019target}. The Polyak-averaging update is simply EMA update using in \citep{song2023consistency}, and standard update corresponds to \citep{song2023improved} which $\theta^- \leftarrow \theta$ every iteration. The periodic update is a standard update but after every fixed number of iterations. In RL, Polyak-averaging and periodic updates are more stable but slowly convergent \citep{lee2019target}. Even using these stable target updates, there is still instability in TD training. Since the target network needs to change along with the online model, the target value of TD can never be fixed, which makes the loss highly oscillate. Therefore, even though the pixel data is very well-bounded within [-1, 1], the CM training is still affected by impulsive outliers due to the nature of TD loss.

To mitigate the impact of impulsive outliers, we could use more stable target updates like Polyak or periodic in TD loss \cite{lee2019target}, but they lead to very slow convergence, as shown in \citep{song2023consistency}. Even though CM is initialized by a pretrained diffusion model, the Polyak update still takes a long time to converge. Therefore, using Polyak or periodic updates is computationally expensive, and we keep the standard target update as in \citep{song2023improved}. Another direction is using a special metric for latent like LPIPS on pixel space \citep{song2023consistency}. \citep{kang2024diffusion2gan} proposes the E-LatentLPIPS as a metric for distillation and performs well on distillation tasks. However, this requires training a network as a metric and using this metric during the training process will also increase the training budget. To avoid the overhead of the training, we seek a simple loss function like Pseudo-Huber but be more effective with outliers. We find that the Cauchy loss function \citep{black1996robust, barron2019general} could be a promising candidate in place of Pseudo-Huber for latent space.
\subsection{Cauchy Loss against Impulsive Outlier} \label{sec:cauchy}
In this section, we introduce the Cauchy loss \citep{black1996robust, barron2019general} function to deal with extreme impulsive outliers. The Cauchy loss function has the following form:
\begin{equation}
    d_{\text{Cauchy}}(\rvx, \rvy)=  \log \left(1+\frac{||\rvx-\rvy||_2^2}{2c^2}\right), \label{loss:cauchy}
\end{equation}
and we also consider two additional robust losses, which are Pseudo-Huber \citep{song2023improved, barron2019general} and Geman-McClure \citep{geman1986bayesian, barron2019general}
\begin{equation}
    d_{\text{Pseudo-Huber}}(\rvx, \rvy)= \sqrt{||\rvx-\rvy||_2^2 + c^2} - c, \label{loss:huber}
\end{equation}
\begin{equation}
    d_{\text{Geman-McClure}}(\rvx, \rvy)= \frac{2||\rvx-\rvy||_2^2}{||\rvx-\rvy||_2^2 + 4c^2}, \label{loss:gm}
\end{equation}
where $c$ is the scaling parameter to control how robust the loss is to the outlier. We analyze their robustness behavior against outliers. As shown in \cref{fig:loss_val}, the Pseudo-Huber loss linearly increases like $L_1$ loss for the large residuals $\rvx-\rvy$. In contrast, the Cauchy loss only grows logarithmically, and the Geman-McClure suppresses the loss value to $1$ for the outliers. 

The Pseudo-Huber loss works well if the residual value does not grow too high and, therefore, has a good performance on the pixel space. However, for the latent space, as shown in the bottom right of \cref{fig:impulsive_noise}, the TD suffers from extremely high values coming from the impulsive outlier in the latent dataset, the Cauchy loss could be more suitable since it significantly dampens the influence of extreme outliers. Otherwise, even Geman-McClure is very highly effective for removing outlier effects than two previous losses; it gives a gradient $0$ for high TD value and completely ignores the impulsive outliers as \cref{fig:loss_derivative}. This is unexpected behavior because even though we call the high-value latent impulsive outlier, they actually could encode important information from original data. Completely ignoring them could significantly hurt the performance of training model. Based on this analysis, we choose Cauchy loss as the default loss for latent CM for the rest of the paper. The loss ablation is provided in \cref{tab:ablate_robust}.


\begin{figure}[!ht]
    \centering
    \begin{subfigure}[t]{0.40\textwidth}
        \centering
        \includegraphics[width=1.0\textwidth]{figures/func.png}
        \caption{Robust Loss}
        \label{fig:loss_val}
    \end{subfigure}%
    ~ 
    \begin{subfigure}[t]{0.40 \textwidth}
        \centering
        \includegraphics[width=1.0\textwidth]{figures/derivative.png}
        \caption{Derivative of Robust Loss}
        \label{fig:loss_derivative}
    \end{subfigure}
    \caption{Analysis of robust loss: Pseudo-Huber, Cauchy, and Geman-McClure}
    \label{fig:loss}
\end{figure}
% \vspace{-3mm}


\subsection{Diffusion Loss at small timestep} \label{sec:diff_loss}
For small noise level $\sigma$, the ground truth of $f(\rvx_\sigma, \sigma)$ can be well approximated by $\rvx_0$, but this does not hold for large noise levels. Therefore, for low-level noise, the consistency objective seems to be overkill and harms the model's performance since instead of optimizing $f_\theta(\rvx_\sigma, \sigma)$ to approximated ground truth $\rvx_0$, the consistency objective optimizes through a proxy estimator $f_{\theta^-}(\rvx_{<\sigma}, <\sigma)$ leading to error accumulation over timestep. To regularize this overkill, we propose to apply an additional diffusion loss on small noise level as follows:

\begin{equation}
    L_{diff} = ||f_\theta(\rvx_{t_i}, t_i) - \rvx_0||^2_2 \quad \forall i \leq \text{int(N $\cdot$ r)}, \label{loss:diff}
\end{equation}

where N is the number of training discretization steps and $r\in[0;1]$ is the diffusion threshold, and we heuristicly choose $r=0.25$. We do not apply diffusion loss for large noise levels since $f(\rvx_\sigma, \sigma)$ will differ greatly from the target $\rvx_0$, leading to very high $L_2$ diffusion loss. This could harm the training consistency process, misleading to the wrong solution. We provide the ablation study in \cref{tab:diff_loss}. Furthermore, CTM \citep{kim2023consistency} also proposes to use diffusion loss, but they use them on both high and low-level noise, which is different from us. 

\subsection{OT matching reduces the variance} \label{sec:ot}
% \vspace{-5mm}
In this section, we adopt the OT matching technique from previous works \citep{pooladian2023multisample, lee2023minimizing}. \citep{pooladian2023multisample} proposes to use OT to match noise and data in the training batch, such as the moving $L_2$ cost is optimal. On the other hand, \citep{lee2023minimizing} introduces $\beta\text{VAE}$ for creating noise corresponding to data and train flow matching on the defined data-noise pairs. By reassigning noise-data pairs, these works significantly reduce the variance during the diffusion/flow matching training process, leading to a faster and more stable training process. According to \citep{zhang2023emergence}, the consistency training and diffusion models produce highly similar images given the same noise input. Therefore, the final output solution of the consistency and diffusion models should be close to each other. Since OT matching helps reduce the variance during training diffusion, it could be useful to reduce the variance of consistency training. In our implementation, we follow \citep{pooladian2023multisample, tong2023improving} using the POT library to map from noise to data in the training batch. The overhead caused by minibatch OT is relatively small, only around $0.93\%$ training time, but gains significant performance improvement as shown in \cref{tab:strategy}.

\subsection{Adaptive $c$ scheduler} \label{sec:c}

% \begin{figure}[h!]
%     \centering
%     \includegraphics[width=0.5\linewidth]{figures/C_by_NFE.pdf}
%     \caption{Our robust adaptive $c$ scheduler}
%     \label{fig:proposed_c}
% \end{figure}

\begin{figure}[h!]
    \centering
    \includegraphics[width=0.8\linewidth]{figures/C_merge.pdf}
    \caption{Model convergence plot on different $c$ schedule. (Left) Our proposed $c$ values. Performance on FID (Middle) and Recall (Right) of our proposed $c$ in comparison with different choices.}
    \label{fig:fid_vary_c}
\end{figure}
% \vspace{-5mm}

In this section, we examine the choice of scaling parameter $c$ in robust loss functions. The scaling parameter controls the robustness level, which is very important for model performance. The previous work \citep{song2023improved} proposes to use fixed constant $c_0 = 0.00054\sqrt{d}$, where $d$ is the dimension of data. We find that using this simple fixed $c$ is not yet optimal for the training consistency model. Especially in this paper, we follow the Exp curriculum specified by \cref{exp_cur} in \citep{song2023improved}, which doubles the total discretization step after a defined number of training iterations. 
\begin{equation}
    \text{NFE}(k)=\min \left(s_0 2^{\left\lfloor\frac{k}{K^{\prime}}\right\rfloor}, s_1\right)+1, \quad K^{\prime}=\left\lfloor\frac{K}{\log _2\left\lfloor s_1 / s_0\right\rfloor+1}\right\rfloor, \label{exp_cur}
\end{equation}
where $k$ is current training iteration, $K$ is total training iteration and $s_0 = 10, s_1=640$. During training, we notice that the variance of TD is significantly reduced as doubling total discretization steps using \cref{exp_cur}. Since the more discretization steps, the closer distance of $\rvx_{t_i}$ and $\rvx_{t_{i+1}}$, the TD value's range between them should be smaller. However, the impulsive outlier still exists regardless of the number of discretization steps. Intuitively, we propose a heuristic adaptive $c$ scheduler where the $c$ is scaled down proportional to the reduction rate of TD variance as the number of discretization steps increases. We plot our $c$ scheduler versus discretization steps in \cref{fig:fid_vary_c} and we fit the $c$ scheduler to get the scheduler equation as following:

\begin{equation}
    c = \exp(-1.18 * \log(\text{NFE}(k) - 1) - 0.72) \label{eq:c_scheduler}
\end{equation}

\subsection{Non-scaling Layernorm} \label{sec:norm}
As mentioned in \cref{sec:analysis}, the statistic of training data could play an important role in the success of consistency training. Furthermore, in architecture design, the normalization layer specifically handles the statistics of input, output, and hidden features. In this section, we investigate the normalization layer choice for consistency training, which is sensitive to training data statistics. 

Currently, both \citep{song2023improved, song2023consistency} use the UNet architecture from \citep{dhariwal2021diffusion}. In UNet \citep{dhariwal2021diffusion}, GroupNorm is used in every layer by default. The GroupNorm only captures the statistics over groups of local channels, while the LayerNorm further captures the statistics' overall features. Therefore, LayerNorm is better at capturing fine-grained statistics over the entire feature. We further carry out the experiments for other types of normalization, such as LayerNorm, InstanceNorm, RMSNorm in \cref{tab:norm_layer} and observe that the GroupNorm and InstanceNorm perform relatively well compared to others, especially LayerNorm. This could be due to that they are less sensitive to the outliers since they only capture the statistic over groups of channels. Therefore, the impulsive features only affect the normalization of a group containing them. For the LayerNorm, the impulsive features could negatively impact the overall features's normalization. We further look into the LayerNorm implementation and suspect that the scaling term could significantly amplify the outliers across features by serving as a shared parameter. This observation is also mentioned in \citep{wei2022outlier} for LLM quantization. In implementation, we set the \textbf{scaling term of LayerNorm to $1$} and \textbf{disabled the gradient update} for it \eqref{operation:layernorm}. We refer to it as Non-scaling LayerNorm (NsLN) as \citep{wei2022outlier}.

\begin{equation}
    \text{LN}_{\gamma, \beta}(\rvx) = \frac{\rvx - u(\rvx)}{\sqrt{\sigma^{2}(\rvx) + \epsilon}} \cdot \gamma + \beta, \quad
    \text{NsLN}_{\beta}(\rvx) = \frac{\rvx - u(\rvx)}{\sqrt{\sigma^{2}(\rvx) + \epsilon}} + \beta, \label{operation:layernorm}
\end{equation}

where $u(\rvx)$ and $\sigma^{2}(\rvx)$ are mean and variance of $\rvx$.

% \subsection{Improve Consistency Distillation}

% \vspace{-15mm}
\section{Experiment} \label{exp}

\subsection{Performance of our training technique} \label{exp:main}
% \vspace{-3mm}
\begin{table}[t]
    \centering
    \begin{tabular}{cc}
        \begin{minipage}[c]{0.58\textwidth}
            \centering
            \begin{subtable}[t]{\textwidth}
                \resizebox{\textwidth}{!}{%
                \begin{tabular}{l c c c c c}
                    \toprule
                    Model & NFE$\downarrow$ & FID$\downarrow$ & Recall$\uparrow$ & Epochs & Total Bs\\
                    \midrule 
                    \multicolumn{5}{c}{\textbf{Pixel Diffusion Model}}\\
                    \midrule
                    WaveDiff \citep{phung2023wavediff} & 2 & 5.94 & 0.37 & 500 & 64\\
                    Score SDE \citep{song2020score} & 4000 & 7.23 & - & ~6.2K & - \\
                    DDGAN \citep{xiao2021tackling} & 2 & 7.64 & 0.36 & 800 & 32 \\
                    RDUOT \citep{dao2024high} & 2 & 5.60 & 0.38 & 600 & 24 \\
                    RDM \citep{teng2023relay} & 270 & 3.15 & 0.55 & 4K & - \\
                    UNCSN++ \citep{kim2021soft} & 2000 & 7.16 & - & - & -\\
                    \midrule 
                    \multicolumn{5}{c}{\textbf{Latent Diffusion Model}}\\
                    \midrule
                    LFM-8 \citep{dao2023flow} & 85 & 5.82 & 0.41 & 500 & 112\\ 
                    LDM-4 \citep{rombach2021highresolution} & 200 & 5.11 & 0.49 & 600 &48 \\
                    LSGM \citep{vahdat2021score} & 23 & 7.22 & - & 1K &-\\
                    DDMI \citep{park2024ddmi} & 1000 & 7.25 & - & - &-\\
                    
                    DIMSUM \citep{phung2024dimsum} & 73 & 3.76  & 0.56 & 395 &32\\
                    $\text{LDM-8}^\dagger$ & 250 & {8.85}  & - & 1.4K &128\\
                    
                    \midrule
                    \multicolumn{5}{c}{\textbf{Latent Consistency Model}}\\
                    \midrule
                    iLCT \citep{song2023improved} & 1 & 37.15 & 0.12 & 1.4K &128\\
                    iLCT \citep{song2023improved} & 2 & 16.84 & 0.24 & 1.4K &128\\
                    Ours  & 1 & 7.27 & 0.50 & 1.4K &128\\
                    Ours  & 2 & 6.93 & 0.52 & 1.4K &128\\
                    \bottomrule
                \end{tabular}%
                }
            \caption{CelebA-HQ}
            \label{tab:celeb}
            \end{subtable}
        \end{minipage}
        \hfill
        \begin{minipage}[c]{0.42\textwidth}
            \centering
            \begin{subtable}[t]{\textwidth}
                \resizebox{\textwidth}{!}{%
                \begin{tabular}{l c c c c c}
                    \toprule
                    Model & NFE$\downarrow$ & FID$\downarrow$ & Recall$\uparrow$ & Epochs & Total Bs\\
                    \midrule 
                    \multicolumn{5}{c}{\textbf{Pixel Diffusion Model}}\\
                    \midrule
                    WaveDiff \citep{phung2023wavediff} & 2 & 5.94 & 0.37 & 500 & 64\\
                    Score SDE \citep{song2020score} & 4000 & 7.23 & - &6.2K & -\\
                    DDGAN \citep{xiao2021tackling} & 2 & 5.25 & 0.36 & 500 & 32\\
                    \midrule 
                    \multicolumn{5}{c}{\textbf{Latent Diffusion Model}}\\
                    \midrule
                    LFM-8 \citep{dao2023flow} & 90 & 7.70 & 0.39 & 90 &112\\
                    LDM-8 \citep{rombach2021highresolution} & 400 & 4.02 & 0.52 & 400 &96\\
                    $\text{LDM-8}^\dagger$ & 250 & {10.81} & - & 1.8K &256\\
                    \midrule
                    \multicolumn{5}{c}{\textbf{Latent Consistency Model}}\\
                    \midrule
                    iLCT \citep{song2023improved} & 1 &52.45  &0.11  & 1.8K &256\\
                    iLCT \citep{song2023improved} & 2 &24.67  &0.17  & 1.8K &256\\
                    Ours  & 1 &8.87  &0.47  & 1.8K &256\\
                    Ours  & 2 &7.71  &0.48  & 1.8K &256\\
                    \bottomrule
                \end{tabular}%
                }
            \caption{LSUN Church}
            \label{tab:lsun}
            \end{subtable}
            \hfill
            \begin{subtable}[t]{\textwidth}
                \resizebox{\textwidth}{!}{%
                \begin{tabular}{l c c c c c}
                    \toprule
                    Model & NFE$\downarrow$ & FID$\downarrow$ & Recall$\uparrow$ & Epochs &Total Bs \\
                    \midrule 
                    \multicolumn{5}{c}{\textbf{Latent Diffusion Model}}\\
                    \midrule
                    LFM-8 \citep{dao2023flow} & 84 & 8.07 & 0.40 & 700 &128\\
                    LDM-4 \citep{rombach2021highresolution} & 200 & 4.98 & 0.50 & 400 &42\\
                    $\text{LDM-8}^\dagger$ & 250 &{10.23} & - & 1.4K &128\\
                    \midrule
                    \multicolumn{5}{c}{\textbf{Latent Consistency Model}}\\
                    \midrule
                    iLCT \citep{song2023improved} & 1 & 48.82  & 0.15 & 1.4K &128 \\
                    iLCT \citep{song2023improved} & 2 & 21.15 & 0.19 & 1.4K &128\\
                    Ours  & 1 & 8.72  &0.42 & 1.4K &128\\
                    Ours  & 2 & 8.29  &0.43  & 1.4K &128\\
                    \bottomrule
                \end{tabular}%
                }
            \caption{FFHQ}
            \label{tab:ffhq}
            \end{subtable}
        \end{minipage}
    \end{tabular}
    \caption{Our performance on CelebA-HQ, LSUN Church, FFHQ datasets at resolution $256 \times 256$. ($\dagger$) means training on our machine with the same diffusion forward and equivalent architecture.}
    \label{tab:main_exp}
\end{table}


\minisection{Experiment Setting:}
We measure the performance of our proposed technique on three datasets: CelebA-HQ \citep{celeba}, FFHQ \citep{karras2019style}, and LSUN Church \citep{lsun}, at the same resolution of $256 \times 256$. Following LDM \citep{rombach2021highresolution}, we use pretrained VAE KL-8 \footnote{https://huggingface.co/stabilityai/sd-vae-ft-ema} to obtain latent data with the dimensionality of $32 \times 32 \times 4$. We adopt the OpenAI UNet architecture \citep{dhariwal2021diffusion} as the default architecture throughout the paper. Furthermore, we use the variance exploding (VE) forward process for all the consistency and diffusion experiments following \citep{song2023consistency, song2023improved}. The baseline iCT is self-implemented based on official implementation CM \citep{song2023consistency} and iCT \citep{song2023improved}. We refer to this baseline as iLCT. Furthermore, we also train the latent diffusion model for each dataset using the same VE forward noise process for fair comparisons with our technique. This LDM model is referred to as $\text{LDM-8}^{\dagger}$ in \cref{tab:main_exp}. All three frameworks, including ours, iLCT, and $\text{LDM-8}^{\dagger}$, use the same architecture.

\minisection{Evaluation:} During the evaluation, we first generate 50K latent samples and then pass them through VAE's decoder to obtain the pixel images. We use two well-known metrics, Fréchet Inception Distance (FID) \citep{fid} and Recall \citep{kynkaanniemi2019improved}, for measuring the performance of the model given the training data and 50K generated images. 

\minisection{Model Performance:} We report the performance of our model across all three datasets in \cref{tab:main_exp}, primarily to compare it with the baseline iLCT \citep{song2023improved} and LDM \citep{rombach2021highresolution}. For both 1 and 2 NFE sampling, we observe that the FIDs of iLCT for all datasets are notably high (over 30 for 1-NFE sampling and over 16 for 2-NFE sampling), consistent with the qualitative results shown in \cref{fig:qualitative_ict}, where the generated image is unrealistic and contain many artifacts. This poor performance of iLCT in latent space is expected, as the Pseudo-Huber training losses are insufficient in mitigating extreme impulsive outliers, as discussed in \cref{sec:analysis} and \cref{sec:cauchy}. In contrast, our proposed framework demonstrates significantly better FID and Recall than iLCT. Specifically, we achieve 1-NFE sampling FIDs of 7.27, 8.87, and 8.29 for CelebA-HQ, LSUN Church, and FFHQ, respectively. For 2-NFE sampling, our FID scores improve across all three datasets. Notably, our 1-NFE sampling outperforms $\text{LDM-8}^{\dagger}$, using the same noise scheduler and architecture. However, our models still exhibit higher FIDs compared to LDM \citep{rombach2021highresolution} and LFM \citep{dao2023flow}. In contrast, we only need 1 or 2 timestep sampling, whereas they require multiple timesteps for high-fidelity generation.
 It's important to note that we employ the VE forward process, whereas these other methods use VP and flow-matching forward processes. Furthermore, the qualitative results of our framework, as shown in \cref{fig:qualitative_1nfe}, highlight our ability to generate high-quality images.

\begin{figure}[ht]
\centering
    \begin{subfigure}[b]{0.3\textwidth}
    \centering
    \includegraphics[width=\textwidth]{figures/lct_celeba.pdf}
    \caption{CelebA-HQ}
    \label{fig:qualitative_celeba}
    \end{subfigure}
    \hfill
    \begin{subfigure}[b]{0.3\textwidth}
    \centering
    \includegraphics[width=\textwidth]{figures/lct_church.pdf}
    \caption{LSUN Church}
    \label{fig:qualitative_lsun_church}
    \end{subfigure}
    \hfill
    \begin{subfigure}[b]{0.3\textwidth}
    \centering
    \includegraphics[width=\textwidth]{figures/ffhq.pdf}
    \caption{FFHQ}
    \label{fig:qualitative_ffhq}
    \end{subfigure}
    \caption{Our qualitative results using 1-NFE at resolution $256 \times 256$}
    \label{fig:qualitative_1nfe}
\end{figure}

\begin{figure}[ht]
\centering
    \begin{subfigure}[b]{0.3\textwidth}
    \centering
    \includegraphics[width=\textwidth]{figures/lct_celeba_baseline.pdf}
    \caption{CelebA-HQ}
    \label{fig:qualitative_ict_celeba}
    \end{subfigure}
    \hfill
    \begin{subfigure}[b]{0.3\textwidth}
    \centering
    \includegraphics[width=\textwidth]{figures/lsun_church.pdf}
    \caption{LSUN Church}
    \label{fig:qualitative_ict_lsun_church}
    \end{subfigure}
    \hfill
    \begin{subfigure}[b]{0.3\textwidth}
    \centering
    \includegraphics[width=\textwidth]{figures/lct_ffhq_baseline.pdf}
    \caption{FFHQ}
    \label{fig:qualitative_ict_ffhq}
    \end{subfigure}
    \caption{iLCT qualitative results using 1-NFE at resolution  $256 \times 256$}
    \label{fig:qualitative_ict}
\end{figure}


\subsection{Ablation of proposed framework} \label{exp:ablation}

We ablate our proposed techniques on the CelebA-HQ $256\times256$ dataset, with all FID and Recall metrics measured using 1-NFE sampling. All models are trained for 1,400 epochs with the same hyperparameters. As shown in \cref{tab:strategy}, replacing Pseudo-Huber losses with Cauchy losses makes our model's training less sensitive to impulsive outliers, resulting in a significant FID reduction from $37.15$ to $13.02$. This demonstrates the effectiveness of Cauchy losses in handling extremely high-value outliers, as discussed in \cref{sec:cauchy}. Additionally, applying diffusion loss at small timesteps further reduces FID by approximately 4 points to $9.11$, as this loss term stabilizes the training process at small timesteps, as described in \cref{sec:diff_loss}. Introducing OT coupling during minibatch training reduces training variance, improving the FID to $8.89$. Notably, by replacing the fixed scaling term $c=c_0$, \citep{song2023improved} with an adaptive scaling schedule, our model achieves an additional FID reduction of more than 1 point, reaching $7.76$, highlighting the importance of the scaling term $c$ in robustness control. Finally, we propose using NsLN, which removes the scaling term from LayerNorm to handle outliers more effectively. NsLN captures feature statistics while mitigating the negative impact of outliers, resulting in our best FID of $7.27$.

\minisection{Robustness Loss} \label{exp:ablation:robust_loss}
To analyze the impact of different robust loss functions, we conduct an ablation study using our best settings but replace the Cauchy loss with alternatives such as L2, E-LatentLPIPS \cite{kang2024diffusion2gan}, the Huber and the Geman-McClure loss. The results, shown in \cref{tab:ablate_robust}, indicate that both Huber and Geman-McClure underperform compared to the Cauchy loss when applied in the latent space. This is because the Huber loss remains too sensitive to extremely impulsive outliers, while the Geman-McClure loss tends to ignore such outliers entirely, leading to a loss of important information. This behavior is also discussed in \cref{sec:cauchy}.

% \vspace{-2mm}
\begin{table}[h!]
    \centering
    \begin{tabular}{cc}
        \begin{minipage}[c]{0.40\textwidth}
            \centering
            
            \begin{subtable}[t]{\textwidth}
                \centering
                \begin{tabular}{lcc}
                    \toprule
                    Framework                      & FID $\downarrow$   & Recall $\uparrow$   \\
                    \midrule
                    iLCT                           & 37.15              & 0.12                \\
                    \midrule
                    Cauchy                         & 13.02              & 0.36                \\
                    + Diff                         & 9.11               & 0.41                \\
                    + OT                           & 8.89               & 0.42                \\
                    + Scaled $c$                   & 7.76               & 0.47                \\
                    + NsLN       & \textbf{7.27}               &\textbf{0.50}                \\
                    % \rowcolor{pink!60}+ NsLN       & 7.27               & 0.50                \\
                    \bottomrule
                \end{tabular}
                \caption{Components of proposed framework}
                \label{tab:strategy}
            \end{subtable}
            \hfill
            % \vspace{2mm}
            \begin{subtable}[t]{\textwidth}
                \centering
                \begin{tabular}{lcc}
                    \toprule
                    $r$           & FID $\downarrow$   & Recall $\uparrow$   \\
                    \midrule
                    1.0                    & 7.47               & 0.49                \\
                    0.6                      & 7.33               & 0.49                \\
                    % \rowcolor{pink!60}0.25   & 7.27               & 0.50                \\
                    0.25   & \textbf{7.27}               & \textbf{0.50}                \\
                    \bottomrule
                \end{tabular}
                \caption{Threshold using Diffusion loss}
                \label{tab:diff_loss}
            \end{subtable}
            
        \end{minipage}
        \hfill
        \begin{minipage}[c]{0.40\textwidth}
            \centering
            \begin{subtable}[t]{\textwidth}
                \centering
                \begin{tabular}{lcc}
                    \toprule
                    Loss                        & FID $\downarrow$   & Recall $\uparrow$   \\
                    \midrule
                     L2                          & 50.40              & 0.04                \\
                    E-LatentLPIPS               & 11.49              & 0.47                \\
                    \midrule
                    Huber                       & 9.97               & 0.44                \\
                    Geman McClure               & 11.28              & 0.44                \\
                    % \rowcolor{pink!60} Cauchy   & 7.27               & 0.50                \\
                    Cauchy   & \textbf{7.27}               & \textbf{0.50}                \\
                    \bottomrule
                \end{tabular}
                \caption{Robust losses.}
                \label{tab:ablate_robust}
            \end{subtable}
            \hfill
            % \vspace{2mm}
            \begin{subtable}[t]{\textwidth}
                \centering
                \begin{tabular}{lcc}
                    \toprule
                    Norm layer                             & FID $\downarrow$   & Recall $\uparrow$   \\
                    \midrule
                    $\text{GN}$                           & 7.76               & 0.47                \\
                    \midrule
                    % \midrule
                    IN                                      & 8.47               & 0.43                \\
                    % $\text{IN}^\dagger$                     & 8.03               & 0.46                \\
                    % \midrule
                    LN                                      & 9.05               & 0.46                \\
                    % $\text{LN}^\dagger$                     & 7.92               & 0.47                \\
                    % \midrule
                    RMS                                     & 8.96               & 0.46                \\
                    % $\text{RMS}^\dagger$                    & 7.62               & 0.47                \\
                    % \midrule
                    NsLN                 &\textbf{7.27}               &\textbf{0.50}                \\
                    % \rowcolor{pink!60} NsLN                 & 7.27               & 0.50                \\
                    % \rowcolor{white}$\text{NsLN}^\dagger$   & 7.64               & 0.47                \\
                    \bottomrule
                \end{tabular}
                \caption{Norm Layer}
                \label{tab:norm_layer}
            \end{subtable}
        \end{minipage}
    \end{tabular}
    \caption{Ablation Studies on CelebA-HQ $256\times256$ dataset at epoch 1400}
    \label{tab:ablation}
\end{table}

\minisection{Diffusion Threshold} \label{exp:ablation:diff_loss}
In this section, we explore the impact of varying the threshold for applying the diffusion loss function in combination with the consistency loss. We observe that using the diffusion loss at every timestep improves consistency training; however, it underperforms compared to applying the diffusion loss selectively at smaller timesteps such as $r=0.25$ as shown in \cref{tab:diff_loss}. This suggests that applying diffusion losses primarily at small noise levels improves performance as discussed \cref{sec:diff_loss}. At larger timesteps, the diffusion loss may conflict with the consistency loss, potentially guiding the model toward incorrect solutions, thereby reducing overall performance.



\minisection{Scaling term $c$ scheduler} \label{exp:ablation:vary_c}
In this section, we compare the performance of our adaptive scaling $c$ scheduler with the fixed scaling $c$ scheduler proposed in \citep{song2023improved}. Our model demonstrates better convergence with the proposed adaptive $c$ scheduler. The rationale behind this improvement lies in the fact that, as the discretization steps increases using the exponential curriculum, the value of the TD scales down. Despite the reduced TD value, impulsive outliers still persist. A fixed large scaling $c$ is not effective in handling these outliers. To address this, we scale $c$ down as discretization steps increases, which leads to better performance, as shown in \cref{fig:fid_vary_c}.


\minisection{Normalizing Layer} \label{exp:ablation:norm_layer}
We denote GN, IN, LN, RMS, and NsLN as GroupNorm, InstanceNorm, LayerNorm, RMSNorm, and Non-scaling LayerNorm, respectively. The baseline UNet architecture from \citep{dhariwal2021diffusion} uses GroupNorm by default. We replace the normalization layers in the baseline with each of these types and train the model on CelebA-HQ using the best settings. The results are reported in \cref{tab:norm_layer}. GN and IN only capture local statistics, making them more robust to outliers, as outliers in one region do not affect others. In contrast, LN captures statistics from all features, making it more vulnerable to outliers because an outlier affects all features through a shared scaling term. By removing the scaling term in LN, we obtain NsLN, which is both effective in capturing feature statistics and resistant to outliers. As shown in \cref{tab:norm_layer}, NsLN outperforms the second-best GN by 0.5 FID and significantly outperforms LN.
\section{Conclusion}
CT is highly sensitive to the statistical properties of the training data. In particular, when the data contains impulsive noise, such as latent data, CT becomes unstable, leading to poor performance. In this work, we propose using the Cauchy loss, which is more robust to outliers, along with several improved training strategies to enhance model performance. As a result, we can generate high-fidelity images from latent CT, effectively bridging the gap between latent diffusion models and consistency models. Future work could explore further improvements to the architecture, specifically by investigating normalization methods that reduce the impact of outliers. For example, removing the scaling term from group normalization or instance normalization may help mitigate outlier effects. Another promising future direction is the integration of this technique with Consistency Trajectory Models (CTM) \cite{kim2023consistency}, as CTM has demonstrated improved performance compared to traditional Consistency Models (CM) \cite{song2023consistency}.
\section*{Acknowledgements}
Research funded by research grants to Prof. Dimitris Metaxas from NSF: 2310966, 2235405, 2212301, 2003874, 1951890, AFOSR 23RT0630, and NIH 2R01HL127661.


\bibliography{iclr2025_conference}
\bibliographystyle{iclr2025_conference}

\newpage
\appendix
\section{Appendix}
We provide additional uncurated samples of our models for three datasets: CelebaA-HQ (\ref{fig:appendix:celeba_onestep}, \ref{fig:appendix:celeba_twostep}), LSUN Church (\ref{fig:appendix:lsun_onestep}, \ref{fig:appendix:lsun_twostep}), and FFHQ (\ref{fig:appendix:ffhq_onestep}, \ref{fig:appendix:ffhq_twostep}). We also provide additional uncurated samples of our models on CelebaA-HQ trained with L2 loss (\ref{fig:appendix:celeba_onestep_ilct_l2}) and E-LatentLPIPS loss (\ref{fig:appendix:celeba_onestep_ilct_elatentlpips}).

\begin{figure}[h]
\centering
    \includegraphics[width=0.6\textwidth]{figures/lct_celeba_more.pdf}
    \caption{One-step samples on CelebA-HQ $256 \times 256$}
    \label{fig:appendix:celeba_onestep}
\end{figure}

\begin{figure}[h]
\centering
    \includegraphics[width=0.6\textwidth]{figures/lct_celeba_more_2step.pdf}
    \caption{Two-step samples on CelebA-HQ $256 \times 256$}
    \label{fig:appendix:celeba_twostep}
\end{figure}

\begin{figure}[h]
\centering
    \includegraphics[width=0.6\textwidth]{figures/lct_lsun_more.pdf}
    \caption{One-step samples on LSUN Church $256 \times 256$}
    \label{fig:appendix:lsun_onestep}
\end{figure}

\begin{figure}[h]
\centering
    \includegraphics[width=0.6\textwidth]{figures/lct_lsun_more_2step.pdf}
    \caption{Two-step samples on LSUN Church $256 \times 256$}
    \label{fig:appendix:lsun_twostep}
\end{figure}

\begin{figure}[h]
\centering
    \includegraphics[width=0.6\textwidth]{figures/lct_ffhq_more.pdf}
    \caption{One-step samples on FFHQ $256 \times 256$}
    \label{fig:appendix:ffhq_onestep}
\end{figure}

\begin{figure}[h]
\centering
    \includegraphics[width=0.6\textwidth]{figures/lct_ffhq_more_2step.pdf}
    \caption{Two-step samples on FFHQ $256 \times 256$}
    \label{fig:appendix:ffhq_twostep}
\end{figure}


\begin{figure}[h]
\centering
    \includegraphics[width=0.6\textwidth]{figures/ilct_l2_celeba.pdf}
    \caption{One-step samples on CelebA-HQ $256 \times 256$ (L2 loss)}
    \label{fig:appendix:celeba_onestep_ilct_l2}
\end{figure}

\begin{figure}[h]
\centering
    \includegraphics[width=0.6\textwidth]{figures/ilct_latentlpips_celeba.pdf}
    \caption{One-step samples on CelebA-HQ $256 \times 256$ (E-LatentLPIPS loss)}
    \label{fig:appendix:celeba_onestep_ilct_elatentlpips}
\end{figure}

\end{document}

\bibliography{iclr2025_conference}
\bibliographystyle{iclr2025_conference}
\newpage
\appendix
%%%%%%%%%%%%%%%%%%%%%%%%%%%%%%%%%%%%%%%%%%%%%%%%%%%%%%%%%%%%%%%%%%%%%%%%%%%%%%%
%%%%%%%%%%%%%%%%%%%%%%%%%%%%%%%%%%%%%%%%%%%%%%%%%%%%%%%%%%%%%%%%%%%%%%%%%%%%%%%
% APPENDIX
%%%%%%%%%%%%%%%%%%%%%%%%%%%%%%%%%%%%%%%%%%%%%%%%%%%%%%%%%%%%%%%%%%%%%%%%%%%%%%%
%%%%%%%%%%%%%%%%%%%%%%%%%%%%%%%%%%%%%%%%%%%%%%%%%%%%%%%%%%%%%%%%%%%%%%%%%%%%%%%
\newpage
\appendix
\onecolumn

\section{Related Work} \label{app:related_work}
\textbf{Personalized Generation} 
Due to the considerable success of large text-to-image models \cite{ramesh2022hierarchical, ramesh2021zero, saharia2022photorealistic, rombach2022high}, the field of personalized generation has been actively developed. The challenge is to customize a text-to-image model to generate specific concepts that are specified using several input images. Many different approaches \cite{DB, TI, CD, svdiff, ortogonal, profusion, elite, r1e} have been proposed to solve this problem and can be divided into the following groups: pseudo-token optimization \cite{TI, profusion, disenbooth, r1e}, diffusion fune-tuning \cite{DB, CD, profusion}, and encoder-based \cite{elite}. The pseudo-token paradigm adjusts the text encoder to convert the concept token into the proper embedding for the diffusion model. Such embedding can be optimized directly \cite{TI, r1e} or can be generated by other neural networks \cite{disenbooth, profusion}. Such approaches usually require a small number of parameters to optimize but lose the visual features of the target concept. Diffusion fine-tuning-based methods optimize almost all \cite{DB} or parts \cite{CD} of the model to reconstruct the training images of the concept. This allows the model to learn the input concept with high accuracy, but the model due to overfitting may lose the ability to edit it when generated with different text prompts. To reduce overfitting and memory usage, lightweight parameterizations \cite{svdiff, r1e, lora} have been proposed that preserve edibility but at the cost of degrading concept fidelity. Encoder-based methods \cite{elite} allow one forward pass of an encoder that has been trained on a large dataset of many different objects to embed the input concept. This dramatically speeds up the process of learning a new concept and such a model is highly editable, but the quality of recovering concept details may be low. Generally, the main problem with existing personalized generation approaches is that they struggle to simultaneously recover a concept with high quality and generate it in a variety of scenes.

\textbf{Sampling strategies}
Much research has been devoted to sampling techniques for text-to-image diffusion models, focusing not only on personalized generation but also on image editing. In this paper, we address a more specific question: how can the two trajectories -- superclass and concept -- be optimally combined to achieve both high concept fidelity and high editability? The ProFusion paper \cite{profusion} considered one way of combining these trajectories (Mixed sampling), which we analyze in detail in our paper (see Section \ref{sec:mixed_sampling}) and show its properties and problems. In ProFusion, authors additionally proposed a more complex sampling procedure, which we observed to be redundant compared to Mixed sampling, as can be seen in our experiments (see Section \ref{sec:experiments}). In Photoswap \cite{photoswap}, authors consider another way of combining trajectories by superclass and concept, which turns out to be almost identical to the Switching sampling strategy that we discuss in detail in Section \ref{sec:switching_sampling}. We show why this strategy fails to achieve simultaneous improvements in concept reconstruction and editability. In the paper, we propose a more efficient way of combining these two trajectories that achieves an optimal balance between the two key features of personalized generation: concept reconstruction and editability.

\section{Training details} \label{sec:training-details}
The Stable Diffusion-2-base model is used for all experiments. For the Dreambooth, Custom Diffusion, and Textual Inversion methods, we used the implementation from \url{https://github.com/huggingface/diffusers}.

\textbf{SVDiff} We implement the method based on \url{https://github.com/mkshing/svdiff-pytorch}. The parameterization is applied to all Text Encoder and U-Net layers. The models for all concepts were trained for $1600$ using Adam optimizer with $\text{batch size} = 1$, $\text{learning rate} = 0.001$, $\text{learning rate 1d} = 0.000001$, $\text{betas} = (0.9, 0.999)$, $\text{epsilon} = 1e\!-\!8$, and $\text{weight decay} = 0.01$. 

\textbf{Dreambooth} All query, key, and value layers in Text Encoder and U-Net were trained during fine-tuning. The models for all concepts were trained for $400$ steps using Adam optimizer with $\text{batch size} = 1$, $\text{learning rate} = 2e\!-\!5$, $\text{betas} = (0.9, 0.999)$, $\text{epsilon} = 1e\!-\!8$, and $\text{weight decay} = 0.01$. 

\textbf{Custom Diffusion} The models for all concepts were trained for $1600$ steps using Adam optimizer with $\text{batch size} = 1$, $\text{learning rate} = 0.00001$, $\text{betas} = (0.9, 0.999)$, $\text{epsilon} = 1e\!-\!8$, and $\text{weight decay} = 0.01$. 

\textbf{Textual Inversion} The models for all concepts were trained for $10000$ steps using Adam optimizer with $\text{batch size} = 1$, $\text{learning rate} = 0.005$, $\text{betas} = (0.9, 0.999)$, $\text{epsilon} = 1e\!-\!8$, and $\text{weight decay} = 0.01$. 

\textbf{ELITE} We used the pre-trained model from the official repo \url{https://github.com/csyxwei/ELITE} with $\lambda=0.6$ and inference hyperparams from the original paper.

\clearpage
\section{Superclass and concept trajectory choice}\label{app:hyper_theta}

 \begin{wrapfigure}{r}{0.45\textwidth}
    % \centering
    \includegraphics[trim={3cm 10cm 3cm 10cm},clip,width=\linewidth]{imgs/mixed_noft_nosup.pdf}
    \caption{The Pareto frontiers for original Mixed sampling and Mixed sampling in the Superclass, NoFT, and Empty Prompt setups. Mixed NoFT and Mixed Empty Prompt configurations overlap with the Pareto frontier of the original mixed sampling, but primarily in regions associated with low image similarity, which compromises concept fidelity.} \label{fig:mixed_noft_ep}
    \vspace{-0.14in}
\end{wrapfigure}

There are multiple ways to define sampling with maximized textual alignment to the prompt. However, the arbitrary choice can harm the alignment between Base sampling~\ref{eq:concept_sampling} and the selected trajectory. We use the Sampling with superclass (\ref{eq:superclass_sampling}) as it's the default choice in the literature and guarantees the maximized alignment between noise predictions $\tilde{\varepsilon}_{\theta}(p^C)$ and $\tilde{\varepsilon}_{\theta}(p^S)$. 

The several natural ways to adjust Sampling with superclass can be presented by varying $\theta$ and $p^{S}$ in (\ref{eq:superclass_sampling}). We explore two additional options with decreased alignment with (\ref{eq:concept_sampling}): (1) NoFT -- weights of base model $\theta^{\text{orig}}$ instead fine-tuned weights, (2) Empty Prompt -- prompt without any reference to a concept, even to its superclass category, i.e. $p^{\hat{S}} = \textit{"with a city in the background"}$ instead of $p^{S} = \textit{"a backpack with a city in the background"}$.

To validate the robustness of our framework for sampling method selection, we employ the original experimental protocol, supplementing the results shown in Figures~\ref{fig:examples} and~\ref{fig:profusion-photoswap}. Our analysis of Figures~\ref{fig:multi-stage_noft_ep} and~\ref{fig:masked_noft_ep} reveals that trajectories generated under the NoFT and Empty Prompt configurations (second and third columns, respectively) maintain identical method ordering to those produced by Superclass sampling ((\ref{eq:superclass_sampling}), first column).

Notably, Figure~\ref{fig:mixed_noft_ep} shows that Empty Prompt configuration demonstrates weaker alignment with Base sampling compared to NoFT, particularly at higher values of the superclass guidance scale $\omega_{s}$. This divergence manifests as reduced concept fidelity for Empty Prompt under large $\omega_{s}$. These findings highlight a practical adjustment: prioritizing smaller $\omega_{s}$ values in Empty Prompt setup preserves concept fidelity without altering the framework’s core selection logic. 

A key limitation of increased misalignment is the gradual erosion of superclass category information from generated images, which can lead to semantically inconsistent outputs. For instance, Figure~\ref{fig:examples_noft_ep} illustrates how the Mixed Empty Prompt setup, despite the strong animal prior in Base sampling, can produce human-like features in an image of a cat described as \textit{"in a chef outfit"}. This suggests that when superclass information is weakened, the model may introduce unexpected visual artifacts, impacting the fidelity of the intended concept.

Concept sampling (\ref{eq:concept_sampling}) can also be adjusted to better capture a concept’s visual characteristics, further decoupling fidelity from editability. For example, this can be achieved by (1) using the weights of a highly overfitted model (e.g., DreamBooth) or (2) selecting a prompt that omits contextual details, such as $p^{\hat{C}} = \textit{"a photo of V*"}$ instead of $p^{C} = \textit{"a V* with a city in the background"}$. Combining superclass sampling under NoFT or Empty Prompt with Base sampling configured via (1) or (2) could enhance both image and text similarity. We leave this direction for future work.

\begin{figure}[b]
    \centering
    \includegraphics[trim={3cm 10cm 3cm 10cm},clip,width=0.32\linewidth]{imgs/multi-stage_original.pdf}
    \hfill
    \includegraphics[trim={3cm 10cm 3cm 10cm},clip,width=0.32\linewidth]{imgs/multi-stage_noft.pdf}
    \hfill
    \includegraphics[trim={3cm 10cm 3cm 10cm},clip,width=0.32\linewidth]{imgs/multi-stage_nosup.pdf}
    \caption{Pareto Frontier Curves for Mixed, Switching, and Multi-Stage Sampling Methods in the Superclass, NoFT and Empty Prompt setups.
The NoFT and Empty Prompt configurations (second and third columns, respectively) preserve the same method ordering as those produced by Superclass sampling (first column).} \label{fig:multi-stage_noft_ep}
\end{figure}
\begin{figure}[t]
    \centering
    \includegraphics[trim={3cm 10cm 3cm 10cm},clip,width=0.32\linewidth]{imgs/masked_profusion.pdf}
    \hfill
    \includegraphics[trim={3cm 10cm 3cm 10cm},clip,width=0.32\linewidth]{imgs/masked_noft.pdf}
    \hfill
    \includegraphics[trim={3cm 10cm 3cm 10cm},clip,width=0.32\linewidth]{imgs/masked_nosup.pdf}
    \caption{Pareto Frontier Curves for Mixed, Switching, Masked, and ProFusion Sampling Methods in the Superclass, NoFT, and Empty Prompt setups.
The NoFT and Empty Prompt configurations (second and third columns, respectively) preserve the same method ordering as those produced by Superclass sampling (first column).} \label{fig:masked_noft_ep}
\end{figure}

\begin{figure}[b]
    \centering
    \includegraphics[width=\linewidth]{imgs/examples_noft_ep.pdf}
    \caption{Examples of the generation outputs for Mixed and ProFusion sampling methods for their optimal metrics point in the Superclass, NoFT, and Empty Prompt (EP) setups.} \label{fig:examples_noft_ep}
\end{figure}

\clearpage

\begin{figure}[ht!]
  \centering
  \includegraphics[trim={0 5cm 0 5cm},clip,width=0.95\linewidth]{imgs/us_example_new.pdf}
  \caption{An example of a task in the user study}
  \label{fig:us_ex}
  \vspace{-0.19in}
\end{figure}

\section{Data preparation}\label{app:data}
For each concept, we used inpainting augmentations to create the training dataset. We took an original image and automatically segmented it using the Segment Anything model on top of the CLIP cross-attention maps. Then we crop the concept from the original image, apply affine transformations to it, and inpaint the background. We used $10$ augmentation prompts, different from the evaluation prompts, and sampled $3$ images per prompt, resulting in a total of $30$ training images per concept. We commit to open-source the augmented datasets for each concept after publication.

\section{User Study}\label{app:us}

An example task from the user study is shown in Figure~\ref{fig:us_ex}. In total, we collected 48,864 responses from 200 unique users for 16,000 unique pairs. For each task, users were asked three questions: 1) "Which image is more consistent with the text prompt?" 2) "Which image better represents the original image?" 3) "Which image is generally better in terms of alignment with the prompt and concept identity preservation?" For each question, users selected one of three responses: "1", "2", or "Can't decide."

\section{Complex Prompts Setting}\label{app:long_prompts}

We conduct a comparison of different sampling methods using a set of complex prompts. For this analysis, we collected 10 prompts, each featuring multiple scene changes simultaneously, including stylization, background, and outfit:

\adjustbox{max width=\linewidth}{
\begin{lstlisting}
live_long = [
  "V* in a chief outfit in a nostalgic kitchen filled with vintage furniture and scattered biscuit",
  "V* sitting on a windowsill in Tokyo at dusk, illuminated by neon city lights, using neon color palette",
  "a vintage-style illustration of a V* sitting on a cobblestone street in Paris during a rainy evening, showcasing muted tones and soft grays",
  "an anime drawing of a V* dressed in a superhero cape, soaring through the skies above a bustling city during a sunset",
  "a cartoonish illustration of a V* dressed as a ballerina performing on a stage in the spotlight",
  "oil painting of a V* in Seattle during a snowy full moon night",
  "a digital painting of a V* in a wizard's robe in a magical forest at midnight, accented with purples and sparkling silver tones",
  "a drawing of a V* wearing a space helmet, floating among stars in a cosmic landscape during a starry night",
  "a V* in a detective outfit in a foggy London street during a rainy evening, using muted grays and blues",
  "a V* wearing a pirate hat exploring a sandy beach at the sunset with a boat floating in the background",
]

object_long = [
  "a digital illustration of a V* on a windowsill in Tokyo at dusk, illuminated by neon city lights, using neon color palette",
  "a sketch of a V* on a sofa in a cozy living room, rendered in warm tones",
  "a watercolor painting of a V* on a wooden table in a sunny backyard, surrounded by flowers and butterflies",
  "a V* floating in a bathtub filled with bubbles and illuminated by the warm glow of evening sunlight filtering through a nearby window",
  "a charcoal sketch of a giant V* surrounded by floating clouds during a starry night, where the moonlight creates an ethereal glow",
  "oil painting of a V* in Seattle during a snowy full moon night",
  "a drawing of a V* floating among stars in a cosmic landscape during a starry night with a spacecraft in the background",
  "a V* on a sandy beach next to the sand castle at the sunset with a floaing boat in the background",
  "an anime drawing V* on top of a white rug in the forest with a small wooden house in the background",
  "a vintage-style illustration of a V* on a cobblestone street in Paris during a rainy evening, showcasing muted tones and soft grays",
]
\end{lstlisting}
}

The results of this comparison are presented in Figures~\ref{fig:add_long},~\ref{fig:add_long_metrics}. We observe that Base sampling may struggle to preserve all the features specified by the prompts, whereas advanced sampling techniques effectively restore them. The overall arrangement of methods in the metric space closely mirrors that observed in the setting with simple prompts.

\begin{figure}[h!]
  \centering
  \includegraphics[width=\linewidth]{imgs/long_prompts_examples.pdf}
  \caption{Additional examples of the generation outputs for different sampling methods with \textbf{complex prompts}. We highlight parts of the prompt that are missing in Base sampling while appearing in other methods.}
  \label{fig:add_long}
\end{figure}

\clearpage
\section{Dreambooth results}\label{app:dreambooth}

We conduct additional analysis of different sampling methods in combination with Dreambooth. Figure~\ref{fig:add_db_metrics} shows that Mixed Sampling still overperforms Switching and Photoswap,  while Multi-stage and Masked struggle to provide an additional improvement over the simple baseline. Figure~\ref{fig:add_db} shows that all methods allow for improvement TS with a negligent decrease in IS while Mixed Sampling provides the best IS among all samplings.

\begin{figure*}[!ht]
\centering
\begin{minipage}{.477\textwidth}
  \centering
  \includegraphics[trim={3cm 10cm 3cm 10cm},clip,width=\linewidth]{imgs/long_prompts.pdf}
  \caption{CLIP metrics for different sampling methods estimated on \textbf{complex prompts}.}
  \label{fig:add_long_metrics}
\end{minipage}
\hfill
\begin{minipage}{.477\textwidth}
  \centering
  \includegraphics[trim={3cm 10cm 3cm 10cm},clip,width=\linewidth]{imgs/db_samplings.pdf}
  \caption{CLIP metrics for different sampling strategies on top of a Dreambooth fine-tuning method.}
  \label{fig:add_db_metrics}
\end{minipage}
\end{figure*} 

\begin{figure}[h!]
  \centering
  \includegraphics[trim={0 1cm 0 1cm},clip,width=\linewidth]{imgs/db_sampling_examples.pdf}
  \caption{Additional examples of the generation outputs for different sampling methods on top of a Dreambooth fine-tuning method.}
  \label{fig:add_db}
\end{figure}

\section{PixArt-alpha \& SD-XL}\label{app:add_backbones}
We conducted a series of experiments using different backbones. For SD-XL~\cite{podell2023sdxlimprovinglatentdiffusion}, we used SVDDiff as the fine-tuning method, while PixArt-alpha~\citep{chen2023pixartalphafasttrainingdiffusion} employed standard Dreambooth training. Hyperparameters for Switching, Masked, and ProFusion were selected in the same manner as in the experiments with SD2.

Figures~\ref{fig:pixart} and~\ref{fig:sdxl} demonstrate that Mixed Sampling follows a similar pattern to SD2, improving TS without a significant loss in IS. Notably, Mixed Sampling for SD-XL achieves simultaneous improvements in both IS and TS. ProFusion exhibits behavior consistent with SD2, enhancing IS more effectively than Mixed Sampling but performing worse at improving TS while also requiring twice the computational resources. 

\begin{figure}[h]
\centering
\begin{minipage}{.49\textwidth}
  \centering
  \includegraphics[trim={3cm 10cm 3cm 10cm},clip,width=\linewidth]{imgs/pixart.pdf}
  \captionof{figure}{CLIP metrics for different sampling methods estimated on PixArt model.}
  \label{fig:pixart}
\end{minipage}%
\hfill
\begin{minipage}{.49\textwidth}
  \centering
  \includegraphics[trim={3cm 10cm 3cm 10cm},clip,width=\linewidth]{imgs/sdxl.pdf}
  \captionof{figure}{CLIP metrics for different sampling methods estimated on SD-XL model.}
  \label{fig:sdxl}
\end{minipage}
\end{figure}

\clearpage
\section{Cross-Attention Masks}\label{app:cross_attn}

\begin{figure}[h!]
  \centering
  \includegraphics[trim={3cm 0cm 3cm 0cm},clip,width=\linewidth]{imgs/cross_attention_masks.pdf}
  \caption{Visualization of the cross-attention masks for Masked sampling examples. Here, $q$ defines the thresholding quantile and $t$ the denoising step.}
  \label{fig:cross_attn_add_ex}
\end{figure}

\clearpage
\section{Additional Examples}\label{app:add_example}

\begin{figure}[h!]
  \centering
  \includegraphics[trim={0 2cm 0 2cm},clip,width=\linewidth]{imgs/additional_examples.pdf}
  \caption{Additional examples of the generation outputs for different sampling methods.}
  \label{fig:add_ex}
\end{figure}

\begin{figure}[h!]
  \centering
  \includegraphics[trim={0 2cm 0 2cm},clip,width=\linewidth]{imgs/additional_examples_all.pdf}
  \caption{Additional examples of the generation outputs for Mixed and ProFusion sampling methods in comparison to the main personalized generation baselines.}
  \label{fig:add_ex_all}
\end{figure}

\clearpage
\section{DINO Image Similarity}\label{app:add_dino}

We compare CLIP-IS (left column) and DINO-IS~\citep{oquab2024dinov2learningrobustvisual} (right column) in Figures~\ref{fig:profusion_photoswap_dino},~\ref{fig:all_methods_dino}. We observe that despite the choice of metric, different sampling techniques and finetuning strategies have the same arrangement. The most noticeable difference is that SVDDiff superiority over ELITE and TI is more pronounced. That strengthens our motivation to select SVDDiff as the main backbone.

\begin{figure}[h]
\centering
\begin{minipage}{.49\textwidth}
  \centering
  \includegraphics[trim={3cm 10cm 3cm 10cm},clip,width=\linewidth]{imgs/profusion_photoswap.pdf}
\end{minipage}%
\hfill
\begin{minipage}{.49\textwidth}
  \centering
  \includegraphics[trim={3cm 10cm 3cm 10cm},clip,width=\linewidth]{imgs/profusion_photoswap_dino.pdf}
\end{minipage}
\caption{Pareto frontiers curves for Photoswap~\citep{photoswap} and ProFusion~\citep{profusion}.}\label{fig:profusion_photoswap_dino}
\end{figure}

\begin{figure}[h]
\centering
\begin{minipage}{.49\textwidth}
  \centering
  \includegraphics[trim={3cm 10cm 3cm 10cm},clip,width=\linewidth]{imgs/all_methods.pdf}
\end{minipage}%
\hfill
\begin{minipage}{.49\textwidth}
  \centering
  \includegraphics[trim={3cm 10cm 3cm 10cm},clip,width=\linewidth]{imgs/all_methods_dino.pdf}
\end{minipage}
\caption{The overall results of different sampling methods against main personalized generation baselines.}\label{fig:all_methods_dino}
\end{figure}

% You may include other additional sections here.

\end{document}
