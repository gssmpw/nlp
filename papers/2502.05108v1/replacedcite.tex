\section{Background and Related work}
\subsubsection{EI and its Relevance for SE Education and Practice} EI encompasses self-awareness of one’s own goals, intentions, and behaviors, as well as the understanding of others' emotions and actions ____. Mayer and Salovey ____ posited that EI comprises four abilities: (1) the ability to perceive emotions in oneself and others, as well as in objects, art, and stories (perception of emotion), (2) the ability to generate emotions in order to use them in other mental processes (emotional facilitation of thought), (3) the ability to understand and reason about emotional information and how emotions combine and progress through relationship transitions (understanding emotions), and (4) the ability to be open to emotions and to moderate them in oneself and others (managing emotions). %

EI has gained increasing relevance in SE, where its impact on developers' productivity, emotional triggers, and well-being is a growing area of interest ____. Two decades ago, Hidalgo et al. ____ made early efforts to integrate EI into SE education by embedding EI skills into a course structure, though their work primarily focused on course design rather than student perspectives. More recently, Arroyo-Herrera et al. ____ launched the "Experiencia 360º" project, employing action research and gaming techniques to strengthen socio-emotional skills among Computing Engineering students.



\subsubsection{Emotional Awareness in SE} 
Kosti et al. ____ examined personality clusters among software engineers, revealing how distinct personality traits correlate with work preferences. Building on the intersection of EI and work performance, Khatun and Salleh ____ employed hierarchical multiple regression analysis to investigate the moderating role of EI in the relationship between work ethics and performance among software engineers. Wrobel ____ surveyed 49 software developers, revealing how emotions affect performance and introducing ``emotional risk" to productivity.


More recently, Novielli and Serebrenik ____ assessed emotions in software ecosystems, revealing how developer emotions can indicate ecosystem health and development issues. This review also summarized tools and datasets for emotion analysis. In another study, Serebrenik ____ shifted the focus to the emotional labor experienced by software developers. Despite stereotypes of developers as isolated workers, the study argued that the growing social demands in software development make emotional labor increasingly relevant. 

The role of emotional awareness in software development and productivity was further explored by Fountaine and Sharif____, who investigated how emotional clarity affects task progress. Their study proposed that enhancing emotional awareness in developer environments could improve productivity. In addition, Girardi et al. ____ conducted a study with 23 students to identify emotional triggers and coping strategies during programming tasks. Their research highlighted the effectiveness of wristbands measuring electrodermal activity and heart metrics in recognizing emotions, contributing to the development of emotion recognition classifiers. In a follow-up study, Girardi et al. ____ examined the impact of emotions on perceived productivity among 21 professionals from five companies. Replicating Graziotin et al.____, they confirmed a positive correlation between emotional valence and productivity, especially in the afternoon, highlighting the importance of individualized emotion detection models.



Graziotin et al. ____ explored the impact of happiness and unhappiness on software developers during the development process. Analyzing responses from 317 participants, they identified 42 consequences of unhappiness and 32 of happiness, influencing cognitive and behavioral states as well as external outcomes. Also related to the perception and management of one's emotions, Guenes et al. ____ examined the impostor phenomenon in software engineers. From responses of 624 participants, they found that frequent and intense impostor feelings led to lower self-perceived productivity.

\subsubsection{Emotions in Agile and Socio-Technical Settings} Luong et al. ____ investigated EI in addressing human-related challenges within Agile Managed Information Systems projects. Analyzing data from 194 agile practitioners, they found significant correlations between EI and key issues such as anxiety, motivation, trust, and communication within agile teams. This intersection of EI in agile contexts, particularly focusing on requirements change handling, was further expanded by Madampe et al. ____, who studied how software practitioners emotionally react to requirements changes throughout their lifecycle, based on a global survey of 201 practitioners. They identified common emotional responses and triggers,  highlighting the negative impact of last-minute changes near deadlines. Madampe et al. ____ developed a framework for handling emotion-oriented requirements changes, based on a survey of 241 participants, identifying emotional challenges and mitigation strategies in socio-technical environments. Their later work linked negative emotions in agile projects to reduced developer satisfaction and productivity, offering practical solutions. Additionally, they showed that EI and cognitive intelligence are key to managing requirements changes in agile settings, using a mixed-method study with 124 practitioners to suggest strategies for emotion regulation, relationship management, and maintaining productivity ____.


\subsubsection{Research Gap} 
In summary, early initiatives and recent studies in SE highlight the relevant role of EI in developers' well-being and productivity. However, most research emphasizes external observations, leaving a gap in understanding SE students' self-perceptions of EI. While professional settings have been explored, little attention has been given to how EI is developed and perceived in educational contexts. Specifically, research on how SE students evaluate their own EI through Mayer and Salovey's framework ____ remains limited. This study aims to address these gaps by investigating SE students' self-perceptions of EI within a project-based learning context.