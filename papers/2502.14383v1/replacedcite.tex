\section{Related Work}
\subsection{Temporal Misinformation Detection}

Posts on social media often come with timestamps. Therefore, many studies use time information to assist in detecting false information. ____ propose a temporal Bi-GCN model to learn representations for rumor propagation by encoding the temporal information for graph structures. ____ design multimodal graph neural networks that can form a temporal news cluster and learn temporal features for fake news detection. ____ suggest a block-based representation and fusion mechanism that can utilize the information from both spatial and temporal perspectives. ____ adopt an attention-based method to combine linguistic and user features alongside temporal features. ____ propose the time-aware crowd signals method to explore the temporal nature of news propagation, utilizing users’ reputations obtained from their public behavior when spreading news in the past.

% \subsection{Multi-task misinformation detection}

% Multi-task learning is a multi-objective optimization method that can effectively improve the generalization ability of models ____. ____ design various auxiliary tasks (e.g. sentiment, novelty, domain) to improve the main misinformation detection task based on LSTM structure. 
% ____ fed features extracted by BERT into multiple linear classifiers to incorporate auxiliary sentiment and stance tasks. ____ introduces a multitask framework for video-based multimodal fake news detection, in which emotion recognition is the auxiliary task. It can be seen that sentiments and emotions play important roles in multi-task fake news detection.

\subsection{Emotion-based Misinformation Detection}

Emotion and sentiment are important features in detecting misinformation ____. 
% {\bf I ADDED THIS SENTENCE TO FIX THE EMOTION-SENTIMENT PROBLEM...PLS CONFIRM} In our work we recognize that positive sentiment correlates with positive emotion and vice versa and use emotion as the principal labeling. 
____ introduce multitask frameworks for incorporating sentiment/emotion as auxiliary tasks to enhance the main misinformation detection task. ____ propose the FakeFlow for fake news detection that can capture the flow of affective information by combining topic and sentiments. ____ adapt LLMs for misinformation detection by levering affective information based on fine-tuning and RAG techniques respectively. However, these models are only applicable to fake source text and do not consider social content. ____ develop an attention-based mechanism by combining CNN and Bi-GRU for enriched extraction of dual emotion features. ____ analyze the relationship between dual emotions and propose dual emotion features, which can be plugged into existing fake news detectors. ____ further fuses the author's visual emotion for multi-modal rumor detection. These works indicate the importance of dual emotion features. 

\subsection{Comment-based Rumor Detection}

Social media posts often come with rich comments. Combining information from the source and tweets can help improve rumor detection models. Some studies focus on the propagation pattern in rumors. ____ apply two recursive neural models based on tree-structured neural networks to learn the propagation layout of tweets. ____ adopt a bi-directional graph model to learn the patterns of rumor propagation. ____ propose a post-level attention model, which incorporates tree structure information into the transformer network.  ____ design a Crowd Intelligence and ChatGPT-Assisted Network, which combines crowd intelligence-based feature learning, ChatGPT-enhanced knowledge mining, and an entity-aware heterogeneous graph. Also, there are many studies using the features of comments. Like ____ apply the emotion and sentiment features and ____ leverage stance features for misinformation detection.

% develop a Claim-guided Hierarchical Graph Attention Network that models conversation threads as an interaction graph to enhance rumor classification.

% LLMs perform well in tasks involving misinformation due to their outstanding capabilities ____. In addition to the studies in the above sections, ____ design an adaptive rationale guidance network that leverages insights on news analysis from LLM to guide small language model in fake news detection. ____ propose a framework that utilizes prompting strategy to enable an LLM to become Generator and Detector and for fake news generation and detection. ____ enhance fact-checking performance of LLMs by leveraging search engines to retrieve external knowledge.  ____ design a multimodal fake news detection framework that leverages forgery-specific knowledge to augment one multi-modal LLM, enabling them to reason about manipulations effectively. ____ design prompts to teach LLMs to reason over clues in news and comments and divide the propagation information into a chain of propagation for rumor detection. 

\subsection{Prompt-tuning and Prefix-tuning}

Prompt-tuning ____ and prefix-tuning ____ typically involve fine-tuning a small number of parameters to adapt to specific tasks, rather than modifying the entire LLM. ____ develop a soft prompt-based learning architecture based on a clinical LLM for clinical concept and relation extraction. ____ provide a prompt-able architecture for medical image segmentation, which is frozen post pre-training but remains flexible with class-specific learnable prompt tokens. ____ propose a super-resolution training method based on prefix and prompt fine-tuning, in which only a few prefix and prompt parameters are added to the self-attention module. ____ design an adaptive prefix-tuning technique involving the training of prefix parameters on adapting tasks, followed by fine-tuning on downstream tasks. ____ propose focused prefix tuning to enable the control to focus on the desired attribute for controllable text generation. ____ propose a zero-shot framework for rumor detection based on prompt learning. They first represent Rumors as different propagation threads, then a hierarchical prompt encoding mechanism is designed to learn context representations.