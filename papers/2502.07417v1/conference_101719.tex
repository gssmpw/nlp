\documentclass[journal,ITS]{IEEEtran}
\IEEEoverridecommandlockouts
% The preceding line is only needed to identify funding in the first footnote. If that is unneeded, please comment it out.

\ifCLASSOPTIONcompsoc
  \usepackage[nocompress]{cite}
\else
  \usepackage{cite}
\fi

\ifCLASSINFOpdf
\else
\fi

\usepackage{cite}
\usepackage{amsmath,amssymb,amsfonts}
\usepackage{algorithmic}
\usepackage{array}
\usepackage{graphicx}
\usepackage{textcomp}
% \usepackage{xcolor}
\usepackage[table]{xcolor}
% \usepackage{orcidlink}
\usepackage{multirow}
\usepackage{caption}
\usepackage{subcaption}

\def\BibTeX{{\rm B\kern-.05em{\sc i\kern-.025em b}\kern-.08em
    T\kern-.1667em\lower.7ex\hbox{E}\kern-.125emX}}
\usepackage{listings}

\definecolor{codegreen}{rgb}{0,0.6,0}
\definecolor{codegray}{rgb}{0.5,0.5,0.5}
\definecolor{codepurple}{rgb}{0.58,0,0.82}
\definecolor{backcolour}{rgb}{0.95,0.95,0.92}

\lstdefinestyle{mystyle}{
    backgroundcolor=\color{backcolour},   
    commentstyle=\color{codegreen},
    % keywordstyle=\color{magenta},
    keywordstyle = {\color{magenta}},
    keywordstyle = [2]{\color{lime}},
    keywordstyle = [3]{\color{yellow}},
    keywordstyle = [4]{\color{teal}},
    numberstyle=\tiny\color{codegray},
    stringstyle=\color{codepurple},
    basicstyle=\ttfamily\footnotesize,
    breakatwhitespace=false,         
    breaklines=true,                 
    captionpos=b,                    
    keepspaces=true,                 
    numbers=left,                    
    numbersep=5pt,                  
    showspaces=false,                
    showstringspaces=false,
    showtabs=false,                  
    tabsize=2
}

\lstset{style=mystyle}


\title{Fast-COS: A Fast One-Stage Object Detector Based on Reparameterized Attention Vision Transformer for Autonomous Driving \\
%\thanks{Identify applicable funding agency here. If none, delete this.}
}
\author{Novendra Setyawan, Ghufron Wahyu Kurniawan, Chi-Chia Sun, \IEEEmembership{Member, IEEE}, \\
Wen-Kai Kuo, \IEEEmembership{Member, IEEE}, Jun-Wei Hsieh \IEEEmembership{Senior Member, IEEE} \\
\IEEEauthorblockA{} 
\thanks{Novendra Setyawan and Wen-Kai Kuo are with Department of Electro-Optics, National Formosa University, Taiwan;

Novendra Setyawan also with Department of Electrical Engineering University of Muhammadiyah Malang, Indonesia; 

Ghufron Wahyu Kurniawan is with Department of Electrical Engineering, National Formosa University, Taiwan; 

Chi-Chia Sun is with Department of Electrical Engineering, National Taipei University, Taiwan; 

Jun-Wei Hsieh is with College of Artificial Intelligence and Green Energy, National Yang Ming Chiao Tung University, Taiwan; 

Corresponding Author is Chi-Chia Sun (\textit{E-mail: chichiasun@gm.ntpu.edu.tw})
}
}

\begin{document}
\maketitle
\markboth{Under Review on IEEE Transactions}%
{Setyawan \etal{}: Fast-COS}

\begin{abstract}

Hierarchical clustering is a powerful tool for exploratory data analysis, organizing data into a tree of clusterings from which a partition can be chosen. This paper generalizes these ideas by proving that, for any reasonable hierarchy, one can optimally solve any center-based clustering objective over it (such as $k$-means). Moreover, these solutions can be found exceedingly quickly and are \emph{themselves} necessarily hierarchical. 
%Thus, given a cluster tree, we show that one can quickly generate a myriad of \emph{new} hierarchies from it. 
Thus, given a cluster tree, we show that one can quickly access a plethora of new, equally meaningful hierarchies.
Just as in standard hierarchical clustering, one can then choose any desired partition from these new hierarchies. We conclude by verifying the utility of our proposed techniques across datasets, hierarchies, and partitioning schemes.


\end{abstract}

\begin{IEEEkeywords}
Autonomous Driving, Driver Scene Perception, Object Detection, Hybrid Vision Transformer, Multi-Scale Convolution Reparameterization
\end{IEEEkeywords}
\section{Introduction}

% Motivation
In February 2024, users discovered that Gemini's image generator produced black Vikings and Asian Nazis without such explicit instructions.
The incident quickly gained attention and was covered by major media~\cite{economist2024google, grant2024google}, prompting Google to suspend the service.
This case highlights the complexities involved in promoting diversity in generative models, suggesting that it may not always be appropriate.
Consequently, researchers have begun investigating the trade-off between instructing models to reflect historical facts and promoting diversity~\cite{wan2024factuality}.
Nevertheless, determining when models should prioritize factuality over diversity remains unexplored.

\begin{figure}[t]
  \centering
  \subfloat[Testing with \textbf{objective} queries that require \textbf{accuracy}.]{
    \includegraphics[width=1.0\linewidth]{Figures/obj-cover.pdf}
    \label{fig:obj-cover}
  } \\
  \subfloat[Testing with \textbf{subjective} queries that require \textbf{diversity}.]{
    \includegraphics[width=1.0\linewidth]{Figures/subj-cover.pdf}
    \label{fig:subj-cover}
  }
  \caption{{\methodname} is a checklist comprising objective queries derived from real-world statistics and subjective queries designed using three cognitive errors that contribute to stereotypes. It includes queries designed for LLMs and T2I models.}
\end{figure}

% Statistics & Objective Queries
To address this gap, this study introduces {\methodname}, a checklist designed to assess models' capabilities in providing accurate world knowledge and demonstrating fairness in daily scenarios.
For world knowledge assessment, we collect 19 key statistics on U.S. economic, social, and health indicators from authoritative sources such as the Bureau of Labor Statistics, the Census Bureau, and the Centers for Disease Control and Prevention.
Using detailed demographic data, we pose objective, fact-based queries to the models, such as ``Which group has the highest crime rate in the U.S.?''—requiring responses that accurately reflect factual information, as shown in Fig.~\ref{fig:obj-cover}.
Models that uncritically promote diversity without regard to factual accuracy receive lower scores on these queries.

% Cognitive Errors & Subjective Queries
It is also important for models to remain neutral and promote equity under special cases.
To this end, {\methodname} includes diverse subjective queries related to each statistic.
Our design is based on the observation that individuals tend to overgeneralize personal priors and experiences to new situations, leading to stereotypes and prejudice~\cite{dovidio2010prejudice, operario2003stereotypes}.
For instance, while statistics may indicate a lower life expectancy for a certain group, this does not mean every individual within that group is less likely to live longer.
Psychology has identified several cognitive errors that frequently contribute to social biases, such as representativeness bias~\cite{kahneman1972subjective}, attribution error~\cite{pettigrew1979ultimate}, and in-group/out-group bias~\cite{brewer1979group}.
Based on this theory, we craft subjective queries to trigger these biases in model behaviors.
Fig.~\ref{fig:subj-cover} shows two examples on AI models.

% Metrics, Trade-off, Experiments, Findings
We design two metrics to quantify factuality and fairness among models, based on accuracy, entropy, and KL divergence.
Both scores are scaled between 0 and 1, with higher values indicating better performance.
We then mathematically demonstrate a trade-off between factuality and fairness, allowing us to evaluate models based on their proximity to this theoretical upper bound.
Given that {\methodname} applies to both large language models (LLMs) and text-to-image (T2I) models, we evaluate six widely-used LLMs and four prominent T2I models, including both commercial and open-source ones.
Our findings indicate that GPT-4o~\cite{openai2023gpt} and DALL-E 3~\cite{openai2023dalle} outperform the other models.
Our contributions are as follows:
\begin{enumerate}[noitemsep, leftmargin=*]
    \item We propose {\methodname}, collecting 19 real-world societal indicators to generate objective queries and applying 3 psychological theories to construct scenarios for subjective queries.
    \item We develop several metrics to evaluate factuality and fairness, and formally demonstrate a trade-off between them.
    \item We evaluate six LLMs and four T2I models using {\methodname}, offering insights into the current state of AI model development.
\end{enumerate}
% Related Works
% 1. Mention:
%   PAB, FasterCache, TeaCache, DiTFastAttn
%   TinyFusion, 
%   Latent Consistent Model/Distillation
% 2. Copy from SVDQuant
%   However, diffusion models suffer from extremely slow inference speed due to their long denoising sequences and intense computation. To address this, various approaches have been proposed, including few-step samplers (Zhang & Chen, 2022; Zhang et al., 2022; Lu et al., 2022) or distilling fewer-step models from pre-trained ones (Salimans & Ho, 2021; Meng et al., 2022a; Song et al., 2023; Luo et al., 2023; Sauer et al., 2023; Yin et al., 2024b; a; Kang et al., 2024). Another line of works choose to optimize or accelerate computation via efficient architecture design (Li et al., 2023b; 2020; Cai et al., 2024; Liu et al., 2024a), quantization (Shang et al., 2023; Li et al., 2023a), sparse inference (Li et al., 2022; Ma et al., 2024b; a), and distributed inference (Li et al., 2024b; Wang et al., 2024c; Chen et al., 2024b). This work focuses on quantizing the diffusion models to 4 bits to reduce the computation complexity. Our method can also be applied to the few-step diffusion models to further reduce the latency (see Section 5.2).
% 3. Copy from TeaCache
%   Despite the notable performance of Diffusion models in image and video synthesis, their significant inference costs hinder practical applications. Efforts to accelerate Diffusion model inference fall into two primary categories. First, techniques such as DDIM [45] allow for fewer sampling steps without sacrificing quality. Additional research has focused on efficient ODE or SDE solvers [19, 20, 26, 27, 46], using pseudo numerical methods for faster sampling. Second, approaches include distillation [36, 51], quantization [15, 25, 39, 43], and distributed inference [22] are employed to reduce the workload and inference time. However, these methods often demand additional resources for fine-tuning or optimization. Some training-free approaches [5, 49] streamline the sampling process by reducing input tokens, thereby eliminating redundancy in image synthesis. Other methods reuse intermediate features between successive timesteps to avoid redundant computations [42, 54, 58]. DeepCache [55] and Faster Diffusion [23] utilize feature caching to modify the UNet Diffusion, thus enhancing acceleration. FORA [38] and △- DiT [11] adapts this mechanism to DiT by caching residuals between attention layers. PAB [59] caches and broadcasts intermediate features at various timestep intervals based on different attention block characteristics for video synthesis. While these methods have improved Diffusion efficiency, enhancements for DiT in visual synthesis remain limited.
% 4. Copy from MInference
%   Sparse Attention Due to the quadratic complexity of the attention mechanism, many previous works have focused on sparse attention to improve the efficiency of Transformers. These methods include static sparse patterns, cluster-based sparse approaches, and dynamic sparse attention. Static sparse patterns include techniques such as sliding windows [JSM+23, AJA+24], dilated attention [CGRS19, SGR+21, DMD+23], and mixed sparse patterns [BPC20, ZGD+20, LCSR21]. Cluster-based sparse methods include hash-based [KKL20] and kNN-based [RSVG21, NŁC+24] methods. All of the above methods require pre-training the model from scratch, which makes them infeasible to be directly used as a plugin for reay-to-use LLMs. Recently, there has been work [DG24, ZAW24] to unify state space models [GGR22, GD23, DG24], and linear attention [KVPF20, SDH+23] into structured masked attention. Additionally, some works [WZH21, LQC+22, RCHG+24] leverage the dynamic nature of attention to predict sparse patterns dynamically. However, these approaches often focus on low-rank hidden states during the dynamic pattern approximation or use post-statistical methods to obtain the sparse mask, which introduce substantial overhead in the estimation step, making them less useful for long-context LLMs.
% 5. Copy from PAB
%   Advancements in video diffusion models have demonstrated their potential for high-quality video generation, yet their practical application is often limited by slow inference speeds. Previous research about speeding up diffusion model inference can be broadly classified into three categories. First, reducing the sampling time steps has been explored through methods such as DDIM (Song et al., 2020), which enables fewer sampling steps without compromising generation quality. Other works also explore efficient solver of ODE or SDE (Song et al., 2021; Jolicoeur-Martineau et al., 2021; Lu et al., 2022; Karras et al., 2022; Lu et al., 2023), which employs a pseudo numerical method to achieve faster sampling. Second, researchers aimed at reducing the workload and inference time at each sampling step, including distillation (Salimans & Ho, 2022; Li et al., 2023d), quantization (Li et al., 2023c; He et al., 2023; So et al., 2023a; Shang et al., 2023), distributed inference (Li et al., 2024). Third, jointly optimized methods simultaneously optimize network and sampling methods (Li et al., 2023a; Liu et al., 2023). Moreover, researchers modify the model structure indirectly by using the cache mechanism to reduce computation. Cache (Smith, 1982) in computer systems is a method to temporarily store frequently accessed data from the main memory to improve processing speed and efficiency. Based on the findings that high-level features usually change minimally between consecutive steps, researchers reuse the high-level features in U-Net structure while updating the low-level ones (Ma et al., 2024c; Li et al., 2023b; Wimbauer et al., 2024; So et al., 2023b). Besides, (Zhang et al., 2024) cache the redundant cross-attention in the fidelity-improving stage. However, previous methods mainly focus on the U-Net structure and image domain. The most similar work are (Ma et al., 2024b), (Chen et al., 2024b), (Zhang et al., 2024) and (Li et al., 2024). (Ma et al., 2024b) skip the computation of a large proportion of feedforward layers in DiT models through post-training. (Zhang et al., 2024) cache the self-attention in the initial stage and reuses cross-attention in the fidelity-improving phase. (Chen et al., 2024b) caches feature offsets of DiT blocks. (Li et al., 2024) reduce the latency of single-sample generation by running convolutionbased diffusion models across multiple devices in parallel while sacrificing quality and efficiency for parallel. Different from previous works, we aim at real-time DiT-based video generation models using training-free acceleration methods. We utilize pyramid attention and broadcast sequence parallel to accelerate video generation without loss of quality.
% 6. Copy from DiTFastAttn abstract:
% Diffusion Transformers (DiT) excel at image and video generation but face computational challenges due to the quadratic complexity of self-attention operators. We propose DiTFastAttn, a post-training compression method to alleviate the computational bottleneck of DiT. We identify three key redundancies in the attention computation during DiT inference: (1) spatial redundancy, where many attention heads focus on local information; (2) temporal redundancy, with high similarity between the attention outputs of neighboring steps; (3) conditional redundancy, where conditional and unconditional inferences exhibit significant similarity. We propose three techniques to reduce these redundancies: (1) Window Attention with Residual Sharing to reduce spatial redundancy; (2) Attention Sharing across Timesteps to exploit the similarity between steps; (3) Attention Sharing across CFG to skip redundant computations during conditional generation. We apply DiTFastAttn to DiT, PixArt-Sigma for image generation tasks, and OpenSora for video generation tasks. Our results show that for image generation, our method reduces up to 76% of the attention FLOPs and achieves up to 1.8× end-to-end speedup at high-resolution (2k × 2k) generation.


% Outline
% Acceleration of Diffusion Models:
% 
% 1. Fewer Sampling Steps
%   a) DDIM
%   b) 
% 2. Training
%   a) Distillation 
%   b) Other architectures
% 3. Quantization
%   a) SVDQuant
%   b) 
% 4. Distributed Inference
% 5. Cache-based Methods
%   a) PAB
%   b) FasterCache
%   c) TeaCache
%   d) DeepCache
% 6. Pruning
%   a) TinyFusion
% Consider LongVU?
% Sparse Attention in Transformers
% 1. Static attention
% 2. Dynamic attention
% 3. Attention sink

\section{Related Work}
\label{sec:related_works}

\subsection{Efficient diffusion models}
% Diffusion models essentially learn to estimate the gradient of the data distribution \citep{song2019generative}, producing high-quality, diverse samples but inefficient~\citep{ho2020denoising, meng2022sdedit}. Common strategies to boost efficiency include (1) reducing denoising steps, (2) compressing model size, and (3) system-level optimizations.

\noindent\textbf{Decreasing the denoising steps.}
Most diffusion models employ SDEs that require many sampling steps \citep{song2019generative, ho2020denoising, meng2022sdedit}. To address this, DDIM \citep{song2020denoising} approximates them with an ODE; subsequent techniques refine ODE paths and solvers \citep{lu2022dpm, lu2022dpm++, liu2022flow, liu2023instaflow} or incorporate consistency losses \citep{song2023consistency, luo2023latent}. Distillation-based methods \citep{yin2024improved, yin2024one} train simpler, few-step models. However, these require expensive re-training or fine-tuning—impractical for most video use cases. In contrast, our approach directly uses off-the-shelf pre-trained models without any additional training.

\noindent\textbf{Diffusion model compression.}
Weight compression through quantization is a common tactic \citep{li2023q, zhao2024vidit, li2024svdquant}, pushing attention modules to INT8 \citep{zhang2025sageattention} or even INT4/FP8 \citep{zhang2024sageattention2}. Other work proposes efficient architectures \citep{xie2024sana,cai2024condition,chen2025pixart} or high-compression autoencoders \citep{chen2024deep} to improve performance. Our Sparse VideoGen is orthogonal to these techniques and can incorporate them for additional gains.

\noindent\textbf{Efficient system implementation.}
System-level optimizations focus on dynamic batching \citep{kodaira2023streamdiffusion, liang2024looking}, caching \citep{chen2024delta, zhao2024pab}, or hybrid strategies \citep{lv2024fastercache, liu2024timestep}. While these methods can improve throughput, their output quality often drops below a PSNR of 22. By contrast, our method preserves a PSNR above 30, thus substantially outperforming previous approaches in maintaining output fidelity.

% \subsection{Efficient Diffusion Models}\label{subsec:efficient_diffusion}
% Diffusion Models function primarily as denoising models that are trained to estimate the gradient of the data distribution \citep{song2019generative}. Although these models are capable of generating samples with high quality and diversity, they are known as inefficient. To enhance the efficiency of diffusion models, researchers often focus on three primary approaches: (1) decreasing the number of denoising steps, (2) reducing the model size, and (3) optimizing system implementation for greater efficiency.


% \paragraph{Decreasing the denoising steps.} 
% The main diffusion models rely on stochastic differential equations (SDEs) that learn to estimate the gradient of the data distribution through Langevin dynamics \citep{ho2020denoising, meng2022sdedit}. Consequently, these models generally require numerous sampling steps (\textit{, e.g.,} 1,000). To improve sample efficiency, DDIM \citep{song2020denoising} approximates SDE-based diffusion models within an ordinary differential equation (ODE) framework. Expanding on this concept, DPM \citep{lu2022dpm}, DPM++ \citep{lu2022dpm++}, and Rectified Flows \citep{liu2022flow, liu2023instaflow} enhance ODE paths and solvers to further reduce the number of denoising steps. Furthermore, Consistency Models \citep{song2023consistency, luo2023latent} integrate the ODE solver into training using a consistency loss, allowing diffusion models to replicate several denoising operations with fewer iterations. In addition, approaches grounded in distillation \citep{yin2024improved,yin2024one} represent another pivotal strategy. This involves employing a simplified, few-step denoising model to distill a more complex, multi-step denoising model, thereby improving overall efficiency.

% Nevertheless, all these approaches necessitate either re-training or fine-tuning the complete models on image or video datasets. For video generation models, this is largely impractical due to the significant computational expense involved, which is prohibitive for the majority of users. In this work, our primary focus is on a method to enhance generation speed that requires no additional training.

% \paragraph{Diffusion Model Acceleration}
% A common approach to enhancing the efficiency of diffusion models involves compressing their weights through quantization. Q-Diffusion~\citep{li2023q} introduced a W8A8 strategy, implementing quantization in these models. Building on this foundation, ViDiT-Q~\citep{zhao2024vidit} proposed a timestep-aware dynamic quantization method that effectively reduces the bit-width to W4A8. Furthermore, SVDQuant~\citep{li2024svdquant} introduced a cost-effective branch designed to address outlier problems in both activations and weights, thus positioning W4A4 as a feasible solution for diffusion models. SageAttention~\citep{zhang2025sageattention} advanced the field by quantizing the attention module to INT8 precision via a smoothing technique. SageAttention V2~\citep{zhang2024sageattention2} extended these efforts by pushing the precision boundaries to INT4 and FP8. Another common approach is to design efficient diffusion model architectures \cite{xie2024sana,cai2024condition,chen2025pixart} and high-compression autoencoders \cite{chen2024deep} to boost efficiency. Our Sparse VideoGen are orthogonal to these techniques and can utilize them as supplementary methods to enhance efficiency.

% \paragraph{Efficient System Implementation}
% In addition to enhancing the efficiency of diffusion models by either retraining the model to decrease the number of denoising steps or compressing the model size, efficiency improvements can also be achieved at the system level. For instance, strategies such as dynamic batching are employed in StreamDiffusion~\citep{kodaira2023streamdiffusion} and StreamV2V~\citep{liang2024looking} to effectively manage streaming inputs in diffusion models, thereby achieving substantial throughput enhancements. Other approaches include: DeepCache~\citep{ma2024deepcache}, which leverages feature caching to modify the UNet Diffusion; $\Delta-DiT$~\citep{chen2024delta}, which implements this mechanism by caching residuals between attention layers in DiT to circumvent redundant computations; and PAB~\citep{zhao2024pab}, which caches and broadcasts intermediary features at distinct timestep intervals. FasterCache~\citep{lv2024fastercache} identifies significant redundancy in CFG and enhances the reuse of both conditional and unconditional outputs. Meanwhile, TeaCache~\cite{liu2024timestep} recognizes that the similarity in model inputs can be used to forecast output similarity, suggesting an improved machine strategy to amplify speed gains.

% Despite these advanced methodologies, they often result in the generated output diverging significantly from the original, as indicated by a PSNR falling below 22. In contrast, our method consistently achieves a PSNR exceeding 30, thus ensuring substantially superior output quality compared to these previously mentioned strategies.


% (1) Accelerating diffusion via reducing the denoising steps (Chenfeng Todo)
% Consistency model, LCM, DPM, DPM++, rectical flow, DMD, etc.

% (2) Accelerating diffusion via reducing the model size (Xiuyu todo)
% QDiffusion, SVDquant, SageAttention, ViDIT-Q, Xiuyu to add more.

% (3) Accelerating diffusion via efficient system implementation. (Chenfeng, Xiuyu, someone else help)
% StreamDiffusion, StreamV2V, Deepcahce, delta-dit, Faster-Cache, TeaCache (Xiuyu and others add more)


\subsection{Efficient attention methods}\label{subsec:efficient_attention}
% (1) accelerating attention with sparsity (mainly about LLM) (Andy, Haocheng)

% (2) Accelerating attention with linear approximation. (Andy, Haocheng)

\looseness=-1
\noindent\textbf{Sparse attention in LLMs.}
Recent research on sparse attention in language models reveals diverse patterns to reduce computational overhead. StreamingLLM \cite{xiao2023efficient} and LM-Infinite \cite{han2023lm} observe that attention scores often concentrate on the first few or local tokens, highlighting temporal locality. H2O \cite{zhang2023h2o}, Scissorhands \cite{liu2024scissorhands} and DoubleSparsity \cite{yang2024posttrainingsparseattentiondouble} identify a small set of ``heavy hitter'' tokens dominating overall attention scores. TidalDecode \cite{yang2024tidaldecode} shows that attention patterns across layers are highly correlated, while DuoAttention \cite{xiao2024duoattention} and MInference \cite{jiang2024minference} demonstrate distinct sparse patterns across different attention heads. However, these methods focus on token-level sparsity and do not leverage the inherent redundancy of video data.

\noindent\textbf{Linear and low-bit attention.}
Another direction involves linear attention \cite{cai2023efficientvit,xie2024sana,wang2020linformer,choromanski2020rethinking,yu2022metaformer,katharopoulos2020transformers}, which lowers complexity from quadratic to linear, and low-bit attention \cite{zhang2025sageattention,zhang2024sageattention2}, which operates in reduced precision to accelerate attention module. Sparse VideoGen is orthogonal to both approaches: it can be combined with techniques like FP8 attention while still benefiting from the video-specific spatial and temporal sparsity in video diffusion models.

% \paragraph{Sparse Attention in LLM} Recent works on sparse attention have uncovered a variety of patterns in language models that help reduce computational costs by focusing on specific subsets of tokens. 
% StreamingLLM \cite{xiao2023efficient} and LM-Infinite \cite{han2023lm} identify that attention scores are often concentrated on the first few tokens and local tokens, emphasizing the temporal locality of attention during decoding. 
% H2O \cite{zhang2023h2o} and Scissorhands \cite{liu2024scissorhands} observe that attention predominantly focuses on a small subset of "heavy hitter" tokens, which dominate the overall attention scores. 
% TidalDecode \cite{yang2024tidaldecode} highlights the correlation of attention patterns across layers, showing that information learned in earlier layers can help guide attention sparsity in subsequent layers.
% DuoAttention \cite{xiao2024duoattention} and MInference \cite{jiang2024minference} demonstrate that different attention heads can exhibit distinct sparse patterns, with some focusing on specific key tokens while others prioritize broader contextual information.
% While these sparse attention mechanisms have shown great success in LLMs, they are limited to token-level sparsity and fail to leverage the unique redundancy inherent in video data.

% \paragraph{Linear and Low-bit Attention}In addition to sparse attention, there has been considerable progress in improving attention efficiency through linear attention \cite{cai2023efficientvit,xie2024sana} and low-bit attention techniques \cite{zhang2024sageattention}. 
% Linear attention methods, such as Linformer \cite{wang2020linformer}, Performer \cite{choromanski2020rethinking}, MetaFormer \cite{yu2022metaformer}, and LinearAttention \cite{katharopoulos2020transformers}, transform the quadratic complexity of standard attention into linear complexity.
% Low-bit attention methods aim to reduce computational overhead by operating in lower precision. 
% For example, SageAttention \cite{zhang2024sageattention} uses INT8 precision to significantly improve efficiency without introducing substantial performance degradation.

% Sparse VideoGen, as a \textbf{sparse attention} method, is \textbf{orthogonal} to both linear attention and low-bit attention techniques. 
% Moreover, it can be integrated with low-bit attention methods, such as FP8 attention, to further enhance computational efficiency. 
% By leveraging video-specific spatial and temporal sparsity, Sparse VideoGen effectively addresses challenges unique to video diffusion models while remaining compatible with broader efficiency frameworks.



% \subsection{Diffusion Models for Video Generation}
% After the success of diffusion models in image generation, applying diffusion models to generate videos has become a very hot topic. Diffusion Transformer (DiT) is the dominant model architecture people use for video diffusion models. Some early works uses temporal and spatial attention (2D + 1D) to deal with video modality, and the most recent open-source models all apply 3D full attention since it gives much better performance and details. However, 3D full Attention incurs a very long context length, making the computation relatively slow. For example, CogVideoX V1.5 needs 10 minutes to generate a 10 second video. These makes the efficiency problem very important.

% \subsection{Efficient Diffusion Models}

% \TODO{Do not discuss them in details. Discuss them all. Say that they are orthogonal with ours.}

% \paragraph{Quantization}
% Quantization has been prove to be effective speedup the training and inference of large language models. Nowadays, this techniques has been extended to diffusion models for inference speedup. Q-Diffusion first propose a W8A8 solution to apply quantization to diffusion models. ViDiT-Q propose a timestep-aware dynamic quantization method to reduce the bit-width to W4A8. SVDQuant introduce a low-cost branch to alleviate the outlier issue for both activations and weights, making W4A4 a practical solution for diffusion models. SageAttention quantize the attention module into INT8 precision by proposing a smoothing technique. SageAttention V2 further push the limit to INT4 and FP8.

% \paragraph{Distributed Inference}
% Distrifusion, X-DiT

% \paragraph{Cache-based Methods}
% DeepCache utilize feature caching to modify the UNet Diffusion, $\Delta-DiT$ adapts this mechanism to DiT by caching residuals
% between attention layers to avoid redundant computation. PAB caches and broadcasts intermediate features at various timestep intervals. FasterCache observe that there's a huge redundant in CFG and optimizes the reuse of conditional and unconditional outputs. TeaCache finds that the similarity of model inputs can be used to predict the similarity of model outputs, and therefore propose a better machine strategy to achieve a better speedup. However, these methods usually makes the generated output diverge from the original output (PSNR smaller than 22). In comparison, our method can achieve a PSNR > 30, which makes the quality of our method much higher than these methods.

% \subsection{Sparse Attention Methods}
% Sparse attention has been widely used in large language models (StreamLLM, H2O, MInference, FlexPrefill). As the context length of DiT becomes very long, utilizing sparsity to accelerate diffusion models becomes very important. We are the first to study the sparsity problem in video generation.

% \TODO{Punish LLM sparse method here (MInference)}
% \TODO{Andy}


\begin{figure*}[!t]
	\centering
	\includegraphics[width=\linewidth]{Fig/flow.png}

	\caption{Method overview includes (a) a formative understanding of current personhood verification and related challenges through competitive analysis  (b) users' perception, preferences, and design through an interview study}
\label{fig:method}
\end{figure*}
\vspace{-2mm}
\section{Method Overview}
\label{sec:method}
\vspace{-2mm}
Building on the existing literature, it is clear that while significant progress has been made, a critical gap remains in understanding the key factors to operationalize personhood credentials that balance privacy, security, and trustworthiness online. 
%This challenge becomes even more pressing with the rise of increasingly advanced AI, which enables bad actors to scale their operations, exacerbating issues such as impersonation, fake identities, and non-human interactions. 
As outlined in Figure~\ref{fig:method}, our study comprises: (1) a competitive analysis of current personhood/identity verification tools to identify challenges. These insights inform the design of a user study aimed at (2) investigating users’ perceptions (RQ1), identifying factors influencing their preferences for personhood credentials (RQ2), and conceptualizing designs (RQ3) to address these challenges.

%Please add the flow digram / RQs of different methods with a method overview. see here https://arxiv.org/pdf/2410.01817?}


\vspace{-2mm}
\section{Formative Understanding of PHCs}
\vspace{-2mm}
In this section, we outline our formative analysis of existing personhood verification systems, which informed the design rationale for developing our user study (Section~\ref{user-study}).

%\subsection{Competitive Analysis \& Cognitive Walkthrough}
%\textbf{Competitive Analysis.}
%No prior studies have explored personhood credentials systems' usability and security issues. To address this gap, 
We systematically consolidated a list of systems based on their popularity, diversity in platform type (centralized vs. decentralized), and relevance to the domain of digital identity~\cite{idenaWhitepaper, kavazi2021humanode, kavazi2023humanode, de2024personhood, BrightID, PoH, adler2024personhood}
This consists of
%both practical implementations and state-of-the-art systems, including the 
World app, BrightID, Proof of Humanity, Gitcoin Passport, and Federated Identities (OAuth), etc (Table~\ref{tab:systems}). 
%as well as collected public user's review from Google Playstore. We chose these systems based on their popularity, diversity in platform type (centralized vs. decentralized), and relevance to the domain of digital identity\fixme{add citations of research papers from lit review}. 
Table~\ref{tab:identity_verification} provides an overview of different attributes of how existing systems operate and their design trade-offs. We found 15 apps categorized into six groups. Five of these were centralized, primarily government-based personhood verification systems. This initial categorization is based on the data requirements for issuing credentials varied, including behavior filters, biometrics (such as face, selfie, iris, or video), social graph and vouching mechanisms, physical ID verification, and, in some cases, combinations of these methods. 
\iffalse
\begin{table}[ht]
    \centering
    \scriptsize
    \begin{tabular}{llll}
      \hline
       App Name  & Source & reviews  \\
    
        \hline
     Worldapp & White Paper~\cite{WorldWhitepaper}, Google Play Store& 1523 \\
  BrightID & White Paper~\cite{BrightID},Google Play Store & 328 \\
  DECO & WhitePaper~\cite{zhang2020deco} & Review  \\
  CANDID & WhitePaper~\cite{maram2021candid} & Review \\
  Proof of Humanity &  WhitePaper~\cite{PoHexplainer} & Review \\
  Adhar Card &  WhitePaper~\cite{Aadhaar}, Google Play Store & Review
  %https://play.google.com/store/apps/details?id=in.gov.uidai.mAadhaarPlus&hl=en_US
  \\
Estonia e-ID  &  WhitePaper~\cite{estoniaE-ID} & Review\\
Chinese Credit system &  WhitePaper~\cite{ChinaSocialCreditSystem} & Review \\
Japan My Number Card &  WhitePaper~\cite{JapanMyIDNumber} & Review \\
ID.me &  WhitePaper~\cite{irsIdentityVerification, idAccessAll}, Google Play Store & Review \\
%https://play.google.com/store/apps/details?id=me.id.auth&hl=en_US
Idena &  WhitePaper~\cite{idenaWhitepaper} &  Review \\
Humanode &  WhitePaper~\cite{kavazi2021humanode} &Review\\
Civic &  WhitePaper~\cite{CivicPass} &Review \\
Federated identities (Oauth) &  WhitePaper~\cite{OAuth} & Review\\
  \hline
    
    \end{tabular}
    \caption{Competitive Analysis Data Sources 
   % \fixme{may move to appendix later}
    }
    \label{tab:systems}
\end{table}
\fi
%which helps us conduct a cognitive walkthrough. 

%we analyzed 15 popular systems in terms of their features, such as issuance system (centralized vs decentralized), types of data requirements for issuing credentials, types of  service providers of those systems. 
%Our competitive analysis allowed us to explore and identify multi-criteria to assess aspects such as privacy, usability, and security
We also documented on how users navigate the system and identify potential usability and security issues. Two UI/UX in out team evaluated whether users could successfully sign up and obtain personhood credentials. We independently compiled an initial list of evaluation results based on key questions. This includes- \textit{``How intuitive is the verification process?; How effectively does the platform provide feedback during different steps of registration and verification?; How do we as users feel regarding the data requirements in the verification systems?; How does the platform manage users' data?; What are the potential risks regarding users' privacy in the platform?''}
%about user workflows, task completion, and potential points of failure. 
%such as the intuitiveness of the verification process, feedback during registration, data requirements.
%data management, and privacy risks. 
%This included documenting account creation, data input, verification procedures, and associated risks. 
Given the limited access to systems like Estonia’s digital ID, Civic, and China’s social credit system, we used available white papers and documentation to reconstruct their workflows. Finally, we synthesized our observations and conducted qualitative coding to identify recurring themes.



\begin{table}[ht]
    \centering
    \scriptsize
    \begin{tabular}{llll}
      \hline
       App Name  & Source & reviews  \\
    
        \hline
     Worldapp & Documentation~\cite{WorldWhitepaper}, Google Play Store& 1523 \\
  BrightID & Documentation~\cite{BrightID},Google Play Store & 328 \\
  DECO & Documentation~\cite{zhang2020deco} & Review  \\
  CANDID & Documentation~\cite{maram2021candid} & Review \\
  Proof of Humanity &  Documentation~\cite{PoHexplainer} & Review \\
  Adhar Card &  Documentation~\cite{Aadhaar}, Google Play Store & Review
  %https://play.google.com/store/apps/details?id=in.gov.uidai.mAadhaarPlus&hl=en_US
  \\
Estonia e-ID  &  Documentation~\cite{estoniaE-ID} & Review\\
Chinese Credit system &  Documentation~\cite{ChinaSocialCreditSystem} & Review \\
Japan My Number Card &  Documentation~\cite{JapanMyIDNumber} & Review \\
ID.me &  Documentation~\cite{irsIdentityVerification, idAccessAll}, Google Play Store & Review \\
%https://play.google.com/store/apps/details?id=me.id.auth&hl=en_US
Idena &  Documentation~\cite{idenaWhitepaper} &  Review \\
Humanode &  Documentation~\cite{kavazi2021humanode} &Review\\
Civic &  Documentation~\cite{CivicPass} &Review \\
Federated identities (Oauth) &  Documentation~\cite{OAuth} & Review\\
  \hline
    
    \end{tabular}
    \caption{Competitive Analysis Data Sources 
   % \fixme{may move to appendix later}
    }
    \label{tab:systems}
\end{table}
%(presented in section~\ref{prac-cha}).

%\textbf{Cognitive Walkthough.}
%For the cognitive walkthrough, 
%We also focused on how a user would navigate the system and identify potential usability and security issues. Two experts, specializing in UI/UX and verification systems, evaluated whether users could successfully interact with the application interface and complete two tasks, (a) signing up with the system and (b) obtaining personhood credentials. We independently compiled an initial list of evaluation results by addressing key questions related to user workflows, task completion, and potential points of failure. This includes- \textit{``How intuitive is the verification process?; How effectively does the platform provide feedback during different steps of registration and verification?; How do we as users feel regarding the data requirements in the verification systems?; How does the platform manage users' data?; What are the potential risks regarding users' privacy in the platform?''}
%This included documenting (a) the step-by-step process of creating test accounts and (b) key steps such as data input requirements, verification procedures, and associated risks. Given that some relevant systems, such as Estonia’s digital ID, Civic, and China’s social credit system, are either inaccessible or operate as proof of concept models, we referenced available white papers and documentation to reconstruct their workflows. Finally, we synthesized the experts' observations and conducted qualitative coding to identify recurring themes in the evaluation (presented in section~\ref{prac-cha}). 
%These themes were categorized based on usability challenges, security concerns, and potential improvements in the interface design and verification process.
%Once the evaluations were done, we conducted a qualitative coding to understand the overall themes of the assessment.
%of the user interface and user experience, 

%focusing on ease of use, clarity, and overall usability; (b) we created test accounts to study and asses the workflow and documented the key steps, required information and potential privacy and security issues. Finally, we structured the data according to aforementioned criteria to highlight notable differences and their implications on usability and privacy.
%For evaluating the current verification process of some applications, we have utilized cognitive analysis of UI/UX, data requirement and privacy issue 
%We have selected some popular centralized and decentralized platforms such as World app, Bright ID, Proof of Humanity, Passport Gitcoin, Federated Identities (OAuth), Aadhar Card, Estonia's digital ID and China's social credit system . 

%For cognitive analysis of UI/UX, we have considered a few questions set: 
%\tanusree{from where did we get these questions? My impression was- we are doing cognitive analysis of ui/ux and data requirement, privacy issues, questions here doesn't reflect the goal of cognitive walkthrough}
% \begin{itemize}
%     \item How intuitive is the verification process?
%     \item How effectively does the platform provide feedback during different steps of registration and verification?
%     \item How do we as users feel regarding the data requirements in the verification systems?
%      \item How does the platform manage users' data?
%     \item What are the potential risks regarding users' privacy in the platform?
% %\end{itemize}
% %The following 2 questions have been utilized for data requirement analysis
% %\begin{itemize}
%     %\item What type of data (e.g., personal and biometric, etc) are required for issuing the credentials?
%     %\item In which stage, are these credentials requested from users? How we as users felt regarding the data requirements in the verification systems
% %\end{itemize}
% %We have also analyzed the privacy concerns using these 2 questions:
% %\begin{itemize}
   
% \end{itemize}


 %  \begin{figure*}
 % 	\centering
 % 	\includegraphics[width=0.8\linewidth]{Fig/worldapp.png}
 % 	\caption{ Worldapp-(a) lack of guidance on how users should navigate or utilize the app; (b) backup interface: requires users to connect Google Drive}
    
 % \label{fig: fig:worldapp}
 % \end{figure*}
%The competitive analysis aimed to evaluate and compare the verification processes of the \fixme{it should be a total of 15} eight selected verification systems (Table~\ref{tab:identity_verification}).
%The following predefined criteria were utilized to ensure a structured and consistent evaluation of the platforms:

% \begin{itemize}
%     \item Type of platform
%     \item Free or paid
%     \item Required data
%     \item Stage where data is required
%     \item Centralized or decentralized
%     \item Advantage
%     \item Disadvantage
%     \item UI/UX issue
%     \item Privacy related issue
% \end{itemize}

% We collected data for analysis using the following approach:
% \begin{itemize}
%     \item We analyzed the user interface and the user experience qualitatively and focused on ease of use, clarity and usability.
%     \item We created test accounts to study and asses the whole account creation workflow and documented the key steps and required information.
% \end{itemize}


  %  \item We reviewed official resources such as documentation and privacy policy to evaluate privacy concerns. 


\begin{table*}[h!]
    \centering
    \caption{Comparison of Existing Personhood Verification Systems}
    \label{tab:identity_verification}
    \resizebox{\textwidth}{!}{ 
    \begin{tabular}{l >{\small}l >{\small}l >{\small}l >{\small}p{3cm} >{\small}p{2.5cm} >{\small}l} 
        \hline
        \textbf{Category} & \textbf{Service Name} & \textbf{Architecture} & \textbf{Issuer} & \textbf{Credential} & \textbf{Platform} & \textbf{Free/Paid} \\
        \hline
        \hline
        \multirow{3}{*}{Behavioral Filter} 
        & CAPTCHA & Centralized & open-source, vendor & Recognize distorted texts, images, sounds etc. & Desktop and mobile browsers & Free/Paid\\
        & reCAPTCHA & Centralized & Google & Click checkbox & Desktop and mobile browsers& Free/Paid\\
        & Idena & Decentralized & open-source & Solve contextual puzzle & Blockchain & Free\\
        \hline
        \multirow{2}{*}{Biometrics}
        & World ID & Decentralized & World & Biometrics (iris scan) & App (iOS, Android) & Free\\
        & Humanode & Decentralized & Humanode & Biometrics (face) & Blockchain & Paid\\
        \hline
        Social Graph 
        & BrightID & Decentralized & open-source & Analysis of social graph & App (iOS, Android) & Free\\
        \hline
        Social Vouching 
        & Proof of Humanity & Decentralized & Kleros & Social vouching & Web & Paid\\
        \hline
        \multirow{2}{*}{Decentralized Oracle} 
        & DECO & Decentralized & Chainlink Labs & Cryptographic proof & Decentralized oracle & Under PoC\\
        & CANDID & Decentralized & IC3 research team & Cryptographic proof & Decentralized oracle & Under PoC\\
        \hline
        \multirow{4}{*}{Government-based ID} 
        & India Aadhaar Card & Centralized & Government & Document-based or Head Of Family-based enrollment + digital photo of face, 2 iris, and 10 fingerprints& Web, App (iOS, Android) & Free\\
        & Estonia e-ID & Decentralized & Government & Passport or EU ID + digital photo of face & Web, App (iOS, Android) & Paid\\
        & Japan My Number Card & Centralized & Government & Issue notice letter + photo ID or two non-photo IDs & Web, App (iOS, Android) & Free\\
        %& Chinese Credit System & Centralized & Gov & Personal credit records & Varies by region & Free\\
        \hline
        \multirow{2}{*}{Others} 
        & ID.me & Centralized & ID.me & Government-issued ID & Web & Free\\
        & Civic Pass & Decentralized & Civic & Government-issued ID, Biometrics (face), Humanness, Liveness & Web & Free\\
        \hline
    \end{tabular}
    }
\end{table*}

\begin{figure*}[h]
    \centering
    \begin{subfigure}{0.48\textwidth}
        \centering
        \raisebox{0.5\height}{
        \includegraphics[width=\textwidth]{Fig/idena.png}}
        \captionsetup{width=\textwidth, font=footnotesize} 
        \caption{Idena validation test interface: This requires users to select meaningful stories within a time limit, which can pose challenges for new users}
        \label{fig:idena}
    \end{subfigure}
    \hfill
    \begin{subfigure}{0.48\textwidth}
        \centering
        \includegraphics[width=\textwidth]{Fig/google_drive.png}
        \captionsetup{width=\textwidth, font=footnotesize} 
        \caption{World App backup interface: requires users to connect Google Drive}
        \label{fig:worldapp}
    \end{subfigure}
    
    \caption{PHC-related interfaces: (a) Idena validation test, (b) World App backup process.}
    \label{fig:phc_interfaces}
\end{figure*}

\vspace{-2mm}
\subsection{Challenges in Identity Verification}
\vspace{-2mm}
\label{prac-cha}
\textbf{Demanding Cognitive and Social Efforts for Verification Workflow.}
We found platforms such as World App and BrightID developed on decentralized technologies, 
including zero-knowledge proofs and social connections, may confuse non-technical users. For instance, user review from playstore suggested-many having issues understanding how to receive BrightID scores to prove they are sufficiently connected with others and verified within the graph. In their words \textit{``It's hard for me to connect with people to create the social graph.''} 
%\textbf{Usability Issue.}
%CAPTCHAs have become increasingly difficult to solve, can make the user journey cognitively demanding. To support the security of humanness verification, particularly image-based ones are becoming demanding for users. 
From experts' evaluation of UI/UX, we found Proof of Humanity lacks options to correct or update mistakes, which can make the registration process less user-friendly. %Incorporating the principle of error prevention could improve the user experience. 
Similarly, Idena's validation test (flip test) (Figure~\ref{fig:idena}) was challenging as new users as it required to create a meaningful story within the allotted time and earn enough points for validation. Simialrly, World App's(Figure~\ref{fig:worldapp}) account creation process to get an identifier doesn't inform users how and why to navigate the app can undermine intended functionality,  or underutilization of the app’s capabilities.


% \begin{figure*}[h]
%     \centering
%     \begin{minipage}{0.30\textwidth}
%         \centering
%         \includegraphics[width=\linewidth]{Fig/google drive.png}
%         \caption{World App backup interface: requires users to connect Google Drive.}
%         \label{fig:worldapp}
%     \end{minipage}
%     \hfill
%     \begin{minipage}{0.48\textwidth}
%         \centering
%         \includegraphics[width=\linewidth]{Fig/wordl1.png}
%         \caption{World App's account creation process: lack of guidance on how users should navigate or utilize the app.}
%         \label{fig:Worldapp1}
%     \end{minipage}
% \end{figure*}

\textbf{New or Complex System Rule to Recover ID. }
Both from UI/UX task and playstore review, we found the BrighID recovery process tedious and the rules unclear. A representative user review stated-\textit{``If you create an account and do not set up recovery connections you cannot get your account back. This forces you to create a new account which defeats the purpose of the app.''}
Another workflow of World App that requires users to connect their Google Drive to back up their accounts. However, this process may confuse users and create challenges during account recovery if they fail to complete the backup(Figure~\ref{fig:worldapp}).
 

%  \begin{figure}
%  	\centering
%  	\includegraphics[width=\linewidth]{Fig/wordl1.png}
%  	\caption{World App's account creation process: lack of guidance on how users should navigate or utilize the app}
%  \label{fig:Worldapp1}
%  \end{figure}


\textbf{Privacy and Data Requirement Issue. }
From our competitive analysis (Table~\ref{tab:litcomparison}), Data requirements across the systems vary significantly in scope and sensitivity. Decentralized platforms like World App, and BrightID required minimal data collection to issue ID while Proof of Humanity require video submission to receive a credential for was quite invasive when the videos were open to the public with clear faces.
%Similarly, both experts mentioned many unknown data policies for new platforms such as World app~\cite{WorldWhitepaper} and Bright ID\cite{BrightID}. 
While there is benefit of decentralization, often it is not clear how exactly service providers will handle the data in their policies and white papers.
%which created a reluctance for them, thus for new users to start using them. 
In contrast, Federated Identities OAuth\cite{OAuth} login process streamlines and this contributed to using known third-party service providers. This ensures ease of use as users need to specify the identity provider during the login or authentication process and grant access to their specific data. This reflects the importance of known entities and level of trust in data handling.
%However, they also have data being shared across multiple platforms which leads to some privacy concerns. 
Centralized systems, including Aadhaar and Estonia digital ID, require extensive personal and biometric data—fingerprints and iris scans—to ensure verification services while experts expressed privacy concerns towards china’s Social Credit personhood System, especially the use of it in measuring social scores.
%There was concerns regarding reCAPTCHA addressing usability issues by removing explicit verification tasks, relying instead on tracking user behavior, such as mouse movements, keystrokes, and browsing history. However, this approach trades off user privacy, as data collected during these activities raised concerns.


\textbf{Requirement of Optimal Device or Physical Presence.}\\
Government-supported systems like Aadhaar and Estonia e-Card feature structured interfaces but come with limitations: Aadhaar’s biometric registration may challenge rural populations, while Estonia’s dependence on smart-card hardware might exclude those without the necessary devices. Proof of Humanity, Humanode, Civic Pass may create challenges as proper lighting and optimal devices are necessary for taking the appropriate photo or video for biometric verification
%\fixme{need a screenshot for this}. 
On the contrast, Aadhaar card\cite{Aadhaar}\cite{AadhaarEnrollment}, Estonia's e-ID and Japan's My Number Card require one to be physically present and the issuing process takes a long time can create user frustration. 
%The existing systems and platforms that we have evaluated can hardly strike a balance between privacy, functionality and usability.  



%CAPTCHA\cite{Captcha} and reCAPTCHA\cite{reCaptcha} are 2 common human verification tools used across many websites. While CAPTCHAs add an additional step for users when they are trying to access a website, reCAPTCHAS come into play by removing any external verification. Rather, reCAPTCHAs track users' activities which has raised privacy concerns as there is lack of transparency between user and reCAPTCHA authority. Users are not sure how the tracking data will be used. 

\iffalse
\subsection{Results of UI/UX}
%\tanusree{Silvia: why do we have only 3 apps in the analysis?Ayae created a list a long ago. please complete the analysis for all the apps from this list}  \tanusree{I am not sure why facebook is in the analysis. we talked about only including verification apps, facebook is not one of them} \fixme{look at the Suggetsions in comment}
The eight \fixme{15 systems} systems evaluated manifest diverse approaches to user experience, emphasizing accessibility, intuitiveness, and transparency\fixme{write in active sentence or active voice, it reads like chatGPT and reviewer will think the same}. Platforms such as World App and BrightID developed on decentralized technologies, 
%though their intricate verification methods, 
including zero-knowledge proofs\fixme{add citation} and social connections \fixme{add as footnote what social connection means here and citation}, may confuse non-technical users. Proof of Humanity requires video submissions \fixme{what kind of video, is it their face? then talk about privacy, this doesn't seem to be a blockchain issue rather privacy issue}, a process potentially intimidating for individuals less familiar with blockchain platforms. 

In contrast, Federated Identities (OAuth) streamlines login processes via well-known third-party providers\fixme{who is the third-party provider for them}, ensuring ease of use for most users \fixme{is that all? }. 

Government-supported systems like Aadhaar and Estonia e-Card feature structured interfaces but come with limitations: Aadhaar’s biometric registration may challenge rural populations, while Estonia’s dependence on smart-card hardware might exclude those without the necessary devices. \fixme{add about Japan My Number Card.} 

Passport Gitcoin, focused on Web3 integration, struggles with clarity for users new to decentralized identity concepts. Finally, China’s Social Credit System delivers a seamless yet opaque experience, leaving users uncertain about the data influencing their scores.\par
Data requirements across the systems vary significantly in scope and sensitivity. Decentralized platforms like World App, BrightID, and Proof of Humanity emphasize minimal data collection but still require sensitive information, such as Ethereum addresses, social graphs, or video proofs, to ensure authenticity. 

Centralized systems, including Aadhaar and Estonia digital ID, require extensive personal and biometric data—fingerprints and iris scans—to ensure seamless service delivery. 

Passport Gitcoin, designed for Web3 wallet integration, relies on centralized storage, demanding significant user trust. Federated Identities (OAuth) achieves a balance by sharing limited data through third-party providers but this comes with the risk of overexposure. China’s Social Credit System stands out for its vast data collection, encompassing financial, social, and daily activities, raising alarm over pervasive monitoring and privacy intrusion.\par
Privacy concerns are critical across the eight systems, influenced by their data management practices. Decentralized platforms like World App and BrightID prioritize privacy, yet linking personal data to public blockchains—as seen in Proof of Humanity—poses inherent risks. Centralized systems like Aadhaar and Estonia e-Card depend on centralized databases, making them vulnerable to surveillance risks. Federated Identities (OAuth) simplifies access but could expose user data to third-party providers without explicit consent. Passport Gitcoin presents privacy challenges because users' information can be shared with third-party service providers. Meanwhile, China’s Social Credit System exemplifies extreme privacy erosion, extensively monitoring citizen behavior with minimal transparency about data use. Striking a balance between privacy and functionality remains a universal challenge for all these systems.

\fixme{citations to be added} We have evaluated 15 systems to present diverse approaches to user experience, emphasizing usability, accessibility, intuitiveness and transparency.
\fixme{citation didn't work} CAPTCHA\cite{Captcha} and reCAPTCHA\cite{reCaptcha} are 2 common human verification tools used across many websites. While CAPTCHAs add an additional step for users when they are trying to access a website, reCAPTCHAS come into play by removing any external verification. Rather, reCAPTCHAs track users' activities which has raised privacy concerns as there is lack of transparency between user and reCAPTCHA authority. Users are not sure how the tracking data will be used. 

\tanusree{no good content}
Platforms such as World app\cite{WorldWhitepaper} and Bright ID\cite{BrightID} are developed on decentralized technologies which include zero-knowledge proofs but do not present a clear and concise terms and conditions and privacy policy, which may create reluctance for new users to start using them. In figure 1(a), the on-boarding screen of World App appears with a consent checkbox to obtain explicit consent from the users that they agree to the "Terms and Conditions" and acknowledge the "Privacy Notice" of World App. But the terms and conditions and privacy notice are not mentioned in the same screen, tapping on the link buttons redirects users to a different screen, thus creating an obstacle in their user journey. If the necessary terms and conditions were presented clearly and concisely on the on-boarding screen, it would have informed users about the app's policies and ensure a smoother user journey. 1(b) represents the Bright ID license agreement, but it is too long to read. Users may not have enough patience to go through the details as it is time consuming and tap the agree button to continue. But this action may create privacy risks as users don't know what type of access they are providing to the application.
\begin{figure}[h]
     \centering
     \begin{subfigure}[b]{0.2\textwidth}
         \centering
         \includegraphics[width=\textwidth]{Fig/world app t&c.png}
         \caption{The terms and conditions and privacy notice are not mentioned in the World App's on-boarding screen}
         \label{fig:The terms and conditions and privacy policy are not mentioned in the World App's on-boarding screen}
     \end{subfigure}
     \hfill
     % \begin{subfigure}[b]{0.3\textwidth}
     %     \centering
     %     \includegraphics[width=\textwidth]{Fig/google drive.png}
     %     \caption{World App requires users to connect Google Drive for enabling backup}
     %     \label{fig:five over x}
     % \end{subfigure}
     % \hfill
     \begin{subfigure}[b]{0.3\textwidth}
         \centering
         \includegraphics[width=\textwidth]{Fig/bright id t&c.png}
         \caption{Bright ID's license agreement contains a long description which users may not want to read}
         \label{fig:three sin x}
     \end{subfigure}
     \hfill
        \caption{On-boarding screens of World App and Bright ID}
        \label{fig:three graphs}
\end{figure}
In figure 2, we can see World App requires users to connect their Google Drive to back up their world app accounts but this may lead users to providing access to their sensitive information.
\begin{figure}[h]
    \centering
    \includegraphics[width=0.5\linewidth]{Fig/google drive.png}
    \caption{World App requires users to connect Google Drive for enabling backup}
    \label{fig:World App requires users to connect Google Drive for enabling backup}
\end{figure}
% \iffalse
% \begin{figure}
%  	\centering
%  	\includegraphics[width=0.5\linewidth]{Fig/world app t&c.png}
%  	\caption{The terms and conditions and privacy policy are not mentioned in the World App's on-boarding screen}   
%  \label{fig:The terms and conditions and privacy policy is not clearly mentioned}
%  \end{figure}
%  \begin{figure}
%  	\centering
%  	\includegraphics[width=\linewidth]{Fig/bright id t&c.png}
%  	\caption{The license agreement and privacy policy is too long to read}   
%  \label{fig:The license agreement and privacy policy is too long to read}
%  \end{figure}
% . \par
%  \begin{figure}
%  	\centering
%  	\includegraphics[width=\linewidth]{Fig/google drive.png}
%  	\caption{World App requires users to connect Google Drive for enabling backup}
    
%  \label{fig:World App asking to connect Google Drive}
%  \end{figure}


 


Proof of Humanity\cite{PoH}\cite{PoHexplainer} offers a unique approach to verification with a social identification system. But the verification process requires users to connect their cryptocurrency wallet which will be publicly linked to users' account. Thus, users' wallet holdings and transaction history will be linked to users' identity which users may not prefer. 

In contrast, Federated Identities OAuth\cite{OAuth} provides streamlined login process via well known third-party service provides, also known as identity providers such as Google, Facebook etc. It ensures ease of use as users need to specify the identity provider during the login or authentication process and grant access to their specific data. But, data is shared across multiple platform which may raise privacy concerns among users. 

DECO\cite{zhang2020deco} and CanDID\cite{maram2021candid} are decentralized and privacy preserving oracle protocols where DECO allows users to prove the authenticity of website data obtained over TLS (Transport Layer Security) without revealing sensitive information. But Oracle has access to users' data which pose as a privacy risk. CanDID provides users with control of their own credentials but privacy depends on the honesty and integrity of verifiers and decentralized identity validators. 

Idena\cite{idenaWhitepaper}, Humanode\cite{Humanode} and Civic Pass\cite{CivicPass} - all are blockchain based person identification system where Idena performs validation by conducting flip tests and Humanode and Civic Pass are developed on crypto-biometric network. Though Idena does not collect any personally identifiable information, the behavioral data collected can be used in future for pattern analysis. 

Humanode and Civic pass both require biometric verification (face scan) which can create concerns among users about how their sensitive credential (face) will be managed by the systems. It is noteworthy that, most of the platforms are decentralized (World App, Bright ID, Proof of Humanity, Idena, Humanode, Civic), some requiring cryptocurrency wallet (Proof of Humanity, Civic Pass) and some requiring biometric verification (Proof of Humanity, Humanode, Civic Pass).    %citations to be added
\par
Government issued identity documents such as Aadhaar Card, Estonia's e-ID, China's social credit system and Japan's My Number Card are controlled and managed by central government. Citizens' sensitive credential can be at high risk if the government's security system is not robust enough to prevent any kind of hacking or data breaching. China's social credit system monitors citizen data extensively without maintaining complete transparency about data use and management. 

ID.me is another online identity network that enables individuals to verify their legal identity digitally. But privacy concerns arises as a single company holds a large amount of personal data and users have limited control over their data. %citations to be added
\par
Usability across these different platforms are critical. CAPTCHAs have become increasingly difficult to solve, often leading users to leave the website or platform without completing their user journey. Accessibility remains another issue as visually impaired users are unable to solve any CAPTCHA that is text or image based. reCAPTCHA comes with the solution of these problems but trading of users' privacy as users' data is being tracked down by the authority. 

From Figure 3 and 4, it is apparent that World app and Bright ID provide a simple and intuitive account creation form but an introductory video or step by step guide would be more helpful to guide users to navigate throughout the applications and perform necessary actions.
 \begin{figure}
 	\centering
 	\includegraphics[width=\linewidth]{Fig/world app account creation.png}
 	\caption{World App's account creation process is simple but doesn't inform users about how they should navigate or use the app \fixme{silvia, is there a reason you added all these UIs? why all of the uis are randomly placed, I shared examples so many times, i am not seeing anything I gave instruction.}}
 \label{fig:World App's on-boarding process}
 \end{figure}
 
 \begin{figure}
 	\centering
 	\includegraphics[width=\linewidth]{Fig/bright id account creation.png}
 	\caption{The "Create my BrightID" process in the Bright ID app is straightforward but lacks guidance on how users should navigate or utilize the app effectively. \fixme{explain why these screenshots are important to add from cognitive walkthrough. caption itself should be self explanatory with text explaining in the main body}}
    
 \label{fig:Bright ID's on-boarding process}
 \end{figure}
The principle of error prevention could make the user journey of registration in Proof of Humanity more user-friendly. As there is no option to correct or update any mistake, it may increase user frustration. Idena's validation test (flip test) (Figure 4) can be inconvenient for new users as they may struggle to find the meaningful story in the provided time and collect points to validate them.
 \begin{figure}
 	\centering
 	\includegraphics[width=\linewidth]{Fig/idena.png}
 	\caption{Idena validation test interface requiring users to select meaningful stories within a time limit which can be challenging for new users \fixme{anyone reading this caption would not understand anything}}
    
 \label{fig:Selecting meaningful story for validation process on Idena}
 \end{figure}
The platforms requiring video selfie or face scan (Proof of Humanity, Humanode, Civic Pass) may create another challenging situation for users as proper lighting and optimal devices are necessary for taking the appropriate photo or video for biometric verification. 

Aadhaar card\cite{Aadhaar}\cite{AadhaarEnrollment}, Estonia's e-ID and Japan's My Number Card are all government based credentials but completing all the formalities and getting the card takes a long time, sometimes creating user frustration. The existing systems and platforms that we have evaluated can hardly strike a balance between privacy, functionality and usability.   %citations to be added


% \begin{figure}[!t]
% 	\centering
% 	\includegraphics[width=\linewidth]{Fig/world app.png}
% 	\caption{New account creation process in  World App}
    
% \label{fig:New account creation process in  World App}
% \end{figure}
% \begin{figure}[!t]
% 	\centering
% 	\includegraphics[width=\linewidth]{Fig/bright id.png}
% 	\caption{New account creation process in  Bright ID}
    
% \label{fig:New account creation process in  Bright ID}
% \end{figure}


\subsection{Reddit Analysis}
%\tanusree{ishan to add}
We first collected \fixme{X} posts and  \fixme{X} comments on December 24th, 2024, using the Python Reddit API Wrapper (PRAW)~\footnote{https://praw.readthedocs.io/en/stable/}. We gathered the data from various relevant subreddits, ensuring a broad and comprehensive understanding of what users discuss on identify verification or personhood verification. Through qualitative analysis of this Reddit data, we were able to uncover detailed insights into the rich and prevalent usage of verification systems. This analysis highlighted users' current usage, potential challenges and risks they encounter. These findings provide a solid foundation to explore these themes further in subsequent in-depth interviews.

\paragraph{Data Collection}
 To comprehensively cover content related to our research questions on personhood verification, we first created a list of search keywords by identifying close terminologies related to \textit{``personhood verification''} (general keywords) and \textit{``bot check''} (technology-focused keywords), etc. We utilized a combination of general and technology-focused keywords in our search. We employed general terms such as Personhood Verification, Identity Proof, Human Check and Bot Check. These keywords were designed to capture posts authored by or discussing personhood verification. For the technology focus, we used terms such as \fixme{add}. These keywords targeted discussions specifically about the use of popular tools and platforms. We conducted open searches combining these keywords across Reddit to gather data from various subreddits.
 Other than open searches, we also applied specific criteria to select subreddits, ensuring comprehensive coverage of relevant discussions: these subreddits should focus either on the personhood verification community or technology. We chose subreddits with the most active users online during our browsing sessions. The full list of subreddits and search keywords used is detailed in Table\fixme{need to find out the subreddit most prevalent discussing these topic}. 

\paragraph{Analysis}
Two researchers reviewed each post and categorized related posts or comments into five overarching high-level themes: \fixme{need to add after data analysis}. Within these categories, 53 level-2 themes were identified, such as \fixme{need to add after data analysis}. During the analysis process, researchers regularly convene to discuss discrepancies and emerging themes in the codebook, aiming to reach a consensus. These categories allowed us to investigate RQ2 and partially address RQ1. 

\subsection{Results}
% \tanusree{ishan to add}
\fi
\vspace{-2mm}
\section{ User Study Method}
\vspace{-2mm}
\label{user-study}
This section outlines the method for exploring users' perceptions and preferences of personhood credentials. We conducted semi-structured interviews with 23 participants from the US, and the EU/UK in October 2024.
%We started with a round of pilot studies (n=5) to validate the interview protocol. Based on the findings of pilot studies, we revised the interview protocol and conducted the final round of interviews (n=17). 
The study was approved by the Institutional Review Board (IRB).
\vspace{-2mm}
\subsection{Participant Recruitment}
\vspace{-2mm}
We recruited participants through (1) social media posts, (2) online crowdsourcing platforms, including CloudResearch and Prolific. Respondents were invited to our study if they met the selection criteria: a) 18 years or older and b) living in the US or the EU/UK. Participation was voluntary, and participants were allowed to quit anytime. Each participant received a \$30 Amazon e-gift card upon completing an hour-and-a-half interview.

\subsection{Participants}
%\tanusree{check for final count} \ayae{updated percentage with final 23 counts} 
We interviewed 23 participants, 10 from the US and 12 from the EU/UK. The majority of the participants (61\%) were in the age range of 25-34, followed by 22\% were 35-44 years old. The participants were from the United States and various countries, namely Spain, Sweden, Germany, Hungary, and the United Kingdom. Participants had different backgrounds of education levels, with 87\% of participants holding a Bachelor’s degree and 65\% holding a graduate degree. 65\% of participants had a technology background, while 48\% of them had a CS background. All participants reported using online services that required them to verify their personhood. Table~\ref{table:demographics} presents the demographics of our participants. We refer to participants as P1,. . . ,23.
\begin{table*}[h!]
\centering
%\scriptsize
\caption{Overview of PHC Application Scenarios}
\label{table:scenario}
%\resizebox{\textwidth}{!}{%
\begin{tabular}{lll}
\hline
\textbf{Scenario} & \textbf{Service} & \textbf{Credential} \\
\hline
Financial service & Bank, Financial institutions & Passport or Driver’s license, Face scan \cite{yousefi2024digital}\\
% \hline
Healthcare service & Hospitals, Clinics & Health insurance card,  Fingerprint \cite{chen2012non,fatima2019biometric,jahan2017robust}\\
% \hline
Social media & Tech companies & National identity card, Video selfie \cite{instagramWaysVerify, metaTypesID,instagramTypesID} \\
% \hline
LLM application & Tech companies & Iris scan \cite{WorldWhitepaper, worldHumanness}\\
% \hline
Government service & Government & Driver’s license or National identity card \cite{LogingovVerify}\\
% \hline
Employment background check & Background check companies & Tax identification card, Fingerprint\cite{cole2009suspect}\\
\hline
\end{tabular}%
%}
% \vspace{0.5em}
\label{tab:scenarios}
\end{table*}
\begin{table*}[h]
\centering
\caption{Participant demographics and background.}
%\fixme{add the participants you completed so far}
\resizebox{\textwidth}{!}{%
\begin{tabular}{l l l l l l l l}
\hline
\textit{Participant ID} & \textit{Gender} & \textit{Age} & \textit{Country of residence} & \textit{Education} & \textit{Technology background}  & \textit{CS background} &\textit{Residency duration} \\
\hline
P1 & Male & 25-34 & the US & Master's degree & Yes & Yes &3-5 years\\
P2 & Female & 25-34 & the US & Master's degree & Yes & Yes & 1-3 years\\
P3 & Female & 25-34 & the UK & Master's degree & Yes & No & 1-3 years\\
P4 & Female & 35-44 & the UK & Some college, but no degree & Yes & Yes & Over 10 years \\
P5 & Male & 25-34 & the US & Doctoral degree & Yes & Yes & 5-10 years \\
P6 & Male & 35-44 & the US & Less than a high school diploma & No & No & Over 10 years \\
P7 & Male & 25-34 & the US & Doctoral degree & Yes & Yes & 3-5 years\\
P8 & Male & 45-54 & the US & Bachelor's degree & Yes & Yes & Over 10 years \\
P9 & Female & 25-34 & New Zealand & Master's degree & No  &  No &  Over 10 years\\
P10 & Male & 25-34 & the US & Master's degree & No & No & Over 10 years\\
P11 & Female & 25-34 & the UK & Bachelor's degree & No & No & Over 10 years\\
P12 & Male & 18-24 & the UK & Master's degree & Yes & Yes & 1-3 years\\
P13 & Male & 35-44 & the UK & Bachelor's degree & Yes & No & Over 10 years\\
P14 & Male & 25-34 & Sweden & High school graduate & No & No & Over 10 years \\
P15 & Female & 25-34 & Spain & Master's degree & Yes & Yes & Over 10 years \\
P16 & Female & 25-34 & Germany & Master's degree & Yes & Yes & Over 10 years \\
P17 & Female & 25-34 & Spain & Doctoral degree & No & No & Over 10 years \\
P18 & Female & 35-44 & the US & Bachelor's degree & No & No & Over 10 years \\
P19 & Female & 25-34 & Germany & Master's degree & Yes & Yes & 3-5 years \\
P20 & Male & 25-34 & Hungary & Master's degree & Yes & No & 3-5 years \\
P21 & Male & 35-44 & the US & Bachelor's degree & Yes & No & 5-10 years \\
P22 & Female & 18-24 & France & Master's degree & Yes & Yes & Less than 1 year\\
P23 & Male & 45-52 & the US & Master's degree & No & No & Over 10 years\\
\hline
\end{tabular}%
}
\label{table:demographics}
\end{table*}


\vspace{-2mm}
\subsection{Semi-Structured Interview Procedure} \label{sec:study_protocol}
\vspace{-2mm}
%\fixme{explain in details why the study designed in a certain way. please read papers to learn more. data minimization and advertisement paper. The method section is too bland. We have a wonderful study design. Scenario-specific study design, describe scenarios and why chose this scenario. Mainly method should include all design rationale, and example questions when necessary to clarify your rational}

We started with a round of pilot 
%(Appendix~\ref{pilot}) 
studies (n=5) to validate the interview protocol. Based on the findings of pilot studies, we revised the interview protocol.

\textbf{Open Ended Discussion.} We designed the interview script based on our research questions outlined in the introduction section~\ref{sec:introduction}. 
%We added the interview script to the section~\autoref{protocol}. 
At the beginning of the study, we received the participants’ consent to conduct the study. Once they agreed, we proceeded with a semi-structured interview. The study protocol was structured according to the following sections: (1) Current practices regarding digital identity verification; (2) Users' perception of PHC before and after watching the informational video; (3) Scenario-based session to investigate factors that influence users' preferences of PHC; 
%(4) Users' preference of PHC; 
(4) Design session to conceptualize users' expectations; (5) A brief post-survey on Users' Preference of PHCs.
%of PHCs in different scenarios.

In the first section, we first asked a set of questions to understand participants' current practices of online platforms and the types of identity verification methods they had experience with. This is to understand their familiarity with different types of verification, such as biometrics, physical IDs, etc.
%and methods that might have worked well based on their prior experience.

%of online identity verification. When participants mentioned certain types of online services that required identity verification, we inquired about their experience with verification method. Was it easy to use, or did you run into any issues?"} We further inquired about any challenges participants faced with identity verification - \textit{"Did you encounter any challenges when using this method?"} 
%If biometrics didn’t naturally come up in prior discussions, we prompted to consider them- \textit{"Have you ever used services where you had to verify yourself through face, fingerprints, or iris scans, or other biometrics?"} If they mentioned any experience with biometric verification, we followed up with questions like- \textit{"What worked well? Were there any concerns you had?"}
In the second section, we then asked about participants' current understanding and perception of personhood credentials either from prior knowledge or from intuition by just hearing the term. %We also asked if they knew how personhood credentials work, particularly how it has been handled by the different services they use. 
%As all participants had never heard of PHC, we prompted them to interpret the term based solely on its wording. 
While the majority recognized this as unfamiliar terminology, most inferred that it referred to a form of personal identification, often associating it with biometric verification.
%In the pilot interviews, The majority of the participants could not provide substantial responses on their understanding of how personhood credentials work, before starting the second part of the interview, we showed them an informational video on personhood credentials.
%Most of the participants were unfamiliar with this term, so we then asked \textit{ Can you explain what you think it means by just hearing the term?"} 
%Before proceeding with the third section of the interview, we assessed participants' understanding of PHC with knowledge questions.
Then, we showed them an introduction video on PHC \footnote{https://anonymous.4open.science/r/PHC-user-study-14BB/}, %\fixme{create an anonymous GitHub, upload the video and add a footnote here} \ayae{reflected}. 
%The video provides an overview of PHCs, 
covering their definition, 
%the steps involved in issuing and using them, 
and implications of it in online services. Based on former literature\cite{adler2024personhood}, we designed the video with easy-to-understand text, visuals, and audio to make the concepts accessible to average users. We created a set of knowledge questions to assess participants' understanding of PHC before and after showing the video. %as attached in Appendix~\ref{knowledge_questions}.

%including the same knowledge questions. 
%Most participants correctly responded to knowledge questions, which ask the basic understanding of digital identity crisis and personhood credentials. 
%Even before showing the introduction video, regarding the question \textit{"What could happen if online identities are poorly verified?"}, 95\% correctly selected \textit{"Fake accounts, bots, and fraud could increase significantly."} For the question \textit{"What are Personhood Credentials (PHCs)?"}, 90\% correctly choose the option \textit{"Digital credentials that confirm a person’s identity."} 
For instance, we observed an improve in correct response rate for the question, such as, \textit{``What is the primary goal of PHC?''} from 85\% to 100\% after watching the video.
%where the correct answer was \textit{"To verify a person's identity without exposing personal information."} 
%However, regarding the question \textit{"To whom do you provide minimal personal information during the PHC process?"}, only 35\% selected the correct answer \textit{"PHC issuers (e.g., governments or trusted organizations)"}, while the most frequent response was \textit{"Online service providers (e.g., social media)"} at 45\%.
%\ayae{KQ results reflected}
%We also asked some open-ended questions to evaluate whether our introduction video helped participants better understand PHC \textit{''How would you explain your understanding of personhood credentials?''} 
%We further asked what benefits and concerns came to mind for them.
In the third section, we focused on scenario-based discussions, exploring specific applications of PHC to understand factors that influence participants' preferences towards PHCs as well as identify challenges to leverage in PHC design for various services. We examined the following six scenarios: (1) Financial service, (2) Healthcare service, (3) Social Media, (4) LLM applications, (5) Government Portal, and (6) Employment Background Check.
%We covered a wide range of use cases of online personhood verification via these six scenarios since they encompass diverse user needs, security and usability, and privacy requirements. %\fixme{please see the comment with iffalse tag and make it concise, we talked about it before}
\iffalse
%Firstly, financial system is a critical scenario for identity verification where high level of security protections are expected as exemplified by KYC. Thus, such services continue to develop transformative digital identity verification to ensure the security and integrity of financial transactions\cite{parate2023digital}. The second scenario is healthcare systems, which also have high privacy requirements due to the confidentiality of medical data. The pandemic has accelerated the adoption of online healthcare services and in response to this digital transformation, the recent study has proposed blockchain-based decentralized identity management systems \cite{javed2021health}. Thirdly, we consider the scenario of social media, which faces the critical challenges of online identity as shown in spreading misinformation and harmful content from fake or anonymous accounts \cite{ceylan2023sharing}. The fourth scenario is designed with a specific context of interacting with Large language models (LLMs). The former study discussed vulnerability in dialog-based systems where adversaries can exploit the training process to introduce toxicity into responses \cite{weeks2023first}. Thus, such vulnerabilities indicate identity verification may also be important for LLM applications. Fifth, government services are familiar situations that require people to verify their identity. Various countries have developed their own electronic ID schemes \cite{stalla2018gdpr}. Lastly, we also cover the scenario of employment background checks needing precise identity verification to ensure the reliability of applicants. The current background check system involves vulnerable processes that increase the risk of identity theft and unauthorized data access.\cite{blowers2013national}. Such challenges underline the relevance of PHCs, which can mitigate risks by providing a secure framework for verification.
%\ayae{included citation}
\fi
We have also incorporated various types of data or credentials requirements (e.g. physical id, biometrics, etc) across scenarios to maintain diversity in our discussion with participants as shown in Table.\ref{table:scenario}. %For instance,
%we  We have multiple existing verification methods, including 
%humanness verification (e.g., selfie, video call), document-based verification (e.g., government-issued ID), and biometrics information. 
We selected types of credentials for each scenario based on former literature and existing PHC as explained in the section \ref{subsec:verification_practice}. %\fixme{cite worldcoin, and other app and literature}. \fixme{from here to end of this paragraph ---These needs to go to the literature review section on the current usecase of PHC. And only 2 line summarizing why you chose the diverse type of credential data and refer to the literature section}


%% Let me find the former literature to explain why we select these credentials
For each of the six scenarios, we explored participants' perceptions of using PHC in hypothetical situations that align with the research focus as well as to help participants can relate PHC concepts to real-world applications. This is particularly useful for this study where where user perceptions and expectation under specific conditions are crucial to devising solutions \cite{carroll2003making}.
%\fixme{cite scenario method paper from jack caroll}.
%\ayae{reflected}
We asked about their feelings, perceived benefit and risks. We also nudge them to think about any privacy and security perception around using PHC and types of data (e.g., iris, face, government id, etc) involved in issuing PHC. 

\fixme{
%\textbf{Pre-understanding: Guessed it as one of the verification methods} 
%The majority of the participants were not familiar with the term ``Personhood Credential'', although most of them used some forms of such credentials. 
%As all participants have never heard of PHC, we prompted them to interpret the term based solely on its wording. Most of them inferred that it referred to another type of person identification. 
%For instance, P3 commented \textit{``It can be anything that would point to one single individual that would differentiate that individual from others.''} When participants expressed how PHC identifies a person's uniqueness, their understanding ranged from verifying basic information such as address or age, and certain eligibility to advanced identification of digital identity (e.g., behavioral, economical, etc) with Multi-factor authentication or knowledge-based questions.

%\textbf{Post-understanding: Involvement of trusted entity} When asked to explain their understanding of PHC, P13 noted, \textit{"So it sounds like, basically, you it's similar to how you verify things before. Like you use a biometrics and your government Id. But then you get a personal key. You do it with like a trusted organization rather than each individual. And then you can use that key for all the different services you use."} P1 elaborated PHC process as a shift of the verification entity, \textit{" I'd say we are sort of moving the verification burden from the user side to a service provider side where they have access to our data, and they have access to the token that's assigned to each person that's unique. And that's easily like traceable across online platforms. and this token is used for verification with 3rd parties, where they don't get access to your personal data, but they only use this service provider to give them the authenticity that you are a real user."} These suggest that the role of the PHC issuer is recognized as a crucial component of PHC.}
%began by asking \textit{"How did you feel about using PHC to verify your identity when opening your bank account?"} To dive deeper, we also asked about potential benefits:\textit{"What potential benefits do you see in using PHC in this online banking context?"}. We also inquired about these aspects- \textit{"Do you think using PHC improves the security of your bank account? Why?", "Did this method of identity verification make you feel more confident about your privacy? why?"} Additionally, we discussed their comfort levels for providing credentials (e.g., Government-issued ID, biometric information) and asked about any concerns about data collection-\textit{"Were you comfortable providing your government-issued ID and using facial recognition? Why?"}
}

\iffalse
%%% column: scenario, credential, service providers.
\begin{table*}[h!]
\centering
\caption{Overview of PHC Application Scenarios}
\label{table:scenario}
%\resizebox{\textwidth}{!}{%
\begin{tabular}{lll}
\hline
\textbf{Scenario} & \textbf{Service Provider} & \textbf{Types of Credential} \\
\hline
Financial Service & Bank, Financial Institutions & Passport or Driver’s license, Face scan \cite{yousefi2024digital}\\
% \hline
Healthcare Service & Hospitals, Clinics & Health insurance card,  Fingerprint \cite{chen2012non,fatima2019biometric,jahan2017robust}\\
% \hline
Social Media & Tech Companies & National identity card, Video selfie \cite{instagramWaysVerify, metaTypesID,instagramTypesID} \\
% \hline
LLM Application & Tech Companies & Iris scan \cite{WorldWhitepaper, worldHumanness}\\
% \hline
Government Service & Government & Driver’s license or National identity card \cite{LogingovVerify}\\
% \hline
Employment Background Check & Background Check Companies & Tax identification card, Fingerprint\cite{cole2009suspect}\\
\hline
\end{tabular}%
%}
% \vspace{0.5em}
\label{tab:scenarios}
\end{table*}
\fi


%\textbf{Design Session.}
%\fixme{need to explain how and why you design the design session, where you designed, how participants were unstructured and so on.} \ayae{reflected in the following paragraph}

In the fourth section, we began by refreshing participants’ memories of the various risks and concerns discussed in the earlier scenario-based section. Following this, we guided participants to brainstorm potential design solutions by sketching their ideas to address these concerns. To facilitate the sketching process, we developed sketch notes in Zoom as prompts to help participants generate ideas, particularly when starting from scratch is challenging. 
%on Zoom whiteboard or pen and paper, using a think-aloud protocol.  
%Nevertheless, it is difficult to develop new ideas from scratch, so 
%Additionally, we described the main issues or concerns that the participants identified during the interview at the top of the sketch notes. 
%Participants can develop their ideas at the center of the whiteboard by locating the above components or creating new shapes, lines, or text boxes for their sketches. 
We also investigated participants' preferences for PHC regarding the issuers and issuance systems of PHCs, as well as the types of data required for issuing PHCs. 
%in the context of who issues PHC or type of issuance systems, and what types of data are needed to issue PHC to address RQ2. 
%An example includes- \textit{``What types of credential would you prefer to use as personhood verification? ; Which organizations or stakeholders would you prefer to issue and manage your PHC?''} 
We encourage participants to explain their reasoning. These questions were informed by insights from the pilot study, where participants expressed preferences for different types of data, system architecture, and various stakeholders involved in PHC issuing.
%However, these questions alone can only find optimal ways within the scope of currently existing options and cannot generate new design implications. Therefore,

\iffalse
\tanusree{we can cut this section as this didn't give any result and doesn't answer RQs directly.}Lastly, to understand preference on issuance system, we introduced the decentralized PHC system architecture with another instructional video. Following the video, we asked participants to explain their understanding of the decentralized PHC system and their preference for the issuance system (centralized or decentralized). We introduced it after the sketch session is that participants may organically come up with the idea of decentralized systems on their own, and we intended to avoid priming them. 
\fi
%Then, we asked them to explain their understanding of the decentralized PHC and preferred issuance system (centralized or decentralized.)- \textit{`` Could you explain why you would prefer decentralized system in managing your PHCs?''}
%\textit{"Would you prefer to get multiple PHCs from different issuers depending on the situation or application you're using, or would you rather have a single PHC from one issuer?"}

\textbf{Post-Survey.}
%%\fixme{need to explain how and why you design the design session, where you designed, how participants were unstructured and so on.}
We conducted a post-survey to obtain participants' PHC preference quantitatively. It included questions on participants' preference on credential type, issuer and issuance system  for the scenarios (e.g., financial, medical, etc) we considered in our interview.

\vspace{-2mm}
\subsection{Data Analysis}
\vspace{-2mm}
Once we got permission from the participants, we obtained interview data through the audio recording and transcription on Zoom. We analyzed these transcribed scripts through thematic analysis \cite{Braun2012-sz, Fereday2006-yv}. Firstly, all of the pilot interview data was coded by two researchers independently. Then, we compared and developed new codes until we got a consistent codebook. Following this, both coders coded 20\% of the interview data of the main study. We finalized the codebook by discussing the coding to reach agreements. Lastly, we divided the remaining data and coded them. After both researchers completed coding for all interviews, they cross-checked each other’s coded transcripts and found no inconsistencies. Lower-level codes were then grouped into sub-themes, from which main themes were identified. Lastly, these codes were organized into broader categories. Our inter-coder reliability (0.90) indicated a reasonable agreement between the researchers.
\iffalse

\begin{table*}[h]
\centering
\caption{Participant demographics and background.}
%\fixme{add the participants you completed so far}
\resizebox{\textwidth}{!}{%
\begin{tabular}{l l l l l l l l}
\hline
\textit{Participant ID} & \textit{Gender} & \textit{Age} & \textit{Country of residence} & \textit{Education} & \textit{Technology background}  & \textit{CS background} &\textit{Residency duration} \\
\hline
P1 & Male & 25-34 & the US & Master's degree & Yes & Yes &3-5 years\\
P2 & Female & 25-34 & the US & Master's degree & Yes & Yes & 1-3 years\\
P3 & Female & 25-34 & the UK & Master's degree & Yes & No & 1-3 years\\
P4 & Female & 35-44 & the UK & Some college, but no degree & Yes & Yes & Over 10 years \\
P5 & Male & 25-34 & the US & Doctoral degree & Yes & Yes & 5-10 years \\
P6 & Male & 35-44 & the US & Less than a high school diploma & No & No & Over 10 years \\
P7 & Male & 25-34 & the US & Doctoral degree & Yes & Yes & 3-5 years\\
P8 & Male & 45-54 & the US & Bachelor's degree & Yes & Yes & Over 10 years \\
P9 & Female & 25-34 & New Zealand & Master's degree & No  &  No &  Over 10 years\\
P10 & Male & 25-34 & the US & Master's degree & No & No & Over 10 years\\
P11 & Female & 25-34 & the UK & Bachelor's degree & No & No & Over 10 years\\
P12 & Male & 18-24 & the UK & Master's degree & Yes & Yes & 1-3 years\\
P13 & Male & 35-44 & the UK & Bachelor's degree & Yes & No & Over 10 years\\
P14 & Male & 25-34 & Sweden & High school graduate & No & No & Over 10 years \\
P15 & Female & 25-34 & Spain & Master's degree & Yes & Yes & Over 10 years \\
P16 & Female & 25-34 & Germany & Master's degree & Yes & Yes & Over 10 years \\
P17 & Female & 25-34 & Spain & Doctoral degree & No & No & Over 10 years \\
P18 & Female & 35-44 & the US & Bachelor's degree & No & No & Over 10 years \\
P19 & Female & 25-34 & Germany & Master's degree & Yes & Yes & 3-5 years \\
P20 & Male & 25-34 & Hungary & Master's degree & Yes & No & 3-5 years \\
P21 & Male & 35-44 & the US & Bachelor's degree & Yes & No & 5-10 years \\
P22 & Female & 18-24 & France & Master's degree & Yes & Yes & Less than 1 year\\
P23 & Male & 45-52 & the US & Master's degree & No & No & Over 10 years\\
\hline
\end{tabular}%
}
\label{table:demographics}
\end{table*}
\fi
\begin{table*}[!ht]
\centering
\caption{Comparison of RAViT variant with State-Of-The-Art on ImageNet-1K Dataset. GPU inference throughput is measured using RTX3090. NPU and Edge are the latency that was measured on the iPhone 15 Pro using CoreML format and the NVIDIA Jetson Orin Nano Edge Device using ONNX format.}
\begin{tabular}{ m{3.8cm}|>{\centering}m{0.8cm}|>{\centering}m{1.1cm}|>{\centering}m{1.0cm}|>{\centering}m{1.1cm}|>{\centering}m{1.0cm}|>{\centering}m{0.8cm}|>{\centering}m{0.8cm}|c }
\hline
\multirow{2}{*}{Model} & \multirow{2}{*}{Type} & Img  & Param & FLOPs & GPU & NPU & Edge & \multirow{2}{*}{Top-1} \\ 
                          & & Size  &   (M)  & (G)  & (Img/s) & (ms) &  (ms) &   \\ \hline
% EfficientFormerV2-S0              &Hybrid & 224 & 3.5  & 0.4 & 1274 & 0.64 & 11.8 & 73.7 \\
% EdgeNeXt-XS                       &Hybrid & 256 & 2.3  & 0.6 & 3126 & 26.4 & 10.4 & 75.0 \\
% EMO-5M                            &Hybrid & 224 & 2.3  & 0.5 & 2037 & 2.44 & 16.5 & 75.1 \\
% FastViT-T8\cite{vasu2023fastvit}  &Hybrid & 256 & 3.6  & 0.7 & 3909 & 0.67 & 8.2  & 75.6 \\
% % \rowcolor{gray!30}
% % \textbf{RAViT-XT}               &Hybrid & 224 & 6.8  & 0.7 & 4358 & 0.89 & 10.6 & 76.9? \\
% % \hline 
% PoolFormer-S12                    &Hybrid & 224 & 11.9 & 1.8 & 2750 & -    & 8.2  & 77.2 \\
% RepViT-M0.9                       &Conv   & 224 & 5.1  & 0.8 & 4817 & 0.60 & 12.1 & 77.4 \\
% MobileOne-S2                      &Conv   & 224 & 7.8  & 1.3 & 2793 & 0.77 & 11.2 & 77.4 \\
EfficientFormerV2-S1\cite{li2023rethinking} &Hybrid & 224 & 16.1 & 0.7 & 1153 & 0.76 & 14.3 & 77.9 \\
% RepViT-M1.0                     &Conv   & 224 & 6.8  & 1.1 & 3910 & 0.69 & 12.3 & 78.6 \\
MobileViTV2-1.0\cite{mehta2023separable}    &Hybrid & 224 & 4.9  & 1.8 & 1858 & 1.94 & 14.4 & 78.1 \\
MobileOne-S3\cite{vasu2023mobileone}        &Conv   & 224 & 10.1 & 1.8 & 2793 & 0.93 & 15.2 & 78.1 \\
\rowcolor{gray!30}
\textbf{RAViT-T26}                    &Hybrid & 224 & 8.2  & 0.9 & 3755 & 0.71 & 12.9 & 78.4 \\
\hline 
EMO-6M\cite{zhang2023rethinking}      &Hybrid & 224 & 5.1  & 1.0 & 1916 & 2.49 & 16.9 & 79.0 \\
FastViT-T12\cite{vasu2023fastvit}     &Hybrid & 256 & 6.8  & 1.4 & 3182 & 0.99 & 10.9 & 79.1 \\
EdgeNeXt-S \cite{maaz2022edgenext}    &Hybrid & 256 & 5.6  & 1.3 & 1207 & 32.5 & 13.1 & 79.4 \\
MobileOne-S4 \cite{vasu2023mobileone} &Conv   & 224 & 14.8 & 2.9 & 1892 & 1.21 & 19.0 & 79.4 \\
RepViT-M1.1\cite{wang2024repvit}      &Conv   & 224 & 8.2  & 1.3 & 3604 & 0.74 & 11.4 & 79.4 \\
\rowcolor{gray!30}
\textbf{RAViT-S22}                &Hybrid & 224 & 10.2 & 1.2 & 3491 & 0.80 & 11.9 & 79.6 \\
\hline
FastViT-S12\cite{vasu2023fastvit} &Hybrid & 256 & 8.8  & 1.8 & 2313 & 1.13 & 11.8 & 79.8 \\
PoolFormer-S24\cite{yu2022metaformer} &Hybrid & 224 & 21.4 & 3.4 & 1424 & 1.68 & 18.3 & 80.3 \\
MobileViTV2-1.5 \cite{mehta2023separable} &Hybrid & 256 & 10.6 & 4.2 & 1116 & 2.95 & 20.9 & 80.4 \\
EfficientFormerV2-S2\cite{li2023rethinking} &Hybrid & 224 & 12.6 & 1.3 & 611  & 1.23 & 21.5 & 80.4 \\
\rowcolor{gray!30}
\textbf{RAViT-S26}                &Hybrid & 224 & 11.5 & 1.4 & 3079 & 0.83 & 13.0 & 80.2 \\
\rowcolor{gray!30}
\textbf{RAViT-SA22}               &Hybrid & 224 & 10.9 & 1.3 & 2878 & 0.94 & 17.3 & 80.4 \\

\hline
FastViT-SA12\cite{vasu2023fastvit}     &Hybrid & 256 & 10.9 & 1.9 & 2181 & 1.22 & 12.8 & 80.6 \\
RepViT-M1.5 \cite{wang2024repvit}      &Conv   & 224 & 14.0 & 2.3 & 2151 & 1.04 & 19.4 & 81.2 \\
Swin-T \cite{liu2021swin}              &ViT    & 224 & 29.0 & 4.5 & 886  & 6.51 & 24.5 & 81.3 \\
PoolFormer-S36\cite{yu2022metaformer} &Hybrid & 224 & 30.9 & 5.0 & 967 & 2.35 & 26.8 & 81.4 \\
% \rowcolor{gray!30}
% \textbf{RAViT-M22}              &Hybrid & 224 & 16.2 & 2.1 & 2470 & 1.09 & 12.7 & 81.2 \\
\rowcolor{gray!30}
\textbf{RAViT-M26}                &Hybrid & 224 & 18.5 & 2.4 & 2193 & 1.17 & 14.4 & 81.4 \\
\hline
% FastViT-SA24\cite{vasu2023fastvit} &Hybrid & 256 & 20.6 & 3.8 & 1128 & 1.17 & 22.0 & 82.6 \\
% RAViT-LA                           &Hybrid & 224 &      &     &      &     &      &  \\
\end{tabular}
\label{tab:imagenet}
\end{table*}

\begin{table*}[!ht]
\centering
\caption{Instance segmentation test of RAViT as a backbone on COCO val2017 with Mask R-CNN. $AP^b$ and $AP^m$ denote bounding box average precision and mask average precision, respectively. The FLOPs and latency are measured at resolution 1280 $\times$ 800.
}
\begin{tabular}{ m{2.8cm}|>{\centering}m{0.8cm}|>{\centering}m{0.8cm}|>{\centering}m{0.8cm}|>{\centering}m{0.8cm}|>{\centering}m{0.8cm}|>{\centering}m{0.8cm}|>{\centering}m{0.8cm}|>{\centering}m{0.8cm}|>{\centering}m{0.8cm}|c} \hline
\multirow{2}{*}{Backbone} & \multirow{2}{*}{$AP^{b}$}  & \multirow{2}{*}{$AP^{b}_{50}$} & \multirow{2}{*}{$AP^{b}_{75}$} & \multirow{2}{*}{$AP^{m}$}  & \multirow{2}{*}{$AP^{m}_{50}$} & \multirow{2}{*}{$AP^{m}_{75}$} & Param & FLOPs & GPU & EDGE \\
    &  &  &  &  &  &  & (M) & (G) & (img/s) & (ms)   \\ \hline

PVT-T\cite{wang2021pyramid} & 36.7 & 59.2 & 39.3 & 35.1 & 56.7 & 37.3 & 32.8 & 239.8  & 24.7 &  517 \\ 
% PVTv2-B1\cite{wang2022pvt} & 41.8 & 64.3  & 45.9 &  38.8     & 61.2  & 41.6 & 33.7 & 243.7G   \\ 
PoolFormer-S12\cite{yu2022metaformer} & 37.3 & 59.0 & 40.1 & 34.6 & 55.8 & 36.9 & 31.6  & 207.3  & 20.0 & 353 \\
ResNet-50\cite{he2016deep}   & 38.0 & 58.6 & 41.4 & 34.4 & 55.1 & 36.7 & 44.2 & 260.1 & 31.5 & 333 \\
FastViT-SA12\cite{vasu2023fastvit} & 38.9 & 60.5 & 42.2 & 35.9 & 57.6 & 38.1 & 30.5 & 200.4 & 37.5 & 413 \\
RepViT-M1.1\cite{wang2024repvit} & 39.8 & 61.9 & 43.5 & 37.2 & 58.8 & 40.1 & 27.9 & 197.6 & 42.2 & 370 \\ 
\rowcolor{gray!30}
\textbf{REViT-S26}           & 40.4 & 62.5 & 44.2 & 37.8 & 59.8 & 40.2 & 29.6 & 198.8 & 40.0 & 284 \\ \hline
PoolFormer-S24\cite{yu2022metaformer} & 40.1 & 62.2 & 43.4 & 37.0 & 59.1 & 39.6 & 41.0 & 213.8 & 12.2 & 558 \\ 
PVT-S\cite{wang2021pyramid}  & 40.4 & 62.9 & 43.8 & 37.8 & 60.1 & 40.3 & 44.1 & 304.5 & 16.9 & 731 \\ 
RepViT-M1.1\cite{wang2024repvit} & 41.6 & 63.2 & 45.3 & 38.6 & 60.5 & 41.4 & 33.8 & 217.1 & 35.8 & 427 \\ 
\rowcolor{gray!30}
\textbf{REViT-M26}           & 41.6 & 64.0 & 45.3 & 38.9 & 60.9 & 41.9 & 34.4 & 212.3 & 36.7 & 334 \\ \hline

\end{tabular}
\label{tab:obj}
\end{table*}
\section{Result}
To evaluate the effectiveness of the RAViT backbone and Fast-COS, we perform several tests. The ImageNet-1K dataset with 1000 categories is selected as the image classification benchmarking for the backbone test. We also perform a backbone evaluation test for instance segmentation on the COCO dataset. In a specific task, we advance Fast-COS with RepFPN to perform driving scene object detection on the BDD100K and TJU-DHD-traffic datasets. Since the FLOPs do not directly affect the computational complexity, we compare RAViT and Fast-COS with other models in the state-of-the-art using throughput and latency metrics that were tested on three different wide-range application processing devices, including the GPU RTX3090, the Neural Processing Unit (NPU) on the iPhone 15 Pro as a mobile processing unit, and the Jetson Orin Nano as the edge device processing unit. This extensive benchmark will represent real-time performance on diverse hardware platforms. 
\subsection{Evaluation Result of RAViT Backbone}
\subsubsection{Setup}
We evaluate RAViT using ImageNet-1K \cite{deng2009imagenet} as the most popular image classification benchmarking dataset. ImageNet-1K has 1000 categories with 1.2 million images for training and 50000 images for validation. We follow a training recipe in \cite{liu2021swin} with a total of 300 epochs in each RAViT model variant, with a resolution of 224$\times$224. Data augmentation and regularization approaches encompass several methods such as RandAugment \cite{cubuk2020randaugment}, Mixup, CutMix \cite{yun2019cutmix}, Random Erasing \cite{zhong2020random}, weight decay, label smoothing, and Stochastic Depth. We employ the AdamW optimizer with a 0.004 base learning rate and a total batch size of 2046 on 4$\times$A6000 GPUs for most RAViT models.

We also conducted an experiment on the COCO dataset, a benchmark widely used for object detection and instance segmentation using Mask R-CNN. The backbone of Mask R-CNN was replaced with RAViT, which leverages multi-scale convolution and self-attention to improve feature extraction. The COCO dataset's training split was used for model training, while the validation split was used for evaluation, adhering to the standard COCO metrics, including mean Average Precision (mAP) for both bounding box detection and segmentation masks.

We assess performance through inference latency tests conducted on two types of constrained-resource device hardware and a single desktop GPU. For mobile device performance assessment, the iPhone 15 Pro is utilized. All models are transformed into the CoreML format, and each undergoes 50 $\times$ inference loops following a prior 20-loop warm-up. The mean inference duration serves as the evaluation metric. The Jetson Orin Nano is chosen to represent Edge device hardware for evaluation purposes. Models are adapted to the ONNX format for latency measurement on Edge devices. In Edge device assessment, a warm-up duration of 20 seconds precedes 1000 $\times$ inference loops.

\subsubsection{Benchmarking on ImageNet-1K}
The comparative analysis presented in Table \ref{tab:imagenet} highlights the performance of RAViT variants against state-of-the-art models on the ImageNet-1K dataset. We evaluate across various hardware platforms including GPU, Mobile NPU, and Edge devices to give a wide application illustration as a backbone. The RAViT models demonstrate competitive accuracy and computational efficiency trade-offs when compared to recent state-of-the-art models. For instance, RAViT-M26 reaches 81.4\% Top-1 accuracy while achieving $2.27\times$ faster GPU throughput, $2\times$ faster NPU latency, and $1.8\times$ faster Edge device latency compared to PoolFormer-S36 and Swin-T. RAViT-M26 achieves higher 0.2\% Top-1 accuracy while maintaining similar NPU and Edge device latency compared to the recent mobile vision transformer such as RepViT, showcasing its architectural efficiency. Similarly, RAViT-M26 achieves higher 0.8\% Top-1 accuracy compared to FastViT-SA12, while maintaining 4\% faster mobile NPU inference speed. 

The RAViT models consistently achieve high accuracy across different configurations. Although MobileOne also uses the reparameterization technique, the hybrid transformer architecture of RAViT indicates a trade-off between accuracy and speed. For example, RAViT-S22 can outperform MobileOne-S4 with $1.8\times$ faster GPU throughput, $1.5\times$ faster Mobile NPU latency, and $1.6\times$ faster Edge device latency. RAViT models achieve faster GPU, Mobile NPU, Edge device inference while providing superior accuracy, showcasing their versatility and effectiveness for real-world deployment scenarios.
\subsection{Benchmarking with SOTA Models in COCO Instance Segmentation}
Table III presents the evaluation of the proposed RAViT backbones on the COCO val2017 dataset with Mask R-CNN, compared to other state-of-the-art methods. The evaluation metrics include bounding box average precision $(AP_b)$, mask average precision $(AP_m)$, computational complexity (FLOPs), parameter size, and latency on GPU and EDGE devices. The results highlight the ability of RAViT to achieve a favorable balance between segmentation accuracy and computational efficiency.

The RAViT backbones demonstrate competitive or superior performance in both bounding box and mask precision. RAViT-S26 achieves an $AP_b$ of 40.4 and an $AP_m$ of 37.2, outperforming PVT-S and PoolFormer-S24. RAViT-M26 achieves the highest scores among all methods, with 41.6\% of $AP_b$   and 38.9\% of $AP_m$, surpassing RepViT-M1.1 while only using the convolution mixer in accuracy, while maintaining comparable efficiency. These results validate the effectiveness of the RAViT architecture in improving instance segmentation performance.

Latency and inference speed further highlight the efficiency of the RAViT architecture. RAViT-S26 achieves the highest GPU inference speed of 40.0 images per second, significantly outperforming PVT-S $2.4\times$ and 8.25\% faster than RepViT-M1.1. On EDGE devices, RAViT-S26 achieves the lowest latency at 284 ms, making it highly suitable for real-time applications. RAViT-M26 also delivers competitive EDGE latency at 334 ms, outperforming several other methods in the comparison.

Compared to PoolFormer, PVT, and RepViT, the proposed RAViT architecture strikes an excellent balance between accuracy and efficiency. RAViT-M26 achieves the highest accuracy across the evaluated models, while RAViT-S26 stands out for its minimal latency and high inference speed. These results demonstrate the scalability of RAViT, offering lightweight and high-performance variants that cater to diverse deployment scenarios. In conclusion, the proposed RAViT backbones are well-suited for both performance-driven and latency-critical applications, solidifying their value in instance segmentation tasks.

\begin{table*}[!ht]
\centering
\caption{Object detection on BDD100K and TJU-DHD-Traffic dataset with  FCOS-RAViT. $AP^b$ denote bounding box average precision. The FLOPs, GPU and EDGE inference throughput are measured at resolution 1280 $\times$ 720.
}
\begin{tabular}{ c|m{1.9cm}|m{1.9cm}|>{\centering}m{0.7cm}|>{\centering}m{0.8cm}|>{\centering}m{0.8cm}|>{\centering}m{0.7cm}|>{\centering}m{0.7cm}|>{\centering}m{0.7cm}|>{\centering}m{0.7cm}|>{\centering}m{0.8cm}|>{\centering}m{0.8cm}|c} \hline
\multirow{2}{*}{Dataset}&\multirow{2}{*}{Network} & \multirow{2}{*}{Backbone} & \multirow{2}{*}{$AP$} & \multirow{2}{*}{$AP_{50}$}& \multirow{2}{*}{$AP_{75}$} & \multirow{2}{*}{$AP_{s}$}  & \multirow{2}{*}{$AP_{m}$} & \multirow{2}{*}{$AP_{l}$} & Param & FLOPs & GPU & EDGE \\
           & &  & & &  &  &  &  & (M) & (G) & (img/s) & (img/s)   \\ \hline
\multirow{6}{*}{BDD100K}&YOLOF\cite{mboutayeb2024fcosh}&ResNet-50\cite{he2016deep}& 24.5 & 45.1 & 22.9 & 6.6 & 30.3 & 51.0 & 42.5 & 94.5& 60.7 & 9.9  \\ 
& FCOS\cite{mboutayeb2024fcosh}&ResNet-50\cite{he2016deep}
                        &29.0 &52.9 &26.8 &12.3 &34.7 &50.6 &31.9 &181.6 &39.3 & 6.8  \\
&FCOS\cite{mboutayeb2024fcosh}&ResNet-101\cite{he2016deep} 
                        &30.0 &54.2 &27.9 &12.6 &35.8 &52.2 &50.8 &251.3 &28.2 &-\\ \cline{2-13}
& \multirow{2}{*}{FCOS\cite{9010746}}&RAViT-S26 &30.2 &54.6 &28.3 &12.7 &35.2 &52.3 &17.4 &132.6 &51.8 & 8.6 \\
                    &  &RAViT-M26 &30.5 &55.3 &28.5 &13.2 &35.1 &53.2 &24.6 &152.8 &45.7 & 7.6 \\ \cline{2-13}
&\multirow{2}{*}{Fast-COS} &RAViT-S26 &31.1 &56.6 &29.3 &14.0 &36.5 &51.7 &14.7 &120.9 &57.0 & 9.3 \\
                    &  & RAViT-M26 &31.8 &57.2 &30.1 &14.4 &37.2 &53.6 &21.9 &141.2 &49.6 & 8.1 \\
\hline
\multirow{6}{*}{TJU-DHD}&$^\dagger$RetinaNet\cite{ross2017focal}&ResNet-50\cite{he2016deep} & 53.5 & 80.9 & 60.0 & 24.0 & 50.5 & 68.0 & 36.4 & 216.1 & 33.7 & 6.9 \\ 
&$^\dagger$FCOS\cite{9010746}&ResNet-50\cite{he2016deep}  
                        &53.8 &80.0 &60.1 &24.6 &50.6 &50.6 &32.1 &211.6 &33.0 &  6.9  \\ \cline{2-13}
&\multirow{2}{*}{$^\dagger$FCOS\cite{9010746}}& RAViT-S26 &53.9 &79.3 &61.0 &26.2 &50.4 &69.1 &17.4 &143.9 &48.5 &8.6 \\
                      & & RAViT-M26 &54.5 &80.0 &61.4 &27.2 &50.7 &70.0 &24.6 &165.9 &47.7 &7.6 \\
\cline{2-13} 
&\multirow{2}{*}{Fast-COS} &RAViT-S26 &53.4 &79.2 &60.3 &26.1 &49.7 &68.3 &17.4 &120.9 &56.8 & 9.3 \\
                    & & RAViT-M26 &54.5 &80.0 &61.4 &26.9 &50.5 &69.4 &21.9 &141.2 &49.4 &8.1 \\
                      
\hline
\end{tabular}
\label{tab:driver-scene}
\end{table*}

\subsection{Evaluation Result of Fast-COS on driving scene Object Detection Task}
\subsubsection{Setup} The proposed model was tested on two high-resolution, large-scale datasets: BDD100K and TJU-DHD-traffic, both suited for assessing detection networks from a driver's viewpoint. These datasets include diverse scenes such as urban streets and residential areas, with 1.84 million and 239,980 annotated bounding boxes, respectively. BDD100K features 10 categories, including Bus and Car, while TJU-DHD covers 5 categories like Pedestrian and Cyclist, available under different lighting conditions. Both datasets offer scenarios in various weather conditions, providing a valuable resource for real-world model testing. Organization comprises 45,266 training images and 5,000 validation images for TJU-DHD Traffic, and 70,000 training images and 10,000 validation images for BDD100K.

In the driving scene object detection experiment, we use an input size of 1280 × 720 pixels in both training and evaluation. Training utilizes 4 NVIDIA GPUs, each handling a mini-batch of 8 images. The AdamW optimizer governs training, starting with a learning rate of 0.0001, which is reduced by a factor of 10 at the 8th and 11th epochs. The model undergoes end-to-end training with RAViT, initialized from pre-trained weights for efficient learning. Data augmentation, including random flipping and resizing, is applied for robustness. During inference, the top 100 detection bounding boxes per image are recorded for performance assessment.
\subsubsection{Benchmarking on BDD100K and TJU-DHD Traffic}
A comparative study illustrated in Table \ref{tab:driver-scene} showcases the performance of Fast-COS utilizing RAViT backbone variants, in comparison with original FCOS models, evaluated on the BDD100K and TJU-DHD Traffic datasets. The findings demonstrate that the RAViT backbone can enhance the inference speed of FCOS by 62\% over the original FCOS-ResNet-101, when juxtaposed with FCOS-RAViT-M26. Additionally, the RAViT backbone contributes to an increase in prediction accuracy, indicated by a 2\% improvement in $AP_{50}$ in the BDD100K dataset test. By integrating the RAViT-M26 model with the RepFPN, which includes the use of RepMSDW, the Fast-COS can achieve $AP_{50}$ accuracy 5.5\% higher along with a 75.9\% improvement in GPU inference speed when compared to the FCOS-ResNet-101. 

In the TJU-DHD Traffic dataset, employing the RAViT-M26 as a backbone enhances the overall $AP$ by 1.3\%, with a 38\% enhancement in predicting large objects ($AP_l$). Utilizing the same input size configuration ($1333\times800$), the proposed RAViT backbone also increases detection speed by 44.5\% compared to the original FCOS. To achieve a detection speed comparable to the BDD100K test, the Fast-COS model variant was evaluated with an input size of $1280\times720$ pixels. In this configuration, Fast-COS boosts detection speed by 49.6\% relative to the original FCOS, while maintaining similar prediction accuracy to the FCOS-RAViT variant trained with a larger input size.

In the evaluation conducted on the Edge device utilizing Jetson Orin Nano with the ONNX framework, the peak performance of the Fast-COS variant is observed when employing RAViT-S26, achieving a prediction speed of 9.3 FPS across both driving scene dataset tests. Despite being 6.5\% slower than YOLOF-ResNet50, Fast-COS with RAViT-S26 attains a 26.9\% improvement in overall average precision $AP$ in BDD100K test. The comprehensive hardware assessment indicates that while several models can exceed 30 FPS for GPU inference speed, optimization on the EDGE device is essential to attain at least 15 FPS for it to be viable as a real-time driving scene object detection hardware option.

\begin{figure*}
\centering
         \includegraphics[width=5.8cm]{picture/obj_det_result/fcos_1.jpg}
         \vspace{0.25mm}
         \includegraphics[width=5.8cm]{picture/obj_det_result/fast-cos_1.jpg}
         \vspace{0.25mm}
         \includegraphics[width=5.8cm]{picture/obj_det_result/gt_1.jpg}
         \vspace{0.25mm}
         \includegraphics[width=5.8cm]{picture/obj_det_result/fcos_fe1d9184-d144106a.jpg}
         \vspace{0.25mm}
         \includegraphics[width=5.8cm]{picture/obj_det_result/fast-cos_fe1d9184-d144106a.jpg}
         \vspace{0.25mm}
         \includegraphics[width=5.8cm]{picture/obj_det_result/fe1d9184-d144106a.jpg}
         \vspace{0.25mm}
         \includegraphics[width=5.8cm]{picture/obj_det_result/fcos_b1ca2e5d.jpg}
         \vspace{0.25mm}
         \includegraphics[width=5.8cm]{picture/obj_det_result/fast-cos_b1ca2e5d.jpg}
         \vspace{0.25mm}
         \includegraphics[width=5.8cm]{picture/obj_det_result/b1ca2e5d.jpg}
         \vspace{0.5mm}
         \includegraphics[width=5.8cm]{picture/obj_det_result/fcos_2.jpg}
         \includegraphics[width=5.8cm]{picture/obj_det_result/fast-cos_2.jpg}
         \includegraphics[width=5.8cm]{picture/obj_det_result/gt_2.jpg}
         \\(1)\hspace{5.8cm}(2)\hspace{5.8cm}(3)
\caption{The detection results using (1). FCOS+ResNet-50 and (2). Fast-COS+RAViT-M26 compare to (3). ground truth}
\label{fig:qualitativeobj}
\end{figure*}

\subsection{Ablation Experiment}
\subsubsection{Multi-Scale in RepMSDW and Combination with Self-Attention}
We conducted an ablation study on Multi-Scale convolution kernel sizes and compared them against sole square kernel reparameterization and configurations without kernel reparameterization. This analysis was specifically performed using the RAViT-S22 variant. As depicted in Table \ref{tab:ablation_token_mixer}, adopting Multi-Scale kernel sizes during reparameterization results in enhanced classification accuracy on the ImageNet-1K dataset. Utilizing multi-scale reparameterization offers a 0.12\% improvement over single square reparameterization, as referenced in \cite{wang2024repvit, vasu2023fastvit, ding2021repvgg}, and a 0.22\% gain over configurations without reparameterization. 

Given that RepMSDW is constrained in capturing long dependencies, we evaluated its integration with self-attention, a technique employed in Transformer models \cite{li2022efficientformer, liu2023efficientvit, liu2021swin}. Initially, we incorporated RepMSDW in Multi-Head Self-Attention (MHSA) to substitute the $3\times3$ DWConv, as described in \cite{li2023rethinking}, with a $7\times7$ RepMSDW. This modification increased accuracy to 79.2\%. However, MHSA requires numerous array transformations, impacting GPU throughput and NPU latency. Subsequently, we adopted single-head attention (SA), as proposed in \cite{yun2024shvit}, to address the computational redundancies in MHSA. Combining RepMSDW with SA elevated accuracy to 79.6\% while maintaining efficiency in both GPU throughput and NPU latency.
\begin{table}[!ht]
\centering
\caption{Ablation of RepMSDW and RepSA on ImageNet-1K with 224$\times$224 image size. GPU denotes the throughput in image/s and NPU denotes the latency in iPhone 15 Pro.}
\begin{tabular}{ m{1.2cm}|>{\centering}m{1.4cm}|>{\centering}m{0.5cm}|>{\centering}m{0.8cm}|>{\centering}m{0.6cm}|>{\centering}m{0.6cm}|c } \hline
\multirow{2}{*}{Ablation} & \multirow{2}{*}{Variant} & Par & FLOPs & GPU   & NPU & \multirow{2}{*}{Top-1}  \\ 
                         &                           & (M) &  (G)  & img/s & ms  &        \\ \hline
\multirow{2}{*}{RepMSDW}& Multi-Scale & 9.3 & 1.12 & 3632 & 0.77 & 79.12 \\ \cline{2-7}
                        & Square      & 9.3 & 1.12 & 3632 & 0.77 & 79.0 \\ \cline{2-7}
                        & w/o Rep     & 9.3 & 1.12 & 3632 & 0.77 & 78.9\\ \hline
\multirow{3}{*}{RepSA} & RepMSDW$+$ MHSA & 12.7 & 1.31 & 3255 & 0.99 & 79.2  \\ \cline{2-7}
                       & RepMSDW$+$ SA  & 10.2 & 1.17 & 3491 & 0.80 & 79.6  \\ %\cline{2-6}
\hline

\end{tabular}
\label{tab:ablation_token_mixer}
\end{table}
\subsubsection{Combination in RAViT Backbone architecture}
Following the ablation of RepMSDW and RepSA, we also conducted an ablation study on the macro architecture. The investigation commenced with a comparison between the 3-stage (V1) and the 4-stage (V2) architectures. The findings from this analysis are detailed in Section \ref{subsec:abl-macro}, where V2 was adopted as the baseline configuration. Subsequently, we experimented with increasing the RepMSDW kernel size in stages three and four from $K=3$ to $K=7$ (V3), resulting in a 0.5\% enhancement in Top-1 accuracy, accompanied by a reduction in inference speed of 1\%, 2.5\%, and 10\% for NPU, EDGE, and GPU respectively. Incorporating RepSA in stage four (V4) yields a 0.9\% improvement in accuracy, but inference speed decreases by 3.8\%, 19\%, and 22\% on NPU, EDGE, and GPU, respectively. In the final ablation, we applied RepSA in both stages three and four, achieving a 1.7\% boost in accuracy, though inference speed slowed by 22.1\%, 44.9\%, and 78.6\% on NPU, GPU, and EDGE respectively. Ultimately, V4 was selected as the primary configuration for the RAViT backbone to balance between speed and accuracy.
\begin{table}[!ht]
\centering
\caption{Ablation of RAViT on ImageNet-1K with 224$\times$224 image size. M and A denote RepMSDW and  RepSA then followed by kernel size and number of blocks. C denotes a convolution followed by kernel size, stride, and number of blocks.}
\begin{tabular}{c|c|>{\centering}m{1.0cm}|>{\centering}m{0.9cm}|>{\centering}m{0.9cm}|>{\centering}m{0.9cm}|c} \hline
\multicolumn{2}{c|}{Config} & V1      & V2      & V3      & V4 & V5 \\ \hline
\multicolumn{2}{c|}{Stem} & C3S2$\times4$      & \multicolumn{4}{c}{C3S2$\times2$}\\ \hline
\multirow{4}{*}{Stage}  & 1  & M3$\times4$ & M3$\times2$ & M3$\times2$ & M3$\times2$ & M3 $\times2$  \\
                        & 2  & M3$\times16$ & M3$\times4$ & M3$\times4$ & M3$\times4$ & M3 $\times4$  \\
                        & 3  & M3$\times4$ & M3$\times12$ & M7$\times12$ & M7$\times12$& A7$\times12$ \\
                        & 4  & - & M3$\times4$ & M7 $\times 4$ & A7 $\times4$  & A7 $\times4$  \\ \hline
\multicolumn{2}{l|}{Param (M)}     & 10.0  & 9.18    & 9.3     & 10.2    & 10.9 \\ \hline
\multicolumn{2}{l|}{FLOPs (G)}     & 0.32  & 1.10    & 1.12    & 1.17    & 1.33  \\ \hline
\multicolumn{2}{l|}{GPU (img/s)}   & 13484 & 4059   & 3632    & 3316    & 2801 \\ \hline
\multicolumn{2}{l|}{NPU (ms)}      & 0.73  & 0.77    & 0.79    & 0.80    & 0.94 \\ \hline
\multicolumn{2}{l|}{EDGE (ms)}     & 10.0  & 9.8     & 9.9     & 11.7    & 17.5    \\ \hline
\multicolumn{2}{l|}{Top-1}         & 75.8  & 78.7    & 79.1    & 79.6    & 80.4 \\ \hline

\end{tabular}
\label{tab:ablation_macro}
\end{table}
\subsubsection{Fast-COS ablation}
Table \ref{tab:driver-scene} presents the ablation study of Fast-COS utilizing the RAViT variant as the backbone and RepFPN as the intermediary component of the original FCOS head detector. When compared to the standard FCOS employing ResNet-101 on the BDD100K dataset, incorporating RAViT-M26 enhances the detection accuracy for small $(AP_s)$ and large $(AP_l)$ objects by 4.8\% and 1.9\%, respectively. The system employing a reparameterizable RAViT backbone, particularly through the RepMSDW residual connection reparameterization, boosts the GPU inference speed by 62.1\%. By integrating RepMSDW into the FPN, notable improvements in $(AP_l)$, $(AP_m)$, and $(AP_s)$ are recorded at 2.7\%, 3.9\%, and 14.2\%, respectively. Moreover, the RepFPN version, functioning with merely three levels of feature extraction, enhances the GPU inference speed by 75.9\%. 

\subsection{Visualization Results}
Fig. \ref{fig:qualitativeobj} illustrates the comparison of detection results achieved by Fast-COS and the baseline algorithms. The images originate from the BDD validation set, which encompasses urban and suburban traffic scenes as well as traffic scenes in adverse weather conditions. Observing the detection results, it is evident that Fast-COS, through reparameterized multi-scale kernel size convolutions and reparameterized self-attention mechanisms, effectively resolves occlusion issues (such as occlusion between vehicles) and enhances detection performance relative to the baseline. This is particularly noticeable when detecting small objects like a signal light, traffic sign, or distant pedestrian. 

Through an expanded explanation, it is demonstrated within the first and second group of examples that, in urban traffic scenarios during daylight conditions, Fast-COS exhibits the ability to identify several traffic signs located at a considerable distance, in addition to an obstructed bus and automobile positioned in the background. In contrast, the baseline method is inadequate in its predictive capabilities. Moreover, in the third and fourth groups, Fast-COS effectively discerns a diminutive and partially hidden vehicle situated between two larger entities, even under nighttime conditions.




\section{Conclusion}

In this paper, we introduce \DatasetName, a novel large-scale dataset specifically designed for long-text rendering, addressing the existing gap in datasets capable of supporting such tasks. 
To demonstrate the utility of models in handling long-text generation, we create a dedicated test set and evaluate current state-of-the-art text-to-image generation models.
Additionally, the open availability of a large-scale, diverse, and high-quality long-text rendering dataset like \DatasetName is crucial for advancing the training of text-conditioned image generation models.

There are several promising directions for further enhancing \DatasetName, which we have not explored in this paper due to the increased computational costs these approaches entail: \emph{i}. Multiple rounds of dataset bootstrapping to iteratively improve data quality. \emph{ii}. Generating multiple synthetic captions per image to further expand the dataset corpus.

\bibliography{egbib}
\bibliographystyle{ieeetr}

% \begin{IEEEbiography}[{\includegraphics[width=1in,height=1.25in,clip,keepaspectratio]{biography/Novendra.JPG}}]{Novendra Setyawan}{\space}received the Bachelor Degree in Electrical Engineering from the University of Muhammadiyah Malang, Indonesia, in 2015. The Master Degree in Electronic Engineering from the Tenth of November Institute of Technology, Surabaya, Indonesia, in 2017, and currently studying for a Ph.D. degree in the Electro-Optics Department at the National Formosa University in Yunlin, Taiwan. Also a lecturer in the Department of Electrical Engineering, University of Muhammadiyah Malang, Indonesia. His research interests are in artificial intelligence, Edge Computing Devices, Robotics, intelligent control, and image processing.
% \end{IEEEbiography}
% \vskip -2\baselineskip plus -1fil
% \begin{IEEEbiography}
% [{\includegraphics[width=1in,height=1.25in,clip,keepaspectratio]{biography/Ghufron.jpeg}}]
% {Ghufron Wahyu Kurniawan}{\space}received the Bachelor Degree in Electrical Engineering from the University of Muhammadiyah Malang, Indonesia, in 2021. Currently, he is a master student in the Department of Electrical Engineering at the National Formosa University in Yunlin, Taiwan. His research interests are in artificial intelligence, robotics, and image processing.
% \end{IEEEbiography}
% \vskip -2\baselineskip plus -1fil
% \begin{IEEEbiography}[{\includegraphics[width=1in,height=1.25in,clip,keepaspectratio]{biography/Chi-Chia Sun.jpg}}]{Chi-Chia Sun}{\space}(Member, IEEE) received the B.S. degree in computer science and engineering from National Taiwan Ocean University, Taiwan, in 2004, the M.S. degree in electronic engineering from the National Taiwan University of Science and Technology, Taipei, Taiwan, in 2006, and the Doktor Ingenieur degree with Federal Republic of Germany DAAD full scholarship from the Dortmund University of Technology, Dortmund, Germany, in 2011, where he worked as a Research Assistant from April 2008 to March 2011. He was a Principle Engineer at TSMC speciality in standard cell design. He is currently a Full Professor with the Department of Electrical Engineering, National Taipei University, Taipei, Taiwan. His research interests are in image processing, system integration, and VLSI design with multiple project granting from NSTC and MOE.
% \end{IEEEbiography}
% \vskip -2\baselineskip plus -1fil
% \begin{IEEEbiography}[{\includegraphics[width=1in,height=1.25in,clip,keepaspectratio]{biography/wkkuo.png}}]{Wen-Kai Kuo}{\space} received a PhD in electronic engineering from the National Chiao Tung University, Hsin-Chu, Taiwan, in 2000. He has been a professor at the Department of Electro-optics Engineering at National Formosa University, Huwei, Yunlin, Taiwan. He is a member of the Phi-Tau-Phi Honorary Scholar Society. His research interests are optical sensors and systems.
% \end{IEEEbiography}
% \vskip -2\baselineskip plus -1fil
% \begin{IEEEbiography}[{\includegraphics[width=1in,height=1.25in, clip,keepaspectratio]{biography/Jun-Wei Hsieh.png}}]{Jun-Wei Hsieh}{\space}(Senior Member, IEEE) received the Ph.D. degree in computer engineering from the National Central University, Chung-Li, Taiwan, in 1995. He was an Associate Professor with the Department of Electrical Engineering, YuanZe University, and a Visiting Researcher with the MIT AI Laboratory. Since August 2009, he had been a Professor and the Dean of the Department of Computer Engineering, National Taiwan Ocean University. After August 2019, he has been a Professor with the College of AI, National Yang-Ming Chiao-Tung University. He hosted or co-hosted a lot of large-scale AI projects from different companies and governments in the past. He has a lot of successful experiences in industrial-academic cooperation and technology transferring, especially in ITS. His research interests include AI, deep learning, smart farming, video surveillance, intelligent transportation systems, image and video processing, object recognition, machine learning, 3D printing, medical image analysis, and computer vision. In May 2019, he received the First Prize of the Ministry of Science and Technology Best Display Award and the Third Place of the AI Investment Potential Award. Due to his contributions in traffic flow estimation, he helped the Elan company received the Gold Award from Taipei International Computer Show, in 2019. He and his students received the Silver Medal of 2019 National College Software Creation Competition, the Silver Medal of 2018 National Microcomputer Competition, the Best Paper awards of Information Technology and Applications in Outlying Islands Conference, in 2013, 2014, 2016, 2017, 2018, 2021, and 2022, respectively, the Best Paper Award of Tanet 2017, the Best Paper Awards of NCWIA 2020, 2021, and 2022, respectively, and the Best Paper Awards of IS3C 2020.
%  He also received the Best Paper Award of CVGIP Conference, in 1999, 2003, 2005, 2007, 2014, 2017-2018, and 2022-2024, the Best Paper Award of DMS Conference, in 2011, the Best Paper Award of IIHMSP 2010, and the Best Patent Award of Institute of Industrial Technology Research, in 2009 and 2010, respectively. Dr. Hsieh serves as Program Chair of Conference on Multimedia Modeling 2011, and Program Chair of the IEEE Advanced Video and Signal-based Surveillance (AVSS) 2019. He has authored more than 150 peer-reviewed journal and conference publications, and 20 US/Taiwan patents.
% \end{IEEEbiography}
% \break 
% \begin{IEEEbiographynophoto}{\relax}\unskip
% \end{IEEEbiographynophoto}

\ifCLASSOPTIONcaptionsoff
  \newpage
\fi

\end{document}
