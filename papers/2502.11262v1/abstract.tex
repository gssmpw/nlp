%% The abstract is a short summary of the work to be presented in the
%% article.
\begin{abstract}
Preparing high-quality datasets required 
by various data-driven AI and machine learning 
models has become a cornerstone task in 
data-driven analysis. 
Conventional data discovery methods typically 
integrate datasets towards a single 
pre-defined quality measure 
that may lead to bias for 
downstream tasks.   
This paper introduces \textbf{MODis}, a 
framework that %can %automatically 
discovers %and recommend datasets
datasets %from multiple source 
%datasets %to improve the
%expected performances of 
%for an input model  
by optimizing {\em multiple} user-defined, 
model-performance measures. 
Given a set of data sources and a model, 
 \textbf{MODis} 
%consults 
%a model performance 
%estimator to dynamically 
selects and integrates data sources  
into a skyline dataset, 
over which 
the model is expected to have the
desired performance in all the 
performance measures. 
We formulate \textbf{MODis} as 
a multi-goal finite 
state transducer,   
%in terms of 
%by interleaving both data augmentation and reduction 
%operators 
%Pareto optimality. 
\eat{
in terms of 
finite state transducer with 
finer grained operators enhanced by 
selection conditions. }
%analyze its expressiveness, 
%and establish the hardness 
%and 
and derive three feasible algorithms 
to generate skyline datasets. Our first 
algorithm adopts a ``reduce-from-universal'' strategy, 
that starts with a universal schema and 
iteratively prunes unpromising data. 
Our second algorithm further reduces 
the cost with a bi-directional strategy 
that interleaves 
data augmentation and 
reduction. We also introduce 
a diversification algorithm to mitigate 
the bias in skyline datasets. 
We experimentally verify the 
efficiency and effectiveness of 
our skyline data discovery algorithms, and showcase 
their applications in optimizing %accelerating 
data science pipelines.   
\end{abstract}

% 12.27. Wu note: for a 9 pager plan: %%%%%
% main setted as sigir24: main.tex; as I'm cleaning an outline for SIGIR. 
% if you want to work on full.tex: compile by selecting full.tex. 
% abstract updated.  
% page 1 - 1.75 Sec 1 Introduction
%  page 1.75 - 2.5 Sec 2: preliminaris: models, performance evaluation 
%  page 2.5 - 3: problem formulation; hardness. Leave formulations of transducers in full version. 
%  page 3 - 4: framework and algorithm framework;
%  page  4 - 5.5: forward algorithm; FPTAS results; 
%  page 5.5 - 6.5: bi-directional search. 
%  page 6.5 - 7: optimization techs. 
%  page 7 - 8.5: experimental study. 
%  page 8.5 - 9: conclusion. 
%  motivate algorithm with goal-driven AI search: 
% https://www.uobabylon.edu.iq/eprints/publication_1_2422_213.pdf
%%%%%%%%%%%%%%%%%%%%%%%%

%%
%% The code below is generated by the tool at http://dl.acm.org/ccs.cfm.
%% Please copy and paste the code instead of the example below.
% \begin{CCSXML}
% <ccs2012>
%  <concept>
%   <concept_id>10010520.10010553.10010562</concept_id>
%   <concept_desc>Computer systems organization~Embedded systems</concept_desc>
%   <concept_significance>500</concept_significance>
%  </concept>
%  <concept>
%   <concept_id>10010520.10010575.10010755</concept_id>
%   <concept_desc>Computer systems organization~Redundancy</concept_desc>
%   <concept_significance>300</concept_significance>
%  </concept>
%  <concept>
%   <concept_id>10010520.10010553.10010554</concept_id>
%   <concept_desc>Computer systems organization~Robotics</concept_desc>
%   <concept_significance>100</concept_significance>
%  </concept>
%  <concept>
%   <concept_id>10003033.10003083.10003095</concept_id>
%   <concept_desc>Networks~Network reliability</concept_desc>
%   <concept_significance>100</concept_significance>
%  </concept>
% </ccs2012>
% \end{CCSXML}

% \ccsdesc[500]{Computer systems organization~Embedded systems}
% \ccsdesc[300]{Computer systems organization~Redundancy}
% \ccsdesc{Computer systems organization~Robotics}
% \ccsdesc[100]{Networks~Network reliability}

%%
%% Keywords. The author(s) should pick words that accurately describe
%% the work being presented. Separate the keywords with commas.
% \keywords{a, b, c, d}