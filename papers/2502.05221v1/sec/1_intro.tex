\section{Introduction} \label{sec:intro}

Combinatorial optimization (CO) involves finding the optimal solution from a finite, discrete solution space. These NP-hard problems, such as the Traveling Salesman Problem (TSP), Knapsack Problem, and Job Scheduling, require significant computational resources to solve exactly, motivating the development of efficient approximation methods.

Among CO problems, TSP is a widely studied example. It entails finding the shortest route to visit all cities exactly once and return to the starting point. The problem's factorial complexity makes brute-force solutions impractical, especially for large instances, necessitating heuristic and learning-based approaches.

To address NP-hard CO problems, supervised learning (SL), reinforcement learning (RL), and semi-supervised learning (SSL) methods have been explored. SL approaches generate approximate solutions using historical optimal data, often producing stable results via heatmap-based decoding \cite{joshi2019tsp, sun2023difusco, li2023t2t}. RL methods train agents to construct solutions sequentially or iteratively improve solutions through metaheuristics \cite{kool2019attention, bresson2021transformer, zheng2021rl-lkh}. However, RL approaches can suffer from inefficiencies during training due to large initial Optimality Gaps. Hybrid methods that integrate SL into RL training have been proposed to address this challenge.

Recently, DIFUSCO \cite{sun2023difusco} introduced diffusion generative models for CO problems, achieving state-of-the-art performance on TSP. DIFUSCO generates heatmaps representing edge probabilities and decodes them into feasible solutions. Building on this, the T2T model \cite{li2023t2t} introduced objective-guided sampling, further improving performance and efficiency while retaining DIFUSCO's non-autoregressive structure. DIFUSCO and T2T operate in a discrete time-space framework, with categorical-state diffusion (D3PM) outperforming continuous-state diffusion (DDPM) in discrete optimization tasks.

Inspired by these advancements, we explore Blackout Diffusion \cite{santos2023blackout}, a framework that bridges discrete state spaces and continuous-time modeling, for CO problems. Continuous-time diffusion, as demonstrated in "Score-Based Generative Modeling through Stochastic Differential Equations" \cite{song2021sde}, has shown performance gains in generative tasks compared to discrete-time methods like DDPM \cite{ho2020ddpm}. We hypothesize that applying continuous-time discrete-state diffusion to CO tasks could improve solution quality by enabling smoother state transitions.

This work represents the first attempt to combine DIFUSCO with Blackout Diffusion. By incorporating insights from diffusion scheduling and optimal observation time design, we aim to refine sampling strategy and solution quality. Although the performance was not satisfactory, our study lays a foundation for future research into continuous-time diffusion models for combinatorial optimization.

\subsection*{Contributions}

\begin{itemize} \item \textbf{Integration of DIFUSCO and Blackout Diffusion for Combinatorial Optimization}
\item \textbf{Sampling Optimization for Efficient Blackout Diffusion}
\item \textbf{Design of Heuristic Diffusion Processes}
\item \textbf{Exploration of Diffusion Models for Discrete State-Space Optimization}
\end{itemize}