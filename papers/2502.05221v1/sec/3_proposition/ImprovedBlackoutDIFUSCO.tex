\begin{figure*}[ht]
    \centering
    \begin{minipage}{0.32\textwidth}
        \centering
        \includegraphics[width=\linewidth]{sec/original.png}
        \caption*{(a) Original Blackout DIFUSCO}
    \end{minipage}
    \begin{minipage}{0.32\textwidth}
        \centering
        \includegraphics[width=\linewidth]{sec/improved.png}
        \caption*{(b) Improved Blackout DIFUSCO}
    \end{minipage}
    \begin{minipage}{0.32\textwidth}
        \centering
        \includegraphics[width=\linewidth]{sec/moreimproved.png}
        \caption*{(c) More Improved Blackout DIFUSCO}
    \end{minipage}
    \caption{
        Visualization of the forward diffusion process in Blackout DIFUSCO variants. 
        The plots illustrate the evolution of the corrupted heatmaps as the process progresses across the time axis. 
        Each design leverages the binomial distribution for modeling the forward process, and the variance of the heatmaps (\(\text{std}\)) is analyzed to optimize observation time scheduling. The red markers in the plots indicate the sampling points during the generation step, highlighting the regions where the model focuses its computations.
        \textbf{(a) Original Blackout DIFUSCO}: Displays the standard forward process with no modifications.
        \textbf{(b) Improved Blackout DIFUSCO}: Implements a peaked observation time schedule inspired by the cosine scheduler, ensuring that the standard deviation grows linearly to a sharp midpoint and tapers symmetrically.
        \textbf{(c) More Improved Blackout DIFUSCO}: Further enhances the scheduling by introducing a hyperparameter \(\alpha = 0.2\), which redistributes sampling density to focus on regions with maximum \(\text{std}\), ensuring improved reconstruction of challenging adjacency matrix states.
    }
    \label{fig:blackout_difusco_comparison}
\end{figure*}


\subsection{Improved Blackout DIFUSCO}
To enhance the performance of Blackout DIFUSCO, we propose a modification to the observation time scheduling inspired by the cosine scheduler from Improved DDPM \cite{nichol2021improved}. Specifically, we analyze the standard deviation (\(\text{std}\)) of the corrupted heatmaps across the time axis and observe that, unlike Gaussian-based diffusion models, the variance in Blackout Diffusion is smallest at the final corruption stage (\(t = T\)) and largest at the midpoint. 

To address this, we redesign the observation time scheduling such that the standard deviation increases linearly throughout the diffusion process, producing a sharper peak at intermediate timesteps. This adjustment ensures that the model allocates more computational focus during the middle stages, where the variance is highest, and less attention towards the extremes, where the variance is minimal. This "peaked" observation time distribution aligns with the unique characteristics of Blackout Diffusion, allowing the reverse process to reconstruct adjacency matrices more effectively.
