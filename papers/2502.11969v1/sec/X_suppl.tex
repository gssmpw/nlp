\clearpage
\appendix
\setcounter{page}{1}
\maketitlesupplementary

\definecolor{codegreen}{rgb}{0,0.6,0}
\definecolor{codegray}{rgb}{0.5,0.5,0.5}
\definecolor{codepurple}{rgb}{0.58,0,0.82}
\definecolor{backcolour}{rgb}{0.95,0.95,0.92}

\section{Prompt for ChatGPT-4o}

\begin{lstlisting}[
    language={},
    label=lst:gpt,
    backgroundcolor=\color{backcolour},   
    commentstyle=\color{codegreen},
    keywordstyle=\color{magenta},
    numberstyle=\tiny\color{codegray},
    stringstyle=\color{codepurple},
    basicstyle=\ttfamily\footnotesize,
    breakatwhitespace=false,         
    breaklines=true,                 
    keepspaces=true,                 
    numbers=none,        
    numbersep=5pt,                  
    showspaces=false,                
    showstringspaces=false,
    showtabs=false,                  
    tabsize=2,
    frame=single,
    rulecolor=\color{black}
]
You are a helpful assistant that identifies the common, specific concept among a given set of words and provides additional words that fit within that concept. For example, given ["rose," "daisy," "tulip"], you might infer the concept of "plants," with a more specific focus on "flowers." Once you've identified the concept, generate exactly 200 words that belong to it.

The new words should be as semantically distinct from each other as possible while staying relevant to the shared concept. Aim for diversity within the category to showcase a broad range of examples. Please return the words in a Python list format.

Here are the my words: [`Annual Crop Land', 'Forest', 'Herbaceous Vegetation Land', 'Highway or Road', 'Industrial Buildings']
\end{lstlisting}

We provide the prompt given to ChatGPT-4o for generating novel classes for EuroSAT. The prompt is designed based on the templates from \cite{anperceptionclip}. For other datasets, only the list of base classes is modified.

\section{Prompts for Ensembling}
\begin{table}[ht]
    \centering
    \renewcommand{\arraystretch}{1.0}
    \label{tab:pe}
    \resizebox{\columnwidth}{!}{
    \begin{tabular}{p{2.8cm} p{9cm}}
    \toprule
    \textbf{Dataset} & \textbf{Hand-crafted prompt} \\
    \midrule
    \multirow{7}{*}{All} & ``\texttt{itap of a [CLASS].}'' \\ 
                         & ``\texttt{a bad photo of the [CLASS].}'' \\ 
                         & ``\texttt{a origami [CLASS].}'' \\ 
                         & ``\texttt{a photo of the large [CLASS].}'' \\ 
                         & ``\texttt{a [CLASS] in a video game.}'' \\ 
                         & ``\texttt{art of the [CLASS].}'' \\ 
                         & ``\texttt{a photo of the small [CLASS].}'' \\ 
    \midrule
    ImageNet & ``\texttt{a photo of a [CLASS].}'' \\
    Caltech101 & ``\texttt{a photo of a [CLASS].}'' \\ 
    OxfordPets & ``\texttt{a photo of a [CLASS], a type of pet.}'' \\
    StanfordCars & ``\texttt{a photo of a [CLASS].}'' \\
    Flowers102 & ``\texttt{a photo of a [CLASS], a type of flower.}'' \\
    Food101 & ``\texttt{a photo of [CLASS], a type of food.}'' \\
    FGVCAircraft & ``\texttt{a photo of an aircraft [CLASS].}'' \\
    SUN397  & ``\texttt{a photo of a [CLASS].}'' \\
    DTD & ``\texttt{a photo of a [CLASS], a type of texture.}'' \\
    EuroSAT & ``\texttt{a centered satellite photo of [CLASS].}'' \\
    UCF101 & ``\texttt{a photo of a person doing [CLASS].}'' \\
    \bottomrule
    \end{tabular}}
\end{table}

We use the prompt templates provided in the implementation of CoOp \cite{zhou2022learning} for prompt ensembling. For each dataset, a total of eight prompts are ensembled, comprising seven general prompts and one dataset-specific custom prompt.

\section{Word Examples}
We compare the words generated by ChatGPT-4o with the actual new classes in the dataset. This qualitative comparison suggests that the words generated by ChatGPT-4o are semantically aligned with the new classes, making them suitable candidates for unseen classes.

\begin{table}[ht]
    \centering
    \footnotesize
    \setlength{\tabcolsep}{4pt}
    \renewcommand{\arraystretch}{0.90}
    \label{tab:wordex}

    % ----- 1) DTD -----
    \begin{subtable}{\linewidth}
        \centering
        \subcaption{DTD}
        \vspace{2mm}
        \begin{tabular}{p{1.5cm} >{\raggedright\arraybackslash}p{6cm}}
        \toprule
        & \textbf{Words} \\
        \midrule
        Base class & \texttt{banded, blotchy, braided, bubbly, bumpy, chequered, cobwebbed, cracked, crosshatched, crystalline, ...} \\
        \midrule
        ChatGPT-4o & \texttt{acid-washed, aerated, airy, angular, anodized, antiqued, arced, asymmetrical, beaded, bizarre, ...} \\
        \midrule
        New class & \texttt{matted, meshed, paisley, perforated, pitted, pleated, polka-dotted, porous, potholed, scaly, ...} \\
        \bottomrule
        \end{tabular}
    \end{subtable}

    \vspace{3mm}

    % ----- 2) FGVCAircraft -----
    \begin{subtable}{\linewidth}
        \centering
        \subcaption{FGVCAircraft}
        \vspace{2mm}
        \begin{tabular}{p{1.5cm} >{\raggedright\arraybackslash}p{6cm}}
        \toprule
        & \textbf{Words} \\
        \midrule
        Base class & \texttt{707-320, 727-200, 737-200, 737-300, 737-400, 737-500, 737-600, 737-700, 737-800, 737-900, ...} \\
        \midrule
        ChatGPT-4o & \texttt{Airbus A220-100, Airbus A220-300, Airbus A300, Airbus A310, Airbus A320neo, Airbus A321neo, ...} \\
        \midrule
        New class & \texttt{Cessna 560, Challenger 600, DC-10, DC-3, DC-6, DC-8, DC-9-30, DH-82, DHC-1, DHC-6, ...} \\
        \bottomrule
        \end{tabular}
    \end{subtable}

    \vspace{3mm}

    % ----- 3) StanfordCars -----
    \begin{subtable}{\linewidth}
        \centering
        \subcaption{StanfordCars}
        \vspace{2mm}
        \begin{tabular}{p{1.5cm} >{\raggedright\arraybackslash}p{6cm}}
        \toprule
        & \textbf{Words} \\
        \midrule
        Base class & \texttt{2000 AM General Hummer SUV, 2012 Acura RL Sedan, 2012 Acura TL Sedan, 2008 Acura TL Type-S, ...} \\
        \midrule
        ChatGPT-4o & \texttt{2019 Ford Mustang Coupe, 2020 Toyota Camry Sedan, 2018 Honda Civic Hatchback, 2021 Chevrolet Tahoe SUV, ...} \\
        \midrule
        New class & \texttt{2012 FIAT 500 Abarth, 2012 FIAT 500 Convertible, 2012 Ferrari FF Coupe, 2012 Ferrari California Convertible, ...} \\
        \bottomrule
        \end{tabular}
    \end{subtable}

\end{table}
