LawPal, a Retrieval-Augmented Generation (RAG)-based legal chatbot, enhances legal knowledge accessibility in India by leveraging DeepSeek-R1:5B for language understanding and FAISS for efficient document retrieval. It effectively addresses challenges in legal research, including accessibility, misinformation, and complexity, making it a valuable tool for legal professionals and the public.  

Evaluation results show over 90\% accuracy in retrieving and interpreting legal information, with strong efficiency, consistency, and robustness against adversarial inputs. Comparative benchmarking confirms LawPal’s superiority over existing legal AI tools. However, limitations persist in handling multi-jurisdictional queries, long-context arguments, and specialized legal topics like international law.  

Future improvements will focus on multilingual support, enabling access to legal texts in regional Indian languages, and jurisdictional adaptability, ensuring location-based legal filtering. Enhancing long-context understanding will allow better synthesis of interconnected legal provisions. Integration with government legal databases will keep the system up-to-date with new laws and judgments. Beyond research assistance, LawPal’s expansion into legal workflow automation, including document summarization, contract analysis, and compliance verification, will further increase its utility for legal professionals.  

In conclusion, LawPal is a major step in AI-driven legal assistance. With advancements in multilingual capabilities, jurisdictional specialization, and workflow integration, it is set to democratize legal knowledge and enhance access to legal information for all.