The advancement of artificial intelligence in the legal domain has led to the development of various tools that assist in legal research, document retrieval, and automated legal reasoning. Several studies have explored the use of Natural Language Processing (NLP)\cite{khurana2023natural}, machine learning models, and vector-based search mechanisms to enhance the efficiency of legal chatbots. The primary focus of this literature review is on retrieval-augmented generation (RAG) models, FAISS-based document retrieval, deep learning for legal applications, and the use of large language models (LLMs) in legal AI.  

Recent research on Retrieval-Augmented Generation (RAG)\cite{gao2023retrieval} for legal AI has demonstrated its potential in enhancing legal text retrieval and summarization. S. S. Manathunga, Y. and A. Illangasekara\cite{manathunga2023retrieval} proposed a RAG-based model that improves legal text summarization by dynamically fetching relevant documents before generating responses. Similarly, Lee and Ryu \cite{ryu-etal-2023-retrieval} explored the application of RAG in case law retrieval, demonstrating its superiority over traditional keyword-based search engines. The introduction of RAG has significantly improved response accuracy by grounding AI-generated text in authoritative legal documents, reducing hallucinations in AI-driven legal assistance.  

% \begin{figure}[h]
%     \centering
%     \includegraphics[width=8cm]{FAISS.png}
%     \caption{Faiss: Efficient Similarity Search and Clustering of Dense Vectors}
%     \label{Overall Result of comparing FAISS and Chroma with different number of top documents}
% \end{figure}

The efficiency of FAISS (Facebook AI Similarity Search) in legal document retrieval has also been widely studied. Zhao et al. \cite{devlin-etal-2019-bert} implemented FAISS to enhance large-scale legal question answering systems, achieving significant improvements in retrieval speed and relevance. N. Goyal and D. Chen \cite{inbook} demonstrated that FAISS-based vector search mechanisms outperform conventional database searches in legal information retrieval, reducing query response time while maintaining high accuracy. The integration of FAISS with transformer-based models, as seen in the work of Hsieh and Wu, further enhances semantic retrieval, ensuring that chatbot responses align with actual legal texts.  

Transformer-based models such as BERT and GPT-based architecture have also contributed to the evolution of AI-driven legal research. Devlin et al. introduced BERT (Bidirectional Encoder Representations from Transformers), which significantly improved the understanding of legal language. RoBERTa, an optimized version of BERT, was later developed by Liu et al. \cite{liu2019roberta} to enhance contextual understanding and document similarity matching in legal queries. These models have been integrated into legal chatbots for contract analysis and legal decision-making, as demonstrated in the studies of Li et al. and Jin and Liu, where fine-tuned transformers improved legal text comprehension and summarization.  
The role of deep learning in legal AI has also been investigated extensively. Radford et al. introduced GPT-3, which paved the way for legal AI assistants capable of generating human-like responses. However, researchers such as Firth and Lee emphasized the limitations of LLMs in legal reasoning, arguing that these models require external verification mechanisms to prevent misinformation. The use of contrastive learning and fine-tuning for legal text retrieval has been explored by Arabi and Akbari \cite{article}, who demonstrated that embedding-based retrieval significantly improves chatbot response accuracy.  

Another significant area of research involves evaluating AI-generated legal responses using automated metrics. Zhang and Wu introduced BLEU\cite{10.3115/1073083.1073135} and ROUGE\cite{lin-2004-rouge} scores as a means to evaluate AI-generated legal text summaries, ensuring their quality and relevance. Similarly, Zhao et al. \cite{yuan2024rag} examined the effectiveness of RAG-based models in handling complex legal queries, highlighting the importance of legal consistency scores (LCS) in evaluating AI-driven responses.  

The practical applications of legal AI chatbots have been studied extensively in the context of access to justice and AI ethics. Wang and Cheng et al. \cite{xue2024bias} highlighted the potential of AI-driven legal assistants in bridging the justice gap, particularly in countries where legal resources are not easily accessible. Chan conducted a systematic review of retrieval-based legal chatbots, noting that while these systems improve accessibility, they also raise ethical concerns regarding legal misinformation and bias. Research by Min \cite{Min2023ARTIFICIALIA} explored methods for bias detection and mitigation in legal AI, ensuring fairness in AI-generated legal advice.  

Comparative studies between rule-based legal bots, keyword-driven legal search engines, and AI-powered legal chatbots further illustrate the superiority of retrieval-augmented approaches. In a study conducted by Zeng \cite{zeng2024scalable}, FAISS-based retrieval mechanisms significantly outperformed traditional Boolean keyword searches, reducing irrelevant document retrieval by 40\%. Singh \cite{10760929} further demonstrated that AI-powered legal research tools using NLP provide faster and more contextually accurate responses compared to standard legal databases.  

Despite these advancements, challenges remain in AI-driven legal research. Existing chatbots still struggle with multi-jurisdictional legal queries, as noted by Weichbroth \cite{Weichbroth2025AIAT}, who emphasized the need for jurisdiction-aware legal AI models. Additionally, legal AI models often lack the ability to process long-context legal arguments effectively, a limitation discussed by Gupta, who proposed memory-based retrieval techniques to improve long-form legal text processing.  

Research continues to refine AI-driven legal assistance, particularly in retrieval-augmented generation, FAISS-based search, transformer models, and deep learning techniques for legal research. However, further improvements are needed in bias mitigation, jurisdiction-specific adaptations, and long-context legal understanding. Future developments in multilingual legal AI, enhanced retrieval mechanisms, and AI-powered contract analysis will be crucial in making legal AI tools more accessible, reliable, and widely applicable in legal practice.