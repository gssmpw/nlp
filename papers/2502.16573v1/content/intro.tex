LawPal’s effectiveness relies on a well-structured data pipeline that ensures accurate, legally valid, and contextually appropriate responses. The process begins with gathering legal texts from authoritative sources like government websites, Supreme Court archives, and legal research papers. To keep the dataset updated, automated web scraping tools extract the latest amendments and court decisions.

The data undergoes rigorous preprocessing, including cleaning, OCR digitization and text normalization. It is then segmented into smaller chunks using LangChain’s RecursiveCharacterTextSplitter to maintain logical continuity. Each chunk is encoded into vector embeddings using DeepSeek’s model\cite{liu2024deepseek}, which are indexed with FAISS\cite{douze2024faiss} for fast retrieval. Hierarchical indexing categorizes legal texts into domains like Criminal Law and Civil Law to improve search precision.

LawPal also implements caching for frequently searched queries and parallelized FAISS searches for scalability. Continuous quality assurance ensures the dataset remains accurate and up-to-date. This structured approach allows LawPal to handle legal queries with efficiency, semantic accuracy, and contextual awareness, making legal knowledge more accessible in India.