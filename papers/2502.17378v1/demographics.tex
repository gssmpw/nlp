\section{Demographics of Study Participants}
\label{sec:demographics}

This section provides an overview of the demographics of our study participants, highlighting their general experience in software development, their experience in ML projects, their primary roles and activities in ML software development, their familiarity with CI concepts, and their knowledge of the CI pipelines employed in the ML projects they contribute to.

Figure~\ref{fig:q1_1_and_q1_2_developer_expericel_level} presents the general software development experience of the study participants and their experience developing ML projects.
The results indicate that the majority of participants (68.4\%, \nicefrac{106}{155}) have five or more years of general software development experience, with 31.6\% of the participants having ten or more years of experience. This highlights the relatively high level of expertise of our participants, suggesting that the insights drawn from their responses are informed by substantial professional experience. Nevertheless, the collected responses encompass a variety of perspectives that also include those of less experienced developers.

Regarding experience in ML projects, the responses reveal a balanced distribution: 37.4\% of participants have less than 2 years of experience, 32.3\% have 2 to 4 years, and 30.4\% have 5 or more years of experience.
This distribution reflects the relative recency of ML projects as a field, where it is expected that many practitioners are still accumulating experience. While a significant portion of participants has considerable expertise, the presence of less experienced practitioners is equally valuable. These participants might provide unique perspectives on the challenges faced when starting to work on ML projects, particularly in adopting and using CI pipelines.

% \begin{figure}[H]
% 	\centering
% 	\includegraphics[ width=12cm]{q1_1_Experience_developing_software.pdf}
% 	% figure caption is below the figure
% 	\caption{Participants' experience developing software in general.}
% 	\label{fig:q1_1_Experience_developing_software}       % Give a unique label
% \end{figure}

\begin{figure}
	\centering
	\includegraphics[width=12cm]{q1_1_and_q1_2_developer_expericel_level.pdf}
	% figure caption is below the figure
	\caption{Participants' experience in general software development and ML project development.}
	\label{fig:q1_1_and_q1_2_developer_expericel_level}       % Give a unique label
\end{figure}

% Figure~\ref{fig:q1_2_Experience_developing_ML_projects} presents a detailed view of participants' experience contributing to ML projects. The responses reveal a balanced distribution: 37.4\% of participants have less than 2 years of experience, 32.3\% have 2 to 4 years, and 30.7\% have 5 or more years of experience.
% This distribution reflects the relative recency of ML projects as a field, where it is expected that many practitioners are still accumulating experience. While a significant portion of participants has considerable expertise, the presence of less experienced practitioners is equally valuable. These participants might provide unique perspectives on the challenges faced when starting to work on ML projects, particularly in adopting and using CI pipelines.

% \begin{figure}[H]
% 	\centering
% 	\includegraphics[ width=12cm]{q1_2_Experience_developing_ML_projects.pdf}
% 	% figure caption is below the figure
% 	\caption{Participants' experience developing ML projects.}
% 	\label{fig:q1_2_Experience_developing_ML_projects}       % Give a unique label
% \end{figure}

The distribution of participants' primary roles in ML projects, as shown in Figure~\ref{fig:q1_3_Primary_roles_in_ML_projects}, highlights the diverse expertise within our dataset.
Participants could select multiple roles in our form. Consequently, the reported percentages do not sum to 100\%, reflecting the fact that individuals often assume multiple responsibilities in ML projects.
The majority of participants identify as Developers (73\%, 114 participants), emphasizing their central role in ML project workflows. ML Engineers, responsible for model deployment and monitoring, comprise a significant portion of the participants (42\%, 65 individuals), while Data Scientists, focused on model development and validation, represent 39\% (60 participants).

Participants involved in code review and integration, crucial for maintaining code quality and consistency, account for 26\% (40 participants). Data Engineers, handling tasks such as data ingestion and storage, make up 23\% (36 participants). DevOps Engineers, who oversee the deployment and maintenance of CI/CD pipelines, constitute 19\% (29 participants). A smaller but essential group includes Testers (15\%, 24 participants), who ensure quality through systematic testing, and ML Researchers (1.9\%, 3 participants), who focus on advancing the theoretical aspects of ML. Finally, documentation roles, while less represented, remain critical for maintaining project records, with 0.6\% (1 participant) listing this as their primary responsibility.
This broad range of roles reflects the diverse expertise of our participants, capturing insights across the full lifecycle of ML project development and deployment.

\begin{figure}
	\centering
	\includegraphics[width=12cm]{q1_3_Primary_roles_in_ML_projects.pdf}
	% figure caption is below the figure
	\caption{Participants' primary roles in ML projects.}
	\label{fig:q1_3_Primary_roles_in_ML_projects}       % Give a unique label
\end{figure}

Regarding participants' familiarity with CI, we found that most have a strong understanding of the concept, which enhances the credibility and depth of their responses on CI practices in both ML and non-ML projects.
Figure~\ref{fig:q1_4_Familiarity_with_CI_concepts} illustrates the distribution of participants' familiarity levels with CI. A significant proportion reported a high level of familiarity, with 32.9\% (51 participants) indicating they are ``Very familiar'' and 17.4\% (27 participants) describing themselves as ``Extremely familiar''. Additionally, 29.7\% (46 participants) rated their familiarity as ``Moderately familiar''. Lower levels of familiarity were reported by smaller groups, with 18.1\% (28 participants) identifying as ``Somewhat familiar'' and only 1.9\% (3 participants, comprising 2 Developers and 1 Data Scientist primarily involved in model development and validation) as ``Not familiar at all''.

\begin{figure}
	\centering
	\includegraphics[ width=12cm]{q1_4_Familiarity_with_CI_concepts.pdf}
	% figure caption is below the figure
	\caption{Participants' familiarity with CI.}
	\label{fig:q1_4_Familiarity_with_CI_concepts}       % Give a unique label
\end{figure}

Beyond general familiarity with CI concepts, Figure~\ref{fig:q4_1_Familiarity_with_the_CI_pipeline_of_the_studied_project} reveals that most participants also possess a strong understanding of the CI pipelines used in the studied ML projects. Specifically, 31.6\% rated their familiarity as ``Fair'', 31.0\% as ``Good'', and 18.1\% as ``Excellent''. 
Only a smaller proportion of participants reported limited familiarity, with 11.0\% rating it as ``Poor'' and 7.7\% as ``Very Poor'', while an additional 7.7\% did not respond (NA). 
These findings highlight the strength of the dataset, as the participants' high level of familiarity ensures they are well-equipped to provide valuable and informed insights into the CI pipelines of ML projects, thereby enhancing the reliability and depth of their responses.

\begin{figure}
	\centering
	\includegraphics[width=12cm]{q4_1_Familiarity_with_the_CI_pipeline_of_the_studied_project.pdf}
	% figure caption is below the figure
	\caption{Participants' familiarity with the CI pipeline of the studied projects.}	\label{fig:q4_1_Familiarity_with_the_CI_pipeline_of_the_studied_project}       % Give a unique label
\end{figure}

% \textit{\textbf{The majority of participants (75.5\%, 116 out of 155) state that their ML teams strive to incorporate CI practices into their workflows.}} 
Additionally, in \textsc{Question \#2.1}, we asked participants whether they felt their teams were committed to key CI practices, including frequent commits, maintaining short build durations, promptly addressing broken builds, and ensuring high test coverage. The results show that 63.9\% (99 out of 155) of participants reported that their teams consistently prioritize these practices. In contrast, 11.6\% (18 out of 155) indicated partial commitment, with CI practices applied inconsistently or skipped in certain situations. Additionally, 14.2\% (22 out of 155) stated that their teams showed no commitment to CI practices, while 10.3\% (16 out of 155) did not provide a response.
These findings provide crucial context for understanding the broader landscape of CI adoption within ML project. Projects actively engaging with CI practices form a valuable foundation for participants’ responses, as they reflect both individual familiarity and team-level implementation of CI. 

% According to the participants, most ML projects are committed to incorporating CI practices into their workflows. As P112 emphasized, \textit{``As an integrator of an ML project, it is crucial for the team to strive to incorporate CI practices to ensure the smooth and efficient development and deployment of machine learning models''}. When participants mentioned partial commitment to CI practices, they often highlighted that only some practices are prioritized. For instance, P75 shared, \textit{``I attempt to incorporate CI practices, but the emphasis is less on quick builds. However, I strive to quickly address broken builds''}. Similarly, P78 stated, \textit{``Our team strives to achieve most of the mentioned points. Having said that, high test coverage is often overlooked while prioritizing fast R\&D results''}. Some participants also indicated that the adoption of CI practices depends on the nature of the project. P87 explained, \textit{``For research project, I think `high test coverage' is not mandatory. For other projects, I agree''}.

% In cases where there was a lack of commitment to adopting CI practices, participants cited reasons such as the \textit{absence of best practices}, \textit{business pressures}, and a \textit{general lack of interest in integrating CI} into the workflow. Regarding the absence of best practices, P21 commented, \textit{``Overall, the adoption of best software engineering practices in the ML-space is still lacking. There's always arguments that it's ``harder" to break down changes into smaller chunks and writing tests, even though this is not necessarily true"}.
% Business pressures were also mentioned, with P50 stating, \textit{``My team mostly strives to meet immediate business needs and remain agile above all"}. 
% Finally, P136 highlighted a lack of interest in CI practices in the teams they work, noting, \textit{``In my experience, there wasn’t much testing involved for ML models, and no pressure to keep build durations short".}

% In summary, the demographic analysis highlights the diverse expertise of the study participants, encompassing various roles, years of experience in both general software development and ML-specific projects, and higher levels of familiarity with CI. This diversity ensures that our study captures a wide range of perspectives, offering a comprehensive understanding of the adoption and challenges associated with CI practices in the ML domain.
