\begin{abstract}
Continuous Integration (CI) is a cornerstone of modern software development, promoting stability and ensuring quality. 
However, while widely adopted in traditional software projects, applying CI practices to Machine Learning (ML) projects presents distinctive characteristics. For example, our previous work revealed that ML projects often experience longer build durations and lower test coverage rates compared to their non-ML counterparts.
Building on these quantitative findings, this work investigates the underlying reasons for these distinctive characteristics through a qualitative perspective.
We surveyed 155 practitioners from 47 ML projects and conducted a thematic analysis to identify the key differences in CI adoption in the ML domain compared to non-ML domains. 
Practitioners highlighted eight key differences, including test complexity, infrastructure requirements, and build duration and stability. Common challenges mentioned by practitioners include higher project complexity, model training demands, extensive data handling, increased computational resource needs, and dependency management, all contributing to extended build durations. Furthermore, ML systems' non-deterministic nature, data dependencies, and computational constraints were identified as significant barriers to effective testing.
The key takeaway from this study is that while foundational CI principles remain valuable, ML projects require tailored approaches to address their unique challenges.
To bridge this gap, we propose a set of ML-specific CI practices, including tracking model performance metrics and prioritizing test execution within CI pipelines. Additionally, our findings highlight the importance of fostering interdisciplinary collaboration to strengthen the testing culture in ML projects. Furthermore, we emphasize the need for standardized guidelines to address key CI challenges in ML workflows, such as dependency management.
By bridging quantitative findings with practitioners' insights, this study provides a deeper understanding of the interplay between CI practices and the unique demands of ML projects, laying the groundwork for more efficient and robust CI strategies in this domain.

\keywords{Continuous Integration \and Machine Learning \and Build Duration \and Test Coverage}
% \PACS{PACS code1 \and PACS code2 \and more}
\subclass{68N01 \and 68T05 \and 68M15}
\end{abstract}