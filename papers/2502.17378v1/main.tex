%%%%%%%%%%%%%%%%%%%%%%% file template.tex %%%%%%%%%%%%%%%%%%%%%%%%%
%
% This is a general template file for the LaTeX package SVJour3
% for Springer journals.          Springer Heidelberg 2010/09/16
%
% Copy it to a new file with a new name and use it as the basis
% for your article. Delete % signs as needed.
%
% This template includes a few options for different layouts and
% content for various journals. Please consult a previous issue of
% your journal as needed.
%
%%%%%%%%%%%%%%%%%%%%%%%%%%%%%%%%%%%%%%%%%%%%%%%%%%%%%%%%%%%%%%%%%%%
%
% First comes an example EPS file -- just ignore it and
% proceed on the \documentclass line
% your LaTeX will extract the file if required
\begin{filecontents*}{example.eps}
%!PS-Adobe-3.0 EPSF-3.0
%%BoundingBox: 19 19 221 221
%%CreationDate: Mon Sep 29 1997
%%Creator: programmed by hand (JK)
%%EndComments
gsave
newpath
  20 20 moveto
  20 220 lineto
  220 220 lineto
  220 20 lineto
closepath
2 setlinewidth
gsave
  .4 setgray fill
grestore
stroke
grestore
\end{filecontents*}
%
\RequirePackage{fix-cm}
%
%\documentclass{svjour3}                     % onecolumn (standard format)
%\documentclass[smallcondensed]{svjour3}     % onecolumn (ditto)
\documentclass[smallextended]{svjour3}       % onecolumn (second format)
%\documentclass[twocolumn]{svjour3}          % twocolumn
%
\smartqed  % flush right qed marks, e.g. at end of proof
%
\usepackage{graphicx}
\usepackage{lipsum}
%\usepackage{sectsty}
\usepackage{pdfpages}
\usepackage{float}
\usepackage{caption}  % subfigures
\usepackage{subcaption} % subfigures
\usepackage{nicefrac}
\usepackage{url}
\usepackage{booktabs}
\usepackage{comment}
\usepackage{color,soul} % package to highlight text
\usepackage{array}
\usepackage{multirow}
\usepackage{bigstrut}
\usepackage{subcaption}
\usepackage{units}
\usepackage{flushend}
\usepackage{url}
\usepackage{natbib}
\usepackage{hyperref}
\usepackage{colortbl}
\usepackage{booktabs}
\usepackage{natbib}
\usepackage{makecell}
\usepackage{mathptm}
\usepackage{tcolorbox}

\usepackage{fancyvrb} % breaking lines with verbatim
\usepackage{fvextra} % breaking lines with verbatim
\usepackage{appendix}
\usepackage{microtype}

\usepackage{tabularx}

\usepackage{pdfpages}

\usepackage{xcolor}

\usepackage{tcolorbox}
\usepackage{listings}
\usepackage{fvextra}
\DefineVerbatimEnvironment{Verbatim}{Verbatim}{breaklines=true}
%
% \usepackage{mathptmx}      % use Times fonts if available on your TeX system
%
% insert here the call for the packages your document requires
%\usepackage{latexsym}
% etc.
%
% please place your own definitions here and don't use \def but
% \newcommand{}{}
%
% Insert the name of "your journal" with
% \journalname{myjournal}
%

% Command to create comments in the document

% \renewcommand{\arraystretch}{1.5}  % Adjust row height

% DEFINING THE HIGHLIGHT BOX
\newtcolorbox{highlightbox}{
	colback=gray!20, % Background color
	colframe=black,   % Border color
	boxsep=2pt,       % Padding around the box content
	arc=0pt,          % Corner radius
	boxrule=1pt       % Border thickness
}

\newcommand{\joao}[1] {\textcolor{red}{[\textbf{Jo\~{a}o}: #1]} }
\newcommand{\filipe}[1] {\textcolor{purple}{[\textbf{Filipe}: #1]} }
\newcommand{\dan}[1] {\textcolor{blue}{[\textbf{Dan}: #1]} }
\newcommand{\uira}[1] {\textcolor{green}{[\textbf{Uira}: #1]} }
\newcommand{\sergio}[1] {\textcolor{orange}{[\textbf{Sergio}: #1]} }



% \newcommand{\RQone}{RQ1: Are there perceived differences between ML and non-ML projects with respect to their CI practices?}

\newcommand{\RQone}{RQ1: What are the perceived reasons behind the differences between ML and non-ML projects with respect to their CI practices?}
\newcommand{\RQtwo}{RQ2: How do ML practitioners manage the build duration of their CI pipeline?}
\newcommand{\RQthree}{RQ3: Are there differences when handling test coverage rates between ML and non-ML projects?}

\begin{document}

\title{Continuous Integration Practices in Machine Learning Projects: The Practitioners' Perspective%\thanks{Grants or other notes
%about the article that should go on the front page should be
%placed here. General acknowledgments should be placed at the end of the article.}
}
% \subtitle{Do you have a subtitle?\\ If so, write it here}

\titlerunning{CI Practices in ML Projects: The Practitioners' Perspective}        % if too long for running head

\author{
    João Helis Bernardo \and 
    Daniel Alencar da Costa \and
    Filipe Roseiro Cogo \and
    Sérgio Queiróz de Medeiros \and    
    Uirá Kulesza        
}

%\authorrunning{Short form of author list} % if too long for running head

% \institute{
% João Helis Bernardo \at
% Federal Institute of Rio Grande do Norte \\
% Federal University of Rio Grande do Norte \\
% Natal, Brazil \\              
% \email{joao.helis@ifrn.edu.br}         \and
% Daniel Alencar da Costa \at
% University of Otago \\
% Dunedin, New Zealand \\              
% \email{danielcalencar@otago.ac.nz}      \and
% Sérgio Queiroz de Medeiros \at
% Federal University of Rio Grande do Norte \\
% Natal, Brazil \\                        
% \email{sergio.medeiros@ufrn.br}         
% \and
% Filipe Roseiro Cogo \at
% Centre for Software Excellence, Huawei \\
% Kingston, Canada \\                     
% \email{filipe.roseiro.cogo1@huawei.com}
% \and
% Uirá Kulesza \at
% Federal University of Rio Grande do Norte \\
% Natal, Brazil \\                        
% \email{uira@dimap.ufrn.br}         
% }

\institute{
% João Helis Bernardo  \at
% Federal Institute of Rio Grande do Norte \\
% Federal University of Rio Grande do Norte \\
% Natal, Brazil \\              
% \email{joao.helis@ifrn.edu.br}         \and
João Helis Bernardo, Sérgio Queiroz de Medeiros, and Uirá Kulesza  \at
Federal University of Rio Grande do Norte \\
Natal, Brazil \\              
\email{joao.helis@ifrn.edu.br, sergio.medeiros@ufrn.br, uira@dimap.ufrn.br}
\and
Daniel Alencar da Costa \at
University of Otago \\
Dunedin, New Zealand \\              
\email{danielcalencar@otago.ac.nz}      \and
% Sérgio Queiroz de Medeiros \at
% Federal University of Rio Grande do Norte \\
% Natal, Brazil \\                        
% \email{sergio.medeiros@ufrn.br}         
% \and
Filipe Roseiro Cogo \at
Centre for Software Excellence, Huawei \\
Kingston, Canada \\                     
\email{filipe.roseiro.cogo1@huawei.com}
% \and
% Uirá Kulesza \at
% Federal University of Rio Grande do Norte \\
% Natal, Brazil \\                        
% \email{uira@dimap.ufrn.br}         
}

\date{Received: date / Accepted: date}
% The correct dates will be entered by the editor


\maketitle

\begin{abstract}
Retrieval-Augmented Generation (RAG) is often used with Large Language Models (LLMs) to infuse domain knowledge or user-specific information. In RAG, given a user query, a retriever extracts chunks of relevant text from a knowledge base. These chunks are sent to an LLM as part of the input prompt. Typically, any given chunk is repeatedly retrieved across user questions. However, currently, for every question, attention-layers in LLMs fully compute the key values (KVs) repeatedly for the input chunks, as state-of-the-art methods cannot reuse KV-caches when chunks appear at arbitrary locations with arbitrary contexts. Naive reuse leads to output quality degradation.  This leads to potentially redundant computations on expensive GPUs and increases latency. In this work, we propose \sys, a system for managing and reusing precomputed KVs corresponding to the text chunks (we call \textit{chunk-caches}) in RAG-based systems. We present how to identify \hl{\textit{chunk-caches} that are reusable}, how to efficiently perform a small fraction of recomputation to \textit{fix} the cache to maintain output quality, and how to efficiently store and evict \textit{chunk-caches} in the hardware for maximizing reuse while masking any overheads. With real production workloads as well as synthetic datasets, we show that \sys reduces redundant computation by \textbf{51\%} over SOTA prefix-caching and \textbf{75\%} over full recomputation.
\hl{Additionally, with continuous batching on a real production workload, we get a \textbf{1.6$\times$} speedup in throughput and a \textbf{2$\times$} reduction in end-to-end response latency over prefix-caching while maintaining quality, for both the \llama-3-8B and \llama-3-70B models. 
}
\end{abstract}





\documentclass[../main.tex]{subfiles}
\graphicspath{{../images/}}
\makeatletter
\def\input@path{{../images/}}
\makeatother
\begin{document}
\section{Introduction}
\begin{figure}
\centering
\begin{tikzpicture}
\node[inner sep=0pt] (ws) at (0, 0) {
\includegraphics[height=.4\textwidth, trim={10cm 0 10cm 0},clip]{world_space.png}};
\node[inner sep=0pt] (cs) at (6,0) {\includegraphics[height=.4\textwidth, trim={10cm 1cm 10cm 4cm},clip]{conf_space.png}};
\end{tikzpicture}
\vspace{-5pt}
\label{fig:pbrm_intro}
\caption{\textbf{Left}: Shows world space obstacles as grey spheres. Robots start and goal configuration is colored red and green, respectively. Configurations along the computed path are colored transparent blue. \textbf{Right:} Mapped world space scenario to configuration space. Obstacle region is the grey mesh. Red spheres are collision-free regions computed by the neural SCDF. The optimized shortest path in the convex corridor is the blue curve.}
\vspace{-25pt}
\end{figure}
Motion planning is the problem of finding a collision-free trajectory that connects a given start and goal configuration. The planning takes place in the configuration space of the robot. For single body robots, like mobile robots or drones, the configuration space and the world space are usually the same. This simplifies the planning, since explicit obstacle representations are available which enables geometrical tools like separating hyperplanes, smallest distance to obstacles etc., to be used when designing motion planning algorithms. For multi-body robots like manipulators, the situation is completely different. The world space obstacles are usually mapped to non-convex regions, and to make the problem even harder, the mapping is usually not known. Forming explicit representations of the obstacle region in the configuration space is usually too expensive or intractable. Despite all of this, sampling based planners are used with great success, which mainly is due to their use of implicit representations of the obstacle region. The basic idea is to construct a graph in the configuration space that covers and connects the collision-free region. From this graph, a path can be extracted that connects a given start and goal configuration. The approach is computationally expensive, since the graph is constructed with the smallest geometrical building block available, points, which represents a collision-check. Furthermore, the extracted paths from the graph are non-smooth and jagged due to the stochastic nature of the approach. This adds an additional post-processing step to the process, where the paths are shortcutted and smoothened, before the path can be used for tracking. Clearly a lot of time is invested to form this graph and produce smooth paths. Thus, if the obstacles start to move, then all of this work is done in no use, since all points that make up this graph need to be re-verified, which is simply too time consuming to be done in real time.
\\\\
In this work, we want to address the existing drawbacks of the sampling based planners. Our main contribution is an improved motion planner where each vertex in the graph covers a collision-free region in the form of a sphere instead of a point and where the edges are formed with neighboring intersecting spheres. This representation has the advantage of instead of returning piecewise linear paths, returning a sequence of overlapping spheres, i.e. a convex corridor, that connects a given start and goal configuration, illustrated in Figure \ref{fig:pbrm_intro}. This convex corridor allows us to use convex optimization to produce smooth trajectories, instead of computationally expensive post-processing methods. The representation further allows us to estimate the coverage of the collision-free space, which gives us awareness and feedback in the offline roadmap construction phase. Finally, our representation is simple to adapt to moving obstacles, simply requery for the new radii and recheck for intersections. 
\\\\
The spherical collision-free regions are formed using a signed distance function (SDF), which is a function that returns the smallest distance from an arbitrary point to the boundary of an obstacle. As the name implies, the distance is signed, thus if the point is inside the obstacle it is negative otherwise positive. If the distance is positive, a sphere with radius equal to the distance is guaranteed to cover a collision-free region. Using an SDF in motion planning is not new, but what is novel about our approach is that we express the distance in the configuration space instead of the world space and by doing so allows us to form these convex collision-free regions. We refer to the resulting SDF as a signed configuration distance function (SCDF). Computing an SCDF analytically is non-trivial, our approach is therefore to parameterize the SCDF with a deep neural network and learn the mapping by supervised learning. Our resulting neural SCDF can compute distances for different parameter values of obstacle shapes and we also show how multiple distances can be combined, thus making our approach flexible.
\section{Related work}
Motion planning algorithms can roughly be divided into three families, grid-based, sampling based and optimization based methods. Grid-based methods (GBM) discretize the planning space from which a graph is then compiled. A standard search method is A$^\star$ \citep{a_star}, which is classified as an \textit{informed} search method, since it employs a heuristic function to speed up the search. A$^\star$ guarantees to return an optimal path at the level of discretization used. GBMs usually discretize the planning space by a regular lattice and this limits the GBMs to problems with low dimensionality due to the curse of dimensionality. Thus, GBMs are usually limited to single-body robots where the degrees of freedom (DOF) are low. To overcome the inherent scaling problem with the GBMs, stochastic methods are usually used for multi-body robots. These methods are termed as sampling-based methods (SBM) and core members within this family are the rapidly-exploring random trees (RRT) \citep{rrt} and the probabilistic roadmap (PRM) \citep{prm}. RRT grows a tree from the start configuration and explores the collision-free region in a rapid way until it is able to connect to the goal region. RRT is usually improved by bi-directional planning \citep{rrt_connect}, i.e. an additional tree is grown from the goal configuration and the trees are tested for connection after any tree has been expanded. RRT is a single-query method, thus it searches for a path from scratch each time it is queried. Contrary to this, PRM is a multi-query method, which solves for multiple queries without starting from scratch. PRM does this by creating a roadmap (graph) that covers the collision-free space as an offline step. The graph is then used to solve for multiple queries. PRMs are used in cases where the environment does not change since the extra offline step is too computationally costly and needs to be re-done if the environment is changed. In our work, we address this inherent issue by using a different roadmap representation. Our vertices in the graph cover a collision-free region in the form of spheres and we form the edges by checking for intersecting spheres. If something in the environment changes, we recompute the spheres radii and recheck the intersections, without relying on collision detection. We use a trained neural network to compute the sphere radius, therefore querying for the radius can be done fast, hence our representation enables the PRM for dynamic environments.
\\\\
In the recent decades, optimization based methods (OBM) \citep{chomp, schulman, itomp, stomp} have been introduced as an alternative to SBM for multi-body robots. Like the SBM, the OBMs scale well to higher dimensional problems and produce smoother motion. It is common to use a SDF in the optimization since it is a smooth function, thus enabling gradient-based methods. However, the standard way of expressing the SDF is in world space. The distance therefore needs to be mapped to the configuration space by the forward kinematics. This mapping makes the optimization problem a non-linear program (NLP), which is computationally expensive to solve. Recently, a different approach has been proposed. In \cite{mp_gcs} motion planning is formulated as a convex optimization problem by using the graph of convex sets framework \citep{gcs}. The underlying idea is to decompose the collision-free space into intersecting convex sets from which a convex optimization problem is formulated. In cases where an explicit representation of the obstacles in the configuration space exists, like for single-body robots, creating collision-free convex regions can be done fast \citep{iris}. For multi-body robots, this is non-trivial. Existing work does this successfully \citep{iris_nlp, iris_c} by an optimization based approach, but the methods are still too time consuming to be used in the presence of moving obstacles. Our approach is instead to use deep learning to learn an SDF expressed in the configuration space. With this, we can query for shortest distances to the collision boundary, which allows us to expand spherical regions which are collision-free. Our approach is fast and therefore enables our suggested roadmap planner to be used in dynamic environments.
\\\\
Recent research has focused on learning collision detection \citep{fk_kernel_distance, diffco, graphdistnet} by predicting the signed distance between the robot links and the surrounding obstacles in the world space. The learned SDF is used in trajectory optimization but since the distance is expressed in the world space, the problem becomes an NLP and therefore takes a long time to solve. We take a novel approach and suggest to instead express the signed distance in the configuration space. This allows us to improve the PRM at the same time as it enables convex optimization for trajectory optimization, which runs faster and is more reliable than NLP solvers. In \cite{cspf} a learned signed distance function in the configuration space is proposed similar to our approach. However, their approach is restricted to point cloud representations, while we propose to represent the obstacles as parameterized geometric shapes, e.g. spheres. Furthermore, we also show how to use our learned SCDF to improve an existing roadmap planner.
\section{Problem formulation}
A robot is located in the world space, $\W \subset \R^3 $. The unique location of the robot is given by its configuration $\q \in \C$, where $\C$ is the configuration space. The set of points covered by the robots bodies at a certain configuration is expressed as $\B(\q) \subset \W$. The robot is surrounded by $\NrObst$ obstacles $\O = \bigcup_{i=1}^{\NrObst} \O_i$, where  $\O_i \subset \W$. The representation of the obstacle in the configuration space is the set $\C\O_i = \{\q \in \C \: |\: \B(\q) \cap \O_i \neq \emptyset \}$. The obstacle space is formed as $\Co = \bigcup_{i=1}^{\NrObst} \C \O_i$. The complement is referred to as the free space, $\Cf = \C \setminus \Co$. The path planning problem is a tuple, ($\Cf$, $\qStart$, $\qGoal$), where we want to connect a query pair, consisting of a start, $\qStart$, and goal configuration, $\qGoal$, with a geometric path, $\q(s): [0, 1] \mapsto \Cf$, such that $\q(0)=\qStart$ and $\q(1)=\qGoal$, or report correctly when such a path does not exist.
\end{document}

\section{Related Work}
Alongside a discussion of what is meant by LLM harmfulness,
this section covers two distinct strands of related work: measuring types of harm in LLMs, and LLMs for diverse annotation tasks. %First,

%Different kinds of 
Diverse undesirable LLM outputs, from toxic language to privacy invasion, have been discussed in the observed \cite{banko-etal-2020-unified}. Here we review the ones we include in our definition of ``harm.'' %definition. Plus, we review LLMs as judges. 
Toxic content can be elicited from both generative  \cite{deshpande2023toxicity} and masked LLMs \cite{ousidhoum-etal-2021-probing}. 
%Among ways 
To measure toxic or hateful language, some use APIs such as PerspectiveAPI \cite{lees2022new} or HateBERT \cite{caselli-etal-2021-hatebert}. \citet{openai2024gpt4technicalreport} report that GPT4 produces toxic content 0.78\% of the time, versus 6.48\% in GPT3.5.
%as opposed to GPT3.5 with 6.48\%. On the other hand,
\citet{dubey2024llama} report that llama3-70B produces harmful content 5\% of the time, %whereas the 405B model generates harm 3\% of the time. 
compared to 3\% in the 405B model.
Instead of %single value classifiers to measure harm, 
reporting an absolute rate, we focus on relative harmfulness of different LLMs. %, so we point to recent work on LLMs for annotation.

The first category of harm we consider is social stereotyping and bias. %discrimination. It has been shown that 
LLMs can perpetuate social bias based on gender, race, religion etc. \cite{lin-etal-2022-gendered,bender2021dangers,field-etal-2021-survey,gupta-etal-2024-sociodemographic,andriushchenko2024agentharm,mazeika2024harmbench}. This can marginalize these groups more, and results in less fair model performance. \citet{guo2024hey} designed a competition to elicit biased output from LLMs to assess the perception of bias from non-expert users. %The first part of our work is similar to this analysis, but 
We also intentionally elicit harmful output, going %we look at other types of harms besides bias.
beyond social bias.

%When the models become stronger, they become more robust to jailbreaking attacks to elicit harmful content. However, there are datasets that can still jailbreak models to produce harmful content \cite{andriushchenko2024agentharm,mazeika2024harmbench}.

Our second category of harm is offensiveness and toxicity, which %. As opposed to stereotyping or social discrimination, this harm 
%is more subjective and harder to define than the previous category, so there 
lacks an established definition due to its greater subjectivity \cite{dev-etal-2022-measures,korre-etal-2023-harmful}. We include hate speech (HS) and abusive language as toxic content. HS can be defined as expressions of offensive and discriminatory discourse towards a group or an individual based on characteristics such as race, religion, nationality, or other group characteristics \cite{john2000hate,jahan2023systematic,basile2019semeval,davidson2017automated}. It includes racism, negative stereotyping, and sexist language. On the other hand, abusive language is content with inappropriate words such as profanity or disrespectful terms. It also includes psychological threats such as humiliation. %or constant criticism. %Toxic content can be elicited from both generative models \cite{deshpande2023toxicity} and masked language models \cite{ousidhoum-etal-2021-probing}.

%In addition to obvious toxic content, LLMs can generate diverse implicit toxic outputs using reinforcement learning with favoring toxic content in the reward function \cite{wen-etal-2023-unveiling}.  Regarding the subjectivity of this task, \cite{korre-etal-2023-harmful} reannotate the existing datasets with different definitions of toxicity and show that broader definitions result in more robust annotations, but interannotator agreements are still lower than 0.5. \cite{dev-etal-2022-measures} also point out the lack of definition for bias and harm in general and propose a framework to guide researchers during the development of bias measures.

Harm can be implicit, such as privacy invasion
%We are also interested in privacy invasion,
where there is 
leakage of personal information. %leakage from the model. 
%LLMs can memorize details of the training data and then leak private information such as 
This includes social security numbers, phone numbers, or bank account information \cite{carlini2021extracting,brown2022does}. 
%There are several frameworks to test the privacy of LLMs \cite{li2024llm} and generate data for personal attribute inference \cite{yukhymenko2024synthetic,kim2024propile}.

%Our definition of harm includes hate speech (HS) as well. HS can be defined as \textcolor{red}{expressions of} hatred towards a social group, the humiliation of the members of a group, or %communication disparaging  extreme disparagement of a person or a group based on race, color, ethnicity, gender, sexual orientation, nationality, religion, or other group characteristics .

For data annotation, LLMs
%Besides text generation, 
%LLMs have been used to annotate data because they 
can %be comparable to 
replace humans for some tasks, %and make the annotation process faster and cheaper 
with gains in efficiency and economy \cite{tan2024large}. They have been used for sociological annotations such as for classification of stance, bots or humor  \cite{ziems2024can,zhu2023can}. For tasks such as topic and frame detection or sentence segmentation they can surpass crowd-workers
%Some works show that they can surpass crowd-workers for some tasks such as topic and frame detection or sentence segmentation %into research aspects 
\cite{he2024if,gilardi2023chatgpt}. Some have argued that human-LLM collaboration results in more reliable annotation \cite{he2024if,zhang2023llmaaa,kim2024meganno+}. In addition to more objective tasks,
%LLMs have been used to annotate data %even 
they have been applied to subjective annotations such as offensiveness and abusiveness \cite{pavlovic-poesio-2024-effectiveness,zhu2023can,he2023annollm}, %. For example, LLMs are used as judges to rank responses from different LLMs 
or to rank outputs from different LLMs based on helpfulness, accuracy, or relevance \cite{zheng2023judging,lin2024wildbench,dubois2024length}. These works tend to focus on human-large LLM interactions, whereas we focus on single-turn responses from smaller LLMs. We inspire from \citet{zheng2023judging} but we only measure harm instead of overall performance. Plus, we use 3 LLMs to evaluate smaller LLMs.
% \label{demographics}

\section{Research Methodology}
\label{sec:methodology}

In this section, we describe how we selected the ML practitioners for our study, the data collection process, and the research approach we used to perform our analysis.

\subsection{Subject Projects}

Our prior work~\citep{bernardo2024machine} quantitatively analyzed the adoption of CI practices across 93 ML projects and 92 non-ML projects. This study revealed that ML projects often experience longer build durations and that medium-sized ML projects tend to have lower test coverage compared to non-ML projects. Building on these findings, our current study seeks to deepen our understanding of the specific factors driving these differences in CI practice adoption. To achieve this, we employed a qualitative, survey-based approach, focusing on the perceptions of ML practitioners (i.e., contributors and integrators of ML projects) regarding the challenges, barriers, and strategies associated with CI adoption in ML projects.

To maintain consistency with prior analyses and minimize potential biases associated with using an unverified or outdated collection of projects, we based our study on the 
93 ML projects investigated in our prior work~\citep{bernardo2024machine}. 
This dataset is both up-to-date and meticulously curated, representing a diverse collection of actively maintained ML projects that successfully integrate CI workflows into their pipelines. Additionally, the projects are categorized by size—small, medium, or large—based on their Lines of Code (LOC), ensuring a comprehensive representation across varying scales of ML development.

In addition to investigating general differences in CI adoption between ML and non-ML projects, this study focuses on the factors influencing build durations in ML projects, as outlined in RQ2. 
To capture diverse and meaningful insights, we adopted a sampling approach designed to reflect a wide range of experiences related to build durations. Specifically, we targeted projects with the shortest and those with the longest build durations to ensure that our analysis encompasses the full spectrum of challenges and characteristics encountered in ML workflows.

To sample practitioners for the survey, we selected those that are associated with the top 25\% of projects with the shortest build durations and the top 25\% with the longest build durations. 
As a result, we identified practitioners from 47 ML projects.
This targeted selection enhances the relevance of our findings by ensuring representation from both extremes of build duration characteristics, providing a comprehensive view of the characteristics and challenges faced in the CI workflows of ML projects. The list of the investigated ML projects, along with their characteristics (e.g., size and median build duration), is provided in Table~\ref{tab:project_characteristics}. 

% Table generated by Excel2LaTeX from sheet 'Sheet2'
\begin{table}[H]
  \centering
  \caption{Characteristics of Investigated Machine Learning Projects.}
    \begin{tabular}{cp{3.6cm}
    >{\raggedright\arraybackslash}p{2cm}
    >{\raggedleft\arraybackslash}p{2cm}
    >{\raggedright\arraybackslash}p{2cm}}
    \toprule
    \multirow[t]{3}{*}{\textbf{\#}} & \multirow[t]{3}{*}{\textbf{Project}} & \multirow[t]{3}{*}{\textbf{LOC Size}} & \textbf{Median build duration (minutes)} & \textbf{Build duration category} \bigstrut\\
    \midrule
    \textbf{1} & alan-turing-institute/sktime & large & 127.3 & longer \bigstrut[t]\\
    \textbf{2} & amark/gun & large & 1.3   & shorter \\
    \textbf{3} & apache/incubator-mxnet & large & 192.8 & longer \\
    \textbf{4} & apache/spark & large & 123.7 & longer \\
    \textbf{5} & apache/superset & large & 10.2  & shorter \\
    \textbf{6} & AUTOMATIC1111/stable-diffusion-webui & medium & 7.4   & shorter \\
    \textbf{7} & BehaviorTree/BehaviorTree.CPP & medium & 3.7   & shorter \\
    \textbf{8} & BLKSerene/Wordless & medium & 37.4  & longer \\
    \textbf{9} & chakki-works/doccano & medium & 2.8   & shorter \\
    \textbf{10} & criteo/tf-yarn & small & 2.4   & shorter \\
    \textbf{11} & DandyDev/slack-machine & small & 2.5   & shorter \\
    \textbf{12} & diffgram/diffgram & large & 5.1   & shorter \\
    \textbf{13} & dmlc/tvm & large & 90.2  & longer \\
    \textbf{14} & FluxML/Metalhead.jl & small & 28.4  & longer \\
    \textbf{15} & FluxML/NNlib.jl & medium & 37.4  & longer \\
    \textbf{16} & huggingface/pytorch-pretrained-BERT & large & 5.4   & shorter \\
    \textbf{17} & huggingface/transformers & large & 5.4   & shorter \\
    \textbf{18} & JohnSnowLabs/spark-nlp & large & 47.0    & longer \\
    \textbf{19} & jtablesaw/tablesaw & medium & 4.3   & shorter \\
    \textbf{20} & kendryte/nncase & large & 147.4 & longer \\
    \textbf{21} & LaurentMazare/tch-rs & large & 7.1   & shorter \\
    \textbf{22} & microsoft/dowhy & medium & 29.8  & longer \\
    \textbf{23} & microsoft/LightGBM & medium & 21.8  & longer \\
    \textbf{24} & microsoft/onnxruntime & large & 99.6  & longer \\
    \textbf{25} & microsoft/pai & medium & 3.9   & shorter \\
    \textbf{26} & mlpack/mlpack & large & 126.1 & longer \\
    \textbf{27} & mne-tools/mne-cpp & large & 62.4  & longer \\
    \textbf{28} & msdslab/automated-systematic-review & medium & 4.7   & shorter \\
    \textbf{29} & nilearn/nilearn & medium & 31.0    & longer \\
    \textbf{30} & opencv/dldt & large & 2.5   & shorter \\
    \textbf{31} & OpenKore/openkore & large & 5.0     & shorter \\
    \textbf{32} & OpenNMT/OpenNMT-py & medium & 3.4   & shorter \\
    \textbf{33} & pytorch/ignite & medium & 29.0    & longer \\
    \textbf{34} & pytorch/tnt & medium & 4.1   & shorter \\
    \textbf{35} & RubixML/RubixML & medium & 3.6   & shorter \\
    \textbf{36} & scikit-learn/scikit-learn & large & 0.3   & shorter \\
    \textbf{37} & SeldonIO/seldon-core & large & 4.6   & shorter \\
    \textbf{38} & shimat/opencvsharp & medium & 20.0    & longer \\
    \textbf{39} & skorch-dev/skorch & medium & 5.3   & shorter \\
    \textbf{40} & smistad/FAST & medium & 46.6  & longer \\
    \textbf{41} & sorgerlab/indra & medium & 19.3  & longer \\
    \textbf{42} & tensorflow/addons & medium & 19.4  & longer \\
    \textbf{43} & tensorly/tensorly & medium & 46.6  & longer \\
    \textbf{44} & tesseract-ocr/tesseract & large & 124.9 & longer \\
    \textbf{45} & Texera/texera & large & 6.8   & shorter \\
    \textbf{46} & TuringLang/Turing.jl & small & 107.1 & longer \\
    \textbf{47} & zhenghaoz/gorse & medium & 6.9   & shorter \bigstrut[b]\\
    \bottomrule
    \end{tabular}%
  \label{tab:project_characteristics}%
\end{table}%

\subsection{Data Collection}

To identify practitioners within the 47 investigated ML projects, we focused on individuals who actively contributed to the projects after the adoption of \textsc{GitHub Actions} CI workflows. Specifically, we selected integrators who either merged or closed at least one pull request (PR) or submitted at least one PR that was successfully merged into the \textit{main/master} branch of the project codebase during this period. This approach ensures that the selected practitioners directly contributed to the project while CI workflows were in use, enabling them to provide relevant and informed insights into CI practices in ML projects.

We collected PR metadata for the studied projects using the \textsc{GitHub API} on June 14, 2024. 
The PR's metadata include details of the PR number, state, author login, base branch, number of additions, deletions, changed files, commit count, and whether the PR was merged or closed, along with the login of the user who closed it.
By analyzing data from the period following the adoption of \textsc{GitHub Actions}, we identified 114,598 PRs reviewed by 3,276 unique integrators. Additionally, 6,861 contributors had at least one PR successfully merged. Importantly, these represent two distinct groups of practitioners in our study: contributors, who authored PRs, and integrators, who merged or rejected them. Among the contributors, 1,909 also acted as integrators, reflecting some overlap between the two roles. After accounting for this overlap, we identified a total of 4,952 unique contributors. Combined with the 3,276 unique integrators, this results in 8,228 practitioners involved in the studied projects.

To contact the practitioners involved in these projects, we collected their email addresses using the \textsc{GitHub API}, ensuring we only collected publicly available information. Of the 3,276 unique integrators, we retrieved 2,060 email addresses; for the 4,952 contributors, we retrieved 2,947. In total, we collected 5,007 unique email addresses from practitioners in the 47 analyzed projects.

To collect our data, we designed a web-based survey and sent invitations by email to all 5,007 ML practitioners whose email addresses were available.
The invitation letter is included in Appendix~\ref{sec:appendix_invitation_email_example}. To encourage participation, we offered respondents the opportunity to win one of ten \$50 Amazon or Steam gift cards, distributed through a random drawing. Participants were eligible for the draw only if they completed all survey questions and explicitly indicated their willingness to participate.
We used \textsc{Mailgun}\footnote{\url{https://www.mailgun.com}} to send personalized email invitations to each practitioner.
If a practitioner was associated with multiple investigated projects, we sent only one invitation, prioritizing the project where they had the highest number of integrated PRs.

In total, we received 155 responses, resulting in a response rate of 3.1\% (\nicefrac{155}{5007}). These responses came from practitioners associated with 30 of the 47 investigated projects. Table~\ref{tab:number_responses_per_respondent_type} presents details on the number of practitioners contacted per project, the responses received, and the corresponding response rates. To maintain anonymity, practitioners' names have been replaced with unique identifiers in the table. For example, practitioner 01 is labeled as ``P1'' and is associated with the \textit{alan-turing-institute/sktime} project.

% Table generated by Excel2LaTeX from sheet 'Sheet1'
\begin{table}[H]
  \centering
  \caption{Number of responses and practitioners of the studied ML projects that were invited to participate.}
\begin{tabular}{cp{3cm}>{\raggedleft\arraybackslash}p{1.7cm}
>{\raggedleft\arraybackslash}p{1.5cm}
>{\raggedleft\arraybackslash}p{1.3cm}
>{\raggedright\arraybackslash}p{1.5cm}}
\toprule
\textbf{\#} & \textbf{Project} & \textbf{Number of practitioners} & \textbf{Number of responses} & {\textbf{Response rate}} & \textbf{Practitioner IDs} \\
\midrule
1 & alan-turing-institute/sktime & 123 & 8 & 6.5\% & P1--P8 \\
2 & amark/gun & 34 & 2 & 5.9\% & P9--P10 \\
3 & apache/spark & 189 & 1 & 0.5\% & P11--P11 \\
4 & apache/superset & 435 & 12 & 2.8\% & P12--P23 \\
5 & AUTOMATIC1111/stable-diffusion-webui & 293 & 8 & 2.7\% & P24--P31 \\
6 & BehaviorTree/BehaviorTree.CPP & 56 & 3 & 5.4\% & P32--P34 \\
7 & chakki-works/doccano & 37 & 1 & 2.7\% & P35--P35 \\
8 & diffgram/diffgram & 8 & 2 & 25.0\% & P36--P37 \\
9 & dmlc/tvm & 399 & 7 & 1.8\% & P38--P44 \\
10 & FluxML/NNlib.jl & 24 & 1 & 4.2\% & P45--P45 \\
11 & huggingface/pytorch-pretrained-BERT & 700 & 18 & 2.6\% & P46--P63 \\
12 & huggingface/transformers & 689 & 19 & 2.8\% & P64--P82 \\
13 & kendryte/nncase & 7 & 1 & 14.3\% & P83--P83 \\
14 & LaurentMazare/tch-rs & 39 & 2 & 5.1\% & P84--P85 \\
15 & microsoft/LightGBM & 89 & 2 & 2.2\% & P86--P87 \\
16 & microsoft/onnxruntime & 196 & 10 & 5.1\% & P88--P97 \\
17 & microsoft/pai & 12 & 1 & 8.3\% & P98--P98 \\
18 & mlpack/mlpack & 38 & 4 & 10.5\% & P99--P102 \\
19 & msdslab/automated-systematic-review & 23 & 1 & 4.3\% & P103--P103 \\
20 & nilearn/nilearn & 58 & 6 & 10.3\% & P104--P109 \\
21 & opencv/dldt & 394 & 4 & 1.0\% & P110--P113 \\
22 & pytorch/tnt & 10 & 1 & 10.0\% & P114--P114 \\
23 & scikit-learn/scikit-learn & 627 & 25 & 4.0\% & P115--P139 \\
24 & SeldonIO/seldon-core & 69 & 2 & 2.9\% & P140--P141 \\
25 & shimat/opencvsharp & 19 & 1 & 5.3\% & P142--P142 \\
26 & sorgerlab/indra & 9 & 1 & 11.1\% & P143--P143 \\
27 & tensorflow/addons & 50 & 3 & 6.0\% & P144--P146 \\
28 & tesseract-ocr/tesseract & 46 & 6 & 13.0\% & P147--P152 \\
29 & TuringLang/Turing.jl & 26 & 2 & 7.7\% & P153--P154 \\
30 & zhenghaoz/gorse & 24 & 1 & 4.2\% & P155--P155 \\
31 & apache/incubator-mxnet & 40 & 0 & 0.0\% & --- \\
32 & BLKSerene/Wordless & 3 & 0 & 0.0\% & --- \\
33 & criteo/tf-yarn & 4 & 0 & 0.0\% & --- \\
34 & DandyDev/slack-machine & 4 & 0 & 0.0\% & --- \\
35 & FluxML/Metalhead.jl & 12 & 0 & 0.0\% & --- \\
36 & JohnSnowLabs/spark-nlp & 30 & 0 & 0.0\% & --- \\
37 & jtablesaw/tablesaw & 18 & 0 & 0.0\% & --- \\
38 & microsoft/dowhy & 20 & 0 & 0.0\% & --- \\
39 & mne-tools/mne-cpp & 5 & 0 & 0.0\% & --- \\
40 & OpenKore/openkore & 11 & 0 & 0.0\% & --- \\
41 & OpenNMT/OpenNMT-py & 18 & 0 & 0.0\% & --- \\
42 & pytorch/ignite & 55 & 0 & 0.0\% & --- \\
43 & RubixML/RubixML & 11 & 0 & 0.0\% & --- \\
44 & skorch-dev/skorch & 9 & 0 & 0.0\% & --- \\
45 & smistad/FAST & 4 & 0 & 0.0\% & --- \\
46 & tensorly/tensorly & 16 & 0 & 0.0\% & --- \\
47 & Texera/texera & 24 & 0 & 0.0\% & --- \\
\midrule
& \textbf{Total} & \textbf{5,007} & \textbf{155} & \textbf{3.1\%} & \\
\bottomrule
\end{tabular}  \label{tab:number_responses_per_respondent_type}%
\end{table}%

Our survey is organized into five major sections, as described in Table \ref{tab:survey_structure}. It includes 20 questions, combining 8 closed- and 12 open-ended questions, designed to collect both quantitative and qualitative data. The estimated completion time is approximately 10 minutes.
To ensure relevance and foster more thoughtful responses, we designed 47 unique questionnaires, each tailored to specific characteristics and statistics of an associated project.
For example, in \textsc{Question \#4.3} of the form sent to the practitioners of the project \textit{tesseract-ocr/tesseract}, we asked: \textit{"When analyzing the data of the \textit{tesseract-ocr/tesseract} project, we observed that this project has a median build duration of 124.9 minutes, which is longer than 90\% of the investigated projects of similar size. Do you have any insights into why this project has a longer build duration?"}.
This customization allowed us to provide participants with context-specific data, enabling them to offer richer and more meaningful insights about their respective projects.
A complete example of the survey is available in our online Appendix\footnote{\url{https://zenodo.org/records/14902811}}, which includes the customized questionnaire sent to participants of the \textit{tesseract-ocr/tesseract}\footnote{\url{http://github.com/tesseract-ocr/tesseract}} project.


\begin{table}
\centering
\caption{Survey Structure and Description.}
\label{tab:survey_structure}
\begin{tabular}{p{3cm}p{8cm}}
\toprule
\textbf{Section} & \textbf{Description} \\
\midrule
Participant Information & Collects demographic data and information about participants' experience, including their experience contributing to ML projects, as well as familiarity with CI practices. \\
\midrule
Perceptions about CI Practices & Gathers insights into the challenges and differences when adopting CI practices in ML projects compared to non-ML projects. Focuses on team practices such as maintaining short build durations, frequent commits, and upholding high test coverage. \\
\midrule
Reflection on Previous Findings & Explores participants' views on results from our prior study, focusing on disparities in CI adoption between ML and non-ML projects. The questions explore the underlying factors contributing to differences in build durations, test coverage, and potential strategies for enhancement. \\
\midrule
Project-Specific Analysis & Presents data derived from a specific project (e.g., \textit{tesseract-ocr/tesseract}) to solicit feedback on unique challenges and techniques for enhancing CI practices in a real-world scenario. \\
\midrule
Conclusion and Follow-Up & Allows participants to opt into follow-up interviews, request updates on study findings, and share additional comments. Ensures eligibility for the gift card drawing by confirming survey completion. \\
\bottomrule
\end{tabular}
\end{table}

To encourage participation and a higher response rate, none of the questions in our survey were mandatory. As a result, the number of responses for each question varied, as not all participants answered every question. Responses were marked as ``NA'' (No Answer) if a participant left a question blank.
Table~\ref{tab:response_rates} provides a detailed overview of each survey question, including its description, type (open-ended or close-ended), and corresponding response rates. Close-ended questions generally achieved higher response rates, with several receiving complete responses (e.g., Questions 1.1–1.4 at 100\%). Conversely, open-ended questions exhibited slightly lower response rates, with the lowest being 85.2\% (Question 3.7). This trend indicates that open-ended questions, which typically require more effort and time to answer, may discourage some participants from responding. Nonetheless, the consistently high response rates across all question types highlight strong engagement from the participants.

\begin{table}
\centering
\caption{Survey Questions and Response Rates.}
\label{tab:response_rates}
\begin{tabular}{cp{6cm}lr}
\toprule
\textbf{\#} & \textbf{Question Description} & \textbf{Question Type} & \textbf{Responses (Rate)} \\
\midrule
1.1 & Experience developing software & Close-ended & 155/155 (100\%) \\
1.2 & Experience developing ML projects & Close-ended & 155/155 (100\%) \\
1.3 & Primary roles in ML projects & Close-ended & 155/155 (100\%) \\
1.4 & Familiarity with CI concepts & Close-ended & 155/155 (100\%) \\
2.1 & ML projects strive to incorporate CI practices & Open-ended & 146/155 (94.2\%) \\
2.2 & Challenges or differences when implementing a CI pipeline & Open-ended & 142/155 (91.6\%) \\
2.3 & ML projects commit more frequently & Close-ended & 149/155 (96.1\%) \\
2.4 & ML projects have longer build durations & Close-ended & 151/155 (97.4\%) \\
2.5 & ML projects have lower test coverage & Close-ended & 150/155 (96.8\%) \\
2.6 & ML projects fix broken builds more quickly & Close-ended & 149/155 (96.1\%) \\
3.1 & Importance of ML projects keeping a short build duration & Close-ended & 151/155 (97.4\%) \\
3.2 & Perceptions about previous study results on build duration in ML projects & Open-ended & 141/155 (91\%) \\
3.3 & Strategies to reduce build duration in ML projects & Open-ended & 138/155 (89\%) \\
3.4 & Acceptable test coverage rate for an ML project & Close-ended & 153/155 (98.7\%) \\
3.5 & Perceptions about previous study results on test coverage in ML projects & Open-ended & 135/155 (87.1\%) \\
3.6 & Challenges in testing ML projects & Open-ended & 133/155 (85.8\%) \\
3.7 & Strategies to enhance test coverage in ML projects & Open-ended & 132/155 (85.2\%) \\
4.1 & Familiarity with the CI pipeline of the studied project & Close-ended & 154/155 (99.4\%) \\
4.2 & Acceptable build duration for ML projects & Close-ended & 150/155 (96.8\%) \\
4.3 & Perceptions about the build duration of the studied project & Open-ended & 135/155 (87.1\%) \\
\bottomrule
\end{tabular}
\end{table}


\subsection{Analytical Approach}

We applied an inductive thematic analysis to identify, analyze, and report themes within the qualitative data collected from our questionnaire, following the approach outlined by \cite{braun2006using}. To ensure rigor and transparency in the process, we adhered to the guidelines proposed by \cite{nowell2017thematic}.

The initial step of our thematic analysis involved open-coding the qualitative data. This process refers to assigning codes to relevant segments of data collected from the responses to our open-ended survey questions. Each question was coded by at least two authors, enhancing the robustness of the analysis and mitigating potential bias.
The first author conducted open coding for all eight open-ended questions in the survey, and to ensure reliability in the coding process, the second and third authors coded responses for three questions, and the fourth author coded two questions. 
Afterwards, the fifth author reviewed the entire set of codes generated by the two coders of each question. This review process helped resolve disagreements or ambiguities, refine the coding, and add additional entries where necessary.

Once the coding process was completed, the first author performed axial coding, grouping codes into higher-level themes. These themes represented broader conceptual constructs, organizing multiple related codes under a common idea. For example, a single theme might encompass several related codes addressing a specific aspect of CI practices in ML projects.


% To gain deeper insights into the relationships among the themes identified in our thematic analysis, we conducted an exploratory co-occurrence analysis. This approach examines the frequency with which different themes appear together in participants' responses, uncovering patterns and connections that provide a more nuanced understanding of the data.

Finally, we report the codes and themes derived from our thematic analysis in the results section.
When presenting our findings, we indicate the number of quotes associated with each code and theme using superscripts. 
However, it is important to note that these numbers do not necessarily indicate the relevance or significance of a code. For instance, a code may be cited in more quotes simply because it is more easily remembered by participants, rather than due to its importance.
To provide further context and depth, we include representative quotes from participants. To maintain anonymity, participant names are replaced with unique IDs.

While textual representations highlight key insights, we also employ network mapping charts to provide a structured visual representation of the relationships between themes and codes.
At the center of the network lies the core theme, encapsulating the primary focus of the RQ. Surrounding it are second-level themes, which further break down into third-level themes (codes), organized based on their conceptual relationships. Figure~\ref{fig:network_mapping_chart_example} presents an example of a Network Mapping Chart, illustrating these relationships in the thematic analysis.
Each third-level theme (code) offers granular insights into specific aspects of the data. The thickness of the edges in the network represents frequency, indicating how prominently each code appeared during the analysis.

\begin{figure}[H]
	\centering
	\includegraphics[width=0.7\textwidth]{network_graph_example.png}
	% figure caption is below the figure
	\caption{Example of a Network Mapping Chart visualizing the relationships between themes and codes in thematic analysis.}
	\label{fig:network_mapping_chart_example}       % Give a unique label
\end{figure}


% Each second-level theme is further broken down into third-level themes (or codes), which provide more granular insights into specific aspects of the data. These codes are grouped based on their conceptual or contextual relationship with the higher-level themes. The thickness of the edges in the network reflects the frequency of occurrences, indicating how prominently each code appeared during the analysis.

% To enhance the interpretation of the codes, we include representative quotes from our participants. To maintain anonymity, participants’ names are replaced with unique IDs. 
% For example, when discussing perceived differences and challenges in adopting a CI pipeline in ML projects (RQ1), practitioner 57, who contributed to the \textit{huggingface/pytorch-pretrained-BERT} project (see Table \ref{tab:number_responses_per_respondent_type}), stated: \textit{``ML projects often are stochastic or make statistical guarantees in nature, and require different forms of testing/verification as a result''}. In this instance, the participant's name was replaced with P57. This specific response was coded under \textit{``Non-determinism''} and grouped into the broader theme of \textit{``Testing Complexity''}.
\begin{table*}[!ht]\scriptsize
\centering
\caption{Experiment - Participants demographics}
\label{tab:partdemograph}
\begin{tabular}{cccccccccccccccccccc}
\hline
\multicolumn{10}{c}{\textbf{Plugin Group}} & \multicolumn{10}{c}{\textbf{Control Group}} \\ \hline
\multicolumn{1}{c|}{\multirow{2}{*}{\textbf{ID}}} & \multicolumn{1}{c|}{\multirow{2}{*}{\textbf{Gender}}} & \multicolumn{1}{c|}{\multirow{2}{*}{\textbf{Persona}}} & \multicolumn{2}{c|}{\textbf{Experience}} & \multicolumn{5}{c|}{\textbf{Facets}} & \multicolumn{1}{c|}{\multirow{2}{*}{\textbf{ID}}} & \multicolumn{1}{c|}{\multirow{2}{*}{\textbf{Gender}}} & \multicolumn{1}{c|}{\multirow{2}{*}{\textbf{Persona}}} & \multicolumn{2}{c|}{\textbf{Experience}} & \multicolumn{5}{c}{\textbf{Facets}} \\ \cline{4-10} \cline{14-20} 
\multicolumn{1}{c|}{} & \multicolumn{1}{c|}{} & \multicolumn{1}{c|}{} & \multicolumn{1}{c|}{\textbf{GitHub}} & \multicolumn{1}{c|}{\textbf{OSS}} & \multicolumn{1}{c|}{\textbf{MT}} & \multicolumn{1}{c|}{\textbf{SE}} & \multicolumn{1}{c|}{\textbf{R}} & \multicolumn{1}{c|}{\textbf{IP}} & \multicolumn{1}{c|}{\textbf{L}} & \multicolumn{1}{c|}{} & \multicolumn{1}{c|}{} & \multicolumn{1}{c|}{} & \multicolumn{1}{c|}{\textbf{GitHub}} & \multicolumn{1}{c|}{\textbf{OSS}} & \multicolumn{1}{c|}{\textbf{MT}} & \multicolumn{1}{c|}{\textbf{SE}} & \multicolumn{1}{c|}{\textbf{R}} & \multicolumn{1}{c|}{\textbf{IP}} & \multicolumn{1}{c}{\textbf{L}} \\ \hline \hline

\multicolumn{1}{c|}{P1} & \multicolumn{1}{c|}{M} & \multicolumn{1}{c|}{\tikzcirclenew[fill=blue]{3pt}} & \multicolumn{1}{c|}{Never} & \multicolumn{1}{c|}{No} & \multicolumn{1}{c|}{\tikzcirclenew[fill=blue]{3pt}} & \multicolumn{1}{c|}{\tikzcirclenew[fill=blue]{3pt}} & \multicolumn{1}{c|}{\tikzcirclenew[fill=blue]{3pt}} & \multicolumn{1}{c|}{\tikzcircle[fill=orange]{3pt}} & \multicolumn{1}{c|}{\tikzcirclenew[fill=blue]{3pt}} & \multicolumn{1}{c|}{P40} & \multicolumn{1}{c|}{W} & \multicolumn{1}{c|}{\tikzcircle[fill=orange]{3pt}} & \multicolumn{1}{c|}{Once} & \multicolumn{1}{c|}{No} & \multicolumn{1}{c|}{\tikzcircle[fill=orange]{3pt}} & \multicolumn{1}{c|}{\tikzcirclenew[fill=blue]{3pt}} & \multicolumn{1}{c|}{\tikzcircle[fill=orange]{3pt}} & \multicolumn{1}{c|}{\tikzcircle[fill=orange]{3pt}} & \multicolumn{1}{c}{\tikzcirclenew[fill=blue]{3pt}} \\ \hline

\multicolumn{1}{c|}{P2} & \multicolumn{1}{c|}{W} & \multicolumn{1}{c|}{\tikzcirclenew[fill=blue]{3pt}} & \multicolumn{1}{c|}{Once} & \multicolumn{1}{c|}{No} & \multicolumn{1}{c|}{\tikzcirclenew[fill=blue]{3pt}} & \multicolumn{1}{c|}{\tikzcirclenew[fill=blue]{3pt}} & \multicolumn{1}{c|}{\tikzcirclenew[fill=blue]{3pt}} & \multicolumn{1}{c|}{\tikzcircle[fill=orange]{3pt}} & \multicolumn{1}{c|}{\tikzcirclenew[fill=blue]{3pt}} & \multicolumn{1}{c|}{P41} & \multicolumn{1}{c|}{M} & \multicolumn{1}{c|}{\tikzcirclenew[fill=blue]{3pt}} & \multicolumn{1}{c|}{Once} & \multicolumn{1}{c|}{No} & \multicolumn{1}{c|}{\tikzcirclenew[fill=blue]{3pt}} & \multicolumn{1}{c|}{\tikzcirclenew[fill=blue]{3pt}} & \multicolumn{1}{c|}{\tikzcircle[fill=orange]{3pt}} & \multicolumn{1}{c|}{\tikzcircle[fill=orange]{3pt}} & \multicolumn{1}{c}{\tikzcirclenew[fill=blue]{3pt}} \\ \hline

\multicolumn{1}{c|}{P3} & \multicolumn{1}{c|}{M} & \multicolumn{1}{c|}{\tikzcirclenew[fill=blue]{3pt}} & \multicolumn{1}{c|}{Never} & \multicolumn{1}{c|}{No} & \multicolumn{1}{c|}{\tikzcircle[fill=orange]{3pt}} & \multicolumn{1}{c|}{\tikzcirclenew[fill=blue]{3pt}} & \multicolumn{1}{c|}{\tikzcirclenew[fill=blue]{3pt}} & \multicolumn{1}{c|}{\tikzcircle[fill=orange]{3pt}} & \multicolumn{1}{c|}{\tikzcirclenew[fill=blue]{3pt}} & \multicolumn{1}{c|}{P42} & \multicolumn{1}{c|}{M} & \multicolumn{1}{c|}{\tikzcircle[fill=orange]{3pt}} & \multicolumn{1}{c|}{Never} & \multicolumn{1}{c|}{No} & \multicolumn{1}{c|}{\tikzcircle[fill=orange]{3pt}} & \multicolumn{1}{c|}{\tikzcirclenew[fill=blue]{3pt}} & \multicolumn{1}{c|}{\tikzcircle[fill=orange]{3pt}} & \multicolumn{1}{c|}{\tikzcircle[fill=orange]{3pt}} & \multicolumn{1}{c}{\tikzcirclenew[fill=blue]{3pt}} \\ \hline

\multicolumn{1}{c|}{P4} & \multicolumn{1}{c|}{M} & \multicolumn{1}{c|}{\tikzcirclenew[fill=blue]{3pt}} & \multicolumn{1}{c|}{Never} & \multicolumn{1}{c|}{No} & \multicolumn{1}{c|}{\tikzcircle[fill=orange]{3pt}} & \multicolumn{1}{c|}{\tikzcirclenew[fill=blue]{3pt}} & \multicolumn{1}{c|}{\tikzcirclenew[fill=blue]{3pt}} & \multicolumn{1}{c|}{\tikzcircle[fill=orange]{3pt}} & \multicolumn{1}{c|}{\tikzcirclenew[fill=blue]{3pt}} & \multicolumn{1}{c|}{P43} & \multicolumn{1}{c|}{M} & \multicolumn{1}{c|}{\tikzcirclenew[fill=blue]{3pt}} & \multicolumn{1}{c|}{Never} & \multicolumn{1}{c|}{No} & \multicolumn{1}{c|}{\tikzcirclenew[fill=blue]{3pt}} & \multicolumn{1}{c|}{\tikzcirclenew[fill=blue]{3pt}} & \multicolumn{1}{c|}{\tikzcirclenew[fill=blue]{3pt}} & \multicolumn{1}{c|}{\tikzcircle[fill=orange]{3pt}} & \multicolumn{1}{c}{\tikzcirclenew[fill=blue]{3pt}} \\ \hline

\multicolumn{1}{c|}{P5} & \multicolumn{1}{c|}{M} & \multicolumn{1}{c|}{\tikzcirclenew[fill=blue]{3pt}} & \multicolumn{1}{c|}{Once} & \multicolumn{1}{c|}{No} & \multicolumn{1}{c|}{\tikzcirclenew[fill=blue]{3pt}} & \multicolumn{1}{c|}{\tikzcirclenew[fill=blue]{3pt}} & \multicolumn{1}{c|}{\tikzcirclenew[fill=blue]{3pt}} & \multicolumn{1}{c|}{\tikzcircle[fill=orange]{3pt}} & \multicolumn{1}{c|}{\tikzcirclenew[fill=blue]{3pt}} & \multicolumn{1}{c|}{P44} & \multicolumn{1}{c|}{M} & \multicolumn{1}{c|}{\tikzcirclenew[fill=blue]{3pt}} & \multicolumn{1}{c|}{Never} & \multicolumn{1}{c|}{No} & \multicolumn{1}{c|}{\tikzcirclenew[fill=blue]{3pt}} & \multicolumn{1}{c|}{\tikzcirclenew[fill=blue]{3pt}} & \multicolumn{1}{c|}{\tikzcircle[fill=orange]{3pt}} & \multicolumn{1}{c|}{\tikzcircle[fill=orange]{3pt}} & \multicolumn{1}{c}{\tikzcirclenew[fill=blue]{3pt}} \\ \hline

\multicolumn{1}{c|}{P6} & \multicolumn{1}{c|}{M} & \multicolumn{1}{c|}{\tikzcirclenew[fill=blue]{3pt}} & \multicolumn{1}{c|}{Once} & \multicolumn{1}{c|}{No} & \multicolumn{1}{c|}{\tikzcirclenew[fill=blue]{3pt}} & \multicolumn{1}{c|}{\tikzcirclenew[fill=blue]{3pt}} & \multicolumn{1}{c|}{\tikzcircle[fill=orange]{3pt}} & \multicolumn{1}{c|}{\tikzcircle[fill=orange]{3pt}} & \multicolumn{1}{c|}{\tikzcirclenew[fill=blue]{3pt}} & \multicolumn{1}{c|}{P45} & \multicolumn{1}{c|}{M} & \multicolumn{1}{c|}{\tikzcirclenew[fill=blue]{3pt}} & \multicolumn{1}{c|}{Never} & \multicolumn{1}{c|}{No} & \multicolumn{1}{c|}{\tikzcirclenew[fill=blue]{3pt}} & \multicolumn{1}{c|}{\tikzcirclenew[fill=blue]{3pt}} & \multicolumn{1}{c|}{\tikzcirclenew[fill=blue]{3pt}} & \multicolumn{1}{c|}{\tikzcircle[fill=orange]{3pt}} & \multicolumn{1}{c}{\tikzcirclenew[fill=blue]{3pt}} \\ \hline

\multicolumn{1}{c|}{P7} & \multicolumn{1}{c|}{W} & \multicolumn{1}{c|}{\tikzcircle[fill=orange]{3pt}} & \multicolumn{1}{c|}{Never} & \multicolumn{1}{c|}{No} & \multicolumn{1}{c|}{\tikzcircle[fill=orange]{3pt}} & \multicolumn{1}{c|}{\tikzcirclenew[fill=blue]{3pt}} & \multicolumn{1}{c|}{\tikzcircle[fill=orange]{3pt}} & \multicolumn{1}{c|}{\tikzcircle[fill=orange]{3pt}} & \multicolumn{1}{c|}{\tikzcirclenew[fill=blue]{3pt}} & \multicolumn{1}{c|}{P46} & \multicolumn{1}{c|}{M} & \multicolumn{1}{c|}{\tikzcircle[fill=orange]{3pt}} & \multicolumn{1}{c|}{Never} & \multicolumn{1}{c|}{No} & \multicolumn{1}{c|}{\tikzcirclenew[fill=blue]{3pt}} & \multicolumn{1}{c|}{\tikzcircle[fill=orange]{3pt}} & \multicolumn{1}{c|}{\tikzcircle[fill=orange]{3pt}} & \multicolumn{1}{c|}{\tikzcircle[fill=orange]{3pt}} & \multicolumn{1}{c}{\tikzcirclenew[fill=blue]{3pt}} \\ \hline

\multicolumn{1}{c|}{P8} & \multicolumn{1}{c|}{W} & \multicolumn{1}{c|}{\tikzcircle[fill=orange]{3pt}} & \multicolumn{1}{c|}{Never} & \multicolumn{1}{c|}{No} & \multicolumn{1}{c|}{\tikzcirclenew[fill=blue]{3pt}} & \multicolumn{1}{c|}{\tikzcircle[fill=orange]{3pt}} & \multicolumn{1}{c|}{\tikzcircle[fill=orange]{3pt}} & \multicolumn{1}{c|}{\tikzcircle[fill=orange]{3pt}} & \multicolumn{1}{c|}{\tikzcirclenew[fill=blue]{3pt}} & \multicolumn{1}{c|}{P47} & \multicolumn{1}{c|}{M} & \multicolumn{1}{c|}{\tikzcirclenew[fill=blue]{3pt}} & \multicolumn{1}{c|}{Never} & \multicolumn{1}{c|}{No} & \multicolumn{1}{c|}{\tikzcirclenew[fill=blue]{3pt}} & \multicolumn{1}{c|}{\tikzcirclenew[fill=blue]{3pt}} & \multicolumn{1}{c|}{\tikzcirclenew[fill=blue]{3pt}} & \multicolumn{1}{c|}{\tikzcircle[fill=orange]{3pt}} & \multicolumn{1}{c}{\tikzcirclenew[fill=blue]{3pt}} \\ \hline

\multicolumn{1}{c|}{P9} & \multicolumn{1}{c|}{M} & \multicolumn{1}{c|}{\tikzcirclenew[fill=blue]{3pt}} & \multicolumn{1}{c|}{Once} & \multicolumn{1}{c|}{No} & \multicolumn{1}{c|}{\tikzcirclenew[fill=blue]{3pt}} & \multicolumn{1}{c|}{\tikzcirclenew[fill=blue]{3pt}} & \multicolumn{1}{c|}{\tikzcirclenew[fill=blue]{3pt}} & \multicolumn{1}{c|}{\tikzcircle[fill=orange]{3pt}} & \multicolumn{1}{c|}{\tikzcirclenew[fill=blue]{3pt}} & \multicolumn{1}{c|}{P48} & \multicolumn{1}{c|}{M} & \multicolumn{1}{c|}{\tikzcirclenew[fill=blue]{3pt}} & \multicolumn{1}{c|}{Never} & \multicolumn{1}{c|}{No} & \multicolumn{1}{c|}{\tikzcircle[fill=orange]{3pt}} & \multicolumn{1}{c|}{\tikzcirclenew[fill=blue]{3pt}} & \multicolumn{1}{c|}{\tikzcirclenew[fill=blue]{3pt}} & \multicolumn{1}{c|}{\tikzcircle[fill=orange]{3pt}} & \multicolumn{1}{c}{\tikzcirclenew[fill=blue]{3pt}} \\ \hline

\multicolumn{1}{c|}{P10} & \multicolumn{1}{c|}{W} & \multicolumn{1}{c|}{\tikzcirclenew[fill=blue]{3pt}} & \multicolumn{1}{c|}{Never} & \multicolumn{1}{c|}{No} & \multicolumn{1}{c|}{\tikzcirclenew[fill=blue]{3pt}} & \multicolumn{1}{c|}{\tikzcirclenew[fill=blue]{3pt}} & \multicolumn{1}{c|}{\tikzcircle[fill=orange]{3pt}} & \multicolumn{1}{c|}{\tikzcircle[fill=orange]{3pt}} & \multicolumn{1}{c|}{\tikzcirclenew[fill=blue]{3pt}} & \multicolumn{1}{c|}{P49} & \multicolumn{1}{c|}{M} & \multicolumn{1}{c|}{\tikzcircle[fill=orange]{3pt}} & \multicolumn{1}{c|}{Never} & \multicolumn{1}{c|}{No} & \multicolumn{1}{c|}{\tikzcircle[fill=orange]{3pt}} & \multicolumn{1}{c|}{\tikzcirclenew[fill=blue]{3pt}} & \multicolumn{1}{c|}{\tikzcircle[fill=orange]{3pt}} & \multicolumn{1}{c|}{\tikzcircle[fill=orange]{3pt}} & \multicolumn{1}{c}{\tikzcirclenew[fill=blue]{3pt}} \\ \hline

\multicolumn{1}{c|}{P11} & \multicolumn{1}{c|}{M} & \multicolumn{1}{c|}{\tikzcircle[fill=orange]{3pt}} & \multicolumn{1}{c|}{Never} & \multicolumn{1}{c|}{Some} & \multicolumn{1}{c|}{\tikzcircle[fill=orange]{3pt}} & \multicolumn{1}{c|}{\tikzcircle[fill=orange]{3pt}} & \multicolumn{1}{c|}{\tikzcircle[fill=orange]{3pt}} & \multicolumn{1}{c|}{\tikzcircle[fill=orange]{3pt}} & \multicolumn{1}{c|}{\tikzcircle[fill=orange]{3pt}} & \multicolumn{1}{c|}{P50} & \multicolumn{1}{c|}{M} & \multicolumn{1}{c|}{\tikzcirclenew[fill=blue]{3pt}} & \multicolumn{1}{c|}{Never} & \multicolumn{1}{c|}{No} & \multicolumn{1}{c|}{\tikzcirclenew[fill=blue]{3pt}} & \multicolumn{1}{c|}{\tikzcirclenew[fill=blue]{3pt}} & \multicolumn{1}{c|}{\tikzcirclenew[fill=blue]{3pt}} & \multicolumn{1}{c|}{\tikzcircle[fill=orange]{3pt}} & \multicolumn{1}{c}{\tikzcirclenew[fill=blue]{3pt}} \\ \hline

\multicolumn{1}{c|}{P12} & \multicolumn{1}{c|}{M} & \multicolumn{1}{c|}{\tikzcirclenew[fill=blue]{3pt}} & \multicolumn{1}{c|}{Never} & \multicolumn{1}{c|}{No} & \multicolumn{1}{c|}{\tikzcirclenew[fill=blue]{3pt}} & \multicolumn{1}{c|}{\tikzcirclenew[fill=blue]{3pt}} & \multicolumn{1}{c|}{\tikzcirclenew[fill=blue]{3pt}} & \multicolumn{1}{c|}{\tikzcircle[fill=orange]{3pt}} & \multicolumn{1}{c|}{\tikzcirclenew[fill=blue]{3pt}} & \multicolumn{1}{c|}{P51} & \multicolumn{1}{c|}{M} & \multicolumn{1}{c|}{\tikzcirclenew[fill=blue]{3pt}} & \multicolumn{1}{c|}{Never} & \multicolumn{1}{c|}{No} & \multicolumn{1}{c|}{\tikzcirclenew[fill=blue]{3pt}} & \multicolumn{1}{c|}{\tikzcirclenew[fill=blue]{3pt}} & \multicolumn{1}{c|}{\tikzcirclenew[fill=blue]{3pt}} & \multicolumn{1}{c|}{\tikzcircle[fill=orange]{3pt}} & \multicolumn{1}{c}{\tikzcirclenew[fill=blue]{3pt}} \\ \hline

\multicolumn{1}{c|}{P13} & \multicolumn{1}{c|}{W} & \multicolumn{1}{c|}{\tikzcircle[fill=orange]{3pt}} & \multicolumn{1}{c|}{Once} & \multicolumn{1}{c|}{No} & \multicolumn{1}{c|}{\tikzcircle[fill=orange]{3pt}} & \multicolumn{1}{c|}{\tikzcircle[fill=orange]{3pt}} & \multicolumn{1}{c|}{\tikzcircle[fill=orange]{3pt}} & \multicolumn{1}{c|}{\tikzcircle[fill=orange]{3pt}} & \multicolumn{1}{c|}{\tikzcircle[fill=orange]{3pt}} & \multicolumn{1}{c|}{P52} & \multicolumn{1}{c|}{M} & \multicolumn{1}{c|}{\tikzcircle[fill=orange]{3pt}} & \multicolumn{1}{c|}{Never} & \multicolumn{1}{c|}{Some} & \multicolumn{1}{c|}{\tikzcircle[fill=orange]{3pt}} & \multicolumn{1}{c|}{\tikzcircle[fill=orange]{3pt}} & \multicolumn{1}{c|}{\tikzcirclenew[fill=blue]{3pt}} & \multicolumn{1}{c|}{\tikzcircle[fill=orange]{3pt}} & \multicolumn{1}{c}{\tikzcircle[fill=orange]{3pt}} \\ \hline

\multicolumn{1}{c|}{P14} & \multicolumn{1}{c|}{W} & \multicolumn{1}{c|}{\tikzcircle[fill=orange]{3pt}} & \multicolumn{1}{c|}{Never} & \multicolumn{1}{c|}{Some} & \multicolumn{1}{c|}{\tikzcirclenew[fill=blue]{3pt}} & \multicolumn{1}{c|}{\tikzcirclenew[fill=blue]{3pt}} & \multicolumn{1}{c|}{\tikzcircle[fill=orange]{3pt}} & \multicolumn{1}{c|}{\tikzcircle[fill=orange]{3pt}} & \multicolumn{1}{c|}{\tikzcircle[fill=orange]{3pt}} & \multicolumn{1}{c|}{P53} & \multicolumn{1}{c|}{M} & \multicolumn{1}{c|}{\tikzcircle[fill=orange]{3pt}} & \multicolumn{1}{c|}{Once} & \multicolumn{1}{c|}{No} & \multicolumn{1}{c|}{\tikzcircle[fill=orange]{3pt}} & \multicolumn{1}{c|}{\tikzcircle[fill=orange]{3pt}} & \multicolumn{1}{c|}{\tikzcirclenew[fill=blue]{3pt}} & \multicolumn{1}{c|}{\tikzcircle[fill=orange]{3pt}} & \multicolumn{1}{c}{\tikzcircle[fill=orange]{3pt}} \\ \hline

\multicolumn{1}{c|}{P15} & \multicolumn{1}{c|}{M} & \multicolumn{1}{c|}{\tikzcircle[fill=orange]{3pt}} & \multicolumn{1}{c|}{Never} & \multicolumn{1}{c|}{No} & \multicolumn{1}{c|}{\tikzcircle[fill=orange]{3pt}} & \multicolumn{1}{c|}{\tikzcirclenew[fill=blue]{3pt}} & \multicolumn{1}{c|}{\tikzcirclenew[fill=blue]{3pt}} & \multicolumn{1}{c|}{\tikzcircle[fill=orange]{3pt}} & \multicolumn{1}{c|}{\tikzcircle[fill=orange]{3pt}} & \multicolumn{1}{c|}{P54} & \multicolumn{1}{c|}{W} & \multicolumn{1}{c|}{\tikzcircle[fill=orange]{3pt}} & \multicolumn{1}{c|}{Never} & \multicolumn{1}{c|}{No} & \multicolumn{1}{c|}{\tikzcircle[fill=orange]{3pt}} & \multicolumn{1}{c|}{\tikzcirclenew[fill=blue]{3pt}} & \multicolumn{1}{c|}{\tikzcircle[fill=orange]{3pt}} & \multicolumn{1}{c|}{\tikzcircle[fill=orange]{3pt}} & \multicolumn{1}{c}{\tikzcircle[fill=orange]{3pt}} \\ \hline

\multicolumn{1}{c|}{P16} & \multicolumn{1}{c|}{M} & \multicolumn{1}{c|}{\tikzcircle[fill=orange]{3pt}} & \multicolumn{1}{c|}{Once} & \multicolumn{1}{c|}{No} & \multicolumn{1}{c|}{\tikzcirclenew[fill=blue]{3pt}} & \multicolumn{1}{c|}{\tikzcircle[fill=orange]{3pt}} & \multicolumn{1}{c|}{\tikzcirclenew[fill=blue]{3pt}} & \multicolumn{1}{c|}{\tikzcircle[fill=orange]{3pt}} & \multicolumn{1}{c|}{\tikzcircle[fill=orange]{3pt}} & \multicolumn{1}{c|}{P55} & \multicolumn{1}{c|}{W} & \multicolumn{1}{c|}{\tikzcircle[fill=orange]{3pt}} & \multicolumn{1}{c|}{Once} & \multicolumn{1}{c|}{No} & \multicolumn{1}{c|}{\tikzcirclenew[fill=blue]{3pt}} & \multicolumn{1}{c|}{\tikzcircle[fill=orange]{3pt}} & \multicolumn{1}{c|}{\tikzcircle[fill=orange]{3pt}} & \multicolumn{1}{c|}{\tikzcircle[fill=orange]{3pt}} & \multicolumn{1}{c}{\tikzcircle[fill=orange]{3pt}} \\ \hline

\multicolumn{1}{c|}{P17} & \multicolumn{1}{c|}{W} & \multicolumn{1}{c|}{\tikzcircle[fill=orange]{3pt}} & \multicolumn{1}{c|}{Once} & \multicolumn{1}{c|}{No} & \multicolumn{1}{c|}{\tikzcircle[fill=orange]{3pt}} & \multicolumn{1}{c|}{\tikzcirclenew[fill=blue]{3pt}} & \multicolumn{1}{c|}{\tikzcircle[fill=orange]{3pt}} & \multicolumn{1}{c|}{\tikzcircle[fill=orange]{3pt}} & \multicolumn{1}{c|}{\tikzcircle[fill=orange]{3pt}} & \multicolumn{1}{c|}{P56} & \multicolumn{1}{c|}{M} & \multicolumn{1}{c|}{\tikzcircle[fill=orange]{3pt}} & \multicolumn{1}{c|}{Never} & \multicolumn{1}{c|}{No} & \multicolumn{1}{c|}{\tikzcircle[fill=orange]{3pt}} & \multicolumn{1}{c|}{\tikzcircle[fill=orange]{3pt}} & \multicolumn{1}{c|}{\tikzcirclenew[fill=blue]{3pt}} & \multicolumn{1}{c|}{\tikzcircle[fill=orange]{3pt}} & \multicolumn{1}{c}{\tikzcircle[fill=orange]{3pt}} \\ \hline

\multicolumn{1}{c|}{P18} & \multicolumn{1}{c|}{W} & \multicolumn{1}{c|}{\tikzcirclenew[fill=blue]{3pt}} & \multicolumn{1}{c|}{Once} & \multicolumn{1}{c|}{No} & \multicolumn{1}{c|}{\tikzcirclenew[fill=blue]{3pt}} & \multicolumn{1}{c|}{\tikzcirclenew[fill=blue]{3pt}} & \multicolumn{1}{c|}{\tikzcirclenew[fill=blue]{3pt}} & \multicolumn{1}{c|}{\tikzcircle[fill=orange]{3pt}} & \multicolumn{1}{c|}{\tikzcirclenew[fill=blue]{3pt}} & \multicolumn{1}{c|}{P57} & \multicolumn{1}{c|}{M} & \multicolumn{1}{c|}{\tikzcircle[fill=orange]{3pt}} & \multicolumn{1}{c|}{Few times} & \multicolumn{1}{c|}{No} & \multicolumn{1}{c|}{\tikzcircle[fill=orange]{3pt}} & \multicolumn{1}{c|}{\tikzcirclenew[fill=blue]{3pt}} & \multicolumn{1}{c|}{\tikzcircle[fill=orange]{3pt}} & \multicolumn{1}{c|}{\tikzcircle[fill=orange]{3pt}} & \multicolumn{1}{c}{\tikzcirclenew[fill=blue]{3pt}} \\ \hline

\multicolumn{1}{c|}{P19} & \multicolumn{1}{c|}{M} & \multicolumn{1}{c|}{\tikzcirclenew[fill=blue]{3pt}} & \multicolumn{1}{c|}{Never} & \multicolumn{1}{c|}{No} & \multicolumn{1}{c|}{\tikzcirclenew[fill=blue]{3pt}} & \multicolumn{1}{c|}{\tikzcirclenew[fill=blue]{3pt}} & \multicolumn{1}{c|}{\tikzcirclenew[fill=blue]{3pt}} & \multicolumn{1}{c|}{\tikzcircle[fill=orange]{3pt}} & \multicolumn{1}{c|}{\tikzcirclenew[fill=blue]{3pt}} & \multicolumn{1}{c|}{P58} & \multicolumn{1}{c|}{M} & \multicolumn{1}{c|}{\tikzcircle[fill=orange]{3pt}} & \multicolumn{1}{c|}{Once} & \multicolumn{1}{c|}{Some} & \multicolumn{1}{c|}{\tikzcircle[fill=orange]{3pt}} & \multicolumn{1}{c|}{\tikzcircle[fill=orange]{3pt}} & \multicolumn{1}{c|}{\tikzcirclenew[fill=blue]{3pt}} & \multicolumn{1}{c|}{\tikzcirclenew[fill=blue]{3pt}} & \multicolumn{1}{c}{\tikzcircle[fill=orange]{3pt}} \\ \hline

\multicolumn{1}{c|}{P20} & \multicolumn{1}{c|}{M} & \multicolumn{1}{c|}{\tikzcircle[fill=orange]{3pt}} & \multicolumn{1}{c|}{Few times} & \multicolumn{1}{c|}{No} & \multicolumn{1}{c|}{\tikzcircle[fill=orange]{3pt}} & \multicolumn{1}{c|}{\tikzcirclenew[fill=blue]{3pt}} & \multicolumn{1}{c|}{\tikzcircle[fill=orange]{3pt}} & \multicolumn{1}{c|}{\tikzcircle[fill=orange]{3pt}} & \multicolumn{1}{c|}{\tikzcircle[fill=orange]{3pt}} & \multicolumn{1}{c|}{P59} & \multicolumn{1}{c|}{M} & \multicolumn{1}{c|}{\tikzcirclenew[fill=blue]{3pt}} & \multicolumn{1}{c|}{Never} & \multicolumn{1}{c|}{No} & \multicolumn{1}{c|}{\tikzcirclenew[fill=blue]{3pt}} & \multicolumn{1}{c|}{\tikzcirclenew[fill=blue]{3pt}} & \multicolumn{1}{c|}{\tikzcirclenew[fill=blue]{3pt}} & \multicolumn{1}{c|}{\tikzcirclenew[fill=blue]{3pt}} & \multicolumn{1}{c}{\tikzcirclenew[fill=blue]{3pt}} \\ \hline

\multicolumn{1}{c|}{P21} & \multicolumn{1}{c|}{M} & \multicolumn{1}{c|}{\tikzcirclenew[fill=blue]{3pt}} & \multicolumn{1}{c|}{Often} & \multicolumn{1}{c|}{No} & \multicolumn{1}{c|}{\tikzcirclenew[fill=blue]{3pt}} & \multicolumn{1}{c|}{\tikzcirclenew[fill=blue]{3pt}} & \multicolumn{1}{c|}{\tikzcircle[fill=orange]{3pt}} & \multicolumn{1}{c|}{\tikzcircle[fill=orange]{3pt}} & \multicolumn{1}{c|}{\tikzcirclenew[fill=blue]{3pt}} & \multicolumn{1}{c|}{P60} & \multicolumn{1}{c|}{M} & \multicolumn{1}{c|}{\tikzcirclenew[fill=blue]{3pt}} & \multicolumn{1}{c|}{Once} & \multicolumn{1}{c|}{No} & \multicolumn{1}{c|}{\tikzcirclenew[fill=blue]{3pt}} & \multicolumn{1}{c|}{\tikzcircle[fill=orange]{3pt}} & \multicolumn{1}{c|}{\tikzcirclenew[fill=blue]{3pt}} & \multicolumn{1}{c|}{\tikzcirclenew[fill=blue]{3pt}} & \multicolumn{1}{c}{\tikzcirclenew[fill=blue]{3pt}} \\ \hline

\multicolumn{1}{c|}{P22} & \multicolumn{1}{c|}{W} & \multicolumn{1}{c|}{\tikzcirclenew[fill=blue]{3pt}} & \multicolumn{1}{c|}{Once} & \multicolumn{1}{c|}{No} & \multicolumn{1}{c|}{\tikzcirclenew[fill=blue]{3pt}} & \multicolumn{1}{c|}{\tikzcirclenew[fill=blue]{3pt}} & \multicolumn{1}{c|}{\tikzcirclenew[fill=blue]{3pt}} & \multicolumn{1}{c|}{\tikzcircle[fill=orange]{3pt}} & \multicolumn{1}{c|}{\tikzcircle[fill=orange]{3pt}} & \multicolumn{1}{c|}{P61} & \multicolumn{1}{c|}{M} & \multicolumn{1}{c|}{\tikzcircle[fill=orange]{3pt}} & \multicolumn{1}{c|}{Few times} & \multicolumn{1}{c|}{No} & \multicolumn{1}{c|}{\tikzcircle[fill=orange]{3pt}} & \multicolumn{1}{c|}{\tikzcircle[fill=orange]{3pt}} & \multicolumn{1}{c|}{\tikzcirclenew[fill=blue]{3pt}} & \multicolumn{1}{c|}{\tikzcirclenew[fill=blue]{3pt}} & \multicolumn{1}{c}{\tikzcircle[fill=orange]{3pt}} \\ \hline

\multicolumn{1}{c|}{P23} & \multicolumn{1}{c|}{W} & \multicolumn{1}{c|}{\tikzcirclenew[fill=blue]{3pt}} & \multicolumn{1}{c|}{Often} & \multicolumn{1}{c|}{No} & \multicolumn{1}{c|}{\tikzcirclenew[fill=blue]{3pt}} & \multicolumn{1}{c|}{\tikzcirclenew[fill=blue]{3pt}} & \multicolumn{1}{c|}{\tikzcirclenew[fill=blue]{3pt}} & \multicolumn{1}{c|}{\tikzcircle[fill=orange]{3pt}} & \multicolumn{1}{c|}{\tikzcirclenew[fill=blue]{3pt}} & \multicolumn{1}{c|}{P62} & \multicolumn{1}{c|}{M} & \multicolumn{1}{c|}{\tikzcircle[fill=orange]{3pt}} & \multicolumn{1}{c|}{Few times} & \multicolumn{1}{c|}{No} & \multicolumn{1}{c|}{\tikzcircle[fill=orange]{3pt}} & \multicolumn{1}{c|}{\tikzcirclenew[fill=blue]{3pt}} & \multicolumn{1}{c|}{\tikzcircle[fill=orange]{3pt}} & \multicolumn{1}{c|}{\tikzcircle[fill=orange]{3pt}} & \multicolumn{1}{c}{\tikzcircle[fill=orange]{3pt}} \\ \hline

\multicolumn{1}{c|}{P24} & \multicolumn{1}{c|}{M} & \multicolumn{1}{c|}{\tikzcirclenew[fill=blue]{3pt}} & \multicolumn{1}{c|}{Often} & \multicolumn{1}{c|}{No} & \multicolumn{1}{c|}{\tikzcirclenew[fill=blue]{3pt}} & \multicolumn{1}{c|}{\tikzcircle[fill=orange]{3pt}} & \multicolumn{1}{c|}{\tikzcirclenew[fill=blue]{3pt}} & \multicolumn{1}{c|}{\tikzcircle[fill=orange]{3pt}} & \multicolumn{1}{c|}{\tikzcirclenew[fill=blue]{3pt}} & \multicolumn{1}{c|}{P63} & \multicolumn{1}{c|}{W} & \multicolumn{1}{c|}{\tikzcirclenew[fill=blue]{3pt}} & \multicolumn{1}{c|}{Never} & \multicolumn{1}{c|}{No} & \multicolumn{1}{c|}{\tikzcircle[fill=orange]{3pt}} & \multicolumn{1}{c|}{\tikzcirclenew[fill=blue]{3pt}} & \multicolumn{1}{c|}{\tikzcirclenew[fill=blue]{3pt}} & \multicolumn{1}{c|}{\tikzcircle[fill=orange]{3pt}} & \multicolumn{1}{c}{\tikzcirclenew[fill=blue]{3pt}} \\ \hline

\multicolumn{1}{c|}{P25} & \multicolumn{1}{c|}{M} & \multicolumn{1}{c|}{\tikzcirclenew[fill=blue]{3pt}} & \multicolumn{1}{c|}{Often} & \multicolumn{1}{c|}{No} & \multicolumn{1}{c|}{\tikzcircle[fill=orange]{3pt}} & \multicolumn{1}{c|}{\tikzcirclenew[fill=blue]{3pt}} & \multicolumn{1}{c|}{\tikzcirclenew[fill=blue]{3pt}} & \multicolumn{1}{c|}{\tikzcirclenew[fill=blue]{3pt}} & \multicolumn{1}{c|}{\tikzcircle[fill=orange]{3pt}} & \multicolumn{1}{c|}{P64} & \multicolumn{1}{c|}{W} & \multicolumn{1}{c|}{\tikzcirclenew[fill=blue]{3pt}} & \multicolumn{1}{c|}{Never} & \multicolumn{1}{c|}{No} & \multicolumn{1}{c|}{\tikzcirclenew[fill=blue]{3pt}} & \multicolumn{1}{c|}{\tikzcirclenew[fill=blue]{3pt}} & \multicolumn{1}{c|}{\tikzcirclenew[fill=blue]{3pt}} & \multicolumn{1}{c|}{\tikzcircle[fill=orange]{3pt}} & \multicolumn{1}{c}{\tikzcirclenew[fill=blue]{3pt}} \\ \hline

\multicolumn{1}{c|}{P26} & \multicolumn{1}{c|}{W} & \multicolumn{1}{c|}{\tikzcircle[fill=orange]{3pt}} & \multicolumn{1}{c|}{Few times} & \multicolumn{1}{c|}{No} & \multicolumn{1}{c|}{\tikzcircle[fill=orange]{3pt}} & \multicolumn{1}{c|}{\tikzcircle[fill=orange]{3pt}} & \multicolumn{1}{c|}{\tikzcirclenew[fill=blue]{3pt}} & \multicolumn{1}{c|}{\tikzcircle[fill=orange]{3pt}} & \multicolumn{1}{c|}{\tikzcircle[fill=orange]{3pt}} & \multicolumn{1}{c|}{P65} & \multicolumn{1}{c|}{M} & \multicolumn{1}{c|}{\tikzcirclenew[fill=blue]{3pt}} & \multicolumn{1}{c|}{Never} & \multicolumn{1}{c|}{No} & \multicolumn{1}{c|}{\tikzcircle[fill=orange]{3pt}} & \multicolumn{1}{c|}{\tikzcirclenew[fill=blue]{3pt}} & \multicolumn{1}{c|}{\tikzcirclenew[fill=blue]{3pt}} & \multicolumn{1}{c|}{\tikzcircle[fill=orange]{3pt}} & \multicolumn{1}{c}{\tikzcirclenew[fill=blue]{3pt}} \\ \hline

\multicolumn{1}{c|}{P27} & \multicolumn{1}{c|}{M} & \multicolumn{1}{c|}{\tikzcircle[fill=orange]{3pt}} & \multicolumn{1}{c|}{Few times} & \multicolumn{1}{c|}{No} & \multicolumn{1}{c|}{\tikzcircle[fill=orange]{3pt}} & \multicolumn{1}{c|}{\tikzcirclenew[fill=blue]{3pt}} & \multicolumn{1}{c|}{\tikzcircle[fill=orange]{3pt}} & \multicolumn{1}{c|}{\tikzcircle[fill=orange]{3pt}} & \multicolumn{1}{c|}{\tikzcircle[fill=orange]{3pt}} & \multicolumn{1}{c|}{P66} & \multicolumn{1}{c|}{W} & \multicolumn{1}{c|}{\tikzcirclenew[fill=blue]{3pt}} & \multicolumn{1}{c|}{Never} & \multicolumn{1}{c|}{No} & \multicolumn{1}{c|}{\tikzcircle[fill=orange]{3pt}} & \multicolumn{1}{c|}{\tikzcirclenew[fill=blue]{3pt}} & \multicolumn{1}{c|}{\tikzcirclenew[fill=blue]{3pt}} & \multicolumn{1}{c|}{\tikzcircle[fill=orange]{3pt}} & \multicolumn{1}{c}{\tikzcirclenew[fill=blue]{3pt}} \\ \hline

\multicolumn{1}{c|}{P28} & \multicolumn{1}{c|}{M} & \multicolumn{1}{c|}{\tikzcircle[fill=orange]{3pt}} & \multicolumn{1}{c|}{Once} & \multicolumn{1}{c|}{No} & \multicolumn{1}{c|}{\tikzcircle[fill=orange]{3pt}} & \multicolumn{1}{c|}{\tikzcirclenew[fill=blue]{3pt}} & \multicolumn{1}{c|}{\tikzcirclenew[fill=blue]{3pt}} & \multicolumn{1}{c|}{\tikzcircle[fill=orange]{3pt}} & \multicolumn{1}{c|}{\tikzcircle[fill=orange]{3pt}} & \multicolumn{1}{c|}{P67} & \multicolumn{1}{c|}{M} & \multicolumn{1}{c|}{\tikzcircle[fill=orange]{3pt}} & \multicolumn{1}{c|}{Never} & \multicolumn{1}{c|}{No} & \multicolumn{1}{c|}{\tikzcircle[fill=orange]{3pt}} & \multicolumn{1}{c|}{\tikzcircle[fill=orange]{3pt}} & \multicolumn{1}{c|}{\tikzcirclenew[fill=blue]{3pt}} & \multicolumn{1}{c|}{\tikzcircle[fill=orange]{3pt}} & \multicolumn{1}{c}{\tikzcircle[fill=orange]{3pt}} \\ \hline

\multicolumn{1}{c|}{P29} & \multicolumn{1}{c|}{M} & \multicolumn{1}{c|}{\tikzcircle[fill=orange]{3pt}} & \multicolumn{1}{c|}{Never} & \multicolumn{1}{c|}{No} & \multicolumn{1}{c|}{\tikzcircle[fill=orange]{3pt}} & \multicolumn{1}{c|}{\tikzcircle[fill=orange]{3pt}} & \multicolumn{1}{c|}{\tikzcircle[fill=orange]{3pt}} & \multicolumn{1}{c|}{\tikzcircle[fill=orange]{3pt}} & \multicolumn{1}{c|}{\tikzcircle[fill=orange]{3pt}} & \multicolumn{1}{c|}{P68} & \multicolumn{1}{c|}{M} & \multicolumn{1}{c|}{\tikzcirclenew[fill=blue]{3pt}} & \multicolumn{1}{c|}{Few times} & \multicolumn{1}{c|}{No} & \multicolumn{1}{c|}{\tikzcirclenew[fill=blue]{3pt}} & \multicolumn{1}{c|}{\tikzcirclenew[fill=blue]{3pt}} & \multicolumn{1}{c|}{\tikzcirclenew[fill=blue]{3pt}} & \multicolumn{1}{c|}{\tikzcircle[fill=orange]{3pt}} & \multicolumn{1}{c}{\tikzcircle[fill=orange]{3pt}} \\ \hline

\multicolumn{1}{c|}{P30} & \multicolumn{1}{c|}{M} & \multicolumn{1}{c|}{\tikzcirclenew[fill=blue]{3pt}} & \multicolumn{1}{c|}{Few times} & \multicolumn{1}{c|}{No} & \multicolumn{1}{c|}{\tikzcirclenew[fill=blue]{3pt}} & \multicolumn{1}{c|}{\tikzcirclenew[fill=blue]{3pt}} & \multicolumn{1}{c|}{\tikzcirclenew[fill=blue]{3pt}} & \multicolumn{1}{c|}{\tikzcircle[fill=orange]{3pt}} & \multicolumn{1}{c|}{\tikzcirclenew[fill=blue]{3pt}} & \multicolumn{1}{c|}{P69} & \multicolumn{1}{c|}{M} & \multicolumn{1}{c|}{\tikzcirclenew[fill=blue]{3pt}} & \multicolumn{1}{c|}{Few times} & \multicolumn{1}{c|}{No} & \multicolumn{1}{c|}{\tikzcirclenew[fill=blue]{3pt}} & \multicolumn{1}{c|}{\tikzcirclenew[fill=blue]{3pt}} & \multicolumn{1}{c|}{\tikzcirclenew[fill=blue]{3pt}} & \multicolumn{1}{c|}{\tikzcirclenew[fill=blue]{3pt}} & \multicolumn{1}{c}{\tikzcircle[fill=orange]{3pt}} \\ \hline

\multicolumn{1}{c|}{P31} & \multicolumn{1}{c|}{M} & \multicolumn{1}{c|}{\tikzcirclenew[fill=blue]{3pt}} & \multicolumn{1}{c|}{Never} & \multicolumn{1}{c|}{No} & \multicolumn{1}{c|}{\tikzcircle[fill=orange]{3pt}} & \multicolumn{1}{c|}{\tikzcirclenew[fill=blue]{3pt}} & \multicolumn{1}{c|}{\tikzcirclenew[fill=blue]{3pt}} & \multicolumn{1}{c|}{\tikzcirclenew[fill=blue]{3pt}} & \multicolumn{1}{c|}{\tikzcirclenew[fill=blue]{3pt}} & \multicolumn{1}{c|}{P70} & \multicolumn{1}{c|}{M} & \multicolumn{1}{c|}{\tikzcirclenew[fill=blue]{3pt}} & \multicolumn{1}{c|}{Few times} & \multicolumn{1}{c|}{Some} & \multicolumn{1}{c|}{\tikzcirclenew[fill=blue]{3pt}} & \multicolumn{1}{c|}{\tikzcirclenew[fill=blue]{3pt}} & \multicolumn{1}{c|}{\tikzcirclenew[fill=blue]{3pt}} & \multicolumn{1}{c|}{\tikzcircle[fill=orange]{3pt}} & \multicolumn{1}{c}{\tikzcirclenew[fill=blue]{3pt}} \\ \hline

\multicolumn{1}{c|}{P32} & \multicolumn{1}{c|}{M} & \multicolumn{1}{c|}{\tikzcircle[fill=orange]{3pt}} & \multicolumn{1}{c|}{Never} & \multicolumn{1}{c|}{No} & \multicolumn{1}{c|}{\tikzcirclenew[fill=blue]{3pt}} & \multicolumn{1}{c|}{\tikzcirclenew[fill=blue]{3pt}} & \multicolumn{1}{c|}{\tikzcircle[fill=orange]{3pt}} & \multicolumn{1}{c|}{\tikzcircle[fill=orange]{3pt}} & \multicolumn{1}{c|}{\tikzcircle[fill=orange]{3pt}} & \multicolumn{1}{c|}{P71} & \multicolumn{1}{c|}{M} & \multicolumn{1}{c|}{\tikzcircle[fill=orange]{3pt}} & \multicolumn{1}{c|}{Few times} & \multicolumn{1}{c|}{No} & \multicolumn{1}{c|}{\tikzcircle[fill=orange]{3pt}} & \multicolumn{1}{c|}{\tikzcircle[fill=orange]{3pt}} & \multicolumn{1}{c|}{\tikzcircle[fill=orange]{3pt}} & \multicolumn{1}{c|}{\tikzcircle[fill=orange]{3pt}} & \multicolumn{1}{c}{\tikzcirclenew[fill=blue]{3pt}} \\ \hline

\multicolumn{1}{c|}{P33} & \multicolumn{1}{c|}{M} & \multicolumn{1}{c|}{\tikzcirclenew[fill=blue]{3pt}} & \multicolumn{1}{c|}{Few times} & \multicolumn{1}{c|}{No} & \multicolumn{1}{c|}{\tikzcircle[fill=orange]{3pt}} & \multicolumn{1}{c|}{\tikzcirclenew[fill=blue]{3pt}} & \multicolumn{1}{c|}{\tikzcirclenew[fill=blue]{3pt}} & \multicolumn{1}{c|}{\tikzcircle[fill=orange]{3pt}} & \multicolumn{1}{c|}{\tikzcirclenew[fill=blue]{3pt}} & \multicolumn{1}{c|}{P72} & \multicolumn{1}{c|}{M} & \multicolumn{1}{c|}{\tikzcirclenew[fill=blue]{3pt}} & \multicolumn{1}{c|}{Few times} & \multicolumn{1}{c|}{No} & \multicolumn{1}{c|}{\tikzcirclenew[fill=blue]{3pt}} & \multicolumn{1}{c|}{\tikzcircle[fill=orange]{3pt}} & \multicolumn{1}{c|}{\tikzcirclenew[fill=blue]{3pt}} & \multicolumn{1}{c|}{\tikzcircle[fill=orange]{3pt}} & \multicolumn{1}{c}{\tikzcirclenew[fill=blue]{3pt}} \\ \hline

\multicolumn{1}{c|}{P34} & \multicolumn{1}{c|}{M} & \multicolumn{1}{c|}{\tikzcirclenew[fill=blue]{3pt}} & \multicolumn{1}{c|}{Few times} & \multicolumn{1}{c|}{No} & \multicolumn{1}{c|}{\tikzcirclenew[fill=blue]{3pt}} & \multicolumn{1}{c|}{\tikzcircle[fill=orange]{3pt}} & \multicolumn{1}{c|}{\tikzcirclenew[fill=blue]{3pt}} & \multicolumn{1}{c|}{\tikzcircle[fill=orange]{3pt}} & \multicolumn{1}{c|}{\tikzcirclenew[fill=blue]{3pt}} & \multicolumn{1}{c|}{P73} & \multicolumn{1}{c|}{M} & \multicolumn{1}{c|}{\tikzcirclenew[fill=blue]{3pt}} & \multicolumn{1}{c|}{Once} & \multicolumn{1}{c|}{No} & \multicolumn{1}{c|}{\tikzcirclenew[fill=blue]{3pt}} & \multicolumn{1}{c|}{\tikzcirclenew[fill=blue]{3pt}} & \multicolumn{1}{c|}{\tikzcirclenew[fill=blue]{3pt}} & \multicolumn{1}{c|}{\tikzcircle[fill=orange]{3pt}} & \multicolumn{1}{c}{\tikzcirclenew[fill=blue]{3pt}} \\ \hline

\multicolumn{1}{c|}{P35} & \multicolumn{1}{c|}{W} & \multicolumn{1}{c|}{\tikzcirclenew[fill=blue]{3pt}} & \multicolumn{1}{c|}{Few times} & \multicolumn{1}{c|}{No} & \multicolumn{1}{c|}{\tikzcirclenew[fill=blue]{3pt}} & \multicolumn{1}{c|}{\tikzcirclenew[fill=blue]{3pt}} & \multicolumn{1}{c|}{\tikzcircle[fill=orange]{3pt}} & \multicolumn{1}{c|}{\tikzcircle[fill=orange]{3pt}} & \multicolumn{1}{c|}{\tikzcirclenew[fill=blue]{3pt}} & \multicolumn{1}{c|}{P74} & \multicolumn{1}{c|}{M} & \multicolumn{1}{c|}{\tikzcircle[fill=orange]{3pt}} & \multicolumn{1}{c|}{Never} & \multicolumn{1}{c|}{No} & \multicolumn{1}{c|}{\tikzcircle[fill=orange]{3pt}} & \multicolumn{1}{c|}{\tikzcirclenew[fill=blue]{3pt}} & \multicolumn{1}{c|}{\tikzcirclenew[fill=blue]{3pt}} & \multicolumn{1}{c|}{\tikzcircle[fill=orange]{3pt}} & \multicolumn{1}{c}{\tikzcircle[fill=orange]{3pt}} \\ \hline

\multicolumn{1}{c|}{P36} & \multicolumn{1}{c|}{M} & \multicolumn{1}{c|}{\tikzcirclenew[fill=blue]{3pt}} & \multicolumn{1}{c|}{Never} & \multicolumn{1}{c|}{No} & \multicolumn{1}{c|}{\tikzcirclenew[fill=blue]{3pt}} & \multicolumn{1}{c|}{\tikzcirclenew[fill=blue]{3pt}} & \multicolumn{1}{c|}{\tikzcirclenew[fill=blue]{3pt}} & \multicolumn{1}{c|}{\tikzcircle[fill=orange]{3pt}} & \multicolumn{1}{c|}{\tikzcirclenew[fill=blue]{3pt}} & \multicolumn{1}{c|}{P75} & \multicolumn{1}{c|}{M} & \multicolumn{1}{c|}{\tikzcirclenew[fill=blue]{3pt}} & \multicolumn{1}{c|}{Few times} & \multicolumn{1}{c|}{No} & \multicolumn{1}{c|}{\tikzcirclenew[fill=blue]{3pt}} & \multicolumn{1}{c|}{\tikzcirclenew[fill=blue]{3pt}} & \multicolumn{1}{c|}{\tikzcirclenew[fill=blue]{3pt}} & \multicolumn{1}{c|}{\tikzcircle[fill=orange]{3pt}} & \multicolumn{1}{c}{\tikzcirclenew[fill=blue]{3pt}} \\ \hline

\multicolumn{1}{c|}{P37} & \multicolumn{1}{c|}{M} & \multicolumn{1}{c|}{\tikzcirclenew[fill=blue]{3pt}} & \multicolumn{1}{c|}{Few times} & \multicolumn{1}{c|}{Some} & \multicolumn{1}{c|}{\tikzcirclenew[fill=blue]{3pt}} & \multicolumn{1}{c|}{\tikzcircle[fill=orange]{3pt}} & \multicolumn{1}{c|}{\tikzcirclenew[fill=blue]{3pt}} & \multicolumn{1}{c|}{\tikzcircle[fill=orange]{3pt}} & \multicolumn{1}{c|}{\tikzcirclenew[fill=blue]{3pt}} &

\multicolumn{10}{c}{\multirow{3}{*}{\begin{tabular}[c]{@{}c@{}}\textbf{Legend:} M: Man | W: Woman | \tikzcirclenew[fill=blue]{3pt}: Tim | \tikzcircle[fill=orange]{3pt}: Abi\\ MT: Motivation | SE: Self-efficacy | R: Risk \\ IP: Information processing | L: Learning\end{tabular}}} \\ \cline{1-10}

\multicolumn{1}{c|}{P38} & \multicolumn{1}{c|}{M} & \multicolumn{1}{c|}{\tikzcirclenew[fill=blue]{3pt}} & \multicolumn{1}{c|}{Never} & \multicolumn{1}{c|}{No} & \multicolumn{1}{c|}{\tikzcirclenew[fill=blue]{3pt}} & \multicolumn{1}{c|}{\tikzcirclenew[fill=blue]{3pt}} & \multicolumn{1}{c|}{\tikzcirclenew[fill=blue]{3pt}} & \multicolumn{1}{c|}{\tikzcircle[fill=orange]{3pt}} & \multicolumn{1}{c|}{\tikzcircle[fill=orange]{3pt}} & \multicolumn{10}{l}{} \\ \cline{1-10}

\multicolumn{1}{c|}{P39} & \multicolumn{1}{c|}{M} & \multicolumn{1}{c|}{\tikzcirclenew[fill=blue]{3pt}} & \multicolumn{1}{c|}{Few times} & \multicolumn{1}{c|}{Some} & \multicolumn{1}{c|}{\tikzcirclenew[fill=blue]{3pt}} & \multicolumn{1}{c|}{\tikzcirclenew[fill=blue]{3pt}} & \multicolumn{1}{c|}{\tikzcirclenew[fill=blue]{3pt}} & \multicolumn{1}{c|}{\tikzcircle[fill=orange]{3pt}} & \multicolumn{1}{c|}{\tikzcirclenew[fill=blue]{3pt}} & \multicolumn{10}{l}{} \\ \hline
\end{tabular}
\end{table*}



\begin{comment}


The five facets used by the GenderMag method are presented in Table~\ref{tab:gendermagfactes}. The facets are used to define personas (e.g., Abi and Tim). GenderMag highlights that differences relevant to inclusiveness lie not in a person's gender identity but in the facet values themselves~\cite{hill2017gender}. Nevertheless, Abi's facet values are more frequent in women than in other genders, and Tim's facet values are more frequent in men than in other genders. 

%(Figure~\ref{fig:abbypersona})

\begin{table}[!ht]\scriptsize
\centering
\vspace{-2.5mm}
\caption{GenderMag facets~\cite{burnett2016gendermag}}
\label{tab:gendermagfactes}
\newcommand{\pb}[1]{\parbox[t][][t]{1.0\linewidth}{#1} \vspace{-2pt}}

\begin{tabular}{p{12mm}|p{62mm}}
\hline
\multicolumn{1}{>{\centering\arraybackslash}m{12mm}|}{\textbf{GenderMag Facets}} & \multicolumn{1}{>{\centering\arraybackslash}m{62mm}}{\textbf{Definition}} \\ \hline \hline

Motivation & \pb{Women tend (statistically) to be motivated to use technology for what they can accomplish with it, whereas men are often motivated by their enjoyment of technology per se~\cite{simon2000impact, cassell2002hand, margolis2002unlocking, hou2006girls, kelleher2009barriers, burnett2010gender, burnett2011gender, hallstrom2015gender}. This difference can affect which software features users choose to use}. \\ \hline 

Information processing styles & \pb{To solve problems, people often need to process new information. Women are more likely (statistically) to process new information comprehensively—gathering fairly complete information before proceeding—but men are more likely to use selective styles—following the first promising information, then backtracking if needed~\cite{cafferata1989gender, meyers1991exploring, coursaris2008empirical, riedl2010there, meyers2015revisiting}. Each style has advantages, but either is at a disadvantage when not supported by the software.} \\ \hline

Computer self-efficacy & \pb{Self-efficacy is a person's confidence about succeeding at a specific task, which influences their use of cognitive strategies, persistence, and strategies for coping with obstacles. Empirical data have shown that women often have lower computer self-efficacy than men, which can affect their behavior with technology~\cite{margolis2002unlocking, durndell2002computer, hartzel2003self, beckwith2005effectiveness, beckwith2006tinkering, burnett2010gender, burnett2011gender, singh2013role, huffman2013using}.} \\ \hline

Risk aversion & \pb{Research shows that women statistically tend to be more risk-averse than men~\cite{weber2002domain, dohmen2011individual, charness2012strong}. These results span numerous decision-making domains, such as ethics, investment, gambling, health/safety, and career. Risk aversion with software usage can impact users' decisions as to which feature sets to use.} \\ \hline

Learning: by Process vs. by Tinkering & \pb{Research across age groups and professions reports women being statistically less likely to playfully experiment (“tinker”) with software features new to them, compared to men~\cite{beckwith2006tinkering, hou2006girls, rosner2009learning, burnett2010gender, cao2010debugging, chang2014specialization}. However, when women do tinker, they tend to be more likely to reflect during the process and thereby sometimes profit from it more than men do.} \\ \hline \hline
\end{tabular}
\end{table}

\end{comment}
&&&&
\begin{table}[ht!]
\centering
\caption{\textbf{Super Resolution Performance Results.} Our proposed WGAN EEG Spatial Upsampling method significantly outperforms a baseline of Bicubic Interpolation commonly used in EEG upsampling pipelines.}
\label{tab:results}
\resizebox{0.8\linewidth}{!}{%
\begin{tabular}{@{}cccccc@{}}
\toprule
\multirow{2}{*}{\textbf{Dataset}} & \multirow{2}{*}{\textbf{Scale}} & \multicolumn{2}{c}{\textbf{Bicubic}} & \multicolumn{2}{c}{\textbf{WGAN}} \\ \cmidrule(l){3-6} 
                      &   & \textbf{MSE} & \textbf{MAE} & \textbf{MSE}    & \textbf{MAE}   \\
\toprule
\multirow{2}{*}{Val}  & 2 & 3.71E7       & 3.89E3       & \textbf{2.01E3} & \textbf{24.38} \\
                      & 4 & 7.23E7       & 6.42E3       & \textbf{8.53E3} & \textbf{63.83} \\
\midrule
\multirow{2}{*}{Test} & 2 & 3.75E7       & 3.91E3       & \textbf{2.06E3} & \textbf{24.66} \\
                      & 4 & 7.30E7       & 6.45E3       & \textbf{8.68E3} & \textbf{64.39} \\
\bottomrule
\end{tabular}%
}
\end{table}
This work identifies signal collapse as a critical bottleneck in one-shot neural network pruning. Performance loss in pruned networks is due to \textbf{signal collapse} in addition to the removal of critical parameters. We propose \textbf{REFLOW} (\textbf{Re}storing \textbf{F}low of \textbf{Low}-variance signals), a simple yet effective method that mitigates signal collapse without computationally expensive weight updates. By focusing on signal preservation, REFLOW highlights the importance of mitigating signal collapse in sparse networks and enables magnitude pruning to match or surpass state-of-the-art one-shot pruning methods such as CHITA, CBS, and WF.

REFLOW consistently achieves state-of-the-art accuracy across diverse architectures, restoring ResNeXt-101 from under 4.1\% to 78.9\% top-1 accuracy at 80\% sparsity on ImageNet. Its lightweight design makes it a practical solution for both research and deployment, delivering high-quality sparse models without the overhead of traditional approaches. These findings challenge the traditional emphasis on weight selection strategies and underscore the critical role of signal propagation for achieving high-quality sparse networks in the context of one-shot pruning.



\section{Limitations} 

In this work, we compared the effectiveness and interplay of SFT and RL-based methods, under fixed data constraints. In particular, we chose offline methods like DPO and KTO as the baseline implementation of the RL method because it eliminates the need for reward modeling or iterative finetuning. This means that the process of development is limited to collecting an offline dataset and fientuning it - making it the most fair comparable to SFT in terms of implementation effort, compute costs and annotation efforts. Since this baseline RL method shows optimal performance over SFT, we hope that this motivates future work to study more complex RL-based methods and their interplay with SFT. In addition, we used GPT4o annotation for synthetic data generation, and also for evaluating Summarization and Helpfulness, which could include potential biases inherited from the model. 

In addition, we limited the size of the model to under 10 Billion parameters, to keep the finetuning cost low enough to ignore as compared to the data annotation costs. In addition, it would be extremely compute resource intensive to run thousands of finetuning runs with larger model sizes like 70B parameters. We hope that future work would study the scaling trends of RL-based methods against different model sizes, and also study the compute-data trade-off in-depth.

\section*{Conclusion}
This paper aims to enhance our understanding of the computational complexity of computing various Shapley value variants. We found that for various ML models --- including decision trees, regression tree ensembles, weighted automata, and linear regression --- both local and global interventional and baseline SHAP can be computed in polynomial time under HMM modeled distributions. This extends popular algorithms, such as TreeSHAP, beyond their empirical distributional scope. We also establish strict complexity gaps between the various SHAP variants (baseline, interventional, and conditional) and prove the intractability of computing SHAP for tree ensembles and neural networks in simplified scenarios. Overall, we present SHAP as a versatile framework whose complexity depends on four key factors: \begin{inparaenum}[(i)] \item model type, \item SHAP variant, \item distribution modeling approach, \item and local vs. global explanations\end{inparaenum}. We believe this perspective provides deeper insight into the computational complexity of SHAP, paving the way for future work.




%We believe that our framework provides a more intricate understanding of SHAP computation complexity across different models, distributions, and variants, paving the way for further research.

Our work opens promising directions for future research. First, expanding our computational analysis to other SHAP-related metrics, such as asymmetric SHAP~\citep{frye20} and SAGE~\citep{covert2020understanding}, would be valuable. Additionally, we aim to explore more expressive distribution classes and relaxed assumptions beyond those in Section \ref{sec:tractable} while maintaining tractable SHAP computation. Finally, when exact computation is intractable (Section \ref{sec:intractable}), investigating the approximability of SHAP metrics through approximation and parameterized complexity theory~\citep{downey2012parameterized} is an important direction.

%Our work opens several promising avenues for future research on the computational properties of explainable AI methods, with a particular focus on SHAP. First, it would be interesting to broaden the computational analysis conducted in this work to include other popular SHAP-related metrics in the literature, such as asymmetric SHAP \cite{frye20} and SAGE \cite{covert2020understanding}. Also, in the future, we aim to explore more expressive distribution classes and relaxed distributional assumptions—extending beyond those examined in Section \ref{sec:tractable} —that still yield tractable SHAP computation. Finally, when exact computation proves intractable (Section \ref{sec:intractable}), it is worthwhile to theoretically investigate the question of the approximability of computing the SHAP metrics across various configurations, through the lens of approximation and parametrized complexity theory \cite{arora2009computational}.

%This paper aims to deepen our understanding of the computational complexity involved in obtaining different Shapley value variants. We found that for a variety of ML models, including decision trees, tree ensembles for regression, weighted automata, and linear regression models — computing both local and global interventional and baseline SHAP can be done in polynomial time when distributions are modeled by HMMs. This extends the distributional scope of popular algorithms like TreeSHAP, which is limited to empirical distributions. Additionally, we demonstrate a strict complexity gap between SHAP variants, showing that interventional and baseline SHAP can be strictly easier to compute than conditional SHAP. Despite these positive results, we uncovered intractability for various SHAP variants in neural networks and tree ensembles. Finally, we provided generalized complexity relations across SHAP variants. We believe that our framework offers a deeper understanding of the complexity involved in computing SHAP across various variants, models, distributions, as well as in both local and global computations, laying the groundwork for future research.

% Your text comes here. Separate text sections with
% \section{Section title}
% \label{sec:1}
% Text with citations \cite{RefB} and \cite{RefJ}.
% \subsection{Subsection title}
% \label{sec:2}
% as required. Don't forget to give each section
% and subsection a unique label (see Sect.~\ref{sec:1}).
% \paragraph{Paragraph headings} Use paragraph headings as needed.
% \begin{equation}
% a^2+b^2=c^2
% \end{equation}

% For one-column wide figures use
% \begin{figure}
% % Use the relevant command to insert your figure file.
% % For example, with the graphicx package use
%   \includegraphics{example.eps}
% % figure caption is below the figure
% \caption{Please write your figure caption here}
% \label{fig:1}       % Give a unique label
% \end{figure}
%
% For two-column wide figures use
% \begin{figure*}
% % Use the relevant command to insert your figure file.
% % For example, with the graphicx package use
%   \includegraphics[width=0.75\textwidth]{example.eps}
% % figure caption is below the figure
% \caption{Please write your figure caption here}
% \label{fig:2}       % Give a unique label
% \end{figure*}
%
% For tables use
% \begin{table}
% % table caption is above the table
% \caption{Please write your table caption here}
% \label{tab:1}       % Give a unique label
% % For LaTeX tables use
% \begin{tabular}{lll}
% \hline\noalign{\smallskip}
% first & second & third  \\
% \noalign{\smallskip}\hline\noalign{\smallskip}
% number & number & number \\
% number & number & number \\
% \noalign{\smallskip}\hline
% \end{tabular}
% \end{table}


%\begin{acknowledgements}
%If you'd like to thank anyone, place your comments here
%and remove the percent signs.
%\end{acknowledgements}


% Authors must disclose all relationships or interests that 
% could have direct or potential influence or impart bias on 
% the work: 
%
% \section*{Conflict of interest}
%
% The authors declare that they have no conflict of interest.


% BibTeX users please use one of
%\bibliographystyle{spbasic}      % basic style, author-year citations
%\bibliographystyle{spmpsci}      % mathematics and physical sciences
%\bibliographystyle{spphys}       % APS-like style for physics
%\bibliography{}   % name your BibTeX data base

% % Non-BibTeX users please use
% \begin{thebibliography}{}
% %
% % and use \bibitem to create references. Consult the Instructions
% % for authors for reference list style.
% %
% \bibitem{RefJ}
% % Format for Journal Reference
% Author, Article title, Journal, Volume, page numbers (year)
% % Format for books
% \bibitem{RefB}
% Author, Book title, page numbers. Publisher, place (year)
% % etc
% \end{thebibliography}

\bibliographystyle{spbasic}      % basic style, author-year citations
%	\bibliographystyle{spmpsci}      % mathematics and physical sciences
%	\bibliographystyle{spphys}       % APS-like style for physics
	%\bibliography{}   % name your BibTeX data base
	
	% Non-BibTeX users please use
\bibliography{main}

\appendix
\begin{appendices}
\renewcommand{\thetable}{A\arabic{table}}
\counterwithin{table}{section}
\section{Appendices}
The full list of examined primary publications is enumerated in Table\ref{tab:flep}.

\begin{table}[ht!]
\caption{Full List Of Examined Publications
\label{tab:flep}}
\begin{adjustbox}{width=\linewidth, center}
\centering
\begin{tabular}{llll}
\toprule
\textbf{Year} & \textbf{Venue Type} & \textbf{Venue}               & \textbf{Title}              \\

\textsc{Conference} & & &\\   
\midrule
2023          & Conference          & AINA                         & A Vulnerability Detection Method for SDN with Optimized Fuzzing\citep{10.1007/978-3-031-28451-9_46}                                                                            \\
2022          & Conference          & ICAIS                        & Test Traffic Control Based on REST API for Software-Defined Networking\citep{10.1007/978-3-031-06788-4_40}                                                                     \\
2021          & Conference          & SOSR                         & Tardis: A Fault-Tolerant Design for Network Control Planes\citep{10.1145/3482898.3483355}                                                                                      \\
2021          & Conference          & NISS                         & SDN Control Plane Security: Attacks and Mitigation Techniques\citep{10.1145/3454127.3456612}                                                                                   \\
2021          & Conference          & DSN                          & A Comprehensive Study of Bugs in Software Defined Networks\citep{9505089}                                                                                                      \\
2021          & Conference          & NTMS                         & SDN Security through System Call Learning\citep{9432640}                                                                                                                       \\
2020          & Conference          & CIC                          & \begin{tabular}[c]{@{}l@{}}ParaSDN: An Access Control Model for SDN Applications based on Parameterized Roles and \\ Permissions\citep{9319021}\end{tabular}                   \\
2020          & Conference          & SP                           & Unexpected Data Dependency Creation and Chaining: A New Attack to SDN\citep{9152642}                                                                                           \\
2020          & Conference          & CSPS                         & Deep Learning Based Detection Method for SDN Malicious Applications\citep{10.1007/978-981-13-6508-9_13}                                                                        \\
2020          & Conference          & INFOCOM                      & AudiSDN: Automated Detection of Network Policy Inconsistencies in Software-Defined Networks\citep{9155378}                                                                     \\
2020          & Conference          & WISA                         & FSF: Code Coverage-Driven Fuzzing for Software-Defined Networking\citep{10.1007/978-3-030-39303-8_4}                                                                           \\
2020          & Conference          & CAV                          & Towards Model Checking Real-World Software-Defined Networks\citep{10.1007/978-3-030-53291-8_8}                                                                                 \\
2019          & Conference          & IM                           & Mining Software Repositories for Predictive Modelling of Defects in SDN Controller\citep{8717837}                                                                              \\
2019          & Conference          & IM                           & Thinking inside the Box: Differential Fault Localization for SDN Control Plane\citep{8717815}                                                                                  \\
2018          & Conference          & NFV-SDN                      & RE-CHECKER: Towards Secure RESTful Service in Software-Defined Networking\citep{8725649}                                                                                       \\
2018          & Conference          & ICNP                         & INDAGO: A New Framework For Detecting Malicious SDN Applications\citep{8526819}                                                                                                \\
% 2018          & Chapter             &                              & Security Analysis of FloodLight, ZeroSDN, Beacon and POX SDN Controllers\citep{Ilyas2018}                                                                                      \\
2018          & Conference          & CCS                          & AIM-SDN: Attacking Information Mismanagement in SDN-datastores\citep{10.1145/3243734.3243799}                                                                                  \\
2018          & Conference          & ICC                          & SENAD: Securing Network Application Deployment in Software Defined Networks\citep{8422405}                                                                                     \\
% 2018          & Chapter             &                              & Security Analysis of SDN Routing Applications\citep{Sagare2018}                                                                                                                \\
2017          & Conference          & NSDI                         & Automated Bug Removal for Software-Defined Networks\citep{10.5555/3154630.3154688}                                                                                             \\
2017          & Conference          & MASCOTS                      & Testing Black-Box SDN Applications with Formal Behavior Models\citep{8107437}                                                                                                  \\
2017          & Conference          & ICC                          & Controller DAC: Securing SDN controller with dynamic access control\citep{7997249}                                                                                             \\
2017          & Conference          & ISSREW                       & Analytics-Enhanced Automated Code Verification for Dependability of Software-Defined Networks\citep{8109275}                                                                   \\
2017          & Conference          & SIGCOMM                      & BigBug: Practical Concurrency Analysis for SDN\citep{10.1145/3050220.3050230}                                                                                                  \\
2017          & Conference          & ICCNT                        & Requirement analysis for abstracting security in software-defined network\citep{8204161}                                                                                       \\
2017          & Conference          & CNSM                         & An empirical study of software reliability in SDN controllers\citep{8256002}                                                                                                   \\
2017          & Conference          & BWCCA                        & A Comprehensive Security Analysis Checksheet for OpenFlow Networks\citep{10.1007/978-3-319-49106-6_22}                                                                         \\
2017          & Conference          & RAID                         & BEADS: Automated Attack Discovery in OpenFlow-Based SDN Systems\citep{10.1007/978-3-319-66332-6_14}                                                                            \\
2017          & Conference          & MIWAI                        & DREAD-R: Severity Assessment of ONOS SDN Controller\citep{10.1007/978-3-319-69456-6_27}                                                                                        \\
2016          & Conference          & CODASPY                      & SHIELD: An Automated Framework for Static Analysis of SDN Applications\citep{10.1145/2876019.2876026}                                                                          \\
2016          & Conference          & CODASPY                      & The Smaller, the Shrewder: A Simple Malicious Application Can Kill an Entire SDN Environment\citep{10.1145/2876019.2876024}                                                    \\
2016          & Conference          & ISSREW                       & Programming the Network: Application Software Faults in Software-Defined Networks\citep{7789391}                                                                               \\
2016          & Conference          & ICSP                         & BuDDI: Bug detection, debugging, and isolation middlebox for software-defined network controllers\citep{7818438}                                                               \\
2016          & Conference          & ICIEV                        & SDN testing and debugging tools: A survey\citep{7760078}                                                                                                                       \\
2015          & Conference          & ICUFN                        & Secure your Northbound SDN API\citep{7182679}                                                                                                                                  \\
2015          & Conference          & TSA                          & Model-Based Testing of SDN Firewalls: A Case Study\citep{7335947}                                                                                                              \\
2015          & Conference          & SIGCOMM                          & Automated Network Repair with Meta Provenance\citep{10.1145/2834050.2834112}                                                                                               \\
2015          & Conference          & GPCE                         & Safer SDN programming through Arbiter\citep{10.1145/2936314.2814218}                                                                                                           \\
2015          & Conference          & NetSoft                      & Design and deployment of secure, robust, and resilient SDN controllers\citep{7258233}                                                                                          \\
2014          & Conference          & CONEXT                       & Controller-agnostic SDN Debugging\citep{10.1145/2674005.2674993}                                                                                                               \\
2014          & Conference          & PLDI                         & VeriCon: towards verifying controller programs in software-defined networks\citep{10.1145/2666356.2594317}                                                                     \\
2014          & Conference          & NSDI                         & Tierless programming and reasoning for software-defined networks\citep{10.5555/2616448.2616496}                                                                                \\
2014          & Conference          & CONEXT                       & Model Based Black-Box Testing of SDN Applications\citep{10.1145/2680821.2680828}                                                                                               \\
2014          & Conference          & SIGCOMM                      & Troubleshooting blackbox SDN control software with minimal causal sequences\citep{10.1145/2740070.2626304}                                                                     \\
2013          & Conference          & PLDI                         & Machine-verified network controllers\citep{10.1145/2499370.2462178}                                                                                                            \\
2012          & Conference          & NSDI                         & A NICE Way to Test OpenFlow Applications\citep{180591}                                                                                                                         \\
              &                     &                              &                                                                                                                                                                                \\
\textsc{Journal} & & &\\ \hline
2023          & Journal             & WWW                          & DACAS: integration of attribute-based access control for northbound interface security in SDN\citep{Liu2023}                                                                   \\
2022          & Journal             & ToN                          & A Framework for Policy Inconsistency Detection in Software-Defined Networks\citep{9681706}                                                                                     \\
2021          & Journal             & CSUR                         & Application Threats to Exploit Northbound Interface Vulnerabilities in Software Defined Networks\citep{10.1145/3453648}                                                        \\
2021          & Journal             & JPDC                         & \begin{tabular}[c]{@{}l@{}}SEAPP: A secure application management framework based on REST API access control in SDN-enabled \\ cloud environment\citep{HU2021108}\end{tabular} \\
2020          & Journal             & IJCNA                        & Security in SDN: A comprehensive survey\citep{CORREACHICA2020102595}                                                                                                           \\
2019          & Journal             & COMST                        & Fault Management in Software-Defined Networking: A Survey\citep{8456508}                                                                                                       \\
2019          & Journal             & COMST                        & A Survey on Network Verification and Testing With Formal Methods: Approaches and Challenges\citep{8453007}                                                                     \\
2019          & Journal             & JAMT                         & Modeling and Verifying Basic Modules of Floodlight\citep{Xiang2019}                                                                                                            \\
2018          & Journal             & TNSM                         & Assessing the Maturity of SDN Controllers With Software Reliability Growth Models\citep{8386840}                                                                               \\
2018          & Journal             & JSAC                         & MORPH: An Adaptive Framework for Efficient and Byzantine Fault-Tolerant SDN Control Plane\citep{8490892}                                                                       \\
2017          & Journal             & COMST                        & A Survey on Fault Management in Software-Defined Networks\citep{7959044}                                                                                                       \\
2016          & Journal             & IJCNA                        & Secure and dependable software defined networks\citep{AKHUNZADA2016199}                                                                                                        \\
2016          & Journal             & MONET                        & Security in Software-Defined Networking: Threats and Countermeasures\citep{Shu2016}                                                                                            \\

\bottomrule

\end{tabular}
\end{adjustbox}
\end{table}
\end{appendices}


\end{document}
% end of file template.tex

