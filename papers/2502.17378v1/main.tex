%%%%%%%%%%%%%%%%%%%%%%% file template.tex %%%%%%%%%%%%%%%%%%%%%%%%%
%
% This is a general template file for the LaTeX package SVJour3
% for Springer journals.          Springer Heidelberg 2010/09/16
%
% Copy it to a new file with a new name and use it as the basis
% for your article. Delete % signs as needed.
%
% This template includes a few options for different layouts and
% content for various journals. Please consult a previous issue of
% your journal as needed.
%
%%%%%%%%%%%%%%%%%%%%%%%%%%%%%%%%%%%%%%%%%%%%%%%%%%%%%%%%%%%%%%%%%%%
%
% First comes an example EPS file -- just ignore it and
% proceed on the \documentclass line
% your LaTeX will extract the file if required
\begin{filecontents*}{example.eps}
%!PS-Adobe-3.0 EPSF-3.0
%%BoundingBox: 19 19 221 221
%%CreationDate: Mon Sep 29 1997
%%Creator: programmed by hand (JK)
%%EndComments
gsave
newpath
  20 20 moveto
  20 220 lineto
  220 220 lineto
  220 20 lineto
closepath
2 setlinewidth
gsave
  .4 setgray fill
grestore
stroke
grestore
\end{filecontents*}
%
\RequirePackage{fix-cm}
%
%\documentclass{svjour3}                     % onecolumn (standard format)
%\documentclass[smallcondensed]{svjour3}     % onecolumn (ditto)
\documentclass[smallextended]{svjour3}       % onecolumn (second format)
%\documentclass[twocolumn]{svjour3}          % twocolumn
%
\smartqed  % flush right qed marks, e.g. at end of proof
%
\usepackage{graphicx}
\usepackage{lipsum}
%\usepackage{sectsty}
\usepackage{pdfpages}
\usepackage{float}
\usepackage{caption}  % subfigures
\usepackage{subcaption} % subfigures
\usepackage{nicefrac}
\usepackage{url}
\usepackage{booktabs}
\usepackage{comment}
\usepackage{color,soul} % package to highlight text
\usepackage{array}
\usepackage{multirow}
\usepackage{bigstrut}
\usepackage{subcaption}
\usepackage{units}
\usepackage{flushend}
\usepackage{url}
\usepackage{natbib}
\usepackage{hyperref}
\usepackage{colortbl}
\usepackage{booktabs}
\usepackage{natbib}
\usepackage{makecell}
\usepackage{mathptm}
\usepackage{tcolorbox}

\usepackage{fancyvrb} % breaking lines with verbatim
\usepackage{fvextra} % breaking lines with verbatim
\usepackage{appendix}
\usepackage{microtype}

\usepackage{tabularx}

\usepackage{pdfpages}

\usepackage{xcolor}

\usepackage{tcolorbox}
\usepackage{listings}
\usepackage{fvextra}
\DefineVerbatimEnvironment{Verbatim}{Verbatim}{breaklines=true}
%
% \usepackage{mathptmx}      % use Times fonts if available on your TeX system
%
% insert here the call for the packages your document requires
%\usepackage{latexsym}
% etc.
%
% please place your own definitions here and don't use \def but
% \newcommand{}{}
%
% Insert the name of "your journal" with
% \journalname{myjournal}
%

% Command to create comments in the document

% \renewcommand{\arraystretch}{1.5}  % Adjust row height

% DEFINING THE HIGHLIGHT BOX
\newtcolorbox{highlightbox}{
	colback=gray!20, % Background color
	colframe=black,   % Border color
	boxsep=2pt,       % Padding around the box content
	arc=0pt,          % Corner radius
	boxrule=1pt       % Border thickness
}

\newcommand{\joao}[1] {\textcolor{red}{[\textbf{Jo\~{a}o}: #1]} }
\newcommand{\filipe}[1] {\textcolor{purple}{[\textbf{Filipe}: #1]} }
\newcommand{\dan}[1] {\textcolor{blue}{[\textbf{Dan}: #1]} }
\newcommand{\uira}[1] {\textcolor{green}{[\textbf{Uira}: #1]} }
\newcommand{\sergio}[1] {\textcolor{orange}{[\textbf{Sergio}: #1]} }



% \newcommand{\RQone}{RQ1: Are there perceived differences between ML and non-ML projects with respect to their CI practices?}

\newcommand{\RQone}{RQ1: What are the perceived reasons behind the differences between ML and non-ML projects with respect to their CI practices?}
\newcommand{\RQtwo}{RQ2: How do ML practitioners manage the build duration of their CI pipeline?}
\newcommand{\RQthree}{RQ3: Are there differences when handling test coverage rates between ML and non-ML projects?}

\begin{document}

\title{Continuous Integration Practices in Machine Learning Projects: The Practitioners' Perspective%\thanks{Grants or other notes
%about the article that should go on the front page should be
%placed here. General acknowledgments should be placed at the end of the article.}
}
% \subtitle{Do you have a subtitle?\\ If so, write it here}

\titlerunning{CI Practices in ML Projects: The Practitioners' Perspective}        % if too long for running head

\author{
    João Helis Bernardo \and 
    Daniel Alencar da Costa \and
    Filipe Roseiro Cogo \and
    Sérgio Queiróz de Medeiros \and    
    Uirá Kulesza        
}

%\authorrunning{Short form of author list} % if too long for running head

% \institute{
% João Helis Bernardo \at
% Federal Institute of Rio Grande do Norte \\
% Federal University of Rio Grande do Norte \\
% Natal, Brazil \\              
% \email{joao.helis@ifrn.edu.br}         \and
% Daniel Alencar da Costa \at
% University of Otago \\
% Dunedin, New Zealand \\              
% \email{danielcalencar@otago.ac.nz}      \and
% Sérgio Queiroz de Medeiros \at
% Federal University of Rio Grande do Norte \\
% Natal, Brazil \\                        
% \email{sergio.medeiros@ufrn.br}         
% \and
% Filipe Roseiro Cogo \at
% Centre for Software Excellence, Huawei \\
% Kingston, Canada \\                     
% \email{filipe.roseiro.cogo1@huawei.com}
% \and
% Uirá Kulesza \at
% Federal University of Rio Grande do Norte \\
% Natal, Brazil \\                        
% \email{uira@dimap.ufrn.br}         
% }

\institute{
% João Helis Bernardo  \at
% Federal Institute of Rio Grande do Norte \\
% Federal University of Rio Grande do Norte \\
% Natal, Brazil \\              
% \email{joao.helis@ifrn.edu.br}         \and
João Helis Bernardo, Sérgio Queiroz de Medeiros, and Uirá Kulesza  \at
Federal University of Rio Grande do Norte \\
Natal, Brazil \\              
\email{joao.helis@ifrn.edu.br, sergio.medeiros@ufrn.br, uira@dimap.ufrn.br}
\and
Daniel Alencar da Costa \at
University of Otago \\
Dunedin, New Zealand \\              
\email{danielcalencar@otago.ac.nz}      \and
% Sérgio Queiroz de Medeiros \at
% Federal University of Rio Grande do Norte \\
% Natal, Brazil \\                        
% \email{sergio.medeiros@ufrn.br}         
% \and
Filipe Roseiro Cogo \at
Centre for Software Excellence, Huawei \\
Kingston, Canada \\                     
\email{filipe.roseiro.cogo1@huawei.com}
% \and
% Uirá Kulesza \at
% Federal University of Rio Grande do Norte \\
% Natal, Brazil \\                        
% \email{uira@dimap.ufrn.br}         
}

\date{Received: date / Accepted: date}
% The correct dates will be entered by the editor


\maketitle

\begin{abstract}  
Test time scaling is currently one of the most active research areas that shows promise after training time scaling has reached its limits.
Deep-thinking (DT) models are a class of recurrent models that can perform easy-to-hard generalization by assigning more compute to harder test samples.
However, due to their inability to determine the complexity of a test sample, DT models have to use a large amount of computation for both easy and hard test samples.
Excessive test time computation is wasteful and can cause the ``overthinking'' problem where more test time computation leads to worse results.
In this paper, we introduce a test time training method for determining the optimal amount of computation needed for each sample during test time.
We also propose Conv-LiGRU, a novel recurrent architecture for efficient and robust visual reasoning. 
Extensive experiments demonstrate that Conv-LiGRU is more stable than DT, effectively mitigates the ``overthinking'' phenomenon, and achieves superior accuracy.
\end{abstract}  
\section{Introduction}
\label{sec:introduction}
The business processes of organizations are experiencing ever-increasing complexity due to the large amount of data, high number of users, and high-tech devices involved \cite{martin2021pmopportunitieschallenges, beerepoot2023biggestbpmproblems}. This complexity may cause business processes to deviate from normal control flow due to unforeseen and disruptive anomalies \cite{adams2023proceddsriftdetection}. These control-flow anomalies manifest as unknown, skipped, and wrongly-ordered activities in the traces of event logs monitored from the execution of business processes \cite{ko2023adsystematicreview}. For the sake of clarity, let us consider an illustrative example of such anomalies. Figure \ref{FP_ANOMALIES} shows a so-called event log footprint, which captures the control flow relations of four activities of a hypothetical event log. In particular, this footprint captures the control-flow relations between activities \texttt{a}, \texttt{b}, \texttt{c} and \texttt{d}. These are the causal ($\rightarrow$) relation, concurrent ($\parallel$) relation, and other ($\#$) relations such as exclusivity or non-local dependency \cite{aalst2022pmhandbook}. In addition, on the right are six traces, of which five exhibit skipped, wrongly-ordered and unknown control-flow anomalies. For example, $\langle$\texttt{a b d}$\rangle$ has a skipped activity, which is \texttt{c}. Because of this skipped activity, the control-flow relation \texttt{b}$\,\#\,$\texttt{d} is violated, since \texttt{d} directly follows \texttt{b} in the anomalous trace.
\begin{figure}[!t]
\centering
\includegraphics[width=0.9\columnwidth]{images/FP_ANOMALIES.png}
\caption{An example event log footprint with six traces, of which five exhibit control-flow anomalies.}
\label{FP_ANOMALIES}
\end{figure}

\subsection{Control-flow anomaly detection}
Control-flow anomaly detection techniques aim to characterize the normal control flow from event logs and verify whether these deviations occur in new event logs \cite{ko2023adsystematicreview}. To develop control-flow anomaly detection techniques, \revision{process mining} has seen widespread adoption owing to process discovery and \revision{conformance checking}. On the one hand, process discovery is a set of algorithms that encode control-flow relations as a set of model elements and constraints according to a given modeling formalism \cite{aalst2022pmhandbook}; hereafter, we refer to the Petri net, a widespread modeling formalism. On the other hand, \revision{conformance checking} is an explainable set of algorithms that allows linking any deviations with the reference Petri net and providing the fitness measure, namely a measure of how much the Petri net fits the new event log \cite{aalst2022pmhandbook}. Many control-flow anomaly detection techniques based on \revision{conformance checking} (hereafter, \revision{conformance checking}-based techniques) use the fitness measure to determine whether an event log is anomalous \cite{bezerra2009pmad, bezerra2013adlogspais, myers2018icsadpm, pecchia2020applicationfailuresanalysispm}. 

The scientific literature also includes many \revision{conformance checking}-independent techniques for control-flow anomaly detection that combine specific types of trace encodings with machine/deep learning \cite{ko2023adsystematicreview, tavares2023pmtraceencoding}. Whereas these techniques are very effective, their explainability is challenging due to both the type of trace encoding employed and the machine/deep learning model used \cite{rawal2022trustworthyaiadvances,li2023explainablead}. Hence, in the following, we focus on the shortcomings of \revision{conformance checking}-based techniques to investigate whether it is possible to support the development of competitive control-flow anomaly detection techniques while maintaining the explainable nature of \revision{conformance checking}.
\begin{figure}[!t]
\centering
\includegraphics[width=\columnwidth]{images/HIGH_LEVEL_VIEW.png}
\caption{A high-level view of the proposed framework for combining \revision{process mining}-based feature extraction with dimensionality reduction for control-flow anomaly detection.}
\label{HIGH_LEVEL_VIEW}
\end{figure}

\subsection{Shortcomings of \revision{conformance checking}-based techniques}
Unfortunately, the detection effectiveness of \revision{conformance checking}-based techniques is affected by noisy data and low-quality Petri nets, which may be due to human errors in the modeling process or representational bias of process discovery algorithms \cite{bezerra2013adlogspais, pecchia2020applicationfailuresanalysispm, aalst2016pm}. Specifically, on the one hand, noisy data may introduce infrequent and deceptive control-flow relations that may result in inconsistent fitness measures, whereas, on the other hand, checking event logs against a low-quality Petri net could lead to an unreliable distribution of fitness measures. Nonetheless, such Petri nets can still be used as references to obtain insightful information for \revision{process mining}-based feature extraction, supporting the development of competitive and explainable \revision{conformance checking}-based techniques for control-flow anomaly detection despite the problems above. For example, a few works outline that token-based \revision{conformance checking} can be used for \revision{process mining}-based feature extraction to build tabular data and develop effective \revision{conformance checking}-based techniques for control-flow anomaly detection \cite{singh2022lapmsh, debenedictis2023dtadiiot}. However, to the best of our knowledge, the scientific literature lacks a structured proposal for \revision{process mining}-based feature extraction using the state-of-the-art \revision{conformance checking} variant, namely alignment-based \revision{conformance checking}.

\subsection{Contributions}
We propose a novel \revision{process mining}-based feature extraction approach with alignment-based \revision{conformance checking}. This variant aligns the deviating control flow with a reference Petri net; the resulting alignment can be inspected to extract additional statistics such as the number of times a given activity caused mismatches \cite{aalst2022pmhandbook}. We integrate this approach into a flexible and explainable framework for developing techniques for control-flow anomaly detection. The framework combines \revision{process mining}-based feature extraction and dimensionality reduction to handle high-dimensional feature sets, achieve detection effectiveness, and support explainability. Notably, in addition to our proposed \revision{process mining}-based feature extraction approach, the framework allows employing other approaches, enabling a fair comparison of multiple \revision{conformance checking}-based and \revision{conformance checking}-independent techniques for control-flow anomaly detection. Figure \ref{HIGH_LEVEL_VIEW} shows a high-level view of the framework. Business processes are monitored, and event logs obtained from the database of information systems. Subsequently, \revision{process mining}-based feature extraction is applied to these event logs and tabular data input to dimensionality reduction to identify control-flow anomalies. We apply several \revision{conformance checking}-based and \revision{conformance checking}-independent framework techniques to publicly available datasets, simulated data of a case study from railways, and real-world data of a case study from healthcare. We show that the framework techniques implementing our approach outperform the baseline \revision{conformance checking}-based techniques while maintaining the explainable nature of \revision{conformance checking}.

In summary, the contributions of this paper are as follows.
\begin{itemize}
    \item{
        A novel \revision{process mining}-based feature extraction approach to support the development of competitive and explainable \revision{conformance checking}-based techniques for control-flow anomaly detection.
    }
    \item{
        A flexible and explainable framework for developing techniques for control-flow anomaly detection using \revision{process mining}-based feature extraction and dimensionality reduction.
    }
    \item{
        Application to synthetic and real-world datasets of several \revision{conformance checking}-based and \revision{conformance checking}-independent framework techniques, evaluating their detection effectiveness and explainability.
    }
\end{itemize}

The rest of the paper is organized as follows.
\begin{itemize}
    \item Section \ref{sec:related_work} reviews the existing techniques for control-flow anomaly detection, categorizing them into \revision{conformance checking}-based and \revision{conformance checking}-independent techniques.
    \item Section \ref{sec:abccfe} provides the preliminaries of \revision{process mining} to establish the notation used throughout the paper, and delves into the details of the proposed \revision{process mining}-based feature extraction approach with alignment-based \revision{conformance checking}.
    \item Section \ref{sec:framework} describes the framework for developing \revision{conformance checking}-based and \revision{conformance checking}-independent techniques for control-flow anomaly detection that combine \revision{process mining}-based feature extraction and dimensionality reduction.
    \item Section \ref{sec:evaluation} presents the experiments conducted with multiple framework and baseline techniques using data from publicly available datasets and case studies.
    \item Section \ref{sec:conclusions} draws the conclusions and presents future work.
\end{itemize}
\section{Related Work}
The landscape of large language model vulnerabilities has been extensively studied in recent literature \cite{crothers2023machinegeneratedtextcomprehensive,shayegani2023surveyvulnerabilitieslargelanguage,Yao_2024,Huang2023ASO}, that propose detailed taxonomies of threats. These works categorize LLM attacks into distinct types, such as adversarial attacks, data poisoning, and specific vulnerabilities related to prompt engineering. Among these, prompt injection attacks have emerged as a significant and distinct category, underscoring their relevance to LLM security.

The following high-level overview of the collected taxonomy of LLM vulnerabilities is defined in \cite{Yao_2024}:
\begin{itemize}
    \item Adversarial Attacks: Data Poisoning, Backdoor Attacks
    \item Inference Attacks: Attribute Inference, Membership Inferences
    \item Extraction Attacks
    \item Bias and Unfairness
Exploitation
    \item Instruction Tuning Attacks: Jailbreaking, Prompt Injection.
\end{itemize}
Prompt injection attacks are further classified in \cite{shayegani2023surveyvulnerabilitieslargelanguage} into the following: Goal hijacking and \textbf{Prompt leakage}.

The reviewed taxonomies underscore the need for comprehensive frameworks to evaluate LLM security. The agentic approach introduced in this paper builds on these insights, automating adversarial testing to address a wide range of scenarios, including those involving prompt leakage and role-specific vulnerabilities.

\subsection{Prompt Injection and Prompt Leakage}

Prompt injection attacks exploit the blending of instructional and data inputs, manipulating LLMs into deviating from their intended behavior. Prompt injection attacks encompass techniques that override initial instructions, expose private prompts, or generate malicious outputs \cite{Huang2023ASO}. A subset of these attacks, known as prompt leakage, aims specifically at extracting sensitive system prompts embedded within LLM configurations. In \cite{shayegani2023surveyvulnerabilitieslargelanguage}, authors differentiate between prompt leakage and related methods such as goal hijacking, further refining the taxonomy of LLM-specific vulnerabilities.

\subsection{Defense Mechanisms}

Various defense mechanisms have been proposed to address LLM vulnerabilities, particularly prompt injection and leakage \cite{shayegani2023surveyvulnerabilitieslargelanguage,Yao_2024}. We focused on cost-effective methods like instruction postprocessing and prompt engineering, which are viable for proprietary models that cannot be retrained. Instruction preprocessing sanitizes inputs, while postprocessing removes harmful outputs, forming a dual-layer defense. Preprocessing methods include perplexity-based filtering \cite{Jain2023BaselineDF,Xu2022ExploringTU} and token-level analysis \cite{Kumar2023CertifyingLS}. Postprocessing employs another set of techniques, such as censorship by LLMs \cite{Helbling2023LLMSD,Inan2023LlamaGL}, and use of canary tokens and pattern matching \cite{vigil-llm,rebuff}, although their fundamental limitations are noted \cite{Glukhov2023LLMCA}. Prompt engineering employs carefully designed instructions \cite{Schulhoff2024ThePR} and advanced techniques like spotlighting \cite{Hines2024DefendingAI} to mitigate vulnerabilities, though no method is foolproof \cite{schulhoff-etal-2023-ignore}. Adversarial training, by incorporating adversarial examples into the training process, strengthens models against attacks \cite{Bespalov2024TowardsBA,Shaham2015UnderstandingAT}.

\subsection{Security Testing for Prompt Injection Attacks}

Manual testing, such as red teaming \cite{ganguli2022redteaminglanguagemodels} and handcrafted "Ignore Previous Prompt" attacks \cite{Perez2022IgnorePP}, highlights vulnerabilities but is limited in scale. Automated approaches like PAIR \cite{chao2024jailbreakingblackboxlarge} and GPTFUZZER \cite{Yu2023GPTFUZZERRT} achieve higher success rates by refining prompts iteratively or via automated fuzzing. Red teaming with LLMs \cite{Perez2022RedTL} and reinforcement learning \cite{anonymous2024diverse} uncovers diverse vulnerabilities, including data leakage and offensive outputs. Indirect Prompt Injection (IPI) manipulates external data to compromise applications \cite{Greshake2023NotWY}, adapting techniques like SQL injection to LLMs \cite{Liu2023PromptIA}. Prompt secrecy remains fragile, with studies showing reliable prompt extraction \cite{Zhang2023EffectivePE}. Advanced frameworks like Token Space Projection \cite{Maus2023AdversarialPF} and Weak-to-Strong Jailbreaking Attacks \cite{zhao2024weaktostrongjailbreakinglargelanguage} exploit token-space relationships, achieving high success rates for prompt extraction and jailbreaking.

\subsection{Agentic Frameworks for Evaluating LLM Security}

The development of multi-agent systems leveraging large language models (LLMs) has shown promising results in enhancing task-solving capabilities \cite{Hong2023MetaGPTMP, Wang2023UnleashingTE, Talebirad2023MultiAgentCH, Wu2023AutoGenEN, Du2023ImprovingFA}. A key aspect across various frameworks is the specialization of roles among agents \cite{Hong2023MetaGPTMP, Wu2023AutoGenEN}, which mimics human collaboration and improves task decomposition.

Agentic frameworks and the multi-agent debate approach benefit from agent interaction, where agents engage in conversations or debates to refine outputs and correct errors \cite{Wu2023AutoGenEN}. For example, debate systems improve factual accuracy and reasoning by iteratively refining responses through collaborative reasoning \cite{Du2023ImprovingFA}, while AG2 allows agents to autonomously interact and execute tasks with minimal human input.

These frameworks highlight the viability of agentic systems, showing how specialized roles and collaborative mechanisms lead to improved performance, whether in factuality, reasoning, or task execution. By leveraging the strengths of diverse agents, these systems demonstrate a scalable approach to problem-solving.

Recent research on testing LLMs using other LLMs has shown that this approach can be highly effective \cite{chao2024jailbreakingblackboxlarge, Yu2023GPTFUZZERRT, Perez2022RedTL}. Although the papers do not explicitly employ agentic frameworks they inherently reflect a pattern similar to that of an "attacker" and a "judge". \cite{chao2024jailbreakingblackboxlarge}  This pattern became a focal point for our work, where we put the judge into a more direct dialogue, enabling it to generate attacks based on the tested agent response in an active conversation.

A particularly influential paper in shaping our approach is Jailbreaking Black Box Large Language Models in Twenty Queries \cite{chao2024jailbreakingblackboxlarge}. This paper not only introduced the attacker/judge architecture but also provided the initial system prompts used for a judge.
% \label{demographics}

\section{Research Methodology}
\label{sec:methodology}

In this section, we describe how we selected the ML practitioners for our study, the data collection process, and the research approach we used to perform our analysis.

\subsection{Subject Projects}

Our prior work~\citep{bernardo2024machine} quantitatively analyzed the adoption of CI practices across 93 ML projects and 92 non-ML projects. This study revealed that ML projects often experience longer build durations and that medium-sized ML projects tend to have lower test coverage compared to non-ML projects. Building on these findings, our current study seeks to deepen our understanding of the specific factors driving these differences in CI practice adoption. To achieve this, we employed a qualitative, survey-based approach, focusing on the perceptions of ML practitioners (i.e., contributors and integrators of ML projects) regarding the challenges, barriers, and strategies associated with CI adoption in ML projects.

To maintain consistency with prior analyses and minimize potential biases associated with using an unverified or outdated collection of projects, we based our study on the 
93 ML projects investigated in our prior work~\citep{bernardo2024machine}. 
This dataset is both up-to-date and meticulously curated, representing a diverse collection of actively maintained ML projects that successfully integrate CI workflows into their pipelines. Additionally, the projects are categorized by size—small, medium, or large—based on their Lines of Code (LOC), ensuring a comprehensive representation across varying scales of ML development.

In addition to investigating general differences in CI adoption between ML and non-ML projects, this study focuses on the factors influencing build durations in ML projects, as outlined in RQ2. 
To capture diverse and meaningful insights, we adopted a sampling approach designed to reflect a wide range of experiences related to build durations. Specifically, we targeted projects with the shortest and those with the longest build durations to ensure that our analysis encompasses the full spectrum of challenges and characteristics encountered in ML workflows.

To sample practitioners for the survey, we selected those that are associated with the top 25\% of projects with the shortest build durations and the top 25\% with the longest build durations. 
As a result, we identified practitioners from 47 ML projects.
This targeted selection enhances the relevance of our findings by ensuring representation from both extremes of build duration characteristics, providing a comprehensive view of the characteristics and challenges faced in the CI workflows of ML projects. The list of the investigated ML projects, along with their characteristics (e.g., size and median build duration), is provided in Table~\ref{tab:project_characteristics}. 

% Table generated by Excel2LaTeX from sheet 'Sheet2'
\begin{table}[H]
  \centering
  \caption{Characteristics of Investigated Machine Learning Projects.}
    \begin{tabular}{cp{3.6cm}
    >{\raggedright\arraybackslash}p{2cm}
    >{\raggedleft\arraybackslash}p{2cm}
    >{\raggedright\arraybackslash}p{2cm}}
    \toprule
    \multirow[t]{3}{*}{\textbf{\#}} & \multirow[t]{3}{*}{\textbf{Project}} & \multirow[t]{3}{*}{\textbf{LOC Size}} & \textbf{Median build duration (minutes)} & \textbf{Build duration category} \bigstrut\\
    \midrule
    \textbf{1} & alan-turing-institute/sktime & large & 127.3 & longer \bigstrut[t]\\
    \textbf{2} & amark/gun & large & 1.3   & shorter \\
    \textbf{3} & apache/incubator-mxnet & large & 192.8 & longer \\
    \textbf{4} & apache/spark & large & 123.7 & longer \\
    \textbf{5} & apache/superset & large & 10.2  & shorter \\
    \textbf{6} & AUTOMATIC1111/stable-diffusion-webui & medium & 7.4   & shorter \\
    \textbf{7} & BehaviorTree/BehaviorTree.CPP & medium & 3.7   & shorter \\
    \textbf{8} & BLKSerene/Wordless & medium & 37.4  & longer \\
    \textbf{9} & chakki-works/doccano & medium & 2.8   & shorter \\
    \textbf{10} & criteo/tf-yarn & small & 2.4   & shorter \\
    \textbf{11} & DandyDev/slack-machine & small & 2.5   & shorter \\
    \textbf{12} & diffgram/diffgram & large & 5.1   & shorter \\
    \textbf{13} & dmlc/tvm & large & 90.2  & longer \\
    \textbf{14} & FluxML/Metalhead.jl & small & 28.4  & longer \\
    \textbf{15} & FluxML/NNlib.jl & medium & 37.4  & longer \\
    \textbf{16} & huggingface/pytorch-pretrained-BERT & large & 5.4   & shorter \\
    \textbf{17} & huggingface/transformers & large & 5.4   & shorter \\
    \textbf{18} & JohnSnowLabs/spark-nlp & large & 47.0    & longer \\
    \textbf{19} & jtablesaw/tablesaw & medium & 4.3   & shorter \\
    \textbf{20} & kendryte/nncase & large & 147.4 & longer \\
    \textbf{21} & LaurentMazare/tch-rs & large & 7.1   & shorter \\
    \textbf{22} & microsoft/dowhy & medium & 29.8  & longer \\
    \textbf{23} & microsoft/LightGBM & medium & 21.8  & longer \\
    \textbf{24} & microsoft/onnxruntime & large & 99.6  & longer \\
    \textbf{25} & microsoft/pai & medium & 3.9   & shorter \\
    \textbf{26} & mlpack/mlpack & large & 126.1 & longer \\
    \textbf{27} & mne-tools/mne-cpp & large & 62.4  & longer \\
    \textbf{28} & msdslab/automated-systematic-review & medium & 4.7   & shorter \\
    \textbf{29} & nilearn/nilearn & medium & 31.0    & longer \\
    \textbf{30} & opencv/dldt & large & 2.5   & shorter \\
    \textbf{31} & OpenKore/openkore & large & 5.0     & shorter \\
    \textbf{32} & OpenNMT/OpenNMT-py & medium & 3.4   & shorter \\
    \textbf{33} & pytorch/ignite & medium & 29.0    & longer \\
    \textbf{34} & pytorch/tnt & medium & 4.1   & shorter \\
    \textbf{35} & RubixML/RubixML & medium & 3.6   & shorter \\
    \textbf{36} & scikit-learn/scikit-learn & large & 0.3   & shorter \\
    \textbf{37} & SeldonIO/seldon-core & large & 4.6   & shorter \\
    \textbf{38} & shimat/opencvsharp & medium & 20.0    & longer \\
    \textbf{39} & skorch-dev/skorch & medium & 5.3   & shorter \\
    \textbf{40} & smistad/FAST & medium & 46.6  & longer \\
    \textbf{41} & sorgerlab/indra & medium & 19.3  & longer \\
    \textbf{42} & tensorflow/addons & medium & 19.4  & longer \\
    \textbf{43} & tensorly/tensorly & medium & 46.6  & longer \\
    \textbf{44} & tesseract-ocr/tesseract & large & 124.9 & longer \\
    \textbf{45} & Texera/texera & large & 6.8   & shorter \\
    \textbf{46} & TuringLang/Turing.jl & small & 107.1 & longer \\
    \textbf{47} & zhenghaoz/gorse & medium & 6.9   & shorter \bigstrut[b]\\
    \bottomrule
    \end{tabular}%
  \label{tab:project_characteristics}%
\end{table}%

\subsection{Data Collection}

To identify practitioners within the 47 investigated ML projects, we focused on individuals who actively contributed to the projects after the adoption of \textsc{GitHub Actions} CI workflows. Specifically, we selected integrators who either merged or closed at least one pull request (PR) or submitted at least one PR that was successfully merged into the \textit{main/master} branch of the project codebase during this period. This approach ensures that the selected practitioners directly contributed to the project while CI workflows were in use, enabling them to provide relevant and informed insights into CI practices in ML projects.

We collected PR metadata for the studied projects using the \textsc{GitHub API} on June 14, 2024. 
The PR's metadata include details of the PR number, state, author login, base branch, number of additions, deletions, changed files, commit count, and whether the PR was merged or closed, along with the login of the user who closed it.
By analyzing data from the period following the adoption of \textsc{GitHub Actions}, we identified 114,598 PRs reviewed by 3,276 unique integrators. Additionally, 6,861 contributors had at least one PR successfully merged. Importantly, these represent two distinct groups of practitioners in our study: contributors, who authored PRs, and integrators, who merged or rejected them. Among the contributors, 1,909 also acted as integrators, reflecting some overlap between the two roles. After accounting for this overlap, we identified a total of 4,952 unique contributors. Combined with the 3,276 unique integrators, this results in 8,228 practitioners involved in the studied projects.

To contact the practitioners involved in these projects, we collected their email addresses using the \textsc{GitHub API}, ensuring we only collected publicly available information. Of the 3,276 unique integrators, we retrieved 2,060 email addresses; for the 4,952 contributors, we retrieved 2,947. In total, we collected 5,007 unique email addresses from practitioners in the 47 analyzed projects.

To collect our data, we designed a web-based survey and sent invitations by email to all 5,007 ML practitioners whose email addresses were available.
The invitation letter is included in Appendix~\ref{sec:appendix_invitation_email_example}. To encourage participation, we offered respondents the opportunity to win one of ten \$50 Amazon or Steam gift cards, distributed through a random drawing. Participants were eligible for the draw only if they completed all survey questions and explicitly indicated their willingness to participate.
We used \textsc{Mailgun}\footnote{\url{https://www.mailgun.com}} to send personalized email invitations to each practitioner.
If a practitioner was associated with multiple investigated projects, we sent only one invitation, prioritizing the project where they had the highest number of integrated PRs.

In total, we received 155 responses, resulting in a response rate of 3.1\% (\nicefrac{155}{5007}). These responses came from practitioners associated with 30 of the 47 investigated projects. Table~\ref{tab:number_responses_per_respondent_type} presents details on the number of practitioners contacted per project, the responses received, and the corresponding response rates. To maintain anonymity, practitioners' names have been replaced with unique identifiers in the table. For example, practitioner 01 is labeled as ``P1'' and is associated with the \textit{alan-turing-institute/sktime} project.

% Table generated by Excel2LaTeX from sheet 'Sheet1'
\begin{table}[H]
  \centering
  \caption{Number of responses and practitioners of the studied ML projects that were invited to participate.}
\begin{tabular}{cp{3cm}>{\raggedleft\arraybackslash}p{1.7cm}
>{\raggedleft\arraybackslash}p{1.5cm}
>{\raggedleft\arraybackslash}p{1.3cm}
>{\raggedright\arraybackslash}p{1.5cm}}
\toprule
\textbf{\#} & \textbf{Project} & \textbf{Number of practitioners} & \textbf{Number of responses} & {\textbf{Response rate}} & \textbf{Practitioner IDs} \\
\midrule
1 & alan-turing-institute/sktime & 123 & 8 & 6.5\% & P1--P8 \\
2 & amark/gun & 34 & 2 & 5.9\% & P9--P10 \\
3 & apache/spark & 189 & 1 & 0.5\% & P11--P11 \\
4 & apache/superset & 435 & 12 & 2.8\% & P12--P23 \\
5 & AUTOMATIC1111/stable-diffusion-webui & 293 & 8 & 2.7\% & P24--P31 \\
6 & BehaviorTree/BehaviorTree.CPP & 56 & 3 & 5.4\% & P32--P34 \\
7 & chakki-works/doccano & 37 & 1 & 2.7\% & P35--P35 \\
8 & diffgram/diffgram & 8 & 2 & 25.0\% & P36--P37 \\
9 & dmlc/tvm & 399 & 7 & 1.8\% & P38--P44 \\
10 & FluxML/NNlib.jl & 24 & 1 & 4.2\% & P45--P45 \\
11 & huggingface/pytorch-pretrained-BERT & 700 & 18 & 2.6\% & P46--P63 \\
12 & huggingface/transformers & 689 & 19 & 2.8\% & P64--P82 \\
13 & kendryte/nncase & 7 & 1 & 14.3\% & P83--P83 \\
14 & LaurentMazare/tch-rs & 39 & 2 & 5.1\% & P84--P85 \\
15 & microsoft/LightGBM & 89 & 2 & 2.2\% & P86--P87 \\
16 & microsoft/onnxruntime & 196 & 10 & 5.1\% & P88--P97 \\
17 & microsoft/pai & 12 & 1 & 8.3\% & P98--P98 \\
18 & mlpack/mlpack & 38 & 4 & 10.5\% & P99--P102 \\
19 & msdslab/automated-systematic-review & 23 & 1 & 4.3\% & P103--P103 \\
20 & nilearn/nilearn & 58 & 6 & 10.3\% & P104--P109 \\
21 & opencv/dldt & 394 & 4 & 1.0\% & P110--P113 \\
22 & pytorch/tnt & 10 & 1 & 10.0\% & P114--P114 \\
23 & scikit-learn/scikit-learn & 627 & 25 & 4.0\% & P115--P139 \\
24 & SeldonIO/seldon-core & 69 & 2 & 2.9\% & P140--P141 \\
25 & shimat/opencvsharp & 19 & 1 & 5.3\% & P142--P142 \\
26 & sorgerlab/indra & 9 & 1 & 11.1\% & P143--P143 \\
27 & tensorflow/addons & 50 & 3 & 6.0\% & P144--P146 \\
28 & tesseract-ocr/tesseract & 46 & 6 & 13.0\% & P147--P152 \\
29 & TuringLang/Turing.jl & 26 & 2 & 7.7\% & P153--P154 \\
30 & zhenghaoz/gorse & 24 & 1 & 4.2\% & P155--P155 \\
31 & apache/incubator-mxnet & 40 & 0 & 0.0\% & --- \\
32 & BLKSerene/Wordless & 3 & 0 & 0.0\% & --- \\
33 & criteo/tf-yarn & 4 & 0 & 0.0\% & --- \\
34 & DandyDev/slack-machine & 4 & 0 & 0.0\% & --- \\
35 & FluxML/Metalhead.jl & 12 & 0 & 0.0\% & --- \\
36 & JohnSnowLabs/spark-nlp & 30 & 0 & 0.0\% & --- \\
37 & jtablesaw/tablesaw & 18 & 0 & 0.0\% & --- \\
38 & microsoft/dowhy & 20 & 0 & 0.0\% & --- \\
39 & mne-tools/mne-cpp & 5 & 0 & 0.0\% & --- \\
40 & OpenKore/openkore & 11 & 0 & 0.0\% & --- \\
41 & OpenNMT/OpenNMT-py & 18 & 0 & 0.0\% & --- \\
42 & pytorch/ignite & 55 & 0 & 0.0\% & --- \\
43 & RubixML/RubixML & 11 & 0 & 0.0\% & --- \\
44 & skorch-dev/skorch & 9 & 0 & 0.0\% & --- \\
45 & smistad/FAST & 4 & 0 & 0.0\% & --- \\
46 & tensorly/tensorly & 16 & 0 & 0.0\% & --- \\
47 & Texera/texera & 24 & 0 & 0.0\% & --- \\
\midrule
& \textbf{Total} & \textbf{5,007} & \textbf{155} & \textbf{3.1\%} & \\
\bottomrule
\end{tabular}  \label{tab:number_responses_per_respondent_type}%
\end{table}%

Our survey is organized into five major sections, as described in Table \ref{tab:survey_structure}. It includes 20 questions, combining 8 closed- and 12 open-ended questions, designed to collect both quantitative and qualitative data. The estimated completion time is approximately 10 minutes.
To ensure relevance and foster more thoughtful responses, we designed 47 unique questionnaires, each tailored to specific characteristics and statistics of an associated project.
For example, in \textsc{Question \#4.3} of the form sent to the practitioners of the project \textit{tesseract-ocr/tesseract}, we asked: \textit{"When analyzing the data of the \textit{tesseract-ocr/tesseract} project, we observed that this project has a median build duration of 124.9 minutes, which is longer than 90\% of the investigated projects of similar size. Do you have any insights into why this project has a longer build duration?"}.
This customization allowed us to provide participants with context-specific data, enabling them to offer richer and more meaningful insights about their respective projects.
A complete example of the survey is available in our online Appendix\footnote{\url{https://zenodo.org/records/14902811}}, which includes the customized questionnaire sent to participants of the \textit{tesseract-ocr/tesseract}\footnote{\url{http://github.com/tesseract-ocr/tesseract}} project.


\begin{table}
\centering
\caption{Survey Structure and Description.}
\label{tab:survey_structure}
\begin{tabular}{p{3cm}p{8cm}}
\toprule
\textbf{Section} & \textbf{Description} \\
\midrule
Participant Information & Collects demographic data and information about participants' experience, including their experience contributing to ML projects, as well as familiarity with CI practices. \\
\midrule
Perceptions about CI Practices & Gathers insights into the challenges and differences when adopting CI practices in ML projects compared to non-ML projects. Focuses on team practices such as maintaining short build durations, frequent commits, and upholding high test coverage. \\
\midrule
Reflection on Previous Findings & Explores participants' views on results from our prior study, focusing on disparities in CI adoption between ML and non-ML projects. The questions explore the underlying factors contributing to differences in build durations, test coverage, and potential strategies for enhancement. \\
\midrule
Project-Specific Analysis & Presents data derived from a specific project (e.g., \textit{tesseract-ocr/tesseract}) to solicit feedback on unique challenges and techniques for enhancing CI practices in a real-world scenario. \\
\midrule
Conclusion and Follow-Up & Allows participants to opt into follow-up interviews, request updates on study findings, and share additional comments. Ensures eligibility for the gift card drawing by confirming survey completion. \\
\bottomrule
\end{tabular}
\end{table}

To encourage participation and a higher response rate, none of the questions in our survey were mandatory. As a result, the number of responses for each question varied, as not all participants answered every question. Responses were marked as ``NA'' (No Answer) if a participant left a question blank.
Table~\ref{tab:response_rates} provides a detailed overview of each survey question, including its description, type (open-ended or close-ended), and corresponding response rates. Close-ended questions generally achieved higher response rates, with several receiving complete responses (e.g., Questions 1.1–1.4 at 100\%). Conversely, open-ended questions exhibited slightly lower response rates, with the lowest being 85.2\% (Question 3.7). This trend indicates that open-ended questions, which typically require more effort and time to answer, may discourage some participants from responding. Nonetheless, the consistently high response rates across all question types highlight strong engagement from the participants.

\begin{table}
\centering
\caption{Survey Questions and Response Rates.}
\label{tab:response_rates}
\begin{tabular}{cp{6cm}lr}
\toprule
\textbf{\#} & \textbf{Question Description} & \textbf{Question Type} & \textbf{Responses (Rate)} \\
\midrule
1.1 & Experience developing software & Close-ended & 155/155 (100\%) \\
1.2 & Experience developing ML projects & Close-ended & 155/155 (100\%) \\
1.3 & Primary roles in ML projects & Close-ended & 155/155 (100\%) \\
1.4 & Familiarity with CI concepts & Close-ended & 155/155 (100\%) \\
2.1 & ML projects strive to incorporate CI practices & Open-ended & 146/155 (94.2\%) \\
2.2 & Challenges or differences when implementing a CI pipeline & Open-ended & 142/155 (91.6\%) \\
2.3 & ML projects commit more frequently & Close-ended & 149/155 (96.1\%) \\
2.4 & ML projects have longer build durations & Close-ended & 151/155 (97.4\%) \\
2.5 & ML projects have lower test coverage & Close-ended & 150/155 (96.8\%) \\
2.6 & ML projects fix broken builds more quickly & Close-ended & 149/155 (96.1\%) \\
3.1 & Importance of ML projects keeping a short build duration & Close-ended & 151/155 (97.4\%) \\
3.2 & Perceptions about previous study results on build duration in ML projects & Open-ended & 141/155 (91\%) \\
3.3 & Strategies to reduce build duration in ML projects & Open-ended & 138/155 (89\%) \\
3.4 & Acceptable test coverage rate for an ML project & Close-ended & 153/155 (98.7\%) \\
3.5 & Perceptions about previous study results on test coverage in ML projects & Open-ended & 135/155 (87.1\%) \\
3.6 & Challenges in testing ML projects & Open-ended & 133/155 (85.8\%) \\
3.7 & Strategies to enhance test coverage in ML projects & Open-ended & 132/155 (85.2\%) \\
4.1 & Familiarity with the CI pipeline of the studied project & Close-ended & 154/155 (99.4\%) \\
4.2 & Acceptable build duration for ML projects & Close-ended & 150/155 (96.8\%) \\
4.3 & Perceptions about the build duration of the studied project & Open-ended & 135/155 (87.1\%) \\
\bottomrule
\end{tabular}
\end{table}


\subsection{Analytical Approach}

We applied an inductive thematic analysis to identify, analyze, and report themes within the qualitative data collected from our questionnaire, following the approach outlined by \cite{braun2006using}. To ensure rigor and transparency in the process, we adhered to the guidelines proposed by \cite{nowell2017thematic}.

The initial step of our thematic analysis involved open-coding the qualitative data. This process refers to assigning codes to relevant segments of data collected from the responses to our open-ended survey questions. Each question was coded by at least two authors, enhancing the robustness of the analysis and mitigating potential bias.
The first author conducted open coding for all eight open-ended questions in the survey, and to ensure reliability in the coding process, the second and third authors coded responses for three questions, and the fourth author coded two questions. 
Afterwards, the fifth author reviewed the entire set of codes generated by the two coders of each question. This review process helped resolve disagreements or ambiguities, refine the coding, and add additional entries where necessary.

Once the coding process was completed, the first author performed axial coding, grouping codes into higher-level themes. These themes represented broader conceptual constructs, organizing multiple related codes under a common idea. For example, a single theme might encompass several related codes addressing a specific aspect of CI practices in ML projects.


% To gain deeper insights into the relationships among the themes identified in our thematic analysis, we conducted an exploratory co-occurrence analysis. This approach examines the frequency with which different themes appear together in participants' responses, uncovering patterns and connections that provide a more nuanced understanding of the data.

Finally, we report the codes and themes derived from our thematic analysis in the results section.
When presenting our findings, we indicate the number of quotes associated with each code and theme using superscripts. 
However, it is important to note that these numbers do not necessarily indicate the relevance or significance of a code. For instance, a code may be cited in more quotes simply because it is more easily remembered by participants, rather than due to its importance.
To provide further context and depth, we include representative quotes from participants. To maintain anonymity, participant names are replaced with unique IDs.

While textual representations highlight key insights, we also employ network mapping charts to provide a structured visual representation of the relationships between themes and codes.
At the center of the network lies the core theme, encapsulating the primary focus of the RQ. Surrounding it are second-level themes, which further break down into third-level themes (codes), organized based on their conceptual relationships. Figure~\ref{fig:network_mapping_chart_example} presents an example of a Network Mapping Chart, illustrating these relationships in the thematic analysis.
Each third-level theme (code) offers granular insights into specific aspects of the data. The thickness of the edges in the network represents frequency, indicating how prominently each code appeared during the analysis.

\begin{figure}[H]
	\centering
	\includegraphics[width=0.7\textwidth]{network_graph_example.png}
	% figure caption is below the figure
	\caption{Example of a Network Mapping Chart visualizing the relationships between themes and codes in thematic analysis.}
	\label{fig:network_mapping_chart_example}       % Give a unique label
\end{figure}


% Each second-level theme is further broken down into third-level themes (or codes), which provide more granular insights into specific aspects of the data. These codes are grouped based on their conceptual or contextual relationship with the higher-level themes. The thickness of the edges in the network reflects the frequency of occurrences, indicating how prominently each code appeared during the analysis.

% To enhance the interpretation of the codes, we include representative quotes from our participants. To maintain anonymity, participants’ names are replaced with unique IDs. 
% For example, when discussing perceived differences and challenges in adopting a CI pipeline in ML projects (RQ1), practitioner 57, who contributed to the \textit{huggingface/pytorch-pretrained-BERT} project (see Table \ref{tab:number_responses_per_respondent_type}), stated: \textit{``ML projects often are stochastic or make statistical guarantees in nature, and require different forms of testing/verification as a result''}. In this instance, the participant's name was replaced with P57. This specific response was coded under \textit{``Non-determinism''} and grouped into the broader theme of \textit{``Testing Complexity''}.

\begin{table}[htbp!]
\resizebox{\columnwidth}{!}{%
\begin{tabular}{@{}l|ccc|c@{}}
 & Liberal & Moderate & Conservative & Total \\ \hline
Female & 223 & 114 & 45 & 382 \\
Male & 102 & 78 & 53 & 233 \\
Prefer not to say & 2 & 0 & 0 & 2 \\ \hline
Total & 327 & 192 & 98 & 617
\end{tabular}%
}
\caption{Annotator Demographics. All annotators are based in the United States. The table shows the number of annotators across ideology and sex categories, as self-reported to Prolific. The mean age is 38.3 (SD=12.7), and 45 annotators are immigrants (7.3\%).}
\label{tab:demographics}
\end{table}



\begin{table*}[t]
\centering
\fontsize{11pt}{11pt}\selectfont
\begin{tabular}{lllllllllllll}
\toprule
\multicolumn{1}{c}{\textbf{task}} & \multicolumn{2}{c}{\textbf{Mir}} & \multicolumn{2}{c}{\textbf{Lai}} & \multicolumn{2}{c}{\textbf{Ziegen.}} & \multicolumn{2}{c}{\textbf{Cao}} & \multicolumn{2}{c}{\textbf{Alva-Man.}} & \multicolumn{1}{c}{\textbf{avg.}} & \textbf{\begin{tabular}[c]{@{}l@{}}avg.\\ rank\end{tabular}} \\
\multicolumn{1}{c}{\textbf{metrics}} & \multicolumn{1}{c}{\textbf{cor.}} & \multicolumn{1}{c}{\textbf{p-v.}} & \multicolumn{1}{c}{\textbf{cor.}} & \multicolumn{1}{c}{\textbf{p-v.}} & \multicolumn{1}{c}{\textbf{cor.}} & \multicolumn{1}{c}{\textbf{p-v.}} & \multicolumn{1}{c}{\textbf{cor.}} & \multicolumn{1}{c}{\textbf{p-v.}} & \multicolumn{1}{c}{\textbf{cor.}} & \multicolumn{1}{c}{\textbf{p-v.}} &  &  \\ \midrule
\textbf{S-Bleu} & 0.50 & 0.0 & 0.47 & 0.0 & 0.59 & 0.0 & 0.58 & 0.0 & 0.68 & 0.0 & 0.57 & 5.8 \\
\textbf{R-Bleu} & -- & -- & 0.27 & 0.0 & 0.30 & 0.0 & -- & -- & -- & -- & - &  \\
\textbf{S-Meteor} & 0.49 & 0.0 & 0.48 & 0.0 & 0.61 & 0.0 & 0.57 & 0.0 & 0.64 & 0.0 & 0.56 & 6.1 \\
\textbf{R-Meteor} & -- & -- & 0.34 & 0.0 & 0.26 & 0.0 & -- & -- & -- & -- & - &  \\
\textbf{S-Bertscore} & \textbf{0.53} & 0.0 & {\ul 0.80} & 0.0 & \textbf{0.70} & 0.0 & {\ul 0.66} & 0.0 & {\ul0.78} & 0.0 & \textbf{0.69} & \textbf{1.7} \\
\textbf{R-Bertscore} & -- & -- & 0.51 & 0.0 & 0.38 & 0.0 & -- & -- & -- & -- & - &  \\
\textbf{S-Bleurt} & {\ul 0.52} & 0.0 & {\ul 0.80} & 0.0 & 0.60 & 0.0 & \textbf{0.70} & 0.0 & \textbf{0.80} & 0.0 & {\ul 0.68} & {\ul 2.3} \\
\textbf{R-Bleurt} & -- & -- & 0.59 & 0.0 & -0.05 & 0.13 & -- & -- & -- & -- & - &  \\
\textbf{S-Cosine} & 0.51 & 0.0 & 0.69 & 0.0 & {\ul 0.62} & 0.0 & 0.61 & 0.0 & 0.65 & 0.0 & 0.62 & 4.4 \\
\textbf{R-Cosine} & -- & -- & 0.40 & 0.0 & 0.29 & 0.0 & -- & -- & -- & -- & - & \\ \midrule
\textbf{QuestEval} & 0.23 & 0.0 & 0.25 & 0.0 & 0.49 & 0.0 & 0.47 & 0.0 & 0.62 & 0.0 & 0.41 & 9.0 \\
\textbf{LLaMa3} & 0.36 & 0.0 & \textbf{0.84} & 0.0 & {\ul{0.62}} & 0.0 & 0.61 & 0.0 &  0.76 & 0.0 & 0.64 & 3.6 \\
\textbf{our (3b)} & 0.49 & 0.0 & 0.73 & 0.0 & 0.54 & 0.0 & 0.53 & 0.0 & 0.7 & 0.0 & 0.60 & 5.8 \\
\textbf{our (8b)} & 0.48 & 0.0 & 0.73 & 0.0 & 0.52 & 0.0 & 0.53 & 0.0 & 0.7 & 0.0 & 0.59 & 6.3 \\  \bottomrule
\end{tabular}
\caption{Pearson correlation on human evaluation on system output. `R-': reference-based. `S-': source-based.}
\label{tab:sys}
\end{table*}



\begin{table}%[]
\centering
\fontsize{11pt}{11pt}\selectfont
\begin{tabular}{llllll}
\toprule
\multicolumn{1}{c}{\textbf{task}} & \multicolumn{1}{c}{\textbf{Lai}} & \multicolumn{1}{c}{\textbf{Zei.}} & \multicolumn{1}{c}{\textbf{Scia.}} & \textbf{} & \textbf{} \\ 
\multicolumn{1}{c}{\textbf{metrics}} & \multicolumn{1}{c}{\textbf{cor.}} & \multicolumn{1}{c}{\textbf{cor.}} & \multicolumn{1}{c}{\textbf{cor.}} & \textbf{avg.} & \textbf{\begin{tabular}[c]{@{}l@{}}avg.\\ rank\end{tabular}} \\ \midrule
\textbf{S-Bleu} & 0.40 & 0.40 & 0.19* & 0.33 & 7.67 \\
\textbf{S-Meteor} & 0.41 & 0.42 & 0.16* & 0.33 & 7.33 \\
\textbf{S-BertS.} & {\ul0.58} & 0.47 & 0.31 & 0.45 & 3.67 \\
\textbf{S-Bleurt} & 0.45 & {\ul 0.54} & {\ul 0.37} & 0.45 & {\ul 3.33} \\
\textbf{S-Cosine} & 0.56 & 0.52 & 0.3 & {\ul 0.46} & {\ul 3.33} \\ \midrule
\textbf{QuestE.} & 0.27 & 0.35 & 0.06* & 0.23 & 9.00 \\
\textbf{LlaMA3} & \textbf{0.6} & \textbf{0.67} & \textbf{0.51} & \textbf{0.59} & \textbf{1.0} \\
\textbf{Our (3b)} & 0.51 & 0.49 & 0.23* & 0.39 & 4.83 \\
\textbf{Our (8b)} & 0.52 & 0.49 & 0.22* & 0.43 & 4.83 \\ \bottomrule
\end{tabular}
\caption{Pearson correlation on human ratings on reference output. *not significant; we cannot reject the null hypothesis of zero correlation}
\label{tab:ref}
\end{table}


\begin{table*}%[]
\centering
\fontsize{11pt}{11pt}\selectfont
\begin{tabular}{lllllllll}
\toprule
\textbf{task} & \multicolumn{1}{c}{\textbf{ALL}} & \multicolumn{1}{c}{\textbf{sentiment}} & \multicolumn{1}{c}{\textbf{detoxify}} & \multicolumn{1}{c}{\textbf{catchy}} & \multicolumn{1}{c}{\textbf{polite}} & \multicolumn{1}{c}{\textbf{persuasive}} & \multicolumn{1}{c}{\textbf{formal}} & \textbf{\begin{tabular}[c]{@{}l@{}}avg. \\ rank\end{tabular}} \\
\textbf{metrics} & \multicolumn{1}{c}{\textbf{cor.}} & \multicolumn{1}{c}{\textbf{cor.}} & \multicolumn{1}{c}{\textbf{cor.}} & \multicolumn{1}{c}{\textbf{cor.}} & \multicolumn{1}{c}{\textbf{cor.}} & \multicolumn{1}{c}{\textbf{cor.}} & \multicolumn{1}{c}{\textbf{cor.}} &  \\ \midrule
\textbf{S-Bleu} & -0.17 & -0.82 & -0.45 & -0.12* & -0.1* & -0.05 & -0.21 & 8.42 \\
\textbf{R-Bleu} & - & -0.5 & -0.45 &  &  &  &  &  \\
\textbf{S-Meteor} & -0.07* & -0.55 & -0.4 & -0.01* & 0.1* & -0.16 & -0.04* & 7.67 \\
\textbf{R-Meteor} & - & -0.17* & -0.39 & - & - & - & - & - \\
\textbf{S-BertScore} & 0.11 & -0.38 & -0.07* & -0.17* & 0.28 & 0.12 & 0.25 & 6.0 \\
\textbf{R-BertScore} & - & -0.02* & -0.21* & - & - & - & - & - \\
\textbf{S-Bleurt} & 0.29 & 0.05* & 0.45 & 0.06* & 0.29 & 0.23 & 0.46 & 4.2 \\
\textbf{R-Bleurt} & - &  0.21 & 0.38 & - & - & - & - & - \\
\textbf{S-Cosine} & 0.01* & -0.5 & -0.13* & -0.19* & 0.05* & -0.05* & 0.15* & 7.42 \\
\textbf{R-Cosine} & - & -0.11* & -0.16* & - & - & - & - & - \\ \midrule
\textbf{QuestEval} & 0.21 & {\ul{0.29}} & 0.23 & 0.37 & 0.19* & 0.35 & 0.14* & 4.67 \\
\textbf{LlaMA3} & \textbf{0.82} & \textbf{0.80} & \textbf{0.72} & \textbf{0.84} & \textbf{0.84} & \textbf{0.90} & \textbf{0.88} & \textbf{1.00} \\
\textbf{Our (3b)} & 0.47 & -0.11* & 0.37 & 0.61 & 0.53 & 0.54 & 0.66 & 3.5 \\
\textbf{Our (8b)} & {\ul{0.57}} & 0.09* & {\ul 0.49} & {\ul 0.72} & {\ul 0.64} & {\ul 0.62} & {\ul 0.67} & {\ul 2.17} \\ \bottomrule
\end{tabular}
\caption{Pearson correlation on human ratings on our constructed test set. 'R-': reference-based. 'S-': source-based. *not significant; we cannot reject the null hypothesis of zero correlation}
\label{tab:con}
\end{table*}

\section{Results}
We benchmark the different metrics on the different datasets using correlation to human judgement. For content preservation, we show results split on data with system output, reference output and our constructed test set: we show that the data source for evaluation leads to different conclusions on the metrics. In addition, we examine whether the metrics can rank style transfer systems similar to humans. On style strength, we likewise show correlations between human judgment and zero-shot evaluation approaches. When applicable, we summarize results by reporting the average correlation. And the average ranking of the metric per dataset (by ranking which metric obtains the highest correlation to human judgement per dataset). 

\subsection{Content preservation}
\paragraph{How do data sources affect the conclusion on best metric?}
The conclusions about the metrics' performance change radically depending on whether we use system output data, reference output, or our constructed test set. Ideally, a good metric correlates highly with humans on any data source. Ideally, for meta-evaluation, a metric should correlate consistently across all data sources, but the following shows that the correlations indicate different things, and the conclusion on the best metric should be drawn carefully.

Looking at the metrics correlations with humans on the data source with system output (Table~\ref{tab:sys}), we see a relatively high correlation for many of the metrics on many tasks. The overall best metrics are S-BertScore and S-BLEURT (avg+avg rank). We see no notable difference in our method of using the 3B or 8B model as the backbone.

Examining the average correlations based on data with reference output (Table~\ref{tab:ref}), now the zero-shoot prompting with LlaMA3 70B is the best-performing approach ($0.59$ avg). Tied for second place are source-based cosine embedding ($0.46$ avg), BLEURT ($0.45$ avg) and BertScore ($0.45$ avg). Our method follows on a 5. place: here, the 8b version (($0.43$ avg)) shows a bit stronger results than 3b ($0.39$ avg). The fact that the conclusions change, whether looking at reference or system output, confirms the observations made by \citet{scialom-etal-2021-questeval} on simplicity transfer.   

Now consider the results on our test set (Table~\ref{tab:con}): Several metrics show low or no correlation; we even see a significantly negative correlation for some metrics on ALL (BLEU) and for specific subparts of our test set for BLEU, Meteor, BertScore, Cosine. On the other end, LlaMA3 70B is again performing best, showing strong results ($0.82$ in ALL). The runner-up is now our 8B method, with a gap to the 3B version ($0.57$ vs $0.47$ in ALL). Note our method still shows zero correlation for the sentiment task. After, ranks BLEURT ($0.29$), QuestEval ($0.21$), BertScore ($0.11$), Cosine ($0.01$).  

On our test set, we find that some metrics that correlate relatively well on the other datasets, now exhibit low correlation. Hence, with our test set, we can now support the logical reasoning with data evidence: Evaluation of content preservation for style transfer needs to take the style shift into account. This conclusion could not be drawn using the existing data sources: We hypothesise that for the data with system-based output, successful output happens to be very similar to the source sentence and vice versa, and reference-based output might not contain server mistakes as they are gold references. Thus, none of the existing data sources tests the limits of the metrics.  


\paragraph{How do reference-based metrics compare to source-based ones?} Reference-based metrics show a lower correlation than the source-based counterpart for all metrics on both datasets with ratings on references (Table~\ref{tab:sys}). As discussed previously, reference-based metrics for style transfer have the drawback that many different good solutions on a rewrite might exist and not only one similar to a reference.


\paragraph{How well can the metrics rank the performance of style transfer methods?}
We compare the metrics' ability to judge the best style transfer methods w.r.t. the human annotations: Several of the data sources contain samples from different style transfer systems. In order to use metrics to assess the quality of the style transfer system, metrics should correctly find the best-performing system. Hence, we evaluate whether the metrics for content preservation provide the same system ranking as human evaluators. We take the mean of the score for every output on each system and the mean of the human annotations; we compare the systems using the Kendall's Tau correlation. 

We find only the evaluation using the dataset Mir, Lai, and Ziegen to result in significant correlations, probably because of sparsity in a number of system tests (App.~\ref{app:dataset}). Our method (8b) is the only metric providing a perfect ranking of the style transfer system on the Lai data, and Llama3 70B the only one on the Ziegen data. Results in App.~\ref{app:results}. 


\subsection{Style strength results}
%Evaluating style strengths is a challenging task. 
Llama3 70B shows better overall results than our method. However, our method scores higher than Llama3 70B on 2 out of 6 datasets, but it also exhibits zero correlation on one task (Table~\ref{tab:styleresults}).%More work i s needed on evaluating style strengths. 
 
\begin{table}%[]
\fontsize{11pt}{11pt}\selectfont
\begin{tabular}{lccc}
\toprule
\multicolumn{1}{c}{\textbf{}} & \textbf{LlaMA3} & \textbf{Our (3b)} & \textbf{Our (8b)} \\ \midrule
\textbf{Mir} & 0.46 & 0.54 & \textbf{0.57} \\
\textbf{Lai} & \textbf{0.57} & 0.18 & 0.19 \\
\textbf{Ziegen.} & 0.25 & 0.27 & \textbf{0.32} \\
\textbf{Alva-M.} & \textbf{0.59} & 0.03* & 0.02* \\
\textbf{Scialom} & \textbf{0.62} & 0.45 & 0.44 \\
\textbf{\begin{tabular}[c]{@{}l@{}}Our Test\end{tabular}} & \textbf{0.63} & 0.46 & 0.48 \\ \bottomrule
\end{tabular}
\caption{Style strength: Pearson correlation to human ratings. *not significant; we cannot reject the null hypothesis of zero corelation}
\label{tab:styleresults}
\end{table}

\subsection{Ablation}
We conduct several runs of the methods using LLMs with variations in instructions/prompts (App.~\ref{app:method}). We observe that the lower the correlation on a task, the higher the variation between the different runs. For our method, we only observe low variance between the runs.
None of the variations leads to different conclusions of the meta-evaluation. Results in App.~\ref{app:results}.
\section{Discussion of Assumptions}\label{sec:discussion}
In this paper, we have made several assumptions for the sake of clarity and simplicity. In this section, we discuss the rationale behind these assumptions, the extent to which these assumptions hold in practice, and the consequences for our protocol when these assumptions hold.

\subsection{Assumptions on the Demand}

There are two simplifying assumptions we make about the demand. First, we assume the demand at any time is relatively small compared to the channel capacities. Second, we take the demand to be constant over time. We elaborate upon both these points below.

\paragraph{Small demands} The assumption that demands are small relative to channel capacities is made precise in \eqref{eq:large_capacity_assumption}. This assumption simplifies two major aspects of our protocol. First, it largely removes congestion from consideration. In \eqref{eq:primal_problem}, there is no constraint ensuring that total flow in both directions stays below capacity--this is always met. Consequently, there is no Lagrange multiplier for congestion and no congestion pricing; only imbalance penalties apply. In contrast, protocols in \cite{sivaraman2020high, varma2021throughput, wang2024fence} include congestion fees due to explicit congestion constraints. Second, the bound \eqref{eq:large_capacity_assumption} ensures that as long as channels remain balanced, the network can always meet demand, no matter how the demand is routed. Since channels can rebalance when necessary, they never drop transactions. This allows prices and flows to adjust as per the equations in \eqref{eq:algorithm}, which makes it easier to prove the protocol's convergence guarantees. This also preserves the key property that a channel's price remains proportional to net money flow through it.

In practice, payment channel networks are used most often for micro-payments, for which on-chain transactions are prohibitively expensive; large transactions typically take place directly on the blockchain. For example, according to \cite{river2023lightning}, the average channel capacity is roughly $0.1$ BTC ($5,000$ BTC distributed over $50,000$ channels), while the average transaction amount is less than $0.0004$ BTC ($44.7k$ satoshis). Thus, the small demand assumption is not too unrealistic. Additionally, the occasional large transaction can be treated as a sequence of smaller transactions by breaking it into packets and executing each packet serially (as done by \cite{sivaraman2020high}).
Lastly, a good path discovery process that favors large capacity channels over small capacity ones can help ensure that the bound in \eqref{eq:large_capacity_assumption} holds.

\paragraph{Constant demands} 
In this work, we assume that any transacting pair of nodes have a steady transaction demand between them (see Section \ref{sec:transaction_requests}). Making this assumption is necessary to obtain the kind of guarantees that we have presented in this paper. Unless the demand is steady, it is unreasonable to expect that the flows converge to a steady value. Weaker assumptions on the demand lead to weaker guarantees. For example, with the more general setting of stochastic, but i.i.d. demand between any two nodes, \cite{varma2021throughput} shows that the channel queue lengths are bounded in expectation. If the demand can be arbitrary, then it is very hard to get any meaningful performance guarantees; \cite{wang2024fence} shows that even for a single bidirectional channel, the competitive ratio is infinite. Indeed, because a PCN is a decentralized system and decisions must be made based on local information alone, it is difficult for the network to find the optimal detailed balance flow at every time step with a time-varying demand.  With a steady demand, the network can discover the optimal flows in a reasonably short time, as our work shows.

We view the constant demand assumption as an approximation for a more general demand process that could be piece-wise constant, stochastic, or both (see simulations in Figure \ref{fig:five_nodes_variable_demand}).
We believe it should be possible to merge ideas from our work and \cite{varma2021throughput} to provide guarantees in a setting with random demands with arbitrary means. We leave this for future work. In addition, our work suggests that a reasonable method of handling stochastic demands is to queue the transaction requests \textit{at the source node} itself. This queuing action should be viewed in conjunction with flow-control. Indeed, a temporarily high unidirectional demand would raise prices for the sender, incentivizing the sender to stop sending the transactions. If the sender queues the transactions, they can send them later when prices drop. This form of queuing does not require any overhaul of the basic PCN infrastructure and is therefore simpler to implement than per-channel queues as suggested by \cite{sivaraman2020high} and \cite{varma2021throughput}.

\subsection{The Incentive of Channels}
The actions of the channels as prescribed by the DEBT control protocol can be summarized as follows. Channels adjust their prices in proportion to the net flow through them. They rebalance themselves whenever necessary and execute any transaction request that has been made of them. We discuss both these aspects below.

\paragraph{On Prices}
In this work, the exclusive role of channel prices is to ensure that the flows through each channel remains balanced. In practice, it would be important to include other components in a channel's price/fee as well: a congestion price  and an incentive price. The congestion price, as suggested by \cite{varma2021throughput}, would depend on the total flow of transactions through the channel, and would incentivize nodes to balance the load over different paths. The incentive price, which is commonly used in practice \cite{river2023lightning}, is necessary to provide channels with an incentive to serve as an intermediary for different channels. In practice, we expect both these components to be smaller than the imbalance price. Consequently, we expect the behavior of our protocol to be similar to our theoretical results even with these additional prices.

A key aspect of our protocol is that channel fees are allowed to be negative. Although the original Lightning network whitepaper \cite{poon2016bitcoin} suggests that negative channel prices may be a good solution to promote rebalancing, the idea of negative prices in not very popular in the literature. To our knowledge, the only prior work with this feature is \cite{varma2021throughput}. Indeed, in papers such as \cite{van2021merchant} and \cite{wang2024fence}, the price function is explicitly modified such that the channel price is never negative. The results of our paper show the benefits of negative prices. For one, in steady state, equal flows in both directions ensure that a channel doesn't loose any money (the other price components mentioned above ensure that the channel will only gain money). More importantly, negative prices are important to ensure that the protocol selectively stifles acyclic flows while allowing circulations to flow. Indeed, in the example of Section \ref{sec:flow_control_example}, the flows between nodes $A$ and $C$ are left on only because the large positive price over one channel is canceled by the corresponding negative price over the other channel, leading to a net zero price.

Lastly, observe that in the DEBT control protocol, the price charged by a channel does not depend on its capacity. This is a natural consequence of the price being the Lagrange multiplier for the net-zero flow constraint, which also does not depend on the channel capacity. In contrast, in many other works, the imbalance price is normalized by the channel capacity \cite{ren2018optimal, lin2020funds, wang2024fence}; this is shown to work well in practice. The rationale for such a price structure is explained well in \cite{wang2024fence}, where this fee is derived with the aim of always maintaining some balance (liquidity) at each end of every channel. This is a reasonable aim if a channel is to never rebalance itself; the experiments of the aforementioned papers are conducted in such a regime. In this work, however, we allow the channels to rebalance themselves a few times in order to settle on a detailed balance flow. This is because our focus is on the long-term steady state performance of the protocol. This difference in perspective also shows up in how the price depends on the channel imbalance. \cite{lin2020funds} and \cite{wang2024fence} advocate for strictly convex prices whereas this work and \cite{varma2021throughput} propose linear prices.

\paragraph{On Rebalancing} 
Recall that the DEBT control protocol ensures that the flows in the network converge to a detailed balance flow, which can be sustained perpetually without any rebalancing. However, during the transient phase (before convergence), channels may have to perform on-chain rebalancing a few times. Since rebalancing is an expensive operation, it is worthwhile discussing methods by which channels can reduce the extent of rebalancing. One option for the channels to reduce the extent of rebalancing is to increase their capacity; however, this comes at the cost of locking in more capital. Each channel can decide for itself the optimum amount of capital to lock in. Another option, which we discuss in Section \ref{sec:five_node}, is for channels to increase the rate $\gamma$ at which they adjust prices. 

Ultimately, whether or not it is beneficial for a channel to rebalance depends on the time-horizon under consideration. Our protocol is based on the assumption that the demand remains steady for a long period of time. If this is indeed the case, it would be worthwhile for a channel to rebalance itself as it can make up this cost through the incentive fees gained from the flow of transactions through it in steady state. If a channel chooses not to rebalance itself, however, there is a risk of being trapped in a deadlock, which is suboptimal for not only the nodes but also the channel.

\section{Conclusion}
This work presents DEBT control: a protocol for payment channel networks that uses source routing and flow control based on channel prices. The protocol is derived by posing a network utility maximization problem and analyzing its dual minimization. It is shown that under steady demands, the protocol guides the network to an optimal, sustainable point. Simulations show its robustness to demand variations. The work demonstrates that simple protocols with strong theoretical guarantees are possible for PCNs and we hope it inspires further theoretical research in this direction.
\section{Limitation}
The use of 3D-printed PLA for structural components improves improving ease of assembly and reduces weight and cost, yet it causes deformation under heavy load, which can diminish end-effector precision. Using metal, such as aluminum, would remedy this problem. Additionally, \robot relies on integrated joint relative encoders, requiring manual initialization in a fixed joint configuration each time the system is powered on. Using absolute joint encoders could significantly improve accuracy and ease of use, although it would increase the overall cost. 

%Reliance on commercially available actuators simplifies integration but imposes constraints on control frequency and customization, further limiting the potential for tailored performance improvements.

% The 6 DoF configuration provides sufficient mobility for most tasks; however, certain bimanual operations could benefit from an additional degree of freedom to handle complex joint constraints more effectively. Furthermore, the limited torque density of commercially available proprioceptive actuators restricts the payload and torque output, making the system less suitability for handling heavier loads or high-torque applications. 

The 6 DoF configuration of the arm provides sufficient mobility for single-arm manipulation tasks, yet it shows a limitation in certain bimanual manipulation problems. Specifically, when \robot holds onto a rigid object with both hands, each arm loses 1 DoF because the hands are fixed to the object during grasping. This leads to an underactuated kinematic chain which has a limited mobility in 3D space. We can achieve more mobility by letting the object slip inside the grippers, yet this renders the grasp less robust and simulation difficult. Therefore, we anticipate that designing a lightweight 3 DoF wrist in place of the current 2 DoF wrist allows a more diverse repertoire of manipulation in bimanual tasks.

Finally, the limited torque density of commercially available proprioceptive actuators restricts the performance. Currently, all of our actuators feature a 1:10 gear ratio, so \robot can handle up to 2.5 kg of payload. To handle a heavier object and manipulate it with higher torque, we expect the actuator to have 1:20$\sim$30 gear ratio, but it is difficult to find an off-the-shelf product that meets our requirements. Customizing the actuator to increase the torque density while minimizing the weight will enable \robot to move faster and handle more diverse objects.

%These constraints highlight opportunities for improvement in future iterations, including alternative materials for enhanced rigidity, custom actuator designs for higher control precision and torque density, the adoption of absolute joint encoders, and optimized configurations to balance dexterity and weight.


\section{Conclusion}
In this work, we propose a simple yet effective approach, called SMILE, for graph few-shot learning with fewer tasks. Specifically, we introduce a novel dual-level mixup strategy, including within-task and across-task mixup, for enriching the diversity of nodes within each task and the diversity of tasks. Also, we incorporate the degree-based prior information to learn expressive node embeddings. Theoretically, we prove that SMILE effectively enhances the model's generalization performance. Empirically, we conduct extensive experiments on multiple benchmarks and the results suggest that SMILE significantly outperforms other baselines, including both in-domain and cross-domain few-shot settings.

% Your text comes here. Separate text sections with
% \section{Section title}
% \label{sec:1}
% Text with citations \cite{RefB} and \cite{RefJ}.
% \subsection{Subsection title}
% \label{sec:2}
% as required. Don't forget to give each section
% and subsection a unique label (see Sect.~\ref{sec:1}).
% \paragraph{Paragraph headings} Use paragraph headings as needed.
% \begin{equation}
% a^2+b^2=c^2
% \end{equation}

% For one-column wide figures use
% \begin{figure}
% % Use the relevant command to insert your figure file.
% % For example, with the graphicx package use
%   \includegraphics{example.eps}
% % figure caption is below the figure
% \caption{Please write your figure caption here}
% \label{fig:1}       % Give a unique label
% \end{figure}
%
% For two-column wide figures use
% \begin{figure*}
% % Use the relevant command to insert your figure file.
% % For example, with the graphicx package use
%   \includegraphics[width=0.75\textwidth]{example.eps}
% % figure caption is below the figure
% \caption{Please write your figure caption here}
% \label{fig:2}       % Give a unique label
% \end{figure*}
%
% For tables use
% \begin{table}
% % table caption is above the table
% \caption{Please write your table caption here}
% \label{tab:1}       % Give a unique label
% % For LaTeX tables use
% \begin{tabular}{lll}
% \hline\noalign{\smallskip}
% first & second & third  \\
% \noalign{\smallskip}\hline\noalign{\smallskip}
% number & number & number \\
% number & number & number \\
% \noalign{\smallskip}\hline
% \end{tabular}
% \end{table}


%\begin{acknowledgements}
%If you'd like to thank anyone, place your comments here
%and remove the percent signs.
%\end{acknowledgements}


% Authors must disclose all relationships or interests that 
% could have direct or potential influence or impart bias on 
% the work: 
%
% \section*{Conflict of interest}
%
% The authors declare that they have no conflict of interest.


% BibTeX users please use one of
%\bibliographystyle{spbasic}      % basic style, author-year citations
%\bibliographystyle{spmpsci}      % mathematics and physical sciences
%\bibliographystyle{spphys}       % APS-like style for physics
%\bibliography{}   % name your BibTeX data base

% % Non-BibTeX users please use
% \begin{thebibliography}{}
% %
% % and use \bibitem to create references. Consult the Instructions
% % for authors for reference list style.
% %
% \bibitem{RefJ}
% % Format for Journal Reference
% Author, Article title, Journal, Volume, page numbers (year)
% % Format for books
% \bibitem{RefB}
% Author, Book title, page numbers. Publisher, place (year)
% % etc
% \end{thebibliography}

\bibliographystyle{spbasic}      % basic style, author-year citations
%	\bibliographystyle{spmpsci}      % mathematics and physical sciences
%	\bibliographystyle{spphys}       % APS-like style for physics
	%\bibliography{}   % name your BibTeX data base
	
	% Non-BibTeX users please use
\bibliography{main}

\appendix
% --------------------------------------------------------------------------
% --------------------------------------------------------------------------
\section{Supplementary Material}\label{sec:appendix}

% --------------------------------------------------------------------------
% --------------------------------------------------------------------------

% \subsection{Western United States Learned Masks}\label{sec:masks}

\begin{figure}[h!]
    \centering
    \noindent\includegraphics[width=0.95\textwidth]{supplemental_figures/wus_masks_ltall.png}
    \caption[Western United States learned masks, 1-10 years]{
        Model learned masks for the western United States for lead times 1-10 years.
    }
    \label{fig:wus_all_learned_masks}
\end{figure}

\begin{figure}[h!]
    \centering
    \noindent\includegraphics[width=0.95\textwidth]{supplemental_figures/wus_tlmasks_ltall.png}
    \caption[Western United States transfer learned masks, 1-10 years]{
        Transfer learned masks for the western United States for lead times 1-10 years.
    }
    \label{fig:wus_all_tl_masks}
\end{figure}

% --------------------------------------------------------------------------
% --------------------------------------------------------------------------

% \subsection{Metrics Versus Number of Analogs}\label{sec:number_analogs}

\begin{figure}[h!]
    \centering
    \noindent\includegraphics[width=0.95\textwidth]{supplemental_figures/wus_mse_nana_full.png}
    \caption[MSE versus number of analogs]{
        MSE versus number analogs used to calculate the mean prediction, for all lead times.
        Also shown are the global and regional mask results.
    }
    \label{fig:mse_nana_full}
\end{figure}

\begin{figure}[h!]
    \centering
    \noindent\includegraphics[width=0.95\textwidth]{supplemental_figures/wus_crps_nana_full.png}
    \caption[CRPS versus number of analogs]{
        Same as Figure \ref{fig:mse_nana_full}, but for CRPS.
    }
    \label{fig:crps_nana_full}
\end{figure}

% --------------------------------------------------------------------------
% --------------------------------------------------------------------------

% \subsection{Extra Metrics: EMD and Class Accuracy}\label{sec:extra_metrics}

\begin{figure}[h!]
    \centering
    \noindent\includegraphics[width=0.8\textwidth]{supplemental_figures/wus_emd_lead.png}
    \caption[Earth mover's distance western United States results]{
        EMD versus lead time for the western United States for mean predictions (line plots) and individual analogs (distributions).
        The top two panels show EMD for the time period $1864$-$2023$, while the bottom two panels show EMD for $2009$-$2018$.
    }
    \label{fig:emd_lead}
\end{figure}

\begin{figure}[h!]
    \centering
    \noindent\includegraphics[width=0.8\textwidth]{supplemental_figures/wus_acc4_lead.png}
    \caption[4 class accuracy for western United States]{
        Same as Figure \ref{fig:emd_lead}, but for class accuracy, with four classes.
    }
    \label{fig:acc4_lead}
\end{figure}

\begin{figure}[h!]
    \centering
    \includegraphics[width=0.6\textwidth]{supplemental_figures/regions_emd.png}
    \includegraphics[width=0.6\textwidth]{supplemental_figures/regions_acc4.png}
    \caption[EMD and class accuracy for 5 regions]{
        EMD (top) and class accuracy (bottom, 4 classes).
        The circles are the mean metric covering the time period $1956$-$2023$, the squares cover $1999$-$2018$.
    }
    \label{fig:emdacc_region}
\end{figure}


% --------------------------------------------------------------------------
% --------------------------------------------------------------------------

% \subsection{Prediction Year Versus Analog Year}\label{sec:year_comp}

\begin{figure}[h!]
    \centering
    \includegraphics[width=0.48\textwidth]{supplemental_figures/year_comp_wus5.png}
    \includegraphics[width=0.48\textwidth]{supplemental_figures/year_comp_regional_wus5.png}
    \caption[Analog year versus prediction year]{
        For each prediction year, the range of years for the top ten analogs (top panel) as well as the error over that time (bottom).
        While the model learned mask (left) begins following the one-to-one line (indicating that the selected analog year is the same as the prediction year) by the $1970$s, the regional mask (right) only does so in the final few years.
        In the bottom panels, it can be seen that the model learned mask predictions are not biased toward underestimating or overestimating at any point, while the regional mask predictions tends towards underestimation starting in the $1970$s.
        This is likely due to the regional mask not being able to pick up on the forced warming signal, which may be most apparent outside the target region.
    }
    \label{fig:year_comp}
\end{figure}



\end{document}
% end of file template.tex

