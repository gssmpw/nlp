

\section{Conclusion}
\label{sec:conclusion}

Our study qualitatively investigated practitioners' perspectives on the adoption of CI practices in ML projects, drawing insights from a survey of 155 participants. 
By exploring key challenges, such as build duration management, test coverage expectations, and reasons behind the differences between ML and non-ML when adopting CI practices, the study provides a comprehensive understanding of the unique demands of ML projects in the context of CI.

The key takeaway from this study is that while foundational CI principles remain valuable, ML projects require tailored approaches to address their unique challenges, including longer build durations, lower test coverage, complex dependency management, and the non-deterministic nature of ML workflows.
To bridge this gap, we propose a set of ML-specific CI practices, such as tracking model performance metrics on each commit, prioritizing test execution, and versioning datasets and models. Additionally, our study underscores the importance of fostering interdisciplinary collaboration to strengthen the testing culture in ML projects. Finally, our discussion highlights the need for standardized CI guidelines, particularly in dependency management, to improve CI effectiveness in ML workflows.
This study identifies critical gaps between current CI practices and practitioner expectations and provides a roadmap for advancing CI in ML projects through targeted research and practical interventions.

\section*{Data Availability Statements}

The replication package for this study, including anonymized datasets, survey responses, and analysis scripts, is available at \url{https://zenodo.org/records/14902811}.
Due to privacy concerns, raw survey responses containing non-anonymized text are not publicly available.

\section*{Funding}

This work is partially supported by \textsc{INES} (National Institute of Software Engineering), \textsc{CNPq} grant 465614/2014-0, \textsc{CAPES} grant 88887.136410/2017-00, \textsc{FACEPE} grants APQ-0399-1.03/17, \textsc{PRONEX} APQ/0388-1.03/14, and \textsc{CNPq} grant 425211/2018-5.

\section*{Author Contributions}

\begin{itemize}
    \item João Helis Bernardo: Conceptualization, Methodology, Data Collection, Thematic Analysis, Writing—Original Draft.
    \item Daniel A. da Costa: Methodology, Writing—Review \& Editing, Thematic Analysis, Supervision, Research Direction.
    \item Filipe R. Cogo: Methodology, Writing—Review \& Editing, Thematic Analysis.
    \item Sérgio Q. de Medeiros: Methodology, Writing—Review \& Editing, Thematic Analysis.
    \item Uirá Kulesza: Methodology, Writing—Review \& Editing, Thematic Analysis, Supervision, Research Direction.
\end{itemize}

\section*{Ethical Approval}

The study involved a survey with software practitioners regarding their experiences with continuous integration in machine learning projects. The research did not involve personal or sensitive data and adhered to ethical guidelines for research with human participants. 
The study did not collect personally identifiable information beyond publicly available \textsc{GitHub} information. Furthermore, participants were informed about the research purpose, and their responses were anonymized.

\section*{Informed Consent}

The invitation email included a description of the study, its purpose, and the voluntary nature of participation. No personally identifiable information was collected beyond publicly available \textsc{GitHub} information.

\section*{Conflict of interest}

The authors declare that they have no conflicts of interest related to this work.