% \vspace{-10pt}
\section{Introduction} \label{sec:introduction}

\begin{figure}
    \centering
    \includegraphics[width=\linewidth]{figures/PZ_DEMO.png}
        \vspace{-2em}
    \caption{An overview of data processing using \chat and \sys}
    \label{fig:architecture-overview}
\end{figure}

Generative AI has transformed our interactions with data and computing by introducing Large Language Models (LLMs) capable of complex tasks such as multimodal extraction, reasoning, and code synthesis~\cite{yao2022react,hong2023metagpt}. 
Yet, implementing such functionality often requires coordinating multiple software stacks—vector databases~\cite{lewis2020retrieval}, relational operators~\cite{palimpzestCIDR}, and novel programming practices like prompt engineering~\cite{dspy}. 
As these AI systems grow in scope, users face major challenges around runtime cost, resource consumption, and the coordination of diverse tools.

\sys{} is one of several new declarative AI systems (e.g., \textsc{Lotus}~\cite{patel2024lotus}, \textsc{DocETL}~\cite{shankar2024docetl}, \textsc{Caesura}~\cite{urban2024caesura}) that adopt a declarative approach for building AI pipelines over unstructured datasets. 
These systems greatly simplify this task for proficient programmers by allowing them to write high-level \emph{logical} specifications - and let automated optimization produce efficient \emph{physical} implementations of these pipelines~\cite{palimpzestCIDR}.
However, this programming model remains inaccessible to less technical users. 
To bridge that gap, we propose \chat{}, a chat-based interface that integrates \sys{} with \archytas{}, a framework to design reasoning agents.
Our system lets users specify data processing pipelines in natural language, while still benefiting from the underlying automated optimization provided by declarative AI frameworks.

In this demonstration, we showcase how both developers and non-experts can collaboratively harness our system. 
All users, regardless of their coding expertise, can design and execute pipelines entirely through \chat{}. 
Furthermore, expert users can either further iterate on the code produced using the chat interface, or program their pipelines directly within \sys{}.
To illustrate the versatility of this approach, we present a real-world use case in scientific discovery, where medical researchers want to identify relevant genomic and proteomic datasets in cancer-related papers. 
Conference attendees can similarly explore additional cases or upload their own unstructured data for processing.

\chat\ is an early, but promising, prototype, and we welcome suggestions for further optimizations and features. 
% We encourage readers to consult our extended technical description in~\cite{palimpzestCIDR}, experiment with our open-source \sys{} code at \systemurl, and interact with \chat{} demo at \demourl and code at \chatcode.
We encourage readers to consult our extended technical description in~\cite{palimpzestCIDR}, experiment with our open-source \sys{} code \footnote{\systemurl}, interact with the \chat{} demo online\footnote{\demourl} or locally \footnote{\chatcode}, or view the video\footnote{\demovideo}.

The remainder of this paper is structured as follows:  
section~\ref{sec:architecture} describes the core architecture of \sys\ and the reasoning framework driving \chat.
Section~\ref{sec:demoscenario} demonstrates a concrete usage scenario through the chat-based interface, and Section~\ref{sec:conclusion} briefly concludes the paper.
