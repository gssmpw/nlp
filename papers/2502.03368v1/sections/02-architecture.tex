\section{System Architecture}
\label{sec:architecture}

The \chat{} system is composed of two interconnected systems: the \sys{} framework, a declarative system to build and automatically optimize data processing pipelines; and the \archytas{} framework, a system to code reasoning agents with access to external tools.
We briefly describe both individually, and then detail how they integrate to form \chat{}.

\subsection{\sys{}}
\sys{} is a declarative system designed to build and optimize pipelines for unstructured data processing, harnessing both LLM-based and conventional operations. At its core, \sys{} programs can be viewed as collections of relational operators. First, users must specify one or more input datasets and their corresponding “Schema.” A schema consists of the attribute names, types, and descriptions used to process the dataset. Given that a dataset can contain unstructured data, \sys{} automatically extracts attribute values from each data record using either LLM-based or conventional methods. Subsequently, users specify a series of transformations over these data records to form the final output. 
% Figure~\ref{fig:operators} highlights the logical operators that \sys{} supports, and Figure~\ref{fig:code} shows a short example program.

Although \sys{} implements most relational algebra operators, our demo emphasizes two special operators: (1) \emph{Convert}, which transforms an object of schema~A into an object of schema~B by computing the fields in~B that do not explicitly exist in~A, and (2) \emph{Filter}, which applies a natural language predicate or UDF on a dataset’s records and returns the subset that satisfies the predicate. All other operations (e.g., \emph{Aggregation}) follow conventional database semantics.
% describe program 

% \subsection{Physical plans}
A \sys{} plan is a sequence of these operators over a dataset. By design, users write \emph{logical} plans only; the choice of the physical implementation is deferred until runtime. For each logical operator, multiple equivalent physical implementations may be available. For instance, a filter operation might be performed via different LLM models, each representing a distinct physical method. More details about logical and physical implementations are available in~\cite{palimpzestCIDR}.

% \subsection{Plan optimization}
After the user specifies their pipelines, 
\sys\ creates a search space of all possible physical plans that implement such plan, which are effectively logically equivalent but may yield outputs of different quality, with a different cost, or with a different runtime.

In a subsequent optimization phase, \sys\ automatically ranks physical plans and selects the most optimal one that meets user-defined preferences.
Users can specify whether they are interested in quality, runtime, or cost of executing their pipelines. 
They may instruct the system to narrow its optimization on one of these dimensions (e.g., to minimize the cost no matter the quality), or specify a meaningful combination of them (e.g., maximize the output quality while being under a certain latency).

% \begin{figure*}
%     \centering
%     \includegraphics[width=1.6\columnwidth]{figures/chat.png}
%     \caption{Chat Interface available at \demourl}
%     \label{fig:schematic}\vspace{-10pt}
% \end{figure*}
























