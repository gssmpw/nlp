\begin{figure}[t]
\begin{minted}[escapeinside=||,texcomments]{python}
    @tool()
    async def create_schema(
        self, schema_name: str, schema_description: str, 
        field_names: list, field_descriptions: list, 
        agent: AgentRef) -> str:
        """
        This tool should be used to generate a new extraction schema. 
        The inputs are a schema name and a set of fields. For example, 
        let's say the user is interested in extracting author 
        information from a paper. In this case the schema name might 
        be 'Author' and the fields may be 'name', 'email', 'affiliation'.
        You should provide a short description for each field. 
        Field names cannot have spaces or special characters.
        """
        
        class_name = "{{ schema_name }}"
        schema = {"__doc__": "{{ schema_description }}"}
        for idx, field in enumerate(|\bf{\{\{ field\_names \}\}}|)):
            desc = |\bf{\{\{ field\_descriptions \}\}}|[idx]
            schema[field] = pz.Field(desc=desc)       
        new_schema = type(class_name, (pz.Schema,), schema)
        return new_schema
\end{minted}
\caption{An example Archytas tool used to generate an extraction schema with \sys. Docstrings are important to provide context for the reasoning agent. Inputs variables are injected with the template syntax \texttt{\{\{variable\}\}}.}
\label{fig:toolcode}
\end{figure}