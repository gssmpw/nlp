\usepackage{xcolor}
\usepackage{colortbl}
\usepackage{xspace}
\usepackage{subcaption}
\usepackage{amsmath}
\usepackage{listings}
\usepackage{algorithm}
\usepackage{algpseudocodex}
\usepackage{multirow}
\usepackage{multicol}
\usepackage{tikz}
\newcommand*\circled[1]{\tikz[baseline=(char.base)]{
            \node[shape=circle,draw,inner sep=2pt] (char) {#1};}}
\usepackage{bbm}
\usepackage{pifont}
\usepackage{hhline}
\usepackage{arydshln}
\usepackage{graphicx}
\usepackage[clock, misc]{ifsym}
\usepackage[inline]{enumitem}
\usepackage[percent]{overpic}
\usepackage{tcolorbox}
\tcbuselibrary{breakable}
\usepackage{tikz}
\usetikzlibrary{arrows.meta, positioning, shapes.geometric}
\settopmatter{authorsperrow=5}



\usepackage[frozencache,cachedir=.]{minted}
% Code style
% Customize minted settings
\setminted{
    frame=lines,
    framesep=1mm,
    baselinestretch=1,
    fontsize=\footnotesize,
    linenos,
    numbersep=5pt,
    autogobble,
    breaklines,
    tabsize=2,
    breakautoindent=true,
    mathescape,
}


\newcommand{\user}{Alex\xspace}
\newcommand{\challengeAname}{Deriving the Schema\xspace}
\newcommand{\challengeBname}{Distilling the Data\xspace}
\newcommand{\challengeCname}{Obtaining the Causal Model\xspace}


\let\originalinput\input 

\renewcommand{\input}[1]{%
  \originalinput{#1}
%  \clearpage
 % \pagebreak
}

\newcommand\encircle[1]{%
  \raisebox{.5pt}{\textcircled{\raisebox{-.3pt} {\footnotesize #1}}} }


\newcommand{\LineComment}[1]{\State \textcolor{blue}{// #1}}

\newcommand{\Elif}[1]{\State\hspace{-\algorithmicindent} \textbf{else if} #1 \textbf{then}}
%\numberwithin{algorithm}{section}

\newtheorem{episode}{\textbf{Episode}}
\newtheorem{definition}{\textbf{Definition}}
\newtheorem{innerproblem}{\textbf{Problem}}
\newenvironment{problem}[2]{\renewcommand\theinnerproblem{#1}\innerproblem[#2]}{\endinnerproblem}

\newcommand{\doop}{{\tt do}}

\definecolor{darkgreen}{RGB}{0, 100, 0}
\definecolor{experimentblue}{RGB}{127, 159, 186}
\definecolor{experimentgreen}{RGB}{127, 186, 130}

\newcommand{\todo}[1]{\textcolor{red}{\bf (TODO)~[#1]}{\typeout{#1}}}
\newcommand{\chunwei}[1]{\textcolor{teal}{\bf (Chunwei)~[#1]}{\typeout{#1}}}
\newcommand{\mike}[1]{\textcolor{purple}{\bf (Mike)~[#1]}{\typeout{#1}}}
\newcommand{\gerado}[1]{\textcolor{blue}{\bf (Gerardo)~[#1]}{\typeout{#1}}}
\newcommand{\brandon}[1]{\textcolor{magenta}{\bf (Brandon)~[#1]}{\typeout{#1}}}

\newcommand{\sparagraph}[1]{\vspace{1mm}\noindent {\bf #1}}


\newcommand{\cdsuccess}{{\textcolor{darkgreen}{\ding{51}}}}
\definecolor{darkorange}{RGB}{240,131,15}
\newcommand{\cdtimeout}{{\textcolor{darkorange}{\ding{115}}}}
\newcommand{\cderror}{{\textcolor{red}{\ding{55}}}}
\newcommand{\cdempty}{{\textcolor{blue}{\ding{108}}}}


\newcommand{\dataset}[1]{\textsf{\textsc{#1}}}
\newcommand{\baseline}[1]{\textsf{#1}}
\newcommand{\prepvar}[1]{\texttt{#1}}
\newcommand{\funcname}[1]{\textsc{#1}}

\newcommand{\sys}{\textsc{Palimpzest}\xspace}
\newcommand{\system}{\textsc{Palimpzest}\xspace}
\newcommand{\chat}{\textsc{PalimpChat}\xspace}
\newcommand{\archytas}{\textsc{Archytas}\xspace}

\newcommand{\systemurl}{\url{www.palimpzest.org}\xspace}

\newcommand{\demourl}{\url{www.palimpzest.org/chat}\xspace}
% \newcommand{\demourl}{\url{http://3.213.4.62:8888}\xspace}

\newcommand{\demovideo}{\url{https://people.csail.mit.edu/chunwei/demo/palimpchat.mp4}\xspace}

\newcommand{\chatcode}{\url{https://github.com/mitdbg/PalimpChat}\xspace}










