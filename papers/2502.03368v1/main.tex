%%
%% This is file `sample-manuscript.tex',
%% generated with the docstrip utility.
%%
%% The original source files were:
%%
%% samples.dtx  (with options: `manuscript')
%% 
%% IMPORTANT NOTICE:
%% 
%% For the copyright see the source file.
%% 
%% Any modified versions of this file must be renamed
%% with new filenames distinct from sample-manuscript.tex.
%% 
%% For distribution of the original source see the terms
%% for copying and modification in the file samples.dtx.
%% 
%% This generated file may be distributed as long as the
%% original source files, as listed above, are part of the
%% same distribution. (The sources need not necessarily be
%% in the same archive or directory.)
%%
%% Commands for TeXCount
%TC:macro \cite [option:text,text]
%TC:macro \citep [option:text,text]
%TC:macro \citet [option:text,text]
%TC:envir table 0 1
%TC:envir table* 0 1
%TC:envir tabular [ignore] word
%TC:envir displaymath 0 word
%TC:envir math 0 word
%TC:envir comment 0 0
%%
%%
%% The first command in your LaTeX source must be the \documentclass command.
%%%% Small single column format, used for CIE, CSUR, DTRAP, JACM, JDIQ, JEA, JERIC, JETC, PACMCGIT, TAAS, TACCESS, TACO, TALG, TALLIP (formerly TALIP), TCPS, TDSCI, TEAC, TECS, TELO, THRI, TIIS, TIOT, TISSEC, TIST, TKDD, TMIS, TOCE, TOCHI, TOCL, TOCS, TOCT, TODAES, TODS, TOIS, TOIT, TOMACS, TOMM (formerly TOMCCAP), TOMPECS, TOMS, TOPC, TOPLAS, TOPS, TOS, TOSEM, TOSN, TQC, TRETS, TSAS, TSC, TSLP, TWEB.
% \documentclass[acmsmall]{acmart}

%%%% Large single column format, used for IMWUT, JOCCH, PACMPL, POMACS, TAP, PACMHCI
% \documentclass[acmlarge,screen]{acmart}

%%%% Large double column format, used for TOG
% \documentclass[acmtog, authorversion]{acmart}

%%%% Generic manuscript mode, required for submission
%%%% and peer review
\documentclass[sigconf, 9pt]{acmart}
%% Fonts used in the template cannot be substituted; margin 
%% adjustments are not allowed.
%%
%% \BibTeX command to typeset BibTeX logo in the docs
\AtBeginDocument{% 
    \providecommand\BibTeX{{%
        Bib\TeX}}}


%% Rights management information.  This information is sent to you
%% when you complete the rights form.  These commands have SAMPLE
%% values in them; it is your responsibility as an author to replace
%% the commands and values with those provided to you when you
%% complete the rights form.
\copyrightyear{2024}
\acmYear{2024}
\setcopyright{rightsretained} 
\acmConference[SIGMOD-Companion '25]{Companion of the 2025 International Conference on Management of Data}{June 22--27, 2025}{Berlin,Germany}
\acmBooktitle{Companion of the 2025 International Conference on Management of Data (SIGMOD-Companion '25), June 22--27, 2025, Berlin, Germany}
\acmDOI{10.1145/XXXXX}
\acmISBN{979-8-4007-0422-2/24/06}

% The following includes the CC license icon appropriate for your paper.
% Download the image from www.scomminc.com/pp/acmsig/4ACM-CC-by-88x31.eps
% and place within your figs or figures folder

\makeatletter

\gdef\@copyrightpermission{
  * Denotes equal first author contribution.
  
  \begin{minipage}{0.3\columnwidth}
   \href{https://creativecommons.org/licenses/by/4.0/}{\includegraphics[width=0.90\textwidth]{figures/cc-by-nc-sa4acm.png}}
  \end{minipage}\hfill
  \begin{minipage}{0.7\columnwidth}
   \href{https://creativecommons.org/licenses/by/4.0/}{This work is licensed under a Creative Commons Attribution International 4.0 License.}
  \end{minipage}
  \vspace{5pt}
}
\makeatother




\newcommand{\ones}{\mathbf 1}
\newcommand{\reals}{{\mbox{\bf R}}}
\newcommand{\integers}{{\mbox{\bf Z}}}
\newcommand{\symm}{{\mbox{\bf S}}}  % symmetric matrices

\newcommand{\nullspace}{{\mathcal N}}
\newcommand{\range}{{\mathcal R}}
\newcommand{\Rank}{\mathop{\bf Rank}}
%\newcommand{\Tr}{\mathop{\bf Tr}}
\newcommand{\diag}{\mathop{\bf diag}}
\newcommand{\card}{\mathop{\bf card}}
\newcommand{\rank}{\mathop{\bf rank}}
\newcommand{\conv}{\mathop{\bf conv}}
\newcommand{\prox}{\mathbf{prox}}

\newcommand{\Expect}{\mathop{\bf E{}}}
\newcommand{\var}{\mathop{\bf var{}}}
\newcommand{\Prob}{\mathop{\bf Prob}}
\newcommand{\Co}{{\mathop {\bf Co}}} % convex hull
\newcommand{\dist}{\mathop{\bf dist{}}}
%\newcommand{\argmin}{\mathop{\rm argmin}}
%\newcommand{\argmax}{\mathop{\rm argmax}}
\newcommand{\epi}{\mathop{\bf epi}} % epigraph
\newcommand{\Vol}{\mathop{\bf vol}}
\newcommand{\dom}{\mathop{\bf dom}} % domain
\newcommand{\intr}{\mathop{\bf int}}
%\newcommand{\sign}{\mathop{\bf sign}}

\newcommand{\cf}{{\it cf.}}
\newcommand{\eg}{{\it e.g.}}
\newcommand{\ie}{{\it i.e.}}
\newcommand{\etc}{{\it etc.}}

\newcommand{\todo}{{\bf TODO}}

\newcommand{\bone}{\boldsymbol{1}}
\newcommand{\balpha}{\boldsymbol{\alpha}}
\newcommand{\bbeta}{\boldsymbol{\beta}}
\newcommand{\bdelta}{\boldsymbol{\delta}}
\newcommand{\bepsilon}{\boldsymbol{\epsilon}}
\newcommand{\blambda}{\boldsymbol{\lambda}}
\newcommand{\bomega}{\boldsymbol{\omega}}
\newcommand{\bpi}{\boldsymbol{\pi}}
\newcommand{\bnu}{\boldsymbol{\nu}}
\newcommand{\bphi}{\boldsymbol{\phi}}
\newcommand{\bvphi}{\boldsymbol{\varphi}}
\newcommand{\bpsi}{\boldsymbol{\psi}}
\newcommand{\bsigma}{\boldsymbol{\sigma}}
\newcommand{\btheta}{\boldsymbol{\theta}}
\newcommand{\bzeta}{\boldsymbol{\zeta}}
\newcommand{\bxi}{\boldsymbol{\xi}}
\newcommand{\ba}{\boldsymbol{a}}
\newcommand{\bb}{\boldsymbol{b}}
\newcommand{\bc}{\boldsymbol{c}}
\newcommand{\bd}{\boldsymbol{d}}
\newcommand{\be}{\boldsymbol{e}}
\newcommand{\boldf}{\boldsymbol{f}}
\newcommand{\bg}{\boldsymbol{g}}
\newcommand{\bh}{\boldsymbol{h}}
\newcommand{\bi}{\boldsymbol{i}}
\newcommand{\bj}{\boldsymbol{j}}
\newcommand{\bk}{\boldsymbol{k}}
\newcommand{\bell}{\boldsymbol{\ell}}
\newcommand{\bp}{\boldsymbol{p}}
\newcommand{\br}{\boldsymbol{r}}
\newcommand{\bs}{\boldsymbol{s}}
\newcommand{\bt}{\boldsymbol{t}}
\newcommand{\bu}{\boldsymbol{u}}
\newcommand{\bv}{\boldsymbol{v}}
\newcommand{\bw}{\boldsymbol{w}}
\newcommand{\bx}{{\boldsymbol{x}}}
\newcommand{\by}{\boldsymbol{y}}
\newcommand{\bz}{\boldsymbol{z}}
\newcommand{\bA}{\boldsymbol{A}}
\newcommand{\bB}{\boldsymbol{B}}
\newcommand{\bC}{\boldsymbol{C}}
\newcommand{\bD}{\boldsymbol{D}}
\newcommand{\bE}{\boldsymbol{E}}
\newcommand{\bF}{\boldsymbol{F}}
\newcommand{\bG}{\boldsymbol{G}}
\newcommand{\bH}{\boldsymbol{H}}
\newcommand{\bI}{\boldsymbol{I}}
\newcommand{\bJ}{\boldsymbol{J}}
\newcommand{\bL}{\boldsymbol{L}}
\newcommand{\bM}{\boldsymbol{M}}
\newcommand{\bP}{\boldsymbol{P}}
\newcommand{\bQ}{\boldsymbol{Q}}
\newcommand{\bR}{\boldsymbol{R}}
\newcommand{\bS}{\boldsymbol{S}}
\newcommand{\bT}{\boldsymbol{T}}
\newcommand{\bU}{\boldsymbol{U}}
\newcommand{\bV}{\boldsymbol{V}}
\newcommand{\bW}{\boldsymbol{W}}
\newcommand{\bX}{\boldsymbol{X}}
\newcommand{\bY}{\boldsymbol{Y}}
\newcommand{\bZ}{\boldsymbol{Z}}

% new theorems
% \newtheorem{theorem}{Theorem}
%\newtheorem*{proof}{Proof}

% usepackages
\usepackage{amsmath}
\usepackage{amsfonts}
\usepackage{textcomp} % for \textlangle and \textrangle macros
\newcommand{\qdist}[1]{\ifmmode\langle#1\rangle\else\textlangle#1\textrangle\fi}
\usepackage{xcolor}
\usepackage{algorithm} % for algorithms
\usepackage{algpseudocode} % for pseudocode
\usepackage{comment} % for large comments
\usepackage{bbm}
\usepackage{dsfont}
\usepackage{subfigure}
\usepackage{bm}
\usepackage{booktabs} % For better table lines
\usepackage{array} % For better column formatting
%\usepackage{appendix}
%\usepackage[english]{babel}
%\usepackage{amsthm}
\usepackage{graphicx} % for graphs






%%
%% Submission ID.
%% Use this when submitting an article to a sponsored event. You'll
%% receive a unique submission ID from the organizers
%% of the event, and this ID should be used as the parameter to this command.
%%\acmSubmissionID{123-A56-BU3}

%%
%% For managing citations, it is recommended to use bibliography
%% files in BibTeX format.
%%
%% You can then either use BibTeX with the ACM-Reference-Format style,
%% or BibLaTeX with the acmnumeric or acmauthoryear sytles, that include
%% support for advanced citation of software artefact from the
%% biblatex-software package, also separately available on CTAN.
%%
%% Look at the sample-*-biblatex.tex files for templates showcasing
%% the biblatex styles.
%%

%%
%% The majority of ACM publications use numbered citations and
%% references.  The command \citestyle{authoryear} switches to the
%% "author year" style.
%%
%% If you are preparing content for an event
%% sponsored by ACM SIGGRAPH, you must use the "author year" style of
%% citations and references.
%% Uncommenting
%% the next command will enable that style.
%%\citestyle{acmauthoryear}

%%
%% end of the preamble, start of the body of the document source.
\begin{document}

%%
%% The "title" command has an optional parameter,
%% allowing the author to define a "short title" to be used in page headers.
\title{\chat{}: Declarative and Interactive AI analytics}

%%
%% The "author" command and its associated commands are used to define
%% the authors and their affiliations.
%% Of note is the shared affiliation of the first two authors, and the
%% "authornote" and "authornotemark" commands
%% used to denote shared contribution to the research.

\author{Zihan Wang}
\orcid{0000-0003-0493-2668}
\authornote{Both authors contributed equally to the paper.}
\affiliation{%
  \institution{University of Amsterdam}
  \city{Amsterdam}
  \country{The Netherlands}
}
\email{zhw.cypher@gmail.com}

\author{Ziqi Zhao}
\orcid{0009-0008-3011-5745}
\authornotemark[1]
\affiliation{
    \institution{Shandong University}
    \city{Qingdao}
    \country{China}
}
\email{ziqizhao.work@gmail.com}

\author{Yougang Lyu}
\orcid{0009-0000-1082-9267}
\affiliation{
    \institution{University of Amsterdam}
    \city{Amsterdam}
    \country{The Netherlands}
}
\email{youganglyu@gmail.com}

\author{Zhumin Chen}
\orcid{0000-0003-4592-4074}
\affiliation{
    \institution{Shandong University}
    \city{Jinan}
    \country{China}
}
\email{chenzhumin@sdu.edu.cn}

\author{Maarten de Rijke}
\orcid{0000-0002-1086-0202}
\affiliation{%
  \institution{University of Amsterdam}
  \city{Amsterdam}
  \country{The Netherlands}
}
\email{m.derijke@uva.nl}

\author{Zhaochun Ren}
\orcid{0000-0002-9076-6565}
%\authornote{Corresponding author.}
\affiliation{%
  \institution{Leiden University}
  \city{Leiden}
  \country{The Netherlands}
}
\email{z.ren@liacs.leidenuniv.nl}



%%
%% The abstract is a short summary of the work to be presented in the
%% article.
\begin{abstract}

Thanks to the advances in generative architectures and large language models, data scientists can now code pipelines of AI operations to process large collections of unstructured data.
Recent progress has seen the rise of declarative AI frameworks (e.g., \textsc{palimp}-\textsc{zest}, \textsc{Lotus}, and \textsc{DocETL}) to build optimized and increasingly complex pipelines, but these systems often remain accessible only to expert programmers. 
In this demonstration, we present \chat, a chat-based interface to \sys{} that bridges this gap by letting users create and run sophisticated AI pipelines through natural language alone. By integrating \archytas{}, a ReAct-based reasoning agent, and \sys{}’s suite of relational and LLM-based operators, \chat{} provides a practical illustration of how a chat interface can make declarative AI frameworks truly accessible to non-experts.

Our demo system is publicly available online. At SIGMOD’25, participants can explore three real-world scenarios—scientific discovery, legal discovery, and real estate search—or apply \chat{} to their own datasets. In this paper, we focus on how \chat{}, supported by the \sys{} optimizer, simplifies complex AI workflows such as extracting and analyzing biomedical data. A companion video is also available at \demovideo.
% \cite{chatdemovideo}.
    % Thanks to the advances in generative architectures and large language models, data scientists can now code pipelines of machine-learning operations to process large collections of unstructured data.
    % We present \sys: a declarative framework to automatically build pipelines, and \chat, a chat-based interface that allows users to build \sys pipelines purely in natural language.
    % we showcase not only the potential of optimized unstructured data processing with \sys, but also the ease of getting started with our system without the steep learning curve of mastering a new programming framework.
    % %
    % %
    % The core architecture of \sys is based on an extended set of relational operations which are implemented using conventional as well as LLM-based physical operators.
    % % Users can specify whether to optimize the execution of their pipelines according to runtime, quality of the outputs, or processing cost.
    % The natural language interface to build these pipelines is based on the \archytas{} system, which implements the ``ReAct''  protocol to design reasoning agents.

    % % In this system, an LLM agent is provided with coding capabilities as well as templates of \sys primitives and it is instructed to build data processing pipelines.
    % Our demo system is publicly available online.
    % SIGMOD'25 participants can leverage \chat on three use cases: scientific discovery, legal discovery, and real estate search - as well as upload their own datasets for analysis.
    % In this paper, we will describe the demonstration of the first use case.
    % A companion video for this submission is available online.

\end{abstract}

%%
%% The code below is generated by the tool at http://dl.acm.org/ccs.cfm.
%% Please copy and paste the code instead of the example below.
%%
\begin{CCSXML}
<ccs2012>
   <concept>
       <concept_id>10011007.10011074.10011111.10011697</concept_id>
       <concept_desc>Software and its engineering~System administration</concept_desc>
       <concept_significance>500</concept_significance>
       </concept>
   <concept>
       <concept_id>10010147.10010178.10010187.10010192</concept_id>
       <concept_desc>Computing methodologies~Causal reasoning and diagnostics</concept_desc>
       <concept_significance>500</concept_significance>
       </concept>
   <concept>
       <concept_id>10010147.10010178.10010179.10010182</concept_id>
       <concept_desc>Computing methodologies~Natural language generation</concept_desc>
       <concept_significance>100</concept_significance>
       </concept>
 </ccs2012>
\end{CCSXML}

\ccsdesc[500]{Information systems~ Data analytics}
\ccsdesc[500]{Computing methodologies~Natural language processing}


%%
%% Keywords. The author(s) should pick words that accurately describe
%% the work being presented. Separate the keywords with commas.
\keywords{Compound AI, LLMs,  AI programming, Chat-box}

%%
%% This command processes the author and affiliation and title
%% information and builds the first part of the formatted document.
\maketitle

% \vspace{-10pt}
\section{Introduction} \label{sec:introduction}

\begin{figure}
    \centering
    \includegraphics[width=\linewidth]{figures/PZ_DEMO.png}
        \vspace{-2em}
    \caption{An overview of data processing using \chat and \sys}
    \label{fig:architecture-overview}
\end{figure}

Generative AI has transformed our interactions with data and computing by introducing Large Language Models (LLMs) capable of complex tasks such as multimodal extraction, reasoning, and code synthesis~\cite{yao2022react,hong2023metagpt}. 
Yet, implementing such functionality often requires coordinating multiple software stacks—vector databases~\cite{lewis2020retrieval}, relational operators~\cite{palimpzestCIDR}, and novel programming practices like prompt engineering~\cite{dspy}. 
As these AI systems grow in scope, users face major challenges around runtime cost, resource consumption, and the coordination of diverse tools.

\sys{} is one of several new declarative AI systems (e.g., \textsc{Lotus}~\cite{patel2024lotus}, \textsc{DocETL}~\cite{shankar2024docetl}, \textsc{Caesura}~\cite{urban2024caesura}) that adopt a declarative approach for building AI pipelines over unstructured datasets. 
These systems greatly simplify this task for proficient programmers by allowing them to write high-level \emph{logical} specifications - and let automated optimization produce efficient \emph{physical} implementations of these pipelines~\cite{palimpzestCIDR}.
However, this programming model remains inaccessible to less technical users. 
To bridge that gap, we propose \chat{}, a chat-based interface that integrates \sys{} with \archytas{}, a framework to design reasoning agents.
Our system lets users specify data processing pipelines in natural language, while still benefiting from the underlying automated optimization provided by declarative AI frameworks.

In this demonstration, we showcase how both developers and non-experts can collaboratively harness our system. 
All users, regardless of their coding expertise, can design and execute pipelines entirely through \chat{}. 
Furthermore, expert users can either further iterate on the code produced using the chat interface, or program their pipelines directly within \sys{}.
To illustrate the versatility of this approach, we present a real-world use case in scientific discovery, where medical researchers want to identify relevant genomic and proteomic datasets in cancer-related papers. 
Conference attendees can similarly explore additional cases or upload their own unstructured data for processing.

\chat\ is an early, but promising, prototype, and we welcome suggestions for further optimizations and features. 
% We encourage readers to consult our extended technical description in~\cite{palimpzestCIDR}, experiment with our open-source \sys{} code at \systemurl, and interact with \chat{} demo at \demourl and code at \chatcode.
We encourage readers to consult our extended technical description in~\cite{palimpzestCIDR}, experiment with our open-source \sys{} code \footnote{\systemurl}, interact with the \chat{} demo online\footnote{\demourl} or locally \footnote{\chatcode}, or view the video\footnote{\demovideo}.

The remainder of this paper is structured as follows:  
section~\ref{sec:architecture} describes the core architecture of \sys\ and the reasoning framework driving \chat.
Section~\ref{sec:demoscenario} demonstrates a concrete usage scenario through the chat-based interface, and Section~\ref{sec:conclusion} briefly concludes the paper.

\section{System Architecture}
\label{sec:architecture}

The \chat{} system is composed of two interconnected systems: the \sys{} framework, a declarative system to build and automatically optimize data processing pipelines; and the \archytas{} framework, a system to code reasoning agents with access to external tools.
We briefly describe both individually, and then detail how they integrate to form \chat{}.

\subsection{\sys{}}
\sys{} is a declarative system designed to build and optimize pipelines for unstructured data processing, harnessing both LLM-based and conventional operations. At its core, \sys{} programs can be viewed as collections of relational operators. First, users must specify one or more input datasets and their corresponding “Schema.” A schema consists of the attribute names, types, and descriptions used to process the dataset. Given that a dataset can contain unstructured data, \sys{} automatically extracts attribute values from each data record using either LLM-based or conventional methods. Subsequently, users specify a series of transformations over these data records to form the final output. 
% Figure~\ref{fig:operators} highlights the logical operators that \sys{} supports, and Figure~\ref{fig:code} shows a short example program.

Although \sys{} implements most relational algebra operators, our demo emphasizes two special operators: (1) \emph{Convert}, which transforms an object of schema~A into an object of schema~B by computing the fields in~B that do not explicitly exist in~A, and (2) \emph{Filter}, which applies a natural language predicate or UDF on a dataset’s records and returns the subset that satisfies the predicate. All other operations (e.g., \emph{Aggregation}) follow conventional database semantics.
% describe program 

% \subsection{Physical plans}
A \sys{} plan is a sequence of these operators over a dataset. By design, users write \emph{logical} plans only; the choice of the physical implementation is deferred until runtime. For each logical operator, multiple equivalent physical implementations may be available. For instance, a filter operation might be performed via different LLM models, each representing a distinct physical method. More details about logical and physical implementations are available in~\cite{palimpzestCIDR}.

% \subsection{Plan optimization}
After the user specifies their pipelines, 
\sys\ creates a search space of all possible physical plans that implement such plan, which are effectively logically equivalent but may yield outputs of different quality, with a different cost, or with a different runtime.

In a subsequent optimization phase, \sys\ automatically ranks physical plans and selects the most optimal one that meets user-defined preferences.
Users can specify whether they are interested in quality, runtime, or cost of executing their pipelines. 
They may instruct the system to narrow its optimization on one of these dimensions (e.g., to minimize the cost no matter the quality), or specify a meaningful combination of them (e.g., maximize the output quality while being under a certain latency).

% \begin{figure*}
%     \centering
%     \includegraphics[width=1.6\columnwidth]{figures/chat.png}
%     \caption{Chat Interface available at \demourl}
%     \label{fig:schematic}\vspace{-10pt}
% \end{figure*}

























\subsection{\archytas{}}
\label{sec:chat}

\begin{figure}[t]
\begin{minted}[escapeinside=||,texcomments]{python}
    @tool()
    async def create_schema(
        self, schema_name: str, schema_description: str, 
        field_names: list, field_descriptions: list, 
        agent: AgentRef) -> str:
        """
        This tool should be used to generate a new extraction schema. 
        The inputs are a schema name and a set of fields. For example, 
        let's say the user is interested in extracting author 
        information from a paper. In this case the schema name might 
        be 'Author' and the fields may be 'name', 'email', 'affiliation'.
        You should provide a short description for each field. 
        Field names cannot have spaces or special characters.
        """
        
        class_name = "{{ schema_name }}"
        schema = {"__doc__": "{{ schema_description }}"}
        for idx, field in enumerate(|\bf{\{\{ field\_names \}\}}|)):
            desc = |\bf{\{\{ field\_descriptions \}\}}|[idx]
            schema[field] = pz.Field(desc=desc)       
        new_schema = type(class_name, (pz.Schema,), schema)
        return new_schema
\end{minted}
\caption{An example Archytas tool used to generate an extraction schema with \sys. Docstrings are important to provide context for the reasoning agent. Inputs variables are injected with the template syntax \texttt{\{\{variable\}\}}.}
\label{fig:toolcode}
\end{figure}

% \subsection{Archytas}
Archytas is a toolbox for enabling LLM agents to interact with various tools in order to solve tasks more effectively, following the ReAct (Reason \& Action) paradigm ~\cite{yao2022react}. 
It is similar in functionality to existing solutions like LangChain~\cite{langchain}, but focuses on providing a streamlined interface for tools. 
By implementing ReAct, an agent can decompose a user request into smaller steps, decide which tools to invoke for each step, provide corresponding input to those tools, and iterate until the task is complete. 
This flexible interface is well-suited for both purely textual tasks and more complex operations (e.g., looking up a dataset, calling AI-based filtering routines, etc.).

\subsection{\chat{}}
The \chat{} interface integrates with \system{} with \archytas{} by exposing a series of tools that the LLM-based agent can leverage. 
Essentially, these tools 
% (shown in Table \ref{tab:agent_tools_short}) 
correspond to templated code snippets that can 1. perform fundamental \sys\ operations (e.g., registering a dataset, generating schemas, filtering records) and 2. orchestrate entire pipelines of transformations.
% without requiring the user to manually write code.
Figure~\ref{fig:toolcode} provides an example of a tool implementation.

% \begin{table}[t]
\scriptsize
\centering
\label{tab:agent_tools_short}
\begin{tabular}{p{2.0cm}p{5.1cm}}
\toprule
\textbf{Tool} & \textbf{Description} \\
\midrule

\texttt{register\_dataset} 
& \textbf{Inputs:} (\texttt{path} (str), \texttt{name} (str)) 
\newline \textbf{Function:} Register a dataset for subsequent use. \\[5pt]

\texttt{unregister\_dataset}
& \textbf{Inputs:} (\texttt{dataset\_name} (str))
\newline \textbf{Function:} Remove a dataset from the system. \\[5pt]

\texttt{list\_datasets}
& \textbf{Inputs:} None  
\newline \textbf{Function:} Show all tracked datasets. \\[5pt]

\texttt{retrieve\_dataset}
& \textbf{Inputs:} (\texttt{dataset\_path} (str)) 
\newline \textbf{Function:} List records in a dataset. \\[5pt]

\texttt{generate\_extraction\_}\\ \texttt{schema}
& \textbf{Inputs:} (\texttt{schema\_name}, \texttt{schema\_description}, \texttt{field\_names}, \texttt{field\_descriptions}, \texttt{field\_required})
\newline \textbf{Function:} Define a new schema for extracting fields. \\[5pt]

\texttt{filter\_data}
& \textbf{Inputs:} (\texttt{input\_dataset} (str), \texttt{filter\_expression} (str))
\newline \textbf{Function:} Filter a dataset using a natural-language condition. \\[5pt]

\texttt{convert\_dataset}
& \textbf{Inputs:} (\texttt{input\_dataset}, \texttt{schema}, \texttt{cardinality})
\newline \textbf{Function:} Transform a dataset into a new schema. \\[5pt]

\texttt{override\_dataset}
& \textbf{Inputs:} (\texttt{input\_dataset} (str))
\newline \textbf{Function:} Reset the current “active” dataset. \\[5pt]

\texttt{set\_input\_dataset}
& \textbf{Inputs:} (\texttt{dataset} (str), \texttt{input\_schema} (str))
\newline \textbf{Function:} Establish a default dataset and schema. \\[5pt]

\texttt{pick\_schema}
& \textbf{Inputs:} (\texttt{schema\_name} (str))
\newline \textbf{Function:} Retrieve (or create) a named schema. \\[5pt]

\texttt{list\_schemas}
& \textbf{Inputs:} None
\newline \textbf{Function:} Show all known schemas. \\[5pt]

\texttt{execute\_workload}
& \textbf{Inputs:} (\texttt{output\_dataset}, \texttt{policy\_method}, \texttt{allow\_code\_synth}, \texttt{allow\_token\_reduction})
\newline \textbf{Function:} Execute a pipeline under a chosen policy. \\
\bottomrule
\end{tabular}
\caption{Core \chat\ Tools.}
\end{table}

\begin{figure}[t]
    \centering
    \includegraphics[width=\linewidth]{figures/demo-dataset.png}
    \caption{Setting an input dataset through \chat{}}
    \label{fig:dataset_ops}
\end{figure}

The Archytas agent will read tool code as natural language, and consider its doc-string and input/output parameters in order to decide whether to use it to satisfy the user requests.
All tools adhere to a similar pattern in terms of input and output. The general docstring of a tool summarizes what each tool accomplishes and when it is appropriate to use.
The \texttt{Args} section of the docstring can be used to describe the input and output arguments expected for each tool.
Providing a few examples of usage within the docstring proved to be the most efficient solution to improve the quality of the reasoning agent.

The code of each tool is a Python function with the \verb|@tool()| annotation, and a Jinja-based templated syntax can be used to inject run-time variables.
Within the tool code, if a variable is expressed in round brackets as \texttt{\{\{variable\}\}}, the Archytas agent will fill the variable with a variable available at run-time in the Python execution environment.
When parsing a single user request, the reasoning Archytas agent can decide to chain several tool invocations if it deems it necessary to fulfill the desired request (See Section~\ref{sec:demoscenario} for an example).

% \subsection{Beaker}
% Beaker is a custom Jupyter kernel framework designed to integrate seamlessly with web applications. Rather than restricting all interactions to a classic notebook UI, Beaker permits front-end components of an application to leverage long-lived Jupyter sessions in the background.

We host our \chat{} system within a hybrid notebook-chat environment, called Beaker. 
Beaker is an implementation of computational notebooks that integrates AI capabilities into the interactive coding environment. 
Building on Project Jupyter's foundation, Beaker incorporates an AI agent that facilitates code generation and execution while maintaining awareness of the complete notebook state.
This enables context-aware coding, automated debugging, and dependency management, along with comprehensive state management that allows users to restore previous notebook states.

The combination of notebook as well as chat-based features allows users to interact with code, data, and LLM features all from one seamless interface. 
% This powerful combination means a user can, for instance, register a dataset, set a schema, filter, then convert and extract information without ever leaving a single chat or environment.

% While maintaining full compatibility with Jupyter notebook formats, Beaker supports seamless transitions between notebook-style coding and conversational interfaces. 
% Advanced users can customize agents with specific toolsets using the ReAct paradigm, allowing for adaptation to diverse use cases while preserving the familiar notebook interface paradigm.
% \subsection{\sys\ Task Abstraction}
% 
% \chunwei{I think this is a good opportunity to mention something about language-independent abstraction of \sys\ workload. This abstraction could serve as unified representation across any \sys\ language interface.}


\section{Demonstration}
\label{sec:demoscenario}

In this demonstration scenario, we will present a real-world use case leveraging \chat{} to build an end-to-end pipeline for scientific discovery.

\textbf{Use case:} Consider medical researchers interested in performing a literature survey on \emph{colorectal cancer}. 
They are interested in finding studies that report on the correlation between gene mutation and tumor cells, providing or using publicly available datasets of genomic data, and fetching these datasets to run their own analysis.
Assume they have access digital library of scientific papers, but this collection is potentially large, containing unrelated papers, and it is not annotated with metadata about the data sources.

To automate this task, they can build an AI data-processing pipeline using \sys{} through the \chat{} interface.
At a high level, the medical researcher has to upload a collection of papers, filter for those about colorectal cancer, and then extract any reference of publicly available datasets.

% \subsection{Dataset Operations}
% \textbf{Dataset operations:}
At the core of \sys{}, there are \emph{datasets}: collections of input records.
The first step when building a pipeline is to define an input dataset - this could either be a local folder, for which every file will constitute an individual record; or an iterable object in memory, for which every item will be a record.
Additionally, more experienced users can define any custom logic to marshal arbitrary objects or paths into input datasets.

% offers several interfaces for managing datasets, including listing available datasets, registering new ones, and removing unused entries. It also supports retrieving and displaying specific datasets. 

For our demonstration, the user instructs \chat{} to load an input dataset from PDFs of scientific papers contained in a local folder (See Figure~\ref{fig:dataset_ops}).

%\subsection{Data transformations}
The core \chat{} system includes a native \texttt{PDFFile} schema, which is automatically chosen to parse the files in this dataset given their extension. 
However, this schema only represents the filename and the raw textual content extracted for a given paper.

Next, the user informs \chat{} that they are interested in papers that are about colorectal cancer, and for these papers, that they would like to extract whatever public dataset is been used by the study in the paper.
From these requests, the reasoning agent creates a pipeline of \sys{} operations: it first uses the \texttt{filter()} function to select papers that satisfy the desired condition.
Then, it dynamically generates an extraction schema with the fields \texttt{dataset\_name}, \texttt{description}, and \texttt{URL}.
Finally, it applies the extraction schema to the filtered dataset with a \texttt{convert()} operation.
Figure~\ref{fig:chat-extract} highlights these steps. 
In the background, \chat{} decomposes the user’s request into sub-steps for data loading, filtering, conversion, and execution.
Before running the full pipeline to process the inputs, the users can specify their optimization goal: minimum cost, maximum quality, or minimum runtime.
\sys{} will automatically determine the appropriate physical plan that implements the logical plan built so far.

The final code generated can be seen in Figure~\ref{fig:finalcode}.

\begin{figure}[t]
    \centering
    \includegraphics[width=\linewidth]{figures/demo-pipeline.png}
    \caption{Building a pipeline through natural language. As seen in the last call, the agent reasons and may decide to decompose a user question into several tasks required  before execution.}
    \label{fig:chat-extract}
\end{figure}

When the users prompt the chat to run the pipeline, they will visualize the extracted dataset names and URLs.
Moreover, users can gain insights into the workload execution by asking the system to provide statistics such as how much runtime was needed to produce the output, and how much the LLM invocations costed.
A sample of this output can be seen in Figure~\ref{fig:demo-results}.
For the scientific discovery use case, out of an input dataset of 11 papers, the pipeline managed to extract 6 publicly available datasets related to colorectal cancers, together with the associated URLs.
We manually verified the validity of these URLs.

The execution statistics of the pipeline show that the workload was executed in about 240s and with a cost of about 0.35 USD.

Finally, after building a pipeline, users may continue to iterate on the code produced either through the chat interface or by downloading a Jupyter notebook that contains all inputs and generated snippets of code.

% \begin{figure}[t]
%     \centering
%     \includegraphics[width=\linewidth]{figures/filter_result.png}
%     \caption{\chat{} reasons and decomposes the complex task into compatible steps before execution. 
%     % The outputs of running a \sys{} pipeline. 
%     Users may specify whether they want to optimize for cost, quality, or runtime.
%     }
%     \label{fig:chat-extract}
% \end{figure}

\begin{figure}[t]
    \centering
    \includegraphics[width=\linewidth]{figures/demo-results.png}
    \caption{The output of the execution of the scientific discovery use case. Users can visualize both output records, as well as summary information about the plan execution such as the operators chosen and the total pipeline cost and runtime. 
    }
    \label{fig:demo-results}
\end{figure}

% to create or select a schema that precisely defines the attributes to be extracted or transformed. During this step, users provide a broad schema description—optionally clarifying field requirements, providing attribute descriptions, and indicating which fields are mandatory. If the schema definition is ambiguous or lacks detail, \chat{} prompts the user to supply additional information. This iterative process ensures that \sys{} has a precise understanding of the dataset's structure and context for downstream processing tasks.

% \subsection{Task Execution}
% Finally, we demonstrate how \chat{} orchestrates the user-defined tasks over the dataset. The system interprets the steps needed—such as converting documents to a target schema or filtering records based on user-defined criteria—and executes the pipeline. 

\begin{figure}[t]
\begin{minted}[escapeinside=||,texcomments]{python}
#Set input dataset
schema = PDFFile
dataset = pz.Dataset(source="sigmod-demo", schema=schema)

#Filter dataset
dataset = dataset.filter("The papers are about colorectal cancer")

#Create new schema
class_name = "ClinicalData"
schema = {"__doc__": "A schema for extracting clinical data datasets from papers."}
field_names = ['name', 'description', 'url']
field_descriptions = ['The name of the clinical data dataset', 
    'A short description of the content of the dataset', 
    'The public URL where the dataset can be accessed']
for idx, field enumerate(field_names):
    desc = field_descriptions[idx]
    attributes[name] = pz.Field(desc=desc)
new_class = type(class_name, (pz.Schema,), attributes)

#Perform conversion
convert_schema = ClinicalData
cardinality = pz.Cardinality.ONE_TO_MANY
dataset = dataset.convert(convert_schema, desc=ClinicalData.__doc__, cardinality=cardinality)

#Execute workload
output = dataset
policy = pz.MaxQuality()
records, execution_stats = Execute(output, policy=policy)
\end{minted}
\caption{The final \system{} pipeline built iteratively using the chat interface for the scientific discovery page.}
\label{fig:finalcode}
\end{figure}


We introduced \name{}, a dataset designed to evaluate how LLMs are affected by different framings, in comparison to humans. The statements in the dataset are start from a clear positive or negative base statement, reframed with an opposite-sentiment suffix or prefix. By collecting human annotations, we quantify the strength of the framing effect for each instance and assess how LLMs react to the same reframing, focusing on sentiment shifts -- cases where the perceived sentiment aligns with the opposite framing sentiment.

As LLMs become increasingly integrated into decision-making systems, it becomes crucial to understand how framing influences their outputs. 
In some applications, e.g., in virtual companions, framing can be harnessed to produce human-like behavior leading to better engagement.
In contrast, in other applications, such as financial or legal advice, mitigating the framing effect can lead to less biased decisions.
In both cases, a better understanding of framing in LLMs can help develop application-appropriate strategies.

We find that LLMs and humans exhibit similar behavior on \name{}, with all tested models achieving a strong correlation  $(r \geq 0.57)$ with human responses. Notably, \gpt{} showed the weakest correlation to humans, despite being widely regarded as a very capable model in other contexts.

We hope this work encourages research into distinguishing between scenarios where human-like behavior is desirable and those where models should surpass human limitations to achieve above-human performance. These insights are essential for developing LLMs that are both interpretable and optimized for their intended applications.

\begin{acks}
    We gratefully acknowledge the support of the DARPA ASKEM project (Award No. HR00112220042), and the ARPA-H Biomedical Data Fabric project (NSF DBI 2327954).
\end{acks}

\bibliographystyle{ACM-Reference-Format}
\bibliography{bib}


\end{document}
\endinput
%%
%% End of file `sample-authordraft.tex'.
