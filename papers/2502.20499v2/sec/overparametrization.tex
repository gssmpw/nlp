\section{Lack of Capacity does not explain SG}

Given that \textit{diversity} has shown to improve SG performance, we would like to gain a deeper understanding of why this is the case. Our first hypothesis is the following: adding more colors to the training distribution strains the capacity of the model such that it is forced to learn simpler learning solutions \textit{shape} and \textit{color} independently of each other. In other words, lack of capacity may force the model to find a more systematic solution. To test this, we test two cases: 1) We artificially restrict the capacity of a baseline Transformer model trained on 8 colors by limiting the size of the residual stream dimension ($d \in \{ 32, 64, 128 \}$). 2) We augment the capacity of the model trained on 216 colors by increasing this dimension by $\{512, 1024\}$.

Figure \ref{fig:smaller-models} shows results for these experiments. As can be seen, performance in both $\mathcal{D}_{test-ID}$ and $\mathcal{D}_{test-OOD}$ decreases as model capacity is reduced in the first case, and the converse happens when we increase capacity. So, lack of capacity does not seem to be the mechanism by which \textit{diversity} achieves SG. We now turn to mutual information between latent attributes as a possible explanation.



\begin{figure}[h]
    \centering
    \begin{subfigure}{0.45\linewidth}
    \includegraphics[width=\linewidth]{figures/results/acc-color-smaller-models.pdf}
    \caption{\textit{Color} Task}
    \end{subfigure}
    \hfill
    \begin{subfigure}{0.45\linewidth}
    \includegraphics[width=\linewidth]{figures/results/acc-shape-smaller-models.pdf}
    \caption{\textit{Shape} Task}
    \end{subfigure}

    \bigskip

    \begin{subfigure}{0.45\linewidth}
    \includegraphics[width=\linewidth]{figures/results/acc-color-larger-models.pdf}
    \caption{\textit{Color} Task}
    \end{subfigure}
    \hfill
    \begin{subfigure}{0.45\linewidth}
    \includegraphics[width=\linewidth]{figures/results/acc-shape-larger-models.pdf}
    \caption{\textit{Shape} Task}
    \end{subfigure}
    \caption{In-Distribution and Out-of-Distribution accuracy for models trained with different values for their hidden dimensions on a training set with 216 colors for the \textit{color} and \textit{shape} task. (a-b) When exposed to low diversity (8 colors) both metrics suffer as model capacity is reduced. (c-d) Both metrics increase as model capacity is increased. These results suggest that lack of capacity of the models is not the cause for achieving greater SG when augmenting the number of colors in the training set.}
    \label{fig:smaller-models}

\end{figure}