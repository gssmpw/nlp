
\subsection{Burstiness}

\begin{figure}[h]
    \centering
    
    \begin{subfigure}[b]{0.475\linewidth}
        \captionsetup{justification=centerlast, width=0.9\linewidth}
        \includegraphics[width=\linewidth]{figures/results/delta-acc-colors-by-p-burstiness.pdf}
        \caption{\textit{Color} Task}
        \label{fig:colors-common-colors-8}
    \end{subfigure}
    \hspace{\fill}
    \begin{subfigure}[b]{0.475\linewidth}
        \captionsetup{justification=centerlast, width=0.9\linewidth}
        \includegraphics[width=\linewidth]{figures/results/delta-acc-shapes-by-p-burstiness.pdf}
        \caption{\textit{Shape} Task}
        \label{fig:shapes-common-colors-8}
    \end{subfigure}
    
    
    \caption{Change of accuracy with respect to no burstiness for the \textit{color} and \textit{shape} tasks in test-ID and test-OOD for different levels of \textit{burstiness} over \textit{color} for various numbers of colors. Limiting the number of colors available for each image during training allows the model to gain up to 14.8\% more accuracy over the baseline. The \textit{color} task, however, suffers up to 14.3\% decline as the \textit{color} task becomes easier to memorize.}
    \label{fig:burstiness}
\end{figure}


As was mentioned in Section \ref{sec:distributional_properties_training_data}, another way of disrupting the link between \textit{color} and \textit{shape}, is to hinder the learning of \textit{color} by limiting the number of colors contained in a given image, something we call \textit{burstiness}. Figure \ref{fig:burstiness} shows the results. As can be seen, increasing the \textit{burstiness} of the training data increases the SG capability of the model in the \textit{shape} task by up to 14.8\%. As expected,  however, the \textit{color} task is significantly affected, both in $\mathcal{D}_{test-ID}$ and $\mathcal{D}_{test-OOD}$, losing up to 14.3\% accuracy in $\mathcal{D}_{test-OOD}$ for 64 colors. This last effect is especially pronounced as we increase the number of colors contained in the dataset, which is to be expected as the \textit{color} task becomes significantly harder as the number of colors increases dramatically, while the limit of only 3 colors per image remains fixed. Finally, the effect of \textit{burstiness} on \textit{material} and \textit{size} is very small, with at most 1\% gain in $\mathcal{D}_{test-OOD}$.


% These results supports the finding of \cite{chan2022data-properties-drives-in-context-learning}, that contextually limiting the number of colors in the image 

% Furthermore, 
