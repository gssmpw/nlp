\section{Impact of Altering the Training Distribution on SG}
\label{sec:results}
We investigate the impact on SG of altering $\mathcal{D}_{train}$ by manipulating properties specifically tailored to perturbate the link between $a$ and $b$.


\subsection{Diversity}

\begin{figure}[h]
    \centering
    \begin{subfigure}[b]{0.45\linewidth}
        \includegraphics[width=\linewidth]{figures/results/color_task_by_colors.pdf}
        \caption{\textit{Color} Task}
        \label{fig:colors-acc-by-num-colors}
    \end{subfigure}
    \hspace{\fill} % Add horizontal fill space
    \begin{subfigure}[b]{0.45\linewidth}
        \includegraphics[width=\linewidth]{figures/results/shape_task_by_colors.pdf}
        \caption{\textit{Shape} Task}
        \label{fig:shapes-acc-by-num-colors}
    \end{subfigure}
    
    \vspace{1em} % Adjust spacing between rows if needed
    
    \begin{subfigure}[b]{0.45\linewidth}
        \includegraphics[width=\linewidth]{figures/results/material_task_by_colors.pdf}
        \caption{\textit{Material} Task}
        \label{fig:materials-acc-by-num-colors}
    \end{subfigure}
    \hspace{\fill} % Add horizontal fill space
    \begin{subfigure}[b]{0.45\linewidth}
        \includegraphics[width=\linewidth]{figures/results/size_task_by_colors.pdf}
        \caption{\textit{Size} Task}
        \label{fig:size-acc-by-num-colors}
    \end{subfigure}

    % \caption{ \label{fig:acc_by_colors} Accuracy versus the number of colors in $\mathcal{D}_{train}$ for $\mathcal{D}_{test-ID}$ and $\mathcal{D}_{test-OOD}$ for the \textit{color} and \textit{shape} tasks. Performance for the \textit{shape} task increases drastically for the OOD split as we increase colors, increasing 86\% in absolute terms over the 8-color baseline. Moreover, performance in the color task also tends to increase in the out-of-distribution split, while in-distribution only suffers slightly, even though the task becomes significantly harder. Remarkably, the \textit{material} and \textit{size} task rapidly increase their $\mathcal{D}_{test-OOD}$ performance as color increases.}
    \caption{ \label{fig:acc_by_colors} Accuracy versus the number of colors in $\mathcal{D}_{train}$ for $\mathcal{D}_{test-ID}$ and $\mathcal{D}_{test-OOD}$ for the \textit{color} and \textit{shape} tasks. Performance for the \textit{shape} task increases drastically for the OOD split as we increase colors, increasing 86\% in absolute terms over the 8-color baseline. Moreover, performance in the color task also tends to increase in the out-of-distribution split, while in-distribution only suffers slightly, even though the task becomes significantly harder. Remarkably, the \textit{material} and \textit{size} task rapidly increase their $\mathcal{D}_{test-OOD}$ performance as color increases.}

    
\end{figure}

To increase \textit{diversity}, we run experiments where we increase colors in $\mathcal{D}_{train}$ from 8 to 216. Figure \ref{fig:acc_by_colors} shows these results. We observe that augmenting \textit{color diversity} improves SG on the \textit{shape} task (see monotonic increase of {\color{OliveGreen}green bars}. This increase is of an absolute 89\%, a staggering amount that reflects the difference between a model that does not generalize systematically to one that does. Unexpectedly, the performance on color prediction barely decreases in test-ID (around 6\%) (Fig. \ref{fig:colors-acc-by-num-colors}), in spite of the increased difficulty of the task given by a higher number of classes. Moreover, increased color diversity reduces the performance gap between $\mathcal{D}_{test-ID}$ and $\mathcal{D}_{test-OOD}$ in the \textit{color} task (Fig. \ref{fig:colors-acc-by-num-colors}). Remarkably, out-of-distribution performance for the $material$ and $size$ tasks also increases significantly when augmenting colors. This suggests that increased diversity seems to be creating an inductive bias where at least \textit{color} is being disassociated from \textit{shape}, \textit{material} and \textit{size}. Moreover, this increase for \textit{material} and \textit{size} occurs rapidly with just 27 colors, while for \textit{shape} this keeps increasing until reaching 216 colors.

Finally, we test whether increasing diversity affects the model's sample efficiency. We proceed to train models on datasets with 216 colors with size $\{0.25, 0.5, 0.75\}$ of the original training size and compare them to the 8-color baseline at full training size. Results are shown in Figure \ref{fig:sample-efficiency}. As can be seen, even training with just one quarter of the original training size with increased diversity produces much stronger out-of-distribution generalization than the baseline, while the gap between in-distribution and out-of-distribution remains relatively low.

%Importantly, the trade-off between SG and memorization shows that increasing color diversity leads to significant SG performance gains on shapes, an absolute increase of 89\%, despite a slight decrease in color ID performance, an absolute decrease of 6\%. This shows that properties of the data distribution play a relevant role in biasing the latent space to focus on the disentangling of semantic factors, here color and shape.
\begin{comment}{

\begin{figure}[h]
    \centering
    
    \begin{subfigure}[b]{0.45\linewidth}
    \includegraphics[width=\linewidth]{figures/results/acc-colors-by-common-colors-8.pdf}
    \caption{Color Task on datasets with 8 Colors \label{fig:colors-common-colors-8}}
    \end{subfigure}
    \hfill
    \begin{subfigure}[b]{0.45\linewidth}
    \includegraphics[width=\linewidth]{figures/results/acc-shapes-by-common-colors-8.pdf}
    \caption{Shape Task on datasets with 8 Colors \label{fig:shapes-common-colors-8}}
    \end{subfigure}
    
    \bigskip
    
    \begin{subfigure}[b]{0.45\linewidth}
    \includegraphics[width=\linewidth]{figures/results/acc-colors-by-common-colors-216.pdf}
    \caption{Color Task on datasets with 216 Colors \label{fig:colors-common-colors-216}}
    \end{subfigure}
    \hfill
    \begin{subfigure}[b]{0.45\linewidth}
    \includegraphics[width=\linewidth]{figures/results/acc-shapes-by-common-colors-216.pdf}
    \caption{Shape Task on datasets with 216 Colors \label{fig:shapes-common-colors-216}}
    \end{subfigure}
    \caption{Accuracy in test-ID and test-OOD for different number of colors (\textit{diversity}). Left corresponds to the \textit{color} task and right to \textit{shape} task. Increasing the number of colors produces a massive increase in systematic generalization on the \textit{shape} task, increasing up to an absolute 89\% over the 8-color baseline.}
    \label{fig:common-colors}
\end{figure}
}
\end{comment}