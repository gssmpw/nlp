
\begin{abstract}


Deep neural networks (DNNs) struggle at systematic generalization (SG). Several studies have evaluated the possibility to promote SG through the proposal of novel architectures, loss functions or training methodologies. Few studies, however, have focused on the role of training data properties in promoting SG. In this work, we investigate the impact of certain data distributional properties, as inductive biases for the SG ability of a multi-modal language model. To this end, we study three different properties. First, data diversity, instantiated as an increase in the possible values a latent property in the training distribution may take. Second, burstiness, where we probabilistically restrict the number of possible values of latent factors on particular inputs during training. Third, latent intervention, where a particular latent factor is altered randomly during training. We find that all three factors significantly enhance SG, with diversity contributing an 89\% absolute increase in accuracy in the most affected property. Through a series of experiments, we test various hypotheses to understand why these properties promote SG. Finally, we find that Normalized Mutual Information (NMI) between latent attributes in the training distribution is strongly predictive of out-of-distribution generalization. We find that a mechanism by which lower NMI induces SG is in the geometry of representations. In particular, we find that NMI induces more parallelism in neural representations (i.e., input features coded in parallel neural vectors) of the model, a property related to the capacity of reasoning by analogy.



%Diversity
%Burstiness
%Intervention
%Parallel-Scores
%VQA

%Based on CLEVR, a popular benchmark to test multimodal reasoning abilities in AI models, we explore data configurations that promote the emergence of SG abilities in DNNs. Our results show that SG strongly depends on data distribution properties. Notably, we observe that SG is significantly enhanced by promoting diversity (e.g., augmenting object colors in the dataset). Increasing the model’s SG performance by up to 60\% absolute increase in accuracy in the most affected attributes. Moreover, we find that label overloading—where a single label represents multiple values for a particular property, such as color—is a key feature for SG. Decreases overfitting for all attributes.

%Traditional artificial neural networks (ANNs) struggle at systematic generalization (SG). However, few studies have focused on how training data configuration can act as inductive biases to promote SG. In this work, we investigate the impact of data properties on the SG ability of a multi-modal language model, text and images. Based on CLEVR, a popular dataset to test multimodal reasoning abilities in AI models, we explore data configurations that promote the emergence of SG abilities in ANN models. Our results show that SG strongly depends on data distribution properties. Notably, we observe that SG is significantly enhanced by promoting diversity (e.g., augmenting object colors in the dataset). Increasing the model’s SG performance by up to 60\% absolute increase in accuracy in the most affected attributes. Moreover, we find that label overloading—where a single label represents multiple values for a particular property, such as color—is a key feature for SG. Decreases overfitting for all attributes.
\end{abstract}