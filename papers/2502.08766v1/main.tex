\documentclass[conference]{IEEEtran}

\IEEEoverridecommandlockouts
% The preceding line is only needed to identify funding in the first footnote. If that is unneeded, please comment it out.

\usepackage{multirow}
\usepackage[table,xcdraw]{xcolor}
\usepackage{threeparttable}

\usepackage{changes}
\definechangesauthor[name={Wei}, color=blue]{YN}
\definechangesauthor[name={Wei}, color=red]{DEL}

\def\method{\text MixMin~}
\def\methodnospace{\text MixMin}
\def\genmethod{$\mathbb{R}$\text Min~}
\def\genmethodnospace{ $\mathbb{R}$\text Min}


\def\BibTeX{{\rm B\kern-.05em{\sc i\kern-.025em b}\kern-.08em
    T\kern-.1667em\lower.7ex\hbox{E}\kern-.125emX}}
\begin{document}

\title{Unlocking Mental Health: Exploring College Students' Well-being through Smartphone Behaviors
}

\author{\IEEEauthorblockN{1\textsuperscript{st} Wei Xuan}
\IEEEauthorblockA{\textit{Department of Economics} \\
\textit{University of Southern California}\\
Los Angeles, CA, USA \\
wxuan@usc.edu}
\and
\IEEEauthorblockN{2\textsuperscript{nd} Meghna Roy Chowdhury, 3\textsuperscript{rd} Yi Ding}
\IEEEauthorblockA{\textit{Electrical and Computer Engineering} \\
\textit{Purdue University}\\
West Lafayette, IN, USA \\
mroycho@purdue.edu, yiding@purdue.edu}
% \and
% \IEEEauthorblockN{3\textsuperscript{rd} Yi Ding}
% \IEEEauthorblockA{\textit{Electrical and Computer Engineering} \\
% \textit{Purdue University}\\
% West Lafayette, IN, USA \\
% yiding@purdue.edu}
\and
\IEEEauthorblockN{4\textsuperscript{th} Yixue Zhao}
\IEEEauthorblockA{\textit{Information Sciences Institute} \\
\textit{University of Southern California}\\
Arlington, VA, USA \\
yzhao@isi.edu}
}

\maketitle

\begin{abstract}
% Space: 1/4 of Page 1
The global mental health crisis is a pressing concern, with college students particularly vulnerable to rising mental health disorders. The widespread use of smartphones among young adults, while offering numerous benefits, has also been linked to negative outcomes such as addiction and regret, significantly impacting well-being. Leveraging the longest longitudinal dataset collected over four college years through passive mobile sensing, this study is the first to examine the relationship between students' smartphone unlocking behaviors and their mental health at scale in real-world settings. We provide the first evidence demonstrating the predictability of phone unlocking behaviors for mental health outcomes based on a large dataset, highlighting the potential of these novel features for future predictive models. Our findings reveal important variations in smartphone usage across genders and locations, offering a deeper understanding of the interplay between digital behaviors and mental health. We highlight future research directions aimed at mitigating adverse effects and promoting digital well-being in this population.
\end{abstract}

% \begin{IEEEkeywords}

% \end{IEEEkeywords}
\vspace{-2pt}
\section{Introduction}\label{intro}
\vspace{-2pt}
% Space: 1/2 of Page 1
College years are pivotal for young adults' personal, social, and academic development, often accompanied by significant stressors \cite{magolda2006intellectual,anderson2012college}. Mobile devices have become integral to college students' daily lives, with U.S. students averaging 8–10 hours of smartphone use per day \cite{roberts2014invisible}. 
A considerable body of literature suggests that excessive smartphone usage interferes with daily activities and diminishes mental health \cite{ames2013managing,cho2021reflect,ruensuk2022sad,lin2016association}. However, most prior work overlooks the multifaceted nature of smartphone use. When used mindfully, smartphones can be powerful tools that enhance productivity and well-being \cite{buchi2024digital, zhao2024digital}. Furthermore, smartphones' impact likely varies across user groups and contexts, highlighting the need for a more nuanced understanding of its relationship with mental health.

To gain a comprehensive understanding of college students' smartphone usage and its association with mental well-being, we utilized data from the {College Experience Study} (CES) dataset~\cite{nepal2024capturing}, the longest-running mobile sensing study to date. This study provides a rich dataset of passive mobile sensing data and Ecological Momentary Assessment (EMA) surveys collected from over 200 Dartmouth students between 2017 and 2022. 
Our analysis specifically focuses on the fundamental smartphone behavior of \textit{unlocking}, considering both its \emph{frequency} and \emph{duration}, to explore its relationship with students' mental health as measured by the Patient Health Questionnaire-4 (PHQ4)\cite{lowe20104,kroenke2009ultra}.
We focus on \emph{unlocking} behaviors due to empirical evidence linking them to mental health and their coarse-grained nature, which allows prediction without compromising user privacy (e.g., revealing app usage). 

This is the first large-scale study based on over 210,000 data points collected throughout entire college years that utilizes {real-world} unlocking data to explore its predictability for mental health, moving beyond prior work that has relied on indirect methods to study phone usage~\cite{abdrabou2024eyegaze, harbach2014survey, li2019swipevlock}. % Our study is the first large-scale investigation of college students' smartphone unlocking behaviors in real-life settings, based on over 210,000 data points collected throughout their entire college years. 
We first conduct correlation analysis to assess the feasibility of using smartphone unlocking behaviors to predict students' mental health status. To guide the granularity of future predictive models, we further explored the impact of gender and location differences (e.g., study spaces, social venues, and homes) on the correlations.
Following the standardization approach for PHQ4 Score in \cite{wicke2022update}, we categorize the PHQ4 Score into four groups---“Normal,” “Mild,” “Moderate,” and “Severe”---to facilitate easier interpretation. Due to the discrete nature of these outcome variables, we employ multinomial logistic regression~\cite{bohning1992multinomial} to analyze the relationships between smartphone unlocking behaviors and mental well-being. Rather than using a single global model for the entire population, we apply separate multinomial logistic regression models for different genders and locations to capture the dataset's heterogeneity.
% \hl{To further explore these novel features' predictability, we employ a multinomial logistic regression model given the discrete nature of the outcome variable (PHQ4 scores ranging from 0 to 12 where lower score indicates better mental health) with multiple categories. For interpretability, PHQ4 scores are grouped into four categories, with the lowest-score group serving as the baseline.}
% \yixue{@Yi, the highlighted parts may need to be revised. Also, the reason to use logistic regression isn't highlighted either. I remember I didn't get logistic regression can understand the *combined* features' predictability instead of individual features that's studied in a regular Pearson correlation. This point should be highlighted too as reviewers are like me who wouldn't get it if we don't point it out. Here's the relevant review regarding the discrete nature: Reviwer 1: Section 2.E: The authors need to better explain the merit of their categories and how sensitive the survey is to changes in overall score. Why did the authors not work with a continuous scale? With the categories they still have the problem that scores of 2 and 3 are in different categories. Why is this less of an issue than having 1 and 2 in different categories? My main concern is that whever you try to fit a continuous scale into categories, there may be oddities at the boundaries between categories.}
% The predictive features include unlock duration and unlock frequency. 
%Given the discrete outcome of interest (PHQ4 scores ranging from 0 to 12) with multiple categories, we then employ a multinomial logistic regression model to explore the predictability of the unlocking behaviors on students' well-being. For interpretability, PHQ4 scores are grouped into four categories, with the lowest score group serving as the baseline. Unlock duration and the number of unlocks are included as predictive features in the model. 

Our analysis uncovers nuanced phenomena in terms of gender differences: for male students, increased phone usage is linked to poorer mental health, while for female students, it appears to have a positive impact, suggesting fundamentally different dynamics in how phone usage affects well-being across genders. 
% Finally, we explore phone usage patterns across different locations, including study spaces, social venues, dining areas, dormitories, and homes, uncovering contextual relationships with mental well-being. 
These findings provide insights and actionable items for future work to build predictive models and design digital well-being interventions, as discussed in Section~\ref{result}. % \yixue{@Yi, any novelty for research approach/methodology to add in this paragraph above?}
%Specifically, we distinguish between two aspects of phone usage: frequency, as indicated by the number of locks and unlocks, and duration, as indicated by the time elapsed between a lock and unlock. Using a multiple linear regression model, we examine the relationship between different dimensions of smartphone usage and college students' mental health status. To explore diversity in these relationships, we develop separate regression models for different genders. Additionally, we analyze phone usage patterns in various locations, such as study spaces, social venues, dining areas, dormitories, and homes, to uncover contextual relationships with mental well-being.

This paper makes these contributions: 
1) We conduct the first large-scale study of real-world smartphone unlocking behaviors and mental health among college students, encompassing both iOS and Android users; 
2) We provide the first empirical evidence on the predictability of novel smartphone unlocking features, opening up future directions in predicting mental health; 
3) We demonstrate the gender differences and location variations in students' smartphone behaviors and their relationships with mental health, providing actionable insights for building predictive models;
4) We open-source the data analysis pipeline and artifacts to foster future research~\cite{OurRepo}.
%including distribution analysis at individual level, Pearson correlation analysis, and multinomial logistic regression analysis, to foster future research upon acceptance.
%First, we differentiate between smartphone usage frequency and duration, demonstrating how these dimensions relate to college students' mental health in distinct ways. This provides a holistic view of the interplay between phone usage and students' well-being. 
%Second, we highlight gender differences in the correlations between phone usage and mental health. Third, we present the first empirical evidence linking phone usage with mental well-being across different locations and contexts. Lastly, by leveraging insights from the most extensive longitudinal mobile sensing study to date, we provide valuable knowledge that can inform interventions aimed at improving phone usage habits and enhancing mental health.


% The remainder of this paper is structured as follows. Section \ref{design} introduces the dataset and explains the design of our models. Section \ref{result} details the empirical results. Section \ref{literature} discusses related work in this domain. Finally, Section \ref{conclusion} concludes the paper and outlines future research directions.


\section{The \search\ Search Algorithm}
\label{sec:search}

%In traditional ML, structure changes and step (operator) changes are performed before model training, \ie, fixed to the training process, and weights are updated with SGD, because weights are continous, differentiable values, and there are significantly more weights than structure and operator changes. In workflow autotuning, all three types of cogs can be chosen with a unified search-based approach, because all of them are non-differentiable configurations and the number of cogs in different types are all small.
%Thus, \sysname\ only needs to navigate the search space of combination of cogs as the search space to produce its workflow optimization results.

%We propose, \textit{\textbf{\search}}, an adaptive hierarchical search algorithm that autotunes gen-AI workflows based on observed end-to-end workflow results. In each search iteration, \search\ selects a combination of cogs to apply to the workflow and executes the resulting workflow with user-provided training inputs. \search\ evaluates the final generation quality using the user-specified evaluator and measures the execution time and cost for each training input. These results are aggregated and serve as BO observations and pruning criteria.
%the optimizer can condition on and propose better configurations in later trials. The optimizer will also be informed about the violation of any user-specified metric thresholds. More details of this mechanism can be found in Appendix ~\ref{appdx:TPE}.

With our insights in Section~\ref{sec:theory}, we believe that search methods based on Bayesian Optimizer (BO) can work for all types of cogs in gen-AI workflow autotuning because of BO's efficiency in searching discrete search space.
A key challenge in designing a BO-based search is the limited search budgets that need to be used to search a high-dimensional cog space. 
For example, for 4 cogs each with 4 options and a workflow of 3 LLM steps, the search space is $4^{12}$. Suppose each search uses GPT-4o and has 1000 output tokens, the entire space needs around \$168K to go through. A user search budget of \$100 can cover only 0.06\% of the search space. A traditional BO approach cannot find good results with such small budgets.
%The entire search space grows exponentially with the number of cogs and the number of steps in a workflow. Moreover, different cogs and different combinations of cogs can have varying impacts on different workflows. 
%Without prior knowledge, it is difficult to determine the amount of budget to give to each cog.

To confront this challenge, we propose \textit{\textbf{\search}}, an adaptive hierarchical search algorithm that efficiently assigns search budget across cogs based on budget size and observed workflow evaluation results, as defined in Algorithms~\ref{alg:main} and \ref{alg:outer} and described below.
%autotunes gen-AI workflows based on observed end-to-end workflow results.
%\search\ includes a search layer partitioning method, a search budget initial assignment method, an evaluation-guided budget re-allocation mechanism, and a convergence-based early-exiting strategy. We discuss them in details below.

%\zijian{\search\ allows users to specify the optimization budget allowed in terms of the maximum number of search iterations. Based on the relationship between the complexity of the search space and the available budget, we will separate all tunable parameters into different layers each optimized by independent Bayesian optimization routines. Then we will decide the maximum budget each layer can get with a bottom-up partition strategy. Besides search space and resource partition, we also employ a novel allocation algorithm that integrates successive halving~\cite{successivehalving} and a convergence-based early exiting strategy to facilitate efficient usage of assigned budget.}


% The outermost layer searches and selects structures for a workflow; the middle layer searches and selects step options under the workflow structure selected in the outermost layer; the innermost layer searches and selects weights with the given workflow structure and steps. 

\begin{algorithm}[h]
    \caption{\search\ Algorithm}
    \label{alg:main}
      \small
\begin{algorithmic}[1]
\STATE \textbf{Global Value:} $R = \emptyset$ \COMMENT{Global result set}
%\STATE \textbf{Global Value:} $F = \emptyset$ \COMMENT{Global observation set}

%Reduct factor $\eta > 1$, explore width $W$
\STATE \textbf{Input:} User-specified Total Budget $TB$
\STATE \textbf{Input:} Cog set $C = \{c_{11},c_{12},...\}, \{c_{21},c_{22},...\}, \{c_{31},c_{32},...\}$

    \STATE
%\FOR{$i = 1,2,3$}
    %\COMMENT{$\alpha$ is a configurable value default to 1.1}
%\ENDFOR
%\STATE
%    \STATE \{$B_1,B_2,B_3$\} = LayerPartition($C$) \COMMENT{Calculate ideal layer budget}
    %\STATE \textbf{Glob}.budgets = budgets
%    \STATE opt\_layers = init\_opt\_routines() \COMMENT{A list of optimize routine each layer will use for search}
%\STATE
%    \FOR{$i \in L, \dots, 1$}
%        \IF{$i == L$}
 %           \STATE opt\_layers[L] = InnerLayerOpt
  %      \ELSE
   %         \STATE opt\_layers[i] = OuterLayerOpt
            %\STATE opt\_layers[i].next\_layer\_budgets = B[i+1]
            %\STATE opt\_layers[i].next\_layer\_routine = opt\_layers[i+1]
    %    \ENDIF
    %\ENDFOR
%\STATE opt\_layers[1].invoke($\emptyset$, B[1])
\STATE $U = 0$ \COMMENT{Used budget so far, initialize to 0}

\STATE \COMMENT{Perform search with 1 to 3 layers until budget runs out}
\FOR{$L = 1,2,3$} 
        \IF{$L=1$}
            \STATE $C_1 = C_1 \cup C_2 \cup C_3$ \COMMENT{Merge all cogs into a single layer}
        \ENDIF
        \IF{$L==2$}
            \STATE $C_1 = C_1 \cup C_2$ \COMMENT{Merge step and weight cogs}
            \STATE $C_2 = C_3$ \COMMENT{Architecture cog becomes the second layer}
        \ENDIF
        \STATE
    \FOR{$i = 1,..,L$}
    \STATE $NC_i = |C_i|$ \COMMENT{Total number of cogs in layer $L$} 
%    NO_i &= \sum_{L} \{\text{number of possible options in cog } c_{ij}\} \\
    \STATE $S_i = NC_i^\alpha$ \COMMENT{Estimated expected search size in layer $i$}
    \ENDFOR
    \STATE $E_L = \prod\limits_{i=1}^{L}S_i$ \COMMENT{Expected total search size in the current round}
    \STATE $E = TB - U > E_L$ ? $E_L$ : $(TB - U)$ \COMMENT{Consider insufficient budget} 
    \IF{$L==3$ and $(TB - U)$ > $E_L$}
         \STATE $E = TB - U$ \COMMENT{Spend all remaining budget if at 3 layer}
    \ENDIF
    %\STATE$TL = |N|$ \COMMENT{number of layers}
    \FOR{$i = 1,..,L$}
        \STATE $B_i =  \lfloor S_i \times \sqrt[L]{\frac{E}{E_L}}\rfloor$
        %$B$ = BudgetAssign($N$, $TL$, $TB$)
        \COMMENT{Assign budget proportionally to $S_i$}
    \ENDFOR
    \STATE
\STATE \texttt{LayerSearch} ($\emptyset$, $B$, $L$, $B_L$) \COMMENT{Hierarchical search from layer $L$}
\STATE
\STATE $U = U + E$
\IF{$U \geq TB$}
\STATE break \COMMENT{Stop search when using up all user budget}
\ENDIF
\ENDFOR
%\STATE
%\STATE $O$ = \texttt{SelectBestConfigs} ($R$)
%\IF{$L == 1$}
%    \STATE InnerLayerOpt($\emptyset$, B[1])
%\ELSE
%    \STATE OuterLayerOpt($\emptyset$, B[1], 1)
%\ENDIF
\STATE
\STATE \textbf{Output:} $O$ = \texttt{SelectBestConfigs} ($R$) \COMMENT{Return best optimizations}
\end{algorithmic}
\end{algorithm}

\subsection{Hierarchical Layer and Budget Partition}
\label{sec:ssp}

%We motivate \search's adaptive hierarchical search 
A non-hierarchical search has all cog options in a single-layer search space for an optimizer like BO to search, an approach taken by prior workflow optimizers~\cite{dspy-2-2024,gptswarm}.
With small budgets, a single-layer hierarchy allows BO-like search to spend the budget on dimensions that could potentially generate some improvements.
%While given enough budget, the single-layer space can be extensively searched to find global optimal, with little budget, 
However, a major issue with a single-layer search space is that a search algorithm like BO can be stuck at a local optimum even when budgets increase.
% (unless the budget is close to covering a very large space across dimensions).
To mitigate this issue, our idea is to perform a hierarchical search that works by choosing configurations in the outermost layer first, then under each chosen configuration, choosing the next layer's configurations until the innermost layer. 
With such a hierarchy, a search algorithm could force each layer to sample some values. Given enough budget, each dimension will receive some sampling points, allowing better coverage in the entire search space. However, with high dimensionality (\ie, many types of cogs) and insufficient budget, a hierarchical search may not be able to perform enough local search to find any good optimizations.

To support different user-specified budgets and to get the best of both approaches, we propose an adaptive hierarchical search approach, as shown in Algorithm~\ref{alg:main}.
\search\ starts the search by combining all cogs into one layer ($L=1$, line 9 in Algorithm~\ref{alg:main}) and estimating the expected search budget of this single layer to be the total number of cogs to the power of $\alpha$ (lines 16-19, by default $\alpha = 1.1$). This budget is then passed to the \texttt{LayerSearch} function (Algorithm~\ref{alg:outer}) to perform the actual cog search. When the user-defined budget is no larger than this estimated budget, we expect the single-layer, non-hierarchical search to work better than hierarchical search.
%as the budget for this single layer.

If the user-defined budget is larger, \search\ continues the search with two layers ($L=2$), combining step and weight cogs into the inner layer and architecture cogs as the outer layer (lines 11-14).
\search\ estimates the total search budget for this round as the product of the number of cogs in each of the two layers to the power of $\alpha$ (lines 16-20). It then distributes the estimated search budget between the two layers proportionally to each layer's complexity (lines 22-24) and calls the upper layer's \texttt{LayerSearch} function. Afterward, if there is still budget left, \search\ performs a last round of search using three layers and the remaining budget in a similar way as described above but with three separate layers (architecture as the outermost, step as the middle, and weight cogs as the innermost layer). Two or three layers work better for larger user-defined budgets, as they allow for a larger coverage of the high-dimensional search space.

Finally, \search\ combines all the search results to select the best configurations based on user-defined metrics (line 34).

%\search\ organizes cogs by having architecture cogs in the outer-most search layer, step cogs in the middle layer, and weight cogs in the inner-most layer (line 4 in Algorithm~\ref{alg:main}).
%This is because step cogs' input and output format are dependent on the workflow structure, and the effectiveness of weights (\eg, prompting) are dependent on steps (\eg, LLM model). 

% increases the number of layers until hitting the user-specified total search budget, $TB$

%Thus, the first step of \search\ is to determine the number of layers in its hierarchy and what cogs to include in a layer.
%Intuitively, structure cogs should be placed in the outer-most search layer to be determined first before exploring other cogs. This is because other cogs change node and edge values, and it is easier for 
%However, instead of a fixed number of layers in the hierarchy, we adapt the cog layering according to user-specified total search budgets, $TB$, and the complexity of each layer, using Algorithm~\ref{alg:main}.

% the following \texttt{LayerPartition} method.
%We begin by modeling the relationship between the expected number of evaluations and the number of cogs as well as the number of options in each layer:

%We first consider the identity of each cog in the search space. All structure-cogs will be placed in the outer-most search layer exclusively, which is similar to non-differentiable NAS in traditional ML training. This layer will fix the workflow graph and pass it to the following layer, allowing a stabilized search space for faster convergence.

%Since step-cogs will not create a changing search space, the partition of step-cogs and weight-cogs is conditioned on the search space complexity and the given total budget. Separating step-cogs out can benefit from a more flexible budget allocation strategy and broader exploration for local search at weight-cogs but performs poorly when the given budget is more constrained, in that case, we will optimize them jointly in the same layer.


%\small
%\begin{align*}
%    C &= \{c_{11},c_{12},...\}, \{c_{21},c_{22},...\}, \{c_{31},c_{32},...\} \\
%    NC_i &= \text{total number of cogs in layer i} \\
%    NO_i &= \sum_{j} \{\text{number of possible options in cog } c_{ij}\} \\
%    N_i &= max(NC_i^\alpha,NO_i) \\
%    N_i &= \sum_{j} \{\text{number of possible options in } C_{ij}\} \\
%    N_i &= max(|C_i|^\alpha, N_i) \\
%    B_j &= \prod\limits_{i=1}^{j}N_i, j \in \{1,2,3\}
%\end{align*}

%\normalsize
%where $L$ represents the total number of layers and can be 1, 2, or 3. 
%$C$ represents the entire cog search space, with each row $c_{i*}$ being one of the three types of cogs and lower layers having lower-numbered rows (\eg, $c_{1*}$ being weight cogs). $NC_i$ is the number of cogs in layer $i$, and $NO_i$ is the total number of options across all cogs in layer $i$. $N_i$ is our estimation of the complexity of layer $i$ based on $NC_i$ and $NO_i$ ($\alpha$ is a configurable weight to control the importance between $NC_i$ and $NO_i$; by default $\alpha = 1.1$). 
%$\alpha$ stands for a control parameter, setting the intensity of this scaling behavior w.r.t the number of cogs, we found that $\alpha = 1.2$ is empirically sufficient and efficient for optimizing real workloads. 
%$B_j$ is the expected total number of workflow evaluations for all the lower $j$ layers.
%After calculating $B_1$, $B_2$, and $B_3$, we compare the total budget $TB$ with them.
%When $TB \geq B_3$, we set the total number of layers, $TL$, to 3. When $B_2 \leq TB < B_3$, we set the total number of layers to 2 and merge the step and weight cogs into one layer. When $TB < B_1$, we put all cogs in one layer.
%We only create a separate layer for step-cogs when the given budget $TB$ is greater or equal to the total expected budget for three layers.

%\subsection{Seach Budget Partition}
%\label{sec:sbp}
%After determining cog layers, we distribute the total budget, $TB$, across the layers proportionally to each layer's expected budget $N_i$: , which is the \texttt{BudgetAssign} function.
%We follow a bottom-up partition strategy, where lower layers will try to greedily take the expected budget. This stems from two simple heuristics: (1) feedback to the upper layer is more accurate when the succeeding layer is trained with enough iterations, and (2) the effectiveness of a structure change depends on the setting of individual steps in the workflow (\eg, majority voting is more powerful when each LLM-agent is embedded with diverse few-shot examples or reasoning styles). In cases where the given resource exceeds the total expected budget, 
%We assign $TB$ across layers proportionally to their expected budget $N_i$. 
%The budget assigned at each layer $B_i$ given the total available number of evaluations $TB$ is obtained as follows:

%\small
%\begin{align}
%B_i &=  \lfloor N_i \times \sqrt[L]{\frac{TB}{B^*}}\rfloor
%    B_L &= \begin{cases}
%        min(N_L, TB) & TB < B^* \\
%        \lfloor N_L \times \sqrt[L]{\frac{TB}{B^*}}\rfloor & TB \geq B^*
%    \end{cases}
%    \\
%    B_i &= \begin{cases}
%        min(N_i, \lfloor\frac{TB}{\prod_{j=i+1}^L B_j}\rfloor) & TB < B^* \\
%        \lfloor N_i \times \sqrt[L]{\frac{TB}{B^*}}\rfloor & TB \geq B^*
%    \end{cases}
%\end{align}

%\normalsize


\subsection{Recursive Layer-Wise Search Algorithm}
%The calculation above pre-assigns cogs to layers and search budgets to each layer. 
We now introduce how \search\ performs the actual search in a recursive manner until the inner-most layer is searched, as presented in Algorithm~\ref{alg:outer} \texttt{LayerSearch}. 
Our overall goal is to ensure strong cog option coverage within each layer while quickly directing budgets to more promising cog options based on evaluation results.
%So far, we have determined the optimization layer structure and the maximum allowed search iteration each layer will get. Next, we introduce how the budget is consumed in each layer. The inner-most layer, where weight-cogs, and potentially step-cogs, reside, follows the conventional Bayesian optimization process, exhausting all budgets unless an early stop signal is sent. This signal will be triggered when the current optimizer witnesses $p$ consecutive iterations without any improvements above the threshold. All optimization layers use early stopping to avoid budget waste.
%Algorithm~\ref{alg:inner} describes the search happening at the inner-most (bottom) layer, and 
Specifically, every layer's search is under a chosen set of cog configurations from its upper layers ($C_{chosen}$) and is given a budget $b$. 
In the inner-most layer (lines 7-20), \search\ samples $b$ configurations and evaluates the workflow for each of them together with the configurations from all upper layers ($C_{chosen}$). The evaluation results are added to the feedback set $F$ as the return of this layer.

\begin{algorithm}[h]
  %\algsetup{linenosize=\tiny}
  \small
    \caption{\texttt{LayerSearch} Function}
    \label{alg:outer}
\begin{algorithmic}[1]
%\STATE \textbf{Global Config:} Reduct factor $\eta > 1$, explore width $W$
\STATE \textbf{Global Value:} $R$ \COMMENT{Global result set}
%\STATE \textbf{Global Value:} $F$ \COMMENT{Global observation set}
\STATE \textbf{Input:} $C_{chosen}$: configs chosen in upper layers
\STATE \textbf{Input:} $B$: Array storing assigned budgets to different layers
\STATE \textbf{Input:} $curr\_layer$: this layer's level
\STATE \textbf{Input:} $curr\_b$: this layer's assigned budget
%\STATE
%\FUNCTION{LayerSearch\hspace{0.4em}($C_{chosen}$, $B$, $curr\_layer$, $curr\_b$)}

    \STATE
    \STATE \COMMENT{Search for inner-most layer}
    \IF{curr\_layer == 1}
        \STATE $F = \emptyset$ \COMMENT{Init this layer's feedback set to empty}
        %\STATE $F^{\prime} = match(C_{chosen}, F)$ \COMMENT{Local feedback set}
        \FOR{$k = 0, \dots, curr\_b$}
            \STATE $\lambda$ = \texttt{TPESample} (1) \COMMENT{Sample one configuration using TPE}
            \STATE $f = $ \texttt{EvaluateWorkflow} ($C_{chosen} \cup \lambda$)
            \STATE $R = R \cup \{C_{chosen} \cup \lambda\}$ \COMMENT{Add configuration to global $R$}
            \IF{\texttt{EarlyStop} (f)}
            \STATE break
            \ENDIF
            \STATE $F = F \cup \{f\}$ \COMMENT{Add evaluate result to feedback $F$}
        \ENDFOR
        %\STATE $F = F \cup F^{\prime}$
        \STATE \textbf{Return} $F$
    \ENDIF
    \STATE
    \STATE \COMMENT{Search for non-inner-most layer}
    %\STATE $K = \lfloor \frac{b}{W} \rfloor$, 
    \STATE $b\_used = 0$, $TF = \emptyset$ \COMMENT{Init this layer's used budget and feedback set}
    \STATE $R = \lceil\frac{curr\_b}{\eta}\rceil$, $S = \lfloor\frac{curr\_b}{R}\rfloor$ \COMMENT{Set $R$ and $S$ based on $curr\_b$}
    \STATE
    \WHILE{$b\_{used}$ $\leq$ $curr\_b$}
        \STATE \COMMENT{Sample $W$ configs at a time until running out of $curr\_b$}
        \STATE $n = (curr\_b - b_{used})$ > $W$ ? $W$ : $(curr\_b - b_{used})$
        %\IF{$b - b_{used} < W$}
        %    \STATE $n = b_l - b_{used}$
        %\ELSE
         %   \STATE $n=W$
        %\ENDIF
        \STATE $b\_used$ += $n$
        %\STATE $n = \text{min}(W,\ b_l - kW)$ \COMMENT{Propose $W$ configs and meet $b_l$ constraint}
        \STATE $\Theta = $ \texttt{TPESample} ($n$) \COMMENT{Sample a chunk of $n$ configs in the layer} 
        %\STATE $F^{\prime} = match(C_{chosen}, F)$ \COMMENT{Per-chunk feedback set}
        \STATE $F = \emptyset$ \COMMENT{Init this layer's feedback set to empty}
        \STATE
        \FOR{$s = 0, 1, \dots, S$}
            \STATE $r_s = R\cdot \eta^s$
            \FOR{$\theta \in \Theta$}
                %\IF{$curr\_layer < max\_layer$}
                    \STATE $f =$ \texttt{LayerSearch} ($C_{chosen} \cup \{\theta\}$, $B$, curr\_layer$-1$, $r_s$)
                %\ELSE
                %    \STATE $f =$ InnerOpt($\gamma \cup \{\theta\}$, $r_s$)
                %\STATE $f$ = $opt\_layers[current\_layer+1](\gamma \cup \{\theta\}, r_s)$ \{Optimize the current config at the next layer with $r_s$ budget \}
                %\ENDIF
                \STATE $F = F \cup f$ \COMMENT{Add evaluate result to feedback}
                \IF{\texttt{EarlyStop} ($f$)}
                    \STATE $\Theta = \Theta - \{\theta\}$ \COMMENT{Skip converged configs}
                \ENDIF
            \ENDFOR
            \STATE $\Theta$ = Select top $\lfloor \frac{|\Theta|}{\eta}\rfloor$ configs from $F$ based on user-specified metrics
        \ENDFOR
        \STATE
        \IF{\texttt{EarlyStop} ($F$)}
            \STATE break \COMMENT{Skip remaining search if results converged}
        \ENDIF
        \STATE $TF = TF \cup F$
    \ENDWHILE
    %\STATE $F = F \cup TF$
        \STATE \textbf{Return} $TF$

%\ENDFUNCTION

%\STATE \textbf{Output:} Best metrics in all trials
\end{algorithmic}
\end{algorithm}

% consumption\_nextlayer\_bucket = WSR

% for s in 0, 1,...S do
%     w = W*\eta^{s}
%     r = R*\eta^{-s}

% total budget at next layer = b_l / W * WSR = b_l * SR

% b_l * SR <= b_l * B_l+1

% S = B_{l+1} / R



For a non-inner-most layer, \search\ samples a chunk ($W$) of points at a time using the TPE BO algorithm~\cite{bergstra2011tpe} until all this layer's pre-assigned budget is exhausted (lines 27-30). Within a chunk, \search\ uses a successive-halving-like approach to iteratively direct the search budget to more promising configurations within the chunk (the dynamically changing set, $\Theta$). In each iteration, \search\ calls the next-level search function for each sampled configuration in $\Theta$ with a budget of $r_s$ and adds the evaluation observations from lower layers to the feedback set $F$ for later TPE sampling to use (lines 35-37).
In the first iteration ($s=0$), $r_s$ is set to $R\cdot \eta^0=R$ (line 34). After the inner layers use this budget to search, \search\ filters out configurations with lower performance and only keeps the top $\lfloor \frac{|\Theta|}{\eta}\rfloor$ configurations as the new $\Theta$ to explore in the next iteration (line 42). In each next iteration, \search\ increases $r_s$ by $\eta$ times (line 34), essentially giving more search budget to the better configurations from the previous iteration.

The successive halving method effectively distributes the search budget to more promising configurations, while the chunk-based sampling approach allows for evaluation feedback to accumulate quickly so that later rounds of TPE can get more feedback (compared to no chunking and sampling all $b$ configurations at the same time). To further improve the search efficiency, we adopt an {\em early stop} approach where we stop a chunk or a layer's search when we find its latest few searches do not improve workflow results by more than a threshold, indicating convergence (lines 14,38,45).

%algorithm takes as input other cog settings from previous layers and the assigned budget at the current layer. It tiles the search loop into fixed-size blocks (line 4), each runs the SuccessiveHalving (SH) subroutine in the inner loop (line 7-15). In each SH iteration, only top-$1/\eta$-quantile configurations in $\Theta$ will continue in the next round with $\eta$ times larger budget consumption. As a result, exponentially more trials will be performed by more promising configurations. 

%On average, \textit{Outer-layer search} will create $K$ brackets, each granting approximately $WRS$ budget to the next layer. $R$ represents the smallest amount of resource allocated to any configurations in $\Theta$. 

% layer - 1: budget = 4
% K * W <= b\_current layer
% layer -1: itear 0: propose 2

%     SH:
%     2 config -> R
%     1 config -> 2R

%     iteration 1: propose 2 = W
%     SH:
%     2 config -> R
%     1 config -> 2R

% W configs; each has R resource

% W / eta configs; each has R * eta resource

% R -> least resource one config can get = B2 - smth
% R + R*eta + ... + R*eta\^s -> most promising = B2 + smth


% $L2$ is the middle layer where structure-cogs and step-cogs may be placed exclusively. We employ hyperband for its robustness in exploration and exploitation trace-off. If this layer exists, it will instruct $L1$ the number of search iterations to run in each invocation. Specifically, in each iteration at line 4, \sysname will propose $n$ configurations and run SuccessiveHalving (SH) subroutine (line 8-15). SH will optimize each proposal and use the search results from $L1$ to rank their performance. Each time only the top-performing $n \cdot \eta^{-i}$ can continue in the next round with a larger budget. With this strategy, exponentially more search budgets are allocated to more promising configs at $L2$.

% \input{algo-l2-search}

% $L3$ is the outer-most layer for structure-cogs only when $L2$ is created. For this layer, we employ plain SH without hyperband because of its predictable convergence behavior. This is mainly due to two factors: (1) structure change to the workflow is more significant thus different configurations are more likely to deviate after training with the following layers. (2) with the search space partition strategy in Sec ~\ref{sec:ssp}, we can assume the available budget at each layer is substantial when $L3$ exists. Given this prior knowledge, we can avoid grid searching control parameter $n$ as in the hyperband but adopt a more aggressive allocation scheme to bias towards better proposals and moderate search wastes.



%\subsubsection{Runtime Budget Adaptation}
%Using static estimation of the expected budget for each layer is not enough, we also adjust the assignment during the optimization based on real convergence behavior. Specifically, for layer $i$, we record the number of configurations evaluated in each optimize routine. We set the convergence indicator $C_{ij}$ of $j^{th}$ routine with this number if the search early exits before reaching the budget limit, otherwise 2\x of its assigned resource. Then we update $E_i$ with $\frac{\sum_{j}^M C_{ij}}{M}$. \sysname\ will update the budget partition according to Sec~\ref{sec:sbp} for any newly spawned optimizer routines. Besides controlling the proportion of budgets across layers, a smaller/larger $B_{l+1}$ will also guide the SH in Alg~\ref{alg:outer} to shrink/extend the budget $R$ for differentiating config performance.


\section{\sysname\ Design}
\label{sec:cognify}

We build \sysname, an extensible gen-AI workflow autotuning platform based on the \search\ algorithm. The input to \sysname\ is the user-written gen-AI workflow (we currently support LangChain \cite{langchain-repo}, DSPy \cite{khattab2024dspy}, and our own programming model), a user-provided workflow training set, a user-chosen evaluator, and a user-specified total search budget. \sysname\ currently supports three autotuning objectives: generation quality (defined by the user evaluator), total workflow execution cost, and total workflow execution latency. Users can choose one or more of these objectives and set thresholds for them or the remaining metrics (\eg, optimize cost and latency while ensuring quality to be at least 5\% better than the original workflow). 
\sysname\ uses the \search\ algorithm to search through the cog space.
When given multiple optimization objectives, \sysname\ maintains a sorted optimization queue for each objective and performs its pruning and final result selection from all the sorted queues (possibly with different weighted numbers).
To speed up the search process, we employ parallel execution, where a user-configurable number of optimizers, each taking a chunk of search load, work together in parallel. %Below, we introduce each type of cogs in more details.
\sysname\ returns multiple autotuned workflow versions based on user-specified objectives.
\sysname\ also allows users to continue the auto-tuning from a previous optimization result with more budgets so that users can gradually increase their search budget without prior knowledge of what budget is sufficient.
Appendix~\ref{sec:apdx-example} shows an example of \sysname-tuned workflow outputs. 
\sysname\ currently supports six cogs in three categories, as discussed below. 

%In \sysname, we call every workflow optimization technique a {\em cog}, including structure-changing cogs like task decomposition, step-changing cogs like model selection, and weight-changing cogs like adding few-shot examples to prompts. 
%\sysname\ places structure-changing cogs in the outermost layer, step cogs in the middle layer, and weight cogs in the innermost layer, because \fixme{TODO}.


\subsection{Architecture Cogs}
\label{sec:structure-cog}
%Changing the structure of a workflow can potentially improve its generation quality (\eg, by using multiple steps to attempt at a task in parallel or in chain) or reduce its execution cost and latency (\eg, by merging or removing steps).
\sysname\ currently supports two architecture cogs: task decomposition and task ensemble.
Task decomposition~\cite{khot2023decomposed} breaks a workflow step into multiple sub-steps and can potentially improve generation quality and lower execution costs, as decomposed tasks are easier to solve even with a small (cheaper) model.
There are numerous ways to perform task decomposition in a workflow. 
%, as all LM steps can potentially be decomposed and into different numbers of sub-steps in different ways. Throwing all options to the Bayesian Optimizer would drastically increase the search space for \sysname. 
To reduce the search space, we propose several ways to narrow down task decomposition options. Even though we present these techniques in the setting of task decomposition, they generalize to many other structure-changing tuning techniques.

%First, we narrow down a selected set of steps in a workflow to decompose. 
Intuitively, complex tasks are the hardest to solve and worth decomposition the most. We use a combination of LLM-as-a-judge \cite{vicuna_share_gpt} and static graph (program) analysis to identify complex steps. We instruct an LLM to give a rating of the complexity of each step in a workflow. We then analyze the relationship between steps in a workflow and find the number of out-edges of each step (\ie, the number of subsequent steps getting this step's output). More out-edges imply that a step is likely performing more tasks at the same time and is thus more complex. We multiply the LLM-produced rating and the number of out-edges for each step and pick the modules with scores above a learnable threshold as the target for task decomposition. We then instruct an LLM to propose a decomposition (\ie, generate the submodules and their prompts) for each candidate step. %We provide the LLM with few-shot examples for what proposed modules for a separate task could look like. We also add a refinement step that validates whether the proposition decomposition maintains the semantics of the original trajectory. Once candidate decompositions are generated, those are used for the entirety of the optimization.

{
\begin{figure*}[t!]
\begin{center}
\centerline{\includegraphics[width=0.95\textwidth]{Figures/big_grid.pdf}}
\vspace{-0.1in}
\mycap{Generation Quality vs Cost/Latency.}{Dashed lines show the Pareto frontier (upper left is better). Cost shown as model API dollar cost for every 1000 requests. Cognify selects models
from GPT-4o-mini and Llama-8B. DSPy and Trace do not support model selection and are given GPT-4o-mini for all steps. Trace results for Text-2-SQL and FinRobot have 0 quality and are not included.} 
\Description{Eight graphs with different shapes representing baselines compared to points on a Pareto frontier.}
\label{fig-biggrid}
\end{center}
\end{figure*}
}


The second structure-changing cog that \sysname\ supports is task ensembling. This cog spawns multiple parallel steps (or samplers) for a single step in the original workflow, as well as an aggregator step that returns the best output (or combination of outputs). By introducing parallel steps, \sysname\ can optimize these independently with step and weight cogs. This provides the aggregator with a diverse set of outputs to choose from. 
%The aggregator is prompted with the role of the samplers, as well as the inputs to each. It also receives a criteria by which it should make a decision. We choose to prompt it with a qualitative description of how it should resolve discrepancies between outputs. 


\subsection{Step Cogs}
We currently support two step-changing cogs: model selection for language-model (LM) steps and code rewriting for code steps.

For model selection, to reduce its search space, we identify ``important'' LM steps---steps that most critically impact the final workflow output to reduce the set \search\ performs TPE sampling on. Our approach is to test each step in isolation by freezing other steps with the cheapest model and trying different models on the step under testing. 
We then calculate the difference between the model yielding the best and worst workflow results as the importance of the step under testing. %For each model, we get the workflow output quality score using sampled user-supplied inputs and user-specific evaluator. We then calculate the difference between the highest and lowest scores as this module's importance. 
After testing all the steps, we choose the steps with the highest K\% importance as the ones for TPE to sample from.
%, where K is determined based on user-chosen stop criteria. We then initialize the Bayesian optimization to start with the state where important modules use the largest model and all other modules use the cheapest model. We set the TPE optimization bandwidth of each module to be the inverse of importance, \ie, the more important a module is the more TPE spends on optimizing.

The second step cog \sysname\ supports is code rewriting, where it automatically changes code steps to use better implementation. To rewrite a code step, \sysname\ finds the $k$ worst- and best-performing training data points and feeds their corresponding input and output pairs of this code step to an LLM. We let the LLM propose $n$ new candidate code pieces for the step at a time.
%in parallel to generate a set of $n$ candidates.
In subsequent trials, the optimizer dynamically updates the candidate set using feedback from the evaluator.


\subsection{Weight Cogs}
\sysname\ currently supports two weight-changing cogs: reasoning and few-shot examples.
First, \sysname\ supports adding reasoning capability to the user's original prompt, with two options: zero-shot Chain-of-Thought \cite{wei2022chain} (\ie, ``think step-by-step...'') and dynamic planning \cite{huang2022language} (\ie, ``break down the task into simpler sub-tasks...''). These prompts are appended to the user's prompt. In the case where the original module relies on structured output, we support a reason-then-format option that injects reasoning text into the prompt while maintaining the original output schema.

Second, \sysname\ supports dynamically adding few-shot examples to a prompt. At the end of each iteration, we choose the top-$k$-performing examples for an LM step in the training data and use their corresponding input-output pairs of the LM step as the few-shot examples to be appended to the original prompt to the LM step for later iterations' TPE sampling. As such, the set of few-shot examples is constantly evolving during the optimization process based on the workflow's evaluation results. 
%Few-shot examples are available to all modules, even intermediate steps in the workflow. We use the full trajectory of each request to generate examples for the intermediate steps. Furthermore, we automatically filter out examples that do not meet a user-specified threshold. 




\vspace{-2pt}
\section{Results and Lessons Learned}
\label{result}
\vspace{-2pt}
% Space: 1/2 of Page 2 + 1/2 of Page 3
\subsection{Insights for Unlocking Behavior}
\vspace{-2pt}
%\yixue{each sub-section can be per RQ}
\begin{figure}[b]
\centering
\includegraphics[width=\linewidth]{figures/distribution_highlight.pdf}
\caption{Distributions of Unlock Number, Unlock Duration, and Duration per Unlock at the individual level with mean values (top) and maximum values (bottom). Excluded data points described in Section \ref{data} are shown in red.}
\label{fig1}
\end{figure}

\begin{figure}[t]
\centering
\includegraphics[width=\linewidth]{figures/distribution_by_gender.pdf}
\caption{Gender differences highlighted based on the same data shown in Fig.~\ref{fig1}}
\label{fig2}
\end{figure}

%\yixue{For Fig 1. 1. In the caption, I was referring to the exclusion criteria that's written in the regression model section. Doesn't Pearson use that exclusion criteria as well? If so, it doesn't make sense to have it in the regression section before. Maybe move it here when showing Fig.1. Otherwise, it's mysterious why the exclusions are in red and it wasn't explained here. Reader will be puzzled when seeing Fig 1. After you decide what to do, make sure the caption is correct as well. 2. keep the feature names consistent in the fig. Unit: None change to Unit: Times 3. keep Fig 2 consistent with Fig 1 after you change Fig 1}

To answer \textbf{RQ$_1$}, we analyze the data distribution as discussed in Section~\ref{sec:data:distribution}. Fig.~\ref{fig1} highlights the mean and maximum statistics of the results. For instance, Fig.\ref{fig1}-a (Mean Unlock Number) shows the distribution of the mean value of each student's Unlock Number during the study period.
%The results are shown in Fig \ref{fig1}. 

\textbf{Overall Pattern}. On average, most students unlock their phones 50–100 times daily (Fig.\ref{fig1}-a) totaling 2–4 hours (Fig.\ref{fig1}-b), and each unlock session averages 2-4 minutes (Fig.\ref{fig1}-c). On high-usage days, most students unlock their phones 100-300 times (Fig.\ref{fig1}-d)
% \added{100-300} \deleted{150–370}, 
totaling 6–10 hours. Interestingly, for most students, the duration per unlock typically falls under 4 minutes (Fig.\ref{fig1}-c), and the majority of students (96\%) use their phones under 1 hour per unlock even on the highest-usage days (Fig.\ref{fig1}-f). This suggests that students primarily engage with their phones for brief, fleeting interactions rather than extended, focused tasks, likely to satisfy momentary curiosity or to simply mindlessly unlock rather than sustained activities.

\textbf{Gender Difference}. As shown in Fig. \ref{fig2}, notable gender differences in phone usage emerge. Male students display a broader range of unlock frequencies, indicating more diverse usage habits. In contrast, female students exhibit more consistent patterns but spend significantly more time per unlock. This behavior suggests that female students may engage more deeply with their phones during each interaction, potentially reflecting a more intentional and focused usage style.



\vspace{-2pt}
\subsection{Unlocking Behavior and Mental Well-being Correlation}
\vspace{-2pt}


% \begin{table}[t]
% \centering
% \caption{Pearson Correlation with PHQ4 Score}
% \begin{tabular}{|l|l|l|l|l}
% \cline{1-4}
%                     & Overall    & Male       & Female     &  \\ \cline{1-4}
% Unlock Duration     & 0.0321 & 0.2130 & -0.0388 &  \\ \cline{1-4}
% Unlock Number       & -0.0107 & 0.0964 & -0.0743 &  \\ \cline{1-4}
% Duration per Unlock & 0.0380 & 0.0778 & 0.0283 &  \\ \cline{1-4}
% \end{tabular}
% \label{corre}
% \end{table}

\begin{table}[b]
\centering
\caption{Pearson Correlation with PHQ4 Score across Genders}
\begin{tabular}{|l|c|c|c|}
\hline
\textbf{}                    & \textbf{Overall}                                                                                  & \textbf{Male}                                                                                    & \textbf{Female}                                                                                   \\ \hline
\textbf{Unlock Duration}     & {\color[HTML]{3166FF} \begin{tabular}[c]{@{}c@{}}0.0321 \\      (p\textless{}0.01)\end{tabular}}  & {\color[HTML]{3166FF} \begin{tabular}[c]{@{}c@{}}0.2131 \\      (p\textless{}0.01)\end{tabular}} & {\color[HTML]{FE0000} \begin{tabular}[c]{@{}c@{}}-0.0388 \\      (p\textless{}0.01)\end{tabular}} \\ \hline
\textbf{Unlock Number}       & {\color[HTML]{FE0000} \begin{tabular}[c]{@{}c@{}}-0.0107 \\      (p\textless{}0.01)\end{tabular}} & {\color[HTML]{3166FF} \begin{tabular}[c]{@{}c@{}}0.0964 \\      (p\textless{}0.01)\end{tabular}} & {\color[HTML]{FE0000} \begin{tabular}[c]{@{}c@{}}-0.0743 \\      (p\textless{}0.01)\end{tabular}} \\ \hline
\textbf{Duration per Unlock} & {\color[HTML]{3166FF} \begin{tabular}[c]{@{}c@{}}0.0380 \\      (p\textless{}0.01)\end{tabular}}  & {\color[HTML]{3166FF} \begin{tabular}[c]{@{}c@{}}0.0778 \\      (p\textless{}0.01)\end{tabular}} & {\color[HTML]{3166FF} \begin{tabular}[c]{@{}c@{}}0.0284 \\      (p\textless{}0.01)\end{tabular}}  \\ \hline
\end{tabular}
\label{corre}
\begin{tablenotes}
\item Positive coefficients are in blue, while negative coefficients are in red. 
 %Information about p-values is provided in parentheses.
\end{tablenotes}
\end{table}






% Please add the following required packages to your document preamble:
% \usepackage{multirow}
% \usepackage[table,xcdraw]{xcolor}
% Beamer presentation requires \usepackage{colortbl} instead of \usepackage[table,xcdraw]{xcolor}
\begin{table}[t]
\centering
\caption{Pearson Correlation with PHQ4 Score at Different Locations}
\begin{tabular}{|l|l|l|l|l|}
\hline
                                     &                          & \textbf{Overall}                                                                               & \textbf{Male}                                                                                  & \textbf{Female}                                                                               \\ \hline
                                     & \textbf{Duration/Unlock} & {\color[HTML]{3166FF} \begin{tabular}[c]{@{}l@{}}0.0115\\ (p\textgreater{}0.1)\end{tabular}}   & {\color[HTML]{3166FF} \begin{tabular}[c]{@{}l@{}}0.0791\\ (p\textless{}0.01)\end{tabular}}     & {\color[HTML]{3166FF} \begin{tabular}[c]{@{}l@{}}0.0043\\ (p\textgreater{}0.1)\end{tabular}}  \\ \cline{2-5} 
                                     & \textbf{Duration}        & {\color[HTML]{FE0000} \begin{tabular}[c]{@{}l@{}}-0.0125\\ (p\textless{}0.05)\end{tabular}}    & {\color[HTML]{3166FF} \begin{tabular}[c]{@{}l@{}}0.0296\\ (p\textless{}0.01)\end{tabular}}     & {\color[HTML]{FE0000} \begin{tabular}[c]{@{}l@{}}-0.0265\\ (p\textless{}0.01)\end{tabular}}   \\ \cline{2-5} 
\multirow{-3}{*}{\textbf{Food}}      & \textbf{Number}          & {\color[HTML]{FE0000} \begin{tabular}[c]{@{}l@{}}-0.0197\\ (p\textless{}0.01)\end{tabular}}    & {\color[HTML]{3166FF} \begin{tabular}[c]{@{}l@{}}0.0214\\ (p\textless{}0.05)\end{tabular}}     & {\color[HTML]{FE0000} \begin{tabular}[c]{@{}l@{}}-0.0408\\ (p\textless{}0.01)\end{tabular}}   \\ \hline
                                     & \textbf{Duration/Unlock} & {\color[HTML]{3166FF} \begin{tabular}[c]{@{}l@{}}0.038\\ (p\textless{}0.01)\end{tabular}}      & {\color[HTML]{3166FF} \begin{tabular}[c]{@{}l@{}}0.0479\\ (p\textless{}0.01)\end{tabular}}     & {\color[HTML]{3166FF} \begin{tabular}[c]{@{}l@{}}0.0343\\ (p\textless{}0.01)\end{tabular}}    \\ \cline{2-5} 
                                     & \textbf{Duration}        & {\color[HTML]{3166FF} \begin{tabular}[c]{@{}l@{}}0.0014\\ (p\textgreater{}0.1)\end{tabular}}   & {\color[HTML]{FE0000} \begin{tabular}[c]{@{}l@{}}-0.0055\\ (p\textgreater{}0.1)\end{tabular}}  & {\color[HTML]{3166FF} \begin{tabular}[c]{@{}l@{}}0.009\\ (p\textgreater{}0.1)\end{tabular}}   \\ \cline{2-5} 
\multirow{-3}{*}{\textbf{Study}}     & \textbf{Number}          & {\color[HTML]{FE0000} \begin{tabular}[c]{@{}l@{}}-0.0151\\ (p\textless{}0.01)\end{tabular}}    & {\color[HTML]{FE0000} \begin{tabular}[c]{@{}l@{}}-0.0218\\ (p\textless{}0.05)\end{tabular}}    & {\color[HTML]{FE0000} \begin{tabular}[c]{@{}l@{}}-0.0077\\ (p\textgreater{}0.1)\end{tabular}} \\ \hline
                                     & \textbf{Duration/Unlock} & {\color[HTML]{FE0000} \begin{tabular}[c]{@{}l@{}}-0.0005\\ (p\textgreater{}0.1)\end{tabular}}  & {\color[HTML]{3166FF} \begin{tabular}[c]{@{}l@{}}0.0371\\ (p\textless{}0.05)\end{tabular}}     & {\color[HTML]{FE0000} \begin{tabular}[c]{@{}l@{}}-0.0082\\ (p\textgreater{}0.1)\end{tabular}} \\ \cline{2-5} 
                                     & \textbf{Duration}        & {\color[HTML]{FE0000} \begin{tabular}[c]{@{}l@{}}-0.01\\ (p\textless{}0.1)\end{tabular}}       & {\color[HTML]{FE0000} \begin{tabular}[c]{@{}l@{}}-0.0046\\ (p\textgreater{}0.1)\end{tabular}}  & {\color[HTML]{FE0000} \begin{tabular}[c]{@{}l@{}}-0.0073\\ (p\textgreater{}0.1)\end{tabular}} \\ \cline{2-5} 
\multirow{-3}{*}{\textbf{Social}}    & \textbf{Number}          & {\color[HTML]{FE0000} \begin{tabular}[c]{@{}l@{}}-0.0157\\ (p\textless{}0.01)\end{tabular}}    & {\color[HTML]{FE0000} \begin{tabular}[c]{@{}l@{}}-0.0057\\ (p\textgreater{}0.1)\end{tabular}}  & {\color[HTML]{FE0000} \begin{tabular}[c]{@{}l@{}}-0.0181\\ (p\textless{}0.01)\end{tabular}}   \\ \hline
                                     & \textbf{Duration/Unlock} & {\color[HTML]{FE0000} \begin{tabular}[c]{@{}l@{}}0.02478\\ (p\textless{}0.01)\end{tabular}}    & {\color[HTML]{FE0000} \begin{tabular}[c]{@{}l@{}}-0.00979\\ (p\textgreater{}0.1)\end{tabular}} & {\color[HTML]{3166FF} \begin{tabular}[c]{@{}l@{}}0.04242\\ (p\textless{}0.01)\end{tabular}}   \\ \cline{2-5} 
                                     & \textbf{Duration}        & {\color[HTML]{FE0000} \begin{tabular}[c]{@{}l@{}}-0.00215\\ (p\textgreater{}0.1)\end{tabular}} & {\color[HTML]{FE0000} \begin{tabular}[c]{@{}l@{}}-0.01371\\ (p\textgreater{}0.1)\end{tabular}} & {\color[HTML]{3166FF} \begin{tabular}[c]{@{}l@{}}0.00987\\ (p\textgreater{}0.1)\end{tabular}} \\ \cline{2-5} 
\multirow{-3}{*}{\textbf{Dormitory}} & \textbf{Number}          & {\color[HTML]{FE0000} \begin{tabular}[c]{@{}l@{}}-0.01505\\ (p\textless{}0.01)\end{tabular}}   & {\color[HTML]{FE0000} \begin{tabular}[c]{@{}l@{}}-0.00768\\ (p\textgreater{}0.1)\end{tabular}} & {\color[HTML]{FE0000} \begin{tabular}[c]{@{}l@{}}-0.01684\\ (p\textless{}0.01)\end{tabular}}  \\ \hline
                                     & \textbf{Duration/Unlock} & {\color[HTML]{3166FF} \begin{tabular}[c]{@{}l@{}}0.0181\\ (p\textless{}0.01)\end{tabular}}     & {\color[HTML]{3166FF} \begin{tabular}[c]{@{}l@{}}0.0562\\ (p\textless{}0.01)\end{tabular}}     & {\color[HTML]{3166FF} \begin{tabular}[c]{@{}l@{}}0.0145\\ (p\textless{}0.05)\end{tabular}}    \\ \cline{2-5} 
                                     & \textbf{Duration}        & {\color[HTML]{3166FF} \begin{tabular}[c]{@{}l@{}}0.0188\\ (p\textless{}0.01)\end{tabular}}     & {\color[HTML]{3166FF} \begin{tabular}[c]{@{}l@{}}0.1474\\ (p\textless{}0.01)\end{tabular}}     & {\color[HTML]{FE0000} \begin{tabular}[c]{@{}l@{}}-0.0334\\ (p\textless{}0.01)\end{tabular}}   \\ \cline{2-5} 
\multirow{-3}{*}{\textbf{Home}}      & \textbf{Number}          & {\color[HTML]{3166FF} \begin{tabular}[c]{@{}l@{}}0.0033\\ (p\textgreater{}0.1)\end{tabular}}   & {\color[HTML]{3166FF} \begin{tabular}[c]{@{}l@{}}0.092\\ (p\textless{}0.01)\end{tabular}}      & {\color[HTML]{FE0000} \begin{tabular}[c]{@{}l@{}}-0.047\\ (p\textless{}0.01)\end{tabular}}    \\ \hline
\end{tabular}
\label{corre_loc}
\begin{tablenotes}
\item Positive coefficients are in blue, while negative coefficients are in red. 
\end{tablenotes}
\vspace{-2pt}
\end{table}


\textbf{Overall Correlations}. To answer \textbf{RQ$_2$}, we conduct correlation analysis as described in Section~\ref{sec:data:correlation} and the results are shown in Table~\ref{corre}.
Notably, all p-values for the correlation coefficients are well below 0.01, indicating strong associations between phone usage patterns and mental health. This finding highlights the potential of our novel unlocking features as effective predictors of mental health outcomes. Column 1 reveals an intriguing pattern: longer phone usage durations correlate with poorer mental health outcomes, whereas higher unlock frequencies are associated with better mental health. These findings underscore the importance of considering both frequency and duration when assessing the impact of smartphone use on mental health, and call for further studies to identify the root causes of these interesting phenomena.

\textbf{Gender Difference}. 
Table \ref{corre} reveals significant gender differences in phone usage. For males, both unlock duration and frequency are positively correlated with PHQ4 Scores, indicating a link to poorer mental health. In contrast, females exhibit the opposite trend, suggesting higher phone usage may be associated with better well-being or act as a coping mechanism. Additionally, the magnitude of correlation coefficients is notably larger for male students, suggesting they are more sensitive to the mental health impacts of phone usage. These findings imply that male students may engage with their phones in less constructive ways than their female peers, with higher interaction levels correlating more strongly with adverse mental health outcomes, highlighting the need for personalized predictive models in different user groups.

\textbf{Location Difference}. Table \ref{corre_loc} shows the correlations between unlocking behaviors and mental well-being at different locations, revealing fine-grained insights to guide further studies. For example, unlocking frequency and duration are both negatively correlated with PHQ4 Score (better mental health) at food and social places, suggesting beneficial effects in social contexts. In contrast, these behaviors correlate with poorer mental health at home, possibly reflecting less meaningful usage when in private. These findings emphasize the importance of context when examining phone usage and mental health.

\subsection{Insights from Multinomial Logistic Regression}
\vspace{-2pt}


\begin{table}[t]
\centering
\caption{Coefficients Results of Multinomial Logistic Regression}
\label{logit}
\begin{tabular}{|l|l|l|l|l|}
\hline
                               & \textbf{Features} & \textbf{Mild}                                                                                          & \textbf{Moderate}                                                                                       & \textbf{Severe}                                                                                        \\ \hline
                               & \textbf{Duration} & {\color[HTML]{3166FF} \begin{tabular}[c]{@{}l@{}}0.1169\\  (p\textless{}0.01)\end{tabular}}    & {\color[HTML]{FE0000} \begin{tabular}[c]{@{}l@{}}-0.0339\\  (p\textless{}0.01)\end{tabular}}    & {\color[HTML]{3166FF} \begin{tabular}[c]{@{}l@{}}0.1575\\  (p\textless{}0.01)\end{tabular}}    \\ \cline{2-5} 
\multirow{-2}{*}{\textbf{Overall}}      & \textbf{Number}   & {\color[HTML]{3166FF} \begin{tabular}[c]{@{}l@{}}0.0903\\  (p\textless{}0.01)\end{tabular}}    & {\color[HTML]{FE0000} \begin{tabular}[c]{@{}l@{}}-0.0378\\  (p\textless{}0.01)\end{tabular}}    & {\color[HTML]{FE0000} \begin{tabular}[c]{@{}l@{}}-0.1270\\  (p\textless{}0.01)\end{tabular}}   \\ \hline
                               & \textbf{Duration} & {\color[HTML]{3166FF} \begin{tabular}[c]{@{}l@{}}0.2398\\  (p\textless{}0.01)\end{tabular}}    & {\color[HTML]{3166FF} \begin{tabular}[c]{@{}l@{}}0.0298\\  (p\textgreater{}0.1)\end{tabular}}   & {\color[HTML]{3166FF} \begin{tabular}[c]{@{}l@{}}0.5741\\  (p\textless{}0.01)\end{tabular}}    \\ \cline{2-5} 
\multirow{-2}{*}{\textbf{Male}}         & \textbf{Number}   & {\color[HTML]{3166FF} \begin{tabular}[c]{@{}l@{}}0.2215\\  (p\textless{}0.01)\end{tabular}}    & {\color[HTML]{FE0000} \begin{tabular}[c]{@{}l@{}}-0.1896\\  (p\textless{}0.01)\end{tabular}}    & {\color[HTML]{FE0000} \begin{tabular}[c]{@{}l@{}}-0.1731\\  (p\textless{}0.01)\end{tabular}}   \\ \hline
                               & \textbf{Duration} & {\color[HTML]{3166FF} \begin{tabular}[c]{@{}l@{}}0.1000\\  (p\textless{}0.01)\end{tabular}}    & {\color[HTML]{FE0000} \begin{tabular}[c]{@{}l@{}}-0.0386\\  (p\textless{}0.01)\end{tabular}}    & {\color[HTML]{FE0000} \begin{tabular}[c]{@{}l@{}}-0.0053\\  (p\textgreater{}0.1)\end{tabular}} \\ \cline{2-5} 
\multirow{-2}{*}{\textbf{Female}}       & \textbf{Number}   & {\color[HTML]{FE0000} \begin{tabular}[c]{@{}l@{}}-0.0810\\  (p\textless{}0.05)\end{tabular}}   & {\color[HTML]{3166FF} \begin{tabular}[c]{@{}l@{}}0.0147\\  (p\textgreater{}0.1)\end{tabular}}   & {\color[HTML]{FE0000} \begin{tabular}[c]{@{}l@{}}-0.2997\\  (p\textless{}0.01)\end{tabular}}   \\ \hline
                               & \textbf{Duration} & {\color[HTML]{3166FF} \begin{tabular}[c]{@{}l@{}}0.0547\\  (p\textgreater{}0.1)\end{tabular}}  & {\color[HTML]{FE0000} \begin{tabular}[c]{@{}l@{}}-0.0283\\  (p\textgreater{}0.1)\end{tabular}}  & {\color[HTML]{3166FF} \begin{tabular}[c]{@{}l@{}}0.1400\\  (p\textless{}0.01)\end{tabular}}    \\ \cline{2-5} 
\multirow{-2}{*}{\textbf{Food Places}}   & \textbf{Number}   & {\color[HTML]{FE0000} \begin{tabular}[c]{@{}l@{}}-0.0059\\  (p\textgreater{}0.1)\end{tabular}} & {\color[HTML]{FE0000} \begin{tabular}[c]{@{}l@{}}-0.0444\\  (p\textless{}0.05)\end{tabular}}    & {\color[HTML]{FE0000} \begin{tabular}[c]{@{}l@{}}-0.3357\\  (p\textless{}0.01)\end{tabular}}   \\ \hline
                               & \textbf{Duration} & {\color[HTML]{3166FF} \begin{tabular}[c]{@{}l@{}}0.0586\\  (p\textgreater{}0.1)\end{tabular}}  & {\color[HTML]{FE0000} \begin{tabular}[c]{@{}l@{}}-0.0294\\  (p\textgreater{}0.1)\end{tabular}}  & {\color[HTML]{3166FF} \begin{tabular}[c]{@{}l@{}}0.1417\\  (p\textless{}0.01)\end{tabular}}    \\ \cline{2-5} 
\multirow{-2}{*}{\textbf{Social Places}} & \textbf{Number}   & {\color[HTML]{FE0000} \begin{tabular}[c]{@{}l@{}}-0.0062\\  (p\textgreater{}0.1)\end{tabular}} & {\color[HTML]{FE0000} \begin{tabular}[c]{@{}l@{}}-0.0450\\  (p\textless{}0.05)\end{tabular}}    & {\color[HTML]{FE0000} \begin{tabular}[c]{@{}l@{}}-0.3365\\  (p\textless{}0.01)\end{tabular}}   \\ \hline
                               & \textbf{Duration} & {\color[HTML]{3166FF} \begin{tabular}[c]{@{}l@{}}0.0060\\  (p\textgreater{}0.1)\end{tabular}}  & {\color[HTML]{FE0000} \begin{tabular}[c]{@{}l@{}}-0.0688\\  (p\textless{}0.01)\end{tabular}}    & {\color[HTML]{3166FF} \begin{tabular}[c]{@{}l@{}}0.0864\\  (p\textgreater{}0.1)\end{tabular}}  \\ \cline{2-5} 
\multirow{-2}{*}{\textbf{Study Places}}        & \textbf{Number}   & {\color[HTML]{3166FF} \begin{tabular}[c]{@{}l@{}}0.0111\\  (p\textgreater{}0.1)\end{tabular}}  & {\color[HTML]{FE0000} \begin{tabular}[c]{@{}l@{}}-0.0306\\  (p\textless{}0.05)\end{tabular}} & {\color[HTML]{FE0000} \begin{tabular}[c]{@{}l@{}}-0.2858\\  (p\textless{}0.01)\end{tabular}}   \\ \hline
                               & \textbf{Duration} & {\color[HTML]{3166FF} \begin{tabular}[c]{@{}l@{}}0.1253\\  (p\textless{}0.01)\end{tabular}}    & {\color[HTML]{FE0000} \begin{tabular}[c]{@{}l@{}}-0.0583\\  (p\textless{}0.01)\end{tabular}}    & {\color[HTML]{3166FF} \begin{tabular}[c]{@{}l@{}}0.0735\\  (p\textless{}0.05)\end{tabular}} \\ \cline{2-5} 
\multirow{-2}{*}{\textbf{Home}}         & \textbf{Number}   & {\color[HTML]{3166FF} \begin{tabular}[c]{@{}l@{}}0.0845\\  (p\textless{}0.01)\end{tabular}}    & {\color[HTML]{FE0000} \begin{tabular}[c]{@{}l@{}}-0.0455\\  (p\textless{}0.01)\end{tabular}}    & {\color[HTML]{FE0000} \begin{tabular}[c]{@{}l@{}}-0.0965\\  (p\textless{}0.05)\end{tabular}}   \\ \hline
                               & \textbf{Duration} & {\color[HTML]{3166FF} \begin{tabular}[c]{@{}l@{}}0.0032\\  (p\textgreater{}0.1)\end{tabular}}  & {\color[HTML]{FE0000} \begin{tabular}[c]{@{}l@{}}-0.0789\\  (p\textless{}0.01)\end{tabular}}    & {\color[HTML]{FE0000} \begin{tabular}[c]{@{}l@{}}-0.0187\\  (p\textgreater{}0.1)\end{tabular}} \\ \cline{2-5} 
\multirow{-2}{*}{\textbf{Dormitories}}    & \textbf{Number}   & {\color[HTML]{3166FF} \begin{tabular}[c]{@{}l@{}}0.0687\\  (p\textless{}0.05)\end{tabular}}    & {\color[HTML]{FE0000} \begin{tabular}[c]{@{}l@{}}-0.0019\\  (p\textgreater{}0.1)\end{tabular}}  & {\color[HTML]{FE0000} \begin{tabular}[c]{@{}l@{}}-0.1317\\  (p\textless{}0.05)\end{tabular}}   \\ \hline
\end{tabular}
\begin{tablenotes}
\item Positive coefficients are in blue, while negative coefficients are in red. 
\end{tablenotes}
\vspace{-2pt}
\end{table}
\textbf{Impact of Phone Usage is Multifaceted}. Table \ref{logit} answers \textbf{RQ$_3$} as described in Section~\ref{sec:data:regression}
% the results of the Multinomial Logistic Regression model
%where the PHQ4 score is classified into four categories: (1) “Normal” (0–2), (2) “Mild” (3–5), (3) “Moderate” (6–8), and (4) “Severe” (9–12), 
with ``Normal'' as the baseline group, uncovering interesting patterns. For instance, the overall results (first row) suggest that longer unlock durations are associated with higher likelihoods of being in the ``Mild'' or ``Severe'' mental health categories compared to the ``Normal'' group. However, for ``Moderate'' cases, longer unlock durations are more likely to be observed in individuals classified under the ``Normal'' mental status.
% increased unlock duration raises the likelihood of falling into the ``Mild'' or ``Severe'' groups but lowers the likelihood of being in the ``Moderate'' group. 
This finding suggests that the impact of phone usage on mental health is nuanced and not in one single direction. Phone usage is not inherently beneficial or harmful; its effect varies by individual. Wise phone usage may support better mental well-being, increasing the chances of being in less severe categories, while excessive or addictive use can worsen mental health outcomes.

\textbf{Gender Differences in Predictive Patterns}. 
% Motivated by the observed differences in unlocking behaviors and their correlations with PHQ4 scores, we applied the Multinomial Logistic Regression model separately for male and female students, uncovering distinct patterns. 
Among males, longer unlock durations are associated with a higher likelihood of being in the ``Severe'' mental health category, whereas for females, longer durations are more likely to be linked to ``Normal'' state rather than ``Severe''. 
Notably, male students exhibit greater sensitivity to unlocking behaviors with a strong probability linking longer unlock duration to ``Severe'' mental status (coefficient\textgreater{}0.5).
This reinforces our earlier observations regarding gender differences, highlighting the critical role of phone usage in male students' mental health, as it appears to significantly increase the risk of severe stress and anxiety.
% Higher unlock frequency decreases the probability of being in the ``Moderate'' group for males but increases it for females. 
% Notably, male students exhibit greater sensitivity to unlocking behaviors, with a stronger association between phone usage and the likelihood of being in the ``Severe'' group.
These preliminary findings on the gender differences suggest that the mechanisms linking phone usage to mental health may vary significantly across groups. Consequently, predictive models for mental health based on phone behaviors should be tailored for specific groups—or even individuals—to capture these variations effectively.

\textbf{The Effects of Location Contexts}. Table~\ref{logit} further shows the regression results in different locations, 
% To examine whether the multinomial logistic regression model varies across different contexts, we fit separate models for behaviors at various locations, 
including food places, social places, study places, home, and dormitories,  revealing significant contextual differences. For instance, unlock duration is positively associated with the likelihood of falling into the “Severe” group at social places, study places, food places, and home but shows a negative association in dormitories. 
These variations underscore the importance of incorporating context as a critical dimension in predictive models. Future models should be context-aware, accounting for location-specific behaviors to improve accuracy and relevance in predicting mental health outcomes.







\putsec{related}{Related Work}

\noindent \textbf{Efficient Radiance Field Rendering.}
%
The introduction of Neural Radiance Fields (NeRF)~\cite{mil:sri20} has
generated significant interest in efficient 3D scene representation and
rendering for radiance fields.
%
Over the past years, there has been a large amount of research aimed at
accelerating NeRFs through algorithmic or software
optimizations~\cite{mul:eva22,fri:yu22,che:fun23,sun:sun22}, and the
development of hardware
accelerators~\cite{lee:cho23,li:li23,son:wen23,mub:kan23,fen:liu24}.
%
The state-of-the-art method, 3D Gaussian splatting~\cite{ker:kop23}, has
further fueled interest in accelerating radiance field
rendering~\cite{rad:ste24,lee:lee24,nie:stu24,lee:rho24,ham:mel24} as it
employs rasterization primitives that can be rendered much faster than NeRFs.
%
However, previous research focused on software graphics rendering on
programmable cores or building dedicated hardware accelerators. In contrast,
\name{} investigates the potential of efficient radiance field rendering while
utilizing fixed-function units in graphics hardware.
%
To our knowledge, this is the first work that assesses the performance
implications of rendering Gaussian-based radiance fields on the hardware
graphics pipeline with software and hardware optimizations.

%%%%%%%%%%%%%%%%%%%%%%%%%%%%%%%%%%%%%%%%%%%%%%%%%%%%%%%%%%%%%%%%%%%%%%%%%%
\myparagraph{Enhancing Graphics Rendering Hardware.}
%
The performance advantage of executing graphics rendering on either
programmable shader cores or fixed-function units varies depending on the
rendering methods and hardware designs.
%
Previous studies have explored the performance implication of graphics hardware
design by developing simulation infrastructures for graphics
workloads~\cite{bar:gon06,gub:aam19,tin:sax23,arn:par13}.
%
Additionally, several studies have aimed to improve the performance of
special-purpose hardware such as ray tracing units in graphics
hardware~\cite{cho:now23,liu:cha21} and proposed hardware accelerators for
graphics applications~\cite{lu:hua17,ram:gri09}.
%
In contrast to these works, which primarily evaluate traditional graphics
workloads, our work focuses on improving the performance of volume rendering
workloads, such as Gaussian splatting, which require blending a huge number of
fragments per pixel.

%%%%%%%%%%%%%%%%%%%%%%%%%%%%%%%%%%%%%%%%%%%%%%%%%%%%%%%%%%%%%%%%%%%%%%%%%%
%
In the context of multi-sample anti-aliasing, prior work proposed reducing the
amount of redundant shading by merging fragments from adjacent triangles in a
mesh at the quad granularity~\cite{fat:bou10}.
%
While both our work and quad-fragment merging (QFM)~\cite{fat:bou10} aim to
reduce operations by merging quads, our proposed technique differs from QFM in
many aspects.
%
Our method aims to blend \emph{overlapping primitives} along the depth
direction and applies to quads from any primitive. In contrast, QFM merges quad
fragments from small (e.g., pixel-sized) triangles that \emph{share} an edge
(i.e., \emph{connected}, \emph{non-overlapping} triangles).
%
As such, QFM is not applicable to the scenes consisting of a number of
unconnected transparent triangles, such as those in 3D Gaussian splatting.
%
In addition, our method computes the \emph{exact} color for each pixel by
offloading blending operations from ROPs to shader units, whereas QFM
\emph{approximates} pixel colors by using the color from one triangle when
multiple triangles are merged into a single quad.


% Space: 1/4 of Page 3

\vspace{-2pt}
\section{Conclusion and Future Directions}\label{conclusion}
% Space: 1/4 of Page 3 + 1/4 of Page 4
\vspace{-2pt}
This study presents the first large-scale investigation of college students' smartphone unlocking behaviors and their association with mental well-being, leveraging over 210,000 data points from a four-year longitudinal dataset. Our findings highlight significant gender and contextual differences, emphasizing the need for fine-grained analyses in mental health research. We demonstrate that unlocking behaviors alone offer strong predictive power, enabling lightweight models for practical adoption.
Future work should refine predictive models beyond gender differences, incorporate temporal dynamics such as high-stress periods, and develop personalized, context-aware interventions, including location-based recommendations, to promote healthier phone usage and digital well-being.
% This paper presents the first large-scale study of college students' smartphone unlocking behaviors and mental health, leveraging over 210,000 data points from a four-year longitudinal dataset. 
% Our findings reveal significant gender and contextual differences,  
% longer unlock durations correlate with poorer mental health for male students but show a positive effect for females, while location-specific behaviors highlight the importance of context in understanding mental health outcomes.
% Our findings reveal significant gender and contextual differences, highlighting the need for fine-grained analyses in future work, such as identifying root causes for mental health issues. 
% We take an initial step toward building predictive models based on unlocking behaviors, demonstrating that these features alone offer substantial predictive power, enabling smaller and more efficient models suitable for practical adoption. Future work should explore clustering students to develop tailored models beyond gender differences and integrate location data to promote healthier phone usage. Additionally, analyzing temporal dynamics, such as changes during high-stress exam periods, and developing personalized, context-aware interventions, such as location-based recommendations, could further enhance digital well-being solutions.

% We demonstrate that unlocking behaviors alone hold strong predictive power, enabling lightweight models for practical use. Future work should refine prediction models beyond gender differences, incorporate temporal dynamics, and develop personalized interventions to promote healthier phone usage and digital well-being.

\bibliographystyle{IEEEtran}
\bibliography{ref}



\end{document}
