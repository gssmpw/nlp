\documentclass[conference]{IEEEtran}

\IEEEoverridecommandlockouts
% The preceding line is only needed to identify funding in the first footnote. If that is unneeded, please comment it out.

\usepackage{multirow}
\usepackage[table,xcdraw]{xcolor}
\usepackage{threeparttable}

\usepackage{changes}
\definechangesauthor[name={Wei}, color=blue]{YN}
\definechangesauthor[name={Wei}, color=red]{DEL}

%%%%%%%%%%%%%%%%%%%%%%%%%%%%%%%%%%%%%%%%%%%%%%%%%%%%%%%
%%%%%%%%%%%%%%%    theorems %%%%%%%%%%%%%%%%%%%%%%%%%%%
%%%%%%%%%%%%%%%%%%%%%%%%%%%%%%%%%%%%%%%%%%%%%%%%%%%%%%%
% \usepackage{mdframed}
\usepackage{kantlipsum}

%%%%%%%%%%%%%%%%%%%%%%%%%%%%%%%%%%%%%%%%%%%%%%%%%%%%%%%
%%%%%%%%%%%%%%%    theorems %%%%%%%%%%%%%%%%%%%%%%%%%%%
%%%%%%%%%%%%%%%%%%%%%%%%%%%%%%%%%%%%%%%%%%%%%%%%%%%%%%%
\theoremstyle{plain}
\newtheorem{theorem}{Theorem}[section]
\newtheorem{proposition}[theorem]{Proposition}
\newtheorem{lemma}[theorem]{Lemma}
\newtheorem{example}[theorem]{Example}
\newtheorem{corollary}[theorem]{Corollary}
\theoremstyle{definition}
\newtheorem{definition}[theorem]{Definition}
\newtheorem{assumption}[theorem]{Assumption}
\theoremstyle{remark}
\newtheorem{remark}[theorem]{Remark}


% \titleformat{\subsection}[runin]% runin puts it in the same paragraph
%        {\normalfont\bfseries}% formatting commands to apply to the whole heading
%        {\thesubsection}% the label and number
%        {0.5em}% space between label/number and subsection title
%        {}% formatting commands applied just to subsection title
%        [.]% punctuation or other commands following subsection title


%%%%%%%%%%%%%%%%%%%%%%%%%%%%%%%%%%%%%%%%%%%%%%%%%%%%%%%
%%%%%%%%%%%%%%%  mathematical notations%%%%%%%%%%%%%%%%
% \usepackage[english]{babel}
% \usepackage[utf8]{inputenc}
% \usepackage[T1]{fontenc}

%% Figures, tables and lists
\usepackage[dvipsnames]{xcolor}
\usepackage{paralist}
\usepackage{graphicx}
\usepackage{subcaption}
\usepackage{longtable} 
\usepackage{multirow}
\usepackage{listings}
\usepackage{makecell}
\usepackage{array}
\usepackage{float}
\usepackage{dsfont}
\usepackage{rotating}
\usepackage{booktabs}
\usepackage{enumerate}
\usepackage{tikz}
\usepackage{pgf}
\usepackage{enumitem}
\usepackage{lipsum} % for generating filler text
\usepackage{titlesec}

%% Math
% \usepackage{amssymb, amsthm,bbm}
\usepackage{mathtools}
\usepackage{mathrsfs}
%% References and author info 
\mathtoolsset{showonlyrefs}
\usepackage{natbib}
\usepackage{authblk}
\usepackage{todonotes}
\usepackage{xr-hyper}


%%%%%%%%%%%%%%%%%%%%%%%%%%%%%%%%%%%%%%%%%%%%%%%%%%%%%%%
\newcommand{\R}{\mathbb R}
\newcommand{\EE}{\mathbb{E}}

\DeclareMathOperator{\Tr}{Tr}
\DeclareMathOperator*{\argmin}{argmin}
\DeclareMathOperator*{\argmax}{argmax}

\newcommand{\bs}[1]{\ensuremath{\boldsymbol{#1}}}
\newcommand{\mc}{\mathcal}
\newcommand{\opt}{^\star}


\newcommand{\diff}{\textnormal{d}}


\def \iid {\stackrel{\textnormal{i.i.d.}}{\sim}}
\def \iidtext {\textnormal{i.i.d.}}





%%%%%%%%%%%%%%%%%%%%%%%%%%%%%%%%%%%%%%%%%%%%%%%%%%%%%%%
%%%%%%%%%%%%%%%%%%%%% colors     %%%%%%%%%%%%%%%%%%%%%%
%%%%%%%%%%%%%%%%%%%%%%%%%%%%%%%%%%%%%%%%%%%%%%%%%%%%%%%
\definecolor{myblue}{rgb}{.8, .8, 1}
\definecolor{mathblue}{rgb}{0.2472, 0.24, 0.6} % mathematica's Color[1, 1--3]
\definecolor{mathred}{rgb}{0.6, 0.24, 0.442893}
\definecolor{mathyellow}{rgb}{0.6, 0.547014, 0.24}


% May add more in future.







\def\BibTeX{{\rm B\kern-.05em{\sc i\kern-.025em b}\kern-.08em
    T\kern-.1667em\lower.7ex\hbox{E}\kern-.125emX}}
\begin{document}

\title{Unlocking Mental Health: Exploring College Students' Well-being through Smartphone Behaviors
}

\author{\IEEEauthorblockN{1\textsuperscript{st} Wei Xuan}
\IEEEauthorblockA{\textit{Department of Economics} \\
\textit{University of Southern California}\\
Los Angeles, CA, USA \\
wxuan@usc.edu}
\and
\IEEEauthorblockN{2\textsuperscript{nd} Meghna Roy Chowdhury, 3\textsuperscript{rd} Yi Ding}
\IEEEauthorblockA{\textit{Electrical and Computer Engineering} \\
\textit{Purdue University}\\
West Lafayette, IN, USA \\
mroycho@purdue.edu, yiding@purdue.edu}
% \and
% \IEEEauthorblockN{3\textsuperscript{rd} Yi Ding}
% \IEEEauthorblockA{\textit{Electrical and Computer Engineering} \\
% \textit{Purdue University}\\
% West Lafayette, IN, USA \\
% yiding@purdue.edu}
\and
\IEEEauthorblockN{4\textsuperscript{th} Yixue Zhao}
\IEEEauthorblockA{\textit{Information Sciences Institute} \\
\textit{University of Southern California}\\
Arlington, VA, USA \\
yzhao@isi.edu}
}

\maketitle

\begin{abstract}
% Space: 1/4 of Page 1
The global mental health crisis is a pressing concern, with college students particularly vulnerable to rising mental health disorders. The widespread use of smartphones among young adults, while offering numerous benefits, has also been linked to negative outcomes such as addiction and regret, significantly impacting well-being. Leveraging the longest longitudinal dataset collected over four college years through passive mobile sensing, this study is the first to examine the relationship between students' smartphone unlocking behaviors and their mental health at scale in real-world settings. We provide the first evidence demonstrating the predictability of phone unlocking behaviors for mental health outcomes based on a large dataset, highlighting the potential of these novel features for future predictive models. Our findings reveal important variations in smartphone usage across genders and locations, offering a deeper understanding of the interplay between digital behaviors and mental health. We highlight future research directions aimed at mitigating adverse effects and promoting digital well-being in this population.
\end{abstract}

% \begin{IEEEkeywords}

% \end{IEEEkeywords}
\vspace{-2pt}
\section{Introduction}\label{intro}
\vspace{-2pt}
% Space: 1/2 of Page 1
College years are pivotal for young adults' personal, social, and academic development, often accompanied by significant stressors \cite{magolda2006intellectual,anderson2012college}. Mobile devices have become integral to college students' daily lives, with U.S. students averaging 8–10 hours of smartphone use per day \cite{roberts2014invisible}. 
A considerable body of literature suggests that excessive smartphone usage interferes with daily activities and diminishes mental health \cite{ames2013managing,cho2021reflect,ruensuk2022sad,lin2016association}. However, most prior work overlooks the multifaceted nature of smartphone use. When used mindfully, smartphones can be powerful tools that enhance productivity and well-being \cite{buchi2024digital, zhao2024digital}. Furthermore, smartphones' impact likely varies across user groups and contexts, highlighting the need for a more nuanced understanding of its relationship with mental health.

To gain a comprehensive understanding of college students' smartphone usage and its association with mental well-being, we utilized data from the {College Experience Study} (CES) dataset~\cite{nepal2024capturing}, the longest-running mobile sensing study to date. This study provides a rich dataset of passive mobile sensing data and Ecological Momentary Assessment (EMA) surveys collected from over 200 Dartmouth students between 2017 and 2022. 
Our analysis specifically focuses on the fundamental smartphone behavior of \textit{unlocking}, considering both its \emph{frequency} and \emph{duration}, to explore its relationship with students' mental health as measured by the Patient Health Questionnaire-4 (PHQ4)\cite{lowe20104,kroenke2009ultra}.
We focus on \emph{unlocking} behaviors due to empirical evidence linking them to mental health and their coarse-grained nature, which allows prediction without compromising user privacy (e.g., revealing app usage). 

This is the first large-scale study based on over 210,000 data points collected throughout entire college years that utilizes {real-world} unlocking data to explore its predictability for mental health, moving beyond prior work that has relied on indirect methods to study phone usage~\cite{abdrabou2024eyegaze, harbach2014survey, li2019swipevlock}. % Our study is the first large-scale investigation of college students' smartphone unlocking behaviors in real-life settings, based on over 210,000 data points collected throughout their entire college years. 
We first conduct correlation analysis to assess the feasibility of using smartphone unlocking behaviors to predict students' mental health status. To guide the granularity of future predictive models, we further explored the impact of gender and location differences (e.g., study spaces, social venues, and homes) on the correlations.
Following the standardization approach for PHQ4 Score in \cite{wicke2022update}, we categorize the PHQ4 Score into four groups---“Normal,” “Mild,” “Moderate,” and “Severe”---to facilitate easier interpretation. Due to the discrete nature of these outcome variables, we employ multinomial logistic regression~\cite{bohning1992multinomial} to analyze the relationships between smartphone unlocking behaviors and mental well-being. Rather than using a single global model for the entire population, we apply separate multinomial logistic regression models for different genders and locations to capture the dataset's heterogeneity.
% \hl{To further explore these novel features' predictability, we employ a multinomial logistic regression model given the discrete nature of the outcome variable (PHQ4 scores ranging from 0 to 12 where lower score indicates better mental health) with multiple categories. For interpretability, PHQ4 scores are grouped into four categories, with the lowest-score group serving as the baseline.}
% \yixue{@Yi, the highlighted parts may need to be revised. Also, the reason to use logistic regression isn't highlighted either. I remember I didn't get logistic regression can understand the *combined* features' predictability instead of individual features that's studied in a regular Pearson correlation. This point should be highlighted too as reviewers are like me who wouldn't get it if we don't point it out. Here's the relevant review regarding the discrete nature: Reviwer 1: Section 2.E: The authors need to better explain the merit of their categories and how sensitive the survey is to changes in overall score. Why did the authors not work with a continuous scale? With the categories they still have the problem that scores of 2 and 3 are in different categories. Why is this less of an issue than having 1 and 2 in different categories? My main concern is that whever you try to fit a continuous scale into categories, there may be oddities at the boundaries between categories.}
% The predictive features include unlock duration and unlock frequency. 
%Given the discrete outcome of interest (PHQ4 scores ranging from 0 to 12) with multiple categories, we then employ a multinomial logistic regression model to explore the predictability of the unlocking behaviors on students' well-being. For interpretability, PHQ4 scores are grouped into four categories, with the lowest score group serving as the baseline. Unlock duration and the number of unlocks are included as predictive features in the model. 

Our analysis uncovers nuanced phenomena in terms of gender differences: for male students, increased phone usage is linked to poorer mental health, while for female students, it appears to have a positive impact, suggesting fundamentally different dynamics in how phone usage affects well-being across genders. 
% Finally, we explore phone usage patterns across different locations, including study spaces, social venues, dining areas, dormitories, and homes, uncovering contextual relationships with mental well-being. 
These findings provide insights and actionable items for future work to build predictive models and design digital well-being interventions, as discussed in Section~\ref{result}. % \yixue{@Yi, any novelty for research approach/methodology to add in this paragraph above?}
%Specifically, we distinguish between two aspects of phone usage: frequency, as indicated by the number of locks and unlocks, and duration, as indicated by the time elapsed between a lock and unlock. Using a multiple linear regression model, we examine the relationship between different dimensions of smartphone usage and college students' mental health status. To explore diversity in these relationships, we develop separate regression models for different genders. Additionally, we analyze phone usage patterns in various locations, such as study spaces, social venues, dining areas, dormitories, and homes, to uncover contextual relationships with mental well-being.

This paper makes these contributions: 
1) We conduct the first large-scale study of real-world smartphone unlocking behaviors and mental health among college students, encompassing both iOS and Android users; 
2) We provide the first empirical evidence on the predictability of novel smartphone unlocking features, opening up future directions in predicting mental health; 
3) We demonstrate the gender differences and location variations in students' smartphone behaviors and their relationships with mental health, providing actionable insights for building predictive models;
4) We open-source the data analysis pipeline and artifacts to foster future research~\cite{OurRepo}.
%including distribution analysis at individual level, Pearson correlation analysis, and multinomial logistic regression analysis, to foster future research upon acceptance.
%First, we differentiate between smartphone usage frequency and duration, demonstrating how these dimensions relate to college students' mental health in distinct ways. This provides a holistic view of the interplay between phone usage and students' well-being. 
%Second, we highlight gender differences in the correlations between phone usage and mental health. Third, we present the first empirical evidence linking phone usage with mental well-being across different locations and contexts. Lastly, by leveraging insights from the most extensive longitudinal mobile sensing study to date, we provide valuable knowledge that can inform interventions aimed at improving phone usage habits and enhancing mental health.


% The remainder of this paper is structured as follows. Section \ref{design} introduces the dataset and explains the design of our models. Section \ref{result} details the empirical results. Section \ref{literature} discusses related work in this domain. Finally, Section \ref{conclusion} concludes the paper and outlines future research directions.


\section{Design}\label{sec:design}

%%%%%%%%%%%%%%%%%%%%%%%%%%%%%%


\begin{figure*}[t]
    \centering
    \includegraphics[trim = 15 530 15 15, width=1\textwidth]{Algorithm_drawio.pdf}
    \caption{Overview of KiSS}
    \label{fig:overview}
\end{figure*}


The results we gleaned from the previous section (see Section~\ref{sec:work_anly}) helped in developing our policy: KiSS. The KiSS or \textbf{Keep it Separated Serverless} policy aims to address critical challenges in Function-as-a-Service (FaaS) platforms, particularly in edge computing environments, by achieving the following objectives:

\begin{itemize}
    \item \textbf{Reduced Cold Start Latency:} Prioritizes high-frequency functions to minimize delays in real-time applications.
    \item \textbf{Improved Resource Efficiency:} Optimizes memory and compute usage while avoiding unnecessary overhead from static warm states.
    \item \textbf{Minimized Inter-Function Interference:} Enhances throughput and scalability through modular resource partitioning.
    \item \textbf{Improved Function Service Rate:} Adopts resource-aware policies to reduce dropped requests and maximize system reliability.
\end{itemize}


\subsection{KiSS Policy Overview}

KiSS introduces a modular, data-driven orchestration strategy designed to optimize serverless execution in resource-constrained environments, particularly at the edge. By leveraging our workload analysis (refer Section 2.5), our policy segments functions based on key metrics—memory footprint, invocation frequency, and execution time—to optimize performance across diverse workloads.

The edge computing context introduces unique challenges like limited memory, heterogeneous resources, and dynamic workloads. Generalized cloud strategies often fail to adapt to such constraints. KiSS addresses this gap by analyzing workload characteristics and implementing a resource-efficient, modular strategy that aligns with edge-specific demands.

\subsection{Components of KiSS Policy Design}
Figure~\ref{fig:overview} shows the overall architecture of KiSS. 
The incoming \textit{FaaS traffic} will include both small and large functions. 
The \textit{request handler} accepts the incoming functions and shares the function information to the workload analyzer. 
The \textit{workload analyser} processes the function information to profile the incoming function traffic information and generate data such as invocation frequency, memory footprint etc.
The \textit{KiSS policy} uses this data to estimate where this function will be placed between the two different warm pool partitions.

The \textit{load balancer} implements a partitioning logic where functions are allocated to distinct warm pools using (\textit{invoker 1} and \textit{invoker 2}) based on profiling thresholds:

(i)~Small Functions Pool: Dedicated to high-frequency, low-memory functions to ensure low latency, and (ii)~Large Functions Pool: Allocated for low-frequency, memory-intensive functions, minimizing contention with smaller containers.
Each warm pool operates autonomously achieving Policy Independence.
The \textit{Warm Pool Replacement Policy} for each warm container pool can independently implement different workload-specific strategies to reduce contention and enhance temporal locality.


These factors form the foundation of KiSS’s multi-tiered warm pool framework, allowing it to effectively manage serverless resources and enhance performance in edge computing. By addressing these challenges, KiSS positions itself as a practical and scalable solution for FaaS platforms in environments with diverse and demanding resource constraints.


\subsection{Innovations of KiSS Policy}

One of the most innovative features of KiSS is its multi-level warm pool partitioning, which isolates high- and low-frequency functions into separate pools. This design eliminates inefficiencies inherent in monolithic resource strategies by ensuring that small, frequently invoked functions are always ready to execute, while larger, less frequent functions remain accessible without competing for resources. This adaptability extends to the ability to add more pools as workload patterns evolve, making KiSS a flexible and future-proof solution. Moreover, its modular architecture supports diverse deployment scenarios, from centralized clouds to resource-constrained edge environments. Integration with traffic-aware schedulers ensures that KiSS maintains scalability and responsiveness even under fluctuating workloads.


\subsubsection{Advantages of KiSS}

The advantages of KiSS are particularly pronounced in edge environments. By keeping frequently accessed containers in warm states, it drastically reduces cold start latency, which is critical for real-time applications such as IoT and AI analytics. Static warm pool partitioning, based on workload analysis, optimizes memory usage by eliminating unnecessary overhead, ensuring that resources are used efficiently even in environments with stringent memory constraints. This strategy not only enhances performance but also reduces operational costs by consolidating memory usage and minimizing cold starts. KiSS’s platform-agnostic design further enhances its versatility, enabling seamless deployment across various serverless frameworks.



\vspace{-2pt}
\section{Results and Lessons Learned}
\label{result}
\vspace{-2pt}
% Space: 1/2 of Page 2 + 1/2 of Page 3
\subsection{Insights for Unlocking Behavior}
\vspace{-2pt}
%\yixue{each sub-section can be per RQ}
\begin{figure}[b]
\centering
\includegraphics[width=\linewidth]{figures/distribution_highlight.pdf}
\caption{Distributions of Unlock Number, Unlock Duration, and Duration per Unlock at the individual level with mean values (top) and maximum values (bottom). Excluded data points described in Section \ref{data} are shown in red.}
\label{fig1}
\end{figure}

\begin{figure}[t]
\centering
\includegraphics[width=\linewidth]{figures/distribution_by_gender.pdf}
\caption{Gender differences highlighted based on the same data shown in Fig.~\ref{fig1}}
\label{fig2}
\end{figure}

%\yixue{For Fig 1. 1. In the caption, I was referring to the exclusion criteria that's written in the regression model section. Doesn't Pearson use that exclusion criteria as well? If so, it doesn't make sense to have it in the regression section before. Maybe move it here when showing Fig.1. Otherwise, it's mysterious why the exclusions are in red and it wasn't explained here. Reader will be puzzled when seeing Fig 1. After you decide what to do, make sure the caption is correct as well. 2. keep the feature names consistent in the fig. Unit: None change to Unit: Times 3. keep Fig 2 consistent with Fig 1 after you change Fig 1}

To answer \textbf{RQ$_1$}, we analyze the data distribution as discussed in Section~\ref{sec:data:distribution}. Fig.~\ref{fig1} highlights the mean and maximum statistics of the results. For instance, Fig.\ref{fig1}-a (Mean Unlock Number) shows the distribution of the mean value of each student's Unlock Number during the study period.
%The results are shown in Fig \ref{fig1}. 

\textbf{Overall Pattern}. On average, most students unlock their phones 50–100 times daily (Fig.\ref{fig1}-a) totaling 2–4 hours (Fig.\ref{fig1}-b), and each unlock session averages 2-4 minutes (Fig.\ref{fig1}-c). On high-usage days, most students unlock their phones 100-300 times (Fig.\ref{fig1}-d)
% \added{100-300} \deleted{150–370}, 
totaling 6–10 hours. Interestingly, for most students, the duration per unlock typically falls under 4 minutes (Fig.\ref{fig1}-c), and the majority of students (96\%) use their phones under 1 hour per unlock even on the highest-usage days (Fig.\ref{fig1}-f). This suggests that students primarily engage with their phones for brief, fleeting interactions rather than extended, focused tasks, likely to satisfy momentary curiosity or to simply mindlessly unlock rather than sustained activities.

\textbf{Gender Difference}. As shown in Fig. \ref{fig2}, notable gender differences in phone usage emerge. Male students display a broader range of unlock frequencies, indicating more diverse usage habits. In contrast, female students exhibit more consistent patterns but spend significantly more time per unlock. This behavior suggests that female students may engage more deeply with their phones during each interaction, potentially reflecting a more intentional and focused usage style.



\vspace{-2pt}
\subsection{Unlocking Behavior and Mental Well-being Correlation}
\vspace{-2pt}
\begin{figure}[t]
    \centering
    \includegraphics[width=0.85\linewidth]{images/spearman_heatmap.png}
    \vspace{-0.3cm}
    \caption{Spearman correlation between Insighter-generated ratings and user explicit rating. \label{fig:spearman}}
    \vspace{-0.5cm}
\end{figure}

\textbf{Overall Correlations}. To answer \textbf{RQ$_2$}, we conduct correlation analysis as described in Section~\ref{sec:data:correlation} and the results are shown in Table~\ref{corre}.
Notably, all p-values for the correlation coefficients are well below 0.01, indicating strong associations between phone usage patterns and mental health. This finding highlights the potential of our novel unlocking features as effective predictors of mental health outcomes. Column 1 reveals an intriguing pattern: longer phone usage durations correlate with poorer mental health outcomes, whereas higher unlock frequencies are associated with better mental health. These findings underscore the importance of considering both frequency and duration when assessing the impact of smartphone use on mental health, and call for further studies to identify the root causes of these interesting phenomena.

\textbf{Gender Difference}. 
Table \ref{corre} reveals significant gender differences in phone usage. For males, both unlock duration and frequency are positively correlated with PHQ4 Scores, indicating a link to poorer mental health. In contrast, females exhibit the opposite trend, suggesting higher phone usage may be associated with better well-being or act as a coping mechanism. Additionally, the magnitude of correlation coefficients is notably larger for male students, suggesting they are more sensitive to the mental health impacts of phone usage. These findings imply that male students may engage with their phones in less constructive ways than their female peers, with higher interaction levels correlating more strongly with adverse mental health outcomes, highlighting the need for personalized predictive models in different user groups.

\textbf{Location Difference}. Table \ref{corre_loc} shows the correlations between unlocking behaviors and mental well-being at different locations, revealing fine-grained insights to guide further studies. For example, unlocking frequency and duration are both negatively correlated with PHQ4 Score (better mental health) at food and social places, suggesting beneficial effects in social contexts. In contrast, these behaviors correlate with poorer mental health at home, possibly reflecting less meaningful usage when in private. These findings emphasize the importance of context when examining phone usage and mental health.

\subsection{Insights from Multinomial Logistic Regression}
\vspace{-2pt}


\begin{table}[t]
\centering
\caption{Coefficients Results of Multinomial Logistic Regression}
\label{logit}
\begin{tabular}{|l|l|l|l|l|}
\hline
                               & \textbf{Features} & \textbf{Mild}                                                                                          & \textbf{Moderate}                                                                                       & \textbf{Severe}                                                                                        \\ \hline
                               & \textbf{Duration} & {\color[HTML]{3166FF} \begin{tabular}[c]{@{}l@{}}0.1169\\  (p\textless{}0.01)\end{tabular}}    & {\color[HTML]{FE0000} \begin{tabular}[c]{@{}l@{}}-0.0339\\  (p\textless{}0.01)\end{tabular}}    & {\color[HTML]{3166FF} \begin{tabular}[c]{@{}l@{}}0.1575\\  (p\textless{}0.01)\end{tabular}}    \\ \cline{2-5} 
\multirow{-2}{*}{\textbf{Overall}}      & \textbf{Number}   & {\color[HTML]{3166FF} \begin{tabular}[c]{@{}l@{}}0.0903\\  (p\textless{}0.01)\end{tabular}}    & {\color[HTML]{FE0000} \begin{tabular}[c]{@{}l@{}}-0.0378\\  (p\textless{}0.01)\end{tabular}}    & {\color[HTML]{FE0000} \begin{tabular}[c]{@{}l@{}}-0.1270\\  (p\textless{}0.01)\end{tabular}}   \\ \hline
                               & \textbf{Duration} & {\color[HTML]{3166FF} \begin{tabular}[c]{@{}l@{}}0.2398\\  (p\textless{}0.01)\end{tabular}}    & {\color[HTML]{3166FF} \begin{tabular}[c]{@{}l@{}}0.0298\\  (p\textgreater{}0.1)\end{tabular}}   & {\color[HTML]{3166FF} \begin{tabular}[c]{@{}l@{}}0.5741\\  (p\textless{}0.01)\end{tabular}}    \\ \cline{2-5} 
\multirow{-2}{*}{\textbf{Male}}         & \textbf{Number}   & {\color[HTML]{3166FF} \begin{tabular}[c]{@{}l@{}}0.2215\\  (p\textless{}0.01)\end{tabular}}    & {\color[HTML]{FE0000} \begin{tabular}[c]{@{}l@{}}-0.1896\\  (p\textless{}0.01)\end{tabular}}    & {\color[HTML]{FE0000} \begin{tabular}[c]{@{}l@{}}-0.1731\\  (p\textless{}0.01)\end{tabular}}   \\ \hline
                               & \textbf{Duration} & {\color[HTML]{3166FF} \begin{tabular}[c]{@{}l@{}}0.1000\\  (p\textless{}0.01)\end{tabular}}    & {\color[HTML]{FE0000} \begin{tabular}[c]{@{}l@{}}-0.0386\\  (p\textless{}0.01)\end{tabular}}    & {\color[HTML]{FE0000} \begin{tabular}[c]{@{}l@{}}-0.0053\\  (p\textgreater{}0.1)\end{tabular}} \\ \cline{2-5} 
\multirow{-2}{*}{\textbf{Female}}       & \textbf{Number}   & {\color[HTML]{FE0000} \begin{tabular}[c]{@{}l@{}}-0.0810\\  (p\textless{}0.05)\end{tabular}}   & {\color[HTML]{3166FF} \begin{tabular}[c]{@{}l@{}}0.0147\\  (p\textgreater{}0.1)\end{tabular}}   & {\color[HTML]{FE0000} \begin{tabular}[c]{@{}l@{}}-0.2997\\  (p\textless{}0.01)\end{tabular}}   \\ \hline
                               & \textbf{Duration} & {\color[HTML]{3166FF} \begin{tabular}[c]{@{}l@{}}0.0547\\  (p\textgreater{}0.1)\end{tabular}}  & {\color[HTML]{FE0000} \begin{tabular}[c]{@{}l@{}}-0.0283\\  (p\textgreater{}0.1)\end{tabular}}  & {\color[HTML]{3166FF} \begin{tabular}[c]{@{}l@{}}0.1400\\  (p\textless{}0.01)\end{tabular}}    \\ \cline{2-5} 
\multirow{-2}{*}{\textbf{Food Places}}   & \textbf{Number}   & {\color[HTML]{FE0000} \begin{tabular}[c]{@{}l@{}}-0.0059\\  (p\textgreater{}0.1)\end{tabular}} & {\color[HTML]{FE0000} \begin{tabular}[c]{@{}l@{}}-0.0444\\  (p\textless{}0.05)\end{tabular}}    & {\color[HTML]{FE0000} \begin{tabular}[c]{@{}l@{}}-0.3357\\  (p\textless{}0.01)\end{tabular}}   \\ \hline
                               & \textbf{Duration} & {\color[HTML]{3166FF} \begin{tabular}[c]{@{}l@{}}0.0586\\  (p\textgreater{}0.1)\end{tabular}}  & {\color[HTML]{FE0000} \begin{tabular}[c]{@{}l@{}}-0.0294\\  (p\textgreater{}0.1)\end{tabular}}  & {\color[HTML]{3166FF} \begin{tabular}[c]{@{}l@{}}0.1417\\  (p\textless{}0.01)\end{tabular}}    \\ \cline{2-5} 
\multirow{-2}{*}{\textbf{Social Places}} & \textbf{Number}   & {\color[HTML]{FE0000} \begin{tabular}[c]{@{}l@{}}-0.0062\\  (p\textgreater{}0.1)\end{tabular}} & {\color[HTML]{FE0000} \begin{tabular}[c]{@{}l@{}}-0.0450\\  (p\textless{}0.05)\end{tabular}}    & {\color[HTML]{FE0000} \begin{tabular}[c]{@{}l@{}}-0.3365\\  (p\textless{}0.01)\end{tabular}}   \\ \hline
                               & \textbf{Duration} & {\color[HTML]{3166FF} \begin{tabular}[c]{@{}l@{}}0.0060\\  (p\textgreater{}0.1)\end{tabular}}  & {\color[HTML]{FE0000} \begin{tabular}[c]{@{}l@{}}-0.0688\\  (p\textless{}0.01)\end{tabular}}    & {\color[HTML]{3166FF} \begin{tabular}[c]{@{}l@{}}0.0864\\  (p\textgreater{}0.1)\end{tabular}}  \\ \cline{2-5} 
\multirow{-2}{*}{\textbf{Study Places}}        & \textbf{Number}   & {\color[HTML]{3166FF} \begin{tabular}[c]{@{}l@{}}0.0111\\  (p\textgreater{}0.1)\end{tabular}}  & {\color[HTML]{FE0000} \begin{tabular}[c]{@{}l@{}}-0.0306\\  (p\textless{}0.05)\end{tabular}} & {\color[HTML]{FE0000} \begin{tabular}[c]{@{}l@{}}-0.2858\\  (p\textless{}0.01)\end{tabular}}   \\ \hline
                               & \textbf{Duration} & {\color[HTML]{3166FF} \begin{tabular}[c]{@{}l@{}}0.1253\\  (p\textless{}0.01)\end{tabular}}    & {\color[HTML]{FE0000} \begin{tabular}[c]{@{}l@{}}-0.0583\\  (p\textless{}0.01)\end{tabular}}    & {\color[HTML]{3166FF} \begin{tabular}[c]{@{}l@{}}0.0735\\  (p\textless{}0.05)\end{tabular}} \\ \cline{2-5} 
\multirow{-2}{*}{\textbf{Home}}         & \textbf{Number}   & {\color[HTML]{3166FF} \begin{tabular}[c]{@{}l@{}}0.0845\\  (p\textless{}0.01)\end{tabular}}    & {\color[HTML]{FE0000} \begin{tabular}[c]{@{}l@{}}-0.0455\\  (p\textless{}0.01)\end{tabular}}    & {\color[HTML]{FE0000} \begin{tabular}[c]{@{}l@{}}-0.0965\\  (p\textless{}0.05)\end{tabular}}   \\ \hline
                               & \textbf{Duration} & {\color[HTML]{3166FF} \begin{tabular}[c]{@{}l@{}}0.0032\\  (p\textgreater{}0.1)\end{tabular}}  & {\color[HTML]{FE0000} \begin{tabular}[c]{@{}l@{}}-0.0789\\  (p\textless{}0.01)\end{tabular}}    & {\color[HTML]{FE0000} \begin{tabular}[c]{@{}l@{}}-0.0187\\  (p\textgreater{}0.1)\end{tabular}} \\ \cline{2-5} 
\multirow{-2}{*}{\textbf{Dormitories}}    & \textbf{Number}   & {\color[HTML]{3166FF} \begin{tabular}[c]{@{}l@{}}0.0687\\  (p\textless{}0.05)\end{tabular}}    & {\color[HTML]{FE0000} \begin{tabular}[c]{@{}l@{}}-0.0019\\  (p\textgreater{}0.1)\end{tabular}}  & {\color[HTML]{FE0000} \begin{tabular}[c]{@{}l@{}}-0.1317\\  (p\textless{}0.05)\end{tabular}}   \\ \hline
\end{tabular}
\begin{tablenotes}
\item Positive coefficients are in blue, while negative coefficients are in red. 
\end{tablenotes}
\vspace{-2pt}
\end{table}
\textbf{Impact of Phone Usage is Multifaceted}. Table \ref{logit} answers \textbf{RQ$_3$} as described in Section~\ref{sec:data:regression}
% the results of the Multinomial Logistic Regression model
%where the PHQ4 score is classified into four categories: (1) “Normal” (0–2), (2) “Mild” (3–5), (3) “Moderate” (6–8), and (4) “Severe” (9–12), 
with ``Normal'' as the baseline group, uncovering interesting patterns. For instance, the overall results (first row) suggest that longer unlock durations are associated with higher likelihoods of being in the ``Mild'' or ``Severe'' mental health categories compared to the ``Normal'' group. However, for ``Moderate'' cases, longer unlock durations are more likely to be observed in individuals classified under the ``Normal'' mental status.
% increased unlock duration raises the likelihood of falling into the ``Mild'' or ``Severe'' groups but lowers the likelihood of being in the ``Moderate'' group. 
This finding suggests that the impact of phone usage on mental health is nuanced and not in one single direction. Phone usage is not inherently beneficial or harmful; its effect varies by individual. Wise phone usage may support better mental well-being, increasing the chances of being in less severe categories, while excessive or addictive use can worsen mental health outcomes.

\textbf{Gender Differences in Predictive Patterns}. 
% Motivated by the observed differences in unlocking behaviors and their correlations with PHQ4 scores, we applied the Multinomial Logistic Regression model separately for male and female students, uncovering distinct patterns. 
Among males, longer unlock durations are associated with a higher likelihood of being in the ``Severe'' mental health category, whereas for females, longer durations are more likely to be linked to ``Normal'' state rather than ``Severe''. 
Notably, male students exhibit greater sensitivity to unlocking behaviors with a strong probability linking longer unlock duration to ``Severe'' mental status (coefficient\textgreater{}0.5).
This reinforces our earlier observations regarding gender differences, highlighting the critical role of phone usage in male students' mental health, as it appears to significantly increase the risk of severe stress and anxiety.
% Higher unlock frequency decreases the probability of being in the ``Moderate'' group for males but increases it for females. 
% Notably, male students exhibit greater sensitivity to unlocking behaviors, with a stronger association between phone usage and the likelihood of being in the ``Severe'' group.
These preliminary findings on the gender differences suggest that the mechanisms linking phone usage to mental health may vary significantly across groups. Consequently, predictive models for mental health based on phone behaviors should be tailored for specific groups—or even individuals—to capture these variations effectively.

\textbf{The Effects of Location Contexts}. Table~\ref{logit} further shows the regression results in different locations, 
% To examine whether the multinomial logistic regression model varies across different contexts, we fit separate models for behaviors at various locations, 
including food places, social places, study places, home, and dormitories,  revealing significant contextual differences. For instance, unlock duration is positively associated with the likelihood of falling into the “Severe” group at social places, study places, food places, and home but shows a negative association in dormitories. 
These variations underscore the importance of incorporating context as a critical dimension in predictive models. Future models should be context-aware, accounting for location-specific behaviors to improve accuracy and relevance in predicting mental health outcomes.








\section{Related Work} \label{sec:related}

% \textbf{Adversarial Attack}
\textbf{Attacks on SLAM.} 
%With the rise of machine learning, 
The robustness of computer vision systems is being actively investigated. With the emergence of adversarial images in the digital domain by adding optimized noise directly to images~\cite{szegedy2013intriguing,carlini2017towards}, researchers find that such attacks also exist physically in the real world \cite{eykholt2018robust,song2018physical,zhao2019seeing}. To fill the gap between attacks in the digital and physical worlds, recent studies have demonstrated that attacks on real-world computer vision systems are practical \cite{eykholt2018robust,li2019adversarial,man2020ghostimage,sharif2016accessorize,zhao2019seeing,zhou2018invisible}. However, attacks on traditional computer vision methods such as SLAM are relatively less explored. \cite{yoshida2022adversarial} proposes an attack against the scan matching algorithm in LiDAR-based SLAM, while most SLAMs in AR/VR devices rely on different sensors like RGB/depth cameras and IMUs. \cite{ikram2022perceptual} and \cite{chen2024adversary} mislead visual SLAM by poisoning the images with special patterns, and \cite{wang2021can} causes the camera to fail using infrared light. In our work, we demonstrate attacks on Visual-Inertial SLAM (VI-SLAM) by perturbing the IMU readings, rather than cameras, and showing its impact on XR user experience. 

\textbf{Acoustic Injection Attacks.} Among various physical attacks, acoustic injection attacks are attractive due to their low cost. Son~\etal~\cite{son2015rocking} were the first to introduce acoustic attacks on MEMS gyroscopes, demonstrating how these attacks could lead to sensor denial-of-service and result in drone crashes. WALNUT~\cite{trippel2017walnut} expanded on this by developing output biasing and control attacks that enable precise manipulation of MEMS accelerometer outputs using modulated sound waves. Wang et al.~\cite{wang2017sonic} demonstrated a sonic gun, showcasing the vulnerability of various smart devices (\eg drones and self-balancing vehicles) to acoustic attacks. Tu et al. \cite{tu2018injected} designed side-swing and switching attacks to alter the outputs of MEMS gyroscopes and accelerometers. Furthermore, Ji et al. \cite{ji2021poltergeist} fool the object detectors by applying acoustic attack to the image stabilizers commonly used in modern cameras. However, none of the existing works study the relationship between the acoustic injections and SLAM outputs on recent XR devices. 

% \zijian{Do we need one session about security in AR/VR?}
% \yicheng{TODO}
%\jiasi{cite the AIVR paper (UMass Amherst?) paper is we have not already. They add IMU perturbation but w/o SLAM, iirc} \yicheng{Cited}

\textbf{XR Security and Privacy.} 
%Security and privacy concerns in XR systems have gained significant attention. 
For single-user XR systems, researchers have demonstrated various side-channel attacks to extract sensitive information (\eg keystrokes) through video feeds~\cite{ling2019know}, head movements~\cite{nair2023unique, slocum2023going}, architectural hints~\cite{zhang2023its,shang2020arspy}, power usage~\cite{li2024dangers}, and EM side-channel leakages~\cite{al2021vr}. In multi-user XR systems, Su et al.~\cite{su2024remote} use avatar motion data to infer keystrokes in shared VR environments. Slocum et al.~\cite{slocum2024doesn} reveal vulnerabilities in the shared state frameworks of multi-user AR. Similarly, Lebeck et al.~\cite{lebeck2017securing} highlight risks like deceptive virtual objects and emphasize access control for managing shared physical and virtual spaces. Ruth et al.~\cite{ruth2019secure} further propose a secure multi-user AR framework focusing on content sharing and permissions.
Chandio et al.~\cite{chandio2024stealthy} %introduced a multi-modal spatiotemporal attack that 
simultaneously manipulated visual and inertial sensors to disrupt XR pose estimation. However, their study evaluated the attack using offline datasets and assumed the attacker's capability to manipulate IMU data streams through acoustic means, without real experiments. Ours is the first to demonstrate acoustic injection attacks on recent XR devices, like the Hololens 2, in the real world.
 


% Space: 1/4 of Page 3

\vspace{-2pt}
\section{Conclusion and Future Directions}\label{conclusion}
% Space: 1/4 of Page 3 + 1/4 of Page 4
\vspace{-2pt}
This study presents the first large-scale investigation of college students' smartphone unlocking behaviors and their association with mental well-being, leveraging over 210,000 data points from a four-year longitudinal dataset. Our findings highlight significant gender and contextual differences, emphasizing the need for fine-grained analyses in mental health research. We demonstrate that unlocking behaviors alone offer strong predictive power, enabling lightweight models for practical adoption.
Future work should refine predictive models beyond gender differences, incorporate temporal dynamics such as high-stress periods, and develop personalized, context-aware interventions, including location-based recommendations, to promote healthier phone usage and digital well-being.
% This paper presents the first large-scale study of college students' smartphone unlocking behaviors and mental health, leveraging over 210,000 data points from a four-year longitudinal dataset. 
% Our findings reveal significant gender and contextual differences,  
% longer unlock durations correlate with poorer mental health for male students but show a positive effect for females, while location-specific behaviors highlight the importance of context in understanding mental health outcomes.
% Our findings reveal significant gender and contextual differences, highlighting the need for fine-grained analyses in future work, such as identifying root causes for mental health issues. 
% We take an initial step toward building predictive models based on unlocking behaviors, demonstrating that these features alone offer substantial predictive power, enabling smaller and more efficient models suitable for practical adoption. Future work should explore clustering students to develop tailored models beyond gender differences and integrate location data to promote healthier phone usage. Additionally, analyzing temporal dynamics, such as changes during high-stress exam periods, and developing personalized, context-aware interventions, such as location-based recommendations, could further enhance digital well-being solutions.

% We demonstrate that unlocking behaviors alone hold strong predictive power, enabling lightweight models for practical use. Future work should refine prediction models beyond gender differences, incorporate temporal dynamics, and develop personalized interventions to promote healthier phone usage and digital well-being.

\bibliographystyle{IEEEtran}
\bibliography{ref}



\end{document}
