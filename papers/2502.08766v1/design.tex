%\vspace{-2pt}
\section{Empirical Study Design}\label{design}
% Space: 1/4 of Page 1 + 1/2 of Page 2
%\vspace{-2pt}
%This section overviews our empirical study design, including the research questions we focus on, dataset used, and our data analysis methodology.

%\vspace{-4pt}
\subsection{Research Questions}
%\vspace{-2pt}

To investigate the nuanced relationship between college students' smartphone behaviors and mental health, as well as to guide future predictive models on mental health outcomes, we address the following key research questions (RQs).
\begin{itemize}
%\vspace{-0.5mm}
	\item \textbf{RQ$_1$} -- Do college students of different genders interact with their smartphones differently across college years?
	\item \textbf{RQ$_2$} -- Are there any correlations between students' smartphone behaviors and mental health? Do gender or location differences change these correlations? %\yixue{pearson correlation of 3 features, all and gender diff and location diff. }
	\item \textbf{RQ$_3$} -- Can we build predictive models of students' mental health with smartphone behaviors? Do predictive patterns differ across genders and locations?
    %\yixue{logistic regression with 2 features and 4 categories. }
% \vspace{-1mm}
\end{itemize}


\subsection{Dataset Overview and Pre-processing}\label{data}
\vspace{-2pt}
%\yixue{Here we should say the "meta-processing" of the dataset. 1. we identified relevant features to focus on from this giant dataset; 2. we introduce our new features to facilitate the study, e.g., duration/num, and counting the avg of unlocking in the past 2 weeks based on EMA data points, and also categorizing them into "mild", "normal", etc. ANYTHING ELSE? more technical stuff? If we're eliminating outliers up front, we should have that here; if we're only excluding outliers in the regression, we should mention it later when talking about regression.}

To answer the RQs, we utilize the CES dataset \cite{nepal2024capturing}, the most extensive longitudinal mobile sensing study to date. This dataset spans from 2017 to 2022 and includes 215 first-year undergraduate students from Dartmouth College in the United States, with 146 (67.9\%) female and 69 (32.1\%) male participants.
% \deleted{217 first-year undergraduate students from Dartmouth College in the United States, with 129 (69.35\%) female and 57 (30.65\%) male participants.}
From the dataset, we select the most relevant features to represent smartphone behaviors, namely, the number of phone unlocks (\textit{Unlock Number}) and the duration of phone unlocks (\textit{Unlock Duration}) on a given day.
Students' mental health was assessed using the \textit{PHQ4 Score} from the EMA surveys based on the PHQ9 and the Generalized Anxiety Disorder-7 (GAD7)~\cite{nepal2024capturing,williams2014gad}. PHQ4 Score ranges from 0 to 12, where higher scores denote greater levels of depression and anxiety.
The EMA surveys are delivered randomly once per week through a mobile app. In total, the dataset contains 213,408 data entries collected for unlocking behaviors and 34,581 data entries for PHQ4 Score across all participants.

% Table~\ref{tbl:data_points} shows the total number of data points (All Users) used in our study and the statistics per user (Min, Avg, Max). showing consistency across {Unlock Number} and {Unlock Duration}, which highlights the dataset's quality.
%We leverage the number of locks/unlocks and duration between lock and unlocks to represent different aspects of phone usage. 

We augment the dataset by first introducing a new feature to represent the average time spent on each unlock on a given day (\textit{Duration per Unlock}). To align the data with the PHQ4 Score, which reflects mental health status over the past two weeks, we introduce three additional features and calculate their values: the moving averages of {Unlock Duration}, {Unlock Number}, and {Duration per Unlock} over the same two-week period. 
%Together with gender information and activities at different locations, we construct various datasets to analyze phone usage and mental health status across different student groups and contexts.

To avoid tainted results, we exclude outlier data points in our analyses 
%As the unlock data is collected through a mobile app with non-trivial technical challenges \yi{what is non-trivial technical challenges, unclear}, it may include certain invalid data points due to technical glitches \yi{this whole sentence repeats itself. needs to rewrite}. 
based on the following criteria: (1) data points where the {Unlock Duration} exceeds 16 hours; (2) students whose maximum {Unlock Number} or {Unlock Duration} is 0; and (3) students whose data shows no variation, defined as a standard deviation of 0 for {Unlock Number}, {Unlock Duration}, or {PHQ4 Score} throughout the study period.
% These data points in red color in our distribution analysis. 
% To avoid tainted results, we exclude these data points when conducting Pearson Correlation Analysis and Mutinomial Logistic Regression Model 
This process resulted in 214 students with 213,360 data points in total.

% \begin{table}[t]
% \centering
% \caption{The number of data points across the dataset}
% \label{tbl:data_points}
% \begin{tabular}{|l|l|l|l|
% >{\columncolor[HTML]{EFEFEF}}l |}
% \hline
% \multicolumn{1}{|c|}{\textbf{Number of Data Points / User}} & \multicolumn{1}{c|}{\textbf{Min}} & \multicolumn{1}{c|}{\textbf{Avg}} & \multicolumn{1}{c|}{\textbf{Max}} & \multicolumn{1}{c|}{\cellcolor[HTML]{EFEFEF}\textbf{All Users}} \\ \hline
% \textbf{Unlock Number}                            &        2                       &                  994               &       1370                        &                    214751                                         \\ \hline
% \textbf{Unlock Duration}                             &        2                           &                  994                 &     1370                              &      214751                                                       \\ \hline
% \textbf{PHQ4 Score}                                  &       1                            &                 162                  &     441                              &              35126                                               \\ \hline
% \end{tabular}
% \end{table}

\subsection{Unlocking Behavior Patterns}
\label{sec:data:distribution}
\vspace{-2pt}
To first understand unlocking behavior patterns among college students (\textbf{RQ$_1$}), we calculate the detailed statistics (e.g., mean, standard deviation) of {Unlock Number}, {Unlock Duration}, and {Duration per Unlock} for each student. We then plot the distributions of these features across all students and further stratify the distributions by gender to reveal potential differences between male and female students.



%\yixue{we just introduced these features, so keep their names consistent. even better if we keep the order consistent too, first number, then duration, and duration per unlock. This should be consistent in the diagrams' captions etc as well throughout the paper}


\subsection{Pearson Correlation Analysis}
\label{sec:data:correlation}
\vspace{-2pt}
To understand whether unlocking behaviors are promising features to predict mental health, we investigate the direction and strength of correlations between unlocking behaviors and students' mental health (\textbf{RQ$_2$}) by calculating the Pearson correlation coefficients between PHQ4 Scores and {Unlock Duration}, {Unlock Number}, and {Duration per Unlock}. These coefficients are stratified by gender and location to uncover potential variations, providing further insights.


\subsection{Multinomial Logistic Regression Model}
\label{sec:data:regression}
\vspace{-2pt}
To explore the extent to which unlock features can predict students' mental health (\textbf{RQ$_3$}), we formulate it as a classification problem by categorizing the outcome variable {PHQ4 Score} into four groups using standardization approach~\cite{wicke2022update}: (1) ``Normal'' (0–2); (2) ``Mild'' (3-5); (3) ``Moderate'' (6-8); (4) ``Severe'' (9-12). This approach accommodates the subjectivity inherent in PHQ4 Scores, as students may rate the same mental status with different scores. By allowing variation within each group, we reduce the sensitivity and improve the robustness of the classification. Given the discrete and multi-class nature of the outcome variables, we use multinomial logistic regression~\cite{bohning1992multinomial} to model the relationship between unlocking behaviors and the likelihood of belonging to different mental status groups. This method is chosen for its suitability in handling categorical outcomes and its inherent linearity, ensuring interpretability and computational efficiency.

%To better understand the relationship between students’ unlocking behavior and mental health status, we build a multinomial logistic regression model. 

%The indicator for mental health status, PHQ4 score, is scaled from 0 to 12. We first classify the outcome into four categories based on PHQ4 score. We label mental health status as “Normal” if PHQ4 score is between 0 and 2, “Mild” if PHQ4 score is between 3 and 5, “Moderate” if PHQ4 score is between 6 and 8, and “Severe” if PHQ4 score is between 9 and 12. 


%There are some unusual observations and students in the dataset which might contaminate our regression results. To address this issue, We drop those observations where duration of unlocks in one day is over 1000 minutes (about 16 hours). We further drop those students whose maximum of unlock duration or numbers is 0. We also leave out those students whose behaviors are unusally stable, that is, those whose standard deviation of unlock duration, number of unlocks, and PHQ4 score across the study period is 0. 
% \yixue{@Yi, we need to add why choose multinomial logistic regression model here. Relevant review: m) Section 2.E: The authors use a linear model for their regression. While this model may be perfectly fine, the authors must provide reasoning as to why this is a suitable model.}
% After these pre-processing steps, we build the following multinomial logistic regression model:
% \begin{equation*}
%     score_{i,k}= \beta_{k0}+\beta_{k1}Unlock\_Num_i+\beta_{k2}Unlock\_Duration_{i}
% \end{equation*}
% where $k$ is one of the four groups of mental health status, $i$ is student $i$. $score_{i,k}$ is the score associated with assigning student $i$ to mental health group $k$. The higher $score_{i,k}$, the more likely student $i$ falls into group $k$. $Unlock\_Num_i$ and $Unlock\_Duration_{i}$ are student $i$'s moving average Unlock Number and Unlock Duration, respectively.
