\vspace{-2pt}
\section{Related Work}\label{literature}
\vspace{-2pt}
% To the best of our knowledge, 
% in mobile software engineering community and digital well-being community, this paper isa rare work that classifies smartphone usage into different aspects and provides empirical insights about how students' well-being is linked to them in a different way. Also, we provide first piece of evidence that mental well-being is correlated with phone usage differently for different group of people and at different contexts.
This paper is the first study to investigate college students' smartphone unlocking behaviors and mental well-being using a longitudinal dataset spanning entire college years, including both iOS and Android users.
Previous research has explored unlocking behaviors through indirect methods, such as examining eye gaze during unlocking~\cite{abdrabou2024eyegaze}, conducting self-reported online surveys~\cite{harbach2014survey}, predicting unlock patterns based on swipe behaviors~\cite{li2019swipevlock}, and analyzing lock screens on Android devices only~\cite{harbach2016anatomy}. However, none of these studies have directly measured real-world unlocking behaviors in practice.

Using mobile sensing techniques to understand mental health issues has gained prominence, such as identifying mental health risks~\cite{adler2022machine, faurholt2019objective, wang2022first}, modeling behaviors and capturing mental health parameters~\cite{likamwa2013moodscope, macleod2021mobile, mehrotra2017mytraces, tag2022emotion}. While some studies focus on college students' behaviors~\cite{macleod2021mobile, meegahapola2020smartphone, meegahapola2023generalization, wang2021transition}, the specific role of smartphone unlocking behaviors remains underexplored.

% There has been a flourish in previous literature utilizing mobile sensing data and EMA surveys to identify mental health risks \cite{adler2022machine, faurholt2019objective,wang2022first}, model behavior for mental health assessment and capture mental health parameters \cite{likamwa2013moodscope,macleod2021mobile,mehrotra2017mytraces, tag2022emotion}. Some of them specifically target at students' behaviors \cite{macleod2021mobile,meegahapola2020smartphone,meegahapola2023generalization,wang2021transition, xu2019leveraging}. This branch of research mainly focused on building prediction model for early detection of mental health issues with all available passive sensing data. Smartphone usage behavior is only part of the predictors and its specific role remains untouched. This directly motivates us to dig further into smartphone usage behavior of college students to recognize the importance of phone usage in prediction and intervention of mental health issues.


 Another area has linked smartphone usage to diminishing well-being, with studies examining their relationship~\cite{buchi2024digital, allcott2022digital, peper2018digital, thomas2022digital} and interventions to reduce digital usage~\cite{lyngs2020just, orzikulova2023finerme, roffarello2023achieving, zimmermann2023digital}. 
 However, they often overlook that smartphone behaviors involve both \textit{duration} and \textit{frequency}, with patterns varying across different user groups and contexts. Our work fills this gap by providing insights to inform the design of personalized, context-aware interventions for healthier smartphone usage.


% Another stream of research has linked smartphone usage to diminishing well-being \cite{buchi2024digital,allcott2022digital,peper2018digital,thomas2022digital}, and proposed approaches, including limiting screen time, incorporating friction, etc., to reduce digital usage \cite{lyngs2020just,orzikulova2023finerme, roffarello2023achieving,zimmermann2023digital}. However, they often missed the point that phone usage behavior involves not only duration but also frequency of human-phone interactions. More importantly, there are different patterns in terms of phone usage and mental well-being for different group of people and/or at different context. Our paper provides empirical evidence about the heterogeneity of links between phone usage and mental well-being for different genders and at different locations. This new piece knowledge can be leveraged to better inform group-specific and context-aware interventions for phone usage.