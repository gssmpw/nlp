\section{Related Work}
\label{sec:relatedWork}

The efficient evaluation of \acp{KG} accuracy has been largely overlooked. The first approach, KGEval \cite{ojha_talukdar-2017}, is an iterative algorithm that alternates between control and inference stages. In the control stage, KGEval uses crowdsourcing to select facts for evaluation. In the inference stage, it applies type consistency and Horn-clause coupling constraints~\cite{mitchell_etal-2018,lao_etal-2011} to the evaluated facts to automatically estimate the correctness of additional facts. This process repeats until no more facts are evaluated. KGEval does not scale to real-life \acp{KG}~\cite{gao_etal-2019} and is susceptible to error propagation due to its probabilistic inference mechanism, making it unsuitable for our work.

To overcome KGEval limitations, \citet{gao_etal-2019} resorted to sampling strategies and estimators that gauge \ac{KG} accuracy with statistical guarantees. To minimize annotation costs, \cite{gao_etal-2019} proposed the use of cluster sampling techniques (i.e., \ac{TWCS}) over standard sampling (i.e., \ac{SRS}). On the other hand, the authors relied on the Wald interval~\cite{casella_berger-2002}, known to have reliability issues when used on binomial proportions~\cite{brown_etal-2001,wallis-2013}. To address Wald issues for \ac{KG} accuracy evaluation, \citet{marchesin_silvello-2024} proposed to use the Wilson interval~\cite{wilson-1927}, being it better suited for binomial proportions.

Although reliable, the Wilson interval shares Wald's frequentist interpretation, preventing a one-shot probabilistic understanding of its reliability. Additionally, the interval balances efficiency and reliability, sometimes requiring more annotations than Wald for greater reliability. In contrast, our solution offers a one-shot interpretation of interval confidence, making it better suited for the task while remaining the most efficient option. Moreover, the proposed approach, $a$HPD, eliminates the need for analysts to select a specific prior, addressing a key barrier to using Bayesian methods.

\citet{qi_etal-2022} developed an efficient human-machine collaborative framework to minimize annotation costs through inference. This framework leverages statistical techniques similar to those in~\cite{gao_etal-2019} and combines inference mechanisms akin to those in~\cite{ojha_talukdar-2017}. Their work focuses on the interplay between human annotations and inference mechanisms rather than on \acp{CI} and their impact on the minimization problem. Therefore, we do not include their method in our experimental analysis, but the $a$HPD method can be integrated into \citet{qi_etal-2022}'s framework to enhance efficiency.