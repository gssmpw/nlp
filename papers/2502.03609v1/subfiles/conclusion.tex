



\section{Conclusion}
We have proposed \OTCP, a new approach that can leverage a recently proposed formulation for multivariate quantiles that uses optimal transport theory and optimal transport map estimators. We show the theoretical soundness of this approach, but, most importantly, demonstrate its applicability throughout a broad range of tasks compiled by \citep{dheur2025multioutputconformalregressionunified}. Compared to similar baselines that either use a conditional mean regression estimator (\MergeCP), or more involved quantile regression estimators (\MCP), \OTCP\, shows overall superior performance, while incurring, predictably, a higher train / calibration time cost. The challenges brought forward by the estimation of OT maps in high dimensions~\citep{chewi2024statistical} require being particularly careful when tuning entropic regularization and grid size. However, we show that there exists a reasonable setting for both of these parameters that delivers good performance across most tasks.


\begin{figure}
    \centering
    \includegraphics[width=\linewidth]{figures/fig1_big_region_size.pdf}
    \vskip-.5cm
    \caption{As in \ref{fig:small-region}, we report mean and standard errors for region size (log scale) across 10 different seeds for larger datasets. We keep the same parameters and importantly $\varepsilon = 0.1$ and $2^{15}=32768$ points in the uniform target measure. We expect the performance of \OTCP\, to decrease with dimensionality, but it does provide a convincing alternative to the other approaches.}
    \label{fig:big-region}
\end{figure}

\begin{figure}
    \centering
    \includegraphics[width=\linewidth]{figures/taxi.pdf}
    \caption{Conformal sets recovered by mapping back the reduced sphere on the Manhattan map, in agreement with Equation~\ref{eq:transport_oracle_coverage}, on a prediction for the \texttt{taxi} dataset. We use the inverse entropic map mentioned in Section~\ref{subsec:entropic}, mapping back the gridded sphere of size $m=2^{15}$ for each level, and plotting its outer contour.}
    \label{fig:taxi}
\end{figure}

