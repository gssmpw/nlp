

While the tightness of the sequential fat-shattering dimension is already known, and the tightness of the minimax regret was established in \cref{sec:tightness_gen}, we emphasize that the results were derived for a hypothesis class being induced by the set of all $n$-qubit quantum states. In this section, we consider a more restrictive setting where the hypothesis class is induced solely by the set of all $n$-qubit pure states. The derivation closely follows the techniques developed in the previous section with a notable distinction: whereas the previous sections relied on matrix completion results (without defining the states $\boldsymbol{\omega}(\boldsymbol{\epsilon})$ explicitly), here we shall construct the states $\boldsymbol{\omega}(\boldsymbol{\epsilon}) = \vonneu{\psi(\boldsymbol{\epsilon})}$ explicitly. We show that the bounds on the sequential fat-shattering dimension and consequently the minimax regret remain almost tight in this setting. This result highlights that, irrespective of whether the hypothesis class is induced by pure or mixed states, one can achieve the same minimax regret in both cases.


\begin{theorem} \label{theo:final_pure}
    Let $\mathcal{X} = \{E\in\mathrm{Herm}_{\mathbb{C}}(2^n), \operatorname{Spec}(E)\subset[0,1]\}$ be the sample space. Define $\mathcal{H}=\{\operatorname{Tr}_\omega, \omega\in\mathcal C_n, \mathrm{Tr}[\omega^2]=1\}$ as the hypothesis class, where $\mathcal{C}_n$ is the set of all $n$-qubit quantum states. Then, for $\delta=2^{-\frac n\eta}$, we have $\text{sfat}_\delta(\mathcal H, \mathcal{X})=\Omega(\frac n{\delta^\eta}), \forall\eta < 2$.
\end{theorem}

We prove this result in \cref{pf:final_pure}.

\begin{remark}
    The lower bounds presented in both \cref{thm:final,theo:final_pure} hold only if $\delta$ decays to zero with the right asymptotic rate with respect to $n$. This condition does not pose a problem as it is enough to conclude the optimality of the known upper bound on the sequential fat-shattering dimension. However, one might still seek to derive a more general lower bound (one with a different or potentially no dependence between $\delta$ and $n$). In such cases it might not be feasible to explicitly construct the states $\boldsymbol{\omega} (\boldsymbol{\epsilon})$ as done in \cref{theo:final_pure}. In such situations we conjecture that the techniques from \cref{proof:final} involving matrix completion would serve as a valuable foundation.
\end{remark}
Now, building on the result in \cref{theo:final_pure}, we proceed to demonstrate the tightness of the minimax regret $\mathcal{V}_T$:

\begin{corollary}\label{cor:bis}
    Let $\mathcal{X} = \{E\in\mathrm{Herm}_{\mathbb{C}}(2^n), \operatorname{Spec}(E)\subset[0,1]\}$ be the sample space. Define $\mathcal{H}=\{\operatorname{Tr}_\omega, \omega\in\mathcal C_n, \mathrm{Tr}[\omega^2]=1\}$ as the hypothesis class, where $\mathcal{C}_n$ is the set of all $n$-qubit quantum states. Then we have $\mathcal{V}_T=\Omega(\sqrt{nT})$, assuming the loss function under consideration is the $L_1$-loss.
\end{corollary}
We prove this result in \cref{pf:cor}.

An important aspect to notice here is that the significance of the sequential fat-shattering dimension extends beyond regret analysis. Several corollaries follow from \cref{theo:final_pure}. One such example is the fact that Theorem 1 from \citet{aaronson2019online}, which has been shown to be optimal for mixed states, is also optimal for pure states.

\begin{corollary}
\label{cor:mistake_bds}
Let $\rho$ be an $n$-qubit mixed state, and let $E_1, E_2, \dots$ be a sequence of 2-outcome measurements that are revealed to the learner one by one, each followed by a value $b_t \in [0,1]$ such that $\lvert \mathrm{Tr}(E_t \rho) - b_t \rvert \leq \varepsilon/3$. Then there is an explicit strategy for outputting hypothesis states $\omega_1, \omega_2, \dots$ such that $\lvert \mathrm{Tr}(E_t \omega_t) - \mathrm{Tr}(E_t \rho) \rvert > \varepsilon$ for at most $O\left(\frac{n}{\varepsilon^2}\right)$ values of $t$. This mistake bound is almost asymptotically optimal, even if we restrict $\rho$ to be a pure state.
\end{corollary}

