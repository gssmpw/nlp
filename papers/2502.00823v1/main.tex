\documentclass{article}
\newcommand{\vect}[1]{| #1 \rangle}
\newcommand{\meas}[2]{| #1 \rangle \langle #2 | + | #2 \rangle \langle #1 |}
\newcommand{\vonneu}[1]{| #1 \rangle \langle #1 |}
\usepackage{physics} % for \ket
\usepackage{stmaryrd} % For \llbracket and \rrbracket
\usepackage{tikz}
\usetikzlibrary{positioning}
\usepackage{forest}

\usepackage[authoryear]{natbib} % For author-year citations
\usepackage{microtype}
\usepackage{graphicx}
\usepackage{subfigure}
\usepackage{booktabs} % For professional tables
\usepackage{hyperref}
\usepackage[capitalize,noabbrev]{cleveref} % Must come after hyperref
\usepackage{arxiv} % Using updated single-column arxiv.sty

% Theorems
\usepackage{amsmath,amssymb,mathtools,amsthm}
\newtheorem{theorem}{Theorem}[section]
\newtheorem{proposition}[theorem]{Proposition}
\newtheorem{lemma}[theorem]{Lemma}
\newtheorem{corollary}[theorem]{Corollary}
\newtheorem{definition}[theorem]{Definition}
\newtheorem{assumption}[theorem]{Assumption}
\newtheorem{remark}[theorem]{Remark}


% Document start
\begin{document}

\arxivtitle{Online Learning of Pure States is as Hard as Mixed States}

\arxivauthor{Maxime Meyer$^{1}$ \quad Soumik Adhikary$^{2}$ \quad Naixu Guo$^{2}$ \quad Patrick Rebentrost$^{2,3}$}

\arxivaffiliation{$^1$Department of Mathematics, National University of Singapore, 117543, Singapore \\
$^2$Centre for Quantum Technologies, National University of Singapore, 117543, Singapore \\
$^3$School of Computing, National University of Singapore, 117543, Singapore}

\arxivcorrespondingauthor{Maxime Meyer}{e1124885@u.nus.edu}

\begin{abstract}
Quantum state tomography, the task of learning an unknown quantum state, is a fundamental problem in quantum information. In standard settings, the complexity of this problem depends significantly on the type of quantum state that one is trying to learn, with pure states being substantially easier to learn than general mixed states. A natural question is whether this separation holds for any quantum state learning setting. In this work, we consider the online learning framework and prove the surprising result that learning pure states in this setting is as hard as learning mixed states. More specifically, we show that both classes share almost the same sequential fat-shattering dimension, leading to identical regret scaling under the $L_1$-loss. We also generalize previous results on \textit{full} quantum state tomography in the online setting to learning only partially the density matrix, using \textit{smooth analysis}.
\end{abstract}

\section{Introduction}\label{sec:intro}
Learning information from an unknown quantum state is a fundamental task in quantum physics. Given several perfect copies of an $N$-dimensional quantum state $\rho \in \{\mathrm{Herm}_{\mathbb{C}} (N)\mid \ \operatorname{Tr}(\rho) = 1, \ \rho \succeq 0\}$, quantum full state tomography seeks to reconstruct the complete matrix representation of $\rho$ via measurements. It has a wide range of practical applications, including tasks such as characterizing qubit states for superconducting circuits \citep{PhysRevLett.100.247001}, nitrogen-vacancy (NV) centers in diamond \citep{doi:10.1126/science.1189075}, and verifying successful quantum teleportation \citep{Bouwmeester_1997}. The ease of learning a quantum state is often characterized by the sample complexity, \textit{i.e.} the number of independent copies of the quantum state required for an accurate reconstruction. For a general density matrix $\rho$, the sample complexity scales as $\tilde{\Theta}(N^3)$ for incoherent measurements. However, if the state is known to be pure, the sample complexity can be improved to $\tilde{\Theta}(N)$ \citep{kueng2014lowrankmatrixrecovery, harrow16sampletomography, chen2023doesadaptivityhelpquantum} . Nevertheless, for $n$-qubit states, where $N = 2^n$, this scaling implies an exponential dependence on the number of qubits.


However, for many practical problems, such as certification of a quantum device \citep{Gross_2010, Flammia_2011, eisert2020quantumcertification} and property estimation for quantum chemistry \citep{Huang_2021, Wu2023overlappedgrouping, zhang2023composite, guo2024estimatingpropertiesquantumstate, Miller_2024}, only partial information about the quantum state is needed. To address such cases, quantum PAC learning, also known as pretty good tomography, was introduced by \citet{aaronson2007learnability}. In this framework, a fixed distribution $\mathcal{D}$ governs the selection of two-outcome measurements $E$, which are represented by $N\times N$ Hermitian matrices with eigenvalues in $[0,1]$. The learner then tries to output a hypothesis state $\omega$ for which $\operatorname{Tr} (E \rho) \approx \operatorname{Tr} (E \omega)$ with high probability. The number of measurements required to output this hypothesis state was shown to scale only linearly with the number of qubits $n$, yielding an exponential speed-up over full-state tomography. However, a key limitation of both these approaches is that they do not account for adversarial environments, where the set of realizable measurements may evolve over time.
 


This limitation can be circumvented by generalizing to the online learning setting \citep{aaronson2019online, chen2020practicaladaptivealgorithmsonline,Chen2024adaptive}, where learning a quantum state $\rho$ is posed as a $T$-round repeated two-player game. In each round $t \in [T]$, Nature -- also called the adversary -- chooses a measurement $E_t$ from the set of two-outcome measurements. The task of the learner is to predict the value $\mathrm{Tr}(E_t\rho)$ by selecting a hypothesis $\omega_t$ and computing $\mathrm{Tr}(E_t\omega_t)$ based on previous results. Thereafter, Nature returns the loss, a metric quantifying the difference between the prediction and the true value. The most adversarial scenario arises when the measurement at each round is chosen from the set of all two-outcome measurements without any constraints, \textit{i.e.}, it can be chosen adversarially and adaptively. It has been shown that in such cases, a learner can output a hypothesis state which incurs an additional $O(\sqrt{nT})$ loss compared to the best possible hypothesis state after $T$ rounds of the game \citep{aaronson2019online}. This measure of how much worse a learner performs compared to the best possible strategy in hindsight is called regret and serves as a fundamental measure of performance for any online learning problem.


Given the clear separation in sample complexity between pure and mixed state tomography \citep{kueng2014lowrankmatrixrecovery, harrow16sampletomography}, we ask the central question that this work aims to address:

\begin{center}
\textit{Is there any separation between online learning of pure and mixed quantum states?}
\end{center}


\noindent In this work, we show that the answer is \textbf{No} when comparing regrets with respect to the $L_1$ loss. We believe that this result is surprising. Indeed, not only is there a significant difference in sample complexities between learning pure and mixed states in the standard tomographic settings, but this distinction also holds in specific online learning scenarios. For instance, under bandit feedback with adaptive measurements, \citet{Lumbreras_2022, lumbreras2024learningpurequantumstates} shows that the regret for mixed states grows exponentially faster than for pure states.

Our proof is based on the analysis of the sequential fat-shattering dimension of those states. Informally, it can be seen as the minimum number of mistakes -- defined as errors exceeding a threshold $\delta$ -- that a learner must make before successfully learning a quantum state $\rho$ against a perfect adversary. It has been demonstrated that both upper and lower bounds on regret can be expressed in terms of this dimension \citet{rakhlin2015sequential}. Therefore, it suffices to study the bounds of the latter, for the specific cases of pure and mixed state learning, to address the question posed above. Prior work \citep{aaronson2019online} established tight bounds on the sequential fat-shattering dimension of general mixed states. In this work, we establish lower bounds on the sequential fat-shattering dimension -- and consequently on the regret -- of several subclasses of quantum states, the most important of which is the class of pure states. Our approach employs a distinct proof strategy, providing new insights and extending naturally to various subsettings of the online quantum state learning problem. As a result, we show that the regret for online learning of both pure and mixed quantum state scales as $\Theta(\sqrt{nT})$.

A key feature of the online learning setting considered here is the adversary's ability to select measurements in a completely unconstrained manner. As a result, this setting may effectively incur full state tomography, even in cases where only partial information about the state is needed.  We therefore introduce the concept of smoothness for online learning of quantum states. Smoothed analysis was first introduced in \citet{spielman2004} as a tool that allows for interpolation between the worst and the average case analysis. Later, \citet{Haghtalab2024} extended this concept to the online setting, where the degree of adversariality is quantified by a smoothness parameter $\sigma \in [0,1]$. The particular value $\sigma = 1$ corresponds to $\mathrm{i.i.d.}$ inputs, while the limit $\sigma \rightarrow 0$ corresponds to fully adversarial inputs. In this work, we establish an upper bound on regret for smooth online learning of quantum states, providing insights into the effect of adversariality on regret scaling.


\subsection{Structure of the paper}

We start with a formal description of online learning and its application towards quantum state learning in \cref{sec:preliminaries}. In particular, we focus on regret and sequential fat-shattering dimension as fundamental measures of performance for any online learning problem. In \cref{sec:sfat_intro}, we proceed to derive bounds on sequential fat-shattering dimension, and hence regret, for several subsettings of quantum state learning. We use the techniques developed in this section to prove our main result in \cref{sec:main}, where we show that online learning of pure states is as hard as mixed states. In \cref{sec:smooth}, we extend our analysis to a smooth version of online learning of quantum states and derive an upper bound on the associated regret. Finally, we conclude the paper in \cref{sec:concl}.


\section{Related Works}\label{sec:related}

\textbf{Pure and mixed state tomography:}
In full state tomography of an $N$ dimensional quantum state $\rho$, the goal is to reconstruct its complete classical representation given several independent copies. The associated sample complexity refers to the number of copies required to obtain a classical description of $\rho$ up to an accuracy  $\epsilon$. It has been shown that the sample complexity up to trace distance $\epsilon$ is $\tilde{\Theta}(N r^2/\epsilon^2)$ for incoherent measurements \citep{harrow16sampletomography, kueng2014lowrankmatrixrecovery, chen2023doesadaptivityhelpquantum}, where $r$ is the rank of the density matrix $\rho$.
This result highlights a fundamental separation in the sample complexities between pure and mixed state tomography, as the rank of pure states is $r=1$.
Given the fundamental nature of this separation in quantum information science, one might expect it to persist in the online learning setting.
In fact, \citep{Lumbreras_2022} studies the online learning of properties of quantum states under bandit feedback with adaptive measurements, obtaining a regret scaling of $\Theta(\sqrt{T})$ for mixed states.
Subsequently, \citep{lumbreras2024learningpurequantumstates} improved this result to $\Theta(\mathrm{polylog}\ T)$ for pure states with rank-$1$ projective measurements, showing an exponential separation.
In contrast, our results demonstrate that, in the general online learning setting, the regret scaling for pure and mixed states is identical.


\textbf{Existing bounds:} \citet{aaronson2019online} showed that the sequential fat-shattering dimension of quantum states with parameter $\delta$ is tight, of order $\Theta(\frac{n}{\delta^2})$. He also uses a result from \citet{arora2012multiplicative} to show that the regret is tight for the $L_1$ loss in the non-realizable case (that is when the data isn't assumed to come from an actual quantum state), being of order $\Theta(\sqrt{nT})$. In this paper, we generalize, although non-constructively, the tightness of regret to the realizable case in \cref{cor:tight_reg_gen}. We also provide lower bounds for sequential fat-shattering dimension for several restricted settings, notably proving almost tightness for pure states in \cref{theo:final_pure}.

\section{Preliminaries}\label{sec:preliminaries}

\subsection{Online Learning}

Online learning, or the sequential prediction model, is a $T$ round repeated two-player game \citep{online_book}. In each round $t \in [T]$ of the game, the learner is presented with an input from the sample space $x_t \in \mathcal{X}$. Without any loss of generality, we can assume that $x_t$ is sampled from a distribution $\mathcal{D}_t(\mathcal{X})$ where $\mathcal{D}_t$ may be chosen adversarially. The learner's goal is to learn an unknown function $f: \mathcal{X} \rightarrow \mathcal{Y}$ from the data they receive. Here, $\mathcal{Y}$ denotes the space of possible labels for each input $x_t$. The learning proceeds by designing an algorithm that outputs a sequence of functions $h_t:\mathcal{X} \rightarrow \mathcal{Y}$ chosen from a hypothesis class $\mathcal H$. After each round, the learner incurs a loss and aims to minimize the cumulative regret at the end of all $T$ rounds of the game \citep{cesa1997_expert, arora2012multiplicative}:

\begin{definition}[Regret]
\label{def:regret}
    Let $\mathcal{X}$ denote the sample space, $\mathcal{Y}$ its associated label space, and $\mathcal{H}$ the hypothesis class. In each round $t \in [T]$ of the online learning process, the learner incurs a loss $\ell_t (h_t(x_t), y_t)$, where $y_t \in \mathcal{Y}$ is the true label associated to $x_t$. The regret is then defined as:
    \begin{equation*}
    R_T = \sum_{t=1}^T \ell_t (h_t (x_t), y_t) - \inf_{h \in \mathcal{H}} \sum_{t=1}^T \ell_t(h(x_t), y_t).
    \end{equation*}
\end{definition}

Note here that the hypothesis class $\mathcal{H}$ may or may not contain the target function $f$. For a given pair $(\mathcal{H}, \mathcal{X})$, if the regret grows sublinearly in T, we have
\begin{equation}
    \lim_{T \rightarrow \infty} \  \frac{1}{T}R_T = 0,
\end{equation}
{\it i.e.} the online decision is as good as the offline decision asymptotically.
We say a problem is online learnable for such cases. An alternate figure of merit that is often considered in the literature is the minimax regret \citep{rakhlin2015online}.

\begin{definition}[Minimax regret]
    Consider the online learning setting described in \cref{def:regret}. Let $\mathcal{P}$ and $\mathcal{Q}$ be sets of probability measures defined on $\mathcal{X}$ and $\mathcal{H}$ respectively. The minimax regret is then defined as:
\begin{equation*}
    \label{eq:minmax_reg}
    \begin{aligned}
     \mathcal{V}_T = &\Big< \inf_{\mathcal{Q} \in \Delta(\mathcal{H})} \ \sup_{y_t} \  \sup_{\mathcal{D}_t \in \mathcal{P}} \  \mathop{\mathbb{E}}_{h_t \sim \mathcal{Q}} \  \mathop{\mathbb{E}}_{x_t \sim \mathcal{D}_t} \Big>_{t=1}^T \\&\Big[ \sum_{t=1}^T \ell_t (h_t (x_t), y_t) - \inf_{h \in \mathcal{H}} \sum_{t=1}^T \ell_t(h(x_t), y_t) \Big],
     \end{aligned}
\end{equation*}
    where $\big< \cdot \big>_{t=1}^T$ denotes iterated application of the enclosed operators. 
\end{definition}
The question of learnability of an online learning problem can then be reduced to the study of $\mathcal{V}_T$. Given a pair $(\mathcal{H}, \mathcal{X})$, a problem is said to be online learnable if and only if $\lim_{T \rightarrow \infty} \mathcal{V}_T/T = 0$.

\subsection{Online Learning of Quantum States}

An $n$-qubit quantum state can be written as a density matrix $\rho \in \{\omega \in\mathrm{Herm}_{\mathbb{C}} (2^n); \ \operatorname{Tr}(\omega) = 1, \ \omega \succeq 0\}$. In the context of quantum state learning, the sample space $\mathcal{X}$ is composed of two-outcome quantum measurements, represented by two-element positive operator-valued measure (POVM) $\{E, \mathbf{1} - E\}$, where $E\in\mathrm{Herm}_{\mathbb{C}}(2^n)$ and $\operatorname{Spec}(E) \subset[0, 1]$. Since the second element of the POVM is uniquely determined by the first, a two-outcome measurement can effectively be represented by a single operator $E$. Accordingly, we define the sample space as $\mathcal{X} \subset \{E\in\mathrm{Herm}_{\mathbb{C}}(2^n), \operatorname{Spec}(E) \subset[0, 1]\}$.

A measurement $E$  is said to {\it accept} a quantum state $\rho$ with probability $\operatorname{Tr}(E\rho)$ and {\it reject} it with probability $1 - \operatorname{Tr}(E\rho)$. For a given quantum state $\rho$, predicting its acceptance probabilities for all measurements $E$ is tantamount to characterizing it completely. Hence learning a quantum state $\rho$ is equivalent to learning the function $\operatorname{Tr}_\rho \colon \mathcal X \rightarrow [0,1]$, defined as $\operatorname{Tr}_\rho (E) = \operatorname{Tr}(E \rho)$. Therefore, denoting $\mathcal C_n$ as the set of all $n$-qubit quantum states, we  set the hypothesis class to be $\mathcal{H}_n=\{\operatorname{Tr}_\omega, \omega\in\mathcal C_n\}$, and $\mathcal Q$ can be seen as a distribution over $\mathcal C_n$. Here, the learner receives a sequence of measurements  $(E_t)_{t \in [T]}$, each drawn from a distribution $\mathcal{D}_t$ (chosen adversarially) one at a time. Upon receiving each measurement, the learner selects a hypothesis $\omega_t \in \mathcal{C}_n$ and thereby incurs a loss of $\ell_t (\operatorname{Tr}_{\omega_t} (E_t), \operatorname{Tr}_{\rho} (E_t)) = \ell_t (\operatorname{Tr} (E_t\omega_t ), \operatorname{Tr}(E_t\rho ))$. Drawing parallels to \cref{sec:related}, $\operatorname{Tr}(E_t\rho )$ can be seen as the label associated to each measurement $E_t$, leading to the label space $\mathcal Y =[0,1]$. Note that for a fixed state $\rho$, the label $\operatorname{Tr} (E_t \rho)$ is completely determined by $E_t$, assuming ideal situations where $\operatorname{Tr} (E_t \rho)$ can be determined perfectly. Under this scenario, we introduce a slightly modified version of the minimax regret than the one considered in \cref{eq:minmax_reg} as the figure of merit:
\begin{equation}
    \label{eq:minmax_reg_no_label}
    \begin{aligned}
         \Bar{\mathcal{V}}_T = &\Big< \inf_{\mathcal{Q} \in \Delta(\mathcal{C}_n)} \ \sup_{\mathcal{D}_t \in \mathcal{P}} \   \mathop{\mathbb{E}}_{{\omega_t} \sim \mathcal{Q}} \  \mathop{\mathbb{E}}_{E_t \sim \mathcal{D}_t} \Big>_{t=1}^T \\
         &\Big[ \sum_{t=1}^T \ell_t (\operatorname{Tr} (E_t\omega_t ), \operatorname{Tr}(E_t\rho )) \\&- \inf_{\omega \in \mathcal{C}_n} \sum_{t=1}^T \ell_t (\operatorname{Tr} (E_t\omega ), \operatorname{Tr}(E_t\rho )) \Big].
    \end{aligned}
\end{equation}
The key difference here is that the label associated to each $E_t$ is not chosen adversarially as was the case in \cref{eq:minmax_reg}. Nevertheless, if estimating $\operatorname{Tr} (E_t \rho)$ is only approximate (due to the presence of noise or finite number of measurement shots), one can still use $\mathcal{V}_T$ as the figure of merit with $y_t \in [0,1]$. It is easy to see that $\Bar{\mathcal{V}}_T \leq \mathcal{V}_T$. We say that the problem is online learnable if, for any adversarially chosen sequence of measurements  $(E_t)_{t\in [T]}$, there exists a strategy $(\mathcal{Q}_t)_{t \in [T]}$ for which the minimax regret grows sub-linearly with respect to $T$.

\subsection{Sequential fat-shattering dimension}

The main notion we will be studying in this paper is that of sequential fat-shattering dimension. 

\begin{definition}[Sequential fat-shattering dimension]

A $\mathcal{X}$-valued complete binary tree $\mathbf{x}$ of depth $T$ is deemed to be $\delta$-shattered by a hypothesis class $\mathcal{H}$ if there exists a $\mathbb{R}$-valued complete binary tree $\mathbf{v}$ of same depth $T$ such that for all paths $\ \boldsymbol{\epsilon} \in \{\pm 1\}^{T-1}$,
\begin{equation*}
     \exists \ h \in \mathcal{H} \ : \ \forall t \in [T] \ \ \epsilon_t [h(\mathbf{x}_t(\boldsymbol{\epsilon})) - \mathbf{v}_t (\boldsymbol{\epsilon})] \geq \frac{\delta}{2}.
\end{equation*}
The sequential fat-shattering dimension at scale $\delta$, $\text{sfat}_\delta (\mathcal{H}, \mathcal{X})$, is defined to be the largest $T$ for which $\mathcal{H}$ $\delta$-shatters a $\mathcal{X}$-valued tree of depth $T$.
\end{definition}
Recall that a $\mathcal{X}$-valued complete binary tree of depth $T$, $\mathbf{x}$, is defined as a sequence of $T$ mappings $(\mathbf{x}_1, \mathbf{x}_2, \cdots, \mathbf{x}_T)$, where $\mathbf{x}_t: \{\pm 1\}^{t-1} \rightarrow \mathcal{X}$, with a constant function $\mathbf{x}_1 \in \mathcal{X}$ as the root. In simpler terms the tree can be seen as a collection of $T$ length paths $\boldsymbol{\epsilon} = (\epsilon_1, \epsilon_2, \cdots, \epsilon_{T-1}) \in \{\pm1\}^{T-1}$ ($+1$ indicating right and $-1$ indicating left from any given node) and $\mathbf{x}_t(\boldsymbol{\epsilon}) \equiv \mathbf{x}_t (\epsilon_1, \cdots, \epsilon_{t-1}) \in \mathcal{X}$ denoting the label of the $t$-th node on the corresponding path $\mathbf{\epsilon}$. 

This dimension is a fundamental property in online learning, as it both upper and lower bounds regret \citep{rakhlin2015online, rakhlin2015sequential}, as shown in \cref{eq:fin_class_bnd,eq:Regret_LB}.
\begin{equation}
\label{eq:fin_class_bnd}
\begin{aligned}
    \mathcal{V}_T \leq & \inf_{\alpha > 0} \Big\{ 4 \alpha T L - \\ & 12 L \sqrt{T} \int_{\alpha}^1 \sqrt{\text{sfat}_\delta (\mathcal{H}, \mathcal{X}) \log \left(\frac{2 e T}{\delta} \right)} d\delta \Big\}.
\end{aligned}
\end{equation}
Note that this upper bound also holds for $\Bar{\mathcal{V}}_T$. The bound in \cref{eq:fin_class_bnd} was used in \citet{aaronson2019online} to derive the regret upper bounds for online quantum state learning. Similarly, the minimax regret can also be lower bounded by the sequential fat-shattering dimension, provided that $\ell_t (h_t(x_t), y_t) = \vert h_t(x_t) - y_t \vert$ and that $\mathcal{P}$ is taken to be the whole set of all distributions on $\mathcal{X}$. 
\begin{equation}
\label{eq:Regret_LB}
    \mathcal{V}_T \geq \frac{1}{4 \sqrt{2}} \sup_{\delta > 0} \Big\{ \sqrt{\delta^2 T \min\{ \text{sfat}_\delta (\mathcal{H}, \mathcal{X}), T\}} \Big\}.
\end{equation}


\section{Lower bounds for online learning of quantum states}\label{sec:sfat_intro}


In this section, we employ a distinct proof strategy than \citet{aaronson2019online} to obtain lower bounds on sequential fat-shattering dimension -- recall that he proved that $\text{sfat}_\delta (\mathcal{H}_n, \mathcal{X})=\Theta(\frac{n}{\delta^2})$. This allows us to extend those bounds naturally to various subsettings of the online quantum state learning problem,  characterized by a restricted hypothesis class $\mathcal{H}\subset\mathcal H_n$ and a constrained sample space $\mathcal X$. Such settings frequently arise in practical applications, where the focus is on characterizing specific subsets of quantum states. Furthermore, experimental constraints often limit the implementable set of measurement operations. We derive bounds on $\text{sfat} (\mathcal{H}, \mathcal{X})$ for several such practically relevant subsettings, leading up to the most general formulation of the learning problem. In addition, we extend the tightness of minimax regret in the non-realizable case $\mathcal{V}_T=\tilde \Theta(\sqrt{nT})$ \citep{aaronson2019online} to the realizable case.


In our approach, we explicitly construct an $\mathcal{X}$-valued complete binary tree $\mathbf{x}$ of depth $T(\delta, n)$ and prove that it is $\delta$-shattered by $\mathcal{H}$. That is we will define an explicit $[0,1]$-valued complete binary tree $\mathbf{v}$ of same depth $T(\delta, n)$, and a function $\boldsymbol \omega:\{\pm 1\}^{T-1}\rightarrow{\mathcal C}_n$ such that for all paths $\ \boldsymbol{\epsilon} \in \{\pm 1\}^{T-1}$,
\begin{equation}
     \forall t \in [T(\delta,n)] \ \ \epsilon_t [\operatorname{Tr}_{\boldsymbol \omega(\boldsymbol \epsilon)}(\mathbf{x}_t(\boldsymbol{\epsilon})) - \mathbf{v}_t (\boldsymbol{\epsilon})] \geq \frac{\delta}{2}.
\end{equation}
This directly implies that $\text{sfat}_\delta (\mathcal{H}, \mathcal{X}) = \Omega(T(\delta, n))$.
% While proving this, we will also provide lower bounds for several simplified settings, when we further restrict the quantum states studied, or the measurements that we are allowed to implement. Not only will this allow to slowly build up to the final proof, it will also show regret lower bounds for several interesting quantum state learning problems.

To set the stage for the following sections, we first establish a few notations: since we will frequently consider pure states, the quantum state $\boldsymbol\omega (\boldsymbol{\epsilon})$ will be denoted by its associated vector $\ket{\psi (\boldsymbol{\epsilon})}$, where $\boldsymbol\omega (\boldsymbol{\epsilon})=\ket{\psi(\boldsymbol{\epsilon})} \bra{\psi(\boldsymbol{\epsilon})}$. Furthermore, we define $N=2^n$  to be the dimension of the Hilbert space under consideration. Additionally, we will denote the pair of binary trees $(\mathbf{x}, \mathbf{v})$ by a single tree $\mathbf{T}$. We call $\mathbf{x}$ the $\mathcal{X}$-valued part of $\mathbf{T}$, and $\mathbf{v}$ the real-valued part of $\mathbf{T}$.

\subsection{Learning with respect to a single measurement}

We start with the learning problem of estimating the expectation value of an unknown $n$-qubit quantum state with respect to a fixed measurement operator $E$. This learning problem is key to practical tasks such as quantum state discrimination and hypothesis testing \citep{barnett2008quantumstatediscrimination, Bae_2015}.
Formally, we take $\mathcal{X}=\{E\}$ to be the sample space and keep $\mathcal{H}_n$ as the hypothesis class. As mentioned previously, we are focused on providing a lower bound on $\text{sfat}_\delta(\mathcal{H}_n, \mathcal{X})$, which could then be used to lower bound $\mathcal{V}_T$. We achieve this by constructing what we call the Halving tree $\mathbf{T}_h$, which has a constant $\mathcal X$-valued part, and a real-valued part as shown in \cref{fig:halving_tree}. Every halving tree $\mathbf{T}_h [i, T]=(\mathbf x,\mathbf v)$ will thus be entirely determined by its depth $T$ and the constant measurement $\vonneu i$. The name follows from the distinctive structure exhibited by the real part of the tree $\mathbf{T}_h [i, T]$, as shown in \cref{fig:halving_tree}. This construction will prove useful as a crucial building block for establishing the regret bounds in more general settings (see \cref{theo:vnh,thm:final,theo:final_pure}). 

\begin{figure}[h]
    \centering
    \begin{forest}
    for tree={
        math content, % Ensures math mode inside nodes
        align=center, % Center-aligns the text inside the nodes
        s sep=0.45mm, % Further reduce width between siblings
        l sep=0.6mm, % Further reduce height between levels
        inner sep=1mm, % Reduce padding inside nodes
    }
    [$\frac12$
        [$\frac14$
            [$\frac18$
                [$\vdots$
                    [$\frac1{2^T}$]
                    [$\frac3{2^T}$]
                ]
                [$\vdots$
                    [$\cdots$, no edge]]
            ]
            [$\frac38$
                [$\vdots$
                     [$\cdots$, no edge]]
                [$\vdots$
                    [$\cdots$, no edge]]
            ]
        ]
        [$\frac34$
            [$\frac58$
                [$\vdots$
                    [$\cdots$, no edge]]
                [$\vdots$
                     [$\cdots$, no edge]]
            ]
            [$\frac78$
                [$\vdots$
                    [$\cdots$, no edge]]
                [$\vdots$
                    [$\frac{2^T-3}{2^T}$]
                    [$\frac{2^T-1}{2^T}$]
                ]
            ]
        ]
    ]
    \end{forest}
    \caption{Real valued part of a $T$-depth halving tree $\mathbf{T}_h$, up to a constant multiplicative factor $\frac1N$.}
    \label{fig:halving_tree}
\end{figure}


\begin{theorem}\label{theo:single_meas}
   Let $E\in\mathrm{Herm}_{\mathbb{C}}(2^n), \operatorname{Spec}(E)\subset[0,1]$ be a fixed measurement, and $\mathcal{X}=\{E\}$ be the sample space.
    Let $\mathcal{H}_n=\{\operatorname{Tr}_\omega, \omega\in\mathcal C_n\}$ be the hypothesis class, where $\mathcal{C}_n$ is the set of all $n$-qubit quantum states. Then we have $\text{sfat}_\delta(\mathcal H_n, \mathcal{X})=\Omega(\log_2(\frac1\delta))$.
\end{theorem}

We prove this result in \cref{pf:single_meas}.

\begin{remark}
    Note that the lower bound on the fat-shattering dimension obtained above is independent of $n$, and therefore still holds if the hypothesis class is induced by 1-qubit pure states.
\end{remark}




\subsection{Learning uniform superposition states}
\label{sec:vn}


In the previous section, the focus was on learning the expectation value of a single measurement. We now shift our attention to a harder setting. Consider the sample space $\mathcal{X}$ consisting of the $N$ measurements corresponding to an orthogonal basis of $\mathcal C_n$. The hypothesis class will be induced by the uniform superpositions of basis states.
Such states play an important role in fundamental quantum algorithms \citep{365701, Shor_1997, Grover_1997}, and for quantum random number generators \citep{Mannalatha_2023}.
We now provide a lower bound to the sequential fat-shattering dimension for this specific setting. We accomplish this by constructing what we call the Von Neumann tree $\mathbf{T}_{vn}$.
The name follows from the fact that, while its real-valued part is constant, all nodes in the $\mathcal{X}$-valued part of $\mathbf{T}_{vn}$ are labeled by Von Neumann measurements as shown in \cref{fig:vn_tree}.
\begin{figure}[h]
    \centering
    \begin{forest}
    for tree={
        math content, % Ensures math mode inside nodes
        align=center, % Center-aligns the text inside the nodes
        s sep=0.45mm, %  Width
        l sep=6mm, % Height
        inner sep=1mm, % Increase the padding inside the nodes
    }
    [$\vonneu{0}$
        [$\vonneu{1}$
            [$\vdots$
                [$\vonneu{T-1}$]
                [$\cdots$]
            ]
            [$\vdots$
                [$\cdots$, no edge]]
        ]
        [$\vonneu{1}$
            [$\vdots$
                [$\cdots$, no edge]]
            [$\vdots$
                [$\cdots$]
                [$\vonneu{T-1}$]
            ]
        ]
    ]
    \end{forest}
    \caption{The $\mathcal{X}$-valued part of the Von Neumann tree $\mathbf{T}_{vn}$.}
    \label{fig:vn_tree}
\end{figure}


\begin{theorem}\label{theo:uniform}
    Let $(\vect {0},...,\vect{N-1})$ be an orthogonal basis of $\mathcal C_n$, where $\mathcal{C}_n$ is the set of all $n$-qubit quantum states. Denote the sample space as $\mathcal{X}=\{\vonneu{i},i\in\llbracket 0, N-1 \rrbracket\}$. Let $\mathcal{H}=\{\operatorname{Tr}_\omega,\omega=\frac1{\sqrt{\vert I\vert}}\sum_{i\in I}\vect i, I\subset\llbracket0,N-1\rrbracket\}$ be the hypothesis class. Then we have $\text{sfat}_\delta(\mathcal H, \mathcal{X})=\Omega(\min(\frac1\delta,2^n))$.

\end{theorem}

We prove this result in \cref{pf:uniform}.

\subsection{Learning general states using Von Neumann measurements}


Building on the setting established in the previous section, we consider the sample space $\mathcal{X}$ consisting of the $N$ measurements corresponding to an orthogonal basis of $\mathcal C_n$. The hypothesis class in the present setting is however induced by the set of all pure quantum states. The corresponding learning problem involves a learner estimating the expectation values of an unknown $n$-qubit quantum state with respect to $N$ measurement operators, where the hypothesis is chosen from the set of all $n$-qubit pure quantum states.
Related problems have been considered, for example, in \citet{zhao2023provablelearningquantumstates}.

We now provide a lower bound to the sequential fat-shattering dimension for this specific setting. We accomplish this by constructing what we call the Von Neumann Halving tree $\mathbf{T}_{vnh}$, which is constructed by combining $\mathbf{T}_h$ and $\mathbf{T}_{vn}$ as shown in \cref{fig:vn_halving_tree}. Our construction illustrates the utility of $\mathbf{T}_h$, demonstrating its effectiveness as a tool to multiply existing lower bounds by a factor of $n$.
\begin{figure}[h]
    \centering
    \begin{forest}
    for tree={
        math content, % Ensures math mode inside nodes
        align=center, % Center-aligns the text inside the nodes
        s sep=5mm, %  Width
        l sep=10mm, % Height
        % inner sep=5mm, % Increase the padding inside the nodes
    }
    [$\vonneu{0}$
        [$\vonneu{1}$, edge label={node[midway,left] {$\mathbf{T}_h [0, T]$}}
            [$\vdots$, edge label={node[midway,left] {$\mathbf{T}_h [1, T]$}}
                [$\vonneu{N-2}$, edge label={node[midway,left] {$\mathbf{T}_h [N-3, T]$}}
                    [, edge label={node[midway,left] {$\mathbf{T}_h [N-2, T]$}}]
                ]
            ]
        ]
    ]
    \end{forest}
    \caption{Von Neumann Halving Tree}
    \label{fig:vn_halving_tree}
\end{figure}
\begin{theorem}
\label{theo:vnh}
Let $(\vect {0},...,\vect{N-1})$ be an orthogonal basis of $\mathcal C_n$ and $\mathcal{X}=\{\vonneu{i},i\in\llbracket 0, N-1 \rrbracket\}$ be the sample space. Let $\mathcal{H}=\{\operatorname{Tr}_\omega, \omega \in \mathcal{C}_n, \operatorname{Tr}(\omega^2) = 1\}$ be the hypothesis class. Here $\mathcal{C}_n$ is the set of all $n$-qubit quantum states. Then,  for $\delta=2^{-\frac n\eta}$, we have $\text{sfat}_\delta(\mathcal H_n, \mathcal{X})=\Omega(\frac n{\delta^\eta}) \ \forall\eta<1$.


\end{theorem}

We prove this result in \cref{pf:vnh}.

\subsection{Tightness of sequential fat-shattering dimension of quantum states}
\label{sec:tightness_gen}

Having considered several restricted settings in the previous sections, we now turn to the most general formulation of the online quantum state learning problem. Formally, we define the sample space to be the space of all 2-outcome measurements $\mathcal{X}$, while the Hypothesis class is given as $\mathcal{H}_n$. Our objective is to derive a tight lower bound on $\text{sfat}_\delta (\mathcal{H}_n, \mathcal{X})$. To accomplish this, we will use the techniques developed in the previous sections. In addition, we will establish new results on completion of partial matrices, which are essential for our analysis. We start with the latter. 

Let us first introduce a few necessary definitions on partial matrices and their completions. A partial matrix is a matrix in which certain entries are specified while the other entries are free to be chosen. It is called partial symmetric if it is symmetric on the specified entries. And a completion of a partial matrix refers to a specific assignment of values to its unspecified entries. 



\begin{theorem}\label{lemma_completion}
    Any real partial symmetric matrix $\omega$ satisfying the following conditions
    \begin{enumerate}
        \item $w_{11}=\frac{1}{2}$, and $w_{ii}=\frac{1}{2(N-1)}$ $\forall i\in\llbracket2,N\rrbracket$,
        \item Elements are specified on the set $\{w_{1i}, w_{i1}\}$, where $i\in [N]$,
        % Only the following elements are specified: $\{w_{1j}, w_{i1}\}$.
        \item $\forall i\in\llbracket2,N\rrbracket,|w_{1i}|\leq\frac1{2\sqrt{N-1}}$,
    \end{enumerate}
can be completed to a density matrix.
\end{theorem}
We prove this result in \cref{pf:lemma_completion}.
\begin{figure}[h]
    \centering
    % \includegraphics[width=0.5\linewidth]{}
\[
% \text{part} (w_{12}, w_{13}, \cdots, w_{1N}) =
\begin{bmatrix}
\frac12 & w_{12} & \cdots & \cdots & w_{1N} \\
w_{12} & \frac1{2(N-1)} & ? & \cdots & ? \\
\vdots & ? & \frac1{2(N-1)} & \cdots & ? \\
\vdots & \vdots & \vdots & \ddots & \vdots \\
w_{1N} & ? & ? & \cdots & \frac1{2(N-1)}
\end{bmatrix}
\]
\caption{General form of the partial matrix described in \cref{lemma_completion}. The interrogation marks denote the unspecified entries in the matrix.}
    \label{fig:part_matrix}
\end{figure}
Any partial matrix of the form shown in \cref{fig:part_matrix} shall be denoted as $\text{part} (w_{12}, w_{13}, \cdots, w_{1N})$.

We now derive the lower bound for $\text{sfat}_\delta (\mathcal{H}_n, \mathcal{X})$. The key idea is to construct a new tree analogous to the Von Neumann halving tree, with a crucial distinction that it accommodates more general measurements, extending beyond the $\vonneu i$ type measurements that have been considered thus far. Furthermore, given this tree, we apply \cref{lemma_completion} to ensure that all paths on the tree can be associated to a valid density matrix. We show that this allows us to obtain the quadratic dependence on $\frac{1}{\delta}$ in the lower bound of $\text{sfat}_\delta (\mathcal{H}_n, \mathcal{X})$, thus establishing the almost tightness.
\begin{theorem}\label{thm:final}
    Let $\mathcal{X} = \{E\in\mathrm{Herm}_{\mathbb{C}}(2^n), \operatorname{Spec}(E)\subset[0,1]\}$ be the sample space.
    Define $\mathcal{H}_n=\{\operatorname{Tr}_\omega, \omega\in\mathcal C_n\}$ as the hypothesis class, where $\mathcal{C}_n$ is the set of all $n$-qubit quantum states. Then, for $\delta=2^{-\frac n\eta}$, we have $\text{sfat}_\delta(\mathcal H_n, \mathcal{X})=\Omega(\frac n{\delta^\eta}), \forall\eta < 2$.
\end{theorem}
We prove this result in \cref{pf:final}.
\begin{remark}

    As mentioned in the beginning of this section, our method for lower bounding the sequential fat-shattering dimension differs from that employed in \citet{aaronson2007learnability}, and is adaptable to various restricted online quantum state learning settings. For cases involving further restrictions, whether on the sample space, the hypothesis class or the relationship between $\delta$ and $n$ (Recall that \citet{aaronson2007learnability} required $\delta\geq\sqrt{n2^{-(n-5)/35}/8}$) we conjecture that the techniques from \cref{proof:final} involving matrix completion would serve as a valuable foundation. 


\end{remark}

% \end{proof}
Building on this result, we proceed to establish the tightness of the minimax regret $\mathcal{V}_T$ with the $L_1$-loss. While the bound has already been shown to be tight in the non-realizable case \citep{arora2012multiplicative, aaronson2019online}, we extend the tightness result in the realizable setting.



\begin{corollary}
\label{cor:tight_reg_gen}
    Let $\mathcal{X} = \{E\in\mathrm{Herm}_{\mathbb{C}}(2^n), \operatorname{Spec}(E)\subset[0,1]\}$ be the sample space.
    Define $\mathcal{H}_n=\{\operatorname{Tr}_\omega, \omega\in\mathcal C_n\}$ as the hypothesis class, where $\mathcal{C}_n$ is the set of all $n$-qubit quantum states. Then we have $\mathcal{V}_T=\Omega(\sqrt{nT})$, assuming the loss function under consideration is the $L_1$-loss.
\end{corollary}
We prove this result in \cref{pf:cor}.


\section{Online learning of pure state is as hard as mixed state}\label{sec:main}


While the tightness of the sequential fat-shattering dimension is already known, and the tightness of the minimax regret was established in \cref{sec:tightness_gen}, we emphasize that the results were derived for a hypothesis class being induced by the set of all $n$-qubit quantum states. In this section, we consider a more restrictive setting where the hypothesis class is induced solely by the set of all $n$-qubit pure states. The derivation closely follows the techniques developed in the previous section with a notable distinction: whereas the previous sections relied on matrix completion results (without defining the states $\boldsymbol{\omega}(\boldsymbol{\epsilon})$ explicitly), here we shall construct the states $\boldsymbol{\omega}(\boldsymbol{\epsilon}) = \vonneu{\psi(\boldsymbol{\epsilon})}$ explicitly. We show that the bounds on the sequential fat-shattering dimension and consequently the minimax regret remain almost tight in this setting. This result highlights that, irrespective of whether the hypothesis class is induced by pure or mixed states, one can achieve the same minimax regret in both cases.


\begin{theorem} \label{theo:final_pure}
    Let $\mathcal{X} = \{E\in\mathrm{Herm}_{\mathbb{C}}(2^n), \operatorname{Spec}(E)\subset[0,1]\}$ be the sample space. Define $\mathcal{H}=\{\operatorname{Tr}_\omega, \omega\in\mathcal C_n, \mathrm{Tr}[\omega^2]=1\}$ as the hypothesis class, where $\mathcal{C}_n$ is the set of all $n$-qubit quantum states. Then, for $\delta=2^{-\frac n\eta}$, we have $\text{sfat}_\delta(\mathcal H, \mathcal{X})=\Omega(\frac n{\delta^\eta}), \forall\eta < 2$.
\end{theorem}

We prove this result in \cref{pf:final_pure}.

\begin{remark}
    The lower bounds presented in both \cref{thm:final,theo:final_pure} hold only if $\delta$ decays to zero with the right asymptotic rate with respect to $n$. This condition does not pose a problem as it is enough to conclude the optimality of the known upper bound on the sequential fat-shattering dimension. However, one might still seek to derive a more general lower bound (one with a different or potentially no dependence between $\delta$ and $n$). In such cases it might not be feasible to explicitly construct the states $\boldsymbol{\omega} (\boldsymbol{\epsilon})$ as done in \cref{theo:final_pure}. In such situations we conjecture that the techniques from \cref{proof:final} involving matrix completion would serve as a valuable foundation.
\end{remark}
Now, building on the result in \cref{theo:final_pure}, we proceed to demonstrate the tightness of the minimax regret $\mathcal{V}_T$:

\begin{corollary}\label{cor:bis}
    Let $\mathcal{X} = \{E\in\mathrm{Herm}_{\mathbb{C}}(2^n), \operatorname{Spec}(E)\subset[0,1]\}$ be the sample space. Define $\mathcal{H}=\{\operatorname{Tr}_\omega, \omega\in\mathcal C_n, \mathrm{Tr}[\omega^2]=1\}$ as the hypothesis class, where $\mathcal{C}_n$ is the set of all $n$-qubit quantum states. Then we have $\mathcal{V}_T=\Omega(\sqrt{nT})$, assuming the loss function under consideration is the $L_1$-loss.
\end{corollary}
We prove this result in \cref{pf:cor}.

An important aspect to notice here is that the significance of the sequential fat-shattering dimension extends beyond regret analysis. Several corollaries follow from \cref{theo:final_pure}. One such example is the fact that Theorem 1 from \citet{aaronson2019online}, which has been shown to be optimal for mixed states, is also optimal for pure states.

\begin{corollary}
\label{cor:mistake_bds}
Let $\rho$ be an $n$-qubit mixed state, and let $E_1, E_2, \dots$ be a sequence of 2-outcome measurements that are revealed to the learner one by one, each followed by a value $b_t \in [0,1]$ such that $\lvert \mathrm{Tr}(E_t \rho) - b_t \rvert \leq \varepsilon/3$. Then there is an explicit strategy for outputting hypothesis states $\omega_1, \omega_2, \dots$ such that $\lvert \mathrm{Tr}(E_t \omega_t) - \mathrm{Tr}(E_t \rho) \rvert > \varepsilon$ for at most $O\left(\frac{n}{\varepsilon^2}\right)$ values of $t$. This mistake bound is almost asymptotically optimal, even if we restrict $\rho$ to be a pure state.
\end{corollary}



\section{Smoothed online learning}\label{sec:smooth}


The online learning framework discussed so far is fully adversarial, since the adversary is free to select any measurement at each round $t$. However, as seen in \cref{sec:intro}, the learner often aims to learn specific properties of $\rho$ rather than reconstructing it entirely in practical scenarios. In the PAC learning framework, such properties are captured by a fixed distribution $\mathcal D$ over the set of two-outcome measurements. We apply smooth analysis to extend these restrictions to the online setting, imposing the condition that the distributions $\mathcal{D}_t$ chosen by the adversary at every round must remain close to the original distribution $\mathcal D$.


\begin{definition}[Smooth distributions]
    A distribution $\mu$ is said to be $\sigma$-smooth with respect to a fixed distribution $\mathcal{D}$ for a $\sigma \in (0,1]$ if and only if \citep{smooth_distrib_def}:
\begin{enumerate}
    \item $\mu$ is absolutely continuous with respect to $\mathcal{D}$, {\it i.e.} every measurable set $A$ such that $\mathcal{D}(A) = 0$ satisfies $\mu(A) = 0$

    \item The Radon Nikodym derivative $d\mu/d\mathcal{D}$ satisfies the following relation:
\begin{equation}
    \text{ess} \sup \frac{d \mu}{d \mathcal{D}} \leq \frac{1}{\sigma}.
\end{equation}
\end{enumerate}
\end{definition}








Let $\mathcal{B}(\sigma, \mathcal{D})$ be the set of all $\sigma$-smooth distributions with respect to $\mathcal{D}$. In smoothed online learning, the adversary is restricted by the condition $\mathcal{D}_t \in \mathcal{B} (\sigma, \mathcal{D})$. Note that in this setting we recover the case of an oblivious adversary for $\sigma = 1$, while we get the completely adversarial case for $\sigma \rightarrow 0$. To establish regret bounds in smoothed online learning, \citet{Haghtalab2024, block2022smoothed} introduced the concept of coupling. The key idea here is that if the distributions $(\mathcal{D}_t)_{t=1}^T$ are $\sigma$-smooth with respect to $\mathcal{D}$, we may pretend that in expectation the data is sampled i.i.d from $\mathcal{D}$ instead of $(\mathcal{D}_t)_{t=1}^T$. For a more formal description, define $\mathcal{B}_T (\sigma, \mathcal{D})$ to be the set of joint distributions $\mathcal{D}_\wedge$ on $\mathcal{X}^T$, where each marginal distribution $\mathcal{D}_t(\cdot \vert x_1,...,x_{t-1})$ is conditioned on the previous draws. 

\begin{definition}[Coupling]
\label{def:coupling}
    A distribution $\mathcal{D}_\wedge \in \mathcal{B}_T (\sigma, \mathcal{D})$ is said to be coupled to independent random variables drawn according to $\mathcal{D}$ if there exists a probability measure $\Pi$ with random variables $(x_t, Z_t^j)_{t \in [T], j \in [k]}\sim\Pi$ satisfying the following conditions:
    \begin{enumerate}
        \item $x_t \sim \mathcal{D}_t (\cdot \vert x_1. x_2, \cdots, x_{t-1})$,
    
        \item $\{ Z_t^j \}_{t \in [T], j \in [k]} \sim \mathcal{D}^{\otimes kT}$,
    
        \item With probability at least $1 - T e^{-\sigma k}$, we have $x_t \in \{ Z_t^j \}_{ j \in [k]} \ \forall t \in [T]$. 
    \end{enumerate}
\end{definition}


The last relation is particularly interesting and is used to derive the regret bounds for smoothed online quantum state learning.


\subsection{Smooth online learning of quantum states}

Recall the expression of minimax regret in \cref{eq:minmax_reg}. In the smoothed setting, the expression gets slightly modified accounting for the restriction imposed on the adversary:
\begin{equation}
\begin{aligned}
    \label{eq:minmax_reg_smooth}
     \mathcal{V}_T =& \Big< \inf_{\mathcal{Q} \in \Delta(\mathcal{H})} \  \sup_{\mathcal{D}_t \in \mathcal{B}(\sigma, \mathcal{D})} \  \mathop{\mathbb{E}}_{h_t \sim \mathcal{Q}} \  \mathop{\mathbb{E}}_{x_t \sim \mathcal{D}_t} \Big>_{t=1}^T \\&\Big[ \sum_{t=1}^T \ell_t (h_t (x_t)) - \inf_{h \in \mathcal{H}} \sum_{t=1}^T \ell_t(h(x_t)) \Big].
     \end{aligned}
\end{equation}
Here, the key difference with \cref{eq:minmax_reg} is that $\mathcal{D}_t$ is now restricted to the set of all $\sigma$-smooth distributions with respect to $\mathcal{D}$ instead of all possible distributions on $\mathcal{X}$. We recall that for quantum state learning, we have $\mathcal{X} \subset \{E\in\mathrm{Herm}_{\mathbb{C}}(2^n), \operatorname{Spec}(E)\subset[0,1]\}$, $\mathcal{H}_n=\{\operatorname{Tr}_\omega, \omega\in\mathcal C_n\}$ and a target state $\rho$. For the sake of brevity, we will continue using the notation $x_t$ to indicate input data and $h$ to indicate the hypothesis. The derivation here closely follows the approach in \citet{block2022smoothed} (which derives the regret bounds for classical smoothed online supervised learning)  with one important difference; in the original derivation, for a given input $x_t \in \mathcal{X}$, the authors distinguish between a predicted label $\hat{y}_t \in \mathcal{Y}$ and $h_t(x_t) \in \mathcal{Y}$ where $\mathcal{Y}$ is the label space. We do not make this distinction as our labels are always related to our inputs via the hypothesis. 

 

\begin{theorem}
\label{theo:smooth_ub}
 Let $\mathcal{X} = \{E\in\mathrm{Herm}_{\mathbb{C}}(2^n), \operatorname{Spec}(E)\subset[0,1]\}$ be the sample space. Define $\mathcal{H}=\{\operatorname{Tr}_\omega, \omega\in\mathcal C_n, \mathrm{Tr}[\omega^2]=1\}$ as the hypothesis class, where $\mathcal{C}_n$ is the set of all $n$-qubit quantum states.  Furthermore let $\sigma \in (0,1]$ be the smoothness parameter. Then we have $\mathcal{V}_T = O\Bigg(\sqrt{\frac{nT \log T}{\sigma}} \ \Bigg).$
\end{theorem}

We prove this result in \cref{pf:smooth}.

\section{Conclusion and Future Works}\label{sec:concl}

\textbf{Lower bounds on fat-shattering dimension:} In this work, we established lower bounds on sequential fat-shattering dimension of various subproblems within the quantum state learning framework. Crucially, we showed that pure and mixed states almost share the same asymptotical dimension. Note that, although our construction directly implies that the regular $\delta$-fat-shattering dimension of pure states scales as $\Omega(\frac 1{\delta^2})$, whether we can recover $\Omega(\frac n{\delta^2})$ for pure $n$-qubits in the offline setting remains an open question.  

\textbf{Consequences on regret:} This lower bound on sequential fat-shattering dimension has several implications, including the key result that learning pure and mixed states in the online setting will incur the same asymptotical regret for the $L_1$-loss. However, there are no known tight lower bounds on regret for more general loss functions. Additionally, sequential fat-shattering dimension may serve as a fundamental tool for deriving bounds on other key complexity measures in quantum state learning.

\textbf{Smooth online learning:} Finally, we extend our analysis from standard online learning of quantum states to the smooth online learning setting. To our knowledge, this work represents the first application of smooth analysis to quantum state learning. In this setting, we establish an upper bound on the regret. However a key open question is whether this bound is tight.

\section*{Acknowledgements}

The work of Maxime Meyer, Soumik Adhikary, Naixu Guo, and Patrick Rebentrost was supported by the National
Research Foundation, Singapore and A*STAR under its CQT Bridging Grant and its Quantum
Engineering Programme under grant NRF2021-QEP2-02-P05.
Authors thank Josep Lumbreras, Aadil Oufkir, and Jonathan Allcock for their insightful discussions and kind suggestions.

\bibliography{references}
\bibliographystyle{plainnat}

\appendix
\onecolumn
\section{Proofs}
\label{sec:appendix}



\end{document}
