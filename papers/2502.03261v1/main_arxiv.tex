\documentclass{article}

% if you need to pass options to natbib, use, e.g.:
%     \PassOptionsToPackage{numbers, compress}{natbib}
% before loading neurips_2021

% ready for submission 
\PassOptionsToPackage{round}{natbib}
% \usepackage[preprint]{neurips_2024}

\usepackage{colm2024_conference}
\colmfinalcopy



\usepackage{hyperref}
\usepackage{xcolor}
% \usepackage{tcolorbox}
\usepackage{algorithmic}
\usepackage{longtable}
\usepackage{wrapfig}

\newtheorem{ex}{Example}[section]
\newcounter{excounter}[section]
\renewcommand{\theexcounter}{\thesection.\arabic{excounter}}
%\usepackage{minted}



\newenvironment{myex}[1][]
  {\stepcounter{excounter}\begin{mybox}{yellow!70!black}\textbf{Example~\theexcounter\ #1}\newline}
  {\end{mybox}}

% to compile a preprint version, e.g., for submission to arXiv, add add the
% [preprint] option:
%     \usepackage[preprint]{neurips_2021}

% to compile a camera-ready version, add the [final] option, e.g.:
%     \usepackage[final]{neurips_2021}

% to avoid loading the natbib package, add option nonatbib:
%    \usepackage[nonatbib]{neurips_2021}

\usepackage[utf8]{inputenc} % allow utf-8 input
\usepackage[T1]{fontenc}    % use 8-bit T1 fonts
\usepackage{hyperref}       % hyperlinks
\usepackage{url}            % simple URL typesetting
\usepackage{booktabs}       % professional-quality tables
\usepackage{amsfonts}       % blackboard math symbols
\usepackage{nicefrac}       % compact symbols for 1/2, etc.
\usepackage{microtype}      % microtypography
\usepackage{subcaption} 
\usepackage[symbol]{footmisc}
\usepackage{tcolorbox}
\tcbuselibrary{listings,breakable}
\usepackage{verbatimbox}
%\usepackage{minted}
%\usepackage[finalizecache,cachedir=.]{minted}
 \usepackage[frozencache,cachedir=.]{minted}
\usepackage{mdframed}

\hypersetup{
    colorlinks,
    % linkcolor={black},
    linkcolor={red!50!black},
    citecolor={blue!60!black},
    urlcolor={blue!80!black}
}


\newcommand\blfootnote[1]{%
  \begingroup
  \renewcommand\thefootnote{}\footnote{#1}%
  \addtocounter{footnote}{-1}%
  \endgroup
}
\colorlet{darkred}{red!60!black}

\usepackage{algorithm}
% \usepackage{algpseudocode}
\usepackage{enumitem}
  \setlist{leftmargin=*}
\usepackage{graphicx}
  \graphicspath{ {./figs/} }
\usepackage{lineno}
  \linenumbers
\usepackage{YK}
\usepackage{wrapfig}  % Required for wrapping figures

%
\setlength\unitlength{1mm}
\newcommand{\twodots}{\mathinner {\ldotp \ldotp}}
% bb font symbols
\newcommand{\Rho}{\mathrm{P}}
\newcommand{\Tau}{\mathrm{T}}

\newfont{\bbb}{msbm10 scaled 700}
\newcommand{\CCC}{\mbox{\bbb C}}

\newfont{\bb}{msbm10 scaled 1100}
\newcommand{\CC}{\mbox{\bb C}}
\newcommand{\PP}{\mbox{\bb P}}
\newcommand{\RR}{\mbox{\bb R}}
\newcommand{\QQ}{\mbox{\bb Q}}
\newcommand{\ZZ}{\mbox{\bb Z}}
\newcommand{\FF}{\mbox{\bb F}}
\newcommand{\GG}{\mbox{\bb G}}
\newcommand{\EE}{\mbox{\bb E}}
\newcommand{\NN}{\mbox{\bb N}}
\newcommand{\KK}{\mbox{\bb K}}
\newcommand{\HH}{\mbox{\bb H}}
\newcommand{\SSS}{\mbox{\bb S}}
\newcommand{\UU}{\mbox{\bb U}}
\newcommand{\VV}{\mbox{\bb V}}


\newcommand{\yy}{\mathbbm{y}}
\newcommand{\xx}{\mathbbm{x}}
\newcommand{\zz}{\mathbbm{z}}
\newcommand{\sss}{\mathbbm{s}}
\newcommand{\rr}{\mathbbm{r}}
\newcommand{\pp}{\mathbbm{p}}
\newcommand{\qq}{\mathbbm{q}}
\newcommand{\ww}{\mathbbm{w}}
\newcommand{\hh}{\mathbbm{h}}
\newcommand{\vvv}{\mathbbm{v}}

% Vectors

\newcommand{\av}{{\bf a}}
\newcommand{\bv}{{\bf b}}
\newcommand{\cv}{{\bf c}}
\newcommand{\dv}{{\bf d}}
\newcommand{\ev}{{\bf e}}
\newcommand{\fv}{{\bf f}}
\newcommand{\gv}{{\bf g}}
\newcommand{\hv}{{\bf h}}
\newcommand{\iv}{{\bf i}}
\newcommand{\jv}{{\bf j}}
\newcommand{\kv}{{\bf k}}
\newcommand{\lv}{{\bf l}}
\newcommand{\mv}{{\bf m}}
\newcommand{\nv}{{\bf n}}
\newcommand{\ov}{{\bf o}}
\newcommand{\pv}{{\bf p}}
\newcommand{\qv}{{\bf q}}
\newcommand{\rv}{{\bf r}}
\newcommand{\sv}{{\bf s}}
\newcommand{\tv}{{\bf t}}
\newcommand{\uv}{{\bf u}}
\newcommand{\wv}{{\bf w}}
\newcommand{\vv}{{\bf v}}
\newcommand{\xv}{{\bf x}}
\newcommand{\yv}{{\bf y}}
\newcommand{\zv}{{\bf z}}
\newcommand{\zerov}{{\bf 0}}
\newcommand{\onev}{{\bf 1}}

% Matrices

\newcommand{\Am}{{\bf A}}
\newcommand{\Bm}{{\bf B}}
\newcommand{\Cm}{{\bf C}}
\newcommand{\Dm}{{\bf D}}
\newcommand{\Em}{{\bf E}}
\newcommand{\Fm}{{\bf F}}
\newcommand{\Gm}{{\bf G}}
\newcommand{\Hm}{{\bf H}}
\newcommand{\Id}{{\bf I}}
\newcommand{\Jm}{{\bf J}}
\newcommand{\Km}{{\bf K}}
\newcommand{\Lm}{{\bf L}}
\newcommand{\Mm}{{\bf M}}
\newcommand{\Nm}{{\bf N}}
\newcommand{\Om}{{\bf O}}
\newcommand{\Pm}{{\bf P}}
\newcommand{\Qm}{{\bf Q}}
\newcommand{\Rm}{{\bf R}}
\newcommand{\Sm}{{\bf S}}
\newcommand{\Tm}{{\bf T}}
\newcommand{\Um}{{\bf U}}
\newcommand{\Wm}{{\bf W}}
\newcommand{\Vm}{{\bf V}}
\newcommand{\Xm}{{\bf X}}
\newcommand{\Ym}{{\bf Y}}
\newcommand{\Zm}{{\bf Z}}

% Calligraphic

\newcommand{\Ac}{{\cal A}}
\newcommand{\Bc}{{\cal B}}
\newcommand{\Cc}{{\cal C}}
\newcommand{\Dc}{{\cal D}}
\newcommand{\Ec}{{\cal E}}
\newcommand{\Fc}{{\cal F}}
\newcommand{\Gc}{{\cal G}}
\newcommand{\Hc}{{\cal H}}
\newcommand{\Ic}{{\cal I}}
\newcommand{\Jc}{{\cal J}}
\newcommand{\Kc}{{\cal K}}
\newcommand{\Lc}{{\cal L}}
\newcommand{\Mc}{{\cal M}}
\newcommand{\Nc}{{\cal N}}
\newcommand{\nc}{{\cal n}}
\newcommand{\Oc}{{\cal O}}
\newcommand{\Pc}{{\cal P}}
\newcommand{\Qc}{{\cal Q}}
\newcommand{\Rc}{{\cal R}}
\newcommand{\Sc}{{\cal S}}
\newcommand{\Tc}{{\cal T}}
\newcommand{\Uc}{{\cal U}}
\newcommand{\Wc}{{\cal W}}
\newcommand{\Vc}{{\cal V}}
\newcommand{\Xc}{{\cal X}}
\newcommand{\Yc}{{\cal Y}}
\newcommand{\Zc}{{\cal Z}}

% Bold greek letters

\newcommand{\alphav}{\hbox{\boldmath$\alpha$}}
\newcommand{\betav}{\hbox{\boldmath$\beta$}}
\newcommand{\gammav}{\hbox{\boldmath$\gamma$}}
\newcommand{\deltav}{\hbox{\boldmath$\delta$}}
\newcommand{\etav}{\hbox{\boldmath$\eta$}}
\newcommand{\lambdav}{\hbox{\boldmath$\lambda$}}
\newcommand{\epsilonv}{\hbox{\boldmath$\epsilon$}}
\newcommand{\nuv}{\hbox{\boldmath$\nu$}}
\newcommand{\muv}{\hbox{\boldmath$\mu$}}
\newcommand{\zetav}{\hbox{\boldmath$\zeta$}}
\newcommand{\phiv}{\hbox{\boldmath$\phi$}}
\newcommand{\psiv}{\hbox{\boldmath$\psi$}}
\newcommand{\thetav}{\hbox{\boldmath$\theta$}}
\newcommand{\tauv}{\hbox{\boldmath$\tau$}}
\newcommand{\omegav}{\hbox{\boldmath$\omega$}}
\newcommand{\xiv}{\hbox{\boldmath$\xi$}}
\newcommand{\sigmav}{\hbox{\boldmath$\sigma$}}
\newcommand{\piv}{\hbox{\boldmath$\pi$}}
\newcommand{\rhov}{\hbox{\boldmath$\rho$}}
\newcommand{\upsilonv}{\hbox{\boldmath$\upsilon$}}

\newcommand{\Gammam}{\hbox{\boldmath$\Gamma$}}
\newcommand{\Lambdam}{\hbox{\boldmath$\Lambda$}}
\newcommand{\Deltam}{\hbox{\boldmath$\Delta$}}
\newcommand{\Sigmam}{\hbox{\boldmath$\Sigma$}}
\newcommand{\Phim}{\hbox{\boldmath$\Phi$}}
\newcommand{\Pim}{\hbox{\boldmath$\Pi$}}
\newcommand{\Psim}{\hbox{\boldmath$\Psi$}}
\newcommand{\Thetam}{\hbox{\boldmath$\Theta$}}
\newcommand{\Omegam}{\hbox{\boldmath$\Omega$}}
\newcommand{\Xim}{\hbox{\boldmath$\Xi$}}


% Sans Serif small case

\newcommand{\Gsf}{{\sf G}}

\newcommand{\asf}{{\sf a}}
\newcommand{\bsf}{{\sf b}}
\newcommand{\csf}{{\sf c}}
\newcommand{\dsf}{{\sf d}}
\newcommand{\esf}{{\sf e}}
\newcommand{\fsf}{{\sf f}}
\newcommand{\gsf}{{\sf g}}
\newcommand{\hsf}{{\sf h}}
\newcommand{\isf}{{\sf i}}
\newcommand{\jsf}{{\sf j}}
\newcommand{\ksf}{{\sf k}}
\newcommand{\lsf}{{\sf l}}
\newcommand{\msf}{{\sf m}}
\newcommand{\nsf}{{\sf n}}
\newcommand{\osf}{{\sf o}}
\newcommand{\psf}{{\sf p}}
\newcommand{\qsf}{{\sf q}}
\newcommand{\rsf}{{\sf r}}
\newcommand{\ssf}{{\sf s}}
\newcommand{\tsf}{{\sf t}}
\newcommand{\usf}{{\sf u}}
\newcommand{\wsf}{{\sf w}}
\newcommand{\vsf}{{\sf v}}
\newcommand{\xsf}{{\sf x}}
\newcommand{\ysf}{{\sf y}}
\newcommand{\zsf}{{\sf z}}


% mixed symbols

\newcommand{\sinc}{{\hbox{sinc}}}
\newcommand{\diag}{{\hbox{diag}}}
\renewcommand{\det}{{\hbox{det}}}
\newcommand{\trace}{{\hbox{tr}}}
\newcommand{\sign}{{\hbox{sign}}}
\renewcommand{\arg}{{\hbox{arg}}}
\newcommand{\var}{{\hbox{var}}}
\newcommand{\cov}{{\hbox{cov}}}
\newcommand{\Ei}{{\rm E}_{\rm i}}
\renewcommand{\Re}{{\rm Re}}
\renewcommand{\Im}{{\rm Im}}
\newcommand{\eqdef}{\stackrel{\Delta}{=}}
\newcommand{\defines}{{\,\,\stackrel{\scriptscriptstyle \bigtriangleup}{=}\,\,}}
\newcommand{\<}{\left\langle}
\renewcommand{\>}{\right\rangle}
\newcommand{\herm}{{\sf H}}
\newcommand{\trasp}{{\sf T}}
\newcommand{\transp}{{\sf T}}
\renewcommand{\vec}{{\rm vec}}
\newcommand{\Psf}{{\sf P}}
\newcommand{\SINR}{{\sf SINR}}
\newcommand{\SNR}{{\sf SNR}}
\newcommand{\MMSE}{{\sf MMSE}}
\newcommand{\REF}{{\RED [REF]}}

% Markov chain
\usepackage{stmaryrd} % for \mkv 
\newcommand{\mkv}{-\!\!\!\!\minuso\!\!\!\!-}

% Colors

\newcommand{\RED}{\color[rgb]{1.00,0.10,0.10}}
\newcommand{\BLUE}{\color[rgb]{0,0,0.90}}
\newcommand{\GREEN}{\color[rgb]{0,0.80,0.20}}

%%%%%%%%%%%%%%%%%%%%%%%%%%%%%%%%%%%%%%%%%%
\usepackage{hyperref}
\hypersetup{
    bookmarks=true,         % show bookmarks bar?
    unicode=false,          % non-Latin characters in AcrobatÕs bookmarks
    pdftoolbar=true,        % show AcrobatÕs toolbar?
    pdfmenubar=true,        % show AcrobatÕs menu?
    pdffitwindow=false,     % window fit to page when opened
    pdfstartview={FitH},    % fits the width of the page to the window
%    pdftitle={My title},    % title
%    pdfauthor={Author},     % author
%    pdfsubject={Subject},   % subject of the document
%    pdfcreator={Creator},   % creator of the document
%    pdfproducer={Producer}, % producer of the document
%    pdfkeywords={keyword1} {key2} {key3}, % list of keywords
    pdfnewwindow=true,      % links in new window
    colorlinks=true,       % false: boxed links; true: colored links
    linkcolor=red,          % color of internal links (change box color with linkbordercolor)
    citecolor=green,        % color of links to bibliography
    filecolor=blue,      % color of file links
    urlcolor=blue           % color of external links
}
%%%%%%%%%%%%%%%%%%%%%%%%%%%%%%%%%%%%%%%%%%%


% \usepackage{todonotes}
\def\logit{\mathrm{logit}}

\newcommand{\my}[1]{\textcolor{orange}{[MY: #1]}}

\title{CARROT:  A Cost Aware Rate Optimal Router}

% The \author macro works with any number of authors. There are two commands
% used to separate the names and addresses of multiple authors: \And and \AND.
%
% Using \And between authors leaves it to LaTeX to determine where to break the
% lines. Using \AND forces a line break at that point. So, if LaTeX puts 3 of 4
% authors names on the first line, and the last on the second line, try using
% \AND instead of \And before the third author name.
\author{%
  Seamus Somerstep$^{\dagger \square}$ \ \ \
  Felipe Maia Polo$^{\dagger  \square}$ \ \ \
  Allysson Flavio Melo de Oliveira $^ {\ddag \circ}$ \\ 
  \textbf{Prattyush Mangal} $^ \ddag$ \ \ \
  \textbf{Mírian Silva} $^ {\ddag \circ \triangle}$  \ \ \
  \textbf{Onkar Bhardwaj} $^ {\ddag \circ}$ \\ \vspace{0.5cm}
  \textbf{Mikhail Yurochkin}* $^ {\ddag \circ}$ \ \ \
  \textbf{Subha Maity}* $^ \spadesuit$ \\ 
\normalsize $^\square$ Department of Statistics, University of Michigan\\
$^\ddagger$ IBM Research  ~~~ $^\circ$ MIT-IBM Watson AI Lab  ~~~
$\triangle$ Federal University of Minas Gerais\\
$^ \spadesuit$ Department of Statistics and Actuarial Science, University of Waterloo 
  % examples of more authors
  % \And
  % Coauthor \\
  % Affiliation \\
  % Address \\
  % \texttt{email} \\
  % \AND
  % Coauthor \\
  % Affiliation \\
  % Address \\
  % \texttt{email} \\
  % \And
  % Coauthor \\
  % Affiliation \\
  % Address \\
  % \texttt{email} \\
  % \And
  % Coauthor \\
  % Affiliation \\
  % Address \\
  % \texttt{email} \\
}

%\author{%
  %Seamus Somerstep$^{\dagger \square}$\\
  %\texttt{smrstep@umich.edu}\\ 
  %\And
  %Felipe Maia Polo$^{\dagger  \square}$\\
  %\texttt{maiapolo@umich.edu}\\
  %\AND
  %Allysson Flavio Melo de Oliveira $^ {\ddag \circ}$ \\
  %\texttt{allysson@br.ibm.com} \\
  %\And
  %Prattyush Mangal $^ \ddag$  \\
  %\texttt{prattyush.mangal@ibm.com} \\
  %\AND
  %Mírian Silva $^ {\ddag \circ \triangle}$  \\
  %\texttt{mirianfrsilva@ibm.com} \\
  %\And
  %Onkar Bhardwaj $^ {\ddag \circ}$  \\
  %\texttt{onkarbhardwaj@ibm.com} \\
  %\AND
  %Mikhail Yurochkin* $^ {\ddag \circ}$  \\
  %\texttt{mikhail.yurochkin@ibm.com} \\
 % \And
  %Subha Maity* $^ \spadesuit$ \\
  %\texttt{smaity@uwaterloo.ca} \\
  % examples of more authors
  % \And
  % Coauthor \\
  % Affiliation \\
  % Address \\
  % \texttt{email} \\
  % \AND
  % Coauthor \\
  % Affiliation \\
  % Address \\
  % \texttt{email} \\
  % \And
  % Coauthor \\
  % Affiliation \\
  % Address \\
  % \texttt{email} \\
  % \And
  % Coauthor \\
  % Affiliation \\
  % Address \\
  % \texttt{email} \\
%}

% \usepackage{setspace}


\begin{document}
\nolinenumbers


\maketitle


% \begin{figure*}
%     \centering
    
    
%     \href{https://huggingface.co/CARROT-LLM-Routing}{https://huggingface.co/CARROT-LLM-Routing}

% \end{figure*}

\vspace{-0.8cm}
\begin{center}
   \large{ %Hugging Face repository:
   \href{https://huggingface.co/CARROT-LLM-Routing}{\textcolor{purple!80!black}{\texttt{huggingface.co/CARROT-LLM-Routing}}}}
   % \large{\url{https://huggingface.co/CARROT-LLM-Routing}}
\end{center}
  % 


% \linenumbers


\begin{abstract}
    \begin{abstract}  
Test time scaling is currently one of the most active research areas that shows promise after training time scaling has reached its limits.
Deep-thinking (DT) models are a class of recurrent models that can perform easy-to-hard generalization by assigning more compute to harder test samples.
However, due to their inability to determine the complexity of a test sample, DT models have to use a large amount of computation for both easy and hard test samples.
Excessive test time computation is wasteful and can cause the ``overthinking'' problem where more test time computation leads to worse results.
In this paper, we introduce a test time training method for determining the optimal amount of computation needed for each sample during test time.
We also propose Conv-LiGRU, a novel recurrent architecture for efficient and robust visual reasoning. 
Extensive experiments demonstrate that Conv-LiGRU is more stable than DT, effectively mitigates the ``overthinking'' phenomenon, and achieves superior accuracy.
\end{abstract}  
\end{abstract}


% {\bf TL;DR:} We present a theoretical and empirical investigation of performance and cost trade-offs in LLM routing. We introduce 

%\bibliographystyle{plainnat}
%\begin{thebibliography}{9}

%\bibitem[Christiano et~al.(2017)]{Christiano2017DeepRLHF}
%Paul Christiano, Jan Leike, Tom Brown, et~al.
%\newblock Deep reinforcement learning from human preferences.
%\newblock In \emph{Advances in Neural Information Processing Systems (NeurIPS)}, 2017.

%\bibitem[Rafailov et~al.(2023)]{Rafailov2023DirectPreference}
%R~Rafailov, Mitchell Wortsman, Ludwig Schmidt, et~al.
%\newblock Direct preference optimization: Your language model is secretly a reward model.
%\newblock \emph{arXiv preprint arXiv:2305.17888}, 2023.

%\bibitem[Von Stackelberg(1934)]{VonStackelberg1934Marktform}
%Heinrich Von Stackelberg.
%\newblock \emph{Marktform und Gleichgewicht}.
%\newblock Julius Springer, 1934.

%\bibitem[Basar and Olsder(1999)]{Basar1999DynamicNG}
%Tamer Basar and Geert~Jan Olsder.
%\newblock \emph{Dynamic Noncooperative Game Theory}.
%\newblock SIAM, 1999.

%\bibitem[Villani(2008)]{Villani2008OptimalTW}
%Cédric Villani.
%\newblock \emph{Optimal transport: old and new}.
%\newblock Springer, 2008.

%\end{thebibliography}

\section{Introduction}
\label{sec:intro}
Recent breakthroughs in large language models (LLMs) have made it increasingly crucial to \emph{align} generated text with human preferences for both usability and safety \citep{Ouyang2022Training,Bai2022Training}. Traditional approaches such as Reinforcement Learning from Human Feedback (RLHF) \citep{Christiano2017Deep} and Direct Preference Optimization (DPO) \citep{Rafailov2023Direct} often require \emph{massive} amounts of meticulously curated preference data. Not only is gathering such a dataset expensive and time-consuming, but any mislabeling can propagate through iterative alignment stages \citep{Casper2023Open}, leading to suboptimal or even unsafe model behaviors. This raises a critical challenge: \emph{How can we achieve preference data-efficient alignment of language models while maintaining robustness to annotation noise?}

From the perspective of data efficiency and robustness, existing alignment approaches often suffer from two main issues:\emph{Self-annotation gaps} and \emph{Lack of equilibrium guarantees under noise}. Recent work explores self-annotation \citep{Lee2024Rlaif,Yuan2024Self,Kim2025Spread}, where an LLM generates labels for new prompt--response pairs instead of relying on humans. While this indeed lowers annotation cost, such methods often treat policy updates and preference annotation as disconnected processes. Consequently, once noisy synthetic preferences are generated, there is limited recourse if the LLM’s self-labels embed systematic biases or errors that can corrupt future training \citep{Chowdhury2024Provably}. Some adversarial training approaches \citep{Cheng2023Adversarial,Wu2024Towards} attempt to counter distributional shifts in preference data, but they often lack formal \emph{equilibrium} guarantees and can lead to unstable optimization cycles in practice. Furthermore, these adversarial approaches are not specifically tailored for data-scarce alignment regimes, thus limiting their applicability when human labels are extremely expensive. \emph{We delay a more thorough \textbf{related work} section in the Appendix~\ref{sec:related_supp}}.

To address these issues, we propose \textbf{Stackelberg Game Preference Optimization (SGPO)}, a framework that models alignment as a \emph{two-player} Stackelberg game between: a \emph{policy} (the leader), which aims to satisfy real human preferences, and an \emph{adversarial preference distribution} (the follower), which explores worst-case shifts within a defined Wasserstein ball of radius $\epsilon$. Drawing inspiration from Stackelberg dynamics \citep{Bacsar1998Dynamic} and optimal transport \citep{Villani2009Optimal}, SGPO ensures that the policy optimizes against the worst plausible shifts in preference data. Specifically, under $\epsilon$-bounded shifts (or annotation noise), we prove that the resulting policy’s regret is at most $\mathcal{O}(\epsilon)$ (see Section~\ref{sec:theory}), whereas standard DPO can incur \emph{linear} regret growth with respect to the magnitude of distribution mismatch. Although our analysis uses $\epsilon$-Wasserstein balls as a tractable model of moderate noise or mismatch, it remains relevant for practical alignment scenarios where annotation errors are not unbounded but still matter.

On top of SGPO, we develop the \textbf{Stackelberg Self-Annotated Preference Optimization (SSAPO)} (Section~\ref{sec:ssapo}) algorithm, a procedure aimed at drastically lowering human annotation needs. SSAPO starts with a small human-labeled seed (about $1/30$ of the usual scale in our experiments) and then: \emph{(1) Self-Annotates} newly sampled prompts by generating responses and extracting winner--loser pairs from the current policy's own comparisons. \emph{(2) Adversarially Reweights} these pairs within a Wasserstein ball of radius $\epsilon$, by solving a distributionally robust optimization (DRO) \citep{Esfahani2018Data} program,ensuring that potentially corrupted or unrepresentative synthetic preferences do not overwhelm the policy update.
By iterating these two steps, SSAPO instantiates the SGPO framework, preserving theoretical bounded-regret guarantees while yielding significant data-efficiency gains. In practice, we find that with only 1/30 of the usual human annotations (from the UltraFeedback dataset \citep{Cui2023Ultrafeedback}), SSAPO attains 35.82~\% GPT4 win-rate (24.44\% LC win-rate) on Mistral-7B, 40.12\% win-rate (33.33\% LC win-rate) on Llama3-8B-instruct. Which matches Mixtral Large (21.4\% win-rate and 32.7\% LC win-rate) and Llama3-70B-instruct (33.2\% winrate and 34.4\% LC win-rate) according to the AlphacaEval~2.0 leaderboard~\citep{dubois2024length}.% Concretely, within 3 self-annotation rounds, our model matches methods that use \textbf{30 times} more human-labeled data.

In summary, We formulate DPO-like alignment as a Stackelberg game, demonstrate \emph{existence} of an equilibrium, and establish that \emph{SGPO} achieves an $\mathcal{O}(\epsilon)$ regret bound under $\epsilon$-bounded noise, in contrast to the linear regret behavior of DPO when facing similarly scaled shifts.(Section~\ref{sec:theory}). We implement SGPO via \emph{SSAPO}, which combines self-annotation with distributionaly robust optimization.(Section~\ref{sec:ssapo}) %We tackle the concavity requirement of Standard DRO with piecewise-linear \emph{concave} envelope. We handle large-scale preference data via a \emph{uniform grouping} strategy for parallel subproblem solutions.  
 Extensive experiments show that SSAPO attains strong alignment performance with substantially fewer human labels, thereby showing its real-world viability for cost-effective preference alignment of LLMs (Section~\ref{sec:experiments}).

%Overall, our main contributions are: (1) \textbf{Theoretical Foundation} (Section~\ref{sec:theory}): We formulate alignment under limited labels as a Stackelberg game, demonstrate \emph{existence} of an equilibrium, and establish that \emph{SGPO} achieves an $\mathcal{O}(\epsilon)$ regret bound under $\epsilon$-bounded noise. This stands in contrast to the linear regret behavior of DPO when facing similarly scaled shifts.(Theorems~\ref{thm:existence_se}--\ref{thm:sgpo_regret}). (2) \textbf{Scalable Algorithm} (Section~\ref{sec:ssapo}):  We implement SGPO via \emph{SSAPO}, which combines self-annotation with distributionaly robust optimization (DRO). We tackle the concavity requirement of Standard DRO with piecewise-linear \emph{concave} envelope. We handle large-scale preference data via a \emph{uniform grouping} strategy for parallel subproblem solutions. (3) \textbf{Empirical Validation} (Section~\ref{sec:experiments}): Extensive experiments show that SSAPO attains strong alignment performance with substantially fewer human labels, thereby showing its real-world viability for cost-effective preference alignment of LLMs.
%By furnishing both formal guarantees and an effective, scalable method, we hope to advance the practice of LLM alignment under realistic labeling budgets.
\newcommand{\tabincell}[2]{\begin{tabular}{@{}#1@{}}#2\end{tabular}}
\newcommand{\rowstyle}[1]{\gdef\currentrowstyle{#1}%
	#1\ignorespaces
}

\newcommand{\className}[1]{\textbf{\sf #1}}
\newcommand{\functionName}[1]{\textbf{\sf #1}}
\newcommand{\objectName}[1]{\textbf{\sf #1}}
\newcommand{\annotation}[1]{\textbf{#1}}
\newcommand{\todo}[1]{\textcolor{blue}{\textbf{[[TODO: #1]]}}}
\newcommand{\change}[1]{\textcolor{blue}{#1}}
\newcommand{\fetch}[1]{\textbf{\em #1}}
\newcommand{\phead}[1]{\vspace{1mm} \noindent {\bf #1}}
\newcommand{\wei}[1]{\textcolor{blue}{{\it [Wei says: #1]}}}
\newcommand{\peter}[1]{\textcolor{red}{{\it [Peter says: #1]}}}
\newcommand{\zhenhao}[1]{\textcolor{dkblue}{{\it [Zhenhao says: #1]}}}
\newcommand{\feng}[1]{\textcolor{magenta}{{\it [Feng says: #1]}}}
\newcommand{\jinqiu}[1]{\textcolor{red}{{\it [Jinqiu says: #1]}}}
\newcommand{\shouvick}[1]{\textcolor{violet(ryb)}{{\it [Shouvick says: #1]}}}
\newcommand{\pattern}[1]{\emph{#1}}
%\newcommand{\tool}{{{DectGUILag}}\xspace}
\newcommand{\tool}{{{GUIWatcher}}\xspace}


\newcommand{\guo}[1]{\textcolor{yellow}{{\it [Linqiang says: #1]}}}

\newcommand{\rqbox}[1]{\begin{tcolorbox}[left=4pt,right=4pt,top=4pt,bottom=4pt,colback=gray!5,colframe=gray!40!black,before skip=2pt,after skip=2pt]#1\end{tcolorbox}}




% \section{Introduction}


\begin{figure}[t]
\centering
\includegraphics[width=0.6\columnwidth]{figures/evaluation_desiderata_V5.pdf}
\vspace{-0.5cm}
\caption{\systemName is a platform for conducting realistic evaluations of code LLMs, collecting human preferences of coding models with real users, real tasks, and in realistic environments, aimed at addressing the limitations of existing evaluations.
}
\label{fig:motivation}
\end{figure}

\begin{figure*}[t]
\centering
\includegraphics[width=\textwidth]{figures/system_design_v2.png}
\caption{We introduce \systemName, a VSCode extension to collect human preferences of code directly in a developer's IDE. \systemName enables developers to use code completions from various models. The system comprises a) the interface in the user's IDE which presents paired completions to users (left), b) a sampling strategy that picks model pairs to reduce latency (right, top), and c) a prompting scheme that allows diverse LLMs to perform code completions with high fidelity.
Users can select between the top completion (green box) using \texttt{tab} or the bottom completion (blue box) using \texttt{shift+tab}.}
\label{fig:overview}
\end{figure*}

As model capabilities improve, large language models (LLMs) are increasingly integrated into user environments and workflows.
For example, software developers code with AI in integrated developer environments (IDEs)~\citep{peng2023impact}, doctors rely on notes generated through ambient listening~\citep{oberst2024science}, and lawyers consider case evidence identified by electronic discovery systems~\citep{yang2024beyond}.
Increasing deployment of models in productivity tools demands evaluation that more closely reflects real-world circumstances~\citep{hutchinson2022evaluation, saxon2024benchmarks, kapoor2024ai}.
While newer benchmarks and live platforms incorporate human feedback to capture real-world usage, they almost exclusively focus on evaluating LLMs in chat conversations~\citep{zheng2023judging,dubois2023alpacafarm,chiang2024chatbot, kirk2024the}.
Model evaluation must move beyond chat-based interactions and into specialized user environments.



 

In this work, we focus on evaluating LLM-based coding assistants. 
Despite the popularity of these tools---millions of developers use Github Copilot~\citep{Copilot}---existing
evaluations of the coding capabilities of new models exhibit multiple limitations (Figure~\ref{fig:motivation}, bottom).
Traditional ML benchmarks evaluate LLM capabilities by measuring how well a model can complete static, interview-style coding tasks~\citep{chen2021evaluating,austin2021program,jain2024livecodebench, white2024livebench} and lack \emph{real users}. 
User studies recruit real users to evaluate the effectiveness of LLMs as coding assistants, but are often limited to simple programming tasks as opposed to \emph{real tasks}~\citep{vaithilingam2022expectation,ross2023programmer, mozannar2024realhumaneval}.
Recent efforts to collect human feedback such as Chatbot Arena~\citep{chiang2024chatbot} are still removed from a \emph{realistic environment}, resulting in users and data that deviate from typical software development processes.
We introduce \systemName to address these limitations (Figure~\ref{fig:motivation}, top), and we describe our three main contributions below.


\textbf{We deploy \systemName in-the-wild to collect human preferences on code.} 
\systemName is a Visual Studio Code extension, collecting preferences directly in a developer's IDE within their actual workflow (Figure~\ref{fig:overview}).
\systemName provides developers with code completions, akin to the type of support provided by Github Copilot~\citep{Copilot}. 
Over the past 3 months, \systemName has served over~\completions suggestions from 10 state-of-the-art LLMs, 
gathering \sampleCount~votes from \userCount~users.
To collect user preferences,
\systemName presents a novel interface that shows users paired code completions from two different LLMs, which are determined based on a sampling strategy that aims to 
mitigate latency while preserving coverage across model comparisons.
Additionally, we devise a prompting scheme that allows a diverse set of models to perform code completions with high fidelity.
See Section~\ref{sec:system} and Section~\ref{sec:deployment} for details about system design and deployment respectively.



\textbf{We construct a leaderboard of user preferences and find notable differences from existing static benchmarks and human preference leaderboards.}
In general, we observe that smaller models seem to overperform in static benchmarks compared to our leaderboard, while performance among larger models is mixed (Section~\ref{sec:leaderboard_calculation}).
We attribute these differences to the fact that \systemName is exposed to users and tasks that differ drastically from code evaluations in the past. 
Our data spans 103 programming languages and 24 natural languages as well as a variety of real-world applications and code structures, while static benchmarks tend to focus on a specific programming and natural language and task (e.g. coding competition problems).
Additionally, while all of \systemName interactions contain code contexts and the majority involve infilling tasks, a much smaller fraction of Chatbot Arena's coding tasks contain code context, with infilling tasks appearing even more rarely. 
We analyze our data in depth in Section~\ref{subsec:comparison}.



\textbf{We derive new insights into user preferences of code by analyzing \systemName's diverse and distinct data distribution.}
We compare user preferences across different stratifications of input data (e.g., common versus rare languages) and observe which affect observed preferences most (Section~\ref{sec:analysis}).
For example, while user preferences stay relatively consistent across various programming languages, they differ drastically between different task categories (e.g. frontend/backend versus algorithm design).
We also observe variations in user preference due to different features related to code structure 
(e.g., context length and completion patterns).
We open-source \systemName and release a curated subset of code contexts.
Altogether, our results highlight the necessity of model evaluation in realistic and domain-specific settings.





\section{The lower bound}\label{sec:lb}
In this section, we provide a sample complexity lower bound proof with regard to the error tolerance $\eps$. In fact, we prove \Cref{thm:main}, which is a formal version of \Cref{thm:main-lb}. %All distributions constructed in this section will satisfy \Cref{assump:moment} and \Cref{assump:smooth}.

% A sampling algorithm here is a procedure with query access to $f$, $\grad f$ or $\grad^2 f$, which outputs a sample point based on a series of queries. We aim to prove the following theorem in this note.
% \htodo{Here $c$ is the constant in \Cref{lem:disjointcap}.}

Recall that we use $\@D_{L,M}$ to denote the collection of distributions which are $L$-log-smooth and have second moment at most $M$. 
\begin{theorem}\label{thm:main}
    % There exist a universal constant $C>0$ such that 
    For any $L,M>0$ satisfying $LM\ge d$ and for any $\eps\in(0,1/32)$, $d\geq 5$, if a sampling algorithm $\+A$ always terminates within 
\[
    \frac{\eps (d-2)^{\frac{3}{2}}}{8}\cdot \tp{\frac{9}{256} \cdot \frac{LM}{d\eps} \cdot \frac{1}{\log \frac{LM}{d\eps}}}^{\frac{d-1}{2}}
    % \frac{\eps}{4}\cdot \tp{\frac{C\cdot LM}{2 d\eps} \cdot \frac{1}{\log \frac{LM}{d\eps}}}^{\frac{d-1}{2}}% \approx \frac{\eps}{4}\exp\set{\frac{d}{2}\cdot \Omega\tp{\log \frac{LM}{d\eps}}}
\]
queries on every input instance in $\@D_{L,M}$, then there must exist some distribution $\mu\in \@D_{L,M}$ such that when the underlying instance is $\mu$, the distribution of $\+A$'s output, denoted as $\tilde \mu$, is $\eps$ away from $\mu$ in total variation distance, i.e., $\DTV(\mu,\tilde \mu)\geq \eps$.
\end{theorem}
%\htodo{Actually we consider those distributions with second moment $O(M)$ and $O(L)$-smooth. Not exactly $M$ and $L$.}

%\mn{When $d$ is even, $\Gamma\tp{\frac{d}{2}+1} = \tp{\frac{d}{2}}!$. When $d$ is odd, $\Gamma\tp{\frac{d}{2}+1} = \frac{\sqrt{\pi}}{2^d}\cdot \frac{d!}{\tp{\frac{d-1}{2}}!}$.}

\subsection{The base instance}

We first construct a base distribution $\mu_0$. Let $R= \tp{\frac{M}{\eps}}^{\frac{1}{2}}$, and let
\[
    \mathfrak{g}_{[\frac{R}{4},\frac{R}{2}]}(x)=q_{\!{mol}}\tp{\frac{ \|x\|^2- \frac{R^2}{16}}{\frac{R^2}{4} - \frac{R^2}{16}}} \quad \mbox{and} \quad \mathfrak{g}_{[R,2R]}(x)=q_{\!{mol}}\tp{\frac{\|x\|^2-R^2}{4R^2-R^2}},
\]
% Let $\alpha\in (0,2)$ be a constant to be determined later.
With constant function $h_1 \equiv \log \tp{\!{vol}(\+B_{3 R})} + \log \frac{1}{\eps}$ and function $h_0(x)=\frac{d\|x\|^2}{2M} + \frac{d}{2}\log \frac{2\pi M}{d}$, define the function $f_0$ as
\[
    \forall x\in \bb R^d, f_0(x) = \begin{cases} h_0(x), & \|x\|\leq \frac{R}{4} \\
    \mathfrak{g}_{[\frac{R}{4},\frac{R}{2}]}(x)\cdot h_1 + \tp{1-\mathfrak{g}_{[\frac{R}{4},\frac{R}{2}]}(x)}\cdot h_0(x), & \frac{R}{4}<\|x\| \leq \frac{R}{2}\\
    h_1, & \frac{R}{2} <\|x\|\leq R\\
    \mathfrak{g}_{[R,2R]}(x)\cdot h_0(x) + \tp{1-\mathfrak{g}_{[R,2R]}(x)}\cdot h_1, & R<\|x\|\leq 2R \\
    h_0(x), & \|x\|>2R
    \end{cases}.
\]

%\mn{Here we use a constant function $h_1$ rather than using Gaussian directly. This is crucial.}
Consider the distribution $\mu_0$ with density $p_{\mu_0} \propto \exp\tp{-f_0(x)}$ and its normalizing factor $Z_0 = \int_{\bb R^d} \exp\tp{-f_0(x)} \d x$.
\begin{lemma}\label{lem:Z_0}
    The normalizing constant $1-16\eps \leq Z_0 \leq  1+\eps$. 
\end{lemma}
\begin{proof}
    On one hand, from Markov's inequality,
    $$
        Z_0\geq \int_{\+B_{\frac{R}{4}}} e^{-f_0(x)} \dd x = \int_{\+B_{\frac{R}{4}}} e^{-h_0(x)} \dd x = 1- \Pr[X\sim \+N\tp{0,\frac{M}{d}\cdot \!{Id}_d}]{\|X\|^2\geq \frac{R^2}{16}} \geq 1-16\eps.
    $$
    On the other hand, 
    $$
        Z_0\leq \int_{\bb R^d} e^{-h_0(x)} \dd x + \vol (\+B_{2R})\cdot e^{-h_1} \leq 1 + \eps\cdot \frac{\vol (\+B_{2R})}{\vol(\+B_{3R})}\leq 1+\eps.
    $$
    
    % Note that $h_1=\log \frac{1}{\eps} + \frac{d}{2}\log\tp{\pi \alpha^2 R^2} - \log \Gamma\tp{\frac{d}{2}+1}$.
    
    % Note that for each fixed $x\in \bb R^d$, $f_0$ is a non-decreasing function wrt $\alpha$. So $Z_0$ is decreasing as $\alpha$ increases. When $\alpha=1$,
\end{proof}

% We fixed the value of $\alpha$ to be the one in \Cref{lem:Z_0}.
\begin{lemma}\label{lem:propertymu0}
    The distribution $\mu_0$ is $\+O(L)$-log-smooth and has second moment $\+O(M)$.
\end{lemma}
%\htodo{$O(M)$ and $O(L)$, not exactly $M$ and $L$} 
\begin{proof}
    We first calculate the second moment of $\mu_0$. We have
    $$
        \E[\mu_0]{\|X\|^2} \leq \frac{\E[X\sim \+N\tp{0,\frac{M}{d}\cdot \!{Id}_d}]{\|X\|^2} + \vol (\+B_{2R})\cdot e^{-h_1}\cdot 4R^2}{Z_0} \leq \frac{M + \eps\cdot \frac{\vol (\+B_{2R})}{\vol (\+B_{3R})}\cdot 4R^2}{Z_0} \leq \frac{3M}{Z_0}\leq 6M,
    $$
    where the last inequality is due to \Cref{lem:Z_0} and the fact that $\eps<\frac{1}{32}$.
    % \htodo{We may need $\eps<\frac{1}{32}$.}

    For the smoothness, we only need to check $\|\grad^2 f_0(x)\|$ for those $x$ with $\|x\|\in (\frac{R}{4},\frac{R}{2}]$ and $\|x\|\in (R,2R]$ since clearly $f_0\in C^2(\bb R^d)$. %\ctodo{need $f_0\in C^2$ here.}

    First, for $\|x\|\in (\frac{R}{4},\frac{R}{2}]$,
    \[
        \grad f_0(x) = \grad \mathfrak{g}_{[\frac{R}{4},\frac{R}{2}]}(x) \cdot (h_1 - h_0(x) ) + (1-\mathfrak{g}_{[\frac{R}{4},\frac{R}{2}]}(x))\cdot \grad h_0(x), 
    \]
    and
    \begin{align*}
        \grad^2 f_0(x) &= \underbrace{\grad^2 \mathfrak{g}_{[\frac{R}{4},\frac{R}{2}]}(x) \cdot (h_1 - h_0(x))}_{\mbox{(a)}} - \underbrace{\tp{\grad \mathfrak{g}_{[\frac{R}{4},\frac{R}{2}]}(x) \cdot \grad h_0(x)^{\top} +  \grad h_0(x)\cdot \grad \mathfrak{g}_{[\frac{R}{4},\frac{R}{2}]}(x)^{\top}}}_{\mbox{(b)}}\\
        &\quad + \underbrace{ \tp{1-\mathfrak{g}_{[\frac{R}{4},\frac{R}{2}]}(x)}\cdot \grad^2 h_0(x)}_{\mbox{(c)}}.
        \end{align*}
    Recall that $LM\geq d$, so it is easy to get $0\preceq \mbox{(c)}\preceq L\cdot \!{Id}_d$. Since 
    \[
        \grad \mathfrak{g}_{[\frac{R}{4},\frac{R}{2}]}(x) = \frac{2x}{\frac{R^2}{4} - \frac{R^2}{16}} \cdot q_{\!{mol}}'\tp{\frac{ \|x\|^2- \frac{R^2}{16}}{\frac{R^2}{4} - \frac{R^2}{16}}},
    \]
    we have
    \[
        \mbox{(b)} = \frac{4d \cdot xx^{\top}}{M\tp{\frac{R^2}{4} - \frac{R^2}{16}}} \cdot q_{\!{mol}}'\tp{\frac{ \|x\|^2- \frac{R^2}{16}}{\frac{R^2}{4} - \frac{R^2}{16}}}.
    \]
    % \mn{For $x,y\in \bb R^d$, we can show that $\|x\|\|y\|\!{Id}_d - xy^T\succeq 0$: for any $z\in \bb R^d$,
    % \begin{align*}
    %     &\phantom{{}={}}z^T\tp{\|x\|\|y\|\!{Id}_d - xy^T}z \\
    %     & =\|x\|\|y\|\|z\|^2 - (z^Tx)(y^Tz)\\
    %     &\geq \|x\|\|y\|\|z\|^2 - \|x\|\|y\|\|z\|^2\\
    %     &= 0.
    % \end{align*}}
    Therefore, $-\+O(L)\cdot\!{Id}_d \preceq \mbox{(b)} \preceq \+O(L)\cdot\!{Id}_d$.
    By direct calculation,
    \[
        \grad^2 \mathfrak{g}_{[\frac{R}{4},\frac{R}{2}]}(x) = \frac{4xx^{\top}}{\tp{\frac{R^2}{4} - \frac{R^2}{16}}^2} \cdot q''_{\!{mol}}\tp{\frac{ \|x\|^2- \frac{R^2}{16}}{\frac{R^2}{4} - \frac{R^2}{16}}} + \frac{2\!{Id}_d}{\frac{R^2}{4} - \frac{R^2}{16}} \cdot q'_{\!{mol}}\tp{\frac{ \|x\|^2- \frac{R^2}{16}}{\frac{R^2}{4} - \frac{R^2}{16}}}
    \]
    and 
    \begin{align}
        \abs{h_1 - h_0(x)} &= \abs{\log \frac{1}{\eps} + \frac{d}{2}\log\tp{9\pi R^2} - \log \Gamma\tp{\frac{d}{2}+1} - \frac{d\|x\|^2}{2M} - \frac{d}{2}\log \frac{2\pi M}{d}} \notag\\
        &= \abs{\log \frac{1}{\eps} + \frac{d}{2}\log\frac{9d}{2\eps} - \log \Gamma\tp{\frac{d}{2}+1} - \frac{d\|x\|^2}{2M}} \label{eq:1}
        % &\leq \log \frac{1}{\eps} + \frac{d}{2}\log\tp{9\pi R^2} - \log \Gamma\tp{\frac{d}{2}+1} + \frac{R^2 d}{8M} + \frac{d}{2}\log \frac{2\pi M}{d}.
    \end{align}
    % \htodo{Here we need $R^2=\frac{M}{\eps}\log \frac{M}{\eps}$?}
    From Stirling's formula, we know that for any $d>0$,
    \[
        \log\sqrt{\pi d} + \frac{d}{2}\log \frac{d}{2e} \leq \log \Gamma\tp{\frac{d}{2}+1}\leq \log\sqrt{\pi d} + \frac{d}{2}\log \frac{d}{2e} + 1.
    \]
    % \[
    %     \log \Gamma\tp{\frac{d}{2}+1} \leq \begin{cases}
    %         \log\sqrt{\pi d} + \frac{d}{2}\log \frac{d}{2e} + 1, & d \mbox{ is even}\\
    %         \log \sqrt{\frac{2\pi d}{d-1}} + \frac{d+1}{2}\log \frac{d}{2e} + 1, & d \mbox{ is odd}
    %     \end{cases},
    % \]
    % and 
    % \[
    %     \log \Gamma\tp{\frac{d}{2}+1} \geq \begin{cases}
    %         \log\sqrt{\pi d} + \frac{d}{2}\log \frac{d}{2e} , & d \mbox{ is even}\\
    %         \log \sqrt{\frac{2\pi d}{d-1}} + \frac{d+1}{2}\log \frac{d}{2e} , & d \mbox{ is odd}
    %     \end{cases}.
    % \]
    Back to \Cref{eq:1}, we have
    \begin{align*}
        \abs{h_1 - h_0(x)}&\leq \abs{\log \frac{1}{\eps} + \frac{d}{2}\log\frac{9e}{\eps} - \frac{d\|x\|^2}{2M} -\log\sqrt{\pi d}} + 1 \\
        &\leq \log \frac{1}{\eps} + \frac{d}{2}\log\frac{9e}{\eps} + \frac{dR^2}{8M} + \log\sqrt{\pi d} +1.
    \end{align*}
    Since $LM\geq d$, we have $-\+O(L)\cdot\!{Id}_d \preceq \mbox{(a)}\preceq \+O(L)\cdot\!{Id}_d$.
    
    For $\|x\|\in (R,2R]$, 
    \[
        \grad^2 f_0(x) = \grad^2 \mathfrak{g}_{[R,2R]}(x)(h_0(x)-h_1) + 2\grad \mathfrak{g}_{[R,2R]}(x) \grad h_0(x)^{\top} + \mathfrak{g}_{[R,2R]}(x) \cdot \grad^2 h_0(x).
    \]
    The remaining calculations are similar. %\ctodo{Maybe say more here.}
\end{proof}


\subsection{Perturb the base instance}\label{sec:hardinstance}
We then construct instances via perturbing $\mu_0$. Let $r_1= \sqrt{\frac{d}{L}\log \frac{LM}{d\eps}}$, $r_2=\sqrt{2}r_1$. Let $h_2\defeq h_1 - \gamma$, where $\gamma$ is a value to be determined later.
% \htodo{Assume the value of $\eps,L,M,d$ satisfy $4r_2\leq R$ and $r_1=\Omega(1)$.}

Note that when $\eps<1/32$, we have $4r_2\leq R$. For a point $v\in \bb R^d$ with $\|v\|=\frac{3R}{4}$, let $\mathfrak{g}_v(x) = q_{\!{mol}}\tp{\frac{\|x-v\|^2-r_1^2}{r_2^2-r_1^2}}$ and $f_v(x)=\mathfrak{g}_v(x)f_0(x)+(1-\mathfrak{g}_v(x))h_2$. This means that, outside the ball $\+B_{r_2}(v)$, $f_v\equiv f_0$, and inside the ball $\+B_{r_1}(v)$, $f_v\equiv h_2$. Define density of the distribution $\mu_v$ over $\bb R^d$ as $p_{\mu_v}\propto e^{-f_v}$. Let $Z_v = \int_{\bb R^d} e^{-f_v(x)} \dd x$.

\begin{lemma}\label{lem:gamma}
    There exists a $\gamma>0$ such that the following holds at the same time:
    \begin{itemize}
        \item  $\int_{\+B_{r_2}(v)} \tp{e^{-f_v(x)} - e^{-h_1}}  \dd x = 9\eps$;
        %\item $9 \tp{\frac{3R}{r_2}}^d \leq e^{\gamma}\leq 18\tp{\frac{3R}{r_1}}^d $;
        % \item $\eps \leq \DTV(\mu_0,\mu_v)\leq 10\eps$;
        \item $\frac{9\eps e^{h_1}}{\!{vol}(\+B_{r_2})}\le e^\gamma-1 \le \frac{9\eps e^{h_1}}{\!{vol}(\+B_{r_1})}$.
        \item $Z_0\leq Z_v\leq 1+10\eps$.
    \end{itemize}
    % There exists a $\gamma$ with $\eps\abs{h_1-\gamma} \leq ()$, such that $\eps \leq \DTV(\mu_0,\mu_v)\leq 10\eps$ for each $\|v\| = \frac{3R}{4}$. This $\gamma$ also satisfies that $Z_0\leq Z_v\leq 1+10\eps$
\end{lemma}

Before proving the lemma, let us examine the information it brings. Recall that $f_0(x)\equiv h_1$ when $x\in \+B_{r_2}(v)$ and we would like to perturb $h_1$ by amount of $\gamma$ to obtain $f_v$ so that there will be $\Theta(\eps)$ more probability mass in $\+B_{r_2}(v)$. For fixed $r_1$ and $r_2$, the lemma says that the order of $\gamma$ is roughly proportional to $h_1$. 

\begin{proof}[Proof of \Cref{lem:gamma}]
    Consider the value $\int_{\+B_{r_2}(v)} \tp{e^{-f_v(x)} - e^{-h_1}}\dd x$. It is continuous and increasing in $\gamma$ when $\gamma\ge 0$. When $\gamma = 0$, $\int_{\+B_{r_2}(v)} \tp{e^{-f_v(x)} - e^{-h_1}}\dd x = 0$. When $\gamma \to \infty $, this value goes to $\infty$. So we can find a $\gamma$ such that $\int_{\+B_{r_2}(v)} \tp{e^{-f_v(x)} - e^{-h_1}}\dd x = 9\eps$ holds exactly.

    For such a $\gamma$, we have
    \[
        Z_v = \int_{\bb R^d} e^{-f_v(x)} \dd x \leq \int_{\bb R^d} e^{-f_0(x)} \dd x + \int_{\+B_{r_2}(v)} \tp{ e^{-f_v(x)} - e^{-h_1}}\dd x = Z_0+9\eps \leq 1+10\eps.
    \]
    Also
    $$
        Z_v = \int_{\bb R^d} e^{-f_v(x)} \dd x \geq \int_{\bb R^d} e^{-f_0(x)} \dd x = Z_0.
    $$
    
    % Then we caculate $\DTV(\mu_0,\mu_v)$. On one hand, since $\+B(v,r_1)\subseteq \+B(v,r_2)\subseteq \set{x\in \bb R^d:\ \|x\| \in (\frac{R}{2},R]}$,
    % \begin{align*}
    %     \DTV(\mu_0,\mu_v) &= \frac{1}{2}\int_{\bb R^d} \abs{\frac{e^{-f_0(x)}}{Z_0} - \frac{e^{-f_v(x)}}{Z_v}} \dd x \\ 
    %     &\geq \frac{1}{2}\int_{\+B(v,r_2)} \abs{\frac{e^{-f_0(x)}}{Z_0} - \frac{e^{-f_v(x)}}{Z_v}} \dd x \\
    %     &=  \frac{1}{2}\int_{\+B(v,r_2)}  \abs{\frac{e^{-f_0(x)}}{Z_v} - \frac{e^{-f_v(x)}}{Z_v}} - \abs{\frac{e^{-f_0(x)}}{Z_0} -  \frac{e^{-f_0(x)}}{Z_v}}  \dd x  \\
    %     &= \frac{1}{2}\tp{\frac{9\eps}{Z_v} - e^{-h_1}\cdot \abs{\frac{1}{Z_0} - \frac{1}{Z_v}} \cdot \vol\tp{\+B(v,r_2)}} \\
    %     &\geq \eps.
    % \end{align*}

    % On the other hand, \begin{align*}
    %     \DTV(\mu_0,\mu_v) &= \frac{1}{2}\int_{\bb R^d} \abs{\frac{e^{-f_0(x)}}{Z_0} - \frac{e^{-f_v(x)}}{Z_v}} \dd x \\ 
    %     &\leq \frac{1}{2}\int_{\bb R^d }  \abs{\frac{e^{-f_0(x)}}{Z_v} - \frac{e^{-f_v(x)}}{Z_v}} + \abs{\frac{e^{-f_0(x)}}{Z_0} -  \frac{e^{-f_0(x)}}{Z_v}}  \dd x \\
    %     &\leq \frac{1}{2}\tp{\frac{9\eps}{Z_v} + Z_0\cdot \abs{\frac{1}{Z_0} - \frac{1}{Z_v}}}\\
    %     &\leq 10\eps.
    % \end{align*}
    
    It remains to calculate $e^\gamma$. We have that
    \[
        \vol\tp{\+B_{r_1}(v)} \cdot e^{-h_1} \tp{e^\gamma - 1}\leq \int_{\+B_{r_2}(v)} \tp{e^{-f_v(x)} - e^{-h_1}}  \dd x =9\eps \leq \vol\tp{\+B_{r_2}(v)} \cdot e^{-h_1} \tp{e^\gamma - 1}.
    \]
\end{proof}

\begin{corollary}\label{cor:gamma-bound}
    For our choice of $h_1$, $r_1$ and $r_2$, it holds that
    \[
        9\tp{\frac{3R}{r_2}}^d \leq e^{\gamma}\leq 18\tp{\frac{3R}{r_1}}^d.
    \]
\end{corollary}
\begin{proof}
    Recall that $e^{h_1}=\eps^{-1}\vol\tp{\+B_{3R}}$ and $\vol\tp{\+B_r} = \frac{\tp{\pi R^2}^{\frac{d}{2}}}{\Gamma\tp{\frac{d}{2}+1}}$. We have
    \[
        e^\gamma-1 \leq  9\tp{\frac{3R}{r_1}}^d \mbox{ and } e^\gamma-1 \geq 9 \tp{\frac{3R}{r_2}}^d.
    \]
    Therefore
    \[
        18\tp{\frac{3R}{r_1}}^d \geq e^\gamma\geq 9 \tp{\frac{3R}{r_2}}^d.
    \]
\end{proof}

\subsection{Properties of the perturbed distributions}
For every $v$ with $\|v\|= \frac{3R}{4}$, we first analyze the smoothness and second moment of the distribution $\mu_v$.

\begin{lemma}\label{lem:moment}
    For $\|v\|=\frac{3R}{4}$, $\E[\mu_v]{\|X\|^2}= \+O\tp{M}$.
\end{lemma}
\begin{proof}
    Direct calculation gives
    \begin{align*}
        \E[\mu_v]{\|X\|^2} & \leq \frac{\E[X\sim \+N\tp{0,\frac{M}{d}\cdot \!{Id}_d}]{\|X\|^2} + \vol (B_{2R})\cdot e^{-h_1}\cdot 4R^2 + \int_{\+B_{r_2}(v)} \tp{e^{-f_v(x)} - e^{-h_1}} \|X\|^2 \dd x}{Z_v} \\
        &\leq \frac{3M + R^2\cdot 9\eps}{Z_v} \leq 24 M.
    \end{align*}
    % \htodo{Here only $\tilde O(M)$. It seems that we cannot guarantee the $O(L)$-smooth of $f_0$ and $O(M)$ second moment of $\mu_v$. Or equivalently, we can choose $M=\frac{M_0}{\log \frac{M_0}{\eps}}$. We require $LM_0\geq d\log\frac{M_0}{\eps}$.}
\end{proof}

\begin{lemma}\label{smooth}
  For $\|v\|=\frac{3R}{4}$, the function $f_v$ is $\+O(L)$-smooth. 
\end{lemma}
\begin{proof}
    We only need to consider those $x\in \+B_{r_2}(v)\setminus \+B_{r_1}(v)$. For such $x$, $f_0(x)=h_1$. Therefore,
    \[
        \grad^2 f_v(x) = \gamma\cdot \grad^2 g_v(x).
    \]
    Note that
    \[
        \grad g_v(x) = \frac{2(x-v)}{r_2^2-r_1^2}\cdot q_{\!{mol}}'\tp{\frac{\|x-v\|^2-r_1^2}{r_2^2-r_1^2}}
    \]
    and
    \[
        \grad^2 g_v(x) = \frac{4(x-v)(x-v)^{\top}}{\tp{r_2^2-r_1^2}^2}\cdot q''_{\!{mol}}\tp{y=\frac{\|x-v\|^2-r_1^2}{r_2^2-r_1^2}} + \frac{2\!{Id}_d}{r_2^2-r_1^2}\cdot q'_{\!{mol}}\tp{\frac{\|x-v\|^2-r_1^2}{r_2^2-r_1^2}}.
    \]
    From \Cref{cor:gamma-bound}, 
    \[
        \gamma \leq \log 18 + \frac{d}{2}\log\frac{9LM}{\eps d} - \frac{d}{2}\log\log \frac{LM}{d\eps}.
    \]
    Therefore, $-\+O(L)\cdot\!{Id}_d \preceq \grad^2 f_v(x) \preceq \+O(L)\cdot\!{Id}_d $.
\end{proof}

We remark that the constants hidden in the $\+O(\cdot)$ in the above two lemmas are universal constants and do not depend on $d$ and $\eps$. 


\begin{lemma}\label{lem:TV}
    For $u,v\in \bb R^d$ such that $\|v\|=\|u\|=\frac{3R}{4}$ and $\+B_{r_2}(u)\cap \+B_{r_2}(v)=\emptyset$, $\DTV\tp{\mu_u,\mu_v}> 4\eps$.
\end{lemma}
\begin{proof}
    By the definition of total variation distance, 
    \begin{align*}
        \DTV(\mu_u,\mu_v) & = \frac{1}{2}\int_{\bb R^d} \abs{\frac{e^{-f_u(x)}}{Z_u} - \frac{e^{-f_v(x)}}{Z_v}} \dd x \\
        \mr{$Z_u=Z_v$} & =\frac{1}{2Z_v} \tp{\int_{\+B_{r_2}(u)}\abs{e^{-f_u(x)} - e^{-h_1}} \dd x + \int_{\+B_{r_2}(v)}\abs{e^{-f_v(x) }- e^{-h_1}} \dd x} \\
        &= \frac{9\eps}{Z_v} > 4\eps.
    \end{align*}
\end{proof}

\subsection{The number of disjoint $\+B_{r_2}(v)$'s}

\begin{lemma}\label{lem:disjointcap}
    Suppose $d\geq 5$. There exist $n=\frac{(d-1)\sqrt{d-2}}{2}\cdot \tp{\frac{3R}{8\sqrt{2}r_2}}^{d-1}$ vectors $v_1,v_2,\dots,v_n \in \bb R^d$ such that
    \begin{itemize}
        \item for each $i\in[n]$, $\|v_i\| = \frac{3R}{4}$;
        \item for each $i,j\in [n]$, if $i\ne j$, then $\+B_{r_2}(v_i)\cap \+B_{r_2}(v_j)=\emptyset$.
    \end{itemize}
\end{lemma}
\begin{proof}
    Let $S$ be the sphere $\set{x\in \bb R^d: \|x\| = \frac{3R}{4}}$. For two vectors $x,y\in \bb R^d$, let $\theta(x,y)$ represent the angle between $x$ and $y$. We first try to find $n$ disjoint caps $C_1,C_2,\dots,C_n$ on $S$. Denoting $v_i$ as the central vector of cap $C_i$, $C_i=\set{x\in \bb R^d: \|x\|=\frac{3R}{4}, \cos(\theta(x,v_i))\geq \ell}$ with $\ell=\frac{\sqrt{(\tp{\frac{3R}{4}}^2-2r_2^2)}}{\frac{3R}{4}}$. 
    
    In contrast, suppose we can only find $n'<\frac{(d-2)^{\frac{3}{2}}}{2}\cdot \tp{\frac{3R}{8\sqrt{2}r_2}}^{d-1}$ such disjoint caps $\set{C_i}_{1\leq i\leq n'}$ with central vectors $\set{v_i}_{1\leq i\leq n'}$. Then for any $w\in S$, there exist $i\in[n']$ and $x\in S$ such that $\cos(\theta(x,v_i))\geq \ell$ and $\cos(\theta(x,w))\geq \ell$. Otherwise we can find a new cap with center $w$.
    
    Via \Cref{lem:cos}, we know that $\cos(\theta(w,v_i))\geq \ell^2 - \tp{1-\ell^2}  \ell'$ with $\ell'=\frac{\tp{\frac{3R}{4}}^2-4r_2^2}{\tp{\frac{3R}{4}}^2}$. 
    This means we can find $n'$ larger caps with central vectors $\set{v_i}_{1\leq i\leq n'}$ and angle $\arccos(\ell')$ to cover the sphere.
    From \cite{L11}, however, the area of a cap with angle $\theta = \arccos(\ell')$ is $\frac{\Gamma\tp{\frac{d-1}{2}}}{\sqrt{\pi}\Gamma\tp{\frac{d}{2}}} \int_{0}^{\theta} \sin^{d-2}(\phi)\d \phi$ times of the total sphere. We have
    % \ctodo{Find the constant $c$.}
    \[
         \int_{0}^{\theta} \sin^{d-2}(\phi)\d \phi \leq \frac{1}{\ell'} \int_{0}^{\theta} \sin^{d-2}(\phi)\cos (\phi)\d \phi = \frac{1}{\ell'} \int_0^{\sin(\theta)} s^{d-2} \d s = \frac{1}{\ell'} \frac{\tp{\sin \theta}^{d-1}}{d-1}.
    \]
    Since $\sin \theta = \sqrt{1-\tp{\ell'}^2}\leq \frac{2\sqrt{2}r_2}{\frac{3R}{4}}$, the ratio between the area of this larger cap and the sphere can be bounded by
    \begin{align*}
        \frac{\Gamma\tp{\frac{d-1}{2}}}{\sqrt{\pi}\Gamma\tp{\frac{d}{2}}} \int_{0}^{\theta} \sin^{d-2}(\phi)\d \phi & \leq \frac{1}{\ell' (d-1)}\cdot \frac{\Gamma\tp{\frac{d-1}{2}}}{\sqrt{\pi}\Gamma\tp{\frac{d}{2}}}\cdot \tp{\frac{2\sqrt{2}r_2}{\frac{3R}{4}}}^{d-1} \\
        \mr{$R\geq 4r_2$}
        &\leq \frac{9}{5\sqrt{\pi}(d-1)} \cdot \frac{\Gamma\tp{\frac{d-1}{2}}}{\Gamma\tp{\frac{d}{2}}} \cdot \tp{\frac{8\sqrt{2}r_2}{3R}}^{d-1} \\
        \mr{Gautschi's inequality}
        &\leq \frac{9\sqrt{2}}{5\sqrt{\pi}}\cdot \frac{1}{(d-1)\sqrt{d-2}}\cdot \tp{\frac{8\sqrt{2}r_2}{3R}}^{d-1}\\
        &\leq \frac{2}{(d-1)\sqrt{d-2}} \tp{\frac{8\sqrt{2}r_2}{3R}}^{d-1}.
    \end{align*}
    
    % So there exists some universal constant $c'$ such that the ratio between the area of this larger cap and the sphere is no larger than $\tp{\frac{c' r_2}{R}}^{d-1}$. 
    % By choosing $c= \frac{1}{c'}$, 
    This will lead to a conflict since $n'$ such caps cannot cover the sphere. Therefore, we can find $n= \frac{(d-1)\sqrt{d-2}}{2}\cdot \tp{\frac{3R}{8\sqrt{2}r_2}}^{d-1}$ such $C_i$'s.
    
    Furthermore, from \Cref{lem:cosinBall}, $\+B_{r_2}(v_i)\cap \+B_{r_2}(v_j)=\emptyset$.
\end{proof}

\subsection{Proof of the lower bound}
\begin{theorem}[\Cref{thm:main} restated]
    % There exist a universal constant $C>0$ such that 
    For any $L,M>0$ satisfying $LM\ge d$ and for any $\eps\in(0,1/32)$, $d\geq 5$, if a sampling algorithm $\+A$ always terminates within 
    \[
        \frac{\eps (d-2)^{\frac{3}{2}}}{8}\cdot \tp{\frac{9}{256} \cdot \frac{LM}{d\eps} \cdot \frac{1}{\log \frac{LM}{d\eps}}}^{\frac{d-1}{2}}
        % \frac{\eps}{4}\cdot \tp{\frac{C\cdot LM}{2 d\eps} \cdot \frac{1}{\log \frac{LM}{d\eps}}}^{\frac{d-1}{2}}
        % % \approx \frac{\eps}{4}\exp\set{\frac{d}{2}\cdot \Omega\tp{\log \frac{LM}{d\eps}}}
    \]
    queries on every input instance in $\@D_{L,M}$, then there must exist some distribution $\mu\in \@D_{L,M}$ such that when the underlying instance is $\mu$, the distribution of $\+A$'s output, denoted as $\tilde \mu$, is $\eps$ away from $\mu$ in total variation distance, i.e., $\DTV(\mu,\tilde \mu)\geq \eps$.
\end{theorem}
\begin{proof}
    Let $v_1,v_2,\dots,v_n$ be the $n$ vectors in \Cref{lem:disjointcap}. For each $i\in [n]$, construct a distribution $\mu_{v_i}$ with density $p_i \propto e^{-f_{v_i}}$ as described in \Cref{sec:hardinstance}. From the discussion in previous sections, we can assume that every $\mu_{v_i}$ as well as the base instance are $L$-log-smooth and have second moment at most $M$. For simplicity, we write $\mu_{v_i}$ as $\mu_i$.

    We use $\Pr[\mu]{\cdot}$ and $\E[\mu]{\cdot}$ to denote the probability and expectation when the underlying instance is some distribution $\mu$. Let $\+E_{k,i}$ be the event that the algorithm $\+A$ queries a value in zone $\+B_{r_2}(v_i)$ in the $k$-th query and it is the first time that $\+A$ queries the points in $\+B_{r_2}(v_i)$. If the algorithm terminates before the $k$-th query, we regard $\+E_{k,i}$ as an impossible event.

    Assume $\+A$ always terminates in $N$ queries for some 
    $$
        N< \frac{\eps (d-2)^{\frac{3}{2}}}{8}\cdot \tp{\frac{9}{256}\cdot \frac{LM}{d\eps} \cdot \frac{1}{\log \frac{LM}{d\eps}}}^{\frac{d-1}{2}} < \frac{\eps}{4}\cdot \frac{(d-1)\sqrt{d-2}}{2}\cdot \tp{\frac{3R}{8\sqrt{2}r_2}}^{d-1} = \frac{\eps n}{4}.
    $$ 
    Let $\+E_{i}$ be the event that $\+A$ queries the points in $\+B_{r_2}(v_i)$ at least once. Then
    \begin{align*}
        \sum_{i=1}^n \Pr[\mu_0]{\+E_i}& = \sum_{i=1}^n \sum_{k=1}^N \Pr[\mu_0]{\+E_{k,i}} = \sum_{k=1}^N \sum_{i=1}^n \Pr[\mu_0]{\+E_{k,i}} \leq N<\frac{\eps n}{4}.
    \end{align*}
    So there exists some $i_0,j_0\in[n]$ and $i_0\neq j_0$ such that $\Pr[\mu_0]{\+E_{i_0}}<\frac{\eps}{3}$ and $\Pr[\mu_0]{\+E_{j_0}}<\frac{\eps}{3}$. Otherwise $\sum_{i=1}^n \Pr[\mu_0]{\+E_i}\geq \frac{\eps(n-1)}{3}\geq \frac{\eps n}{4}$ since $n\geq 4$ when $d\geq 5$. From union bound, $\Pr[\mu_0]{\ol{\+E_{i_0}}\cap \ol{\+E_{j_0}}}> 1-\frac{2\eps}{3}$. 
    
    We know that on $\bb R^d\setminus \+B_{r_2}(v_i)$, $f_{v_i}(x)=f_0(x)$ for each $i\in[n]$. Therefore, via coupling arguments,
    \[
        \Pr[\mu_{i_0}]{\ol{\+E_{i_0}}\cap \ol{\+E_{j_0}}} = \Pr[\mu_{j_0}]{\ol{\+E_{i_0}}\cap \ol{\+E_{j_0}}} = \Pr[\mu_0]{\ol{\+E_{i_0}}\cap \ol{\+E_{j_0}}}> 1-\frac{2\eps}{3}.
    \]

    Let $\+E=\+E_{i_0}\cup \+E_{j_0}$. Let $\tilde \mu_0^{\ol{\+E}}$ be the output distribution of $\+A$ with input distribution $\mu_0$ when $\ol{\+E}$ happens. 
    Since
    \begin{align*}
        4\eps\leq \DTV(\mu_{i_0},\mu_{j_0}) \leq \DTV(\mu_{i_0},\tilde \mu_0^{\ol{\+E}}) + \DTV(\mu_{j_0},\tilde \mu_0^{\ol{\+E}}),
    \end{align*}
    we have either $\DTV(\mu_{i_0},\tilde \mu_0^{\ol{\+E}})>2\eps$ or $\DTV(\mu_{j_0},\tilde \mu_0^{\ol{\+E}})>2\eps$. W.l.o.g., assume $\DTV(\mu_{i_0},\tilde \mu_0^{\ol{\+E}})>2\eps$. Assume the output distribution of $\+A$ is $\tilde \mu_{i_0}$ when the input is $\mu_{i_0}$. Let $\tilde \mu_{i_0}^{\+E}$ and $\tilde \mu_{i_0}^{\ol{\+E}}$ be $\tilde \mu_{i_0}$ conditioned on $\+E$ and $\ol{\+E}$ respectively and denote their density functions as $\tilde p_{i_0}^{\+E}$, $\tilde p_{i_0}^{\ol{\+E}}$ and $\tilde p_{i_0}$. Then $\tilde \mu_{i_0}^{\ol{\+E}}=\tilde \mu_0^{\ol{\+E}}$. 
    Then we have
    \begin{align*}
        \DTV(\mu_{i_0},\tilde \mu_{i_0}) &= \frac{1}{2}\int_{\bb R^d} \abs{p_{i_0}(x) - \tilde p_{i_0} (x)} \dd x\\
        &=\frac{1}{2}\int_{\bb R^d} \abs{p_{i_0}(x) - \Pr[\mu_{i_0}]{\+E}\tilde p_{i_0}^{\+E}(x) - \Pr[\mu_{i_0}]{\ol{\+E}}\tilde p_{i_0}^{\ol{\+E}}(x)} \dd x \\
        &\geq \Pr[\mu_{i_0}]{\ol{\+E}}\cdot \frac{1}{2} \int_{\bb R^d}\abs{p_{i_0}(x) - \tilde p_{i_0}^{\ol{\+E}}(x)} \dd x  - \Pr[\mu_{i_0}]{\+E}\cdot \frac{1}{2} \int_{\bb R^d}\abs{p_{i_0}(x) - \tilde p_{i_0}^{\+E}(x)} \dd x \\
        & = \Pr[\mu_{i_0}]{\ol{\+E}}\cdot \DTV(\mu_{i_0}, \tilde \mu_{i_0}^{\ol{\+E}}) - \Pr[\mu_{i_0}]{\+E}\cdot \DTV(\mu_{i_0}, \tilde \mu_{i_0}^{\+E}) \\
        & = \Pr[\mu_{i_0}]{\ol{\+E}}\cdot \DTV(\mu_{i_0}, \tilde \mu^{\ol{\+E}}_0) - \Pr[\mu_{i_0}]{\+E}\cdot \DTV(\mu_{i_0}, \tilde \mu_{i_0}^{\+E}) \\
        &\geq \tp{1-\frac{2\eps}{3}}\cdot 2\eps - \frac{2\eps}{3}\cdot 1\\
        &>\eps.
    \end{align*}
    This means, on instance $\mu_{i_0}$, the algorithm $\+A$ will fail to output a distribution which is $\eps$-close to $\mu_{i_0}$ in total variation distance.
\end{proof}



% \section{Efficient learning of Pareto routers} 
\label{sec:efficient-learning-algorithm}


\subsection{The upper bound}\label{sec:upper-bound}

Recall in Algorithm \ref{alg:pareto-routers} we say that we require an ``efficient estimate'' $\widehat \Phi$ of the function $\Phi^\star$, whose $l$-th coordinate is $\Phi^\star_l(x) = \Ex_P[\ell\{Y, f_l(X)\mid X = x]$. Now, we make this statement precise and show that for such an $\widehat \Phi $ the excess risk \eqref{eq:excess-risk} achieves the rate in the lower bound \eqref{eq:lower-bound}. 

\begin{theorem}[Upper bound]\label{thm:upper-bound}
    Suppose that for some $\rho_1, \rho_2 > 0$ and any $n \ge 1$ and $t > 0$ and almost all $x$ with respect to $P_X$ we have the following concentration bound for $\widehat \Phi$:
    \begin{equation}\label{eq:concentration-phi}
        \max_{P\in \cP} P_{\cD_n} \big \{ \|\widehat \Phi(x) - \Phi^\star (x)\|_1 \ge t\big \} \le  \rho_1 \exp\big (- \rho_2 a_n t^2 \big )\,, 
    \end{equation}
    where $\{a_n; n \ge 1\}\subset \reals$ is a sequence that increases to $\infty$.  
    Fix a $\lambda \in [0, 1]$.  Then, if all the $P\in \cP$ satisfies the margin condition \ref{assmp:margin} with the parameters $(\alpha, K_\alpha)$ then there exists a $K> 0$ such that for any $n \ge 1$ the excess risk for the router $\widehat g_\lambda$ in \eqref{eq:eff-estimate-router} is upper bounded as 
    \begin{equation}
        \max_{P\in \cP} \Ex_{\cD_n}\big [\cE_P(\widehat g_\lambda,\lambda)\big ] \le K a_n^{-\frac{1+ \alpha}{2}}\,. 
    \end{equation}
    Thus, as long as $a_n = n^{{  2\beta/(2\beta + d)}}$  for any $\lambda \in [0, 1]$ the excess risk for the router $\widehat g_\lambda$ in \eqref{eq:eff-estimate-router} has the rate of convergence $a_n^{- {(1 + \alpha)}/{2}} = n^{- {\beta(1 + \alpha)}/{(2\beta + d)}}$ that matches with the lower bound rate in \eqref{eq:lower-bound}. 
\end{theorem}




% {\bf Question:} How can we find Pareto front solutions efficiently without using a back-tracking technique, \ie\ fit different $g$ for different $\lambda$? The hint is to learn $f^\star(x)$ as $\widehat f(x)$ and for a $\lambda > 0$ then define 
% \[
% \textstyle \widehat g _\lambda (x) = \delta _{\widehat l(x)}, ~~ \widehat l(x) =  \argmin_ l\big\{\ell (\widehat f(x_0), f_l(x_0)) + \lambda c_l (x_0)\big\}
% \] Show that $\widehat g_\lambda$ achieved the rate at lower bound. 


\subsection{Efficient learning of the loss regression function}
\label{sec:reg-fn-estimate}
We consider \emph{local polynomial regression} (LPR) to estimate the loss regression function. For a sample point $Z_i = (X_i, Y_i)$ we define the observed loss vector $\ell_i \in \reals^L$ where $[\ell_i]_l = \ell\{Y_i,f_l(X_i)\}$. Consider a symmetric kernel $\psi: \reals^d \to \reals_+$ and define $\Theta(\lfloor \beta \rfloor)$ as the class of all $\lfloor \beta \rfloor$-degree polynomials from $\reals^d$ to $\reals$. For a bandwidth $h > 0$ we define the LPR estimate as 
\begin{equation}
    \widehat \Phi_l(x) = \widehat\theta_x^{(l)}(0), ~~ \widehat \theta_x^{(l)} \in \argmin_{\theta \in \Theta(\lfloor \beta \rfloor)}  \textstyle \sum_{i = 1}^n \psi (\frac{X_i - x}{h}) \big [\ell \{Y_i, f_l(X_i)\} - \theta (X_i -x )\big]^2 \,. 
\end{equation}
In their Theorem 3.2 \citet{audibert2007Fast} establishes that for $h = n ^{-1/(2\beta + d)}$ the concentration bound in \eqref{eq:concentration-phi} is satisfied with $a_n = n^{-2\beta/(2\beta + d)}$, therefore the router derived from such $\widehat\Phi$ using \eqref{eq:eff-estimate-router} achieves the same rate of convergence as in the lower bound \eqref{eq:lower-bound}, \ie\ rate optimal. 
\section{Experiments}
\label{sec:exp}
Following the settings in Section \ref{sec:existing}, we evaluate \textit{NovelSum}'s correlation with the fine-tuned model performance across 53 IT datasets and compare it with previous diversity metrics. Additionally, we conduct a correlation analysis using Qwen-2.5-7B \cite{yang2024qwen2} as the backbone model, alongside previous LLaMA-3-8B experiments, to further demonstrate the metric's effectiveness across different scenarios. Qwen is used for both instruction tuning and deriving semantic embeddings. Due to resource constraints, we run each strategy on Qwen for two rounds, resulting in 25 datasets. 

\subsection{Main Results}

\begin{table*}[!t]
    \centering
    \resizebox{\linewidth}{!}{
    \begin{tabular}{lcccccccccc}
    \toprule
    \multirow{3}*{\textbf{Diversity Metrics}} & \multicolumn{10}{c}{\textbf{Data Selection Strategies}} \\
    \cmidrule(lr){2-11}
    & \multirow{2}*{\textbf{K-means}} & \multirow{2}*{\vtop{\hbox{\textbf{K-Center}}\vspace{1mm}\hbox{\textbf{-Greedy}}}}  & \multirow{2}*{\textbf{QDIT}} & \multirow{2}*{\vtop{\hbox{\textbf{Repr}}\vspace{1mm}\hbox{\textbf{Filter}}}} & \multicolumn{5}{c}{\textbf{Random}} & \multirow{2}{*}{\textbf{Duplicate}} \\ 
    \cmidrule(lr){6-10}
    & & & & & \textbf{$\mathcal{X}^{all}$} & ShareGPT & WizardLM & Alpaca & Dolly &  \\
    \midrule
    \rowcolor{gray!15} \multicolumn{11}{c}{\textit{LLaMA-3-8B}} \\
    Facility Loc. $_{\times10^5}$ & \cellcolor{BLUE!40} 2.99 & \cellcolor{ORANGE!10} 2.73 & \cellcolor{BLUE!40} 2.99 & \cellcolor{BLUE!20} 2.86 & \cellcolor{BLUE!40} 2.99 & \cellcolor{BLUE!0} 2.83 & \cellcolor{BLUE!30} 2.88 & \cellcolor{BLUE!0} 2.83 & \cellcolor{ORANGE!20} 2.59 & \cellcolor{ORANGE!30} 2.52 \\    
    DistSum$_{cosine}$  & \cellcolor{BLUE!30} 0.648 & \cellcolor{BLUE!60} 0.746 & \cellcolor{BLUE!0} 0.629 & \cellcolor{BLUE!50} 0.703 & \cellcolor{BLUE!10} 0.634 & \cellcolor{BLUE!40} 0.656 & \cellcolor{ORANGE!30} 0.578 & \cellcolor{ORANGE!10} 0.605 & \cellcolor{ORANGE!20} 0.603 & \cellcolor{BLUE!10} 0.634 \\
    Vendi Score $_{\times10^7}$ & \cellcolor{BLUE!30} 1.70 & \cellcolor{BLUE!60} 2.53 & \cellcolor{BLUE!10} 1.59 & \cellcolor{BLUE!50} 2.23 & \cellcolor{BLUE!20} 1.61 & \cellcolor{BLUE!30} 1.70 & \cellcolor{ORANGE!10} 1.44 & \cellcolor{ORANGE!20} 1.32 & \cellcolor{ORANGE!10} 1.44 & \cellcolor{ORANGE!30} 0.05 \\
    \textbf{NovelSum (Ours)} & \cellcolor{BLUE!60} 0.693 & \cellcolor{BLUE!50} 0.687 & \cellcolor{BLUE!30} 0.673 & \cellcolor{BLUE!20} 0.671 & \cellcolor{BLUE!40} 0.675 & \cellcolor{BLUE!10} 0.628 & \cellcolor{BLUE!0} 0.591 & \cellcolor{ORANGE!10} 0.572 & \cellcolor{ORANGE!20} 0.50 & \cellcolor{ORANGE!30} 0.461 \\
    \midrule    
    \textbf{Model Performance} & \cellcolor{BLUE!60}1.32 & \cellcolor{BLUE!50}1.31 & \cellcolor{BLUE!40}1.25 & \cellcolor{BLUE!30}1.05 & \cellcolor{BLUE!20}1.20 & \cellcolor{BLUE!10}0.83 & \cellcolor{BLUE!0}0.72 & \cellcolor{ORANGE!10}0.07 & \cellcolor{ORANGE!20}-0.14 & \cellcolor{ORANGE!30}-1.35 \\
    \midrule
    \midrule
    \rowcolor{gray!15} \multicolumn{11}{c}{\textit{Qwen-2.5-7B}} \\
    Facility Loc. $_{\times10^5}$ & \cellcolor{BLUE!40} 3.54 & \cellcolor{ORANGE!30} 3.42 & \cellcolor{BLUE!40} 3.54 & \cellcolor{ORANGE!20} 3.46 & \cellcolor{BLUE!40} 3.54 & \cellcolor{BLUE!30} 3.51 & \cellcolor{BLUE!10} 3.50 & \cellcolor{BLUE!10} 3.50 & \cellcolor{ORANGE!20} 3.46 & \cellcolor{BLUE!0} 3.48 \\ 
    DistSum$_{cosine}$ & \cellcolor{BLUE!30} 0.260 & \cellcolor{BLUE!60} 0.440 & \cellcolor{BLUE!0} 0.223 & \cellcolor{BLUE!50} 0.421 & \cellcolor{BLUE!10} 0.230 & \cellcolor{BLUE!40} 0.285 & \cellcolor{ORANGE!20} 0.211 & \cellcolor{ORANGE!30} 0.189 & \cellcolor{ORANGE!10} 0.221 & \cellcolor{BLUE!20} 0.243 \\
    Vendi Score $_{\times10^6}$ & \cellcolor{ORANGE!10} 1.60 & \cellcolor{BLUE!40} 3.09 & \cellcolor{BLUE!10} 2.60 & \cellcolor{BLUE!60} 7.15 & \cellcolor{ORANGE!20} 1.41 & \cellcolor{BLUE!50} 3.36 & \cellcolor{BLUE!20} 2.65 & \cellcolor{BLUE!0} 1.89 & \cellcolor{BLUE!30} 3.04 & \cellcolor{ORANGE!30} 0.20 \\
    \textbf{NovelSum (Ours)}  & \cellcolor{BLUE!40} 0.440 & \cellcolor{BLUE!60} 0.505 & \cellcolor{BLUE!20} 0.403 & \cellcolor{BLUE!50} 0.495 & \cellcolor{BLUE!30} 0.408 & \cellcolor{BLUE!10} 0.392 & \cellcolor{BLUE!0} 0.349 & \cellcolor{ORANGE!10} 0.336 & \cellcolor{ORANGE!20} 0.320 & \cellcolor{ORANGE!30} 0.309 \\
    \midrule
    \textbf{Model Performance} & \cellcolor{BLUE!30} 1.06 & \cellcolor{BLUE!60} 1.45 & \cellcolor{BLUE!40} 1.23 & \cellcolor{BLUE!50} 1.35 & \cellcolor{BLUE!20} 0.87 & \cellcolor{BLUE!10} 0.07 & \cellcolor{BLUE!0} -0.08 & \cellcolor{ORANGE!10} -0.38 & \cellcolor{ORANGE!30} -0.49 & \cellcolor{ORANGE!20} -0.43 \\
    \bottomrule
    \end{tabular}
    }
    \caption{Measuring the diversity of datasets selected by different strategies using \textit{NovelSum} and baseline metrics. Fine-tuned model performances (Eq. \ref{eq:perf}), based on MT-bench and AlpacaEval, are also included for cross reference. Darker \colorbox{BLUE!60}{blue} shades indicate higher values for each metric, while darker \colorbox{ORANGE!30}{orange} shades indicate lower values. While data selection strategies vary in performance on LLaMA-3-8B and Qwen-2.5-7B, \textit{NovelSum} consistently shows a stronger correlation with model performance than other metrics. More results are provided in Appendix \ref{app:results}.}
    \label{tbl:main}
    \vspace{-4mm}
\end{table*}


\begin{table}[t!]
\centering
\resizebox{\linewidth}{!}{
\begin{tabular}{lcccc}
\toprule
\multirow{2}*{\textbf{Diversity Metrics}} & \multicolumn{3}{c}{\textbf{LLaMA}} & \textbf{Qwen}\\
\cmidrule(lr){2-4} \cmidrule(lr){5-5} 
& \textbf{Pearson} & \textbf{Spearman} & \textbf{Avg.} & \textbf{Avg.} \\
\midrule
TTR & -0.38 & -0.16 & -0.27 & -0.30 \\
vocd-D & -0.43 & -0.17 & -0.30 & -0.31 \\
\midrule
Facility Loc. & 0.86 & 0.69 & 0.77 & 0.08 \\
Entropy & 0.93 & 0.80 & 0.86 & 0.63 \\
\midrule
LDD & 0.61 & 0.75 & 0.68 & 0.60 \\
KNN Distance & 0.59 & 0.80 & 0.70 & 0.67 \\
DistSum$_{cosine}$ & 0.85 & 0.67 & 0.76 & 0.51 \\
Vendi Score & 0.70 & 0.85 & 0.78 & 0.60 \\
DistSum$_{L2}$ & 0.86 & 0.76 & 0.81 & 0.51 \\
Cluster Inertia & 0.81 & 0.85 & 0.83 & 0.76 \\
Radius & 0.87 & 0.81 & 0.84 & 0.48 \\
\midrule
NovelSum & \textbf{0.98} & \textbf{0.95} & \textbf{0.97} & \textbf{0.90} \\
\bottomrule
\end{tabular}
}
\caption{Correlations between different metrics and model performance on LLaMA-3-8B and Qwen-2.5-7B.  “Avg.” denotes the average correlation (Eq. \ref{eq:cor}).}
\label{tbl:correlations}
\vspace{-2mm}
\end{table}

\paragraph{\textit{NovelSum} consistently achieves state-of-the-art correlation with model performance across various data selection strategies, backbone LLMs, and correlation measures.}
Table \ref{tbl:main} presents diversity measurement results on datasets constructed by mainstream data selection methods (based on $\mathcal{X}^{all}$), random selection from various sources, and duplicated samples (with only $m=100$ unique samples). 
Results from multiple runs are averaged for each strategy.
Although these strategies yield varying performance rankings across base models, \textit{NovelSum} consistently tracks changes in IT performance by accurately measuring dataset diversity. For instance, K-means achieves the best performance on LLaMA with the highest NovelSum score, while K-Center-Greedy excels on Qwen, also correlating with the highest NovelSum. Table \ref{tbl:correlations} shows the correlation coefficients between various metrics and model performance for both LLaMA and Qwen experiments, where \textit{NovelSum} achieves state-of-the-art correlation across different models and measures.

\paragraph{\textit{NovelSum} can provide valuable guidance for data engineering practices.}
As a reliable indicator of data diversity, \textit{NovelSum} can assess diversity at both the dataset and sample levels, directly guiding data selection and construction decisions. For example, Table \ref{tbl:main} shows that the combined data source $\mathcal{X}^{all}$ is a better choice for sampling diverse IT data than other sources. Moreover, \textit{NovelSum} can offer insights through comparative analyses, such as: (1) ShareGPT, which collects data from real internet users, exhibits greater diversity than Dolly, which relies on company employees, suggesting that IT samples from diverse sources enhance dataset diversity \cite{wang2024diversity-logD}; (2) In LLaMA experiments, random selection can outperform some mainstream strategies, aligning with prior work \cite{xia2024rethinking,diddee2024chasing}, highlighting gaps in current data selection methods for optimizing diversity.



\subsection{Ablation Study}


\textit{NovelSum} involves several flexible hyperparameters and variations. In our main experiments, \textit{NovelSum} uses cosine distance to compute $d(x_i, x_j)$ in Eq. \ref{eq:dad}. We set $\alpha = 1$, $\beta = 0.5$, and $K = 10$ nearest neighbors in Eq. \ref{eq:pws} and \ref{eq:dad}. Here, we conduct an ablation study to investigate the impact of these settings based on LLaMA-3-8B.

\begin{table}[ht!]
\centering
\resizebox{\linewidth}{!}{
\begin{tabular}{lccc}
\toprule
\textbf{Variants} & \textbf{Pearson} & \textbf{Spearman} & \textbf{Avg.} \\
\midrule
NovelSum & 0.98 & 0.96 & 0.97 \\
\midrule
\hspace{0.10cm} - Use $L2$ distance & 0.97 & 0.83 & 0.90\textsubscript{↓ 0.08} \\
\hspace{0.10cm} - $K=20$ & 0.98 & 0.96 & 0.97\textsubscript{↓ 0.00} \\
\hspace{0.10cm} - $\alpha=0$ (w/o proximity) & 0.79 & 0.31 & 0.55\textsubscript{↓ 0.42} \\
\hspace{0.10cm} - $\alpha=2$ & 0.73 & 0.88 & 0.81\textsubscript{↓ 0.16} \\
\hspace{0.10cm} - $\beta=0$ (w/o density) & 0.92 & 0.89 & 0.91\textsubscript{↓ 0.07} \\
\hspace{0.10cm} - $\beta=1$ & 0.90 & 0.62 & 0.76\textsubscript{↓ 0.21} \\
\bottomrule
\end{tabular}
}
\caption{Ablation Study for \textit{NovelSum}.}
\label{tbl:ablation}
\vspace{-2mm}
\end{table}

In Table \ref{tbl:ablation}, $\alpha=0$ removes the proximity weights, and $\beta=0$ eliminates the density multiplier. We observe that both $\alpha=0$ and $\beta=0$ significantly weaken the correlation, validating the benefits of the proximity-weighted sum and density-aware distance. Additionally, improper values for $\alpha$ and $\beta$ greatly reduce the metric's reliability, highlighting that \textit{NovelSum} strikes a delicate balance between distances and distribution. Replacing cosine distance with Euclidean distance and using more neighbors for density approximation have minimal impact, particularly on Pearson's correlation, demonstrating \textit{NovelSum}'s robustness to different distance measures.






% \section{Routing in a general multi-objective framework}\label{sec:general-multi-obj}

In this section, we discuss a natural extension of the routing problem with general metrics. Since most of the constructions are very similar to the construction in the Introduction section \ref{sec:intro}, we keep our discussion short.  Say, we are provided $M$ pre-trained models, denoted as $f_1, \dots,f_M$. Given model $f_m$ and a sample point $(X, Y)$,  we are interested in $L$ different loss metrics, among which the first $L_1 $ many, denoted as $\ell_1\{f_m; X, Y\} , \dots, \ell_{L_1}\{f_m; X, Y\}$, depends on $f_m$, $X$ and $Y$, while the next $L_2 = L - L_1$ many, denoted as $\ell_{L_1+ 1}(f_m; X), \dots , \ell_{L}(f_m; X)$, depends only on $f_m$ and $X$.  In that case we consider a compromise between these $L$ losses: for an $\lambda = (\lambda_1 , \dots, \lambda_L) \in\Delta^{L-1} $ the linear loss trade-off in eq. \eqref{eq:linearized-loss} naturally extends to 
\begin{equation} \label{eq:linearized-loss-gen}
 \textstyle \eta_{\lambda, m}(X, Y) =  \sum_{l = 1}^{L_1}\lambda_l \ell_l\{f_m; X, Y\} + \sum_{l = L_1 + 1}^{L}\lambda_l \ell_l\{f_m; X\}   \,.
\end{equation} 
To make it even more general, assume that we have a classification task with $M$ classes, \ie \ the outcome $Y \in \{1, \dots, M\}$ is now categorical. Additionally, as a multi-objective problem we have $L$ different loss metrics, among which the first $L_1 $ many $\ell_1\{m; X, Y\} , \dots, \ell_{L_1}\{m; X, Y\}$ are losses  of predicting a sample $(X, Y)$ as the class $m$, while the next $L_2 = L - L_1$ many $\ell_{L_1+ 1}\{m; X\}, \dots , \ell_{L}\{m; X\}$ is the cost of predicting the sample $X$ as the class $m$. Then the aggregated loss is 
\begin{equation} \label{eq:linearized-loss-gen-2}
 \textstyle \eta_{\lambda, m}(X, Y) =  \sum_{l = 1}^{L_1}\lambda_l \ell_l\{m; X, Y\} + \sum_{l = L_1 + 1}^{L}\lambda_l \ell_l\{m; X\}   \,.
\end{equation} 

For a particular $\lambda \in \Delta^{L-1}$ we want to estimate the optimal/Bayes/admissible classifier $g_\lambda^\star$ that minimized the average of the aggregates loss trade-off:
\begin{equation}
    \textstyle g_\lambda^\star = \argmin_{g: \cX \to [M]}  \Ex_P\big[\sum_{m = 1}^M \bbI\{g(X) = m\} \eta_{\lambda, m} (X, Y) \big ]
\end{equation}
Similar to Lemma \ref{lemma:oracle-router}, defining $\Phi_{l, m}^\star(X) = \Ex[\ell_l\{f_m; X,Y\} \mid X]$ for $l \in [L_1]$ and $\Phi_{l, m}^\star(X) = \ell_l\{f_m; X\} $ for $l \in \{L_1 + 1, \dots, L\}$ the Bayes classifier $g_\lambda^\star$ has the explicit form  (\cf\ Lemma \ref{lemma:bayes-classifier-gen})
\begin{equation}\label{eq:oracle-router-gen}
  \textstyle  g_\lambda^\star (X) = \argmin_m  \eta_{\lambda, m}(X), ~~ \eta_{\lambda, m}(X) = \sum_{l = 1}^L \lambda_l \Phi_{l, m}^\star (X)\,.
\end{equation}
A natural extension of Algorithm \ref{alg:pareto-routers} requires us to estimate each of the functions $\Phi_{l, m}^\star$ as $\widehat \Phi_{l, m}$ then estimate all Bayes classifiers in a one-shot plug-in approach as:
\[
\textstyle \widehat g_\lambda(X) = \argmin_m  \{\sum_{l = 1}^L \lambda_l \widehat \Phi_{l, m} (X) \}\,.
\] 
Note that, for $l \ge L_1+ 1$ the functions $\Phi_{l, m}^\star$ are known, so they need not be estimated; in these cases, we simply let $\widehat \Phi_{l, m} = \Phi_{l ,m}^\star$. 
Having estimated $\widehat g_\lambda$, we can then evaluate them on a test data split with respect to each of the individual metrics to estimate the Pareto frontier and examine the optimal trade-off between the objectives.
We describe the procedure in Algorithm \ref{alg:pareto-routers-gen}.  

\begin{algorithm}
    \begin{algorithmic}[1]
\Require Dataset $\cD_n$
\State Randomly split the dataset into training and test splits: $\cD_n = \cD_{\text{tr}} \cup \cD_{\text{test}}$. 
\State  Learn estimates $\widehat \Phi_{l, m} (X)$ of the function $\Phi_{l, m}^\star(X)$ using the training split $\cD_{\text{tr}}$. 
\For{$\lambda \in \Delta^{L-1}$}
\State  Define $\widehat \eta_{\lambda, m}(X) =  \sum_{l = 1}^L \lambda_l \widehat \Phi_{l, m} (X)  $ and 
 $\widehat g_\lambda(X) = \argmin_m \widehat \eta_{\lambda, m}(X)$
 \For{$l \in \{1, \dots, L\}$}
 \State Calculate $\widehat \cE_{l, \lambda}  =  \frac1{|\cD_{\text{test}}|} \sum_{(X, Y) \in \cD_{\text{test}}}  \ell_l\{Y, f_{\widehat g_\lambda(X)}(X)\}$
 \EndFor
\EndFor

\Return $\{g_\lambda: \lambda \in \Delta^{L-1}\}$ and $\widehat\cF = \{(\widehat \cE_{1, \lambda}, \dots, \widehat \cE_{L, \lambda}): \lambda \in \Delta^{L-1}\}$. 
\end{algorithmic}
\caption{Learning of Bayes classifiers}
\label{alg:pareto-routers-gen}
\end{algorithm}


Except for some minor differences, the minimax rate analysis of the estimate of $g_\lambda^\star$ is identical to Section \ref{sec:lower-bound}. We assume that
\begin{enumerate}
    \item For $1 \le l \le L_1$ the $\Phi_{l, m}^\star$ functions are $(\beta_l, K_{\beta, l})$-smooth, which is similar to the Assumption \ref{assmp:smooth}. Recall the discussion following Assumption \ref{assmp:smooth}; (1) Since $\Phi_{l, m}$ are known for $l \ge L_1 + 1$ a smoothness assumption on them is not necessary, and (2) We could assume different smoothness for different $\Phi_{l, m}^\star$, \ie\ they are $\beta_{l, m}$ smooth, but the rate will only involve the smallest smoothness, \ie\ $\beta_{l, \min} = \min_m \beta_{l, m}$. Thus, for simplicity, we assume that for a given $l$ the $\beta_{l, m}$ are identical for different $m$. 
    \item The margin condition in Assumption \ref{assmp:margin} is satisfied with the new definition of $\eta_{\lambda, m}(X) = \Ex[\eta_{\lambda, m}(X, Y)\mid X]$ with parameters $(\alpha, K_\alpha)$.
    \item  The $P_X$ is compactly supported and satisfies the strong density condition, as in the assumption \ref{assmp:strong-density}.
\end{enumerate}

%  and that the margin condition is satisfied  Furthermore, we assume the 
% \SM{talk about how the worst smoothness parameter is the one that matters.}


Under these assumptions, we establish the upper and lower bound analysis for the minimax rate of convergence in excess risk
\[
\textstyle \cE_P(g, \lambda) = \cL_P(g, \lambda) - \cL_P(g_\lambda^\star, \lambda)\,.
\]
For this purpose, let us quickly recall the notation and definitions necessary to describe the lower and upper bounds. We denote the class of all probability distributions on $\cX \times \cY$ space by $\cP$.
We also denote $\Gamma = \{g: \cX \to [M]\}$ as the set of all classifiers and $\cA_n$ as the class of learning algorithms, such that an algorithm $\cA_n \ni A_n: \cZ^n \to \Gamma $ that takes the dataset $\cD_n $ as input and produces a classifier $A_n(\cD_n): \cX \to [M]$.

\begin{theorem}
\label{thm:bound-gen}
Suppose that $\max_{l \in  [L_1]}\beta_l <  \nicefrac{d}{\alpha}$. Then we have the following lower and upper bounds for the excess risk. 
\begin{itemize}
    \item {\bf Lower bound:} For $n \ge 1$  and $A_n \in \cA_n$ define $\cE_P(A_n, \lambda) = \Ex_{\cD_n}\big[\cE_P\big(A_n(\cD_n), \lambda\big)\big]$. There exist constants $c_1, c_2> 0$ that are independent of $n$ and $\lambda$ such that for any $n\ge 1$ and $ \lambda \in \Delta^{L-1}$ we have the lower bound
    \begin{equation} \label{eq:lower-bound-gen}
      \textstyle  \min\limits_{A_n \in \cA_n} \max\limits_{P \in \cP} ~~ \cE_P(A_n, \lambda) \ge c \big \{\sum_{l = 1}^{L_1}\lambda_l n^{- \frac{\beta_l}{2\beta_l + d}}\big\}^{1+\alpha} \,.
    \end{equation}  
    \item  {\bf Upper bound:} For $l \le L_1$ and $m\in [M]$ suppose that there are estimators $\widehat\Phi_{l, m}$ for $\Phi^\star_{l, m}$ that satisfies the following: for constants $\rho_{l, 1}, \rho_{l, 2} > 0$ and any $n \ge 1$ and $t > 0$ and almost all $X$ with respect to $P_X$ we have 
    \begin{equation} \label{eq:concentration-phi-gen}
        \max_{P\in \cP} P \big \{ \max_m \big |\widehat \Phi_{l, m} (X) - \Phi^\star_{l, m}  (X)\big |  \ge t\big \} \le  \rho_{l, 1} \exp\big (- \rho_{l,2} a_{l, n}^2 t^2 \big )\,,  
    \end{equation}  where the sequence $\{a_{l, n}; n \ge 1\}\subset (0, \infty)$  increases to $\infty$. Then there exists a $K> 0$ such that for any $n \ge 1$ and $\lambda \in \Delta^{L-1} $ the excess risk for the router $\widehat g_\lambda$ in Algorithm \ref{alg:pareto-routers-gen} is upper bounded as 
    \begin{equation} \label{eq:upper-bound-gen}
      \textstyle   \max\limits_{P\in \cP} \Ex_{\cD_n}\big [\cE_P(\widehat g_\lambda,\lambda)\big ] \le K \big\{ \sum_{l = 1}^{L_1}\lambda_l a_{l, n}^{-1}\big\} ^{1+ \alpha}\,. 
    \end{equation}
    \item  {\bf Local polynomial estimator:} Assume that $\ell_l \{Y_i, f_m(X_i)\}$ are sub-Gaussian, \ie\ there exists constants $c_{l, 1}$ and $c_{l, 2}$ such that  
    \[
    \textstyle P\big ( |\ell_l \{Y, f_m(X)\} | > t \mid X\big ) \le c_{l, 1} e^{-c_{l, 2}t^2}\,. 
    \] If $\psi$ satisfies the regularity conditions with parameter $\beta_l$ in (\cf\ Definition \ref{def:kernel-reg})  and $k = \lfloor \beta_l \rfloor$ then for $h = n^{-\frac{1}{2\beta_l + d}}$ the inequality in \eqref{eq:concentration-phi-gen} holds for 
    \begin{equation}\label{LPR-gen}
    \widehat \Phi_{l, m}(x_0) = \widehat\theta_{x_0}^{(m)}(0), ~~ \widehat \theta_x^{(m)} \in \argmin_{\theta \in \Theta(k)}  \textstyle \sum_{i = 1}^n \psi (\frac{X_i - x_0}{h}) \big [\ell \{Y_i, f_m(X_i)\} - \theta (X_i -x_0 )\big]^2 \,. 
\end{equation}
    with $a_{l, n} = n^{\nicefrac{2\beta_l}{(2\beta_l + d)}}$, where $\Theta(k)$ is the class of all $k$ degree polynomials. 
\end{itemize} 

  
\end{theorem}
% \SM{Discuss the optimality in remarks.} 
\SM{comment about $\max_{l \in  [L_1]}\beta_l <  \nicefrac{d}{\alpha}$.}
We end our section with a few remarks. 

\begin{remark}[Rate optimality for plug-in routers]

Firstly, given that the upper bound in eq.  \eqref{eq:upper-bound-gen} for plug-in router described in Algorithm \ref{alg:pareto-routers-gen} with local-polynomial estimator for $\widehat \Phi_{l, m}(X)$ achieves the same rate of convergence as the lower bound in \eqref{eq:lower-bound} we conclude that the minimax optimal rate of convergence in excess risk is 
\begin{equation} \label{eq:optimal-rate-gen}
    \textstyle \min\limits_{A_n \in \cA_n} \max\limits_{P \in \cP} ~~ \cE_P(A_n, \lambda) \asymp \cO \Big(\big \{\sum_{l = 1}^{L_1}\lambda_l n^{- \frac{\beta_l}{2\beta_l + d}}\big\}^{1+\alpha}\Big)\,. 
\end{equation} Moreover, this also implies that the plug-in router can achieve this optimal rate as long as the bounds in eq. \eqref{eq:concentration-phi-gen} are satisfied with $a_{l, n} = n^{\nicefrac{2\beta_l}{(2\beta_l + d)}}$. 
    
\end{remark}

\begin{remark}[Difficulty in routing with respect to $\lambda$]
Moreover, the rate in eq. \eqref{eq:optimal-rate-gen} implies that for a smaller $\lambda_l$ the errors in estimating $\Phi_{l, m}^\star$ have a lesser impact on the excess risk convergence. This conclusion is very much related to our Remark \ref{remark:diff-in-lambda-lb}, and thus we keep our discussion brief and ask readers to revisit the said remark. We end with a quick observation that the hardest instance to classify is when $\lambda_{l_{\min}} = 1$ for the lowest smoothness parameter, \ie\ $l_{\min} = \argmin_{l \le L_1} \beta_l$. 

    
\end{remark}


\begin{remark}

{\bf (Minimax rate study for non-parametric classification with multiple objectives)}
Compared to \citet{audibert2007Fast}, our study of the optimal minimax rate provided in this section generalizes the setting on three fronts: (1) the number of classes can be more than two, (2) general classification loss functions beyond $0/1$-loss and (3) multiple objectives. In this general setting, we show that the plug-in classifiers described in Algorithm \ref{alg:pareto-routers-gen} are both computationally and statistically (rate optimal) efficient for estimating the entire class of Bayes classifiers $\{g_\lambda^\star: \lambda\in \Delta^{L-1}\}$. 
    
\end{remark}
\section{Conclusion}
In this work, we propose a simple yet effective approach, called SMILE, for graph few-shot learning with fewer tasks. Specifically, we introduce a novel dual-level mixup strategy, including within-task and across-task mixup, for enriching the diversity of nodes within each task and the diversity of tasks. Also, we incorporate the degree-based prior information to learn expressive node embeddings. Theoretically, we prove that SMILE effectively enhances the model's generalization performance. Empirically, we conduct extensive experiments on multiple benchmarks and the results suggest that SMILE significantly outperforms other baselines, including both in-domain and cross-domain few-shot settings.












 

\bibliography{seamus,YK,RAGBench}
\bibliographystyle{abbrvnat}

\newpage
\appendix
% \section{Minimax study under mild density condition}

%\title{Generating 3D \hl{Small} Binding Molecules Using Shape-Conditioned Diffusion Models with Guidance}
%\date{\vspace{-5ex}}

%\author{
%	Ziqi Chen\textsuperscript{\rm 1}, 
%	Bo Peng\textsuperscript{\rm 1}, 
%	Tianhua Zhai\textsuperscript{\rm 2},
%	Xia Ning\textsuperscript{\rm 1,3,4 \Letter}
%}
%\newcommand{\Address}{
%	\textsuperscript{\rm 1}Computer Science and Engineering, The Ohio Sate University, Columbus, OH 43210.
%	\textsuperscript{\rm 2}Perelman School of Medicine, University of Pennsylvania, Philadelphia, PA 19104.
%	\textsuperscript{\rm 3}Translational Data Analytics Institute, The Ohio Sate University, Columbus, OH 43210.
%	\textsuperscript{\rm 4}Biomedical Informatics, The Ohio Sate University, Columbus, OH 43210.
%	\textsuperscript{\Letter}ning.104@osu.edu
%}

%\newcommand\affiliation[1]{%
%	\begingroup
%	\renewcommand\thefootnote{}\footnote{#1}%
%	\addtocounter{footnote}{-1}%
%	\endgroup
%}



\setcounter{secnumdepth}{2} %May be changed to 1 or 2 if section numbers are desired.

\setcounter{section}{0}
\renewcommand{\thesection}{S\arabic{section}}

\setcounter{table}{0}
\renewcommand{\thetable}{S\arabic{table}}

\setcounter{figure}{0}
\renewcommand{\thefigure}{S\arabic{figure}}

\setcounter{algorithm}{0}
\renewcommand{\thealgorithm}{S\arabic{algorithm}}

\setcounter{equation}{0}
\renewcommand{\theequation}{S\arabic{equation}}


\begin{center}
	\begin{minipage}{0.95\linewidth}
		\centering
		\LARGE 
	Generating 3D Binding Molecules Using Shape-Conditioned Diffusion Models with Guidance (Supplementary Information)
	\end{minipage}
\end{center}
\vspace{10pt}

%%%%%%%%%%%%%%%%%%%%%%%%%%%%%%%%%%%%%%%%%%%%%
\section{Parameters for Reproducibility}
\label{supp:experiments:parameters}
%%%%%%%%%%%%%%%%%%%%%%%%%%%%%%%%%%%%%%%%%%%%%

We implemented both \SE and \methoddiff using Python-3.7.16, PyTorch-1.11.0, PyTorch-scatter-2.0.9, Numpy-1.21.5, Scikit-learn-1.0.2.
%
We trained the models using a Tesla V100 GPU with 32GB memory and a CPU with 80GB memory on Red Hat Enterprise 7.7.
%
%We released the code, data, and the trained model at Google Drive~\footnote{\url{https://drive.google.com/drive/folders/146cpjuwenKGTd6Zh4sYBy-Wv6BMfGwe4?usp=sharing}} (will release to the public on github once the manuscript is accepted).

%===================================================================
\subsection{Parameters of \SE}
%===================================================================


In \SE, we tuned the dimension of all the hidden layers including VN-DGCNN layers
(Eq.~\ref{eqn:shape_embed}), MLP layers (Eq.~\ref{eqn:se:decoder}) and
VN-In layer (Eq.~\ref{eqn:se:decoder}), and the dimension $d_p$ of generated shape latent embeddings $\shapehiddenmat$ with the grid-search algorithm in the 
parameter space presented in Table~\ref{tbl:hyper_se}.
%
We determined the optimal hyper-parameters according to the mean squared errors of the predictions of signed distances for 1,000 validation molecules that are selected as described in Section ``Data'' 
in the main manuscript.
%
The optimal dimension of all the hidden layers is 256, and the optimal dimension $d_p$ of shape latent embedding \shapehiddenmat is 128.
%
The optimal number of points $|\pc|$ in the point cloud \pc is 512.
%
We sampled 1,024 query points in $\mathcal{Z}$ for each molecule shape.
%
We constructed graphs from point clouds, which are employed to learn $\shapehiddenmat$ with VN-DGCNN layer (Eq.~\ref{eqn:shape_embed}), using the $k$-nearest neighbors based on Euclidean distance with $k=20$.
%
We set the number of VN-DGCNN layers as 4.
%
We set the number of MLP layers in the decoder (Eq.~\ref{eqn:se:decoder}) as 2.
%
We set the number of VN-In layers as 1.

%
We optimized the \SE model via Adam~\cite{adam} with its parameters (0.950, 0.999), %betas (0.95, 0.999), 
learning rate 0.001, and batch size 16.
%
We evaluated the validation loss every 2,000 training steps.
%
We scheduled to decay the learning rate with a factor of 0.6 and a minimum learning rate of 1e-6 if 
the validation loss does not decrease in 5 consecutive evaluations.
%
The optimal \SE model has 28.3K learnable parameters. 
%
We trained the \SE model %for at most 80 hours 
with $\sim$156,000 training steps.
%
The training took 80 hours with our GPUs.
%
The trained \SE model achieved the minimum validation loss at 152,000 steps.


\begin{table*}[!h]
  \centering
      \caption{{Hyper-Parameter Space for \SE Optimization}}
  \label{tbl:hyper_se}
  \begin{threeparttable}
 \begin{scriptsize}
      \begin{tabular}{
%	@{\hspace{2pt}}l@{\hspace{2pt}}
	@{\hspace{2pt}}l@{\hspace{5pt}} 
	@{\hspace{2pt}}r@{\hspace{2pt}}         
	}
        \toprule
        %Notation &
          Hyper-parameters &  Space\\
        \midrule
        %$t_a$    & 
         %hidden layer dimension         & \{16, 32, 64, 128\} \\
         %atom/node embedding dimension &  \{16, 32, 64, 128\} \\
         %$\latent^{\add}$/$\latent^{\delete}$ dimension        & \{8, 16, 32, 64\} \\
         hidden layer dimension            & \{128, 256\}\\
         dimension $d_p$ of \shapehiddenmat        &  \{64, 128\} \\
         \#points in \pc        & \{512, 1,024\} \\
         \#query points in $\mathcal{Z}$                & 1,024 \\%1024 \\%\bo{\{1024\}}\\
         \#nearest neighbors              & 20          \\
         \#VN-DGCNN layers (Eq~\ref{eqn:shape_embed})               & 4            \\
         \#MLP layers in Eq~\ref{eqn:se:decoder} & 4           \\
        \bottomrule
      \end{tabular}
%  	\begin{tablenotes}[normal,flushleft]
%  		\begin{footnotesize}
%  	
%  	\item In this table, hidden dimension represents the dimension of hidden layers and 
%  	atom/node embeddings; latent dimension represents the dimension of latent embedding \latent.
%  	\par
%  \end{footnotesize}
%  
%\end{tablenotes}
%      \begin{tablenotes}
%      \item 
%      \par
%      \end{tablenotes}
\end{scriptsize}
  \end{threeparttable}
\end{table*}

%
\begin{table*}[!h]
  \centering
      \caption{{Hyper-Parameter Space for \methoddiff Optimization}}
  \label{tbl:hyper_diff}
  \begin{threeparttable}
 \begin{scriptsize}
      \begin{tabular}{
%	@{\hspace{2pt}}l@{\hspace{2pt}}
	@{\hspace{2pt}}l@{\hspace{5pt}} 
	@{\hspace{2pt}}r@{\hspace{2pt}}         
	}
        \toprule
        %Notation &
          Hyper-parameters &  Space\\
        \midrule
        %$t_a$    & 
         %hidden layer dimension         & \{16, 32, 64, 128\} \\
         %atom/node embedding dimension &  \{16, 32, 64, 128\} \\
         %$\latent^{\add}$/$\latent^{\delete}$ dimension        & \{8, 16, 32, 64\} \\
         scalar hidden layer dimension         & 128 \\
         vector hidden layer dimension         & 32 \\
         weight of atom type loss $\xi$ (Eq.~\ref{eqn:loss})  & 100           \\
         threshold of step weight $\delta$ (Eq.~\ref{eqn:diff:obj:pos}) & 10 \\
         \#atom features $K$                   & 15 \\
         \#layers $L$ in \molpred             & 8 \\
         %\# \eqgnn/\invgnn layers     &  8 \\
         %\# heads {$n_h$} in $\text{MHA}^{\mathtt{x}}/\text{MHA}^{\mathtt{v}}$                               & 16 \\
         \#nearest neighbors {$N$}  (Eq.~\ref{eqn:geometric_embedding} and \ref{eqn:attention})            & 8          \\
         {\#diffusion steps $T$}                  & 1,000 \\
        \bottomrule
      \end{tabular}
%  	\begin{tablenotes}[normal,flushleft]
%  		\begin{footnotesize}
%  	
%  	\item In this table, hidden dimension represents the dimension of hidden layers and 
%  	atom/node embeddings; latent dimension represents the dimension of latent embedding \latent.
%  	\par
%  \end{footnotesize}
%  
%\end{tablenotes}
%      \begin{tablenotes}
%      \item 
%      \par
%      \end{tablenotes}
\end{scriptsize}
  \end{threeparttable}

\end{table*}


%===================================================================
\subsection{Parameters of \methoddiff}
%===================================================================

Table~\ref{tbl:hyper_diff} presents the parameters used to train \methoddiff.
%
In \methoddiff, we set the hidden dimensions of all the MLP layers and the scalar hidden layers in GVPs (Eq.~\ref{eqn:pred:gvp} and Eq.~\ref{eqn:mess:gvp}) as 128. %, including all the MLP layers in \methoddiff and the scalar dimension of GVP layers in Eq.~\ref{eqn:pred:gvp} and Eq.~\ref{eqn:mess:gvp}. %, and MLP layer (Eq.~\ref{eqn:diff:graph:atompred}) as 128.
%
We set the dimensions of all the vector hidden layers in GVPs as 32.
%
We set the number of layers $L$ in \molpred as 8.
%and the number of layers in graph neural networks as 8.
%
Both two GVP modules in Eq.~\ref{eqn:pred:gvp} and Eq.~\ref{eqn:mess:gvp} consist of three GVP layers. %, which consisa GVP modset the number of layer of GVP modules %is a multi-head attention layer ($\text{MHA}^{\mathtt{x}}$ or $\text{MHA}^{\mathtt{h}}$) with 16 heads.
% 
We set the number of VN-MLP layers in Eq.~\ref{eqn:shaper} as 1 and the number of MLP layers as 2 for all the involved MLP functions.
%

We constructed graphs from atoms in molecules, which are employed to learn the scalar embeddings and vector embeddings for atoms %predict atom coordinates and features  
(Eq.~\ref{eqn:geometric_embedding} and \ref{eqn:attention}), using the $N$-nearest neighbors based on Euclidean distance with $N=8$. 
%
We used $K=15$ atom features in total, indicating the atom types and its aromaticity.
%
These atom features include 10 non-aromatic atoms (i.e., ``H'', ``C'', ``N'', ``O'', ``F'', ``P'', ``S'', ``Cl'', ``Br'', ``I''), 
and 5 aromatic atoms (i.e., ``C'', ``N'', ``O'', ``P'', ``S'').
%
We set the number of diffusion steps $T$ as 1,000.
%
We set the weight $\xi$ of atom type loss (Eq.~\ref{eqn:loss}) as $100$ to balance the values of atom type loss and atom coordinate loss.
%
We set the threshold $\delta$ (Eq.~\ref{eqn:diff:obj:pos}) as 10.
%
The parameters $\beta_t^{\mathtt{x}}$ and $\beta_t^{\mathtt{v}}$ of variance scheduling in the forward diffusion process of \methoddiff are discussed in 
Supplementary Section~\ref{supp:forward:variance}.
%
%Please note that as in \squid, we did not perform extensive hyperparameter optimization for \methoddiff.
%
Following \squid, we did not perform extensive hyperparameter tunning for \methoddiff given that the used 
hyperparameters have enabled good performance.

%
We optimized the \methoddiff model via Adam~\cite{adam} with its parameters (0.950, 0.999), learning rate 0.001, and batch size 32.
%
We evaluated the validation loss every 2,000 training steps.
%
We scheduled to decay the learning rate with a factor of 0.6 and a minimum learning rate of 1e-5 if 
the validation loss does not decrease in 10 consecutive evaluations.
%
The \methoddiff model has 7.8M learnable parameters. 
%
We trained the \methoddiff model %for at most 60 hours 
with $\sim$770,000 training steps.
%
The training took 70 hours with our GPUs.
%
The trained \methoddiff achieved the minimum validation loss at 758,000 steps.

During inference, %the sampling, 
following Adams and Coley~\cite{adams2023equivariant}, we set the variance $\phi$ 
of atom-centered Gaussians as 0.049, which is used to build a set of points for shape guidance in Section ``\method with Shape Guidance'' 
in the main manuscript.
%
We determined the number of atoms in the generated molecule using the atom number distribution of training molecules that have surface shape sizes similar to the condition molecule.
%
The optimal distance threshold $\gamma$ is 0.2, and the optimal stop step $S$ for shape guidance is 300.
%
With shape guidance, each time we updated the atom position (Eq.~\ref{eqn:shape_guidance}), we randomly sampled the weight $\sigma$ from $[0.2, 0.8]$. %\bo{(XXX)}.
%
Moreover, when using pocket guidance as mentioned in Section ``\method with Pocket Guidance'' in the main manuscript, each time we updated the atom position (Eq.~\ref{eqn:pocket_guidance}), we randomly sampled the weight $\epsilon$ from $[0, 0.5]$. 
%
For each condition molecule, it took around 40 seconds on average to generate 50 molecule candidates with our GPUs.



%%%%%%%%%%%%%%%%%%%%%%%%%%%%%%%%%%%%%%%%%%%%%%
\section{Performance of \decompdiff with Protein Pocket Prior}
\label{supp:app:decompdiff}
%%%%%%%%%%%%%%%%%%%%%%%%%%%%%%%%%%%%%%%%%%%%%%

In this section, we demonstrate that \decompdiff with protein pocket prior, referred to as \decompdiffbeta, shows very limited performance in generating drug-like and synthesizable molecules compared to all the other methods, including \methodwithpguide and \methodwithsandpguide.
%
We evaluate the performance of \decompdiffbeta in terms of binding affinities, drug-likeness, and diversity.
%
We compare \decompdiffbeta with \methodwithpguide and \methodwithsandpguide and report the results in Table~\ref{tbl:comparison_results_decompdiff}.
%
Note that the results of \methodwithpguide and \methodwithsandpguide here are consistent with those in Table~\ref{tbl:overall_results_docking2} in the main manuscript.
%
As shown in Table~\ref{tbl:comparison_results_decompdiff}, while \decompdiffbeta achieves high binding affinities in Vina M and Vina D, it substantially underperforms \methodwithpguide and \methodwithsandpguide in QED and SA.
%
Particularly, \decompdiffbeta shows a QED score of 0.36, while \methodwithpguide substantially outperforms \decompdiffbeta in QED (0.77) with 113.9\% improvement.
%
\decompdiffbeta also substantially underperforms \methodwithpguide in terms of SA scores (0.55 vs 0.76).
%
These results demonstrate the limited capacity of \decompdiffbeta in generating drug-like and synthesizable molecules.
%
As a result, the generated molecules from \decompdiffbeta can have considerably lower utility compared to other methods.
%
Considering these limitations of \decompdiffbeta, we exclude it from the baselines for comparison.

\begin{table*}[!h]
	\centering
		\caption{Comparison on PMG among \methodwithpguide, \methodwithsandpguide and \decompdiffbeta}
	\label{tbl:comparison_results_decompdiff}
\begin{threeparttable}
	\begin{scriptsize}
	\begin{tabular}{
		@{\hspace{2pt}}l@{\hspace{2pt}}
		%
		%@{\hspace{2pt}}l@{\hspace{2pt}}
		%
		@{\hspace{2pt}}r@{\hspace{2pt}}
		@{\hspace{2pt}}r@{\hspace{2pt}}
		%
		@{\hspace{6pt}}r@{\hspace{6pt}}
		%
		@{\hspace{2pt}}r@{\hspace{2pt}}
		@{\hspace{2pt}}r@{\hspace{2pt}}
		%
		@{\hspace{5pt}}r@{\hspace{5pt}}
		%
		@{\hspace{2pt}}r@{\hspace{2pt}}
		@{\hspace{2pt}}r@{\hspace{2pt}}
		%
		@{\hspace{5pt}}r@{\hspace{5pt}}
		%
		@{\hspace{2pt}}r@{\hspace{2pt}}
	         @{\hspace{2pt}}r@{\hspace{2pt}}
		%
		@{\hspace{5pt}}r@{\hspace{5pt}}
		%
		@{\hspace{2pt}}r@{\hspace{2pt}}
		@{\hspace{2pt}}r@{\hspace{2pt}}
		%
		@{\hspace{5pt}}r@{\hspace{5pt}}
		%
		@{\hspace{2pt}}r@{\hspace{2pt}}
		@{\hspace{2pt}}r@{\hspace{2pt}}
		%
		@{\hspace{5pt}}r@{\hspace{5pt}}
		%
		@{\hspace{2pt}}r@{\hspace{2pt}}
		@{\hspace{2pt}}r@{\hspace{2pt}}
		%
		@{\hspace{5pt}}r@{\hspace{5pt}}
		%
		@{\hspace{2pt}}r@{\hspace{2pt}}
		%@{\hspace{2pt}}r@{\hspace{2pt}}
		%@{\hspace{2pt}}r@{\hspace{2pt}}
		}
		\toprule
		\multirow{2}{*}{method} & \multicolumn{2}{c}{Vina S$\downarrow$} & & \multicolumn{2}{c}{Vina M$\downarrow$} & & \multicolumn{2}{c}{Vina D$\downarrow$} & & \multicolumn{2}{c}{{HA\%$\uparrow$}}  & & \multicolumn{2}{c}{QED$\uparrow$} & & \multicolumn{2}{c}{SA$\uparrow$} & & \multicolumn{2}{c}{Div$\uparrow$} & %& \multirow{2}{*}{SR\%$\uparrow$} & 
		& \multirow{2}{*}{time$\downarrow$} \\
	    \cmidrule{2-3}\cmidrule{5-6} \cmidrule{8-9} \cmidrule{11-12} \cmidrule{14-15} \cmidrule{17-18} \cmidrule{20-21}
		& Avg. & Med. &  & Avg. & Med. &  & Avg. & Med. & & Avg. & Med.  & & Avg. & Med.  & & Avg. & Med.  & & Avg. & Med.  & & \\ %& & \\
		%\multirow{2}{*}{method} & \multirow{2}{*}{\#c\%} &  \multirow{2}{*}{\#u\%} &  \multirow{2}{*}{QED} & \multicolumn{3}{c}{$\nmax=50$} & & \multicolumn{2}{c}{$\nmax=1$}\\
		%\cmidrule(r){5-7} \cmidrule(r){8-10} 
		%& & & & \avgshapesim(std) & \avggraphsim(std  &  \diversity(std  & & \avgshapesim(std) & \avggraphsim(std \\
		\midrule
		%Reference                          & -5.32 & -5.66 & & -5.78 & -5.76 & & -6.63 & -6.67 & & - & - & & 0.53 & 0.49 & & 0.77 & 0.77 & & - & - & %& 23.1 & & - \\
		%\midrule
		%\multirow{4}{*}{PM} 
		%& \AR & -5.06 & -4.99 & &  -5.59 & -5.29 & &  -6.16 & -6.05 & &  37.69 & 31.00 & &  0.50 & 0.49 & &  0.66 & 0.65 & & - & - & %& 7.0 & 
		%& 7,789 \\
		%& \pockettwomol   & -4.50 & -4.21 & &  -5.70 & -5.27 & &  -6.43 & -6.25 & &  48.00 & 51.00 & &  0.58 & 0.58 & &  \textbf{0.77} & \textbf{0.78} & &  0.69 & 0.71 &  %& 24.9 & 
		%& 2,544 \\
		%& \targetdiff     & -4.88 & \underline{-5.82} & &  -6.20 & \underline{-6.36} & &  \textbf{-7.37} & \underline{-7.51} & &  57.57 & 58.27 & &  0.50 & 0.51 & &  0.60 & 0.59 & &  0.72 & 0.71 & % & 10.4 & 
		%& 1,252 \\
		 \decompdiffbeta             & -4.72 & -4.86 & & \textbf{-6.84} & \textbf{-6.91} & & \textbf{-8.85} & \textbf{-8.90} & &  {72.16} & {72.16} & &  0.36 & 0.36 & &  0.55 & 0.55 & & 0.59 & 0.59 & & 3,549 \\ 
		%-4.76 & -6.18 & &  \textbf{-6.86} & \textbf{-7.52} & &  \textbf{-8.85} & \textbf{-8.96} & &  \textbf{72.7} & \textbf{89.8} & &  0.36 & 0.34 & &  0.55 & 0.57 & & 0.59 & 0.59 & & 15.4 \\
		%& \decompdiffref  & -4.58 & -4.77 & &  -5.47 & -5.51 & &  -6.43 & -6.56 & &  47.76 & 48.66 & &  0.56 & 0.56 & &  0.70 & 0.69  & &  0.72 & 0.72 &  %& 15.2 & 
		%& 1,859 \\
		%\midrule
		%\multirow{2}{*}{PC}
		\methodwithpguide       &  \underline{-5.53} & \underline{-5.64} & & {-6.37} & -6.33 & &  \underline{-7.19} & \underline{-7.52} & &  \underline{78.75} & \textbf{94.00} & &  \textbf{0.77} & \textbf{0.80} & &  \textbf{0.76} & \textbf{0.76} & & 0.63 & 0.66 & & 462 \\
		\methodwithsandpguide   & \textbf{-5.81} & \textbf{-5.96} & &  \underline{-6.50} & \underline{-6.58} & & -7.16 & {-7.51} & &  \textbf{79.92} & \underline{93.00} & &  \underline{0.76} & \underline{0.79} & &  \underline{0.75} & \underline{0.74} & & 0.64 & 0.66 & & 561\\
		\bottomrule
	\end{tabular}%
	\begin{tablenotes}[normal,flushleft]
		\begin{footnotesize}
	\item 
\!\!Columns represent: {``Vina S'': the binding affinities between the initially generated poses of molecules and the protein pockets; 
		``Vina M'': the binding affinities between the poses after local structure minimization and the protein pockets;
		``Vina D'': the binding affinities between the poses determined by AutoDock Vina~\cite{Eberhardt2021} and the protein targets;
		``QED'': the drug-likeness score;
		``SA'': the synthesizability score;
		``Div'': the diversity among generated molecules;
		``time'': the time cost to generate molecules.}
		
		\par
		\par
		\end{footnotesize}
	\end{tablenotes}
	\end{scriptsize}
\end{threeparttable}
  \vspace{-10pt}    
\end{table*}



%===================================================================
\section{{Additional Experimental Results on SMG}}
\label{supp:app:results}
%===================================================================

%-------------------------------------------------------------------------------------------------------------------------------------
\subsection{Comparison on Shape and Graph Similarity}
\label{supp:app:results:overall_shape}
%-------------------------------------------------------------------------------------------------------------------------------------

%\ziqi{Outline for this section:
%	\begin{itemize}
%		\item \method can consistently generate molecules with novel structures (low graph similarity) and similar shapes (high shape similarity), such that these molecules have comparable binding capacity with the condition molecules, and potentially better properties as will be shown in Table~\ref{tbl:overall_results_quality_10}.
%	\end{itemize}
%}

\begin{table*}[!h]
	\centering
		\caption{Similarity Comparison on SMG}
	\label{tbl:overall_sim}
\begin{threeparttable}
	\begin{scriptsize}
	\begin{tabular}{
		@{\hspace{0pt}}l@{\hspace{8pt}}
		%
		@{\hspace{8pt}}l@{\hspace{8pt}}
		%
		@{\hspace{8pt}}c@{\hspace{8pt}}
		@{\hspace{8pt}}c@{\hspace{8pt}}
		%
	    	@{\hspace{0pt}}c@{\hspace{0pt}}
		%
		@{\hspace{8pt}}c@{\hspace{8pt}}
		@{\hspace{8pt}}c@{\hspace{8pt}}
		%
		%@{\hspace{8pt}}r@{\hspace{8pt}}
		}
		\toprule
		$\delta_g$  & method          & \avgshapesim$\uparrow$(std) & \avggraphsim$\downarrow$(std) & & \maxshapesim$\uparrow$(std) & \maxgraphsim$\downarrow$(std)       \\ %& \#n\%$\uparrow$  \\ 
		\midrule
		%\multirow{5}{0.079\linewidth}%{\hspace{0pt}0.1} & \dataset   & 0.0             & 0.628(0.139)          & 0.567(0.068)          & 0.078(0.010)          &  & 0.588(0.086)          & 0.081(0.013)          & 4.7              \\
		%&  \squid($\lambda$=0.3) & 0.0             & 0.320(0.000)          & 0.420(0.163)          & \textbf{0.056}(0.032) &  & 0.461(0.170)          & \textbf{0.065}(0.033) & 1.4              \\
		%& \squid($\lambda$=1.0) & 0.0             & 0.414(0.177)          & 0.483(0.184)          & \underline{0.064}(0.030)  &  & 0.531(0.182)          & \underline{0.073}(0.029)  & 2.4              \\
		%& \method               & \underline{1.6}     & \textbf{0.857}(0.034) & \underline{0.773}(0.045)  & 0.086(0.011)          &  & \underline{0.791}(0.053)  & 0.087(0.012)          & \underline{5.1}      \\
		%& \methodwithsguide      & \textbf{3.7}    & \underline{0.833}(0.062)  & \textbf{0.812}(0.037) & 0.088(0.009)          &  & \textbf{0.835}(0.047) & 0.089(0.010)          & \textbf{6.2}     \\ 
		%\cmidrule{2-10}
		%& improv\% & - & 36.5 & 43.2 & -53.6 &  & 42.0 & -33.8 & 31.9  \\
		%\midrule
		\multirow{6}{0.059\linewidth}{\hspace{0pt}0.3} & \dataset             & 0.745(0.037)          & \textbf{0.211}(0.026) &  & 0.815(0.039)          & \textbf{0.215}(0.047)      \\ %    & \textbf{100.0}   \\
			& \squid($\lambda$=0.3) & 0.709(0.076)          & 0.237(0.033)          &  & 0.841(0.070)          & 0.253(0.038)        \\ %  & 45.5             \\
		    & \squid($\lambda$=1.0) & 0.695(0.064)          & \underline{0.216}(0.034)  &  & 0.841(0.056)          & 0.231(0.047)        \\ %  & 84.3             \\
			& \method               & \underline{0.770}(0.039)  & 0.217(0.031)          &  & \underline{0.858}(0.038)  & \underline{0.220}(0.046)  \\ %& \underline{87.1}     \\
			& \methodwithsguide     & \textbf{0.823}(0.029) & 0.217(0.032)          &  & \textbf{0.900}(0.028) & 0.223(0.048)  \\ % & 86.0             \\ 
		%\cmidrule{2-7}
		%& improv\% & 10.5 & -2.8 &  & 7.0 & -2.3  \\ % & %-12.9  \\
		\midrule
		\multirow{6}{0.059\linewidth}{\hspace{0pt}0.5} & \dataset & 0.750(0.037)          & \textbf{0.225}(0.037) &  & 0.819(0.039)          & \textbf{0.236}(0.070)          \\ %& \textbf{100.0}   \\
			& \squid($\lambda$=0.3)  & 0.728(0.072)          & 0.301(0.054)          &  & \underline{0.888}(0.061)  & 0.355(0.088)          \\ %& 85.9             \\
			& \squid($\lambda$=1.0)  & 0.699(0.063)          & 0.233(0.043)          &  & 0.850(0.057)          & 0.263(0.080)          \\ %& \underline{99.5}     \\
			& \method               & \underline{0.771}(0.039)  & \underline{0.229}(0.043)  &  & 0.862(0.036)          & \textbf{0.236}(0.065) \\ %& 99.2             \\
			& \methodwithsguide    & \textbf{0.824}(0.029) & \underline{0.229}(0.044)  &  & \textbf{0.903}(0.027) & \underline{0.242}(0.069)  \\ %& 99.0             \\ 
		%\cmidrule{2-7}
		%& improv\% & 9.9 & -1.8 &  & 1.7 & 0.0 \\ %& -0.8  \\
		\midrule
		\multirow{6}{0.059\linewidth}{\hspace{0pt}0.7} 
		& \dataset &  0.750(0.037) & \textbf{0.226}(0.038) & & 0.819(0.039) & \underline{0.240}(0.081) \\ %& \textbf{100.0} \\
		%& \dataset & 12.3            & 0.736(0.076)          & 0.768(0.037)          & \textbf{0.228}(0.042) &  & 0.819(0.039)          & \underline{0.242}(0.085)  & \textbf{100.0}   \\
			& \squid($\lambda$=0.3) &  0.735(0.074)          & 0.328(0.070)          &  & \underline{0.900}(0.062)  & 0.435(0.143)          \\ %& 95.4             \\
			& \squid($\lambda$=1.0) &  0.699(0.064)          & 0.234(0.045)          &  & 0.851(0.057)          & 0.268(0.090)          \\ %& \underline{99.9}     \\
			& \method               &  \underline{0.771}(0.039)  & \underline{0.229}(0.043)  &  & 0.862(0.036)          & \textbf{0.237}(0.066) \\ %& 99.3             \\
			& \methodwithsguide     &  \textbf{0.824}(0.029) & 0.230(0.045)          &  & \textbf{0.903}(0.027) & 0.244(0.074)          \\ %& 99.2             \\ 
		%\cmidrule{2-7}
		%& improv\% & 9.9 & -1.3 &  & 0.3 & 1.3 \\%& -0.7  \\
		\midrule
		\multirow{6}{0.059\linewidth}{\hspace{0pt}1.0} 
		& \dataset & 0.750(0.037)          & \textbf{0.226}(0.038) &  & 0.819(0.039)          & \underline{0.242}(0.085)  \\%& \textbf{100.0}  \\
		& \squid($\lambda$=0.3) & 0.740(0.076)          & 0.349(0.088)          &  & \textbf{0.909}(0.065) & 0.547(0.245)       \\ %   & \textbf{100.0}  \\
		& \squid($\lambda$=1.0) & 0.699(0.064)          & 0.235(0.045)          &  & 0.851(0.057)          & 0.271(0.097)          \\ %& \textbf{100.0}   \\
		& \method               & \underline{0.771}(0.039)  & \underline{0.229}(0.043)  &  & 0.862(0.036)          & \textbf{0.237}(0.066) \\ %& \underline{99.3}  \\
		& \methodwithsguide      & \textbf{0.824}(0.029) & 0.230(0.045)          &  & \underline{0.903}(0.027)  & 0.244(0.076)          \\ %& 99.2            \\
		%\cmidrule{2-7}
		%& improv\% &  9.9               & -1.3              &  & -0.7              & -2.1           \\ %       & -0.7 \\
		\bottomrule
	\end{tabular}%
	\begin{tablenotes}[normal,flushleft]
		\begin{footnotesize}
	\item 
\!\!Columns represent: ``$\delta_g$'': the graph similarity constraint; 
%``\#d\%'': the percentage of molecules that satisfy the graph similarity constraint and are with high \shapesim ($\shapesim>=0.8$);
%``\diversity'': the diversity among the generated molecules;
``\avgshapesim/\avggraphsim'': the average of shape or graph similarities between the condition molecules and generated molecules with $\graphsim<=\delta_g$;
``\maxshapesim'': the maximum of shape similarities between the condition molecules and generated molecules with $\graphsim<=\delta_g$;
``\maxgraphsim'': the graph similarities between the condition molecules and the molecules with the maximum shape similarities and $\graphsim<=\delta_g$;
%``\#n\%'': the percentage of molecules that satisfy the graph similarity constraint ($\graphsim<=\delta_g$).
%
``$\uparrow$'' represents higher values are better, and ``$\downarrow$'' represents lower values are better.
%
 Best values are in \textbf{bold}, and second-best values are \underline{underlined}. 
\par
		\par
		\end{footnotesize}
	\end{tablenotes}
\end{scriptsize}
\end{threeparttable}
  \vspace{-10pt}    
\end{table*}
%\label{tbl:overall_sim}


{We evaluate the shape similarity \shapesim and graph similarity \graphsim of molecules generated from}
%Table~\ref{tbl:overall_sim} presents the comparison of shape-conditioned molecule generation among 
\dataset, \squid, \method and \methodwithsguide under different graph similarity constraints  ($\delta_g$=1.0, 0.7, 0.5, 0.3). 
%
%During the evaluation, for each molecule in the test set, all the methods are employed to generate or identify 50 molecules with similar shapes.
%
We calculate evaluation metrics using all the generated molecules satisfying the graph similarity constraints.
%
Particularly, when $\delta_g$=1.0, we do not filter out any molecules based on the constraints and directly calculate metrics on all the generated molecules.
%
When $\delta_g$=0.7, 0.5 or 0.3, we consider only generated molecules with similarities lower than $\delta_g$.
%
Based on \shapesim and \graphsim as described in Section ``Evaluation Metrics'' in the main manuscript,
we calculate the following metrics using the subset of molecules with \graphsim lower than $\delta_g$, from a set of 50 generated molecules for each test molecule and report the average of  these metrics across all test molecules:
%
(1) \avgshapesim\ measures the average \shapesim across each subset of generated molecules with $\graphsim$ lower than $\delta_g$; %per test molecule, with the overall average calculated across all test molecules; }%the 50 generated molecules for each test molecule, averaged across all test molecules;
(2) \avggraphsim\ calculates the average \graphsim for each set; %, with these means averaged across all test molecules}; %} 50 molecules, %\bo{@Ziqi rephrase}, with results averaged on the test set;\ziqi{with the average computed over the test set; }
(3) \maxshapesim\ determines the maximum \shapesim within each set; %, with these maxima averaged across all test molecules; }%\hl{among 50 molecules}, averaged across all test molecules;
(4) \maxgraphsim\ measures the \graphsim of the molecule with maximum \shapesim in each set. %, averaged across all test molecules; }%\hl{among 50 molecules}, averaged across all test molecules;

%
As shown in Table~\ref{tbl:overall_sim}, \method and \methodwithsguide demonstrate outstanding performance in terms of the average shape similarities (\avgshapesim) and the average graph similarities (\avggraphsim) among generated molecules.
%
%\ziqi{
%Table~\ref{tbl:overall} also shows that \method and \methodwithsguide consistently outperform all the baseline methods in average shape similarities (\avgshapesim) and only slightly underperform 
%the best baseline \dataset in average graph similarities (\avggraphsim).
%}
%
Specifically, when $\delta_g$=0.3, \methodwithsguide achieves a substantial 10.5\% improvement in \avgshapesim\ over the best baseline \dataset. 
%
In terms of \avggraphsim, \methodwithsguide also achieves highly comparable performance with \dataset (0.217 vs 0.211, in \avggraphsim, lower values indicate better performance).
%
%This trend remains consistent across various $\delta_g$ values.
This trend remains consistent when applying various similarity constraints (i.e., $\delta_g$) as shown in Table~\ref{tbl:overall_sim}.


Similarly, \method and \methodwithsguide demonstrate superior performance in terms of the average maximum shape similarity across generated molecules for all test molecules (\maxshapesim), as well as the average graph similarity of the molecules with the maximum shape similarities (\maxgraphsim). %maximum shape similarities of generated molecules (\maxshapesim) and the average graph similarities of molecules with the maximum shape similarities (\maxgraphsim). %\bo{\maxgraphsim is misleading... how about $\text{avgMSim}_\text{g}$}
%
%\bo{
%in terms of the maximum shape similarities (\maxshapesim) and the maximum graph similarities (\maxgraphsim) among all the generated molecules.
%@Ziqi are the metrics maximum values or the average of maximum values?
%}
%
Specifically, at \maxshapesim, Table~\ref{tbl:overall_sim} shows that \methodwithsguide outperforms the best baseline \squid ($\lambda$=0.3) when $\delta_g$=0.3, 0.5, and 0.7, and only underperforms
it by 0.7\% when $\delta$=1.0.
%
We also note that the molecules generated by {\methodwithsguide} with the maximum shape similarities have substantially lower graph similarities ({\maxgraphsim}) compared to those generated by {\squid} ({$\lambda$}=0.3).
%\hl{We also note that the molecules with the maximum shape similarities generated by {\methodwithsguide} are with significantly lower graph similarities ({\maxgraphsim}) than those generated by {\squid} ({$\lambda$}=0.3).}
%
%\bo{@Ziqi please rephrase the language}
%
%\bo{
%@Ziqi the conclusion is not obvious. You may want to remind the meaning of \maxshapesim and \maxgraphsim here, and based on what performance you say this.
%}
%
%\bo{\st{This also underscores the ability of {\methodwithsguide} in generating molecules with similar shapes to condition molecules and novel graph structures.}}
%
As evidenced by these results, \methodwithsguide features strong capacities of generating molecules with similar shapes yet novel graph structures compared to the condition molecule, facilitating the discovery of promising drug candidates.
%

\begin{comment}
\ziqi{replace \#n\% with the percentage of novel molecules that do not exist in the dataset and update the discussion accordingly}
%\ziqi{
Table~\ref{tbl:overall_sim} also presents \bo{\#n\%}, the percentage of molecules generated by each method %\st{(\#n\%)} 
with graph similarities lower than the constraint $\delta_g$. 
%
%\bo{
%Table~\ref{tbl:overall_sim} also presents \#n\%, the percentage of generated molecules with graph similarities lower than the constraint $\delta_g$, of different methods. 
%}
%
As shown in Table~\ref{tbl:overall_sim},  when a restricted constraint (i.e., $\delta_g$=0.3) is applied, \method and \methodwithsguide could still generate a sufficient number of molecules satisfying the constraint.
%
Particularly, when $\delta_g$=0.3, \method outperforms \squid with $\lambda$=0.3 by XXX and \squid with $\lambda$=1.0 by XXX.
% achieve the second and the third in \#n\% and only underperform the best baseline \dataset.
%
This demonstrates the ability of \method in generating molecules with novel structures. 
%
When $\delta_g$=0.5, 0.7 and 1.0, both methods generate over 99.0\% of molecules satisfying the similarity constraint $\delta_g$.
%
%Note that \dataset is guaranteed to identify at least 50 molecules satisfying the $\delta_g$ by searching within a training dataset of diverse molecules.
%
Note that \dataset is a search algorithm that always first identifies the molecules satisfying $\delta_g$ and then selects the top-50 molecules of the highest shape similarities among them. 
%
Due to the diverse molecules in %\hl{the subset} \bo{@Ziqi why do you want to stress subset?} of 
the training set, \dataset can always identify at least 50 molecules under different $\delta_g$ and thus achieve 100\% in \#n\%.
%
%\bo{
%Note that \dataset is a search algorithm that always generate molecules XXX
%@Ziqi
%We need to discuss here. For \dataset, \#n\% in this table does not look aligned with that in Fig 1 if the highlighted defination is correct...
%}
%
%Thus, \dataset achieves 100.0\% in \#n\% under different $\delta_g$.
%
It is also worth noting that when $\delta_g$=1.0, \#n\% reflects the validity among all the generated molecules. 
%
As shown in Table~\ref{tbl:overall_sim}, \method and \methodwithsguide are able to generate 99.3\% and 99.2\% valid molecules.
%
This demonstrates their ability to effectively capture the underlying chemical rules in a purely data-driven manner without relying on any prior knowledge (e.g., fragments) as \squid does.
%
%\bo{
%@Ziqi I feel this metric is redundant with the avg graph similarity when constraint is 1.0. Generally, if the avg similarity is small. You have more mols satisfying the requirement right?
%}
\end{comment}

Table~\ref{tbl:overall_sim} also shows that by incorporating shape guidance, \methodwithsguide
%\bo{
%@Ziqi where does this come from...
%}
substantially outperforms \method in both \avgshapesim and \maxshapesim, while maintaining comparable graph similarities (i.e., \avggraphsim\ and \maxgraphsim).
%
Particularly, when $\delta_g$=0.3, \methodwithsguide 
establishes a considerable improvement of 6.9\% and 4.9\%
%\bo{\st{achieves 6.9\% and 4.9\% improvements}} 
over \method in \avgshapesim and \maxshapesim, respectively. 
%
%\hl{In the meanwhile}, 
%\bo{@Ziqi it is not the right word...}
Meanwhile, \methodwithsguide achieves the same \avggraphsim with \method and only slightly underperforms \method in \maxgraphsim (0.223 vs 0.220).
%\bo{
%XXX also achieves XXX
%}
%it maintains the same \avggraphsim\ with \method and only slightly underperforms \method in \maxgraphsim (0.223 vs 0.220).
%
%Compared with \method, \methodwithsguide consistently generates molecules with higher shape similarities while maintaining comparable graph similarities.
%
%\bo{
%@Ziqi you may want to highlight the utility of "generating molecules with higher shape similarities while maintaining comparable graph similarities" in real drug discovery applications.
%
%
%\bo{
%@Ziqi You did not present the details of method yet...
%}
%
%\methodwithsguide leverages additional shape guidance to push the predicted atoms to the shape of condition molecules \bo{and XXX (@Ziqi boosts the shape similarities XXX)} , as will be discussed in Section ``\method with Shape Guidance'' later.
%
The superior performance of \methodwithsguide suggests that the incorporation of shape guidance effectively boosts the shape similarities of generated molecules without compromising graph similarities.
%
%This capability could be crucial in drug discovery, 
%\bo{@Ziqi it is a strong statement. Need citations here}, 
%as it enables the discovery of drug candidates that are both more potentially effective due to the improved shape similarities and novel induced by low graph similarities.
%as it could enable the identification of candidates with similar binding patterns %with the condition molecule (i.e., high shape similarities) 
%(i.e., high shape similarities) and graph structures distinct from the condition molecules (i.e., low graph similarities).
%\bo{\st{and enjoys novel structures (i.e., low graph similarities) with potentially better properties. } \ziqi{change enjoys}}
%\bo{
%and enjoys potentially better properties (i.e., low graph similarities). \ziqi{this looks weird to me... need to discuss}
%}
%\st{potentially better properties (i.e., low graph similarities).}}

%-------------------------------------------------------------------------------------------------------------------------------------
\subsection{Comparison on Validity and Novelty}
\label{supp:app:results:valid_novel}
%-------------------------------------------------------------------------------------------------------------------------------------

We evaluate the ability of \method and \squid to generate molecules with valid and novel 2D molecular graphs.
%
We calculate the percentages of the valid and novel molecules among all the generated molecules.
%
As shown in Table~\ref{tbl:validity_novelty}, both \method and \methodwithsguide outperform \squid with $\lambda$=0.3 and $\lambda$=1.0 in generating novel molecules.
%
Particularly, almost all valid molecules generated by \method and \methodwithsguide are novel (99.8\% and 99.9\% at \#n\%), while the best baseline \squid with $\lambda$=0.3 achieves 98.4\% in novelty.
%
In terms of the percentage of valid and novel molecules among all the generated ones (\#v\&n\%), \method and \methodwithsguide again outperform \squid with $\lambda$=0.3 and $\lambda$=1.0.
%
We also note that at \#v\%,  \method (99.1\%) and \methodwithsguide (99.2\%) slightly underperform \squid with $\lambda$=0.3 and $\lambda$=1.0 (100.0\%) in generating valid molecules.
%
\squid guarantees the validity of generated molecules by incorporating valence rules into the generation process and ensuring it to avoid fragments that violate these rules.
%
Conversely, \method and \methodwithsguide use a purely data-driven approach to learn the generation of valid molecules.
%
These results suggest that, even without integrating valence rules, \method and \methodwithsguide can still achieve a remarkably high percentage of valid and novel generated molecules.

\begin{table*}
	\centering
		\caption{Comparison on Validity and Novelty between \method and \squid}
	\label{tbl:validity_novelty}
	\begin{scriptsize}
\begin{threeparttable}
%	\setlength\tabcolsep{0pt}
	\begin{tabular}{
		@{\hspace{3pt}}l@{\hspace{10pt}}
		%
		@{\hspace{10pt}}r@{\hspace{10pt}}
		%
		@{\hspace{10pt}}r@{\hspace{10pt}}
		%
		@{\hspace{10pt}}r@{\hspace{3pt}}
		}
		\toprule
		method & \#v\% & \#n\% & \#v\&n\% \\
		\midrule
		\squid ($\lambda$=0.3) & \textbf{100.0} & 96.7 & 96.7 \\
		\squid ($\lambda$=1.0) & \textbf{100.0} & 98.4 & 98.4 \\
		\method & 99.1 & 99.8 & 98.9 \\
		\methodwithsguide & 99.2 & \textbf{99.9} & \textbf{99.1} \\
		\bottomrule
	\end{tabular}%
	%
	\begin{tablenotes}[normal,flushleft]
		\begin{footnotesize}
	\item 
\!\!Columns represent: ``\#v\%'': the percentage of generated molecules that are valid;
		``\#n\%'': the percentage of valid molecules that are novel;
		``\#v\&n\%'': the percentage of generated molecules that are valid and novel.
		Best values are in \textbf{bold}. 
		\par
		\end{footnotesize}
	\end{tablenotes}
\end{threeparttable}
\end{scriptsize}
\end{table*}


%-------------------------------------------------------------------------------------------------------------------------------------
\subsection{Additional Quality Comparison between Desirable Molecules Generated by \method and \squid}
\label{supp:app:results:quality_desirable}
%-------------------------------------------------------------------------------------------------------------------------------------

\begin{table*}[!h]
	\centering
		\caption{Comparison on Quality of Generated Desirable Molecules between \method and \squid ($\delta_g$=0.5)}
	\label{tbl:overall_results_quality_05}
	\begin{scriptsize}
\begin{threeparttable}
	\begin{tabular}{
		@{\hspace{0pt}}l@{\hspace{16pt}}
		@{\hspace{0pt}}l@{\hspace{2pt}}
		%
		@{\hspace{6pt}}c@{\hspace{6pt}}
		%
		%@{\hspace{3pt}}c@{\hspace{3pt}}
		@{\hspace{3pt}}c@{\hspace{3pt}}
		@{\hspace{3pt}}c@{\hspace{3pt}}
		@{\hspace{3pt}}c@{\hspace{3pt}}
		@{\hspace{3pt}}c@{\hspace{3pt}}
		%
		%
		}
		\toprule
		group & metric & 
        %& \dataset 
        & \squid ($\lambda$=0.3) & \squid ($\lambda$=1.0)  &  \method & \methodwithsguide  \\
		%\multirow{2}{*}{method} & \multirow{2}{*}{\#c\%} &  \multirow{2}{*}{\#u\%} &  \multirow{2}{*}{QED} & \multicolumn{3}{c}{$\nmax=50$} & & \multicolumn{2}{c}{$\nmax=1$}\\
		%\cmidrule(r){5-7} \cmidrule(r){8-10} 
		%& & & & \avgshapesim(std) & \avggraphsim(std  &  \diversity(std  & & \avgshapesim(std) & \avggraphsim(std \\
		\midrule
		\multirow{2}{*}{stability}
		& atom stability ($\uparrow$) & 
        %& 0.990 
        & \textbf{0.996} & 0.995 & 0.992 & 0.989     \\
		& mol stability ($\uparrow$) & 
        %& 0.875 
        & \textbf{0.948} & 0.947 & 0.886 & 0.839    \\
		%\midrule
		%\multirow{3}{*}{Drug-likeness} 
		%& QED ($\uparrow$) & 
        %& \textbf{0.805} 
        %& 0.766 & 0.760 & 0.755 & 0.751    \\
	%	& SA ($\uparrow$) & 
        %& \textbf{0.874} 
        %& 0.814 & 0.813 & 0.699 & 0.692    \\
	%	& Lipinski ($\uparrow$) & 
        %& \textbf{4.999} 
        %& 4.979 & 4.980 & 4.967 & 4.975    \\
		\midrule
		\multirow{4}{*}{3D structures} 
		& RMSD ($\downarrow$) & 
        %& \textbf{0.419} 
        & 0.907 & 0.906 & 0.897 & \textbf{0.881}    \\
		& JS. bond lengths ($\downarrow$) & 
        %& \textbf{0.286} 
        & 0.457 & 0.477 & 0.436 & \textbf{0.428}    \\
		& JS. bond angles ($\downarrow$) & 
        %& \textbf{0.078} 
        & 0.269 & 0.289 & \textbf{0.186} & 0.200    \\
		& JS. dihedral angles ($\downarrow$) & 
        %& \textbf{0.151} 
        & 0.199 & 0.209 & \textbf{0.168} & 0.170    \\
		\midrule
		\multirow{5}{*}{2D structures} 
		& JS. \#bonds per atoms ($\downarrow$) & 
        %& 0.325 
        & 0.291 & 0.331 & \textbf{0.176} & 0.181    \\
		& JS. basic bond types ($\downarrow$) & 
        %& \textbf{0.055} 
        & \textbf{0.071} & 0.083 & 0.181 & 0.191    \\
		%& JS. freq. bond types ($\downarrow$) & 
        %& \textbf{0.089} 
        %& 0.123 & 0.130 & 0.245 & 0.254    \\
		%& JS. freq. bond pairs ($\downarrow$) & 
        %& \textbf{0.078} 
        %& 0.085 & 0.089 & 0.209 & 0.221    \\
		%& JS. freq. bond triplets ($\downarrow$) & 
        %& \textbf{0.089} 
        %& 0.097 & 0.114 & 0.211 & 0.223    \\
		%\midrule
		%\multirow{3}{*}{Rings} 
		& JS. \#rings ($\downarrow$) & 
        %& 0.142 
        & 0.280 & 0.330 & \textbf{0.043} & 0.049    \\
		& JS. \#n-sized rings ($\downarrow$) & 
        %& \textbf{0.055} 
        & \textbf{0.077} & 0.091 & 0.099 & 0.112    \\
		& \#Intersecting rings ($\uparrow$) & 
        %& \textbf{6} 
        & \textbf{6} & 5 & 4 & 5    \\
		%\method (+bt)            & 100.0 & 98.0 & 100.0 & 0.742 & 0.772 (0.040) & 0.211 (0.033) & & 0.862 (0.036) & 0.211 (0.033) & 0.743 (0.043) \\
		%\methodwithguide (+bt)    & 99.8 & 98.0 & 100.0 & 0.736 & 0.814 (0.031) & 0.193 (0.042) & & 0.895 (0.029) & 0.193 (0.042) & 0.745 (0.045) \\
		%
		\bottomrule
	\end{tabular}%
	\begin{tablenotes}[normal,flushleft]
		\begin{footnotesize}
	\item 
\!\!Rows represent:  {``atom stability'': the proportion of stable atoms that have the correct valency; 
		``molecule stability'': the proportion of generated molecules with all atoms stable;
		%``QED'': the drug-likeness score;
		%``SA'': the synthesizability score;
		%``Lipinski'': the Lipinski 
		``RMSD'': the root mean square deviation (RMSD) between the generated 3D structures of molecules and their optimal conformations; % identified via energy minimization;
		``JS. bond lengths/bond angles/dihedral angles'': the Jensen-Shannon (JS) divergences of bond lengths, bond angles and dihedral angles;
		``JS. \#bonds per atom/basic bond types/\#rings/\#n-sized rings'': the JS divergences of bond counts per atom, basic bond types, counts of all rings, and counts of n-sized rings;
		%``JS. \#rings/\#n-sized rings'': the JS divergences of the total counts of rings and the counts of n-sized rings;
		``\#Intersecting rings'': the number of rings observed in the top-10 frequent rings of both generated and real molecules. } \par
		\par
		\end{footnotesize}
	\end{tablenotes}
\end{threeparttable}
\end{scriptsize}
\end{table*}

%\label{tbl:overall_quality05}

\begin{table*}[!h]
	\centering
		\caption{Comparison on Quality of Generated Desirable Molecules between \method and \squid ($\delta_g$=0.7)}
	\label{tbl:overall_results_quality_07}
	\begin{scriptsize}
\begin{threeparttable}
	\begin{tabular}{
		@{\hspace{0pt}}l@{\hspace{14pt}}
		@{\hspace{0pt}}l@{\hspace{2pt}}
		%
		@{\hspace{4pt}}c@{\hspace{4pt}}
		%
		%@{\hspace{3pt}}c@{\hspace{3pt}}
		@{\hspace{3pt}}c@{\hspace{3pt}}
		@{\hspace{3pt}}c@{\hspace{3pt}}
		@{\hspace{3pt}}c@{\hspace{3pt}}
		@{\hspace{3pt}}c@{\hspace{3pt}}
		%
		%
		}
		\toprule
		group & metric & 
        %& \dataset 
        & \squid ($\lambda$=0.3) & \squid ($\lambda$=1.0)  &  \method & \methodwithsguide  \\
		%\multirow{2}{*}{method} & \multirow{2}{*}{\#c\%} &  \multirow{2}{*}{\#u\%} &  \multirow{2}{*}{QED} & \multicolumn{3}{c}{$\nmax=50$} & & \multicolumn{2}{c}{$\nmax=1$}\\
		%\cmidrule(r){5-7} \cmidrule(r){8-10} 
		%& & & & \avgshapesim(std) & \avggraphsim(std  &  \diversity(std  & & \avgshapesim(std) & \avggraphsim(std \\
		\midrule
		\multirow{2}{*}{stability} 
		& atom stability ($\uparrow$) & 
        %&  0.990 
        & \textbf{0.995} & 0.995 & 0.992 & 0.988 \\
		& molecule stability ($\uparrow$) & 
        %& 0.876 
        & 0.944 & \textbf{0.947} & 0.885 & 0.839 \\
		\midrule
		%\multirow{3}{*}{Drug-likeness} 
		%& QED ($\uparrow$) & 
        %& \textbf{0.805} 
        %& 0.766 & 0.760 & 0.755 & 0.751    \\
	%	& SA ($\uparrow$) & 
        %& \textbf{0.874} 
        %& 0.814 & 0.813 & 0.699 & 0.692    \\
	%	& Lipinski ($\uparrow$) & 
        %& \textbf{4.999} 
        %& 4.979 & 4.980 & 4.967 & 4.975    \\
	%	\midrule
		\multirow{4}{*}{3D structures} 
		& RMSD ($\downarrow$) & 
        %& \textbf{0.420} 
        & 0.897 & 0.906 & 0.897 & \textbf{0.881}    \\
		& JS. bond lengths ($\downarrow$) & 
        %& \textbf{0.286} 
        & 0.457 & 0.477 & 0.436 & \textbf{0.428}    \\
		& JS. bond angles ($\downarrow$) & 
        %& \textbf{0.078} 
        & 0.269 & 0.289 & \textbf{0.186} & 0.200    \\
		& JS. dihedral angles ($\downarrow$) & 
        %& \textbf{0.151} 
        & 0.199 & 0.209 & \textbf{0.168} & 0.170    \\
		\midrule
		\multirow{5}{*}{2D structures} 
		& JS. \#bonds per atoms ($\downarrow$) & 
        %& 0.325 
        & 0.285 & 0.329 & \textbf{0.176} & 0.181    \\
		& JS. basic bond types ($\downarrow$) & 
        %& \textbf{0.055} 
        & \textbf{0.067} & 0.083 & 0.181 & 0.191    \\
	%	& JS. freq. bond types ($\downarrow$) & 
        %& \textbf{0.089} 
        %& 0.123 & 0.130 & 0.245 & 0.254    \\
	%	& JS. freq. bond pairs ($\downarrow$) & 
        %& \textbf{0.078} 
        %& 0.085 & 0.089 & 0.209 & 0.221    \\
	%	& JS. freq. bond triplets ($\downarrow$) & 
        %& \textbf{0.089} 
        %& 0.097 & 0.114 & 0.211 & 0.223    \\
	%	\midrule
	%	\multirow{3}{*}{Rings} 
		& JS. \#rings ($\downarrow$) & 
        %& 0.143 
        & 0.273 & 0.328 & \textbf{0.043} & 0.049    \\
		& JS. \#n-sized rings ($\downarrow$) & 
        %& \textbf{0.055} 
        & \textbf{0.076} & 0.091 & 0.099 & 0.112    \\
		& \#Intersecting rings ($\uparrow$) & 
        %& \textbf{6} 
        & \textbf{6} & 5 & 4 & 5    \\
		%\method (+bt)            & 100.0 & 98.0 & 100.0 & 0.742 & 0.772 (0.040) & 0.211 (0.033) & & 0.862 (0.036) & 0.211 (0.033) & 0.743 (0.043) \\
		%\methodwithguide (+bt)    & 99.8 & 98.0 & 100.0 & 0.736 & 0.814 (0.031) & 0.193 (0.042) & & 0.895 (0.029) & 0.193 (0.042) & 0.745 (0.045) \\
		%
		\bottomrule
	\end{tabular}%
	\begin{tablenotes}[normal,flushleft]
		\begin{footnotesize}
	\item 
\!\!Rows represent:  {``atom stability'': the proportion of stable atoms that have the correct valency; 
		``molecule stability'': the proportion of generated molecules with all atoms stable;
		%``QED'': the drug-likeness score;
		%``SA'': the synthesizability score;
		%``Lipinski'': the Lipinski 
		``RMSD'': the root mean square deviation (RMSD) between the generated 3D structures of molecules and their optimal conformations; % identified via energy minimization;
		``JS. bond lengths/bond angles/dihedral angles'': the Jensen-Shannon (JS) divergences of bond lengths, bond angles and dihedral angles;
		``JS. \#bonds per atom/basic bond types/\#rings/\#n-sized rings'': the JS divergences of bond counts per atom, basic bond types, counts of all rings, and counts of n-sized rings;
		%``JS. \#rings/\#n-sized rings'': the JS divergences of the total counts of rings and the counts of n-sized rings;
		``\#Intersecting rings'': the number of rings observed in the top-10 frequent rings of both generated and real molecules. } \par
		\par
		\end{footnotesize}
	\end{tablenotes}
\end{threeparttable}
\end{scriptsize}
\end{table*}

%\label{tbl:overall_quality07}

\begin{table*}[!h]
	\centering
		\caption{Comparison on Quality of Generated Desirable Molecules between \method and \squid ($\delta_g$=1.0)}
	\label{tbl:overall_results_quality_10}
	\begin{scriptsize}
\begin{threeparttable}
	\begin{tabular}{
		@{\hspace{0pt}}l@{\hspace{14pt}}
		@{\hspace{0pt}}l@{\hspace{2pt}}
		%
		@{\hspace{4pt}}c@{\hspace{4pt}}
		%
		%@{\hspace{3pt}}c@{\hspace{3pt}}
		@{\hspace{3pt}}c@{\hspace{3pt}}
		@{\hspace{3pt}}c@{\hspace{3pt}}
		@{\hspace{3pt}}c@{\hspace{3pt}}
		@{\hspace{3pt}}c@{\hspace{3pt}}
		%
		%
		}
		\toprule
		group & metric & 
        %& \dataset 
        & \squid ($\lambda$=0.3) & \squid ($\lambda$=1.0)  &  \method & \methodwithsguide \\
		%\multirow{2}{*}{method} & \multirow{2}{*}{\#c\%} &  \multirow{2}{*}{\#u\%} &  \multirow{2}{*}{QED} & \multicolumn{3}{c}{$\nmax=50$} & & \multicolumn{2}{c}{$\nmax=1$}\\
		%\cmidrule(r){5-7} \cmidrule(r){8-10} 
		%& & & & \avgshapesim(std) & \avggraphsim(std  &  \diversity(std  & & \avgshapesim(std) & \avggraphsim(std \\
		\midrule
		\multirow{2}{*}{stability}
		& atom stability ($\uparrow$) & 
        %& 0.990 
        & \textbf{0.995} & \textbf{0.995} & 0.992 & 0.988     \\
		& mol stability ($\uparrow$) & 
        %& 0.876 
        & 0.942 & \textbf{0.947} & 0.885 & 0.839    \\
		\midrule
	%	\multirow{3}{*}{Drug-likeness} 
	%	& QED ($\uparrow$) & 
        %& \textbf{0.805} 
        %& \textbf{0.766} & 0.760 & 0.755 & 0.751    \\
	%	& SA ($\uparrow$) & 
        %& \textbf{0.874} 
        %& \textbf{0.813} & \textbf{0.813} & 0.699 & 0.692    \\
	%	& Lipinski ($\uparrow$) & 
        %& \textbf{4.999} 
        %& 4.979 & \textbf{4.980} & 4.967 & 4.975    \\
	%	\midrule
		\multirow{4}{*}{3D structures} 
		& RMSD ($\downarrow$) & 
        %& \textbf{0.420} 
        & 0.898 & 0.906 & 0.897 & \textbf{0.881}    \\
		& JS. bond lengths ($\downarrow$) & 
        %& \textbf{0.286} 
        & 0.457 & 0.477 & 0.436 & \textbf{0.428}    \\
		& JS. bond angles ($\downarrow$) & 
        %& \textbf{0.078} 
        & 0.269 & 0.289 & \textbf{0.186} & 0.200   \\
		& JS. dihedral angles ($\downarrow$) & 
        %& \textbf{0.151} 
        & 0.199 & 0.209 & \textbf{0.168} & 0.170    \\
		\midrule
		\multirow{5}{*}{2D structures} 
		& JS. \#bonds per atoms ($\downarrow$) & 
        %& 0.325 
        & 0.280 & 0.330 & \textbf{0.176} & 0.181    \\
		& JS. basic bond types ($\downarrow$) & 
        %& \textbf{0.055} 
        & \textbf{0.066} & 0.083 & 0.181 & 0.191   \\
	%	& JS. freq. bond types ($\downarrow$) & 
        %& \textbf{0.089} 
        %& \textbf{0.123} & 0.130 & 0.245 & 0.254    \\
	%	& JS. freq. bond pairs ($\downarrow$) & 
        %& \textbf{0.078} 
        %& \textbf{0.085} & 0.089 & 0.209 & 0.221    \\
	%	& JS. freq. bond triplets ($\downarrow$) & 
        %& \textbf{0.089} 
        %& \textbf{0.097} & 0.114 & 0.211 & 0.223    \\
		%\midrule
		%\multirow{3}{*}{Rings} 
		& JS. \#rings ($\downarrow$) & 
        %& 0.143 
        & 0.269 & 0.328 & \textbf{0.043} & 0.049    \\
		& JS. \#n-sized rings ($\downarrow$) & 
        %& \textbf{0.055} 
        & \textbf{0.075} & 0.091 & 0.099 & 0.112    \\
		& \#Intersecting rings ($\uparrow$) & 
        %& \textbf{6} 
        & \textbf{6} & 5 & 4 & 5    \\
		%\method (+bt)            & 100.0 & 98.0 & 100.0 & 0.742 & 0.772 (0.040) & 0.211 (0.033) & & 0.862 (0.036) & 0.211 (0.033) & 0.743 (0.043) \\
		%\methodwithguide (+bt)    & 99.8 & 98.0 & 100.0 & 0.736 & 0.814 (0.031) & 0.193 (0.042) & & 0.895 (0.029) & 0.193 (0.042) & 0.745 (0.045) \\
		%
		\bottomrule
	\end{tabular}%
	\begin{tablenotes}[normal,flushleft]
		\begin{footnotesize}
	\item 
\!\!Rows represent:  {``atom stability'': the proportion of stable atoms that have the correct valency; 
		``molecule stability'': the proportion of generated molecules with all atoms stable;
		%``QED'': the drug-likeness score;
		%``SA'': the synthesizability score;
		%``Lipinski'': the Lipinski 
		``RMSD'': the root mean square deviation (RMSD) between the generated 3D structures of molecules and their optimal conformations; % identified via energy minimization;
		``JS. bond lengths/bond angles/dihedral angles'': the Jensen-Shannon (JS) divergences of bond lengths, bond angles and dihedral angles;
		``JS. \#bonds per atom/basic bond types/\#rings/\#n-sized rings'': the JS divergences of bond counts per atom, basic bond types, counts of all rings, and counts of n-sized rings;
		%``JS. \#rings/\#n-sized rings'': the JS divergences of the total counts of rings and the counts of n-sized rings;
		``\#Intersecting rings'': the number of rings observed in the top-10 frequent rings of both generated and real molecules. } \par
		\par
		\end{footnotesize}
	\end{tablenotes}
\end{threeparttable}
\end{scriptsize}
\end{table*}

%\label{tbl:overall_quality10}

Similar to Table~\ref{tbl:overall_results_quality_desired} in the main manuscript, we present the performance comparison on the quality of desirable molecules generated by different methods under different graph similarity constraints $\delta_g$=0.5, 0.7 and 1.0, as detailed in Table~\ref{tbl:overall_results_quality_05}, Table~\ref{tbl:overall_results_quality_07}, and Table~\ref{tbl:overall_results_quality_10}, respectively.
%
Overall, these tables show that under varying graph similarity constraints, \method and \methodwithsguide can always generate desirable molecules with comparable quality to baselines in terms of stability, 3D structures, and 2D structures.
%
These results demonstrate the strong effectiveness of \method and \methodwithsguide in generating high-quality desirable molecules with stable and realistic structures in both 2D and 3D.
%
This enables the high utility of \method and \methodwithsguide in discovering promising drug candidates.


\begin{comment}
The results across these tables demonstrate similar observations with those under $\delta_g$=0.3 in Table~\ref{tbl:overall_results_quality_desired}.
%
For stability, when $\delta_g$=0.5, 0.7 or 1.0, \method and \methodwithsguide achieve comparable performance or fall slightly behind \squid ($\lambda$=0.3) and \squid ($\lambda$=1.0) in atom stability and molecule stability.
%
For example, when $\delta_g$=0.5, as shown in Table~\ref{tbl:overall_results_quality_05}, \method achieves similar performance with the best baseline \squid ($\lambda$=0.3) in atom stability (0.992 for \method vs 0.996 for \squid with $\lambda$=0.3).
%
\method underperforms \squid ($\lambda$=0.3) in terms of molecule stability.
%
For 3D structures, \method and \methodwithsguide also consistently generate molecules with more realistic 3D structures compared to \squid.
%
Particularly, \methodwithsguide achieves the best performance in RMSD and JS of bond lengths across $\delta_g$=0.5, 0.7 and 1.0.
%
In JS of dihedral angles, \method achieves the best performance among all the methods.
%
\method and \methodwithsguide underperform \squid in JS of bond angles, primarily because \squid constrains the bond angles in the generated molecules.
%
For 2D structures, \method and \methodwithsguide again achieve the best performance 
\end{comment}

%===================================================================
\section{Additional Experimental Results on PMG}
\label{supp:app:results_PMG}
%===================================================================

%\label{tbl:comparison_results_decompdiff}


%-------------------------------------------------------------------------------------------------------------------------------------
%\subsection{{Additional Comparison for PMG}}
%\label{supp:app:results:docking}
%-------------------------------------------------------------------------------------------------------------------------------------

In this section, we present the results of \methodwithpguide and \methodwithsandpguide when generating 100 molecules. 
%
Please note that both \methodwithpguide and \methodwithsandpguide show remarkable efficiency over the PMG baselines.
%
\methodwithpguide and \methodwithsandpguide generate 100 molecules in 48 and 58 seconds on average, respectively, while the most efficient baseline \targetdiff requires 1,252 seconds.
%
We report the performance of \methodwithpguide and \methodwithsandpguide against state-of-the-art PMG baselines in Table~\ref{tbl:overall_results_docking_100}.


%
According to Table~\ref{tbl:overall_results_docking_100}, \methodwithpguide and \methodwithsandpguide achieve comparable performance with the PMG baselines in generating molecules with high binding affinities.
%
Particularly, in terms of Vina S, \methodwithsandpguide achieves very comparable performance (-4.56 kcal/mol) to the third-best baseline \decompdiff (-4.58 kcal/mol) in average Vina S; it also achieves the third-best performance (-4.82 kcal/mol) among all the methods and slightly underperforms the second-best baseline \AR (-4.99 kcal/mol) in median Vina S
%
\methodwithsandpguide also achieves very close average Vina M (-5.53 kcal/mol) with the third-best baseline \AR (-5.59 kcal/mol) and the third-best performance (-5.47 kcal/mol) in median Vina M.
%
Notably, for Vina D, \methodwithpguide and \methodwithsandpguide achieve the second and third performance among all the methods.
%
In terms of the average percentage of generated molecules with Vina D higher than those of known ligands (i.e., HA), \methodwithpguide (58.52\%) and \methodwithsandpguide (58.28\%) outperform the best baseline \targetdiff (57.57\%).
%
These results signify the high utility of \methodwithpguide and \methodwithsandpguide in generating molecules that effectively bind with protein targets and have better binding affinities than known ligands.

In addition to binding affinities, \methodwithpguide and \methodwithsandpguide also demonstrate similar performance compared to the baselines in metrics related to drug-likeness and diversity.
%
For drug-likeness, both \methodwithpguide and \methodwithsandpguide achieve the best (0.67) and the second-best (0.66) QED scores.
%
They also achieve the third and fourth performance in SA scores.
%
In terms of the diversity among generated molecules,  \methodwithpguide and \methodwithsandpguide slightly underperform the baselines, possibly due to the design that generates molecules with similar shapes to the ligands.
%
These results highlight the strong ability of \methodwithpguide and \methodwithsandpguide in efficiently generating effective binding molecules with favorable drug-likeness and diversity.
%
This ability enables them to potentially serve as promising tools to facilitate effective and efficient drug development.

\begin{table*}[!h]
	\centering
		\caption{Additional Comparison on PMG When All Methods Generate 100 Molecules}
	\label{tbl:overall_results_docking_100}
\begin{threeparttable}
	\begin{scriptsize}
	\begin{tabular}{
		@{\hspace{2pt}}l@{\hspace{2pt}}
		%
		@{\hspace{2pt}}r@{\hspace{2pt}}
		%
		@{\hspace{2pt}}r@{\hspace{2pt}}
		@{\hspace{2pt}}r@{\hspace{2pt}}
		%
		@{\hspace{6pt}}r@{\hspace{6pt}}
		%
		@{\hspace{2pt}}r@{\hspace{2pt}}
		@{\hspace{2pt}}r@{\hspace{2pt}}
		%
		@{\hspace{5pt}}r@{\hspace{5pt}}
		%
		@{\hspace{2pt}}r@{\hspace{2pt}}
		@{\hspace{2pt}}r@{\hspace{2pt}}
		%
		@{\hspace{5pt}}r@{\hspace{5pt}}
		%
		@{\hspace{2pt}}r@{\hspace{2pt}}
	         @{\hspace{2pt}}r@{\hspace{2pt}}
		%
		@{\hspace{5pt}}r@{\hspace{5pt}}
		%
		@{\hspace{2pt}}r@{\hspace{2pt}}
		@{\hspace{2pt}}r@{\hspace{2pt}}
		%
		@{\hspace{5pt}}r@{\hspace{5pt}}
		%
		@{\hspace{2pt}}r@{\hspace{2pt}}
		@{\hspace{2pt}}r@{\hspace{2pt}}
		%
		@{\hspace{5pt}}r@{\hspace{5pt}}
		%
		@{\hspace{2pt}}r@{\hspace{2pt}}
		@{\hspace{2pt}}r@{\hspace{2pt}}
		%
		@{\hspace{5pt}}r@{\hspace{5pt}}
		%
		@{\hspace{2pt}}r@{\hspace{2pt}}
		%@{\hspace{2pt}}r@{\hspace{2pt}}
		%@{\hspace{2pt}}r@{\hspace{2pt}}
		}
		\toprule
		\multirow{2}{*}{method} & \multicolumn{2}{c}{Vina S$\downarrow$} & & \multicolumn{2}{c}{Vina M$\downarrow$} & & \multicolumn{2}{c}{Vina D$\downarrow$} & & \multicolumn{2}{c}{{HA\%$\uparrow$}}  & & \multicolumn{2}{c}{QED$\uparrow$} & & \multicolumn{2}{c}{SA$\uparrow$} & & \multicolumn{2}{c}{Div$\uparrow$} & %& \multirow{2}{*}{SR\%$\uparrow$} & 
		& \multirow{2}{*}{time$\downarrow$} \\
	    \cmidrule{2-3}\cmidrule{5-6} \cmidrule{8-9} \cmidrule{11-12} \cmidrule{14-15} \cmidrule{17-18} \cmidrule{20-21}
		 & Avg. & Med. &  & Avg. & Med. &  & Avg. & Med. & & Avg. & Med.  & & Avg. & Med.  & & Avg. & Med.  & & Avg. & Med.  & & \\ %& & \\
		%\multirow{2}{*}{method} & \multirow{2}{*}{\#c\%} &  \multirow{2}{*}{\#u\%} &  \multirow{2}{*}{QED} & \multicolumn{3}{c}{$\nmax=50$} & & \multicolumn{2}{c}{$\nmax=1$}\\
		%\cmidrule(r){5-7} \cmidrule(r){8-10} 
		%& & & & \avgshapesim(std) & \avggraphsim(std  &  \diversity(std  & & \avgshapesim(std) & \avggraphsim(std \\
		\midrule
		Reference                          & -5.32 & -5.66 & & -5.78 & -5.76 & & -6.63 & -6.67 & & - & - & & 0.53 & 0.49 & & 0.77 & 0.77 & & - & - & %& 23.1 & 
		& - \\
		\midrule
		\AR & \textbf{-5.06} & -4.99 & &  -5.59 & -5.29 & &  -6.16 & -6.05 & &  37.69 & 31.00 & &  0.50 & 0.49 & &  0.66 & 0.65 & & 0.70 & 0.70 & %& 7.0 & 
		& 7,789 \\
		\pockettwomol   & -4.50 & -4.21 & &  -5.70 & -5.27 & &  -6.43 & -6.25 & &  48.00 & 51.00 & &  0.58 & 0.58 & &  \textbf{0.77} & \textbf{0.78} & &  0.69 & 0.71 &  %& 24.9 & 
		& 2,150 \\
		\targetdiff     & -4.88 & \textbf{-5.82} & &  \textbf{-6.20} & \textbf{-6.36} & &  \textbf{-7.37} & \textbf{-7.51} & &  57.57 & 58.27 & &  0.50 & 0.51 & &  0.60 & 0.59 & &  \textbf{0.72} & 0.71 & % & 10.4 & 
		& 1,252 \\
		%& \decompdiffbeta                    & 63.03 & %-4.72 & -4.86 & & \textbf{-6.84} & \textbf{-6.91} & & \textbf{-8.85} & \textbf{-8.90} & &  \textbf{72.16} & \textbf{72.16} & &  0.36 & 0.36 & &  0.55 & 0.55 & & 0.59 & 0.59 & & 14.9 \\ 
		%-4.76 & -6.18 & &  \textbf{-6.86} & \textbf{-7.52} & &  \textbf{-8.85} & \textbf{-8.96} & &  \textbf{72.7} & \textbf{89.8} & &  0.36 & 0.34 & &  0.55 & 0.57 & & 0.59 & 0.59 & & 15.4 \\
		\decompdiffref  & -4.58 & -4.77 & &  -5.47 & -5.51 & &  -6.43 & -6.56 & &  47.76 & 48.66 & &  0.56 & 0.56 & &  0.70 & 0.69  & &  \textbf{0.72} & \textbf{0.72} &  %& 15.2 & 
		& 1,859 \\
		\midrule
		%\method & 14.04 & 9.74 & &  -2.80 & -3.87 & &  -6.32 & -6.41 & &  42.37 & 40.40 & &  0.70 & 0.71 & &  0.73 & 0.72 & & 0.71 & 0.74 & & 42 \\
		%\methodwithsguide & 1.04 & -0.33 & &  -4.23 & -4.39 & &  -6.31 & -6.46 & &  46.18 & 44.00 & &  0.69 & 0.71 & &  0.72 & 0.71 & & 0.70 & 0.73 & 53 \\
		\methodwithpguide      & -4.15 & -4.59 & &  -5.41 & -5.34 & &  -6.49 & -6.74 & &  \textbf{58.52} & 59.00 & &  \textbf{0.67} & \textbf{0.69} & &  0.68 & 0.68 & & 0.67 & 0.70 & %& 28.0 & 
		& 48 \\
		\methodwithsandpguide  & -4.56 & -4.82 & &  -5.53 & -5.47 & &  -6.60 & -6.78 & &  58.28 & \textbf{60.00} & &  0.66 & 0.68 & &  0.67 & 0.66 & & 0.68 & 0.71 &
		& 58 \\
		\bottomrule
	\end{tabular}%
	\begin{tablenotes}[normal,flushleft]
		\begin{footnotesize}
	\item 
\!\!Columns represent: {``Vina S'': the binding affinities between the initially generated poses of molecules and the protein pockets; 
		``Vina M'': the binding affinities between the poses after local structure minimization and the protein pockets;
		``Vina D'': the binding affinities between the poses determined by AutoDock Vina~\cite{Eberhardt2021} and the protein pockets;
		``HA'': the percentage of generated molecules with Vina D higher than those of condition molecules;
		``QED'': the drug-likeness score;
		``SA'': the synthesizability score;
		``Div'': the diversity among generated molecules;
		``time'': the time cost to generate molecules.}
		\par
		\par
		\end{footnotesize}
	\end{tablenotes}
	\end{scriptsize}
\end{threeparttable}
\end{table*}


%\label{tbl:overall_results_docking_100}

%-------------------------------------------------------------------------------------------------------------------------------------
%\subsection{{Comparison of Pocket Guidance}}
%\label{supp:app:results:docking}
%-------------------------------------------------------------------------------------------------------------------------------------


\begin{comment}
%-------------------------------------------------------------------------------------------------------------------------------------
\subsection{\ziqi{Simiarity Comparison for Pocket-based Molecule Generation}}
%-------------------------------------------------------------------------------------------------------------------------------------


\begin{table*}[t!]
	\centering
	\caption{{Overall Comparison on Similarity for Pocket-based Molecule Generation}}
	\label{tbl:docking_results_similarity}
	\begin{small}
		\begin{threeparttable}
			\begin{tabular}{
					@{\hspace{0pt}}l@{\hspace{5pt}}
					%
					@{\hspace{3pt}}l@{\hspace{3pt}}
					%
					@{\hspace{3pt}}r@{\hspace{8pt}}
					@{\hspace{3pt}}c@{\hspace{3pt}}
					%
					@{\hspace{3pt}}c@{\hspace{3pt}}
					@{\hspace{3pt}}c@{\hspace{3pt}}
					%
					@{\hspace{0pt}}c@{\hspace{0pt}}
					%
					@{\hspace{3pt}}c@{\hspace{3pt}}
					@{\hspace{3pt}}c@{\hspace{3pt}}
					%
					@{\hspace{3pt}}r@{\hspace{3pt}}
				}
				\toprule
				$\delta_g$  & method          & \#d\%$\uparrow$ & $\diversity_d$$\uparrow$(std) & \avgshapesim$\uparrow$(std) & \avggraphsim$\downarrow$(std) & & \maxshapesim$\uparrow$(std) & \maxgraphsim$\downarrow$(std)       & \#n\%$\uparrow$  \\ 
				\midrule
				%\multirow{6}{0.059\linewidth}{\hspace{0pt}0.1} 
				%& \AR   & 4.4 & 0.781(0.076) & 0.511(0.197) & \textbf{0.056}(0.020) & & 0.619(0.222) & 0.074(0.024) & 21.4  \\
				%& \pockettwomol & 6.6 & 0.795(0.099) & 0.519(0.216) & 0.063(0.020) & & 0.608(0.236) & 0.076(0.022) & \textbf{24.1}  \\
				%& \targetdiff & 2.0 & 0.872(0.041) & 0.619(0.110) & 0.068(0.018) & & 0.721(0.146) & 0.075(0.023) & 17.7  \\
				%& \decompdiffbeta & 0.0 & - & 0.374(0.138) & 0.059(0.031) & & 0.414(0.141) & \textbf{0.058}(0.032) & 9.8  \\
				%& \decompdiffref & 8.5 & 0.805(0.096) & 0.810(0.070) & 0.076(0.018) & & 0.861(0.085) & 0.076(0.020) & 11.3  \\
				%& \methodwithpguide   &  9.9 & \textbf{0.876}(0.041) & 0.795(0.058) & 0.073(0.015) & & 0.869(0.073) & 0.076(0.020) & 17.7  \\
				%& \methodwithsandpguide & \textbf{11.9} & 0.872(0.036) & \textbf{0.813}(0.052) & 0.075(0.014) & & \textbf{0.874}(0.069) & 0.080(0.014) & 17.0  \\
				%\cmidrule{2-10}
				%& improv\% & 40.4$^*$ & 8.8$^*$ & 0.4 & -30.4$^*$ &  & 1.6 & -30.0$^*$ & -26.3$^*$  \\
				%\midrule
				\multirow{7}{0.059\linewidth}{\hspace{0pt}1.0} 
				& \AR & 14.6 & 0.681(0.163) & 0.644(0.119) & 0.236(0.123) & & 0.780(0.110) & 0.284(0.177) & 95.8  \\
				& \pockettwomol & 18.6 & 0.711(0.152) & 0.654(0.131) &   \textbf{0.217}(0.129) & & 0.778(0.121) &   \textbf{0.243}(0.137) &  \textbf{98.3}  \\
				& \targetdiff & 7.1 &  \textbf{0.785}(0.085) & 0.622(0.083) & 0.238(0.122) & & 0.790(0.102) & 0.274(0.158) & 90.4  \\
				%& \decompdiffbeta & 0.1 & 0.589(0.030) & 0.494(0.124) & 0.263(0.143) & & 0.567(0.143) & 0.275(0.162) & 67.7  \\
				& \decompdiffref & 37.3 & 0.721(0.108) & 0.770(0.087) & 0.282(0.130) & & \textbf{0.878}(0.059) & 0.343(0.174) & 83.7  \\
				& \methodwithpguide   &  27.4 & 0.757(0.134) & 0.747(0.078) & 0.265(0.165) & & 0.841(0.081) & 0.272(0.168) & 98.1  \\
				& \methodwithsandpguide &\textbf{45.2} & 0.724(0.142) &   \textbf{0.789}(0.063) & 0.265(0.162) & & 0.876(0.062) & 0.264(0.159) & 97.8  \\
				\cmidrule{2-10}
				& Improv\%  & 21.2$^*$ & -3.6 & 2.5$^*$ & -21.7$^*$ &  & -0.1 & -8.4$^*$ & -0.2  \\
				\midrule
				\multirow{7}{0.059\linewidth}{\hspace{0pt}0.7} 
				& \AR   & 14.5 & 0.692(0.151) & 0.644(0.119) & 0.233(0.116) & & 0.779(0.110) & 0.266(0.140) & 94.9  \\
				& \pockettwomol & 18.6 & 0.711(0.152) & 0.654(0.131) & \textbf{0.217}(0.129) & & 0.778(0.121) & \textbf{0.243}(0.137) & \textbf{98.2}  \\
				& \targetdiff & 7.1 & \textbf{0.786}(0.084) & 0.622(0.083) & 0.238(0.121) & & 0.790(0.101) & 0.270(0.151) & 90.3  \\
				%& \decompdiffbeta & 0.1 & 0.589(0.030) & 0.494(0.124) & 0.263(0.142) & &0.567(0.143) & 0.273(0.156) & 67.6  \\
				& \decompdiffref & 36.2 & 0.721(0.113) & 0.770(0.086) & 0.273(0.123) & & \textbf{0.876}(0.059) & 0.325(0.139) & 82.3  \\
				& \methodwithpguide   &  27.4 & 0.757(0.134) & 0.746(0.078) & 0.263(0.160) & & 0.841(0.081) & 0.271(0.164) & 96.8  \\
				& \methodwithsandpguide      & \textbf{45.0} & 0.732(0.129) & \textbf{0.789}(0.063) & 0.262(0.157) & & \textbf{0.876}(0.063) & 0.262(0.153) & 96.2  \\
				\cmidrule{2-10}
				& Improv\%  & 24.3$^*$ & -3.6 & 2.5$^*$ & -20.8$^*$ &  & 0.0 & -7.6$^*$ & -1.5  \\
				\midrule
				\multirow{7}{0.059\linewidth}{\hspace{0pt}0.5} 
				& \AR   & 14.1 & 0.687(0.160) & 0.639(0.124) & 0.218(0.097) & & 0.778(0.110) & 0.260(0.130) & 89.8  \\
				& \pockettwomol & 18.5 & 0.711(0.152) & 0.649(0.134) & \textbf{0.209}(0.114) & & 0.777(0.121) & \textbf{0.240}(0.131) & \textbf{93.2}  \\
				& \targetdiff & 7.1 & \textbf{0.786}(0.084) & 0.621(0.083) & 0.230(0.111) & & 0.788(0.105) & 0.254(0.127) & 86.5  \\
				%&\decompdiffbeta & 0.1 & 0.595(0.025) & 0.494(0.124) & 0.254(0.129) & & 0.565(0.142) & 0.259(0.138) & 63.9  \\
				& \decompdiffref & 34.7 & 0.730(0.105) & 0.769(0.086) & 0.261(0.109) & & 0.874(0.080) & 0.301(0.117) & 77.3   \\
				& \methodwithpguide  &  27.2 & 0.765(0.123) & 0.749(0.075) & 0.245(0.135) & & 0.840(0.082) & 0.252(0.137) & 88.6  \\
				& \methodwithsandpguide & \textbf{44.3} & 0.738(0.122) & \textbf{0.791}(0.059) & 0.247(0.132) &  & \textbf{0.875}(0.065) & 0.249(0.130) & 88.8  \\
				\cmidrule{2-10}
				& Improv\%   & 27.8$^*$ & -2.7 & 2.9$^*$ & -17.6$^*$ &  & 0.2 & -3.4 & -4.7$^*$  \\
				\midrule
				\multirow{7}{0.059\linewidth}{\hspace{0pt}0.3} 
				& \AR   & 12.2 & 0.704(0.146) & 0.614(0.146) & 0.164(0.059) & & 0.751(0.138) & 0.206(0.059) & 66.4  \\
				& \pockettwomol & 17.1 & 0.731(0.129) & 0.617(0.163) & \textbf{0.155}(0.056) & & 0.740(0.159) & \textbf{0.190}(0.076) & \textbf{71.0}  \\
				& \targetdiff & 6.2 & \textbf{0.809}(0.061) & 0.619(0.087) & 0.181(0.068) & & 0.768(0.119) & 0.196(0.076) & 61.7  \\				
                %& \decompdiffbeta & 0.0 & - & 0.489(0.124) & 0.195(0.080) & & 0.547(0.139) & 0.203(0.087) & 42.0  \\
				& \decompdiffref & 27.7 & 0.775(0.081) & 0.767(0.086) & 0.202(0.062) & & 0.854(0.093) & 0.216(0.068) & 52.6  \\
				& \methodwithpguide   &  24.4 & 0.805(0.084) & 0.763(0.066) & 0.180(0.074) & & 0.847(0.080) & \textbf{0.190}(0.059) & 61.4  \\
				& \methodwithsandpguide & \textbf{36.3} & 0.789(0.081) & \textbf{0.800}(0.056) & 0.181(0.071) & &\textbf{0.878}(0.067) & \textbf{0.190}(0.078) & 61.8  \\
				\cmidrule{2-10}
				& improv\% & 31.1$^*$ & 3.9$^*$ & 4.3$^*$ & -16.5$^*$ &  & 2.8$^*$ & 0.0 & -12.9$^*$  \\
				\bottomrule
			\end{tabular}%
			\begin{tablenotes}[normal,flushleft]
				\begin{footnotesize}
					\item 
					\!\!Columns represent: \ziqi{``$\delta_g$'': the graph similarity constraint; ``\#n\%'': the percentage of molecules that satisfy the graph similarity constraint ($\graphsim<=\delta_g$);
						``\#d\%'': the percentage of molecules that satisfy the graph similarity constraint and are with high \shapesim ($\shapesim>=0.8$);
						``\avgshapesim/\avggraphsim'': the average of shape or graph similarities between the condition molecules and generated molecules with $\graphsim<=\delta_g$;
						``\maxshapesim'': the maximum of shape similarities between the condition molecules and generated molecules with $\graphsim<=\delta_g$;
						``\maxgraphsim'': the graph similarities between the condition molecules and the molecules with the maximum shape similarities and $\graphsim<=\delta_g$;
						``\diversity'': the diversity among the generated molecules.
						%
						``$\uparrow$'' represents higher values are better, and ``$\downarrow$'' represents lower values are better.
						%
						Best values are in \textbf{bold}, and second-best values are \underline{underlined}. 
					} 
					%\todo{double-check the significance value}
					\par
					\par
				\end{footnotesize}
			\end{tablenotes}
		\end{threeparttable}
	\end{small}
	\vspace{-10pt}    
\end{table*}
%\label{tbl:docking_results_similarity}

\bo{@Ziqi you may want to check my edits for the discussion in Table 1 first.
%
If the pocket if known, do you still care about the shape similarity in real applications?
}

\ziqi{Table~\ref{tbl:docking_results_similarity} presents the overall comparison on similarity-based metrics between \methodwithpguide, \methodwithsandpguide and other baselines under different graph similarity constraints  ($\delta_g$=1.0, 0.7, 0.5, 0.3), similar to Table~\ref{tbl:overall}. 
%
As shown in Table~\ref{tbl:docking_results_similarity}, regarding desirable molecules,  \methodwithsandpguide consistently outperforms all the baseline methods in the likelihood of generating desirable molecules (i.e., $\#d\%$).
%
For example, when $\delta_g$=1.0, at $\#d\%$, \methodwithsandpguide (45.2\%) demonstrates significant improvement of $21.2\%$ compared to the best baseline \decompdiff (37.3\%).
%
In terms of $\diversity_d$, \methodwithpguide and \methodwithsandpguide also achieve the second and the third best performance. 
%
Note that the best baseline \targetdiff in $\diversity_d$ achieves the least percentage of desirable molecules (7.1\%), substantially lower than \methodwithpguide and \methodwithsandpguide.
%
This makes its diversity among desirable molecules incomparable with other methods. 
%
When $\delta_g$=0.7, 0.5, and 0.3, \methodwithsandpguide also establishes a significant improvement of 24.3\%, 27.8\%, and 31.1\% compared to the best baseline method \decompdiff.
%
It is also worth noting that the state-of-the-art baseline \decompdiff underperforms \methodwithpguide and \methodwithsandpguide in binding affinities as shown in Table~\ref{tbl:overall_results_docking}, even though it outperforms \methodwithpguide in \#d\%.
%
\methodwithpguide and \methodwithsandpguide also achieve the second and the third best performance in $\diversity_d$ at $\delta_g$=0.7, 0.5, and 0.3. 
%
The superior performance of \methodwithpguide and \methodwithsandpguide in $\#d\%$ at small $\delta_g$ indicates their strong capacity in generating desirable molecules of novel graph structures, thereby facilitating the discovery of novel drug candidates.
%
}

\ziqi{Apart from the desirable molecules, \methodwithpguide and \methodwithsandpguide also demonstrate outstanding performance in terms of the average shape similarities (\avgshapesim) and the average graph similarities (\avggraphsim).
%
Specifically, when $\delta_g$=1.0, \methodwithsandpguide achieves a significant 2.5\% improvement in \avgshapesim\ over the best baseline \decompdiff. 
%
In terms of \avggraphsim, \methodwithsandpguide also achieves higher performance than the baseline \decompdiff of the highest \avgshapesim (0.265 vs 0.282).
%
Please note that all the baseline methods except \decompdiff achieve substantially lower performance in \avgshapesim than \methodwithpguide and \methodwithsandpguide, even though these methods achieve higher \avggraphsim values.
%
This trend remains consistent when applying various similarity constraints (i.e., $\delta_g$) as shown in Table~\ref{tbl:overall_results_docking}.
}

\ziqi{Similarly, \methodwithpguide and \methodwithsandpguide also achieve superior performance in \maxshapesim and \maxgraphsim.
%
Specifically, when $\delta_g$=1.0, for \maxshapesim, \methodwithsandpguide achieves highly comparable performance in \maxshapesim\ compared to the best baseline \decompdiff (0.876 vs 0.878).
%
We also note that \methodwithsandpguide achieves lower \maxgraphsim\ than the \decompdiff with 23.0\% difference. 
%
When $\delta_g$ gets smaller from 0.7 to 0.3, \methodwithsandpguide maintains a high \maxshapesim value around 0.876, while the best baseline \decompdiff has \maxshapesim decreased from 0.878 to 0.854.
%
This demonstrates the superior ability of \methodwithsandpguide in generating molecules with similar shapes and novel structures.
%
}

\ziqi{
In terms of \#n\%, when $\delta_g$=1.0, the percentage of molecules with \graphsim below $\delta_g$ can be interpreted as the percentage of valid molecules among all the generated molecules. 
%
As shown in Table~\ref{tbl:docking_results_similarity}, \methodwithpguide and \methodwithsandpguide are able to generate 98.1\% and 97.8\% of valid molecules, slightly below the best baseline \pockettwomol (98.3\%). 
%
When $\delta_g$=0.7, 0.5, or 0.3, all the methods, including \methodwithpguide and \methodwithsandpguide, can consistently find a sufficient number of novel molecules that meet the graph similarity constraints.
%
The only exception is \decompdiff, which substantially underperforms all the other methods in \#n\%.
}
\end{comment}

%%%%%%%%%%%%%%%%%%%%%%%%%%%%%%%%%%%%%%%%%%%%%
\section{Properties of Molecules in Case Studies for Targets}
\label{supp:app:results:properties}
%%%%%%%%%%%%%%%%%%%%%%%%%%%%%%%%%%%%%%%%%%%%%

%-------------------------------------------------------------------------------------------------------------------------------------
\subsection{Drug Properties of Generated Molecules}
\label{supp:app:results:properties:drug}
%-------------------------------------------------------------------------------------------------------------------------------------

Table~\ref{tbl:drug_property} presents the drug properties of three generated molecules: NL-001, NL-002, and NL-003.
%
As shown in Table~\ref{tbl:drug_property}, each of these molecules has a favorable profile, making them promising drug candidates. 
%
{As discussed in Section ``Case Studies for Targets'' in the main manuscript, all three molecules have high binding affinities in terms of Vina S, Vina M and Vina D, and favorable QED and SA values.
%
In addition, all of them meet the Lipinski's rule of five criteria~\cite{Lipinski1997}.}
%
In terms of physicochemical properties, all these properties of NL-001, NL-002 and NL-003, including number of rotatable bonds, molecule weight, LogP value, number of hydrogen bond doners and acceptors, and molecule charges, fall within the desired range of drug molecules. 
%
This indicates that these molecules could potentially have good solubility and membrane permeability, essential qualities for effective drug absorption.

These generated molecules also demonstrate promising safety profiles based on the predictions from ICM~\cite{Neves2012}.
%
In terms of drug-induced liver injury prediction scores, all three molecules have low scores (0.188 to 0.376), indicating a minimal risk of hepatotoxicity. 
%
NL-001 and NL-002 fall under `Ambiguous/Less concern' for liver injury, while NL-003 is categorized under 'No concern' for liver injury. 
%
Moreover, all these molecules have low toxicity scores (0.000 to 0.236). 
%
NL-002 and NL-003 do not have any known toxicity-inducing functional groups. 
%
NL-001 and NL-003 are also predicted not to include any known bad groups that lead to inappropriate features.
%
These attributes highlight the potential of NL-001, NL-002, and NL-003 as promising treatments for cancers and Alzheimer’s disease.

%\begin{table*}
	\centering
		\caption{Drug Properties of Generated Molecules}
	\label{tbl:binding_drug_mols}
	\begin{scriptsize}
\begin{threeparttable}
	\begin{tabular}{
		@{\hspace{6pt}}r@{\hspace{6pt}}
		@{\hspace{6pt}}r@{\hspace{6pt}}
		@{\hspace{6pt}}r@{\hspace{6pt}}
		@{\hspace{6pt}}r@{\hspace{6pt}}
		@{\hspace{6pt}}r@{\hspace{6pt}}
		@{\hspace{6pt}}r@{\hspace{6pt}}
		@{\hspace{6pt}}r@{\hspace{6pt}}
		@{\hspace{6pt}}r@{\hspace{6pt}}
		@{\hspace{6pt}}r@{\hspace{6pt}}
		%
		}
		\toprule
Target & Molecule & Vina S & Vina M & Vina D & QED   & SA   & Logp  & Lipinski \\
\midrule
\multirow{3}{*}{CDK6} & NL-001 & -6.817      & -7.251    & -8.319     & 0.834 & 0.72 & 1.313 & 5        \\
& NL-002 & -6.970       & -7.605    & -8.986     & 0.851 & 0.74 & 3.196 & 5        \\
\cmidrule{2-9}
& 4AU & 0.736       & -5.939    & -7.592     & 0.773 & 0.79 & 2.104 & 5        \\
\midrule
\multirow{2}{*}{NEP} & NL-003 & -11.953     & -12.165   & -12.308    & 0.772 & 0.57 & 2.944 & 5        \\
\cmidrule{2-9}
& BIR & -9.399      & -9.505    & -9.561     & 0.463 & 0.73 & 2.677 & 5        \\
		\bottomrule
	\end{tabular}%
	\begin{tablenotes}[normal,flushleft]
		\begin{footnotesize}
	\item Columns represent: {``Target'': the names of protein targets;
		``Molecule'': the names of generated molecules and known ligands;
		``Vina S'': the binding affinities between the initially generated poses of molecules and the protein pockets; 
		``Vina M'': the binding affinities between the poses after local structure minimization and the protein pockets;
		``Vina D'': the binding affinities between the poses determined by AutoDock Vina~\cite{Eberhardt2021} and the protein pockets;
		``HA'': the percentage of generated molecules with Vina D higher than those of condition molecules;
		``QED'': the drug-likeness score;
		``SA'': the synthesizability score;
		``Div'': the diversity among generated molecules;
		``time'': the time cost to generate molecules.}
\!\! \par
		\par
		\end{footnotesize}
	\end{tablenotes}
\end{threeparttable}
\end{scriptsize}
  \vspace{-10pt}    
\end{table*}

%\label{tbl:binding_drug_mols}

\begin{table*}
	\centering
		\caption{Drug Properties of Generated Molecules}
	\label{tbl:drug_property}
	\begin{scriptsize}
\begin{threeparttable}
	\begin{tabular}{
		@{\hspace{0pt}}p{0.23\linewidth}@{\hspace{5pt}}
		%
		@{\hspace{1pt}}r@{\hspace{2pt}}
		@{\hspace{2pt}}r@{\hspace{6pt}}
		@{\hspace{6pt}}r@{\hspace{6pt}}
		%
		}
		\toprule
		Property Name & NL-001 & NL-002 & NL-003 \\
		\midrule
Vina S & -6.817 &  -6.970 & -11.953 \\
Vina M & -7.251 & -7.605 & -12.165 \\
Vina D & -8.319 & -8.986 & -12.308 \\
QED    & 0.834  & 0.851  & 0.772 \\
SA       & 0.72    & 0.74    & 0.57    \\
Lipinski & 5 & 5 & 5 \\
%bbbScore          & 3.386                                                                                        & 4.240                                                                                        & 3.892      \\
%drugLikeness      & -0.081                                                                                       & -0.442                                                                                       & -0.325     \\
%molLogP1          & 1.698                                                                                        & 2.685                                                                                        & 2.382      \\
\#rotatable bonds          & 3                                                                                        & 2                                                                                        & 2      \\
molecule weight         & 267.112                                                                                      & 270.117                                                                                      & 390.206    \\
molecule LogP           & 1.698                                                                                        & 2.685                                                                                        & 2.382     \\
\#hydrogen bond doners           & 1                                                                                        & 1                                                                                        & 2      \\
\#hydrogen bond acceptors           & 5                                                                                       & 3                                                                                        & 5      \\
\#molecule charges   & 1                                                                                        & 0                                                                                        & 0      \\
drug-induced liver injury predScore    & 0.227                                                                                        & 0.376                                                                                        & 0.188      \\
drug-induced liver injury predConcern  & Ambiguous/Less concern                                                                       & Ambiguous/Less concern                                                                       & No concern \\
drug-induced liver injury predLabel    & Warnings/Precautions/Adverse reactions & Warnings/Precautions/Adverse reactions & No match   \\
drug-induced liver injury predSeverity & 2                                                                                        & 3                                                                                        & 2      \\
%molSynth1         & 0.254                                                                                        & 0.220                                                                                        & 0.201      \\
%toxicity class         & 0.480                                                                                        & 0.480                                                                                        & 0.450      \\
toxicity names         & hydrazone                                                                                    &   -                                                                                           &   -         \\
toxicity score         & 0.236                                                                                        & 0.000                                                                                        & 0.000      \\
bad groups         & -                                                                                             & Tetrahydroisoquinoline:   allergies                                                          &   -         \\
%MolCovalent       &                                                                                              &                                                                                              &            \\
%MolProdrug        &                                                                                              &                                                                                              &            \\
		\bottomrule
	\end{tabular}%
	\begin{tablenotes}[normal,flushleft]
		\begin{footnotesize}
	\item ``-'': no results found by algorithms
\!\! \par
		\par
		\end{footnotesize}
	\end{tablenotes}
\end{threeparttable}
\end{scriptsize}
  \vspace{-10pt}    
\end{table*}

%\label{tbl:drug_property}

%-------------------------------------------------------------------------------------------------------------------------------------
\subsection{Comparison on ADMET Profiles between Generated Molecules and Approved Drugs}
\label{supp:app:results:properties:admet}
%-------------------------------------------------------------------------------------------------------------------------------------

\paragraph{Generated Molecules for CDK6}
%
Table~\ref{tbl:admet_cdk6} presents the comparison on ADMET profiles between two generated molecules for CDK6 and the approved CDK6 inhibitors, including Abemaciclib~\cite{Patnaik2016}, Palbociclib~\cite{Lu2015}, and Ribociclib~\cite{Tripathy2017}.
%
As shown in Table~\ref{tbl:admet_cdk6}, the generated molecules, NL-001 and NL-002, exhibit comparable ADMET profiles with those of the approved CDK6 inhibitors. 
%
Importantly, both molecules demonstrate good potential in most crucial properties, including Ames mutagenesis, favorable oral toxicity, carcinogenicity, estrogen receptor binding, high intestinal absorption and favorable oral bioavailability.
%
Although the generated molecules are predicted as positive in hepatotoxicity and mitochondrial toxicity, all the approved drugs are also predicted as positive in these two toxicity.
%
This result suggests that these issues might stem from the limited prediction accuracy rather than being specific to our generated molecules.
%
Notably, NL-001 displays a potentially better plasma protein binding score compared to other molecules, which may improve its distribution within the body. 
%
Overall, these results indicate that NL-001 and NL-002 could be promising candidates for further drug development.


\begin{table*}
	\centering
		\caption{Comparison on ADMET Profiles among Generated Molecules and Approved Drugs Targeting CDK6}
	\label{tbl:admet_cdk6}
	\begin{scriptsize}
\begin{threeparttable}
	\begin{tabular}{
		%@{\hspace{0pt}}p{0.23\linewidth}@{\hspace{5pt}}
		%
		@{\hspace{6pt}}l@{\hspace{5pt}}
		@{\hspace{6pt}}r@{\hspace{6pt}}
		@{\hspace{6pt}}r@{\hspace{6pt}}
		@{\hspace{6pt}}r@{\hspace{6pt}}
		@{\hspace{6pt}}r@{\hspace{6pt}}
		@{\hspace{6pt}}r@{\hspace{6pt}}
		%
		%
		@{\hspace{6pt}}r@{\hspace{6pt}}
		%@{\hspace{6pt}}r@{\hspace{6pt}}
		%
		}
		\toprule
		\multirow{2}{*}{Property name} & \multicolumn{2}{c}{Generated molecules} & & \multicolumn{3}{c}{FDA-approved drugs} \\
		\cmidrule{2-3}\cmidrule{5-7}
		 & NL--001 & NL--002 & & Abemaciclib & Palbociclib & Ribociclib \\
		\midrule
\rowcolor[HTML]{D2EAD9}Ames   mutagenesis                             & --   &  --  & & + &  --  & --  \\
\rowcolor[HTML]{D2EAD9}Acute oral toxicity (c)                           & III & III & &  III          & III          & III         \\
Androgen receptor binding                         & +                          & +            &              & +            & +            & +             \\
Aromatase binding                                 & +                          & +            &              & +            & +            & +            \\
Avian toxicity                                    & --                          & --          &                & --            & --            & --            \\
Blood brain barrier                               & +                          & +            &              & +            & +            & +            \\
BRCP inhibitior                                   & --                          & --          &                & --            & --            & --            \\
Biodegradation                                    & --                          & --          &                & --            & --            & --           \\
BSEP inhibitior            & +                          & +            &              & +            & +            & +        \\
Caco-2                                            & +                          & +            &              & --            & --            & --            \\
\rowcolor[HTML]{D2EAD9}Carcinogenicity (binary)                          & --                          & --             &             & --            & --            & --          \\
\rowcolor[HTML]{D2EAD9}Carcinogenicity (trinary)                         & Non-required               & Non-required   &            & Non-required & Non-required & Non-required  \\
Crustacea aquatic toxicity & --                          & --            &              & --            & --            & --            \\
 CYP1A2 inhibition                                 & +                          & +            &              & --            & --            & +             \\
CYP2C19 inhibition                                & --                          & +            &              & +            & --            & +            \\
CYP2C8 inhibition                                 & --                          & --           &               & +            & +            & +            \\
CYP2C9 inhibition                                 & --                          & --           &               & --            & --            & +             \\
CYP2C9 substrate                                  & --                          & --           &               & --            & --            & --            \\
CYP2D6 inhibition                                 & --                          & --           &               & --            & --            & --            \\
CYP2D6 substrate                                  & --                          & --           &               & --            & --            & --            \\
CYP3A4 inhibition                                 & --                          & +            &              & --            & --            & --            \\
CYP3A4 substrate                                  & +                          & --            &              & +            & +            & +            \\
\rowcolor[HTML]{D2EAD9}CYP inhibitory promiscuity                        & +                          & +                    &      & +            & --            & +            \\
Eye corrosion                                     & --                          & --           &               & --            & --            & --            \\
Eye irritation                                    & --                          & --           &               & --            & --            & --             \\
\rowcolor[HTML]{D8E7FF}Estrogen receptor binding                         & +                          & +                    &      & +            & +            & +            \\
Fish aquatic toxicity                             & --                          & +            &              & +            & --            & --            \\
Glucocorticoid receptor   binding                 & +                          & +             &             & +            & +            & +            \\
Honey bee toxicity                                & --                          & --           &               & --            & --            & --            \\
\rowcolor[HTML]{D2EAD9}Hepatotoxicity                                    & +                          & +            &              & +            & +            & +             \\
Human ether-a-go-go-related gene inhibition     & +                          & +               &           & +            & --            & --           \\
\rowcolor[HTML]{D8E7FF}Human intestinal absorption                       & +                          & +             &             & +            & +            & +    \\
\rowcolor[HTML]{D8E7FF}Human oral bioavailability                        & +                          & +              &            & +            & +            & +     \\
\rowcolor[HTML]{D2EAD9}MATE1 inhibitior                                  & --                          & --              &            & --            & --            & --    \\
\rowcolor[HTML]{D2EAD9}Mitochondrial toxicity                            & +                          & +                &          & +            & +            & +    \\
Micronuclear                                      & +                          & +                          & +            & +            & +           \\
\rowcolor[HTML]{D2EAD9}Nephrotoxicity                                    & --                          & --             &             & --            & --            & --             \\
Acute oral toxicity                               & 2.325                      & 1.874    &     & 1.870        & 3.072        & 3.138        \\
\rowcolor[HTML]{D8E7FF}OATP1B1 inhibitior                                & +                          & +              &            & +            & +            & +             \\
\rowcolor[HTML]{D8E7FF}OATP1B3 inhibitior                                & +                          & +              &            & +            & +            & +             \\
\rowcolor[HTML]{D2EAD9}OATP2B1 inhibitior                                & --                          & --             &             & --            & --            & --             \\
OCT1 inhibitior                                   & --                          & --        &                  & +            & --            & +             \\
OCT2 inhibitior                                   & --                          & --        &                  & --            & --            & +             \\
P-glycoprotein inhibitior                         & --                          & --        &                  & +            & +            & +     \\
P-glycoprotein substrate                          & --                          & --        &                  & +            & +            & +     \\
PPAR gamma                                        & +                          & +          &                & +            & +            & +      \\
\rowcolor[HTML]{D8E7FF}Plasma protein binding                            & 0.359        & 0.745     &    & 0.865        & 0.872        & 0.636       \\
Reproductive toxicity                             & +                          & +          &                & +            & +            & +           \\
Respiratory toxicity                              & +                          & +          &                & +            & +            & +         \\
Skin corrosion                                    & --                          & --        &                  & --            & --            & --           \\
Skin irritation                                   & --                          & --        &                  & --            & --            & --         \\
Skin sensitisation                                & --                          & --        &                  & --            & --            & --          \\
Subcellular localzation                           & Mitochondria               & Mitochondria  &             & Lysosomes    & Mitochondria & Mitochondria \\
Tetrahymena pyriformis                            & 0.398                      & 0.903         &             & 1.033        & 1.958        & 1.606         \\
Thyroid receptor binding                          & +                          & +             &             & +            & +            & +           \\
UGT catelyzed                                     & --                          & --           &               & --            & --            & --           \\
\rowcolor[HTML]{D8E7FF}Water solubility                                  & -3.050                     & -3.078              &       & -3.942       & -3.288       & -2.673     \\
		\bottomrule
	\end{tabular}%
	\begin{tablenotes}[normal,flushleft]
		\begin{footnotesize}
	\item Blue cells highlight crucial properties where a negative outcome (``--'') is desired; for acute oral toxicity (c), a higher category (e.g., ``III'') is desired; and for carcinogenicity (trinary), ``Non-required'' is desired.
	%
	Green cells highlight crucial properties where a positive result (``+'') is desired; for plasma protein binding, a lower value is desired; and for water solubility, values higher than -4 are desired~\cite{logs}.
\!\! \par
		\par
		\end{footnotesize}
	\end{tablenotes}
\end{threeparttable}
\end{scriptsize}
  \vspace{--10pt}    
\end{table*}

%\label{tbl:admet_cdk6}

\paragraph{Generated Molecule for NEP}
%
Table~\ref{tbl:admet_nep} presents the comparison on ADMET profiles between a generated molecule for NEP targeting Alzheimer's disease and three approved drugs, Donepezil, Galantamine, and Rivastigmine, for Alzheimer's disease~\cite{Hansen2008}.
%
Overall, NL-003 exhibits a comparable ADMET profile with the three approved drugs.  
%
Notably, same as other approved drugs, NL-003 is predicted to be able to penetrate the blood brain barrier, a crucial property for Alzheimer's disease.
%  
In addition, it demonstrates a promising safety profile in terms of Ames mutagenesis, favorable oral toxicity, carcinogenicity, estrogen receptor binding, high intestinal absorption, nephrotoxicity and so on.
%
These results suggest that NL-003 could be promising candidates for the drug development of Alzheimer's disease.

\begin{table*}
	\centering
		\caption{Comparison on ADMET Profiles among Generated Molecule Targeting NEP and Approved Drugs for Alzhimer's Disease}
	\label{tbl:admet_nep}
	\begin{scriptsize}
\begin{threeparttable}
	\begin{tabular}{
		@{\hspace{6pt}}l@{\hspace{5pt}}
		%
		@{\hspace{6pt}}r@{\hspace{6pt}}
		@{\hspace{6pt}}r@{\hspace{6pt}}
		@{\hspace{6pt}}r@{\hspace{6pt}}
		@{\hspace{6pt}}r@{\hspace{6pt}}
		@{\hspace{6pt}}r@{\hspace{6pt}}
		%
		%
		%@{\hspace{6pt}}r@{\hspace{6pt}}
		%
		}
		\toprule
		\multirow{2}{*}{Property name} & Generated molecule & & \multicolumn{3}{c}{FDA-approved drugs} \\
\cmidrule{2-2}\cmidrule{4-6}
			& NL--003 & & Donepezil	& Galantamine & Rivastigmine \\
		\midrule
\rowcolor[HTML]{D2EAD9} 
Ames   mutagenesis                            & --                      &              & --                                    & --                                 & --                     \\
\rowcolor[HTML]{D2EAD9}Acute oral toxicity (c)                       & III           &                       & III                                  & III                               & II                      \\
Androgen receptor binding                     & +      &      & +            & --         & --         \\
Aromatase binding                             & --     &       & +            & --         & --        \\
Avian toxicity                                & --     &                               & --                                    & --                                 & --                        \\
\rowcolor[HTML]{D8E7FF} 
Blood brain barrier                           & +      &                              & +                                    & +                                 & +                        \\
BRCP inhibitior                               & --     &       & --            & --         & --         \\
Biodegradation                                & --     &                               & --                                    & --                                 & --                        \\
BSEP inhibitior                               & +      &      & +            & --         & --         \\
Caco-2                                        & +      &      & +            & +         & +         \\
\rowcolor[HTML]{D2EAD9} 
Carcinogenicity (binary)                      & --     &                               & --                                    & --                                 & --                        \\
\rowcolor[HTML]{D2EAD9} 
Carcinogenicity (trinary)                     & Non-required    &                     & Non-required                         & Non-required                      & Non-required             \\
Crustacea aquatic toxicity                    & +               &                     & +                                    & +                                 & --                        \\
CYP1A2 inhibition                             & +               &                     & +                                    & --                                 & --                        \\
CYP2C19 inhibition                            & +               &                     & --                                    & --                                 & --                        \\
CYP2C8 inhibition                             & +               &                     & --                                    & --                                 & --                        \\
CYP2C9 inhibition                             & --              &                      & --                                    & --                                 & --                        \\
CYP2C9 substrate                              & --              &                      & --                                    & --                                 & --                        \\
CYP2D6 inhibition                             & --              &                      & +                                    & --                                 & --                        \\
CYP2D6 substrate                              & --              &                      & +                                    & +                                 & +                        \\
CYP3A4 inhibition                             & --              &                      & --                                    & --                                 & --                        \\
CYP3A4 substrate                              & +               &                     & +                                    & +                                 & --                        \\
\rowcolor[HTML]{D2EAD9} 
CYP inhibitory promiscuity                    & +               &                     & +                                    & --                                 & --                        \\
Eye corrosion                                 & --     &       & --            & --         & --         \\
Eye irritation                                & --     &       & --            & --         & --         \\
Estrogen receptor binding                     & +      &      & +            & --         & --         \\
Fish aquatic toxicity                         & --     &                               & +                                    & +                                 & +                        \\
Glucocorticoid receptor binding             & --      &      & +            & --         & --         \\
Honey bee toxicity                            & --    &                                & --                                    & --                                 & --                        \\
\rowcolor[HTML]{D2EAD9} 
Hepatotoxicity                                & +     &                               & +                                    & --                                 & --                        \\
Human ether-a-go-go-related gene inhibition & +       &     & +            & --         & --         \\
\rowcolor[HTML]{D8E7FF} 
Human intestinal absorption                   & +     &                               & +                                    & +                                 & +                        \\
\rowcolor[HTML]{D8E7FF} 
Human oral bioavailability                    & --    &                                & +                                    & +                                 & +                        \\
\rowcolor[HTML]{D2EAD9} 
MATE1 inhibitior                              & --    &                                & --                                    & --                                 & --                        \\
\rowcolor[HTML]{D2EAD9} 
Mitochondrial toxicity                        & +     &                               & +                                    & +                                 & +                        \\
Micronuclear                                  & +     &       & --            & --         & +         \\
\rowcolor[HTML]{D2EAD9} 
Nephrotoxicity                                & --    &                                & --                                    & --                                 & --                        \\
Acute oral toxicity                           & 2.704  &      & 2.098        & 2.767     & 2.726     \\
\rowcolor[HTML]{D8E7FF} 
OATP1B1 inhibitior                            & +      &                              & +                                    & +                                 & +                        \\
\rowcolor[HTML]{D8E7FF} 
OATP1B3 inhibitior                            & +      &                              & +                                    & +                                 & +                        \\
\rowcolor[HTML]{D2EAD9} 
OATP2B1 inhibitior                            & --     &                               & --                                    & --                                 & --                        \\
OCT1 inhibitior                               & +      &      & +            & --         & --         \\
OCT2 inhibitior                               & --     &       & +            & --         & --         \\
P-glycoprotein inhibitior                     & +      &      & +            & --         & --         \\
\rowcolor[HTML]{D8E7FF} 
P-glycoprotein substrate                      & +      &                              & +                                    & +                                 & --                        \\
PPAR gamma                                    & +      &      & --            & --         & --         \\
\rowcolor[HTML]{D8E7FF} 
Plasma protein binding                        & 0.227   &                             & 0.883                                & 0.230                             & 0.606                    \\
Reproductive toxicity                         & +       &     & +            & +         & +         \\
Respiratory toxicity                          & +       &     & +            & +         & +         \\
Skin corrosion                                & --      &      & --            & --         & --         \\
Skin irritation                               & --      &      & --            & --         & --         \\
Skin sensitisation                            & --      &      & --            & --         & --         \\
Subcellular localzation                       & Mitochondria & &Mitochondria & Lysosomes & Mitochondria  \\
Tetrahymena pyriformis                        & 0.053           &                     & 0.979                                & 0.563                             & 0.702                        \\
Thyroid receptor binding                      & +       &     & +            & +         & --             \\
UGT catelyzed                                 & --      &      & --            & +         & --             \\
\rowcolor[HTML]{D8E7FF} 
Water solubility                              & -3.586   &                            & -2.425                               & -2.530                            & -3.062                       \\
		\bottomrule
	\end{tabular}%
	\begin{tablenotes}[normal,flushleft]
		\begin{footnotesize}
	\item Blue cells highlight crucial properties where a negative outcome (``--'') is desired; for acute oral toxicity (c), a higher category (e.g., ``III'') is desired; and for carcinogenicity (trinary), ``Non-required'' is desired.
	%
	Green cells highlight crucial properties where a positive result (``+'') is desired; for plasma protein binding, a lower value is desired; and for water solubility, values higher than -4 are desired~\cite{logs}.
\!\! \par
		\par
		\end{footnotesize}
	\end{tablenotes}
\end{threeparttable}
\end{scriptsize}
  \vspace{--10pt}    
\end{table*}

%\label{tbl:admet_nep}

\clearpage
%%%%%%%%%%%%%%%%%%%%%%%%%%%%%%%%%%%%%%%%%%%%%
\section{Algorithms}
\label{supp:algorithms}
%%%%%%%%%%%%%%%%%%%%%%%%%%%%%%%%%%%%%%%%%%%%%

Algorithm~\ref{alg:shapemol} describes the molecule generation process of \method.
%
Given a known ligand \molx, \method generates a novel molecule \moly that has a similar shape to \molx and thus potentially similar binding activity.
%
\method can also take the protein pocket \pocket as input and adjust the atoms of generated molecules for optimal fit and improved binding affinities.
%
Specifically, \method first calculates the shape embedding \shapehiddenmat for \molx using the shape encoder \SEE described in Algorithm~\ref{alg:see_shaperep}.
%
Based on \shapehiddenmat, \method then generates a novel molecule with a similar shape to \molx using the diffusion-based generative model \methoddiff as in Algorithm~\ref{alg:diffgen}.
%
During generation, \method can use shape guidance to directly modify the shape of \moly to closely resemble the shape of \molx.
%
When the protein pocket \pocket is available, \method can also use pocket guidance to ensure that \moly is specifically tailored to closely fit within \pocket.

\begin{algorithm}[!h]
    \caption{\method}
    \label{alg:shapemol}
         %\hspace*{\algorithmicindent} 
	\textbf{Required Input: $\molx$} \\
 	%\hspace*{\algorithmicindent} 
	\textbf{Optional Input: $\pocket$} 
    \begin{algorithmic}[1]
        \FullLineComment{calculate a shape embedding with Algorithm~\ref{alg:see_shaperep}}
        \State $\shapehiddenmat$, $\pc$ = $\SEE(\molx)$
        \FullLineComment{generate a molecule conditioned on the shape embedding with Algorithm~\ref{alg:diffgen}}
         \If{\pocket is not available}
        \State $\moly = \diffgenerative(\shapehiddenmat, \molx)$
        \Else
        \State $\moly = \diffgenerative(\shapehiddenmat, \molx, \pocket)$
        \EndIf
        \State \Return \moly
    \end{algorithmic}
\end{algorithm}
%\label{alg:shapemol}

\begin{algorithm}[!h]
    \caption{\SEE for shape embedding calculation}
    \label{alg:see_shaperep}
    \textbf{Required Input: $\molx$}
    \begin{algorithmic}[1]
        %\Require $\molx$
        \FullLineComment{sample a point cloud over the molecule surface shape}
        \State $\pc$ = $\text{samplePointCloud}(\molx)$
        \FullLineComment{encode the point cloud into a latent embedding (Equation~\ref{eqn:shape_embed})}
        \State $\shapehiddenmat = \SEE(\pc)$
        \FullLineComment{move the center of \pc to zero}
        \State $\pc = \pc - \text{center}(\pc)$
        \State \Return \shapehiddenmat, \pc
    \end{algorithmic}
\end{algorithm}
%\label{alg:see_shaperep}

\begin{algorithm}[!h]
    \caption{\diffgenerative for molecule generation}
    \label{alg:diffgen}
    	\textbf{Required Input: $\molx$, \shapehiddenmat} \\
 	%\hspace*{\algorithmicindent} 
	\textbf{Optional Input: $\pocket$} 
    \begin{algorithmic}[1]
        \FullLineComment{sample the number of atoms in the generated molecule}
        \State $n = \text{sampleAtomNum}(\molx)$
        \FullLineComment{sample initial positions and types of $n$ atoms}
        \State $\{\pos_T\}^n = \mathcal{N}(0, I)$
        \State $\{\atomfeat_T\}^n = C(K, \frac{1}{K})$
        \FullLineComment{generate a molecule by denoising $\{(\pos_T, \atomfeat_T)\}^n$ to $\{(\pos_0, \atomfeat_0)\}^n$}
        \For{$t = T$ to $1$}
            \IndentLineComment{predict the molecule without noise using the shape-conditioned molecule prediction module \molpred}{1.5}
            \State $(\tilde{\pos}_{0,t}, \tilde{\atomfeat}_{0,t}) = \molpred(\pos_t, \atomfeat_t, \shapehiddenmat)$
            \If{use shape guidance and $t > s$}
                \State $\tilde{\pos}_{0,t} = \shapeguide(\tilde{\pos}_{0,t}, \molx)$
                %\State $\tilde{\pos}_{0,t} = \pos^*_{0,t}$
            \EndIf
            \IndentLineComment{sample $(\pos_{t-1}, \atomfeat_{t-1})$ from $(\pos_t, \atomfeat_t)$ and $(\tilde{\pos}_{0,t}, \tilde{\atomfeat}_{0,t})$}{1.5}
            \State $\pos_{t-1} = P(\pos_{t-1}|\pos_t, \tilde{\pos}_{o,t})$
            \State $\atomfeat_{t-1} = P(\atomfeat_{t-1}|\atomfeat_t, \tilde{\atomfeat}_{o,t})$
            \If{use pocket guidance and $\pocket$ is available}
                \State $\pos_{t-1} = \pocketguide(\pos_{t-1}, \pocket)$
                %\State $\pos_{t-1} = \pos_{t-1}^*$
            \EndIf  
        \EndFor
        \State \Return $\moly = (\pos_0, \atomfeat_0)$
    \end{algorithmic}
\end{algorithm}
%\label{alg:diffgen}

%\input{algorithms/train_SE}
%\label{alg:train_se}

%\begin{algorithm}[!h]
    \caption{Training Procedure of \methoddiff}
    \label{alg:diffgen}
    \begin{algorithmic}[1]
        \Require $\shapehiddenmat, \molx, \pocket$
        \FullLineComment{sample the number of atoms in the generated molecule}
    \end{algorithmic}
\end{algorithm}
%\label{alg:train_diff}

%---------------------------------------------------------------------------------------------------------------------
\section{{Equivariance and Invariance}}
\label{supp:ei}
%---------------------------------------------------------------------------------------------------------------------

%.................................................................................................
\subsection{Equivariance}
\label{supp:ei:equivariance}
%.................................................................................................

{Equivariance refers to the property of a function $f(\pos)$ %\bo{is it the property of the function or embedding (x)?} 
that any translation and rotation transformation from the special Euclidean group SE(3)~\cite{Atz2021} applied to a geometric object
$\pos\in\mathbb{R}^3$ is mirrored in the output of $f(\pos)$, accordingly.
%
This property ensures $f(\pos)$ to learn a consistent representation of an object's geometric information, regardless of its orientation or location in 3D space.
%
%As a result, it provides $f(\pos)$ better generalization capabilities~\cite{Jonas20a}.
%
Formally, given any translation transformation $\mathbf{t}\in\mathbb{R}^3$ and rotation transformation $\mathbf{R}\in\mathbb{R}^{3\times3}$ ($\mathbf{R}^{\mathsf{T}}\mathbf{R}=\mathbb{I}$), %\xia{change the font types for $^{\mathsf{T}}$ and $\mathbb{I}$ in the entire manuscript}), 
$f(\pos)$ is equivariant with respect to these transformations %$g$ (\bo{where is $g$...})
if it satisfies
\begin{equation}
f(\mathbf{R}\pos+\mathbf{t}) = \mathbf{R}f(\pos) + \mathbf{t}. %\ \text{where}\ \hiddenpos = f(\pos).
\end{equation}
%
%where $\hiddenpos=f(\pos)$ is the output of $\pos$. 
%
In \method, both \SE and \methoddiff are developed to guarantee equivariance in capturing the geometric features of objects regardless of any translation or rotation transformations, as will be detailed in the following sections.
}

%.................................................................................................
\subsection{Invariance}
\label{supp:ei:invariance}
%.................................................................................................

%In contrast to equivariance, 
Invariance refers to the property of a function that its output {$f(\pos)$} remains constant under any translation and rotation transformations of the input $\pos$. %a geometric object's feature $\pos$.
%
This property enables $f(\pos)$ to accurately capture %a geometric object's 
the inherent features (e.g., atom features for 3D molecules) that are invariant of its orientation or position in 3D space.
%
Formally, $f(\pos)$ is invariant under any translation $\mathbf{t}$ and  rotation $\mathbf{R}$ if it satisfies
%
\begin{equation}
f(\mathbf{R}\pos+\mathbf{t}) = f(\pos).
\end{equation}
%
In \method, both \SE and \methoddiff capture the inherent features of objects in an invariant way, regardless of any translation or rotation transformations, as will be detailed in the following sections.

%%%%%%%%%%%%%%%%%%%%%%%%%%%%%%%%%%%%%%%%%%%%%
\section{Point Cloud Construction}
\label{supp:point_clouds}
%%%%%%%%%%%%%%%%%%%%%%%%%%%%%%%%%%%%%%%%%%%%%

In \method, we represented molecular surface shapes using point clouds (\pc).
%
$\pc$
serves as input to \SE, from which we derive shape latent embeddings.
%
To generate $\pc$, %\bo{\st{create this}}, \bo{generate $\pc$}
we initially generated a molecular surface mesh using the algorithm from the Open Drug Discovery Toolkit~\cite{Wjcikowski2015oddt}.
%
Following this, we uniformly sampled points on the mesh surface with probability proportional to the face area, %\xia{how to uniformly?}, ensuring the sampling is done proportionally to the face area, with
using the algorithm from PyTorch3D~\cite{ravi2020pytorch3d}.
%
This point cloud $\pc$ is then centralized by setting the center of its points to zero.
%
%

%%%%%%%%%%%%%%%%%%%%%%%%%%%%%%%%%%%%%%%%%%%%%
\section{Query Point Sampling}
\label{supp:training:shapeemb}
%%%%%%%%%%%%%%%%%%%%%%%%%%%%%%%%%%%%%%%%%%%%%

As described in Section ``Shape Decoder (\SED)'', the signed distances of query points $z_q$ to molecule surface shape $\pc$ are used to optimize \SE.
%
In this section, we present how to sample these points $z_q$ in 3D space.
%
Particularly, we first determined the bounding box around the molecular surface shape, using the maximum and minimum \mbox{($x$, $y$, $z$)-axis} coordinates for points in our point cloud \pc,
denoted as $(x_\text{min}, y_\text{min}, z_\text{min})$ and $(x_\text{max}, y_\text{max}, z_\text{max})$.
%
We extended this box slightly by defining its corners as \mbox{$(x_\text{min}-1, y_\text{min}-1, z_\text{min}-1)$} and \mbox{$(x_\text{max}+1, y_\text{max}+1, z_\text{max}+1)$}.
%
For sampling $|\mathcal{Z}|$ query points, we wanted an even distribution of points inside and outside the molecule surface shape.
%
%\ziqi{Typically, within this bounding box, molecules occupy only a small portion of volume, which makes it more likely to sample
%points outside the molecule surface shape.}
%
When a bounding box is defined around the molecule surface shape, there could be a lot of empty spaces within the box that the molecule does not occupy due to 
its complex and irregular shape.
%
This could lead to that fewer points within the molecule surface shape could be sampled within the box.
%
Therefore, we started by randomly sampling $3k$ points within our bounding box to ensure that there are sufficient points within the surface.
%
We then determined whether each point lies within the molecular surface, using an algorithm from Trimesh~\footnote{https://trimsh.org/} based on the molecule surface mesh.
%
If there are $n_w$ points found within the surface, we selected $n=\min(n_w, k/2)$ points from these points, 
and randomly choose the remaining 
%\bo{what do you mean by remaining? If all the 3k sampled points are inside the surface, you get no points left.} 
$k-n$ points 
from those outside the surface.
%
For each query point, we determined its signed distance to the molecule surface by its closest distance to points in \pc with a sign indicating whether it is inside the surface.

%%%%%%%%%%%%%%%%%%%%%%%%%%%%%%%%%%%%%%%%%%%%%
\section{Forward Diffusion (\diffnoise)}
\label{supp:forward}
%%%%%%%%%%%%%%%%%%%%%%%%%%%%%%%%%%%%%%%%%%%%%

%===================================================================
\subsection{{Forward Process}}
\label{supp:forward:forward}
%===================================================================

Formally, for atom positions, the probability of $\pos_t$ sampled given $\pos_{t-1}$, denoted as $q(\pos_t|\pos_{t-1})$, is defined as follows,
%\xia{revise the representation, should be $\beta^x_t$ -- note the space} as follows,
%
\begin{equation}
q(\pos_t|\pos_{t-1}) = \mathcal{N}(\pos_t|\sqrt{1-\beta^{\mathtt{x}}_t}\pos_{t-1}, \beta^{\mathtt{x}}_t\mathbb{I}), 
\label{eqn:noiseposinter}
\end{equation}
%
%\xia{should be a comma after the equation. you also missed it. }
%\st{in which} 
where %\hl{$\pos_0$ denotes the original atom position;} \xia{no $\pos_0$ in the equation...}
%$\mathbf{I}$ denotes the identity matrix;
$\mathcal{N}(\cdot)$ is a Gaussian distribution of $\pos_t$ with mean $\sqrt{1-\beta_t^{\mathtt{x}}}\pos_{t-1}$ and covariance $\beta_t^{\mathtt{x}}\mathbf{I}$.
%\xia{what is $\mathcal{N}$? what is $q$? you abused $q$. need to be crystal clear... }
%\bo{Should be $\sim$ not $=$ in the equation}
%
Following Hoogeboom \etal~\cite{hoogeboom2021catdiff}, 
%the forward process for the discrete atom feature $\atomfeat_t\in\mathbb{R}^K$ adds 
%categorical noise into $\atomfeat_{t-1}$ according to a variance schedule $\beta_t^v\in (0, 1)$. %as follows, %\hl{$\betav_t\in (0, 1)$} as follows,
%\xia{presentation...check across the entire manuscript... } as follows,
%
%\ziqi{Formally, 
for atom features, the probability of $\atomfeat_t$ across $K$ classes given $\atomfeat_{t-1}$ is defined as follows,
%
\begin{equation}
q(\atomfeat_t|\atomfeat_{t-1}) = \mathcal{C}(\atomfeat_t|(1-\beta^{\mathtt{v}}_t) \atomfeat_{t-1}+\beta^{\mathtt{v}}_t\mathbf{1}/K),
\label{eqn:noisetypeinter}
\end{equation}
%
where %\hl{$\atomfeat_0$ denotes the original atom positions}; 
$\mathcal{C}$ is a categorical distribution of $\atomfeat_t$ derived from the %by 
noising $\atomfeat_{t-1}$ with a uniform noise $\beta^{\mathtt{v}}_t\mathbf{1}/K$ across $K$ classes.
%adding an uniform noise $\beta^v_t$ to $\atomfeat_{t-1}$ across K classes.
%\xia{there is always a comma or period after the equations. Equations are part of a sentence. you always missed it. }
%\xia{what is $\mathcal{C}$? what does $q$ mean? it is abused. }

Since the above distributions form Markov chains, %} \xia{grammar!}, 
the probability of any $\pos_t$ or $\atomfeat_t$ can be derived from $\pos_0$ or $\atomfeat_0$:
%samples $\mol_0$ as follows,
%
\begin{eqnarray}
%\begin{aligned}
& q(\pos_t|\pos_{0}) & = \mathcal{N}(\pos_t|\sqrt{\cumalpha^{\mathtt{x}}_t}\pos_0, (1-\cumalpha^{\mathtt{x}}_t)\mathbb{I}), \label{eqn:noisepos}\\
& q(\atomfeat_t|\atomfeat_0)  & = \mathcal{C}(\atomfeat_t|\cumalpha^{\mathtt{v}}_t\atomfeat_0 + (1-\cumalpha^{\mathtt{v}}_t)\mathbf{1}/K), \label{eqn:noisetype}\\
& \text{where }\cumalpha^{\mathtt{u}}_t & = \prod\nolimits_{\tau=1}^{t}\alpha^{\mathtt{u}}_\tau, \ \alpha^{\mathtt{u}}_\tau=1 - \beta^{\mathtt{u}}_\tau, \ {\mathtt{u}}={\mathtt{x}} \text{ or } {\mathtt{v}}.\;\;\;\label{eqn:noiseschedule}
%\end{aligned}
\label{eqn:pos_prior}
\end{eqnarray}
%\xia{always punctuations after equations!!! also use ``eqnarray" instead of ``equation" + ``aligned" for multiple equations, each
%with a separate reference numbering...}
%\st{in which}, 
%where \ziqi{$\cumalpha^u_t = \prod_{\tau=1}^{t}\alpha^u_\tau$ and $\alpha^u_\tau=1 - \beta^u_\tau$ ($u$=$x$ or $v$)}.
%\xia{no such notations in the above equations; also subscript $s$ is abused with shape};
%$K$ is the number of categories for atom features.
%
%The details about noise schedules $\beta^x_t$ and $\beta^v_t$ are available in Supplementary Section \ref{XXX}. \ziqi{add trend}
%
Note that $\bar{\alpha}^{\mathtt{u}}_t$ ($\mathtt{u}={\mathtt{x}}\text{ or }{\mathtt{v}}$)
%($u$=$x$ or $v$) 
is monotonically decreasing from 1 to 0 over $t=[1,T]$. %\xia{=???}. 
%
As $t\rightarrow 1$, $\cumalpha^{\mathtt{x}}_t$ and $\cumalpha^{\mathtt{v}}_t$ are close to 1, leading to that $\pos_t$ or $\atomfeat_t$ approximates 
%the original data 
$\pos_0$ or $\atomfeat_0$.
%
Conversely, as  $t\rightarrow T$, $\cumalpha^{\mathtt{x}}_t$ and $\cumalpha^{\mathtt{v}}_t$ are close to 0,
leading to that $q(\pos_T|\pos_{0})$ %\st{$\rightarrow \mathcal{N}(\mathbf{0}, \mathbf{I})$} 
resembles  {$\mathcal{N}(\mathbf{0}, \mathbb{I})$} 
and $q(\atomfeat_T|\atomfeat_0)$ %\st{$\rightarrow \mathcal{C}(\mathbf{I}/K)$} 
resembles {$\mathcal{C}(\mathbf{1}/K)$}.

Using Bayes theorem, the ground-truth Normal posterior of atom positions $p(\pos_{t-1}|\pos_t, \pos_0)$ can be calculated in a
closed form~\cite{ho2020ddpm} as below,
%
\begin{eqnarray}
& p(\pos_{t-1}|\pos_t, \pos_0) = \mathcal{N}(\pos_{t-1}|\mu(\pos_t, \pos_0), \tilde{\beta}^\mathtt{x}_t\mathbb{I}), \label{eqn:gt_pos_posterior_1}\\
&\!\!\!\!\!\!\!\!\!\!\!\mu(\pos_t, \pos_0)\!=\!\frac{\sqrt{\bar{\alpha}^{\mathtt{x}}_{t-1}}\beta^{\mathtt{x}}_t}{1-\bar{\alpha}^{\mathtt{x}}_t}\pos_0\!+\!\frac{\sqrt{\alpha^{\mathtt{x}}_t}(1-\bar{\alpha}^{\mathtt{x}}_{t-1})}{1-\bar{\alpha}^{\mathtt{x}}_t}\pos_t, 
\tilde{\beta}^\mathtt{x}_t\!=\!\frac{1-\bar{\alpha}^{\mathtt{x}}_{t-1}}{1-\bar{\alpha}^{\mathtt{x}}_{t}}\beta^{\mathtt{x}}_t.\;\;\;
\end{eqnarray}
%
%\xia{Ziqi, please double check the above two equations!}
Similarly, the ground-truth categorical posterior of atom features $p(\atomfeat_{t-1}|\atomfeat_{t}, \atomfeat_0)$ can be calculated~\cite{hoogeboom2021catdiff} as below,
%
\begin{eqnarray}
& p(\atomfeat_{t-1}|\atomfeat_{t}, \atomfeat_0) = \mathcal{C}(\atomfeat_{t-1}|\mathbf{c}(\atomfeat_t, \atomfeat_0)), \label{eqn:gt_atomfeat_posterior_1}\\
& \mathbf{c}(\atomfeat_t, \atomfeat_0) = \tilde{\mathbf{c}}/{\sum_{k=1}^K \tilde{c}_k}, \label{eqn:gt_atomfeat_posterior_2} \\
& \tilde{\mathbf{c}} = [\alpha^{\mathtt{v}}_t\atomfeat_t + \frac{1 - \alpha^{\mathtt{v}}_t}{K}]\odot[\bar{\alpha}^{\mathtt{v}}_{t-1}\atomfeat_{0}+\frac{1-\bar{\alpha}^{\mathtt{v}}_{t-1}}{K}], 
\label{eqn:gt_atomfeat_posterior_3}
%\label{eqn:atomfeat_posterior}
\end{eqnarray}
%
%\xia{Ziqi: please double check the above equations!}
%
where $\tilde{c}_k$ denotes the likelihood of $k$-th class across $K$ classes in $\tilde{\mathbf{c}}$; 
$\odot$ denotes the element-wise product operation;
$\tilde{\mathbf{c}}$ is calculated using $\atomfeat_t$ and $\atomfeat_{0}$ and normalized into $\mathbf{c}(\atomfeat_t, \atomfeat_0)$ so as to represent
probabilities. %\xia{is this correct? is $\tilde{c}_k$ always greater than 0?}
%\xia{how is it calculated?}.
%\ziqi{the likelihood distribution $\tilde{c}$ is calculated by $p(\atomfeat_t|\atomfeat_{t-1})p(\atomfeat_{t-1}|\atomfeat_0)$, according to 
%Equation~\ref{eqn:noisetypeinter} and \ref{eqn:noisetype}.
%\xia{need to write the key idea of the above calculation...}
%
The proof of the above equations is available in Supplementary Section~\ref{supp:forward:proof}.

%===================================================================
\subsection{Variance Scheduling in \diffnoise}
\label{supp:forward:variance}
%===================================================================

Following Guan \etal~\cite{guan2023targetdiff}, we used a sigmoid $\beta$ schedule for the variance schedule $\beta_t^{\mathtt{x}}$ of atom coordinates as below,

\begin{equation}
\beta_t^{\mathtt{x}} = \text{sigmoid}(w_1(2 t / T - 1)) (w_2 - w_3) + w_3
\end{equation}
in which $w_i$($i$=1,2, or 3) are hyperparameters; $T$ is the maximum step.
%
We set $w_1=6$, $w_2=1.e-7$ and $w_3=0.01$.
%
For atom types, we used a cosine $\beta$ schedule~\cite{nichol2021} for $\beta_t^{\mathtt{v}}$ as below,

\begin{equation}
\begin{aligned}
& \bar{\alpha}_t^{\mathtt{v}} = \frac{f(t)}{f(0)}, f(t) = \cos(\frac{t/T+s}{1+s} \cdot \frac{\pi}{2})^2\\
& \beta_t^{\mathtt{v}} = 1 - \alpha_t^{\mathtt{v}} = 1 - \frac{\bar{\alpha}_t^{\mathtt{v}} }{\bar{\alpha}_{t-1}^{\mathtt{v}} }
\end{aligned}
\end{equation}
in which $s$ is a hyperparameter and set as 0.01.

As shown in Section ``Forward Diffusion Process'', the values of $\beta_t^{\mathtt{x}}$ and $\beta_t^{\mathtt{v}}$ should be 
sufficiently small to ensure the smoothness of forward diffusion process. In the meanwhile, their corresponding $\bar{\alpha}_t$
values should decrease from 1 to 0 over $t=[1,T]$.
%
Figure~\ref{fig:schedule} shows the values of $\beta_t$ and $\bar{\alpha}_t$ for atom coordinates and atom types with our hyperparameters.
%
Please note that the value of $\beta_{t}^{\mathtt{x}}$ is less than 0.1 for 990 out of 1,000 steps. %\bo{\st{, though it increases when $t$ is close to 1,000}}.
%
This guarantees the smoothness of the forward diffusion process.
%\bo{add $\beta_t^{\mathtt{x}}$ and $\beta_t^{\mathtt{v}}$ in the legend of the figure...}
%\bo{$\beta_t^{\mathtt{v}}$ does not look small when $t$ is close to 1000...}

\begin{figure}
	\begin{subfigure}[t]{.45\linewidth}
		\centering
		\includegraphics[width=.7\linewidth]{figures/var_schedule_beta.pdf}
	\end{subfigure}
	%
	\hfill
	\begin{subfigure}[t]{.45\linewidth}
		\centering
		\includegraphics[width=.7\linewidth]{figures/var_schedule_alpha.pdf}
	\end{subfigure}
	\caption{Schedule}
	\label{fig:schedule}
\end{figure}

%===================================================================
\subsection{Derivation of Forward Diffusion Process}
\label{supp:forward:proof}
%===================================================================

In \method, a Gaussian noise and a categorical noise are added to continuous atom position and discrete atom features, respectively.
%
Here, we briefly describe the derivation of posterior equations (i.e., Eq.~\ref{eqn:gt_pos_posterior_1}, and   \ref{eqn:gt_atomfeat_posterior_1}) for atom positions and atom types in our work.
%
We refer readers to Ho \etal~\cite{ho2020ddpm} and Kong \etal~\cite{kong2021diffwave} %\bo{add XXX~\etal here...} \cite{ho2020ddpm,kong2021diffwave} 	
for a detailed description of diffusion process for continuous variables and Hoogeboom \etal~\cite{hoogeboom2021catdiff} for
%\bo{add XXX~\etal here...} \cite{hoogeboom2021catdiff} for
the description of diffusion process for discrete variables.

For continuous atom positions, as shown in Kong \etal~\cite{kong2021diffwave}, according to Bayes theorem, given $q(\pos_t|\pos_{t-1})$ defined in Eq.~\ref{eqn:noiseposinter} and 
$q(\pos_t|\pos_0)$ defined in Eq.~\ref{eqn:noisepos}, the posterior $q(\pos_{t-1}|\pos_{t}, \pos_0)$ is derived as below (superscript $\mathtt{x}$ is omitted for brevity),

\begin{equation}
\begin{aligned}
& q(\pos_{t-1}|\pos_{t}, \pos_0)  = \frac{q(\pos_t|\pos_{t-1}, \pos_0)q(\pos_{t-1}|\pos_0)}{q(\pos_t|\pos_0)} \\
& =  \frac{\mathcal{N}(\pos_t|\sqrt{1-\beta_t}\pos_{t-1}, \beta_{t}\mathbf{I}) \mathcal{N}(\pos_{t-1}|\sqrt{\bar{\alpha}_{t-1}}\pos_{0}, (1-\bar{\alpha}_{t-1})\mathbf{I}) }{ \mathcal{N}(\pos_{t}|\sqrt{\bar{\alpha}_t}\pos_{0}, (1-\bar{\alpha}_t)\mathbf{I})}\\
& =  (2\pi{\beta_t})^{-\frac{3}{2}} (2\pi{(1-\bar{\alpha}_{t-1})})^{-\frac{3}{2}} (2\pi(1-\bar{\alpha}_t))^{\frac{3}{2}} \times \exp( \\
& -\frac{\|\pos_t - \sqrt{\alpha}_t\pos_{t-1}\|^2}{2\beta_t} -\frac{\|\pos_{t-1} - \sqrt{\bar{\alpha}}_{t-1}\pos_{0} \|^2}{2(1-\bar{\alpha}_{t-1})} \\
& + \frac{\|\pos_t - \sqrt{\bar{\alpha}_t}\pos_0\|^2}{2(1-\bar{\alpha}_t)}) \\
& = (2\pi\tilde{\beta}_t)^{-\frac{3}{2}} \exp(-\frac{1}{2\tilde{\beta}_t}\|\pos_{t-1}-\frac{\sqrt{\bar{\alpha}_{t-1}}\beta_t}{1-\bar{\alpha}_t}\pos_0 \\
& - \frac{\sqrt{\alpha_t}(1-\bar{\alpha}_{t-1})}{1-\bar{\alpha}_t}\pos_{t}\|^2) \\
& \text{where}\ \tilde{\beta}_t = \frac{1-\bar{\alpha}_{t-1}}{1-\bar{\alpha}_t}\beta_t.
\end{aligned}
\end{equation}
%\bo{marked part does not look right to me.}
%\bo{How to you derive from the second equation to the third one?}

Therefore, the posterior of atom positions is derived as below,

\begin{equation}
q(\pos_{t-1}|\pos_{t}, \pos_0)\!\!=\!\!\mathcal{N}(\pos_{t-1}|\frac{\sqrt{\bar{\alpha}_{t-1}}\beta_t}{1-\bar{\alpha}_t}\pos_0 + \frac{\sqrt{\alpha_t}(1-\bar{\alpha}_{t-1})}{1-\bar{\alpha}_t}\pos_{t}, \tilde{\beta}_t\mathbf{I}).
\end{equation}

For discrete atom features, as shown in Hoogeboom \etal~\cite{hoogeboom2021catdiff} and Guan \etal~\cite{guan2023targetdiff},
according to Bayes theorem, the posterior $q(\atomfeat_{t-1}|\atomfeat_{t}, \atomfeat_0)$ is derived as below (supperscript $\mathtt{v}$ is omitted for brevity),

\begin{equation}
\begin{aligned}
& q(\atomfeat_{t-1}|\atomfeat_{t}, \atomfeat_0) =  \frac{q(\atomfeat_t|\atomfeat_{t-1}, \atomfeat_0)q(\atomfeat_{t-1}|\atomfeat_0)}{\sum_{\scriptsize{\atomfeat}_{t-1}}q(\atomfeat_t|\atomfeat_{t-1}, \atomfeat_0)q(\atomfeat_{t-1}|\atomfeat_0)} \\
%& = \frac{\mathcal{C}(\atomfeat_t|(1-\beta_t)\atomfeat_{t-1} + \beta_t\frac{\mathbf{1}}{K}) \mathcal{C}(\atomfeat_{t-1}|\bar{\alpha}_{t-1}\atomfeat_0+(1-\bar{\alpha}_{t-1})\frac{\mathbf{1}}{K})} \\
\end{aligned}
\end{equation}

For $q(\atomfeat_t|\atomfeat_{t-1}, \atomfeat_0)$, we have % $\atomfeat_t=\atomfeat_{t-1}$ with probability $1-\beta_t+\beta_t / K$, and $\atomfeat_t \neq \atomfeat_{t-1}$
%with probability $\beta_t / K$.
%
%Therefore, this function can be symmetric, that is, 
%
\begin{equation}
\begin{aligned}
q(\atomfeat_t|\atomfeat_{t-1}, \atomfeat_0) & = \mathcal{C}(\atomfeat_t|(1-\beta_t)\atomfeat_{t-1} + \beta_t/{K})\\
& = \begin{cases}
1-\beta_t+\beta_t/K,\!&\text{when}\ \atomfeat_{t} = \atomfeat_{t-1},\\
\beta_t / K,\! &\text{when}\ \atomfeat_{t} \neq \atomfeat_{t-1},
\end{cases}\\
& = \mathcal{C}(\atomfeat_{t-1}|(1-\beta_t)\atomfeat_{t} + \beta_t/{K}).
\end{aligned}
%\mathcal{C}(\atomfeat_{t-1}|(1-\beta_{t})\atomfeat_{t} + \beta_t\frac{\mathbf{1}}{K}).
\end{equation}
%
Therefore, we have
%\bo{why it can be symmetric}
%
\begin{equation}
\begin{aligned}
& q(\atomfeat_t|\atomfeat_{t-1}, \atomfeat_0)q(\atomfeat_{t-1}|\atomfeat_0) \\
& = \mathcal{C}(\atomfeat_{t-1}|(1-\beta_t)\atomfeat_{t} + \beta_t\frac{\mathbf{1}}{K}) \mathcal{C}(\atomfeat_{t-1}|\bar{\alpha}_{t-1}\atomfeat_0+(1-\bar{\alpha}_{t-1})\frac{\mathbf{1}}{K}) \\
& = [\alpha_t\atomfeat_t + \frac{1 - \alpha_t}{K}]\odot[\bar{\alpha}_{t-1}\atomfeat_{0}+\frac{1-\bar{\alpha}_{t-1}}{K}].
\end{aligned}
\end{equation}
%
%\bo{what is $\tilde{\mathbf{c}}$...}
Therefore, with $q(\atomfeat_t|\atomfeat_{t-1}, \atomfeat_0)q(\atomfeat_{t-1}|\atomfeat_0) = \tilde{\mathbf{c}}$, the posterior is as below,

\begin{equation}
q(\atomfeat_{t-1}|\atomfeat_{t}, \atomfeat_0) = \mathcal{C}(\atomfeat_{t-1}|\mathbf{c}(\atomfeat_t, \atomfeat_0)) = \frac{\tilde{\mathbf{c}}}{\sum_{k}^K\tilde{c}_k}.
\end{equation}

%%%%%%%%%%%%%%%%%%%%%%%%%%%%%%%%%%%%%%%%%%%%%
\section{{Backward Generative Process} (\diffgenerative)}
\label{supp:backward}
%%%%%%%%%%%%%%%%%%%%%%%%%%%%%%%%%%%%%%%%%%%%%

Following Ho \etal~\cite{ho2020ddpm}, with $\tilde{\pos}_{0,t}$, the probability of $\pos_{t-1}$ denoised from $\pos_t$, denoted as $p(\pos_{t-1}|\pos_t)$,
can be estimated %\hl{parameterized} \xia{???} 
by the approximated posterior $p_{\boldsymbol{\Theta}}(\pos_{t-1}|\pos_t, \tilde{\pos}_{0,t})$ as below,
%
\begin{equation}
\begin{aligned}
p(\pos_{t-1}|\pos_t) & \approx p_{\boldsymbol{\Theta}}(\pos_{t-1}|\pos_t, \tilde{\pos}_{0,t}) \\
& = \mathcal{N}(\pos_{t-1}|\mu_{\boldsymbol{\Theta}}(\pos_t, \tilde{\pos}_{0,t}),\tilde{\beta}_t^{\mathtt{x}}\mathbb{I}),
\end{aligned}
\label{eqn:aprox_pos_posterior}
\end{equation}
%
where ${\boldsymbol{\Theta}}$ is the learnable parameter; $\mu_{\boldsymbol{\Theta}}(\pos_t, \tilde{\pos}_{0,t})$ is an estimate %estimation
of $\mu(\pos_t, \pos_{0})$ by replacing $\pos_0$ with its estimate $\tilde{\pos}_{0,t}$ 
in Equation~{\ref{eqn:gt_pos_posterior_1}}.
%
Similarly, with $\tilde{\atomfeat}_{0,t}$, the probability of $\atomfeat_{t-1}$ denoised from $\atomfeat_t$, denoted as $p(\atomfeat_{t-1}|\atomfeat_t)$, 
can be estimated %\hl{parameterized} 
by the approximated posterior $p_{\boldsymbol{\Theta}}(\atomfeat_{t-1}|\atomfeat_t, \tilde{\atomfeat}_{0,t})$ as below,
%
\begin{equation}
\begin{aligned}
p(\atomfeat_{t-1}|\atomfeat_t)\approx p_{\boldsymbol{\Theta}}(\atomfeat_{t-1}|\atomfeat_{t}, \tilde{\atomfeat}_{0,t}) 
=\mathcal{C}(\atomfeat_{t-1}|\mathbf{c}_{\boldsymbol{\Theta}}(\atomfeat_t, \tilde{\atomfeat}_{0,t})),\!\!\!\!
\end{aligned}
\label{eqn:aprox_atomfeat_posterior}
\end{equation}
%
where $\mathbf{c}_{\boldsymbol{\Theta}}(\atomfeat_t, \tilde{\atomfeat}_{0,t})$ is an estimate of $\mathbf{c}(\atomfeat_t, \atomfeat_0)$
by replacing $\atomfeat_0$  
with its estimate $\tilde{\atomfeat}_{0,t}$ in Equation~\ref{eqn:gt_atomfeat_posterior_1}.



%===================================================================
\section{\method Loss Function Derivation}
\label{supp:training:loss}
%===================================================================

In this section, we demonstrate that a step weight $w_t^{\mathtt{x}}$ based on the signal-to-noise ratio $\lambda_t$ should be 
included into the atom position loss (Eq.~\ref{eqn:diff:obj:pos}).
%
In the diffusion process for continuous variables, the optimization problem is defined 
as below~\cite{ho2020ddpm},
%
\begin{equation*}
\begin{aligned}
& \arg\min_{\boldsymbol{\Theta}}KL(q(\pos_{t-1}|\pos_t, \pos_0)|p_{\boldsymbol{\Theta}}(\pos_{t-1}|\pos_t, \tilde{\pos}_{0,t})) \\
& = \arg\min_{\boldsymbol{\Theta}} \frac{\bar{\alpha}_{t-1}(1-\alpha_t)}{2(1-\bar{\alpha}_{t-1})(1-\bar{\alpha}_{t})}\|\tilde{\pos}_{0, t}-\pos_0\|^2 \\
& = \arg\min_{\boldsymbol{\Theta}} \frac{1-\alpha_t}{2(1-\bar{\alpha}_{t-1})\alpha_{t}} \|\tilde{\boldsymbol{\epsilon}}_{0,t}-\boldsymbol{\epsilon}_0\|^2,
\end{aligned}
\end{equation*}
where $\boldsymbol{\epsilon}_0 = \frac{\pos_t - \sqrt{\bar{\alpha}_t}\pos_0}{\sqrt{1-\bar{\alpha}_t}}$ is the ground-truth noise variable sampled from $\mathcal{N}(\mathbf{0}, \mathbf{1})$ and is used to sample $\pos_t$ from $\mathcal{N}(\pos_t|\sqrt{\cumalpha_t}\pos_0, (1-\cumalpha_t)\mathbf{I})$ in Eq.~\ref{eqn:noisetype};
$\tilde{\boldsymbol{\epsilon}}_0 = \frac{\pos_t - \sqrt{\bar{\alpha}_t}\tilde{\pos}_{0, t}}{\sqrt{1-\bar{\alpha}_t}}$ is the predicted noise variable. 

%A simplified training objective is proposed by Ho \etal~\cite{ho2020ddpm} as below,
Ho \etal~\cite{ho2020ddpm} further simplified the above objective as below and
demonstrated that the simplified one can achieve better performance:
%
\begin{equation}
\begin{aligned}
& \arg\min_{\boldsymbol{\Theta}} \|\tilde{\boldsymbol{\epsilon}}_{0,t}-\boldsymbol{\epsilon}_0\|^2 \\
& = \arg\min_{\boldsymbol{\Theta}} \frac{\bar{\alpha}_t}{1-\bar{\alpha}_t}\|\tilde{\pos}_{0,t}-\pos_0\|^2,
\end{aligned}
\label{eqn:supp:losspos}
\end{equation}
%
where $\lambda_t=\frac{\bar{\alpha}_t}{1-\bar{\alpha}_t}$ is the signal-to-noise ratio.
%
While previous work~\cite{guan2023targetdiff} applies uniform step weights across
different steps, we demonstrate that a step weight should be included into the atom position loss according to Eq.~\ref{eqn:supp:losspos}.
%
However, the value of $\lambda_t$ could be very large when $\bar{\alpha}_t$ is close to 1 as $t$ approaches 1.
%
Therefore, we clip the value of $\lambda_t$ with threshold $\delta$ in Eq.~\ref{eqn:diff:obj:pos}.




\end{document}