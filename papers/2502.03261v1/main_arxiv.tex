\documentclass{article}

% if you need to pass options to natbib, use, e.g.:
%     \PassOptionsToPackage{numbers, compress}{natbib}
% before loading neurips_2021

% ready for submission 
\PassOptionsToPackage{round}{natbib}
% \usepackage[preprint]{neurips_2024}

\usepackage{colm2024_conference}
\colmfinalcopy



\usepackage{hyperref}
\usepackage{xcolor}
% \usepackage{tcolorbox}
\usepackage{algorithmic}
\usepackage{longtable}
\usepackage{wrapfig}

\newtheorem{ex}{Example}[section]
\newcounter{excounter}[section]
\renewcommand{\theexcounter}{\thesection.\arabic{excounter}}
%\usepackage{minted}



\newenvironment{myex}[1][]
  {\stepcounter{excounter}\begin{mybox}{yellow!70!black}\textbf{Example~\theexcounter\ #1}\newline}
  {\end{mybox}}

% to compile a preprint version, e.g., for submission to arXiv, add add the
% [preprint] option:
%     \usepackage[preprint]{neurips_2021}

% to compile a camera-ready version, add the [final] option, e.g.:
%     \usepackage[final]{neurips_2021}

% to avoid loading the natbib package, add option nonatbib:
%    \usepackage[nonatbib]{neurips_2021}

\usepackage[utf8]{inputenc} % allow utf-8 input
\usepackage[T1]{fontenc}    % use 8-bit T1 fonts
\usepackage{hyperref}       % hyperlinks
\usepackage{url}            % simple URL typesetting
\usepackage{booktabs}       % professional-quality tables
\usepackage{amsfonts}       % blackboard math symbols
\usepackage{nicefrac}       % compact symbols for 1/2, etc.
\usepackage{microtype}      % microtypography
\usepackage{subcaption} 
\usepackage[symbol]{footmisc}
\usepackage{tcolorbox}
\tcbuselibrary{listings,breakable}
\usepackage{verbatimbox}
%\usepackage{minted}
%\usepackage[finalizecache,cachedir=.]{minted}
 \usepackage[frozencache,cachedir=.]{minted}
\usepackage{mdframed}

\hypersetup{
    colorlinks,
    % linkcolor={black},
    linkcolor={red!50!black},
    citecolor={blue!60!black},
    urlcolor={blue!80!black}
}


\newcommand\blfootnote[1]{%
  \begingroup
  \renewcommand\thefootnote{}\footnote{#1}%
  \addtocounter{footnote}{-1}%
  \endgroup
}
\colorlet{darkred}{red!60!black}

\usepackage{algorithm}
% \usepackage{algpseudocode}
\usepackage{enumitem}
  \setlist{leftmargin=*}
\usepackage{graphicx}
  \graphicspath{ {./figs/} }
\usepackage{lineno}
  \linenumbers
\usepackage{YK}
\usepackage{wrapfig}  % Required for wrapping figures
\newcommand{\thought}[1]{{\color[rgb]{0.2,0.39,0.66}(#1)}}
\newcommand{\todo}[1]{{\color[rgb]{1.0,0.0,0.0}(#1)}}
\newcommand{\hsh}[1]{{\color{green!50!black} Henrik: #1}}
\newcommand{\st}[1]{{\color{red!50!black} Sebastian: #1}}

\newcommand{\ulm}[1]{_{\scaleto{\mathrm{#1}}{3pt}}}
\newcommand\at[2]{\left.#1\right|_{#2}}











\newtheorem{assumption}{Assumption}

\DeclareMathOperator*{\argmax}{arg\,max}
\DeclareMathOperator*{\argmin}{arg\,min}

\newcommand{\swname}[1]{\texttt{#1}}
\newcommand{\ie}{i\/.\/e\/.,\/~}
\newcommand{\eg}{e\/.\/g\/.,\/~}
\newcommand{\cf}{cf\/.\/~}

\newcommand{\fig}{Fig\/.\/~}
\newcommand{\defn}{Def\/.\/~}
\newcommand{\sect}{Sec\/.\/~}
\newcommand{\tabl}{Tab\/.\/~}
\newcommand{\algo}{Algorithm~}
\newcommand{\theo}{Theorem~}

\newcommand{\bnnl}{3 hidden layers}
\newcommand{\bnnn}{50 neurons}
\newcommand{\bnna}{tanh activations}

\newcommand{\capt}[1]{\mdseries{\emph{#1}}}

\newcommand{\videolink}{at \url{https://youtu.be/_d7AqTRjz6g}}
\newcommand{\codelink}{\url{https://github.com/wheelbot/mini-wheelbot}}

\newcommand{\fakepar}[1]{\vspace{0mm}\noindent\textbf{#1.}}

\newcommand{\needref}{\textcolor{red}{[REF]}}

\newcommand{\plotfontsize}{9pt}

% \usepackage{todonotes}
\def\logit{\mathrm{logit}}

\newcommand{\my}[1]{\textcolor{orange}{[MY: #1]}}

\title{CARROT:  A Cost Aware Rate Optimal Router}

% The \author macro works with any number of authors. There are two commands
% used to separate the names and addresses of multiple authors: \And and \AND.
%
% Using \And between authors leaves it to LaTeX to determine where to break the
% lines. Using \AND forces a line break at that point. So, if LaTeX puts 3 of 4
% authors names on the first line, and the last on the second line, try using
% \AND instead of \And before the third author name.
\author{%
  Seamus Somerstep$^{\dagger \square}$ \ \ \
  Felipe Maia Polo$^{\dagger  \square}$ \ \ \
  Allysson Flavio Melo de Oliveira $^ {\ddag \circ}$ \\ 
  \textbf{Prattyush Mangal} $^ \ddag$ \ \ \
  \textbf{Mírian Silva} $^ {\ddag \circ \triangle}$  \ \ \
  \textbf{Onkar Bhardwaj} $^ {\ddag \circ}$ \\ \vspace{0.5cm}
  \textbf{Mikhail Yurochkin}* $^ {\ddag \circ}$ \ \ \
  \textbf{Subha Maity}* $^ \spadesuit$ \\ 
\normalsize $^\square$ Department of Statistics, University of Michigan\\
$^\ddagger$ IBM Research  ~~~ $^\circ$ MIT-IBM Watson AI Lab  ~~~
$\triangle$ Federal University of Minas Gerais\\
$^ \spadesuit$ Department of Statistics and Actuarial Science, University of Waterloo 
  % examples of more authors
  % \And
  % Coauthor \\
  % Affiliation \\
  % Address \\
  % \texttt{email} \\
  % \AND
  % Coauthor \\
  % Affiliation \\
  % Address \\
  % \texttt{email} \\
  % \And
  % Coauthor \\
  % Affiliation \\
  % Address \\
  % \texttt{email} \\
  % \And
  % Coauthor \\
  % Affiliation \\
  % Address \\
  % \texttt{email} \\
}

%\author{%
  %Seamus Somerstep$^{\dagger \square}$\\
  %\texttt{smrstep@umich.edu}\\ 
  %\And
  %Felipe Maia Polo$^{\dagger  \square}$\\
  %\texttt{maiapolo@umich.edu}\\
  %\AND
  %Allysson Flavio Melo de Oliveira $^ {\ddag \circ}$ \\
  %\texttt{allysson@br.ibm.com} \\
  %\And
  %Prattyush Mangal $^ \ddag$  \\
  %\texttt{prattyush.mangal@ibm.com} \\
  %\AND
  %Mírian Silva $^ {\ddag \circ \triangle}$  \\
  %\texttt{mirianfrsilva@ibm.com} \\
  %\And
  %Onkar Bhardwaj $^ {\ddag \circ}$  \\
  %\texttt{onkarbhardwaj@ibm.com} \\
  %\AND
  %Mikhail Yurochkin* $^ {\ddag \circ}$  \\
  %\texttt{mikhail.yurochkin@ibm.com} \\
 % \And
  %Subha Maity* $^ \spadesuit$ \\
  %\texttt{smaity@uwaterloo.ca} \\
  % examples of more authors
  % \And
  % Coauthor \\
  % Affiliation \\
  % Address \\
  % \texttt{email} \\
  % \AND
  % Coauthor \\
  % Affiliation \\
  % Address \\
  % \texttt{email} \\
  % \And
  % Coauthor \\
  % Affiliation \\
  % Address \\
  % \texttt{email} \\
  % \And
  % Coauthor \\
  % Affiliation \\
  % Address \\
  % \texttt{email} \\
%}

% \usepackage{setspace}


\begin{document}
\nolinenumbers


\maketitle


% \begin{figure*}
%     \centering
    
    
%     \href{https://huggingface.co/CARROT-LLM-Routing}{https://huggingface.co/CARROT-LLM-Routing}

% \end{figure*}

\vspace{-0.8cm}
\begin{center}
   \large{ %Hugging Face repository:
   \href{https://huggingface.co/CARROT-LLM-Routing}{\textcolor{purple!80!black}{\texttt{huggingface.co/CARROT-LLM-Routing}}}}
   % \large{\url{https://huggingface.co/CARROT-LLM-Routing}}
\end{center}
  % 


% \linenumbers


\begin{abstract}
    \begin{abstract}
Retrieval-Augmented Generation (RAG) is often used with Large Language Models (LLMs) to infuse domain knowledge or user-specific information. In RAG, given a user query, a retriever extracts chunks of relevant text from a knowledge base. These chunks are sent to an LLM as part of the input prompt. Typically, any given chunk is repeatedly retrieved across user questions. However, currently, for every question, attention-layers in LLMs fully compute the key values (KVs) repeatedly for the input chunks, as state-of-the-art methods cannot reuse KV-caches when chunks appear at arbitrary locations with arbitrary contexts. Naive reuse leads to output quality degradation.  This leads to potentially redundant computations on expensive GPUs and increases latency. In this work, we propose \sys, a system for managing and reusing precomputed KVs corresponding to the text chunks (we call \textit{chunk-caches}) in RAG-based systems. We present how to identify \hl{\textit{chunk-caches} that are reusable}, how to efficiently perform a small fraction of recomputation to \textit{fix} the cache to maintain output quality, and how to efficiently store and evict \textit{chunk-caches} in the hardware for maximizing reuse while masking any overheads. With real production workloads as well as synthetic datasets, we show that \sys reduces redundant computation by \textbf{51\%} over SOTA prefix-caching and \textbf{75\%} over full recomputation.
\hl{Additionally, with continuous batching on a real production workload, we get a \textbf{1.6$\times$} speedup in throughput and a \textbf{2$\times$} reduction in end-to-end response latency over prefix-caching while maintaining quality, for both the \llama-3-8B and \llama-3-70B models. 
}
\end{abstract}





\end{abstract}


% {\bf TL;DR:} We present a theoretical and empirical investigation of performance and cost trade-offs in LLM routing. We introduce 

\vspace{-0.5cm} 
\section{Introduction}
% Event cameras are innovative bio-inspired sensors.
% Unlike traditional frame cameras, Event cameras do not operate at a fixed rate but asynchronously report pixel-wise intensity changes, known as events (\fig \ref{relatedwork}a). 
% With microsecond level resolution and an asynchronous, differential operating principle, event cameras excel at capturing high-speed motions that cause severe motion blur in frame cameras. 
% Additionally, Event cameras have a very high dynamic range (HDR) of 140dB compared to 60dB in frame cameras, performing well under varied illumination conditions. 
% Consequently, event cameras are considered an important sensing modality and are increasingly used for tasks like motion tracking and Simultaneous Localization and Mapping (SLAM).


% Event cameras are innovative bio-inspired sensors that report pixel-wise intensity changes as events asynchronously with microsecond sensing latency (\fig \ref{intro}a), rather than fixed interval frames \tocite. sensing latency is the time from a visual stimulus appearing to its sensor readout.
% Event cameras are innovative bio-inspired sensors that asynchronously report pixel brightness changes as events with \textit{millisecond latency} (\fig \ref{intro}a), instead of fixed interval frames \cite{he2024microsaccade, gehrig2024low}.  
% % sensing latency is the time from visual stimulus to sensor readout \cite{gehrig2024low}.
% With high temporal resolution and a high dynamic range, event cameras excel at capturing high-speed motions without blurring and perform well under varied illumination conditions \cite{falanga2020dynamic, xu2023taming}.
% Thus, event cameras are envisioned as an ideal solution for challenging 2D vision tasks, such as low latency and accurate object detection in \fig \ref{intro}b \cite{gallego2020event}.

Event cameras are innovative bio-inspired sensors that report changes in pixel brightness asynchronously as events with \textit{millisecond latency} (\fig \ref{intro}a), rather than at fixed-time
intervals \cite{he2024microsaccade, gehrig2024low}.  
% sensing latency is the time from visual stimulus to sensor readout \cite{gehrig2024low}.
With high temporal resolution and a high dynamic range, event cameras excel at capturing high-speed motions without blurring and perform well under varied illumination conditions \cite{falanga2020dynamic, xu2023taming}.
Thus, event cameras are envisioned as an ideal solution for 2D vision tasks, such as low latency and accurate object detection as shown in \fig \ref{intro}b \cite{gallego2020event}.

% Similar to frame cameras, event cameras encounter scale uncertainty (\aka, they struggle to accurately estimate object depth) \tocite.
% This challenge hinders event cameras from fully realizing their potential in 3D object localization \tocite. 
% Similar to frame cameras, event cameras encounter scale uncertainty (\aka, they struggle to accurately estimate object depth) \tocite.
% This challenge hinders event cameras from fully realizing their potential in 3D object localization \tocite. 
% 尽管事件相机在上述 2D vision tasks取得了不错的表现,
% However, event cameras struggle to fully realize their potential in low-latency 3D object localization, which has various potential applications (\eg, drone localization,入侵物体定位等)
% because they encounter scale uncertainty (\aka, they struggle to accurately estimate depth) \cite{zhang2022mobidepth}. 
% Although event cameras perform well in the aforementioned 2D vision tasks, they struggle to fully realize their potential in low-latency 3D object localization, which 对事件相机视野中出现的物体进行三维定位, has various potential applications (\eg, drone localization, intruding object detection, AR/MR) due to scale uncertainty (\aka, difficulty in accurately estimating depth) \cite{zhang2022mobidepth}.
% Although event cameras excel in the 2D vision tasks, they struggle to fully realize their potential in 3D vision tasks 以 low-latency 3D object localization为代表, which involves localizing the object within the camera's field of view in three dimensions (\fig \ref{intro}c). 
% Although event cameras excel in aforementioned 2D vision tasks, they struggle to fully realize their potential in 3D vision tasks, particularly in low-latency 3D object localization, which involves localizing objects within the camera's field of view in three dimensions (\fig \ref{intro}c) \cite{qin2019monogrnet}.
% The latency measures the time elapsed from visual stimulus to resulting localization output.
% This limitation, due to scale uncertainty (\ie, difficulty in accurately estimating depth) \cite{zhang2022mobidepth}, affects various potential vital applications of event cameras (\eg, drone localization \cite{wang2022micnest}, intruding object detection\cite{han2015twins}).

% Although event cameras excel in 2D vision tasks, they face fundamental challenges in 3D vision, preventing their full potential from being realized.
% Specifically, 3D object localization, identifying the location of objects within the camera's field of view in three dimensions, is a fundamental function for various vital 3D vision tasks (\eg, drone localization \cite{wang2022micnest}, AR/MR \cite{xu2021followupar}).
% However, event cameras, capturing per-pixel brightness changes in 2D without depth details, can't directly gauge object distance, causing scale uncertainty. 
% This limitation restricts event cameras in 3D object localization , hindering the exploitation of their low-latency advantage in 3D vision tasks.
% % To address this, 
% Two types of solutions are proposed:

However, event cameras face significant challenges when applied to more complex 3D vision tasks.
% which prevents their full potential from being realized. 
For instance, 3D object localization, which identifies the location of objects within the camera's field of view in three dimensions, is a fundamental block for various vital 3D vision tasks (\eg, drone navigation \cite{wang2022micnest}, augmented/mixed reality \cite{xu2021followupar}).
Event cameras only capture per-pixel brightness changes in 2D devoid of depth details, resulting in scale uncertainty that hinders their effectiveness in 3D object localization (\fig \ref{intro}c).
This limitation further restricts the exploitation of their potential in various 3D vision tasks.
To address this, two primary types of solutions are proposed to enhance event cameras:

% However, event cameras can only capture 2D images and lack depth information, making it impossible to directly measure the actual distance of the object. 
% This leads to scale uncertainty, preventing event cameras from performing 3D object localization (\fig \ref{intro}c).
% This prevents 3D object localization from leveraging the low-latency advantage of event cameras and hinders their use in various vital 3D vision applications.
% Two type solutions are proposed to augment event cameras:

% However, similar to frame cameras, event cameras face scale uncertainty (\aka, they cannot accurately estimate the depth of objects) \tocite.
% This is a fundamental challenge that prevents event cameras from fully realizing their potential in 3D object localization and tracking \tocite. 
% There are mainly two types of solutions proposed to address this issue, supplementing event cameras with depth information of objects:

\noindent $\bullet$ \textbf{Events only-based solutions.}
These methods rely solely on event data for object depth estimation and fall into two types. 
$(i)$ Incorporating known geometric information with observations to deduce depth. 
These methods rely heavily on prior knowledge, leading to poor performance in new scenes or with new objects \cite{falanga2020dynamic}.
$(ii)$ Employing machine learning algorithms that either stick events within a time window (\eg, $1ms$) into an image for DNN-based estimation \cite{guo2022low}, or devise event-oriented networks (\eg, SNN) for object localization \cite{zhou2023computational, barchid2023spiking}. 
These methods are computationally intensive during network inference \cite{diehl2015unsupervised, guo2021toward}, potentially entailing significant latency (\eg, tens to hundreds of milliseconds) in practice.

% However, they often entail significant delays (\eg, tens to hundreds of milliseconds) for network inference \cite{diehl2015unsupervised, guo2021toward}, negating low-latency benefits of event cameras.

% CNNs struggle to process event data directly due to its asynchronous nature \tocite. 
% Current practices

% These methods use only event data for object depth estimation, which can be categorized into two types.
% One type is machine learning algorithms. 
% Convolutional neural networks (CNNs) cannot directly process event data due to its asynchronous nature \tocite. 
% Current methods either stick events within a short time window (\eg, $<1ms$) into an image for CNN-based depth estimation \tocite or design event-oriented networks (\eg, spiking neural networks) for object localization \tocite. 
% However, these methods often require significant delays (\eg, tens to hundreds of milliseconds) for inference \cite{diehl2015unsupervised, guo2021toward}, negating the low-latency benefits of event cameras\tocite.
% The other type of methods incorporates known geometric information of the target object combined with observational data to infer depth, which heavily rely on prior knowledge, resulting in poor performance in new scenes and with unfamiliar objects.

\noindent $\bullet$ \textbf{Fusion-based solutions.}
These methods enhance event cameras for 3D object localization through sensor fusion, categorized into two types.
$(i)$ Involving dual event cameras \cite{zhou2021event, xu2023taming}. 
These methods often require meticulous calibration and feature matching between event cameras, which are time-consuming and sensitive to environmental noise \cite{falanga2020dynamic}.
$(ii)$ Introducing dedicated depth estimation sensors (\eg, depth cameras \cite{he2021fast}, LiDAR \cite{cui2022dense}) to provide event cameras with depth information \cite{li2022motion}. 
% However, these sensors typically operate at 10$Hz$ $\sim$ 30$Hz$ \tocite, requiring downsampling event cameras to synchronize, which nullify the low-latency benefits of event cameras \tocite.
However, these sensors typically operate with latencies ranging from 30$ms$ to 100$ms$ \cite{li2023leovr}, necessitating the downsampling of event data in the temporal domain for synchronization.
% requiring downsampling event cameras to synchronize, which nullify the low-latency benefits of event cameras.

\noindent \textbf{Remark.}
% In summary, current methods entail lengthy processing times or necessitate downsampling event cameras for synchronization with other sensors, presenting significant challenges in fully harnessing the potential of event cameras for low-latency 3D object localization.
% Inappropriate sensor choice for fusion and the absence of suitable algorithms negate the low-latency advantages of event cameras, posing challenges in fully leveraging their potential for low-latency 3D object localization.
In summary, the absence of efficient algorithms and the sensor with matched frequencies for depth estimation introduces substantial delays in 3D object localization. 
This limitation prevents the complete exploitation of the low-latency benefits of event cameras.

% "In summary, the lack of efficient algorithms and appropriately synchronized sensors for depth estimation causes significant delays in 3D object localization. This hurdle hinders the complete exploitation of the low-latency advantages offered by event cameras."
% event cameras’ potential.

% \noindent $\bullet$ \textbf{Dedicated depth sensors-based solution.}
% By introducing dedicated depth estimation sensors (\eg, depth cameras and LiDAR), these solutions provide event cameras with depth information of objects. 
% Specifically, these sensors emit light in a specific spectrum and calculate depth based on reflection time.
% However, they typically operate at frequencies of 10Hz $\sim$ 30Hz, much lower than the sampling frequency of event cameras, degrading localization performance.

% \noindent $\bullet$ \textbf{Learning-based solution.}
% Machine learning algorithms, such as convolutional neural networks (CNNs), cannot directly process event camera data because it consists of asynchronous events, not fixed-rate frames. 
% Current practices either $(i)$ stick all events within a short time window (\eg, $<1ms$) into an image for CNN-based depth estimation, or $(ii)$ design event-oriented networks (\eg, spiking neural networks) for object localization. 
% They rely heavily on labeled training data, leading to poor performance with new scenes and objects. Also, they introduce significant delays, negating the low-latency benefits of event cameras.

% \noindent $\bullet$ \textbf{Motion-based solution.}
% These methods combine information from inertial measurement units (IMUs) and use visual-inertial odometry to estimate 3D location of the target object. 
% Although they do not rely on dedicated sensors or training data, they require camera movement while the object remains stationary, which severely limits usage scenarios. 
% Additionally, current practices involve using dual event cameras with known pose relationships for 3D object localization. 
% However, these methods often require meticulous calibration and feature matching between cameras, which are highly sensitive to unexpected noise in the environment.

\begin{figure}[t]
    \setlength{\abovecaptionskip}{0.25cm} % height above Figure X caption
    \setlength{\belowcaptionskip}{-0.3cm}
    \setlength{\subfigcapskip}{-0.25cm}
    \centering
        \includegraphics[width=0.98\columnwidth]{Figs/intro_new.png}
        \vspace{-0.2cm}
    \caption{Illustration of events generation and applications of event cameras.}
    \label{intro}
    \vspace{-0.3cm}
\end{figure} 

\begin{figure*}[t]
    \setlength{\abovecaptionskip}{0.2cm} % height above Figure X caption
    \setlength{\belowcaptionskip}{-0.3cm}
    \setlength{\subfigcapskip}{-0.25cm}
    \centering
        \includegraphics[width=2\columnwidth]{evaFigs/relatedall_2.png}
        \vspace{-0.2cm}
    \caption{Benchmark study on drone localization and performance of existing solutions at different settings.}
    \label{relatedwork}
    \vspace{-0.2cm}
\end{figure*} 

\noindent \textbf{Enhance event camera with mmWave radar.}
% MmWave radar, utilizing frequency-modulated continuous waves (FM-CW) with microsecond level latency, measures relative angle and distance of moving objects, generating sparse point cloud \cite{woodford2023metasight, zheng2023neuroradar}. 
% with microsecond level latency
% MmWave radar, utilizing frequency-modulated continuous waves (FM-CW), has been widely employed to measure the relative angle and distance of moving objects, resulting in a sparse point cloud \cite{woodford2023metasight, zheng2023neuroradar}.
% The mmWave radar, utilizing frequency-modulated continuous waves (FM-CW), has been widely employed in detection and tracking of moving objects, resulting in a sparse point cloud \cite{woodford2023metasight, zheng2023neuroradar}.
% Inspired by achievements of mmWave-based sensing techniques, we observe that both event camera and mmWave radar share microsecond time resolution, making mmWave radar a promising modality to enhance the event camera in 3D object localization.
% This presents a significant opportunity for event-based accurate and low-latency localization.
mmWave signals, operating at high frequencies (30 $\sim$ 300 GHz) with wide bandwidth, offer high sensing sensitivity and precision \cite{fiandrino2019scaling, zhang2023survey}.
Endowed with fine-grained, directional sensing capability, and resistance to weather and illumination conditions, mmWave sensing has great advantages in object depth estimation \cite{sie2023batmobility, iizuka2023millisign, lu2020see, lu2020milliego}.
More importantly, both event cameras and mmWave radar feature \textit{millisecond latency} \cite{mmWaveUser}. These factors make mmWave a promising enhancement for event cameras in low-latency 3D object localization.
% Meanwhile, this fusion also holds potential in solving the issues of limited spatial resolution and scatter center drift faced by mmWave radar.

% mmWave signals, operating at high frequencies (30-300 GHz) with wide bandwidth, offer high sensing sensitivity. With fine-grained, directional sensing capability, mmWave sensing excels in object depth estimation. Both event cameras and mmWave radar share millisecond latency, making mmWave a promising enhancement for event cameras in low-latency 3D object localization. Additionally, this fusion can address the issues of limited spatial resolution and scatter center drift faced by mmWave radar.

% and resistance to weather and illumination conditions, 
% More importantly,尽管 mmWave 面临limited angular resolution和scatter center drift问题, both event cameras and mmWave radar share \textit{millisecond latency} \cite{mmWaveUser}, making mmWave a promising enhancement for event cameras in low-latency 3D object localization.
% Despite the issues of limited spatial resolution and scatter center drift faced by mmWave, both event cameras and mmWave radar share \textit{millisecond latency} \cite{mmWaveUser}. 
% This makes mmWave a promising enhancement for event cameras in low-latency 3D object localization, while the event camera also holds potential in solving mmWave radar issues.

% To better understand the potential of fusing the event camera and mmWave radar for low-latency and accurate localization, we conduct a benchmark study on landing drone localization at a real-world drone delivery airport (\fig \ref{relatedwork}a), as accurate and low-latency localization is essential for effective drone landing \cite{sun2023indoor}. 
% This is because landing is a critical phase where drones are most vulnerable \cite{wang2022micnest, xu2023taming, floreano2015science}, posing financial risks and safety threats \cite{Russiandrone}. 
% Higher accuracy improves landing success on designated platforms, while lower latency allows more reaction time to unexpected situations \cite{famili2022pilot, he2023acoustic, chi2022wi}.

To explore the potential of fusing the event camera and mmWave radar for improved 3D object localization, we conduct a benchmark study on drone localization during landing phase at a real-world drone delivery airport (\fig \ref{relatedwork}a). 
Accurate and low-latency localization is crucial for effective landing of the drone \cite{wang2022micnest, sun2023indoor}, as in this phase the drone is most vulnerable, posing financial risks and safety threats \cite{floreano2015science, Russiandrone}. 
Enhanced accuracy ensures successful landing on designated platforms, while reduced latency provides more reaction time for unexpected situations \cite{famili2022pilot, he2023acoustic, chi2022wi}.
Our benchmark study reveals that existing methods face fundamental challenges in 3D object localization, as elaborated below:

% \noindent $\bullet$ \textbf{C1: Millisecond sensing latency amplifies sensing noise, impairing drone detection.}
% \noindent $\bullet$ \textbf{C1: Differing noise distribution characteristics of both modalities hinder drone detection.}
% Unexpected environmental changes introduce irrelevant information as noise in sensing results \cite{xu2023taming}.
% Although both sensors have matched sensing latency, their noise distribution characteristics differ significantly due to their different mechanisms, hindering system's ability to identify signals changes caused by drone in both modalities \cite{zuo2024cross}, especially in millisecond latency (\fig \ref{relatedwork}b).
% However, traditional single modality-oriented noise filtering algorithms \cite{cao2024virteach, liu2024pmtrack, wang2021asynchronous, alzugaray2018asynchronous} achieve a low event and point cloud filtering rate (recall and precision < 65\% in \fig \ref{relatedwork}c) due to their rule-based pipelines struggle to distinguish drone-induced signal changes from scene dynamics.
% This results in detection precision bottlenecks, significantly diminishing the efficiency and accuracy of localization.

% 事件相机容易由于什么产生噪声,雷达容易由于什么产生噪声。对于同一个目标,这两种不同的传感器不仅产生异构的target-trigger的信息,也产生了空间上不同分布(dimentions,patterns)的噪声,而且这些噪声时间上也可能不同步,特别是在高时间分辨率的情况下。不幸的是,传统的方法要不就是针对单模态的滤波,要不就是对两个相似的信号进行滤波,不能应用到我们这个异构的高频场景下。
% Unexpected environmental changes introduce irrelevant information as noise in sensing results \cite{xu2023taming}.
% \noindent $\bullet$ \textbf{C1: Differing noise distribution characteristics of both modalities hinder drone detection.}
% Both sensor modalities yield not only heterogeneous information about the drone but also generate significantly heterogeneous noise. 
% Event cameras produce noise due to unexpected changes in brightness conditions, while mmWave radar struggles with noise caused by signal multipath effects.
% This noise differs greatly in dimensions and patterns, and it may lacks temporal synchronization, particularly under millisecond latency (\fig \ref{relatedwork}b). 
% These factors make noise filtering challenging, causing detection bottlenecks and reducing localization efficiency and accuracy \cite{xu2023taming}.
% Traditional noise filtering algorithms \cite{cao2024virteach, liu2024pmtrack, wang2021asynchronous, alzugaray2018asynchronous} target a specific modality, resulting in low noise event and point cloud filtering rates (recall and precision < 65\% in \fig \ref{relatedwork}c), limiting their effectiveness in our scenario.

% 一句背景,一句现象,一句结果,一句实验数据。
\noindent $\bullet$ \textbf{C1: Noise distribution characteristics of both modalities differ, hindering drone detection.}
% Both sensor modalities yield not only heterogeneous information about the drone but also generate significantly heterogeneous noise. 
These two sensor modalities not only provide different types of information but also generate significantly heterogeneous noise. 
Event cameras produce noise due to unexpected changes in brightness conditions, whereas mmWave radar struggles with noise caused by signal multipath effects.
These noises differ greatly in both dimension and pattern, which can also be asynchronous, especially under high temporal resolution (\fig \ref{relatedwork}b).
This spatial and temporal heterogeneity complicates noise filtering, causing detection bottlenecks \cite{xu2023taming}.
Unfortunately, existing traditional noise filtering algorithms \cite{cao2024virteach, liu2024pmtrack, wang2021asynchronous, alzugaray2018asynchronous} typically target a single modality, resulting in low noise event and point cloud filtering rates (recall and precision < 65\% in \fig \ref{relatedwork}c), limiting their effectiveness in our scenario.


% However, traditional noise filtering algorithms \cite{cao2024virteach, liu2024pmtrack, wang2021asynchronous, alzugaray2018asynchronous} are solely targeted at a specific modality and fail to exploit the consistency among different modalities, achieving a low event and point cloud filtering rate (recall and precision < 65\% in \fig \ref{relatedwork}c), which cannot be utilized in effective noise filtering in our scenario.
% This results in detection precision bottlenecks, significantly diminishing the efficiency and accuracy of localization.
% Although both sensors have matched sensing latency, their noise distribution characteristics differ significantly due to their different mechanisms, hindering system's ability to identify signals changes caused by drone in both modalities \cite{zuo2024cross}, especially in millisecond latency (\fig \ref{relatedwork}b).
% However, traditional single modality-oriented noise filtering algorithms \cite{cao2024virteach, liu2024pmtrack, wang2021asynchronous, alzugaray2018asynchronous} achieve a low event and point cloud filtering rate (recall and precision < 65\% in \fig \ref{relatedwork}c) due to their rule-based pipelines struggle to distinguish drone-induced signal changes from scene dynamics.

% \noindent $\bullet$ \textbf{C2: Ultra-large amount data burden the heterogeneous data fusion, delaying drone localization.}
% Once the drone is detected, accurate 3D spatial location estimation of it is essential, which is more time-consuming than detection due to additional processing (\eg, sensor fusion and optimization).
% The ultra-large amount of data generated by the millisecond latency further burdens the time consumption . 
% Although the localization accuracy is boosted, existing methods \cite{zhao20213d, falanga2020dynamic, mitrokhin2018event} introduces significant delays (\fig \ref{relatedwork}d).
% Moreover, asynchronous event streams and sparse point clouds from mmWave radar are heterogeneous in terms of precision, scale, and density. 
% Previous fusion methods (\eg, Extended kalman filter, particle filter, and graph optimization \cite{grisetti2010tutorial} in \fig \ref{relatedwork}d) suffer from severe cumulative drift error and lengthy processing latency, rendering them inadequate for accurate and low-latency localization.

\noindent $\bullet$ \textbf{C2: Ultra-large data volume burdens the heterogeneous data fusion, delaying drone localization.}
Accurately estimating 3D location of the drone after detection involves time-consuming processing steps, such as sensor fusion and optimization. 
% Once the drone is detected, we proceed to perform 3D localization on it.
% Accurately estimating 3D spatial location of drone involves several time-consuming processing steps, including the sensor fusion and optimization.
The ultra-large amount of data due to the high frequency further exacerbates the processing time, causing significant delays \cite{xu2021followupar}.
Meanwhile, the asynchronous event streams and sparse point clouds are heterogeneous in terms of precision, scale, and density, adding complexity to the sensor fusion.
Existing methods (\eg, Extended Kalman filter, particle filter, and graph optimization) suffer from cumulative drift error, heterogeneity issues, and lengthy processing latency, rendering them inadequate for accurate and low-latency localization as shown in \fig \ref{relatedwork}d \cite{zhao20213d, falanga2020dynamic, mitrokhin2018event, grisetti2010tutorial}.


\noindent \textbf{Our work.}
% We explore the sensing principles of the event camera and mmWave radar and propose EventLoc, an low latency-oriented event camera enhancement system that provides cm-level accurate 3D object localization with millisecond level latency to enable application of event camera in various 3D vision tasks.
We delve into the sensing principles of event cameras and mmWave radar, introducing EventLoc. 
This system enhances event camera functionality with a focus on low-latency 3D object localization, providing cm-level accuracy with millisecond latency on average. 
As a result, EventLoc broadens event camera application in diverse 3D vision tasks.
In detail, EventLoc features three key designs to fully unleash the potential of event camera and mmWave radar for 3D object localization, as elaborated below: \\
% and is implemented with adaptively acceleration algorithms to further improve accuracy and reduce latency, 
\noindent $\bullet$ \textbf{On system architecture front.}
By incorporating mmWave radar with millisecond latency, we enhance the performance of event camera and improve 3D localization performance at the data source.
EventLoc features a carefully designed system architecture that tightly couples event camera and mmWave radar. 
This integration spans from early-stage filtering to later-stage fusion and optimization, fully leveraging the unique advantages of both sensors (§\ref{3.2}). \\
\noindent $\bullet$ \textbf{On system algorithm front.}
We first introduce the Consi-stency-Instructed Collaborative Tracking (\textit{CCT}) algorithm to extract \textit{consistent information} in sensing data from both modalities to filter out environment-triggered noise with a low false positive rate, enhancing the detection performance with a low-latency (§\ref{4.1}). 
We then present the Graph-Informed Adaptive Joint Optimization (\textit{GAJO}) algorithm to fully fuse \textit{complementary information} from both modalities, accelerating the optimization in localizing the object (§\ref{4.2}). \\
\noindent $\bullet$ \textbf{On system implementation front.}
We further analyze the sources of latency and propose an Adaptive Optimization method for boosting the \textit{GAJO}. 
This method dynamically optimizes the set of locations rather than relying on a fixed sliding window, further enhancing the accuracy of localization and reducing latency (§\ref{5.1}).

\begin{figure*}[t]
    \setlength{\abovecaptionskip}{0.4cm} % height above Figure X caption
    \setlength{\belowcaptionskip}{-0.5cm}
    \setlength{\subfigcapskip}{-0.25cm}
    \centering
        \includegraphics[width=2\columnwidth]{Figs/overview2.png}
        \vspace{-0.2cm}
    \caption{System architecture of EventLoc.}
    \label{overview}
    % \vspace{-0.2cm}
\end{figure*} 

\noindent \textbf{Evaluation and Result.} 
We fully implement EventLoc with COTS event camera and mmWave radar.
Extensive experiments in indoor/outdoor environments are conducted with different drone flight conditions to comprehensively evaluate performance of EventLoc.
We compare the end-to-end drone localization accuracy and latency of EventLoc with three SOTA methods.
% Through over 30 hours of real-world experiments, we demonstrate that EventLoc enhances event camera with mmWave radar by achieving a localization accuracy of 1.01 $dm$, surpassing all baselines by >50\%. EventLoc further achieves localization latency of 5.15 $ms$, outperforming baselines by >50\% in average.
Through over 30 hours of experiments, we demonstrate that EventLoc enhances event camera with mmWave radar by achieving an average localization accuracy of 0.101 $m$ and latency of 5.15 $ms$, surpassing all baselines by >50\% on average.
Additionally, EventLoc is marginally affected by factors such as drone type and envir. conditions.\\
\textbf{Real-world deployment.}
We have deployed the sensor platform with EventLoc at a real-world drone delivery airport as shown in \fig \ref{relatedwork}a to demonstrate practicability of the system.
10 hours study shows that EventLoc meets drone landing demands within the constraints of available resources.

\noindent \textbf{Contributions.} This paper makes following contributions.

\noindent $(1)$ We propose EventLoc, a novel low latency-oriented event camera enhancement system. It tightly integrates asynchronous events and mmWave radar sparse point clouds, achieving accurate drone localization with millisecond latency.\\
\noindent $(2)$ We propose the $CCT$, a light-weight cross-modal noise filter to push the limit of detection accuracy by leveraging the \textit{consistent information} from both modalities. \\
\noindent $(3)$  We propose the $GAJO$, a factor graph-based optimization framework that fully harnessing \textit{complementary information} from both modalities to enhance localization performance.\\
% accuracy and latency
\noindent $(4)$ We implement and extensively evaluate EventLoc by comparing it with three SOTA methods, showing its effectiveness. We also deploy EventLoc in a real-world drone delivery airport, demonstrating feasibility of EventLoc.
% The remainder of the paper is organized as follows:
% §2 provides an overview of EventLoc, with detailed descriptions of the Consistency-Instructed Collaborative Tracking algorithm in §3.1 and the Graph-Informed Adaptive Joint Optimization algorithm in §3.2. §4 showcases the adaptive acceleration algorithms and implementation. §5 details our extensive indoor and outdoor experiments with EventLoc. §6 presents our experiments conducted at a real-world delivery airport. §7 discusses related work. §8 concludes the paper.

\iffalse
\begin{table*}[htbp]
\tiny
\begin{center}
\begin{tabular}{lccccccccccccc}\toprule
Model, ft setting & mem & \#param & ARC-c & ARC-e & BoolQ & HS & OBQA & PIQA & rte & SIQA & WG & Avg
%\\\cmidrule(lr){2-3}\cmidrule(lr){4-5} \cmidrule(lr){6-7} \cmidrule(lr){8-9}\cmidrule(lr){10-11} \cmidrule(lr){12-13} \cmidrule(lr){14-15} \cmidrule(lr){16-17} 
\\\cmidrule(lr){1-13}
Llama2(7B), LoRA, $r=64$ & 23.46GB & 159.9M(2.37\%) & \textbf{44.97} & 77.02 & 77.43 & \textbf{57.75} & 32.0 & \textbf{78.45} & 62.09 & \textbf{47.75} & 68.75 & 60.69\\
Llama2(7B), SPruFT, $r=128$ & \textbf{17.62GB} & 145.8M(2.16\%) & 43.60 & \textbf{77.26} & \textbf{77.77} & 57.47 & \textbf{32.6} & 78.07 & \textbf{64.98} & 46.67 & \textbf{69.30} & \textbf{60.86} \\\cmidrule(lr){2-13}
Llama2(7B), FA-LoRA, $r=64$ & 17.25GB & 92.8M(1.38\%) & 43.77 & \textbf{77.57} & 77.74 & \textbf{57.45} & 31.0 & 77.86 & \textbf{66.06} & \textbf{47.13} & 69.06 & 60.85\\
Llama2(7B), FA-SPruFT, $r=128$ & \textbf{15.21GB} & 78.6M(1.17\%) & \textbf{43.94} & 77.22 & \textbf{77.83} & 57.11 & \textbf{32.0} & \textbf{78.18} & 65.70 & 46.47 & \textbf{69.38} & \textbf{60.87}\\\midrule
Llama3(8B), LoRA, $r=64$ & 30.37GB & 167.8M(2.09\%) & \textbf{53.07} & \textbf{81.40} & \textbf{82.32} & \textbf{60.67} & 34.2 & \textbf{79.98} & 69.68 & \textbf{48.52} & \textbf{73.56} & \textbf{64.82}\\
Llama3(8B), SPruFT, $r=128$ & \textbf{24.49GB} & 159.4M(1.98\%) & 52.47 & 81.10 & 81.28 & 60.29 & \textbf{34.6} & 79.76 & \textbf{70.04} & 47.75 & 73.24 & 64.50 \\\cmidrule(lr){2-13}
Llama3(8B), FA-LoRA, $r=64$ & 24.55GB & 113.2M(1.41\%) & \textbf{52.47} & \textbf{81.36} & \textbf{82.23} & 60.17 & \textbf{35.0} & \textbf{79.76} & \textbf{70.04} & \textbf{48.31} & \textbf{73.56} & \textbf{64.77}\\
Llama3(8B), FA-SPruFT, $r=128$ & \textbf{22.41GB} & 92.3M(1.15\%) & 52.22 & 81.19 & 81.35 & \textbf{60.20} & 34.2 & 79.71 & 69.31 & 47.13 & 73.01 & 64.26 \\\bottomrule
\end{tabular}
%\vspace{-0.2cm}
\caption{Fine-tuning Llama on Alpaca dataset for 5 epochs and evaluating on 9 tasks from EleutherAI LM Harness. "mem" represents the memory usage, with further details provided in Appendix~\ref{apdx:measure}. \#param is the number of trainable parameters, where the difference of \#param between the two approaches depends on the architecture of Llama, as some layers have $d_{in} \neq d_{out}$. Note that 10 million trainable parameters only account for less than 0.15GB of memory requirement. FA indicates that we freeze attention layers, but not including MLP layers followed by attention blocks. HS, OBQA, and WG represent HellaSwag, OpenBookQA, and WinoGrande datasets. More details of datasets can be found in Appendix~\ref{apdx:data}. The ablation study for different $r$ and the comparison with other LoRA variants can be found in Appendix~\ref{apdx:ablation}. All reported results are accuracies on the corresponding tasks. \textbf{Bold} indicates the best results of two approaches on the same task.} \label{tab:llm} 
\end{center}
\end{table*}
\fi

\begin{table*}[htbp]
\tiny
\begin{center}
\begin{tabular}{lccccccccccccc}\toprule
Model, ft setting & mem & \#param & ARC-c & ARC-e & BoolQ & HS & OBQA & PIQA & rte & SIQA & WG & Avg
\\\cmidrule(lr){1-13}
Llama2(7B)\\ \cmidrule(lr){1-1} 
LoRA, $r=64$ & 23.46GB & 159.9M(2.37\%) & \textbf{44.97} & 77.02 & 77.43 & 57.75 & 32.0 & \textbf{78.45} & 62.09 & 47.75 & 68.75 & 60.69\\
VeRA, $r=64$ & 22.97GB & 1.374M(0.02\%) & 43.26 & 76.43 & 77.40 & 57.26 & 31.6 & 78.02 & 62.09 & 45.85 & 68.75 & 60.07\\
DoRA, $r=64$ & 44.85GB & 161.3M(2.39\%) & 44.71 & 77.02 & 77.55 & \textbf{57.79} & 32.4 & 78.29 & 61.73 & \textbf{47.90} & 68.98 & 60.71\\
RoSA, $r=32, d=1.2\%$ & 44.69GB & 157.7M(2.34\%) & 43.86 & \textbf{77.48} & \textbf{77.86} & 57.42 & 32.2 & 77.97 & 63.90 &  47.29 & 69.06 & 60.78\\
SPruFT, $r=128$ & \textbf{17.62GB} & 145.8M(2.16\%) & 43.60 & 77.26 & 77.77 & 57.47 & \textbf{32.6} & 78.07 & \textbf{64.98} & 46.67 & \textbf{69.30} & \textbf{60.86} %\\\cmidrule(lr){2-13}
%FA-LoRA, $r=64$ & 17.25GB & 92.8M(1.38\%) & 43.77 & \textbf{77.57} & 77.74 & \textbf{57.45} & 31.0 & 77.86 & 66.06 & \textbf{47.13} & 69.06 & 60.85\\
%FA-DoRA, $r=64$ & 30.61GB & 93.6M(1.39\%) & 43.94 & 77.44 & 77.49 & 57.44 & 31.0 & 77.86 & \textbf{66.43} & 46.98 & 69.14 & 60.86\\
%FA-RoSA, $r=32, d=1.2\%$ & 38.34GB & 98.3M(1.46\%) & \textbf{44.28} & 77.02 & 77.68 & 57.22 & 31.0 & 77.97 & 64.26 & 46.32 & 69.22 & 60.55\\
%FA-SPruFT, $r=128$ & \textbf{15.21GB} & 78.6M(1.17\%) & 43.94 & 77.22 & \textbf{77.83} & 57.11 & \textbf{32.0} & \textbf{78.18} & 65.70 & 46.47 & \textbf{69.38} & \textbf{60.87}
\\\midrule
Llama3(8B)\\ \cmidrule(lr){1-1} 
LoRA, $r=64$ & 30.37GB & 167.8M(2.09\%) & 53.07 & 81.40 & 82.32 & 60.67 & 34.2 & 79.98 & 69.68 & 48.52 & 73.56 & 64.82\\
VeRA, $r=64$ & 29.49GB & 1.391M(0.02\%) & 50.26 & 80.30 & 81.41 & 60.16 & 34.4 & 79.60 & 69.31 & 46.93 & 72.77 & 63.90\\
DoRA, $r=64$ & 51.45GB & 169.1M(2.11\%) & \textbf{53.33} & \textbf{81.57} & \textbf{82.45} & \textbf{60.71} & 34.2 & \textbf{80.09} & 69.31 & \textbf{48.67} & \textbf{73.64} & \textbf{64.88}\\
RoSA, $r=32, d=1.2\%$ & 48.40GB & 167.6M(2.09\%) & 51.28 & 81.27 & 81.80 & 60.18 & 34.4 & 79.87 & 69.31 & 47.95 & 73.16 & 64.36\\
SPruFT, $r=128$ & \textbf{24.49GB} & 159.4M(1.98\%) & 52.47 & 81.10 & 81.28 & 60.29 & \textbf{34.6} & 79.76 & \textbf{70.04} & 47.75 & 73.24 & 64.50 %\\\cmidrule(lr){2-13}
%FA-LoRA, $r=64$ & 24.55GB & 113.2M(1.41\%) & 52.47 & 81.36 & 82.23 & 60.17 & \textbf{35.0} & 79.76 & 70.04 & 48.31 & \textbf{73.56} & 64.77\\
%FA-DoRA, $r=64$ & 40.62GB & 114.3M(1.42\%) & \textbf{52.56} & \textbf{81.69} & \textbf{82.26} & \textbf{60.20} & 34.4 & \textbf{79.82} & \textbf{70.40} & \textbf{48.46} & 73.40 & \textbf{64.80}\\
%FA-RoSA, $r=32, d=1.2\%$ & 42.31GB & 124.3M(1.55\%) & 52.22 & 81.19 & 82.05 & 60.11 & 34.4 & 79.76 & 69.31 & 47.70 & 73.16 & 64.43\\
%FA-SPruFT, $r=128$ & \textbf{22.41GB} & 92.3M(1.15\%) & 52.22 & 81.19 & 81.35 & \textbf{60.20} & 34.2 & 79.71 & 69.31 & 47.13 & 73.01 & 64.26 
\\\bottomrule
\end{tabular}
%\vspace{-0.2cm}
\caption{Fine-tuning Llama on Alpaca dataset for 5 epochs and evaluating on 9 tasks from EleutherAI LM Harness. ``mem" represents the memory usage, with further details provided in Appendix~\ref{apdx:measure}. \#param is the number of trainable parameters, where the difference of \#param between the two approaches depends on the architecture of Llama, as some layers have $d_{in} \neq d_{out}$. %FA indicates that we freeze attention layers, but not including MLP layers followed by attention blocks. 
HS, OBQA, and WG represent HellaSwag, OpenBookQA, and WinoGrande datasets. %More details of datasets can be found in Appendix~\ref{apdx:data}. 
The ablation study for different $r$ can be found in Appendix~\ref{apdx:ranks}. All reported results are accuracies on the corresponding tasks. \textbf{Bold} indicates the best result on the same task. } \label{tab:llm} 
\end{center}
\end{table*}

\section{Experimental Setup}\label{sec:setup}

%(0.5 page)
%Why the chosen framework?
%Some prior approaches

%- parameter settings
%- uniform across layers vs greedy ... 
%- potential transformer-specific details

%Equations about what these methods do.. 

%(0.5 page)
%Which NN architectures are used, why?
%Number of parameters, layers, ...

%(Potential prior work on compression -- )

\subsection{Datasets} \label{subsec:dataset}
We use multiple datasets for different tasks. For image classification, we fine-tune models on the training split and evaluate it on the validation split of Tiny-ImageNet~\citep{tavanaei2020embedded}, CIFAR100~\citep{alex2009learning}, and Caltech101~\citep{li_andreeto_ranzato_perona_2022}. For text generation, we fine-tune LLMs on 256 samples from Stanford-Alpaca~\citep{alpaca} and assess zero-shot performance on nine EleutherAI LM Harness tasks~\citep{gao2021framework}. See Appendix~\ref{apdx:data} for details.

\subsection{Models and Baselines} \label{subsec:models}

We fine-tune full-precision Llama-2-7B and Llama-3-8B (float32) using our SPruFT, LoRA~\citep{hulora}, VeRA~\citep{kopiczko2024vera}, DoRA~\citep{liu2024dora}, and RoSA~\citep{nikdan2024rosa}. RoSA is chosen as the representative SFT method and is the only SFT due to the high memory demands of other SFT approaches, while full fine-tuning is excluded for the same reason. We freeze Llama’s classification layers and fine-tune only the linear layers in attention and MLP blocks.

Next, we evaluate importance metrics by fine-tuning Llamas and image models, including DeiT~\citep{touvron2021training}, ViT~\citep{dosovitskiy2020image}, ResNet101~\citep{he2016deep}, and ResNeXt101~\citep{xie2017aggregated} on CIFAR100, Caltech101, and Tiny-ImageNet. For image tasks, we set the fine-tuning ratio at 5\%, meaning the trainable parameters are a total of 5\% of the backbone plus classification layers.

\subsection{Training Details} \label{subsec:training}
Our fine-tuning framework is built on torch-pruning\footnote{Torch-pruning is not required, all their implementations are based on PyTorch.}~\citep{fang2023depgraph}, PyTorch~\citep{paszke2019pytorch}, PyTorch-Image-Models~\citep{rw2019timm}, and HuggingFace Transformers~\citep{wolf2020transformers}. Most experiments run on a single A100-80GB GPU, while DoRA and RoSA use an H100-96GB GPU. We use the Adam optimizer~\citep{KingBa15} and fine-tune all models for a fixed number of epochs without validation-based model selection.

%Structured pruning often considers parameter dependencies in importance evaluation~\citep{liu2021group, fang2023depgraph, ma2023llmpruner}. This becomes the following process in our work: first, searching for dependencies by tracing the computation graph of gradient; next, evaluating the importance of parameter groups; and finally, fine-tuning the parameters within those important groups collectively. For instance, if $\W^{a}_{\cdot j}$ and $\W^{b}_{i\cdot}$ are dependent, where $\W^{a}_{\cdot j}$ is the $j$-th column in parameter matrix (or the $j$-th input channels/features) of layer $a$ and $\W^{b}_{i\cdot}$ is the $i$-th row in parameter matrix (or the $i$-th output channels/features) of layer $b$, then $\W^{a}_{\cdot j}$ and $\W^{b}_{i\cdot}$ will be fine-tuned simultaneously while the corresponding $\M^{a}_{dep}$ for $\W^{a}_{\cdot j}$ becomes column selection matrix and $\W^a_s$ becomes $\W^a_{f,dep}\M^a_{dep}$. Consequently, fine-tuning $2.5\%$ output channels for layer $b$ will result in fine-tuning additional $2.5\%$ input channels in each dependent layer. Therefore, for the $5\%$ of desired fine-tuning ratio, the fine-tuning ratio with considering dependencies is set to $2.5\%$\footnote{In some complex models, considering dependencies results in slightly more than twice the number of trainable parameters. However, in most cases, the factor is 2.} for the approach that includes dependencies. More details for dependencies of NN can be found in Appendix~\ref{apdx:dep}. 

\textbf{Image models}: The learning rate is set to $10^{-4}$ with cosine annealing decay~\citep{loshchilov2017sgdr}, where the minimum learning rate is $10^{-9}$. All image models used in this study are pre-trained on ImageNet. 

\textbf{Llama}: For LoRA and DoRA, we set $\alpha = 16$, a dropout rate of $0.1$, and a learning rate of $10^{-4}$  with linear decay (
$0.01$ decay rate). For SPruFT, we control trainable parameters using rank instead of fine-tuning ratio for direct comparison. The learning rate is $2 \cdot 10^{-5}$ with the same decay settings. Linear decay is applied after a warmup over the first $3$\% of training steps. The maximum sequence length is $2048$, with truncation for longer inputs and padding for shorter ones.





% \section{Introduction}
\label{sec:intro}

\begin{figure*}[tb]
    \centering
    \includegraphics[width=0.848\linewidth]{figs/circuitnn.pdf} 
    \caption{Illustration of differentiable CircuitNN. CircuitNN is designed based on differentiable NAND gates. After DAS is guided by PI and PO pairs of the truth table, CircuitNN can get the precise circuit architecture logic equivalent to the truth table.}
    \label{fig:circuitnn}
\end{figure*}

% 1. Describe the importance of logic synthesis
% 2. Existing Problems
% (a) Neural Architecture Search: Unstable, Predefined Setting, etc.
% (b) Circuit Generation: Probabilistic Model, Logic Equivalence

With the rapid advancement of technology, the scale of integrated circuits (ICs) has expanded exponentially. 
This expansion has introduced significant challenges in chip manufacturing, particularly concerning power and area metrics.
A primary objective in IC design is achieving the same circuit function with fewer transistors, thereby reducing power usage and area occupancy.

Logic synthesis~\cite{hachtel2005logicsynth}, a critical step in electronic design automation (EDA), transforms behavioral-level circuit designs into optimized gate-level circuits, ultimately yielding the final IC layout. 
The primary goal of logic synthesis is to identify the physical implementation with the fewest gates for a given circuit function. 
This task constitutes a challenging NP-hard combinatorial optimization problem. 
Current logic synthesis tools~\cite{brayton2010abc, wolf2013yosys} rely on human-designed heuristics, often leading to sub-optimal outcomes.

Differentiable architecture search (DAS) techniques~\cite{liu2018darts, chu2020darts} offer novel perspectives on addressing challenges in this problem.
Circuit functions can be represented through truth tables, which map binary inputs to their corresponding outputs. 
Truth tables provide a precise representation of input-output relationships, ensuring the design of functionally equivalent circuits.
Inspired by this, researchers~\cite{deepmind2024ai4sys, wang2024tnet} have begun exploring the application of DAS to synthesize circuits directly from truth tables.
Specifically, \citet{deepmind2024ai4sys} proposed CircuitNN, a framework that learns differentiable connection structures with logic gates, enabling the automatic generation of logic circuits from truth tables.
This approach significantly reduces the complexity of traditional circuit generation. 
Building on this, \citet{wang2024tnet} introduced T-Net, a triangle-shaped variant of CircuitNN, incorporating regularization techniques to enhance the efficiency of DAS.

Despite these advancements, several challenges remain. 
The computational complexity of DAS grows quadratically with the number of gates, posing scalability issues.
Although triangle-shaped architecture~\cite{wang2024tnet} partially mitigates this problem, redundancy persists. 
%Additionally, DAS is susceptible to converging to local optima, limiting the ability to search architectures that satisfy the given truth tables~\cite{liu2018darts}. 
%Furthermore, hyperparameters (network depth and layer width) require extensive searches, introducing complexity and prolonging the synthesis process. 
Additionally, DAS is susceptible to converging to local optima~\cite{liu2018darts} and hyperparameters (network depth and layer width) require extensive searches. 
The challenges arise from the vast search space in DAS. 
% Even with predefined settings for CircuitNN, finding a configuration that meets the truth table requires extensive trial and error during the DAS process. 
Intuitively, limiting the search space through predefined parameters (network depth, gates per layer, and connection probabilities) can significantly reduce the complexity.

Recent advances~\cite{openai2023gpt4, abramson2024alphafold3, esser2024sd3, li2024mar} in conditional generative models have demonstrated remarkable performance across language, vision, and graph generation tasks. 
Motivated by these developments, we propose a novel approach to circuit generation that generates preliminary circuit structures to guide DAS in generating refined circuits matching specified truth tables. 
Firstly, we introduce CircuitVQ, a tokenizer with a discrete codebook for circuit tokenization. 
Built upon our Circuit AutoEncoder framework~\cite{hou2022graphmae,li2023maskgae,wu2025mgvga}, CircuitVQ is trained through a circuit reconstruction task. 
Specifically, the CircuitVQ encoder encodes input circuits into discrete tokens using a learnable codebook, while the decoder reconstructs the circuit adjacency matrix based on these tokens.
Subsequently, the CircuitVQ encoder serves as a circuit tokenizer for CircuitAR pretraining, which employs a masked autoregressive modeling paradigm~\cite{chang2022maskgit, li2023mage}. 
In this process, the discrete codes function as supervision signals. 
After training, CircuitAR can generate discrete tokens progressively, which can be decoded into initial circuit structures by the decoder of the CircuitVQ. 
These prior insights can guide DAS in producing refined circuits that match the target truth tables precisely.

Our key contributions can be summarized as follows:
\begin{itemize}
\item We introduce CircuitVQ, a circuit tokenizer that facilitates graph autoregressive modeling for circuit generation, based on our Circuit AutoEncoder framework;
\item Develop CircuitAR, a model trained using masked autoregressive modeling, which generates initial circuit structures conditioned on given truth tables;
\item Propose a refinement framework that integrates differentiable architecture search to produce functionally equivalent circuits guided by target truth tables;
\item Comprehensive experiments demonstrating the scalability and capability emergence of our CircuitAR and the superior performance of the proposed circuit generation approach.
\end{itemize}

% Motivation
% (a) Diffusion (Vision, Graph), Autoregressive (Language, Vision)
% (b) Circuit Generation for Predefined Setting
% (c) Neural Architecture Search for Strict Logic Equivalence

% Contribution
% (a) Circuit Tokenizer (new transformer arch, training strategy)
% (b) CircuitAR (train and gen strategies, post-ar strategy)
% (c) Extensive Evaluation including BitD (Bit Distance) for Scalability

\section{Statistical efficiency of CARROT} 
\label{sec:lower-bound}

In this section we establish that, under certain conditions, the plug-in approach to routing is minimax optimal. To show this, we follow two steps:
\begin{itemize}
    \item First we establish an information theoretic lower bound on the sample complexity for learning the oracle routers (\cf\ Theorem \ref{thm:lower-bound}). 
    \item Next, establish an upper bound for the minimax risk of plug-in routers (\cf\ Theorem \ref{thm:upper-bound}). We show that under sufficient conditions on the estimates of $\Ex[Y\mid X]$ the sample complexity in the upper bound matches the lower bound. Together, they imply the statistical efficiency of the plug-in approach.  
    % We also suggest an estimate for $\Ex[Y\mid X]$ that meets the needed conditions for CARROT to be rate optimal.  
\end{itemize} 


%For our minimax analysis we begin with some notation. For the probability distribution $P $ defined on the space $\cX \times \reals^{M \times K}$, we denote the marginal distribution of $X$ by $P_X$. Let us denote $\supp(\cdot)$ as the support of a probability distribution. Within the space $\reals^d$, we denote $\Lambda_d$ as the Lebesgue measure, $\|\cdot\|_2$ and $\|\cdot\|_\infty$ as the $\ell_2$ and $\ell_\infty$-norms, and $\cB(x, r, \ell_2)$ and $\cB(x, r, \ell_\infty)$ as closed balls of radius $r$ and centered at $x$ with respect to the $\ell_2$ and $\ell_\infty$-norms. 

% We denote $\Ex_P[\ell \{ Y, f_m(X)\}\mid X]$ as $\Phi^\star_m(X)$. Then, following eq. \eqref{eq:reg-decomposition} the regression function $\eta_{\lambda, m}^\star(X) = \Ex_P[\eta_\lambda(X, Y)]$ has the decomposition $\eta_{\lambda, m}^\star(X) = \lambda \Phi^\star_m(X) + (1 - \lambda) \kappa_m(X)$. In the following lemma, we provide a formulation of oracle routers using this decomposition, which will be useful for developing their computationally efficient estimates. 
% \begin{lemma} \label{lemma:oracle-router}
%     For any $0 \le \lambda \le 1$  the oracle router $g_\lambda^\star$ that minimizes the loss $\cL_P(g, \lambda)$ is 
%     \begin{equation} \label{eq:oracle-router-2}
%         \textstyle g_\lambda^\star(X) = \argmin_m ~ \eta_{\lambda, m} ^\star(X) = \argmin_m ~ \{ \lambda \Phi^\star_m(X) + (1 - \lambda) \kappa_m(X)\}\,.
%     \end{equation}
% \end{lemma}

We begin with a notational convention for $g_\mu^\star(X)$. If the minimum is attained at multiple $m$'s, we consider $g_\mu^\star(X)$ as a subset of $[M]$. On the contrary, if the minimum is uniquely attained, then $g_\mu^\star(X)$ refers to both the index $m_X$ where the minimum is attained and the singleton set $\{m_X\} \subset [M]$. The distinction should be clear from the context.

We also generalize slightly to the setting where the last $K_2$ metrics are known functions of $X$, \ie\ for $m \in [M], k \in \{K - K_2 +1 , \dots K\}$ there exist known functions $f_{m, k}: \cX \to \reals$ such that $[Y]_{m, k} = f_{m, k}(X)$. Since $\Ex[[Y]_{m, k}\mid X] = f_{m, k}(X)$ are known for $k \ge K - K_2 +1 $ they don't need to be estimated. 
% We shall see the presence of known metrics has consequences for the sample complexity (\cf\ Remark \ref{remark:difficulty-routing}). We also define $K_1 = K - K_2$ as the number of known metrics.    

\subsection{Technical Assumptions}

%Let us discuss a notational convention for $g_\lambda^\star(X)$. The minimum can be attained at multiple $m$'s. In that case, $g_\lambda^\star(X) \subset [M]$. However, when the minimum is uniquely attained, the $g_\lambda^\star(X)$ refers to both the index $m_X$ where the minimum is attained and the singleton set $\{m_X\} \subset [M]$. The distinctions should be clear from the contexts.


%We also assume that the last $K_2$ many metrics are known functions of $X$, \ie\ for $m \in [M], k \in \{K - K_2 +1 , \dots K\}$ there exist known functions $f_{m, k}: \cX \to \reals$ such that $[Y]_{m, k} = f_{m, k}(X)$. Since $\Ex[[Y]_{m, k}\mid X] = f_{m, k}(X)$ are known for $k \ge K - K_2 +1 $ within the Algorithm \ref{alg:pareto-routers}, they don't need to be estimated. We shall see its consequence in the study sample complexity. Define $K_1 = K - K_2$.    



The technical assumptions of our minimax study are closely related to those in investigations of non-parametric binary classification problems with $0/1$ loss functions, \eg\  \citet{cai2019Transfer,kpotufe2018Marginal,maity2022minimax,audibert2007Fast}. In fact, our setting generalizes the classification settings considered in these papers on multiple fronts: (i) we allow for general loss functions, (ii) we allow for more than two classes, and (iii) we allow for multiple objectives. %So, before we describe the assumptions, 

To clarify this, we discuss how binary classification is a special case of our routing problem. %This connection will be later used for adapting the standard assumptions considered in these papers to our setting. 

\begin{example}[Binary classification with $0/1$-loss] \label{example:binary-classification}
    Consider a binary classification setting with $0/1$-loss: we have the pairs $(X, Z) \in \cX \times \{0, 1\}$ and we want to learn a classifier $h: \cX \to\{0, 1\} $ to predict $Z$ using $X$. This is a special case of our setting with $M = 2$ and $K= 1$, where for $m \in \{0, 1\}$ the $[Y]_{m, 1} = \bbI\{Z \neq m\}$. Then the risk for the classifier $h$, which can also be thought of as a router, is 
\begin{align*}
\textstyle \cR_P(h) & \textstyle = \Ex\big[\sum_{m \in \{0, 1\}}[Y]_{m, 1} \bbI\{h(X) = m\} \big]\\ 
& = \Ex\big[ \bbI\{h(X) \neq Z\} \big]\,,
\end{align*} the standard misclassification risk for binary classification. 
\end{example}

% \SM{Mention that for $\lambda = 0$ the oracle router is precisely known. Thus, we only focus on the cases of $\lambda > 0$.}

We assume that $\supp(P_X)$ is a compact set in $\reals^d$. This is a standard assumption in minimax investigations for non-parametric classification problems \citep{audibert2007Fast,cai2019Transfer,kpotufe2018Marginal,maity2022minimax}. 
Next,  we place H\"older smoothness conditions on the functions $\Phi_m^\star$. This controls the difficulty of their estimation. For a tuple $s = (s_1 , \dots, s_d) \in (\bN \cup \{0\})^d$ of $d$ non-negative integers  define $|s| = \sum_{j = 1}^d s_j$ and for a function $\phi: \reals^d\to \reals$ and $x = (x_1, \dots, x_d) \in \reals^d$ define the differential operator: 
\begin{equation}
  \textstyle  D_s(\phi, x) = \frac{\partial^{|s|}\phi(x)}{\partial x_1^{s_1} \dots \partial x_d^{s_d}}\,, 
\end{equation} assuming that such a derivative exists. Using this differential operator we now define the H\"older smoothness condition: 

\begin{definition}[H\"older smoothness]
   For $\beta, K_\beta >0$ we say that $\phi:\reals^d \to \reals$ is $(\beta, K_\beta)$-H\"older smooth on a set $ A \subset \reals^d$ if it is $\lfloor \beta \rfloor$-times continuously differentiable on $A$ and for any $x, y \in A $ 
   \begin{equation}
       |\phi(y) - \phi_x ^{(\lfloor \beta \rfloor)}(y)| \le K_\beta \|x - y\|_2^\beta\,,
   \end{equation} where 
$\phi_x ^{(\lfloor \beta \rfloor)}(y) = \sum_{|s| \le \lfloor \beta \rfloor} D_s(\phi, x) \{\prod_{j = 1}^d(y_j - x_j)^{s_j}\} $ is the $\lfloor \beta \rfloor$-order Taylor polynomial approximation of $\phi(y)$ around $x$. 
\end{definition}
With this definition, we assume the following:
\begin{assumption}\label{assmp:smooth}
    For $m \in [M]$ and $k \in [K_1]$ the 
    % functions $\kappa_m$ and 
    $[\Phi(X)]_{m, k}$ is $(\gamma_{k}, K_{\gamma, k})$-H\"older smooth. 
\end{assumption} 
%The H\"older smoothness assumption controls how well a non-parametric function can be estimated \citep{fan1997local}, where higher smoothness parameters lead to a smaller error in estimation. 
This smoothness parameter will appear in the sample complexity of our  plug-in router. Since the $[\Phi(X)]_{m, k}$ are known for $k \ge K_1 + 1$ we do not require any smoothness assumptions on them.

% Note that the assumption implies for any $\lambda\in [0, 1]$ and $m \in [M]$ the $\eta_{\lambda, m}^\star = \lambda \Phi_m^\star + (1 - \lambda) \kappa_m$ are also $(\beta, K_\beta)$-H\"older smooth.
% The smoothness on the $\Phi_m^\star$ functions controls their complexity: the higher smoothness implies a lower complexity and is easier to estimate. \SM{Talk about how different complexity would lead to the rate at lowest smoothness. Argue it through the estimation of the differences. And also talk about how smoothness in $\kappa_m$ is not necessary.}
% Before we move on, we want to make a few remarks about this smoothness assumption. As discussed in Section \ref{sec:setup} the core idea behind our approach is to plug-in an estimate of $[\Phi(X)]_{m, k}$ into the oracle router \eqref{eq:oracle-router-2}
%  \[
%     \textstyle g_\mu^\star(X)  = \argmin_m \big\{ \sum_{k = 1}^K \mu_k [\Phi (X)]_{m, k} \big\} \,.
%     \]
% A similar idea can also be found  in the context of binary classification in non-parametric settings: for $(X, Y) \in \cX \times \{0, 1\}$ drawn from the distribution $P$, they plug-in an estimator of $\eta(X) = P(Y = 1\mid X )$ into the Bayes classifier $f^\star(X) = \bbI\{\eta(X) \ge \nicefrac{1}{2}\}$ to obtain a minimax rate optimal classifier, which they call ``plug-in classifier''. In their context, the smoothness in $\eta$ controls how well it can be estimated from a dataset, which later affects the misclassification error for this plug-in classifier. Drawing a parallel to our context,
% \begin{enumerate}
%     \item Within the regression function $\eta_{\lambda, m}^\star(X) = \lambda \Phi^\star _m(X) + (1 - \lambda) \kappa_m(X)$ the $\kappa_m$ are already known and need not be estimated. Thus, we do not require any smoothness assumption for $\kappa_m$ and only require a smoothness condition for the unknown $\Phi_m^\star$.  
%     \item To make the setting more general, we could assume different smoothness parameters for different $\Phi_m^\star$; \eg\ $\Phi_m^\star$ is $\beta_m$ H\"older smooth, in which case it can be estimated at a minimax optimal $\ell_1$-error rate $\cO_P(n^{-{\beta_m}/{(2\beta_m + d)}})$ \citep{fan1997local}. But then the differences \[
%     \textstyle \eta_{\lambda, m_1}^\star(X) - \eta_{\lambda, m_2}^\star(X) = \lambda \big\{ \Phi^\star _{m_1}(X) -\Phi^\star _{m_2}(X)\big\}  + (1 - \lambda) \big \{ \kappa_{m_1}(X) - \kappa_{m_2}(X)\big \}\,, 
%     \] which are crucial for the prediction of oracle routers, will be estimated at a rate 
%     \[
%     \textstyle \cO_P\big(n^{-\frac{\beta_{m_1}}{2\beta_{m_1} + d}} \vee n^{-\frac{\beta_{m_2}}{2\beta_{m_2} + d}}\big) = \cO_P\big(n^{-\frac{\beta_{m_1} \wedge \beta_{m_2}}{2(\beta_{m_1} \wedge \beta_{m_2}) + d}} \big)\,,
%     \]
%     and in the worst case, at a rate $\cO_P(n^{-{\beta_{\min} }/{(2\beta_{\min}  + d})} )$ with respect to the smallest smoothness parameter $\beta_{\min} = \min_m \beta_m$. Since accurately estimating all of these pairwise differences is important to obtain a prediction similar to that of the oracle routers, the final rate of convergence rate for excess risk will be determined by the smallest smoothness parameter, in which case, the other smoothness parameters become irrelevant. Therefore, for a simpler exposition of the problem setting, we assume that the smoothness parameters of $\Phi_m^\star$ are all identical. 
% \end{enumerate}




Next, we introduce \emph{margin condition}, which quantifies the difficulty in learning the oracle router.  For a given $\mu$ define the margin as the difference between the minimum and second minimum of the risk values: 
{ \begin{equation}\label{eq:margin}
    \begin{aligned}
        & \textstyle \Delta_\mu(x) =  
    \begin{cases}
       \min\limits_{m \notin g_\mu(x)} \eta_{\mu, m}(x) - \min\limits_m \eta_{\mu, m}(x) & \text{if} ~ g_\mu^\star(x) \neq [M]\\ 
       0 & \text{otherwise}.
       \end{cases} 
    \end{aligned}
\end{equation}}

% At an $x$, the margin is simply the gap between the second-lowest and lowest coordinate values of $\eta_{\lambda, m}$. If all the coordinates are the same, then we set the margin at zero. 
\noindent Our definition of a margin generalizes the usual definition of the margin considered for binary classification with $0/1$ loss in \citet{audibert2007Fast}. Recall the binary classification example in \ref{example:binary-classification}, in which case, 
$[\Phi(X)]_{m , 1} =  P(Z \neq m\mid X) $. Since $K = 1$ we have 
$\eta_{\mu, m}(X) = P(Z \neq m\mid X) $, which further implies $\eta_{\mu, 0}(X) + \eta_{\mu, 1}(X) = 1$.
Thus for binary classification with $0/1$ loss, our definition of margin simplifies to 
\begin{align*}
\textstyle \min\limits_{m \notin g_\mu^\star(x)} \eta_{\mu, m}(x) - \min\limits_m \eta_{\mu, m}(x)
=  |\eta_{\mu, 1}(X) - \eta_{\mu, 0}(X)| = 2 |\eta_{\mu, 0}(X) - \nicefrac{1}{2}| \,,
\end{align*}
which is a constant times the margin $  |P(Y = 1\mid X) - \nicefrac{1}{2}| = |\eta_{\mu, 0}(X) - \nicefrac{1}{2}| $ in \citet{audibert2007Fast}. 


% Relating their framework to ours, for them $M = 2$ with the class indices $\{0, 1\}$. Moreover, for $m \in \{0, 1\}$ the loss regression function for classifying a sample $X$ as class $m$ is  $\eta_{\lambda, m}^\star(X) = P(Y \neq m\mid X) $, which satisfies $\eta^\star_{\lambda, 0}(X) + \eta^\star_{\lambda, 1}(X) = 1$. In this case, our definition of margin simplifies to $|\eta_{\lambda, 1}(X) - \eta_{\lambda, 0}(X)| = 2 |\eta_{\lambda, 1}(X) - \nicefrac{1}{2}|$, which is a constant multiplication of their definition of margin $  |P(Y = 1\mid X) - \nicefrac{1}{2}| = |\eta_{\lambda, 1}(X) - \nicefrac{1}{2}| $. 



Clearly, the margin determines the difficulty in learning the oracle router. A query $X$ with a small margin gap is difficult to route, because to have the same prediction as the oracle, \ie\  $\argmin_{m} \hat \eta_{\mu, m}(X) = \argmin_{m} \eta_{\mu, m}^\star(X)$ we need to estimate $ \eta_{\mu, m}^\star(X)$ with high precision. In the following assumption, we control the probability of drawing these ``difficult to route'' queries.

\begin{assumption}[Margin condition]\label{assmp:margin}
    For $\alpha, K_\alpha >0$ and any $t > 0$ the margin $\Delta_{\mu}$ \eqref{eq:margin} satisfies: \begin{equation}
        P_X \big\{0 < \Delta _\mu(X) \le t\big \}  \le K_\alpha t^{\alpha}\,. 
    \end{equation}
\end{assumption}
%From Proposition 3.4 of \cite{audibert2007Fast}, if $\alpha  (1 \wedge \gamma_{k})  \ge d$ for some $k$ then if $\mu = e_k$ for which the $g_\mu^\star$ never changeswe argue that when $\alpha  (1 \wedge \gamma_{k})  \ge d$ for some $k$ then for $\mu = e_k$ for which the $g_\mu^\star$ never changes its decision within the interior of $\supp(P_X)$. 
Following \citet{audibert2007Fast}, we focus on the cases where $\alpha < d$ and for every $k$ the $\alpha \gamma_k < d$. This helps to avoid trivial cases where routing decisions are constant over $P_X$ for some $\mu$.  %These are trivial cases, which we ignore. Thus, throughout our paper, we assume that $\alpha   < d$ and for every $k$ the $\alpha \gamma_k < d$.  
Next, we assume that $P_X$ has a density $p_X$ that satisfies a strong density condition described below.
% We start by formalizing the problem setup. We assume that the covariate space is $\cX$ is a compact set in $\reals^d$. Next, we assume that the density exists for the marginal probability $P_X$ and satisfies a strong density condition, which is formalized below. 
\begin{assumption}[Strong density condition] \label{assmp:strong-density}
Fix constants $c_0, r_0> 0$ and $0 \le \mu_{\min}  \le \mu_{\max} < \infty$. We say $P_X$ satisfies the strong density condition if its support is a compact $(c_0, r_0)$-regular set and it has density $p_X$ which is bounded: $\mu_{\min} \le p_X (x) \le \mu_{\max} $ for all $x$ within $\supp(P_X)$. A set $A \subset \reals^d$ is Lebesgue measurable and %\ie\ for the Lebesgue measure $\Lambda_d$ on $\reals^d$, any Lebesgue measurable set $A \subset \reals^d$ and 
$\text{for any} ~ 0 < r \le r_0, ~  x \in A$ it satisfies
\begin{equation}
    \Lambda_d (A \cap \cB(x, r, \ell_2)) \ge c_0 \Lambda_d(\cB(x, r, \ell_2)). %~ \text{for any} ~ 0 < r \le r_0, ~  x \in A,
\end{equation} %and the density $p_X$ is bounded as: $\mu_{\min} \le p_X (x) \le \mu_{\max} $ for all $x$ within $\supp(P_X)$. 
\end{assumption}
This is another standard assumption required for minimax rate studies in nonparametric classification problems \citep{audibert2007Fast,cai2019Transfer}. All together, we define $\cP(c_0, r_0, \mu_{\min}, \mu_{\max}, \beta_{m ,k}, K_{\beta, m, k}, \alpha, K_\alpha)$, or simply $\cP$, as the class of probabilities $P$ defined on the space $\cX \times \cY$ for which $P_X$  is  compactly supported and satisfies the strong density assumption \ref{assmp:strong-density} with parameters $(c_0, r_0, \mu_{\min}, \mu_{\max})$, and the H\"older smoothness assumption \ref{assmp:smooth} and the $(\alpha, K_\alpha)$-margin condition in Assumption \ref{assmp:margin} hold. We shall establish our minimax rate of convergence within this probability class. 










% In this section we provide a ``mini-max'' investigation on learning of the oracle $g^\star$. For this purpose, assume that 
% \begin{itemize}
    
%     \item {\bf Strong density condition:} $X$ is distributed in the space $[0, 1]^d$ that satisfies the strong density condition \citep{audibert2007Fast}.
%     \item {\bf H\"older smoothness:} The functions $\kappa_\lambda(x, l) \triangleq\ell (f^\star(x), f_l(x)) + \lambda c_l (x)$ are $\alpha$-H\"older smooth. 
% \item {\bf Noise condition:} For $t > 0$ there exists a $\gamma > 0$ such that 
% \[
% \textstyle P_X \big ( 0 <\max_l \big | \kappa_\lambda(x, l) - \min_{l' \neq l} \kappa_\lambda (x, l') \big | \le t \big ) = \cO(t^\gamma)
% \]
% \end{itemize}
% Denote $\cP$ as the class of all probabilities that satisfies the above conditions. 

% \SM{revise the lower bound in light of $\lambda  \ge c n^{\frac{\beta - \nicefrac d\alpha }{2\beta +d }}$ requirement.}
\subsection{The lower bound} 
Rather than the actual risk $\cR_P(\mu, g)$, we establish a lower bound on the excess risk:
\begin{equation}\label{eq:excess-risk}
    \cE_P(\mu, g) = \cR_P(\mu, g) - \cR_P(\mu, g_\mu^\star)\,,
\end{equation} that compares the risk of a proposed router to the oracle one. We denote $\Gamma = \{g: \cX \to [M]\}$ as the class of all routers. For an $n \in \bN$ we refer to the map $A_n: \cZ^n \to \Gamma$, which takes the dataset $\cD_n $ as an input and produces a router $A_n(\cD_n): \cX \to [M]$, as an algorithm. Finally, call the class of all algorithms that operate on $\cD_n$ as $\cA_n$. The following Theorem describes a lower bound on the minimax risk for any such algorithm $A_n$. 
\begin{theorem}\label{thm:lower-bound}
    For an $n \ge 1$  and $A_n \in \cA_n$ define  $\cE_P(\mu, A_n) = \Ex_{\cD_n}\big[\cE_P\big(\mu, A_n(\cD_n)\big)\big]$ as the excess risk of an algorithm $A_n$. There exists a constant $c> 0$ that is independent of both $n$ and $\mu$ such that for any $n\ge 1$ and $\mu\in \Delta^{K-1}$ we have the lower bound
    \begin{equation}\label{eq:lower-bound}
      \textstyle  \min\limits_{A_n \in \cA_n} \max\limits_{P \in \cP} ~~ \cE_P(\mu, A_n) \ge c \big \{\sum_{k = 1}^{K_1} \mu_k n^{- \frac{\gamma_k}{2\gamma_k + d}}\big\}^{1+\alpha} \,.
    \end{equation}
\end{theorem} 
This result is a generalization of that in \citet{audibert2007Fast}, which considers binary classification. 
\begin{remark} \label{remark:minimax-lower-bound}
    Consider the binary classification in Example \ref{example:binary-classification}. Since $K = 1$, the lower bound simplifies to $\cO(n^{-\nicefrac{\gamma_1 (1+ \alpha)}{2\gamma_1 + d}})$,  which matches with the rate in \citet[Theorem 3.5]{audibert2007Fast}. 
    Beyond $0/1$ loss, our lower bound also establishes that the rate remains identical for other classification loss functions as well. 
    
    % The case of $\lambda = 1$ is closely related to the usual classification tasks with a single objective function. In fact, binary classification with $0/1$-loss is a special case. To make this connection clear, consider $M = 2$ and the index set for the classes as $\{0, 1\}$. Moreover, assume that the loss is $0/1$, \ie\ for $m\in \{0, 1\}$ the  $Z_m = \ell\{ Y, f_m(X)\} \in \{0, 1\}$ and $Z_0 + Z_1 = 1$, in which case the loss of a router $g:\cX \to \{0, 1\}$ is 
    % \[
    % \begin{aligned}
    %      \textstyle \ell\{g; X, Y\} & \textstyle= \bbI\{ g(X) = 0\} Z_0 +  \bbI\{ g(X) = 1\} Z_1 \\
    %      & \textstyle= \bbI\{ g(X) = 0\} Z_0 +  \bbI\{ g(X) = 1\} (1 - Z_0) = \bbI \{g(X) \neq Z_0\} \,.
    % \end{aligned}
    % \] Thus, it is not surprising that for $\lambda = 1$ our rate of convergence $\cO(n^{\frac{-\beta(1 + \alpha)}{2\beta +d}})$ for the lower bound is exactly the same as in \citet[Theorem 3.5]{audibert2007Fast}. As such, {we broaden the framework of the minimax lower bound study for non-parametric classification tasks to (1) more than two classes, and (2) general loss functions.} To understand this, consider a classification task with $M$ classes and the loss function of a classifier $g: \cX \to [M]$ is $\ell\{ g; X, Y\} = \sum_{m = 1}^M \bbI \{g(X) = m\} \ell_m(X, Y)$ where $\ell_m(X, Y)$ is the loss incurred when a sample $(X, Y)$ is predicted as the class $m$. In that case, simply letting $\lambda = 1$ and $Z_m = \ell_{m}(X, Y)$ within the lower bound analysis, we obtain a rate of convergence $\cO(n^{\nicefrac{-\beta(1 + \alpha)}{(2\beta +d)}})$. Moreover, the analysis of the upper bound in Section \ref{sec:upper-bound} reveals that this rate is minimax optimal. 
\end{remark}


% \SM{mention how the true difficulty of the problem lies when the $\lambda$ are bounded away from zero because in that case one needs to accurately estimate the $\Phi_m^\star$. At $\lambda \to 0$ the importance is not on the unknown function.}

% \begin{remark}\label{remark:diff-in-lambda-lb} \label{remark:lower-bound-lambda}
%     The lower bound also highlights that ``it is easier to route for a smaller value $\lambda $''. Within the oracle router, we know the $(1 - \lambda) \kappa_m(X)$ part within the function $\eta_{\lambda, m}^\star(X) = \lambda \Phi_m^\star(X) + (1 - \lambda) \kappa_m(X)$. Thus, at an intuitive level, for smaller values of $\lambda$ there is less importance on the unknown $\lambda \Phi_m^\star(X)$ part, and thus the plug-in router in eq. \eqref{eq:plugin-router} can tolerate a larger noise in their estimations. Our lower bound makes this intuition precise: for a smoothness parameter $\beta$ the $\Phi_m^\star$ are estimated at a $\cO_P(n^{-\nicefrac{\beta}{(2\beta+d)}})$-rate, and thus both $\lambda \Phi_m^\star(X)$ and the whole $\eta_{\lambda, m}^\star(X)$ are estimated at a $\cO_P(\lambda n^{-\nicefrac{\beta}{(2\beta+d)}})$-rate. This intuition is more formally exposed in the upper bound study of excess risk in Theorem \ref{thm:upper-bound}, where we precisely quantify the relationship between the error in the estimation of $\Phi_m^\star$ and the convergence rate for excess risk (\cf\ Remark \ref{remark:upper-bound-lambda}).
%     Following this intuition, we can understand that learning a router for a smaller $\lambda$ is easier and for $\lambda = 1$ it is the hardest.
% \end{remark}

% This is the exact same rate of convergence obtained in \citet[Theorem 3.5]{audibert2007Fast}. Intuitively, for a fixed $\lambda$ the task of routing is equivalent to multi-class classification, so, it is not surprising that they have the same rate. 


% \SM{remark about $\lambda$}. 

% \begin{remark}
%     For a fixed $n$ our lower bound analysis in Theorem \ref{thm:lower-bound} is only valid for $\lambda \ge c_1 n^{\frac{\beta - \nicefrac{d}{\alpha}}{2\beta + d}}$. To address this gap, we note: 
%     \begin{itemize}
%         \item Firstly, because we know the cost functions $\kappa_m$, we can precisely find the oracle router $g^\star_{0}(X) = \argmin_m \kappa_m(X)$ at $\lambda = 0$. Furthermore, since we consider $\alpha \beta < d$ the $c_1 n^{\frac{\beta - \nicefrac{d}{\alpha}}{2\beta + d}}$ decreases to zero as $n$ grows to infinity. Thus, the gap in $\lambda$, where this lower bound is invalid, vanishes as the sample size increases to infinity. 
%         \item Regardless of this gap, we argue in Remark \ref{} that we can estimate the Pareto frontier for the performance-cost trade-off efficiently. 
%     \end{itemize}
% \end{remark}


% \SM{compare it to usual rate of convergence for classification when $\lambda = 1$}. 


% \subsection{Efficient learning of oracle routers}
% \label{sec:efficient-learning}
% Let us quickly recall our core idea behind it. For a $\lambda \in [0, 1]$ the true loss regression function $\eta_{\lambda, m}^\star(X)$ for the oracle router $g_\lambda^\star (X) = \argmin_m \eta_{\lambda, m}^\star(X)$ is  decomposed as: 
% \begin{equation} \label{eq:oracle-router}
%     \eta_{\lambda, m}^\star(X) = \lambda \Phi^\star_m(X) + (1 - \lambda) \kappa_m(X), ~ \Phi^\star _m(X) = \Ex_P[\ell\{ Y, f_m(X)\} \mid X ]\,. 
% \end{equation} Since we already know $\kappa_m(X)$ at a new $X$ the only unknown is the $\Phi^\star_m(X)$. Thus, we can plug-in its estimate  $\widehat \Phi_m(X)$ within eq. \eqref{eq:oracle-router} and estimate the oracle router as:
% \begin{equation}\label{eq:plugin-router}
%     \widehat g_\lambda(X) = \argmin_m \hat \eta_{\lambda, m}(X), ~~ \hat \eta_{\lambda, m}(X) = \lambda \widehat \Phi_m(X) + (1 - \lambda) \kappa_m(X)  
% \end{equation} 
% Moreover, we evaluate their prediction errors and costs on a test split of the dataset as: 
% \begin{equation}
%    \textstyle  \hat \cE_\lambda  =  \frac1{n_\text{test}} \sum_{i = 1}^{n_\text{test}}  \ell\{Y_i', f_{\widehat g_\lambda(X_i')}(X_i')\}, ~~ \hat \cC _\lambda =  \frac1{n_\text{test}} \sum_{i = 1}^{n_\text{test}}  \kappa_{\widehat g_\lambda(X_i')}(X_i')\,. 
% \end{equation} 
% This plug-in approach is computationally efficient, as we can estimate oracle routers for all $\lambda$ in one go, instead of minimizing \ref{eq:ERM} at different $\lambda$'s. In addition to being computationally efficient, through a study on the minimax upper bound on excess risk in the next section we establish that this plug-in approach is also statistically efficient. Furthermore, we extend this approach to a general multi-objective classification problem. For now, we end this section by summarizing our steps in Algorithm \ref{alg:pareto-routers}. 


% \begin{algorithm}
%     \begin{algorithmic}[1]
% \Require Dataset $\cD_n$
% \State Randomly split the dataset into training and test splits: $\cD_n = \cD_{\text{tr}} \cup \cD_{\text{test}}$. 
% \State  Learn an estimate $\widehat \Phi_m (X)$ of $\Phi_m^\star(X)$ using the training split $\cD_{\text{tr}}$. 
% \For{$\lambda \in [0, 1]$}
% \State  Define $\hat \eta_{\lambda, m}(X) =  \lambda \widehat \Phi_m(X) + (1 - \lambda) \kappa_m(X)  $ and 
%  $\widehat g_\lambda(X) = \argmin_m \hat \eta_{\lambda, m}(X)$. If there is a tie within the $\argmin$, break the tie randomly.
%  \State Calculate $\hat \cE_\lambda  =  \frac1{|\cD_{\text{test}}|} \sum_{(X, Y) \in \cD_{\text{test}}}  \ell\{Y, f_{\widehat g_\lambda(X)}(X)\}$
%  \State \quad\quad  and $\hat \cC_\lambda  =  \frac1{|\cD_{\text{test}}|} \sum_{(X, Y) \in \cD_{\text{test}}}  \kappa_{\widehat g_\lambda(X)}(X)$
% \EndFor

% \Return $\{g_\lambda: \lambda \in [0, 1]\}$ and $\hat\cF = \{(\hat \cE_\lambda, \hat \cC_\lambda): \lambda \in [0, 1]\}$. 
% \end{algorithmic}
% \caption{Learning of oracle routers}
% \label{alg:pareto-routers}
% \end{algorithm}


% \SM{talk about the computational efficiency, that we can obtain all the routers in one go. Also, provide a teaser that in the next section we shall establish that its statistically efficient as well.}





% Given a router $\widehat g:[0, 1]^d \to \Delta^ L$ we define the excess risk as:
% \begin{equation}
%    \textstyle \cE_P (\widehat g) = \cR_P(\widehat g) - \cR_P(g^\star) = \Ex_P\big[\sum_{l = 1}^L\{g_l(x_0)- g_l^\star(X_0)\}\kappa_\lambda(X_0 , l) \big]
% \end{equation} We shall show that 
% \begin{equation}
%    \inf_{\widehat g} \sup_{P\in \cP} \textstyle \Ex_P \big[ \cE_\lambda (\widehat g) \big ] \asymp n ^{- \frac{\alpha(1 + \gamma)}{2\alpha + d}}\,.
% \end{equation}



\subsection{The upper bound }\label{sec:upper-bound}
Next, we show that if algorithm the $A_n$ corresponds to CARROT, the performance of $\hat{g}_{\mu}$ matches the lower bound in Theorem \ref{thm:lower-bound} (\cf\ equation \ref{eq:lower-bound}). En-route to attaining $\hat{g}_{\mu}$, we need an estimate $\widehat \Phi(X)$ of $\Phi(X) = \Ex_P[Y \mid X ]$. %In this section, we ask what is the needed performance of this estimate? More importantly, we also study if the plug-in approach leads to statistically efficient estimates of the oracle routers, or in other words does the performance of $\hat{g}_{\mu}$ match \ref{eq:lower-bound} 
Our strategy will consist of two steps: 
\begin{itemize}
    \item First,  we establish an upper bound on the rate of convergence for excess risk \eqref{eq:excess-risk} for the plug-in router in terms of the rate of convergence for $\widehat \Phi(X)$. 
    \item Then we discuss the desired rate of convergence in $\widehat \Phi(X)$ so that the upper bound has the identical rate of convergence to the lower bound \eqref{eq:lower-bound}. Later in Appendix \ref{sec:reg-fn-estimate} we provide an estimate $\widehat \Phi(X)$ that has the required convergence rate. 
\end{itemize}
These two steps, together with the lower bound in \eqref{eq:lower-bound} establish that our plug-in router achieves the best possible rate of convergence in excess risk. 

We begin with an assumption that specifies a rate of convergence for $[\widehat \Phi(X)]_{m, k}$. 
\begin{assumption} \label{assmp:convergence}
    For some constants $\rho_1, \rho_2 > 0$ and any $n \ge 1$ and $t > 0$ and almost all $X$ with respect to the distribution $P_X$ we have the following concentration bound:
    \begin{align}\label{eq:concentration-phi}
        \max_{P\in \cP} P \big \{ \max_{m, k} a_{k, n}^{-1}\big |[\widehat \Phi (X)]_{m, k} - [\Phi  (X)]_{m, k}\big |
        \ge t\big \}  
        \le  \rho_1 \exp\big (- \rho_2  t^2 \big )\,,  
    \end{align}  where for each $k$ the  $\{a_{k,n}; n \ge 1\}\subset (0, \infty)$ is a sequence that decreases to zero. 
\end{assumption}
Using this high-level assumption, in the next theorem, we establish an upper bound on the minimax excess risk for CARROT that depends on both $a_{k, n}$ and $\mu$.  
\begin{theorem}[Upper bound]\label{thm:upper-bound}
  Assume \ref{assmp:convergence}.   If all the $P\in \cP$ satisfy the margin condition \ref{assmp:margin} with the parameters $(\alpha, K_\alpha)$ then there exists a $K> 0$ such that for any $n \ge 1$ and $\mu\in \Delta^{K-1} $ the excess risk for the router $\widehat g_\mu$ in Algorithm \ref{alg:pareto-routers} is upper bounded as 
    \begin{equation}
        \max_{P\in \cP} \Ex_{\cD_n}\big [\cE_P(\widehat g_\lambda,\lambda)\big ] \le\textstyle K \big \{\sum_{k = 1}^{K_1} \mu_k a_{k, n}\big\}^{1+\alpha} \,.
    \end{equation}
\end{theorem} 
% \begin{remark} \label{remark:upper-bound-lambda}
%     Recall the Remark \ref{remark:lower-bound-lambda}, where we discuss that it is easier to route for smaller $\lambda$. Indeed, under Assumption \ref{assmp:convergence} the same holds for $\hat \eta_{\lambda, m}(X) -  \eta_{\lambda, m}^\star(X) = \lambda \{ \widehat \Phi_m(X) - \Phi^\star_m(X)\}$ at a rate $\lambda a_n^{-1}$, which, under the $\alpha$-margin condition (Assumption \ref{assmp:margin}), reveals such a dependence for the minimax upper bound on $\lambda$. 
% \end{remark}


\begin{remark}[Rate efficient routers] \label{cor:efficient-routers}
    When $a_{k, n} = n^{-\nicefrac{\gamma_k}{(2\gamma_k +d)}}$ the upper bound in Theorem \ref{thm:upper-bound} has the $\cO(\{\sum_{k = 1}^{K_1}\mu_k n^{-\nicefrac{\gamma_k}{(2\gamma_k+d)}}\}^{1+\alpha})$-rate, which is identical to the rate in the lower bound (\cf\ Theorem \ref{thm:lower-bound}), suggesting that the minimax optimal rate of convergence for the routing problem is 
\begin{equation}
    \label{eq:minimax-rate}
     \textstyle  \min\limits_{A_n \in \cA_n} \max\limits_{P \in \cP} ~~ \cE_P(A_n, \lambda) \asymp \textstyle  \cO\big ( \big \{\sum_{k = 1}^{K_1} \mu_k n^{- \frac{\gamma_k}{2\gamma_k + d}}\big\}^{1+\alpha}\big ) \,.
\end{equation}
   Following this, we conclude: When $a_{k, n} = n^{-\nicefrac{\gamma_k}{(2\gamma_k +d)}}$ the plug-in approach in Algorithm \ref{alg:pareto-routers}, in addition to being computationally efficient, is also minimax rate optimal. 
%\begin{enumerate}
   % \item When $a_{k, n} = n^{-\nicefrac{\gamma_k}{(2\gamma_k +d)}}$ the plug-in approach in Algorithm \ref{alg:pareto-routers}, in addition to being computationally efficient, is also minimax rate optimal in excess risk. 
   % \item  Following up on the Remark  \ref{remark:minimax-lower-bound}, this result generalizes the minimax study of non-parametric binary classification to (a) more than two classes, and (b) classification loss functions beyond $0/1$-loss. 
%\end{enumerate}
\end{remark} 
An example of an estimator $\widehat{\Phi}$ that meets the needed conditions for $a_{k, n} = n^{-\nicefrac{\gamma_k}{(2\gamma_k +d)}}$ to hold is described in Appendix \ref{sec:reg-fn-estimate}. 
% \begin{remark} \label{remark:difficulty-routing}
% The optimal rate of convergence in \eqref{eq:minimax-rate} is particularly small for small values of $\sum_{k = 1}^{K_1} \mu_k$; in fact, it's identical to zero when $\sum_{k = 1}^{K_1} \mu_k = 0$. This is quite intuitive from the expression of $g_\mu^\star$ in Lemma \ref{lemma:oracle-router}.  For $\sum_{k = 1}^{K_1} \mu_k = 0$ the $g_\mu^\star(X) = \argmin_{m} \{\sum_{k\ge K_1+1} \mu_k [\Phi(X)]_{m, k}\}$ is precisely known to us, thus the excess risk for routing is identical to zero.
    
% \end{remark}

   
%     This leads to the two following conclusions: 
%     \begin{enumerate}
%         \item For any $\lambda$ the minimax optimal rate of convergence for the routing problem is 
%         \[
%         \textstyle \textstyle  \min_{A_n \in \cA_n} \max_{P \in \cP} ~~ \cE_P(A_n, \lambda) \asymp  \cO\big ( \big\{\lambda n^{-\nicefrac{\beta}{(2\beta+d)}}\big\}^{1+\alpha}\big ) \,.
%         \] 
%         This reveals 
        
        
%         Unsurprisingly, this is identical to the optimal rate of convergence in non-parametric classification problems \citep{audibert2007Fast}. Repeating the intuition again, for a fixed $\lambda$ the routing problem 
%         is essentially a multiclass identical minimax optimal rate of convergence for excess risk. 
%         \item  A more interesting observation is that if $\widehat \Phi_m$ converges to $\Phi_m^\star$ at a rate $a_n^{-\frac{1}{2}} = n^{-\frac{\beta}{2\beta +d}}$ then our one-shot approach in Algorithm \ref{alg:pareto-routers} achieves this optimal minimax rate of convergence in excess risk and thus is a \emph{rate efficient} estimator for Pareto routers. We shall show later in Section \ref{sec:reg-fn-estimate} that a local polynomial regression estimator with a suitable bandwidth achieves this desired rate. 
%         \end{enumerate}



% \begin{remark}
%     Remark about the importance of $\lambda$. 
% \end{remark}

% \begin{remark}
%     Let us quickly recall the gap in the validity in our lower bound in Theorem \ref{thm:lower-bound}, that the rate in the lower bound is not true when $0 < \lambda < c_1 n ^{\frac{\beta - \nicefrac{d}{\alpha}}{2\beta +d}}$. However, the rate in the upper bound continues to hold for all $0 \le \lambda \le 1$. 
% \end{remark}


% Now, we make this statement precise and show that for such an $\widehat \Phi $ the excess risk \eqref{eq:excess-risk} achieves the rate in the lower bound \eqref{eq:lower-bound}. 

% \begin{theorem}[Upper bound]\label{thm:upper-bound}
%     Suppose that for some $\rho_1, \rho_2 > 0$ and any $n \ge 1$ and $t > 0$ and almost all $x$ with respect to $P_X$ we have the following concentration bound for $\widehat \Phi$:
%     \begin{equation}\label{eq:concentration-phi}
%         \max_{P\in \cP} P_{\cD_n} \big \{ \|\widehat \Phi(x) - \Phi^\star (x)\|_1 \ge t\big \} \le  \rho_1 \exp\big (- \rho_2 a_n t^2 \big )\,, 
%     \end{equation}
%     where $\{a_n; n \ge 1\}\subset \reals$ is a sequence that increases to $\infty$.  
%     Fix a $\lambda \in [0, 1]$.  Then, if all the $P\in \cP$ satisfies the margin condition \ref{assmp:margin} with the parameters $(\alpha, K_\alpha)$ then there exists a $K> 0$ such that for any $n \ge 1$ the excess risk for the router $\widehat g_\lambda$ in \eqref{eq:eff-estimate-router} is upper bounded as 
%     \begin{equation}
%         \max_{P\in \cP} \Ex_{\cD_n}\big [\cE_P(\widehat g_\lambda,\lambda)\big ] \le K a_n^{-\frac{1+ \alpha}{2}}\,. 
%     \end{equation}
%     Thus, as long as $a_n = n^{{  2\beta/(2\beta + d)}}$  for any $\lambda \in [0, 1]$ the excess risk for the router $\widehat g_\lambda$ in \eqref{eq:eff-estimate-router} has the rate of convergence $a_n^{- {(1 + \alpha)}/{2}} = n^{- {\beta(1 + \alpha)}/{(2\beta + d)}}$ that matches with the lower bound rate in \eqref{eq:lower-bound}. 
% \end{theorem}






% \subsection{Efficiency of the estimation Pareto frontier}

% As mentioned in Section \ref{sec:efficient-learning} and Algorithm \ref{alg:pareto-routers} we estimate the performance-cost trade-off for Pareto routers in a very simple manner: first estimate the Pareto routers $\widehat g_\lambda$ in a training split of the dataset, then evaluate their average performances and costs in the remaining test split as
% \[
% \textstyle \hat \cE_\lambda  =  \frac1{n_\text{test}} \sum_{i = 1}^{n_\text{test}}  \ell\{Y_i', f_{\widehat g_\lambda(X_i')}(X_i')\}, ~~ \hat \cC _\lambda =  \frac1{n_\text{test}} \sum_{i = 1}^{n_\text{test}}  \kappa_{\widehat g_\lambda(X_i')}(X_i')\,. 
% \] In this section we evaluate the efficiency of estimating the Pareto front $\{ (\hat \cE_\lambda, \hat \cC_\lambda): 0 \le \lambda \le 1\}$ in this manner. Note that the true Pareto front is $\{ ( \cE_\lambda^\star,  \cC_\lambda^\star): 0 \le \lambda \le 1\}$ where 
% \[
% \textstyle \cE^\star_\lambda = \Ex_P [\ell\{ Y, f_{g_\lambda^\star(X)}(X)\}], ~~ \cC_\lambda^\star =  \Ex_P [\kappa _{g_\lambda^\star(X)}(X)]\,. 
% \] To evaluate their differences, focus only on $\hat \cE_\lambda - \cE_\lambda^\star$, as the analysis of $\hat \cC_\lambda - \cC_\lambda^\star$ is very similar. Defining $\widetilde \cE_\lambda = \Ex_P [\ell\{ Y, f_{\widehat g_\lambda(X)}(X)\}]$, we can decompose the difference as 
% \begin{equation}
%     \textstyle \hat \cE_\lambda - \cE_\lambda^\star = \{\hat \cE_\lambda - \widetilde \cE_\lambda\}  + \{\widetilde \cE_\lambda-  \cE_\lambda^\star\} 
% \end{equation} The first term is $\cO (n^{-\nicefrac12})$. Now, to bound the second term, notice that 
% \[
% \textstyle \cE_P(\widehat g_\lambda,\lambda) \le 
% \]


% that for $h = n ^{-1/(2\beta + d)}$ the concentration bound in \eqref{eq:concentration-phi} is satisfied with $a_n = n^{-2\beta/(2\beta + d)}$, therefore the router derived from such $\hat\Phi$ using \eqref{eq:eff-estimate-router} achieves the same rate of convergence as in the lower bound \eqref{eq:lower-bound}, \ie\ rate optimal. 
% \section{The upper bound}\label{sec:ub}

In this section we prove \Cref{thm:main-ub}. We construct a distribution $\pi$  that satisfies $\DTV(\mu,\pi)\le \frac{\eps}{2}$, has smoothness close to that of $\mu$, is of bounded moment, and whose \Poincare constant is at least $\approx \tp{\frac{LM}{d\eps}}^{-O(d)}$. Then we call known Langevin-based algorithm to sample from $\pi$, whose sample complexity is directly related to the \Poincare constant. Our strategy for the construction of $\pi$ is as follows.

\begin{itemize}
    \item First, observe the following comparison result. Let $p_1(x)$ and $p_2(x)$ be the densities of two distributions supported on $\bb R^n$. If $1/C\le \frac{p_1(x)}{p_2(x)}\le C$ for every $x\in \bb R^n$, then the ratio of their \Poincare constants is at least $C^{-\+O(1)}$. Therefore, we only need to construct a distribution $\pi$ with appropriate smoothness, whose density is pointwise close to a suitable Gaussian. The range of ratios in the densities that we can tolerate is of the order $\tp{\frac{LM}{d\eps}}^{\+O(d)}$.
    \item Clearly the density $p_\mu(x)\propto e^{-f_\mu(x)}$ of $\mu$ does not satisfy our requirement due to the possible existence of certain regions with extremely small probability. The value of $f_\mu$ may be very large in these regions. On the other hand, the measure of these regions under $\mu$ is small, so we can \emph{truncate} $f_\mu$ appropriately to ensure its value is well upper bounded without affecting the measure $\mu$ much. We then use the truncated function to define the distribution $\pi$.
    \item In order to truncate $f_\mu$ appropriately, we need to estimate its minimum value $f^*\defeq \min_{x\in\bb R^d} f_\mu(x)$ and the partition function $Z_\mu \defeq \int_{\bb R^n} \exp\tp{-f_\mu(x)} \d x$ within a certain accuracy. To this end, we divide a compact set containing most mass of $\mu$ into cubes and approximate $\mu$ in each cube respectively using queries to $f_\mu$.
\end{itemize}
%\htodo{We only use queries to $f_\mu$, no $\grad f_{\mu}$ in the estimation?}

We will give the construction of $\pi$ in \Cref{sec:construction-of-pi} and prove its properties in \Cref{sec:properties-of-pi}. Then we show how to estimate the key parameters in our construction in \Cref{sec:estimate-of-pi}. Finally, we combine everything and prove \Cref{thm:main-ub} in \Cref{sec:proof-of-ub}.

\subsection{The construction of $\pi$}\label{sec:construction-of-pi}

The purpose of this section is to construct a distribution $\pi$ whose density function is close to that of $\gamma\sim\+N\tp{0,\frac{M}{\eps d}\!{Id}_d}$ \emph{pointwise} and $\DTV(\mu,\pi)\le \frac{\eps}{2}$. We assume the density of $\pi$ is \emph{proportional to} $\exp\tp{-f_\pi(x)}$. Let $f_\gamma\colon \bb R^d\to\bb R$ be the function $x\mapsto \frac{\eps d\norm{x}^2}{2M}+\frac{d}{2}\log\frac{2\pi M}{d\eps}$. Then the density of $\gamma$ is \emph{equal to} $\exp\tp{-f_\gamma(x)}$.

We use $Z_\pi \defeq \int_{\bb R^d} \exp\tp{-f_\pi(x)}\d x$ and $Z_\mu\defeq \int_{\bb R^d} \exp\tp{-f_\mu(x)} \d x$ to denote the two normalizing factors. Then the density of $\pi$ and $\mu$ are
\[
    p_\pi(x) = \exp\tp{-f_\pi(x)} / Z_\pi \mbox{ and } p_{\mu}(x) = \exp\tp{-f_{\mu}(x)} / Z_\mu
\]
respectively.

Note that the second moment of a random variable $X$ with law $\mu$ is at most $M$. By Markov's inequality, for $R=\sqrt{\frac{32M}{\eps}}$,
\begin{equation}\label{eqn:markov-mu}
    \Pr{\norm{X}^2>R^2} \le \frac{\E{\norm{X}^2}}{R^2}\le \frac{M}{R^2} = \frac{\eps}{32},
\end{equation}
meaning outside a ball of radius $\Theta(R)$, the mass of $\mu$ is $O(\eps)$. We let $f_\pi(x) = f_\gamma(x) - \log\eps$ for $x\in \bb R^d \setminus \+B_{2R}$. We will then construct a function $f_\pi^{\le 2R}$ with support $\+B_{2R}$ and define
\[
    f_\pi(x) = (1-\mathfrak{g}_{[R,2R]}(x)) \cdot f_\pi^{\le 2R}(x) + \mathfrak{g}_{[R,2R]}(x) \cdot \tp{f_\gamma(x)-\log\eps},
\]
where $\mathfrak{g}_{[R,2R]}\defeq q_{\!{mol}}\tp{\frac{\norm{x}^2-R^2}{(2R)^2-R^2}}$ is the smooth function interpolating $f_\pi^{\le 2R}$ and $f_\gamma - \log\eps$ in the region $\norm{x}\in [R,2R]$. 

%Let $Z_\pi\defeq \int_{\bb R^n}$ The construction of $f_\pi^{\le 2R}$ should meet the 
As discussed before, for $x\in \+B_R$, ideally $p_{\pi}(x) = \exp\tp{-f_{\pi}^{\le 2R}(x)} / Z_\pi$ should be close to $p_{\mu}(x) = \exp\tp{-f_\mu(x)} / Z_\mu$ with those points of extremely small probability smoothly truncated. As a result, we first assume that we can find an approximation of $Z_\mu$, denoted as $\wh Z_\mu$. In general calculating a good approximation for $Z_\mu$ is computationally equivalent to sampling from $p_\mu$. However, as our target is a bound for the \Poincare constant of order exponential in $d$, our requirement for the accuracy of the approximation is very loose. We also assume an approximation $\wh f^*$ of the minimum $f^*\defeq \inf_{x\in \+B_{2R}} f_\mu(x)$. In fact, we will prove the following proposition in \Cref{sec:estimate-of-pi}.

\begin{proposition} \label{prop:Z-and-fmin}
    Within $\+O\tp{\frac{LM}{\eps d}}^d$ queries to $f_\mu(x)$, one can find
    \begin{itemize}
        \item a number $\wh Z_\mu$ satisfying $\frac12 e^{-d}\le \frac{\wh Z_\mu}{Z_\mu}\le 1$, and
        \item a number $\wh f^*$ satisfying $f^*\le \wh f^*\le f^*+d$.
    \end{itemize}    
\end{proposition}

Then we turn to truncate the small value of $\exp(-f_\mu(x))$ or equivalently the large value of $f_\mu(x)$ in $\+B_{2R}$. Define two constants
\[
    h_1 \defeq \wh f^* +\log\!{vol}(\+B_{2R}) + \frac{d}{2}\log L+\log\frac{4}{\eps},\; h_2 \defeq h_1+\frac{d}{2}\log\frac{LM}{d\eps}.
\]
We remark that $h_1$ is our threshold for the truncation. The term $\frac{d}{2}\log L$ term is used to guarantee that $\log\!{vol}(\+B_{2R}) + \frac{d}{2}\log L$ is nonnegative and therefore $h_1\ge f^*$. In order to keep the truncated function smooth, we define a \emph{soft threshold} $h_2$ above $h_1$. 

Define the interpolation function $\mathfrak{g}_{[h_1,h2]}(x) \defeq q_{\!{mol}}\tp{\frac{h_2-f_{\mu}(x)}{h_2-h_1}}$. We define
\[
    \ol{f_{\pi}^{\le 2R}}(x)\defeq \mathfrak{g}_{[h_1,h_2]}(x)\cdot f_\mu(x) + (1-\mathfrak{g}_{[h_1,h_2]}(x))\cdot h_2
\]
In other words, for those $x$ with $f_\mu(x)\le h_1$, $\ol{f_{\pi}^{\le 2R}}(x) = f_\mu(x)$; for those $x$ with $f_\mu(x)\ge h_2$, $\ol{f_{\pi}^{\le 2R}}(x) = h_2$; for those $x$ with $f_\mu(x)\in [h_1,h_2]$, the value of $\ol{f_{\pi}^{\le 2R}(x)}$ is smoothly interpolated between $h_1$ and $h_2$. The function $f_\pi^{\le 2R}$ is illustrated in \Cref{fig:ub}. %\ctodo{A figure here.}

Finally, we \emph{approximately normalize} $\exp\tp{-\ol{f_{\pi}^{\le 2R}}(x)}$ into a ``probability'' by dividing our estimate $\wh Z_\mu$, namely that for every $x\in \bb R^d$, let
\[
    f^{\le 2R}_\pi(x) \defeq \ol{f^{\le 2R}_\pi}(x) + \log \wh Z_{\mu}.
\]


\subsection{Properties of $\pi$}\label{sec:properties-of-pi}

In this section we prove some useful properties of the distribution $\pi$ just constructed. We begin with three useful technical lemmas in \Cref{sec:ub-tech}. Then we prove key properties of $\pi$, including showing its closeness to $\mu$ in terms of total variation distance in \Cref{sec:ub-closeness}, bounding its \Poincare constant in \Cref{sec:ub-poincare} and analyzing its smoothness in \Cref{sec:ub-smooth}.

\subsubsection{Technical lemmas}\label{sec:ub-tech}

Recall that $p_\mu(x)\propto \exp\tp{-f_\mu(x)}$ is the density of the $L$-log-smooth distribution $\mu$ and $Z_\mu = \int_{\bb R^d} p_\mu(x)\d x$ is the normalizing factor. The first lemma says that $p_\mu(x)$ has an upper bound since $\mu$ is $L$-log-smooth. 

\begin{lemma}\label{lem:mu-bound}
    $\forall x\in \bb R^d,\;p_\mu(x) \le \tp{\frac{2\pi}{L}}^{-\frac{d}{2}}$.
\end{lemma}
\begin{proof}
    Let $x^* = \argmax_{x\in\bb R^d} \exp\tp{-f_\mu(x)}$. Since $f_\mu$ is $L$-smooth, for each $y\in \bb R^d$,
    \begin{align*}
        f_\mu(y)
        &\le f_\mu(x^*) + \grad f_\mu(x^*)\top (y-x^*) + \frac{L}{2}\norm{y-x^*}^2\\
        \mr{$\grad f_\mu(x^*)=0$}
        &=f_\mu(x^*)+ \frac{L}{2}\norm{y-x^*}^2.
    \end{align*}
    On the other hand, 
    \begin{align*}
        1
        &=Z_\mu^{-1} \int_{\bb R^d} \exp\tp{-f_\mu(y)} \d y\\
        &\ge Z_{\mu}^{-1}\int_{\bb R^d} \exp\tp{-f_\mu(x^*)-\frac{L}{2}\norm{y-x^*}^2} \d y\\
        &= p_\mu(x^*)\cdot\int_{\bb R^d} \exp\tp{-\frac{L}{2}\norm{y-x^*}^2}\d y.
    \end{align*}
    Since $\int_{\bb R^d} \exp\tp{-\frac{L}{2}\norm{y-x^*}^2}\d y = \tp{\frac{2\pi}{L}}^{\frac{d}{2}}$, we conclude the proof. 
\end{proof}

The second lemma shows that the function $\ol{f^{\le 2R}_{\pi}}$, our truncation for $f_\mu$, does not change the mass in the $\+O(R)$-ball much. 

\begin{lemma}\label{lem:Z_R-close-to-one}
    Assume $d\geq 3$. The following holds.
    \[
        Z_{\mu}^{-1}\int_{\+B_{2R}} \exp\tp{-\ol{f^{\le 2R}_{\pi}}(x)}\d x \le 1+\frac{\eps}{32}, \mbox{ and } Z_{\mu}^{-1}\int_{\+B_R} \exp\tp{-\ol{f^{\le 2R}_{\pi}}(x)}\d x \ge 1-\frac{\eps}{16}.
    \]
\end{lemma}
\begin{proof}
    Let $\+L =\set{x\in \+B_{2R}\cmid f_\mu(x)\ge h_1}$ be the set of points in $\+B_{2R}$ where the truncation occurs. Clearly
    \begin{align*}
        Z_{\mu}^{-1}\int_{\+B_{2R}} \exp\tp{-\ol{f^{\le 2R}_\pi}(x)}\d x
        &\leq Z_{\mu}^{-1}\int_{\+B_{2R}\setminus \+L} \exp\tp{-f_\mu(x)}\d x + Z_{\mu}^{-1}\int_{\+L} \exp\tp{-h_1}\d x\\
        &\le Z_{\mu}^{-1}\int_{\bb R^d} \exp\tp{-f_\mu(x)}\d x + Z_\mu^{-1}\cdot \!{vol}(\+B_{2R})\cdot \exp(-h_1)\\
        \mr{by \Cref{lem:mu-bound}, $Z_\mu^{-1}\exp\tp{-f^*} \le \tp{\frac{2\pi}{L}}^{-\frac{d}{2}}$}
        &\le 1+\frac{\eps}{4}\cdot (2\pi)^{-\frac{d}{2}}\\
        \mr{$d\geq 3$} 
        &\le 1+\frac{\eps}{32}.
    \end{align*}
    For the lower bound, we can calculate that
    \begin{align*}
        Z_\mu^{-1}\int_{\+B_R} \exp\tp{-\ol{f^{\le 2R}_\pi}(x)}\d x
        &\ge Z_\mu^{-1}\int_{\+B_R\setminus \+L} \exp\tp{-\ol{f^{\le 2R}_\pi}(x)}\d x\\
        \mr{$\ol{f^{\le 2R}_\pi}(x) = f_\mu(x)$ for $x\in\+B_{2R}\setminus\+L$} 
        &=Z_\mu^{-1}\int_{\+B_R\setminus\+L} \exp\tp{-f_\mu(x)} \d x\\
        &=Z_\mu^{-1}\int_{\+B_R} \exp\tp{-f_\mu(x)} \d x - Z_\mu^{-1}\int_{\+L} \exp\tp{-f_\mu(x)} \d x.
    \end{align*}
    By Markov's inequality,
    \begin{equation}\label{eqn:1st}
        Z_\mu^{-1}\int_{\+B_R} \exp\tp{-f_\mu(x)} \d x = \Pr[X\sim\mu]{X\in \+B_R}\ge 1-\frac{M}{R^2} = 1-\frac{\eps}{32}.
    \end{equation}
    By our definition of $h_1$, 
    \begin{equation}\label{eqn:2nd}
        Z_\mu^{-1}\int_{\+L} \exp\tp{-h_1}\le \!{vol}(\+B_{2R})\cdot Z_\mu^{-1}\exp\tp{-h_1}\le \frac{\eps}{4}\cdot (2\pi)^{-\frac{d}{2}}\le \frac{\eps}{32}.
    \end{equation}
    Combining~\eqref{eqn:1st} and~\eqref{eqn:2nd} finishes the proof.
\end{proof}

Recall that our definition for $f^{\le 2R}_{\pi}$ is an \emph{approximately normalized} $\ol{f^{\le 2R}_{\pi}}$ using our estimate $\wh Z_\mu$ for $Z_\mu$. The following lemma states that provided the estimate is accurate enough, $Z_\pi$ is close to $1$.

\begin{lemma}\label{lem:Zpi-close-to-one}
    Assume $d\geq 3$. It holds that
    \begin{itemize}
        \item $1-\frac{\eps}{16} \le Z_\pi \cdot \frac{\wh Z_\mu}{Z_\mu} \le 1+\frac{\eps}{16}$.
        \item $\frac12\le Z_\pi\le 4e^d$.
    \end{itemize}
\end{lemma}
\begin{proof}
    On the one hand, from \Cref{lem:Z_R-close-to-one},
    \[
        Z_{\pi} \geq \int_{\+B_R} \exp\tp{-f_\pi(x)} \d x = \frac{Z_\mu}{\wh Z_{\mu}} \cdot Z_\mu^{-1}\int_{\+B_R} \exp\tp{-\ol{f^{\le 2R}_\pi}(x)} \d x \geq \frac{Z_\mu}{\wh Z_{\mu}} \cdot \tp{1 - \frac{\eps}{16}}.
    \]
    On the other hand, 
    \begin{align*}
        Z_{\pi} &\leq \int_{\bb R^d\setminus \+B_R} e^{-f_\gamma(x)+\log\eps} \d x + \int_{\+B_{2R}} \exp\tp{-f^{\le 2R}_\pi(x)} \d x\\
        &\le \eps\Pr[X\sim \+N(0,\frac{M}{d\eps}\cdot \!{Id}_d)]{ \|X\|^2 \geq R^2} + \frac{Z_\mu}{\wh Z_{\mu}} \cdot Z_\mu^{-1}\int_{\+B_{2R}} e^{-\ol{f^{\le 2R}_\pi}(x)} \d x \\
        \mr{\Cref{lem:Z_R-close-to-one}}
        &\leq \frac{\eps}{32} + \frac{Z_\mu}{\wh Z_{\mu}} \cdot \tp{1+\frac{\eps}{32}}.
    \end{align*}
    The lemma then follows from \Cref{prop:Z-and-fmin}.
\end{proof} 

\subsubsection{Distance between $\pi$ and $\mu$}\label{sec:ub-closeness}

We now prove that the total variation distance between $\pi$ and $\mu$ is at most $\frac{\eps}{2}$. 

\begin{lemma} \label{lem:pi-mu-close}
    Assume $d\geq 3$. We have $\DTV(\pi,\mu)\le \frac{\eps}{2}$.
\end{lemma}
\begin{proof}
    We still let $\+L = \set{x\in \+B_{2R}\cmid f_{\mu}(x)\ge h_1}$ denote those points that have been truncated. Clearly
    \[
        \DTV(\pi,\mu) = \frac{1}{2}\Big(\underbrace{\int_{\bb R^d\setminus \+B_R}  \abs{p_\pi(x)-p_\mu(x)} \d x}_{\mbox{(a)}} + \underbrace{\int_{ \+B_R\setminus \+L }  \abs{p_\pi(x)-p_\mu(x)} \d x}_{\mbox{(b)}} + \underbrace{\int_{\+L\cap \+B_R}  \abs{p_\pi(x)-p_\mu(x)} \d x}_{\mbox{(c)}}\Big).
    \]
    We then bound terms (a), (b) and (c) respectively. For (a), we have
    \begin{align*}
        \mbox{(a)}
        &\le \int_{\bb R^d\setminus \+B_R} p_\mu(x) \d x + \int_{\bb R^d\setminus \+B_R} p_\pi(x)\d x\\
        &= \Pr[X\sim\mu]{\norm{X}^2>R^2} +\tp{1-Z_\pi^{-1} \int_{\+B_R} \exp\tp{-f^{\le 2R}_\pi(x)} \d x}\\
        \mr{\eqref{eqn:markov-mu} and definition of $f^{\le 2R}_\pi$}
        &\le \frac{\eps}{32} + 1-Z_{\pi}^{-1} \cdot \frac{Z_\mu}{\wh Z_\mu}\cdot Z_\mu^{-1}\cdot \int_{\+B_R} \exp\tp{-\ol{f^{\le 2R}_\pi}(x)} \d x\\
        \mr{\Cref{lem:Z_R-close-to-one}}
        &\le\frac{\eps}{32} + 1-\frac{Z_\mu}{\wh Z_\mu}\cdot Z_\pi^{-1}\tp{1-\frac{\eps}{16}}\\
        \mr{\Cref{lem:Zpi-close-to-one}}
        &\le \frac{\eps}{32}+\frac{\eps}{8} = \frac{5\eps}{32}.
    \end{align*}
    By our construction, for $x\in \+B_R\setminus\+L$, we have that $f_\pi(x) = f_{\mu}(x)+\log\wh Z_\mu$. Therefore, for the term (b), we have
    \begin{align*}
        \mbox{(b)}
        &=\int_{\+B_R\setminus\+L}\abs{Z_\mu^{-1}\exp\tp{-f_{\mu}(x)}-Z_\pi^{-1}\cdot \wh Z_\mu^{-1}\exp\tp{-f_{\mu}(x)}} \d x\\
        &=\abs{1-\frac{Z_\mu}{\wh Z_\mu}\cdot Z_\pi^{-1}} \cdot Z_\mu^{-1}\int_{\+B_R\setminus\+L} \exp\tp{-f_{\mu}(x)} \d x\\
        &\le \abs{1-\frac{Z_\mu}{\wh Z_\mu}\cdot Z_\pi^{-1}}\\
        \mr{\Cref{lem:Zpi-close-to-one}}
        &\le \frac{\eps}{8}.
    \end{align*}
    Finally, for the term (c), we have
    \begin{align*}
        \mbox{(c)}
        &\le \int_{\+L\cap \+B_R} p_\mu(x)\d x + \int_{\+L\cap \+B_R} p_\pi(x) \d x\\
        &\le Z_\mu^{-1}\int_{\+B_R} \exp\tp{-h_1}\d x + Z_\pi^{-1}\int_{\+L\cap \+B_R} \exp\tp{-h_1-\log \wh Z_\mu}\d x\\
        &\le \tp{1+\frac{Z_\mu}{\wh Z_\mu\cdot Z_\pi}}\cdot Z_\mu^{-1}\int_{\+B_R} \exp\tp{-h_1}\d x\\
        \mr{\Cref{lem:Zpi-close-to-one}}
        &\le 3\cdot\!{vol}(\+B_R)\cdot Z_\mu^{-1} \exp\tp{-h_1}\\
        \mr{Definition of $h_1$ and \Cref{lem:mu-bound}}
        &\le\frac{3\eps}{4}\cdot\tp{2\pi}^{-\frac{d}{2}} \le \frac{3\eps}{32}.
    \end{align*}
    In total, we have $\DTV(\pi,\mu) \le \frac{5\eps}{32}+\frac{\eps}{8}+\frac{3\eps}{32}<\frac{\eps}{2}$.
\end{proof}

\subsubsection{The \Poincare constant of $\pi$}\label{sec:ub-poincare}

In this section we bound the \Poincare constant of $\pi$. Recall that the density $p_\gamma$ of $\gamma\sim \+N\tp{0,\frac{M}{\eps d}\!{Id}_d}$ is $p_\gamma(x) = \exp\tp{-f_\gamma(x)}$ where $f_\gamma(x) = \frac{\eps d\norm{x}^2}{2M}+\frac{d}{2}\log\frac{2\pi M}{d\eps}$.  We will show that $p_\pi$ is close to $p_\gamma$ pointwise. 

\begin{lemma}\label{lem:fclose}
    Assume $d\geq 3$. For every $x\in \bb R^d$, $\abs{f_\gamma(x)-f_\pi(x)} \le \+O\tp{d\log\frac{LM}{d\eps}}$.
\end{lemma}
\begin{proof}
    Outside $\+B_{2R}$, we have
    \[
        \abs{f_\pi(x) - f_\gamma(x)} = \log\frac{1}{\eps} = \+O\tp{d\log\frac{LM}{d\eps}}.
    \]
    For $x\in \+B_{2R}$, or equivalently $\norm{x}^2\le \frac{128M}{\eps}$, 
    % $\abs{f_\gamma(x)} = \+O\tp{d\log\frac{LM}{d\eps}}$. 
    \[
        \frac{d}{2}\log\frac{2\pi M}{d\eps} \leq f_\gamma(x) \leq 64d + \frac{d}{2}\log\frac{2\pi M}{d\eps}.
    \]
    
    % It remains to show that $\abs{f_\pi(x)} = \+O\tp{d\log\frac{LM}{d\eps}}$ as well.
    It remains to bound $f_\pi(x)$ inside $\+B_{2R}$. Note that $f_\pi(x) = f^{\le 2R}_\pi(x)$ for $x\in \+B_R$ and $f_\pi(x)$ is an interpolation of $f^{\le 2R}_\pi(x)$ and $f_\gamma(x)-\log\eps$ for $x\in \+B_{2R}\setminus \+B_R$, we only need to bound $f^{\le 2R}_{\pi}$.
    % we only need to verify that $f^{\le 2R}_{\pi} = \+O\tp{d\log\frac{LM}{d\eps}}$ for $x\in \+B_{2R}$. 

    Recall that ${f^{\le 2R}_{\pi}(x)} = {\ol{f^{\le 2R}_{\pi}}(x)+\log \wh Z_\mu}= {\ol{f^{\le 2R}_{\pi}}(x)+\log Z_\mu}+{\log\frac{\wh Z_\mu}{Z_\mu}}$. By \Cref{prop:Z-and-fmin}, $-d -1 \leq \log\frac{\wh Z_\mu}{Z_\mu} \leq 0$.
    % we only need to bound $\abs{\ol{f^{\le 2R}_{\pi}}(x)+\log Z_\mu}$.
    By our construction, for all $x\in \+B_{2R}$, 
    \[
        f^*+\log Z_{\mu} \leq \ol{f^{\le 2R}_{\pi}}(x)+\log Z_{\mu} \leq h_2 + \log Z_{\mu}.
    \]
    From \Cref{lem:mu-bound}, $f^*+\log Z_{\mu} \geq \frac{d}{2}\log\frac{2\pi}{L}$. On the other hand, $h_2 + \log Z_{\mu} \leq f^*+\log Z_{\mu} + d + \log \!{vol}(\+B_{2R}) + \frac{d}{2}\log\frac{L^2M}{d\eps} + \log \frac{4}{\eps}$. Since
    \[
        \!{vol}(\+B_R)\cdot e^{-f^* - \log Z_{\mu}} = \int_{\+B_R} e^{-f^* - \log Z_{\mu}} \dd x \geq \Pr[X\sim \mu]{X\in \+B_R} \geq 1-\frac{\eps}{32}, 
    \]
    we have $f^*+\log Z_{\mu} \leq \log \!{vol}(\+B_R) + 1$. Therefore, 
    \begin{align*}
        \ol{f^{\le 2R}_{\pi}}(x)+\log Z_{\mu} \leq h_2 + \log Z_{\mu} &\leq \log \!{vol}(\+B_R) + \log \!{vol}(\+B_{2R}) + d+1 + \frac{d}{2}\log\frac{L^2M}{d\eps} + \log \frac{4}{\eps} \\
        \mr{\Cref{prop:Gamma}}
        &\leq \log \frac{4}{\eps} + d+1 + \frac{d}{2}\log \frac{ L^2 M}{d\eps} + \frac{d}{2}\log \frac{64e\pi M}{d\eps} + \frac{d}{2}\log \frac{4\cdot 64e\pi M}{d\eps} \\
        & \leq \log \frac{4}{\eps} + d+ 1 + d\log \frac{8 L M}{d\eps} + \frac{d}{2}\log \frac{4\cdot 64 e^2\pi^2 M}{d\eps}.
        % & = \+O\tp{d\log\frac{LM}{d\eps}}.
    \end{align*}

    Combining the above calculations, for $x\in \+B_{2R}$,
    \begin{align*}
        f_\pi(x) - f_\gamma(x) &\leq f^{\le 2R}_{\pi}(x) - f_\gamma(x) + \log \frac{1}{\eps}\\
        &\leq \ol{f^{\le 2R}_{\pi}}(x)+\log Z_{\mu} - f_\gamma(x) + \log\frac{\wh Z_\mu}{Z_\mu} + \log \frac{1}{\eps}\\
        &\leq \log \frac{4}{\eps} + d+ 1 + d\log \frac{8 L M}{d\eps} + \frac{d}{2}\log \frac{4\cdot 64 e^2\pi^2 M}{d\eps} - \frac{d}{2}\log\frac{2\pi M}{d\eps} + \log \frac{1}{\eps}\\
        &= \+O\tp{d\log \frac{LM}{d\eps}}
    \end{align*}
    and 
    \begin{align*}
        f_\pi(x) - f_\gamma(x) &\geq f^{\le 2R}_{\pi}(x) - f_\gamma(x) \\
        &= \ol{f^{\le 2R}_{\pi}}(x)+\log Z_{\mu} - f_\gamma(x) + \log\frac{\wh Z_\mu}{Z_\mu} \\
        &\geq \frac{d}{2}\log\frac{2\pi}{L} - 64d - \frac{d}{2}\log\frac{2\pi M}{d\eps} - d - 1\\
        &= - \+O\tp{d\log \frac{LM}{d\eps}}.
    \end{align*}
\end{proof}

Since $\abs{\log p_\pi(x) - \log p_\gamma(x)} = \abs{f_\gamma(x)-f_\pi(x)-\log Z_\pi} \le \abs{f_\gamma(x)-f_\pi(x)}+\abs{\log Z_\pi}$, by \Cref{lem:Zpi-close-to-one}, we have the following corollary.

\begin{corollary}\label{cor:pclose}
    For every $x\in\bb R^d$, $\tp{\frac{LM}{d\eps}}^{-\+O\tp{d}}\le \frac{p_\pi(x)}{p_\gamma(x)} \le \tp{\frac{LM}{d\eps}}^{\+O\tp{d}}$.
\end{corollary}

Then we come to the bound for the \Poincare constant of $\pi$. 

\begin{lemma}\label{lem:pi-PI}
    $C_{\!{PI}}(\pi) = \frac{2d\eps}{M}\cdot \tp{\frac{LM}{d\eps}}^{-\+O(d)}$.
\end{lemma}

\begin{proof}
    By definition, $C_{\!{PI}}(\pi) = \inf_{h\colon\bb R^d \to \bb R} \frac{\E[\pi]{\|\grad h\|^2}}{\Var[\pi]{h^2}}$. For each $h\colon\bb R^d \to \bb R$,
    \begin{align*}
        \frac{\E[\pi]{\|\grad h\|^2}}{\Var[\pi]{h}} 
        & = \frac{2\int_{\bb R^d} \|\grad h(x)\|^2 p_\pi(x) \dd x}{\int_{\bb R^d\times \bb R^d} \tp{h(x)-h(y)}^2 p_\pi(x)p_\pi(y) \dd x \dd y} \\
        \mr{\Cref{cor:pclose}}
        &\geq \tp{\frac{LM}{d\eps}}^{\+O(d)}\cdot \frac{2\int_{\bb R^d} \|\grad h(x)\|^2 p_\gamma(x) \dd x}{\int_{\bb R^d\times \bb R^d} \tp{h(x)-h(y)}^2 p_\gamma(x)p_\gamma(y) \dd x \dd y}\\
        &= \tp{\frac{LM}{d\eps}}^{\+O(d)}\cdot \frac{\E[\gamma]{\|\grad h\|^2}}{\Var[\gamma]{h}}.
    \end{align*}
    Since $C_{\!{PI}}(\gamma) = \inf_{h\colon\bb R^d \to \bb R} \frac{\E[\gamma]{\|\grad h\|^2}}{\Var[\gamma]{h}} = \frac{2d\eps}{M}$ \cite{HE76}, we know that $C_{\!{PI}}(\pi) \geq \frac{2d\eps}{M}\cdot \tp{\frac{LM}{d\eps}}^{\+O(d)}$.

\end{proof}

\subsubsection{The smoothness and first moment of $\pi$} \label{sec:ub-smooth}

In this section, we prove the smoothness property and bound the first moment of $\pi$. These properties are important in the algorithm to sample from $\pi$ in \Cref{sec:proof-of-ub}. Remember that we assumed $\grad f_\mu(0) = 0$. 

\begin{lemma}\label{lem:smooth1}
    % We have $\grad \ol{f^{\le 2R}_\pi}(0)=0$ and for each $x\in \+B_{2R}$, $\| \grad^2 \ol{f^{\le 2R}_\pi}(x) \| =  \+O\tp{\frac{L^3R^4}{\tp{h_2-h_1}^2}}$ and $\| \grad \ol{f^{\le 2R}_\pi}(x) \| = \+O\tp{\frac{L^2R^3}{h_2-h_1}}$.  
    We have $\grad \ol{f^{\le 2R}_\pi}(0)=0$ and for any $x,y \in \+B_{2R}$, $\| \grad \ol{f^{\le 2R}_\pi}(x) \| = \+O\tp{\frac{L^2R^3}{h_2-h_1}}$ and $\| \grad \ol{f^{\le 2R}_\pi}(x) - \grad \ol{f^{\le 2R}_\pi}(y) \| =  \+O\tp{\frac{L^3R^4}{\tp{h_2-h_1}^2}}\cdot \|x-y\|$.  
\end{lemma}
\begin{proof}
        By the definition of $\ol{f^{\le 2R}_\pi}$, for each $x,y\in \+B_{2R}$, direct calculation gives
        \begin{align*}
            \grad \ol{f^{\le 2R}_\pi}(x) &= \grad \mathfrak{g}_{[h_1,h_2]}(x) 
            \cdot \tp{f_\mu(x) - h_2} + \grad f_\mu(x) \cdot \mathfrak{g}_{[h_1,h_2]}(x)
            % ,\\
            % \grad^2 \ol{f^{\le 2R}_\pi}(x) &= \grad^2 \mathfrak{g}_{[h_1,h_2]}(x)\cdot \tp{f_\mu(x) - h_2} + \mathfrak{g}_{[h_1,h_2]}(x)\cdot \grad^2 f_\mu(x)\\
            % &\quad\quad +\grad \mathfrak{g}_{[h_1,h_2]}(x) \grad f_\mu(x)^{\top} + \grad f_\mu(x)\cdot \grad \mathfrak{g}_{[h_1,h_2]}(x)^{\top}
        \end{align*}
        and 
        \begin{align}
            \grad \ol{f^{\le 2R}_\pi}(x) - \grad \ol{f^{\le 2R}_\pi}(y) &= \tp{\grad \mathfrak{g}_{[h_1,h_2]}(x) - \grad \mathfrak{g}_{[h_1,h_2]}(y) }\cdot \tp{f_\mu(x) - h_2} + \grad \mathfrak{g}_{[h_1,h_2]}(y) \cdot (f_\mu(x)-f_{\mu}(y)) \notag \\
            &\quad\quad +\mathfrak{g}_{[h_1,h_2]}(x) \tp{\grad f_\mu(x) - \grad f_\mu(y)} + \grad f_\mu(y)\cdot \tp{\mathfrak{g}_{[h_1,h_2]}(x) - \mathfrak{g}_{[h_1,h_2]}(y)}. \notag
        \end{align}
        By the definition of $\mathfrak{g}_{[h_1,h_2]}$, we have
        \[
            \grad \mathfrak{g}_{[h_1,h_2]}(x) = \frac{-\grad f_\mu(x)}{h_2-h_1} \cdot q_{\!{mol}}'\tp{\frac{ h_2 - f_\mu(x) }{ h_2 - h_1 }}.
        \]
        % and
        % \[
        %     \grad^2 \mathfrak{g}_{[h_1,h_2]}(x) = \frac{\grad f_\mu(x)\cdot \grad f_\mu(x)^{\top}}{(h_2-h_1)^2} \cdot q_{\!{mol}}''\tp{\frac{ h_2 - f_\mu(x) }{ h_2 - h_1 }} - \frac{\grad^2 f_{\mu}(x)}{h_2-h_1} \cdot q_{\!{mol}}'\tp{\frac{ h_2 - f_\mu(x) }{ h_2 - h_1 }}.
        % \]
        It is easy to see $\grad \ol{f^{\le 2R}_\pi}(0)=0$. Since $f$ is $L$-smooth and $\grad f_\mu(0)=0$, for $x\in \+B_{2R}$, $\|\grad f_\mu(x) \| \leq L\|x\| \leq 2LR$ and $\|\grad f_\mu(x) - \grad f_\mu(y) \|\leq L\|x-y\|$. Recall that $q'_{\!{mol}}$ is always $O(1)$. We have $\|\grad \mathfrak{g}_{[h_1,h_2]}(x)\| =\+O\tp{\frac{LR}{h_2-h_1}}$ and 
        \begin{align*}
            \|\grad \mathfrak{g}_{[h_1,h_2]}(x) - \grad \mathfrak{g}_{[h_1,h_2]}(y) \| &\leq \norm{\frac{\grad f_\mu(x)-\grad f_\mu(y)}{h_2-h_1}}\cdot q_{\!{mol}}'\tp{\frac{ h_2 - f_\mu(x) }{ h_2 - h_1 }} \\
            &\quad + \frac{\|\grad f_{\mu}(y)\|}{h_2-h_1} \cdot \abs{q_{\!{mol}}'\tp{\frac{ h_2 - f_\mu(x) }{ h_2 - h_1 }} - q_{\!{mol}}'\tp{\frac{ h_2 - f_\mu(y) }{ h_2 - h_1 }}}\\
            &\leq \+O\tp{\frac{L}{h_2-h_1}}\cdot \|x-y\| + \+O\tp{\frac{LR}{h_2-h_1}\cdot \frac{LR}{h_2-h_1}}\cdot \|x-y\|\\
            &= \+O\tp{\frac{L^2R^2}{(h_2-h_1)^2}}\cdot \|x-y\|.
        \end{align*}
        Consequently, $\abs{f_\mu(x)-f_\mu(y)} \leq \+O(LR)\|x-y\|$ and $\abs{ \mathfrak{g}_{[h_1,h_2]}(x) - \mathfrak{g}_{[h_1,h_2]}(y) } \leq \+O\tp{\frac{LR}{h_2-h_1}}\cdot \|x-y\|$.
        
         Let $x^* = \arg\min_{x\in \+B_{2R}} f_\mu(x)$. We have for any $x\in \+B_{2R}$
         \begin{equation}\label{eq:x*}
             f^*\leq f_\mu(x) \leq f_\mu(x^*) + \grad f_\mu(x^*)\cdot (x-x^*) + \frac{L}{2}\cdot \|x-x^*\|^2 \leq f^* + 16LR^2. 
         \end{equation}
        Recall the definition of $h_2 =  \wh f^* + \log \!{vol}(\+B_{2R}) + \frac{d}{2}\log L + \log \frac{4}{\eps} + \frac{d}{2}\log\frac{LM}{d\eps}$. According to \Cref{prop:Z-and-fmin} and~\eqref{eq:x*}, for $x\in \+B_{2R}$, 
        \[
             f_\mu(x)-h_2\geq - \frac{d}{2}\log \frac{LM}{d\eps} - \log\frac{4}{\eps} + \log\Gamma\tp{\frac{d}{2}+1} - \frac{d}{2}\log \frac{4\cdot 32\pi LM}{\eps} - d,
        \]
        and
        \[
             f_\mu(x)-h_2\leq - \frac{d}{2}\log \frac{LM}{d\eps} - \log\frac{4}{\eps} + \log\Gamma\tp{\frac{d}{2}+1} - \frac{d}{2}\log \frac{4\cdot 32\pi LM}{\eps} + 16LR^2.
        \]    
        Therefore, $\abs{f_\mu(x)-h_2}= \+O(LR^2)$.

        Combining all above, we have
        \[
             \| \grad \ol{f^{\le 2R}_\pi}(x) \| \leq 2LR \cdot \+O\tp{\frac{\abs{f_\mu(x)-h_2}}{h_2-h_1}}  + 2LR = \+O\tp{\frac{L^2R^3}{h_2-h_1}},
        \]
        and 
        \begin{align*}
            \|\grad \ol{f^{\le 2R}_\pi}(x) - \grad \ol{f^{\le 2R}_\pi}(y)\| &\leq \+O\tp{\frac{L^3R^4}{\tp{h_2-h_1}^2} + \frac{L^2R^2}{h_2-h_1} + L } = \+O\tp{\frac{L^3R^4}{\tp{h_2-h_1}^2}}\cdot \|x-y\|.
        \end{align*}
\end{proof}
% \begin{proof}
%         By the definition of $\ol{f^{\le 2R}_\pi}$, for each $x,y\in \+B_{2R}$, direct calculation gives
%         \begin{align*}
%             \grad \ol{f^{\le 2R}_\pi}(x) &= \grad \mathfrak{g}_{[h_1,h_2]}(x) 
%             \cdot \tp{f_\mu(x) - h_2} + \grad f_\mu(x) \cdot \mathfrak{g}_{[h_1,h_2]}(x)
%             % ,\\
%             % \grad^2 \ol{f^{\le 2R}_\pi}(x) &= \grad^2 \mathfrak{g}_{[h_1,h_2]}(x)\cdot \tp{f_\mu(x) - h_2} + \mathfrak{g}_{[h_1,h_2]}(x)\cdot \grad^2 f_\mu(x)\\
%             % &\quad\quad +\grad \mathfrak{g}_{[h_1,h_2]}(x) \grad f_\mu(x)^{\top} + \grad f_\mu(x)\cdot \grad \mathfrak{g}_{[h_1,h_2]}(x)^{\top}
%         \end{align*}
%         and 
%         \begin{align*}
%             \grad \ol{f^{\le 2R}_\pi}(x) - \grad \ol{f^{\le 2R}_\pi}(y) &= \tp{\grad \mathfrak{g}_{[h_1,h_2]}(x) - \grad \mathfrak{g}_{[h_1,h_2]}(y) }\cdot \tp{f_\mu(x) - h_2} + \mathfrak{g}_{[h_1,h_2]}(x)\cdot \grad^2 f_\mu(x)\\
%             &\quad\quad +\grad \mathfrak{g}_{[h_1,h_2]}(x) \grad f_\mu(x)^{\top} + \grad f_\mu(x)\cdot \grad \mathfrak{g}_{[h_1,h_2]}(x)^{\top}
%         \end{align*}
%         By the definition of $\mathfrak{g}_{[h_1,h_2]}$, we have
%         \[
%             \grad \mathfrak{g}_{[h_1,h_2]}(x) = \frac{-\grad f_\mu(x)}{h_2-h_1} \cdot q_{\!{mol}}'\tp{\frac{ h_2 - f_\mu(x) }{ h_2 - h_1 }},
%         \]
%         and
%         \[
%             \grad^2 \mathfrak{g}_{[h_1,h_2]}(x) = \frac{\grad f_\mu(x)\cdot \grad f_\mu(x)^{\top}}{(h_2-h_1)^2} \cdot q_{\!{mol}}''\tp{\frac{ h_2 - f_\mu(x) }{ h_2 - h_1 }} - \frac{\grad^2 f_{\mu}(x)}{h_2-h_1} \cdot q_{\!{mol}}'\tp{\frac{ h_2 - f_\mu(x) }{ h_2 - h_1 }}.
%         \]
%         It is easy to see $\grad \ol{f^{\le 2R}_\pi}(0)=0$. Since $f$ is $L$-smooth and $\grad f_\mu(0)=0$, for $x\in \+B_{2R}$, $\|\grad f_\mu(x) \| \leq L\|x\| \leq 2LR$ and $\|\grad^2 f_\mu(x) \|\leq L$. 
        
%          Let $x^* = \arg\min_{x\in \+B_{2R}} f_\mu(x)$. We have for any $x\in \+B_{2R}$
%          \begin{equation}\label{eq:x*}
%              f^*\leq f_\mu(x) \leq f_\mu(x^*) + \grad f_\mu(x^*)\cdot (x-x^*) + \frac{L}{2}\cdot \|x-x^*\|^2 \leq f^* + 16LR^2. 
%          \end{equation}
%         Recall the definition of $h_2 =  \wh f^* + \log \!{vol}(\+B_{2R}) + \frac{d}{2}\log L + \log \frac{4}{\eps} + \frac{d}{2}\log\frac{LM}{d\eps}$. According to \Cref{prop:Z-and-fmin} and~\eqref{eq:x*}, for $x\in \+B_{2R}$, 
%         \[
%              f_\mu(x)-h_2\geq - \frac{d}{2}\log \frac{LM}{d\eps} - \log\frac{4}{\eps} + \log\Gamma\tp{\frac{d}{2}+1} - \frac{d}{2}\log \frac{4\cdot 32\pi LM}{\eps} - d,
%         \]
%         and
%         \[
%              f_\mu(x)-h_2\leq - \frac{d}{2}\log \frac{LM}{d\eps} - \log\frac{4}{\eps} + \log\Gamma\tp{\frac{d}{2}+1} - \frac{d}{2}\log \frac{4\cdot 32\pi LM}{\eps} + 16LR^2.
%         \]    
%         Combining all above, we have
%         \[
%              \| \grad \ol{f^{\le 2R}_\pi}(x) \| \leq 2LR \cdot \+O\tp{\frac{\abs{f_\mu(x)-h_2}}{h_2-h_1}}  + 2LR = \+O\tp{\frac{L^2R^3}{h_2-h_1}},
%         \]
%         and 
%         \begin{align*}
%              \| \grad^2 \ol{f^{\le 2R}_\pi}(x) \| &\leq \+O\tp{\frac{(2LR)^2\cdot \abs{f_\mu(x) - h_2}}{(h_2-h_1)^2}} + \+O\tp{\frac{(LR)^2}{h_2-h_1}} = \+O\tp{\frac{L^3R^4}{\tp{h_2-h_1}^2}}.
%         \end{align*}
% \end{proof}
    

\begin{lemma}\label{lem:f2smooth}
    % The function $f_\pi$ is $\+O\tp{\frac{L^3R^4}{\tp{h_2-h_1}^2} + \frac{d\eps}{M}}$-smooth.
    The function $f_\pi$ is $\+O\tp{\frac{L^3R^4}{\tp{h_2-h_1}^2}}$-smooth.
\end{lemma}
\begin{proof}
    We divide $\bb R^d$ into three parts, $\+B_R$, $\+B_{2R}\setminus \+B_{R}$ and $\bb R^d \setminus \+B_{2R}$. Since our construct guarantees that $f_\pi$ and $\grad f_{\pi}$ are continuous functions, to prove the smoothness of $f_{\pi}$, we only need to bound $\|\grad f_{\pi}(x) - \grad f_{\pi}(y)\|$ for those $x,y$ from the same part. For $x,y$ from different parts, for example, if $x\in \+B_R$ and $y\in \+B_{2R}\setminus \+B_{R}$, we can find a $z$ at the intersection of this two parts such that $\|x-y\|=\|x-z\|+\|z-y\|$ and bounding $\|\grad f_{\pi}(x) - \grad f_{\pi}(y)\|$ can be transformed to bounding $\|\grad f_{\pi}(x) - \grad f_{\pi}(z)\|$ and $\|\grad f_{\pi}(z) - \grad f_{\pi}(y)\|$ respectively.

    By construction, for $x,y\in \+B_R$, $\norm{\grad f_\pi(x)-\grad f_\pi(y)}$ is bounded by \Cref{lem:smooth1}. For $x,y\in \bb R^d \setminus \+B_{2R}$, we know $\|\grad f_\pi(x)-\grad f_\pi(y)\|\leq \frac{\eps d}{M}\cdot \|x-y\|$ and $\frac{\eps d}{M} = \+O\tp{\frac{L^3R^4}{\tp{h_2-h_1}^2}}$. It remains to deal with those $x,y\in \+B_{2R}\setminus \+B_{R}$.

    For $x,y\in \+B_{2R}\setminus \+B_{R}$,
    \[
        \grad f_\pi(x) = - \grad f^{\le 2R}_\pi(x) \cdot \mathfrak{g}_{[R,2R]}(x) - \grad \mathfrak{g}_{[R,2R]}(x) \cdot f^{\le 2R}_\pi(x) + \grad \mathfrak{g}_{[R,2R]}(x) \cdot (f_{\gamma}(x)-\log \eps) + \grad f_{\gamma}(x) \cdot  \mathfrak{g}_{[R,2R]}(x)
    \] 
    and
    \begin{align}
        \|\grad f_\pi(x) - \grad f_\pi(y)\| &\leq  \norm{\grad f^{\le 2R}_\pi(x) - \grad f^{\le 2R}_\pi(y)}\cdot \abs{\mathfrak{g}_{[R,2R]}(x)} + \norm{\grad f^{\le 2R}_\pi(y)}\cdot \abs{\mathfrak{g}_{[R,2R]}(x) - \mathfrak{g}_{[R,2R]}(y)} \notag \\
        &\quad + \abs{f^{\le 2R}_\pi(x) - f^{\le 2R}_\pi(y)}\cdot \|\grad \mathfrak{g}_{[R,2R]}(x)\|  + \norm{\grad \mathfrak{g}_{[R,2R]}(x)}\cdot \abs{f_{\gamma}(x) - f_{\gamma}(y)} \notag  \\
        &\quad + \norm{\grad f_{\gamma}(x) - \grad f_{\gamma}(y)}\cdot \abs{\mathfrak{g}_{[R,2R]}(x)} + \norm{\grad f_{\gamma}(y)}\cdot \abs{\mathfrak{g}_{[R,2R]}(x) - \mathfrak{g}_{[R,2R]}(y)}.\notag  \\
        &\quad + \norm{\grad \mathfrak{g}_{[R,2R]}(x) - \grad \mathfrak{g}_{[R,2R]}(y)} \cdot \abs{f^{\le 2R}_\pi(y) -f_{\gamma}(y)+\log \eps} . \label{eq:grad2}
    \end{align}
    For the first term in \Cref{eq:grad2}, we know from \Cref{lem:smooth1} and the fact $\mathfrak{g}_{[R,2R]}(x)=\+O(1)$ that $\norm{\grad f^{\le 2R}_\pi(x) - \grad f^{\le 2R}_\pi(y)}\cdot \abs{\mathfrak{g}_{[R,2R]}(x)} = \+O\tp{\frac{L^3R^4}{\tp{h_2-h_1}^2}}\cdot \|x-y\|$. For the fifth term, similarly, we have $\norm{\grad f_{\gamma}(x) - \grad f_{\gamma}(y)}\cdot \abs{\mathfrak{g}_{[R,2R]}(x)} = \+O\tp{\frac{\eps d}{M}}\cdot \|x-y\|$.

    By the definition of $\mathfrak{g}_{[R,2R]}$, we have $\grad \mathfrak{g}_{[R,2R]}(x) = \frac{2x}{(2R)^2 - R^2} \cdot q'_{\!{mol}}\tp{\frac{\|x\|^2 - R^2}{(2R)^2 - R^2}} = \+O\tp{\frac{1}{R}}$ for $x\in \+B_{2R}$. Therefore, we can bound the second term in \Cref{eq:grad2} by $\norm{\grad f^{\le 2R}_\pi(y)}\cdot \abs{\mathfrak{g}_{[R,2R]}(x) - \mathfrak{g}_{[R,2R]}(y)} = \+O\tp{\frac{L^2R^2}{h_2-h_1}}\cdot \|x-y\|$ and also bound the third term by $\abs{f^{\le 2R}_\pi(x) - f^{\le 2R}_\pi(y)}\cdot \|\grad \mathfrak{g}_{[R,2R]}(x)\| = \+O\tp{\frac{L^2R^2}{h_2-h_1}}\cdot \|x-y\|$. 

    Since $\grad f_{\gamma}(x) = \frac{\eps d x}{M}$, for any $x\in \+B_{2R}$, $\norm{\grad f_{\gamma}(x)} = \+O\tp{\frac{\eps d R }{M}}= \+O\tp{\frac{d }{R}}$. Then both the fourth and the sixth term in \Cref{eq:grad2} can be bounded by $\+O\tp{\frac{d }{R^2}}\cdot \|x-y\|$.

    Still by the definition of $\mathfrak{g}_{[R,2R]}$, for $x\in \+B_{2R}$,
    \begin{align*}
        \|\grad \mathfrak{g}_{[R,2R]}(x) - \grad \mathfrak{g}_{[R,2R]}(y)\| &= \frac{2\|x-y\|}{(2R)^2 - R^2} \cdot \abs{q'_{\!{mol}}\tp{\frac{\|x\|^2 - R^2}{(2R)^2 - R^2}}} \\
        &\quad + \frac{2\|y\|}{(2R)^2 - R^2} \cdot \abs{q'_{\!{mol}}\tp{\frac{\|x\|^2 - R^2}{(2R)^2 - R^2}} - q'_{\!{mol}}\tp{\frac{\|y\|^2 - R^2}{(2R)^2 - R^2}}} \\
        &\leq \+O\tp{\frac{1}{R^2}}\cdot \|x-y\|.
    \end{align*}
    Therefore, from \Cref{lem:fclose} the last term in \Cref{eq:grad2} can be bounded by  $\+O\tp{\frac{d\log \frac{LM}{d\eps}}{R^2}}\cdot \|x-y\|$.

    % Combining the above equation and \Cref{lem:fclose}, we have $\norm{\grad^2 \mathfrak{g}_{[R,2R]}(x) \cdot \tp{f_\gamma(x) - f^{\le 2R}_\pi(x) - \log\eps}} \leq \+O\tp{\frac{d\eps}{M}\cdot \log\frac{LM}{d\eps}}$. From \Cref{lem:smooth1}, we have $\norm{\grad \mathfrak{g}_{[R,2R]}(x) \grad f^{\le 2R}_\pi(x)^\top},\norm{\grad f^{\le 2R}_\pi(x)\grad \mathfrak{g}_{[R,2R]}(x)^\top} \leq \+O\tp{\frac{L^2R^2}{h_2-h_1}}$. For $\norm{\grad \mathfrak{g}_{[R,2R]}(x) \grad f_\gamma(x)^\top}$ and $\norm{ \grad f_\gamma(x)\grad \mathfrak{g}_{[R,2R]}(x)^\top}$, they can be bounded by $\+O\tp{\frac{d\eps}{M}}$.
    In total, $\grad f_\pi(x)$ is $ \+O\tp{\frac{L^3R^4}{\tp{h_2-h_1}^2}}$-Lipschitz.
\end{proof}
% \begin{proof}
%     By construction, for $x,y\in \+B_R$, $\norm{f_\pi(x)-f_\pi(y)}$ is bounded by \Cref{lem:smooth1}. For $x\in \bb R^d \setminus \+B_{2R}$, we know $f_\pi$ is $\frac{\eps d}{M}$-smooth and $\frac{\eps d}{M} = \+O\tp{\frac{L^3R^4}{\tp{h_2-h_1}^2}}$. It remains to deal with those $x\in \+B_{2R}\setminus \+B_{R}$.

%     For $x\in \+B_{2R}\setminus \+B_{R}$, 
%     \begin{align*}
%         \grad^2 f_\pi(x) &= \grad^2 \mathfrak{g}_{[R,2R]}(x) \cdot \tp{f_\gamma(x) - f^{\le 2R}_\pi(x)-\log\eps} + \mathfrak{g}_{[R,2R]}(x) \cdot \grad^2 f_\gamma(x) - \mathfrak{g}_{[R,2R]}(x)\cdot \grad^2 f^{\le 2R}_{\pi}(x) \\
%         &\quad - \grad \mathfrak{g}_{[R,2R]}(x) \grad f^{\le 2R}_\pi(x)^\top - \grad f^{\le 2R}_\pi(x) \grad \mathfrak{g}_{[R,2R]}(x)^\top\\
%         &\quad + \grad \mathfrak{g}_{[R,2R]}(x) \grad f_\gamma(x)^{\top} + \grad f_\gamma(x) \grad\mathfrak{g}_{[R,2R]}(x)^\top\\
%         &\quad +\grad^2 f^{\le 2R}_\pi.
%     \end{align*}
%     It is easy to know that $\norm{\mathfrak{g}_{[R,2R]}(x) \cdot \grad^2 f_\gamma(x)} \leq \frac{\eps d}{M} = \+O\tp{\frac{L^3R^4}{\tp{h_2-h_1}^2}}$. From \Cref{lem:smooth1}, $\norm{\mathfrak{g}_{[R,2R]}(x)\cdot \grad^2 f^{\le 2R}_{\pi}(x)} \leq \norm{\grad^2 f^{\le 2R}_{\pi}(x)} = \+O\tp{\frac{L^3R^4}{\tp{h_2-h_1}^2}}$.

%     By the definition of $\mathfrak{g}_{[R,2R]}$, we have $\grad \mathfrak{g}_{[R,2R]}(x) = \frac{2x}{(2R)^2 - R^2} \cdot q'_{\!{mol}}\tp{\frac{\|x\|^2 - R^2}{(2R)^2 - R^2}}$ and 
%     \[
%         \grad^2 \mathfrak{g}_{[R,2R]}(x) = \frac{2\!{Id}_d}{(2R)^2 - R^2} \cdot q'_{\!{mol}}\tp{\frac{\|x\|^2 - R^2}{(2R)^2 - R^2}} + \frac{4xx^\top}{\tp{(2R)^2 - R^2}^2} \cdot q''_{\!{mol}}\tp{\frac{\|x\|^2 - R^2}{(2R)^2 - R^2}}.
%     \]
%     Combining the above equation and \Cref{lem:fclose}, we have $\norm{\grad^2 \mathfrak{g}_{[R,2R]}(x) \cdot \tp{f_\gamma(x) - f^{\le 2R}_\pi(x) - \log\eps}} \leq \+O\tp{\frac{d\eps}{M}\cdot \log\frac{LM}{d\eps}}$. From \Cref{lem:smooth1}, we have $\norm{\grad \mathfrak{g}_{[R,2R]}(x) \grad f^{\le 2R}_\pi(x)^\top},\norm{\grad f^{\le 2R}_\pi(x)\grad \mathfrak{g}_{[R,2R]}(x)^\top} \leq \+O\tp{\frac{L^2R^2}{h_2-h_1}}$. For $\norm{\grad \mathfrak{g}_{[R,2R]}(x) \grad f_\gamma(x)^\top}$ and $\norm{ \grad f_\gamma(x)\grad \mathfrak{g}_{[R,2R]}(x)^\top}$, they can be bounded by $\+O\tp{\frac{d\eps}{M}}$.
%     In total, we have $\norm{\grad^2 f_\pi(x)} \leq \+O\tp{\frac{L^3R^4}{\tp{h_2-h_1}^2}}$.
% \end{proof}

The following result is a corollary of the above lemmas.
\begin{corollary}\label{coro:gap}
    The function $f_{\pi}$ satisfies $f_{\pi}(0) - \min_{x\in \bb R^d} f_{\pi}(x) = \+O\tp{\frac{L^3R^6}{\tp{h_2-h_1}^2}}$.
\end{corollary}
\begin{proof}
    By the definition of $f_\pi$, it is increasing as $\|x\|$ increases outside $\+B_{2R}$. Therefore, $f_{\pi}(0) - \min_{x\in \bb R^d} f_{\pi}(x) = f_{\pi}(0) - \min_{x\in \+B_{2R}} f_{\pi}(x)$. Since $\grad f_{\pi}(0) = \grad \ol{f^{\le 2R}_\pi}(0)=0$ and $f_{\pi}$ is $\+O\tp{\frac{L^3R^4}{\tp{h_2-h_1}^2}}$-smooth from \Cref{lem:f2smooth}, for arbitrary $x\in \+B_{2R}$, $f_{\pi}(0) - f_{\pi}(x) \leq \+O\tp{\frac{L^3R^6}{\tp{h_2-h_1}^2}}$ and consequently, $f_{\pi}(0) - \min_{x\in \bb R^d} f_{\pi}(x) = \+O\tp{\frac{L^3R^6}{\tp{h_2-h_1}^2}}$.
\end{proof}

Then we bound the first moment of $\pi$.
\begin{lemma}\label{lem:f2moment}
    The first moment of $\pi$ is bounded by $\+O(\sqrt{M})$.
\end{lemma}
\begin{proof}
    By the definition of $\pi$ and $f_\gamma$, we have
    \begin{align*}
        \E[X \sim \pi]{\|X\|}^2 &\leq \E[X \sim \pi]{\|X\|^2} \\
        & \le \frac{1}{Z_{\pi}} \tp{\int_{\bb R^d} \|x\|^2\cdot \exp\tp{-f_\gamma(x)+\log \eps} \dd x + \frac{Z_{\mu}}{\wh Z_{\mu}}\cdot \frac{1}{Z_{\mu}}\cdot \int_{\+B_{2R}} \|x\|^2\cdot \exp\tp{-\ol{f^{\le 2R}_\pi}(x)} \dd x } \\
        \mr{\Cref{lem:Zpi-close-to-one}}
        & \leq \frac{M}{Z_{\pi}} + \frac{2}{Z_{\mu}} \tp{\int_{\bb R^d} \|x\|^2\cdot \exp\tp{-f_\mu(x)} \dd x + \int_{\+B_{2R}} \|x\|^2\cdot \exp\tp{-h_1} \dd x } \\
        &\leq \frac{M}{ Z_{\pi}} + 2M + \frac{2}{Z_\mu}\cdot \int_{\+B_{2R}} (2R)^2\cdot \exp\tp{-h_1} \dd x  \\
        \mr{Definition of $h_1$} &\leq \frac{M}{Z_{\pi}} + 2M + \frac{8R^2}{Z_\mu}\cdot \exp\tp{-\wh f^* - \frac{d}{2}\log L - \log \frac{4}{\eps}} \\
        \mr{\Cref{lem:mu-bound}}
        &\leq \frac{M}{Z_{\pi}} + 2M + 2\eps R^2 \cdot (2\pi)^{-\frac{d}{2}}\\
        \mr{\Cref{lem:Zpi-close-to-one}}
        &= \+O\tp{M}.
    \end{align*}
\end{proof}

% \begin{lemma}\label{lem:f2moment}
%     The second moment of $\pi$ is bounded by $\+O(M)$.
% \end{lemma}
% \begin{proof}
%     By the definition of $\pi$ and $f_\gamma$, we have
%     \begin{align*}
%         \E[X \sim \pi]{\|X\|^2} 
%         & \le \frac{1}{Z_{\pi}} \tp{\int_{\bb R^d} \|x\|^2\cdot \exp\tp{-f_\gamma(x)+\log \eps} \dd x + \frac{Z_{\mu}}{\wh Z_{\mu}}\cdot \frac{1}{Z_{\mu}}\cdot \int_{\+B_{2R}} \|x\|^2\cdot \exp\tp{-\ol{f^{\le 2R}_\pi}(x)} \dd x } \\
%         \mr{\Cref{prop:Z-and-fmin}}
%         & \leq \frac{M}{Z_{\pi}} + \frac{2}{Z_{\mu}} \tp{\int_{\bb R^d} \|x\|^2\cdot \exp\tp{-f_\mu(x)} \dd x + \int_{\+B_{2R}} \|x\|^2\cdot \exp\tp{-h_1} \dd x } \\
%         &\leq \frac{M}{ Z_{\pi}} + 2M + \frac{2}{Z_\mu}\cdot \int_{\+B_{2R}} (2R)^2\cdot \exp\tp{-h_1} \dd x  \\
%         \mr{Definition of $h_1$} &\leq \frac{M}{Z_{\pi}} + 2M + \frac{8R^2}{Z_\mu}\cdot \exp\tp{-\wh f^* - \frac{d}{2}\log L - \log \frac{4}{\eps}} \\
%         \mr{\Cref{lem:mu-bound}}
%         &\leq \frac{M}{Z_{\pi}} + 2M + 2\eps R^2 \cdot (2\pi)^{-\frac{d}{2}}\\
%         \mr{\Cref{lem:Zpi-close-to-one}}
%         &= \+O\tp{M}.
%     \end{align*}
% \end{proof}


\subsection{Estimate $f^*$ and $Z_\mu$}\label{sec:estimate-of-pi}

In this section, we prove \Cref{prop:Z-and-fmin}, namely to show that how to get estimators $\wh f^*$ and $\wh Z_{\mu}$ satisfying
\[
    f^* \leq \wh f^* \leq f^* + d \quad \mbox{and}\quad 
    \frac12 e^{-d}\le \frac{\wh Z_{\mu}}{Z_{\mu}} \le 1.
\]
The idea is to discretize $\bb R^d$ into cubes of side length $\ell$ and use information in each cube to construct the estimation. Let $\ell = \frac{1}{64}\sqrt{\frac{d\eps}{L^2 M}}$ and $R_0 = 2R + \sqrt{d \cdot \ell^2} $. Let $\+Z_{R_0} = \+B_{R_0}\cap \ell \bb Z^d$ be the collection of vertices of the cubes in $\+B_{R_0}$.
%\set{x\in \+B_{R_0}:\ x / \ell\in \bb Z^d}$.

\begin{lemma}\label{lem:cubes}
    There are at most $\tp{\frac{2^{10}\cdot 5 LM}{d\eps}}^{d}$ cubes with side length $\ell$ in $\+B_{R_0}$ whose vertices are all in $\+Z_{R_0}$. 
\end{lemma}
\begin{proof}
    From \Cref{cor:dballvolbound}, $\!{vol}(\+B_{R_0}) = \frac{\tp{\pi R_0^2}^{\frac{d}{2}}}{\Gamma\tp{ \frac{d}{2}+1} } \leq \tp{\frac{2\pi e R_0^2}{d}}^{\frac{d}{2}}$. The volume of a cube with side length $\ell$ is $\tp{\frac{d\eps}{2^{12}\cdot L^2 M}}^{\frac{d}{2}}$. For the cubes whose vertices are all in $\+Z_{R_0}$, the overlapping area is $0$. Therefore, the total number of such cubes is no larger than 
    \begin{align*}
        \tp{2\pi e R_0^2 \cdot \frac{2^{12}\cdot L^2 M}{\eps d^2}}^{\frac{d}{2}} &= \tp{2\pi e \cdot \frac{2^{12}\cdot L^2 M}{\eps d^2}\cdot \tp{4R^2 + d\ell^2 + 4R\cdot \sqrt{d\ell^2}}}^{\frac{d}{2}}\\
        &= \tp{2\pi e \cdot \frac{2^{12}\cdot L^2 M}{\eps d^2}\cdot \tp{\frac{4\cdot 32 M}{\eps} + \frac{d^2\eps}{64^2\cdot L^2 M} + \frac{\sqrt{2}d}{4L}}}^{\frac{d}{2}}\\
        &= \tp{2\pi e \cdot \tp{\frac{2^{19}L^2 M^2}{\eps^2 d^2} + \frac{1}{4} + \frac{2^8\sqrt{2} LM}{\eps d}}}^{\frac{d}{2}}\\
        &\leq \tp{\frac{2^{10}\cdot 5 LM}{d\eps}}^{d}.
    \end{align*}
    % $\tp{2\pi e R_0^2 \cdot \frac{8\cdot 128 L^2 M}{\eps d^2}}^{\frac{d}{2}} = \tp{ 2\pi e \tp{\frac{8\cdot 128^2 M^2 L^2}{\eps^2 d^2} + 1 }}^{\frac{d}{2}} \leq \tp{\frac{2^{10}\cdot 5 LM}{d\eps}}^{d}$.
\end{proof}

We consider those cubes with side length $\ell$ and all vertices in $\+Z_{R_0}$. Let $n$ be the total number of such cubes. From \Cref{lem:cubes}, $n\leq \tp{\frac{2^{10}\cdot 5 LM}{d\eps}}^{d}$. Denote these cubes as $C_1,C_2,\dots,C_n$ and let $v_1,v_2,\dots, v_n$ be the center of these cubes. We first show that these cubes well cover the ball $\+B_{2R}$.
\begin{lemma}\label{lem:cubecover}
    For each $x\in \+B_{2R}$, there exists $i\in[n]$ such that $x\in C_i$.
    % $\|x - v_i\|\leq 2\sqrt{d}\ell$.
\end{lemma}
\begin{proof}
    % Assume in contradiction that we cannot find such a $v_i$ for some point $y\in \+B_{R'}$. 
    For each point $x\in \+B_{2R}$, we define $\ol x\in \bb R^d$ as
    $
        \forall j\in[d],\  \ol x(j) = \begin{cases}
            \lfloor \frac{x(j)}{\ell} \rfloor \cdot \ell, &\mbox{ if } x(j)\geq 0\\
            \lceil \frac{x(j)}{\ell} \rceil\cdot \ell, &\mbox{ if } x(j)< 0
        \end{cases}.
    $
    Consider the following cube
    \[
        C_x = \set{y\in \bb R^d:\ \forall j \in [d], y(j)\in \begin{cases}
            [\ol x(j), \ol x(j) + \ell], &\mbox{ if }\ol x(j)\ge 0\\
            [\ol x(j) - \ell, \ol x(j)], &\mbox{ if }\ol x(j)< 0
        \end{cases}}.
    \]
    It is obvious that $x\in C_x$ and for each vertex $y\in C_x$,
    \begin{align*}
        \|y\|^2 &\leq \sum_{j=1}^d \tp{\ol x(j)^2 + \ell^2 + 2\ell \cdot \abs{\ol x(j)}} \\
        &= \|\ol x\|^2 + d\ell^2 + 2\sqrt{d\ell^2}\cdot \frac{\sum_{j=1}^n \abs{\ol x(j)}}{\sqrt{d}}\\
        \mr{Cauchy-Schwartz inequality}&\leq \|\ol x\|^2 + d\ell^2 + 2\sqrt{d\ell^2}\cdot \|\ol x\|
        \\ 
        &\leq 4R^2 + d\ell^2 + 2\sqrt{d\ell^2}\cdot 2R
        \\ 
        &\leq R_0^2.
    \end{align*}
    Therefore $y\in \+Z_{R_0}$ and there exists $i\in [n]$ such that $C_i=C_x$.

    % We have $\ol x\in \+B_{2R} \cap \+Z_{R_0}$ and $\|\ol x - x\|^2 \leq d\ell^2$. Consider the following cube
    % \[
    %     C_x = \set{y\in \bb R^d:\ \forall j \in [d], y(j)\in \begin{cases}
    %         [\ol x(j), \ol x(j) + \ell], &\mbox{ if }\ol x(j)< 0\\
    %         [\ol x(j) - \ell, \ol x(j)], &\mbox{ if }\ol x(j)\geq 0
    %     \end{cases}}.
    % \]
    % It is obvious that there exists $i\in [n]$ such that $C_i=C_x$ and $\|x - v_i\|\leq \|x - \ol x\| + \|\ol x - v_i\| \leq 2\sqrt{d}\ell$.
\end{proof}

In the following, we will write $v_x$ for $v_i$ where $i$ is the unique $i$ such that $x\in C_i$.
%For each $x\in \+B_{2R}$, we denote $v_x$ as the vector $v_i$ where $i$ is the smallest index in $[n]$ such that $x\in C_i$. 
% For $x\in \+B_{2R} \setminus \tp{\bigcup_{j\in[n]} C_j}$, let $v_i$ be the closest point to $x$ in $\set{v_j}_{j\in[n]}$. 
% We denote the unique $v_i$ associated with each $x$ as $v_x$. 
%Let $\+J = \set{i\in[n]: \exists x\in \+B_{2R}, v_x = v_i}$. 
Let $\+J\defeq \set{i\in [n]\cmid C_i\cap \+B_{2R}\ne\emptyset}$. To estimate $f^*$ and $Z_{\mu}$, we query the value of each $f_\mu(v_i)$ and assign
\begin{equation}\label{eqn:hat}
    \wh f^* = \min_{i\in \+J} f_\mu(v_i) + \frac{d}{2},\quad \wh Z_{\mu} = \sum_{i=1}^n \!{vol}(C_i) \cdot \exp\tp{-f_\mu(v_i) - \frac{d}{2}}.
\end{equation}

%\mn{Here we add a multiplier $\frac{1}{\eps}$ to $\hat Z_{\mu}$ artificially to make $\hat Z_{\mu}$ large enough.}
Then the query complexity to determine $\wh f^*$ and $\wh Z_{\mu}$ is at most $\tp{\frac{2^{10}\cdot 5 LM}{d\eps}}^{d}$. In the following, we show that our construction satisfies the accurancy requirement, namely that



% The following lemma shows that such constructions indeed satisfy \Cref{assump:Z-and-fmin}.
 \begin{lemma}\label{lem:estimate}
     The construction of $\wh f^*$ and $\wh Z_{\mu}$ in~\eqref{eqn:hat} satisfies
     \[
    f^* \leq \wh f^* \leq f^* + d \quad \mbox{and}\quad 
    \frac12 e^{-d}\le \frac{\wh Z_{\mu}}{Z_{\mu}} \le 1.
    \]
 \end{lemma}
 \begin{proof}
    Since the function $f_\mu$ is $L$-smooth and $\grad f_\mu(0)=0$, for each $x\in \+B_{R_0}$, $\|\grad f_\mu(x)\|\leq L\|x\| \leq LR_0$. From \Cref{lem:cubecover}, we always have $\|x - v_x\| \leq \sqrt{d}\ell$. Then by the definition of $L$-smooth,
    \begin{align*}
        f_\mu(v_x) &\leq f_\mu(x) + \grad f_\mu(x)^{\top} (x-v_x) + \frac{L}{2} \|x - v_x\|^2\\
        & \leq f_\mu(x) + \|\grad f_\mu(x)\| \cdot \|x-v_x\| + \frac{L}{2} \|x - v_x\|^2 \\
        &\leq f_\mu(x) + LR_0\cdot \sqrt{d}\ell + \frac{L}{2}\cdot d\ell^2 \leq f_\mu(x) + \frac{d}{2}
    \end{align*}
    % \htodo{The universal constants here might be inaccurate.}
    and similarly
    \begin{align*}
        f_\mu(x) & \leq f_\mu(v_x) + \grad f_\mu(v_x)^{\top} (v_x - x) + \frac{L}{2} \|x - v_x\|^2 \leq f_\mu(v_x) + \frac{d}{2}.
    \end{align*}
    Therefore, $f^* = \min_{x\in \+B_{2R}} f_\mu(x) \leq \min_{x\in \+B_{2R}} f_\mu(v_x) + \frac{d}{2} = \wh f^*$ and $f^* \geq \min_{x\in \+B_{2R}} f_\mu(v_x) - \frac{d}{2} = \wh f^* - d$.  From the same calculation, we know for each $x\in C_i$, $f_\mu(v_i) - \frac{d}{2} \leq f_\mu(x)\leq f_\mu(v_i) + \frac{d}{2}$. For $\wh Z_{\mu}$, we have
    \[
        Z_{\mu} \geq \sum_{i=1}^n \int_{C_i} \exp\tp{-f_\mu(x)} \d x \geq \sum_{i=1}^n \!{vol}(C_i)\cdot \exp\tp{-f_\mu(v_i) - \frac{d}{2}} = \wh Z_{\mu}.
    \]
    On the other hand, since $\+B_R\subseteq \bigcup_{i\in[n]} C_i$, we have $\int_{\bb R^d \setminus \tp{\bigcup_{i\in[n]} C_i}} \exp\tp{-f_\mu(x)} \d x < \int_{\bigcup_{i\in[n]} C_i} \exp\tp{-f_\mu(x)} \d x$. Therefore,
    \[
        Z_{\mu}\leq 2\sum_{i=1}^n \int_{C_i} \exp\tp{-f_\mu(x)} \d x \leq 2\sum_{i=1}^n \!{vol}(C_i)\cdot \exp\tp{-f_\mu(v_i) + \frac{d}{2}} = 2e^d\cdot \wh Z_{\mu}.
    \]
\end{proof}

\subsection{Proof of \Cref{thm:main-ub}} \label{sec:proof-of-ub}

From previous sections we know that the distribution $\pi$ is $\+O\tp{\frac{L^3R^4}{\tp{h_2-h_1}^2}}$-log-smooth (\Cref{lem:f2smooth}), has its first moment bounded by $\+O(\sqrt{M})$ (\Cref{lem:f2moment}), satisfies $\DTV\tp{\pi,\mu}= \frac{\eps}{2}$ (\Cref{lem:pi-mu-close}) and satisfies $C_{\!{PI}}\ge \frac{2d\eps}{M}\cdot \tp{\frac{LM}{d\eps}}^{-\+O(d)}$ (\Cref{lem:pi-PI}). Moreover, we can query $f_\pi(x)$ and $\grad f_\pi(x)$ efficiently, provided query access to $f_\mu$ and $\grad f_\mu$. 

Therefore, we can use the algorithm in~\cite{BCE+22} to sample from $\pi$ (see also~\cite[Chapter 11]{Che24}). Let $N$ be the total steps and $h$ be the step size. To sample from a target distribution $\nu\propto e^{-f}$, their algorithm acts as follows:
\begin{itemize}
    \item[1.] Pick a time $t_0\in[0,Nh]$ uniformly at random.
    \item[2.] Let $k_0$ be the largest integer such that $k_0 h<t$. For each $t<t_0$ and $k\leq k_0$, the process evolves as 
    \begin{equation}
        X_t = X_{kh} - (t-kh) \grad f(X_{kh}) + \sqrt{2}(B_t-B_{kh}), \label{eq:LMC2}
    \end{equation} 
    where $\set{B_t}_{t\geq 0}$ is the standard Brownian motion.
    \item[3.] Output $X_{t_0}$.
\end{itemize}

\begin{theorem}[A direct consequence of Corollary 8 in \cite{BCE+22}]\label{thm:LMCforPI}
    Let $\set{\mu_t}_{t\geq 0}$ denote the law of the interpolation \Cref{eq:LMC2} of LMC. Assume the potential function $f$ is $\+L$-smooth and the target distribution $\nu\propto e^{-f}$ satisfies the \Poincare inequality with constant $\alpha>0$. If $\!{KL}(\mu_0 \| \nu)\leq K_0$, choosing step size $h=\frac{\sqrt{K_0}}{2\+L\sqrt{dN}}$, then for $N\geq \max\set{\frac{32^2 \alpha^{-2} \+L^2 dK_0}{\delta^4}, \frac{9K_0}{d}}$ and $\ol \mu_{Nh}\defeq \frac{\int_0^{Nh} \mu_t \d t}{Nh}$, $\DTV(\ol \mu_{Nh},\nu)\leq \delta$.
\end{theorem}

To get a convergence guarantee for our target distribution $\pi$, it remains to find an initial distribution $\mu_0$ such that $\!{KL}(\mu_0 \| \pi)$ is bounded. \Cref{lem:f2smooth} shows that $f_{\pi}$ is $\+L$-smooth for $\+L=\+O\tp{\frac{L^3R^4}{\tp{h_2-h_1}^2}}$. By choosing $\mu_0$ as $\+N\tp{0, \frac{\!{Id}_d}{2\+L}}$, we can bound $\!{KL}(\mu_0 \| \pi)$ using the following lemma.
\begin{lemma}[A direct corollary of Lemma 32 in \cite{CEL+24}]\label{lem:initial}
    Suppose $\grad f(0)= 0$ and $f$ is $\+L$-smooth. Let $m = \E[X\sim \nu]{\|X\|}$ be the first moment of $\nu \propto e^{-f}$. Then for $\mu_0=\+N\tp{0, \frac{\!{Id}_d}{2\+L}}$,
    \[
        \log \tp{\sup \frac{\d \mu_0}{\d \nu}} \leq 2+\+L + f(0)-\min_{x\in \bb R^d} f(x) + \frac{d}{2}\log \tp{4m^2\+L}.
    \]
\end{lemma}

Combining \Cref{coro:gap}, \Cref{lem:f2moment} and \Cref{lem:initial}, $\!{KL}(\mu_0\|\pi)$ can be bounded by $\!{poly}(d,M,L,\eps^{-1})$. Therefore, we can choose $\delta = \frac{\eps}{2}$ in \Cref{thm:LMCforPI} to sample from a distribution whose total variation distance is at most $\frac{\eps}{2}$ to $\pi$ with $\!{poly}(L,M, d,\eps^{-1})\cdot \tp{\frac{LM}{\eps d}}^{\+O(d)}$ queries to $f_\mu$ and $\grad f_{\mu}$.

% Therefore, we can use the algorithm in~\cite{VW19} to sample from $\pi$.

% \begin{theorem}[A direct consequence of Theorem 1 in~\cite{VW19}]\label{thm:ULA}
%     Let $\nu\propto \exp\tp{-f}$ satisfy log-Sobolev inequality with constant $\alpha>0$ and is $L$-smooth.  Assume $\grad f(0)=0$ and $f(0)=0$. The unadjusted Langevin algorithm outputs a sample from a distribution $\tilde \nu$ with $D_{\!{KL}}(\tilde \nu,\nu)\le \delta$ after $\tilde\Theta\tp{\frac{L^2d}{\alpha^2 n}}$ iterations. 
% \end{theorem}\ctodo{So maybe we shoudl assume $f(0)=0$ as well?}

% Note that we can assume without loss of generality that $f_\pi(0)=0$. We can pick $\delta = \frac{\eps^2}{2}$ and apply \Cref{thm:ULA} to sample from a distribution whose KL divergence is at most $\delta$ to $\pi$ with $\!{poly}(LM, d,\eps^{-1})\cdot \tp{\frac{LM}{\eps d}}^{\+O(d)}$ queries to $\grad f_\mu$ and $\grad f_{\mu}$. Then its total variation distance to $\pi$ is at most $\frac{\eps}{2}$ by Pinsker's inequality.

% Let $V:\bb R^d\to R$ be the potential function and distribution $\pi\propto e^{-V}$.  


% The following theorem is a direct corollary of Theorem 7 in \cite{CEL+24}.

% To Define $\hat V(x) = V(x) + \frac{\gamma}{2}\tp{\|x\|-R}^2_{+}$, where $R$ is chosen to satisfy $R= 2\*m = 2\int_{\bb R^d}\|x\| \d \pi(x)$ and $\gamma = \frac{1}{768 kh}$ ($h$ is the step size and $k$ is total steps).
% \begin{theorem}[A consequence of Theorem 7, Lemma 32 and 33 in \cite{CEL+24}]\label{thm:LMCforPI}
% Assume potential function $f$ is $L$-smooth, $\grad f(0)=0$, and the distribution $\nu\propto e^{-f(x)}$ satisfy a \Poincare inequality with constant $C_{\!{PI}}$. Let $\*m\defeq \int_{\bb R}\|x\| \dd \nu(x)$. Assume $\eps^{-1}, L, \*m,\ C_{PI}^{-1}\geq 1$. Then the LMC with a proper step size satisfy $R_2(p_{k}\|\pi)\leq \delta$ after
% \[
%     k = C_{\!{PI}}^2 \cdot \!{poly}\tp{d,m,L,\delta^{-1}, f(0)-\min_{x\in \bb R^d} f(x)}
% \]
% % \[
% %     k=\Theta\tp{\frac{dL^2 C_{\!{PI}}^2}{\eps}\cdot  R_3(p_0\|\pi)^2 \cdot \max\set{1, \frac{m}{d}, \frac{\sqrt{R_2(p_0\|\hat \pi)}}{d}}}
% % \]
% steps.
% \end{theorem}

% Assume the start distribution of LMC is $p_0$. 

% For $q=2$, we know that $R_2(\nu\| \pi) = \ln\tp{1+\chi^2(\nu\|\pi)}$ for any probability measure $\pi$ and $\nu$.
% We can find $p_0$ such that $R_3(p_0\|\pi)=\tilde O(d)$ and $R_2(p_0\|\hat \pi)=\tilde O(d)$. See discussions in Appendix A in \cite{CEL+24}.

% We can use $R_2(\nu\| \pi)$ to upper bound $D_{TV}(\nu,\pi)$. This is because 
% \begin{align}
%     4D_{TV}(\nu,\pi)^2 &= \tp{\int_{\bb R^d} \abs{\pi(x) - \nu(x)} \dd x}^2 \notag \\
%     \mr{Cauchy-Schwartz inequality}
%     &\leq \int_{\bb R^d} \tp{\frac{{\pi(x)-\nu(x)}}{\sqrt{\pi(x)}}}^2 \dd x \cdot \int_{\bb R^d} \pi(x) \dd x \notag\\
%     &= e^{R_2(\nu\|\pi)} - 1.\label{eq:TV}
% \end{align}
% Therefore, we can choose $\delta = 2\eps$ in \Cref{thm:LMCforPI} to sample from a distribution whose $R_2$ divergence is at most $\delta$ to $\pi$ with $\!{poly}(L,M, d,\eps^{-1})\cdot \tp{\frac{LM}{\eps d}}^{\+O(d)}$ queries to $\grad f_\mu$ and $\grad f_{\mu}$. Then its total variation distance to $\pi$ is at most $\frac{\eps}{2}$ since $2\eps \leq \log (1+\eps)$ for $\eps\in(0,1/32)$.


%\ctodo{Remember to show that each step of the construction can be done efficiently.}
\section{Experiments}\label{sec_exp}
%\hp{Accelerating IM simulation~\cite{tang2015influence}}

% \begin{itemize}
%     \item 6.1. Problem setting of three COPs, including the general model and three specific CO problems 
%     \item 6.2. Experiment Setting (hyperparameters, details of training, evaluation, and test) 写在appendix里吧
%     \item 6.3. Performance analysis 这个要占半页
% \end{itemize}

%\hp{need to think of a way to compress these tables / visuals.} 

%\hp{\cancel{Baselines}; hyperparamters; \cancel{metrics}; etc.}

With theoretical guarantees on the existence and convergence of NE for ACCES games, we are also interested in how our proposed algorithm CCDO-RL works empirically. To evaluate this, we conduct experiments of CCDO-RL on three distinct ACCES game instances introduced in Section \ref{sub_exp_ins} and analyze the performance of CCDO-RL in Section \ref{sub_train_eval}. Section 6.2.1 aims to empirically demonstrate the convergence (Figures \ref{fig_exploit_20} and \ref{fig_exploit_50}) of the algorithm CCDO-RL over realistic CO problems, and show its consistency with Theorem \ref{CCDOA}. Section 6.2.2 intends to show the average reward (to seen training graphs) as well as the generalizability (to unseen test graphs) of the combinatorial player in real-world ACCES games (shown in Tables \ref{tab_aver}, and \ref{tab_gene}).

\subsection{Three Instances of ACCES Games} \label{sub_exp_ins}
% \hp{This para does not make much sense. Need to follow the framework in the Preliminaries section.}
% For combinatorial optimization problems in real-world applications, situations are more complicated and intractable due to changeable environmental or physical parameters. The form of parameter sets is very crucial because different types have different solvability and computation complexity. Forms of parameter sets mainly contain discrete sets, interval sets \cite{buchheim2018robust} like polyhedral and ellipsoid, probability distributions \cite{carlsson2018wasserstein}, and variable functions \cite{krause2008robust}.

% In reality, these parameters are often impacted by some common factors, such as conditions of weather, transportation, and individual personalities. \cite{kalimeris2019robust} proposed an assumption that real instances (e.g. demands in CVRP, coverages in CSP) 
%Considering affected or attacked COPs, the real instance $\{\theta_{i}\}$ always relied on the estimated value $\{\hat{\theta}_{i}$\} and the variation determined by independent factors $\{g_{i}\}$ and environment/physical parameters/attacker actions $\{\eta\}$. The concrete parameter influence model is stated as follows:

We consider a certain COP which is parameterized with $\{\theta_{i}\}$, where $i$ is the index of nodes (such as a target in security games) -- e.g., such parameters can be interpreted as attack probability of targets.
%coverage radius, customer's demands, or attack probability of targets. 
In real-world applications, we often need to estimate such parameters before solving the COPs. Unfortunately, the estimation $\{\hat{\theta}_{i}\}$ often bears a gap to the true value $\{\theta_{i}\}$, which derives from e.g. environment (aleatoric) uncertainty, model (epistemic) uncertainty, or an attacker trying to manipulate the defender's utility. We use a generic model to formulate this gap:
\begin{equation}\label{linrob}
    \theta_{i} = \hat{\theta}_{i} + y \cdot \tau_{i},
\end{equation}
where $y$ represents the strategy of the nature/attacker, $\tau_{i}$ is the environment factors like weather and transportation conditions, or human subjective factors like the preference of the attacker. 
Such abstraction can represent a wide range of ACCES games, such as facility location covering problems \cite{an2020battery, TIRKOLAEE2020340}, CVRP \cite{vehiclerouting.ch8,dinh2018exact, FLORIO20231081}, security patrolling (OP) \citep{xu2021robust}, and influence maximization problem \cite{kalimeris2019robust}. We describe three instances of ACCES games based on the model (\ref{linrob}).%Based on this model (\ref{linrob}), we focus on three combinatorial optimization problems with attacks or environmental/physical influence.

% \hp{Hard to follow. We should point out what are the two players, what are X, Y, u etc}

\textbf{Adversarial Covering Salesman Problem (ACSP):} In a map of cities, every city $i$ has a coverage $\theta_{i}$. A salesman finds the shortest path such that all cities are visited or covered, with $\theta_{i}$ influenced by physical factors $\tau_i$ and transportation parameters $y$ based on Eq.(\ref{linrob}). The salesman is Player 1 where $X$ consists of the feasible paths of the salesman. Nature is Player 2 with $Y$ = $[0, 1]^K \ni y, K \in \mathbb{N}$. The utility function of Player 1 $u$ is the opposite of the total traveling distance.

\textbf{Adversarial Capacitated Vehicle Routing Problem (ACVRP):} A vehicle with a constrained capacity of goods finds the shortest path under the worst case with the $i_{th}$ customer's demand $\theta_i$ changed by environmental factors $\tau_i$ and weather parameter $y$ on Eq.(\ref{linrob}). The vehicle is Player 1 where $X$ is the set of the feasible path $x$. Nature is Player 2 where $Y$ is $[0, 1]^K \ni y, K \in \mathbb{N}$. The utility function of Player 1  $u$ is the opposite of total delivery distance satisfying all the demands of customers.


\textbf{Patrolling Game (PG):} The patrolling game is described in the introduction.

For all the problem instances, we run our algorithm on two problem sizes: 20 nodes and 50 nodes. The detailed description and problem parameters of the three game instances are in Appendix \ref{app_ex_para_set}.

% Similarly, in the vehicle route problem (VRP), conditions with correlated parameters arouse broad attention from scholars \cite{vehiclerouting.ch8,dinh2018exact,FLORIO20231081}. \cite{dinh2018exact} considered the demand correlation by geographical proximity of nodes, described by some independent random variables in the fractional form. \cite{FLORIO20231081} utilized 'external factors' to stand for unknown covariates affecting all demands and presented a Bayesian model to learn correlations. Further more, about IM problems, \cite{kalimeris2019robust} combined node features and uncertain hyperparameters to fit the influence probability on each edge.

% \subsection{Training CCDO-RL}

% For all the problems, CCDO-RL adopts the REINFORCE algorithm with an attention-based encoder-decoder framework \cite{kool2018attention} (used as an inductive graph representation component) to learn a (generalizable) COP solver for one player (protagonist), and PPO \cite{schulman2017proximal} to train a policy for the other player (adversary) whose strategy space is continuous. CCDO-RL is trained with 50 epochs on a set of 10,000 graphs (with 20 or 50 nodes). The hyperparameters of CCDO-RL are specified in Appendix \ref{app_ex_para_set} (Table \ref{tab_hyper_ccdorl}). Our code is included as supplementary material for ease of reproduction. 
% % \hp{need to specify hyperparas}

\subsection{Performance of CCDO-RL}\label{sub_train_eval}

Two aspects are evaluated for the performance of CCDO-RL, i.e., i) Convergence to NE (Section \ref{sub_per_conver}) exploring whether CCDO-RL can compute the NE, and ii) Protagonist policy's average reward and generalizability (Section \ref{sub_per_rob}). Generalizability refers to the ability of RL models trained on previously seen graphs (problem instances), to perform well on a new set of unseen test graphs. The model’s usability is enhanced by generalizability, rather than focusing solely on the average reward, which is a critical motivation in the literature on RL for COPs \citep{khalil2017learning, kool2018attention}.

For all the problems, CCDO-RL adopts the REINFORCE algorithm with an attention-based encoder-decoder framework \citep{kool2018attention} (used as an inductive graph representation component) to learn a generalizable COP solver for Player 1 (protagonist), and PPO to train a policy for Player 2 (adversary) whose strategy space is continuous. CCDO-RL is trained on a set of 10,000 graphs (with 20 or 50 nodes). The hyperparameters of CCDO-RL are specified in Appendix \ref{app_ex_para_set} (Table \ref{tab_hyper_ccdorl}). Our code is included as supplementary material and will be open-sourced for ease of reproduction. 

% \textbf{Training.} For all the problems, CCDO-RL adopts the REINFORCE algorithm with attention-based encoder-decoder framework \cite{kool2018attention} (used as an inductive graph representation component) to learn a (generalizable) COP solver for one player (protagonist), and PPO \cite{schulman2017proximal} to train a policy for the other player (adversary) whose strategy space is continuous. CCDO-RL is trained with 50 epochs on a set of 10,000 graphs (with 20 or 50 nodes). 

% \hp{We should first present results about convergence as it is mostly aligned with the theory.}

\subsubsection{Convergence to NE} \label{sub_per_conver}

Exploitability is a common metric to describe the closeness to true NE by calculating the sum of performance distances between each new best response and subgame NE, i.e. $\sum_{i=1,2} U(\pi_{i,k}^{br}, \sigma_{-i,k}) - U(\sigma)$ in the general two-player game. Since our game is zero-sum, the calculation is as follows:
\begin{equation*}
   \text{Exploitability}(\sigma) = \max_{\pi_1 \in \Sigma_1} U(\pi_1, \sigma_{2}) - \min_{\pi_2 \in \Sigma_2} U(\sigma_1, \pi_2).
\end{equation*}
From Figure \ref{fig_exploit_20}, we can see that CCDO-RL can converge to approximate NE in 25 iterations or less (in the PG setting), reaching 0.05 in ACSP, 0.10 in ACVRP, and 0.03 in PG with 20 nodes. Similar results are observed in problems with 50 nodes (see Figure \ref{fig_exploit_50} in Appendix \ref{app_exp}). These results validate the effectiveness of CCDO-RL in finding the NE for various types of games.

%Similarly, the exploitability of three COPs in 50 nodes is provided in the appendix \ref{app_exp}.
\vspace{-\baselineskip}
\begin{figure}[htbp]
	\centering
    \subfigure[ACSP20]{
    \label{csp20_nashconv}
    \includegraphics[scale=0.20]{Figures/nashconv_log_csp20_sm_7.eps}
    }
    \subfigure[ACVRP20]{
    \label{cvrp20_nashconv}%文中引用该图片代号
    \includegraphics[scale=0.20]{Figures/nashconv_log_svrp20_sm_7.eps}
    }
    \subfigure[PG20]{
    \label{opsa20_nashconv}
    \includegraphics[scale=0.20]{Figures/nashconv_log_pg20_sm_7.eps}
    }
    \caption{Exploitability curve of CCDO-RL on three games of 20 nodes}
    \label{fig_exploit_20}
\end{figure}
\vspace{-\baselineskip}
\subsubsection{Average reward and Generalizability of Combinatorial player} \label{sub_per_rob}
% \subsubsection{Robustness and Generalizability of Protagonist Policy} \label{sub_per_rob}
%\hp{CCDO-RL being better in these following metrics is only kind of a by-product.}

% \textbf{Evaluation.} The learned policies are then tested on 200 graphs, where 100 of them are randomly selected from the 10,000 training graphs, and the other 100 are unseen graphs. 
% We use two metrics to evaluate the performance of different policies for the protagonist player: \textbf{Average proportional loss} $R-$ describes the policy overfitting degree \citep{lanctot2017unified}; \textbf{Reward} evaluates the performance of the protagonist with the adversary under three COPs.  
% \begin{eqnarray}
%         &R- = (\hat{D} - \hat{O}) / \hat{D}.
% \end{eqnarray}
% in which $\hat{D}$ is the mean value of the diagonals and $\hat{O}$ is the mean value of the off-diagonals in the payoff matrix provided in the Appendix \ref{app_exp}.

% Because the protagonist policy is trained against a powerful adversary under our ACCES game setting, the obtained policy is naturally robust against adversarial perturbations. This subsection sheds a bit of light on this perspective and quantifies the extent of robustness of CCDO-RL as well as the ability of RL to generalize to unseen test graphs.

\textbf{Evaluation.} The learned policies are tested on 200 graphs, with 100 being randomly selected from the 10,000 training graphs (to show the average reward), and the other 100 being unseen graphs (to test policy generalization). We evaluate the performance of the protagonist with the adversary under three COPs. For each COP, the performance is considered both on the 20-node and 50-node map.
% We use two metrics to evaluate the performance of different policies for the protagonist player: \textbf{Average proportional loss} $R-$ describes the policy overfitting degree \citep{lanctot2017unified}; \textbf{Reward} evaluates the performance of the protagonist with the adversary under three COPs.

\textbf{Baselines.} There are heuristic algorithms for each game instance (Heuristic in Table \ref{tab_aver} and \ref{tab_gene}) and a single-player RL algorithm. For ACVRP, we adopt the Tabu Search algorithm (Tabu) \citep{li2020improved} as the heuristic algorithm, which is widely applied in the routing problem. For ACSP, the common benchmark local search algorithm, LS2 \citep{golden2012generalized}, is used. For PG, we choose the greedy algorithm as the baseline. The "RL against Stoc" algorithm in Tables \ref{tab_aver} and \ref{tab_gene} is identical to the protagonist model in CCDO-RL but trained in environments with stochastic adversarial perturbations.

% \textbf{Baselines.} There are a heuristic algorithms for each game instance {\color{red} (Heuristic mentioned in the Table \ref{tab_aver} and \ref{tab_gene})} and a single-player RL algorithm. For ACVRP, we adopt the Clarke-Wright (CW) algorithm \citep{pichpibul2013heuristic} and the Tabu Search algorithm (Tabu) \citep{li2020improved} as heuristics, which are applied widely in the routing problem. For ACSP, two common benchmark local search algorithms, LS1 and LS2 \citep{golden2012generalized}, are used. For PG, we choose a local search algorithm \citep{vansteenwegen2009iterated} and the greedy algorithm as the heuristic baselines. {\color{red} The "RL  against Stoc" algorithm referred to Tables \ref{tab_aver} and \ref{tab_gene}} is identical to the protagonist model in CCDO-RL {\color{red} but trained on environments with stochastic adversarial perturbations.} 

\textbf{Average Reward.}  As illustrated in Table \ref{tab_aver}, our algorithm achieves a better average reward than baselines (10.08\% improvement on average of all settings against two baselines), regardless of CO instance or problem size, when confronting the adversary trained by CCDO-RL. In the setting of CSP-20 nodes, the average reward is improved by 46.98\% compared to the heuristic and by 7.14\% compared with the RL against Stoc. For the 50-node setting, the improvements are 45.91\% and 5.28\% respectively. Similarly, the improvements in contrast to Heuristic and RL against Stoc are as follows: 1.72\% and 3.01\%  for CVRP-20 nodes, 0.75\% and 4.46\% for CVRP-50 nodes, 4.17\% and 1.48\% for PG-20 nodes, and 10.60\% and 4.38\% for PG-50 nodes.

\textbf{Generalizability.} From Table \ref{tab_gene}, CCDO-RL continues to achieve a better average reward when facing the adversary, demonstrating that the learned RL policies generalize well to unseen graphs. Even though the non-RL baselines do have access to the graph structures and other problem information of the unseen problem instances, CCDO-RL can obtain comparable performances without re-training on the new problem instances. The improvements versus Heuristic and RL against Stoc are 46.61\% and 7.02\% for CSP-20 nodes, 42.24\% and 3.94\% for CSP-50 nodes, 1.12\% and 1.56\% for CVRP-20 nodes, 0.90\% and 5.05\% for CVRP-50 nodes, 5.35\% and 2.40\% for PG-20 nodes, and 12.17\% and 10.33\% for PG-50 nodes. Even when confronting the stochastic adversary, CCDO shows superior generalizability compared to two baselines across three COPs, with average improvements of 6.31\%, 3.42\%, and 3.95\% respectively. Detailed results are provided in Appendix \ref{app_exp} (Tables \ref{tab_csp_full_20} - \ref{tab_op_full_50}). 
% The model’s usability is enhanced by the ability to generalize rather than focusing solely on the average reward, which is a critical motivation of the RL for combinatorial optimization literature \citep{khalil2017learning, kool2018attention}.  

\begin{remark}
    In CO problems (or more broadly, operations research and economics), it is known that achieving solution quality improvements against strong baselines (e.g., the RL methods trained with a stochastic adversary) is very challenging, and the margins are usually small \citep{kool2018attention}, sometimes even less than 1\%. However, these “tiny” marginal improvements in profits keep small business owners in the real world alive. Last, the improvement depends a lot on the problem settings, and we show that sometimes the improvement can be much more significant.
\end{remark}
\vspace{-\baselineskip}
% \textbf{Performance analysis.} The robustness results of CCDO-RL for ACSP are shown in Table \ref{tab_csp}. We have the following observations: 1) On both of the 100 seen/unseen graphs, single-player RL performs better than heuristic algorithms no matter whether attacked or not. (2) When confronting the adversary trained by CCDO-RL, CCDO-RL exceeds RL by 0.25 and 0.24 on the training set, and by 0.25 and 0.18 on the test set, respectively under the 20-node and 50-node graphs. This demonstrates the robustness of CCDO-RL. 3) Compared to the performance of the training set with that of the test set, we can see that RL and CCDO-RL both maintain a certain degree of generalization. Similar results for ACVRP (Table \ref{tab_cvrp}) and SPG (Table \ref{tab_op}) are provided in Appendix \ref{app_exp}. 

\begin{table}[ht]
  \caption{Average reward against CCDO-RL's adversary (on seen graphs)}
  \vspace{\baselineskip}
  \label{tab_aver}
  \centering
  \small
  \begin{tabular}{lllllll}
    \toprule
    \multirow{2}{*}{method} & \multicolumn{2}{c}{ACSP (Mean$\pm$Std)} & \multicolumn{2}{c}{ACVRP (Mean$\pm$Std)} & \multicolumn{2}{c}{PG (Mean$\pm$Std)} \\
    \cmidrule(r){2-3} \cmidrule{4-5} \cmidrule(r){6-7}
                            & 20 nodes & 50 nodes & 20 nodes & 50 nodes & 20 nodes & 50 nodes\\
    \midrule
    Heuristic & 6.13$\pm$1.20 & 7.55$\pm$1.42 & 7.65$\pm$1.23  & 13.38$\pm$1.70 & 2.64$\pm$1.03 & 4.53$\pm$1.84   \\
    RL against Stoc    & 3.50$\pm$0.47  & 4.55$\pm$0.62  & 7.55$\pm$1.16  & 13.90$\pm$1.63 & 2.71$\pm$0.90 & 4.80$\pm$2.18   \\
    CCDO-RL   & $\pmb{3.25}$$\pm$0.42 & $\pmb{4.31}$$\pm$0.51  & $\pmb{7.42}$$\pm$1.21  & $\pmb{13.28}$$\pm$1.52 &  $\pmb{2.75}$$\pm$0.87 & $\pmb{5.01}$$\pm$1.91  \\
    \bottomrule
  \end{tabular}
\end{table}
\vspace{-\baselineskip}

\begin{table}[htp]
  \caption{Generalizability against CCDO-RL's adversary (on unseen graphs)}
  \vspace{\baselineskip}
  \label{tab_gene}
  \centering
  \small
  \begin{threeparttable}
  \begin{tabular}{lllllll}
    \toprule
    \multirow{2}{*}{method} & \multicolumn{2}{c}{ACSP (Mean$\pm$Std)} & \multicolumn{2}{c}{ACVRP (Mean$\pm$Std)} & \multicolumn{2}{c}{PG (Mean$\pm$Std)} \\
    \cmidrule(r){2-3} \cmidrule{4-5} \cmidrule(r){6-7}
                            & 20 nodes & 50 nodes & 20 nodes & 50 nodes & 20 nodes & 50 nodes\\
    \midrule
    Heuristic & 6.20$\pm$1.33 & 7.60$\pm$1.37   & 7.64$\pm$1.30  & 13.27$\pm$1.87 & 2.43$\pm$0.98 & 4.19$\pm$1.69    \\
    RL against Stoc  & 3.56$\pm$0.37  & 4.57$\pm$0.58  & 7.67$\pm$1.30  & 13.85$\pm$1.53 &  2.50$\pm$0.95 & 4.26$\pm$2.17 \\
    CCDO-RL   & $\pmb{3.31}$$\pm$0.35 & $\pmb{4.39}$$\pm$0.52  & $\pmb{7.55}$$\pm$1.28  & $\pmb{13.15}$$\pm$1.59 & $\pmb{2.56}$$\pm$0.92 & $\pmb{4.70}$$\pm$1.94\\

    \bottomrule
  \end{tabular}
  \begin{tablenotes}
      \footnotesize
      \item[1] For the average reward of ACSP and ACVRP, smaller is better while for that of PG larger is better.
  \end{tablenotes}
  \end{threeparttable}
\end{table}
\vspace{-\baselineskip}
% two heuristics and one RL
% \begin{table}[ht]
%   \caption{{\color{red} Average reward of CCDO-RL (on seen graphs). For the value of CSP and CVRP, larger is better while for that of PG smaller is better.}}
%   \label{tab_aver}
%   \centering
%   \small
%   \begin{tabular}{lllllll}
%     \toprule
%     \multirow{2}{*}{method} & \multicolumn{2}{c}{CSP (Mean$\pm$Std)} & \multicolumn{2}{c}{CVRP (Mean$\pm$Std)} & \multicolumn{2}{c}{PG (Mean$\pm$Std)} \\
%     \cmidrule(r){2-3} \cmidrule{4-5} \cmidrule(r){6-7}
%                             & 20 nodes & 50 nodes & 20 nodes & 50 nodes & 20 nodes & 50 nodes\\
%     \midrule
%     Baseline 1 & 4.52$\pm$0.71  & 5.98$\pm$0.94 & 7.64$\pm$1.56  & 13.49$\pm$2.10 & 2.71$\pm$1.10 & 1.82$\pm$1.40   \\
%     Baseline 2 & 6.13$\pm$1.20 & 7.55$\pm$1.42   & 7.65$\pm$1.23  & 13.38$\pm$1.70 & 2.64$\pm$1.03 & 1.47$\pm$0.99  \\
%     RL {\color{red}against Stoc}    & 3.50$\pm$0.47  & 4.55$\pm$0.62  & 7.55$\pm$1.16  & 13.90$\pm$1.63 & 2.71$\pm$0.90 & 1.54$\pm$1.03   \\
%     CCDO-RL   & $\pmb{3.25}$$\pm$0.42 & $\pmb{4.31}$$\pm$0.51  & $\pmb{7.42}$$\pm$1.21  & $\pmb{13.28}$$\pm$1.52 &  $\pmb{2.75}$$\pm$0.87 & $\pmb{1.87}$$\pm$1.22  \\
%     \bottomrule
%   \end{tabular}
% \end{table}


% \begin{table}[htp]
%   \caption{{\color{red}Generalizability of CCDO-RL (on unseen graphs)}}
%   \label{tab_gene}
%   \centering
%   \small
%   \begin{threeparttable}
%   \begin{tabular}{lllllll}
%     \toprule
%     \multirow{2}{*}{method} & \multicolumn{2}{c}{CSP (Mean$\pm$Std)} & \multicolumn{2}{c}{CVRP (Mean$\pm$Std)} & \multicolumn{2}{c}{PG (Mean$\pm$Std)} \\
%     \cmidrule(r){2-3} \cmidrule{4-5} \cmidrule(r){6-7}
%                             & 20 nodes & 50 nodes & 20 nodes & 50 nodes & 20 nodes & 50 nodes\\
%     \midrule
%     Baseline 1 & 4.53$\pm$0.79  & 5.95$\pm$0.96 & 7.55$\pm$1.39  & 13.35$\pm$2.04 & 2.52$\pm$1.08 & $\pmb{1.86}$$\pm$1.44  \\
%     Baseline 2 & 6.20$\pm$1.33 & 7.60$\pm$1.37   & 7.64$\pm$1.3  & 13.27$\pm$1.87 & 2.43$\pm$0.98 & 1.52$\pm$1.20    \\
%     RL {\color{red}against Stoc}  & 3.56$\pm$0.37  & 4.57$\pm$0.58  & 7.67$\pm$1.30  & 13.85$\pm$1.53 &  2.50$\pm$0.95 & 1.03$\pm$5.05 \\
%     CCDO-RL   & $\pmb{3.31}$$\pm$0.35 & $\pmb{4.39}$$\pm$0.52  & $\pmb{7.55}$$\pm$1.28  & $\pmb{13.15}$$\pm$1.59 & $\pmb{2.56}$$\pm$0.92 & 1.35$\pm$5.09\\

%     \bottomrule
%   \end{tabular}
%   \begin{tablenotes}
%       \footnotesize
%       \item[1] For the value of CSP and CVRP, larger is better while for that of PG smaller is better.
%   \end{tablenotes}
%   \end{threeparttable}
% \end{table}

% \section{Routing in a general multi-objective framework}\label{sec:general-multi-obj}

In this section, we discuss a natural extension of the routing problem with general metrics. Since most of the constructions are very similar to the construction in the Introduction section \ref{sec:intro}, we keep our discussion short.  Say, we are provided $M$ pre-trained models, denoted as $f_1, \dots,f_M$. Given model $f_m$ and a sample point $(X, Y)$,  we are interested in $L$ different loss metrics, among which the first $L_1 $ many, denoted as $\ell_1\{f_m; X, Y\} , \dots, \ell_{L_1}\{f_m; X, Y\}$, depends on $f_m$, $X$ and $Y$, while the next $L_2 = L - L_1$ many, denoted as $\ell_{L_1+ 1}(f_m; X), \dots , \ell_{L}(f_m; X)$, depends only on $f_m$ and $X$.  In that case we consider a compromise between these $L$ losses: for an $\lambda = (\lambda_1 , \dots, \lambda_L) \in\Delta^{L-1} $ the linear loss trade-off in eq. \eqref{eq:linearized-loss} naturally extends to 
\begin{equation} \label{eq:linearized-loss-gen}
 \textstyle \eta_{\lambda, m}(X, Y) =  \sum_{l = 1}^{L_1}\lambda_l \ell_l\{f_m; X, Y\} + \sum_{l = L_1 + 1}^{L}\lambda_l \ell_l\{f_m; X\}   \,.
\end{equation} 
To make it even more general, assume that we have a classification task with $M$ classes, \ie \ the outcome $Y \in \{1, \dots, M\}$ is now categorical. Additionally, as a multi-objective problem we have $L$ different loss metrics, among which the first $L_1 $ many $\ell_1\{m; X, Y\} , \dots, \ell_{L_1}\{m; X, Y\}$ are losses  of predicting a sample $(X, Y)$ as the class $m$, while the next $L_2 = L - L_1$ many $\ell_{L_1+ 1}\{m; X\}, \dots , \ell_{L}\{m; X\}$ is the cost of predicting the sample $X$ as the class $m$. Then the aggregated loss is 
\begin{equation} \label{eq:linearized-loss-gen-2}
 \textstyle \eta_{\lambda, m}(X, Y) =  \sum_{l = 1}^{L_1}\lambda_l \ell_l\{m; X, Y\} + \sum_{l = L_1 + 1}^{L}\lambda_l \ell_l\{m; X\}   \,.
\end{equation} 

For a particular $\lambda \in \Delta^{L-1}$ we want to estimate the optimal/Bayes/admissible classifier $g_\lambda^\star$ that minimized the average of the aggregates loss trade-off:
\begin{equation}
    \textstyle g_\lambda^\star = \argmin_{g: \cX \to [M]}  \Ex_P\big[\sum_{m = 1}^M \bbI\{g(X) = m\} \eta_{\lambda, m} (X, Y) \big ]
\end{equation}
Similar to Lemma \ref{lemma:oracle-router}, defining $\Phi_{l, m}^\star(X) = \Ex[\ell_l\{f_m; X,Y\} \mid X]$ for $l \in [L_1]$ and $\Phi_{l, m}^\star(X) = \ell_l\{f_m; X\} $ for $l \in \{L_1 + 1, \dots, L\}$ the Bayes classifier $g_\lambda^\star$ has the explicit form  (\cf\ Lemma \ref{lemma:bayes-classifier-gen})
\begin{equation}\label{eq:oracle-router-gen}
  \textstyle  g_\lambda^\star (X) = \argmin_m  \eta_{\lambda, m}(X), ~~ \eta_{\lambda, m}(X) = \sum_{l = 1}^L \lambda_l \Phi_{l, m}^\star (X)\,.
\end{equation}
A natural extension of Algorithm \ref{alg:pareto-routers} requires us to estimate each of the functions $\Phi_{l, m}^\star$ as $\widehat \Phi_{l, m}$ then estimate all Bayes classifiers in a one-shot plug-in approach as:
\[
\textstyle \widehat g_\lambda(X) = \argmin_m  \{\sum_{l = 1}^L \lambda_l \widehat \Phi_{l, m} (X) \}\,.
\] 
Note that, for $l \ge L_1+ 1$ the functions $\Phi_{l, m}^\star$ are known, so they need not be estimated; in these cases, we simply let $\widehat \Phi_{l, m} = \Phi_{l ,m}^\star$. 
Having estimated $\widehat g_\lambda$, we can then evaluate them on a test data split with respect to each of the individual metrics to estimate the Pareto frontier and examine the optimal trade-off between the objectives.
We describe the procedure in Algorithm \ref{alg:pareto-routers-gen}.  

\begin{algorithm}
    \begin{algorithmic}[1]
\Require Dataset $\cD_n$
\State Randomly split the dataset into training and test splits: $\cD_n = \cD_{\text{tr}} \cup \cD_{\text{test}}$. 
\State  Learn estimates $\widehat \Phi_{l, m} (X)$ of the function $\Phi_{l, m}^\star(X)$ using the training split $\cD_{\text{tr}}$. 
\For{$\lambda \in \Delta^{L-1}$}
\State  Define $\widehat \eta_{\lambda, m}(X) =  \sum_{l = 1}^L \lambda_l \widehat \Phi_{l, m} (X)  $ and 
 $\widehat g_\lambda(X) = \argmin_m \widehat \eta_{\lambda, m}(X)$
 \For{$l \in \{1, \dots, L\}$}
 \State Calculate $\widehat \cE_{l, \lambda}  =  \frac1{|\cD_{\text{test}}|} \sum_{(X, Y) \in \cD_{\text{test}}}  \ell_l\{Y, f_{\widehat g_\lambda(X)}(X)\}$
 \EndFor
\EndFor

\Return $\{g_\lambda: \lambda \in \Delta^{L-1}\}$ and $\widehat\cF = \{(\widehat \cE_{1, \lambda}, \dots, \widehat \cE_{L, \lambda}): \lambda \in \Delta^{L-1}\}$. 
\end{algorithmic}
\caption{Learning of Bayes classifiers}
\label{alg:pareto-routers-gen}
\end{algorithm}


Except for some minor differences, the minimax rate analysis of the estimate of $g_\lambda^\star$ is identical to Section \ref{sec:lower-bound}. We assume that
\begin{enumerate}
    \item For $1 \le l \le L_1$ the $\Phi_{l, m}^\star$ functions are $(\beta_l, K_{\beta, l})$-smooth, which is similar to the Assumption \ref{assmp:smooth}. Recall the discussion following Assumption \ref{assmp:smooth}; (1) Since $\Phi_{l, m}$ are known for $l \ge L_1 + 1$ a smoothness assumption on them is not necessary, and (2) We could assume different smoothness for different $\Phi_{l, m}^\star$, \ie\ they are $\beta_{l, m}$ smooth, but the rate will only involve the smallest smoothness, \ie\ $\beta_{l, \min} = \min_m \beta_{l, m}$. Thus, for simplicity, we assume that for a given $l$ the $\beta_{l, m}$ are identical for different $m$. 
    \item The margin condition in Assumption \ref{assmp:margin} is satisfied with the new definition of $\eta_{\lambda, m}(X) = \Ex[\eta_{\lambda, m}(X, Y)\mid X]$ with parameters $(\alpha, K_\alpha)$.
    \item  The $P_X$ is compactly supported and satisfies the strong density condition, as in the assumption \ref{assmp:strong-density}.
\end{enumerate}

%  and that the margin condition is satisfied  Furthermore, we assume the 
% \SM{talk about how the worst smoothness parameter is the one that matters.}


Under these assumptions, we establish the upper and lower bound analysis for the minimax rate of convergence in excess risk
\[
\textstyle \cE_P(g, \lambda) = \cL_P(g, \lambda) - \cL_P(g_\lambda^\star, \lambda)\,.
\]
For this purpose, let us quickly recall the notation and definitions necessary to describe the lower and upper bounds. We denote the class of all probability distributions on $\cX \times \cY$ space by $\cP$.
We also denote $\Gamma = \{g: \cX \to [M]\}$ as the set of all classifiers and $\cA_n$ as the class of learning algorithms, such that an algorithm $\cA_n \ni A_n: \cZ^n \to \Gamma $ that takes the dataset $\cD_n $ as input and produces a classifier $A_n(\cD_n): \cX \to [M]$.

\begin{theorem}
\label{thm:bound-gen}
Suppose that $\max_{l \in  [L_1]}\beta_l <  \nicefrac{d}{\alpha}$. Then we have the following lower and upper bounds for the excess risk. 
\begin{itemize}
    \item {\bf Lower bound:} For $n \ge 1$  and $A_n \in \cA_n$ define $\cE_P(A_n, \lambda) = \Ex_{\cD_n}\big[\cE_P\big(A_n(\cD_n), \lambda\big)\big]$. There exist constants $c_1, c_2> 0$ that are independent of $n$ and $\lambda$ such that for any $n\ge 1$ and $ \lambda \in \Delta^{L-1}$ we have the lower bound
    \begin{equation} \label{eq:lower-bound-gen}
      \textstyle  \min\limits_{A_n \in \cA_n} \max\limits_{P \in \cP} ~~ \cE_P(A_n, \lambda) \ge c \big \{\sum_{l = 1}^{L_1}\lambda_l n^{- \frac{\beta_l}{2\beta_l + d}}\big\}^{1+\alpha} \,.
    \end{equation}  
    \item  {\bf Upper bound:} For $l \le L_1$ and $m\in [M]$ suppose that there are estimators $\widehat\Phi_{l, m}$ for $\Phi^\star_{l, m}$ that satisfies the following: for constants $\rho_{l, 1}, \rho_{l, 2} > 0$ and any $n \ge 1$ and $t > 0$ and almost all $X$ with respect to $P_X$ we have 
    \begin{equation} \label{eq:concentration-phi-gen}
        \max_{P\in \cP} P \big \{ \max_m \big |\widehat \Phi_{l, m} (X) - \Phi^\star_{l, m}  (X)\big |  \ge t\big \} \le  \rho_{l, 1} \exp\big (- \rho_{l,2} a_{l, n}^2 t^2 \big )\,,  
    \end{equation}  where the sequence $\{a_{l, n}; n \ge 1\}\subset (0, \infty)$  increases to $\infty$. Then there exists a $K> 0$ such that for any $n \ge 1$ and $\lambda \in \Delta^{L-1} $ the excess risk for the router $\widehat g_\lambda$ in Algorithm \ref{alg:pareto-routers-gen} is upper bounded as 
    \begin{equation} \label{eq:upper-bound-gen}
      \textstyle   \max\limits_{P\in \cP} \Ex_{\cD_n}\big [\cE_P(\widehat g_\lambda,\lambda)\big ] \le K \big\{ \sum_{l = 1}^{L_1}\lambda_l a_{l, n}^{-1}\big\} ^{1+ \alpha}\,. 
    \end{equation}
    \item  {\bf Local polynomial estimator:} Assume that $\ell_l \{Y_i, f_m(X_i)\}$ are sub-Gaussian, \ie\ there exists constants $c_{l, 1}$ and $c_{l, 2}$ such that  
    \[
    \textstyle P\big ( |\ell_l \{Y, f_m(X)\} | > t \mid X\big ) \le c_{l, 1} e^{-c_{l, 2}t^2}\,. 
    \] If $\psi$ satisfies the regularity conditions with parameter $\beta_l$ in (\cf\ Definition \ref{def:kernel-reg})  and $k = \lfloor \beta_l \rfloor$ then for $h = n^{-\frac{1}{2\beta_l + d}}$ the inequality in \eqref{eq:concentration-phi-gen} holds for 
    \begin{equation}\label{LPR-gen}
    \widehat \Phi_{l, m}(x_0) = \widehat\theta_{x_0}^{(m)}(0), ~~ \widehat \theta_x^{(m)} \in \argmin_{\theta \in \Theta(k)}  \textstyle \sum_{i = 1}^n \psi (\frac{X_i - x_0}{h}) \big [\ell \{Y_i, f_m(X_i)\} - \theta (X_i -x_0 )\big]^2 \,. 
\end{equation}
    with $a_{l, n} = n^{\nicefrac{2\beta_l}{(2\beta_l + d)}}$, where $\Theta(k)$ is the class of all $k$ degree polynomials. 
\end{itemize} 

  
\end{theorem}
% \SM{Discuss the optimality in remarks.} 
\SM{comment about $\max_{l \in  [L_1]}\beta_l <  \nicefrac{d}{\alpha}$.}
We end our section with a few remarks. 

\begin{remark}[Rate optimality for plug-in routers]

Firstly, given that the upper bound in eq.  \eqref{eq:upper-bound-gen} for plug-in router described in Algorithm \ref{alg:pareto-routers-gen} with local-polynomial estimator for $\widehat \Phi_{l, m}(X)$ achieves the same rate of convergence as the lower bound in \eqref{eq:lower-bound} we conclude that the minimax optimal rate of convergence in excess risk is 
\begin{equation} \label{eq:optimal-rate-gen}
    \textstyle \min\limits_{A_n \in \cA_n} \max\limits_{P \in \cP} ~~ \cE_P(A_n, \lambda) \asymp \cO \Big(\big \{\sum_{l = 1}^{L_1}\lambda_l n^{- \frac{\beta_l}{2\beta_l + d}}\big\}^{1+\alpha}\Big)\,. 
\end{equation} Moreover, this also implies that the plug-in router can achieve this optimal rate as long as the bounds in eq. \eqref{eq:concentration-phi-gen} are satisfied with $a_{l, n} = n^{\nicefrac{2\beta_l}{(2\beta_l + d)}}$. 
    
\end{remark}

\begin{remark}[Difficulty in routing with respect to $\lambda$]
Moreover, the rate in eq. \eqref{eq:optimal-rate-gen} implies that for a smaller $\lambda_l$ the errors in estimating $\Phi_{l, m}^\star$ have a lesser impact on the excess risk convergence. This conclusion is very much related to our Remark \ref{remark:diff-in-lambda-lb}, and thus we keep our discussion brief and ask readers to revisit the said remark. We end with a quick observation that the hardest instance to classify is when $\lambda_{l_{\min}} = 1$ for the lowest smoothness parameter, \ie\ $l_{\min} = \argmin_{l \le L_1} \beta_l$. 

    
\end{remark}


\begin{remark}

{\bf (Minimax rate study for non-parametric classification with multiple objectives)}
Compared to \citet{audibert2007Fast}, our study of the optimal minimax rate provided in this section generalizes the setting on three fronts: (1) the number of classes can be more than two, (2) general classification loss functions beyond $0/1$-loss and (3) multiple objectives. In this general setting, we show that the plug-in classifiers described in Algorithm \ref{alg:pareto-routers-gen} are both computationally and statistically (rate optimal) efficient for estimating the entire class of Bayes classifiers $\{g_\lambda^\star: \lambda\in \Delta^{L-1}\}$. 
    
\end{remark}
\section*{Conclusion}
This paper aims to enhance our understanding of the computational complexity of computing various Shapley value variants. We found that for various ML models --- including decision trees, regression tree ensembles, weighted automata, and linear regression --- both local and global interventional and baseline SHAP can be computed in polynomial time under HMM modeled distributions. This extends popular algorithms, such as TreeSHAP, beyond their empirical distributional scope. We also establish strict complexity gaps between the various SHAP variants (baseline, interventional, and conditional) and prove the intractability of computing SHAP for tree ensembles and neural networks in simplified scenarios. Overall, we present SHAP as a versatile framework whose complexity depends on four key factors: \begin{inparaenum}[(i)] \item model type, \item SHAP variant, \item distribution modeling approach, \item and local vs. global explanations\end{inparaenum}. We believe this perspective provides deeper insight into the computational complexity of SHAP, paving the way for future work.




%We believe that our framework provides a more intricate understanding of SHAP computation complexity across different models, distributions, and variants, paving the way for further research.

Our work opens promising directions for future research. First, expanding our computational analysis to other SHAP-related metrics, such as asymmetric SHAP~\citep{frye20} and SAGE~\citep{covert2020understanding}, would be valuable. Additionally, we aim to explore more expressive distribution classes and relaxed assumptions beyond those in Section \ref{sec:tractable} while maintaining tractable SHAP computation. Finally, when exact computation is intractable (Section \ref{sec:intractable}), investigating the approximability of SHAP metrics through approximation and parameterized complexity theory~\citep{downey2012parameterized} is an important direction.

%Our work opens several promising avenues for future research on the computational properties of explainable AI methods, with a particular focus on SHAP. First, it would be interesting to broaden the computational analysis conducted in this work to include other popular SHAP-related metrics in the literature, such as asymmetric SHAP \cite{frye20} and SAGE \cite{covert2020understanding}. Also, in the future, we aim to explore more expressive distribution classes and relaxed distributional assumptions—extending beyond those examined in Section \ref{sec:tractable} —that still yield tractable SHAP computation. Finally, when exact computation proves intractable (Section \ref{sec:intractable}), it is worthwhile to theoretically investigate the question of the approximability of computing the SHAP metrics across various configurations, through the lens of approximation and parametrized complexity theory \cite{arora2009computational}.

%This paper aims to deepen our understanding of the computational complexity involved in obtaining different Shapley value variants. We found that for a variety of ML models, including decision trees, tree ensembles for regression, weighted automata, and linear regression models — computing both local and global interventional and baseline SHAP can be done in polynomial time when distributions are modeled by HMMs. This extends the distributional scope of popular algorithms like TreeSHAP, which is limited to empirical distributions. Additionally, we demonstrate a strict complexity gap between SHAP variants, showing that interventional and baseline SHAP can be strictly easier to compute than conditional SHAP. Despite these positive results, we uncovered intractability for various SHAP variants in neural networks and tree ensembles. Finally, we provided generalized complexity relations across SHAP variants. We believe that our framework offers a deeper understanding of the complexity involved in computing SHAP across various variants, models, distributions, as well as in both local and global computations, laying the groundwork for future research.












 

\bibliography{seamus,YK,RAGBench}
\bibliographystyle{abbrvnat}

\newpage
\appendix
% \section{Minimax study under mild density condition}
% ----------------------------------------------
\newpage
\section*{supplementary material}
In this supplementary material, we will provide a theoretical analysis to the proposed memory efficient Transformer adapter (META) in Section~\ref{secS1}, provide a detailed description of the experimental datasets in Section~\ref{secS2}, provide a detailed description of the experimental settings in Section~\ref{secS3},
provide more result comparisons under different pre-trained weights in Section~\ref{secS4},
provide more ablation study results in Section~\ref{secS5}, show class activation map comparisons of instance segmentation before and after adding the Conv branch in Section~\ref{secS6},qualitative visualizations of instance segmentation and semantic segmentation results in Section~\ref{secS7},  as well as the pseudo-code for when the stripe size is set to $2$ in Section~\ref{secS8}. 
% -------------------------------------------
\section{Theoretical Analysis of META}
\label{secS1}
% -------------------------------------------
{\color{red}{\emph{This supplementary is for Section~3 of the main paper.}}} In this section, we will prove that META exhibits superior generalization capability and stronger adaptability compared to existing ViT adapters. 
%
To achieve this goal, we will prove that the proposed memory efficient adapter (MEA) block possesses larger information entropy (IE) than the existing attention-based ViT adapters~\citep{hu2022lora,jie2023fact,chen2022vision,ma2024segment,luo2023forgery,shao2023deepfake}, which provides evidence that the MEA block has more comprehensive feature representations. Then, based on the maximum mean discrepancy (MMD) theory~\citep{cheng2021neural,arbel2019maximum,wang2021rethinking}, larger IE in the ViT adapter framework leads to superior generalization capability and stronger adaptability. The detailed theoretical analysis process is as follows:

\begin{lemma}
% ---------------------------------
In any case of mutual information, the MEA block will gain larger information entropy after fusing $\textbf{X}_{vit}$ and $\textbf{X}_{con}$.
% ---------------------------------
\end{lemma}
% ---------------------------------
\begin{proof}
As introduced in Section~3.2 of the main paper, the proposed MEA block can be viewed as an operation that integrates the ViT features (\ie, the Attn branch and the FFN branch) and the convolution features (\ie, the Conv branch). Therefore, we begin by formalizing the obtained features into the following two basic elements: the ViT features and the convolution features. To formalize the learning setting, we express the ViT features as $\textbf{X}_{vit}$ and the convolution features as $\textbf{X}_{con}$. It is evident that if $\textbf{X}_{vit}$ and $\textbf{X}_{con}$ are extracted from the same image, then $\textbf{X}_{vit}$ and $\textbf{X}_{con}$ are not independently distributed, and there exists some mutual information between them~\citep{zhang2022graph,wu2021cvt,zhang2023cae,peng2021conformer}. Therefore, the IE of the fused feature of $\textbf{X}_{vit}$ and $\textbf{X}_{con}$ within the MEA block can be expressed as:
% ---------------------------------------------------
\begin{equation}
\begin{split}
\label{eqs:1}
\textrm{H}(\textbf{X}_{vit}, \textbf{X}_{con}) = \textrm{H}(\textbf{X}_{vit}) + \textrm{H}(\textbf{X}_{con}) - \textrm{I}(\textbf{X}_{vit}; \textbf{X}_{con}),
\end{split}
\end{equation}
% ---------------------------------------------------
where $\textrm{H}(\cdot)$ is utilized to calculate the IE of the given variate, which can be formulated as:
% ---------------------------------------------------
\begin{equation}
\begin{split}
\label{eqs:2}
\textrm{H}(\textbf{X}_{vit}) = -\sum P(\textbf{x}_{vit}) log(P(\textbf{x}_{vit})),\\
\textrm{H}(\textbf{X}_{con}) = -\sum P(\textbf{x}_{con}) log(P(\textbf{x}_{con})),
\end{split}
\end{equation}
% ---------------------------------------------------
where $P(\textbf{x}_{vit})$ represents the probability of $\textbf{X}_{vit}$ taking on the value of $\textbf{x}_{vit}$. The similar definition of $P(\textbf{x}_{con})$. $\textrm{I}(\cdot;\cdot)$ in Eq.~\eqref{eqs:1} is used to compute the mutual information between $\textbf{X}_{vit}$ and $\textbf{X}_{con}$, which can be expressed as:
% ---------------------------------------------------
\begin{equation}
\begin{split}
\label{eqs:3}
\textrm{I}(\textbf{X}_{vit}; \textbf{X}_{con}) = \sum\sum \textrm{P}(\textbf{X}_{vit}, \textbf{X}_{con}) \textrm{log}(\textrm{P}(\textbf{X}_{vit}, \textbf{X}_{con}) (\textrm{P}(\textbf{X}_{vit}), \textrm{P}(\textbf{X}_{con}))),
\end{split}
\end{equation}
% ---------------------------------------------------
where $\textrm{P}(\textbf{X}_{vit}, \textbf{X}_{con})$ is their joint probability distribution. 
%\textrm{P}(\textbf{X}_{vit})$ and $\textrm{P}(\textbf{X}_{con})$ are the marginal probability distributions of $\textbf{X}_{vit}$ and $\textbf{X}_{con}$, respectively. 
Since $\textrm{I}(\textbf{X}_{vit}; \textbf{X}_{con})$ is always non-negative, $\textrm{H}(\textbf{X}_{vit}, \textbf{X}_{con})$ may still be greater than $\textrm{H}(\textbf{X}_{vit})$ or $\textrm{H}(\textbf{X}_{con})$~\citep{paninski2003estimation,gabrie2018entropy}. This suggests that the IE of the features extracted by MEA is always greater than the feature representation extracted by either of them separately.

Specifically, if $\textrm{I}(\textbf{X}_{vit}; \textbf{X}_{con})$ is small, the IE gain after fusion may still be significant, which is beneficial for improving the generalization capability and adaptability of the block. However, when $\textrm{I}(\textbf{X}_{vit}; \textbf{X}_{con})$ is large, the IE gain after fusion may be reduced. This means that $\textrm{I}(\textbf{X}_{vit}; \textbf{X}_{con})$ may affect the IE improvement of the fused model. Next, we will discuss the impact of $\textbf{X}_{vit}$ and $\textbf{X}_{con}$ on improving the IE of the adapter based on the size of $\textrm{I}(\textbf{X}_{vit}; \textbf{X}_{con})$, which can be divided into the following three cases:

\begin{itemize}
% --------------------------
\item {{Small} $\textrm{I}(\textbf{X}_{vit}; \textbf{X}_{con})$.} This is an ideal state. When the dependency between $\textbf{X}_{vit}$ and $\textbf{X}_{con}$ is small, it indicates that $\textrm{I}(\textbf{X}_{vit}; \textbf{X}_{con})$ is small, that is, $\textbf{X}_{vit}$ and $\textbf{X}_{con}$ respectively represent different information of the image. In this case, fusing $\textbf{X}_{vit}$ and $\textbf{X}_{con}$ can bring a significant increase in IE, which is beneficial to improving the adapter's generalization capability and adaptability.
% --------------------------
\item {{Medium} $\textrm{I}(\textbf{X}_{vit}; \textbf{X}_{con})$.} When $\textrm{I}(\textbf{X}_{vit}; \textbf{X}_{con})$ is between small and large, it indicates that there is a certain degree of correlation between them. In this case, fusing $\textbf{X}_{vit}$ and $\textbf{X}_{con}$ may still bring some IE gain. The specific improvement effect depends on the degree of correlation between $\textbf{X}_{vit}$ and $\textbf{X}_{con}$ and their complementarity in image representations. Fortunately~\citep{zhang2022graph,zhang2023cae,marouf2024mini,liu2023efficientvit}, a large amount of work has validated that ViT and convolutional layers can extract distinctive information from images. Therefore, in this case, fusing $\textbf{X}_{vit}$ and $\textbf{X}_{con}$ can still bring IE gains.
\item {{Large} \myparagraph{$\textrm{I}(\textbf{X}_{vit}; \textbf{X}_{con})$}.} When $\textrm{I}(\textbf{X}_{vit}; \textbf{X}_{con})$ between $\textbf{X}_{vit}$ and $\textbf{X}_{con}$ is large, it indicates that there is a high correlation between them, \ie, global ViT and local convolution features may represent similar or overlapping information of the image. In this case, the IE gain brought by fusing $\textbf{X}_{vit}$ and $\textbf{X}_{con}$ may decrease because there is a lot of information overlap between them. However, in our case, the probability of such a scenario occurring is almost non-existent, fusing $\textbf{X}_{vit}$ and $\textbf{X}_{con}$ may still improve the performance of the model to some extent, because they may capture the detailed information of the image to varying degrees.
% --------------------------
\end{itemize}
% --------------------------

Based on the aforementioned theoretical analysis, we can conclude that the proposed MEA block has a larger IE than existing ViT adapters (which are primarily based on the attention mechanism) under any scenario. This provides evidence that the MEA block has more comprehensive feature representations. 
% ---------------------------------
\end{proof}
% ---------------------------------
As the MEA block includes a parallel convolutional branch, it can better capture local inductive biases compared to the traditional ViT adapter, which mainly uses self-attention~\citep{hu2022lora,jie2023fact,chen2022vision,ma2024segment,luo2023forgery,shao2023deepfake,mercea2024time}. 
%
Therefore, the MEA block's feature space should be more capable of distinguishing different samples, resulting in a larger MMD value. 
%
Our MEA block's feature space is obtained by combining the attention branch, the feed-forward network branch, and the local convolutional branch, enabling it to capture both local and global inductive biases of the given image. 
%
In contrast, the traditional ViT adapter's feature space is mainly obtained through self-attention and may not be able to capture local features well. Therefore, according to the MMD theory~\citep{cheng2021neural,arbel2019maximum,wang2021rethinking}, we can conclude that if the MEA block's feature space is more discriminative than the traditional ViT adapter's feature space, then the MEA block's feature space is more suitable for adapter feature space and can better improve the model's generalization capability and adaptability.

% -------------------------------------------
\section{Introduction of the Experimental Datasets}
\label{secS2}
% -------------------------------------------
{\color{red}{\emph{This supplementary is for Section~4.1 of the main paper.}}}
In our paper, two representative datasets are used to evaluate the effectiveness and efficiency of our method, including MS-COCO~\citep{caesar2018coco} for ODet and ISeg, and ADE20K~\citep{zhou2017scene} for SSeg. Below are the details of the used datasets:

% -------------------------------
\begin{itemize}
% -------------------------------
\item MS-COCO~\citep{caesar2018coco} is a representative yet challenging dataset for common scene IS and object detection, which consists of $118$k, $5$k and $20$k images for the \emph{training} set, the \emph{val} set and the \emph{test} set, respectively. In our experiments, the model is trained on the \emph{training} set and evaluated on the \emph{val} set.
% -------------------------------
\item ADE20K~\citep{zhou2017scene} is a scene parsing dataset with $20$k images and $150$ object categories. Each image has pixel-level annotations for SS of objects and regions within the scene. The dataset is divided into $20$k, $2$k, and $3$k images for \emph{training}, \emph{val} and \emph{test}, respectively. Our model is trained on the \emph{training} set and evaluated on the \emph{val} set.
% -------------------------------
\end{itemize}
% -------------------------------
For data augmentation, random horizontal flip, brightness jittering and random scaling within the range of $[0.5, 2]$ are used in training as in~\citep{chen2022vision,luo2023forgery,zhang2023cae,mercea2024time}. By default, the inference results are obtained at a single scale, unless explicitly specified otherwise.    


% -------------------------------------------
\section{Introduction of the Experimental Settings}
\label{secS3}
% -------------------------------------------
{\color{red}{\emph{This supplementary is for Section~4.2 of the main paper.}}} Experiments on object detection and instance segmentation are conducted using the open-source MMDetection framework~\citep{chen2019mmdetection}. The training batch size is set to $16$, and AdamW~\citep{loshchilov2017decoupled} is used as the optimizer with the initial learning rate of $1 \times 10^{-4}$ and the weight decay of $0.05$. The layer-wise learning rate decay is used and set to $0.9$, and the drop path rate is set to $0.4$. Following~\citep{xiong2024efficient,wang2021pyramid,chen2022vision,liu2022convnet}, to ensure a fair result comparison, we choose two training schedules, 1$\times$ (\ie, $12$ training epochs) and 3$\times$ (\ie, $36$ training epochs). For the 1$\times$ training schedule, images are resized to the shorter side of 800 pixels, with the longer side not exceeding $1,333$ pixels. In inference, the shorter side of images is consistently set to 800 pixels by default. For the 3$\times$ training schedule, the multi-scale training strategy is also used as in~\citep{chen2022vision}, and the shorter side is resized to $480$ to $800$ pixels, while the longer side remains capped at $1,333$ pixels.

{\color{red}{\emph{This supplementary is for Section~4.3 of the main paper.}}} Experiments on semantic segmentation are conducted using the MMSegmentation framework~\citep{mmseg2020}. The input images are cropped to a fix size of 512 $\times$ 512 pixels as in~\citep{xiong2024efficient,chen2022vision}. The training batch size is set to $16$, and AdamW~\citep{loshchilov2017decoupled} is used as the optimizer with the initial learning rate of $1 \times 10^{-5}$ and the weight decay of $0.05$. Following~\citep{li2022exploring,liu2021swin}, the layer-wise learning rate decay is set to $0.9$ and the drop path rate is set to $0.4$. We report the experimental results on both single scale training and multi-scale training strategies. 
% -------------------------------
\begin{table}[t]
\centering
\small
\renewcommand\arraystretch{1.2}
\setlength{\tabcolsep}{6pt}{
\begin{tabular}{r|r|ccl}
\hline \hline 
Methods & Pre-Trained & Params.$\downarrow$ & AP$^\textrm{m}$ $\uparrow$ \\
\hline 
Swin-B~\citep{liu2021swin} & ImageNet-1k~\citep{deng2009imagenet} & 107.1 &  43.3 \\
ViT-Adapter-B~\citep{chen2022vision} & ImageNet-1k~\citep{deng2009imagenet} & 120.2 & 43.6 \\
\cellcolor[gray]{.95}\textbf{META-B$_{{\textrm{(Ours)}}}$} & \cellcolor[gray]{.95}ImageNet-1k~\citep{deng2009imagenet} & \cellcolor[gray]{.95}115.3 & \cellcolor[gray]{.95}44.3$_{\color{red}{+0.7}}$ \\
\cdashline{1-4}[0.8pt/2pt]
Swin-B~\citep{liu2021swin} & ImageNet-22k~\citep{steiner2021train} & 107.1 & 44.3\\
ViT-Adapter-B~\citep{chen2022vision} & ImageNet-22k~\citep{steiner2021train} & 120.2 & 44.6 \\
\cellcolor[gray]{.95}\textbf{META-B$_{{\textrm{(Ours)}}}$} & \cellcolor[gray]{.95}ImageNet-22k~\citep{steiner2021train} & \cellcolor[gray]{.95}115.3  & \cellcolor[gray]{.95}45.2$_{\color{red}{+0.6}}$ \\
\cdashline{1-4}[0.8pt/2pt]
Swin-B~\citep{liu2021swin} & Multi-Modal~\citep{zhu2022uni} & 107.1 &   -- \\
ViT-Adapter-B~\citep{chen2022vision} & Multi-Modal~\citep{zhu2022uni} & 120.2  & 45.3 \\
\cellcolor[gray]{.95}\textbf{META-B$_{{\textrm{(Ours)}}}$} & \cellcolor[gray]{.95}Multi-Modal~\citep{zhu2022uni} & \cellcolor[gray]{.95}115.3  & \cellcolor[gray]{.95}45.9$_{\color{red}{+0.6}}$ \\
\hline \hline 
\end{tabular}
\caption{Result comparisons on Params. (\textbf{M}) and AP (\%) under different pre-trained weights with Mask R-CNN ($3 \times$ +MS schedule)~\citep{he2017mask} as the baseline model on the \emph{val} set of MS-COCO~\citep{caesar2018coco}. ``--'' denotes there is no such a result in its paper.}
\label{tab3}}
\end{table}
% -------------------------------

% -------------------------------------------
\section{Result Comparisons under Different Weights}
\label{secS4}
% -------------------------------------------
{\color{red}{\emph{This supplementary is for Section~4.2 of the main paper.}}} In this section, we present the experimental results of META on object detection and instance segmentation with different pre-trained weights and compare them with other state-of-the-art methods including SwinViT~\citep{liu2021swin} and ViT-Adapter~\citep{chen2022vision} as in~\citep{chen2022vision}. 
Mask R-CNN~\citep{he2017mask} is used as the baseline, and ViT-B~\citep{li2022exploring} is used as the backbone. The 3$\times$ training schedule with MS training strategy is used. The obtained experimental results are given in Table~\ref{tab3}.
%
From this table, we can observe that our method is applicable to different pre-trained weights (\ie, ImageNet-1k~\citep{deng2009imagenet}, ImageNet-22k~\citep{steiner2021train}, and Multi-Modal~\citep{zhu2022uni}), and achieves more accurate AP with fewer model parameters compared to ViT-Adapter~\citep{chen2022vision}, across different pre-trained weights.  

% -------------------------------------------
\section{More Ablation Study Results}
\label{secS5}
% -------------------------------------------
{\color{red}{\emph{This supplementary is for Section~4.4 of the main paper.}}} In our main paper, we present the experimental results of deploying adapters with Attn branch and FFN branch as components on ViT-B~\citep{li2022exploring}. It is noteworthy that the layer normalization operation has been shared between the Attn branch and the FFN branch to reduce the memory access costs associated with the normalization operations. In this section, we demonstrate a result comparison between the experimental results of using shared layer normalization operation and those of not using it in the traditional setting (\ie, the non-shared normalization). The obtained experimental results are shown in Table~\ref{tab:s1}. It can be observed that sharing layer normalization does not significantly improve the performance in terms of AP. However, compared to FPS, FLOPs, MC, our approach can achieve satisfactory performance gains.
% --------------------------
\begin{table*}[t]
\centering
\renewcommand\arraystretch{1.2}
\setlength{\tabcolsep}{1pt}{
\begin{tabular}{r|ccccc|ccccc}
\hline \hline 
Settings & ViT-B & Attn & FFN & Conv & Cascade & AP$^\textrm{m}$ $\uparrow$ & FPS$\uparrow$ & Params.$\downarrow$ & FLOPs$\downarrow$ & MC$\downarrow$ \\
\hline 
Baseline model & \cmark & \xmark & \xmark & \xmark & \xmark & 41.3 & 11.5 & 113.6\textbf{M} & 719\textbf{G} & NA\\
\cdashline{1-11}[0.8pt/2pt]
\cellcolor[gray]{.95}Shared normalization & \cmark & \cmark & \cmark & \xmark & \xmark & \cellcolor[gray]{.95}43.4 & \cellcolor[gray]{.95}11.3 & \cellcolor[gray]{.95}114.4\textbf{M} & \cellcolor[gray]{.95}719\textbf{G} & \cellcolor[gray]{.95}7.5\textbf{GB}\\
Non-shared normalization & \cmark & \cmark & \cmark & \xmark & \xmark & 43.2 & 10.5 & 114.4\textbf{M} & 737\textbf{G} & 8.8\textbf{GB}\\
\hline \hline 
\end{tabular}
\caption{Ablation study results on shared layer normalization.}
\label{tab:s1}}
\end{table*}
% --------------------------

% -------------------------------
{\color{red}{\emph{This supplementary is for Section~4.4 of the main paper.}}} META is proposed as a simple and fast ViT adapter by minimizing inefficient memory access operations. In this section, we compare META with other efficient attention methods and advanced adapter methods~\citep{marouf2024mini,xia2022vision,sung2022vl}. All methods are used with their default settings and the same settings as the injector and extractor in ViT-adapter~\citep{chen2022vision}. Following the same setup as in~\citep{chen2022vision}, the attention mechanism is utilized as the ViT-adapter layer. Therefore, during the experimental comparisons, we replace the attention mechanism in the ViT-adapter with alternative attention mechanisms to ensure a fair comparison. 
The obtained experimental results are given in Table~\ref{tab6}. We can observe that compared to these methods, META achieves new state-of-the-art performance in both accuracy and efficiency. We ultimately achieve an AP of $44.3\%$ with $115.3$\textbf{M} parameters, $720$\textbf{G} FLOPs, $17.4$ FPS, and 8.1 \textbf{GB} MC. 
% -------------------------------
\begin{table}[t]
\centering
\footnotesize
\renewcommand\arraystretch{1.2}
\setlength{\tabcolsep}{5pt}{
\begin{tabular}{r|ccccc}
\hline \hline 
Methods & AP$\uparrow$ & FPS$\uparrow$ & Params. (\textbf{M})$\downarrow$ & FLOPs (\textbf{G})$\downarrow$  & Momory (\textbf{GB})$\downarrow$ \\
\hline 
WindowAtt~\citep{liu2021swin} & 41.2 & 11.6 & 145.0 & 982 & 18.5 \\
PaleAttention~\citep{wu2022pale} & 42.8 & 14.4 & 155.2 & 1,029 & 16.7\\
Attention~\citep{vaswani2017attention} & 43.1 & 5.2 & 188.4 & 1,250 & 18.3 \\
CSWindow~\citep{dong2022cswin}& 43.1 & 13.7 & 144.6 & 990 & 12.9\\
SimplingAtte~\citep{he2023simplifying} & 43.3 & 12.2 & 126.3 & 994 & 17.1\\
DeformableAtt~\citep{xia2022vision} & 43.7 & 13.5 & 166.0 & 988 & 15.2 \\
\cdashline{1-6}[0.8pt/2pt]
MiniAdapters~\citep{marouf2024mini} & 41.9 & 15.0 & 131.8 & 995 & 12.2 \\
VL-Adapter~\citep{sung2022vl} & 42.7 & 14.5 & 167.2 & 993  & 14.0\\
\cellcolor[gray]{.95}\textbf{META-B$_{{\textrm{(Ours)}}}$} & \cellcolor[gray]{.95}44.3 & \cellcolor[gray]{.95}17.4 & \cellcolor[gray]{.95}115.3 & \cellcolor[gray]{.95}720 & \cellcolor[gray]{.95}8.1\\
\hline \hline 
\end{tabular}
\caption{Result comparisons with different adapters.}
\label{tab6}}
\end{table}
% -------------------------------

% -------------------------------------------
\section{Visualizations under the Conv branch}
\label{secS6}
% -------------------------------------------
{\color{red}{\emph{This supplementary is for Section~3.2 of the main paper.}}} In this section, to observe if the adapter has learned local inductive biases through the Conv branch, we visualize the model's class activation maps. The obtained visualizations are given in Figure~\ref{figs1}. From this figure, it can be observed that after adding the Conv branch, the model focuses more on the specific object area (\eg,`` the dog'' and ``the person'') rather than the surrounding area that may extend beyond the object itself, as was the case before adding the Conv branch. This indicates that our method effectively learns local inductive biases after incorporating the Conv branch.
% -------------------------------------------
% This file was created by matlab2tikz.
%
%The latest updates can be retrieved from
%  http://www.mathworks.com/matlabcentral/fileexchange/22022-matlab2tikz-matlab2tikz
%where you can also make suggestions and rate matlab2tikz.
%
\definecolor{mycolor1}{rgb}{0.21569,0.54902,0.72157}%
\definecolor{mycolor2}{rgb}{0.80784,0.16863,0.12157}%
%
\begin{tikzpicture}

\begin{axis}[%
width=0.898in,
height=1.5in,%3.603in,
at={(0.766in,0.486in)},
scale only axis,
xmin=0,
xmax=10,
ymin=0,
ymax=0.8,
xlabel= \phantom{$z$},
ylabel=$p(g_{z^*}|Y)$,
ylabel near ticks,
title={Linearization-based\\ approach},
title style={align=left}, 
axis background/.style={fill=white},
axis x line*=bottom,
axis y line*=left,
legend style={legend cell align=left, align=left, draw=white!15!black}
]
\addplot[ybar interval, fill=mycolor1, fill opacity=0.4, draw=mycolor1, area legend] table[row sep=crcr] {%
x	y\\
3.36	0.0144927536231884\\
3.429	0.0289855072463768\\
3.498	0.0869565217391305\\
3.567	0.217391304347825\\
3.636	0.391304347826087\\
3.705	0.565217391304348\\
3.774	0.449275362318841\\
3.843	0.405797101449276\\
3.912	0.666666666666667\\
3.981	0.420289855072464\\
4.05	0.478260869565218\\
4.119	0.289855072463768\\
4.188	0.289855072463768\\
4.257	0.347826086956522\\
4.326	0.246376811594203\\
4.395	0.304347826086953\\
4.464	0.20289855072464\\
4.533	0.217391304347823\\
4.602	0.246376811594203\\
4.671	0.289855072463768\\
4.74	0.246376811594203\\
4.809	0.188405797101449\\
4.878	0.231884057971015\\
4.947	0.27536231884058\\
5.016	0.391304347826087\\
5.085	0.246376811594203\\
5.154	0.27536231884058\\
5.223	0.217391304347826\\
5.292	0.347826086956518\\
5.361	0.231884057971018\\
5.43	0.2463768115942\\
5.499	0.260869565217395\\
5.568	0.275362318840576\\
5.637	0.289855072463772\\
5.706	0.304347826086953\\
5.775	0.289855072463772\\
5.844	0.362318840579706\\
5.913	0.289855072463768\\
5.982	0.463768115942029\\
6.051	0.420289855072464\\
6.12	0.492753623188406\\
6.189	0.463768115942029\\
6.258	0.463768115942029\\
6.327	0.420289855072459\\
6.396	0.246376811594206\\
6.465	0.202898550724635\\
6.534	0.0869565217391316\\
6.603	0.0579710144927529\\
6.672	0.0289855072463772\\
6.741	0.0144927536231882\\
6.81	0.0144927536231882\\
};
%\addlegendentry{ground truth}

\addplot [color=mycolor2, line width=2.0pt]
  table[row sep=crcr]{%
0	0.00495647934021539\\
0.01	0.00503120737369003\\
0.02	0.00510691052511148\\
0.03	0.00518359894024572\\
0.04	0.00526128282899922\\
0.05	0.00533997246516116\\
0.06	0.00541967818613227\\
0.07	0.00550041039264009\\
0.08	0.00558217954844049\\
0.09	0.00566499618000535\\
0.1	0.00574887087619621\\
0.11	0.00583381428792371\\
0.12	0.00591983712779276\\
0.13	0.00600695016973323\\
0.14	0.006095164248616\\
0.15	0.00618449025985429\\
0.16	0.00627493915898998\\
0.17	0.00636652196126503\\
0.18	0.0064592497411776\\
0.19	0.0065531336320228\\
0.2	0.00664818482541805\\
0.21	0.0067444145708128\\
0.22	0.0068418341749824\\
0.23	0.00694045500150619\\
0.24	0.00704028847022945\\
0.25	0.00714134605670927\\
0.26	0.00724363929164408\\
0.27	0.00734717976028668\\
0.28	0.00745197910184082\\
0.29	0.00755804900884096\\
0.3	0.00766540122651533\\
0.31	0.00777404755213196\\
0.32	0.00788399983432761\\
0.33	0.00799526997241961\\
0.34	0.00810786991570027\\
0.35	0.00822181166271396\\
0.36	0.00833710726051655\\
0.37	0.00845376880391716\\
0.38	0.00857180843470236\\
0.39	0.00869123834084212\\
0.4	0.00881207075567811\\
0.41	0.00893431795709363\\
0.42	0.00905799226666551\\
0.43	0.00918310604879767\\
0.44	0.00930967170983618\\
0.45	0.00943770169716588\\
0.46	0.00956720849828854\\
0.47	0.00969820463988202\\
0.48	0.00983070268684103\\
0.49	0.00996471524129867\\
0.5	0.0101002549416293\\
0.51	0.0102373344614323\\
0.52	0.0103759665084967\\
0.53	0.0105161638237465\\
0.54	0.0106579391801673\\
0.55	0.0108013053817126\\
0.56	0.010946275262192\\
0.57	0.0110928616841387\\
0.58	0.0112410775376588\\
0.59	0.0113909357392594\\
0.6	0.0115424492306589\\
0.61	0.0116956309775763\\
0.62	0.0118504939685016\\
0.63	0.0120070512134457\\
0.64	0.0121653157426715\\
0.65	0.012325300605404\\
0.66	0.012487018868522\\
0.67	0.0126504836152284\\
0.68	0.0128157079437017\\
0.69	0.0129827049657272\\
0.7	0.0131514878053083\\
0.71	0.0133220695972577\\
0.72	0.0134944634857689\\
0.73	0.0136686826229678\\
0.74	0.0138447401674437\\
0.75	0.0140226492827617\\
0.76	0.0142024231359536\\
0.77	0.0143840748959895\\
0.78	0.0145676177322305\\
0.79	0.0147530648128596\\
0.8	0.0149404293032943\\
0.81	0.0151297243645786\\
0.82	0.0153209631517556\\
0.83	0.0155141588122201\\
0.84	0.0157093244840515\\
0.85	0.0159064732943274\\
0.86	0.0161056183574168\\
0.87	0.016306772773255\\
0.88	0.0165099496255975\\
0.89	0.0167151619802561\\
0.9	0.0169224228833143\\
0.91	0.017131745359324\\
0.92	0.0173431424094836\\
0.93	0.0175566270097958\\
0.94	0.0177722121092072\\
0.95	0.0179899106277294\\
0.96	0.0182097354545399\\
0.97	0.018431699446066\\
0.98	0.0186558154240489\\
0.99	0.0188820961735898\\
1	0.0191105544411779\\
1.01	0.0193412029326999\\
1.02	0.0195740543114314\\
1.03	0.0198091211960107\\
1.04	0.0200464161583947\\
1.05	0.0202859517217969\\
1.06	0.0205277403586084\\
1.07	0.0207717944883015\\
1.08	0.0210181264753159\\
1.09	0.0212667486269282\\
1.1	0.0215176731911045\\
1.11	0.0217709123543368\\
1.12	0.0220264782394623\\
1.13	0.0222843829034672\\
1.14	0.0225446383352741\\
1.15	0.0228072564535138\\
1.16	0.0230722491042814\\
1.17	0.0233396280588773\\
1.18	0.0236094050115328\\
1.19	0.0238815915771207\\
1.2	0.0241561992888519\\
1.21	0.0244332395959568\\
1.22	0.0247127238613533\\
1.23	0.0249946633592999\\
1.24	0.0252790692730363\\
1.25	0.0255659526924097\\
1.26	0.0258553246114888\\
1.27	0.026147195926164\\
1.28	0.0264415774317359\\
1.29	0.0267384798204914\\
1.3	0.0270379136792673\\
1.31	0.0273398894870028\\
1.32	0.0276444176122807\\
1.33	0.0279515083108568\\
1.34	0.0282611717231799\\
1.35	0.0285734178718998\\
1.36	0.0288882566593667\\
1.37	0.0292056978651199\\
1.38	0.0295257511433677\\
1.39	0.0298484260204579\\
1.4	0.0301737318923396\\
1.41	0.0305016780220172\\
1.42	0.0308322735369956\\
1.43	0.0311655274267189\\
1.44	0.0315014485400004\\
1.45	0.0318400455824475\\
1.46	0.0321813271138789\\
1.47	0.032525301545736\\
1.48	0.0328719771384891\\
1.49	0.0332213619990379\\
1.5	0.0335734640781073\\
1.51	0.0339282911676388\\
1.52	0.034285850898178\\
1.53	0.0346461507362581\\
1.54	0.0350091979817809\\
1.55	0.0353749997653947\\
1.56	0.0357435630458696\\
1.57	0.0361148946074719\\
1.58	0.0364890010573359\\
1.59	0.0368658888228363\\
1.6	0.0372455641489584\\
1.61	0.0376280330956701\\
1.62	0.0380133015352924\\
1.63	0.0384013751498731\\
1.64	0.0387922594285599\\
1.65	0.0391859596649766\\
1.66	0.0395824809546015\\
1.67	0.0399818281921483\\
1.68	0.040384006068951\\
1.69	0.0407890190703521\\
1.7	0.0411968714730958\\
1.71	0.0416075673427256\\
1.72	0.0420211105309874\\
1.73	0.0424375046732393\\
1.74	0.0428567531858665\\
1.75	0.0432788592637046\\
1.76	0.0437038258774694\\
1.77	0.0441316557711956\\
1.78	0.0445623514596833\\
1.79	0.0449959152259539\\
1.8	0.0454323491187165\\
1.81	0.0458716549498434\\
1.82	0.0463138342918566\\
1.83	0.0467588884754263\\
1.84	0.047206818586881\\
1.85	0.0476576254657295\\
1.86	0.0481113097021966\\
1.87	0.0485678716347722\\
1.88	0.0490273113477746\\
1.89	0.0494896286689283\\
1.9	0.0499548231669573\\
1.91	0.0504228941491942\\
1.92	0.0508938406592059\\
1.93	0.0513676614744362\\
1.94	0.0518443551038658\\
1.95	0.0523239197856913\\
1.96	0.0528063534850216\\
1.97	0.0532916538915952\\
1.98	0.0537798184175165\\
1.99	0.0542708441950129\\
2	0.0547647280742127\\
2.01	0.0552614666209458\\
2.02	0.0557610561145651\\
2.03	0.0562634925457924\\
2.04	0.0567687716145866\\
2.05	0.0572768887280368\\
2.06	0.0577878389982796\\
2.07	0.0583016172404419\\
2.08	0.0588182179706094\\
2.09	0.0593376354038221\\
2.1	0.0598598634520962\\
2.11	0.0603848957224738\\
2.12	0.0609127255151013\\
2.13	0.0614433458213363\\
2.14	0.0619767493218837\\
2.15	0.0625129283849627\\
2.16	0.063051875064503\\
2.17	0.0635935810983738\\
2.18	0.0641380379066434\\
2.19	0.0646852365898719\\
2.2	0.0652351679274361\\
2.21	0.0657878223758891\\
2.22	0.0663431900673527\\
2.23	0.0669012608079454\\
2.24	0.0674620240762456\\
2.25	0.0680254690217902\\
2.26	0.06859158446361\\
2.27	0.0691603588888017\\
2.28	0.0697317804511384\\
2.29	0.0703058369697168\\
2.3	0.070882515927644\\
2.31	0.0714618044707637\\
2.32	0.0720436894064212\\
2.33	0.0726281572022699\\
2.34	0.0732151939851174\\
2.35	0.0738047855398146\\
2.36	0.0743969173081847\\
2.37	0.0749915743879969\\
2.38	0.0755887415319811\\
2.39	0.0761884031468874\\
2.4	0.0767905432925892\\
2.41	0.0773951456812309\\
2.42	0.0780021936764202\\
2.43	0.0786116702924671\\
2.44	0.0792235581936674\\
2.45	0.079837839693634\\
2.46	0.0804544967546748\\
2.47	0.0810735109872175\\
2.48	0.0816948636492834\\
2.49	0.0823185356460092\\
2.5	0.0829445075292169\\
2.51	0.0835727594970344\\
2.52	0.084203271393565\\
2.53	0.0848360227086074\\
2.54	0.0854709925774259\\
2.55	0.0861081597805728\\
2.56	0.0867475027437606\\
2.57	0.0873889995377879\\
2.58	0.0880326278785163\\
2.59	0.0886783651269002\\
2.6	0.0893261882890704\\
2.61	0.0899760740164704\\
2.62	0.0906279986060465\\
2.63	0.0912819380004926\\
2.64	0.0919378677885496\\
2.65	0.0925957632053583\\
2.66	0.0932555991328697\\
2.67	0.0939173501003085\\
2.68	0.0945809902846947\\
2.69	0.0952464935114191\\
2.7	0.0959138332548775\\
2.71	0.0965829826391593\\
2.72	0.0972539144387952\\
2.73	0.0979266010795609\\
2.74	0.0986010146393384\\
2.75	0.0992771268490357\\
2.76	0.0999549090935642\\
2.77	0.100634332412874\\
2.78	0.101315367503047\\
2.79	0.101997984717452\\
2.8	0.102682154067954\\
2.81	0.103367845226185\\
2.82	0.104055027524874\\
2.83	0.104743669959238\\
2.84	0.105433741188425\\
2.85	0.106125209537028\\
2.86	0.106818042996652\\
2.87	0.107512209227537\\
2.88	0.108207675560252\\
2.89	0.108904408997439\\
2.9	0.109602376215622\\
2.91	0.110301543567076\\
2.92	0.111001877081754\\
2.93	0.111703342469276\\
2.94	0.112405905120979\\
2.95	0.113109530112027\\
2.96	0.113814182203577\\
2.97	0.114519825845015\\
2.98	0.115226425176241\\
2.99	0.115933944030024\\
3	0.116642345934409\\
3.01	0.117351594115192\\
3.02	0.118061651498449\\
3.03	0.118772480713129\\
3.04	0.119484044093701\\
3.05	0.120196303682873\\
3.06	0.120909221234355\\
3.07	0.121622758215693\\
3.08	0.122336875811159\\
3.09	0.123051534924699\\
3.1	0.123766696182943\\
3.11	0.124482319938267\\
3.12	0.125198366271925\\
3.13	0.12591479499723\\
3.14	0.126631565662796\\
3.15	0.127348637555838\\
3.16	0.128065969705534\\
3.17	0.128783520886434\\
3.18	0.129501249621939\\
3.19	0.130219114187826\\
3.2	0.130937072615837\\
3.21	0.131655082697321\\
3.22	0.132373101986931\\
3.23	0.133091087806378\\
3.24	0.133808997248241\\
3.25	0.13452678717983\\
3.26	0.135244414247106\\
3.27	0.135961834878651\\
3.28	0.136679005289694\\
3.29	0.137395881486191\\
3.3	0.138112419268956\\
3.31	0.138828574237848\\
3.32	0.139544301796001\\
3.33	0.140259557154114\\
3.34	0.14097429533479\\
3.35	0.141688471176923\\
3.36	0.142402039340136\\
3.37	0.143114954309265\\
3.38	0.143827170398901\\
3.39	0.144538641757969\\
3.4	0.145249322374359\\
3.41	0.145959166079606\\
3.42	0.146668126553618\\
3.43	0.147376157329438\\
3.44	0.148083211798065\\
3.45	0.148789243213316\\
3.46	0.149494204696722\\
3.47	0.150198049242482\\
3.48	0.150900729722448\\
3.49	0.151602198891158\\
3.5	0.152302409390905\\
3.51	0.153001313756856\\
3.52	0.153698864422194\\
3.53	0.154395013723319\\
3.54	0.155089713905069\\
3.55	0.155782917125992\\
3.56	0.156474575463644\\
3.57	0.157164640919932\\
3.58	0.157853065426486\\
3.59	0.158539800850068\\
3.6	0.15922479899801\\
3.61	0.159908011623694\\
3.62	0.160589390432051\\
3.63	0.161268887085104\\
3.64	0.16194645320753\\
3.65	0.16262204039226\\
3.66	0.163295600206099\\
3.67	0.163967084195381\\
3.68	0.164636443891645\\
3.69	0.165303630817341\\
3.7	0.165968596491557\\
3.71	0.166631292435772\\
3.72	0.167291670179631\\
3.73	0.167949681266742\\
3.74	0.168605277260496\\
3.75	0.169258409749901\\
3.76	0.169909030355445\\
3.77	0.170557090734967\\
3.78	0.171202542589549\\
3.79	0.171845337669428\\
3.8	0.17248542777992\\
3.81	0.173122764787353\\
3.82	0.173757300625023\\
3.83	0.174388987299154\\
3.84	0.175017776894874\\
3.85	0.1756436215822\\
3.86	0.176266473622029\\
3.87	0.176886285372142\\
3.88	0.177503009293208\\
3.89	0.178116597954805\\
3.9	0.178727004041434\\
3.91	0.179334180358544\\
3.92	0.179938079838555\\
3.93	0.180538655546892\\
3.94	0.181135860688006\\
3.95	0.181729648611403\\
3.96	0.182319972817676\\
3.97	0.182906786964519\\
3.98	0.183490044872755\\
3.99	0.184069700532347\\
4	0.184645708108407\\
4.01	0.185218021947199\\
4.02	0.185786596582135\\
4.03	0.186351386739756\\
4.04	0.186912347345709\\
4.05	0.187469433530713\\
4.06	0.188022600636505\\
4.07	0.188571804221778\\
4.08	0.189117000068111\\
4.09	0.189658144185865\\
4.1	0.190195192820082\\
4.11	0.190728102456354\\
4.12	0.191256829826677\\
4.13	0.191781331915283\\
4.14	0.192301565964453\\
4.15	0.192817489480306\\
4.16	0.193329060238567\\
4.17	0.193836236290307\\
4.18	0.194338975967659\\
4.19	0.19483723788951\\
4.2	0.195330980967159\\
4.21	0.195820164409952\\
4.22	0.196304747730887\\
4.23	0.196784690752181\\
4.24	0.197259953610816\\
4.25	0.197730496764038\\
4.26	0.198196280994839\\
4.27	0.198657267417387\\
4.28	0.19911341748243\\
4.29	0.199564692982658\\
4.3	0.200011056058033\\
4.31	0.200452469201067\\
4.32	0.200888895262076\\
4.33	0.201320297454377\\
4.34	0.201746639359454\\
4.35	0.202167884932071\\
4.36	0.202583998505353\\
4.37	0.202994944795806\\
4.38	0.203400688908302\\
4.39	0.203801196341013\\
4.4	0.204196432990295\\
4.41	0.204586365155525\\
4.42	0.204970959543887\\
4.43	0.205350183275104\\
4.44	0.205724003886121\\
4.45	0.206092389335735\\
4.46	0.206455308009166\\
4.47	0.206812728722581\\
4.48	0.207164620727553\\
4.49	0.207510953715471\\
4.5	0.207851697821886\\
4.51	0.208186823630804\\
4.52	0.208516302178914\\
4.53	0.208840104959761\\
4.54	0.209158203927852\\
4.55	0.209470571502708\\
4.56	0.209777180572843\\
4.57	0.210078004499687\\
4.58	0.210373017121445\\
4.59	0.210662192756884\\
4.6	0.21094550620906\\
4.61	0.211222932768978\\
4.62	0.211494448219179\\
4.63	0.211760028837266\\
4.64	0.212019651399358\\
4.65	0.212273293183472\\
4.66	0.212520931972839\\
4.67	0.212762546059146\\
4.68	0.212998114245707\\
4.69	0.213227615850562\\
4.7	0.213451030709505\\
4.71	0.213668339179034\\
4.72	0.21387952213923\\
4.73	0.214084560996561\\
4.74	0.214283437686611\\
4.75	0.214476134676732\\
4.76	0.214662634968617\\
4.77	0.214842922100803\\
4.78	0.215016980151093\\
4.79	0.215184793738895\\
4.8	0.21534634802749\\
4.81	0.21550162872622\\
4.82	0.215650622092589\\
4.83	0.215793314934298\\
4.84	0.215929694611184\\
4.85	0.216059749037091\\
4.86	0.216183466681654\\
4.87	0.216300836572001\\
4.88	0.21641184829438\\
4.89	0.21651649199569\\
4.9	0.216614758384949\\
4.91	0.21670663873466\\
4.92	0.216792124882109\\
4.93	0.21687120923057\\
4.94	0.216943884750432\\
4.95	0.21701014498024\\
4.96	0.217069984027651\\
4.97	0.21712339657031\\
4.98	0.217170377856636\\
4.99	0.217210923706529\\
5	0.21724503051199\\
5.01	0.217272695237652\\
5.02	0.217293915421236\\
5.03	0.217308689173911\\
5.04	0.217317015180578\\
5.05	0.217318892700066\\
5.06	0.217314321565234\\
5.07	0.217303302183007\\
5.08	0.217285835534308\\
5.09	0.217261923173913\\
5.1	0.217231567230225\\
5.11	0.217194770404953\\
5.12	0.217151535972715\\
5.13	0.217101867780549\\
5.14	0.217045770247346\\
5.15	0.216983248363191\\
5.16	0.216914307688628\\
5.17	0.21683895435383\\
5.18	0.216757195057696\\
5.19	0.216669037066854\\
5.2	0.216574488214588\\
5.21	0.216473556899676\\
5.22	0.216366252085147\\
5.23	0.216252583296955\\
5.24	0.216132560622568\\
5.25	0.216006194709479\\
5.26	0.215873496763628\\
5.27	0.215734478547749\\
5.28	0.21558915237963\\
5.29	0.215437531130296\\
5.3	0.215279628222107\\
5.31	0.215115457626779\\
5.32	0.214945033863324\\
5.33	0.21476837199591\\
5.34	0.214585487631638\\
5.35	0.21439639691825\\
5.36	0.21420111654175\\
5.37	0.213999663723949\\
5.38	0.213792056219935\\
5.39	0.213578312315465\\
5.4	0.213358450824282\\
5.41	0.213132491085352\\
5.42	0.212900452960033\\
5.43	0.212662356829162\\
5.44	0.212418223590075\\
5.45	0.212168074653546\\
5.46	0.211911931940665\\
5.47	0.211649817879627\\
5.48	0.211381755402469\\
5.49	0.21110776794172\\
5.5	0.210827879426989\\
5.51	0.210542114281486\\
5.52	0.210250497418463\\
5.53	0.209953054237604\\
5.54	0.209649810621332\\
5.55	0.209340792931057\\
5.56	0.209026028003361\\
5.57	0.208705543146111\\
5.58	0.208379366134512\\
5.59	0.2080475252071\\
5.6	0.207710049061664\\
5.61	0.207366966851112\\
5.62	0.207018308179279\\
5.63	0.206664103096667\\
5.64	0.206304382096132\\
5.65	0.205939176108512\\
5.66	0.205568516498194\\
5.67	0.20519243505863\\
5.68	0.204810964007791\\
5.69	0.204424135983574\\
5.7	0.204031984039149\\
5.71	0.203634541638252\\
5.72	0.203231842650435\\
5.73	0.202823921346256\\
5.74	0.202410812392424\\
5.75	0.201992550846891\\
5.76	0.201569172153899\\
5.77	0.201140712138983\\
5.78	0.200707207003916\\
5.79	0.200268693321625\\
5.8	0.199825208031051\\
5.81	0.199376788431969\\
5.82	0.198923472179769\\
5.83	0.198465297280192\\
5.84	0.198002302084028\\
5.85	0.197534525281773\\
5.86	0.197062005898254\\
5.87	0.196584783287207\\
5.88	0.196102897125829\\
5.89	0.195616387409288\\
5.9	0.195125294445206\\
5.91	0.1946296588481\\
5.92	0.194129521533804\\
5.93	0.193624923713846\\
5.94	0.193115906889808\\
5.95	0.192602512847652\\
5.96	0.192084783652017\\
5.97	0.191562761640493\\
5.98	0.19103648941787\\
5.99	0.19050600985036\\
6	0.189971366059799\\
6.01	0.189432601417827\\
6.02	0.188889759540044\\
6.03	0.188342884280149\\
6.04	0.187792019724062\\
6.05	0.187237210184025\\
6.06	0.186678500192687\\
6.07	0.186115934497179\\
6.08	0.185549558053167\\
6.09	0.184979416018895\\
6.1	0.184405553749221\\
6.11	0.183828016789635\\
6.12	0.183246850870269\\
6.13	0.182662101899903\\
6.14	0.182073815959955\\
6.15	0.181482039298473\\
6.16	0.180886818324118\\
6.17	0.18028819960014\\
6.18	0.179686229838354\\
6.19	0.179080955893116\\
6.2	0.17847242475529\\
6.21	0.177860683546225\\
6.22	0.177245779511726\\
6.23	0.176627760016027\\
6.24	0.176006672535775\\
6.25	0.175382564654008\\
6.26	0.174755484054144\\
6.27	0.174125478513977\\
6.28	0.173492595899676\\
6.29	0.1728568841598\\
6.3	0.172218391319313\\
6.31	0.171577165473615\\
6.32	0.170933254782588\\
6.33	0.170286707464643\\
6.34	0.169637571790795\\
6.35	0.168985896078742\\
6.36	0.168331728686962\\
6.37	0.167675118008831\\
6.38	0.167016112466753\\
6.39	0.166354760506308\\
6.4	0.165691110590427\\
6.41	0.16502521119358\\
6.42	0.164357110795988\\
6.43	0.163686857877857\\
6.44	0.163014500913636\\
6.45	0.162340088366297\\
6.46	0.161663668681645\\
6.47	0.160985290282647\\
6.48	0.160305001563796\\
6.49	0.159622850885492\\
6.5	0.158938886568466\\
6.51	0.15825315688822\\
6.52	0.157565710069504\\
6.53	0.156876594280826\\
6.54	0.156185857628989\\
6.55	0.155493548153663\\
6.56	0.154799713821993\\
6.57	0.154104402523241\\
6.58	0.153407662063457\\
6.59	0.152709540160195\\
6.6	0.152010084437263\\
6.61	0.151309342419508\\
6.62	0.15060736152764\\
6.63	0.1499041890731\\
6.64	0.149199872252962\\
6.65	0.148494458144878\\
6.66	0.147787993702068\\
6.67	0.147080525748341\\
6.68	0.146372100973175\\
6.69	0.145662765926825\\
6.7	0.144952567015486\\
6.71	0.144241550496492\\
6.72	0.14352976247357\\
6.73	0.14281724889213\\
6.74	0.142104055534609\\
6.75	0.141390228015859\\
6.76	0.140675811778582\\
6.77	0.139960852088818\\
6.78	0.139245394031472\\
6.79	0.138529482505904\\
6.8	0.137813162221557\\
6.81	0.137096477693644\\
6.82	0.136379473238879\\
6.83	0.135662192971265\\
6.84	0.134944680797929\\
6.85	0.134226980415018\\
6.86	0.133509135303637\\
6.87	0.132791188725845\\
6.88	0.132073183720711\\
6.89	0.131355163100412\\
6.9	0.130637169446397\\
6.91	0.129919245105597\\
6.92	0.129201432186696\\
6.93	0.128483772556459\\
6.94	0.127766307836106\\
6.95	0.127049079397757\\
6.96	0.126332128360921\\
6.97	0.12561549558905\\
6.98	0.124899221686146\\
6.99	0.124183346993427\\
7	0.123467911586051\\
7.01	0.122752955269899\\
7.02	0.122038517578414\\
7.03	0.1213246377695\\
7.04	0.12061135482248\\
7.05	0.119898707435113\\
7.06	0.119186734020668\\
7.07	0.118475472705061\\
7.08	0.117764961324047\\
7.09	0.117055237420477\\
7.1	0.116346338241611\\
7.11	0.11563830073649\\
7.12	0.114931161553372\\
7.13	0.114224957037223\\
7.14	0.113519723227275\\
7.15	0.112815495854637\\
7.16	0.112112310339969\\
7.17	0.111410201791219\\
7.18	0.110709205001419\\
7.19	0.110009354446535\\
7.2	0.109310684283388\\
7.21	0.108613228347628\\
7.22	0.107917020151773\\
7.23	0.107222092883299\\
7.24	0.106528479402805\\
7.25	0.105836212242227\\
7.26	0.105145323603112\\
7.27	0.104455845354962\\
7.28	0.103767809033623\\
7.29	0.103081245839751\\
7.3	0.102396186637321\\
7.31	0.101712661952208\\
7.32	0.101030701970822\\
7.33	0.100350336538802\\
7.34	0.0996715951597692\\
7.35	0.0989945069941437\\
7.36	0.0983191008580132\\
7.37	0.0976454052220644\\
7.38	0.0969734482105724\\
7.39	0.0963032576004468\\
7.4	0.0956348608203369\\
7.41	0.0949682849497942\\
7.42	0.0943035567184924\\
7.43	0.093640702505504\\
7.44	0.0929797483386348\\
7.45	0.0923207198938145\\
7.46	0.0916636424945427\\
7.47	0.0910085411113931\\
7.48	0.0903554403615708\\
7.49	0.0897043645085274\\
7.5	0.0890553374616292\\
7.51	0.0884083827758814\\
7.52	0.0877635236517063\\
7.53	0.0871207829347749\\
7.54	0.0864801831158936\\
7.55	0.0858417463309428\\
7.56	0.0852054943608689\\
7.57	0.0845714486317291\\
7.58	0.083939630214788\\
7.59	0.0833100598266659\\
7.6	0.0826827578295388\\
7.61	0.0820577442313888\\
7.62	0.0814350386863051\\
7.63	0.0808146604948357\\
7.64	0.0801966286043874\\
7.65	0.0795809616096762\\
7.66	0.0789676777532255\\
7.67	0.0783567949259133\\
7.68	0.0777483306675663\\
7.69	0.0771423021676022\\
7.7	0.0765387262657182\\
7.71	0.0759376194526259\\
7.72	0.0753389978708321\\
7.73	0.0747428773154653\\
7.74	0.0741492732351464\\
7.75	0.0735582007329042\\
7.76	0.0729696745671344\\
7.77	0.0723837091526021\\
7.78	0.071800318561487\\
7.79	0.0712195165244711\\
7.8	0.0706413164318677\\
7.81	0.0700657313347914\\
7.82	0.0694927739463699\\
7.83	0.0689224566429943\\
7.84	0.0683547914656103\\
7.85	0.0677897901210473\\
7.86	0.067227463983387\\
7.87	0.0666678240953688\\
7.88	0.0661108811698337\\
7.89	0.0655566455912037\\
7.9	0.0650051274169981\\
7.91	0.0644563363793857\\
7.92	0.0639102818867709\\
7.93	0.0633669730254157\\
7.94	0.0628264185610945\\
7.95	0.0622886269407826\\
7.96	0.0617536062943775\\
7.97	0.0612213644364519\\
7.98	0.060691908868039\\
7.99	0.0601652467784482\\
8	0.0596413850471111\\
8.01	0.0591203302454577\\
8.02	0.0586020886388212\\
8.03	0.0580866661883726\\
8.04	0.0575740685530815\\
8.05	0.0570643010917059\\
8.06	0.0565573688648083\\
8.07	0.0560532766367973\\
8.08	0.0555520288779961\\
8.09	0.0550536297667352\\
8.1	0.0545580831914697\\
8.11	0.0540653927529206\\
8.12	0.0535755617662392\\
8.13	0.0530885932631942\\
8.14	0.0526044899943808\\
8.15	0.052123254431452\\
8.16	0.0516448887693693\\
8.17	0.0511693949286748\\
8.18	0.0506967745577832\\
8.19	0.050227029035292\\
8.2	0.0497601594723113\\
8.21	0.0492961667148101\\
8.22	0.0488350513459819\\
8.23	0.0483768136886253\\
8.24	0.0479214538075419\\
8.25	0.0474689715119493\\
8.26	0.0470193663579096\\
8.27	0.0465726376507721\\
8.28	0.0461287844476299\\
8.29	0.0456878055597898\\
8.3	0.0452496995552554\\
8.31	0.0448144647612222\\
8.32	0.0443820992665838\\
8.33	0.0439526009244504\\
8.34	0.0435259673546765\\
8.35	0.0431021959463999\\
8.36	0.0426812838605893\\
8.37	0.042263228032601\\
8.38	0.041848025174744\\
8.39	0.0414356717788534\\
8.4	0.0410261641188703\\
8.41	0.0406194982534291\\
8.42	0.0402156700284508\\
8.43	0.0398146750797419\\
8.44	0.039416508835599\\
8.45	0.0390211665194178\\
8.46	0.0386286431523058\\
8.47	0.0382389335556999\\
8.48	0.0378520323539862\\
8.49	0.0374679339771224\\
8.5	0.0370866326632633\\
8.51	0.0367081224613873\\
8.52	0.0363323972339243\\
8.53	0.0359594506593843\\
8.54	0.0355892762349867\\
8.55	0.0352218672792885\\
8.56	0.034857216934813\\
8.57	0.0344953181706767\\
8.58	0.0341361637852141\\
8.59	0.0337797464086021\\
8.6	0.0334260585054804\\
8.61	0.0330750923775696\\
8.62	0.0327268401662861\\
8.63	0.0323812938553532\\
8.64	0.0320384452734075\\
8.65	0.0316982860966018\\
8.66	0.0313608078512014\\
8.67	0.0310260019161767\\
8.68	0.030693859525789\\
8.69	0.0303643717721696\\
8.7	0.0300375296078941\\
8.71	0.0297133238485476\\
8.72	0.0293917451752839\\
8.73	0.0290727841373765\\
8.74	0.0287564311547617\\
8.75	0.0284426765205727\\
8.76	0.0281315104036655\\
8.77	0.0278229228511355\\
8.78	0.027516903790824\\
8.79	0.0272134430338157\\
8.8	0.0269125302769252\\
8.81	0.0266141551051738\\
8.82	0.0263183069942544\\
8.83	0.0260249753129863\\
8.84	0.0257341493257578\\
8.85	0.0254458181949572\\
8.86	0.0251599709833921\\
8.87	0.0248765966566958\\
8.88	0.0245956840857209\\
8.89	0.0243172220489209\\
8.9	0.0240411992347177\\
8.91	0.0237676042438558\\
8.92	0.0234964255917429\\
8.93	0.0232276517107765\\
8.94	0.0229612709526558\\
8.95	0.0226972715906801\\
8.96	0.0224356418220311\\
8.97	0.0221763697700415\\
8.98	0.0219194434864471\\
8.99	0.0216648509536248\\
9	0.021412580086814\\
9.01	0.0211626187363224\\
9.02	0.0209149546897159\\
9.03	0.0206695756739917\\
9.04	0.0204264693577359\\
9.05	0.0201856233532632\\
9.06	0.0199470252187409\\
9.07	0.0197106624602952\\
9.08	0.0194765225341003\\
9.09	0.0192445928484506\\
9.1	0.0190148607658151\\
9.11	0.0187873136048738\\
9.12	0.018561938642537\\
9.13	0.0183387231159461\\
9.14	0.0181176542244563\\
9.15	0.0178987191316015\\
9.16	0.01768190496704\\
9.17	0.0174671988284827\\
9.18	0.0172545877836021\\
9.19	0.0170440588719226\\
9.2	0.0168355991066919\\
9.21	0.0166291954767339\\
9.22	0.0164248349482823\\
9.23	0.0162225044667949\\
9.24	0.0160221909587488\\
9.25	0.0158238813334166\\
9.26	0.015627562484623\\
9.27	0.0154332212924815\\
9.28	0.015240844625113\\
9.29	0.0150504193403427\\
9.3	0.0148619322873796\\
9.31	0.0146753703084749\\
9.32	0.0144907202405607\\
9.33	0.0143079689168702\\
9.34	0.0141271031685362\\
9.35	0.0139481098261715\\
9.36	0.0137709757214283\\
9.37	0.0135956876885382\\
9.38	0.0134222325658323\\
9.39	0.0132505971972412\\
9.4	0.013080768433775\\
9.41	0.0129127331349834\\
9.42	0.0127464781703963\\
9.43	0.0125819904209438\\
9.44	0.0124192567803563\\
9.45	0.0122582641565455\\
9.46	0.0120989994729642\\
9.47	0.0119414496699472\\
9.48	0.011785601706032\\
9.49	0.0116314425592593\\
9.5	0.0114789592284545\\
9.51	0.0113281387344882\\
9.52	0.0111789681215181\\
9.53	0.0110314344582108\\
9.54	0.0108855248389431\\
9.55	0.0107412263849851\\
9.56	0.0105985262456626\\
9.57	0.0104574115995006\\
9.58	0.0103178696553468\\
9.59	0.010179887653476\\
9.6	0.0100434528666753\\
9.61	0.00990855260130987\\
9.62	0.00977517419836908\\
9.63	0.00964330503449418\\
9.64	0.00951293252298657\\
9.65	0.00938404411479698\\
9.66	0.00925662729949601\\
9.67	0.00913066960622556\\
9.68	0.00900615860463183\\
9.69	0.00888308190577935\\
9.7	0.00876142716304676\\
9.71	0.00864118207300384\\
9.72	0.00852233437627048\\
9.73	0.00840487185835706\\
9.74	0.00828878235048692\\
9.75	0.0081740537304007\\
9.76	0.00806067392314265\\
9.77	0.00794863090182918\\
9.78	0.0078379126883997\\
9.79	0.00772850735434974\\
9.8	0.00762040302144666\\
9.81	0.00751358786242797\\
9.82	0.00740805010168226\\
9.83	0.00730377801591319\\
9.84	0.00720075993478624\\
9.85	0.00709898424155877\\
9.86	0.00699843937369316\\
9.87	0.00689911382345334\\
9.88	0.00680099613848483\\
9.89	0.00670407492237845\\
9.9	0.00660833883521762\\
9.91	0.00651377659410972\\
9.92	0.00642037697370137\\
9.93	0.00632812880667791\\
9.94	0.00623702098424713\\
9.95	0.00614704245660751\\
9.96	0.00605818223340095\\
9.97	0.00597042938415035\\
9.98	0.00588377303868186\\
9.99	0.00579820238753227\\
10	0.00571370668234159\\
};
%\addlegendentry{linearization}

\addplot [color=mycolor2, line width=2.0pt, forget plot]
  table[row sep=crcr]{%
5.04791147756762	0\\
5.04791147756762	0.6\\
};
\addplot [color=mycolor1, dashed, line width=2.0pt, forget plot]
  table[row sep=crcr]{%
5.0284309151552	0\\
5.0284309151552	0.6\\
};
\end{axis}

\begin{axis}[%
width=0.898in,
height=1.5in,%3.603in,
at={(1.981in,0.486in)},
scale only axis,
xmin=0,
xmax=10,
ymin=0,
ymax=0.8,
axis background/.style={fill=white},
title={Exact moment \\ matching},
title style={align=left}, 
axis x line*=bottom,
axis y line*=left,
legend style={legend cell align=left, align=left, draw=white!15!black}
]
\addplot[ybar interval, fill=mycolor1, fill opacity=0.4, draw=mycolor1, area legend] table[row sep=crcr] {%
x	y\\
3.36	0.0144927536231884\\
3.429	0.0289855072463768\\
3.498	0.0869565217391305\\
3.567	0.217391304347825\\
3.636	0.391304347826087\\
3.705	0.565217391304348\\
3.774	0.449275362318841\\
3.843	0.405797101449276\\
3.912	0.666666666666667\\
3.981	0.420289855072464\\
4.05	0.478260869565218\\
4.119	0.289855072463768\\
4.188	0.289855072463768\\
4.257	0.347826086956522\\
4.326	0.246376811594203\\
4.395	0.304347826086953\\
4.464	0.20289855072464\\
4.533	0.217391304347823\\
4.602	0.246376811594203\\
4.671	0.289855072463768\\
4.74	0.246376811594203\\
4.809	0.188405797101449\\
4.878	0.231884057971015\\
4.947	0.27536231884058\\
5.016	0.391304347826087\\
5.085	0.246376811594203\\
5.154	0.27536231884058\\
5.223	0.217391304347826\\
5.292	0.347826086956518\\
5.361	0.231884057971018\\
5.43	0.2463768115942\\
5.499	0.260869565217395\\
5.568	0.275362318840576\\
5.637	0.289855072463772\\
5.706	0.304347826086953\\
5.775	0.289855072463772\\
5.844	0.362318840579706\\
5.913	0.289855072463768\\
5.982	0.463768115942029\\
6.051	0.420289855072464\\
6.12	0.492753623188406\\
6.189	0.463768115942029\\
6.258	0.463768115942029\\
6.327	0.420289855072459\\
6.396	0.246376811594206\\
6.465	0.202898550724635\\
6.534	0.0869565217391316\\
6.603	0.0579710144927529\\
6.672	0.0289855072463772\\
6.741	0.0144927536231882\\
6.81	0.0144927536231882\\
};
%\addlegendentry{ground truth}

\addplot [color=mycolor2, line width=2.0pt]
  table[row sep=crcr]{%
0	1.57796213037878e-07\\
0.01	1.6732895884297e-07\\
0.02	1.7741695560253e-07\\
0.03	1.88091260969029e-07\\
0.04	1.99384592349323e-07\\
0.05	2.11331411084634e-07\\
0.06	2.239680106454e-07\\
0.07	2.37332609018677e-07\\
0.08	2.51465445472897e-07\\
0.09	2.66408881892191e-07\\
0.1	2.82207508880122e-07\\
0.11	2.98908256840605e-07\\
0.12	3.16560512251946e-07\\
0.13	3.35216239358466e-07\\
0.14	3.54930107512825e-07\\
0.15	3.75759624411323e-07\\
0.16	3.97765275473713e-07\\
0.17	4.21010669628791e-07\\
0.18	4.45562691776965e-07\\
0.19	4.71491662211309e-07\\
0.2	4.98871503289274e-07\\
0.21	5.27779913658211e-07\\
0.22	5.58298550349138e-07\\
0.23	5.90513219064949e-07\\
0.24	6.24514073001227e-07\\
0.25	6.6039582055038e-07\\
0.26	6.982579422525e-07\\
0.27	7.38204917369618e-07\\
0.28	7.80346460473661e-07\\
0.29	8.24797768452321e-07\\
0.3	8.71679778351659e-07\\
0.31	9.21119436488881e-07\\
0.32	9.73249979284244e-07\\
0.33	1.02821122627662e-06\\
0.34	1.08614988580351e-06\\
0.35	1.14721987384284e-06\\
0.36	1.2115826465312e-06\\
0.37	1.27940754689035e-06\\
0.38	1.35087216631238e-06\\
0.39	1.42616272137205e-06\\
0.4	1.50547444655403e-06\\
0.41	1.58901200350241e-06\\
0.42	1.6769899074197e-06\\
0.43	1.76963297126329e-06\\
0.44	1.86717676840831e-06\\
0.45	1.96986811446753e-06\\
0.46	2.07796556898125e-06\\
0.47	2.19173995771237e-06\\
0.48	2.31147491630579e-06\\
0.49	2.4374674560944e-06\\
0.5	2.57002855285888e-06\\
0.51	2.70948375937301e-06\\
0.52	2.8561738425921e-06\\
0.53	3.01045544636769e-06\\
0.54	3.172701780599e-06\\
0.55	3.34330333775836e-06\\
0.56	3.52266863775558e-06\\
0.57	3.71122500213506e-06\\
0.58	3.90941935862813e-06\\
0.59	4.11771907711229e-06\\
0.6	4.33661283806007e-06\\
0.61	4.56661153458997e-06\\
0.62	4.80824920926364e-06\\
0.63	5.06208402680582e-06\\
0.64	5.32869928395452e-06\\
0.65	5.6087044576838e-06\\
0.66	5.90273629307341e-06\\
0.67	6.21145993213505e-06\\
0.68	6.53557008493801e-06\\
0.69	6.87579224441405e-06\\
0.7	7.23288394625544e-06\\
0.71	7.60763607535751e-06\\
0.72	8.00087422029293e-06\\
0.73	8.41346007734252e-06\\
0.74	8.84629290564536e-06\\
0.75	9.30031103506883e-06\\
0.76	9.77649342843766e-06\\
0.77	1.02758612998007e-05\\
0.78	1.07994797904525e-05\\
0.79	1.13484597044685e-05\\
0.8	1.19239593055503e-05\\
0.81	1.25271861770208e-05\\
0.82	1.31593991468468e-05\\
0.83	1.38219102796108e-05\\
0.84	1.45160869373932e-05\\
0.85	1.5243353911568e-05\\
0.86	1.60051956275572e-05\\
0.87	1.68031584246318e-05\\
0.88	1.7638852912887e-05\\
0.89	1.85139564095642e-05\\
0.9	1.9430215456933e-05\\
0.91	2.03894484239879e-05\\
0.92	2.13935481942593e-05\\
0.93	2.24444849420767e-05\\
0.94	2.35443089996663e-05\\
0.95	2.46951538175072e-05\\
0.96	2.58992390204102e-05\\
0.97	2.71588735618243e-05\\
0.98	2.8476458978919e-05\\
0.99	2.98544927510274e-05\\
1	3.12955717640784e-05\\
1.01	3.28023958836825e-05\\
1.02	3.43777716395763e-05\\
1.03	3.60246160241672e-05\\
1.04	3.77459604079575e-05\\
1.05	3.95449545746657e-05\\
1.06	4.14248708788915e-05\\
1.07	4.33891085292137e-05\\
1.08	4.54411979996362e-05\\
1.09	4.75848055723336e-05\\
1.1	4.98237380146774e-05\\
1.11	5.21619473935505e-05\\
1.12	5.4603536029991e-05\\
1.13	5.71527615972265e-05\\
1.14	5.98140423651886e-05\\
1.15	6.25919625946185e-05\\
1.16	6.54912780838939e-05\\
1.17	6.85169218717338e-05\\
1.18	7.16740100989391e-05\\
1.19	7.49678480323591e-05\\
1.2	7.84039362542751e-05\\
1.21	8.19879770204012e-05\\
1.22	8.5725880789721e-05\\
1.23	8.96237729293646e-05\\
1.24	9.36880005977504e-05\\
1.25	9.79251398092017e-05\\
1.26	0.00010234200268325\\
1.27	0.000106945644881829\\
1.28	0.000111743373237537\\
1.29	0.000116742753576166\\
1.3	0.000121951618736632\\
1.31	0.000127378076791452\\
1.32	0.000133030519470876\\
1.33	0.000138917630793744\\
1.34	0.000145048395908117\\
1.35	0.000151432110144684\\
1.36	0.000158078388285897\\
1.37	0.000164997174053755\\
1.38	0.000172198749819085\\
1.39	0.000179693746535117\\
1.4	0.000187493153898097\\
1.41	0.000195608330737591\\
1.42	0.000204051015639069\\
1.43	0.000212833337801297\\
1.44	0.000221967828130926\\
1.45	0.000231467430576641\\
1.46	0.000241345513705077\\
1.47	0.000251615882520645\\
1.48	0.000262292790531263\\
1.49	0.000273390952061908\\
1.5	0.000284925554817743\\
1.51	0.000296912272698468\\
1.52	0.000309367278865386\\
1.53	0.000322307259062539\\
1.54	0.000335749425193115\\
1.55	0.000349711529152155\\
1.56	0.000364211876916429\\
1.57	0.000379269342892173\\
1.58	0.000394903384521187\\
1.59	0.0004111340571456\\
1.6	0.000427982029131418\\
1.61	0.000445468597250736\\
1.62	0.000463615702322315\\
1.63	0.000482445945109952\\
1.64	0.00050198260247786\\
1.65	0.000522249643802037\\
1.66	0.000543271747636321\\
1.67	0.000565074318631597\\
1.68	0.000587683504706312\\
1.69	0.000611126214466206\\
1.7	0.000635430134870859\\
1.71	0.000660623749144346\\
1.72	0.000686736354927006\\
1.73	0.00071379808266499\\
1.74	0.000741839914233901\\
1.75	0.00077089370179259\\
1.76	0.000800992186862692\\
1.77	0.000832169019629243\\
1.78	0.00086445877845729\\
1.79	0.000897896989619027\\
1.8	0.000932520147225665\\
1.81	0.000968365733357731\\
1.82	0.00100547223838721\\
1.83	0.00104387918148447\\
1.84	0.00108362713130246\\
1.85	0.00112475772683027\\
1.86	0.00116731369840771\\
1.87	0.00121133888889208\\
1.88	0.00125687827496791\\
1.89	0.0013039779885898\\
1.9	0.0013526853385483\\
1.91	0.00140304883214798\\
1.92	0.00145511819698661\\
1.93	0.0015089444028236\\
1.94	0.00156457968352566\\
1.95	0.00162207755907679\\
1.96	0.00168149285763936\\
1.97	0.00174288173765267\\
1.98	0.00180630170995434\\
1.99	0.00187181165990996\\
2	0.00193947186953536\\
2.01	0.00200934403959567\\
2.02	0.00208149131166451\\
2.03	0.00215597829012625\\
2.04	0.00223287106410366\\
2.05	0.00231223722929271\\
2.06	0.00239414590968566\\
2.07	0.002478667779163\\
2.08	0.00256587508293435\\
2.09	0.00265584165880763\\
2.1	0.0027486429582654\\
2.11	0.00284435606732658\\
2.12	0.00294305972717127\\
2.13	0.00304483435450568\\
2.14	0.00314976206164373\\
2.15	0.00325792667628116\\
2.16	0.00336941376093756\\
2.17	0.00348431063204102\\
2.18	0.00360270637862959\\
2.19	0.0037246918806432\\
2.2	0.00385035982677906\\
2.21	0.003979804731883\\
2.22	0.00411312295384885\\
2.23	0.00425041270999705\\
2.24	0.00439177409290354\\
2.25	0.00453730908564923\\
2.26	0.00468712157645985\\
2.27	0.00484131737270557\\
2.28	0.00500000421422928\\
2.29	0.00516329178597171\\
2.3	0.00533129172986171\\
2.31	0.00550411765593867\\
2.32	0.00568188515267452\\
2.33	0.00586471179646173\\
2.34	0.00605271716023359\\
2.35	0.00624602282118281\\
2.36	0.00644475236754353\\
2.37	0.00664903140440251\\
2.38	0.00685898755850369\\
2.39	0.00707475048201145\\
2.4	0.00729645185519606\\
2.41	0.00752422538800619\\
2.42	0.0077582068204918\\
2.43	0.00799853392204129\\
2.44	0.00824534648939642\\
2.45	0.00849878634340852\\
2.46	0.00875899732449895\\
2.47	0.00902612528678757\\
2.48	0.00930031809085181\\
2.49	0.00958172559508029\\
2.5	0.00987049964558341\\
2.51	0.010166794064625\\
2.52	0.0104707646375381\\
2.53	0.0107825690980882\\
2.54	0.0111023671122482\\
2.55	0.0114303202603494\\
2.56	0.0117665920175712\\
2.57	0.0121113477327361\\
2.58	0.0124647546053737\\
2.59	0.0128269816610191\\
2.6	0.0131981997247123\\
2.61	0.0135785813926636\\
2.62	0.0139683010020526\\
2.63	0.0143675345989288\\
2.64	0.0147764599041794\\
2.65	0.0151952562775363\\
2.66	0.0156241046795887\\
2.67	0.0160631876317733\\
2.68	0.0165126891743121\\
2.69	0.0169727948220707\\
2.7	0.0174436915183084\\
2.71	0.0179255675862951\\
2.72	0.0184186126787698\\
2.73	0.0189230177252151\\
2.74	0.0194389748769265\\
2.75	0.0199666774498538\\
2.76	0.0205063198651942\\
2.77	0.0210580975877172\\
2.78	0.0216222070618048\\
2.79	0.0221988456451888\\
2.8	0.0227882115403718\\
2.81	0.023390503723716\\
2.82	0.0240059218721913\\
2.83	0.0246346662877682\\
2.84	0.02527693781945\\
2.85	0.0259329377829358\\
2.86	0.0266028678779091\\
2.87	0.0272869301029493\\
2.88	0.0279853266680635\\
2.89	0.0286982599048398\\
2.9	0.0294259321742241\\
2.91	0.0301685457719252\\
2.92	0.0309263028314531\\
2.93	0.031699405224802\\
2.94	0.0324880544607849\\
2.95	0.0332924515810362\\
2.96	0.0341127970536951\\
2.97	0.0349492906647884\\
2.98	0.0358021314073319\\
2.99	0.0366715173681734\\
3	0.0375576456125996\\
3.01	0.0384607120667382\\
3.02	0.0393809113977787\\
3.03	0.0403184368920498\\
3.04	0.041273480330984\\
3.05	0.0422462318650073\\
3.06	0.0432368798853951\\
3.07	0.0442456108941351\\
3.08	0.045272609371842\\
3.09	0.0463180576437729\\
3.1	0.0473821357439927\\
3.11	0.0484650212777422\\
3.12	0.0495668892820659\\
3.13	0.0506879120847558\\
3.14	0.0518282591616751\\
3.15	0.052988096992522\\
3.16	0.0541675889151041\\
3.17	0.0553668949781894\\
3.18	0.0565861717930083\\
3.19	0.057825572383479\\
3.2	0.0590852460352369\\
3.21	0.0603653381435441\\
3.22	0.0616659900601659\\
3.23	0.0629873389392968\\
3.24	0.0643295175826253\\
3.25	0.0656926542836283\\
3.26	0.0670768726711883\\
3.27	0.0684822915526288\\
3.28	0.0699090247562672\\
3.29	0.0713571809735846\\
3.3	0.0728268636011179\\
3.31	0.0743181705821772\\
3.32	0.0758311942484995\\
3.33	0.0773660211619461\\
3.34	0.0789227319563588\\
3.35	0.0805014011796868\\
3.36	0.0821020971365045\\
3.37	0.0837248817310358\\
3.38	0.0853698103108073\\
3.39	0.0870369315110538\\
3.4	0.0887262870999983\\
3.41	0.0904379118251351\\
3.42	0.0921718332606424\\
3.43	0.093928071656055\\
3.44	0.0957066397863258\\
3.45	0.0975075428034124\\
3.46	0.0993307780895168\\
3.47	0.10117633511212\\
3.48	0.103044195280937\\
3.49	0.104934331806946\\
3.5	0.106846709563605\\
3.51	0.10878128495042\\
3.52	0.110738005758985\\
3.53	0.112716811041641\\
3.54	0.114717630982898\\
3.55	0.116740386773744\\
3.56	0.118784990489004\\
3.57	0.120851344967873\\
3.58	0.122939343697765\\
3.59	0.125048870701625\\
3.6	0.127179800428835\\
3.61	0.129331997649863\\
3.62	0.131505317354778\\
3.63	0.133699604655791\\
3.64	0.135914694693932\\
3.65	0.138150412550017\\
3.66	0.14040657316004\\
3.67	0.142682981235104\\
3.68	0.144979431186045\\
3.69	0.147295707052865\\
3.7	0.149631582439104\\
3.71	0.151986820451277\\
3.72	0.154361173643503\\
3.73	0.156754383967443\\
3.74	0.159166182727665\\
3.75	0.161596290542557\\
3.76	0.164044417310898\\
3.77	0.166510262184199\\
3.78	0.168993513544918\\
3.79	0.17149384899066\\
3.8	0.174010935324459\\
3.81	0.176544428551235\\
3.82	0.179093973880531\\
3.83	0.181659205735608\\
3.84	0.184239747768995\\
3.85	0.186835212884572\\
3.86	0.189445203266253\\
3.87	0.192069310413369\\
3.88	0.194707115182791\\
3.89	0.197358187837878\\
3.9	0.200022088104301\\
3.91	0.2026983652328\\
3.92	0.20538655806893\\
3.93	0.208086195129822\\
3.94	0.210796794688034\\
3.95	0.213517864862485\\
3.96	0.216248903716537\\
3.97	0.218989399363226\\
3.98	0.221738830077677\\
3.99	0.224496664416695\\
4	0.227262361345569\\
4.01	0.230035370372056\\
4.02	0.232815131687571\\
4.03	0.235601076315548\\
4.04	0.238392626266977\\
4.05	0.241189194703077\\
4.06	0.243990186105084\\
4.07	0.24679499645112\\
4.08	0.249603013400101\\
4.09	0.252413616482629\\
4.1	0.255226177298835\\
4.11	0.258040059723083\\
4.12	0.260854620115493\\
4.13	0.263669207540208\\
4.14	0.266483163990311\\
4.15	0.269295824619319\\
4.16	0.272106517979162\\
4.17	0.274914566264549\\
4.18	0.277719285563608\\
4.19	0.280519986114707\\
4.2	0.283315972569318\\
4.21	0.286106544260821\\
4.22	0.288890995479109\\
4.23	0.291668615750864\\
4.24	0.294438690125349\\
4.25	0.297200499465598\\
4.26	0.299953320744823\\
4.27	0.302696427347894\\
4.28	0.305429089377727\\
4.29	0.308150573966407\\
4.3	0.310860145590876\\
4.31	0.313557066393003\\
4.32	0.316240596503845\\
4.33	0.318909994371925\\
4.34	0.321564517095314\\
4.35	0.324203420757327\\
4.36	0.326825960765618\\
4.37	0.32943139219449\\
4.38	0.332018970130165\\
4.39	0.334587950018842\\
4.4	0.337137588017281\\
4.41	0.339667141345718\\
4.42	0.342175868642858\\
4.43	0.344663030322738\\
4.44	0.347127888933195\\
4.45	0.349569709515727\\
4.46	0.351987759966485\\
4.47	0.354381311398166\\
4.48	0.356749638502546\\
4.49	0.359092019913418\\
4.5	0.361407738569672\\
4.51	0.363696082078269\\
4.52	0.365956343076854\\
4.53	0.36818781959575\\
4.54	0.37038981541908\\
4.55	0.372561640444749\\
4.56	0.374702611043043\\
4.57	0.376812050413566\\
4.58	0.378889288940274\\
4.59	0.380933664544337\\
4.6	0.382944523034563\\
4.61	0.384921218455149\\
4.62	0.386863113430472\\
4.63	0.38876957950668\\
4.64	0.390639997489835\\
4.65	0.392473757780324\\
4.66	0.394270260703325\\
4.67	0.396028916835047\\
4.68	0.397749147324504\\
4.69	0.399430384210594\\
4.7	0.401072070734212\\
4.71	0.402673661645181\\
4.72	0.404234623503753\\
4.73	0.405754434976448\\
4.74	0.407232587126003\\
4.75	0.408668583695207\\
4.76	0.410061941384398\\
4.77	0.411412190122402\\
4.78	0.412718873330714\\
4.79	0.413981548180693\\
4.8	0.415199785843588\\
4.81	0.416373171733188\\
4.82	0.417501305740899\\
4.83	0.418583802463076\\
4.84	0.419620291420413\\
4.85	0.420610417269232\\
4.86	0.421553840004482\\
4.87	0.422450235154311\\
4.88	0.423299293966035\\
4.89	0.424100723583361\\
4.9	0.424854247214724\\
4.91	0.425559604292603\\
4.92	0.426216550623681\\
4.93	0.426824858529737\\
4.94	0.42738431697915\\
4.95	0.427894731708904\\
4.96	0.428355925337014\\
4.97	0.42876773746526\\
4.98	0.429130024772156\\
4.99	0.429442661096082\\
5	0.429705537508501\\
5.01	0.429918562377215\\
5.02	0.430081661419596\\
5.03	0.430194777745755\\
5.04	0.430257871891611\\
5.05	0.430270921841844\\
5.06	0.430233923042688\\
5.07	0.43014688840459\\
5.08	0.43000984829469\\
5.09	0.429822850519175\\
5.1	0.429585960295475\\
5.11	0.429299260214361\\
5.12	0.428962850191953\\
5.13	0.42857684741169\\
5.14	0.428141386256298\\
5.15	0.427656618229819\\
5.16	0.427122711869768\\
5.17	0.426539852649476\\
5.18	0.425908242870709\\
5.19	0.425228101546648\\
5.2	0.424499664275325\\
5.21	0.423723183103612\\
5.22	0.422898926381879\\
5.23	0.422027178609439\\
5.24	0.421108240270897\\
5.25	0.420142427663547\\
5.26	0.419130072715939\\
5.27	0.418071522797781\\
5.28	0.416967140521312\\
5.29	0.415817303534313\\
5.3	0.414622404304924\\
5.31	0.413382849898431\\
5.32	0.412099061746202\\
5.33	0.410771475406965\\
5.34	0.409400540320603\\
5.35	0.407986719554673\\
5.36	0.406530489543842\\
5.37	0.405032339822449\\
5.38	0.403492772750393\\
5.39	0.401912303232586\\
5.4	0.400291458432158\\
5.41	0.398630777477662\\
5.42	0.396930811164499\\
5.43	0.395192121650785\\
5.44	0.393415282147916\\
5.45	0.391600876606045\\
5.46	0.389749499394726\\
5.47	0.387861754978974\\
5.48	0.38593825759097\\
5.49	0.383979630897678\\
5.5	0.381986507664611\\
5.51	0.379959529416016\\
5.52	0.377899346091709\\
5.53	0.375806615700843\\
5.54	0.373682003972844\\
5.55	0.371526184005785\\
5.56	0.369339835912456\\
5.57	0.367123646464386\\
5.58	0.364878308734077\\
5.59	0.362604521735712\\
5.6	0.360302990064595\\
5.61	0.35797442353558\\
5.62	0.355619536820747\\
5.63	0.353239049086587\\
5.64	0.350833683630945\\
5.65	0.348404167519972\\
5.66	0.345951231225358\\
5.67	0.343475608262067\\
5.68	0.340978034826848\\
5.69	0.338459249437751\\
5.7	0.335919992574906\\
5.71	0.333361006322791\\
5.72	0.330783034014235\\
5.73	0.328186819876394\\
5.74	0.325573108678915\\
5.75	0.322942645384542\\
5.76	0.320296174802359\\
5.77	0.317634441243908\\
5.78	0.314958188182398\\
5.79	0.312268157915207\\
5.8	0.309565091229886\\
5.81	0.306849727073878\\
5.82	0.304122802228139\\
5.83	0.301385050984853\\
5.84	0.298637204829441\\
5.85	0.295879992127034\\
5.86	0.293114137813595\\
5.87	0.290340363091857\\
5.88	0.287559385132253\\
5.89	0.28477191677898\\
5.9	0.281978666261385\\
5.91	0.279180336910783\\
5.92	0.276377626882882\\
5.93	0.273571228885938\\
5.94	0.270761829914778\\
5.95	0.267950110990813\\
5.96	0.265136746908157\\
5.97	0.26232240598598\\
5.98	0.259507749827182\\
5.99	0.256693433083513\\
6	0.253880103227204\\
6.01	0.25106840032923\\
6.02	0.248258956844258\\
6.03	0.24545239740238\\
6.04	0.242649338607682\\
6.05	0.23985038884372\\
6.06	0.237056148085968\\
6.07	0.234267207721267\\
6.08	0.231484150374343\\
6.09	0.228707549741418\\
6.1	0.225937970430952\\
6.11	0.223175967811532\\
6.12	0.220422087866947\\
6.13	0.217676867058438\\
6.14	0.214940832194157\\
6.15	0.212214500305811\\
6.16	0.209498378532504\\
6.17	0.206792964011768\\
6.18	0.20409874377775\\
6.19	0.201416194666557\\
6.2	0.198745783228713\\
6.21	0.196087965648708\\
6.22	0.193443187671595\\
6.23	0.1908118845366\\
6.24	0.188194480917686\\
6.25	0.185591390871025\\
6.26	0.183003017789326\\
6.27	0.180429754362946\\
6.28	0.17787198254772\\
6.29	0.175330073539448\\
6.3	0.172804387754944\\
6.31	0.170295274819593\\
6.32	0.1678030735613\\
6.33	0.165328112010778\\
6.34	0.162870707408052\\
6.35	0.160431166215103\\
6.36	0.158009784134541\\
6.37	0.155606846134221\\
6.38	0.153222626477669\\
6.39	0.150857388760237\\
6.4	0.148511385950859\\
6.41	0.146184860439295\\
6.42	0.14387804408875\\
6.43	0.141591158293747\\
6.44	0.139324414043124\\
6.45	0.137078011988046\\
6.46	0.134852142514887\\
6.47	0.132646985822869\\
6.48	0.130462712006318\\
6.49	0.128299481141408\\
6.5	0.126157443377264\\
6.51	0.124036739031276\\
6.52	0.121937498688509\\
6.53	0.11985984330505\\
6.54	0.117803884315168\\
6.55	0.115769723742151\\
6.56	0.113757454312658\\
6.57	0.111767159574483\\
6.58	0.109798914017555\\
6.59	0.107852783198056\\
6.6	0.105928823865511\\
6.61	0.104027084092708\\
6.62	0.102147603408304\\
6.63	0.100290412931992\\
6.64	0.0984555355120719\\
6.65	0.096642985865296\\
6.66	0.0948527707188535\\
6.67	0.093084888954349\\
6.68	0.0913393317536445\\
6.69	0.0896160827464269\\
6.7	0.0879151181593685\\
6.71	0.0862364069667458\\
6.72	0.0845799110423867\\
6.73	0.0829455853128137\\
6.74	0.0813333779114575\\
6.75	0.0797432303338104\\
6.76	0.0781750775933965\\
6.77	0.0766288483784348\\
6.78	0.0751044652090717\\
6.79	0.0736018445950657\\
6.8	0.072120897193804\\
6.81	0.0706615279685363\\
6.82	0.0692236363467128\\
6.83	0.0678071163783132\\
6.84	0.066411856894059\\
6.85	0.0650377416634018\\
6.86	0.063684649552183\\
6.87	0.0623524546798617\\
6.88	0.0610410265762144\\
6.89	0.059750230337404\\
6.9	0.0584799267813295\\
6.91	0.0572299726021594\\
6.92	0.0560002205239619\\
6.93	0.0547905194533451\\
6.94	0.0536007146310212\\
6.95	0.0524306477822145\\
6.96	0.0512801572658352\\
6.97	0.0501490782223411\\
6.98	0.049037242720213\\
6.99	0.0479444799009781\\
7	0.0468706161227059\\
7.01	0.0458154751019185\\
7.02	0.0447788780538475\\
7.03	0.0437606438309811\\
7.04	0.0427605890598424\\
7.05	0.0417785282759456\\
7.06	0.0408142740568774\\
7.07	0.0398676371534559\\
7.08	0.0389384266189196\\
7.09	0.0380264499361035\\
7.1	0.0371315131425608\\
7.11	0.0362534209535923\\
7.12	0.0353919768831473\\
7.13	0.0345469833625628\\
7.14	0.0337182418571098\\
7.15	0.0329055529803198\\
7.16	0.0321087166060638\\
7.17	0.031327531978362\\
7.18	0.0305617978189021\\
7.19	0.0298113124322489\\
7.2	0.0290758738087266\\
7.21	0.0283552797249624\\
7.22	0.027649327842078\\
7.23	0.0269578158015187\\
7.24	0.0262805413185151\\
7.25	0.0256173022731688\\
7.26	0.0249678967991615\\
7.27	0.024332123370084\\
7.28	0.0237097808833873\\
7.29	0.023100668741957\\
7.3	0.0225045869333163\\
7.31	0.0219213361064622\\
7.32	0.0213507176463456\\
7.33	0.0207925337460005\\
7.34	0.0202465874763387\\
7.35	0.0197126828536178\\
7.36	0.0191906249046004\\
7.37	0.0186802197294185\\
7.38	0.0181812745621606\\
7.39	0.0176935978292\\
7.4	0.0172169992052859\\
7.41	0.0167512896674144\\
7.42	0.0162962815465073\\
7.43	0.0158517885769169\\
7.44	0.0154176259437855\\
7.45	0.014993610328283\\
7.46	0.0145795599507499\\
7.47	0.0141752946117745\\
7.48	0.0137806357312301\\
7.49	0.0133954063853056\\
7.5	0.0130194313415551\\
7.51	0.0126525370920022\\
7.52	0.0122945518843264\\
7.53	0.0119453057511672\\
7.54	0.0116046305375768\\
7.55	0.0112723599266554\\
7.56	0.0109483294634039\\
7.57	0.0106323765768272\\
7.58	0.0103243406003238\\
7.59	0.0100240627903965\\
7.6	0.00973138634372004\\
7.61	0.00944615641260181\\
7.62	0.00916822011887066\\
7.63	0.00889742656623163\\
7.64	0.00863362685112175\\
7.65	0.00837667407210451\\
7.66	0.00812642333783901\\
7.67	0.00788273177366106\\
7.68	0.00764545852681277\\
7.69	0.00741446477035737\\
7.7	0.00718961370581621\\
7.71	0.00697077056456436\\
7.72	0.00675780260802152\\
7.73	0.00655057912667446\\
7.74	0.0063489714379676\\
7.75	0.0061528528830971\\
7.76	0.00596209882274507\\
7.77	0.00577658663178883\\
7.78	0.00559619569302082\\
7.79	0.00542080738991404\\
7.8	0.00525030509846783\\
7.81	0.00508457417816796\\
7.82	0.00492350196209547\\
7.83	0.00476697774621747\\
7.84	0.00461489277789316\\
7.85	0.004467140243628\\
7.86	0.00432361525610802\\
7.87	0.00418421484054641\\
7.88	0.00404883792037373\\
7.89	0.00391738530230245\\
7.9	0.00378975966079661\\
7.91	0.00366586552197609\\
7.92	0.00354560924698525\\
7.93	0.00342889901485462\\
7.94	0.00331564480488402\\
7.95	0.00320575837857484\\
7.96	0.00309915326113891\\
7.97	0.0029957447226103\\
7.98	0.00289544975858655\\
7.99	0.0027981870706246\\
8	0.0027038770463165\\
8.01	0.00261244173906925\\
8.02	0.00252380484761261\\
8.03	0.00243789169525811\\
8.04	0.00235462920893187\\
8.05	0.00227394589800328\\
8.06	0.00219577183293114\\
8.07	0.00212003862374783\\
8.08	0.00204667939840225\\
8.09	0.00197562878098082\\
8.1	0.00190682286982598\\
8.11	0.0018401992155706\\
8.12	0.00177569679910627\\
8.13	0.00171325600950294\\
8.14	0.00165281862189675\\
8.15	0.00159432777536226\\
8.16	0.00153772795078486\\
8.17	0.00148296494874843\\
8.18	0.00142998586745307\\
8.19	0.00137873908067667\\
8.2	0.00132917421579415\\
8.21	0.00128124213186715\\
8.22	0.00123489489781688\\
8.23	0.00119008577069181\\
8.24	0.00114676917404195\\
8.25	0.00110490067641056\\
8.26	0.00106443696995373\\
8.27	0.0010253358491979\\
8.28	0.000987556189944738\\
8.29	0.0009510579283326\\
8.3	0.000915802040062941\\
8.31	0.000881750519800031\\
8.32	0.00084886636075151\\
8.33	0.000817113534437219\\
8.34	0.000786456970653031\\
8.35	0.000756862537636212\\
8.36	0.000728297022438306\\
8.37	0.000700728111511244\\
8.38	0.000674124371511893\\
8.39	0.00064845523032998\\
8.4	0.000623690958343928\\
8.41	0.000599802649908723\\
8.42	0.000576762205079736\\
8.43	0.000554542311575931\\
8.44	0.000533116426985664\\
8.45	0.000512458761217927\\
8.46	0.000492544259201582\\
8.47	0.000473348583834874\\
8.48	0.000454848099187134\\
8.49	0.000437019853954429\\
8.5	0.000419841565170547\\
8.51	0.000403291602174477\\
8.52	0.000387348970835326\\
8.53	0.000371993298035319\\
8.54	0.000357204816411329\\
8.55	0.000342964349355168\\
8.56	0.000329253296272633\\
8.57	0.000316053618101072\\
8.58	0.000303347823085131\\
8.59	0.000291118952810022\\
8.6	0.00027935056849159\\
8.61	0.000268026737522192\\
8.62	0.000257132020271303\\
8.63	0.000246651457139548\\
8.64	0.000236570555864774\\
8.65	0.000226875279078558\\
8.66	0.000217552032111468\\
8.67	0.00020858765104525\\
8.68	0.000199969391009976\\
8.69	0.000191684914724091\\
8.7	0.000183722281275182\\
8.71	0.000176069935139188\\
8.72	0.000168716695435684\\
8.73	0.000161651745416754\\
8.74	0.000154864622186933\\
8.75	0.000148345206651577\\
8.76	0.000142083713690956\\
8.77	0.000136070682557319\\
8.78	0.000130296967492101\\
8.79	0.00012475372856038\\
8.8	0.000119432422699661\\
8.81	0.000114324794980009\\
8.82	0.000109422870072488\\
8.83	0.000104718943922871\\
8.84	0.00010020557562752\\
8.85	9.58755795083083e-05\\
8.86	9.17220173834464e-05\\
8.87	8.77381910310472e-05\\
8.88	8.39176348422412e-05\\
8.89	8.0254108660648e-05\\
8.9	7.67415908050039e-05\\
8.91	7.33742712717205e-05\\
8.92	7.01465451141713e-05\\
8.93	6.70530059954835e-05\\
8.94	6.40884399116234e-05\\
8.95	6.12478190815682e-05\\
8.96	5.85262960013651e-05\\
8.97	5.59191976588905e-05\\
8.98	5.3422019906128e-05\\
8.99	5.10304219858156e-05\\
9	4.87402212093111e-05\\
9.01	4.65473877825589e-05\\
9.02	4.4448039777055e-05\\
9.03	4.24384382427343e-05\\
9.04	4.05149824597284e-05\\
9.05	3.86742053259695e-05\\
9.06	3.69127688776458e-05\\
9.07	3.52274599395366e-05\\
9.08	3.36151859023003e-05\\
9.09	3.20729706238066e-05\\
9.1	3.05979504516463e-05\\
9.11	2.91873703639862e-05\\
9.12	2.78385802259674e-05\\
9.13	2.65490311588883e-05\\
9.14	2.53162720194479e-05\\
9.15	2.413794598636e-05\\
9.16	2.30117872516976e-05\\
9.17	2.19356178143529e-05\\
9.18	2.09073443730507e-05\\
9.19	1.99249553163848e-05\\
9.2	1.89865178073933e-05\\
9.21	1.80901749602264e-05\\
9.22	1.7234143106505e-05\\
9.23	1.64167091490052e-05\\
9.24	1.5636228000354e-05\\
9.25	1.48911201044538e-05\\
9.26	1.41798690384032e-05\\
9.27	1.35010191927194e-05\\
9.28	1.28531735277123e-05\\
9.29	1.22349914038993e-05\\
9.3	1.16451864843962e-05\\
9.31	1.10825247072582e-05\\
9.32	1.05458223257856e-05\\
9.33	1.00339440148562e-05\\
9.34	9.54580104138018e-06\\
9.35	9.08034949702025e-06\\
9.36	8.63658859135671e-06\\
9.37	8.21355900371943e-06\\
9.38	7.81034129194826e-06\\
9.39	7.4260543563832e-06\\
9.4	7.05985395742368e-06\\
9.41	6.71093128503779e-06\\
9.42	6.37851157863778e-06\\
9.43	6.06185279577802e-06\\
9.44	5.76024432816823e-06\\
9.45	5.47300576353238e-06\\
9.46	5.1994856918794e-06\\
9.47	4.93906055478872e-06\\
9.48	4.69113353634792e-06\\
9.49	4.45513349441633e-06\\
9.5	4.23051393092132e-06\\
9.51	4.01675199992853e-06\\
9.52	3.81334755226055e-06\\
9.53	3.61982221547129e-06\\
9.54	3.43571850801515e-06\\
9.55	3.26059898648207e-06\\
9.56	3.0940454248004e-06\\
9.57	2.9356580243395e-06\\
9.58	2.78505465387479e-06\\
9.59	2.64187011840634e-06\\
9.6	2.50575545585127e-06\\
9.61	2.37637726065802e-06\\
9.62	2.25341703341841e-06\\
9.63	2.13657055558007e-06\\
9.64	2.0255472883885e-06\\
9.65	1.92006979521328e-06\\
9.66	1.81987318643892e-06\\
9.67	1.72470458612492e-06\\
9.68	1.63432261966389e-06\\
9.69	1.54849692169029e-06\\
9.7	1.46700766351513e-06\\
9.71	1.38964509938443e-06\\
9.72	1.31620913088152e-06\\
9.73	1.24650888881384e-06\\
9.74	1.18036233194682e-06\\
9.75	1.11759586196676e-06\\
9.76	1.05804395407504e-06\\
9.77	1.00154880263508e-06\\
9.78	9.47959981312304e-07\\
9.79	8.9713411716567e-07\\
9.8	8.48934578167194e-07\\
9.81	8.0323117364319e-07\\
9.82	7.59899867147783e-07\\
9.83	7.18822501295873e-07\\
9.84	6.79886534098404e-07\\
9.85	6.42984786358577e-07\\
9.86	6.0801519970248e-07\\
9.87	5.74880604832494e-07\\
9.88	5.43488499605858e-07\\
9.89	5.1375083655478e-07\\
9.9	4.8558381947767e-07\\
9.91	4.58907708744353e-07\\
9.92	4.33646634970517e-07\\
9.93	4.09728420729044e-07\\
9.94	3.87084409977666e-07\\
9.95	3.65649304893996e-07\\
9.96	3.45361009820035e-07\\
9.97	3.26160482029178e-07\\
9.98	3.07991589039127e-07\\
9.99	2.90800972204353e-07\\
10	2.74537916331524e-07\\
};
%\addlegendentry{moment matching}

\addplot [color=mycolor2, line width=2.0pt, forget plot]
  table[row sep=crcr]{%
5.04760739417037	0\\
5.04760739417037	0.6\\
};
\addplot [color=mycolor1, dashed, line width=2.0pt, forget plot]
  table[row sep=crcr]{%
5.0284309151552	0\\
5.0284309151552	0.6\\
};
\end{axis}

\begin{axis}[%
width=0.898in,
height=1.5in,%3.603in,
at={(3.196in,0.486in)},
scale only axis,
xmin=0,
xmax=10,
ymin=0,
ymax=0.8,
title={Sigma-point \\ propagation},
title style={align=left}, 
axis background/.style={fill=white},
axis x line*=bottom,
axis y line*=left,
legend style={legend cell align=right, align=right, draw=white!15!black, cells={align=right}, font=\tiny}
]
\addplot[ybar interval, fill=mycolor1, fill opacity=0.4, draw=mycolor1, area legend] table[row sep=crcr] {%
x	y\\
3.36	0.0144927536231884\\
3.429	0.0289855072463768\\
3.498	0.0869565217391305\\
3.567	0.217391304347825\\
3.636	0.391304347826087\\
3.705	0.565217391304348\\
3.774	0.449275362318841\\
3.843	0.405797101449276\\
3.912	0.666666666666667\\
3.981	0.420289855072464\\
4.05	0.478260869565218\\
4.119	0.289855072463768\\
4.188	0.289855072463768\\
4.257	0.347826086956522\\
4.326	0.246376811594203\\
4.395	0.304347826086953\\
4.464	0.20289855072464\\
4.533	0.217391304347823\\
4.602	0.246376811594203\\
4.671	0.289855072463768\\
4.74	0.246376811594203\\
4.809	0.188405797101449\\
4.878	0.231884057971015\\
4.947	0.27536231884058\\
5.016	0.391304347826087\\
5.085	0.246376811594203\\
5.154	0.27536231884058\\
5.223	0.217391304347826\\
5.292	0.347826086956518\\
5.361	0.231884057971018\\
5.43	0.2463768115942\\
5.499	0.260869565217395\\
5.568	0.275362318840576\\
5.637	0.289855072463772\\
5.706	0.304347826086953\\
5.775	0.289855072463772\\
5.844	0.362318840579706\\
5.913	0.289855072463768\\
5.982	0.463768115942029\\
6.051	0.420289855072464\\
6.12	0.492753623188406\\
6.189	0.463768115942029\\
6.258	0.463768115942029\\
6.327	0.420289855072459\\
6.396	0.246376811594206\\
6.465	0.202898550724635\\
6.534	0.0869565217391316\\
6.603	0.0579710144927529\\
6.672	0.0289855072463772\\
6.741	0.0144927536231882\\
6.81	0.0144927536231882\\
};
%\addlegendentry{Numerical \\ approx.}

\addplot [color=mycolor2, line width=2.0pt]
  table[row sep=crcr]{%
0	8.87764380319432e-05\\
0.01	9.1700822136384e-05\\
0.02	9.47154517333371e-05\\
0.03	9.78228998119414e-05\\
0.04	0.000101025805675457\\
0.05	0.000104326876427423\\
0.06	0.000107728888484391\\
0.07	0.000111234689115493\\
0.08	0.000114847198009143\\
0.09	0.000118569408867125\\
0.1	0.000122404391026334\\
0.11	0.00012635529110845\\
0.12	0.000130425334697778\\
0.13	0.000134617828047533\\
0.14	0.000138936159814801\\
0.15	0.000143383802824432\\
0.16	0.000147964315862095\\
0.17	0.000152681345496739\\
0.18	0.000157538627932674\\
0.19	0.000162539990891505\\
0.2	0.000167689355524124\\
0.21	0.000172990738352978\\
0.22	0.000178448253244802\\
0.23	0.000184066113414019\\
0.24	0.000189848633456982\\
0.25	0.000195800231417249\\
0.26	0.000201925430882044\\
0.27	0.000208228863110072\\
0.28	0.000214715269190831\\
0.29	0.000221389502235576\\
0.3	0.000228256529600039\\
0.31	0.000235321435139054\\
0.32	0.000242589421493169\\
0.33	0.000250065812407359\\
0.34	0.000257756055081929\\
0.35	0.000265665722555657\\
0.36	0.000273800516121267\\
0.37	0.000282166267773259\\
0.38	0.000290768942688146\\
0.39	0.000299614641737107\\
0.4	0.000308709604031073\\
0.41	0.000318060209498228\\
0.42	0.000327672981493914\\
0.43	0.0003375545894429\\
0.44	0.00034771185151395\\
0.45	0.000358151737326627\\
0.46	0.000368881370690256\\
0.47	0.000379908032374928\\
0.48	0.00039123916291442\\
0.49	0.000402882365440907\\
0.5	0.000414845408551292\\
0.51	0.000427136229204983\\
0.52	0.000439762935652922\\
0.53	0.000452733810397627\\
0.54	0.000466057313184048\\
0.55	0.000479742084020933\\
0.56	0.000493796946232453\\
0.57	0.000508230909539774\\
0.58	0.000523053173172236\\
0.59	0.00053827312900782\\
0.6	0.000553900364742502\\
0.61	0.000569944667088102\\
0.62	0.000586416024998223\\
0.63	0.000603324632921818\\
0.64	0.000620680894083919\\
0.65	0.000638495423793025\\
0.66	0.000656779052774621\\
0.67	0.000675542830530304\\
0.68	0.000694798028721899\\
0.69	0.000714556144579973\\
0.7	0.000734828904336119\\
0.71	0.000755628266678342\\
0.72	0.000776966426228842\\
0.73	0.000798855817043487\\
0.74	0.000821309116132199\\
0.75	0.000844339246999491\\
0.76	0.000867959383204324\\
0.77	0.000892182951938454\\
0.78	0.000917023637622338\\
0.79	0.000942495385517763\\
0.8	0.000968612405356183\\
0.81	0.000995389174981824\\
0.82	0.00102284044400852\\
0.83	0.00105098123748923\\
0.84	0.00107982685959722\\
0.85	0.00110939289731764\\
0.86	0.00113969522414855\\
0.87	0.00117075000380997\\
0.88	0.00120257369395996\\
0.89	0.00123518304991627\\
0.9	0.00126859512838244\\
0.91	0.00130282729117678\\
0.92	0.00133789720896316\\
0.93	0.00137382286498187\\
0.94	0.00141062255877942\\
0.95	0.00144831490993543\\
0.96	0.0014869188617855\\
0.97	0.0015264536851381\\
0.98	0.00156693898198413\\
0.99	0.00160839468919738\\
1	0.00165084108222428\\
1.01	0.00169429877876111\\
1.02	0.00173878874241707\\
1.03	0.00178433228636116\\
1.04	0.00183095107695133\\
1.05	0.00187866713734369\\
1.06	0.00192750285108013\\
1.07	0.00197748096565218\\
1.08	0.00202862459603924\\
1.09	0.00208095722821903\\
1.1	0.00213450272264824\\
1.11	0.00218928531771122\\
1.12	0.0022453296331345\\
1.13	0.00230266067336505\\
1.14	0.00236130383090986\\
1.15	0.00242128488963473\\
1.16	0.0024826300280197\\
1.17	0.00254536582236912\\
1.18	0.00260951924997343\\
1.19	0.0026751176922208\\
1.2	0.00274218893765555\\
1.21	0.00281076118498133\\
1.22	0.00288086304600608\\
1.23	0.00295252354852655\\
1.24	0.00302577213914932\\
1.25	0.00310063868604605\\
1.26	0.00317715348163996\\
1.27	0.00325534724522087\\
1.28	0.0033352511254861\\
1.29	0.00341689670300425\\
1.3	0.00350031599259913\\
1.31	0.00358554144565092\\
1.32	0.00367260595231155\\
1.33	0.0037615428436315\\
1.34	0.00385238589359484\\
1.35	0.00394516932105972\\
1.36	0.00403992779160099\\
1.37	0.00413669641925212\\
1.38	0.00423551076814321\\
1.39	0.00433640685403186\\
1.4	0.0044394211457239\\
1.41	0.00454459056638071\\
1.42	0.00465195249470999\\
1.43	0.00476154476603654\\
1.44	0.00487340567325009\\
1.45	0.00498757396762667\\
1.46	0.00510408885952039\\
1.47	0.00522299001892209\\
1.48	0.00534431757588181\\
1.49	0.00546811212079153\\
1.5	0.00559441470452481\\
1.51	0.00572326683843017\\
1.52	0.00585471049417453\\
1.53	0.00598878810343347\\
1.54	0.00612554255742495\\
1.55	0.00626501720628298\\
1.56	0.00640725585826777\\
1.57	0.00655230277880913\\
1.58	0.00670020268937946\\
1.59	0.00685100076619316\\
1.6	0.00700474263872876\\
1.61	0.0071614743880705\\
1.62	0.00732124254506602\\
1.63	0.00748409408829659\\
1.64	0.00765007644185648\\
1.65	0.00781923747293826\\
1.66	0.00799162548922045\\
1.67	0.00816728923605427\\
1.68	0.00834627789344614\\
1.69	0.00852864107283249\\
1.7	0.00871442881364377\\
1.71	0.00890369157965414\\
1.72	0.00909648025511389\\
1.73	0.00929284614066099\\
1.74	0.00949284094900887\\
1.75	0.00969651680040709\\
1.76	0.00990392621787196\\
1.77	0.0101151221221837\\
1.78	0.0103301578266475\\
1.79	0.0105490870316149\\
1.8	0.0107719638187632\\
1.81	0.0109988426451295\\
1.82	0.0112297783368967\\
1.83	0.0114648260829282\\
1.84	0.0117040414280498\\
1.85	0.0119474802660738\\
1.86	0.0121951988325657\\
1.87	0.0124472536973477\\
1.88	0.0127037017567393\\
1.89	0.0129646002255305\\
1.9	0.0132300066286863\\
1.91	0.0134999787927796\\
1.92	0.0137745748371512\\
1.93	0.0140538531647934\\
1.94	0.0143378724529564\\
1.95	0.0146266916434741\\
1.96	0.0149203699328093\\
1.97	0.0152189667618148\\
1.98	0.0155225418052091\\
1.99	0.0158311549607656\\
2	0.0161448663382137\\
2.01	0.0164637362478493\\
2.02	0.0167878251888549\\
2.03	0.0171171938373269\\
2.04	0.0174519030340098\\
2.05	0.0177920137717355\\
2.06	0.0181375871825679\\
2.07	0.0184886845246504\\
2.08	0.0188453671687578\\
2.09	0.0192076965845497\\
2.1	0.0195757343265274\\
2.11	0.0199495420196917\\
2.12	0.0203291813449033\\
2.13	0.020714714023945\\
2.14	0.0211062018042852\\
2.15	0.0215037064435448\\
2.16	0.0219072896936656\\
2.17	0.0223170132847825\\
2.18	0.0227329389087996\\
2.19	0.0231551282026703\\
2.2	0.0235836427313844\\
2.21	0.0240185439706604\\
2.22	0.0244598932893471\\
2.23	0.0249077519315344\\
2.24	0.0253621809983751\\
2.25	0.0258232414296198\\
2.26	0.0262909939848663\\
2.27	0.0267654992245267\\
2.28	0.0272468174905128\\
2.29	0.0277350088866436\\
2.3	0.0282301332587772\\
2.31	0.0287322501746691\\
2.32	0.029241418903561\\
2.33	0.0297576983955025\\
2.34	0.0302811472604087\\
2.35	0.0308118237468586\\
2.36	0.0313497857206356\\
2.37	0.0318950906430169\\
2.38	0.0324477955488128\\
2.39	0.0330079570241622\\
2.4	0.0335756311840884\\
2.41	0.0341508736498181\\
2.42	0.0347337395258714\\
2.43	0.0353242833769239\\
2.44	0.0359225592044501\\
2.45	0.0365286204231498\\
2.46	0.0371425198371657\\
2.47	0.0377643096160964\\
2.48	0.0383940412708106\\
2.49	0.0390317656290701\\
2.5	0.0396775328109659\\
2.51	0.040331392204175\\
2.52	0.0409933924390448\\
2.53	0.0416635813635106\\
2.54	0.0423420060178543\\
2.55	0.043028712609312\\
2.56	0.0437237464865357\\
2.57	0.0444271521139196\\
2.58	0.0451389730457957\\
2.59	0.0458592519005098\\
2.6	0.0465880303343831\\
2.61	0.0473253490155696\\
2.62	0.0480712475978175\\
2.63	0.048825764694142\\
2.64	0.0495889378504206\\
2.65	0.0503608035189175\\
2.66	0.0511413970317481\\
2.67	0.0519307525742921\\
2.68	0.0527289031585651\\
2.69	0.0535358805965574\\
2.7	0.0543517154735524\\
2.71	0.0551764371214307\\
2.72	0.0560100735919743\\
2.73	0.0568526516301781\\
2.74	0.0577041966475806\\
2.75	0.0585647326956243\\
2.76	0.0594342824390564\\
2.77	0.0603128671293805\\
2.78	0.0612005065783718\\
2.79	0.0620972191316639\\
2.8	0.0630030216424231\\
2.81	0.0639179294451165\\
2.82	0.0648419563293903\\
2.83	0.0657751145140666\\
2.84	0.0667174146212718\\
2.85	0.0676688656507095\\
2.86	0.0686294749540878\\
2.87	0.069599248209715\\
2.88	0.0705781893972751\\
2.89	0.0715663007727959\\
2.9	0.0725635828438214\\
2.91	0.0735700343448021\\
2.92	0.0745856522127145\\
2.93	0.0756104315629238\\
2.94	0.0766443656653014\\
2.95	0.0776874459206114\\
2.96	0.0787396618371763\\
2.97	0.0798010010078386\\
2.98	0.0808714490872275\\
2.99	0.0819509897693467\\
3	0.0830396047654938\\
3.01	0.0841372737825264\\
3.02	0.0852439745014865\\
3.03	0.0863596825565973\\
3.04	0.0874843715146442\\
3.05	0.088618012854755\\
3.06	0.0897605759485901\\
3.07	0.0909120280409574\\
3.08	0.0920723342308645\\
3.09	0.0932414574530208\\
3.1	0.0944193584598023\\
3.11	0.0956059958036935\\
3.12	0.0968013258202172\\
3.13	0.0980053026113663\\
3.14	0.0992178780295499\\
3.15	0.100439001662067\\
3.16	0.101668620816118\\
3.17	0.102906680504373\\
3.18	0.104153123431098\\
3.19	0.105407889978861\\
3.2	0.10667091819583\\
3.21	0.10794214378367\\
3.22	0.109221500086047\\
3.23	0.110508918077769\\
3.24	0.111804326354552\\
3.25	0.113107651123439\\
3.26	0.114418816193881\\
3.27	0.115737742969483\\
3.28	0.117064350440432\\
3.29	0.118398555176626\\
3.3	0.11974027132149\\
3.31	0.121089410586524\\
3.32	0.122445882246559\\
3.33	0.123809593135755\\
3.34	0.125180447644343\\
3.35	0.126558347716114\\
3.36	0.127943192846677\\
3.37	0.129334880082487\\
3.38	0.13073330402065\\
3.39	0.13213835680952\\
3.4	0.133549928150096\\
3.41	0.134967905298216\\
3.42	0.136392173067578\\
3.43	0.137822613833567\\
3.44	0.139259107537919\\
3.45	0.140701531694221\\
3.46	0.142149761394247\\
3.47	0.143603669315142\\
3.48	0.145063125727469\\
3.49	0.146527998504106\\
3.5	0.14799815313001\\
3.51	0.149473452712855\\
3.52	0.15095375799454\\
3.53	0.152438927363579\\
3.54	0.153928816868373\\
3.55	0.155423280231368\\
3.56	0.156922168864102\\
3.57	0.158425331883151\\
3.58	0.159932616126957\\
3.59	0.161443866173569\\
3.6	0.162958924359271\\
3.61	0.164477630798113\\
3.62	0.16599982340235\\
3.63	0.167525337903775\\
3.64	0.169054007875955\\
3.65	0.170585664757372\\
3.66	0.172120137875463\\
3.67	0.17365725447156\\
3.68	0.175196839726729\\
3.69	0.176738716788501\\
3.7	0.17828270679851\\
3.71	0.179828628921003\\
3.72	0.181376300372256\\
3.73	0.182925536450863\\
3.74	0.184476150568913\\
3.75	0.186027954284037\\
3.76	0.187580757332328\\
3.77	0.18913436766213\\
3.78	0.190688591468681\\
3.79	0.192243233229616\\
3.8	0.193798095741313\\
3.81	0.195352980156075\\
3.82	0.196907686020157\\
3.83	0.198462011312601\\
3.84	0.200015752484904\\
3.85	0.201568704501475\\
3.86	0.203120660880904\\
3.87	0.204671413738017\\
3.88	0.206220753826697\\
3.89	0.20776847058349\\
3.9	0.209314352171954\\
3.91	0.210858185527757\\
3.92	0.212399756404511\\
3.93	0.21393884942032\\
3.94	0.215475248105045\\
3.95	0.217008734948253\\
3.96	0.21853909144786\\
3.97	0.220066098159426\\
3.98	0.22158953474612\\
3.99	0.223109180029311\\
4	0.224624812039785\\
4.01	0.226136208069575\\
4.02	0.227643144724377\\
4.03	0.229145397976542\\
4.04	0.230642743218632\\
4.05	0.232134955317513\\
4.06	0.233621808668973\\
4.07	0.235103077252852\\
4.08	0.236578534688654\\
4.09	0.238047954291633\\
4.1	0.239511109129332\\
4.11	0.240967772078555\\
4.12	0.242417715882746\\
4.13	0.243860713209771\\
4.14	0.245296536710069\\
4.15	0.246724959075156\\
4.16	0.248145753096463\\
4.17	0.24955869172449\\
4.18	0.250963548128245\\
4.19	0.252360095754955\\
4.2	0.253748108390029\\
4.21	0.255127360217246\\
4.22	0.256497625879142\\
4.23	0.25785868053759\\
4.24	0.259210299934527\\
4.25	0.260552260452831\\
4.26	0.261884339177307\\
4.27	0.263206313955762\\
4.28	0.264517963460149\\
4.29	0.265819067247764\\
4.3	0.267109405822455\\
4.31	0.268388760695835\\
4.32	0.269656914448466\\
4.33	0.270913650790995\\
4.34	0.27215875462522\\
4.35	0.273392012105055\\
4.36	0.274613210697374\\
4.37	0.275822139242714\\
4.38	0.2770185880158\\
4.39	0.278202348785892\\
4.4	0.279373214876894\\
4.41	0.280530981227242\\
4.42	0.281675444449509\\
4.43	0.282806402889732\\
4.44	0.283923656686419\\
4.45	0.285027007829217\\
4.46	0.286116260217221\\
4.47	0.287191219716896\\
4.48	0.28825169421959\\
4.49	0.289297493698606\\
4.5	0.290328430265828\\
4.51	0.291344318227856\\
4.52	0.292344974141641\\
4.53	0.293330216869595\\
4.54	0.294299867634142\\
4.55	0.2952537500717\\
4.56	0.29619169028607\\
4.57	0.297113516901198\\
4.58	0.298019061113301\\
4.59	0.298908156742324\\
4.6	0.29978064028272\\
4.61	0.300636350953513\\
4.62	0.301475130747636\\
4.63	0.302296824480521\\
4.64	0.303101279837916\\
4.65	0.303888347422908\\
4.66	0.30465788080214\\
4.67	0.30540973655119\\
4.68	0.306143774299102\\
4.69	0.306859856772047\\
4.7	0.307557849836088\\
4.71	0.308237622539045\\
4.72	0.308899047151428\\
4.73	0.309541999206423\\
4.74	0.310166357538923\\
4.75	0.310772004323572\\
4.76	0.311358825111819\\
4.77	0.311926708867954\\
4.78	0.312475548004121\\
4.79	0.313005238414284\\
4.8	0.313515679507133\\
4.81	0.314006774237922\\
4.82	0.314478429139213\\
4.83	0.314930554350523\\
4.84	0.315363063646858\\
4.85	0.315775874466112\\
4.86	0.31616890793534\\
4.87	0.316542088895868\\
4.88	0.316895345927247\\
4.89	0.317228611370034\\
4.9	0.317541821347388\\
4.91	0.317834915785474\\
4.92	0.318107838432669\\
4.93	0.318360536877553\\
4.94	0.318592962565682\\
4.95	0.318805070815141\\
4.96	0.318996820830853\\
4.97	0.319168175717664\\
4.98	0.319319102492166\\
4.99	0.319449572093287\\
5	0.319559559391609\\
5.01	0.319649043197439\\
5.02	0.319718006267613\\
5.03	0.319766435311032\\
5.04	0.31979432099293\\
5.05	0.319801657937876\\
5.06	0.319788444731501\\
5.07	0.319754683920947\\
5.08	0.319700382014057\\
5.09	0.319625549477275\\
5.1	0.319530200732294\\
5.11	0.319414354151413\\
5.12	0.319278032051646\\
5.13	0.319121260687552\\
5.14	0.31894407024281\\
5.15	0.318746494820531\\
5.16	0.318528572432322\\
5.17	0.318290344986095\\
5.18	0.318031858272638\\
5.19	0.317753161950944\\
5.2	0.317454309532312\\
5.21	0.317135358363221\\
5.22	0.316796369606996\\
5.23	0.316437408224253\\
5.24	0.316058542952159\\
5.25	0.315659846282491\\
5.26	0.315241394438521\\
5.27	0.31480326735073\\
5.28	0.314345548631364\\
5.29	0.313868325547838\\
5.3	0.313371688995012\\
5.31	0.312855733466336\\
5.32	0.312320557023887\\
5.33	0.311766261267315\\
5.34	0.311192951301694\\
5.35	0.310600735704312\\
5.36	0.309989726490407\\
5.37	0.309360039077857\\
5.38	0.30871179225085\\
5.39	0.308045108122549\\
5.4	0.307360112096763\\
5.41	0.306656932828639\\
5.42	0.305935702184406\\
5.43	0.305196555200171\\
5.44	0.304439630039798\\
5.45	0.30366506795188\\
5.46	0.302873013225835\\
5.47	0.302063613147121\\
5.48	0.301237017951625\\
5.49	0.300393380779205\\
5.5	0.299532857626443\\
5.51	0.298655607298603\\
5.52	0.297761791360826\\
5.53	0.296851574088582\\
5.54	0.295925122417399\\
5.55	0.294982605891887\\
5.56	0.294024196614091\\
5.57	0.293050069191175\\
5.58	0.292060400682484\\
5.59	0.291055370545979\\
5.6	0.290035160584095\\
5.61	0.288999954889018\\
5.62	0.287949939787436\\
5.63	0.286885303784746\\
5.64	0.285806237508785\\
5.65	0.284712933653071\\
5.66	0.283605586919601\\
5.67	0.28248439396122\\
5.68	0.281349553323585\\
5.69	0.280201265386755\\
5.7	0.279039732306417\\
5.71	0.27786515795479\\
5.72	0.276677747861214\\
5.73	0.275477709152461\\
5.74	0.274265250492784\\
5.75	0.273040582023736\\
5.76	0.271803915303767\\
5.77	0.270555463247651\\
5.78	0.269295440065736\\
5.79	0.268024061203064\\
5.8	0.266741543278376\\
5.81	0.26544810402302\\
5.82	0.264143962219805\\
5.83	0.2628293376418\\
5.84	0.261504450991124\\
5.85	0.260169523837733\\
5.86	0.258824778558242\\
5.87	0.257470438274798\\
5.88	0.25610672679402\\
5.89	0.254733868546051\\
5.9	0.253352088523716\\
5.91	0.251961612221839\\
5.92	0.250562665576713\\
5.93	0.249155474905767\\
5.94	0.247740266847442\\
5.95	0.246317268301297\\
5.96	0.244886706368375\\
5.97	0.24344880829184\\
5.98	0.242003801397917\\
5.99	0.24055191303714\\
6	0.239093370525952\\
6.01	0.237628401088657\\
6.02	0.23615723179975\\
6.03	0.234680089526651\\
6.04	0.233197200872859\\
6.05	0.231708792121536\\
6.06	0.230215089179558\\
6.07	0.228716317522038\\
6.08	0.22721270213734\\
6.09	0.22570446747261\\
6.1	0.224191837379833\\
6.11	0.22267503506244\\
6.12	0.221154283022476\\
6.13	0.21962980300835\\
6.14	0.218101815963184\\
6.15	0.216570541973764\\
6.16	0.215036200220133\\
6.17	0.213499008925817\\
6.18	0.21195918530871\\
6.19	0.210416945532633\\
6.2	0.208872504659575\\
6.21	0.207326076602634\\
6.22	0.205777874079673\\
6.23	0.204228108567692\\
6.24	0.202676990257946\\
6.25	0.201124728011806\\
6.26	0.199571529317377\\
6.27	0.198017600246892\\
6.28	0.196463145414885\\
6.29	0.194908367937152\\
6.3	0.193353469390517\\
6.31	0.191798649773403\\
6.32	0.190244107467226\\
6.33	0.188690039198609\\
6.34	0.187136640002441\\
6.35	0.185584103185767\\
6.36	0.184032620292533\\
6.37	0.182482381069184\\
6.38	0.180933573431129\\
6.39	0.179386383430064\\
6.4	0.177840995222173\\
6.41	0.17629759103721\\
6.42	0.174756351148454\\
6.43	0.173217453843555\\
6.44	0.17168107539627\\
6.45	0.170147390039088\\
6.46	0.16861656993675\\
6.47	0.167088785160669\\
6.48	0.165564203664248\\
6.49	0.164042991259093\\
6.5	0.162525311592137\\
6.51	0.161011326123657\\
6.52	0.159501194106197\\
6.53	0.157995072564391\\
6.54	0.156493116275693\\
6.55	0.154995477751992\\
6.56	0.153502307222147\\
6.57	0.1520137526154\\
6.58	0.15052995954569\\
6.59	0.149051071296865\\
6.6	0.147577228808773\\
6.61	0.146108570664244\\
6.62	0.144645233076952\\
6.63	0.143187349880159\\
6.64	0.141735052516324\\
6.65	0.14028847002759\\
6.66	0.138847729047126\\
6.67	0.137412953791335\\
6.68	0.135984266052907\\
6.69	0.134561785194725\\
6.7	0.133145628144603\\
6.71	0.131735909390866\\
6.72	0.130332740978749\\
6.73	0.128936232507617\\
6.74	0.127546491129\\
6.75	0.126163621545422\\
6.76	0.124787726010036\\
6.77	0.123418904327037\\
6.78	0.122057253852859\\
6.79	0.120702869498137\\
6.8	0.119355843730431\\
6.81	0.1180162665777\\
6.82	0.116684225632517\\
6.83	0.115359806057013\\
6.84	0.114043090588543\\
6.85	0.112734159546067\\
6.86	0.111433090837222\\
6.87	0.11013995996609\\
6.88	0.108854840041642\\
6.89	0.10757780178685\\
6.9	0.106308913548457\\
6.91	0.105048241307385\\
6.92	0.103795848689784\\
6.93	0.102551796978697\\
6.94	0.10131614512634\\
6.95	0.100088949766973\\
6.96	0.0988702652303605\\
6.97	0.0976601435558059\\
6.98	0.0964586345067445\\
6.99	0.0952657855858868\\
7	0.0940816420508967\\
7.01	0.0929062469305942\\
7.02	0.091739641041668\\
7.03	0.0905818630058867\\
7.04	0.0894329492677946\\
7.05	0.088292934112881\\
7.06	0.087161849686207\\
7.07	0.0860397260114811\\
7.08	0.0849265910105658\\
7.09	0.0838224705234065\\
7.1	0.0827273883283668\\
7.11	0.081641366162959\\
7.12	0.0805644237449543\\
7.13	0.0794965787938631\\
7.14	0.0784378470527686\\
7.15	0.0773882423105042\\
7.16	0.0763477764241583\\
7.17	0.0753164593418964\\
7.18	0.0742942991260859\\
7.19	0.0732813019767105\\
7.2	0.0722774722550628\\
7.21	0.0712828125077007\\
7.22	0.0702973234906556\\
7.23	0.0693210041938795\\
7.24	0.0683538518659187\\
7.25	0.0673958620388005\\
7.26	0.0664470285531221\\
7.27	0.0655073435833272\\
7.28	0.0645767976631603\\
7.29	0.063655379711284\\
7.3	0.0627430770570496\\
7.31	0.0618398754664067\\
7.32	0.0609457591679419\\
7.33	0.0600607108790333\\
7.34	0.0591847118321099\\
7.35	0.0583177418010041\\
7.36	0.0574597791273863\\
7.37	0.0566108007472685\\
7.38	0.0557707822175694\\
7.39	0.0549396977427251\\
7.4	0.0541175202013394\\
7.41	0.0533042211728588\\
7.42	0.0524997709642644\\
7.43	0.0517041386367693\\
7.44	0.0509172920325106\\
7.45	0.0501391978012268\\
7.46	0.0493698214269105\\
7.47	0.0486091272544258\\
7.48	0.0478570785160813\\
7.49	0.04711363735815\\
7.5	0.0463787648673247\\
7.51	0.0456524210971024\\
7.52	0.0449345650940864\\
7.53	0.0442251549241985\\
7.54	0.0435241476987926\\
7.55	0.0428314996006617\\
7.56	0.0421471659099276\\
7.57	0.0414711010298096\\
7.58	0.0408032585122601\\
7.59	0.0401435910834611\\
7.6	0.0394920506691755\\
7.61	0.0388485884199429\\
7.62	0.0382131547361147\\
7.63	0.0375856992927223\\
7.64	0.0369661710641687\\
7.65	0.0363545183487397\\
7.66	0.0357506887929274\\
7.67	0.0351546294155594\\
7.68	0.0345662866317275\\
7.69	0.0339856062765119\\
7.7	0.0334125336284925\\
7.71	0.0328470134330438\\
7.72	0.0322889899254083\\
7.73	0.0317384068535412\\
7.74	0.0311952075007242\\
7.75	0.0306593347079419\\
7.76	0.0301307308960168\\
7.77	0.0296093380874989\\
7.78	0.029095097928305\\
7.79	0.0285879517091046\\
7.8	0.0280878403864484\\
7.81	0.0275947046036344\\
7.82	0.0271084847113114\\
7.83	0.0266291207878126\\
7.84	0.0261565526592195\\
7.85	0.0256907199191516\\
7.86	0.02523156194828\\
7.87	0.0247790179335609\\
7.88	0.0243330268871889\\
7.89	0.0238935276652653\\
7.9	0.0234604589861821\\
7.91	0.0230337594487173\\
7.92	0.022613367549841\\
7.93	0.0221992217022314\\
7.94	0.0217912602514977\\
7.95	0.0213894214931097\\
7.96	0.0209936436890328\\
7.97	0.0206038650840679\\
7.98	0.0202200239218939\\
7.99	0.0198420584608144\\
8	0.0194699069892059\\
8.01	0.019103507840669\\
8.02	0.0187427994088807\\
8.03	0.0183877201621492\\
8.04	0.0180382086576705\\
8.05	0.0176942035554866\\
8.06	0.0173556436321467\\
8.07	0.0170224677940708\\
8.08	0.0166946150906171\\
8.09	0.0163720247268531\\
8.1	0.0160546360760318\\
8.11	0.0157423886917736\\
8.12	0.0154352223199552\\
8.13	0.0151330769103064\\
8.14	0.0148358926277161\\
8.15	0.0145436098632491\\
8.16	0.0142561692448743\\
8.17	0.0139735116479077\\
8.18	0.0136955782051693\\
8.19	0.0134223103168583\\
8.2	0.0131536496601461\\
8.21	0.0128895381984906\\
8.22	0.0126299181906737\\
8.23	0.0123747321995629\\
8.24	0.0121239231006\\
8.25	0.0118774340900205\\
8.26	0.0116352086928026\\
8.27	0.0113971907703519\\
8.28	0.0111633245279221\\
8.29	0.0109335545217749\\
8.3	0.0107078256660808\\
8.31	0.0104860832395661\\
8.32	0.0102682728919047\\
8.33	0.0100543406498618\\
8.34	0.00984423292318888\\
8.35	0.00963789651027523\\
8.36	0.00943527860355735\\
8.37	0.00923632679469026\\
8.38	0.00904098907948317\\
8.39	0.00884921386260283\\
8.4	0.00866094996204767\\
8.41	0.00847614661339551\\
8.42	0.00829475347382863\\
8.43	0.00811672062593896\\
8.44	0.00794199858131661\\
8.45	0.00777053828392544\\
8.46	0.00760229111326832\\
8.47	0.00743720888734606\\
8.48	0.00727524386541275\\
8.49	0.0071163487505313\\
8.5	0.00696047669193228\\
8.51	0.00680758128717966\\
8.52	0.00665761658414656\\
8.53	0.00651053708280472\\
8.54	0.00636629773683088\\
8.55	0.00622485395503358\\
8.56	0.00608616160260388\\
8.57	0.00595017700219317\\
8.58	0.00581685693482179\\
8.59	0.00568615864062177\\
8.6	0.00555803981941703\\
8.61	0.00543245863114457\\
8.62	0.00530937369612007\\
8.63	0.00518874409515118\\
8.64	0.00507052936950197\\
8.65	0.00495468952071215\\
8.66	0.00484118501027384\\
8.67	0.00472997675916999\\
8.68	0.00462102614727719\\
8.69	0.00451429501263655\\
8.7	0.00440974565059588\\
8.71	0.00430734081282629\\
8.72	0.00420704370621683\\
8.73	0.00410881799165\\
8.74	0.00401262778266171\\
8.75	0.00391843764398876\\
8.76	0.00382621259000694\\
8.77	0.00373591808306305\\
8.78	0.00364752003170384\\
8.79	0.0035609847888051\\
8.8	0.00347627914960381\\
8.81	0.00339337034963658\\
8.82	0.00331222606258721\\
8.83	0.00323281439804643\\
8.84	0.00315510389918684\\
8.85	0.00307906354035585\\
8.86	0.00300466272458949\\
8.87	0.00293187128105012\\
8.88	0.0028606594623906\\
8.89	0.00279099794204787\\
8.9	0.00272285781146872\\
8.91	0.00265621057727016\\
8.92	0.00259102815833753\\
8.93	0.00252728288286249\\
8.94	0.00246494748532389\\
8.95	0.0024039951034138\\
8.96	0.00234439927491129\\
8.97	0.00228613393450666\\
8.98	0.00222917341057807\\
8.99	0.0021734924219235\\
9	0.00211906607445008\\
9.01	0.00206586985782318\\
9.02	0.00201387964207761\\
9.03	0.00196307167419305\\
9.04	0.00191342257463601\\
9.05	0.00186490933387047\\
9.06	0.00181750930883927\\
9.07	0.0017712002194183\\
9.08	0.00172596014484571\\
9.09	0.00168176752012786\\
9.1	0.00163860113242423\\
9.11	0.00159644011741304\\
9.12	0.00155526395563949\\
9.13	0.00151505246884854\\
9.14	0.00147578581630389\\
9.15	0.00143744449109497\\
9.16	0.00140000931643379\\
9.17	0.00136346144194302\\
9.18	0.00132778233993727\\
9.19	0.00129295380169897\\
9.2	0.00125895793375037\\
9.21	0.00122577715412341\\
9.22	0.00119339418862865\\
9.23	0.0011617920671249\\
9.24	0.00113095411979088\\
9.25	0.00110086397340032\\
9.26	0.00107150554760171\\
9.27	0.00104286305120415\\
9.28	0.00101492097847045\\
9.29	0.000987664105418677\\
9.3	0.000961077486133441\\
9.31	0.000935146449087961\\
9.32	0.000909856593478026\\
9.33	0.000885193785569004\\
9.34	0.000861144155056849\\
9.35	0.000837694091444159\\
9.36	0.000814830240432241\\
9.37	0.000792539500330117\\
9.38	0.000770809018481389\\
9.39	0.000749626187709831\\
9.4	0.000728978642784507\\
9.41	0.000708854256905303\\
9.42	0.000689241138209577\\
9.43	0.000670127626300705\\
9.44	0.000651502288799236\\
9.45	0.000633353917917341\\
9.46	0.000615671527057199\\
9.47	0.000598444347433983\\
9.48	0.000581661824723986\\
9.49	0.000565313615738524\\
9.5	0.000549389585124119\\
9.51	0.000533879802089483\\
9.52	0.000518774537159811\\
9.53	0.000504064258958826\\
9.54	0.000489739631019037\\
9.55	0.000475791508620601\\
9.56	0.000462210935659209\\
9.57	0.000448989141543317\\
9.58	0.000436117538121123\\
9.59	0.000423587716637555\\
9.6	0.000411391444721603\\
9.61	0.000399520663404253\\
9.62	0.000387967484167287\\
9.63	0.000376724186023164\\
9.64	0.000365783212626227\\
9.65	0.00035513716941539\\
9.66	0.000344778820788498\\
9.67	0.00033470108730853\\
9.68	0.000324897042941733\\
9.69	0.000315359912327858\\
9.7	0.000306083068082558\\
9.71	0.000297060028132048\\
9.72	0.000288284453080093\\
9.73	0.00027975014360736\\
9.74	0.000271451037903185\\
9.75	0.000263381209129764\\
9.76	0.000255534862918768\\
9.77	0.000247906334900376\\
9.78	0.000240490088264704\\
9.79	0.000233280711355582\\
9.8	0.000226272915296628\\
9.81	0.000219461531649576\\
9.82	0.000212841510104734\\
9.83	0.000206407916203538\\
9.84	0.000200155929093061\\
9.85	0.000194080839312381\\
9.86	0.000188178046610692\\
9.87	0.000182443057797005\\
9.88	0.000176871484621314\\
9.89	0.000171459041687076\\
9.9	0.000166201544394823\\
9.91	0.000161094906916755\\
9.92	0.000156135140202134\\
9.93	0.000151318350013282\\
9.94	0.000146640734991985\\
9.95	0.000142098584756127\\
9.96	0.000137688278026307\\
9.97	0.000133406280782254\\
9.98	0.000129249144448798\\
9.99	0.000125213504111178\\
10	0.000121296076759446\\
};
%\addlegendentry{sigma points}

\addplot [color=mycolor2, line width=2.0pt, forget plot]
  table[row sep=crcr]{%
5.04857024213078	0\\
5.04857024213078	0.6\\
};
\addplot [color=mycolor1, dashed, line width=2.0pt, forget plot]
  table[row sep=crcr]{%
5.0284309151552	0\\
5.0284309151552	0.6\\
};
\end{axis}
\end{tikzpicture}%
% -------------------------------------------

% -------------------------------------------
\section{Qualitative Visualization results}
\label{secS7}
% -------------------------------------------
{\color{red}{\emph{This supplementary is for Section~4.2 and~4.3 of the main paper.}}} In this section, we show qualitative results on both instance segmentation and semantic segmentation. To demonstrate the superiority of our method, we present visualization results of ablation studies on instance segmentation, as well as comparisons with state-of-the-art methods on both instance segmentation and semantic segmentation. 
% 
The obtained visualization results are shown in Figure~\ref{figs2}. From the results, it can be observed that compared to other methods, our method can achieve more accurate object masks that better fit the actual boundaries of the objects themselves.
% -------------------------------------------
% \begin{figure}[t]
%     \centering
%     \includegraphics[width=\linewidth]{figure/images/concept_affordance_prompt_v2.pdf}
%     \caption{Results of two different levels of prompts.}
%     \label{fig:prompt}
% \end{figure}
% -------------------------------------------

as well as the pseudo-code for when the stripe size is set to $2$ in Section~\ref{secS8}. 
% -------------------------------------------
\section{Pseudo-code fo stripe size = $2$}
\label{secS8}
% -------------------------------------------
In this code snippet, stripe size is set to 2, and relevant features are directly obtained using the gather function instead of reshaping them with img2windows. This operation can reduce unnecessary reshaping operations and improves the efficiency of the code.
% -------------------------------------------
\begin{python}
function cross_shaped_window_attention(x, num_heads, window_size):
    # x: given feature
    # num_heads: head number
    # window_size: window size

    # Get dimensions
    (batch_size, seq_length, d_model) = shape(x)

    # Split into multiple heads
    Q, K, V = split_heads(x, num_heads)

    # Initialize attention output
    attention_output = zeros(batch_size, seq_length, d_model)

    # Initialize previous head's output for cascaded attention
    previous_Q = zeros(batch_size, seq_length, d_model)
    previous_K = zeros(batch_size, seq_length, d_model)
    previous_V = zeros(batch_size, seq_length, d_model)

    # Calculate attention for each head
    for head in range(num_heads):
        for position in range(seq_length):
            # Get cross-shaped window indices
            window_indices = get_cross_shaped_window_indices(position, window_size)

            # Gather Q, K, V for the current window
            Q_window = gather(Q[head], window_indices)
            K_window = gather(K[head], window_indices)
            V_window = gather(V[head], window_indices)

            # Incorporate previous head's output for cascaded attention
            if head > 0:
                Q_window += previous_Q
                K_window += previous_K
                V_window += previous_V

            # Calculate attention scores
            attention_scores = softmax(Q_window * K_window^T / sqrt(d_k))

            # Compute the attention output for the current position
            attention_output[position] = attention_scores * V_window

        # Update previous head's output for the next head
        previous_Q = Q_window
        previous_K = K_window
        previous_V = V_window

    # Final linear transformation
    attention_output = linear_transform(attention_output)
    return attention_output

function feed_forward_network(x):
    # Feed Forward Network
    x = ReLU(linear(x))
    x = linear(x)
    return x
\end{python}

\begin{python}
def get_cross_shaped_window_indices(position, window_size, seq_length):
    # Initialize the list of indices
    indices = []

    # Add the current position
    indices.append(position)

    # Add vertical neighbors (up and down)
    for offset in range(-window_size, window_size + 1):
        if position + offset >= 0 and position + offset < seq_length:
            indices.append(position + offset)

    # Remove duplicates and sort the indices
    indices = list(set(indices))
    indices.sort()

    return indices
\end{python}
% ----------------------------------------------


\end{document}