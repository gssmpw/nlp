\section{Numerical Results}
\label{sec:exp}

Numerical implementation was performed in Python and is available at github.com/lukearcus/fossil\_scenario. 
Simulations were carried out on a server with 80 2.5 GHz CPUs and 125 GB of RAM. 
For all numerical simulations, we considered a confidence level of $\beta=10^{-5}$, $N=1000$ samples; our results are averaged across $5$ independent repetitions, each with different multi-samples. 
By sample complexity, we refer to the number of \emph{trajectory samples}, separate to the states used for the sample-independent loss since these samples can be obtained without accessing the system dynamics.

\subsection{Benchmark Dynamical System}

To demonstrate the efficacy of our techniques across all certificates presented, we use the following dynamical system as benchmark, with state vector $x(k) = (x_1(k), x_2(k)) \in \mathbb{R}^2$, namely,
\begin{equation}
	\label{eq:spiral_dyn}
	\begin{bmatrix}x_1(k+1)\\x_2(k+1)\end{bmatrix}=\begin{bmatrix}x_1(k)-\frac{T_\text{d}}{2}x_2(k)\\ x_2(k)+\frac{T_\text{d}}{2}(x_1(k)-x_2(k))\end{bmatrix},
\end{equation}
where $T_\text{d} = 0.1$ and we use time horizon $T=100$ steps.
The phase plane plot for these dynamics is in Figure \ref{fig:spiral_phase} alongside different sets related to the definition of the reachability, safety and RWA properties are shown. For the reachability property we aim at verifying that trajectories reach the goal set $X_G$ (circle centered at the origin), without 
leaving the domain $X$ until then. The unsafe set is not relevant as far as this property is concerned
For safety, we require that trajectories do not enter the unsafe region $X_U$.
Finally, for the RWA property, we have the domain $X$ and the unsafe set $X_U$.

\begin{figure}[t]
\centering
	\includegraphics[width=0.75\linewidth]{Figures/spiral_plane}
	\caption{Phase plane plot for the dynamical system of \eqref{eq:spiral_dyn}. The different sets shown are related to the sets that appear in the definitions of the reachability, safety and RWA property. For each case, only the relevant sets are considered.}
	\label{fig:spiral_phase}
\end{figure}

\begin{figure*}[ht]
\centering
	\begin{minipage}{0.3\linewidth}
    \centering
		\includegraphics[width=\linewidth]{Figures/spiral_reach_surf}
        \caption{Surface plot of the reachability certificate}
        \label{fig:spiral_reach_surf}
	\end{minipage}\hfill
	\begin{minipage}{0.3\linewidth}
    \centering
		\includegraphics[width=\textwidth]{Figures/spiral_barr_surf}
        \caption{Surface plot of the safety/barrier certificate.}
        \label{fig:spiral_barr_surf}
	\end{minipage}\hfill
	\begin{minipage}{0.3\linewidth}
		\includegraphics[width=\linewidth]{Figures/spiral_rws_surf}
        \caption{Surface plot of the RWA certificate.}
        \label{fig:spiral_rws_surf}
	\end{minipage}\hfill
\end{figure*}

\begin{table*}[t]
\begin{center}
    	\caption{Probabilistic guarantees for the system in \eqref{eq:spiral_dyn}. Standard deviations are shown in parentheses alongside means.}
	\begin{tabular}{p{2.5cm} || p{2cm} | p{2cm}| p{2cm} | p{2cm}| p{2cm}| p{2cm} }
		& Certificate Risk Bound $\varepsilon$ in Theorem \ref{thm:Guarantees} & Empirical Certificate Risk $\hat{\varepsilon}$&  Algorithm~\ref{algo:main} Computation Time (s)  & Property ~~~Risk Bound $\varepsilon$ in Prop. \ref{corr:a_post} & Empirical Property Risk $\hat{\varepsilon}$ & Direct Bound Computation Time (s) \\ \hline \hline
		Reach Certificate (Proposition \ref{cert:reach}) & 0.026 (0.004) & 0 (0) & 15914.6 (11955.3)  & 0.011 (0) & 0 (0) & 2.5 (0.3)  \\ \hline
		Safety Certificate (Proposition \ref{cert:barr}) & 0.044 (0.012) & 0 (0) & 7489.7 (501.8) & 0.011 (0)& 0 (0) & 6.0 (0.4) \\ \hline
		RWA Certificate ~~~(Proposition \ref{cert:RWA}) & 0.035 (0.011) & 0 (0) &  14924.2 (6963.5) & 0.011 (0) & 0 (0) & 6.0 (0.5)  \\ \hline 
	\end{tabular} \label{tab:guarantees}
\end{center}
\end{table*}

Surface plots of the reachability, barrier and RWA certificate are shown in Figure \ref{fig:spiral_reach_surf}, Figure \ref{fig:spiral_barr_surf} and Figure \ref{fig:spiral_rws_surf}, respectively. The zero and $-\delta$-sublevel sets of these certificates are highlighted with dashed black lines. With reference to Figure \ref{fig:spiral_reach_surf} notice that the zero-sublevel set includes both the initial and the goal set, and no states outside the domain as expected. Similarly, in Figure \ref{fig:spiral_barr_surf} the zero-sublevel set of the barrier function does not pass through the unsafe set, while the zero-sublevel set of the RWA certificate does not pass through the unsafe set, and does not include states outside the domain.

The constructed certificates depend on $N$ samples. By means of Algorithm \ref{algo:sub} and Theorem \ref{thm:Guarantees}, these certificates are associated with a theoretical risk bound $\varepsilon$ (that bounds the probability that the certificate will not meet the conditions of the associated property when it comes to a new sample/trajectory). Table \ref{tab:guarantees} shows this risk bound as computed via Theorem \ref{thm:Guarantees}. We quantified empirically this property; namely, we generated additional samples and calculated the number of samples for which the computed certificate violated the associated certificate's conditions. This empirical certificate risk is denoted by $\hat{\varepsilon}$ is shown in the second column of Table \ref{tab:guarantees}. Note that, as expected, the empirical values are lower than the theoretical bounds. 

The fourth column of Table \ref{tab:guarantees} provides the risk bound $\varepsilon$ that would be obtained for direct property violation statements (however, without allowing for certificate construction) as per Proposition \ref{corr:a_post}, this always results in a risk of 0.01825 as no samples are discarded, since the system can be shown to be deterministically safe. 
Recall that the results in the first column of Table \ref{tab:guarantees} bound (implicitly) the probability of property violation, as discussed in the second remark after the proof of Theorem \ref{thm:Guarantees}.

\subsection{Dynamical System of Higher Dimension}

We now investigate a dynamical system of higher dimension with a state $x(k) \in \mathbb{R}^8$, governed by
\begin{equation}
	\begin{aligned}
		x_i(k+1) &= x_i(k) + 0.1x_{i+1}(k),~ i = 1\dots 7,\\
		x_8(k+1) &= x_8(k)- 0.1(576x_1(k)+2400x_2(k)\\
        &+4180x_3(k)+3980x_4(k)+2273x_5(k)\\
             &+800x_6(k)+170x_7(k)+20x_8(k)).
	\end{aligned}
\end{equation}
We define $X=[-2.2,2.2]^8, X_I=[0.9,1.1]^8, X_U = [-2.2,-1.8]^8$.
Once again, the entire of the initial set can be shown to be safe, and so we aim to generate a guarantee as close to $0$ as possible.
We employ Algorithm \ref{algo:sub} to generate a safety certificate. This required an average of $0.280$ seconds, with a standard deviation of $0.004$ seconds. 

This certificate is computed much faster than those in Table~\ref{tab:guarantees}, this is possible since the runtime of our algorithm is primarily constrained by how many samples need to be removed by Algorithm~\ref{algo:main} in order to bring the loss to $0$.
This can be seen as a measure of how ``hard'' the problem is.
In this example, it is likely that the sets are easy to separate whilst still maintaining the difference condition, whereas the system in the previous section required more computation since trajectories move towards the unsafe set before moving away.

Due to the higher-dimensional state space, this certificate is not illustrated pictorially. 
It is accompanied by a probabilistic certificate $\varepsilon = 0.019$ (standard deviation $0.001$) computed by means of Theorem \ref{thm:Guarantees}.
Using Propositon~\ref{corr:a_post}, we find a guarantee of $0.011$ (standard deviation $0$), after $2.27$ seconds (standard deviation $0.02$s).


 \begin{figure*}[t]
 	\begin{minipage}[t]{0.3\linewidth}
	   \includegraphics[width=\linewidth]{Figures/spiral_barr_unsafe_plane}
	   \caption{Phase plane plot, initial and unsafe set for of partially unsafe system.}
	   \label{fig:spiral_barr_unsafe_plane} 
    \end{minipage}\hfill
 	\begin{minipage}[t]{0.3\linewidth}
	   \includegraphics[width=\linewidth]{Figures/spiral_barr_unsafe_surf}
	   \caption{Surface plot of the safety/barrier certificate for the partially unsafe system of Figure \ref{fig:spiral_barr_unsafe_plane}.}
	   \label{fig:spiral_barr_unsafe_surf}
    \end{minipage}\hfill
    \begin{minipage}[t]{0.3\linewidth}
        	\includegraphics[width=\linewidth]{Figures/jet_eng_comparison}
	\caption{Comparison with \cite{DBLP:journals/tac/NejatiLJSZ23}. The zero-level set of the safety certificate of our approach is dashed; level sets that separate the initial and unsafe sets (i.e. $\gamma$- and $\lambda$- level sets) from \cite{DBLP:journals/tac/NejatiLJSZ23} are dotted.}
	\label{fig:jet_eng_comparison}
    \end{minipage}
 \end{figure*}
 
\subsection{Partially Unsafe Systems}
For the numerical experiments so far, all sampled trajectories satisfied the property of interest.
We now consider the problem of safety certificate construction for the system in \eqref{eq:spiral_dyn} with an enlarged unsafe region (see Figure \ref{fig:spiral_barr_unsafe_plane}). We refer to this system as partially unsafe, as some sampled trajectories enter the unsafe set. Unlike existing techniques which require either a deterministically safe system~\cite{DBLP:conf/hybrid/EdwardsPA24}, or stochastic dynamics~\cite{DBLP:conf/cdc/PrajnaJP04}, we are still able to synthesize a probabilistic barrier certificate.  
The zero-sublevel set of the constructed safety certificate is shown by a dashed line in both Figures \ref{fig:spiral_barr_unsafe_plane} and \ref{fig:spiral_barr_unsafe_surf}. Figure \ref{fig:spiral_barr_unsafe_surf} provides a surface of the constructed certificate, and demonstrates that it separates the initial and the unsafe set. The computation time was $32360.0$ seconds (standard deviation $3011.9$s). 

For this certificate, we obtained a theoretical risk bound $\varepsilon = 0.387$ (standard deviation $0.019$) by means of Theorem \ref{thm:Guarantees}, and an empirical property risk of $\hat{\varepsilon} = 0.013$ (standard deviation $0.002$). 
These guarantees are not tight; we could improve these by considering additional samples and performing the discarding procedure of Algorithm \ref{algo:main}, however, this would lead to larger computation times.
To prevent discarding too many trajectories, we only discard those in the compression set at each iteration, which is likely to be a smaller number.

Proposition~\ref{corr:a_post} gives risk bound $0.038$ (standard deviation $0.004$) after $4.4$ seconds (standard deviation $1.4$s).

\subsection{Comparison with \cite{DBLP:journals/tac/NejatiLJSZ23}}

We also compare our work with the one of \cite{DBLP:journals/tac/NejatiLJSZ23}, which has been reviewed in Section \ref{sec:related}. To this end, we construct a safety certificate for a two-dimensional DC Motor as considered in \cite{DBLP:journals/tac/NejatiLJSZ23}. 
We first replicate the methodology of \cite{DBLP:journals/tac/NejatiLJSZ23}, using the Lipschitz constants they provide.

The methodology of \cite{DBLP:journals/tac/NejatiLJSZ23} required $257149$ samples and $748.5$ seconds (standard deviation $246.9$s) of computation time to compute a barrier certificate with confidence at least equal to $0.99$. 
Using $1000$ samples and $0.9$ seconds of computation time (standard deviation $0.7$s), we obtained $\varepsilon = 0.01$ (standard deviation $0$), i.e. we can bound safety with a risk of 1\%, for the same confidence.
It can thus be observed that the numerical computation savings (in terms of number of samples -- this might be an expensive task -- and computation time) are significant. Figure \ref{fig:jet_eng_comparison} illustrates a phase plane plot and the initial and unsafe sets for this problem. The dotted lines correspond to the sublevel sets constructed in \cite{DBLP:journals/tac/NejatiLJSZ23} (one lower bounding the unsafe set, the other upper bounding the initial set). 
The dashed line depicts the zero-sublevel set of the certificate constructed by our approach.

We also performed a comparison on the following four-dimensional system, a discretized version of a model taken from~\cite{DBLP:conf/hybrid/EdwardsPA24}, with the required Lipschitz constants estimated using the technique in \cite{DBLP:journals/jgo/WoodZ96}.
\begin{equation}
\begin{aligned}
        x_1(k+1) &= x_1(k) + 0.1\left(\frac{x_1(k)  x_2(k)}{5} -\frac{x_3(k)x_4(k)}{2}\right),\\
        x_2(k+1) &= x_2(k)+0.1\cos(x_4(k)),\\
        x_3(k+1) &= x_3(k)+0.001\sqrt{|x_1(k)|},\\
        x_4(k+1) &= x_4(k)+0.1\left(-x_1(k) - x_2(k)^2 + \sin(x_4(k))\right).
\end{aligned}
\end{equation}
Due to the reasons outlined in Section \ref{sec:related}, the approach of \cite{DBLP:journals/tac/NejatiLJSZ23} with $10^{19}$ samples results in a confidence of at least $10^{-30}$, which is not practically useful.
In contrast, with our techniques with $1000$ samples we obtain a risk level of $\varepsilon = 0.02039$, with confidence at least $1-10^{-5}$.
