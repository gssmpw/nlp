\section{Introduction}
\let\thefootnote\relax\footnotetext{This work was supported by the EPSRC Centre for Doctoral Training in Autonomous Intelligent Machines and Systems EP/S024050/1.}

%\subsection{Motivation}
Dynamical systems offer a rich class of models for describing the behavior of systems \cite{Hirsch2003DifferentialED}. 
It is often of importance that these systems meet certain properties, for example, stability, safety or reachability\cite{DBLP:conf/hybrid/EdwardsPA24,DBLP:journals/tac/JagtapSZ21,DBLP:journals/tac/NejatiLJSZ23,DBLP:journals/tac/PrajnaJP07}.
Verifying the satisfaction of these properties, also termed as specifications, is a challenging, but important, problem.

%\subsection{Certificates}
A common approach to verify properties of dynamical systems is through the use of \emph{certificates} \cite{DBLP:conf/eucc/AmesCENST19,DBLP:conf/hybrid/PrajnaJ04}.
The goal is to determine a function over the system's state space which exhibits certain properties.
A well-investigated example of such certificate is that of a Lyapunov function, used to verify that dynamics satisfy some stability property \cite{Lyapunov1994TheGP}. Here
we consider constructing reachability, safety, and reach-while-avoid (RWA) certificates for discrete time systems.
In general, constructing such certificates is a case dependent task and requires domain specific expertise. 
Restricting the class of certificates to polynomial functions, 
they can be obtained by solving a convex sum-of-squares problem \cite{DBLP:journals/corr/PapachristodoulouAVPSP13,DBLP:conf/cdc/Papachristodoulou02}.

In this work, rather than imposing assumptions on the class of certificate functions, we view the problem under a data driven lens. To this end, we use a \emph{neural network} as a certificate template, which allows handling a general class of dynamical systems. 
Neural networks are well studied function approximators, with a broad array of applications, from image recognition \cite{DBLP:journals/corr/SimonyanZ14a} to reinforcement learning \cite{DBLP:journals/ral/CaiHXAK21}.
We employ neural networks with no assumptions on their structure to generate certificate parameterizations, and develop an algorithm for training these networks based on past system trajectories that play the role of data. 
We consider a safety informed neural network training process, employing a subgradient descent procedure to tackle the underlying non-convex optimization problem \cite{DBLP:books/cu/BV2014}, and minimizing a loss function encompassing conditions on the certificate to be learned that encode the satisfaction of the associated property.

Besides learning a certificate, we quantify probabilistically its generalization properties, namely, how likely it is for a certificate to be valid (and hence for the associated property
to be satisfied) when it comes to a new system trajectory not included in the training data set. We view this problem under
the realm of probably approximately correct (PAC) learning under the notion of compression \cite{DBLP:journals/jmlr/CampiG23,Floyd_Warmuth_1995,DBLP:journals/tac/MargellosPL15}, and use recent advancements
of the so-called scenario approach \cite{Scen_approach_book,DBLP:journals/siamjo/CampiG08,DBLP:journals/mp/CampiG18,DBLP:journals/tac/CampiGR18,DBLP:journals/mp/GarattiC22} to obtain scalable generalization bounds on the learned certificates.

There has been a considerable amount of work towards data driven verification. One literature stream considers a direct property verification approach with no certificate construction. 
One such approach is to consider discretizing the state space \cite{DBLP:journals/jair/BadingsRAPPSJ23,DBLP:conf/qest/RickardBRA23}, in order to construct a finite model which may be verified using statistical model checkers, or through statistical learning theoretic results \cite{DBLP:journals/sttt/BadingsCJJKT22,DBLP:conf/l4dc/RickardAM24}.
Such techniques frequently suffer from the so-called ``curse of dimensionality'' which prohibits their application to systems of higher dimension.
Therefore, abstraction-free techniques that involve certificate synthesis constitute an alternative approach.
To this end, work has been conducted on synthesizing provably correct neural certificates \cite{DBLP:conf/hybrid/AbateAEGP21,DBLP:conf/hybrid/EdwardsPA24}, making use of SAT-modulo-theory solvers to verify that the synthesized networks meet the requirements.
Alternatively, the techniques in \cite{DBLP:journals/tac/NejatiLJSZ23,DBLP:journals/automatica/SalamatiLSZ24} are the most closely related ones to our work, and consider synthesizing a neural certificate in a data-driven manner, using samples from the state space. However, the associated analysis accompanies the constructed certificates probabilistic guarantees of different nature, and is complementary to our approach. In particular, unlike our work, the generalization guarantees in \cite{DBLP:journals/tac/NejatiLJSZ23} scale linearly in the dimension of the decision space (certificate parameterization) and exponentially in the uncertainty space (where samples are drawn from). This dependency hampers the application of these techniques to systems of higher dimensions.

Our main contributions can be summarized as follows: 
\begin{enumerate}
    \item We develop a novel methodology for the synthesis of neural certificates to verify a wide class of properties, namely, reachability, safety and reach-while-avoid properties, of discrete-time dynamical systems. We accompany the constructed certificates with probabilistic guarantees on their generalization properties, namely, on how likely it is that the certificate fails to be valid for a new trajectory of our system. Our results complement the ones in \cite{DBLP:journals/sttt/BadingsCJJKT22} which are concerned with direct property verification and do not construct certificates. Our framework constitutes a first step towards control synthesis exploiting the constructed certificates.
    \item Our probabilistic guarantees are based on recent advancements in the so-called scenario approach theory, and are based on the notion of compression. This results in \emph{a posteriori} bounds which, however, scale favorably with respect to the system dimension. We offer statements that are different, in some sense complementary to \cite{DBLP:journals/tac/NejatiLJSZ23}, while overcoming the scalability challenges of the bounds therein. We contrast our approach with \cite{DBLP:journals/tac/NejatiLJSZ23} and discuss the relative merits of each, both theoretically (Section \ref{sec:related}) and numerically (Section \ref{sec:exp}).
    \item As a byproduct of our certificate construction algorithm, we provide a novel mechanism to compute the  
quantity termed \emph{compression}, which is instrumental in obtaining meaningful probabilistic guarantees. This process is novel per
se and provides a constructive approach for the general compression set calculation in \cite{DBLP:conf/nips/PaccagnanCG23}, opening the road for its use in general
non-convex optimization problems.
\end{enumerate}
The rest of the paper unfolds as follows: Section \ref{sec:certs} introduces the different certificates under consideration, while Section \ref{sec:learn_certs} provides our probabilistic certificate guarantees. Section \ref{sec:training} provides a data driven algorithm that enjoys such guarantees, while Section 
\ref{sec:related} compares our work with the most closely results in the literature. Section \ref{sec:exp} provides a numerical investigation, while Section \ref{sec:conc} concludes the manuscript and provides some directions for future work.
