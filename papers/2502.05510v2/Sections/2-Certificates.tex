%\section{Notation}
\\

\emph{Notation.}
We use $\{\xi_k\}_{k=0}^K$ to denote a sequence indexed by $k \in \{0,1,\dots,K \}$. $V \models \psi$ defines condition satisfaction i.e., it evaluates to true if the quantity $V$ on the left satisfies the condition $\psi$ on the right, e.g., $x = 1 \models x > 0$ evaluates to true and $x=-1 \models x > 0$ evaluates to false.  Using $\not\models$ represents the logical inverse of this (i.e., condition dissatisfaction). By $(\forall \xi \in \Xi) V\models\psi(\xi)$ we mean that some quantity $V$ satisfies a condition $\psi$ which, in turn, depends on some parameter $\xi$, for all $\xi \in \Xi$.

\section{Certificates}
\label{sec:certs}

We consider a family of certificates that allow us to make (probabilistic) statements on the behavior of a dynamical system, namely, how likely it is that it satisfies certain properties.
To this end, we begin by defining a dynamical system, before considering the various certificates and the properties they verify.

\subsection{Discrete-Time Dynamical Systems}

We consider a bounded state space $X \subseteq \mathbb{R}^n$, and a dynamical system whose evolution starts at an initial state $x(0) \in X_I$, where $X_I\subseteq X$ denotes the set of possible initial conditions. 
From an initial state, we can uncover a finite trajectory, i.e., a sequence of states $\xi = \{x(k)\}_{k=0}^T$, where $T\in \mathbb{N}_+$, by following the dynamics
\begin{equation}
\label{eq:Dyn}
    x(k+1)=f(x(k)).
\end{equation}
We define $f \colon X \rightarrow \mathbb{R}^n$, and assume it to permit unique solutions, but make no further assumptions on its properties.
The set of all possible trajectories $\Xi \subseteq X_I \times X^{T}$ is then the set of all trajectories starting from the initial set $X_I$.
Note that this set-up considers only deterministic systems, but our methods are applicable to systems with stochastic dynamics - we discuss this in further detail in Section \ref{sec:learn_certs:data}. 
We do not consider controller synthesis here, but recognize that our general form of dynamical system allows for verifying systems with controllers ``in the loop'': for instance, our techniques allow us to verify the behavior of a system with a predifined control law structure, such as Model Predictive Control~\cite{DBLP:journals/automatica/GarciaPM89}.

In Section~\ref{sec:learn_certs}, we discuss using a finite set of trajectories in order to provide generalization guarantees for future trajectories. 
Our techniques only require a finite number of samples, and are \emph{theoretically} not restricted on the properties of such samples (i.e. we may have a finite number of samples each with an infinitely long time horizon). 
However, we discuss in Section~\ref{sec:training} how one can synthesize a certificate, and our algorithms are required to store, and perform some calculations, on these trajectories (which is not \emph{practically} possible for $T$ taken to infinity, or continuous time trajectories).

We are interested in verifying whether the behavior of a dynamical system satisfies certain properties.
We use $\phi(\xi)$ to refer to a property of interest (defined concretely in the sequel), which is evaluated on a trajectory $\xi \in \Xi$. Specifically, we
    will define conditions $\psi^s$, that will have to be satisfied over some sub-domains of the state space, and $\psi^\Delta (\xi)$ that will define conditions that will have to be satisfied only at specific points along a trajectory $\xi \in \Xi$. The separate notations $\psi^s$ and $\psi^\Delta$ are used to distinguish between trajectory-independent and -dependent conditions, respectively.
    
In order to verify the satisfaction of a property $\phi$, we consider the problem of finding a \emph{certificate} as follows.

\begin{defn}[Property Verification \& Certificates]\label{prob:property_cert}
    Given a property $\phi(\xi)$, and a function $V\colon \mathbb{R}^n \rightarrow \mathbb{R}$, let $\psi^s$ and $\psi^\Delta (\xi)$ be conditions such that, if$$
            \exists V \colon V \models \psi^s\wedge (\forall \xi \in \Xi) V\models\psi^\Delta(\xi) \implies \phi(\xi), \forall \xi \in \Xi,$$
    then the property $\phi$ is verified for all $\xi \in \Xi$. We then say that such a function $V$ is a \emph{certificate} for the property encoded by $\phi$. 
\end{defn}
     
In words, the implication of Definition \ref{prob:property_cert} is that if a certificate $V$ satisfies the conditions in $\psi^s$, as well as the conditions in $\psi^\Delta (\xi)$, for all $\xi \in \Xi$, then the property $\phi(\xi)$ is satisfied for all trajectories $\xi \in \Xi$. 
\subsection{Certificates}

We now provide a concrete definition for a number of these properties, and the associated certificates (and certificate conditions) that meet the format of Definition \ref{prob:property_cert}. We fix a time horizon $T<\infty$. We assume that $V$ is continuous, so that when considering the supremum/infimum of $V$ over $X$ (already assumed to be bounded) or over some of its subsets, this is well-defined.

\begin{property}[Reachability]\label{prop:reach}
Consider \eqref{eq:Dyn}, and let
    $X_G, X_I \subset X$ denote a goal and initial set, respectively. Assume further that $X_G$ is compact and denote by $\partial X_G$ its boundary. If, for all $\xi \in \Xi$, 
    \begin{align}
        \phi_{\mathrm{reach}}(\xi) \defeq &\exists k \in \{0,\dots,T\} \colon x(k) \in X_G, \nonumber \\
        &\wedge~ \forall j \in \{0,\dots,k\} \colon x(j) \in X
    \end{align}
    holds, then we say that $\phi_{\mathrm{reach}}$ encodes a reachability property.
    $\Xi$ denotes the set of trajectories consistent with \eqref{eq:Dyn} and with initial states contained within $X_I$.
\end{property}
By the definition of $\phi_{\mathrm{reach}}$ it follows that verifying that a system exhibits the reachability property is equivalent to verifying that all trajectories generated from the initial set enter the goal within at most $T$ time steps, and stay within the domain $X$ till then.
In order to verify this property, we consider a certificate that must satisfy a number of conditions. 
These conditions are summarized next. Fix $\delta> -\inf_{x \in X_I} V(x) \geq 0$.  We then have
\begin{align}
	\label{eq:reach_init}	
  &V(x) \leq 0, \; \forall x \in X_I,\\
 \label{eq:reach_goal}
		&V(x) \geq -\delta, \; \forall x \in \partial X_G, \\
  \label{eq:reach_else}
		&V(x) > -\delta, \; \forall x \in X \setminus X_G,\\
        &V(x) > 0, \; \forall x \in \mathbb{R}^N \setminus X,
  \label{eq:reach_dom_border}\\
		&V(x(k+1)) - V(x(k)) \label{eq:reach_deriv} \\ 
        & <- \frac{1}{T} \Big( \sup_{x \in X_I} V(x) + \delta \Big),~ k=0,\dots,k_G-1, \nonumber %\\&\;\forall \{x(k)\}_{k=0,\dots,T} \colon x(0) \in X_I,
		% & \qquad \qquad \forall x(k) \in X \setminus X_G,
\end{align}
       where $k_G \defeq \min \{k \in \{0,\dots,T\} \colon V(x(k)) \leq -\delta\}$, or $k_G=T$, if there is no such $k$.
Conditions (\ref{eq:reach_goal})-(\ref{eq:reach_dom_border}) allow characterizing different parts of the state space by means of specific level sets of $V$. In particular, we require $V$ to be non-positive within the initial set $X_I$ (\ref{eq:reach_init}) and positive outside the domain \eqref{eq:reach_dom_border}, while $V$ should be no more negative than a pre-specified level $-\delta<0$ in the rest of the domain $X$  (\ref{eq:reach_else}), and the sublevel set $V$ less than $-\delta$ should be contained within the goal set $X_G$ \eqref{eq:reach_goal}. 

In the case that $T$ tends to infinity (i.e. an infinite time horizon), the difference condition in \eqref{eq:reach_deriv} is reduced to a negativity requirement, as is standard in the literature \cite{DBLP:conf/hybrid/EdwardsPA24}.
Due to our finite time horizon, we require conditions (\ref{eq:reach_goal})-(\ref{eq:reach_else}) to provide a bound on the value of our function which we must reach within the time horizon.

The condition in \eqref{eq:reach_deriv} is a decrease condition (its right-hand side is negative due to the choice of $\delta$), that implies $V$ is decreasing along system trajectories till the first time the goal set is reached (by the definition of the time instance $k_G$). 
To gain some intuition on \eqref{eq:reach_deriv}, see that if $k_G = T$, its recursive application leads to 
\begin{align}
V(x(T)) < V(x(0)) - T \frac{1}{T} \Big( \sup_{x \in X_I} V(x) + \delta \Big) \leq -\delta, \label{eq:proof_decr}
\end{align}
where the inequality holds since $V(x(0)) \leq \sup_{x \in X_I}V(x)$. Therefore, if the system starts within $X_I$, then it reaches the goal set (see \eqref{eq:reach_goal}) in at most $T$ steps.

A graphical representation of these conditions is provided in Figure~\ref{fig:spiral_reach_plane}.
The inner sublevel set (with dashed line) is the set obtained when the certificate value is less than $-\delta$, whilst the outer one is the set obtained when the certificate is less than $0$. 
The decrease condition then means that we never leave the larger sublevel set and must instead converge to the smaller sublevel set.

\begin{figure}
	\centering
	\includegraphics[width=.7\linewidth]{Figures/spiral_reach_plane}
	\caption{Pictorial illustration of the level sets associated with the reach certificate for the system in \eqref{eq:spiral_dyn}.}
	\label{fig:spiral_reach_plane}
\end{figure}

Now introduce  $\psi^s_{\mathrm{reach}}$ to encode conditions (\ref{eq:reach_init})-(\ref{eq:reach_dom_border}), while
$\psi^\Delta_{\mathrm{reach}}(\xi)$ captures \eqref{eq:reach_deriv}. Notice that the latter depends on $\xi$ as it is enforced on consecutive states $x(k)$ and $x(k+1)$ along a trajectory.

With this in place, we can now define our first certificate.

\begin{prop}[Reachability Certificate]
\label{cert:reach}
    A function $V \colon \mathbb{R}^n \rightarrow \mathbb{R}$ is a reachability certificate if
	 \begin{equation}
	     V \models \psi^s_{\mathrm{reach}} \wedge (\forall \xi \in \Xi) V\models\psi^\Delta_{\mathrm{reach}}(\xi).
	 \end{equation}
\end{prop}
The proof is based on \eqref{eq:proof_decr}; provided formally in Appendix~\ref{app:proofs}.
In words, Proposition~\ref{cert:reach} implies that a function $V$ is a reachability certificate if it satisfies (\ref{eq:reach_init})-(\ref{eq:reach_dom_border}), and (\ref{eq:reach_deriv}) for all trajectories generated by our dynamics.

We now consider a safety property, which is in some sense dual to reachability.

\begin{property}[Safety]\label{prop:safe}
   Consider \eqref{eq:Dyn}, and let $X_I, X_U \subset X$ with $X_I \cap X_U = \emptyset$ denote an initial and an unsafe set, respectively. If for all $\xi \in \Xi$,
    $$
        \phi_\mathrm{safe}(\xi) \defeq \forall k \in \{0,\dots,T\}, x(k) \notin X_U,
    $$
    holds, then we say that $\phi_\mathrm{safe}$ encodes a safety property. $\Xi$ denotes the set of trajectories consistent with \eqref{eq:Dyn} and with initial state contained within $X_I$.
\end{property}
By the definition of $\phi_\mathrm{safe}$, it follows that verifying a system exhibits the safety property is equivalent to verifying all trajectories emanating from the initial set avoid the unsafe set for all time instances, until horizon $T$.

We now define the relevant criteria necessary for a certificate to verify this property.
\begin{align}
	\label{eq:barr_states1}
    &V(x) \leq 0 , \forall x \in X_I,\\
	\label{eq:barr_states2}
    &V(x) > 0, \forall x \in X_U,\\
        &V(x(k+1))-V(x(k)) \label{eq:barr_deriv} \\ &<\frac{1}{T} \Big (\inf_{x \in X_U}V(x)-\sup_{x \in X_I}V(x) \Big),~ k=0,\dots,T-1. \nonumber
\end{align}
Notice that even if $\inf_{x \in X_U}V(x)-\sup_{x \in X_I}V(x) > 0$, i.e., in the case where the last condition encodes an increase of $V$ along the system trajectories, the system still avoids entering the unsafe set. In particular,
\begin{align}
V(x(T)) &< V(x(0)) + \Big (\inf_{x \in X_U}V(x)-\sup_{x \in X_I}V(x) \Big) \nonumber \\
&\leq \inf_{x \in X_U}V(x),
\end{align}
where the inequality is since $V(x(0)) \leq \sup_{x \in X_I} V(x)$.
Therefore, by \eqref{eq:barr_states2}, the resulting inequality implies that even if the system starts at the least negative state within $X_I$, it will still remain safe.

We denote by $\psi_\text{safe}^s$ the conjunction of \eqref{eq:barr_states1} and \eqref{eq:barr_states2}, and by $\psi^\Delta_\text{safe}(\xi)$ the property in \eqref{eq:barr_deriv}. We then have the following safety/barrier certificate

\begin{prop}[Safety/Barrier Certificate]
\label{cert:barr}
    A function $V \colon \mathbb{R}^n \rightarrow \mathbb{R}$ is a safety/barrier certificate if
	 \begin{equation}
	     V \models \psi^s_\mathrm{safe} \wedge (\forall \xi \in \Xi) V\models\psi^\Delta_\mathrm{safe}(\xi).
	 \end{equation}
\end{prop}
The proof can be found in Appendix \ref{app:proofs}.
Combining reachability and safety leads to richer properties. One of these is defined next. 

\begin{property}[Reach-While-Avoid (RWA)]\label{prop:rwa}
     Consi-\\der \eqref{eq:Dyn}, and let $X_I, X_U, X_G \subset X$ with $(X_I \cup X_G) \cap X_U = \emptyset$ denote an initial set, an unsafe set, and a goal set, respectively.
     Assume further that $X_G$ is compact and denote by $\partial X_G$ its boundary. 
     If for all $\xi \in \Xi$,
   \begin{equation*}
    \begin{aligned}
                \phi_\mathrm{RWA}(\xi) \defeq \forall k \in \{0,\dots,T\}, x(k) \notin X_U \cup X^c  \\ \wedge~\exists k \in \{0, \dots, T\}, x(k) \in X_G,
    \end{aligned}
    \end{equation*}
    holds, then we say that $\phi_\mathrm{RWA}$ encodes a RWA property.
    $\Xi$ denotes the set of trajectories consistent with \eqref{eq:Dyn} and with initial state contained within $X_I$.
\end{property}
By the definition of $\phi_\mathrm{RWA}$, it follows that verifying that a system exhibits the RWA property is equivalent to verifying that all trajectories emanating from the initial set $X_I$ avoid entering the unsafe set $X_U$ (and the set complement of the domain $X$), and also eventually enter the goal set $X_G$.

Fix $\delta>0$ such that $\delta > -\inf_{x \in X_I} V(x)$.  
The conditions necessary to verify this property are as follows: 
    \begin{align}
 \label{eq:RWA_init}
		&V(x) \leq 0, \; \forall x \in X_I,\\
  \label{eq:RWA_safe}
		&V(x) > 0, \; \forall x \in X_U,\\
  \label{eq:RWA_goal}
  &V(x) \geq -\delta, \; \forall x \in \partial X_G, \\
		&V(x) > -\delta, \; \forall x \in X \setminus X_G, 
  \label{eq:RWA_else}\\
		&V(x(k+1)) - V(x(k)) \label{eq:RWA_deriv} \\
        &<- \frac{1}{T}\left(\sup_{x \in X_I} V(x) +\delta \right),~ k=0,\dots,k_G-1,\nonumber\\
		&V(x(k+1)) - V(x(k)) \label{eq:RWA_deriv2} \\
        &< \frac{1}{T}\left(\inf_{x \in X_U} V(x) +\delta \right),~ k=k_G,\dots,T-1,\nonumber
    \end{align}
where recall that $k_G$ denotes the first time the system trajectory will ``hit'' the $(-\delta)$-level set of $V$, which is associated with the goal set. 
We use $\psi^s_\mathrm{RWA}$ to denote the conjunction of (\ref{eq:RWA_init})-(~\ref{eq:RWA_else}), and $\psi^\Delta_\mathrm{RWA}(\xi)$ for \eqref{eq:RWA_deriv} and \eqref{eq:RWA_deriv2}. 

These conditions ensure that our initial and unsafe sets (including outside the domain) are separated by a zero-level set of the function $V$, and that there is a minimum inside the goal set.
The difference conditions then ensure that we decrease from the initial set (and hence reach the goal set), and afterward do not increase so much that we enter the unsafe set.

\begin{prop}[RWA Certificate]
\label{cert:RWA}
    A function $V \colon \\\mathbb{R}^n \rightarrow \mathbb{R}$ is a RWA certificate if
	 \begin{equation}
	     V \models \psi^s_\mathrm{RWA} \wedge (\forall \xi \in \Xi) V\models\psi^\Delta_\mathrm{RWA}(\xi).
	 \end{equation}
\end{prop}
The proof can be found in Appendix \ref{app:proofs}.
We provide a graphical representation of the properties in Figure \ref{fig:props}.

\begin{figure}[b]
    \centering
    \begin{subfigure}{0.3\linewidth}
        \includegraphics[width=.75\linewidth]{Figures/reach}
        \caption{Reachability}
    \end{subfigure}\hfill
        \begin{subfigure}{0.3\linewidth}
        \includegraphics[width=.8\linewidth]{Figures/safe}
        \caption{Safety}
    \end{subfigure}\hfill   
    \begin{subfigure}{0.3\linewidth}
        \includegraphics[width=.8\linewidth]{Figures/RWA}
        \caption{RWA}
    \end{subfigure}
    \caption{Pictorial illustration of (a) reachability, (b) safety, and (c) RWA properties, respectively. Black lines illustrate sample trajectories that satisfy the associated properties.}
    \label{fig:props}
\end{figure}

To synthesize one of these deterministic certificates, we require complete knowledge of the behavior $f$ of the dynamical system, to allow us to reason about the space of trajectories $\Xi$.  
This may be impractical, and we therefore consider learning a certificate in a data-driven manner.
