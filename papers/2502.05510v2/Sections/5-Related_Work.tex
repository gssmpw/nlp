\section{Comparison with Related Work}
\label{sec:related}
\subsection{Direct Property Evaluation}
\label{sec:related:direct_prop}
As is known in the case of Lyapunov stability theory, the existence of a certificate is useful per se, and allows one to translate a property to a scalar function. 
As discussed in Corollary~\ref{corr:prop}, a byproduct of this certificate synthesis is that they provide guarantees on the probability of property violation (see \eqref{eq:prop_viol}). 
However, if one is not interested in the construction of a certificate and only in such guarantees, then Theorem 2 in \cite{DBLP:journals/sttt/BadingsCJJKT22} provides an alternative. 
We adapt this result in the proposition below, using the Langford binomial tail bound~\cite{JMLR:v6:langford05a} to obtain a tighter guarantee than the sampling-and-discarding result~\cite{DBLP:journals/jota/CampiG11} used in \cite{DBLP:journals/sttt/BadingsCJJKT22}.
\begin{prop}[Theorem $2$ in \cite{DBLP:journals/sttt/BadingsCJJKT22}]
\label{corr:a_post}
Fix $\beta \in (0,1)$, and for $r = 0,\ldots,N-1$,
determine $\varepsilon(r,\beta,N)$ such that 
    \begin{equation}   
    \label{eq:eps_orig}
     \sum_{k=0}^r\binom{N}{k} \varepsilon^k(1-\varepsilon)^{N-k}=\beta,
    \end{equation}
  while for $r=N$ let $\varepsilon(N,\beta,N) = 1$.   
	Denote by $R_N$ the number of samples in $\{\xi^i\}_{i =1}^{N}$ for which $\phi(\xi^i)$ is violated.
    We then have that
    \begin{align}
	    \mathbb{P}^N &\big \{ \{\xi^i\}_{i=1}^N \in \Xi^N:~\nonumber \\
        &\mathbb{P}\{\xi \in \Xi \colon \neg\phi(\xi) \} \leq \varepsilon(R_N,\beta,N)\big\}
        \geq 1-\beta. \label{eq:prop_direct}
    \end{align}
\end{prop}
% This involves a direct application of the Langford binomial tail bound~\cite{JMLR:v6:langford05a}. % (offering an improvement on the sampling-and-discarding results in \cite{DBLP:journals/jota/CampiG11} which require $\frac{\beta}{N}$ on the right-hand side of \eqref{eq:prop_direct}). 
It is \emph{a posteriori} as $R_N$ can be determined only once the samples are observed. 
% To this end, the term $\beta /N$ appears in the right-hand side of \eqref{eq:prop_direct} to account for the fact that, depending on the samples, up to $N$ terms could appear in the summation.
In this setting, we have a compression set which is the set of all discarded samples, plus an additional one to support the solution after discarding.
Since this additional sample is always present, we incorporate it in the formula in \eqref{eq:eps_orig}. 

We remark that one could obtain  different bounds through alternative statistical techniques, such as Hoeffding's inequality~\cite{Hoeffding01031963} or Chernoff's bound~\cite{10.1214/aoms/1177729330}. 
Since these bounds are of different nature, we do not pursue that avenue further here.

%For property violation Hoeffding bounds expected value (hence violation rate) which is the same as what scenario approach bound does because we have a bernoulli distribution. 
% Hence no reason we couldn't use Hoeffding, maybe we could also apply it to the result from certificate synthesis? Then the below arguments hold.

% However, such techniques have the drawback of only probabilistically bounding a distance from the mean, rather than providing a bound on the probability of drawing a violating sample.}
% \textcolor{red}{[next statement unclear:]}
% Further, one requires bounds on the loss $L(\theta, \xi)$ and hence on the certificate value and Lie derivative, which may not be available.}

% \textcolor{red}{[pls rephrase, very clear:]}

Proposition \ref{corr:a_post} offers an alternative to Theorem \ref{thm:Guarantees} to directly bound property violation. 
We compare the risk levels $\varepsilon$ computed by each approach on our benchmark example in \eqref{eq:spiral_dyn} under a safety specification; general conclusions are case dependent, as both bounds are \emph{a posteriori}.
For a fixed $\beta$, Figure~\ref{fig:bounds} shows the resulting risk levels for varying $N$ across 5 independently sampled sets of trajectories.
The difference of the orange curve from the blue one can be interpreted as the price of certificate generation of Theorem \ref{thm:Guarantees}. For sufficiently large $N$, this price is marginal.  
As the specification is deterministically safe, no discarding is performed for Proposition~\ref{corr:a_post}, hence a smooth curve without variability. 
For non-zero $R_N$ we expect variability as $R_N$ will be randomly distributed. 
\begin{figure}
    \centering
    \includegraphics[width=0.75\linewidth]{Figures/risk_comparison_curves}
    \caption{Comparison of the bounds in Theorem~\ref{thm:Guarantees} and Proposition~\ref{corr:a_post} for direct property evaluation. Median values are shown with a cross, and ranges are indicated by error bars.}
    \label{fig:bounds}
\end{figure}

%Theorem 5.3
\subsection{Certificate Synthesis as in \cite{DBLP:journals/tac/NejatiLJSZ23}}
The results in \cite{DBLP:journals/tac/NejatiLJSZ23} constitute the most closely related ones with respect to our work. 
As no results on reachability and RWA problems were provided in \cite{DBLP:journals/tac/NejatiLJSZ23}, we limit our discussion to the safety property. 
As with our work, a sample-based construction is performed, where samples therein are pairs (state, next-state), as opposed to trajectories as in our work. However, the probabilistic bounds established in \cite{DBLP:journals/tac/NejatiLJSZ23} are structurally different and of complementary nature to our work: next, we review the main result in \cite{DBLP:journals/tac/NejatiLJSZ23}, adapted to our notation.

\begin{thm}[Theorem $5.3$ in~\cite{DBLP:journals/tac/NejatiLJSZ23}]
    Consider \eqref{eq:Dyn}, with initial and unsafe sets $X_I,X_U\subset X \subset \mathbb{R}^n,$ respectively.
    Consider also $N$ samples $\{x_i, f(x_i)\}_{i=1}^N$ from $X$, and assume that the loss function in \eqref{eq:opt_prob} is Lipschitz continuous with constant $\mathcal{L}$.
    Consider then the problem
        \begin{align}
            \eta^\star_N &\in \argmin_{d = (\gamma,\lambda,c,\theta),\eta \in \mathbb{R}}\eta \nonumber \\
            \text{st. }&\; V_\theta(x)-\gamma \leq \eta, \; \forall x \in X_I \nonumber \\
            &\; V_\theta(x)-\lambda \geq -\eta, \; \forall x \in X_U \nonumber \\
            &\; \gamma + cT - \lambda - \mu \leq \eta, ~ c \geq 0, \nonumber \\
            &\; V_\theta(f(x_i)) - V_\theta(x_i) -c \leq \eta, \; i=1,\dots,N,  
        \end{align}
        where $\theta$ parameterizes $V_\theta$, and all other decision variables are scalars leading to level sets of $V_\theta$. Let $\kappa(\delta)$ be such that
    \begin{equation}
	    \label{eq:ball}
        \kappa(\delta) \leq \mathbb{P}\{\mathbb{B}_\delta(x)\}, \forall \delta \in \mathbb{R}_{\geq 0}, \forall x \in X,
    \end{equation}
    where $\mathbb{B}_\delta(x) \subset X$ is a ball of radius $\delta$, centered at $x$.
    Fix $\beta \in (0,1)$ and
    determine $\epsilon(|d|,\beta,N)$ from \eqref{eq:eps_orig}, with $r = d$ and by replacing the right hand-side with $\beta$.
    If $\eta^\star_N \leq \mathcal{L} \kappa^{-1}(\epsilon(|d|,\beta,N)$, we have that
    \begin{equation}
	    \mathbb{P}^N\big \{ \{\xi^i\}_{i=1}^N \in \Xi^N:~\phi_{\mathrm{safe}}(\xi), \; \forall \xi \in \Xi  \big\} \geq 1-\beta.
    \end{equation}
\end{thm}

The following remarks are in order.
\begin{enumerate}[wide, labelwidth=!, labelindent=0pt]
	\item The result in \cite{DBLP:journals/tac/NejatiLJSZ23} is \emph{a priori} (capitalizing on the developments of \cite{6832537}), as opposed to the \emph{a posteriori} assessments of our analysis that are in turn based on \cite{DBLP:journals/jmlr/CampiG23}.
    Moreover, \cite{DBLP:journals/tac/NejatiLJSZ23} offers a
    guarantee that, with a certain confidence, the safety property is \emph{always} satisfied. This is in contrast to Theorem \ref{thm:Guarantees} where we provide such guarantees in probability (up to a quantifiable risk level $\varepsilon$).
However, these ``always'' guarantees come with potential challenges. In particular, the constraint in \eqref{eq:ball} involves the measure of a ``ball'' in the uncertainty space. 
The measure of this ball grows exponentially in the dimension of the uncertainty space (see also Remark 3.9 in 
    \cite{6832537}), while it depends linearly on the dimension of the decision space $|d|$ (see dependence of $\varepsilon$ below \eqref{eq:ball}). This dependence in the results of \cite{DBLP:journals/tac/NejatiLJSZ23} raises computational challenges to obtain useful bounds: we demonstrate this numerically in Section~\ref{sec:exp} employing  one of the examples considered in \cite{DBLP:journals/tac/NejatiLJSZ23}. On the contrary, Theorem \ref{thm:Guarantees} is independent of the dimension of these spaces and only depends on the cardinality of the compression set. 
	\item The result in \cite{DBLP:journals/tac/NejatiLJSZ23} requires inverting $\kappa(\delta)$, which may not have an analytical form in general. Moreover, it implicitly assumes some knowledge of the distribution to obtain $\kappa$, and of the Lipschitz constants of the system dynamics, which we do not require in our analysis. 
\item The results of \cite{DBLP:journals/tac/NejatiLJSZ23} are also extended to continuous-time dynamical systems. 
This is also possible for our results; however, due to practical considerations, we then require knowledge of the Lipschitz constants not required in the discrete time setting. 
This discussion is not pursued further here. 
\end{enumerate}