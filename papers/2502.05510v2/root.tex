\documentclass[twocolumn]{autart}    % Enable this line and disable the 

\newif\ifappendix
\global\appendixtrue

%%%%%%%%%%%%%%%%%%%%%%%%%%%%%%%%%%%%%%%%%%%%%%%%%%%%%%%%%%%%%%%%%%%%%%%%%%%%%%

%% Beautiful mathematics
\usepackage{amsmath, amssymb, amsfonts} 
\usepackage{nicefrac}
\usepackage{mathtools}
\usepackage{bm, bbm}
\usepackage[scr=boondoxo,scrscaled=1.05]{mathalfa}

%% References in the correct format 
%\usepackage[square,numbers]{natbib}
%\def\bibfont{\footnotesize} % fix to have the same font size as without natbib

\usepackage[sort, compress, space]{cite}            


%% Enumerate nicely 
\usepackage{enumitem}

%% Different color comments and commenting large parts of the text
\usepackage{xcolor}
\usepackage{comment}
\usepackage{soul}

%% Hyper references
\usepackage{hyperref}
\usepackage{cleveref}
%\usepackage[numbers]{natbib}

\usepackage{tikz}
%\usepackage{thm-restate}
%% Appendix package
%\usepackage{appendix}

%% Random text to test spacing 
\usepackage{blindtext}

\usepackage{afterpage}

\usepackage{algorithm, algorithmic}    



\usepackage{dsfont}

\usepackage{tikz}
\usepackage{graphicx}
\usepackage{tikzscale}
\usepackage{pgfplots}
\pgfplotsset{compat=newest}
\usepackage{xfrac}

\usepackage{thm-restate}

%\usepackage{subcaption}

\usepackage{balance}

\usepackage{cite}
\usepackage{amsmath,amssymb,amsfonts}
\usepackage{balance}
\usepackage{algorithmic}
\usepackage{graphicx}
\usepackage{textcomp}
\usepackage{xcolor}
\usepackage{amsmath}
\usepackage{amssymb}
\usepackage[mathscr]{euscript}
\usepackage{comment}
\usepackage{xcolor}
\usepackage{enumitem} 
\usepackage{amsthm}



\begin{document}

\begin{frontmatter}

\title{Data-Driven Neural Certificate Synthesis}

\author[Oxford]{Luke Rickard}\ead{rickard@robots.ox.ac.uk}
\author[OxfordCS]{Alessandro Abate}
\author[Oxford]{Kostas Margellos}
\address[Oxford]{Department of Engineering Science, University of Oxford}
\address[OxfordCS]{Department of Computer Science, University of Oxford}

\begin{keyword}
    Verification of Dynamical Systems, Safety, Reachability, Statistical Learning, Scenario Approach.
\end{keyword}

\begin{abstract}
    We investigate the problem of verifying different properties of discrete time dynamical systems, namely, reachability, safety and reach-while-avoid. To achieve this, we adopt a data driven perspective and using past systems' trajectories as data, we aim at learning a specific function termed \emph{certificate} for each property we wish to verify. The certificate construction problem is treated as a safety informed neural network training process, where we use a neural network to learn the parameterization of each certificate, while the loss function we seek to minimize is designed to encompass conditions on the certificate to be learned that encode the satisfaction of the associated property. Besides learning a certificate, we quantify probabilistically its generalization properties, namely, how likely it is for a certificate to be valid (and hence for the associated property to be satisfied) when it comes to a new system trajectory not included in the training data set.
    We view this problem under the realm of probably approximately correct (PAC) learning under the notion of compression, and use recent advancements of the so-called scenario approach to obtain scalable generalization bounds on the learned certificates. To achieve this, we design a novel algorithm that minimizes the loss function and hence constructs a certificate, and at the same time determines a quantity termed compression, which is instrumental in obtaining meaningful probabilistic guarantees. This process is novel per se and provides a constructive mechanism for compression set calculation, thus opening the road for its use to more general non-convex optimization problems. We verify the efficacy of our methodology on several numerical case studies, and compare it (both theoretically and numerically) with closely related results on data-driven property verification.
\end{abstract}

\end{frontmatter}

\section{Introduction}

Chain-of-Thought (CoT) prompting~\cite{Nye:2021, cot, Kojima:2022cotzero} has emerged as a cornerstone strategy for enhancing Large Language Models (LLMs) in complex reasoning tasks. By eliciting step-by-step inference, CoT enables LLMs to decompose intricate problems into manageable subtasks, thereby improving their problem-solving performance~\cite{Yao:2023tot, Wang:2023self-consistency, Zhou:2023least, Shinn:2023Reflexion}. Recent advancements, such as OpenAI's o1~\cite{o1} and DeepSeek-R1~\cite{deepseekr1}, further demonstrate that scaling up CoT lengths from hundreds to thousands of reasoning steps could continuously improve LLM reasoning. These breakthroughs have underscored CoT’s potential to advance LLM capabilities, expanding the boundaries of AI-driven problem-solving.

\begin{figure}[t]
\centering
    \includegraphics[width=0.95\columnwidth]{fig/intro.pdf}
    \caption{In contrast to vanilla CoT that generates all reasoning tokens sequentially, \method enables LLMs to \textit{skip} tokens with less semantic importance (\textit{e.g.,} \includegraphics[width=7pt]{fig/token.pdf}~) and learn shortcuts between critical reasoning tokens, facilitating controllable CoT compression.}
    \label{fig:intro}
\end{figure}

Despite its effectiveness, the increased length of CoT sequences introduces substantial computational overhead. Due to the autoregressive nature of LLM decoding, longer CoT outputs lead to proportional increases in both inference latency and memory footprints of key-value cache. Additionally, the quadratic computational cost of attention layers further exacerbates this burden. These issues become particularly pronounced when CoT sequences extend into thousands of reasoning steps, resulting in significant computational costs and prolonged response times. While prior research has explored methods for selectively skipping reasoning steps~\cite{Ding:2024cotshortcut, liu2024skipstep}, recent findings~\cite{jin:2024cotlength, Merrill:2024cotlength} suggest that such reductions may conflict with test-time scaling~\cite{o1-blog, snell2025scaling}, ultimately impairing LLM reasoning performance. Therefore, striking an optimal balance between CoT efficiency and reasoning accuracy remains a critical open challenge.

In this work, we delve into CoT efficiency and seek the answer to an important question: \textit{``Does every token in the CoT output contribute equally to deriving the answer?''} We empirically analyze the semantic importance of tokens within CoT outputs and reveal that their contributions to the reasoning performance vary, as depicted in Figure 2. Building on this insight, we introduce \method, a simple yet effective approach that enables LLMs to \textit{skip} less important tokens within CoT sequences and learn shortcuts between critical reasoning tokens, thereby allowing for controllable CoT compression with adjustable ratios. Specifically, as shown in Figure~\ref{fig:intro}, \method constructs compressed CoT training data with various compression ratios, by pruning unimportance tokens from original LLM CoT trajectories. Then, it conducts a general supervised fine-tuning process on target LLMs with this training data, facilitating LLMs to automatically trim redundant tokens during reasoning.

We conduct extensive experiments across various models, including LLaMA-3.1-8B-Instruct and the Qwen2.5-Instruct series, using two widely recognized math reasoning benchmarks: GSM8K and MATH-500. The results validate the effectiveness of \method in compressing CoT outputs while maintaining robust reasoning performance. Notably, Qwen2.5-14B-Instruct exhibits almost \textbf{NO} performance drop (less than $0.4\%$) with a $\bm{40\%}$ reduction in token usage on GSM8K. On the challenging MATH-500 dataset, LLaMA-3.1-8B-Instruct effectively reduces CoT token usage by $\bm{30}\%$ with a performance decline of less than $4\%$, resulting in a $\bm{1.4}\times$ inference speedup. Further analysis underscores the coherence of \method in specified compression ratios and its potential scalability with stronger compression techniques.

\method is distinguished by its low training cost. For Qwen2.5-14B-Instruct, \method fine-tunes only 0.2\% of the model's parameters using LoRA. The size of the compressed CoT training data is no larger than that of the original training set, with 7,473 examples in GSM8K and 7,500 in MATH. The training is completed in approximately 2 hours for the 7B model and 2.5 hours for the 14B model on two 3090 GPUs. These characteristics make \method an efficient and reproducible approach, suitable for use in efficient and cost-effective LLM deployment.

To sum up, our key contributions are:
\begin{enumerate}
    \item To the best of our knowledge, this work is the \textit{first} to investigate the potential of enhancing CoT efficiency through \textit{token skipping}, inspired by the varying semantic importance of tokens in CoT trajectories of LLMs.
    \item We introduce \method, a simple yet effective approach that enables LLMs to skip redundant tokens within CoTs and learn shortcuts between critical tokens, facilitating CoT compression with adjustable ratios.
    \item Our experiments validate the effectiveness of \method. When applied to Qwen2.5-14B-Instruct, \method reduces reasoning tokens by $40\%$ (from 313 to 181) on GSM8K, with less than a $0.4\%$ performance drop.
\end{enumerate}

%\section{Notation}
\\

\emph{Notation.}
We use $\{\xi_k\}_{k=0}^K$ to denote a sequence indexed by $k \in \{0,1,\dots,K \}$. $V \models \psi$ defines condition satisfaction i.e., it evaluates to true if the quantity $V$ on the left satisfies the condition $\psi$ on the right, e.g., $x = 1 \models x > 0$ evaluates to true and $x=-1 \models x > 0$ evaluates to false.  Using $\not\models$ represents the logical inverse of this (i.e., condition dissatisfaction). By $(\forall \xi \in \Xi) V\models\psi(\xi)$ we mean that some quantity $V$ satisfies a condition $\psi$ which, in turn, depends on some parameter $\xi$, for all $\xi \in \Xi$.

\section{Certificates}
\label{sec:certs}

We consider a family of certificates that allow us to make (probabilistic) statements on the behavior of a dynamical system, namely, how likely it is that it satisfies certain properties.
To this end, we begin by defining a dynamical system, before considering the various certificates and the properties they verify.

\subsection{Discrete-Time Dynamical Systems}

We consider a bounded state space $X \subseteq \mathbb{R}^n$, and a dynamical system whose evolution starts at an initial state $x(0) \in X_I$, where $X_I\subseteq X$ denotes the set of possible initial conditions. 
From an initial state, we can uncover a finite trajectory, i.e., a sequence of states $\xi = \{x(k)\}_{k=0}^T$, where $T\in \mathbb{N}_+$, by following the dynamics
\begin{equation}
\label{eq:Dyn}
    x(k+1)=f(x(k)).
\end{equation}
We define $f \colon X \rightarrow \mathbb{R}^n$, and assume it to permit unique solutions, but make no further assumptions on its properties.
The set of all possible trajectories $\Xi \subseteq X_I \times X^{T}$ is then the set of all trajectories starting from the initial set $X_I$.
Note that this set-up considers only deterministic systems, but our methods are applicable to systems with stochastic dynamics - we discuss this in further detail in Section \ref{sec:learn_certs:data}. 
We do not consider controller synthesis here, but recognize that our general form of dynamical system allows for verifying systems with controllers ``in the loop'': for instance, our techniques allow us to verify the behavior of a system with a predifined control law structure, such as Model Predictive Control~\cite{DBLP:journals/automatica/GarciaPM89}.

In Section~\ref{sec:learn_certs}, we discuss using a finite set of trajectories in order to provide generalization guarantees for future trajectories. 
Our techniques only require a finite number of samples, and are \emph{theoretically} not restricted on the properties of such samples (i.e. we may have a finite number of samples each with an infinitely long time horizon). 
However, we discuss in Section~\ref{sec:training} how one can synthesize a certificate, and our algorithms are required to store, and perform some calculations, on these trajectories (which is not \emph{practically} possible for $T$ taken to infinity, or continuous time trajectories).

We are interested in verifying whether the behavior of a dynamical system satisfies certain properties.
We use $\phi(\xi)$ to refer to a property of interest (defined concretely in the sequel), which is evaluated on a trajectory $\xi \in \Xi$. Specifically, we
    will define conditions $\psi^s$, that will have to be satisfied over some sub-domains of the state space, and $\psi^\Delta (\xi)$ that will define conditions that will have to be satisfied only at specific points along a trajectory $\xi \in \Xi$. The separate notations $\psi^s$ and $\psi^\Delta$ are used to distinguish between trajectory-independent and -dependent conditions, respectively.
    
In order to verify the satisfaction of a property $\phi$, we consider the problem of finding a \emph{certificate} as follows.

\begin{defn}[Property Verification \& Certificates]\label{prob:property_cert}
    Given a property $\phi(\xi)$, and a function $V\colon \mathbb{R}^n \rightarrow \mathbb{R}$, let $\psi^s$ and $\psi^\Delta (\xi)$ be conditions such that, if$$
            \exists V \colon V \models \psi^s\wedge (\forall \xi \in \Xi) V\models\psi^\Delta(\xi) \implies \phi(\xi), \forall \xi \in \Xi,$$
    then the property $\phi$ is verified for all $\xi \in \Xi$. We then say that such a function $V$ is a \emph{certificate} for the property encoded by $\phi$. 
\end{defn}
     
In words, the implication of Definition \ref{prob:property_cert} is that if a certificate $V$ satisfies the conditions in $\psi^s$, as well as the conditions in $\psi^\Delta (\xi)$, for all $\xi \in \Xi$, then the property $\phi(\xi)$ is satisfied for all trajectories $\xi \in \Xi$. 
\subsection{Certificates}

We now provide a concrete definition for a number of these properties, and the associated certificates (and certificate conditions) that meet the format of Definition \ref{prob:property_cert}. We fix a time horizon $T<\infty$. We assume that $V$ is continuous, so that when considering the supremum/infimum of $V$ over $X$ (already assumed to be bounded) or over some of its subsets, this is well-defined.

\begin{property}[Reachability]\label{prop:reach}
Consider \eqref{eq:Dyn}, and let
    $X_G, X_I \subset X$ denote a goal and initial set, respectively. Assume further that $X_G$ is compact and denote by $\partial X_G$ its boundary. If, for all $\xi \in \Xi$, 
    \begin{align}
        \phi_{\mathrm{reach}}(\xi) \defeq &\exists k \in \{0,\dots,T\} \colon x(k) \in X_G, \nonumber \\
        &\wedge~ \forall j \in \{0,\dots,k\} \colon x(j) \in X
    \end{align}
    holds, then we say that $\phi_{\mathrm{reach}}$ encodes a reachability property.
    $\Xi$ denotes the set of trajectories consistent with \eqref{eq:Dyn} and with initial states contained within $X_I$.
\end{property}
By the definition of $\phi_{\mathrm{reach}}$ it follows that verifying that a system exhibits the reachability property is equivalent to verifying that all trajectories generated from the initial set enter the goal within at most $T$ time steps, and stay within the domain $X$ till then.
In order to verify this property, we consider a certificate that must satisfy a number of conditions. 
These conditions are summarized next. Fix $\delta> -\inf_{x \in X_I} V(x) \geq 0$.  We then have
\begin{align}
	\label{eq:reach_init}	
  &V(x) \leq 0, \; \forall x \in X_I,\\
 \label{eq:reach_goal}
		&V(x) \geq -\delta, \; \forall x \in \partial X_G, \\
  \label{eq:reach_else}
		&V(x) > -\delta, \; \forall x \in X \setminus X_G,\\
        &V(x) > 0, \; \forall x \in \mathbb{R}^N \setminus X,
  \label{eq:reach_dom_border}\\
		&V(x(k+1)) - V(x(k)) \label{eq:reach_deriv} \\ 
        & <- \frac{1}{T} \Big( \sup_{x \in X_I} V(x) + \delta \Big),~ k=0,\dots,k_G-1, \nonumber %\\&\;\forall \{x(k)\}_{k=0,\dots,T} \colon x(0) \in X_I,
		% & \qquad \qquad \forall x(k) \in X \setminus X_G,
\end{align}
       where $k_G \defeq \min \{k \in \{0,\dots,T\} \colon V(x(k)) \leq -\delta\}$, or $k_G=T$, if there is no such $k$.
Conditions (\ref{eq:reach_goal})-(\ref{eq:reach_dom_border}) allow characterizing different parts of the state space by means of specific level sets of $V$. In particular, we require $V$ to be non-positive within the initial set $X_I$ (\ref{eq:reach_init}) and positive outside the domain \eqref{eq:reach_dom_border}, while $V$ should be no more negative than a pre-specified level $-\delta<0$ in the rest of the domain $X$  (\ref{eq:reach_else}), and the sublevel set $V$ less than $-\delta$ should be contained within the goal set $X_G$ \eqref{eq:reach_goal}. 

In the case that $T$ tends to infinity (i.e. an infinite time horizon), the difference condition in \eqref{eq:reach_deriv} is reduced to a negativity requirement, as is standard in the literature \cite{DBLP:conf/hybrid/EdwardsPA24}.
Due to our finite time horizon, we require conditions (\ref{eq:reach_goal})-(\ref{eq:reach_else}) to provide a bound on the value of our function which we must reach within the time horizon.

The condition in \eqref{eq:reach_deriv} is a decrease condition (its right-hand side is negative due to the choice of $\delta$), that implies $V$ is decreasing along system trajectories till the first time the goal set is reached (by the definition of the time instance $k_G$). 
To gain some intuition on \eqref{eq:reach_deriv}, see that if $k_G = T$, its recursive application leads to 
\begin{align}
V(x(T)) < V(x(0)) - T \frac{1}{T} \Big( \sup_{x \in X_I} V(x) + \delta \Big) \leq -\delta, \label{eq:proof_decr}
\end{align}
where the inequality holds since $V(x(0)) \leq \sup_{x \in X_I}V(x)$. Therefore, if the system starts within $X_I$, then it reaches the goal set (see \eqref{eq:reach_goal}) in at most $T$ steps.

A graphical representation of these conditions is provided in Figure~\ref{fig:spiral_reach_plane}.
The inner sublevel set (with dashed line) is the set obtained when the certificate value is less than $-\delta$, whilst the outer one is the set obtained when the certificate is less than $0$. 
The decrease condition then means that we never leave the larger sublevel set and must instead converge to the smaller sublevel set.

\begin{figure}
	\centering
	\includegraphics[width=.7\linewidth]{Figures/spiral_reach_plane}
	\caption{Pictorial illustration of the level sets associated with the reach certificate for the system in \eqref{eq:spiral_dyn}.}
	\label{fig:spiral_reach_plane}
\end{figure}

Now introduce  $\psi^s_{\mathrm{reach}}$ to encode conditions (\ref{eq:reach_init})-(\ref{eq:reach_dom_border}), while
$\psi^\Delta_{\mathrm{reach}}(\xi)$ captures \eqref{eq:reach_deriv}. Notice that the latter depends on $\xi$ as it is enforced on consecutive states $x(k)$ and $x(k+1)$ along a trajectory.

With this in place, we can now define our first certificate.

\begin{prop}[Reachability Certificate]
\label{cert:reach}
    A function $V \colon \mathbb{R}^n \rightarrow \mathbb{R}$ is a reachability certificate if
	 \begin{equation}
	     V \models \psi^s_{\mathrm{reach}} \wedge (\forall \xi \in \Xi) V\models\psi^\Delta_{\mathrm{reach}}(\xi).
	 \end{equation}
\end{prop}
The proof is based on \eqref{eq:proof_decr}; provided formally in Appendix~\ref{app:proofs}.
In words, Proposition~\ref{cert:reach} implies that a function $V$ is a reachability certificate if it satisfies (\ref{eq:reach_init})-(\ref{eq:reach_dom_border}), and (\ref{eq:reach_deriv}) for all trajectories generated by our dynamics.

We now consider a safety property, which is in some sense dual to reachability.

\begin{property}[Safety]\label{prop:safe}
   Consider \eqref{eq:Dyn}, and let $X_I, X_U \subset X$ with $X_I \cap X_U = \emptyset$ denote an initial and an unsafe set, respectively. If for all $\xi \in \Xi$,
    $$
        \phi_\mathrm{safe}(\xi) \defeq \forall k \in \{0,\dots,T\}, x(k) \notin X_U,
    $$
    holds, then we say that $\phi_\mathrm{safe}$ encodes a safety property. $\Xi$ denotes the set of trajectories consistent with \eqref{eq:Dyn} and with initial state contained within $X_I$.
\end{property}
By the definition of $\phi_\mathrm{safe}$, it follows that verifying a system exhibits the safety property is equivalent to verifying all trajectories emanating from the initial set avoid the unsafe set for all time instances, until horizon $T$.

We now define the relevant criteria necessary for a certificate to verify this property.
\begin{align}
	\label{eq:barr_states1}
    &V(x) \leq 0 , \forall x \in X_I,\\
	\label{eq:barr_states2}
    &V(x) > 0, \forall x \in X_U,\\
        &V(x(k+1))-V(x(k)) \label{eq:barr_deriv} \\ &<\frac{1}{T} \Big (\inf_{x \in X_U}V(x)-\sup_{x \in X_I}V(x) \Big),~ k=0,\dots,T-1. \nonumber
\end{align}
Notice that even if $\inf_{x \in X_U}V(x)-\sup_{x \in X_I}V(x) > 0$, i.e., in the case where the last condition encodes an increase of $V$ along the system trajectories, the system still avoids entering the unsafe set. In particular,
\begin{align}
V(x(T)) &< V(x(0)) + \Big (\inf_{x \in X_U}V(x)-\sup_{x \in X_I}V(x) \Big) \nonumber \\
&\leq \inf_{x \in X_U}V(x),
\end{align}
where the inequality is since $V(x(0)) \leq \sup_{x \in X_I} V(x)$.
Therefore, by \eqref{eq:barr_states2}, the resulting inequality implies that even if the system starts at the least negative state within $X_I$, it will still remain safe.

We denote by $\psi_\text{safe}^s$ the conjunction of \eqref{eq:barr_states1} and \eqref{eq:barr_states2}, and by $\psi^\Delta_\text{safe}(\xi)$ the property in \eqref{eq:barr_deriv}. We then have the following safety/barrier certificate

\begin{prop}[Safety/Barrier Certificate]
\label{cert:barr}
    A function $V \colon \mathbb{R}^n \rightarrow \mathbb{R}$ is a safety/barrier certificate if
	 \begin{equation}
	     V \models \psi^s_\mathrm{safe} \wedge (\forall \xi \in \Xi) V\models\psi^\Delta_\mathrm{safe}(\xi).
	 \end{equation}
\end{prop}
The proof can be found in Appendix \ref{app:proofs}.
Combining reachability and safety leads to richer properties. One of these is defined next. 

\begin{property}[Reach-While-Avoid (RWA)]\label{prop:rwa}
     Consi-\\der \eqref{eq:Dyn}, and let $X_I, X_U, X_G \subset X$ with $(X_I \cup X_G) \cap X_U = \emptyset$ denote an initial set, an unsafe set, and a goal set, respectively.
     Assume further that $X_G$ is compact and denote by $\partial X_G$ its boundary. 
     If for all $\xi \in \Xi$,
   \begin{equation*}
    \begin{aligned}
                \phi_\mathrm{RWA}(\xi) \defeq \forall k \in \{0,\dots,T\}, x(k) \notin X_U \cup X^c  \\ \wedge~\exists k \in \{0, \dots, T\}, x(k) \in X_G,
    \end{aligned}
    \end{equation*}
    holds, then we say that $\phi_\mathrm{RWA}$ encodes a RWA property.
    $\Xi$ denotes the set of trajectories consistent with \eqref{eq:Dyn} and with initial state contained within $X_I$.
\end{property}
By the definition of $\phi_\mathrm{RWA}$, it follows that verifying that a system exhibits the RWA property is equivalent to verifying that all trajectories emanating from the initial set $X_I$ avoid entering the unsafe set $X_U$ (and the set complement of the domain $X$), and also eventually enter the goal set $X_G$.

Fix $\delta>0$ such that $\delta > -\inf_{x \in X_I} V(x)$.  
The conditions necessary to verify this property are as follows: 
    \begin{align}
 \label{eq:RWA_init}
		&V(x) \leq 0, \; \forall x \in X_I,\\
  \label{eq:RWA_safe}
		&V(x) > 0, \; \forall x \in X_U,\\
  \label{eq:RWA_goal}
  &V(x) \geq -\delta, \; \forall x \in \partial X_G, \\
		&V(x) > -\delta, \; \forall x \in X \setminus X_G, 
  \label{eq:RWA_else}\\
		&V(x(k+1)) - V(x(k)) \label{eq:RWA_deriv} \\
        &<- \frac{1}{T}\left(\sup_{x \in X_I} V(x) +\delta \right),~ k=0,\dots,k_G-1,\nonumber\\
		&V(x(k+1)) - V(x(k)) \label{eq:RWA_deriv2} \\
        &< \frac{1}{T}\left(\inf_{x \in X_U} V(x) +\delta \right),~ k=k_G,\dots,T-1,\nonumber
    \end{align}
where recall that $k_G$ denotes the first time the system trajectory will ``hit'' the $(-\delta)$-level set of $V$, which is associated with the goal set. 
We use $\psi^s_\mathrm{RWA}$ to denote the conjunction of (\ref{eq:RWA_init})-(~\ref{eq:RWA_else}), and $\psi^\Delta_\mathrm{RWA}(\xi)$ for \eqref{eq:RWA_deriv} and \eqref{eq:RWA_deriv2}. 

These conditions ensure that our initial and unsafe sets (including outside the domain) are separated by a zero-level set of the function $V$, and that there is a minimum inside the goal set.
The difference conditions then ensure that we decrease from the initial set (and hence reach the goal set), and afterward do not increase so much that we enter the unsafe set.

\begin{prop}[RWA Certificate]
\label{cert:RWA}
    A function $V \colon \\\mathbb{R}^n \rightarrow \mathbb{R}$ is a RWA certificate if
	 \begin{equation}
	     V \models \psi^s_\mathrm{RWA} \wedge (\forall \xi \in \Xi) V\models\psi^\Delta_\mathrm{RWA}(\xi).
	 \end{equation}
\end{prop}
The proof can be found in Appendix \ref{app:proofs}.
We provide a graphical representation of the properties in Figure \ref{fig:props}.

\begin{figure}[b]
    \centering
    \begin{subfigure}{0.3\linewidth}
        \includegraphics[width=.75\linewidth]{Figures/reach}
        \caption{Reachability}
    \end{subfigure}\hfill
        \begin{subfigure}{0.3\linewidth}
        \includegraphics[width=.8\linewidth]{Figures/safe}
        \caption{Safety}
    \end{subfigure}\hfill   
    \begin{subfigure}{0.3\linewidth}
        \includegraphics[width=.8\linewidth]{Figures/RWA}
        \caption{RWA}
    \end{subfigure}
    \caption{Pictorial illustration of (a) reachability, (b) safety, and (c) RWA properties, respectively. Black lines illustrate sample trajectories that satisfy the associated properties.}
    \label{fig:props}
\end{figure}

To synthesize one of these deterministic certificates, we require complete knowledge of the behavior $f$ of the dynamical system, to allow us to reason about the space of trajectories $\Xi$.  
This may be impractical, and we therefore consider learning a certificate in a data-driven manner.

\section{Data-Driven Certificates}
\label{sec:learn_certs}

In some cases, one may have access to the underlying dynamics $f$, in which case it is possible to directly find a certificate that satisfies the relevant criteria $\psi^s$, $\psi^{\Delta}$.
However, in many real-world scenarios, access to the true dynamics requires a complete model of the physics of the system, and may not always be possible.
Instead, we take a data-driven approach to synthesize a certificate based only on available trajectories/signatures of that system.

We denote by $(X_I,\mathcal{F},\mathbb{P})$ a probability space, where $\mathcal{F}$ is a $\sigma$-algebra and $\mathbb{P}\colon \mathcal{F}\rightarrow[0,1]$ is a probability measure on our set of initial states $X_I$.
Then, the initial state of our system is randomly distributed according to $\mathbb{P}$.

To obtain our sample set, we consider $N$ initial conditions, according to probability distribution $\mathbb{P}$, namely 
$
    \{x^i(0)\}_{i=1}^N \sim \mathbb{P}^N,$
where we assume that all samples are independent and identically distributed (i.i.d.).
Initializing the dynamics from each of these initial states, we unravel a set of trajectories $\{\xi^i\}_{i=1}^N$.  %$\{ \{x^i(k)\}_{k=0}^T\}_{i=1}^N$. 
Since there is no stochasticity in the dynamics, we can equivalently say that trajectories (generated from the random initial conditions) are distributed according to the same probabilistic law; hence, with a slight abuse of notation, we write $\xi\sim \mathbb{P}$.
In the case of a stochastic dynamical system, the vector field would depend on some additional disturbance vector; our subsequent analysis will remain valid with $\mathbb{P}$ being replaced by the probability distribution that captures both the randomness of the initial state and the distribution of the disturbance. 
 We impose the following assumption.
\begin{assum}[Non-concentrated Mass]\label{ass:non-conc_mass}
	Assume that $\mathbb{P}\{\xi \}=0$, for any $\xi \in \Xi$.
\end{assum}

\subsection{Problem Statement}
\label{sec:learn_certs:data}

Since we are now dealing with a sample-based problem, we will be constructing probabilistic certificates and hence probabilistic guarantees on the satisfaction of a given property. We will present our results for a generic property $\phi \in \{\phi_{\mathrm{reach}}, \phi_{\mathrm{safe}}, \phi_{\mathrm{RWA}}\}$ and associated certificate conditions $\psi^s, \psi^{\Delta}$. 

Denote by $V_N$ a certificate of property $\phi$, we introduce the subscript $N$ to emphasize that this certificate is constructed on the basis of sampled trajectories $\{\xi^i\}_{i=1}^N$.

\begin{prob}[Probabilistic Property Guarantee]\label{prob:guarantees}
   Consider $N$ sampled trajectories, and fix a confidence level $\beta \in (0,1)$. We seek a property violation level $\epsilon \in (0,1)$ such that 
    \begin{align}
       \mathbb{P}^N &\big\{ \{\xi^i\}_{i=1}^N \in \Xi^N:~ \nonumber \\
       &\mathbb{P}\{\xi \in \Xi \colon V_N \not\models \psi^s \wedge \psi^\Delta(\xi)\} \leq \epsilon \big \} \geq 1-\beta. \label{eq:prop_prob}
    \end{align}
\end{prob}

Addressing this problem allows us to provide guarantees even if part of the initial set does not satisfy our specification.
Our statement is in the realm of probably approximately correct (PAC) learning: the probability of sampling a new trajectory $\xi \sim \mathbb{P}$ failing to satisfy our certificate condition is itself a random quantity depending on the samples $\{\xi^i\}_{i=1}^N$, and encompasses the generalization properties of a learned certificate $V_N$. It is thus distributed according to the joint probability measure $\mathbb{P}^N$, hence our results hold with some confidence $(1-\beta)$.

Providing a solution to Problem \ref{prob:guarantees} is equivalent to determining an $\epsilon \in (0,1)$, such that with confidence at least $1-\beta$, the probability that $V_N$ does not satisfy the condition $\psi^s \wedge \psi^\Delta(\xi)$ for another sampled trajectory $\xi \in \Xi$ is at most equal to that $\epsilon$. As such, with a certain confidence, a certificate $V_N$ \emph{trained} on the basis of $N$ sampled trajectories, will remain a valid certificate with probability at least $1-\epsilon$. Therefore, we can argue that  $V_N$ is a \emph{probabilistic} certificate.

\subsection{Probabilistic Guarantees}

We now provide a solution to Problem \ref{prob:guarantees}. 
To this end, we refer to a mapping $\mathcal{A}$ such that $V_N = \mathcal{A}(\{\xi^i\}_{i=1}^{N})$ as an algorithm that, based on $N$ samples, returns a certificate $V_N$. Our main result will apply to a generic algorithm as long this exhibits certain properties that will be outlined as assumptions below. In Section \ref{sec:training} we provide a specific synthesis procedure through which $\mathcal{A}$ (and hence the certificate $V_N$) can be constructed, and show that this algorithm satisfies the considered properties.

The following definition constitutes the backbone of our analysis. The notion introduced below appears with different terms in the literature; we adopt the terminology introduced in \cite{DBLP:journals/jmlr/CampiG23,DBLP:journals/tac/MargellosPL15} (adapted to our purposes) to align with the statistical learning literature. 



\begin{defn}[Compression Set]\label{def:compress}
Fix any $\{\xi^i\}_{i=1}^{N}$, and let $\mathcal{C}_N \subseteq \{\xi^i\}_{i=1}^{N}$ be a subset of the samples with cardinality $C_N = |\mathcal{C}_N| \leq N$.
Define $V_{C_N} = \mathcal{A}(\mathcal{C}_N)$.
We say that $\mathcal{C}_N$ is a compression of $\{\xi^i\}_{i=1}^{N}$ for algorithm $\mathcal{A}$, if
    \begin{equation}
    \begin{aligned}
		V_{C_N} = \mathcal{A}(\mathcal{C}_N) = \mathcal{A}(\{\xi^i\}_{i=1}^{N}) = V_N.
    \end{aligned}
    \end{equation}
\end{defn}
Notice the slight abuse of notation, as the argument of $\mathcal{A}$ might be a set of different cardinality; in the following, its domain will always be clear from the context. 

Figure \ref{fig:Compression} illustrates Definition \ref{def:compress} pictorially. 
It should be noted that compression set cardinalities may be bounded \emph{a priori}~\cite{DBLP:journals/siamjo/CampiG08}, that is, without knowledge of the sample-set, or obtained \emph{a posteriori}, and hence depending on the given set $\{\xi^i\}_{i=1}^{N}$~\cite{DBLP:journals/mp/CampiG18}. 
Trivially, we could take a compression set as the entire sample set, resulting in a trivial risk upper bound of 1. 
    However, it is of benefit to determine a compression set with small (ideally minimal) cardinality, as the smaller $C_N$ is, the smaller risk we can guarantee. 
In this paper we are particularly interested in a posteriori results, since we solve a non-convex problem we cannot in general provide a non-trivial bound to the cardinality of the compression set a priori~\cite{DBLP:journals/tac/CampiGR18}.
Therefore, we introduce the subscript $N$ in our notation for $\mathcal{C}_N$ (set) and $C_N$ (corr. cardinality), respectively.

\begin{figure}
    \centering
    \begin{tikzpicture}[node distance=0.35cm ]
        \node (start) {$\xi^1$};
        \node (blank1) [below of=start] {\phantom{$\xi^N$}};
        \node (next) [below of= blank1] {$\vdots$\phantom{$\xi^i$}};
        \node (blank2) [below  of= next] {\phantom{$\xi^N$}};
        \node (final) [below of =blank2] {$\xi^N$};
        
        \node (alg1) [startstop, right=0.7cm of next] {$\mathcal{A}$};
        
        \node (out1) [right=0.7cm of alg1] {$V_N$};
        
        \node (equals) [right = 0.1cm of out1] { $=$};

        \node (out2) [right = 0.1cm of equals] {$V_{C_N}$};

        \node (alg2) [startstop, right=0.7cm of out2] {$\mathcal{A}$};
        
        \node (next2) [right=0.7cm of alg2] {$\mathcal{C}_N$\phantom{$\xi^i$}};
        \node (blank3) [above of=next2] {\phantom{$\xi^N$}};

        \node (blank4) [below of=next2] {\phantom{$\xi^N$}};

        \draw [arrow] (alg1) -- (out1);
        \draw [arrow] (alg2) -- (out2);
        
        \draw [arrow] (start.east) -- (alg1.west|-start.east);
        \draw [arrow] (blank1.east) -- (alg1.west|-blank1.east);
        \draw [arrow] (next.east) -- (alg1.west|-next.east);
        \draw [arrow] (blank2.east) -- (alg1.west|-blank2.east);
        \draw [arrow] (final.east) -- (alg1.west|-final.east);
        \draw [arrow] (next2.west) -- (alg2.east|-next2.west);
        
        \end{tikzpicture}
    \caption{Pictorial illustration of the compression set notion of Definition \ref{def:compress}.}
    \label{fig:Compression}
\end{figure}

The theorem below provides probabilistic guarantees that are valid irrespective of the cardinality of the underlying compression set. 
However, the quality of these bounds depends significantly on this cardinality, resulting in progressively sharper probabilistic bounds as the compression set cardinality decreases. 
In Algorithm \ref{algo:main} we propose a mechanism to obtain non-trivial compression sets, while avoiding computationally expensive procedures. This can be thought of as a by-product of the certificate synthesis procedure of Section \ref{sec:training}, and constitutes per se a novel  contribution of this work. 

For the guarantees we introduce in the sequel to hold, that the algorithm $\mathcal{A}$ must satisfy the following properties (adapted from \cite{DBLP:journals/jmlr/CampiG23}; note that in \cite{DBLP:journals/jmlr/CampiG23} a non-concentrated mass property is also imposed, which here appears separately as Assumption \ref{ass:non-conc_mass}). 
We later demonstrate that our proposed algorithm satisfies these.

\begin{assum}[Properties of $\mathcal{A}$]\label{ass:alg_prop}
Assume that algorithm $\mathcal{A}$ exhibits the following properties:
\begin{enumerate}[wide, labelwidth=!, labelindent=0pt]
	\item \emph{Preference:} For any pair of multisets $\mathcal{C}_1$ and $\mathcal{C}_2$ of elements of $\{\xi^i\}_{i=1}^N$, with $\mathcal{C}_1 \subseteq \mathcal{C}_2$, if  $\mathcal{C}_1$ does not constitute a compression set of $\mathcal{C}_2$ for algorithm $\mathcal{A}$, then $\mathcal{C}_1$ will not constitute a compression set of $\mathcal{C}_2 \cup\{\xi\}$ for any $\xi \in \Xi$.
	\item \emph{Non-associativity:} Let $\{\xi^i\}_{i=1}^{N+\bar{N}}$ for some $\bar{N} \geq 1$. If $\mathcal{C}$ constitutes a compression set of $\{\xi_i\}_{i=1}^{N} \cup \{\xi\}$ for all $\xi \in \{\xi^i\}_{i=N+1}^{N+\bar{N}}$ for algorithm $\mathcal{A}$, then $\mathcal{C}$ constitutes a compression set of $\{\xi_i\}_{i=1}^{N+\bar{N}}$ (up to a measure-zero set).
\end{enumerate}
\end{assum}

We are now able to state the main result of this section.

\begin{thm}[Probabilistic Guarantees]
\label{thm:Guarantees}
Consider any algorithm $\mathcal{A}$ satisfying Assumption \ref{ass:alg_prop} such that $V_N = \mathcal{A}(\{\xi^i\}_{i=1}^{N})$, with trajectories $\{\xi^i\}_{i=1}^{N}$ generated in an i.i.d. manner from a distribution satisfying Assumption~\ref{ass:non-conc_mass}. 
Fix $\beta \in (0,1)$, and for $k<N$, let
let $\varepsilon(k,\beta,N)$ be the (unique) solution to the polynomial equation in the interval $[k/N,1]$
    \begin{align}
               \frac{\beta}{2N} \sum_{m=k}^{N-1}&\frac{\binom{m}{k}}{\binom{N}{k}}(1-\varepsilon)^{m-N} \nonumber \\
               &+\frac{\beta}{6N}\sum_{m=N+1}^{4N}\frac{\binom{m}{k}}{\binom{N}{k}}(1-\varepsilon)^{m-N} = 1,
     \end{align}
    while for $k=N$ let $\varepsilon(N,\beta,N) =1$. We then have that
    \begin{align}
	    \label{eq:cert_bound}
        &\mathbb{P}^N\big\{ \{\xi^i\}_{i=1}^N \in \Xi^N:~  \\
        &\mathbb{P}\{\xi \in \Xi\colon V_N \not\models \psi^s \wedge \psi^\Delta(\xi)) \} \leq \varepsilon(C_N,\beta,N)\big\} \nonumber \geq 1-\beta.
    \end{align}
\end{thm}
\textbf{Proof}
Fix $\beta \in (0,1)$, and for each $\{\xi^i\}_{i=1}^N$ let $\mathcal{C}_N$ be a compression set for algorithm $\mathcal{A}$. Moreover, note
that letting $V_N = \mathcal{A}(\{\xi^i\}_{i=1}^{N})$ we construct a mapping from samples $\{\xi\}_{i=1}^N$ to a decision, namely, $V_N$, while we impose as an assumption that this mapping satisfies the conditions of Assumption \ref{ass:alg_prop}. 

This framework directly fits into the setting of \cite[Theorem~7]{DBLP:journals/jmlr/CampiG23}, which implies that with confidence at least $1-\beta$, the probability that for a new $\xi \in \Xi$ the compression set changes, is at most $\epsilon(\mathcal{C}_N,\beta,N)$, i.e., 
\begin{align}
\mathbb{P}\{\xi \in \Xi\colon  \mathcal{C}_N^{+} \neq \mathcal{C}_N\} \leq \varepsilon(C_N,\beta,N), \label{eq:proof_thm}
\end{align}
where $\mathcal{C}_N^{+}$ denotes a compression set for algorithm $\mathcal{A}$ when fed with $\{\xi\}_{i=1}^N \cup \{\xi\}$.
However, we then have that
\begin{align}
\{\xi \in \Xi &\colon V_N \not\models \psi^s \wedge \psi^\Delta(\xi)) \} \nonumber \\
&\subseteq \{\xi \in \Xi \colon V_N \neq \mathcal{A}(\{\xi\}_{i=1}^N \cup \{\xi\}) \} \nonumber \\
&= \{\xi \in \Xi \colon \mathcal{A}(\mathcal{C}_N) \neq \mathcal{A}(\mathcal{C}_N^{+})\} \nonumber \\
&\subseteq \{\xi \in \Xi \colon \mathcal{C}_N^{+} \neq \mathcal{C}_N \}, \label{eq:proof_thm1}
\end{align}
where the first inclusion is since for any $\xi \in \Xi$ for which $V_N$ no longer satisfies the certificate condition 
$(\psi^s \wedge \psi^\Delta(\xi))$, we must have that the certificate changes, i.e., $\mathcal{A}(\{\xi^i\}_{i=1}^N \cup \{\xi\})$ (the output/certificate of our algorithm when fed with one more sample) is different from $V_N$. 
Notice that the opposite statement does not always hold, having a different certificate does not necessarily mean the old one violates an existing condition for a new $\xi \in \Xi$. 

The following equality holds as $V_N = \mathcal{A}(\mathcal{C}_N)$, $\mathcal{A}(\{\xi\}_{i=1}^N \cup \{\xi\}) = A(\mathcal{C}_N^{+})$, by definition of a compression set.

Finally, the last inclusion stands since any $\xi \in \Xi$ for which $\mathcal{A}(\mathcal{C}_N) \neq \mathcal{A}(\mathcal{C}_N^{+})$, should be such that $\mathcal{C}_N^{+} \neq \mathcal{C}_N$. 
The opposite direction does not always hold, as if 
$\mathcal{C}_N^{+} \supset \mathcal{C}_N$ then we get another compression set of higher cardinality, and hence we may still have $\mathcal{A}(\mathcal{C}_N) = \mathcal{A}(\mathcal{C}_N^{+})$. By \eqref{eq:proof_thm} and \eqref{eq:proof_thm1}, \eqref{eq:cert_bound} follows, concluding the proof.
 \qed

By the implication in Definition \ref{prob:property_cert}, and under the same conditions as those in Theorem~\ref{thm:Guarantees}, we have the following corollary, which relates correctness of the certificate to bounds on the property under study. 
\begin{cor}[Bounds on Property Satisfaction]\label{corr:prop}
Suppose \eqref{eq:cert_bound} is satisfied for the calculated risk $\epsilon(C_N,\beta,N)$, we can guarantee that 
\begin{equation}
    \begin{aligned}
	\mathbb{P}^N&\big\{\{\xi^i\}_{i=1}^N\in \Xi^N\colon\\&\mathbb{P}\{\xi \in \Xi \colon \neg \phi(\xi)\} \leq \epsilon(C_N,\beta,N)\} \geq 1-\beta. \label{eq:prop_viol}
\end{aligned}
\end{equation}
\end{cor}
That is, if there exists a certificate $V_N$ that satisfies \eqref{eq:cert_bound} up to a level $\epsilon$, then the property $\phi(\xi)$ is satisfied up to the same level (with some confidence). This confidence refers to the selection of sampled trajectories we observed.

The following remarks are in order.
\begin{enumerate}[wide, labelwidth=!, labelindent=0pt]
	\item Notice that Theorem \eqref{thm:Guarantees} involves evaluating $\varepsilon(k,\beta,N)$ at $k=C_N$, i.e., at the cardinality of the compression set. As such, with confidence at least $1-\beta$ with respect to the choice of the trajectories $\{\xi_i\}_{i=1}^N$, the probability that $V_N$ is not a valid certificate when it comes to a new trajectory $\xi$, is at most $\varepsilon(C_N,\beta,N)$. Due to the dependency of $\varepsilon$ on the samples (via $C_N$), the proposed  probabilistic bound is \emph{a posteriori} as it is adapted to the samples we ``see''. As a result, this is often less conservative compared to \emph{a priori} counterparts.
    \item  
    For cases where algorithm $\mathcal{A}$ takes the form of an optimization program that is convex with respect to the parameter vector, determining non-trivial bounds on the cardinality of compression sets is possible \cite{DBLP:journals/siamjo/CampiG08,DBLP:journals/tac/MargellosPL15}, as this is related to the notion of support constraints in convex analysis.
    However, determining compression sets of low cardinality (necessary for small risk bounds) becomes a non-trivial task if $\mathcal{A}$ involves a non-convex optimization program and/or is iterative (as Algorithm \ref{algo:sub}). 
    In such cases, samples that give rise to inactive constraints may still belong to a compression set, as they may affect the optimal parameter implicitly. 
    A direct way of determining the minimal compression set is to resolve the problem with every subset of the samples \cite{DBLP:journals/mp/CampiG18,DBLP:journals/tac/CampiGR18}. 
    Computationally, however, this would be an intense procedure, often prohibitive due to its combinatorial nature \cite{DBLP:journals/tac/FeleM21}. 
\item An alternative procedure could be to use sampled trajectories and check directly whether a property is satisfied for them (by checking the property definition, rather than using the associated certificate's conditions). 
	This is a valid alternative but has the drawback of not providing a certificate $V_N$, but simply provides an answer as far as the property satisfaction is concerned.
    This direction is pursued in \cite{DBLP:journals/sttt/BadingsCJJKT22}; we review this result and compare with our approach in Section \ref{sec:related:direct_prop}. Note that
	having a certificate is interesting per se, and opens the road for control synthesis; which we aim to pursue in future work. 
\end{enumerate}
\section{Certificate Synthesis}
\label{sec:training}

In this section, we propose a mechanism to synthesize a certificate from sampled trajectories, thus offering a constructive approach for algorithm $\mathcal{A}$ in Theorem \ref{thm:Guarantees}.

% \subsection{Neural Networks}
In order to learn a certificate from samples, we consider a neural network, a well-studied class of function approximators that generalize well to a given task.

% \begin{defn}[Neural Network]
%     We denote a neural network by an input layer $z_0 \in \mathbb{R}^n$ (same dimension with the system state vector), a number of hidden layers $z_1 \in \mathbb{R}^{h_1}, \dots, z_k \in \mathbb{R}^{h_k}$, and an output layer $z_{k+1} \in \mathbb{R}$.
%     Each layer, except the input, has an associated set of weights and biases $W_i \in \mathbb{R}^{h_{i-1}\times h_i}, b_i \in \mathbb{R}^{h_i}$, as well as an activation function $\sigma_i \colon \mathbb{R} \rightarrow \mathbb{R}$.

%     The layers are related by the following equations,
%     \begin{align}
%         z_{i} &= \sigma_i(W_i z_{i-1}+b_i),~ i =1,\dots,k,\\
%         z_{k+1} &= W_{k+1} z_k+b_{k+1},
%     \end{align}
%     where the activation function is applied element-wise to its argument. 
% \end{defn}

Such a neural network acts as a ``template'' for our certificate $V_N$. 
Denote all tunable neural network parameters by a vector $\theta$. 
We then have that our certificate $V_N$ depends on $\theta$. 
For the results of this section, we simply write $V_\theta$ and drop the dependency on $N$ to ease notation. 

\newcounter{figure_store}
\setcounter{figure_store}{\arabic{figure}}
\setcounter{figure}{0}

\makeatletter
\renewcommand{\fnum@figure}{\textbf{Algorithm \thefigure}}
\makeatother
\begin{figure*}
\caption{Certificate Synthesis and Compression Set Computation}
\vspace{0.2cm}
\label{algo:sub}
\hrule \vspace{0.05cm} \hrule 
    \begin{algorithmic}[1]
    \Function{}{}$\mathcal{A}(\theta,\mathcal{D})$
    \State Set $k \gets 1$ \Comment{Initialize iteration index}
            \State Set $\mathcal{C}\leftarrow \emptyset$ 
            \Comment{Initialize compression set}
            \State Fix $L_1 < L_0$ with $|L_1-L_0|>\eta$ \Comment{$\eta$ is any fixed tolerance}
            \While{$l^s(\theta)>0$} \Comment{While sample-independent state loss is non-zero}
                \State  $g \gets \nabla_\theta l^s(\theta)$  \Comment{Gradient of loss function} \label{line:state_grad}
                \State $\theta \leftarrow \theta-\alpha g$ \Comment{Step in the direction of sample-independent gradient} \label{line:state_step}
            \EndWhile
            \Statex \vspace{-0.35cm}\hrulefill
             \While{$|L_k-L_{k-1}|>\eta $} \Comment{Iterate until tolerance is met} \label{line:while}
            \State $\mathcal{M} \gets \{\tilde{\xi} \in \mathcal{D} \colon L(\theta,\tilde{\xi}) \geq \max_{\tilde{\xi} \in \mathcal{C}} L(\theta,\tilde{\xi}) \}$ \Comment{Find samples with loss greater than compression set loss} \label{line:max_D}
            \State  $\overline{g}_\mathcal{M} \gets \{\nabla_\theta L(\theta,\tilde{\xi})\}_{\tilde{\xi} \in \mathcal{M}}$  \Comment{ Subgradients of loss function for $\tilde{\xi} \in  \mathcal{M}$} \label{line:subgrad_D}
            \State $\overline{\xi}_{\mathcal{C}} \in \argmax_{\tilde{\xi} \in \mathcal{C}} L(\theta,\tilde{\xi})$ \Comment{Find a sample with maximum loss from $\mathcal{C}$} \label{line:max_C}
            \State $\overline{g}_{\mathcal{C}}  \gets \nabla_\theta L(\theta,\overline{\xi}_{\mathcal{C}})$ \Comment{Approximate subgradient of loss function for $\tilde{\xi} = \overline{\xi}_{\mathcal{C}}$} \label{line:subgrad_C}
            \Statex \vspace{-0.25cm}\hrulefill
             \If{$\exists \overline{g} \in \overline{g}_\mathcal{M} \colon \langle \overline{g},\overline{g}_{\mathcal{C}} \rangle \leq 0 \wedge \overline{g} \neq 0$ \label{line:inner}} \Comment{If there is a misaligned subgradient (take the maximum if multiple)}
                \State $\theta \leftarrow \theta-\alpha\overline{g}$ \Comment{Step in the direction of misaligned subgradient} \label{line:true_step}
                \State $\mathcal{C} \leftarrow \mathcal{C} \cup \{\overline{\xi}\}$ \Comment{Update compression set with sample corresponding to $\overline{g}$} \label{line:update_C}
            \Else
                \State $\theta \leftarrow \theta-\alpha\overline{g}_{\mathcal{C}}$ \Comment{Step in the direction of approximate subgradient} \label{line:approx_step}
            \EndIf \label{line:endif}
	    % \While{$|L_k-L_{k-1}|>\eta $} \Comment{Iterate until tolerance is met} \label{line:while}

     %        \State $\overline{\xi}_\mathcal{D} \in \argmax_{\xi \in \mathcal{D}} L(\theta,\xi)$\Comment{Find a sample with maximum loss from $\mathcal{D}$} \label{line:max_D}
     %        \State  $\overline{g}_\mathcal{D} \gets \nabla_\theta L(\theta,\overline{\xi}_\mathcal{D})$  \Comment{ Subgradient of loss function for $\xi = \overline{\xi}_D$} \label{line:subgrad_D}
     %        \State $\overline{\xi}_{\mathcal{C}} \in \argmax_{\xi \in \mathcal{C}} L(\theta,\xi)$ \Comment{Find a sample with maximum loss from $\mathcal{C}$} \label{line:max_C}
     %        \State $\overline{g}_{\mathcal{C}}  \gets \nabla_\theta L(\theta,\overline{\xi}_{\mathcal{C}})$ \Comment{Approximate subgradient of loss function for $\xi = \overline{\xi}_{\mathcal{C}}$} \label{line:subgrad_C}
     %        \Statex \vspace{-0.25cm}\hrulefill
     %         \If{$\langle \overline{g}_D,\overline{g}_{\mathcal{C}} \rangle \leq 0$} \label{line:inner}
     %         \If{$\overline{g}_D \neq 0$}
     %            \State $\theta \leftarrow \theta-\alpha\overline{g}_\mathcal{D}$ \Comment{Step in the direction of exact subgradient} \label{line:true_step}
     %            \State $\mathcal{C} \leftarrow \mathcal{C} \cup \{\overline{\xi}_D\}$ \Comment{Update compression set with $\xi = \overline{\xi}_D$} \label{line:update_C}
     %            \EndIf
     %        \Else
     %            \State $\theta \leftarrow \theta-\alpha\overline{g}_{\mathcal{C}}$ \Comment{Step in the direction of approximate subgradient} \label{line:approx_step}
     %        \EndIf \label{line:endif}
            \Statex \vspace{-0.35cm}\hrulefill
	    \State $L_k \gets \min\left\{ L_{k-1}, \max_{\xi \in \mathcal{D}} L(\theta,\xi)\right\}$ \Comment{Update ``running'' loss value} \label{line:update_L}
        \State $k \gets k+1$ \Comment{Update iteration index} \label{line:update_k}
            \EndWhile
            \State \Return $\theta, \mathcal{C}$
            \EndFunction
    \end{algorithmic}
               \vspace{0.1cm} \hrule \vspace{0.05cm} \hrule
\end{figure*}
\makeatletter
\renewcommand{\fnum@figure}{Fig. \thefigure}
\makeatother

\setcounter{figure}{\arabic{figure_store}}
\setcounter{algorithm}{1}


\subsection{Certificate and Compression Set Computation}

We provide an algorithm that seeks to optimize the neural network parameters so that it results in a certificate $V_{\theta^\star}$. 
To this end, for a $\xi \in \Xi$ and parameter vector $\theta$, let \begin{equation}
    L(\theta,\xi)=l^\Delta(\theta,\xi)+ l^s(\theta),\label{eq:opt_prob}
\end{equation} represent an associated loss function consisting of a sample-dependent loss $l^\Delta$, and a sample-independent loss $l^s$. 
Without loss of generality, we assume that we can drive the sample-independent loss to be zero (see further discussions later). 
We impose the next mild assumption, needed to prove termination of our algorithm.
\begin{assum}[Minimizers' Existence] \label{ass:exist}
For any $\{\xi\}_{i=1}^N$, and any non-empty $\mathcal{D} \subseteq \{\xi\}_{i=1}^N$, the set of minimizers of $\max_{\xi \in \mathcal{D}} L(\theta,\xi),$ is non-empty.
\end{assum}
We aim at approximating a minimizer $\theta^\star$ of $\max_{\xi \in \mathcal{D}} L(\theta,\xi)$ when $\mathcal{D}=\{\xi\}_{i=1}^N$, which exists due to Assumption \ref{ass:exist}. 
We can then use that minimizer to construct $V_{\theta^\star}$. 
To achieve this, we employ Algorithm~\ref{algo:sub}. 

We provide a graphical representation of the algorithm in Figure~\ref{fig:algorithm}, the loss evaluated only on the support samples is shown as a blue dot, whilst the true loss is shown with a green cross.

Algorithm \ref{algo:sub} takes as inputs some initial (arbitrary) parameter vector $\theta$ and a set of samples $\mathcal{D} \subseteq \{\xi\}_{i=1}^N$. 
First, in steps~\ref{line:state_grad}--\ref{line:state_step}, we optimize for the sample-independent loss until this loss is non-positive, which serves as a form of warm starting.
In step~\ref{line:max_D}, the maximizing samples $\mathcal{M}$ that achieve loss greater than the loss on the compression set is identified, while in step~\ref{line:subgrad_D} the subgradients of the maximizing samples $\nabla_\theta L(\theta,\tilde{\xi}), \tilde{\xi} \in \mathcal{M}$ are computed.  
Steps~\ref{line:max_C}--\ref{line:subgrad_C} perform similar computations but with the set $\mathcal{C}$ maintained throughout the algorithm, in place of $\mathcal{D}$. 
As such, the subgradient in step~\ref{line:subgrad_C} is termed approximate, as the samples in $\mathcal{C}$ may not achieve the worst-case loss value. 
It is to be understood that if $\mathcal{C}$ is empty (as per initialization) steps~\ref{line:max_C}-\ref{line:subgrad_C} are not performed.

Steps~\ref{line:inner}--\ref{line:endif} of Algorithm \ref{algo:sub} involve taking a descent step.  
If the inner product in step~\ref{line:inner} is non-positive (i.e., if the approximate subgradient ``steers'' against a maximizing subgradient) and the maximizing subgradient is non-zero, then we proceed to step~\ref{line:true_step} and follow the maximizing subgradient (with stepsize $\alpha$) to explore the new direction (this can be thought of as an exploration step, and is seen at label $5$ in the Figure); otherwise, we move to step~\ref{line:approx_step} and follow the direction of the approximate subgradient (seen at the ``blue'' dot labelled by $3$) which in this case would point towards a similar direction with the exact one. This logic prevents us from unnecessarily appending to $\mathcal{C}$ more samples.
If the maximizing subgradient is followed, we add the associated sample $\overline{\xi}$ to the set $\mathcal{C}$ (step~\ref{line:update_C}). 
We then iterate till the loss value meets a given tolerance $\eta$ (see steps~\ref{line:while} and~\ref{line:update_L}).

\begin{figure}[b]
    \centering
    \includegraphics[width=0.75\linewidth]{Figures/Algorithm.eps}
    \caption{Graphical Representation of Algorithm~\ref{algo:sub}.}
    \label{fig:algorithm}
\end{figure}

We view Algorithm \ref{algo:sub} as a specific choice for the mapping $\mathcal{A}$ introduced in Section \ref{sec:learn_certs} when fed with $\mathcal{D} = \{\xi_i\}_{i=1}^N$, and some initial choice for $\theta$. 
It terminates returning an updated $\theta$, and a set $\mathcal{C}$ which forms a compression set for this algorithm. 
These are formalized below.

\begin{prop}[Algorithm \ref{algo:sub} Properties] \label{prop:converge}
Consider Assumption \ref{ass:non-conc_mass}, Assumption  \ref{ass:exist} and Algorithm \ref{algo:sub} with $\mathcal{D} = \{\xi_i\}_{i=1}^N$ and a fixed (sample independent) initialization for the parameter $\theta$. We then have:
\begin{enumerate}[wide, labelwidth=!, labelindent=0pt]
\item Algorithm \ref{algo:sub} terminates, returning a parameter vector $\theta^\star$ and a set $\mathcal{C}_N$.
\item The set $\mathcal{C}_N$ with cardinality $C_N = |\mathcal{C}_N|$ forms a compression set for Algorithm \ref{algo:sub}.
\item Algorithm \ref{algo:sub} satisfies Assumption \ref{ass:alg_prop}.
\end{enumerate}
\end{prop}

The proof can be found in Appendix \ref{app:proofs}.

Proposition \ref{prop:converge} implies that we can construct a certificate $V_N = V_{\theta^\star}$, while the algorithm that returns this certificate satisfies Assumption \ref{ass:alg_prop} and admits a compression set $\mathcal{C}_N$ with cardinality $C_N$. 
As such, Algorithm \ref{algo:sub} offers a constructive mechanism to synthesize a certificate, and can be accompanied by the probabilistic guarantees of Theorem \ref{thm:Guarantees}. 
Moreover, Assumptions \ref{ass:non-conc_mass} \& \ref{ass:exist} under which Algorithm \ref{algo:sub} exhibits these properties are rather mild.

Our way of computing a compression set serves as an efficient alternative to existing methodologies, as we construct it iteratively. 
At the same time the constructed compression set is non-trivial as we avoid adding uninformative samples to it, and only add one sample per iteration in the worst case, however, the one that maximizes the loss (see step 12).
This algorithm could be thought of as a constructive procedure for the general methodology proposed recently in \cite{DBLP:conf/nips/PaccagnanCG23}.

Proposition \ref{prop:converge} shows that Algorithm \ref{algo:sub} terminates and produces parameter iterates that yield a non-increasing sequence of loss functions. As such, the algorithm moves towards the direction of the optimum, but we have no guarantees that it indeed reaches some (local) optimum.
We conjecture the approximate subgradient used in our algorithm constitutes a descent direction~\cite{doi:10.1137/1.9781611971309}, and hence if the step size is chosen appropriately the algorithm should converge to a stationary point. Current work focuses on formalizing this claim.


% \end{rem}

In some cases, the parameter returned by Algorithm \ref{algo:sub} may result in a value of the loss function that is considered as undesirable (and as a result the constructed certificate might be far from meeting the desired conditions). 
To achieve a lower loss, we make use of a sample-and-discarding procedure \cite{DBLP:journals/jota/CampiG11,DBLP:journals/tac/RomaoPM23}.
To this end, consider Algorithm \ref{algo:main}. At each iteration of this algorithm, the compression set returned by Algorithm \ref{algo:sub} (step~\ref{line:subgrad}) is appended to a ``running'' set $\widetilde{\mathcal{C}}$ (see step~\ref{line:update_outer_C}). 
We then discard all elements of the compression set from $\mathcal{D}$, and repeat the process till the worst case loss $\max_{\xi \in \mathcal{D}} L(\theta, \xi)\geq0$ is sufficiently small and ideally zero. 
This implies that Algorithm \ref{algo:sub} is invoked each time with fewer samples as its input, while the set $\widetilde{\mathcal{C}}$ progressively increases.
The set $\widetilde{\mathcal{C}}$ is a compression set that includes all samples that lead to a worst case loss, plus all
samples that are removed along the process of Algorithm \ref{algo:main}.
However, it has higher cardinality compared to the original compression set, implying that improving the loss comes at the price of an increased risk level $\varepsilon$ as the cardinality of the compression set increases.

\begin{algorithm}[ht]
\caption{Compression Set Update with Discarding}
\vspace{0.2cm}
\label{algo:main}
\hrule \vspace{0.05cm} \hrule \vspace{0.1cm}
\begin{algorithmic}[1]
\State Fix $ \{\xi^i\}_{i=1}^N$
    \State Set $\widetilde{\mathcal{C}}\gets \emptyset$\Comment{Initialize compression set}
    \State Set $\mathcal{D} \gets \{\xi^i\}_{i=1}^N$ \Comment{Initialize ``running'' samples}
    %\State $\rhd$ While loss is positive
    \While{$\max_{\xi \in \mathcal{D}} L(\theta, \xi)>0$}
            \State $\theta, \mathcal{C} \gets$ $\mathcal{A}(\theta,\mathcal{D})$ \Comment{Call Algorithm \ref{algo:sub}} \label{line:subgrad}
        \State $\widetilde{\mathcal{C}} \gets \widetilde{\mathcal{C}} \cup \mathcal{C}$ \Comment{{Update $\widetilde{\mathcal{C}}$}} \label{line:update_outer_C}
        \State  $\mathcal{D} \gets \mathcal{D} \setminus \widetilde{\mathcal{C}}$ \Comment{Discard $\widetilde{\mathcal{C}}$ from $\mathcal{D}$} \label{line:discard}
    \EndWhile
        \State \Return $\theta$, $\widetilde{\mathcal{C}}$
\end{algorithmic}
\vspace{0.1cm}
\hrule \vspace{0.05cm} \hrule 
\end{algorithm}

\subsection{Choices of Loss Function}
We now provide some choices of the loss function $L(\theta,\xi)=l^\Delta(V_\theta, \xi) + l^s(V_\theta)$ so that minimizing that function we obtain a parameter vector $\theta^\star$, and hence also a certificate $V_{\theta^\star}$, which satisfies the conditions of the property under consideration, namely, reachability, safety, or RWA.
Note that when calculating subgradients to these functions, which as we will see below are non-convex, we effectively have the so-called Clarke subdifferential~\cite{doi:10.1137/1.9781611971309}.

We provide some expressions for $l^s$ and $l^\Delta$ for the reachability property in Property \ref{prop:reach}. 
For the other properties, the loss functions can be defined in an analogous manner. 
To this end, we define
\begin{align}
    &l^s(V_\theta) \defeq \int_{X \setminus X_G}\max\{0,-\delta-V_\theta(x)\} ~\mathrm{d}x \\ &+\int_{X_I}\max\{0,V_\theta(x)\}   ~\mathrm{d}x+\int_{\mathbb{R}^N \setminus X} \max\{0,-V_\theta(x)\}  ~\mathrm{d}x.\nonumber
\end{align}
Focusing on the first of these integrals, if $V(x) > -\delta$ then $\max\{0, -\delta-V_\theta(x)\}=0$, i.e., no loss is incurred, implying satisfaction of \eqref{eq:reach_goal}, \eqref{eq:reach_else}. 
Under a similar reasoning, the other integrals account for \eqref{eq:reach_init} and \eqref{eq:reach_dom_border}, respectively. 
Note that, for a sufficiently expressive neural network, we can find a certificate $V$ which satisfies the state constraints and hence has a sample-independent loss of zero.

In practice, we replace integrals with a summation over points generated deterministically within the relevant domains. 
These points are generated densely enough across the domain of interest, and hence offer an accurate approximation. 
This generation may happen through gridding the relevant domain, or sampling according to a fixed synthetic distribution.
For the last term, we only enforce the positivity condition on the border of the domain $X$.
Thus, we take a deterministically generated discrete set of points on each domain $\mathcal{X}_{\overline{G}}$ for points in the domain but outside the goal region, $\mathcal{X}_I$ from the initial set, and $\mathcal{X}_\partial$ for the border of the domain $X$.
Since these samples do not require access to the dynamics we consider them separate to the sample-set $\{\xi_i\}_{i=1}^N$, and references to the size of the sample set only refer to the trajectory samples (since these are the ``costly'' samples).
Our practical loss function is then of the following form: 
\begin{align}
    &\hat{l^s}(V_\theta) \defeq  \frac{1}{|\mathcal{X}_{\overline{G}}|}\sum_{x \in \mathcal{X}_{\overline{G}}} \max\{0, -\delta-V_\theta(x)\} \\ &+\frac{1}{|\mathcal{X}_I|}\sum_{x \in \mathcal{X}_I}\max\{0,V_\theta(x)\}+\frac{1}{|\mathcal{X}_\partial|}\sum_{x \in \mathcal{X}_\partial} \max\{0,-V_\theta(x)\}.\nonumber
    \end{align}

We define $l^\Delta$ by
\begin{equation}
\label{eq:c_deriv}
        \begin{aligned}            
        l^\Delta&(V_\theta, \xi) \defeq \max \Big \{ 0, \frac{1}{T} \Big (\sup_{x\in \mathcal{X}_I} V_\theta(x) + \delta \Big )\\&-\max_{k=0,\dots,k_G-1} \Big ( V_\theta(x(k+1))-V_\theta(x(k)) \Big ) \Big \}.
        \end{aligned}
\end{equation}
The value of $l^\Delta$ encodes a loss if the condition in \eqref{eq:reach_deriv} is violated.
If both $l^s$ and $l^\Delta$ evaluate to zero for all $\{\xi\}_{i=1}^N$, then we have that 
\begin{equation}
\begin{aligned}
        l^s(V_\theta) + \max_{i = 1, \dots, N} l^\Delta(V_\theta, \xi^i) = 0,
\end{aligned}
\end{equation}
which by Certificate \ref{cert:reach} implies that the constructed certificate $V_\theta$ is such that
\begin{equation}
\begin{aligned}
        V_\theta \models \psi^s_\text{reach} \wedge (i=1,\dots,N) V_\theta \models\psi^\Delta_\text{reach}(\xi^i).
\end{aligned}
\end{equation}
Analogous conclusions hold for all other certificates.
\section{Comparison with Related Work}
\label{sec:related}
\subsection{Direct Property Evaluation}
\label{sec:related:direct_prop}
As is known in the case of Lyapunov stability theory, the existence of a certificate is useful per se, and allows one to translate a property to a scalar function. 
As discussed in Corollary~\ref{corr:prop}, a byproduct of this certificate synthesis is that they provide guarantees on the probability of property violation (see \eqref{eq:prop_viol}). 
However, if one is not interested in the construction of a certificate and only in such guarantees, then Theorem 2 in \cite{DBLP:journals/sttt/BadingsCJJKT22} provides an alternative. 
We adapt this result in the proposition below, using the Langford binomial tail bound~\cite{JMLR:v6:langford05a} to obtain a tighter guarantee than the sampling-and-discarding result~\cite{DBLP:journals/jota/CampiG11} used in \cite{DBLP:journals/sttt/BadingsCJJKT22}.
\begin{prop}[Theorem $2$ in \cite{DBLP:journals/sttt/BadingsCJJKT22}]
\label{corr:a_post}
Fix $\beta \in (0,1)$, and for $r = 0,\ldots,N-1$,
determine $\varepsilon(r,\beta,N)$ such that 
    \begin{equation}   
    \label{eq:eps_orig}
     \sum_{k=0}^r\binom{N}{k} \varepsilon^k(1-\varepsilon)^{N-k}=\beta,
    \end{equation}
  while for $r=N$ let $\varepsilon(N,\beta,N) = 1$.   
	Denote by $R_N$ the number of samples in $\{\xi^i\}_{i =1}^{N}$ for which $\phi(\xi^i)$ is violated.
    We then have that
    \begin{align}
	    \mathbb{P}^N &\big \{ \{\xi^i\}_{i=1}^N \in \Xi^N:~\nonumber \\
        &\mathbb{P}\{\xi \in \Xi \colon \neg\phi(\xi) \} \leq \varepsilon(R_N,\beta,N)\big\}
        \geq 1-\beta. \label{eq:prop_direct}
    \end{align}
\end{prop}
% This involves a direct application of the Langford binomial tail bound~\cite{JMLR:v6:langford05a}. % (offering an improvement on the sampling-and-discarding results in \cite{DBLP:journals/jota/CampiG11} which require $\frac{\beta}{N}$ on the right-hand side of \eqref{eq:prop_direct}). 
It is \emph{a posteriori} as $R_N$ can be determined only once the samples are observed. 
% To this end, the term $\beta /N$ appears in the right-hand side of \eqref{eq:prop_direct} to account for the fact that, depending on the samples, up to $N$ terms could appear in the summation.
In this setting, we have a compression set which is the set of all discarded samples, plus an additional one to support the solution after discarding.
Since this additional sample is always present, we incorporate it in the formula in \eqref{eq:eps_orig}. 

We remark that one could obtain  different bounds through alternative statistical techniques, such as Hoeffding's inequality~\cite{Hoeffding01031963} or Chernoff's bound~\cite{10.1214/aoms/1177729330}. 
Since these bounds are of different nature, we do not pursue that avenue further here.

%For property violation Hoeffding bounds expected value (hence violation rate) which is the same as what scenario approach bound does because we have a bernoulli distribution. 
% Hence no reason we couldn't use Hoeffding, maybe we could also apply it to the result from certificate synthesis? Then the below arguments hold.

% However, such techniques have the drawback of only probabilistically bounding a distance from the mean, rather than providing a bound on the probability of drawing a violating sample.}
% \textcolor{red}{[next statement unclear:]}
% Further, one requires bounds on the loss $L(\theta, \xi)$ and hence on the certificate value and Lie derivative, which may not be available.}

% \textcolor{red}{[pls rephrase, very clear:]}

Proposition \ref{corr:a_post} offers an alternative to Theorem \ref{thm:Guarantees} to directly bound property violation. 
We compare the risk levels $\varepsilon$ computed by each approach on our benchmark example in \eqref{eq:spiral_dyn} under a safety specification; general conclusions are case dependent, as both bounds are \emph{a posteriori}.
For a fixed $\beta$, Figure~\ref{fig:bounds} shows the resulting risk levels for varying $N$ across 5 independently sampled sets of trajectories.
The difference of the orange curve from the blue one can be interpreted as the price of certificate generation of Theorem \ref{thm:Guarantees}. For sufficiently large $N$, this price is marginal.  
As the specification is deterministically safe, no discarding is performed for Proposition~\ref{corr:a_post}, hence a smooth curve without variability. 
For non-zero $R_N$ we expect variability as $R_N$ will be randomly distributed. 
\begin{figure}
    \centering
    \includegraphics[width=0.75\linewidth]{Figures/risk_comparison_curves}
    \caption{Comparison of the bounds in Theorem~\ref{thm:Guarantees} and Proposition~\ref{corr:a_post} for direct property evaluation. Median values are shown with a cross, and ranges are indicated by error bars.}
    \label{fig:bounds}
\end{figure}

%Theorem 5.3
\subsection{Certificate Synthesis as in \cite{DBLP:journals/tac/NejatiLJSZ23}}
The results in \cite{DBLP:journals/tac/NejatiLJSZ23} constitute the most closely related ones with respect to our work. 
As no results on reachability and RWA problems were provided in \cite{DBLP:journals/tac/NejatiLJSZ23}, we limit our discussion to the safety property. 
As with our work, a sample-based construction is performed, where samples therein are pairs (state, next-state), as opposed to trajectories as in our work. However, the probabilistic bounds established in \cite{DBLP:journals/tac/NejatiLJSZ23} are structurally different and of complementary nature to our work: next, we review the main result in \cite{DBLP:journals/tac/NejatiLJSZ23}, adapted to our notation.

\begin{thm}[Theorem $5.3$ in~\cite{DBLP:journals/tac/NejatiLJSZ23}]
    Consider \eqref{eq:Dyn}, with initial and unsafe sets $X_I,X_U\subset X \subset \mathbb{R}^n,$ respectively.
    Consider also $N$ samples $\{x_i, f(x_i)\}_{i=1}^N$ from $X$, and assume that the loss function in \eqref{eq:opt_prob} is Lipschitz continuous with constant $\mathcal{L}$.
    Consider then the problem
        \begin{align}
            \eta^\star_N &\in \argmin_{d = (\gamma,\lambda,c,\theta),\eta \in \mathbb{R}}\eta \nonumber \\
            \text{st. }&\; V_\theta(x)-\gamma \leq \eta, \; \forall x \in X_I \nonumber \\
            &\; V_\theta(x)-\lambda \geq -\eta, \; \forall x \in X_U \nonumber \\
            &\; \gamma + cT - \lambda - \mu \leq \eta, ~ c \geq 0, \nonumber \\
            &\; V_\theta(f(x_i)) - V_\theta(x_i) -c \leq \eta, \; i=1,\dots,N,  
        \end{align}
        where $\theta$ parameterizes $V_\theta$, and all other decision variables are scalars leading to level sets of $V_\theta$. Let $\kappa(\delta)$ be such that
    \begin{equation}
	    \label{eq:ball}
        \kappa(\delta) \leq \mathbb{P}\{\mathbb{B}_\delta(x)\}, \forall \delta \in \mathbb{R}_{\geq 0}, \forall x \in X,
    \end{equation}
    where $\mathbb{B}_\delta(x) \subset X$ is a ball of radius $\delta$, centered at $x$.
    Fix $\beta \in (0,1)$ and
    determine $\epsilon(|d|,\beta,N)$ from \eqref{eq:eps_orig}, with $r = d$ and by replacing the right hand-side with $\beta$.
    If $\eta^\star_N \leq \mathcal{L} \kappa^{-1}(\epsilon(|d|,\beta,N)$, we have that
    \begin{equation}
	    \mathbb{P}^N\big \{ \{\xi^i\}_{i=1}^N \in \Xi^N:~\phi_{\mathrm{safe}}(\xi), \; \forall \xi \in \Xi  \big\} \geq 1-\beta.
    \end{equation}
\end{thm}

The following remarks are in order.
\begin{enumerate}[wide, labelwidth=!, labelindent=0pt]
	\item The result in \cite{DBLP:journals/tac/NejatiLJSZ23} is \emph{a priori} (capitalizing on the developments of \cite{6832537}), as opposed to the \emph{a posteriori} assessments of our analysis that are in turn based on \cite{DBLP:journals/jmlr/CampiG23}.
    Moreover, \cite{DBLP:journals/tac/NejatiLJSZ23} offers a
    guarantee that, with a certain confidence, the safety property is \emph{always} satisfied. This is in contrast to Theorem \ref{thm:Guarantees} where we provide such guarantees in probability (up to a quantifiable risk level $\varepsilon$).
However, these ``always'' guarantees come with potential challenges. In particular, the constraint in \eqref{eq:ball} involves the measure of a ``ball'' in the uncertainty space. 
The measure of this ball grows exponentially in the dimension of the uncertainty space (see also Remark 3.9 in 
    \cite{6832537}), while it depends linearly on the dimension of the decision space $|d|$ (see dependence of $\varepsilon$ below \eqref{eq:ball}). This dependence in the results of \cite{DBLP:journals/tac/NejatiLJSZ23} raises computational challenges to obtain useful bounds: we demonstrate this numerically in Section~\ref{sec:exp} employing  one of the examples considered in \cite{DBLP:journals/tac/NejatiLJSZ23}. On the contrary, Theorem \ref{thm:Guarantees} is independent of the dimension of these spaces and only depends on the cardinality of the compression set. 
	\item The result in \cite{DBLP:journals/tac/NejatiLJSZ23} requires inverting $\kappa(\delta)$, which may not have an analytical form in general. Moreover, it implicitly assumes some knowledge of the distribution to obtain $\kappa$, and of the Lipschitz constants of the system dynamics, which we do not require in our analysis. 
\item The results of \cite{DBLP:journals/tac/NejatiLJSZ23} are also extended to continuous-time dynamical systems. 
This is also possible for our results; however, due to practical considerations, we then require knowledge of the Lipschitz constants not required in the discrete time setting. 
This discussion is not pursued further here. 
\end{enumerate}
\section{Experiments}\label{sec:experiments}

In this section, we conduct experiments with two goals in mind. First, we complement our convergence result from Theorem~\ref{thm:convergence} by demonstrating that, in practice, the belief walk converges rather quickly. Second, using simulated data, we compare the performance of Algorithm~\ref{bb-algorithm} to a state-of-the-art benchmark. 

\paragraph{Simulation details}
We generate instances using a random sampling procedure. We generate $\bm P$ by independently sampling latent vectors for a given number of categories and user types. Namely, we sample a latent vector for each user type and category from a normal distribution, computing entries of $\bm P$ as negated cosine distances (representing a user's affinities to a category), and normalizing these entries. We generate $\bm q$ by independently sampling logits from a normal distribution. Then, we transform them into a categorical distribution through the softmax function. 
The simulations were conducted on a standard CPU-based PC. Further details appear in \apxref{sec:auxiliary-details-about-the-experiments}. 

\begin{figure}
   \centering
   \includegraphics[width=.5\linewidth]{simulations/uncertainty.png}
   \caption{Convergence of beliefs under optimal policies. The number of categories is set to $10$. The x-axis is the number of rounds, and the y-axis is uncertainty in the user type. The transparent lines illustrate a few individual runs and the solid lines are averages over $500$ runs.}
   \label{fig:uncertainty}
\end{figure}

\paragraph{Convergence of beliefs}
Theorem \ref{thm:convergence} establishes that the optimal policy eventually converges to picking a fixed category. While the theorem guarantees convergence, it does not provide explicit rates. Equivalently, convergence can be analyzed in terms of certainty about a user's type, represented by proximity to the vertex to which the belief walk converges (recall the proof sketch of Theorem~\ref{thm:convergence}). Since beliefs update according to Bayes' rule, they converge at a geometric rate once the policy becomes fixed. In other words, further exploration yields diminishing returns when a belief is sufficiently close to a vertex. Thus, it is tempting to assume that the optimal policy myopically maximizes value for that vertex. On the other hand, a poorly chosen myopic policy can fail drastically, as Proposition~\ref{prop:myopic-policy-suboptimality} illustrates. We resolve these conflicting observations through simulations.

Figure \ref{fig:uncertainty} shows how \emph{uncertainty} in user type, defined as the $l_1$-distance from the vertex to which the belief converges under the optimal policy, evolves throughout the session. We vary the number of user types while fixing the number of categories. For each problem size, we report the averaged uncertainty and several individual runs. Despite the heterogeneity of individual runs, their geometric convergence property roughly transfers to averaged curves: Exponential functions fitted to these curves are almost identical to the originals, with correlation coefficients of at least $R^2=0.98$. This matches our intuition that early rounds are most important in terms of both expected reward and information.

Analyzing individual runs reveals notable patterns. While in some sessions, the optimal policies are fixed from the start, in others, recommendations switch (as characterized by jumps in the slope). This reflects the short-term vs. long-term reward trade-off discussed throughout the paper: The optimal policy may initially prioritize immediate rewards before switching to a riskier recommendation that increases certainty and earns more in the long run. 
Despite this, all the presented curves strictly decrease, suggesting that certainty increases monotonically. However, we found that in rare cases, the optimal policy can move away from a vertex before converging to it.
This resolves the above conflict: Even if the belief approaches some vertex, the optimal policy may eventually lead to a different vertex. We exemplify this behavior in \apxref{sec:belief walks}.

\begin{figure*}
\centering
\begin{subfigure}{0.02\textwidth}
   \centering
   \includegraphics[width=\linewidth]{simulations/time_comp_y_axis_caption.png}
   \vspace{0.7cm} % Add vertical space
\end{subfigure}%
\hspace{0.1cm} % Add horizontal space between subfigures
\begin{subfigure}{0.3\textwidth}
   \centering
   \includegraphics[width=\linewidth]{simulations/types_baseline_time.png}
   \caption{x-axis varies types; 10 categories}
   \label{fig:types_time}
   \vspace{0.5cm} % Add vertical space
\end{subfigure}%
\hspace{0.2cm} % Add horizontal space between subfigures
\begin{subfigure}{0.307\textwidth}
   \centering
   \includegraphics[width=\linewidth]{simulations/types_and_actions_baseline_time.png}
   \caption{x-axis varies types and categories}
   \label{fig:actions_types_time}
   \vspace{0.5cm} % Add vertical space
\end{subfigure}%
\hspace{0.2cm} % Add horizontal space between subfigures
\begin{subfigure}{0.3\textwidth}
   \centering
   \includegraphics[width=\linewidth]{simulations/actions_baseline_time.png}
   \caption{x-axis varies categories; 10 types}
   \label{fig:actions_time}
   \vspace{0.5cm} % Add vertical space
\end{subfigure}%
\caption{Runtime comparison (in milliseconds) between Algorithm~\ref{bb-algorithm} and a POMDP solver SARSOP. Each data point represents an average runtime over $500$ of $\prob$ instances. Shaded intervals represent $95\%$ bootstrap confidence intervals of the empirical average. Both algorithms stop when they reach a precision of $\varepsilon=10^{-6}$.\label{fig:sarsop}}
\end{figure*}


\paragraph{Runtime comparison}
Our model is novel, so there are no specifically tailored baselines. However, since it can be cast as a POMDP, we can compare Algorithm~\ref{bb-algorithm} with more general solvers. As a baseline, we have chosen SARSOP, a well-known offline point-based POMDP solver \cite{kurniawati2009sarsop}. While Algorithm~\ref{bb-algorithm} is straightforward, 
SARSOP is rather complex. It represents the optimal policy through $\alpha$-vectors (a convex piece-wise linear approximation of the value function) and clusters sampled beliefs to estimate the values of new ones. We used an open-source implementation of SARSOP,\footnote{\url{https://github.com/AdaCompNUS/sarsop}} and, for a fair comparison, implemented Algorithm~\ref{bb-algorithm} in the same language. 

Figure~\ref{fig:sarsop} presents the runtime comparison between Algorithm~\ref{bb-algorithm} and SARSOP. The statistical tests that support the comparison are deferred to \apxref{sec:auxiliary-details-about-the-experiments}. While Algorithm~\ref{bb-algorithm} dominates SARSOP on rectangular problems with a few categories (Figure~\ref{fig:types_time}) and overperforms SARSOP on square problems (Figure~\ref{fig:actions_types_time}), it underperforms when the number of categories is much higher than the number of user types (Figure~\ref{fig:actions_time}). 

We hypothesize that this result is due to SARSOP more effectively dealing with similar categories (similar associated rows in the matrix $\bm P$) through the $\alpha$-vector representation and clustering heuristic. Another explanation is that Algorithm~\ref{bb-algorithm} explores the policy space; hence, the branching factor is the number of categories. In contrast, SARSOP explores the belief space, whose dimension is the number of types. Consequently, we could expect SARSOP to struggle in cases with many user types and Algorithm~\ref{bb-algorithm} to encounter challenges in cases with many categories. Overall, each algorithm excels under different conditions.

\section{Experiments with Real-World Data} 
%Our model assumes structural properties that differ significantly from standard recommender systems through its requirements for a denser representation of the system through clusters of users and items. This section aims to delineate the essential transformations between these paradigms, demonstrating how to connect conventional sparse rating data with the parameters of our model. 
Beyond the simulations in Section~\ref{sec:experiments}, we also conducted experiments with the Movielens 1M dataset~\cite{harper2015movielens}. To demonstrate the applicability of our approach, we outline below how we transform sparse rating matrices into the required model parameters.

Real-world RS datasets typically consist of a sparse user-item rating matrix, where observed entries represent user ratings for items, with the majority of entries being unobserved. This representation differs from our model's requirements of a dense probability matrix between user types and content categories, accompanied by a prior distribution over user types; hence, our model cannot be applied directly, and the following two transformations are required.

First, the sparse rating matrix must be aggregated into a concise representation through clustering of users and items. This is a well-studied task and several solutions have been proposed in the literature, such as Spectral co-clustering~\cite{coclustering} or DBSCAN~\cite{dbscan}. Second, the clustered data must be transformed into model parameters. One straightforward way to achieve this is to construct the preference matrix by computing mean ratings within cluster pairs and normalizing to $[0,1]$, and deriving the prior distribution from cluster sizes. We defer this analysis to \apxref{sec:movielens}. Our empirical investigation using this real-world dataset substantiates the qualitative patterns observed in Section~\ref{sec:experiments}. Importantly, we stress that the qualitative results we obtain in this section extend beyond the synthetic setup to real-world datasets.


%We demonstrate this methodology using the MovieLens 1M dataset in \apxref{sec:movielens}. Specifically, we provide comprehensive implementation details and extend our comparison with the baseline algorithms from synthetic preference matrices to parameters derived from empirical rating data.  
\section{Conclusion}
We operationalized the theory of instrumental interaction for generative AI, with an in-depth unpacking of the principles of reification of user intent, reflection, and grounding. We argue that leveraging this re-appropriated and refined theory can drive the creation of a \textit{new generation of expressive AI-Instruments} that afford better expression of intent, make it easier to discover what is possible, and provide powerful degrees of freedom for steering the generation towards the best possible results. Those new tools and instruments can truly leverage the polymorphic and non-deterministic behavior of generative AI models, unleashing new and empowering forms of expressive HCI+AI experiences. 

Beyond our focus on AI-Instruments, theories play an important role in the advancement of our wider research field~\cite{rogers_hci_2012, halverson_activity_2002}. Rogers argues that there is a need for theories as lenses bringing critical design characteristics into focus, and which can function as a generative source: providing "\textit{design dimensions and constructs to inform the design and selection of interactive representations}"~\cite{rogers_new_2004}. We hope that our work on operationalizing the theory of instrumental interaction for AI can inspire other new -- and re-appropriated -- theories to advance HCI+AI. 










\bibliographystyle{plain}
\bibliography{neural_certs_paper}


\ifappendix
\appendix   
\section{Proofs}
\label{app:proofs}

\subsection{Certificate Proofs}

\subsubsection{Proof of Proposition \ref{cert:reach} -- Reachability Certificate}
Fix $\delta > -\sup_{x \in X_I} V(x) \geq 0$, and recall that $k_G = \min \{k \in \{0,\dots,T\} \colon V(x(k)) \leq -\delta\}$. Consider then the difference condition in \eqref{eq:reach_deriv}, namely,
    \begin{equation} \label{eq:proof_dec_reach}
        \begin{aligned}
		&V(x(k+1)) - V(x(k)) \\ 
        & < - \frac{1}{T}\Big (\sup_{x \in X_I} V(x)+\delta \Big),~ k=0,\dots,k_G-1,
        \end{aligned}
    \end{equation}
By recursive application of this inequality $k \leq k_G$ times, 
\begin{align}
&V(x(k)) < V(x(0)) -\frac{k}{T} \Big ( \sup_{x \in X_I} V(x) +\delta\Big ) \nonumber \\
&\leq \frac{T-k}{T} \sup_{x \in X_I} V(x) -\frac{k}{T} \delta  \leq -\frac{k}{T} \delta \leq 0, \label{eq:proof_dec_reach1}
\end{align}
where the second inequality is since $V(x(0)) \leq \sup_{x \in X_I} V(x)$, as $x(0) \in X_I$. The third
one is since $\sup_{x \in X_I} V(x) \leq 0$ as by \eqref{eq:reach_init}, $V(x) \leq 0$, for all $x \in X_I$, and  $k \leq k_G \leq T$, while the last inequality is since $\delta>0$.


By \eqref{eq:proof_dec_reach1} we then have that for all $k\leq k_G$, $V(x(k)) <0$, which implies that $x(k)$ does not leave $X$ for all $k\leq k_G$ (see \eqref{eq:reach_else}), while by the definition of $k_G$, $x(k_G) \in X_G$. Notice that if $k_G = T$, then \eqref{eq:proof_dec_reach1} (besides implying that $x(k) \in X$ for all $k\leq T$), also leads to $x(T) \leq -\delta$, which means that $x(T) \in X_G$ after $T$ time steps (see \eqref{eq:reach_goal}), which captures the latest time the goal set is reached.

Therefore, all trajectories that start within $X_I$ reach the goal set $X_G$ in at most $T$ steps, without escaping $X$ till then, thus concluding the proof. \qed

\subsubsection{Proof of Proposition \ref{cert:barr} -- Safety Certificate}

Consider the condition in \eqref{eq:barr_deriv}, namely,
    \begin{equation}
     \begin{aligned}
        &V(x(k+1))-V(x(k)) \\ &< \frac{1}{T} \Big( \inf_{x \in X_U}V(x)-\sup_{x \in X_I}V(x) \Big ),~ k=0,\dots,T-1.
     \end{aligned}
     \end{equation}
     By recursive application of this inequality for $k\leq T$ times, we obtain 
     \begin{align}
         &V(x(k)) < V(x(0)) + \frac{k}{T} \Big( \inf_{x \in X_U}V(x)-\sup_{x \in X_I}V(x) \Big ) \nonumber \\
         &\leq \frac{T-k}{T} \sup_{x \in X_I}V(x) + \frac{k}{T} \inf_{x \in X_U}V(x) \nonumber \\
         &\leq \frac{k}{T} \inf_{x \in X_U}V(x) \leq \inf_{x \in X_U}V(x). \label{eq:proof_safety}
     \end{align}
     where the second inequality is is since $V(x(0)) \leq \sup_{x \in X_I}V(x)$, as $x(0) \in X_I$. The third inequality is since $\sup_{x \in X_I}V(x) \leq 0$ as by \eqref{eq:barr_states1}, $V(x) \leq 0$ for all $x \in X_I$ and $k\leq T$. The last inequality is since $\inf_{x \in X_U}V(x)\geq 0$, as by \eqref{eq:barr_states2} $V(x) >0$ for all $x \in X_U$, and $k\leq T$.
     We thus have
           \begin{align}
       V(x(k)) &< \inf_{x \in X_U}V(x),~ k=1,\dots,T.
       \end{align}
       and hence $x(k) \notin X_U, k=0,\dots,T$ (notice that $x(0) \notin X_U$ holds since $X_I \cap X_U = \emptyset$). The latter implies that all trajectories that start in $X_I$ avoid entering the unsafe set $X_U$, thus concluding the proof. \qed

\subsubsection{Proof of Proposition \ref{cert:RWA} -- RWA Certificaate}

Fix $\delta > -\sup_{x \in X_I} V(x) \geq 0 $, and recall that $k_G = \min \{k \in \{0,\dots,T\} \colon V(x(k)) \leq -\delta\}$.
Consider then the difference condition in \eqref{eq:RWA_deriv}, namely,
    \begin{equation} \label{eq:proof_RWA}
     \begin{aligned}
        &V(x(k+1))-V(x(k)) \\ &<-\frac{1}{T} \Big ( \sup_{x \in X_I}V(x)+\delta \Big ),~ k=0,\dots,k_G-1,
     \end{aligned}
     \end{equation}
    Note that this is identical to the difference condition for our reachability property, and hence following the same arguments with the proof of Proposition \ref{cert:reach}, we can infer that state trajectories emanating from $X_I$ will reach the goal set $X_G$ in at most $T$ time steps.
    
    By \eqref{eq:RWA_else} we have that $V(x) > 0$, for all $x \in _U$ while by \eqref{eq:RWA_init} we have that $V(x) \leq 0$, for all $x \in X_I$. Therefore, $\sup_{x\in X_I} V(x) \leq 0 \leq \inf_{x \in X_U} V(x)$. At the same time by our choice for $\delta$ we have that $\delta>-\sup_{x \in X_I}V(x)$. Combining these, we infer that $\delta >-\inf_{x \in X_U} V(x)$.
    Thus, \eqref{eq:proof_RWA} implies that for all $k=0,\ldots,k_G-1$,
    \begin{align}
        -\frac{1}{T} \Big ( \sup_{x \in X_I}&V(x) +\delta \Big )  \nonumber \\
        &< \frac{1}{T} \Big (\inf_{x \in X_U}V(x)-\sup_{x \in X_I}V(x)\Big ).
    \end{align}
    Therefore,
     \begin{align}
        &V(x(k+1))-V(x(k)) \\ &<\frac{1}{T} \Big (\inf_{x \in X_U}V(x)-\sup_{x \in X_I}V(x)\Big ), k=0,\dots,k_G-1. \nonumber
     \end{align}
     Note that this is identical to the difference condition for our safety property, and hence following the same arguments with the proof of Proposition \ref{cert:barr}, we can infer that state trajectories emanating from $X_I$ will never pass through the unsafe set $X_U$ until time $k=k_G$.

     Moreover, by \eqref{eq:RWA_deriv2}, we have that 
      \begin{equation}
     \begin{aligned}
        &V(x(k+1))-V(x(k)) \\ &<\frac{1}{T} \Big (\inf_{x \in X_U}V(x)+\delta\Big ),~ k=k_G,\dots,T-1.
     \end{aligned}
     \end{equation}
     Note that this is also a difference condition identical to that for our safety property, but with $\delta$ in place of $\sup_{x \in X_I} V(x)$ (since we know that $V(x(k_G)) \leq -\delta$ by definition of $k_G$).
     Hence, we have a safety condition for all trajectories emanating from this sublevel set.
     We know from \eqref{eq:proof_RWA} that trajectories reach this sublevel set, and hence remain safe for $k=k_G,\ldots,T$

    Therefore, we have shown that starting at $X_I$ trajectories reach $X_G$ in at most $T$ time steps, while they never pass through $X_U$ and do not leave the domain $X$, thus concluding the proof. \qed

\subsection{Proof of Proposition \ref{prop:converge} -- Properties of Algorithm 1}
\begin{enumerate}[wide, labelwidth=!, labelindent=0pt]
\item By construction, Algorithm \ref{algo:sub} creates a non-increasing sequence of iterates $\{L_k\}_{k\geq 0}$ that is bounded below by the global minimum of $\max_{\xi \in D} L(\theta,\xi)$ which exists and is finite due to Assumption \ref{ass:exist}.
As such, the sequence $\{L_k\}_{k\geq 0}$ is convergent, which in turn implies that Algorithm \ref{algo:sub} terminates.\\
\item We need to show that the set $\mathcal{C}_N$ is a compression set in the sense of Definition \ref{def:compress} with $\mathcal{A}$ being Algorithm \ref{algo:sub} with $D = \{\xi_i\}_{i=1}^N$. 
To see this, we ``re-run'' Algorithm \ref{algo:sub} from the same initial choice of the parameter vector $\theta$ but with $\mathcal{C}_N$ in place of $D$. 
Notice that exactly the same iterates will be generated, as $\mathcal{C}_N$ contains all samples that have a misaligned subgradient and value greater than the loss evaluated on the running compression set. %all samples that have led to a worst case loss across iterations (step 6). 
As a result, the same output will be returned, which by Definition \ref{def:compress} establishes that $\mathcal{C}_N$ is a compression set.\\
\item We show that all properties of Assumption \ref{ass:alg_prop} are satisfied by Algorithm \ref{algo:sub}. \\

\emph{Preference:} Consider a fixed (sample independent) initialization of Algorithm \ref{algo:sub} in terms of the parameter $\theta$. 
Consider also any subsets $\mathcal{C}_1,\mathcal{C}_2$ of 
$\{\xi^i\}_{i=1}^N$ with $\mathcal{C}_1\subseteq \mathcal{C}_2$. 

Suppose that the compression set returned by Algorithm \ref{algo:sub} when fed with $\mathcal{C}_2$ is different from 
$\mathcal{C}_1$. 
          Fix any $\xi \in \Xi$ and consider the set $\mathcal{C}_2 \cup \{\xi\}$. We will show that the compression set returned by Algorithm \ref{algo:sub} when fed with $\mathcal{C}_2 \cup \{\xi\}$ is different from $\mathcal{C}_1$ as well.\\
          \emph{Case 1}: The new sample $\xi$ does not appear as a maximizing sample in step~\ref{line:max_D} of Algorithm \ref{algo:sub}, or its subgradient is such that the quantity in step~\ref{line:inner} is positive. This implies that step~\ref{line:update_C} is not performed and the algorithm proceeds directly to step~\ref{line:approx_step}. As such, $\xi$ is not added to the compression set returned by Algorithm \ref{algo:sub}, which remains the same with that returned when the algorithm is fed only by $\{\xi^i\}_{i=1}^N$. However, the latter is not equal to $\mathcal{C}_1$, thus establishing the claim.\\  
	       \emph{Case 2}: The new sample $\xi$ appears as a maximizing sample in step~\ref{line:max_D} of Algorithm \ref{algo:sub}, and has a subgradient such that the quantity in step~\ref{line:inner} is non-positive. As such, step~\ref{line:update_C} is performed and $\xi$ is added to the compression returned by Algorithm \ref{algo:sub}.
          If $\xi \notin \mathcal{C}_1$ then the resulting compression set will be different from $\mathcal{C}_1$ as it would contain at least one element that is not $\mathcal{C}_1$, namely, $\xi$.
          
	       If $\xi \in \mathcal{C}_1$ then it must also be in $\mathcal{C}_2$ as $\mathcal{C}_1 \subseteq \mathcal{C}_2$. In that case $\xi$ would appear twice in $\mathcal{C}_2 \cup \{\xi\}$, i.e., the set of samples with which Algorithm \ref{algo:sub} is fed has $\xi$ as a repeated sample (notice that this can happen with zero probability due to Assumption \ref{ass:non-conc_mass}). 
                           
          Once one of these repeated samples is added to the compression set returned by Algorithm \ref{algo:sub}, then the other will never be added. This is since when this other sample appears as a maximizing one in step~\ref{line:max_D} then its duplicate will already be in the compression set, and hence the exact and approximate subgradients in steps~\ref{line:subgrad_D} and~\ref{line:subgrad_C} would be identical. 
          As such, the quantity in step~\ref{line:inner} would be non-negative (and, by positive-definiteness of the inner product, only zero when both vectors are zero-vectors) and hence step \ref{line:update_C} will not be performed, with the duplicate not added to the compression set. 
          As such, one of the repeated $\xi$'s is redundant, which implies that the compression set returned by Algorithm \ref{algo:sub} when fed with $\mathcal{C}_2 \cup \{\xi\}$ is the same with the one that would be returned when it is fed with $\mathcal{C}_2$. 
          However, this would imply that if $\mathcal{C}_1$ is the compression returned by Algorithm \ref{algo:sub} when fed with of $\mathcal{C}_2 \cup \{\xi\}$, it will also be the compression set for $\mathcal{C}_2$ (as the duplicate $\xi$ would be redundant). 
          However, the starting hypothesis has been that $\mathcal{C}_1$ is not a compression of $\mathcal{C}_2$. As such, it is not possible for $\mathcal{C}_1$ to be a compression set of $\mathcal{C}_2 \cup \{\xi\}$ as well, establishing the claim.\\
          
\emph{Non-associativity:} Consider a fixed (sample independent) initialization of Algorithm \ref{algo:sub} in terms of the parameter $\theta$. Let $\{\xi^i\}_{i=1}^{N+\bar{N}}$ for some $\bar{N} \geq 1$.
        Suppose that $\mathcal{C}$ is returned by Algorithm \ref{algo:sub} a compression set of $\{\xi^i\}_{i=1}^{N} \cup \{\xi\}$, for all $\xi \in \{\xi^i\}_{i=N+1}^{N+\bar{N}}$. 
        Therefore, up to a measure zero set we must have that
        \begin{align}
        \mathcal{C} \subset \bigcap_{j=N+1}^{\bar{N}}
       \Big ( \{\xi^i\}_{i=1}^N \cup \{\xi^j\} \Big ) = \{\xi^i\}_{i=1}^N, \label{eq:proof_nonassoc}
        \end{align}
where the inclusion is since $\mathcal{C}$ is assumed to be returned as a compression set by Algorithm \ref{algo:sub} when this is fed with any set within the intersection, while the equality is since by Assumption \ref{ass:non-conc_mass} all samples in $\{\xi^i\}_{i=1}^{N+\bar{N}}$ are distinct up to a measure zero set. This implies that up to a measure zero set $\mathcal{C}$ should be a compression set returned by Algorithm \ref{algo:sub} whenever this is fed with $\{\xi^i\}_{i=1}^N$ as any additional sample would be redundant.

Fix now any $\xi \in \{\xi^i\}_{i=N+1}^{N+\bar{N}}$, and consider 
Algorithm \ref{algo:sub} with $D = \{\xi^i\}_{i=1}^{N} \cup \{\xi\}$. The fact that $\mathcal{C}$ is returned as a compression set for $\{\xi^i\}_{i=1}^{N} \cup \{\xi\}$ implies that whenever $\xi$ is a maximizing sample in step~\ref{line:max_D} of Algorithm \ref{algo:sub}, it should give rise to a subgradient such that the quantity in step $10$ of the algorithm is positive. This implies that step~\ref{line:approx_step} is performed and hence $\xi$ is not added to $\mathcal{C}$. 

Considering Algorithm \ref{algo:sub} this time with $D =\{\xi^i\}_{i=1}^{N+\bar{N}}$, i.e., fed with all samples at once, due to the aforementioned arguments, whenever a $\xi \in \{\xi^i\}_{i=N+1}^{N+\bar{N}}$ is a maximizing sample in step~\ref{line:max_D}, then the algorithm would proceed to step~\ref{line:approx_step}, and steps~\ref{line:true_step}--\ref{line:update_C} will not be executed. As such, no such $\xi$ will be added to $\mathcal{C}$. 

Hence, the compression set returned by Algorithm \ref{algo:sub} when fed with $\{\xi^i\}_{i=1}^{N+\bar{N}}$ would be the same with the one that would be returned if the algorithm was fed with $\{\xi^i\}_{i=1}^{N}$. By \eqref{eq:proof_nonassoc} this then implies that the returned set should be $\mathcal{C}$ up to a measure zero set. \qed

\end{enumerate}
\fi

\end{document}
