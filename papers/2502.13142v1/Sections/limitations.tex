% \vspace{-0.2cm}
\section{Limitations and Future Work} 
\label{sec:limitations}


% \giscard{For @Junyi: 'This poses a potential problem for dynamic robot camera setups, since such camera motion is not a part of the pre-training.'}
While \smodel{} offers substantial benefits for pre-training with human video data for robotic learning, it is important to recognize certain limitations that accompany our approach. Our approach tracks 3D points in camera coordinates, leading to learned representations that combine object and camera motion. This coupling makes it difficult for the model to disentangle the two, leading to potentially inaccurate predictions due to a lack of invariance to camera intrinsics and motion. An improvement addressing this concern could involve pre-training on 3D tracks in world coordinates, leveraging recent dynamic SLAM methods such as MonST3R~\cite{zhang2024monst3r} or MegaSAM~\cite{li2024megasam}, which we leave for future work. Other improvements could include scaling the pre-training data to help the model generalize better to different camera viewpoints, or using multi-view fusion to reduce the dependency on a single viewpoint to improve robustness to occlusions. 
% Another line of future work could focus on selectively tracking only relevant or moving points instead of a fixed uniform grid across frames. This would help the model focus on dynamic elements critical to the task, reducing the influence of static points and improving its ability to disentangle object motion from background noise and camera movement. 
Another line of future work could focus on selectively tracking only relevant or moving points instead of a fixed uniform grid across frames. 
% This would help the model focus on dynamic elements critical to the task, and allow greater resolution in areas with small objects.
This would allow greater resolution in areas with small objects, and also help the model focus on objects critical to the task, improving its ability to disentangle object motion from background noise and camera movement.

% First comment not relevant (pretraining with rtx). Track mnanually or selectively chosen points. 