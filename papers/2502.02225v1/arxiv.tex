% CVPR 2025 Paper Template; see https://github.com/cvpr-org/author-kit

\documentclass[10pt,twocolumn,letterpaper]{article}

%%%%%%%%% PAPER TYPE  - PLEASE UPDATE FOR FINAL VERSION
% \usepackage{cvpr}              % To produce the CAMERA-READY version
\usepackage{cvpr}      % To produce the REVIEW version
\usepackage{algorithm}
\usepackage{algpseudocode}
\usepackage{thmtools,thm-restate}
\usepackage{amssymb} % 如果需要使用 \blacksquare
\usepackage{amsthm}  % 如果需要使用 \qedsymbol
\usepackage{tabularx}
\usepackage{booktabs}
\usepackage{adjustbox}
\newtheorem{theorem}{Theorem}[section]
\renewcommand{\algorithmicrequire}{\textbf{Input:}}
\renewcommand{\algorithmicensure}{\textbf{Output:}}
\renewcommand\qedsymbol{$\blacksquare$}

\newcommand{\ouralgorithm}{Attribute Vector Integration for Image Editing}
\newcommand{\ouracronym}{AVI}
%\newcommand{\oma}[1]{}
\newcommand{\todo}[1]{{\color{red}\bf [TODO: #1]}}

\newcommand{\cut}[1]{}
% \usepackage[pagenumbers]{cvpr} % To force page numbers, e.g. for an arXiv version

% Import additional packages in the preamble file, before hyperref
\newcommand{\CG}{\mathcal{G}\xspace}
\newcommand{\CV}{\mathcal{V}\xspace}
\newcommand{\CE}{\mathcal{E}\xspace}
\newcommand{\CA}{\mathcal{A}\xspace}
\newcommand{\CF}{\mathcal{F}\xspace}
\newcommand{\CR}{\mathcal{R}\xspace}
\newcommand{\CB}{\mathcal{B}\xspace}
\newcommand{\CX}{\mathcal{X}\xspace}
\newcommand{\CK}{\mathcal{K}\xspace}
\newcommand{\CM}{\mathcal{M}\xspace}
\newcommand{\CC}{\mathcal{C}\xspace}
\newcommand{\CL}{\mathcal{L}\xspace}
\newcommand{\CI}{\mathcal{I}\xspace}
\newcommand{\CQ}{\mathcal{Q}\xspace}
\newcommand{\CO}{\mathcal{O}\xspace}
\newcommand{\CP}{\mathcal{P}\xspace}
\newcommand{\CS}{\mathcal{S}\xspace}
\newcommand{\CT}{\mathcal{T}\xspace}
\newcommand{\CJ}{\mathcal{J}\xspace}
\usepackage[para]{footmisc}
\usepackage{subfig}
% \usepackage{subcaption}
% \usepackage{array}
% \usepackage{colortbl}



% It is strongly recommended to use hyperref, especially for the review version.
% hyperref with option pagebackref eases the reviewers' job.
% Please disable hyperref *only* if you encounter grave issues, 
% e.g. with the file validation for the camera-ready version.
%
% If you comment hyperref and then uncomment it, you should delete *.aux before re-running LaTeX.
% (Or just hit 'q' on the first LaTeX run, let it finish, and you should be clear).
\definecolor{cvprblue}{rgb}{0.21,0.49,0.74}
\usepackage[pagebackref,breaklinks,colorlinks,allcolors=cvprblue]{hyperref}

%%%%%%%%% PAPER ID  - PLEASE UPDATE
\def\paperID{8799} % *** Enter the Paper ID here
\def\confName{CVPR}
\def\confYear{2025}

%%%%%%%%% TITLE - PLEASE UPDATE
%\title{The intermediate Latent Space of Diffusion Models is Interpretable}
\title{Exploring the latent space of diffusion models directly through singular value decomposition}

%%%%%%%%% AUTHORS - PLEASE UPDATE
\author{Li Wang\thanks{li.wang@zju.edu.cn}\\
Zhejiang University \\
% For a paper whose authors are all at the same institution,
% omit the following lines up until the closing ``}''.
% Additional authors and addresses can be added with ``\and'',
% just like the second author.
% To save space, use either the email address or home page, not both
\and
Boyan Gao\\
University of Oxford\\
\and
Yanran Li \\
University of Bedfordshire\\
\and
Zhao Wang
Zhejiang University\\
\and
Xiaosong Yang\\
Bournemouth University\\
\and
David A. Clifton\\
University of Oxford\\
\and
Jun Xiao \\
Zhejiang University\\
}

\begin{document}
\maketitle

\begin{abstract}
Despite the groundbreaking success of diffusion models in generating high-fidelity images, their latent space remains relatively under-explored, even though it holds significant promise for enabling versatile and interpretable image editing capabilities. The complicated denoising trajectory and high dimensionality of the latent space make it extremely challenging to interpret. Existing methods mainly explore the feature space of U-Net in Diffusion Models (DMs) instead of the latent space itself. In contrast, we directly investigate the latent space via Singular Value Decomposition (SVD) and discover three useful properties that can be used to control generation results without the requirements of data collection and maintain identity fidelity generated images. Based on these properties, we propose a novel image editing framework that is capable of learning arbitrary attributes from one pair of latent codes destined by text prompts in Stable Diffusion Models. To validate our approach, extensive experiments are conducted to demonstrate its effectiveness and flexibility in image editing. We will release our codes soon to foster further research and applications in this area.


%Alternatively, existing methods explore the feature map of U-Net (denoted as $h$-space) instead of understanding the Gaussian noised latent space directly. In contrast to them, we directly investigate the diffusion latent space and discover it could have an interpretable and semantic structure under Singular Value Decomposition operations. 

%Surprisingly, this approach empowers the diffusion latent space to have excellent editing properties same as the $h$-space and provides more theoretical substantiation and advanced benefits. Compared with $h$-space, it enables more efficient and controllable image editing without extra training data and learns more diverse visual attributes from one single image. we propose a novel image manipulation method based on this latent space and conduct extensive experiments, demonstrating its effectiveness across a range of tasks and datasets. Remarkably, this work is the first exploration to reveal the interpretability of diffusion latent space due to our best knowledge. To foster further research and applications in this area, we will release our project soon. 
\end{abstract}



\section{Introduction}
Diffusion models (DMs)~\cite{ho2020denoising,song2020denoising,song2020score, dhariwal2021diffusion,nichol2021improved, yang2023diffusion, croitoru2023diffusion, zhang2023text} have shown tremendous achievements in various computer vision tasks due to their ability to model complex data distributions and generate high-fidelity images. They have garnered significant attention and are extensively employed across a broad spectrum of academic and practical applications, including Text-based Image Generation~\cite{chen2024textdiffuser, xue2024raphael, zhu2023conditional, saharia2022photorealistic, ho2022cascaded}, Inverse Problems~\cite{cao2024high, li2022srdiff, amit2021segdiff, baranchuk2021label, whang2022deblurring, huang2005inverse, wunsch1996ocean}
%(medical imaging~\cite{cao2024high}, super resolution~\cite{li2022srdiff}, segmentation~\cite{amit2021segdiff, baranchuk2021label}, deblurring~\cite{whang2022deblurring}, weather prediction~\cite{huang2005inverse}, oceanography~\cite{wunsch1996ocean}, etc),
and especially Image Editing~\cite{kawar2023imagic, ruiz2023dreambooth, bar2023multidiffusion, zhang2023sine, yang2023paint}. 


Despite their advanced achievements, the latent space of the DMs has yet to be thoroughly investigated by existing researchers, although it is fundamentally crucial to image manipulation and synthesis. Straightforward editing on it usually leads to undesired results due to the complicated denoising trajectory in the diffusion inversion process. To address this problem, existing works ~\cite{dhariwal2021diffusion, ho2022classifier, kawar2023imagic, ruiz2023dreambooth, bar2023multidiffusion} tend to use classifier guidance technique or fine-tune methods (such as controlnet~\cite{zhang2023adding, zhao2024uni} and LoRAs~\cite{gandikota2023concept, hartley2024domain, pascual2024enhancing}) to manipulate the image generations.
Compared to these works on DMs, researches on GANs~\cite{radford2015unsupervised, voynov2020unsupervised,shen2020interpreting, chen2016infogan,mukherjee2019clustergan} have shown a similar manipulation on image generation but utilizing discovered semantic attribute vectors on their latent space. Thus their manipulation needs much less computational overload and performs with efficiency and flexibility.

%n effective manipulation on their latent space by discovering semantic attribute vectors on it. However, such 


%Due to the complicated trajectory during the denoising process, the latent space $\mathcal{X}$ could not be directly edited to manipulate the final results by simply moving along an axis. Straightforward editing of the latent codes often results in undesired and awkward images. 

%Alternatively, the existing diffusion works~\cite{dhariwal2021diffusion, ho2022classifier, kawar2023imagic, ruiz2023dreambooth, bar2023multidiffusion} tend to use classifier guidance technique or fine-tune methods (such as controlnet~\cite{zhang2023adding, zhao2024uni} and LoRAs~\cite{gandikota2023concept, hartley2024domain, pascual2024enhancing}) to manipulate the image generations. Although they are capable of performing certain types of modifications, they usually need to re-train or fine-tune the diffusion models and collect related training datasets. These editing methods are still limited due to a fundamental lack of understanding of the latent space. This restricts the flexibility and interpretability of image editing, thereby hindering the full potential of diffusion models in relevant applications.


%Compared to DMs, the latent space of Generative Adversarial Networks (GANs) has been well explored and shown that they can be effectively utilized for editing purposes~\cite{radford2015unsupervised, voynov2020unsupervised,shen2020interpreting, chen2016infogan,mukherjee2019clustergan}. Due to the deep understanding of latent space $z$ in GANs, a more flexible and explainable generation process is discovered for image editing. For example, incorporating a weighted direction vector $\alpha v$ via $z+\alpha v$ is capable of editing generated images in various ways, such as interpolations, style transfer and disentanglement analysis. This discovery significantly enlarges the utility of GAN models.  

%Their low-dimensional latent space can be semantically understood, thus it offers a more flexible and explainable generation controlling methods such as interpolations, style transfer and disentanglement analysis. One of the key discoveries indicates that by incorporating a vector $v$ into the noise $z$, the model can linearly manipulate the generated image. The understanding of latent space significantly enlarges the utility of GAN models. 


This has sparked interest in exploring whether the latent space of DMs can be similarly harnessed. Recently, researches~\cite{kwon2022diffusion, park2023understanding, li2024self, yue2024exploring} discovered the $h$-space \cite{kwondiffusion} (i.e. feature maps of the U-Net bottleneck in DMs), has demonstrated promising semantic editing. With the help of this auxiliary space, Park \textit{et al.}~\cite{park2023understanding} utilized a pullback metric in Riemannian geometry to find semantic attribute vectors on latent space $\mathcal{X}$ to manipulate image generation, and Li \textit{et al.}~\cite{li2024self} train a learnable vector $c$ to introduce semantic attributes into generated images. However, these methods mainly focus on leveraging an auxiliary space to perform attribute manipulation rather than investigating the latent space $\mathcal{X}$ itself. As a result, they either need a data collection process or manual interpretation to identify the editing effect on limited attributes.

%\Boyan{1. Math Prove intresptable, 2. NoDATASET, 1-dshot, few-shot, 3. unlimited attributes. 4. consistency is mathematically proved  }

In light of these remarkable works, we are curious about one fundamental question: (1) Can we further explore the interpretability of latent space $\mathcal{X}$ itself and leverage it for effective and flexible image editing?

Motivated by this, we conduct a thorough investigation of the latent space $\mathcal{X}$ through a series of extensive experiments. Surprisingly, we discover that the latent space $\mathcal{X}$ in DMs shows three properties via singular value decomposition across diffusion time steps. Firstly, \textbf{small neighbourhood}, the subspaces constructed by left and right singular vectors (both are orthogonal vectors in descending order based on singular values) remain semantically similar across all time steps, which indicates arbitrary attributes destined by text prompts can be introduced into this small neighbourhood. Secondly, \textbf{attributes encoded in these singular vectors} in the form of vector values and their entangled singular values, and the residential attributes can not be changed if no new singular vectors (presenting new attributes) are added. Thirdly, \textbf{mobility in order}, assuming singular vectors always ordered along with descending of their singular values at all time steps, singular vectors accounting for attributes have mobility in orders across time steps. For example, at later time steps, those vectors responsible for coarse-grained attributes are ranked at higher places, but they will be descended to lower places at earlier time steps. For Stable Diffusion models, given a text prompt, these properties make it difficult to predict what attributes are controlled by which singular vectors at different time steps. However, they also provide an alternative way to introduce new attributes by leveraging the order mobility of singular vectors (regarded as attribute vectors in Section 3.2) between latent codes from two different time steps.    
%the main attributes (e.g., identity, gender and pose etc) destined by text prompts still remain at the relative higher places across time steps as they are the most important ones encoded by singular vectors with realtively larger singular values.}


%the latent space $\mathcal{X}$ already have an admirable semantic structure under a simple Singular Value Decomposition (SVD) operation which enables more flexible and interpretable image editing compared with $h$-space. Specifically, we selected one latent code $x_t$ at the time step $t$ and factorized it by performing the SVD. Similar to the latent space $z$ in GANs, linear editing on the singular vectors of the SVD will manipulate the generated image continuously in a semantically meaningful direction during the diffusion procedure. Furthermore, we validate it shows semantic properties like $h$-space and offers more significant benefits. Compared with existing work~\cite{kwon2022diffusion, park2023understanding, li2024self, yue2024exploring}, \textcolor{red}{the each visual attribute could be learned from just one single latent variable at a specific time step rather than collecting an entire training dataset and manipulating across all time steps.} Thus more abundant visual attributes could be learned efficiently by our method. The image editing is much easier and straightforward without complex pull metrics and extra classifiers. 

%Remarkably, the latent space $\mathcal{X}$ offers a significantly higher degree of interpretability compared to the conventional black box training methodologies. Each singular value represents meaningful visual attributes. Moreover, we introduce mathematical analysis which validates that our method is reasonable and consistent across time steps. 

Based on this observation, we designed a novel image editing framework to learn new attributes for Stable Diffusion Models, which are triggered by a pair of text prompts. For example, an original prompt "A photo of a male person" and a target prompt "A photo of a young male person" are paired to learn the attribute "young" in latent space. This is achieved via a careful design of singular vector integration between latent codes created by two prompts respectively at different time steps. Specifically, we design to integrate singular vectors decomposed from original latent code $x_{T_x}$ and target latent code $z_{T_x+\Delta \tau}$ in a special operation. We also propose an MLP network to predict singular values for re-weighting integrated attributes and loss terms to balance the trade-off expression of original attributes and introduced attributes. To validate our approach, we provide a \text{theoretical analysis} to demonstrate the fidelity of image editing for various attributes and undertake comprehensive experiments across diverse datasets. These results consistently illustrate the efficacy of our method, showcasing its ability to produce high-quality edits while preserving the identity fidelity of original ones.

%Furthermore, we designed a novel editing pipeline to transform visual attributes between a pair of target and original latent codes. For example, the target latent code offers expected attributes like ``female", ``old" or ``pose", while the original latent code contains basic content without them. \textcolor{red}{prompts}.



%We elaborate a combination of mathematical formula to integrate the attributes on the source image through the $\mathcal{X}$ latent space. Specifically, we combined the vectors of $U$ and $V$ of SVD to fuse the attributes of two images and designed a new loss function to constrain the generation process. Four loss terms are designed to guide the generation to converge in a way which mixed the attributes. \Yanran{More explanation?} To validate our approach, we undertake comprehensive experiments across diverse datasets and tasks. These results consistently demonstrate the efficacy of our method, showcasing its ability to produce high-quality edits while preserving the overall coherence of the images. 



In summary, our contributions are threefold: (1) we further explore the latent space in DMs and show three properties that they hold from the perspective of SVD on latent codes. (2) we designed a novel framework with new loss terms for efficient and flexible image editing, which performs directly on the latent space and once at a specific time step in the diffusion process. This requires no data collection process and other auxiliary spaces. (3) we provide extensive experimental validation and thorough mathematical analysis of our approach.
We believe that our new findings not only provide further insights in the latent space of DMs but also pave the way for future innovations in image manipulation. To inspire further exploration, our project will be publicly available for the research community soon. 
%Fruitful insights are discovered by our exploration and contributed to the field.

%We firstly reveal the semantic structure of latent space $\mathcal{X}$ in the diffusion models, demonstrating its superior interoperability and flexibility for image editing. This discovery opens up new avenues for image manipulation that were previously unattainable with diffusion models. (2) We designed a novel method and new loss function for efficient and flexible image manipulation and synthesis via the latent space, which drastically differs from all the existing approaches. (3) We provide extensive experimental validation and thorough mathematical analysis of our approach. Fruitful insights are discovered by our investigation and contributed to the field. We believe that this new finding not only deepens the current understanding of diffusion models but also paves the way for future innovations in generative modelling and image editing. Our code and models are publicly available for the research community to inspire further exploration and development in this promising area. 


\section{Related Work}
%In this section, we comprehensively review the existing work on the latent space of generative models and the interpretability of diffusion models. 

\textbf{Image Manipulation in Diffusion Models.} The mainstream approaches~\cite{huang2024diffusion}, which manipulate the styles, poses or semantic contents of the generated images, are categorized as training-based methods, test-time fine-tuning methods, and training and fine-tuning free methods. A typical training-based approach~\cite{kim2022diffusionclip, wang2023stylediffusion, huang2024diffstyler} introduces a pre-trained classifier (e.g., CLIP~\cite{radford2021learning}) as guidance to adjust the gradient during the diffusion process. Another fine-tuning approach attempts to fine-tune the entire diffusion model~\cite{valevski2023unitune, choi2023custom, huang2023kv}, optimize the latent codes~\cite{mou2023dragondiffusion, shi2024dragdiffusion, nam2024contrastive, yang2023magicremover} or the text-based embedding~\cite{wu2023uncovering, dong2023prompt} to manipulate the output contents. For example, Imagic~\cite{kawar2023imagic} employs a hybrid fine-tuning method to achieve non-rigid text-based image editing by finding a representative latent for the target image. Some of the training and fine-tuning free methods~\cite{kim2023user, elarabawy2022direct, huberman2024edit, gholami2023diffusion, patashnik2023localizing, park2024shape} tried to modify the attention map or the cross-attention map to manipulate outputs. Most of these works modify the generated images in an implicit way, which is based on new training data and motivated intuitively. However, our proposed method based on latent space could be more efficient and explainable without complex extra data collection and model fine-tuning. %\Yanran{I am not sure about this claim} 

\textbf{Interpretable Diffusion Models.} Although the latent space is fundamentally crucial to image manipulation and synthesis, few works have taken in-depth investigations. Some existing works~\cite{choi2021ilvr,meng2108sdedit} attempted to add explicitly guidances into latent codes to manipulate the generation results. Kwon \textit{et al.}~\cite{kwon2022diffusion} instead discovered a feature map, denoted as $h$-space, in between the bottleneck of U-Net in DMs that shows semantic correlation with text embeddings from CLIP. It can be used to learn vectors for manipulating attributes in generated images. Following this idea, Li \textit{et al.}~\cite{li2024self} designed a self-supervised approach to learn vectors in this auxiliary space for the generation of gender fairness and safe content. Park \textit{et al.}~\cite{park2023understanding} attempt to discover the local basis vectors on the latent space for editing attributes, which relies on finding the principal singular vectors from the Jacobian matrix that bridge the latent space $\mathcal{X}$ and $h$-space. Their method needs manual interpretation to discern the impact of found basis vectors. Compared to them, our exploration is steered from this auxiliary space to the latent space $\mathcal{X}$ itself, and we find properties to hold across time steps and utilize them to perform efficient and versatile editing on generated images.


%found from mapping $h$-space to latent space $\mathcal{X}$ through a pullback metric, so that the visual attributes could be linearly edited by moving along the local basis vectors of $\mathcal{X}$ directly. They exclusively choose the principal singular value from the Jacobian matrix and rely on manual interpretation to discern the impact of each component on the editing process.  




\section{Method}
\begin{figure*}[h]
    \centering
    \includegraphics[width=1\textwidth]{experiments/framework_overview.png}
    \caption{Our framework overview for image editing. (1) During the denoising process, we select one time step $T_{x}$ for introducing new attributes. (2) Two latent codes $x_{T_{x}}$ and $z_{T_x + \Delta \tau}$, guided by a pair of text prompts, is fed into our proposed AVI algorithm. (3) The AVI outputs a latent code $y_{pred}$ to replace $x_{T_x}$ to continue the denoising process with the guidance of the original text prompt. Note that the SVD is performed channel-wise.}
\end{figure*}

\subsection{Diffusion Models}
Diffusion models (DMs)~\cite{ho2020denoising,song2020denoising,song2020score, dhariwal2021diffusion,nichol2021improved, yang2023diffusion, croitoru2023diffusion, zhang2023text} are a class of generative models that learn to generate data by simulating a denoising process. In this process, noise is incrementally added to the data, transforming it into pure noise. The model is trained to reverse this, learning to iteratively denoise the data and generate new samples that match the original distribution. Specifically, starting from random noise $x_{T} \in N(0,I)$ the model iteratively subtracts estimated noise at each time step to obtain a denoised latent code, denoted as $x_{t-1} = x_{t} - \epsilon_{\theta}(x_t, t)$, where $\epsilon_{\theta}$ represents the U-Net model of the DMs and $\epsilon_{\theta}(x_t,t)$ is the estimated noise by the model.


%In a forward process, noise is incrementally added to data over multiple time steps, creating a series of noisy versions of the data. A neural network then learns the reverse process, denoising at each step to generate realistic samples formulated as: 
%\begin{equation}
%x_t = \sqrt{\alpha_t} \, x_0 + \sqrt{1 - \alpha_t} \, \epsilon \nonumber
%\end{equation}
%where $\alpha_t$ controls the noise schedule, and $\epsilon \in N(0, I)$ is Gaussian noise.
The training objective is to minimize the discrepancy between the noise predicted by the model and the actual noise introduced at each time step. This optimization function can be formulated as:
\begin{equation}
\mathcal{L}_{\text{diffusion}} = \mathbb{E} \left\| \epsilon_{t} - \epsilon_\theta(x_t, t) \right\|^2 \nonumber
\end{equation}
where $\epsilon_{t} \in N(0,I)$ denotes the ground truth noise added at time step t in the forward process.



\subsection{\ouralgorithm{}}
Taking an original text prompt and a target prompt as a pair, our image editing task aims to generate a new synthesis image which similar to the image $I_x$ guided by original text prompt in terms of pose and content and contains the visual attributes provided by image $I_z$ guided by target prompt. The visual attributes could be gender, action or detail descriptions such as ``smile", ``female" and ``old". In this section, we describe the details of our Attribute Vector Integration (AVI) based Image Editing method and discuss the theoretical analysis. 


% Image editing in DMs involves transforming an original image, $I_x$, toward a target image, $I_z$, based on specified editing prompts $p$. Existing methods \cite{pullback,li2024self} \Wang{more citation?} accomplish this by leveraging the feature maps of bottlenecks in U-Net denoted $h$-space \cite{kwondiffusion}. In this work, we instead steer attentions to discover the potential of singular values from SVD for image editing in DMs.
%steering diffusion models to blending the original and target images within an embedding space, managed by an encoder-decoder pair, $\text{Enc}(\cdot)$ and $\text{Dec}(\cdot)$. In this work, we produce images by compiling the singular vectors from original and target embeddings with predicted \textcolor{red}{single} \Wang{singular?} values.  

%We manipulate the singular vectors to capture the representative information from two distinct embeddings for the \textcolor{red}{perspective} image synthesis. Given a singular value-like set, $S$, we generate a perspective \textcolor{red}{image embedding}\Wang{use latent better than image embedding?}, 

\subsubsection{Attribute Vector Integration}
We discover that the representative information of attributes is captured by singular vectors decomposed from the SVD of latent codes. The generated output image could be reconstructed from an SVD formulation $\hat{y} = \hat{U} \cdot S \cdot \hat{V}$ so that our objective is to find the reasonable left and right orthogonal matrixes $U$ and $V$, the singular value vector $S$ which could fuse the visual attributes of $I_x$ and $I_z$. Motivated by this, we first sample one latent vector $x$ at time step $T_x$ and another latent vector $z$ at time step $T_x + \Delta \tau$ representing $I_x$ and $I_z$ respectively. 

The following formulation could be written:

% Given a vector $S$ representing the singular values, we aim to construct a corresponding latent code, $\hat{y}$ in the form $\hat{y} = \hat{U} \cdot S \cdot \hat{V}$, where $\hat{U}$ and $\hat{V}$ denote left and right orthogonal matrixes, respectively. To distill this representative information in a vector form, we begin by applying singular value decomposition (SVD) on both latent code $x$ and $z$ from $Enc(I_x)$ and $Enc(I_z)$,  
\begin{align}
U_x S_x V_x = \text{SVD}(x),\,\,\, U_z S_z V_z = \text{SVD}(z) \nonumber
\end{align}

We construct the left attribution vector $\hat{U}$ by 
\begin{align*}
\hat{U} \gets [U_{x[:,:k]}, U^{'}_{z[:,:k]}]
\end{align*}
where $[:k,:]$ denotes selecting the top $k$ column vectors based on the magnitude of their singular values. Notably, the matrix $U^{'}_{z[:,:k]}$ is organized in a reverse column order relative to $U_{z[:,:k]}$, which is concatenated column-wise following $U_{x[:,:k]}$. Similarly, we construct the right attribute vectors but row-wise by
\begin{align*}
\hat{V} \gets  \begin{bmatrix} V_{x[:k,:]} \\ V^{'}_{z[:k,:]} \end{bmatrix}
\end{align*}
Following the construction of the attribute vectors, we apply a singular value-like set $S$ to re-weight the contributions of the attribute columns, thereby generating a latent code $\hat{y}$ by optimizing the primary objective function.
%\begin{align*}
%\mathcal{L}_{1}(\hat{y}, z) = \left\|\hat{y} - z \right\|^2_F, \,\,\, \hat{y} = %\hat{U} \cdot S \cdot \hat{V}
%\end{align*}

\begin{align*}
\mathcal{L}_{1}(\hat{y}, z) = \left\|\hat{y} - z \right\|^2_F, \,\,\, \hat{y} = \hat{U} \cdot S \cdot \hat{V}
\end{align*}
However, optimizing the synthetic latent code $\hat{y}$ solely toward the target latent code $z$ results in information loss from the original latent code $x$, potentially degrading output identity from $x$. To mitigate this, we enforce that the produced latent code retains an information block primarily dominated by
\begin{align*}
\mathcal{L}_{2}(\Tilde{y}, x) = \left\|\Tilde{y} - x \right\|^2_F, \,\,\, \Tilde{y} = \Tilde{U} \cdot (S + \Delta s) \cdot \Tilde{V}
\end{align*}
where $\Tilde{U}$ and $\Tilde{V}$ are sorted according to the reverse order of $\hat{U}$ and $\hat{V}$ in the column and row respectively. We will discuss the process of predicting $S$ and $\Delta s$ in the later section.
Due to the concatenation design, the issue of forgetting in editing arises as the generated images tend to lose information from the original image while also implicitly hindering optimization towards the target. This phenomenon can be understood by examining the distance between attribute vectors and the source singular value vectors. Please see the supplementary for proof of Theorem 3.1.


\begin{restatable}{theorem}{eign_distance}  \label{thm:eign_distance}
Given $x, z \in \mathcal{X}$ and their corresponding SVD, $U_x,S_x,V_x = \text{SVD}(x)$ and $U_z, S_z, V_z = \text{SVD}(z)$ where $U_x, \text{ and } U_z \in R^{M\times N}$. Let $k = \frac{N}{2}$, then the distance between attribute vectors $\hat{U}$ and its source singular value vectors $U_x$ and $U_z$ satisfy:
\begin{align*}
\left\| \hat{U} - U_x \right \|  \leq \left\| \hat{U} - U_z \right\|
\end{align*}
where $\hat{U} = [U_{x[:,:k]}, U^{'}_{z[:,:k]}]$. \\
%Proof: \text{Assume} $\sigma^{U_x}_{max} \leq \sigma^{U_z}_{max}$ where $\sigma^{U_x}_{max}$ and $\sigma^{U_z}_{max}$ denotes the maximum singular values from $U_x$ and $U_z$, respectively. $\left\|  \hat{U} - U_x \right\| \leq \left\|  \hat{U} \right\| + \left\| U_x \right\|$ \leq \left\|  \hat{U} \right\| + \sigma^{U_x}_{max}$, $\left\|  \hat{U} - U_z \right\| \leq \left\|  \hat{U} \right\| + \left\| U_z \right\|$ \leq \left\|  \hat{U} \right\| + \sigma^{U_z}_{max}$.
\end{restatable}

\noindent Proof: 
\newline
\noindent Given 
\begin{align*}
\sigma^{U_x}_{max} \leq \sigma^{U_z}_{max} 
\end{align*}
where $\sigma^{U_x}_{max}$ and $\sigma^{U_z}_{max}$ denotes the maximum singular values from $U_x$ and $U_z$, respectively. The following inequality holds:
\begin{align*}
\left\|  \hat{U} - U_x \right\| \leq \left\|  \hat{U} \right\| + \left\| U_x \right\| \leq \left\|  \hat{U} \right\| + \sigma^{U_x}_{max}, 
\end{align*}
then:
\begin{align*}
\left\|  \hat{U} - U_z \right\| \leq \left\|  \hat{U} \right\| + \left\| U_z \right\| \leq \left\|  \hat{U} \right\| + \sigma^{U_z}_{max}. 
\end{align*}
\hfill\qedsymbol

Figure x shows that the maximum singular values from $U_x$ and $U_z$ increases along with the time steps, which indicates the assumption in the proof is satisfied. 

%\Wang{I get it. this paragraph aims to explain why the reverse operation on attribute vector construction}
One may notice that due to this efficient attribute vector integration design, the target semantic contribution merged into the reconstructed original latent code $x$, is limited by the magnitude of the tail of singular values. The same design on the reconstructed target latent code $z$ shares this similar property. The reason behind this design is located in the attribute difference between $x$ and $z$ as they are selected from different time steps, and the order of attribute vectors floats across time steps, thanks to the mobility property. Specifically, assuming attribute vectors are always ordered along with the descending of their singular values, then attribute vectors at the higher places of the order in later time steps will be slightly descended to the lower places at earlier time steps \cite{yue2024exploring}, meanwhile the attribute vectors at the lower places of the order in later time steps will be slightly ascended to higher places at earlier time steps. Thus $x$ is selected at a later time step and the order of its current main attribute vectors will be descended in the earlier time steps, while merged attribute vectors from, $z$ picked at an earlier time step,  will be ascended, which leads to the merged attributes expressed sufficiently in results.

%and $z$ is picked at a earlier time step in diffusion process.

%%{when reconstructing the original latent code with the semantic contribution from target one is limited by the \textcolor{red}{significance or magnitude} of the tail singular values, while the same process on target latent code reconstruction, this property is applied inversely. Additionally, $x$, $z$ are embedded in the different time steps and this leads to the attribute difference between x and z in terms of singular value order~\cite{attribute_chage}, while our design avoids directly aggregating which may lead to an attribute mismatching issue.} 
%Notably, with this efficient attribute vector design, the semantic contribution from the target embedding to the original image embedding reconstruction is limited by the lower-magnitude singular values. Conversely, in the reconstruction of the target embedding, this property is applied in reverse. Additionally, as $x$ and $z$ are embedded at different time steps, there exists an attribute discrepancy between them in terms of singular value order~\cite{attribute_change}. Our design circumvents direct aggregation, which could otherwise lead to attribute mismatching.

\subsubsection{Singular Value Prediction.} 
We assume that the necessary information for generating the synthetic image can be efficiently accessed through the singular vectors of the original and target latent codes. However, the primary challenge lies in devising a coherent approach to blend these singular vectors to capture the attribute information from both latent codes. To address this, we aim to train a neural network, $\Phi$, to produce a singular value-like matrix $S$ that blends these singular vectors, allowing the composited latent codes to inherit attributes from both the original and target ones. Directly predicting weights for this singular-vector space composition poses challenges in neural network training, as it requires balancing the trade-offs necessary for generating the desired latent code. To ease this task, we introduce an auxiliary output branch that generates a small adjustment term, $\Delta s$, which compensates for any prediction error in the singular values of $x$. This adjustment helps optimize learning while keeping the latent codes on target. Additionally, we introduce two regularizers: one to limit the distance between $S$ and $S_z$, and another between $S + \Delta s$ and $S_x$, ensuring accurate representation from both source and target perspectives. We assume that predicting singular values for $z$ involves no substantial differences:
\begin{align*}
\mathcal{L}_{3}(S, S_z)&= \left\| S - S_z \right\|^2_F \\
\mathcal{L}_{4}(S+\Delta s, S_x)&= \left\| S + \Delta s - S_x \right\|^2_F
\end{align*}
where we use square of Frobenius Norm, $ ||\cdot||_F$, to measure the distance between two matrices. By doing this we implicitly regularise the decreasing order prosperity in $S$ to align with the singular vector orders. In summary, the final objective function for learning $\Phi$ is defined as follows: 
\begin{align*}
\mathcal{L}_{AVI}(\phi) := & \mathcal{L}(\phi, \hat{y},\Tilde{y}, S, S_t, \Delta s) \label{eq:main_loss} \\ 
 %= &  \lambda_1 \left\|\hat{y} - z \right\|^2_F  +  \lambda_2 \left\| \Tilde{y}-  x\right|^2_F \nonumber \\ 
%  & + \lambda_3 \left\|S - S_z \right|^2_F \nonumber + \lambda_4 \left\| S + \Delta s - S_x \right\|^2_F \nonumber
 = & \lambda_1 \mathcal{L}_{1}(\hat{y}, z) + \lambda_2 \mathcal{L}_{2}(\Tilde{y}, x) \\
   & + \lambda_3 \mathcal{L}_{3}(S, S_z) + \lambda_4 \mathcal{L}_{4}(S + \Delta s, S_x)
\end{align*}
where $\lambda_i$ are the hyperparameters to balance the strength of each learning term. The general idea of the training and inference is shown in Alg 4 (please see the supplementary for details) with yielding $y_{pred}$ as a result, fed into the diffusion model for further image generation. Theorem 3.1 can also be applied to the inference phase, which demonstrates output $y_{pred}$ preserves better identity fidelity from $x$ than $z$. (please see the supplementary for details). Note that due to the property of the $\Phi$ that $y_{pred}$ stays in the manifold of latent space, thus our proposed MLP network is model-agnostic.
%governed by the encoder and decoder pair, neither the diffusion model nor the pair of the encoder and decoder are required for adaptation.

%The general idea of the training and inference is shown in Alg~\ref{alg:training} with yielding $y_{pred}$ as a result, fed into the diffusion model for further image generation. Note that due to the property of the $\Phi$ that $y_{pred}$ stays in the manifold of latent space governed by the encoder and decoder pair, neither the diffusion model nor the pair of the encoder and decoder are required for adaptation. 

\cut{
To achieve this, a nature selection is to conduct Singular Value Decomposition (SVD) on both embeddings,
\begin{align*}
U_x, S_x, V_x = \text{SVD}(x), \,\,\, x = \text{Enc}(I_x)  \\
U_z, S_z, V_z = \text{SVD}(z), \,\,\, z = \text{Enc}(I_z)  
\end{align*}
and minimising the distance by optimising the SVD decomposition components of $I_x$ towards that of $I_z$ to aggregate the desired attributes. However, this approach may excessively alter the content in $x$, leading to catastrophic forgetting of essential original information and degrading the quality of the generated images. Additionally, it requires frequent querying of the optimization process during denoising $x_t$ to prevent deviations from target attributes~\cite{}. To address these challenges, we propose an intermediate space, generated through a learnable mapping $\Phi_{\phi}: \mathcal{X} \rightarrow \mathcal{Y}$ which bridges the SVD decomposition between two image information sources. This helps mitigate the forgetting problem, and importantly, we train $\Phi_{\phi}$ in a time step-agnostic manner, enabling single-step editing to reduce computational overhead. For simplicity, we denote this mapping as $\Phi$.}
\begin{algorithm}[t]
\caption{Attribute Vector Integration}\label{alg:eigen_man}
\begin{algorithmic}[1]
\Require Original latent code $x$, target latent code $z$, MLP network $\Phi$, top $k$, ratio $\rho$, and stage type.
\Ensure $\hat{y}, \Tilde{y}, S_x, S, \Delta s$ or $\hat{U}, \hat{V}, S, S_x, \Delta s$
\State $S, \Delta s \gets \Phi(x)$
\State $U_x, S_x, V_x = \text{SVD}(x)$ 
\State $U_z, S_z, V_z = \text{SVD}(z)$
\State $U_z^{'} \gets U_{z{[:, ::-1]}}$ \Comment{Reverse column order}
\State $V_z^{'} \gets V_{z{[::-1,:]}}$ \Comment{Reverse row order}
\State $\hat{U} \gets [U_{x[:,:k]} , (1-\rho) \cdot U_{x[:,k:]} + \rho \cdot U^{'}_{z[:,:k]}]$ \Comment{Concatenate columns}
\State $\hat{V} \gets  \begin{bmatrix} V_{x[:k,:]} \\ (1-\rho) \cdot V_{x[k:,:]} + \rho \cdot V^{'}_{z[:k,:]} \end{bmatrix}$ \Comment{Concatenate rows}
\State $\hat{y} = \hat{U} \cdot S \cdot \hat{V}$
\State $\Tilde{U} \gets \hat{U}_{[:, ::-1]}$ \Comment{Reverse column order}
\State $\Tilde{V} \gets \hat{V}_{[::-1,:]})$ \Comment{Reverse row order}
%\State $\Tilde{y} = \Tilde{U} \cdot S \cdot \Tilde{V}$
\State $\Tilde{y} = \Tilde{U} \cdot (S + \Delta s) \cdot \Tilde{V}$  
\If{stage is Training}
\State Return $\hat{y}, \Tilde{y}, S_x, S, \Delta s$
\ElsIf{ stage is Inference}
\State Return $\hat{U}, \hat{V}, S_x, S, \Delta s$
\EndIf
\end{algorithmic}
\end{algorithm}

%\subsection{Inference with Time step selection} \Wang{this section will be discussed in supplementary as hyperparameter choice, so remove this subsection}
%Need more details from Wangli. Please try to use Algorithm 4 to edit the current inference algorithm. \Wang{already re-write the algorithm 4 with algorithm2}
\begin{figure*}[t]
    \centering
    \includegraphics[width=0.97\textwidth]{experiments/CatCelebahq_2.png}
    \caption{Impact of singular values on their main singular vectors on Unconditional Diffusion Models (CelebA-HQ dataset on the left and LSUN-Cat and LSUN-Church datasets on the right). Fine-grained attributes, such as colours and texture appearances, are changing with respect to their singular values at earlier diffusion time steps.}
\end{figure*}

\section{Findings}
In this section, we present the findings about attributes and their control capabilities within DMs. And we also dive into the properties that are held across all time steps in the denoising process.

\subsection{Attributes in Diffusion Models}

\textbf{Unconditional Diffusion Models}
In Figure 2. the left two columns present the impacts of singular values on the main singular vectors (ones with maximum singular values). We observe that at later time steps (e.g., 0.6T and 0.5T) during the denoising process (using DDPM), the slight decrease of singular values will lead to coarse-grained attributes changing, for example, gender transition and ageing. Further, the fine-grained ones, such as makeup and face shape, usually change at earlier time steps (e.g., 0.4T and 0.3T). This observation is aligned with the connections between diffusion time steps and hidden attributes \cite{yue2024exploring}. The right two columns of Figure 2 illustrate a similar observation on other unconditional diffusion models (e.g., LSUN-Cat and LSUN-Church datasets). Specifically, the fine-grained attributes, such as colours and texture appearances, are changing along with the decreasing of singular values at earlier time steps (e.g., 0.4T and 0.3T). These attributes and their effects by singular values are observed at a great chance for most samples on Unconditional Diffusion Models (e.g., CelebA-HQ, LSUN-cat, and LSUN-church datasets), as each of them shares similar structure distribution within latent space. In this work, we leave unconditional diffusion models to future exploration, but we instead delve into the conditional diffusion models (e.g., Stable Diffusion Models), as their latent spaces have a less similar structure distribution.

\textbf{Text-to-Image Diffusion Models}
We continue to investigate the impact of singular values and their singular vectors obtained from SVD of latent codes across diffusion time steps. Surprisingly, similar effects of attribute modification are observed as well. Figure 3 presents the found attributes of Text-to-Image Stable Diffusion Models (ver 2.1). The coarse-grained attributes, for example, gender, pose, background, motion, point views and appearance style, are usually observed at later diffusion time steps (e.g., 0.8T and 0.7T) during the denoising process (both DDPM and DDIM). Additionally, the fine-grained ones shown in Figure 3, such as belt textures, are observed at earlier time steps. 
Another observation is that the attributes that singular vectors are responsible for seem to be random and the residential attributes can not be changed. For example, the first row in Figure 4 shows that attributes like gender and wrinkles can be adjusted, however, this does not happen in the second row. This indicates that the adjustable attributes are related to latent codes guided by text prompts. And new attributes need to be added explicitly.


%\textcolor{red}{Surprisingly, we observe various attributes based on the singular vectors decomposed from Singular Value Decomposition. It would benefit to image editing if these discovered attributes are mastered. However, since the arbitrary eigenspace decomposed from svd, it is hard to directly apply found singular vectors to other images. Thus we propose to fuse singular vectors to learn attributes.}

\begin{figure*}[t]
    \centering
    \includegraphics[width=1\textwidth]{experiments/Analysis_Attributes_crop.png}
    \caption{Representative examples shown on the attributes that one single singular vector affects across the time steps in Stable Diffusion Models (ver-2.1). Starting from the second column, the rows show the impact of singular values, and the columns present the attributes that different singular vectors may affect. Texts on the left side of images denote the attributes that a singular vector affects. It is noticeable that attribute vectors (e.g., belt details) ordered in lower places at later time steps (e.g, last row under 0.7$T$) ascended to higher places at earlier time steps (e.g, 0.6$T$)}
\end{figure*}

\subsection{Analysis of singular vectors and their singular values in Latent Space}
To further understand where the information of attributes is encoded, we conduct extensive experiments to trace them across all time steps. We discover that this information is encoded in the values of singular vectors and their magnitude of change is encoded in their paired singular values. For example, the values of singular vectors and their paired singular values slightly increase from $x_{T}$ to $x_{0}$, which also makes sense since the latent code $x_t$ gradually grows from random noise to real image (Please see supplementary for details).

In addition, we also observe that the order of singular vectors changes across diffusion time steps when assuming their orders always follow the descending of their singular values. For example, the singular vectors that affect coarse-grained attributes (e.g., identity, gender, pose, age, and clothes etc) are ordered at higher places with larger singular values at later time steps (e.g., 0.9T and 0.8T). However, at earlier time steps (e.g., 0.7T - 0.5T), those responsible for coarse-grained attributes are descended to lower places, but those for fine-grained attributes (e.g., glasses, eye pose, fatness, and colours etc.) are ascended to higher places. We refer to this as \textbf{mobility} property. This observation is also aligned that DMs generate samples in a coarse-to-fine manner \cite{pullback,yue2024exploring}. Further, we also observe that extracting the one single singular vector responsible for a semantic attribute and then using it to replace one single singular vector from another latent code will not introduce this semantic attribute. As the extracted singular vectors will be disrupted. (Please see the supplementary for details.)


%\textbf{information of attributes are contained in singular vectors and their expression strength is controlled by their singular values, order of singular vectors are changing along with time steps, if only one singular vector from one latent code composed into another latent code, the information will diluted and not express possible at all. and a coarse-grained attributes are embedded into multiple singular vectors, thus only fuse one singular vectors may fail.}

Figure 4 presents the attributes that one single main singular vector that is capable of manipulating across diffusion time steps. It is noticeable that one single singular vector affects coarse-grained attributes at 0.9T time step, which may involve several fine-grained attributes. For example, the coarse-grained attribute identity is affected by multiple singular vectors (e.g., UV-15 and UV-31) which involves fine-grained attributes such as suit style, face and body pose. Across time steps, one single singular vector tends to be disentangled with other fine-grained attributes. These singular vectors with higher places in order at earlier time steps are ascending from latter places at previous time steps, which also denotes that those singular vectors of coarse-grained attributes have been descended in order. (Please see the supplementary for more details). This is the very reason that makes it possible to learn coarse-grained attributes from the latent codes with interval time steps, as latent codes at earlier time steps already contain that information in the lower places of singular vectors. The order reverse of singular vectors of latent codes at earlier time steps will put these coarse attributes to higher places again, and their magnitude will assigned with larger singular values when the predicted singular values are in descending order.


\subsection{Analysis of Geodesic Distance of Subspaces Constructed by Singular Vectors}
\textbf{The discrepancy of subspaces constructed from singular vectors maintains semantically similar in a small neighbourhood along with the generative process.} To investigate the geometry of the subspaces constructed by singular vectors, a distance metric on the Grassmannian manifold is employed to measure the distance between two subspaces. Each point on the Grassmannian manifold is a vector space, and the metric defined on it represents the distortion among various vector spaces. We employ geodesic metric \cite{choi2021not,xu2023open} to measure the discrepancy between two subspaces, which is related to the principal angles between two subspaces spanned by their columns:

\begin{align*}
d(A,B) = \sqrt{\Sigma^{p}_{k=1} \theta^{2}_{k}}
\end{align*}
where $\theta_{k}$ denotes the principal angles between two subspaces spanned by $A$ and $B$, and $p$ is the dimensionality of the subspace (in our case, $p=4$). Here the two subspaces are represented by singular vectors decomposed from the SVD of two latent codes along the generative process.

\begin{figure}[t]
    \centering
    \includegraphics[width=0.48\textwidth]{experiments/Attributes_sd2.png}
    \caption{Impact of singular values on their main singular vectors on Stable Diffusion Models (ver 2.1).}
\end{figure}

Figure 5 demonstrates the average geodesic distance from subspaces constructed by singular vectors (30 samples) along the diffusion time steps on various datasets. It is noticeable that Unconditional Diffusion Models trained on datasets (e.g, LSUN-Cat, LSUN-Bedroom and LSUN-Church) share similar geodesic distances around 0, and Text-to-image Stable Diffusion Models (ver 2.1) with text prompt "A photo of a celebrity." and Unconditional DM on CelebA-HQ datasets share similar geodesic distance of subspace, which is around $4.3 \times 10^{-4}$. The variance of Stable Diffusion Models is less than that trained on the CelebA-HQ dataset, thus the subspaces constructed by singular vectors are in a small neighbourhood across the time steps, which also indicates that the operations on singular vectors still remain semantically similar across time steps. 


\begin{figure}[t]
    \centering
    \includegraphics[width=0.5\textwidth]{experiments/Geodesic_distance.png}
    \caption{Geodesic Distance across subspaces constructed by singular vectors of different datasets at various diffusion time steps. It is noticeable that the variance of stable diffusion is less than diffusion models trained on the CelebA-HQ dataset, thus we tend to consider the geodesic distance on subspaces to maintain semantically similar across all time steps.}
\end{figure}
\section{Experiments on Learning Attributes}
\textbf{Experiments settings}
%\Yanran{Write the size and training parameter of the setting }
The proposed framework is built on Pytorch 2.1 with CUDA 12.0 in the Ubuntu 20.04 LTS system. The MLP network contains three fully connected linear layers and ReLU layers followed by two branches, in which only one linear layer is contained. The input dimension for all linear layers is fixed to 4096, and the output dimension of the last linear layer is fixed to 64, which is the same dimension as the singular value set. $k$ for the singular vector subset selection is set as 32 for our entire experiments. The hyperparameters are set as followings: for data generation, $T_{x} = 0.8T$, $\Delta \tau = -0.3T$,  $T_{z} = T_{x} + \Delta \tau$, where $T=1000$ denotes the total number of denoising time steps. the sampling number $N$ is set to 5000 when provided only one pair of latent codes, and $N$ is set to 500 when provided over 5 pairs; for training phase, batch size is set $256$, $\rho=1$, $\lambda_{1}=3$, $\lambda_{2}$, $\lambda_{3}$ and $\lambda_{4}$ are set to 10; for inference phase, $\rho$ is customized by users. The training phase uses Adam \cite{kingma2014adam} as optimizer with learning rate $1 \times 10^{-3}$. It usually reaches convergence within 5 epochs which takes around 5 mins in one single RTX 3090 Graphics Card. We set the number of steps to 50 with DDIM inversion \cite{song2020denoising} for generating real images using the Stable Diffusion Model (ver 2.1). All methods in comparison use Stable Diffusion Model version 2.1. The prompts used in all experiments follow a similar pattern that the target text prompts add or remove certain keywords from the original ones. For example, the original prompt is "A realistic portrait of a teacher", then the target prompt is "A realistic portrait of a \textcolor{red}{male} teacher". And the original prompt is "A photo of a dilapidated castle", then the target prompt is "A photo of a castle".  

\begin{table*}[t]
    \caption{Quantitative Evaluation on four attributes. The metrics under each attribute are: FID $\downarrow$, CLIP $\uparrow$ score, and LPIPS $\downarrow$.}
    \centering
    \normalsize
    \label{table1}
    \begin{adjustbox}{width=0.8\textwidth}
    \begin{tabular}{|c|c|c|c|c|c|} % 更新列定义,删除多余的列
        \hline
        Method & Female & Male & Old & Young  & Runtime\\ 
        \hline
        SD~\cite{rombach2022high}   & 31.75 / 26.6  / 0.34 &  13.73 / 25.2 / 0.46 & \underline{16.47} / 24.7  / 0.39 & 35.68 / 25.9  / 0.30 & 6.2s \\  
        Park \textit{et al.} \cite{park2023understanding} & 29.27 / 25.4 / 0.07 & 16.74 / 23.2 / 0.16 & 17.06 / 23.7 / 0.18  & 37.35 / 23.9 / \underline{0.09} & 11.5s\\
        Ours & \underline{19.69} / \underline{28.4}/ \underline{0.33} & \underline{10.08} / \underline{25.4}/ \underline{0.14} & 16.93 / \underline{26.1} / \underline{0.11} & \underline{34.54} / \underline{27.59}/ 0.12 & 6.7s\\
        \hline  
    \end{tabular}
    \end{adjustbox}
\end{table*}

\textbf{Attributes Learned}
%\Yanran{1. demonstrate the visual concept we learned. 2. Interpolation 3. Composition 4. how the visual effects are when using different concepts. } \checkmark
Figure 6 presents the several attributes learned via our proposed method, and it is noticeable that the structure information (e.g., identity) of the original latent code is preserved well and expected attributes are learned to alter in generated images.
\begin{figure}[h]
    \centering
    \includegraphics[width=0.48\textwidth]{experiments/Attributes2.png}
    \caption{Examples of image edition on various learned attributes.}
\end{figure}

\textbf{Comparison to SOTA methods}
Table 1 demonstrates the performance between our method and other state-of-the-art methods ~\cite{park2023understanding,rombach2022high} on attribute manipulation. Lower FID $\downarrow$ values indicate the results present better image quality, higher CLIP $\uparrow$ scores indicate the results are more semantically aligned with the input text, and Lower LPIPS $\downarrow$ values indicate results sharing better identity fidelity of the original images. As can be seen, our method achieves the highest CLIP scores across all four attributes, lowest FID values and the lowest LPIPS values on three attributes except attribute Old and Young, respectively. Park \textit{et al.}~\cite{park2023understanding} achieves the lowest LPIPS values on attribute Young, SD~\cite{rombach2022high} achieves the lowest FID on attribute Old. 

\textbf{Interpolation on learned attributes}
Figure 7 illustrates the impact of manipulating image attributes by linearly controlling the strength of the singular vector column construction, denoted as $\rho$ in lines 6-7 of the pseudo-code of Algorithm 1. The generated image is gradually modified to the introduced attribute by adjusting the strength $\rho$, meanwhile, the identity fidelity is maintained in the transition which indicates the introduced attributes remain approximately disentangled from other semantic factors. 
\begin{figure}[h]
    \centering
    \includegraphics[width=0.45\textwidth]{experiments/Attributes_linear.png}
    \caption{Linear interpolation on learned attributes.}
\end{figure}

\textbf{Ablation Study}
Figure 8 illustrates the proposed loss terms on the impact of the performance. Compared to the full model, the method without $\mathcal{L}_{1}$ leads to fewer attributes introduced into the results. As singular values $S$ are predicted to be unreasonable values for concatenated singular vectors from target latent code $z$. Thus the introduced attributes are not expressed with enough magnitudes. Similarly, the results are closer to images that target latent codes present when $\mathcal{L}_{2}$ is absent, as the original attributes are not expressed with enough magnitudes. Generated images without $\mathcal{L}_{3}$ leads to saturation loss, and the method without $\mathcal{L}_{4}$ causes image quality degraded. Due to the page limit, the discussion about hyperparameters such as loss term weights $\lambda$s, the choice of time step for original latent codes $T_{x}$ and $\Delta \tau$ for target latent codes will be shown in supplementary files. 
%\Yanran{A discussion of the loss why we design it as L2} 
\begin{figure}[h]
    \centering
    \includegraphics[width=0.45\textwidth]{experiments/AblationStudy_crop.png}
    \caption{Ablation study on proposed loss terms.} 
\end{figure}

\section{Discussion}
In this section, we provide some intuitions and implications. It is interesting that we observe that the subspaces constructed by singular vectors remain semantically similar across diffusion time steps within DMs. The generative process seems to involve re-arranging the order of singular vectors and slightly increasing their values and singular values but not their vector directions. Thus the attributes introduced by text prompts are sort of more related to the order of singular vectors and their magnitudes. The mystery of semantic attributes could be more clarified by analysing the mechanism of order changing of these singular vectors and singular values assigned across time steps. Our method has shown the effectiveness of attributes learned from paired latent codes with time step intervals, however, it needs manual selection to find such pairs with text prompts. And it also shows sensitivity to seeds that create latent codes. In future, a promising direction could be analysing normalized singular vectors and singular values, since their directions will remain across time steps, then learning to re-order them and re-assign their singular values for attribute editing.

\section{Conclusion}
In this paper, we further explore the latent space in DMs via applying SVD directly on it and discover three properties that singular vectors hold across diffusion time steps on various datasets. Based on this finding, we propose a novel approach for image editing with attribute vector integration on latent space without data collections and any auxiliary spaces. In addition, it only performs at one specific time step and can be theoretically analysed for fidelity of image editing. Extensive experiments demonstrate the effectiveness and flexibility of our proposed image editing method. We believe our findings and fruitful insights could facilitate future research and applications of the diffusion models. 



\clearpage
{
    \small
    \bibliographystyle{ieeenat_fullname}
    \bibliography{arxiv}
}

% WARNING: do not forget to delete the supplementary pages from your submission 
% 
\clearpage
% \setcounter{page}{1}
% \maketitlesupplementary
\begin{center}
Supplementary Material
\end{center}

% {
%     \onecolumn
%     \centering
%     \Large
%     \textbf{\thetitle}\\
%     \vspace{0.5em}Supplementary Material \\
%     \vspace{1.0em}
% }

\section{Proof of \cref{theorem:dr}}
We require some additional regularity assumptions:
\begin{assumption} 1) The number of classes $C$ is bounded w.r.t the number of samples $N$, 2) the missingness mechanism $P(A=1|Y,\theta)$, as well as its estimated counterpart $P(A=1|Y,\theta)$, are bounded below by some constant $\epsilon > 0$, 3) the quantities $P(Y|X,\theta)$ and $P(A|Y,\theta)$ are estimated using auxiliary samples independent of samples used for the sample averaging.
\label{assumption:extra}
\end{assumption}
Assumptions 1 and 2 are natural. For the missingness mechanism, the ground truth being bounded means that there is a non-vanishing proportion of samples for every class. The boundedness of the estimate can be enforced by clipping the estimate. Assumption 3 is called sample splitting in \cite{kennedy-dr}.

For convenience we use operator $\E_N$ to denote the average of $N$ samples i.e. $\frac{1}{N}\sum_{i=1}^N$. Note that this is by itself a random variable, in contrast to $\E$ which is a fixed number.

\begin{proof}[Proof of \cref{theorem:dr}] Because $C$ is bounded (assumption \ref{assumption:extra}), we can fix a class $c$ and prove the theorem.
Let us define the influence function $\phi$, parameterized by $\theta$, as
\begin{equation}
\phi(O | \theta)(c) = P(Y=c|X,\theta) + \frac{\one(A=1)}{P(A=1|Y,\theta)} (\one(Y=c) - P(Y=c|X,\theta)) - P(Y=c)
\end{equation}
As we have done in the main text, we use $\phi(O)$ to denote the same function but all estimated quantities are replaced with their truths. In other words, we use $\phi(O)$ for $\phi(O|\theta_0)$ where $\theta_0$ is the truth, given that our model contains $\theta_0$ e.g. when the model is consistent.

Recall that:
\begin{equation}
\begin{aligned}
\Psi_{dr}(\theta)(c) &= \frac{1}{N}\sum_{i=1}^N \left\{P(Y=c|X,\theta) + \frac{\one(A=1)}{P(A=1|Y,\theta)} (\one(Y=c) - P(Y=c|X,\theta))\right\}\\
&= \E_N [\phi(O|\theta)(c)] + P(Y=c)
\end{aligned}
\end{equation}

We will show that:
\begin{equation}
\Psi_{dr}(\theta)(c) - P(Y=c) = (\E_N - \E)[\phi(O)(c)] + o_P(N^{-1/2})
\label{eq:proof-linearity}
\end{equation}
To do that, we use the following decomposition
\begin{equation}
\begin{aligned}
\Psi_{dr}(\theta)(c) - P(Y=c) &= \E_N [\phi(O|\theta)(c)] \\
&= (\E_N - \E)[\phi(O)(c)] + (\E_N - \E)[\phi(O|\theta)(c) - \phi(O)(c)] + \E[\phi(O|\theta)(c)]
% &+ (\E_n - \E)[\phi(O;\theta) - \phi(O)]\\
% &+ \E[P(Y=c|X,\theta)] - \E[P(Y=c|X)] + \E[\phi(O,\theta)]
\end{aligned}
\end{equation}
and analyze the second and third term. The third term is:
\begin{equation}
\begin{aligned}
\E[\phi(O|\theta)(c)] &= \E[P(Y=c|X,\theta)] + \E\left[\frac{\one(A=1)}{P(A=1|Y,\theta)}(\one(Y=c) - P(Y=c|X,\theta))\right]- P(Y=c) \\
&= \E\left[P(Y=c|X,\theta) + \frac{P(A=1|Y)}{P(A=1|Y,\theta)}(P(Y=c|X) - P(Y=c|X,\theta))\right] - \E[P(Y=c|X)]\\
&= \E\left[(P(Y=c|X,\theta) - P(Y=c|X)) (P(A=1|Y,\theta) -P(A=1|Y)) \frac{1}{P(A=1|Y,\theta)}\right]\\
\end{aligned}
\end{equation}
by Cauchy-Schwarz inequality:
\begin{equation}
\begin{aligned}
\E[\phi(O|\theta)(c)] &\le \frac{1}{\epsilon} \|P(A=1|Y,\theta) - P(A=1|Y)\|_2 \|P(Y=c|X,\theta) - P(Y=c|X)\|_{L_2(P)}\\
&= \frac{1}{\epsilon} o_P(N^{-1/4} N^{-1/4}) = o_P(N^{-1/2})
\end{aligned}
\end{equation}
by assumption \ref{assumption:4th-root-n} and that $P(A=1|Y,\theta) > \epsilon$ (assumption \ref{assumption:extra}). The second term can be bounded by Chebyshev inequality
% \begin{equation}
% \begin{aligned}
% \E[\E_N[\phi(O|\theta)(c) - \phi(O)(c)]] &= \E[\phi(O|\theta)(c) - \phi(O)(c)]\\
% \var[\E_N[\phi(O|\theta)(c) - \phi(O)(c)]] &= \frac{1}{N}\var[\phi(O|\theta)(c) - \phi(O)(c)] \le 
% \end{aligned}
% \end{equation}
\begin{equation}
P(|(\E_N - \E)[\phi(O|\theta)(c) - \phi(O)(c)]| \ge t) \le \frac{\var[\E_N[\phi(O|\theta)(c) - \phi(O)(c)]]}{t^2} = \frac{\var[\phi(O|\theta)(c) - \phi(O)(c)]}{Nt^2}
\end{equation}
note here that $\theta$ is independent of the samples used for $\E_N$ by assumption \ref{assumption:extra}. For any $\varepsilon > 0$, by picking $t = \frac{1}{\sqrt{N\varepsilon}}$ we get
\begin{equation}
P\left(\left|\frac{(\E_N - \E)[\phi(O|\theta)(c) - \phi(O)(c)]}{N^{-1/2}}\right| \ge \frac{1}{\sqrt{\varepsilon}}\right) \le \varepsilon \var[\phi(O|\theta)(c) - \phi(O)(c)]
\end{equation}
by the definition of $O_P$, we then get
\begin{equation}
(\E_N - \E)[\phi(O|\theta)(c) - \phi(O)(c)] = O_P(N^{-1/2}\var[\phi(O|\theta)(c) - \phi(O)(c)])
\end{equation}
Because $\phi$ is a continuous function of $P(Y|X,\theta)$ and $P(A|Y,\theta)$ (given $P(A|Y,\theta) > \epsilon$, assumption \ref{assumption:extra}), by the continuous mapping theorem and the fact that $P(Y|X,\theta)$ and $P(A|Y,\theta)$ are convergent in probability (assumption \ref{assumption:4th-root-n}), we get $\var[\phi(O|\theta)(c) - \phi(O)(c)] = o_P(1)$. This gives
\begin{equation}
(\E_N - \E)[\phi(O|\theta)(c) - \phi(O)(c)] = o_P(N^{-1/2})
\end{equation}
Therefore, we have shown that the second and third term are both $o_P(N^{-1/2})$, proving \cref{eq:proof-linearity}. As the final step, multiply both sides of this equation by $\sqrt{N}$ we get:
\begin{equation}
\sqrt{N}(\Psi_{dr}(\theta)(c) - P(Y=c)) = \sqrt{N} (\E_N - \E)[\phi(O)(c)] + o_P(1) \rightsquigarrow \mathcal{N}(0, \var[\phi(O)(c)])
\end{equation}
by the central limit theorem, and $\var[\phi(O)(c)] = \E[\phi(O)(c)^2]$ because $\E[\phi(O)(c)] = 0$.
\end{proof}

While we started with the definition of $\phi$, \cref{eq:proof-linearity} shows that $\phi$ is indeed an influence function. Now we show that $\phi$ is also the efficient influence function, by using the characterization of the model's tangent space \cite{tsiatis-missingdata}. Note that the joint probability factorizes as $P(X,A,Y) = P(X)P(Y|X)P(A|Y)$, therefore the tangent space $\mathcal{T}$ factorizes as $\mathcal{T} = \mathcal{T}_{X} \oplus \mathcal{T}_{Y|X} \oplus \mathcal{T}_{A|Y}$ where $\mathcal{T}_X = \{h(X): \E[h] = 0\}$, $\mathcal{T}_{Y|X} = \{h(X,Y): \E[h|X] = 0\}$, $\mathcal{T}_{A|Y} = \{h(A,Y): \E[h|Y] = 0\}$, and the 3 subspaces are pairwise orthogonal. All influence functions are orthogonal to the tangent space, but the influence function that is also in the tangent space has the smallest variance and is called the efficient influence function. As $\phi$ is already an influence function, we need only show that $\phi$ is in $\mathcal{T}$. We write $\phi$ as
\begin{equation}
\phi(O)(c) = (P(Y=c|X) - P(Y=c)) + \left[\frac{\one(A=1)}{P(A=1|Y)} - 1\right](\one(Y=c) - P(Y=c|X)) + (\one(Y=c) - P(Y=c|X))
\end{equation}
and note that the first, second and third term are in $\mathcal{T}_X$, $\mathcal{T}_{A|Y}$ and $\mathcal{T}_{Y|X}$ respectively. Therefore, $\phi$ is indeed in $\mathcal{T}$. The efficient influence function has the smallest variance of all influence function, and therefore our estimator being asymptotically linear in $\phi$ (\cref{eq:proof-linearity}) has the smallest mean squared error in a local asymptotic minimax sense \cite{kennedy-dr, asymptoticstatistics}

\section{Further background and related work}
\paragraph{Discussion on semi-supervised EM.}
It appears that semi-supervised EM was first used for parameter estimation when the missingness mechanism is non-ignorable in \cite{ibrahim1996parameter}, but has not been used for label shift estimation.
Perhaps this is because the semi-supervised situation where additional unlabeled data is available during training is rarer than the test-time adaptation case. EM is well suited to take advantage of the extra unlabeled data to improve the classifier under very scarce and long-tailed labeled data. While the connection between pseudo-labeling and EM has been explored before \cite{entropyminimization}, the situation with label shift has not until recently \cite{simpro}. Here the application of EM is much more interesting, because other than simply giving pseudo-labeling a rigorous formulation, EM also estimates the missingness mechanism (equivalently the label distribution shift), which is important for shift correction and thus high-quality pseudo-labels \cite{acr}. The application of confidence thresholding can be seen as a sparse variant of EM \cite{neal1998view}.

\paragraph{The doubly-robust risk.} 
\label{subsec:dr-risk}
A technique that also derives from the theory of semi-parametric efficiency is orthogonal statistical learning \citep{foster2023orthogonal}. The idea is to minimize the doubly-robust risk:
\label{subsec:method-dr-risk}
\begin{equation}
\label{eq:dr-risk}
\mathcal{R}(\theta_2) = \frac{1}{N} \sum_{i=1}^N \Bigg[ l(x_i, \hat y_i|\theta_2) + \frac{\one(a_i=1)}{P(A=a_i|Y=y_i, \theta_1)} (l(x_i, y_i | \theta_2) - l(x_i, \hat y_i | \theta_2))\Bigg]
\end{equation}
where $l(x,y|\theta) = -\sum_{c=1}^C [y]_c \log P(Y=c|X=x,\theta)$ is the negative cross-entropy. 
The notation $[y]_c$ means that we are using the $c$-entry in a C-dimension probability vector $y$. 
Thus, $y_i$ denotes the one-hot label of observation $i$, while $\hat y_i$ denotes the pseudo-label, which can be one-hot or all-zero. 
Finally, we use $\theta_1$ to denote that $P(a|y,\theta_1)$ is an estimation from a previous stage, but it can be estimated with $\theta_2$ as well. 
The risk $\mathcal{R}(\theta_2)$ can be used as a training loss in a straightforward fashion. 
Similar to the doubly robust estimation of $P(Y)$, the doubly robust risk provides approximately unbiased estimation of the risk. 
This property has been used in \citep{arelabelsinformative, onnonrandommissinglabels, drst} also in the semi-supervised learning setting.
More broadly, it is at the heart of one of the core techniques in heterogenous treatment effect estimation in causal estimation \cite{kennedy2023towards, foster2023orthogonal, wager2018estimation}. 
The focus here is not the estimation of $\mathcal{R}(\theta_2)$ per se, but the quality of the learned model \cite{foster2023orthogonal}.
By using the doubly-robust risk, we can achieve an optimality result similar in spirit to our theorem \cref{theorem:dr}, but for the generalization error.
While this is appealing, in practice there are 2 problems with this approach. First, the inverse probability weight $P(A=a_i|Y=y_i,\theta_1)$ can be very large if the class ratio is highly unlabeled, making training unstable \cite{kallus2020deepmatch, pham2023stable}. 
This problem exists for our estimation as well. However, it is much easier to control for estimation than for training because of the iterative nature of model update. Secondly, we can further write $\mathcal{R}$ as:
\begin{equation}
\mathcal{R}(\theta_2) = \frac{1}{N}\sum_{i=1}^N l\left(x_i, \hat y_i + \frac{\one(a_i=1)}{P(A=a_i|Y=y_i,\theta_1)} (y_i - \hat y_i)\Bigg\vert\theta_2\right)
\end{equation}
which is a cross-entropy loss with new meta-pseudo-labels. However, these labels are not meant to be learned exactly, and furthermore they can be negative. Thus, theoretical works have to put stringent assumptions on the models. In \cref{subsec:ablation-1}, we show that experimentally that the instability problem makes doubly-robust risk performance worse than our 2-stage approach.

\section{Training and hyperparameter settings.}
\label{subsec:training-setting}
For neural network training, we follow the implementation and hyperparameter settings of \cite{simpro}. In particular, we adapt the core code of SimPro for Supervised, MLE and EM. For MLE, we update $P(A|Y)$ using the Adam optimizer with learning rate 1e-3, while for EM we use a momentum update similar to SimPro's update of $P(Y|A)$ because it has a a closed-form solution at each mini-batch. We use Wide ResNet-28-2 on all methods and all datasets in this section, including Imagenet-127, because we are motivated by the fact that stage-1's goal is not classification accuracy but the estimation of a finite-dimensional parameter. When using Wide ResNet-28-2 for Imagenet-127, we use the hyperparameters of CIFAR-100, except we lower the batch size of unlabeled data to 2 times that of labeled data instead of 8 for memory reason. We do not perform additional hyperparameter tuning. All experiments can be performed on 1 A6000 RTX GPU, and are run 3 times. We report the total variation distance between the estimated and the ground truth unlabeled class distribution, similar to its usage in Theorem 3.1 of \cite{lsc}, and the top-1 classification accuracy.

In the second stage of our algorithm, we freeze our estimation and plug it in SimPro and BOAT.
We keep exactly the same hyperparameter settings that SimPro and BOAT use. In particular, for Imagenet-127, we now use ResNet-50 and run each experiment once.
In SimPro, we set the unlabeled class distribution $P(Y|A=0)$ at the E-step;  however, we still keep a running estimate of the class distribution $P(Y)$ in the logit adjustment loss \cref{eq:simpro-la-loss}. While it is possible to use the first stage estimate in the logit adjustment loss, we observe that doing so results in lower accuracy than using the the running average. This is conceptually consistent with the role of the running average - serving not as an accurate estimate of $P(Y)$ but to make the classifier's class distribution uniform through the logit adjustment loss, which is good for the test set. Similarly, in BOAT, we only replace $\Delta_c = \log P(Y|A=1) - \log P(Y|A=0)$ in equation (4) of \cite{boat}, which is adjusting a classifier's predictions from the labeled to the unlabeled class distribution, with our SimPro + DR estimate instead of their on-the-fly estimate. 


% \section{Additional experiments}
% % \begin{table*}[t]
\centering
\caption{Total Variation Distance on CIFAR-10-LT ($N_l = 500$, $M_l = 4000$) with different class imbalance ratios $\gamma_l$ and $\gamma_u$ under five different unlabeled class distributions.}
\label{tab:cifar10-tv}
\resizebox{\textwidth}{!}{
\begin{tabular}{lccccccccccc}
\toprule
& & \multicolumn{2}{c}{consistent} & \multicolumn{2}{c}{uniform} & \multicolumn{2}{c}{reversed} & \multicolumn{2}{c}{middle} & \multicolumn{2}{c}{head-tail} \\
\cmidrule(lr){3-4} \cmidrule(lr){5-6} \cmidrule(lr){7-8} \cmidrule(lr){9-10} \cmidrule(lr){11-12}
& & $\gamma_l = 150$ & $\gamma_l = 100$ & $\gamma_l = 150$ & $\gamma_l = 100$ & $\gamma_l = 150$ & $\gamma_l = 100$ & $\gamma_l = 150$ & $\gamma_l = 100$ & $\gamma_l = 150$ & $\gamma_l = 100$ \\
Model & Estimator & $\gamma_u = 150$ & $\gamma_u = 100$ & $\gamma_u = 1$ & $\gamma_u = 1$ & $\gamma_u = 1/150$ & $\gamma_u = 1/100$ & $\gamma_u = 150$ & $\gamma_u = 100$ & $\gamma_u = 150$ & $\gamma_u = 100$ \\
\midrule
Supervised & MLLS & 0.269 ± 0.252 & 0.038 ± 0.006 & 0.251 ± 0.046 & 0.255 ± 0.060 & 0.429 ± 0.028 & 0.493 ± 0.050 & 0.333 ± 0.042 & 0.320 ± 0.009 & 0.457 ± 0.034 & 0.444 ± 0.043 \\
Supervised & RLLS & 0.043 ± 0.001 & 0.044 ± 0.010 & 0.348 ± 0.034 & 0.305 ± 0.068 & 0.769 ± 0.016 & 0.678 ± 0.028 & 0.430 ± 0.008 & 0.368 ± 0.013 & 0.539 ± 0.018 & 0.503 ± 0.020 \\
\midrule
MLE & IPW & 0.027 ± 0.001 & 0.027 ± 0.000 & 0.319 ± 0.072 & 0.243 ± 0.010 & 0.674 ± 0.020 & 0.646 ± 0.041 & 0.438 ± 0.020 & 0.454 ± 0.026 & 0.547 ± 0.049 & 0.491 ± 0.059 \\
MLE & OR & 0.045 ± 0.004 & 0.042 ± 0.000 & 0.215 ± 0.026 & 0.203 ± 0.032 & 0.433 ± 0.017 & 0.395 ± 0.033 & 0.193 ± 0.006 & 0.209 ± 0.037 & 0.307 ± 0.147 & 0.249 ± 0.130 \\
MLE & DR & 0.090 ± 0.002 & 0.079 ± 0.000 & 0.407 ± 0.027 & 0.360 ± 0.007 & 0.425 ± 0.007 & 0.421 ± 0.029 & 0.256 ± 0.001 & 0.286 ± 0.031 & 0.435 ± 0.136 & 0.362 ± 0.122 \\
\midrule
EM & IPW & 0.035 ± 0.002 & 0.040 ± 0.001 & 0.021 ± 0.001 & 0.029 ± 0.015 & 0.303 ± 0.187 & 0.091 ± 0.010 & 0.119 ± 0.011 & 0.105 ± 0.022 & 0.104 ± 0.026 & 0.104 ± 0.051 \\
EM & OR & 0.037 ± 0.003 & 0.042 ± 0.002 & 0.016 ± 0.001 & 0.024 ± 0.012 & 0.269 ± 0.183 & 0.090 ± 0.008 & 0.122 ± 0.012 & 0.103 ± 0.022 & 0.072 ± 0.012 & 0.073 ± 0.024 \\
EM & DR & 0.034 ± 0.004 & 0.037 ± 0.001 & 0.014 ± 0.001 & 0.027 ± 0.020 & 0.264 ± 0.191 & 0.092 ± 0.005 & 0.111 ± 0.019 & 0.097 ± 0.026 & 0.077 ± 0.016 & 0.073 ± 0.028 \\
\midrule
SimPro & IPW & 0.070 ± 0.011 & 0.058 ± 0.000 & 0.046 ± 0.001 & 0.049 ± 0.005 & 0.254 ± 0.074 & 0.223 ± 0.098 & 0.097 ± 0.025 & 0.067 ± 0.002 & 0.105 ± 0.066 & 0.110 ± 0.079 \\
SimPro & OR & 0.071 ± 0.012 & 0.058 ± 0.000 & 0.045 ± 0.001 & 0.049 ± 0.006 & 0.040 ± 0.003 & 0.059 ± 0.017 & 0.074 ± 0.006 & 0.075 ± 0.002 & 0.033 ± 0.003 & 0.033 ± 0.003 \\
SimPro & DR & 0.017 ± 0.004 & 0.026 ± 0.001 & 0.019 ± 0.002 & 0.018 ± 0.003 & 0.039 ± 0.003 & 0.058 ± 0.025 & 0.091 ± 0.007 & 0.031 ± 0.001 & 0.015 ± 0.003 & 0.019 ± 0.007 \\
\bottomrule
\end{tabular}
}
\end{table*}
% 

\begin{table*}[t]
\centering
\caption{Total Variation Distance on CIFAR-100-LT ($N_l = 50$, $M_l = 400$) with different class imbalance ratios $\gamma_l$ and $\gamma_u$ under five different unlabeled class distributions.}
\label{tab:cifar100-tv}
\resizebox{\textwidth}{!}{
\begin{tabular}{lccccccccccc}
\toprule
& & \multicolumn{2}{c}{consistent} & \multicolumn{2}{c}{uniform} & \multicolumn{2}{c}{reversed} & \multicolumn{2}{c}{middle} & \multicolumn{2}{c}{head-tail} \\
\cmidrule(lr){3-4} \cmidrule(lr){5-6} \cmidrule(lr){7-8} \cmidrule(lr){9-10} \cmidrule(lr){11-12}
& & $\gamma_l = 20$ & $\gamma_l = 10$ & $\gamma_l = 20$ & $\gamma_l = 10$ & $\gamma_l = 20$ & $\gamma_l = 10$ & $\gamma_l = 20$ & $\gamma_l = 10$ & $\gamma_l = 20$ & $\gamma_l = 10$ \\
Model & Estimator & $\gamma_u = 20$ & $\gamma_u = 10$ & $\gamma_u = 1$ & $\gamma_u = 1$ & $\gamma_u = 1/20$ & $\gamma_u = 1/10$ & $\gamma_u = 20$ & $\gamma_u = 10$ & $\gamma_u = 20$ & $\gamma_u = 10$ \\
\midrule
Supervised & MLLS & 0.707 ± 0.016 & 0.313 ± 0.100 & 0.445 ± 0.172 & 0.309 ± 0.119 & 0.383 ± 0.075 & 0.397 ± 0.006 & 0.570 ± 0.001 & 0.373 ± 0.107 & 0.543 ± 0.009 & 0.231 ± 0.057 \\
Supervised & RLLS & 0.520 ± 0.007 & 0.133 ± 0.003 & 0.337 ± 0.125 & 0.253 ± 0.082 & 0.424 ± 0.060 & 0.463 ± 0.003 & 0.454 ± 0.021 & 0.306 ± 0.074 & 0.460 ± 0.028 & 0.241 ± 0.040 \\
\midrule
MLE & IPW & 0.075 ± 0.000 & 0.071 ± 0.001 & 0.229 ± 0.001 & 0.167 ± 0.002 & 0.565 ± 0.005 & 0.443 ± 0.007 & 0.415 ± 0.000 & 0.311 ± 0.005 & 0.343 ± 0.000 & 0.280 ± 0.001 \\
MLE & OR & 0.065 ± 0.002 & 0.061 ± 0.001 & 0.200 ± 0.007 & 0.143 ± 0.001 & 0.526 ± 0.011 & 0.399 ± 0.023 & 0.360 ± 0.003 & 0.256 ± 0.012 & 0.328 ± 0.003 & 0.266 ± 0.005 \\
MLE & DR & 0.149 ± 0.019 & 0.145 ± 0.010 & 0.243 ± 0.004 & 0.214 ± 0.019 & 0.568 ± 0.005 & 0.464 ± 0.014 & 0.403 ± 0.014 & 0.309 ± 0.012 & 0.365 ± 0.007 & 0.320 ± 0.004 \\
\midrule
EM & IPW & 0.097 ± 0.008 & 0.092 ± 0.004 & 0.239 ± 0.007 & 0.179 ± 0.003 & 0.478 ± 0.012 & 0.329 ± 0.020 & 0.262 ± 0.016 & 0.202 ± 0.003 & 0.312 ± 0.002 & 0.227 ± 0.001 \\
EM & OR & 0.121 ± 0.007 & 0.108 ± 0.005 & 0.261 ± 0.007 & 0.189 ± 0.004 & 0.489 ± 0.013 & 0.335 ± 0.020 & 0.274 ± 0.016 & 0.211 ± 0.004 & 0.336 ± 0.003 & 0.235 ± 0.001 \\
EM & DR & 0.125 ± 0.005 & 0.111 ± 0.004 & 0.269 ± 0.007 & 0.194 ± 0.005 & 0.497 ± 0.010 & 0.336 ± 0.024 & 0.281 ± 0.019 & 0.219 ± 0.008 & 0.336 ± 0.007 & 0.233 ± 0.004 \\
\midrule
SimPro & IPW & 0.125 ± 0.001 & 0.100 ± 0.005 & 0.166 ± 0.007 & 0.141 ± 0.009 & 0.353 ± 0.023 & 0.261 ± 0.008 & 0.202 ± 0.003 & 0.158 ± 0.005 & 0.277 ± 0.009 & 0.197 ± 0.003 \\
SimPro & OR & 0.133 ± 0.005 & 0.100 ± 0.004 & 0.160 ± 0.007 & 0.138 ± 0.010 & 0.322 ± 0.014 & 0.253 ± 0.008 & 0.202 ± 0.003 & 0.156 ± 0.005 & 0.269 ± 0.006 & 0.191 ± 0.004 \\
SimPro & DR & 0.122 ± 0.003 & 0.106 ± 0.006 & 0.188 ± 0.009 & 0.149 ± 0.006 & 0.343 ± 0.023 & 0.257 ± 0.007 & 0.219 ± 0.010 & 0.172 ± 0.002 & 0.279 ± 0.007 & 0.198 ± 0.004 \\
\bottomrule
\end{tabular}
}
\end{table*}

\end{document}
