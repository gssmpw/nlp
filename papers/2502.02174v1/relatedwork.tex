\section{RELATED WORK}
\label{sec:RelatedWork}
        Our related work section focuses on game-based learning approaches used in software engineering and, particularly, TD management.
        
        %There is a variety of games that aim to promote understanding of important software development topics through game-based learning. 

        Problems and Programmers is a card game on the subject of software development designed by Baker et al.~\cite{baker_problems_2003}. 
        %CUT: It represents an emulation of the software process in various phases. 
        %The aim is to complete the fictitious software project first.
        %The competitive design allows players to observe other people's strategies and compare them with their own. 
        %In addition to the learning objective of teaching the players about the software development process, there is also the goal of creating a clear, understandable, interactive game with realistic feedback for the players.
        Due to the small number of participants, the findings only indicate that, according to the players' subjective opinions, the game facilitates collaborative learning.
        %CUT: The authors concede that a compromise must be made between the scope of learning aspects about software processes and a functional game.

        PlayScrum and GetKanban v4.0 teach frameworks for agile software development. %~\cite{fernandes_playscrum_2010}. % and the concepts and practices associated with it. 
        %The game emulates the Scrum process and lets the players take on the role of a Scrum Master leading a software development project. 
        %It is played with a game board, product backlog cards, problem cards, concept cards, developer cards, artifact cards, and dice. It is designed for two to five players. 
        Fernandes et al. evaluated PlayScrum with 13 master's students using questionnaires~\cite{fernandes_playscrum_2010}. 
        The players rated it as entertaining and effective for learning Scrum concepts. 
        %However, the study leaders noted that it was difficult for players without knowledge of Scrum to understand the game and its mechanics. 
        %Therefore, they recommended using the game as a supplement during the study and not in isolation for it to be effective. 
        %It was also recommended that the game's maps be further developed and digitized
        %GetKanban v4.0 is a collaborative board game for learning the agile software development method ``Kanban'' 
        GetKanban v4.0  was evaluated against predefined learning objectives~\cite{heikkila_teaching_2016}. %The aim of the game is to generate as much financial value as possible by producing new features in a software system. To achieve this, the players have to make effective resource management decisions in various roles.
        %GetKanban v4.0 has been simplified compared to its predecessor version. Many game elements that had additional rules or complicated mechanics were removed, and the changes were generally intended to provide a simpler, faster, and smoother game experience. 
       % CUT: Heikkilä et al.~\cite{heikkila_teaching_2016} tested and evaluated the game against predefined learning objectives. 
        Heikkilä et al. found that the game was motivating but had no significant effect on the knowledge transfer of Kanban methods.
        
        
        DecidArch (v2) and SmartDecisions are board games that teach decision-making in software architecture.
        %It is an enhanced version of the previous gameDecidArch[Lag+18]. It pursues the same. 
        DecidArch learning objectives are (1)~the rationale of design decisions, (2)~understanding the diversity of solutions, and (3)~the impact of changing design decisions~\cite{DeBoer2019}. 
        The game was evaluated with students in two game iterations to improve the game design.
        SmartDecisions focuses on the ``Attribute-Driven Design'' method (see~\cite{bass_software_2021}) and was evaluated with 41 participants, including students, educators, and practitioners, employing feedback forms and game results~\cite{cervantes_smart_2016}.
        %The game results showed no differences between experienced and inexperienced players.
        %De Boer et al.~\cite{DeBoer2019} evaluated DecidArch in two game iterations with 83 students in 22 groups in the first and 77 students in 20 groups in the second iteration to improve the game design.
        %De Boer et al.~\cite{DeBoer2019} evaluated DecidArch utilizing 83 students in 22 groups. 
        %It proved to be effective in teaching the learning objectives, although the third learning objective (dependencies) was too dependent on chance. Accordingly, the game was revised, and DecidArch v2 was developed to address the weaknesses that had previously been identified and to support the three learning objectives better.
        %The DecidArch v2 study involved 
        %CUT: They evaluated the first version with 83 students in 22 groups and the second with 77 students in 20 groups.
        %CUT: For both versions of the game, participants completed the same questionnaire. %surveys, which consisted of Likert scale questions and open-ended questions. The second evaluation also showed the effectiveness of the game, but some remaining problems were identified. The third learning objective (dependencies) was better supported, but problems remained in supporting the second learning objective (differences) and the third learning objective. 
        %De Boer et al. recognized the need for improvement regarding limited playtime, unclear relationships between design options, and the game's emphasis on individual design decisions.
        
        Hard Choices is the only empirically evaluated board game designed to teach the concept of TD~\cite{Ganesh2014}. % in the software development process. %CUT candidate.
        In this game, players compete against each other to be the first to bring a product to market. %CUT:, earning points as they develop the software. 
        Players can skip some development phases by incurring TD in the form of bridge cards evaluated as minus points in the final calculation. 
        Ganesh et al. tested the game with 41 IT students using pre- and post-tests and a survey. 
        The results revealed a significant improvement in the understanding of TD after the game. 
        %Ganesh was critical of the lack of a control group to better validate the instructional methods of the game and the small sample size of the teachers. She recommended testing the game several times with different teachers to rule out bias.

        Hard Choices is the only game similar to our game as it focuses on TD concepts. 
        However, the main purpose of this game and its evaluation is to educate IT students.
        Our game is designed to foster not only learning but discussion among and behavior change in practitioners, including non-technical stakeholders, and was evaluated using players from practice. 
        %CUT candidate:
        %Furthermore, the game mechanics are distinct: our game incorporates the concept of building a system and completing tickets, while Hard Choices uses dice to move across a path, following actions and consequences along the way.

        Other educational games on the topic of TD, such as the ``Technical Debt Game''~\cite{TDGame1} and the ``Technical Debt Game-for non-technical people''~\cite{TDGame2} were not evaluated, and thus, their effectiveness has not been determined. 
       % However, no studies were publicly available on these games, and thus, their effectiveness cannot be compared.