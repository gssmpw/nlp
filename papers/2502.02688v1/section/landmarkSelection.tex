\section{Landmark Selection}
\label{sec:LandmarkSelection}

There are different methods for selecting landmarks.
    \paragraph*{Random:}         
        A landmark is randomly selected. This method is fast to find landmarks, so we used it to compare to other methods. 
        
    \paragraph*{Outline:}
    The method is based on an approximation of the outlines of a graph.
        \begin{definition}
            The \textbf{outlines of a graph} G are defined by one or more pairs of nodes $(x, y)$ with $x, y \in X$ that maximize the minimum distance between $x$ and $y$ among all pairs of nodes in the graph.
        \end{definition}
To find the pair of nodes representing the outline, we use a well-known 2-approximation. First, we perform a shortest-path search starting from an arbitrary node $x$, then select the node $y$, which is the furthest node from $x$, as a landmark. Next, the shortest paths from $y$ are computed, and $z$ the node furthest from $y$ is selected as the second landmark. The outline is therefore ($y$, $z$) and the landmarks $y$ and $z$.
The complexity of finding a landmark depends on the complexity of computing the two shortest paths, and is therefore in $O(SP)$.        
    \paragraph*{Center:}
        The method is based on an approximation of the center of a graph.
        \begin{definition}
            The \textbf{center of a graph} G is defined by one or more nodes $x \in X$ that minimize the maximum distance from them to any other node in the graph.
        \end{definition}
As the definition of the outlines and the center are 
similar, the selection of landmarks is also similar. We search for the outlines $(x, y)$ with $x, y \in X$ and select as the center the node $z$ that lies halfway between $x$ and $y$. The landmark is $z$. 
The complexity is the same as for the previous method, $O(SP)$.

    \paragraph*{Outline \& center:}
The method is based on both outlines and center of a graph, that is a pair of outlines and a center are selected as described earlier. 

    \paragraph*{Maximum degree:}
        The method is based on the node's degree. We select as a landmark the node $x \in X$ that maximizes $(deg^+(x) + deg^-(x)) \times min(deg^+(x), deg^-(x))$, where $deg^+(x)$ (resp. $deg^-(x)$) is the number of outgoing arcs of $x$ (resp. incoming arcs to $x$). We used this formula to choose nodes with a large number of predecessors and successors. We also expect to choose nodes with a good balance between predecessors and successors. 
        To find landmarks we traverse every node once, giving a complexity of $O(|X|)$. 
        
    All these methods must be applied for each strongly connected component.