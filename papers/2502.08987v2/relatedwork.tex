\section{Related work}
\textbf{Physical reasoning\quad{}}
Research in physical reasoning has progressed along two main trajectories: passive observation and interactive platforms. The passive observation approach, exemplified by the \ac{voe} paradigm \citep{spelke1992origins}, evaluates physical understanding by measuring agents' ability to detect violations of intuitive physics principles \citep{ee1987object,hespos2001infants,dai2023x}. While this approach has provided valuable insights into basic physical comprehension, it is limited by its inability to assess active interventions and complex reasoning.

To enable more comprehensive evaluation, interactive platforms such as PHYRE \citep{bakhtin2019phyre}, the virtual tools game \citep{allen2020rapid}, and \benchmark \citep{li2023phyre} have emerged. These environments require agents to actively manipulate objects to achieve specific goals, testing not only prediction capabilities but also planning and reasoning skills. However, existing approaches in these platforms often struggle with generalization across diverse scenarios, particularly in few-shot settings where limited training data is available \citep{qi2021learning,li2023phyre}.

Current physical reasoning models face two primary challenges: the need for extensive training data and limited cross-scenario transferability. While some methods achieve strong performance within specific scenarios \citep{allen2020rapid}, they often fail to generalize their understanding to novel situations, falling short of human-level reasoning capabilities \citep{kang2024far}. Our work addresses these limitations by introducing a framework that unifies passive observation and interactive learning through dynamic \textbf{force fields}. The \ac{nff} approach enables few-shot learning of physical principles while supporting active reasoning through its interpretable force-based representation, facilitating both accurate prediction and effective intervention planning across diverse physical scenarios.

% Physical reasoning research has evolved significantly over the years, with studies spanning passive observation and interactive platforms. Passive observation frameworks, such as those leveraging the \ac{voe} paradigm \citep{spelke1992origins}, assess an agent's ability to predict physical outcomes by detecting unexpected events that violate intuitive physics principles \citep{ee1987object,hespos2001infants,dai2023x}. These methods provide insights into an agent's physical understanding through visual scenarios but lack the capacity to evaluate interventions. On the other hand, interactive platforms like PHYRE \citep{bakhtin2019phyre}, the virtual tools game \citep{allen2020rapid}, and \benchmark \citep{li2023phyre} address this limitation by presenting physical puzzles that require agents to interact with their environment to accomplish specific objectives. Such environments enable more comprehensive evaluation, encompassing not only prediction capabilities but also planning and reasoning skills \citep{allen2020rapid}. Nevertheless, challenges persist in generalizing to diverse scenarios and integrating effective representation into physical models to reach human-level reasoning \citep{qi2021learning,li2023phyre,kang2024far}.

\begin{figure*}[t!]
    \centering
    \includegraphics[width=\linewidth]{nff/nff}
    \caption{\textbf{Comparison between traditional interaction modeling and \ac{nff}.} (a) Traditional methods encode physical interactions as high-dimensional latent vectors and predict future states through learned transitions between these vectors. This approach requires extensive training data and often leads to overfitting, limiting generalization to novel scenarios. (b) Our \ac{nff} framework represents interactions as learnable force fields predicted from the dynamic object graphs, which are integrated through a differentiable \ac{ode} solver to predict trajectories. \ac{nff} enables few-shot learning of various interaction types from limited interventions while maintaining strong generalization capabilities. The framework consists of four key operators: \textbf{E} (encoder network for processing physical scenes), \textbf{D} (decoder network for state reconstruction), \textbf{F} (force field predictor), and \textbf{$\int$} (numerical integrator for computing trajectories).}
    % \caption{\textbf{Comparison between traditional interaction modeling and \ac{nff}.} (a) Traditional modeling methods, encoding object interactions in high-dimension latent vectors and transitioning them into future latent states, require a large amount of data to train and suffer from overfitting. (b) \ac{nff} learns interaction in the low-dimension representation of force fields, grounded in physics through a differentiable \ac{ode} solver. The \ac{nff} models can inverse the latent forces of various interaction types from a few interventions, and generalize well to \ac{ood} scenarios. The \textbf{E}, \textbf{D}, \textbf{F}, and \textbf{$\int$} operators indicate the encoder network, decoder network, force field predictor, and integrator, respectively.}
    \label{fig:overview}
\end{figure*}

\textbf{Dynamic prediction\quad{}}
The prediction of physical dynamics, fundamental to intuitive physics, has evolved along two methodological branches: discrete and continuous-time approaches. Discrete methods typically combine recurrent architectures with GNNs \citep{battaglia2016interaction,qi2021learning} or transformer modules \citep{wu2022slotformer} for modeling object interactions, while leveraging convolutional operators for processing pixel-based information \citep{shi2015convolutional,wang2022predrnn}. Despite their flexibility in handling various interaction types, these methods often struggle with two key limitations: difficulty in extracting robust physical representations and susceptibility to error accumulation over extended time horizons.

Continuous-time methods address these temporal consistency issues by incorporating physical inductive biases via explicit modeling of state derivatives within dynamical systems \citep{chen2018neural,greydanus2019hamiltonian,zhong2020symplectic,cranmer2020lagrangian,norcliffe2020second}. While this approach improves long-term prediction stability, current systems typically assume energy-conservative systems with simplified dynamics. More importantly, these methods face significant challenges in few-shot learning scenarios and struggle with cross-scenario generalization \citep{chen2020learning}, often requiring specific physical priors and grammars to achieve meaningful transfer \citep{xu2021bayesian}.

Our work advances continuous-time methods by introducing a more general physical representation framework via learnable \textbf{force fields}. This approach enables robust cross-scenario generalization while handling complex non-conservative energy systems. By combining the stability advantages of continuous-time modeling with flexible force field representations, \ac{nff} achieves both accurate long-term predictions and strong generalization capabilities without requiring explicit physical priors.

% Predicting future physical dynamics is a critical component of intuitive physics, essential for supporting downstream tasks such as planning and reasoning. Existing approaches can be broadly categorized into discrete and continuous-time methods. Discrete methods often employ recurrent architectures combined with GNNs \citep{battaglia2016interaction,qi2021learning} or transformer modules \citep{wu2022slotformer} to model object interactions, while convolutional operators are used to process pixel-based information \citep{shi2015convolutional,wang2022predrnn}. However, these methods frequently struggle with extracting robust representation and are prone to error accumulation over extended time horizons. In contrast, continuous-time methods incorporate physical inductive biases by explicitly modeling state derivatives within dynamical systems \citep{chen2018neural,greydanus2019hamiltonian,zhong2020symplectic,cranmer2020lagrangian,norcliffe2020second}, thereby improving long-term physical consistency. Despite these advantages, most studies assume energy-conservative systems with toy dynamics, limiting their applicability. Furthermore, few-shot learning and generalization across different system scenarios remains challenging \citep{chen2020learning} unless specific priors and grammars are given \citep{xu2021bayesian}. To address these limitations, we aim to extend continuous-time methods by introducing more general physical representation that enhances cross-scenario generalization, even in complex non-conservative energy systems.

\setstretch{1}