\section{Related Work}
\myparagraph{Image segmentation of tubular structures.}
Deep learning-based methods have achieved impressive results in segmentation tasks~\cite{long2015fully, ronneberger2015u, chen2018encoder}. To further enhance the segmentation performance of tubular structures, novel network architectures~\cite{jin2019dunet, lou2021dc, shin2019deep, wang2020deep, mei2021coanet, yang2023directional, wang2022pointscatter, qi2023dynamic} and topology-preserving loss functions~\cite{hu2019topology, mosinska2018beyond, shit2021cldice, menten2023skeletonization, qi2023dynamic} have been proposed. For example, in terms of network architecture, DSCNet~\cite{qi2023dynamic} utilizes dynamic snake convolution to capture fine and tortuous local features; PointScatter~\cite{wang2022pointscatter} explores the point set representation of tubular structures and introduces a novel greedy-based region-wise bipartite matching algorithm to improve training efficiency. In terms of loss functions, clDice~\cite{shit2021cldice} proposes a differentiable soft skeletonization method and achieves loss calculation at centerline level, which implicitly helps model focus more on the fine branches; TopoLoss~\cite{hu2019topology} and TCLoss~\cite{qi2023dynamic} measure the topological similarity of the ground truth and the prediction via persistent homology. Despite these advancements, all of the above methods are still confined to the framework of pixel-level classification and can not entirely overcome their inherent limitations. Our method attempts to morph the predicted graphs of tubular structures to let the network focus more on branch-level features, thus ensuring the topological accuracy of predictions.

% \citet{gupta2024topology} investigates structure-level representations of tubular structures like blood vessels by quantifying the uncertainty of each saddle-maxima branch. However, it is a two-step model that requires a pre-trained segmentation network. In addition, evaluating the uncertainty of each "saddle-maxima" pair leads to time-consuming inference. In contrast, our approach requires only one-step training and focuses on branch-level features at training time. Furthermore, the inference of the Morph Module and the post-processing of the segmentation task are relatively time-efficient.

\myparagraph{Image to graph.} There are two mainstream subtasks in this area: road network graph detection~\cite{he2020sat2graph, xu2022rngdet, shit2022relationformer, xu2023rngdet++, hetang2024segment} and scene graph generation~\cite{khandelwal2022iterative, kundu2023ggt}. These tasks usually entail detecting key components as nodes (i.e., key points in roads, objects in scenes) and determining their interrelations as edges (i.e., connectivity in roads, interactions in scenes).
Our work differs from these approaches in three ways. 
Firstly, we use only junctions and endpoints as nodes, which allows for explicit semantic characterization of nodes in our graph representation, unlike road network detection tasks where path points may also be regarded as nodes. Secondly, considering the curved nature of tubular objects, we propose Morph Module to obtain topologically accurate centerline masks, a goal that is not addressed by these works. Finally, our dynamic link prediction module is time-efficient, compared with elaborate and time-consuming designs in these works, such as [\texttt{rln}]-token in RelationFormer~\cite{shit2022relationformer}. For a clear understanding, we experimentally compare the differences between our approach and RelationFormer in Appendix~\ref{appendix_link_prediction}. These distinctions make our model not only time-efficient but also applicable to the task of tubular structure extraction with more complex topology.