\section{Identifying best train-time vocoders}\label{sec:bes_vocoders}
The following tables provide detailed results corresponding to the experiments discussed in \autoref{sec:exp-1}. These tables record EER (\%) across detector systems and distribution-shifts.
\paragraph{Training details} Detection systems are trained on synthetic speech generated from a single vocoder using a learning rate of 1e-5 and batch size of 64 for 20 epochs. Detectors for leave-one-out experiments are trained for 50 epochs with a learning rate of 1e-6, batch size 64 and weight decay of 0.0001. Model checkpoint with lower validation loss is used for evaluation.

Average EER (aEER) column represents the overall generalization of each training model on given test vocoders. On the other hand, aEER row represents the overall difficulty of detecting a particular vocoder considering the performance of all detectors. 

\begin{table}[H]
\caption{EER (\%) on models trained on LJSpeech dataset and generated utterances from train-time vocoder. Evaluation data is generated using vocoded speech from test vocoders with JSUT (single-speaker) as the source dataset. Lowest aEER on all test vocoders is achieved using model trained with HFG vocoded samples. Higher aEER obtained with WaveGlow used in training.}
\label{tab:best_voc_jsut}
\vskip 0.15in
\begin{center}
\scriptsize
\resizebox{\textwidth}{!}{
\begin{tabular}{|c|c|c|c|c|c|c|c|c|c|c|c|c|c|}
\hline
Train set $\downarrow$ Test set $\rightarrow$ & \textbf{PWG} & \textbf{WaveGrad} & \textbf{BigVGAN} & \textbf{BigVSAN}& \textbf{MB-MelGAN}& \textbf{UniVNet v1}& \textbf{UniVNet v2} & \textbf{HiFiGAN} & \textbf{Style-MelGAN} & \textbf{Vocos} & \textbf{APNet2} & \textbf{iSTFTNet} &\textbf{aEER}\\
\hline
PWG & 0.0 & 0.24 & 20.40 &2.72 & 0.0  &0.26 &2.34  & 3.10& 0.0 &  2.28& 4.36& 2.64& 3.19\\
\hline
HiFiGAN & 0.0&0.10 & 10.48& 0.72&0.0 &0.0 &0.24 & 0.34& 0.06&0.70 &1.70 &0.36 &1.22
\\
\hline
MB-Mel &0.0 &0.32&20.50 &2.20 &0.0 &  0.50&  2.46& 2.62&0.02 &3.34 &5.02&3.26&3.35
\\
\hline
FB-Mel & 0.02& 0.56& 23.60 & 6.20 &0.0 & 1.00& 3.80& 4.36& 0.04&6.06 &8.36&4.68&4.89
\\
\hline
Mel & 0.02&3.54 & 34.58&7.60&0.88 & 4.40 & 11.10 &10.80 &0.58 & 14.00&28.78&17.70&11.16
\\
\hline
Mel-L & 0.0&  1.76&29.66 &4.64 &0.24 & 1.76& 5.28&6.84 &0.16 &  6.74&16.10&10.50&6.97
\\
\hline
WaveGLow& 0.12&0.98 &45.10 &5.20 & 0.98&19.24 & 24.44&43.36 & 6.28&8.64 &5.78&31.64&15.98
\\
\hline
\textbf{aEER} & 0.02& 1.07 & 26.33 & 4.18 & 0.30 & 3.88 & 7.09 &10.20 &1.02 &5.96 &10.01 & 10.11 & -\\ 
\hline
\end{tabular}}%
\end{center}
\vskip -0.1in
\end{table}

\begin{table}[H]
\caption{EER (\%) on models trained on LJSpeech dataset and generated utterances from multiple train-time vocoder in leave-one-out fashion. Evaluation data is generated using vocoded speech from test vocoders with JSUT (single-speaker) as the source dataset. Leaving Mel-l, MB-Mel and WaveGlow vocoder from training helps yield low aEER}
\label{tab:leave_voc_jsut}
\vskip 0.15in
\begin{center}
\scriptsize
\resizebox{\textwidth}{!}{
\begin{tabular}{|c|c|c|c|c|c|c|c|c|c|c|c|c|c|}
\hline
Train set $\downarrow$ Test set $\rightarrow$ & \textbf{PWG} & \textbf{WaveGrad} & \textbf{BigVGAN} & \textbf{BigVSAN}& \textbf{MB-MelGAN}& \textbf{UniVNet v1}& \textbf{UniVNet v2} & \textbf{HiFiGAN} & \textbf{Style-MelGAN} & \textbf{Vocos} & \textbf{APNet2} & \textbf{iSTFTNet} &\textbf{aEER}\\
\hline

Leave HFG &0.0  &0.08  &17.80& 1.06&0.0 &0.14 &0.92 & 1.30 &0.0& 1.20 & 2.08 &1.60&2.18
\\ 
\hline
Leave pwg & 0.0 &  0.16 &16.44&1.18 &0.0 &0.12 &0.82 & 1.14 &0.0&1.32  & 2.34&1.20&2.06
\\ 
\hline
Leave mel & 0.0 &0.16  &17.28&1.18 & 0.0&0.18 &0.78 & 1.08 &0.0&1.18  &  1.78&1.22&2.07
\\ 
\hline
Leave mel-l &0.0 &0.06  &16.06& 1.08 &0.0 &0.14 & 0.74&  1.16&0.02& 1.16 &  1.84&1.22&1.95
\\ 
\hline
Leave mb-mel & 0.0 &  0.12&15.66&1.12 &0.0 &  0.14& 0.78&  1.04&0.0& 1.14 & 2.22 & 1.04&1.93
\\ 
\hline
Leave fb-mel & 0.0 &0.18  &17.82&  1.12& 0.0&0.10 &0.82 &1.26  &0.0&1.16  & 1.80&1.22&2.12
\\ 
\hline
Leave waveglow & 0.0 &0.18  &15.58& 1.30 &0.0 &0.08 &0.68 &0.80  &0.02&  1.36& 2.54&0.92&1.95
\\ 
\hline
\end{tabular}}%
\end{center}
\vskip -0.1in
\end{table}

\begin{table}[H]
\caption{EER (\%) on models trained on LJSpeech dataset and generated utterances from train-time vocoder. Evaluation data is generated using vocoded speech from test vocoders with AISHELL (multi-speaker + accented) as the source dataset. BigVGAN and Vocos consistently exhibits higher EER across all training models.}
\label{tab:best_voc_aishell}
\vskip 0.15in
\begin{center}
\scriptsize
\resizebox{\textwidth}{!}{
\begin{tabular}{|c|c|c|c|c|c|c|c|c|c|c|c|c|c|}
\hline
Train set $\downarrow$ Test set $\rightarrow$& \textbf{PWG} & \textbf{WaveGrad} & \textbf{BigVGAN} & \textbf{BigVSAN}& \textbf{MB-MelGAN}& \textbf{UniVNet v1}& \textbf{UniVNet v2} & \textbf{HiFiGAN} & \textbf{Style-MelGAN} & \textbf{Vocos} & \textbf{APNet2} & \textbf{iSTFTNet}& \textbf{aEER}\\
\hline
PWG & 6.16&12.37  & 33.43 & 23.21& 9.69 & 10.53&  14.66 &20.07 &9.09 & 34.60 &  24.65&17.15&17.96\\
\hline
HiFiGAN & 2.24&8.41 &25.30 & 12.41 & 2.74& 3.05 &6.56 &8.47 &4.76 &31.36 & 16.04&6.80&10.67
\\
\hline
MB-Mel &4.66 &11.20&  34.37&19.34  & 5.07&8.94 &14.03 & 16.23&3.62 & 28.16&24.19&17.04&15.57
\\
\hline
FB-Mel &3.21 &6.55 & 31.96& 20.33 & 1.57&  5.32&  10.10& 10.99&3.13 &22.81 &22.32&13.08&12.61
\\
\hline
Mel &6.10 & 15.53&36.53 &24.63& 12.98&16.97  & 21.80 & 20.66& 11.87& 31.70&32.38&25.22&21.36
\\
\hline
Mel-L & 4.73 & 16.80& 37.38&26.93&12.57&11.87 &18.11 &22.68 &8.73&29.41 &34.37&26.47&20.83
\\
\hline

WaveGLow&8.57 &11.87&32.63 &22.24 &16.96 &42.12 & 36.39&  34.15&36.43 &28.17 &21.61&37.27&27.36
\\
\hline
\textbf{aEER} & 5.09 & 11.81 & 33.08 & 21.29 & 8.79 & 14.11 & 17.37 & 19.03 & 11.09 & 29.45 & 25.08 & 20.43 & - \\ 
\hline
\end{tabular}}

\end{center}
\vskip -0.1in
\end{table}

\begin{table}[H]
\caption{EER (\%) on models trained on LJSpeech dataset and generated utterances from multiple train-time vocoder in leave-one-out fashion. Evaluation data is generated using vocoded speech from test vocoders with AISHELL (multi-speaker + accented) as the source dataset. Leaving FB-Mel and MB-Mel vocoders from training helps yield low aEER }
\label{tab:leave_voc_aishell}
\vskip 0.15in
\begin{center}
\scriptsize
\resizebox{\textwidth}{!}{

\begin{tabular}{|c|c|c|c|c|c|c|c|c|c|c|c|c|c|}
\hline
Train set $\downarrow$ Test set $\rightarrow$& \textbf{PWG} & \textbf{WaveGrad} & \textbf{BigVGAN} & \textbf{BigVSAN}& \textbf{MB-MelGAN}& \textbf{UniVNet v1}& \textbf{UniVNet v2} & \textbf{HiFiGAN} & \textbf{Style-MelGAN} & \textbf{Vocos} & \textbf{APNet2} & \textbf{iSTFTNet}& \textbf{aEER}\\
\hline
Leave HFG & 9.93 &21.09  & 34.01&24.62 & 15.38&14.00 &17.89 &  19.98&3.66& 30.11&25.91&20.17&19.73
\\ 
\hline
Leave pwg &10.21  & 18.60 &33.20&22.43 & 13.54&10.95 &13.90 & 15.94&8.54&29.40 &24.70 &15.23&18.05
\\ 
\hline
Leave mel & 13.08 &16.17  &31.49& 20.05&12.83 & 9.88& 12.61& 15.49 &9.88& 26.28 & 20.56  &13.60
&16.82

\\ 
\hline
Leave mel-l & 15.56 & 23.83  & 32.84& 25.07& 18.07&17.64 &20.22 & 20.33 &4.23&30.64  & 24.34 &20.20&21.08
\\ 
\hline
Leave mb-mel & 7.24 &16.48  &30.08&19.14 &10.88 &8.82 &12.48 & 14.42 &4.43& 27.73 & 21.12 &14.15&15.58
\\ 
\hline
Leave fb-mel & 7.93 &15.81  &29.29& 18.96&11.09 &10.46 & 13.37& 14.32 &5.07& 27.06 & 19.89&13.51&15.51
\\ 
\hline
Leave waveglow &9.25  &19.98  &33.06&23.29 &13.26 & 10.82&  14.31& 16.06 &3.97& 30.72 & 24.05&15.30&17.84
\\ 
\hline
\end{tabular}}
\end{center}
\vskip -0.1in
\end{table}

\begin{table}[H]
\caption{EER (\%) on models trained on LJSpeech dataset and generated utterances from train-time vocoder. Evaluation data is generated using vocoded speech from test vocoders with VoxCeleb as the source dataset. Lowest aEER on all vocoders is achieved using model trained with HFG vocoded samples. Highest aEER obtained with WaveGlow used in training. }
\label{tab:bes_voc_voxceleb}
\vskip 0.15in
\begin{center}
\scriptsize
\resizebox{\textwidth}{!}{
\begin{tabular}{|c|c|c|c|c|c|c|c|c|c|c|c|c|c|}
\hline
 Train set $\downarrow$ Test set $\rightarrow$ & \textbf{PWG} & \textbf{WaveGrad} & \textbf{BigVGAN} & \textbf{BigVSAN}& \textbf{MB-MelGAN}& \textbf{UniVNet v1}& \textbf{UniVNet v2} & \textbf{HiFiGAN} & \textbf{Style-MelGAN} & \textbf{Vocos} & \textbf{APNet2} & \textbf{iSTFTNet} &\textbf{aEER}\\
\hline
PWG &  5.64&17.13  & 36.17 &20.76 & 5.88  & 12.66& 17.09 &20.90 & 9.35 &  29.31& 24.21&18.36&18.12\\
\hline
HiFiGAN &1.00 &  5.80& 23.49& 6.81& 1.04&4.22 &5.02 & 7.14&2.31 & 27.39&9.99 &6.70&8.40
\\
\hline
MB-Mel &2.15 &7.03&33.62 & 13.97& 1.95& 8.06 &12.12 &16.31 &  4.06&  20.84&19.94&15.67&12.98
\\
\hline
FB-Mel &4.06 &7.50 &37.64 & 19.75& 2.62& 10.15&15.59 & 17.19& 5.13& 24.10 & 23.82&19.10&15.55
\\
\hline
Mel &3.77 &  9.27&33.54 &12.72&6.13 & 11.26 & 16.31 & 18.79& 8.22&23.20 &27.37&17.23&15.65
\\
\hline
Mel-L &6.34 &15.08 &35.37 &20.61&10.13&14.42 &19.40 &23.71 &11.71& 24.74&33.52&22.75&19.81
\\
\hline
WaveGLow& 25.70&36.58& 42.51& 37.25&23.67 &48.97 &47.29 &41.81 &37.23 & 40.11&34.73&43.43&38.27
\\
\hline 
\textbf{aEER} & 6.95 & 14.05 & 34.62 & 18.83 & 7.34 & 15.67 & 18.97 & 20.83 & 11.14 & 27.09 & 24.79 & 20.46 & - \\ 
\hline
\end{tabular}}
\end{center}
\vskip -0.1in
\end{table}


\begin{table}[H]
\caption{EER (\%) on models trained on LJSpeech dataset and generated utterances from multiple train-time vocoder in leave-one-out fashion. Evaluation data is generated using vocoded speech from test vocoders with VoxCeleb as the source dataset. BigVGAN and Vocos consistently exhibits higher EER across all training models. }
\label{tab:leave_voc_voxceleb}
\vskip 0.15in
\begin{center}
\scriptsize
\resizebox{\textwidth}{!}{
\begin{tabular}{|c|c|c|c|c|c|c|c|c|c|c|c|c|c|}
\hline
 Train set $\downarrow$ Test set $\rightarrow$ & \textbf{PWG} & \textbf{WaveGrad} & \textbf{BigVGAN} & \textbf{BigVSAN}& \textbf{MB-MelGAN}& \textbf{UniVNet v1}& \textbf{UniVNet v2} & \textbf{HiFiGAN} & \textbf{Style-MelGAN} & \textbf{Vocos} & \textbf{APNet2} & \textbf{iSTFTNet} &\textbf{aEER}\\
\hline
Leave HFG & 3.48 &   8.71&31.33&13.66 &5.58 & 7.83&11.75 & 13.68 &4.00&19.45  &  16.72&13.39&12.46
\\ 
\hline
Leave pwg & 6.81 &   11.22&29.62&13.19 &8.24 & 8.08 &10.89 &  13.17 & 6.93& 21.50 & 15.65&12.74&13.17
\\ 
\hline
Leave mel &  8.30 &   10.99&31.04&14.77 & 8.45&9.06 &11.69 &13.62  &7.79&21.15  & 16.16 &12.96&13.83
\\ 
\hline
Leave mel-l &6.77  &  10.58&29.87&14.87 & 7.79 & 9.41&  11.85& 13.60 &6.48& 21.91 &  15.84&13.07&13.50
\\ 
\hline
Leave mb-mel & 4.45 & 9.39 &28.92& 11.96&6.15 & 6.85&9.76 &  12.18&4.82&  21.15& 15.34 &11.75&11.89
\\ 
\hline

Leave fb-mel & 6.19 &  11.12&28.66&13.09 & 7.53 & 8.39&11.16 & 12.66 &6.21& 21.44 &15.16 &11.75&12.78
\\ 
\hline
Leave waveglow & 6.07  &  9.47 &29.54&13.09 & 7.73&  6.66& 9.35& 11.85 &5.15& 20.39 &15.14 &11.22&12.13
\\ 
\hline

\end{tabular}}
\end{center}
\vskip -0.1in
\end{table}

\begin{table}[H]
\caption{EER (\%) on models trained on LJSpeech dataset and generated utterances from train-time vocoder. Evaluation data is generated using vocoded speech from test vocoders with Audiobook as the source dataset.  Lowest aEER on all test vocoders is achieved using model trained with HFG vocoded samples. Highest aEER obtained with WaveGlow used in training.}
\label{tab:best_voc_audiobook}
\vskip 0.15in
\begin{center}
\scriptsize
\resizebox{\textwidth}{!}{
\begin{tabular}{|c|c|c|c|c|c|c|c|c|c|c|c|c|c|}
\hline
Train set $\downarrow$ Test set $\rightarrow$ & \textbf{PWG} & \textbf{WaveGrad} & \textbf{BigVGAN} & \textbf{BigVSAN}& \textbf{MB-MelGAN}& \textbf{UniVNet v1}& \textbf{UniVNet v2} & \textbf{HiFiGAN} & \textbf{Style-MelGAN} & \textbf{Vocos} & \textbf{APNet2} & \textbf{iSTFTNet}&\textbf{aEER}\\
\hline
PWG & 4.37 & 14.53& 38.18 & 23.15& 4.96  & 13.91&19.85 & 22.06& 9.40 &23.02 & 25.55&21.67&18.38\\
\hline
HiFiGAN & 0.98 &  5.21& 26.12& 10.77 &0.75 &3.63 &6.30 &8.02 & 5.56 & 21.85 &12.05 &7.79&9.08
\\
\hline
MB-Mel & 2.77 &10.22& 37.63 &18.79 &3.40 & 11.88& 16.87& 19.90& 5.89 & 21.01&24.55&20.40&16.10
\\
\hline
FB-Mel & 2.89& 9.05& 38.46& 22.38&2.06 & 10.91&17.36 &18.91 &4.51 & 20.84&25.38&21.43&16.18
\\
\hline
Mel & 1.61& 7.84&38.75  &14.67 &5.31 &12.36  & 17.33 & 20.67& 7.88&20.76 & 33.93&21.22&16.86
\\
\hline
Mel-L & 2.73& 10.53& 38.45&19.67&7.40& 13.18&18.18 &22.46 & 8.70&20.79 &33.04&23.23&18.19
\\
\hline

WaveGLow& 18.31&21.47&44.94 & 31.14&22.71 &43.55 &42.34 & 42.50&  35.52& 32.95&33.30&41.91&34.22
\\
\hline
\textbf{aEER} & 6.23 & 11.26 & 37.50 & 20.08 & 6.65 & 15.63 & 19.74 & 22.07 & 11.06 & 23.03 & 26.82 & 22.52 & -\\
\hline
\end{tabular}}%
\end{center}
\vskip -0.1in
\end{table}


\begin{table}[H]
\caption{EER (\%) on models trained on LJSpeech dataset and generated utterances from multiple train-time vocoder in leave-one-out fashion. Evaluation data is generated using vocoded speech from test vocoders with Audiobook as the source dataset. Leaving HFG vocoder degrades the overall performance.}
\label{tab:leave_voc_audiobook}
\vskip 0.15in
\begin{center}
\scriptsize
\resizebox{\textwidth}{!}{

\begin{tabular}{|c|c|c|c|c|c|c|c|c|c|c|c|c|c|}
\hline
Train set $\downarrow$ Test set $\rightarrow$ & \textbf{PWG} & \textbf{WaveGrad} & \textbf{BigVGAN} & \textbf{BigVSAN}& \textbf{MB-MelGAN}& \textbf{UniVNet v1}& \textbf{UniVNet v2} & \textbf{HiFiGAN} & \textbf{Style-MelGAN} & \textbf{Vocos} & \textbf{APNet2} & \textbf{iSTFTNet}&\textbf{aEER}\\
\hline
Leave HFG &  2.88 &  7.46 &35.97&17.26 & 3.62&10.27 &15.37 & 16.94 &4.70& 17.36 &  21.18&18.40&14.28
\\ 
\hline
Leave pwg & 4.47 &  7.56&32.75& 14.75& 4.85&  7.45& 11.23&12.85 &5.53& 15.92 &17.16 &13.73&12.35
\\ 
\hline
Leave mel & 5.21  &  7.25 & 33.32&15.58 &5.41 & 7.45 & 11.82&  12.90 &5.41& 15.06 & 16.88 &13.69&12.49
\\ 
\hline
Leave mel-l & 4.10 &7.86 &33.13& 15.01& 4.59& 7.79& 11.94& 13.15 &3.56& 16.57 & 17.20 &14.14&12.42
\\ 
\hline
Leave mb-mel &3.26  &7.82  &33.92& 15.58& 3.66& 7.25& 11.89& 13.53 &4.10& 16.88 &  18.14&14.36&12.53
\\ 
\hline
Leave fb-mel & 3.88 &  7.57 & 33.24&14.42 &  4.40& 7.27& 11.43&12.66 & 5.46& 15.67 &16.62 &13.10&12.14
\\ 
\hline
Leave waveglow & 3.77 &  7.16 &32.31&14.72 &4.17 & 6.12& 9.94 &11.46  &3.61& 15.50 & 16.67&12.23&11.47
\\ 
\hline
\end{tabular}}
\end{center}
\vskip -0.1in
\end{table}
\hspace{0cm}

\begin{table}[H]
\caption{EER (\%) on models trained on LJSpeech dataset and generated utterances from train-time vocoder. Evaluation data is generated using vocoded speech from test vocoders with Podcast as the source dataset. Lowest aEER achieved using model trained using HFG vocoded samples. Higher aEER obtained with WaveGlow used in training.}
\label{tab:best_voc_podcast}
\vskip 0.15in
\begin{center}
\scriptsize
\resizebox{\textwidth}{!}{
\begin{tabular}{|c|c|c|c|c|c|c|c|c|c|c|c|c|c|}
\hline
Train set $\downarrow$ Test set $\rightarrow$ & \textbf{PWG} & \textbf{WaveGrad} & \textbf{BigVGAN} & \textbf{BigVSAN}& \textbf{MB-MelGAN}& \textbf{UniVNet v1}& \textbf{UniVNet v2} & \textbf{HiFiGAN} & \textbf{Style-MelGAN} & \textbf{Vocos} & \textbf{APNet2} & \textbf{iSTFTNet} &\textbf{aEER}\\
\hline
PWG &  9.34& 18.25 &38.70 & 25.64&  10.71 &17.20 &22.28  & 25.32& 14.45& 26.33 & 26.80&24.04&21.58\\
\hline
HiFiGAN &1.20 & 5.61& 24.46&9.53 & 1.34 &3.90 &5.78 & 8.81 &4.40 & 16.69& 11.34 &7.24&8.35
\\
\hline
MB-Mel & 4.57&11.55&36.55 &19.00 & 5.28& 13.13& 18.03& 22.71& 7.90&21.59 & 24.44&21.03&17.14
\\
\hline
FB-Mel & 7.23&15.63 &  40.16&27.29 & 7.04 & 16.88&23.06 & 25.23&10.19 &26.55 &29.96&26.10&21.27
\\
\hline
Mel & 3.25& 11.35& 34.72&15.64&  7.24&   12.65& 17.11 &21.31 & 8.99 & 20.96& 28.80&20.44&16.87
\\
\hline
Mel-L & 4.14& 12.73 &35.27 &18.85& 8.79& 11.50&16.43 & 22.40&9.23&20.41 &29.63&21.64&17.58
\\
\hline

WaveGLow&  22.00& 24.93& 44.82&  33.78&26.59 &45.39 & 44.29& 43.66&  36.91&35.00 & 34.99&42.13&36.20
\\
\hline
\textbf{aEER} & 7.39 & 14.29 & 36.38 & 21.39 & 9.57 & 17.23 & 20.99 & 24.20 & 13.15 & 23.97 & 26.56 & 23.23 & - \\ 
\hline
\end{tabular}}
\end{center}
\vskip -0.1in
\end{table}

\begin{table}[H]
\caption{EER (\%) on models trained on LJSpeech dataset and generated utterances from multiple train-time vocoder in leave-one-out fashion. Evaluation data is generated using vocoded speech from test vocoders with Podcast as the source dataset. Leaving WaveGlow vocoder from training helps yield low aEER}
\label{tab:leave_voc_podcast}
\vskip 0.15in
\begin{center}
\scriptsize
\resizebox{\textwidth}{!}{
\begin{tabular}{|c|c|c|c|c|c|c|c|c|c|c|c|c|c|}
\hline
Train set $\downarrow$ Test set $\rightarrow$ & \textbf{PWG} & \textbf{WaveGrad } & \textbf{BigVGAN} & \textbf{BigVSAN}& \textbf{MB-MelGAN}& \textbf{UniVNet v1}& \textbf{UniVNet v2} & \textbf{HiFiGAN} & \textbf{Style-MelGAN} & \textbf{Vocos} & \textbf{APNet2} & \textbf{iSTFTNet} &\textbf{aEER}\\
\hline
Leave HFG & 6.61 & 10.22 &31.94&16.93 & 7.37 & 11.57& 15.07& 17.76 &8.13& 17.78 & 18.95 &17.20&
14.96\\ 
\hline
Leave pwg &8.51  &   11.53&30.23&15.96 &9.17 &10.76 &13.50 &  16.02&9.58&17.36  & 17.06 &14.91 &14.54
\\ 
\hline
Leave mel & 9.85 & 11.75 &30.91&16.85 &10.12 & 11.16& 14.04& 15.98 &9.84&17.25  & 17.09 &15.05&
14.99\\ 
\hline
Leave mel-l & 8.14 &11.00  &29.84&16.34 &8.39 &11.09 &13.67 & 15.85 &8.82&18.06  &  16.85&15.29&14.44
\\ 
\hline
Leave mb-mel &7.20  & 10.67 &29.81&15.73 &7.77 &9.88 &13.03 &15.72  &8.16& 17.07 &17.13  &14.71&13.90
\\ 
\hline
Leave fb-mel &7.95  & 11.02 &29.02& 14.91&9.05 &10.64 & 12.81& 14.89 &8.98&16.28  &15.71 &13.83&
13.75\\ 
\hline
Leave waveglow & 7.36 & 10.46 &29.94& 15.51& 7.76 &  8.82& 11.76&14.29  &7.66& 16.94 &16.73 &13.44&13.39
\\ 
\hline
\end{tabular}}%
\end{center}
\vskip -0.1in
\end{table}

\begin{table}[H]
\caption{EER (\%) on models trained on LJSpeech dataset and generated utterances from train-time vocoder. Evaluation data is generated using vocoded speech from test vocoders with Youtube as the source dataset. Lowest aEER on all test vocoders is achieved using model trained with HFG vocoded samples. Highest aEER obtained with WaveGlow used in training.}
\label{tab:best_voc_yt}
\vskip 0.15in
\begin{center}
\scriptsize
\resizebox{\textwidth}{!}{
\begin{tabular}{|c|c|c|c|c|c|c|c|c|c|c|c|c|c|}
\hline
Train set $\downarrow$ Test set $\rightarrow$ & \textbf{PWG} & \textbf{WaveGrad V1} & \textbf{BigVGAN} & \textbf{BigVSAN}& \textbf{MB-MelGAN}& \textbf{UniVNet v1}& \textbf{UniVNet v2} & \textbf{HiFiGAN} & \textbf{Style-MelGAN} & \textbf{Vocos} & \textbf{APNet2} & 
\textbf{iSTFTNet}&
\textbf{aEER}\\
\hline
PWG &17.03 & 23.43 & 40.60 &32.45 &17.98 &20.75 &23.59 & 28.89& 18.24& 35.59 &31.71 &28.39&26.55\\
\hline
HiFiGAN & 9.45&13.29 &30.84&20.65&10.37 & 10.88& 11.51& 16.61& 10.33&37.11 & 19.17&15.92&17.17
\\
\hline
MB-Mel & 13.26&19.71& 39.86& 28.46& 14.97 & 18.54& 22.43&7.44 & 14.68&31.67 & 30.91& 27.59&
22.46\\
\hline
FB-Mel &16.17 & 23.83& 42.25&32.87 & 16.42& 22.87& 26.54&28.61 & 18.41&33.38 &33.94&30.39&27.14
\\
\hline
Mel & 17.66& 21.04&38.61&26.82&20.10 & 22.23 & 24.66&28.33 &19.16 & 31.59&34.86&28.03&26.09
\\
\hline
Mel-L &21.90 &26.80 & 41.45&33.01&25.33&25.40 &28.19 & 31.88&24.89& 33.83&38.47&32.02&30.26
\\
\hline
WaveGLow& 36.53&35.09& 45.99&40.30 & 34.25& 46.89& 48.58&45.02 &43.95 &42.05 & 42.02&46.20&42.23 \\
\hline 
\textbf{aEER} &18.85 & 23.31 & 39.94 & 30.65 & 19.91 & 23.93  & 26.50 & 26.68 & 21.38 & 35.03 & 33.01 & 29.79 & - \\ 
\hline
\end{tabular}}%
\end{center}
\vskip -0.1in
\end{table}

\begin{table}[H]
\caption{EER (\%) on models trained on LJSpeech dataset and generated utterances from multiple train-time vocoder in leave-one-out fashion. Evaluation data is generated using vocoded speech from test vocoders with Youtube as the source dataset. BigVGAN and Vocos consistently exhibits higher EER across all training models.}
\label{tab:leave_voc_yt}
\vskip 0.15in
\begin{center}
\scriptsize
\resizebox{\textwidth}{!}{
\begin{tabular}{|c|c|c|c|c|c|c|c|c|c|c|c|c|c|}
\hline
Train set $\downarrow$ Test set $\rightarrow$ & \textbf{PWG} & \textbf{WaveGrad} & \textbf{BigVGAN} & \textbf{BigVSAN}& \textbf{MB-MelGAN}& \textbf{UniVNet v1}& \textbf{UniVNet v2} & \textbf{HiFiGAN} & \textbf{Style-MelGAN} & \textbf{Vocos} & \textbf{APNet2} & 
\textbf{iSTFTNet}&
\textbf{aEER}\\
\hline
Leave HFG & 21.43 &22.96  &36.63& 26.73& 21.30 &  22.17&23.90 &25.10 &20.94& 27.99 &27.17&25.53&25.15
\\ 
\hline
Leave pwg &23.92  &25.62  &36.05&27.86 & 23.99 &24.06 & 24.86&  26.24&23.96& 30.24 & 27.98&26.26&26.75
\\ 
\hline
Leave mel & 22.85 &24.77  &36.59&27.98 &22.69 &23.17 &24.21 & 25.63 &22.80& 29.42 & 27.37 & 25.67&26.09
\\ 
\hline
Leave mel-l & 20.86 &23.36  &35.31&26.73 &20.36 &21.69 & 23.11 & 24.27 &19.06& 29.06 &  25.80&24.47&24.50
\\ 
\hline
Leave mb-mel & 22.58 & 24.24 &35.53& 26.74& 22.79&22.97 & 23.70 & 25.08 & 22.71&29.04  & 26.79 &25.20&25.61
\\ 
\hline
Leave fb-mel & 22.01 & 23.97 &35.00&26.63 & 21.76& 22.73& 23.77&24.71  &22.04& 29.03 & 26.34&24.73&25.22
\\ 
\hline
Leave waveglow & 23.18 & 24.90 &35.65& 27.06& 23.49&23.32 & 23.87&25.18  &23.03&  29.42& 26.91&25.13&25.92
\\ 
\hline

\end{tabular}}%
\end{center}
\vskip -0.1in
\end{table}

\section{New Generation vocoders}\label{sec:appendix-new-gen}
This section provides additional details on \autoref{sec:exp-3}
\paragraph{Training details} Models are trained for 15 epochs with a learning rate of 1e-6, batch size of 64 and weight decay of 0.0001. 
Following tables provide complementary details for experiments in \autoref{sec:exp-2}. 
Test vocoders (see \autoref{tab:train_test_vocoders}) are divided based on their year of release. 
\begin{table}[H]
 \caption{Vocoders and year of release}
    \label{tab:voc_years}
    \vskip 0.15in
\begin{center}
\begin{small}
\begin{sc}
  \begin{tabular}{lc}
    \toprule
    \textbf{Vocoders} & \textbf{Year}\\ 
    \midrule
      WaveGlow & 2018\\
       \hline
       MelGAN and MelGAN-L & 2019\\ 
       \hline
       PWF, HFG and WaveGrad & 2020\\ 
       \hline
       MB-Mel, UniVNet and Style-Mel & 2021 \\
       \hline
       BigVGAN and iSTFTNet & 2022 \\
\hline 
APNet2 and Vocos & 2023 \\
\hline 
BigVSAN & 2024\\
         \bottomrule 
    \end{tabular}
    \end{sc}
\end{small}
\end{center}
\vskip -0.1in
\end{table}


\begin{table}[H]
    \caption{EER (\%) reported for new generation vocoders. Detectors are inclemently trained by including vocoders released from successive year in the training set. Test vocoders released in year 2020 achieved low EER in general. Detection performance improves with more new generation vocoders in training. Model trained on HFG vocoded samples overall better generalization performance.}
    \label{tab:detail_scores_gen_voc}
    \vskip 0.15in
\begin{center}
    \scriptsize
    \resizebox{\textwidth}{!}{%
    \begin{tabular}{lcccccccccccc}
    \toprule
Train set $\downarrow$ Test set $\rightarrow$  & PWG 2020 & HFG 2020 & WaveGrad 2020 & MB-Mel 2021 & UniVNet c-16 2021 & UniVNet c-32 2021 & Style-Mel 2021 & BigVGAN 2022 & iSTFTNet 2022 & APNet2 2023 & Vocos 2023 & BigVSAN 2024  \\ 
    \midrule
    2018 & 0.12 & 43.36 & 0.98 & 0.98 & 19.24 & 24.44 & 6.28 & 5.20 & 31.64 & 5.78 & 8.64 & 5.20 \\ 
    \hline
    2018 -- 2019 & 0.14 & 4.08 & 0.24 & 0.16 & 0.82 & 2.30 & 0.22 & 21.10 & 3.66 & 3.38 & 2.28 & 1.46  \\ 
    \hline
    2018 -- 2020 & 0.12 & 1.54 & 0.08 & 0.12 & 0.24 & 1.06 & 0.08 & 19.90 & 1.48 & 2.08 & 1.40 & 1.12\\
    \hline
    2018 -- 2021 & 0.02 & 1.34 & 0.10 & 0.02 & 0.30 & 1.06 & 0.02 & 15.56 &  1.26 & 1.54 & 1.16 & 1.08  \\ 
    \hline
    HiFiGAN (best) &  0.0 & 0.34 & 0.10 & 0.0 & 0.0 & 0.24 &0.06 & 10.48 & 0.36 & 1.70 & 0.70 & 0.72  \\ 
         \bottomrule 
    \end{tabular} 
    }%
    \end{center}
\vskip -0.1in
\end{table}

\section{Dataset decription}\label{sec:more_dataset_details}
\paragraph{Language:} For language shift, we use JSUT \cite{sonobe2017jsutcorpusfreelargescale} and AISHELL \cite{bu2017aishell1opensourcemandarinspeech} datasets. JSUT is a Japanese speech corpus, we re-synthesize basic5000 subset of this dataset. It includes one female native Japanese speaker and audio is in reading-style with approximately 6.7 hours of audio. AISHELL is Mandarin speech corpus, with more than 170 hours of speech. We re-synthesize 10 hours of the audio from this dataset for evaluation purposes utilizing test vocoders. In addition, for experiments in \autoref{sec:exp-10} we generate the entire AISHELL-1 dataset using XTTS system \autoref{tab:shifty_speech_tts}.

\paragraph{Celebrities / Speaking-style variation:} Here, the aim is to comprise the test set of varied speaking-style. VoxCeleb \cite{Chung18b} dataset is used for re-synthesis. It contains audios extracted from YouTube videos of various celebrities. In total 11 hours of audio samples are generated for each test vocoder.  

\paragraph{Domain-shift and Emotive speech:} Recording conditions and source of audio can vary during test times. In this study we cover three major recording conditions -- Reading style and clean background (LibriSpeech), Streaming;Emotional speech;Podcast style (MSP-Podcast \cite{Lotfian_2019_3}) and Background noise;spontaneous speech (Youtube \cite{chen2021gigaspeechevolvingmultidomainasr}).

\paragraph{Age, Gender and Accent:} Here, we re-synthesize subset of data from CommonVoice \cite{ardila2020commonvoicemassivelymultilingualspeech} such that it includes utterances from British and American accent. It includes seven age groups spanning from teens -- seventies. We cover four age-groups in our evaluations -- Twenties, Thirties, Forties and Fifties. Since, other subsets have relatively less number of samples they are not included in evaluations but we release the synthesized samples as part of the dataset. Please refer to \autoref{sec:exp-age-accent} for experimentation details.


\section{TTS distribution-shifts}\label{sec:tts_shifts_section}
In this section, we study distribution-shifts relative to TTS systems.

\subsection{Training on Vocoded speech vs training on TTS speech}\label{sec:exp-9}
For this experimental setup, detectors are trained on LJSpeech dataset such that TTS system used to generate synthetic speech utilize HFG vocoder. For the TTS systems, we use the transcripts from LJSpeech dataset to generate corresponding samples. Each of the TTS system we utilize are pre-trained on LJSpeech dataset with HFG as the vocoder. Information about pre-trained models can be found in \autoref{tab:train_details_tts}. For the vocoder, we convert the genuine waveform into acoustic features and then reconstruct the waveform using the HFG vocoder trained on LJSpeech  \footnote{\url{https://huggingface.co/speechbrain/tts-hifigan-ljspeech}}. While we train individual detectors with different TTS systems, we also train detectors with synthetic speech consisting of just the vocoded speech. Additionally, we also train a system using both TTS + vocoded speech. Please note that the set of TTS systems used in training as well as evaluations are -- Grad-TTS, VITS and Glow-TTS. Evaluation data is a test split of LJSpeech dataset (in-domain). 

Detectors are trained with a learning rate of 1e-6, weight decay of 0.0001 and batch size of 64. In particular, the detectors trained on TTS speech are trained for 50 epochs, the detector trained on both TTS and vocoded speech is trained for 30 epochs, and the detector trained on vocoded speech is trained for 20 epochs.


\begin{table}[ht!]
    \caption{\textbf{TTS distribution-shift} Detectors are trained using generated samples from given TTS systems with transcripts from LJSpeech dataset. Poor generalization when the detection model is trained using TTS generated samples. Slight performance improvement when synthetic samples generated with HFG are used in training.}
    \label{tab:d_tts}
    \vskip 0.15in
\begin{center}

\scriptsize
\begin{sc}
    \begin{tabular}{lcccc}
    \toprule
    Training data & Grad-TTS & VITS & Glow-TTS & Vocoded \\ 
    \midrule
      Grad-TTS & 0.45& 38.55 & 1.90 & 38.32\\ 
        \hline
        VITS-TTS &43.81 &0.15  & 0.0 & 46.71 \\ 
        \hline
         Glow-TTS &44.58 & 31.60 & 0.0 & 45.34 \\ 
        \hline
      Vocoded-HFG  &28.39 & 24.50&0.0 & 0.0\\ 
      \hline
      TTS+Vocoded &0.30& 0.30 & 0.0 & 0.38 \\ 
         \bottomrule 
    \end{tabular}
      \end{sc}
\end{center}
\vskip -0.1in
\end{table}

Based on the results from \autoref{tab:d_tts}, it can be observed that training on synthetic speech generated using one end-to-end TTS system generalizes poorly to other end-to-end methods of generation as well as vocoded speech. For example, a model trained on synthetic audio samples from Grad-TTS achieved an EER of 38.55 \% on utterances generated from VITS dataset and 38.32 \% EER on vocoded speech from HFG. Similar can be observed for models trained on VITS and Glow-TTS. On the other hand training on just the vocoded speech led to a slight improvement in generalization with EER of 28.39 \% and 24.50 \% on Grad-TTS and VITS test datasets respectively. However, regardless of the generation system used to generate fake audio samples, Glow-TTS achieved an almost perfect detection score in all cases. 

We also compare the above results with one of the best models trained with fake audio samples generated from more than one vocoder as reported in \autoref{tab:leave-one-out} ($\textbf{D}_{Leave-WaveGlow}$). EER dropped from 28.39 \% using just HFG vocoded speech to 23.43 \% using a detector trained with multi-source vocoded speech. Similarly, for VITS test data EER dropped to 20.45 \%. 
Training on both the TTS and Vocoded speech results in the best overall generalization results. This highlights the effect of distribution-shift with respect to different TTS systems during test time.

\paragraph{Discussion:} Poor generalization to unknown generation system reflects yet another test-time shifts to which detection models are vulnerable. This can also be regarded as one of most the common distribution-shift possible during test time. While the performance gain when the model is trained with fake and real samples generated using all the systems is significant, this is not an ideal real-world scenario. It can also be noted that performance gain using fake audio samples generated using vocoders is significant as compared to training using TTS generated samples.


\subsection{Language distribution-shift in TTS systems}\label{sec:exp-10}
\begin{table}[ht!]
    \caption{\textbf{Language distribution-shift:} Models trained on utterances in Chinese and English, generated using XTTS system. The test set only includes samples generated using XTTS system with corresponding real audio used as reference. The near perfect score was obtained using both detectors.}
    \label{tab:lang_tts_1}
    \vskip 0.15in
\begin{center}
\begin{small}
\scriptsize
\begin{sc}
    \begin{tabular}{lccc}
      \toprule
    Train set & AISH-1(zh) & JSUT (ja) \\
  \hline 
  AISH-1 (zh) & 0.0 & 0.0\\ 
  \hline
  VCTK (en) & 0.04 & 0.06  \\ 
 \bottomrule 
    \end{tabular}
    \end{sc}
\end{small}
\end{center}
\vskip -0.1in
\end{table}
Detection systems are largely trained on English data. However, in real-world scenarios, many different languages are possible during test time. In this section, we systematically study this train-test mismatch by two experiments. For the following experiments, we use XTTS \footnote{\url{https://github.com/coqui-ai/TTS/tree/dev}} multi-lingual pre-trained model for generation. First, we train a detection model with generated utterances using the Chinese language dataset AISHELL as source. During test time we evaluate on English language dataset -- VCTK. The same experiment is repeated by training on VCTK corpus and evaluating on AISHELL dataset. This setup allows us to study cross-language detection performance. Since we only want to account for distribution-shift due to language, we generate utterances with real speakers as reference thus carefully creating the evaluation set. Models are trained with a learning rate of 1e-6, batch size of 128 and weight decay of 0.0001.

It can be observed from \autoref{tab:lang_tts_1} that the model trained on the Chinese dataset generalized well to the English test set, while a model trained on the English dataset also achieves near perfect detection performance on both the English and Chinese datasets. We further extend the evaluation dataset to the Japanese language dataset -- JSUT. While both models can be seen to have a good detection performance, the model trained on the Chinese dataset achieved perfect detection performance on this dataset. 

We further expand the scope of this experiment by utilizing test data for more languages derived from MLAAD \cite{muller2024mlaad} dataset and corresponding real audio samples sourced from M-AILABS \cite{mai} dataset. Surprisingly, the overall generalization of the model trained on Chinese utterances is better than that of the model trained with English. While the overall trend is mostly similar. For example, for both the models, Polish and German are easier to detect. Russian is seemingly harder to detect with EER of 11.50\% using model trained on AISHELL and 10.40\% using a model trained on VCTK corpus. 
\paragraph{Discussion} While we were able to train a detector using utterances solely from the Chinese corpus, data for other languages are not as readily available for training. We observed better generalization in new languages when the model is trained with Chinese language. This indicates that for a better generalization performance on test-time languages training on data other than English should be explored. 
Although, switching the training data from English to Chinese helped reduce the EER \%, the overall trend of detection performance remained the same. For example, some languages like Russian are harder to detect while Polish and Italian are easier to detect. Additional related experiments can be found in \autoref{sec:tts-shift}

\begin{table*}[H]
    \centering
    \scriptsize
    \caption{\textbf{Language distribution-shift:} Models trained on utterances in Chinese and English, generated using XTTS system.}
    \label{tab:lang_tts}
    \begin{tabular}{lcc}
      \toprule
    Train set $\downarrow$ Test set $\rightarrow$ & AISHELL & VCTK \\
  \hline 
  AISHELL (zh) & 0.0 & 0.18\\ 
  \hline
  VCTK (en) & 0.04 & 0.02\\ 
 \bottomrule 
    \end{tabular}
\end{table*}


\begin{table}[t!]
    \caption{\textbf{Language distribution-shift:} Models trained on audio samples in Chinese and English, generated using XTTS system. Real audio samples are derived from MLADD dataset and corresponding fake samples are generated using XTTS system. Difficulty in generalization on Russian and Spanish language across both detectors. }
    \label{tab:lang_tts}
    \vskip 0.15in
\scriptsize
\begin{center}
\begin{sc}
    \begin{tabular}{llllllll}
      \toprule
      Train set $\downarrow$ & \multicolumn{6}{c}{Test set} \\ 
\cmidrule{2-7}
         & pl & it & fr & de & es & ru \\ \hline
  AISH-1 (zh) &0.20 & 2.90 & 4.80& 5.40& 4.90& 11.50\\ 
  \hline
  VCTK (en) &1.20 & 6.70 & 6.60 & 5.40& 7.90 & 10.40\\ 
 \bottomrule 
    \end{tabular}
\end{sc}
\end{center}
\vskip -0.1in
\end{table}


\section{Training details}\label{sec:appendix-train-details}

Speech synthesis is generally a two-stage process --- (i) Stage I deals with generating intermediate representations or acoustic features using source text or aligned linguistic features. These intermediate representations can be time-aligned features like mel-Spectrograms and fundamental frequency (F0), (ii) Stage II deals with transforming these intermediate features to raw waveform or an audio. Vocoder is part of stage II of this synthesis pipeline. There are two distinct family of vocoders based on their nature of generation: Autoregressive (AR) \cite{oord2016wavenetgenerativemodelraw,mehri2017samplernnunconditionalendtoendneural,kalchbrenner2018efficientneuralaudiosynthesis} and Non-Autoregressive (NAR). Most of the new generation vocoder belongs to the latter family. Because AR generation includes sequential sampling and hence computationally inefficient with slow inference speed. Generative Adversarial Network (GAN) \cite{goodfellow2014generativeadversarialnetworks} based vocoders were then proposed to facilitate faster generation. GANs consists of a Generator and Discriminator network. A Generator maps noise vector to a data distribution (raw waveforms) and generate samples. While the Discriminator classifies a generated sample as real or fake with a probability. This includes training generator and discriminator simultaneously with generator minimizing the loss and discriminator maximizing it.
\paragraph{GAN-based vocoders:} We include variety of GAN-based vocoders. Archictetural details of these vocoders are discussed here in brief --- 
MelGAN (Mel) \cite{kumar2019melgangenerativeadversarialnetworks} follows GAN based architecture employing multiple window-based discriminators in order to learn better discriminative features from various frequency ranges. Parallel-WaveGAN (PWG) \cite{yamamoto2020parallelwaveganfastwaveform} uses 
NAR WaveNet based generator with joint adversarial and multi-resolution short-time Fourier Fourier transform loss (STFT). Multi-resolution STFT loss helps learn better time-frequency distribution for producing high quality waveform and also help stabilize GAN training and convergence. HiFiGAN (HFG) \cite{kong2020hifigangenerativeadversarialnetworks} further improved the generation quality of audio by including Multi-Period Discriminator (MPD) to model varied periodic patterns in an audio in addition to Multi-Scale Discriminator (MSD) like MelGAN. \cite{yang2020multibandmelganfasterwaveform} Multi-band MelGAN (MB-Mel) proposed an improved extension of MelGAN by expanding receptive field for improved quality. Instead of working with full-band single, here generator network initially predicts sub-band level signals which are then combined to full-band signal and results in faster generation speed. \cite{mustafa2021stylemelganefficienthighfidelityadversarial} also uses sub-bands and employs temporal adaptive normalization (TADE) by facilitating conditional information about target speech to individual layers of general network. \cite{jang2021univnetneuralvocodermultiresolution} used MPD like HFG, in addition to multi-resolution spectrogram discriminator (MRSD) allowing for multiple spectrograms with different spectral and temporal resolutions. \cite{lee2023bigvganuniversalneuralvocoder} further improves both the generation quality and speed and used MRD. It further adds periodic inductive bias using Snake function. \cite{shibuya2024bigvsanenhancingganbasedneural} BigVSAN employs SAN \cite{takida2024saninducingmetrizabilitygan} instead of traditional GAN training. \cite{kaneko2022istftnetfastlightweightmelspectrogram,siuzdak2024vocosclosinggaptimedomain, ai2023apnetallframelevelneuralvocoder} iSTFTNet, VOCOS and APNet uses inverse short-time Fourier transform (iSTFT) in order to generate a waveform by predicting phase and amplitude. This helps avoid the need of several transposed convolutions to finally upsample input features to match the required sample rate. \cite{du2023apnet2highqualityhighefficiencyneural} APNET2 is an extension of APNET further adding improvements in phase prediction. 

In addition to above we also considered WaveGlow \cite{prenger2018waveglowflowbasedgenerativenetwork} as a Flow-based vocoder and  WaveGrad \cite{chen2020wavegradestimatinggradientswaveform} as a diffusion-based vocoder.

For all the vocoder systems listed below, training data used is LibriTTS. 

\begin{table}[H]
\caption{Training details of vocoders. For the pre-trained models the listed GitHub's are used with models trained on LibriTTS data.}
 \label{tab:train_details_voc}
 \vskip 0.15in
\begin{center}
\scriptsize
\begin{tabular}{lccc}
\hline
\toprule
Vocoder & GitHub  & Pre-trained \\
\hline
\toprule
PWG \multirow{4}{4em}& \multirow{3}{*}{\href{https://github.com/kan-bayashi/ParallelWaveGAN}{Link}} &  $\surd$  \\ 
MB-Mel & &  \\ 
Style-Mel & & \\ 
HiFiGAN & & \\ 
\hline
WaveGrad & \href{https://github.com/lmnt-com/wavegrad/tree/master}{Link}&  $\times$  \\ 
\hline 
UnivNET v1 \multirow{2}{4em}& \href{https://github.com/maum-ai/univnet}{Link}&  $\surd$ \\ 
UnivNET v2 &  & \\ 
\hline
BigVSAN & \href{https://github.com/sony/bigvsan}{Link}  & $\surd$ \\ 
\hline
iSTFTNet &\href{https://github.com/rishikksh20/iSTFTNet-pytorch/tree/master}{Link} & $\times$  \\ 
\hline 
APNet2 &\href{https://github.com/BakerBunker/FreeV/tree/main}{Link}  & $\times$  \\ 
\hline 
Vocos &\href{https://github.com/gemelo-ai/vocos/tree/main}{Link}  &  $\surd$ \\ 
\hline
BigVGAN & \href{https://github.com/NVIDIA/BigVGAN}{Link} & $\surd$  \\ 
\bottomrule
\end{tabular}
\end{center}
\vskip -0.1in
\end{table}

Following are the details for vocoders trained from scratch. WaveGrad is trained for 1M iterations with linear noise schedule, batch size of 64, learning rate of 2e-4 and sample rate of 24000. 
iSTFTNet is trained for 1.05M iterations with batch size of 240, learning rate of 0.0002, learning rate decay of 0.999, adam optimizer with $\beta_1$ as 0.8 and $\beta_2$ as 0.99
APNet2 is trained for 1M iterations with batch size of 480, learning rate of 0.0002, adam optimizer with $\beta_1$ as 0.8 and $\beta_2$ as 0.99
\begin{table}[H]
\caption{Training details of TTS systems used to generate the data.}
\label{tab:train_details_tts}
\vskip 0.15in
\begin{center}
\scriptsize
\begin{tabular}{lcc}
\hline
\toprule
TTS system & GitHub  & Pre-trained \\
\hline
XTTS &\multirow{3}{*}{\href{https://github.com/coqui-ai/TTS/tree/dev}{Link}} &  $\surd$  \\
VITS & & \\
YourTTS & &  \\ 
Glow-TTS & &  \\
\toprule
Grad-TTS & \href{https://github.com/huawei-noah/Speech-Backbones/blob/main/Grad-TTS/README.md}{Link}& $\surd$  \\ 
\bottomrule
\end{tabular}
\end{center}
\vskip -0.1in
\end{table}
A combination of multi-speaker and single-speaker models are used for TTS systems. 
\subsection{TTS distribution-shift}\label{sec:tts-shift}
The below experiment is complementary to that of~\autoref{sec:exp-10}.
Different methods of generation can have different test time implications. In \autoref{tab:diff_tts} we note that models trained with utterances generated from XTTS generalize well to utterances generated using VITS and YourTTS. However, these two models are not generalizable to the older model FastPitch.

\begin{table}[H]
    \caption{\textbf{Language distribution-shift:} Models trained on utterances in Chinese and English, generated using XTTS system. Test data is generated using various TTS systems with VCTK transcript. The reference speaker used to generate the utterance is the same as the corresponding ground-truth utterance. YourTTS generated samples are easier to detect even with a detector trained using a language other than English}
    \label{tab:diff_tts}
     \vskip 0.15in
\begin{center}

\scriptsize
\begin{sc}
    \begin{tabular}{lcccr}
      \toprule
    Train set $\downarrow$ Test set $\rightarrow$ & YourTTS (en) &VITS (en) & FastPitch (en)\\
  \hline 
  AISHELL (zh) & 0.42 & 1.86 & 13.73\\ 
  \hline
  VCTK (en) &0.20 & 0.0 & 0.44 \\ 
 \bottomrule 
    \end{tabular}
    \end{sc}
\end{center}
\vskip -0.1in
\end{table} 

\section{Age and accent}\label{sec:age_and_accent}
\subsection{Impact of age and accent}\label{sec:exp-age-accent}
In this experiment we investigate the effect of age, gender and accent as a distribution-shift. 

We evaluate the 7 trained models from \autoref{sec:exp-1} on CommonVoice dataset for 4 different age-groups (see \autoref{tab:age_groups_accent}). Note that evaluations are done on a subset of test vocoders belonging to new generation vocoder category (BigVGAN, BigVSAN, UniVNET v1, UniVNET v2 and Vocos). 


\begin{table}[H]
  \caption{Detection performance for British accent speech and different age-groups. EER \% is averaged over seven train-time models (see \autoref{tab:age_groups_accent}). Test vocoders considered are BigVGAN, BigVSAN, UniVNET v1, UniVNET v2 and Vocos. Low aEER for test data with male speaker labels}
    \label{tab:bt_five_test_voc}
    \vskip 0.15in
\begin{center}
    \scriptsize
    \begin{tabular}{lccccc}
    \toprule
Gender $\downarrow$ Age $\rightarrow$  & Twenties & Thirties  & Forties & Fifties & \textbf{aEER}\\ 
    \midrule
Male & 22.02 & 21.67 & 23.87 & 23.41 & 22.74\\ 
\hline 
Female& 26.63 & 24.45 & 26.69 & 28.15 & 26.48\\
\bottomrule
    \end{tabular}
    \end{center}
\vskip -0.1in
\end{table}

\begin{table}[H]
 \caption{Detection performance for American accent speech and different age-groups. EER \% is averaged over seven train-time models (See \autoref{tab:age_groups_accent}). Test vocoders considered are -- BigVGAN, BigVSAN, UniVNET v1, UniVNET v2 and Vocos. Comparable aEER for test data with both male and female speaker labels}
    \label{tab:american_five_test_voc}
     \vskip 0.15in
\begin{center}
    \scriptsize

    \begin{tabular}{lccccc}
    \toprule
Gender $\downarrow$ Age $\rightarrow$  & Twenties & Thirties  & Forties & Fifties & \textbf{aEER}\\ 
    \midrule
Male & 22.09 & 21.79 & 24.98 & 26.16 & 23.75\\ 
\hline 
Female& 22.91 & 24.26 & 23.19 & 23.90 & 23.56\\
\bottomrule
    \end{tabular}
     \end{center}
\vskip -0.1in
\end{table}

We further extend the evaluation set of vocoders to include few more vocoders -- PWG, HFG, MB-Mel and Style-Mel. In these evaluations we utilize detector trained on HFG generated samples. Based on these experiments it can be observed that detection of audio samples with male speaker labels are comparatively easier to detect for both American and British accents. However this difference is more profound for British accented speech. For example, as per \autoref{tab:bt_hfg} there is approximately 29\% difference in performance for both genders. While approximately 11.4\% difference for American accented speech \autoref{tab:american_hfg}. Similar was observed based on \autoref{tab:bt_five_test_voc} and \autoref{tab:american_five_test_voc} with 14\% and 0.8\% difference between both genders with British and American accented speech. 
Moreover, age-group fifties were harder to detect in general for both the accents and genders considered.
It can be noted that performance on American accent is slightly better than British accent. We suspect this difference in performance stems from pre-training data used for wav2vec 2.0 xlsr. Moreover, even though the speaker utilized in training has female label, the detection performance was superior for evaluation data with male labels. This experiment again highlights the difference in generalization arising directly from pre-training data of employed front-end features. For better generalization, these differences should be taken into account while carefully selecting pre-training data.
\begin{table}[htbp]
 \caption{Detection performance for British accent speech and different age-groups. Test vocoders considered are -- BigVGAN, BigVSAN, UniVNET v1, UniVNET v2, Vocos, PWG, HFG, MB-Mel and Style-Mel. Detector trained with HFG vocoded samples is used for evaluation. Low aEER for test data with male speaker labels}
    \label{tab:bt_hfg}
    \vskip 0.15in
\begin{center}
    \scriptsize
    \begin{tabular}{lccccc}
    \toprule
Gender $\downarrow$ Age $\rightarrow$  & Twenties & Thirties  & Forties & Fifties & \textbf{aEER}\\ 
    \midrule
Male & 7.62 & 6.86 & 6.26 & 7.16 & 6.97\\ 
\hline 
Female&9.54 & 9.97 & 8.19 & 11.69 & 9.84 \\
\bottomrule
    \end{tabular}
    
    \end{center}
\vskip -0.1in
\end{table}

\begin{table}[ht!]
 \caption{Detection performance for American accent speech and different age-groups. Test vocoders considered are -- BigVGAN, BigVSAN, UniVNET v1, UniVNET v2, Vocos, PWG, HFG, MB-Mel and Style-Mel. Detector trained with HFG vocoded samples is used for evaluation. Low aEER for test data with male speaker labels}
    \label{tab:american_hfg}
    \vskip 0.15in
\begin{center}
    \scriptsize
    \begin{tabular}{lccccc}
    \toprule
Gender $\downarrow$ Age $\rightarrow$  & Twenties & Thirties  & Forties & Fifties & \textbf{aEER}\\ 
    \midrule
Male & 8.04 & 6.86 & 7.70 & 8.13 & 7.68\\ 
\hline 
Female& 8.32 & 9.09 & 8.02 & 9.27 & 8.67\\
\bottomrule
    \end{tabular}
    \end{center}
\vskip -0.1in
\end{table}
 
\begin{table}[H]
\caption{Scores for Each System Across Different Age Groups, Accents, and Genders. Scores are averaged over five test vocoders for simplicity}
\label{tab:age_groups_accent}
\vskip 0.15in
\begin{center}
\scriptsize
\begin{tabular}{|c|c|c|c|c|c|c|c|c|c|}
\hline
\multirow{3}{*}{\textbf{Train Vocoder}} & \multirow{3}{*}{\textbf{Accent}} & \multirow{3}{*}{\textbf{Gender}} & \multicolumn{4}{c|}{\textbf{Age Groups}} \\ \cline{3-7} 
 &  &  &  \textbf{Twenties} & \textbf{Thirties} & \textbf{Forties} & \textbf{Fifties} \\ \hline
\multirow{4}{*}{\textbf{PWG}} & \multirow{2}{*}{British} & Female & 28.69  &25.58  & 26.18 & 32.37 \\ \cline{3-7} 
 &  & Male &  21.51& 24.87 & 23.71 & 22.21 \\ \cline{3-7} 
 & \multirow{2}{*}{American} & Female &26.13  & 26.31 &26.47  & 27.30  \\ \cline{3-7} 
 &  & Male &  24.04 &22.03 &25.08  &29.03\\ \cline{3-7} 
 \hline
\multirow{4}{*}{\textbf{HiFiGAN}} & \multirow{2}{*}{British} & Female  & 13.52 & 14.11 &  11.52&15.86 \\ \cline{3-7} 
 &  & Male & 11.15 &  9.96& 9.41 &  10.42\\ \cline{3-7} 
 & \multirow{2}{*}{American} & Female &12.09  & 12.62 &11.38  &12.98 \\ \cline{3-7} 
 &  & Male  &11.09  &9.94  &10.23 &11.38 \\ \cline{3-7}  \hline
\multirow{4}{*}{\textbf{MB-Mel}} & \multirow{2}{*}{British} & Female & 24.38 & 20.78 & 24.42 & 20.17 \\ \cline{3-7} 
 &  & Male & 17.72 & 18.06 &  18.82&  19.64\\ \cline{3-7} 
 & \multirow{2}{*}{American} & Female&22.41 &22.52  &18.54  & 20.97\\ \cline{3-7} 
 &  & Male &16.34  &16.91  & 20.45 &22.44 \\ \cline{3-7} \hline
\multirow{4}{*}{\textbf{FB-Mel}} & \multirow{2}{*}{British} & Female & 27.53 & 25.70 &31.45 &27.72 \\ \cline{3-7} 
 &  & Male & 18.23 & 17.15 &22.50  &  22.71\\ \cline{3-7} 
 & \multirow{2}{*}{American} & Female  &22.23  &21.62  & 21.59 &25.19 \\ \cline{3-7} 
 &  & Male & 20.13 &19.71  & 24.40 &26.02  \\ \cline{3-7}  \hline
\multirow{4}{*}{\textbf{Mel}} & \multirow{2}{*}{British} & Female & 21.92 &17.28  & 21.62 & 25.47   \\ \cline{3-7} 
 &  & Male & 19.46 & 19.75 & 20.61 & 20.22  \\ \cline{3-7} 
 & \multirow{2}{*}{American} & Female &  19.12&19.60  &19.19  &18.74 \\ \cline{3-7} 
 &  & Male  &18.34  &18.52  &21.00  &21.60\\ \cline{3-7}  \hline
\multirow{4}{*}{\textbf{Mel-L}} & \multirow{2}{*}{British} & Female  & 21.59 &18.19  & 21.59 &26.53  \\ \cline{3-7} 
 &  & Male &19.36  &18.50  & 23.29 &  21.06\\ \cline{3-7} 
 & \multirow{2}{*}{American} & Female  &18.66  &21.72  &19.34  &18.68  \\ \cline{3-7} 
 &  & Male &19.58  &19.61  &24.89 & 23.89 \\ \cline{3-7}  \hline
\multirow{4}{*}{\textbf{WaveGlow}} & \multirow{2}{*}{British} & Female &48.79  & 49.55 & 50.07 & 48.99 \\ \cline{3-7} 
 &  & Male & 46.72 &  43.45& 48.80 &  47.63 \\ \cline{3-7} 
 & \multirow{2}{*}{American} & Female  &39.74  &45.47  &45.84  &43.45  \\ \cline{3-7} 
 &  & Male & 45.16 &45.77  &48.83  &48.80   \\ \cline{3-7}  \hline
\end{tabular}
\end{center}
\vskip -0.1in
\end{table}
For additional test purposes, we also include synthetic speech generated using end-to-end TTS systems using LJSpeech and VCTK transcripts.
\section{Speech Quality and Detectability}\label{sec:quality_and_detectability}

\subsection{Relation between speech quality and vocoder detectability}\label{sec:exp-4}
In this section, we conduct experiments to explore the correlation between quality score (as reported by automated metric UTMOS \cite{saeki2022utmosutokyosarulabvoicemoschallenge}) and detection performance (as measured by EER).
For each of the datasets, we get UTMOS score on all 12 test vocoders. In addition we get the EER for test vocoders by averaging the EER performance obtained on the 7 train vocoders. We particularly utilize stronger learner model proposed as a part of UTMOS metric. This strong learner uses wav2vec2.0 pretrained on LibriSpeech as SSL model. Finally, we calculate the co-relation coefficient using Pearson Correlation. Scores are reported in \autoref{tab:utmos_eers}. We note a strong positive correlation between between two metrics for Audiobook and Podcast datasets with correlation score as 0.89 and 0.85 respectively. While there is moderate correlation between automatic quality metric and EER for VoxCeleb dataset with correlation score of 0.74. In addition, Youtube dataset also exhibits a positive relationship, but not very strongly with correlation score of 0.58. Further, we note that for hard to detect vocoders -- \{BigVGAN, HFG and Vocos\} UTMOS score is in higher range with higher EER (See \autoref{tab:utmos-test-vocoders}). While for compartively easier to detect vocoders like PWG, UTMOS scores are in moderate range of 2 -- 3.

\begin{table}[ht!]
    \small
    \caption{Co-relation score between UTMOS and EER, calculated using Pearson Correlation. All of the 12 test vocoders were considered to get UTMOS score on respective dataset. EER is averaged over score from 7 trained models. Moderate co-relation observed for YouTube and VoxCeleb}
    \label{tab:utmos_eers}
  \vskip 0.15in
\begin{center}
    \begin{tabular}{lcc}
    \toprule
Dataset & Score \\ 
\toprule
Audiobook & 0.89 \\ 
\hline
Podcast & 0.85 \\ 
\hline
YouTube & 0.58 \\ 
\hline
VoxCeleb & 0.74 \\ 
 \bottomrule 
    \end{tabular}
    \end{center}
\vskip -0.1in
\end{table}

\paragraph{Discussion} The results suggest that automatic speech metrics can serve as a proxy for diagnosing fake audio samples, given their strong correlation with EER across most distribution sets. However, these metrics have limitations, particularly in providing consistent results under different conditions. While we adopt UTMOS as the objective metric for datasets with English as the source language in our experiments, exploring other automatic MOS prediction methods could provide valuable insights in the future.
\begin{table*}[h!]
 \caption{UTMOS scores on each dataset. Please note: UTMOS score calculation requires sampling rate to be 16kHz.}
    \label{tab:utmos-test-vocoders}
\vskip 0.15in
\begin{center}
    \scriptsize
    \begin{tabular}{clccccc}
    \toprule
    \multicolumn{2}{c}{\multirow{2}{*}{\textbf{Vocoder}}} & \multicolumn{5}{c}{\textbf{UTMOS} $\uparrow$} \\
    \cmidrule(lr){3-7}
    & & Audiobook & MSP-Podcast & Youtube & Voxceleb & CommonVoice \\
    \midrule
PWG && 3.00   &  2.32&2.01 &2.17 & 2.71\\ \hline
HiFiGAN && 3.56  &2.72  & 2.24&2.47 &  3.11\\ \hline
MB-MelGAN& & 2.99  & 2.29 &2.01 &2.03 &  2.68\\ \hline
Style-MelGAN&&3.25   &2.51  &2.15 & 2.34 &  2.89 \\ \hline
BigVGAN&&3.67   & 2.78 &  2.27& 2.60 &  3.21\\ \hline
BigVSAN&&3.43   & 2.58 & 2.16& 2.39& 3.03\\ \hline
UniVNet v1 &&  3.49  & 2.69  & 2.25&2.50 & 3.13 \\ \hline
UniVNet v2 & & 3.57   & 2.76 & 2.29& 2.55& 3.17 \\ \hline
WaveGrad &&3.20  & 2.42 &2.08 & 2.20& --\\ \hline
APNet2 && 3.56 &2.68& 2.22 &  2.48& -- \\
\hline
Vocos &&3.46 &2.59  &2.14 &  2.29&  2.91\\
\hline
iSTFTNet &&3.50& 2.65 & 2.22&  2.45 & --\\
         \bottomrule
    \end{tabular}
    \end{center}
\vskip -0.1in
\end{table*}


\begin{figure}[H]
 \caption{Relationship between automatic speech metrics and EER. EER is averaged over all 7 trained models\\ trained using generated utterances from respective train-time vocoder. EER and UTMOS follow similar trend \\for Audiobook dataset. While fluctuating behavior for YouTube dataset. BigVGAN vocoder shows higher EER \\and higher UTMOS score, highlighting the relationship between higher-quality audio samples and increased detection\\ difficulty.}
    \label{fig:utmos_eers}
    \vskip 0.15in
\begin{center}
\centering
    \includegraphics[width=0.9\textwidth]{Images/mos_eer_all_datasets.pdf}
   \end{center}
\vskip -0.1in
\end{figure}

\section{Data Augmentations}\label{sec:appendix_data_aug}

\paragraph{Training details} Detectors are trained with a learning rate of 1e-05, batch size of 64 and weight decay 0.0001.  Detector trained with a combination of I+II  augmentations is trained for 45 epochs. The detector trained with augmentation III is trained for 15 epochs. Models with the best validation loss are selected for evaluations.
Below tables provide detailed results for experiment discussed in \autoref{sec:augmentation}. 
\begin{table}[H]
    \caption{Evaluation on test vocoders with JSUT as source dataset. Inconsistent improvement results using combination of I+II augmentations. Overall peformance gain using augmentation of type III}
    \label{tab:data_aug_jsut}
     \vskip 0.15in
\begin{center}
\scriptsize
    \begin{tabular}{lcccccccccccc}
    \toprule
 Augmentation & \multicolumn{12}{c}{Test set }\\ 
      \cmidrule{2-13} & PWG & WaveGrad & BigVGAN & BigVSAN & MB-Mel & Univ-1 & Univ-2 & HFG & Style-Mel & Vocos & APNet2 & iSTFTNet \\
    \midrule
None & 0.0&0.10 &10.48&0.72 &0.0 &0.0 &0.24&0.34 &0.06 &0.70 & 1.70 &0.36  \\ 
\hline 
I + II &0.0 &0.16 &11.38&1.20 &0.0&0.02  &0.18&0.26&0.0&0.88 &1.88&0.44\\ 
\hline 
III &0.0&0.04&6.28&0.40&0.0 &0.0 &0.16&0.14&0.06 &0.58& 1.58 & 0.12  \\ 
         \bottomrule 
    \end{tabular}

\end{center}
\vskip -0.1in
\end{table}

\begin{table}[H]
    \caption{Evaluation on test vocoders with Audiobook as source dataset. Inconsistent improvement results using combination of I+II augmentations. Overall peformance gain using augmentation of type III}
    \label{tab:data_aug_audiobook}
     \vskip 0.15in
\begin{center}
\scriptsize
    \begin{tabular}{lcccccccccccc}
    \toprule
 Augmentation & \multicolumn{12}{c}{Test set }\\ 
      \cmidrule{2-13} & PWG & WaveGrad & BigVGAN & BigVSAN & MB-Mel & Univ-1 & Univ-2 & HFG & Style-Mel & Vocos & APNet2 & iSTFTNet \\
    \midrule
None &0.98&5.21 &26.12&10.77 &0.75 &3.63 &6.30&8.02&5.56 &21.85&12.05 &7.79  \\ 
\hline 
I + II &1.51&6.79 &28.03&13.25&1.29 &3.70&6.78&8.71&4.14 &17.92&13.40 &8.58\\ 
\hline 
III &1.67&4.98&24.25&10.73&1.23&3.10 &5.99&7.19&5.71&17.34 & 12.86 & 6.24 \\ 
         \bottomrule 
    \end{tabular}

\end{center}
\vskip -0.1in
\end{table} 

\begin{table}[H]
    \caption{Evaluation on test vocoders with Podcast as source dataset. No improvement using combination of I+II augmentations. Performance gain observed for -- Univ-1, Vocos and iSTFTNet using augmentation of type III}
    \label{tab:data_aug_podcast}
     \vskip 0.15in
\begin{center}
\scriptsize
    \begin{tabular}{lcccccccccccc}
    \toprule
 Augmentation & \multicolumn{12}{c}{Test set }\\ 
      \cmidrule{2-13} & PWG & WaveGrad & BigVGAN & BigVSAN & MB-Mel & Univ-1 & Univ-2 & HFG & Style-Mel & Vocos & APNet2 & iSTFTNet \\
    \midrule
None &1.20&5.61 &24.46&9.53&1.34 &3.90 &5.78&8.81&4.40 &16.69 &11.34 &7.24  \\ 
\hline 
I + II &3.78&10.95  &29.66&15.08&3.19 &6.32 &9.52&13.25&6.20&18.26&16.45 &11.40\\ 
\hline 
III &2.01&6.31 &26.46&10.76 &1.82&3.62 &6.04&9.25   &5.59 &14.72 & 14.49 & 6.62\\ 
         \bottomrule 
    \end{tabular}

\end{center}
\vskip -0.1in
\end{table}

\begin{table}[H]
    \caption{Evaluation on test vocoders with YouTube as source dataset. No improvement using combination of I+II augmentations. Performance gain observed only for -- BigVGAN using augmentation of type III}
    \label{tab:data_aug_youtube}
     \vskip 0.15in
\begin{center}
\scriptsize
    \begin{tabular}{lcccccccccccc}
    \toprule
 Augmentation & \multicolumn{12}{c}{Test set }\\ 
      \cmidrule{2-13} & PWG & WaveGrad & BigVGAN & BigVSAN & MB-Mel & Univ-1 & Univ-2 & HFG & Style-Mel & Vocos & APNet2 & iSTFTNet \\
    \midrule
None & 9.45&13.29 &30.84&20.65&10.37 &10.88&11.51&16.61   &10.33&37.11&19.17 &15.92 \\ 
\hline 
I + II &10.90&19.86  &34.61&25.49&12.77&12.75&14.22&20.23&14.18 &33.46&23.25 &19.61\\ 
\hline 
III &14.30&16.67 &29.28&21.41&13.76&12.56 &13.15&17.26&13.24 & 35.43 & 20.38 & 17.03 \\ 
         \bottomrule 
    \end{tabular}

\end{center}
\vskip -0.1in
\end{table}
\begin{table}[H]
    \caption{Evaluation on test vocoders with AISHELL as source dataset. Inconsistent improvement using combination of I+II augmentations. Degradation in performance using augmentation of type III}
    \label{tab:data_aug_aishell}
     \vskip 0.15in
\begin{center}
\scriptsize
    \begin{tabular}{lcccccccccccc}
    \toprule
 Augmentation & \multicolumn{12}{c}{Test set }\\ 
      \cmidrule{2-13} & PWG & WaveGrad & BigVGAN & BigVSAN & MB-Mel & Univ-1 & Univ-2 & HFG & Style-Mel & Vocos & APNet2 & iSTFTNet \\
    \midrule
None & 2.24&8.41 &25.30&12.41 &2.74&3.05 &6.56&8.47   &4.76&31.36&16.04   & 6.80 \\ 
\hline 
I + II &2.29&10.89  &27.14&14.22 &1.96 & 2.07&4.08&6.21 &3.13 &27.04&11.45&5.24\\ 
\hline 
III &6.04&15.90 &34.00&20.01&5.37&3.95&8.02&11.99 &8.37&41.10 & 24.08 & 10.14 \\ 
         \bottomrule 
    \end{tabular}
\end{center}
\vskip -0.1in
\end{table}

\begin{table}[H]
    \caption{Evaluation on test vocoders with VoxCeleb as source dataset. Performance improvement only on -- Univ-1 and Vocos using combination of I+II augmentations. Degradation in performance using augmentation of type III}
    \label{tab:data_aug_voxceleb}
     \vskip 0.15in
\begin{center}
\scriptsize
    \begin{tabular}{lcccccccccccc}
    \toprule
 Augmentation & \multicolumn{12}{c}{Test set }\\ 
      \cmidrule{2-13} & PWG & WaveGrad & BigVGAN & BigVSAN & MB-Mel & Univ-1 & Univ-2 & HFG & Style-Mel & Vocos & APNet2 & iSTFTNet \\
    \midrule
None &1.00&5.80 &23.49&6.81 &1.04 &4.22 &5.02&7.14 &2.31 &27.39&9.99&6.70  \\ 
\hline 
I + II &1.80 &10.01  &26.63&9.82 &1.31&3.44&5.25&8.41   &2.80 &23.16&10.97  &7.57\\ 
\hline 
III &1.99&8.20&27.88&9.33  &1.70&4.24 &5.58&9.00 &3.26&28.90 & 13.00 & 8.08 \\ 
         \bottomrule 
    \end{tabular}

\end{center}
\vskip -0.1in
\end{table}

\section{Increasing number of speakers}
\label{sec:more_spks}
Following tables presents detailed results for experiments in \autoref{sec:more_spks}. 
\paragraph{Training details:} The detection models were trained for 40 epochs with a learning rate of 1e-5 and batch size 64, selecting the model with the lowest validation loss. 
\subsection{Performance on in-domain vocoder}
Detector in experiment \autoref{sec:more_spks} was trained on audio samples from HFG vocoder. Here, the evaluations are done on audio samples generated using in-domain vocoder HFG 
\begin{table}[H]
\caption{Model trained on synthetic audio samples generated using HFG vocoder, train-clean360 LibriTTS and tested on HFG vocoder generated samples with JSUT as source dataset. No significant drop in EER \% was observed by increasing number of speakers beyond four }
\label{tab:inc-spk-jsut}
\vskip 0.15in
\begin{center}
\scriptsize
\resizebox{\textwidth}{!}{
\begin{tabular}{|c|c|c|c|c|c|c|c|c|c|c|}
\hline
Exp no. & \textbf{spk1}& \textbf{spks2} &\textbf{spks3}  &\textbf{spks4}  &\textbf{spks5}  &\textbf{spks6}  &\textbf{spks7}  &\textbf{spks8}   &\textbf{spks9}  &\textbf{spks10}\\
\hline 
1 &0.84&0.56&0.82&0.82&1.32&0.86&0.58&0.66&1.22 &1.02\\ 
\hline
2 &1.42&0.98&0.90&1.14&1.04&1.86&0.72&1.06&1.02 &1.28\\ 
\hline
3 &1.88&1.24&1.66&1.28&0.88&1.02&0.90&0.54&1.02 &1.96\\ 
\hline
4 &11.56&4.04&2.66&2.22&0.82&2.60&1.72&0.86&2.20 &1.32\\ 
\hline
5 &1.94&2.14&1.88&1.26&0.68&1.30&0.88&1.18&1.84 &1.24\\ 
\hline
\textbf{aEER} &\textcolor{red}{3.52} &1.79&1.58&1.34&0.94&1.52&0.96&\textcolor{darkgreen}{0.86}&1.46 &1.36\\ 
\bottomrule
\end{tabular}}%
\end{center}
\vskip -0.1in
\end{table}


\begin{table}[H]
\caption{Model trained on synthetic audio samples generated using HFG vocoder, train-clean360 LibriTTS and tested on HFG vocoder generated samples with Audiobook as source dataset. No significant drop in EER \% was observed by increasing number of speakers beyond four}
\label{tab:inc-spk-audiobook}
\vskip 0.15in
\begin{center}
\scriptsize
\resizebox{\textwidth}{!}{
\begin{tabular}{|c|c|c|c|c|c|c|c|c|c|c|}
\hline
Exp no. & \textbf{spk1}& \textbf{spks2} &\textbf{spks3}  &\textbf{spks4}  &\textbf{spks5}  &\textbf{spks6}  &\textbf{spks7}  &\textbf{spks8}   &\textbf{spks9}  &\textbf{spks10}\\
\hline 
1 &29.59&16.91&9.94&9.14&12.84&10.07&12.74&10.51&14.31 &9.11\\ 
\hline
2 &23.29&12.50&10.71&10.30&9.69&9.50&9.00&8.85&10.24 &8.62\\ 
\hline
3 &12.36&10.55&8.00&9.06&8.00&8.26&7.09&7.06&8.24 &9.29\\ 
\hline
4 &23.75&17.48&28.89&9.72&12.35&11.49&10.29&9.98&9.98 &9.31\\ 
\hline
5 &27.66&13.44&11.36&12.22&9.03&10.18&11.07&8.44&9.21 &9.12\\ 
\hline
\textbf{aEER} &\textcolor{red}{23.33}&14.17&13.78&10.08&10.38&9.90&10.03&\textcolor{darkgreen}{8.96} &10.39 & 9.09\\ 
\bottomrule
\end{tabular}}%
\end{center}
\vskip -0.1in
\end{table}


\begin{table}[H]
\caption{Model trained on synthetic audio samples generated using HFG vocoder, train-clean360 LibriTTS and tested on HFG vocoder generated samples with Podcast as source dataset. No significant drop in EER \% was observed by increasing number of speakers beyond four}
\label{tab:inc-spk-podcast}
\vskip 0.15in
\begin{center}
\scriptsize
\resizebox{\textwidth}{!}{
\begin{tabular}{|c|c|c|c|c|c|c|c|c|c|c|}
\hline
Exp no. & \textbf{spk1}& \textbf{spks2} &\textbf{spks3}  &\textbf{spks4}  &\textbf{spks5}  &\textbf{spks6}  &\textbf{spks7}  &\textbf{spks8}   &\textbf{spks9}  &\textbf{spks10}\\
\hline 
1 &33.29&20.36&11.33&11.61&13.78&12.40&21.08&11.85&12.42 &10.58\\ 
\hline
2 &23.01&15.50&11.09&12.48&12.38&12.55&9.79&8.71&12.54 &13.51\\ 
\hline
3 &14.47&15.04&12.79&9.90&10.11&8.88&10.86&7.31&9.17&12.29\\ 
\hline
4 &25.19&21.00&29.80&10.99&15.09&16.03&15.10&12.38&13.02 &10.36\\ 
\hline
5 &20.98&12.11&14.90&17.71&10.28&13.84&12.68&11.89&11.24 &14.49\\ 
\hline
\textbf{aEER} &\textcolor{red}{23.38}&16.80&15.98&12.53&12.32&12.74&13.90&\textcolor{darkgreen}{10.42}&11.67 &12.24\\ 
\bottomrule
\end{tabular}}%
\end{center}
\vskip -0.1in
\end{table}

\begin{table}[H]
\caption{Model trained on synthetic audio samples generated using HFG vocoder, train-clean360 LibriTTS and tested on HFG vocoder generated samples with Youtube as source dataset. No significant drop in EER \% was observed by increasing number of speakers beyond four}
\label{tab:in-spk-yt}
\scriptsize
\vskip 0.15in
\begin{center}
\resizebox{\textwidth}{!}{
\begin{tabular}{|c|c|c|c|c|c|c|c|c|c|c|}
\hline
Exp no. & \textbf{spk1}& \textbf{spks2} &\textbf{spks3}  &\textbf{spks4}  &\textbf{spks5}  &\textbf{spks6}  &\textbf{spks7}  &\textbf{spks8}   &\textbf{spks9}  &\textbf{spks10}\\
\hline 
1 &36.99&28.33&24.41&24.89&23.50&26.08&27.15&23.44&20.96 &22.99\\ 
\hline
2 &31.32&23.81&21.48&21.76&23.75&22.24&21.70&20.05&21.05 &21.45\\ 
\hline
3 &24.45&22.90&20.64&19.77&19.33&19.86&20.07&19.21&20.07 &22.44\\ 
\hline
4 &31.14&28.22&43.36&21.32&25.31&25.52&23.94&21.72&23.13 &24.66\\ 
\hline
5 &27.74&21.44&21.86&24.68&22.74&24.01&21.93&24.64&22.33 &22.37\\ 
\hline
\textbf{aEER} &\textcolor{red}{30.22}&24.94&26.35&22.48&22.92&23.54&22.95&21.81& \textcolor{darkgreen}{21.50}&22.78\\ 
\bottomrule
\end{tabular}}%
\end{center}
\vskip -0.1in
\end{table}

\begin{table}[H]
\caption{Model trained on synthetic audio samples generated using HFG vocoder, train-clean360 LibriTTS and tested on HFG vocoder generated samples with VoxCeleb as source dataset. No significant drop in EER \% was observed by increasing number of speakers beyond four}
\label{tab:inc-spk-voxceleb}
\scriptsize
\vskip 0.15in
\begin{center}
\resizebox{\textwidth}{!}{
\begin{tabular}{|c|c|c|c|c|c|c|c|c|c|c|}
\hline
Exp no. & \textbf{spk1}& \textbf{spks2} &\textbf{spks3}  &\textbf{spks4}  &\textbf{spks5}  &\textbf{spks6}  &\textbf{spks7}  &\textbf{spks8}   &\textbf{spks9}  &\textbf{spks10}\\
\hline 
1 &24.00&16.86&8.24&7.83&10.38&11.75&18.15&9.06&10.23&8.14\\ 
\hline
2 &23.40&17.07&8.71&7.91&8.86&8.69&7.44&6.52&7.63 &9.62\\ 
\hline
3 &12.98&13.25&9.29&7.79&7.32&6.70&7.71&6.23&7.14&11.12\\ 
\hline
4 &20.53&18.85&32.21&7.75&14.01&17.68&13.23&8.76&13.21&8.33\\ 
\hline
5 &15.77&8.06&7.91&11.88&6.81&8.02&10.03&9.72&8.90 &8.67\\ 
\hline
\textbf{aEER} &\textcolor{red}{19.33}&14.81&13.27&8.63&9.47&10.56&11.31&\textcolor{darkgreen}{8.05}&9.42 &9.17\\ 
\bottomrule
\end{tabular}}%
\end{center}
\vskip -0.1in
\end{table}

\begin{table}[h]
\caption{Model trained on synthetic audio samples generated using HFG vocoder, train-clean360 LibriTTS and tested on HFG vocoder generated samples with AISHELL as source dataset. No significant drop in EER \% was observed by increasing number of speakers beyond four}
\label{tab:inc-spks-aishell}
\vskip 0.15in
\begin{center}
\scriptsize
\resizebox{\textwidth}{!}{
\begin{tabular}{|c|c|c|c|c|c|c|c|c|c|c|}
\hline
Exp no. & \textbf{spk1}& \textbf{spks2} &\textbf{spks3}  &\textbf{spks4}  &\textbf{spks5}  &\textbf{spks6}  &\textbf{spks7}  &\textbf{spks8}   &\textbf{spks9}  &\textbf{spks10}\\
\hline 
1 &28.84&24.34&18.91&8.32&29.01&28.31&21.72&22.33&20.19 &15.51\\ 
\hline
2 &38.51&30.63&24.17&11.27&20.13&16.09&18.74&13.54&13.47 &16.90\\ 
\hline
3 &21.40&35.49&13.43&10.98&15.24&12.23&16.57&5.18&11.92 &17.22\\ 
\hline
4 &27.02&34.76&38.68&23.23&24.12&37.90&23.76&19.05&27.85 &11.92\\ 
\hline
5 &18.61&19.74&18.22&21.80&9.04&20.37&15.11&19.07&11.65 &14.11\\ 
\hline
\textbf{aEER} &26.87&\textcolor{red}{28.99}&22.68&\textcolor{darkgreen}{15.12}&19.50&22.98&19.18&15.83&17.01 &15.13\\ 
\bottomrule
\end{tabular}}%
\end{center}
\vskip -0.1in
\end{table}

\subsection{Performance on out-of-domain vocoder}
In the below experiments we investigate if similar behavior is observed with out-of-domain vocoder used in evaluations. Here, the out-of-domain vocoder used is PWG 

\begin{table}[H]
\caption{Model trained on synthetic audio samples generated using HFG vocoder, train-clean360 LibriTTS and tested on PWG vocoder generated samples with JSUT as source dataset.}
\label{tab:inc-spk-jsut-pwg}
\vskip 0.15in
\begin{center}
\scriptsize
\resizebox{\textwidth}{!}{
\begin{tabular}{|c|c|c|c|c|c|c|c|c|c|c|}
\hline
Exp no. & \textbf{spk1}& \textbf{spks2} &\textbf{spks3}  &\textbf{spks4}  &\textbf{spks5}  &\textbf{spks6}  &\textbf{spks7}  &\textbf{spks8}   &\textbf{spks9}  &\textbf{spks10}\\
\hline 
1 &0.02 & 0.02 & 0.08 & 0.0 & 0.10 & 0.02 & 0.40 & 0.0 & 0.0 & 0.0\\ 
\hline
2 &0.02& 0.04 & 0.0 & 0.0 & 0.18 & 0.16 & 0.04 & 0.0 & 0.0 & 0.22\\ 
\hline
3 &0.22 & 0.02 & 0.58 & 0.06 & 0.20 & 0.0& 0.02& 0.02 & 0.08 & 0.02\\ 
\hline
4 &0.30& 0.04 & 0.34 & 0.06 & 0.0 & 0.02 & 0.0& 0.0 & 0.02 & 0.0\\ 
\hline
5 &0.0 & 0.0 & 0.04 & 0.10 & 0.0 & 0.64 & 0.0 & 0.02 & 0.38 & 0.28\\ 
\hline
\textbf{aEER} & \textcolor{red}{0.11}& 0.02 & 0.20 & 0.04 & 0.09 & 0.16 & 0.09 & \textcolor{darkgreen}{0.008} & 0.088 & 0.10\\ 
\bottomrule
\end{tabular}}%
\end{center}
\vskip -0.1in
\end{table}


\begin{table}[H]
\caption{Model trained on synthetic audio samples generated using HFG vocoder, train-clean360 LibriTTS and tested on PWG vocoder generated samples with AISHELL as source dataset.}
\label{tab:inc-spk-aishell-pwg}
\vskip 0.15in
\begin{center}
\scriptsize
\resizebox{\textwidth}{!}{
\begin{tabular}{|c|c|c|c|c|c|c|c|c|c|c|}
\hline
Exp no. & \textbf{spk1}& \textbf{spks2} &\textbf{spks3}  &\textbf{spks4}  &\textbf{spks5}  &\textbf{spks6}  &\textbf{spks7}  &\textbf{spks8}   &\textbf{spks9}  &\textbf{spks10}\\
\hline 
1 &22.71 & 19.28 & 16.86 & 5.58 & 26.10 & 27.45 & 23.02 & 21.12 & 16.57 & 12.09\\ 
\hline
2 &38.51 & 30.63 & 24.17 & 11.27 & 20.13 & 16.09 &  18.74 & 13.54 & 13.47 & 16.90\\ 
\hline
3 &21.40 & 35.49 & 13.43 & 10.98 & 15.24 & 12.23 & 16.57 & 5.18 & 11.92 & 17.22\\ 
\hline
4 &27.02 & 34.76 & 38.68 & 23.23 & 24.12& 37.90 & 23.76 & 19.05 & 27.85 & 11.92\\ 
\hline
5 &18.61 & 19.74 & 18.22 & 21.80 & 9.04 & 20.37 & 15.11 & 19.07 & 11.65 & 14.11\\ 
\hline
\textbf{aEER} &25.65 & \textcolor{red}{27.98} & 22.27 & 14.57 & 18.92 & 22.80 & 19.44 & 15.59 & 16.29 & \textcolor{darkgreen}{14.44}\\ 
\bottomrule
\end{tabular}}%
\end{center}
\vskip -0.1in
\end{table}

\begin{table}[H]
\caption{Model trained on synthetic audio samples generated using HFG vocoder, train-clean360 LibriTTS and tested on PWG vocoder generated samples with VoxCeleb as source dataset.}
\label{tab:inc-spk-}
\vskip 0.15in
\begin{center}
\scriptsize
\resizebox{\textwidth}{!}{
\begin{tabular}{|c|c|c|c|c|c|c|c|c|c|c|}
\hline
Exp no. & \textbf{spk1}& \textbf{spks2} &\textbf{spks3}  &\textbf{spks4}  &\textbf{spks5}  &\textbf{spks6}  &\textbf{spks7}  &\textbf{spks8}   &\textbf{spks9}  &\textbf{spks10}\\
\hline 
1 &16.18 & 12.39 & 3.40 & 2.44 & 4.76& 9.72 & 17.37 & 4.61 & 4.45 & 4\\ 
\hline
2 &16.22 & 12.47 & 4.90 & 5.25 & 8.37 & 7.12 & 4.55 & 2.75 & 5.58 & 10.36\\ 
\hline
3 &10.29 & 11.44 & 6.54 & 2.95 & 3.44 & 4.06 & 4.88 & 2.40 & 3.87 & 7.42\\ 
\hline
4 &11.77 & 16.66 & 32.64 & 4.24 & 12.43 & 16.82 & 11.53 & 8.00 & 13.60 & 4.98\\ 
\hline
5 &5.33 & 2.40 & 4.16 & 9.52 & 3.38 & 5.68 & 5.29 & 6.50 & 3.59 & 6.34\\ 
\hline
\textbf{aEER} &\textcolor{red}{11.95} & 11.07 & 10.32 & 4.88 & 6.47 & 8.68 & 8.72 & \textcolor{darkgreen}{4.85} & 6.21 & 6.74\\ 
\bottomrule
\end{tabular}}%
\end{center}
\vskip -0.1in
\end{table}

\begin{table}[H]
\caption{Model trained on synthetic audio samples generated using HFG vocoder, train-clean360 LibriTTS and tested on PWG vocoder generated samples with Audiobook as source dataset.}
\label{tab:inc-spk-pwg-audiobook}
\vskip 0.15in
\begin{center}
\scriptsize
\resizebox{\textwidth}{!}{
\begin{tabular}{|c|c|c|c|c|c|c|c|c|c|c|}
\hline
Exp no. & \textbf{spk1}& \textbf{spks2} &\textbf{spks3}  &\textbf{spks4}  &\textbf{spks5}  &\textbf{spks6}  &\textbf{spks7}  &\textbf{spks8}   &\textbf{spks9}  &\textbf{spks10}\\
\hline 
1 &20.59 & 12.49 & 4.31 & 2.15 & 6.95 & 6.38 & 11.09 & 4.86 & 6.07 & 3.63\\ 
\hline
2 &14.83 & 7.70 & 6.21 & 5.19 & 7.14 & 5.38 & 4.24 & 2.40 & 5.40 & 6.81\\ 
\hline
3 &9.11 & 5.73 & 3.74 & 3.30 & 2.22 & 2.29 & 2.80 & 1.53 & 2.41 & 3.45\\ 
\hline
4 &10.46 & 11.98 & 28.52 & 3.71 & 8.80 & 7.62 & 6.79 & 6.34 & 6.57 & 2.69\\ 
\hline
5 &14.25 & 3.34 & 6.39 & 8.02 & 2.84 & 5.94 & 3.56 & 3.95 & 3.04 & 6.20\\ 
\hline
\textbf{aEER} &\textcolor{red}{13.84} & 8.24 & 9.83 & 4.47 & 5.59 & 5.52 & 5.69 & \textcolor{darkgreen}{3.81} & 4.69 & 4.55\\ 
\bottomrule
\end{tabular}}%
\end{center}
\vskip -0.1in
\end{table}

\begin{table}[H]
\caption{Model trained on synthetic audio samples generated using HFG vocoder, train-clean360 LibriTTS and tested on PWG vocoder generated samples with Podcast as source dataset.}
\label{tab:inc-spk-pwg-podcast}
\vskip 0.15in
\begin{center}
\scriptsize
\resizebox{\textwidth}{!}{
\begin{tabular}{|c|c|c|c|c|c|c|c|c|c|c|}
\hline
Exp no. & \textbf{spk1}& \textbf{spks2} &\textbf{spks3}  &\textbf{spks4}  &\textbf{spks5}  &\textbf{spks6}  &\textbf{spks7}  &\textbf{spks8}   &\textbf{spks9}  &\textbf{spks10}\\
\hline 
1 &26.42 & 18.61 & 7.37 & 4.77 & 8.09 & 9.69 & 20.09 & 6.48 & 6.00 & 6.85\\ 
\hline
2 &14.14 & 10.18 & 7.02 & 9.46 & 11.58 & 11.19 & 5.94 & 3.55 & 9.64 & 14.49\\ 
\hline
3 &13.03 & 12.72& 9.61 & 5.66 & 5.35 & 7.05 & 7.42 & 3.88 & 6.09 & 7.12\\ 
\hline
4 &12.97 & 16.00 & 30.00& 5.79 & 12.85 & 13.98 & 12.24 & 10.87 & 11.87 & 6.39\\ 
\hline
5 &9.42 & 4.89 & 10.03 & 13.66 & 5.49 & 10.70 & 7.23 & 7.69& 7.21 & 11.48\\ 
\hline
\textbf{aEER} &\textcolor{red}{15.19} & 12.48 & 12.80 & 6.55 & 8.67 & 10.52 & 10.58 & \textcolor{darkgreen}{6.49} & 8.16 & 9.26\\ 
\bottomrule
\end{tabular}}%
\end{center}
\vskip -0.1in
\end{table}

\begin{table}[H]
\caption{Model trained on synthetic audio samples generated using HFG vocoder, train-clean360 LibriTTS and tested on PWG vocoder generated samples with YouTube as source dataset.}
\label{tab:inc-spk-pwg-yt}
\vskip 0.15in
\begin{center}
\scriptsize
\resizebox{\textwidth}{!}{
\begin{tabular}{|c|c|c|c|c|c|c|c|c|c|c|}
\hline
Exp no. & \textbf{spk1}& \textbf{spks2} &\textbf{spks3}  &\textbf{spks4}  &\textbf{spks5}  &\textbf{spks6}  &\textbf{spks7}  &\textbf{spks8}   &\textbf{spks9}  &\textbf{spks10}\\
\hline 
1 &30.80 & 25.93 & 21.61 & 22.12 & 20.37 & 24.08 & 25.88 & 21.30 & 19.20 & 20.76 \\ 
\hline
2 &26.96 & 22.00 & 19.72 & 19.55 & 22.02 & 20.21 & 20.35 & 18.77 & 18.52 & 20.37\\ 
\hline
3 &21.55 & 21.00 & 17.58 & 16.50 & 16.78 & 18.04 & 17.54 & 17.35 & 18.33 & 20.60\\ 
\hline
4 &23.37 & 26.65 & 43.86 & 18.93 & 23.66 & 23.68 & 21.91 & 20.76 & 21.35 & 22.39\\ 
\hline
5 &20.16 & 17.82 & 19.05 & 23.76 & 20.48 & 22.16 & 20.44 & 23.27 & 20.56 & 20.57\\ 
\hline
\textbf{aEER} &24.56 & 22.68 & 24.36 & 20.17 & 20.66 & 21.63 & 21.22 & 20.29 & \textcolor{darkgreen}{19.59} & 20.93\\ 
\bottomrule
\end{tabular}}%
\end{center}
\vskip -0.1in
\end{table}

\begin{figure}[t]
\vskip 0.2in 
\begin{center}
    \centerline{\includegraphics[width=0.9\textwidth]{Images/hfg_train_pwg_test_spks.pdf}}
    \caption{Average EERs reported with models trained on increasing number of speakers with HFG generated audio samples (LibriTTS, train-clean-360). For each dataset average EER on all 12 test vocoders are plotted. Test samples are generated using PWG vocoder. aEER drops significantly when number of speakers are increased to four. No significant performance gain was observed thereafter, for most cases.}
    \label{fig:hfg_out-of-domain}
\end{center}
\vskip -0.2in 
\end{figure}
