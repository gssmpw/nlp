Driven by advances in self-supervised learning for speech, state-of-the-art synthetic speech detectors have achieved low error rates on popular benchmarks such as ASVspoof. However, prior benchmarks do not address the wide range of real-world variability in speech. Are reported error rates realistic in real-world conditions? To assess detector failure modes and robustness under controlled distribution shifts, we introduce \methodName, a benchmark with more than 3000 hours of synthetic speech from 7 domains, 6 TTS systems, 12 vocoders, and 3 languages. We found that \emph{all} distribution shifts degraded model performance, and contrary to prior findings, training on more vocoders, speakers, or with data augmentation did not guarantee better generalization. In fact, we found that training on less diverse data resulted in better generalization, and that a detector fit using samples from a single carefully selected vocoder and a small number of speakers, without data augmentations, achieved state-of-the-art results on the challenging In-the-Wild benchmark.
