
\section{Findings}

To organize our findings, we start by describing high-level findings, including the key characteristics of the practitioners. Next, we present how practitioners used GenAI technologies to transform their overall work, specifically their work processes, role attributes, and professional relationships. 
%Using the theory of job crafting \cite{Wrzesniewski_2001,tims2010job} as an anchor, we start by examining how individuals shaped their tasks and responsibilities through task crafting techniques. We then examine how practitioners have molded their professional networks through relationship crafting techniques. Lastly, we show how practitioners use cognitive crafting techniques to make meaning of their changing roles.




\begin{table}
    \centering
    \begin{tabular}{|c|p{0.7\textwidth}|}
        \hline
        Occupations & Software developer, manager, lecturer, data engineer, UX designer, UX researcher, UX writer, industrial design,  artists,  UI designer, social worker,  architect, fact-checker, knowledge worker, fiction writer, researcher, speech-language pathologist, executive \\
        \hline
        Industries & Education, technology, art/culture, design firm, agroindustry, health,  IT, fact checking, gaming, science fiction, research \\ 
        \hline 
        % GenAI Products & \fixme{ChatGPT, Copilot, Midjourney, Dall-E, Llama, (there are a lot of open-source models listed in the paper that lists Llama/Vicuna)} \\
        % \hline
        % GenAI Modalities & text (large language model or llm chatbots), audio (phone, podcast), images (text-to-image generators), code (programming assistants) \\
        % \hline
        % GenAI Interfaces &  standalone, integrated, custom \\
        % \hline
        
    \end{tabular}
    \caption{Paper Demographics}
    \label{tab:paper_demographics}
\end{table}

% The papers included in our analysis covered 18 different occupations across 11 different industries. The majority (n=15) focused on practitioners in technology-focused or design roles, although a few papers (n=8) also studied roles outside this archetype, including artists, speech-language pathologists, and writers. Sometimes these lines blurred, e.g., in game design settings involving developers and artists. 

The papers included in our analysis covered 18 different occupations across 11 different industries. The majority focused primarily on practitioners in technology or design-focused roles (n=15). However, we found a few papers also studying roles outside this archetype, including artists, speech-language pathologists, and writers (n=8). The sectors in which these practitioners worked spanned from technology and design to fact-checking, research, and health. For a full breakdown of the occupations and industries included in our survey, please see Table~\ref{tab:paper_demographics}.

In the papers examined, practitioners used a wide range of GenAI tools, including off-the-shelf products such as ChatGPT, Midjourney, Copilot, and LLaMA. The modalities of these tools varied, spanning text (e.g., an LLM-driven chatbot), audio (e.g., voice), images (e.g., a text-to-image generator), and code (e.g., an LLM-driven coding assistant). While many practitioners used standalone tools like Midjourney, others utilized GenAI-enabled features integrated into commercial platforms, such as Notion or Figma plugins~\cite{6}, or custom-developed tools designed for specific purposes, such as a chatbot that interacts with patients to populate a health dashboard~\cite{5}.

%Across the papers we surveyed, there were a number of factors that played into when and how practitioners used GenAI tools to craft their roles. These factors included both extrinsic and intrinsic motivators. In an extrinsic sense, practitioners were influenced by contextual factors locally and within their companies. For example, uncertainty around government policies affected a practitioner's decision to adopt GenAI tools~\cite{1}. Similarly, company policies, from whether the company paid for a subscription~\cite{18} to restrictions around data and confidentiality~\cite{9} also play a crucial role in adoption. Some practitioners even wondered whether their employers would at some point mandate the use of GenAI in their work~\cite{9}. Even in the absence of company policies, general opacity around how one's co-workers used GenAI factored into practitioners' own choices~\cite{10}. Finally, perceptions about how well GenAI models portrayed local culture also affected practitioners' decisions about using it~\cite{1}.

%Practitioners also accounted for several intrinsic factors when considering adopting GenAI in their work. These elements were related to practitioners' own values, perceptions of self, and goals. For instance, some practitioners expressed discomfort over directly using images generated by GenAI in case the model appropriated others' work in its training data~\cite{9,22}. Other practitioners had reservations about using GenAI to design things without human input~\cite{2} or struggled with its ability to capture the human emotions they wanted it to convey ~\cite{22}. For some, matters of skill were an important consideration. On the one hand, there were those who had misgivings about their ability to use the tools. When dealing with complex work scenarios, they felt that their technical expertise was not good enough to make the most of GenAI~\cite{9,10}. On the other hand, some expressed a fear of relying too much on GenAI, leading to deleterious consequences by reducing their own autonomy or problem-solving abilities~\cite{10}.

% There were, of course, practical considerations as well, ranging from concerns about hallucinations and mistakes leading to reputational harm~\cite{1} to questions of efficiency. Here there were both those who believed that working with GenAI reduced their productivity~\cite{22} or was difficult to iterate with~\cite{18}, as well as those who saw its use as a way to increase the volume of their output~\cite{3,9}, streamline their processes~\cite{8}, speed up their work~\cite{6,7,9,22,10,14}, remove bottlenecks~\cite{3}, and free up time~\cite{6}. With that said, some practitioners believed that speeding up their work with GenAI came at a significant cost, especially as external pressures adjusted to this speedup. As one graphic designer lamented, 

% \begin{quote}
% \textit{``I know this is not helping me create original, thoughtful works, but I do not want to fall behind in this new commercial art market momentum''~\cite{14}.}
% \end{quote}

%Overall, we found that practitioners across the papers that we surveyed balanced a number of external and internal factors when considering whether and how to make use of GenAI tools for job crafting. 
In the following subsections, we describe different transformations practitioners brought in their work to incorporate GenAI. Throughout our findings, we differentiate between avoidance crafting and approach crafting. Avoidance crafting refers to practices undertaken by an individual to reduce negative work outcomes, while approach crafting refers to the active creation of opportunities that align with one's professional preferences, strengths, and goals~\cite{zhang_2019}.
