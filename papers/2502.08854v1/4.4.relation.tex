\subsection{Transforming Professional Relationships}
\label{sec:relational}
We also saw preliminary evidence of practitioners using GenAI to shape their professional relationships (relational crafting). We believe it may be possible that the ability to interact with the GenAI tools in a two-way synchronous dialogue encouraged practitioners to implicitly have the same expectations from the GenAI that they might have from their human peers. Such expectations were reflected in practitioners' use of GenAI in specific aspects of their work that would usually be conducted through collaborations with peers. This involved participants asking GenAI to reflect on their works, to critique their work, and to provide alternative perspectives. 

For example, while working on their design project, a UI/UX designer used ChatGPT to set up checkpoints across their project life-cycle to help assess whether the current state of their artifact aligned with their envisioned outcome~\cite{6}. A more interesting use case was where fact-checkers in a Sudanese newsroom used GenAI to perform adversarial analysis by asking it to challenge the assumptions in articles written by the team~\cite{1}. Practitioners also established support networks on online platforms \cite{7,18}, such as Meta groups, to exchange specific prompts that enabled them to unlock new ways to engage with GenAI. 
%On several occasions, the comparisons between GenAI and human labor were made more explicit by practitioners anthropomorphizing the GenAI tools. These practitioners treated them as humans by acknowledging the human labor that they were doing with the tools, which was otherwise reserved for human roles.
Creative practitioners in~\cite{10} and~\cite{6} equated working with GenAI to ``\textit{quickly brainstorming with someone}.'' One practitioner shared:

\begin{quote}
    \textit{``I approached it similarly to how I would collaborate with someone. I could just go in with something half-baked and know that the system would ask me to clarify [if it needs it] \dots It really felt like a partnership [that] I found useful.''}
\end{quote}

A few UX designers and researchers replaced their collaborators with GenAI to explore hypothetical what-if scenarios before distilling the insights into eventual designs. For instance, a UX practitioner in \cite{21} fed empathy maps from an interview analysis and asked GenAI to emulate more diverse personas related to the real interviewees to envision a wider range of scenarios. 
%This facilitated a more ``comprehensive understanding of the user’s perspective'' and ``enhanced the design exploration process''. 
In more extreme cases, practitioners experimented with stretching the design boundaries by assigning a non-human actor persona (e.g., technological systems or consumer objects) to GenAI to elicit design ideas. In one instance, the researcher asked ChatGPT to speak on behalf of a set of ESP32 microcontrollers, imagining how they (the microcontroller chips) might perceive their own role in the project~\cite{21}.

The perceived anthropomorphized contributions could directly impact practitioners' existing professional relationships. One way this materialized was when GenAI was used to replace ``taken for granted'' tasks that were outsourced to other roles, such as transcription and translation \cite{1}. In more drastic situations, practitioners used GenAI to reduce their dependencies on different stakeholders. This was partially motivated by \textit{external} factors, such as intent to avoid wasting the time of specialized but overburdened roles or to save time and resources. For instance, UX designers used GenAI in their startup \cite{2} instead of hiring a UX writer to reduce the overhead costs. On the other hand, some of the reasons were \textit{internally} motivated, such as shortening the turn around time that was occurring because of practitioners' dependencies on other roles, such as experts. In \cite{9}, creative leaders, instead of waiting for artists' input, ``side-stepped'' them and developed their own pitch decks for client communication using GenAI. 

Practitioners used GenAI to navigate personal and infrastructural constraints. For example, graphic designers in the Global South faced unreliable internet and power outages, making it difficult to brainstorm with clients~\cite{14}. To maintain pace of progress, practitioners compensated for the gap by brainstorming with ChatGPT. Practitioners also used GenAI to reduce interactions that contributed to increased stress in their jobs. For example, in \cite{5}, telephone operators working in the healthcare domain were tasked with the emotional labor of reaching out to older adults on a regular basis, having conversations with them, and recording any issues impacting their lives. GenAI was introduced in this workflow to reduce the burden of operators doing these calls manually~\cite{5}.


% Some engaged in these processes to overcome personal constraints, such as infrastructural issues. For graphic designers in Global South, unreliable internet and power blackouts made it difficult for them to brainstorm with their clients and have multiple rounds of interactions to refine the project requirements. To work within the constraints, designers reduced their interactions with the client and instead conducted their brainstorming sessions with ChatGPT through prompts to achieve similar, if not perfect, goals. Practitioners also used GenAI to reduce those interactions that contributed to increased stress in their jobs. For example, in \cite{5}, telephone operators working in the healthcare domain were tasked with the emotional labor of reaching out to older adults on regular basis, having conversations with them, and recording any issues impacting their lives. GenAI was introduced in this workflow to reduce the burden of operators doing these calls manually. Interestingly, operators found GenAI extremely helpful in reducing their emotional labor and the resultant emotional stress that they were otherwise experiencing in their day-to-day work.  


