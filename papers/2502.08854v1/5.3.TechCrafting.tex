\subsection{Technology Crafting: A New Form of Job Crafting}

Finally, we observed several instances of job crafting with technology that do not fit the traditional job crafting framework. This form of crafting, which we label \textit{technology crafting}, is characterized by actions that are exclusively aimed at reconfiguring a specific \textit{technology} to improve one's work experience. In particular, this occurred through actions taken to manage or reconfigure GenAI tools that practitioners would not have otherwise engaged in except to integrate said tools into their practices. In these practices, GenAI became the focal point for modification and adoption, just like any other task or relationship.

Unlike prior research on AI \cite{Perez2022} in which users had little control over the structure and function of AI tools (e.g., algorithmic management through a tool), leaving employees to craft only reactively, GenAI was easily configurable and personalizable, supporting proactive crafting. At an extreme, this entailed fine-tuning a model with custom data to make it better at a very specific task or setting up self-hosting to control sensitive data sharing. This kind of labor includes behaviors like curating prompts, tuning model parameters, or managing when to have a tool be active or inactive to better fit one's working style.

These forms of technology crafting differed from previously-studied forms in two key ways. First, it required continuous interaction with GenAI, imposing temporal demands. Second, the black-box nature of the technology introduced variability in control, making it challenging for professionals to accurately predict the outcomes.

% unlike patchwork, our practitioners tend to have agency in the adoption of genai tools, so the invisible labor that they do happens throughout the cycle of their work, not just after the integration of the tool. This is important to recognize


% we argue that the Job crafting framework in its current form is limited in its ability to explain emerging AI managerial labor and reconfigurations that are exclusively aimed to customize the technology itself with an intent to improve practitioners’ work.






% Finally, we observed a new form of job crafting that does not quite fit under the standard definitions of crafting in these works.
% In what we refer to as tech crafting, practitioners expended effort to modify the extent or nature of how GenAI was integrated into their workflows.

% This could take the form of performing the kind of managerial labor the viewed necessary to practically use GenAI. This kind of labor, especially prompt design, is not something that they would otherwise have engaged in for their job--they specifically took on this kind of work in order to use GenAI.

% More interestingly, tech crafting could take the form of configuring or reconfiguring technology to fit the desires, requirements, or values of a practitioner.
% For example, choosing to host an open-source model locally for privacy reasons, and going through the effort and reasoning required to make that possible.
% Similarly, fine-tuning a generative model to fit a specific use case or even simply managing when to have the tool on or off, are examples of crafting the use of technology to fit a desired work experience.