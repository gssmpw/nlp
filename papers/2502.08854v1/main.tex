% This is samplepaper.tex, a sample chapter demonstrating the
% LLNCS macro package for Springer Computer Science proceedings;
% Version 2.21 of 2022/01/12
%
\documentclass[runningheads]{llncs}
%
\usepackage[T1]{fontenc}
% T1 fonts will be used to generate the final print and online PDFs,
% so please use T1 fonts in your manuscript whenever possible.
% Other font encondings may result in incorrect characters.
%
\usepackage{graphicx}

% Used for displaying a sample figure. If possible, figure files should
% be included in EPS format.
%
% If you use the hyperref package, please uncomment the following two lines
% to display URLs in blue roman font according to Springer's eBook style:
%\usepackage{color}
%\renewcommand\UrlFont{\color{blue}\rmfamily}
%\urlstyle{rm}
%
\newcommand{\fixme}[1]{\textcolor{red}{#1}}
% \newcommand{\fcite}{\fixme{~\cite{}}}
\begin{document}
%
\title{Generative AI \& Changing Work: Systematic Review of Practitioner-led Work Transformations through the Lens of Job Crafting}
%
\titlerunning{Gen AI \& Changing Work: Systematic Review}
% If the paper title is too long for the running head, you can set
% an abbreviated paper title here
%
\author{Matthew Law \inst{1}\orcidID{0000-0003-1167-9138} \and \\  Rama Adithya Varanasi  \inst{2}\orcidID{0000-0003-4485-6663} }
%
\authorrunning{Law \& Varanasi., 2025}
% First names are abbreviated in the running head.
% If there are more than two authors, 'et al.' is used.
%
\institute{Denison University, Granville OH, USA 
\and New York University, New York City NY, USA\\}
%
\maketitle              % typeset the header of the contribution
%
\begin{abstract}
Widespread integration of Generative AI tools is transforming white-collar work, reshaping how workers define their roles, manage their tasks, and collaborate with peers. This has created a need to develop an overarching understanding of common worker-driven patterns around these transformations. To fill this gap, we conducted a systematic literature review of 23 studies from the ACM Digital Library that focused on workers' lived-experiences and practitioners with GenAI. Our findings reveal that while many professionals have delegated routine tasks to GenAI to focus on core responsibilities, they have also taken on new forms of AI managerial labor to monitor and refine GenAI outputs. Additionally, practitioners have restructured collaborations, sometimes bypassing traditional peer and subordinate interactions in favor of GenAI assistance. These shifts have fragmented cohesive tasks into piecework creating tensions around role boundaries and professional identity. Our analysis suggests that current frameworks, like job crafting, need to evolve to address the complexities of GenAI-driven transformations.

\keywords{genAI  \and generative AI \and GAI \and work \and labor \and chatGPT \and dalle \and midjourney \and copilot \and practitioner \and meta analysis \and writer \and designer \and software developer \and job crafting}
\end{abstract}
%
%
%

\section{Introduction}

In recent years, with advancements in generative models and the expansion of training datasets, text-to-speech (TTS) models \cite{valle, voicebox, ns3} have made breakthrough progress in naturalness and quality, gradually approaching the level of real recordings. However, low-latency and efficient dual-stream TTS, which involves processing streaming text inputs while simultaneously generating speech in real time, remains a challenging problem \cite{livespeech2}. These models are ideal for integration with upstream tasks, such as large language models (LLMs) \cite{gpt4} and streaming translation models \cite{seamless}, which can generate text in a streaming manner. Addressing these challenges can improve live human-computer interaction, paving the way for various applications, such as speech-to-speech translation and personal voice assistants.

Recently, inspired by advances in image generation, denoising diffusion \cite{diffusion, score}, flow matching \cite{fm}, and masked generative models \cite{maskgit} have been introduced into non-autoregressive (NAR) TTS \cite{seedtts, F5tts, pflow, maskgct}, demonstrating impressive performance in offline inference.  During this process, these offline TTS models first add noise or apply masking guided by the predicted duration. Subsequently, context from the entire sentence is leveraged to perform temporally-unordered denoising or mask prediction for speech generation. However, this temporally-unordered process hinders their application to streaming speech generation\footnote{
Here, “temporally” refers to the physical time of audio samples, not the iteration step $t \in [0, 1]$ of the above NAR TTS models.}.


When it comes to streaming speech generation, autoregressive (AR) TTS models \cite{valle, ellav} hold a distinct advantage because of their ability to deliver outputs in a temporally-ordered manner. However, compared to recently proposed NAR TTS models,  AR TTS models have a distinct disadvantage in terms of generation efficiency \cite{MEDUSA}. Specifically, the autoregressive steps are tied to the frame rate of speech tokens, resulting in slower inference speeds.  
While advancements like VALL-E 2 \cite{valle2} have boosted generation efficiency through group code modeling, the challenge remains that the manually set group size is typically small, suggesting room for further improvements. In addition,  most current AR TTS models \cite{dualsteam1} cannot handle stream text input and they only begin streaming speech generation after receiving the complete text,  ignoring the latency caused by the streaming text input. The most closely related works to SyncSpeech are CosyVoice2 \cite{cosyvoice2.0} and IST-LM \cite{yang2024interleaved}, both of which employ interleaved speech-text modeling to accommodate dual-stream scenarios. However, their autoregressive process generates only one speech token per step, leading to low efficiency.



To seamlessly integrate with  upstream LLMs and facilitate dual-stream speech synthesis, this paper introduces \textbf{SyncSpeech}, designed to keep the generation of streaming speech in synchronization with the incoming streaming text. SyncSpeech has the following advantages: 1) \textbf{low latency}, which means it begins generating speech in a streaming manner as soon as the second text token is received,
and
2) \textbf{high efficiency}, 
which means for each arriving text token, only one decoding step is required to generate all the corresponding speech tokens.

SyncSpeech is based on the proposed \textbf{T}emporal \textbf{M}asked generative \textbf{T}ransformer (TMT).
During inference, SyncSpeech adopts the Byte Pair Encoding (BPE) token-level duration prediction, which can access the previously generated speech tokens and performs top-k sampling. 
Subsequently, mask padding and greedy sampling are carried out based on  the duration prediction from the previous step. 

Moreover, sequence input is meticulously constructed to incorporate duration prediction and mask prediction into a single decoding step.
During the training process, we adopt a two-stage training strategy to improve training efficiency and model performance. First, high-efficiency masked pretraining is employed to establish a rough alignment between text and speech tokens within the sequence, followed by fine-tuning the pre-trained model to align with the inference process.

Our experimental results demonstrate that, in terms of generation efficiency, SyncSpeech operates at 6.4 times the speed of the current dual-stream TTS model for English and at 8.5 times the speed for Mandarin. When integrated with LLMs, SyncSpeech achieves latency reductions of 3.2 and 3.8 times, respectively, compared to the current dual-stream TTS model for both languages.
Moreover, with the same scale of training data, SyncSpeech performs comparably to traditional AR models in terms of the quality of generated English speech. For Mandarin, SyncSpeech demonstrates superior quality and robustness compared to current dual-stream TTS models. This showcases the potential of  SyncSpeech as a foundational model to integrate with upstream LLMs.


\section{Background}

\subsection{Multi-Agent Deep Reinforcement Learning}
 A typical system consists of the following components: agents, an environment, and a training algorithm, as depicted in Figure~\ref{fig:madrl_system}. Formally, we consider a system with $N$ agents, each indexed by $i \in \{1, \dots, N\}$. At each time step, the agent $i$ is presented with an observation $o_i$ and produces an action $a_i$. For the sake of generality, we included a possible communication channel $c_i$, seeing that it is increasingly used \cite{Zhu2022ASO}. In principle, we can extend the definition of communication to include the most common MADRL methods like parameter sharing \cite{Gupta2017CooperativeMC,Chu2017ParameterSD}, which can be seen as a form of latent space communication. Finally, the training algorithm provides feedback $\nabla_i$ to each agent.

Training algorithms in MADRL can be centralized, decentralized, or hybrid. Centralized training uses the joint action $a=(a_1,...,a_N)$ and the state $s$, which can be understood as an observation augmented by information at training time \cite{Lambrechts2023InformedPL}, and consists of applying classical RL to multi-agent problems like for AplhaStar \cite{Mathieu2023AlphaStarUL}. While decentralized training restricts each agent to local observations $o_i$, possibly including a local reward $r_i$, see IDQN \cite{Tampuu2015MultiagentCA} or IPPO \cite{Yu2021TheSE}. Hybrid approaches, such as centralized training with decentralized execution, leverage global information during training but allow agents to act independently using only local observations during execution, see VDN \cite{Sunehag2017ValueDecompositionNF}, QMIX \cite{Rashid2018QMIXMV}, MADPG \cite{Lowe2017MultiAgentAF} or MAPPO \cite{Yu2021TheSE}. Here, we consider agents based on DNNs; therefore, the feedbacks $\nabla_i$ are gradients of a loss $\ell$. Depending on the training algorithm, this loss can be a function of the reward $r$, the state $s$, the actions $a_i$, the observations $o_i$ and the communications $c_i$. For simplicity, we didn't include those dependencies in Figure~\ref{fig:madrl_system}.
 
\begin{figure}[ht]
    \centering
    \includegraphics[width=\linewidth]{figures/MADRL.pdf}
    \caption{Schema of a simplified view of MADRL systems. At each time step, the agent $i$ receives the initial observation $o_i$, complemented by potential communications $c_i$ and produces an action $a_i$. The agent learns throughout training by the means of gradients $\nabla_i$. }
    \label{fig:madrl_system}
\end{figure}


\subsection{Direct Interpretability of DNNs}
\label{sec:background_interp}

We now present an overview of the modern methods widely used to interpret DNNs in Computer Vision (CV) and Natural Language Processing (NLP). As these domains heavily relied on pre-trained models \cite{Simonyan2014VeryDC,He2015DeepRL, Radford2018ImprovingLU}, direct post-hoc methods have dominated the research landscape, providing key hindsight without altering models' architectures.


\paragraph{Feature importance.} Typical methods used in CV to understand convolutional networks involve visualising important pixels, i.e. saliency maps, \cite{Zeiler2013VisualizingAU,Selvaraju2016GradCAMVE}. Other methods compute importance by perturbing the input \cite{Covert2020ExplainingBR}, using the gradients \cite{Radford2015UnsupervisedRL,Selvaraju2016GradCAMVE,Shrikumar2016NotJA, Smilkov2017SmoothGradRN} or locally decomposing relevance \cite{Montavon2015ExplainingNC,Bach2015OnPE}. Recent works in NLP focus on the Transformer architecture and its attention mechanism \cite{Vaswani2017AttentionIA}, providing token-level insights \cite{Wiegreffe2019AttentionIN,Achtibat2024AttnLRPAL}. 


\paragraph{Prototypes:} a class of methods that creates explanations based on characteristic samples. In CV, it is common to analyse neurons using activation maximisation to create pre-images \cite{Mahendran2015VisualizingDC}, or find related images \cite{Chen2020ConceptWF}. Prototypes can be of various forms like perturbed images \cite{Ribeiro2018AnchorsHM}, cropped images \cite{Dreyer2023UnderstandingT} or latent space vector \cite{alain2018understanding,kim2018interpretability}. Recent works based on sparse autoencoders were able to elicit interpretable features in LLMs, i.e., prototypes \cite{Cunningham2023SparseAF}.

\paragraph{Latent manipulation:} techniques that further extend the interpretability of concepts and features by exploring the internal representations learned by models. These methods were introduced in CV with \cite{kim2018interpretability}, later derived as the field of representation engineering \cite{zou2023representation}. Such latent features enable locating, editing, erasing or decoding models' knowledge \cite{Meng2022LocatingAE,belrose2023leace, Ghandeharioun2024PatchscopesAU}, but causally modify or analyse the produced outputs \cite{rimsky2023steering, Kramar2024AtPAE}.

\paragraph{Circuit analysis:} provides a more granular understanding of model internals by examining pathways and dependencies between models' components, usually neurons or attention heads. Circuits were first discovered in CNNs \cite{Olah2020ZoomIA} before being formalised for Transformers \cite{elhage2021mathematical}.  These circuits revealed peculiar models' components that learned precise mechanisms like induction \cite{Olsson2022IncontextLA}. Using specific datasets, relevant circuits can be automatically discovered \cite{conmy2023automated}. More recent works focus on larger models' components at the layer scale \cite{Dunefsky2024TranscodersFI}.


% \begin{table}[H]
%  \begin{center}
%    % \tabcolsep = 2\tabcolsep
%    \begin{tabular}{ll}
%    \toprule
%    \textbf{Methodology} & \textbf{Related Works} \\
%    \midrule
%    Feature Importance & \cite{Zeiler2013VisualizingAU,Selvaraju2016GradCAMVE, Lundberg2017AUA, Bach2015OnPE,Radford2015UnsupervisedRL, Covert2020ExplainingBR, Montavon2015ExplainingNC, Achtibat2024AttnLRPAL, Smilkov2017SmoothGradRN, Wiegreffe2019AttentionIN} \\
%    %Ribeiro2016WhySI
%    %Katz2024BackwardLP
%    Prototypes   & \cite{Ribeiro2018AnchorsHM,Achtibat2022FromAM, Chen2020ConceptWF, Mahendran2015VisualizingDC, alain2018understanding,Cunningham2023SparseAF} \\
%    %Dreyer2023FromHT
%    %bills2023language
%    %Dar2022AnalyzingTI
%    Latent Manipulations & \cite{kim2018interpretability,Meng2022LocatingAE,zou2023representation,rimsky2023steering,belrose2023leace,Kramar2024AtPAE,Ghandeharioun2024PatchscopesAU} \\
%    Circuit Analysis          & \cite{Olah2020ZoomIA,elhage2021mathematical,Olsson2022IncontextLA,conmy2023automated, Dunefsky2024TranscodersFI}\\
%    \bottomrule
%    \end{tabular}
% \caption{Categorisation of modern direct interpretability methods drawn from CV and NLP domains.} \label{tab:interp_methods}
%  \end{center}
% \end{table}




\section{Methods}
We performed a systematic literature review using the thematic synthesis approach to answer our broad research objectives \cite{Thomas2008}. Rather than beginning with a presupposed research question or hypothesis, thematic synthesis enables researchers to start with a high-level objective and use a systematic, inductive analysis to provide comprehensive answers while iteratively refining the objective. This method is particularly suited for reviews aimed not only at summarizing but also at \textit{extending} the existing literature through synthesis, producing higher-order structures between broad concepts \cite{Xiao_Watson_2019}.
% In this sense, the study draws on the long-standing tradition of meta-synthesis, such as meta-ethnography that focuses on producing inferences that expand the current understanding of the literature \cite{Thorne2004}.

% \begin{figure}
%     \centering
%     \includegraphics[width=0.5\linewidth]{images/cscw_analysis.png}
% \caption{A flowchart showing the overall steps taken to screen and shortlist the key literature for analysis}
%     \label{fig:analysis-process}
% \end{figure}

Thematic synthesis is conducted in three steps \cite{Thomas2008}. First, researchers identify, collect, and filter relevant literature to build a study corpus. Second, the study corpus is analyzed to extract core descriptive themes through the coding process, also known as second-order constructs. Third, the themes are clustered and synthesized into analytical themes, or third-order constructs.

\subsection{Searching \& Scoping the Literature}

\subsubsection{Identification}
We began the research process with a broad objective: understanding the relationship between recent AI developments and worker practices. Before initiating the main search, we conducted a preliminary exploration of the research landscape. We looked for papers relevant to AI and work practices in four prominent digital repositories containing HCI work: ACM, Elsevier, Springer, and IEEE. In parallel, we reached out to 15 expert researchers in the fields of technology, organizational studies, and labor studies, asking them to recommend relevant papers and key research gaps. We identified these experts from specific academic groups, mailing lists, prior workshops, and online communities. We shared a brief excerpt of our research objective and our proposed approach before asking them to suggest two to four literature items, excluding gray literature items such as workshops and work-in-progress.

Our preliminary exploration and expert recommendations revealed that papers describing how workers engaged with AI tools and adopted the technology in a bottom-up manner seemed the most relevant to our objective. We also observed a lack of consensus among the qualitative studies regarding how AI use was shaping practices beyond particular working groups. In light of these insights, we refined our final search to reduce the scope exclusively to how workers employed GenAI tools and the bottom-up, proactive efforts through which they used these to transform their work identity, practices, and relationships, for which the ACM Digital Library yielded the most relevant studies.


\subsubsection{Collection}
Based on our refined objective, we started our focused search in the ACM Digital Library through two main areas of keyword search: ones that described and captured \textit{GenAI} technologies and ones that captured practitioner's bottom-up \textit{work practices}. We experimented with the keywords until we achieved a good balance of finding diverse papers while minimizing noise, motivated by a purposive sampling strategy ~\cite{Xiao_Watson_2019}. Purposive sampling focuses on studies that can assist in conceptual synthesis through rich \textit{interpretation} rather than exhaustive search. Table \ref{tab:search_query} presents the final keywords used to find the initial corpus. We constructed our query as the conjunction of the disjunction of terms in each of these areas, over both the title and abstract fields. %Thus, an entry containing at least one of the terms in each area, in either the title or the abstract, was contained in our result set. 

\begin{table}
    \centering
    \begin{tabular}{p{1cm} p{1 cm} p{7.5cm} p{1.3cm}}
        \hline
       \textbf{KC} & \textbf{Op}  & \textbf{Search Terms}   & \textbf{Scope}\\
        \hline
        GenAI &
        &  
       ``generative artificial intelligence'' OR ``generative ai'' OR ``gen AI'' OR ``genAI'' OR ``large language models'' OR ``chatgpt'' OR ``chat-gpt'' OR ``stable diffusion'' OR ``dall-e'' OR ``midjourney'' OR ``AI-generated content'' OR ``text generation AI'' OR ``image generation AI'' OR ``AI creativity tools'' OR ``AI art generation'' OR ``AI-enhanced tools''&  
      
        Title,\ \ \ \ \ \ \ \ \ \ \ \ Abstract\\
       &  (AND) & & \\
        Work Practices &
        &
         ``labor'' OR ``jobs'' OR ``worker'' OR ``practitioner'' OR ``professional'' OR ``staff'' OR ``workforce'' OR ``work practices'' OR ``employment'' OR ``occupational'' OR career OR ``work habits'' OR ``job performance'' OR ``work design'' OR employee OR ``work patterns'' OR ``work routines'' OR ``work strategies'' OR ``employment practices'' OR ``task management'' OR ``job adaptation'' OR ``work changes'' &
         Title,\ \ \ \ \ \ \ \ \ \ \ \ Abstract\\     
        \hline
    \end{tabular}
    \caption{Final search query used to find relevant papers in the ACM Digital Library. The key concepts (KC) were joined by an AND operation (Op) and search terms were separated by OR operation. These keywords were applied to titles and abstracts.}
    \label{tab:search_query}
\end{table}

\subsubsection{Screening}
Our search query included results from 2022 to 2024 (until July), yielding 665 results. From an initial dataset of 665 papers, we performed a pass to identify the most relevant papers in two stages: 1) title screening and abstract screening, and 2) full article screening. To screen the papers effectively, we followed clear inclusion and exclusion criteria. We focused on only those experiences that described and presented evidence of change in work practices, avoiding speculative work \cite{he2024ai} or initial experiences from the system pilots \cite{petridis2023anglekindling}. Papers were included if they talked about a specific working community, their work practices, and the use of GenAI in these practices. We present a more detailed list of inclusion and exclusion criteria in Table \ref{tab:exclusion_inclusion}.

While we mainly focused on qualitative studies over papers focused on lived experiences, we included some survey and/or mixed-method papers that also designed interventions around what they learned.
% mixed-methods papers that used surveys or described intervention designs around experiences. 
Thematic analysis is particularly well-suited to analyzing and synthesizing such rich qualitative research, enabling both consensus-building and the development of extended arguments \cite{Barnett2009}. 
  
The title and abstract screening process was divided between two reviewers who rated papers independently. To reduce preconceived biases, we did not include authors' names or affiliations. To kick-start the process, both authors reviewed a smaller sample of papers together to build consensus on how to apply the filtering criteria. Once the authors started independent review, they used prolonged deliberation \cite{creswell} as a method to discuss, mark, and resolve uncertain cases. During the title and abstract filtering stage, we reduced the set of 665 papers down to 27. During our subsequent full paper review stage, we eliminated four more papers using the same criteria, resulting in a final set of 23 papers.


\begin{table}
    \centering
    \begin{tabular}{p{5cm} p{7cm}}
        \hline
        Inclusion Criteria   & Exclusion Criteria \\
        \hline
        \begin{enumerate}
            \item Studies that captured workers' change of practices as one of their key themes.
            \item Studies that focused on GenAI.
         
            
        \end{enumerate} &
        \begin{enumerate}
            \item Studies that focused on speculative experiences.
            \item Papers that focused on quantitative and computational methodologies. 
            \item Papers that covered experimental studies (e.g., comparative studies between practitioners' use and non-use of GenAI).
            \item Papers that focused on GenAI-based system design and its evaluation through practitioner feedback.
            \item Papers that conducted systematic review.
            \item Low-quality studies.    
            \
        \end{enumerate}
        \\
        \hline
    \end{tabular}
    \caption{Table depicting the inclusion and exclusion criteria for finding relevant papers.}
    \label{tab:exclusion_inclusion}
\end{table}

\subsection{Analysis}
We started our analysis by familiarizing ourselves with the kinds of reported data in the papers and their style of reporting. 
%All the papers in our final list followed a consistent structure of reporting, with most of their empirical data being available in the findings, in the form of key concepts and quotations. This aligned with the recommendations suggested by \cite{Thomas2008}'s best practices. 
Afterward, the authors went through each paper's finding section line-by-line, engaging with their concepts and coding the key insights in order to develop second-order constructs. Once we developed codes for a few papers, we used peer debriefing \cite{creswell} to discuss the codes and work on the disagreements. This process was repeated until all the papers were coded, producing a total of 31 codes. 

In the second phase of our analysis, we focused on developing third-order constructs that went beyond the data presented in the papers. We used an abductive approach, utilizing the theoretical framing of job crafting \cite{Wrzesniewski_2001} to produce interpretations that mapped to our research objective. For instance, the authors employed the notion of task and cognitive crafting to analyze how practitioners reconfigured their tasks and repositioned their beliefs around job identity. This process was repeated until comprehensive analytical themes emerged. To help with the process, authors developed multiple conceptual maps that helped them draw connections across the descriptive themes \cite{Miles2020}. This analysis ended with \textit{thirteen} analytical themes, including \textit{`AI managerial labor,'} \textit{`Expanding role capacities,'} and \textit{`Displacing human dependencies with GenAI.'} 
%A full accounting of the key themes and relevant codes are presented in the attached supplementary materials. 
 




\section{Findings}

To organize our findings, we start by describing high-level findings, including the key characteristics of the practitioners. Next, we present how practitioners used GenAI technologies to transform their overall work, specifically their work processes, role attributes, and professional relationships. 
%Using the theory of job crafting \cite{Wrzesniewski_2001,tims2010job} as an anchor, we start by examining how individuals shaped their tasks and responsibilities through task crafting techniques. We then examine how practitioners have molded their professional networks through relationship crafting techniques. Lastly, we show how practitioners use cognitive crafting techniques to make meaning of their changing roles.




\begin{table}
    \centering
    \begin{tabular}{|c|p{0.7\textwidth}|}
        \hline
        Occupations & Software developer, manager, lecturer, data engineer, UX designer, UX researcher, UX writer, industrial design,  artists,  UI designer, social worker,  architect, fact-checker, knowledge worker, fiction writer, researcher, speech-language pathologist, executive \\
        \hline
        Industries & Education, technology, art/culture, design firm, agroindustry, health,  IT, fact checking, gaming, science fiction, research \\ 
        \hline 
        % GenAI Products & \fixme{ChatGPT, Copilot, Midjourney, Dall-E, Llama, (there are a lot of open-source models listed in the paper that lists Llama/Vicuna)} \\
        % \hline
        % GenAI Modalities & text (large language model or llm chatbots), audio (phone, podcast), images (text-to-image generators), code (programming assistants) \\
        % \hline
        % GenAI Interfaces &  standalone, integrated, custom \\
        % \hline
        
    \end{tabular}
    \caption{Paper Demographics}
    \label{tab:paper_demographics}
\end{table}

% The papers included in our analysis covered 18 different occupations across 11 different industries. The majority (n=15) focused on practitioners in technology-focused or design roles, although a few papers (n=8) also studied roles outside this archetype, including artists, speech-language pathologists, and writers. Sometimes these lines blurred, e.g., in game design settings involving developers and artists. 

The papers included in our analysis covered 18 different occupations across 11 different industries. The majority focused primarily on practitioners in technology or design-focused roles (n=15). However, we found a few papers also studying roles outside this archetype, including artists, speech-language pathologists, and writers (n=8). The sectors in which these practitioners worked spanned from technology and design to fact-checking, research, and health. For a full breakdown of the occupations and industries included in our survey, please see Table~\ref{tab:paper_demographics}.

In the papers examined, practitioners used a wide range of GenAI tools, including off-the-shelf products such as ChatGPT, Midjourney, Copilot, and LLaMA. The modalities of these tools varied, spanning text (e.g., an LLM-driven chatbot), audio (e.g., voice), images (e.g., a text-to-image generator), and code (e.g., an LLM-driven coding assistant). While many practitioners used standalone tools like Midjourney, others utilized GenAI-enabled features integrated into commercial platforms, such as Notion or Figma plugins~\cite{6}, or custom-developed tools designed for specific purposes, such as a chatbot that interacts with patients to populate a health dashboard~\cite{5}.

%Across the papers we surveyed, there were a number of factors that played into when and how practitioners used GenAI tools to craft their roles. These factors included both extrinsic and intrinsic motivators. In an extrinsic sense, practitioners were influenced by contextual factors locally and within their companies. For example, uncertainty around government policies affected a practitioner's decision to adopt GenAI tools~\cite{1}. Similarly, company policies, from whether the company paid for a subscription~\cite{18} to restrictions around data and confidentiality~\cite{9} also play a crucial role in adoption. Some practitioners even wondered whether their employers would at some point mandate the use of GenAI in their work~\cite{9}. Even in the absence of company policies, general opacity around how one's co-workers used GenAI factored into practitioners' own choices~\cite{10}. Finally, perceptions about how well GenAI models portrayed local culture also affected practitioners' decisions about using it~\cite{1}.

%Practitioners also accounted for several intrinsic factors when considering adopting GenAI in their work. These elements were related to practitioners' own values, perceptions of self, and goals. For instance, some practitioners expressed discomfort over directly using images generated by GenAI in case the model appropriated others' work in its training data~\cite{9,22}. Other practitioners had reservations about using GenAI to design things without human input~\cite{2} or struggled with its ability to capture the human emotions they wanted it to convey ~\cite{22}. For some, matters of skill were an important consideration. On the one hand, there were those who had misgivings about their ability to use the tools. When dealing with complex work scenarios, they felt that their technical expertise was not good enough to make the most of GenAI~\cite{9,10}. On the other hand, some expressed a fear of relying too much on GenAI, leading to deleterious consequences by reducing their own autonomy or problem-solving abilities~\cite{10}.

% There were, of course, practical considerations as well, ranging from concerns about hallucinations and mistakes leading to reputational harm~\cite{1} to questions of efficiency. Here there were both those who believed that working with GenAI reduced their productivity~\cite{22} or was difficult to iterate with~\cite{18}, as well as those who saw its use as a way to increase the volume of their output~\cite{3,9}, streamline their processes~\cite{8}, speed up their work~\cite{6,7,9,22,10,14}, remove bottlenecks~\cite{3}, and free up time~\cite{6}. With that said, some practitioners believed that speeding up their work with GenAI came at a significant cost, especially as external pressures adjusted to this speedup. As one graphic designer lamented, 

% \begin{quote}
% \textit{``I know this is not helping me create original, thoughtful works, but I do not want to fall behind in this new commercial art market momentum''~\cite{14}.}
% \end{quote}

%Overall, we found that practitioners across the papers that we surveyed balanced a number of external and internal factors when considering whether and how to make use of GenAI tools for job crafting. 
In the following subsections, we describe different transformations practitioners brought in their work to incorporate GenAI. Throughout our findings, we differentiate between avoidance crafting and approach crafting. Avoidance crafting refers to practices undertaken by an individual to reduce negative work outcomes, while approach crafting refers to the active creation of opportunities that align with one's professional preferences, strengths, and goals~\cite{zhang_2019}.


\subsection{Transforming Work Processes}
\label{sec:workprocesses}
Practitioners primarily transformed their work by leveraging GenAI to craft and reshape various stages of their regular tasks. This concept is known as task crafting, and our analysis revealed it was utilized across seven different stages of work: discovery, analysis, artifact creation, quality assurance, delivery, logistics, and overall management of work.
%(see Figure \fixme{\ref{fig:task-processes}}).

% The second category focused on developing the practitioner's own professional role, both in short- and long-term. The third category focused on the crafting labor that went into the technology to fine-tune and customize it to improve its efficacy the crafting process. 

\subsubsection{Discovery}
A significant proportion of the transformed practices observed in the papers we reviewed involved tasks in the discovery phase of the participants' work. The discovery phase consists of early-stage processes where practitioners learn more about the problem at hand. One way practitioners used GenAI in the discovery phase was by delegating relevant efforts to it, reducing their scope (avoidance crafting). A common theme in this category was offloading time-consuming information foraging tasks to GenAI. For example, a fact checker in~\cite{1} described how they delegated their laborious work of finding trending topics to ChatGPT:

\begin{quote}
    \textit{``We take the top 200 headlines from the last 24 hours from those sites [...] and run them through ChatGPT, asking it to summarize the main narratives [...] and extract the names of people, places, entities [...] and then send that to me by email. So every six hours, [...] we get an email.''}
\end{quote}

Software developers applied similar strategies to more specific instances of information foraging, such as aggregating syntax examples for guidance~\cite{4}. 
%All of these examples reduced the effort needed from practitioners to complete their tasks.

In contrast to delegating effort, several practitioners actively engaged with GenAI tools to improve aspects of their tasks in the discovery stage (approach crafting). This included practitioners seeking inspiration from GenAI as they embarked on projects. For example, developers used GenAI to seek out new problems to solve~\cite{4}, while visual artists engaged with GenAI playfully to find serendipitous inspiration~\cite{9}. It should be noted that seeking inspiration through GenAI was not limited to initial exploration. A participant in~\cite{6} used GenAI to help them flesh out ideas that they had already partially formed, while designers in~\cite{6,18} used GenAI to help develop mood boards. Developers, on the other hand, used it to explore new ways to solve problems, including learning complex logic creation and identifying ways to reuse solutions \cite{4}.

Another common activity observed in this category was using GenAI for brainstorming, with participants finding ways to speed up idea generation~\cite{10}. This involved coaxing responses or even including GenAI in their existing brainstorming techniques like reverse-thinking~\cite{21}. For example, a user interface designer \cite{6} explained using used GenAI to expand the scope of their ideation:   

\begin{quote}
    ``\textit{I would ask him to tell me alternatives. Push it to think about it in a different way. I always ask – any other ideas? And it would always come back with something}.''
\end{quote}

Practitioners also used GenAI to kick-start their creative processes, such as as generating a first draft of a written project ~\cite{10} or providing guidance on what kinds of software libraries might be useful for a problem~\cite{12}. 

%As one UX designer ~\cite{2} put it: 

% \begin{quote}
%     ``\textit{GenAI tools can help nowadays in generating basic ideas to help us populate the blank canvas, thereby aiding in overcoming the fear of the `empty canvas.'}''.
% \end{quote}

Additionally, GenAI helped kick-start work in areas where practitioners' confidence in their expertise wavered, such as an amateur programmer learning programming concepts to help them write code for a new project~\cite{6}, or software developers turning to GenAI to develop deeper understandings of their craft and to strategize solving challenging problems~\cite{16}.

\subsubsection{Analysis} Certain practitioners leveraged GenAI to change how they performed analysis-focused tasks. These tasks typically involved taking raw information and finding specific patterns. Similar to the discovery phase, practitioners delegated analysis tasks to GenAI that they felt were particularly time-consuming. For instance, a UX researcher~\cite{10} used a GenAI tool embedded in Miro\footnote{www.miro.com},
a digital whiteboarding tool, to summarize brainstorming data.

Surprisingly, practitioners also \textit{actively} engaged with GenAI to make their analysis process more efficient. For instance, \cite{10} showed how end-user-facing practitioners repeatedly consulted with GenAI to synthesize insights from different forms of raw data. A product designer shared how they did the analysis:

\begin{quote}
    \textit{``It was like 75 open ended survey responses and I [...] strip them of [the participant number] and dump them into ChatGPT, and asked [ChatGPT] to generate 5 insights based on the 75.'' }
\end{quote}

Practitioners engaged with GenAI for other analysis activities as well. Some used it for text analysis, some for clustering data for analysis (e.g., thematic analysis), others for categorizing user testing observations, and still others for identifying pain points in customer emails~\cite{21}.

\subsubsection{Artifact Creation} 
In the intermediate to late stages of their processes, users relied on GenAI to help create work-related artifacts. The types of artifacts varied, including personas generated from user transcripts~\cite{21}, product descriptions~\cite{21}, merchandise, and game assets~\cite{20}. Surprisingly, the practices tended to lean toward avoidance crafting rather than active collaboration with GenAI. The level of delegation ranged greatly. On the conservative end of the spectrum, some practitioners outsourced only a small portion of artifact creation, e.g.: 

\begin{quote}
``\textit{I could generate any of this in Photoshop or Illustrator [...] the fact that it was able to render these things on an aesthetic level that was exceeding my bar or at my bar, and doing it in an instant, was mind-boggling. What it did is it gave me time to dabble in other areas.}''~\cite{6}
\end{quote}

On the far end of the spectrum, practitioners used GenAI to perform the entire artifact creation, e.g., software developers asking GenAI to generate entire new features or classes from scratch~\cite{19,11}. Amplified output capacity was one of the most positive consequences of task crafting. A set of professionals in diverse roles at a game design firm, for example, ``were unanimous in considering the creation of more content in less time a strength of [GenAI] systems ''~\cite{9}. 
%~\cite{3} particularly associate this effect with content creation and creativity in ways that could influence practitioner's self-perceptions: ``by leveraging ChatGPT's capabilities to expand their creative or informational output, participants experienced a sense of productivity and accomplishment.''

\subsubsection{Quality Assurance \& Delivery} 
Beyond creating task-related artifacts, multiple papers provided evidence of participants using GenAI for quality assurance and delivery tasks. These practices were primarily aimed at offloading participants' responsibilities (avoidance crafting). For quality assurance tasks, this often meant relying on GenAI for code maintenance in software development, such as refactoring code snippets~\cite{4,19}. Those in user-facing roles, such as UX practitioners, focused on designing evaluation methods to maintain or even improve product quality. For example, some used GenAI to generate user test case scenarios, directly integrating the scenarios' output into their projects~\cite{21}.

Reliance on GenAI increased during the delivery stage, with practitioners delegating the creation of resources meant to communicate key ideas to stakeholders. These resources included pitch documents to present content strategy, images to visualize concepts, and presentations to aid communication during handover~\cite{9,10,12,21}. UX designers in~\cite{10,21} described using Midjourney to generate visual representations and mock-up descriptions, all with the goal of reducing communication challenges and enhancing clarity. Developers, in turn, leveraged GenAI extensively for creating various forms of code documentation~\cite{4,19} and reports~\cite{8}.
% This was particularly evident in~\cite{8}, where software developers used GenAI to minimize the effort required for writing reports.
% One developer shared:

% \begin{quote}
% \textit{``The tool’s ability to generate coherent and relevant content and provide valuable insights and suggestions significantly supported participants’ writing activities, enabling them to produce these types of artifacts with greater ease.''}    
% \end{quote}
% matt: this is not a participant quote; it's a quote from a paper (it is paper R8, referring to generation of writing artifacts like reports).

\subsubsection{Task Management \& Logistics} Within this stage, we aggregated all the secondary tasks that practitioners performed to ensure the smooth functioning of their roles. These tasks included planning, coordinating, and managing various resources and processes. Practitioners engaged with GenAI within this stage in two distinct ways. First, they used GenAI to streamline their overall workflow. This included using GenAI to help formulate clear project goals and develop concrete support mechanisms to achieve those goals~\cite{6}. %For example, a sci-fi writer from the same study shared how they leveraged GenAI to create a ``hero's journey'' document -- used by writers to track progress and plot elements. The writer then used ChatGPT to further break down the document into manageable steps, addressing various criteria such as deadlines and project budget.
Speech language pathologists used GenAI to optimize the documentation processes required at different stages of their workflow, such as recording intake forms and writing evaluation reports~\cite{7}. One speech language pathologist posted in an online community:

\begin{quote}
    \textit{``[...] I’m in the process of creating a Google Form for speech/language intake. After collecting responses (ensuring privacy by removing personal details), you can efficiently transfer this data into ChatGPT alongside a preset prompt template and your evaluation notes. [\dots] ''}
    %This approach can greatly expedite the report writing process, using AI to structure and incorporate information seamlessly into a Word document!''}
\end{quote}

Practitioners also used GenAI to delegate specific logistical activities within tasks to enhance work efficiency. One knowledge worker expressed in \cite{3}, ``\textit{It feels good to outsource this [kind of] work to ChatGPT because I don't enjoy it much}''~\cite{3}. Common delegated activities included summarizing material or simplifying dense content. For example, practitioners frequently used GenAI to condense their own written content for presentations or meeting notes while they focused on conveying critical business decisions. We found this strategy to be particularly prevalent in UX practitioners \cite{10} and knowledge workers \cite{3}. 
%In contrast, fact-checkers relied on GenAI tools to break down and restructure ``dense content'' into a format that would be simpler and more digestible for their end-users \cite{1}.

Another common logistical activity was delegating short but highly repetitive tasks to GenAI, such as generating repetitive content. Participants frequently offloaded such tasks to capable programs. For example, software developers used GenAI to generate boiler plate code~\cite{16}, while fact-checkers performed transcription and translation tasks using GenAI~\cite{1}. Similarly, many practitioners relied on GenAI to fix repetitive but predictable issues, such as a software developer using ChatGPT to write a bash script to change the capitalization convention of file names~\cite{11}. GenAI helped practitioners save significant time in these cases by reducing the need for context switching between different applications~\cite{8}.
%Streamlining or delegating work to GenAI allowed practitioners to focus on tasks they found more important or engaging, e.g., preparing for discussions or concentrating on more advanced aspects of their work~\cite{3}. 




\subsection{Transforming Role Attributes} \label{sec:role}
The second transformation involved practitioners using GenAI to enhance their roles, not only shifting their perspectives of these roles (cognitive crafting) but also expanding their role capacities (task crafting). Some of these transformations were temporary, wherein they augmented their abilities through GenAI for the purpose of completing a particular task. Other practices were more permanent, which allowed practitioners to focus on professional development by learning skills relevant to their work. 

\subsubsection{Augmenting Abilities Through GenAI}
In many cases, practitioners used GenAI to expand the range of tasks they could perform without improving their own skills or acquiring additional knowledge in the long term. 
%We refer to this practice as augmenting abilities through GenAI and noticed a high prevalence in our sample studies.  

One area of work in which participants augmented their abilities with GenAI was in the realm of communication. An industrial designer in~\cite{18}, for example, described how CAD-generated images filled in visual details that allowed him to tell a story around the packaging he was designing. In another domain, fact-checkers described using GenAI to connect instantly with audience members on a broader scale, or to convert fact checks into an audio format to share through different modalities like TikTok~\cite{1}. Despite not knowing how to do these tasks before, practitioners could accomplish them successfully using GenAI. 

Other practitioners used GenAI to fill perceived gaps in their critical thinking abilities, such as by using GenAI to make external assessments (e.g., evaluating ideas for a startup)~\cite{21} or to improve their sense-making abilities. A designer in~\cite{18} used GenAI to reveal complex connections between their current work and previous work, connections that they were initially not aware of. A few practitioners used GenAI for more routine but core tasks where they experienced a disadvantage, such as proofreading manuscripts in a non-native language \cite{13}. GenAI also allowed practitioners to perform tasks beyond their skill set. 
%For example, fact-checkers used GenAI to generate search queries in languages they didn't speak or identify the sources of images~\cite{1}. 
This could take the form of GenAI filling in for missing expertise in constrained situations, e.g., ``\textit{In our startup, we don't have a dedicated UX writing role. Our designers often use ChatGPT to assess the appropriateness of the UX content in our design}''~\cite{2}.


\subsubsection{Learning Through GenAI}
Using GenAI in multiple steps enabled practitioners to develop specific abilities and knowledge as a by-product of their dyadic interactions. Multiple developers argued that GenAI provided quicker and more effective access to information for knowledge acquisition than traditional search engines or reading the documentation directly~\cite{8,3}. 
%For example, a software developer claimed that,``\textit{what has been really helping me with this tool is precisely the process of trying to understand something that I’m not currently grasping about what a certain code is supposed to be doing}''~\cite{16}.  

Some practitioners exhibited explicit intent to use GenAI for learning purposes. Examples included developers using GenAI to learn how to do static type checking with type hints in Python~\cite{23}, exploring solutions to problems they'd never needed to solve before~\cite{4}, and understanding machine learning libraries~\cite{23}.

We also observed several counterintuitive examples, where, in attempts to improve productivity through GenAI, some practitioners skipped the learning process altogether. These practitioners experienced significant pressure to use the GenAI tools to decrease their project's delivery time but felt this compromised knowledge acquisition and skill development. For instance, a designer in~\cite{18} shared, ``\textit{there's no longer a day or a week of reading up for the project; instead I collect a bunch of materials and pretty much dump it into GPT-4.}''

Along these lines, the ability to use GenAI in this way reduced practitioners' sense of control and ownership around their work. This could lead practitioners to perceive their roles as evolving into assistants to GenAI. A graphic artist working with Midjourney found his role shifting to handing off ideas to the AI to illustrate, then fixing any errors, a change that he described as ``heartbreaking''~\cite{9}. Nonetheless, other practitioners found the role of GenAI to become more specialized over time, discovered ``progressive ownership'' in adjusting its outputs, or identified the value of human qualities in their work processes.To cope, practitioners engaged in the crafting practice of \textit{role re-framing}, shifting their sense of worth from their technical skills to their original ideas. As their experience grew, they began viewing themselves as managers of the GenAI tools. 



\subsection{Engaging in AI Managerial Labor}
\label{sec:aimanagerial}
The third form of transformation was the introduction of new planning and execution tasks that focused on maximizing the potential of GenAI in practitioners' work (task crafting) and integrating it fully into practitioners' workflows. We describe these tasks as \textit{AI managerial labor}.

One major form of AI managerial labor involved developing prompts, which required the additional work of establishing context. Tasks in this category included explaining big-picture goals, setting character limits for outputs, requesting specific output styles, providing examples of anticipated output, or providing a concrete starting point (e.g., a code to start from)~\cite{4,6,16}. In rare cases, practitioners removed context by redacting sensitive information for compliance reasons from a prompt~\cite{18} or under-specified prompts to provoke surprising outputs~\cite{9}.
% One major form of AI managerial labor involved developing prompts. Practitioners frequently went to great lengths to furnish their prompts with context to help produce useful outputs for their work. For some, establishing context took the form of explaining their big-picture goals, setting character limits for outputs, requesting specific output styles (e.g., bulleted list), providing examples of anticipated output, or providing a concrete starting point (e.g., a code to start from)~\cite{4,6,16}. In rare cases, practitioners removed context, e.g., redacting sensitive information for compliance reasons from a prompt~\cite{18} or under-specifying prompts to provoke surprising outputs~\cite{9}. Some individuals were less willing to expend this kind of effort~\cite{15} and took an alternative route of curating ready-made prompts~\cite{14}. 
In a few instances, curation became a social endeavor of exchanging prompts; we discuss this in more depth in Section~\ref{sec:relational} on relational crafting. 

% We even identified certain practitioners who developed methodological practices around designing their prompts. For example, a developer in ~\cite{16} drafted their prompts in a separate tool before opening the GenAI, 

% \begin{quote}
%     \textit{``So there’s a step today that I often take before talking to ChatGPT, which is creating my prompt, creating my question. So, *I open a notepad*, think about what I’m going to put in the structure, and then I copy and paste it into the chat''}.
% \end{quote}

% A few practitioners viewed prompt designing as an iterative process. For instance, a software developer in~\citet{16} described receiving a wrong version of PHP code snippet from their initial prompt. The GenAI required additional clarification in follow-up prompts to get the syntax the developer was looking for. Others simply found it limiting to provide all the context in a single prompt, preferring to work incrementally towards their ultimate goal with smaller prompts~\cite{4}. Another developer observed,``\textit{Every time you prompt, you’re giving it clues to get closer to the idea you have in your head. The more words that you use, the closer the image can get to what you imagine~\cite{6}}.''  Interestingly, some practitioners included the GenAI in these conversations. For instance, some asked it about its capabilities and limitations or asked it to suggest prompts to use. Others even prompted it to critique its \textit{own} work~\cite{6}. \fixme{Underspecifying prompts [9]. Should it come here?}


A complementary form of AI managerial labor constituted refining GenAI outputs to fit in participants' workflows. Within this category, the most prominent activity consisted of verifying and correcting details in GenAI outputs. ``\textit{I don't trust the AI. \dots So, I have to read everything and validate what it's doing}''~\cite{15}. Beyond verification, some practitioners expended significant manual labor to utilize GenAI outputs. For instance, a UX designer from~\cite{2} spent time switching back and forth between GenAI tools and Photoshop to post-process images. In another domain,~\cite{17} described developers' post-processing pull requests authored by Copilot to either add missing information or remove superfluous context. Sometimes new processes emerged around this post-processing, such as maintaining a changelog to track changes made to the GenAI outputs. 

% In a more subtle form of AI managerial labor, practitioners sometimes exerted themselves in secondary ways around how they engaged with the AI. For example, one software developer described managing when the GenAI tool was on or off, because it gave proactive suggestions that could disrupt their workflow~{}.

AI managerial labor also involved configuring GenAI models at the system or application level, enabling practitioners to exert greater control over (a) data privacy, (b) model adaptation to their use case, and (c) the timing and context of the models' engagement with their workflows. One way practitioners took more control of the privacy of their data was by only using open-source models, as opposed to off-the-shelf models. They self-hosted them to prevent sensitive material from being exposed~\cite{1}. Practitioners also took control over adapting the model to their specific use cases by configuring settings. For example, a developer adjusted an API setting called the model temperature\footnote{https://platform.openai.com/docs/api-reference/chat/create\#chat-create-temperature}, which determines the randomness of the output given by the GenAI model, in order to generate more precise and helpful answers~\cite{6}.

%In a more involved example, a fact-checker fine-tuned GPT-3.5-Turbo with their own custom dataset to generate contextual queries. They could then leverage the contextual queries to perform more sophisticated queries in a different language that the practitioners did not speak ~\cite{1}. Fine-tuning like this is a process by which developers can provide specialized data to create their own custom version of a GenAI model which works better than the off-the-shelf model for their specific use case~\cite{ohm2024focusing}. 

While the above configurations were conducted at the systems-level, practitioners also spent time configuring GenAI tools at the application level, such as by managing when and how a GenAI provided them with suggestions in order to prevent disruptions to their work.  For instance, practitioners found Copilot's proactive suggestion feature a source of interruption. One developer, in \cite{16}, kept Copilot off during the early stages of a project, saying, ``\textit{it tends to provide a lot of suggestions, which kind of hinders my thought process.}'' In contrast, standalone GenAI tools like ChatGPT were more reactive and provided practitioners opportunities to choose when and for what activity to use them. 

% For instance, Copilot, unlike ChatGPT, was a more integrated interface in developers' code editor environments. Although this integration was useful in many ways, some developers described Copilot's proactive suggestion feature as interrupting their work. One developer, in \cite{16}, kept Copilot off during the early stages of a project, saying, 

% \begin{quote}
% \textit{``I believe that when I’m starting a project from scratch, I tend to keep it turned off because, as it still has very little context of what you’re doing, it tends to provide a lot of suggestions, which kind of hinders my thought process.''}
% \end{quote}


% A developer in \cite{22} shared how they selected integrated, proactive GenAI tools or standalone, reactive GenAI tools for different purposes:

% \begin{quote}
%     \textit{``I definitely use [Github Copilot’s Auto-Complete] much more [than Chat GPT]. I rarely actually go into Chat GPT... but I find that when I do use it, it’s more of like- it’s particular use case is kind of getting you unstuck from a spot rather than just integrating it into your regular flow \dots Sometimes it gives solutions and sometimes it’s just kind of to bounce ideas off.''}
% \end{quote}


% \input{4.0.FindingsTable}
\subsection{Transforming Professional Relationships}
\label{sec:relational}
We also saw preliminary evidence of practitioners using GenAI to shape their professional relationships (relational crafting). We believe it may be possible that the ability to interact with the GenAI tools in a two-way synchronous dialogue encouraged practitioners to implicitly have the same expectations from the GenAI that they might have from their human peers. Such expectations were reflected in practitioners' use of GenAI in specific aspects of their work that would usually be conducted through collaborations with peers. This involved participants asking GenAI to reflect on their works, to critique their work, and to provide alternative perspectives. 

For example, while working on their design project, a UI/UX designer used ChatGPT to set up checkpoints across their project life-cycle to help assess whether the current state of their artifact aligned with their envisioned outcome~\cite{6}. A more interesting use case was where fact-checkers in a Sudanese newsroom used GenAI to perform adversarial analysis by asking it to challenge the assumptions in articles written by the team~\cite{1}. Practitioners also established support networks on online platforms \cite{7,18}, such as Meta groups, to exchange specific prompts that enabled them to unlock new ways to engage with GenAI. 
%On several occasions, the comparisons between GenAI and human labor were made more explicit by practitioners anthropomorphizing the GenAI tools. These practitioners treated them as humans by acknowledging the human labor that they were doing with the tools, which was otherwise reserved for human roles.
Creative practitioners in~\cite{10} and~\cite{6} equated working with GenAI to ``\textit{quickly brainstorming with someone}.'' One practitioner shared:

\begin{quote}
    \textit{``I approached it similarly to how I would collaborate with someone. I could just go in with something half-baked and know that the system would ask me to clarify [if it needs it] \dots It really felt like a partnership [that] I found useful.''}
\end{quote}

A few UX designers and researchers replaced their collaborators with GenAI to explore hypothetical what-if scenarios before distilling the insights into eventual designs. For instance, a UX practitioner in \cite{21} fed empathy maps from an interview analysis and asked GenAI to emulate more diverse personas related to the real interviewees to envision a wider range of scenarios. 
%This facilitated a more ``comprehensive understanding of the user’s perspective'' and ``enhanced the design exploration process''. 
In more extreme cases, practitioners experimented with stretching the design boundaries by assigning a non-human actor persona (e.g., technological systems or consumer objects) to GenAI to elicit design ideas. In one instance, the researcher asked ChatGPT to speak on behalf of a set of ESP32 microcontrollers, imagining how they (the microcontroller chips) might perceive their own role in the project~\cite{21}.

The perceived anthropomorphized contributions could directly impact practitioners' existing professional relationships. One way this materialized was when GenAI was used to replace ``taken for granted'' tasks that were outsourced to other roles, such as transcription and translation \cite{1}. In more drastic situations, practitioners used GenAI to reduce their dependencies on different stakeholders. This was partially motivated by \textit{external} factors, such as intent to avoid wasting the time of specialized but overburdened roles or to save time and resources. For instance, UX designers used GenAI in their startup \cite{2} instead of hiring a UX writer to reduce the overhead costs. On the other hand, some of the reasons were \textit{internally} motivated, such as shortening the turn around time that was occurring because of practitioners' dependencies on other roles, such as experts. In \cite{9}, creative leaders, instead of waiting for artists' input, ``side-stepped'' them and developed their own pitch decks for client communication using GenAI. 

Practitioners used GenAI to navigate personal and infrastructural constraints. For example, graphic designers in the Global South faced unreliable internet and power outages, making it difficult to brainstorm with clients~\cite{14}. To maintain pace of progress, practitioners compensated for the gap by brainstorming with ChatGPT. Practitioners also used GenAI to reduce interactions that contributed to increased stress in their jobs. For example, in \cite{5}, telephone operators working in the healthcare domain were tasked with the emotional labor of reaching out to older adults on a regular basis, having conversations with them, and recording any issues impacting their lives. GenAI was introduced in this workflow to reduce the burden of operators doing these calls manually~\cite{5}.


% Some engaged in these processes to overcome personal constraints, such as infrastructural issues. For graphic designers in Global South, unreliable internet and power blackouts made it difficult for them to brainstorm with their clients and have multiple rounds of interactions to refine the project requirements. To work within the constraints, designers reduced their interactions with the client and instead conducted their brainstorming sessions with ChatGPT through prompts to achieve similar, if not perfect, goals. Practitioners also used GenAI to reduce those interactions that contributed to increased stress in their jobs. For example, in \cite{5}, telephone operators working in the healthcare domain were tasked with the emotional labor of reaching out to older adults on regular basis, having conversations with them, and recording any issues impacting their lives. GenAI was introduced in this workflow to reduce the burden of operators doing these calls manually. Interestingly, operators found GenAI extremely helpful in reducing their emotional labor and the resultant emotional stress that they were otherwise experiencing in their day-to-day work.  



\section{Discussion}
Our findings provide a nuanced view of how practitioners engaged in high-skill work have integrated GenAI to reshape their tasks, adjust their relationships, and ultimately shift their perceptions of their roles. These transformations also exposed practitioners to several underlying tensions, which we unpack here.
 
\subsection{The Dynamics of Task Transformation with GenAI}
Our insights show that practitioners primarily turned to GenAI to increase their overall work productivity and efficiency. However, in the process, they created a need for new forms of labor to manage the AI (see Section \ref{sec:aimanagerial}). In some ways, these findings reflect prior work around the long-term transformation of technology-driven automation on work \cite{autor2003skill}. \cite{autor2015there} argued that automation can increase the value of complementary tasks. 
%For example, the rise of AI-driven automation for data collection tasks has created demand for practitioners to make sense of this new data.
While this has traditionally lead to a polarization of labor, where routine tasks are automated and create demand for tasks which require non-routine human abilities (abstract reasoning or manual interaction with messy environments), our findings suggest a shift with GenAI.



% At the same time, our findings diverge from the typical pattern described by \cite{autor2015there} in how this displacement occurs. \cite{autor2015there} suggested that the demand for routine well-structured tasks (also referred to as ``middle-skilled'' activities) tends to give way to tasks that complement automation and require unique human abilities, such as high-level abstract thinking or manual interaction with the messy, unpredictable world. 
% Our GenAI-specific findings demonstrate an opposite trend. 
In addition to routine tasks, some practitioners voluntarily gave GenAI abstract and creative tasks, while regularly taking on tasks to manage GenAI like fixing mistakes in its outputs. This departure reflects the unique capabilities of GenAI tools, which are more capable than previous forms of AI in completing tasks that require complex human abilities. Forms of AI managerial labor like verifying outputs still require human intuition. It remains to be seen whether these will also be susceptible to automation and how the market will value them.

% The pockets of tasks that were still exclusive to practitioners were those that involved human ability to infer contextual preferences and how those map to the outputs. This human quality characterizes much of the AI managerial labor that we observed.

\subsection{Arising Tensions in Shifting Role Identities}
Aforementioned transformations in tasks were also accompanied by practitioners grappling with their shifting role identities.  Workers can perceive threats to their professional identity when they experience changes to their role~\cite{petriglieri2011under}. Workers often respond to such threats by either perceiving a sense of identity crises or perceiving an opportunity for identity growth~\cite{zikic2016happens}. In our findings, GenAI assumed this dual role wherein it was perceived as a threat by the practitioners when they offloaded tasks that they perceived were central to their professional identity (see Section \ref{sec:role}). This contributed to these practitioners perceiving GenAI to devalue their role and contributed to identity crises.

In contrast, others re-framed their roles in response to GenAI, perceived it as a positive force, and focused on identity growth. These findings echo \cite{zikic2016happens}'s work around threat perceptions of immigrant workers and their resultant experiences of identity crisis or growth when adapting to a new workplace. Future studies need to focus on designing support structures that help practitioners handle these threats in ways that lead to growth rather than identity crisis.









% -- role identity shift (start with periphery but leaking core tasks to genai)
% -- inability to control what tasks people give--more and more dependency
% -- tension with high vs low-level skills
% --capacity building vs augmentation
% --balance between increased productivity or freeing time to do other tasks vs increasing pressure/pace demand due to speed of GenAI

% role reduction happening in two ways: one by increasing their capability, they are now less dependent on someone else and also less dependent on themselves


% 2. human-machine collaboration




\subsection{The Versatility of GenAI \& Fractured Roles}
In order to redefine their tasks and roles, practitioners frequently restructured their work in innovative ways to engage effectively with GenAI. This happened across various stages of their workflows (see Section \ref{sec:workprocesses}) and through two primary approaches. First, practitioners segmented their own tasks into distinct parts, identifying elements that GenAI could independently accomplish, and then collaborated on these. Second, practitioners selectively undertook tasks typically managed by external stakeholders, completing them with GenAI’s support. This approach of compartmentalizing tasks with GenAI’s assistance began to resemble a form of piecework\cite{hagan1973piece}. Piecework is a process of decomposing complex work into smaller ``pieces'' that could be performed by individuals with varied skill sets~\cite{alkhatib2017examining}, with recent HCI applying this concept to modern crowd-work \cite{irani2015cultural}. 

It's possible that using GenAI as a pieceworker could alleviate some of the issues that typically arise around piecework. For example, GenAI's versatile features enabled practitioners to assign piecework to it while overseeing the tasks. Unlike traditional piecework dynamics, one might argue that situations where a practitioner chooses to parcel out elements of their work to GenAI gives them unprecedented agency in what aspects of their roles become piecework. In an ideal scenario, this could allow the practitioner to focus on more fulfilling aspects of their roles without exploiting other workers.

However, from practitioners' perspectives, transforming their tasks into piecework brought multiple uncertainties to their roles. One common uncertainty came from the fact that the affordances of the GenAI tool, rather than the preferences of the practitioner, tend to determine what aspects of work get off-loaded. If, for example, tweaking GenAI-generated images is significantly faster than a human artist producing those images, continuing to do this work manually can carry significant risk of falling behind. In addition, the potential for some roles to absorb pieces of other roles can introduce power asymmetries that undermine relationships between neighboring roles. Practitioners can also run the risk of transforming visible aspects of their work into invisible (e.g., prompt designing). This can lead to acrimonious relationships between workers and management~\cite{alkhatib2017examining}.

 
 
 
%  the opportunity cost of not realizing these productivity gains could make doing those aspects of the work untenable, something we saw play into practitioners' choices to use GenAI tools.
% In this case, the practitioner does not really have agency over what aspects of their role become piecework. In cases where most of a role gets transformed into piecework, effected practitioners may end up caught between losing work to GenAI and dealing with the negative dynamics that accompany piecework.

% practitioners, it is questionable the extent to which practitioners can actually control what gets converted to piecework or not.

% Beyond shifting identities, we noticed that crafting with GenAI could lead to traditional roles being broken down into task-sized pieces. We attribute this to the versatility of GenAI tools which allow practitioners to target specific tasks as-needed. 

% Across the papers that we surveyed, it was common for practitioners to break off elements of their work that would not normally be treated separately, in order to delegate them to GenAI. For example, illustrating an idea during the ideation process or seeking inspiration to get unstuck are not typically the kinds of activities that would be outsourced, individually, to another human. However, performing these kinds of microtasks were quite typical uses of GenAI.

% We saw this decomposition happening in two primary ways: (1) practitioners breaking off and absorbing pieces of external roles by asking GenAI to fulfill tasks outside their skillset, and (2) practitioners splitting off tasks that they would normally perform themselves and offloading them to GenAI. An example of GenAI crafting fracturing an external role is a UX design team choosing to use GenAI as-needed instead of hiring a UX writer to evaluate their writing, splitting this kind of work off from UX writers. Alternatively, we saw examples of crafting like a software developer asking GenAI to write class definitions, or a UX designer asking GenAI to craft a persona from user data. In these cases, the crafting practitioner explicitly carves pieces of their own work out of their roles. 

% The practice of decomposing larger tasks to be performed and compensated on-demand relates to the historical phenomenon of piecework~\cite{alkhatib2017examining, hagan1973piece}. Piecework was a movement that emerged in the 19th century around decomposing complex tasks into small pieces that could be performed by individuals with either limited or extremely specialized expertise. Recent scholarship in HCI has sought to apply this concept to modern crowd-work on platforms like Amazon Mechanical Turk~\cite{irani2015cultural}. 

% We argue that practitioners breaking off parts of their roles into tasks that GenAI can perform as needed resembles piecework, with some important distinctions. Most importantly, the piecework in this case is being performed by GenAI, rather than a human worker. GenAI's versatility and the flexibility of the modalities through which we interact with it, such as language and images, position it as an ideal pieceworker, with limited expertise to tackle large problems comprehensively, but perfectly capable of handling constrained tasks with enough oversight. Just like riveters, who do not undergo extended metalwork apprenticeships but become experts at one focused task, GenAI models can be finetuned with custom data for custom tasks, like we saw one fact-checker doing to generate search queries.

% From a worker perspective, transforming more comprehensive roles into piecework can have negative consequences.
% While advocates of piecework have historically pointed to increased worker autonomy and the ability to use it to solve complex problems at scale, detractors have highlighted instances where it enables worker exploitation through asymmetries of power or information, tends to overlook invisible labor performed by workers, and can lead to acrimonious relationships between workers and management~\cite{alkhatib2017examining}.

% It's possible that using GenAI as a pieceworker could alleviate some of the issues that typically arise around piecework.
% Indeed, unlike traditional piecework dynamics, one might argue that situations where a practitioner chooses to parcel out elements of their work to a GenAI gives them unprecedented agency in what aspects of their roles become piecework. In an ideal scenario, this could allow the practitioner to focus on more fulfilling aspects of their roles, parting out less desirable pieces, without exploiting other workers.

% However, based on our earlier discussion of role shifts, it also seems possible that the affordances of the GenAI tool, rather than the preferences of the practitioner, ultimately determine what aspects of work get off-loaded to the GenAI. If, for example, tweaking GenAI-generated images is significantly faster than producing those images as a human artist, the opportunity cost of not realizing these productivity gains could make doing those aspects of the work untenable, something we saw play into practitioners' choices to use GenAI tools.
% In this case, the practitioner does not really have agency over what aspects of their role become piecework. In cases where most of a role gets transformed into piecework, effected practitioners may end up caught between losing work to GenAI and dealing with the negative dynamics that accompany piecework.

% Overall, GenAI's versatility positions it as a sort of ideal pieceworker, enabling the breaking down of traditional roles, either externally or internally. While this may have positive implications for some practitioners, it is questionable the extent to which practitioners can actually control what gets converted to piecework or not. Insofar as piecework can negatively affect work experiences, it is important to determine the extent to which markets can be designed to preserve practitioners' agency in how this plays out. This is complicated by the fact that our findings indicate that many of these shifts are happening organically due to choices that practitioners make on their own. At the very least, conversations need to happen around how we value and incentivize piecework, as it is performed either by a human or an AI.









%  On one hand, unlike traditional instances of piece work, the worker themselves exert agency in choosing to break off aspects of their work. In an ideal scenario, this could lead to situations like those practitioners who used GenAI to save time and allow them to focus on more fulfilling aspects of their roles. However, there is also a possibility that the affordances of the GenAI tool, rather than the preferences of the practitioner, determine how work is off-loaded to the AI.
% For example, if tweaking passable outputs from a GenAI tool is significantly faster than relying on human-generated work, the productivity gains could create market pressures that make doing the work yourself untenable. This could explain the examples we saw of graphic designers feeling forced to use GenAI tools to keep up with the market\fixme{the example in the intro}, or struggling with a sense of lost control and feeling the need to re-imagine themselves as AI managers.


% The flexibility of the modalities through which we interact with GenAI, e.g. language and visual images, position it to act as a sort of chorus of workers, with limited expertise to tackle large wicked problems on its own, but perfectly capable of handling constrained chunks of tasks at a level that is acceptable with enough oversight. The ability to fine-tune models with custom data, like we saw one fact-checker doing, even presents the possibility to create the kinds of focused expertise that ~\citet{alkhatib2017examining} say characterize late-stage, advanced piecework, e.g., workers who don't undergo extended apprenticeships as metalworkers but, by focusing on one task, become expert riveters. The striking compatibility of GenAI to perform this kind of piecework, and the demonstrated willingness of practitioners to engage it thusly, raises the question of what the implications will be for those whose work aligns with this trend. Even though GenAI is performing the piecework, using it in this way is likely to affect the structure and dynamics of work for those who normally perform the tasks being parted out.














% Traditionally, piecework describes this...
% good and bad...



% What we see here has a couple of clear departures from traditional piecework, however.

% a) GenAI is the piece worker
% b) workers can sometimes have agency in the kinds of work they parcel off





% Looking closer at the sense practitioners had of their roles shifting, we also observed increasing complexities in how the tasks responsibilities encompassed by specific roles could fracture in unintended ways due to the choices that they made about how to use GenAI. 

% This is because the versatility of GenAI tools affords targeting individual pieces of work as the need arises. Across the papers that we surveyed, it was common for practitioners to break off elements of their work that would not normally be treated separately, in order to delegate them to GenAI. For example, illustrating an idea during the ideation process, refactoring how filenames are capitalized, or seeking inspiration to get unstuck are not typically the kinds of activities that would be outsourced, individually, to another human. However, performing these kinds of microtasks were quite typical uses of GenAI.

% These dynamics of decomposing larger tasks to be performed on-demand by a GenAI resemble the historical phenomenon of piecework \cite{}. Piecework was a movement that emerged in the 19th century around decomposing complex tasks into small pieces that could be performed by individuals with either limited or extremely specialized expertise. Under this system, workers were paid per piece of work, instead of for a set amount of time.

% % Examples of this spanned from coal miners  being paid per unit of coal that they produced~\cite{hagan1973piece} to astronomer George Airy mailing out pieces of complex computations to distribute them among workers with rudimentary math backgrounds~\cite{alkhatib2017examining}. 
% Recent scholarship in HCI has sought to apply this concept to modern crowd-work on platforms like Amazon Mechanical Turk, what Irani describes as ``cognitive piecework''~\cite{irani2015cultural}. While advocates of piecework have historically pointed to increased worker autonomy and the ability to use it to solve complex problems at scale, detractors have highlighted instances where it enables worker exploitation through asymmetries of power or information, tends to overlook invisible labor performed by workers, and can lead to acrimonious relationships between workers and management~\cite{alkhatib2017examining}.

% The major difference in our study is that the piece work is being performed, not by humans, but by GenAI tools. The flexibility of the modalities through which we interact with GenAI, e.g. language and visual images, position it to act as a sort of chorus of workers, with limited expertise to tackle large wicked problems on its own, but perfectly capable of handling constrained chunks of tasks at a level that is acceptable with enough oversight. The ability to fine-tune models with custom data, like we saw one fact-checker doing, even presents the possibility to create the kinds of focused expertise that ~\citet{alkhatib2017examining} say characterize late-stage, advanced piecework, e.g., workers who don't undergo extended apprenticeships as metalworkers but, by focusing on one task, become expert riveters. The striking compatibility of GenAI to perform this kind of piecework, and the demonstrated willingness of practitioners to engage it thusly, raises the question of what the implications will be for those whose work aligns with this trend. Even though GenAI is performing the piecework, using it in this way is likely to affect the structure and dynamics of work for those who normally perform the tasks being parted out.

% We broadly observed two ways in which the roles were decomposed into tasks that could be delegated to GenAI, with different consequences. In one, the practitioner extends or augments their own abilities using the GenAI to accomplish tasks that are traditionally bound up in other roles. For instance: a manager using a text-to-image generator to illustration presentation materials to sidestep working with an artist, or a UX design team avoiding hiring a UX writer by making use of an LLM when needed. This kind of behavior has consequences for those in the roles that would traditionally be relied on to fulfill these kinds of tasks. At a structural level, those pieces of work that practitioners are turning to GenAI to fulfill are broken out from the traditional role that contained them into work that is requested and fulfilled by the piece. In a practical sense, other roles absorb these pieces, fragmenting the traditional roles into the pieces that are not substitutable by GenAI.

% A second, and more interesting way that we see role fracturing happening is when practitioners reconfigure their own work to give some of it to the GenAI. This does not describe the cases where we saw practitioners inviting GenAI in to expand some aspect of an activity they were doing (e.g. as an extra brainstorming perspective). Instead, these include behaviors like asking the GenAI to make actual design artifacts or analyze user data for them. While this may be done out of exigency, these kinds of choices have the potential to segment and remove those pieces of work from one's own role.

% This a complex dynamic to interpret. On one hand, unlike traditional instances of piece work, the worker themselves exert agency in choosing to break off aspects of their work. In an ideal scenario, this could lead to situations like those practitioners who used GenAI to save time and allow them to focus on more fulfilling aspects of their roles. However, there is also a possibility that the affordances of the GenAI tool, rather than the preferences of the practitioner, determine how work is off-loaded to the AI.
% For example, if tweaking passable outputs from a GenAI tool is significantly faster than relying on human-generated work, the productivity gains could create market pressures that make doing the work yourself untenable. This could explain the examples we saw of graphic designers feeling forced to use GenAI tools to keep up with the market\fixme{the example in the intro}, or struggling with a sense of lost control and feeling the need to re-imagine themselves as AI managers.

% What are the implications of this? At the very least, conversations need to happen around how we value and incentivize both piece work and AI managerial work. It is one thing if a push towards role fragmentation and piece work originated top-down, but our survey suggests that it is emerging naturally in the ecosystem as a result of the decisions practitioners are making to improve their own work experiences.

% Secondly, are there ways to preserve agency over what kinds of work to parcel out, independent of the affordances of GenAI? Are there precedents for successfully structuring markets around drivers other than productivity? Or are there ways that we can shape the development of models and the infrastructure around them to preserve human agency over how to divide work and collaborate with GenAI in a way that doesn't sacrifice its benefits? Are there ways to organize practitioners to resist market pressure to adopt GenAI tools where they do not wish to? (precedent from piece work match girls or slow working)


% ``The literature on the history of piecework does not frame the
% question as whether piecework is inherently ethical or unethical, instead asking what conditions render it exploitative''~\cite{alkhati}
% Argument:
% very common to isolate elements of work to delegate to the GenAI in ways that we normally would not think of doing before
% This can lead to fragmentation of work that would otherwise be part of a human's job into pieces that are fulfilled as-needed.
% The implications of this are twofold:
% (1) GenAI enables the fragmentation of traditional roles
% --> this is bad--piecework is bad
% (2) Piecework is done by tech instead of humans
% --> this may not be bad--piecework is bad


% piecework:
% where it happens, it is dehumanizing
% but humans were always still needed, and we needed infrastructure to transform this work into piecework/manage pieceworkers, etc.
% now humans may not be needed, and the opportunities to piecefy work are frightening


% Closely tied to shifting roles was how the versatility of GenAI corresponded to the fragmentation of traditional roles into pieces that could be absorbed by other roles. 

% The term piecework has traditionally been used to describe compensation that is determined by individual pieces of work, going back to coal miners in the 19th century being paid per unit of coal that they produced~\cite{hagan1973piece}. More recently, the same idea has been applied to gig-based microwork on platforms like Amazon Mechanical Turk, what Irani describes as ``cognitive piecework''~\cite{irani2015cultural}.
% HCI researchers have criticized the consequences of piecework for worker welfare; arguing that it makes work invisible~\cite{irani2013turkopticon} while handcuffing workers' control over their time under the myth of self-reliance~\cite{dubal2020time}.

% There are a few interesting wrinkles in the relationship between GenAI and piecework.
% To start, as Irani and others point out, piecework, particularly with respect to data work, is generaly a key prerequisite to producing data-driven AI models like those used in GenAI.
% Secondly, at one point, crowdsourced piecework was envisioned as a way to make up for AI's shortcomings at the kinds of complex tasks that we saw practitioners employing it towards.
% For example, Bernstein et al. prototyped Soylent, a word processor that integrated crowd workers to perform tasks like shortening or improving text~\cite{bernstein2010soylent}.
% The creators of Soylent saw its key contribution as the idea that integrating distributed human labor into an interface could (1) reduce reliance on individual humans, e.g. colleagues who might help with proofreading, and (2) seamlessly compensate for the inability of computers to do editing tasks requiring complex cognition. While this form of hybrid interface never took off, it is worth pointing out that it sought to achieve its goals by fragmenting the work of one human into something that could be accomplished by many pieceworkers.

% In our survey, however, we saw practitioners using GenAI tools precisely for the kind of small but complex cognition tasks that Bernstein et al. were trying to substitute piecework for--e.g. text summarization; but this time without the human pieceworker.
% And yet we argue that the implications of using GenAI in this way are similar--replacing the need for an individual human who would normally perform this kind of work.
% The flexibility of GenAI, including its ability to handle complex tasks previously thought to require human cognition, has the same fragmentary effect that technologies like the Soylent Word Processor would have for the labor consumer--allowing them to chunk the things they need done into pieces to be completed on-demand.



% As we think about role shifts, one affordance of GenAI that stood out in the diversity of ways that practitioners were crafting their work with it was its versatility to target specific tasks. While we did find examples of practitioners using GenAI to support their processes more generally, we found many cases where it was used for extremely specialized tasks at different stages. One of GenAI’s most unique affordances is its ability to play a variety of specialized roles–some practitioners attributed their productivity gains to being able to consolidate a suite of tools into a single one.

% A consequence of this extreme versatility is the ability to use GenAI to fulfill task-specific needs that may otherwise require a specialist with relevant expertise. This could lead to certain work that is traditionally part of a larger role instead being broken down into smaller, application-specific pieces. When such pieces can be fulfilled using GenAI, practitioners that would otherwise need to collaborate with a professional can simply absorb those capabilities into their own roles.

% Throughout the papers that we surveyed, we saw numerous cases of practitioners thinking about pieces of tasks in isolation that they normally would not have, without GenAI. 
% For example, illustrating an idea during the ideation process, refactoring how filenames are capitalized, or seeking inspiration to get unstuck are not typically the kinds of activities that would be outsourced, individually, to another human.
% However, these kinds of microtasks were quite typical uses of GenAI.




% Interestingly, this kind of crowdsourced cognitive gig-work, what Amazon once referred to as ``artificial artificial intelligence'', was envisioned to substitute for AI's shortcomings.
% Prototypes like Bernstein et al.'s Soylent word processor~\cite{bernstein2010soylent} stood in for shortcomings in text-based AI by using human crowd-workers to do things like shorten or improve sentences, at the click of a button.
% The motivation for creating such a crowd-powered intelligent system was twofold: (1) to reduce reliance on individual humans, e.g. colleagues who might help with proofreading, and (2) to compensate for the inability of computers to do editing tasks requiring complex cognition.
% While this form of hybrid interface never took off, it is worth pointing out that it sought to achieve its goals by fragmenting the work of one human into something that could be accomplished by many pieceworkers.

% With the advent of GenAI, however, we observe practitioners using GenAI tools precisely for the kind of small but complex cognition tasks that Bernstein et al. were trying to substitute piece-work for--e.g. text summarization; but this time without the human piece-worker.
% And yet we argue that the implications of using GenAI in this way are similar--replacing the need for an individual human who would normally perform this kind of work.
% The flexibility of GenAI, including its ability to handle complex tasks previously thought to require human cognition, has the same fragmentary effect that technologies like the Soylent Word Processor would have for the labor consumer--allowing them to chunk the things they need done into pieces to be completed on-demand.

% From the perspective of a practioner employing GenAI to craft their work, this can play in two ways. On one hand, they might use GenAI to address those pieces of work that they would ordinarily rely on human relationships for. In the Soylent example, this would be analogous to foregoing asking a friend or a copy-editor to review something you wrote by using GenAI to proofread each paragraph instead.
% In such cases, the practitioner is simply absorbing tasks that would otherwise be delegated to other roles.
% This of course has implications for the practitioners who would normally be filling those roles, if not the practitioner using GenAI.

% On the other hand, as discussed above, there were cases when humans began to split off tasks they would ordinarily perform to the GenAI.
% This kind of delegation runs the risk of removing those pieces from one's own role, particularly if the GenAI is good enough that the productivity gains outweigh the immediate costs.
% In some cases, such role reduction may be desirable, but in other cases they may ultimately be self-defeating.
% Unfortunately, even if one chooses not to engage in this kind of self-defeating crafting, pressures from others doing so and their short-term productivity gains may force one's hand.

% Even if not all of a role can be absorbed via GenAI, the fracturing of roles can have negative effects.
% Especially through studies of piece-work in the gig economy, HCI researchers have found that it tends to...
% tradeoff--sense of control vs piecework and productivity
% --what if we no longer are able to do our jobs as well as GenAI?



% \subsection{From Human Relationships to Collaboration with Machines}
% Concurrent to this, we observe some shift away from dependence on human relationships enabled by GenAI.
% Motivations for this could span from a desire not to waste other people's time to filling a role that would otherwise not be feasible, e.g. if a company doesn't have access to a UX writer, to capitalizing on opportunities to sidestep engagement with a human that could cost time or money.
% In some cases, GenAI fills roles that humans simply could not fill--acting out user personas or even speaking for inanimate objects in the design process.
% In other cases, it is directly filling roles that would otherwise be filled by a human relationship.


% Interestingly, this is happening at the task level, due to the ability of GenAI tools to target specific tasks.
% This ability to target specific tasks, e.g., illustrating a pitch deck or transcription, allows human roles to expand in ways that absorb other roles.
% For example, a UX designer role can absorb the role typically filled by a UX writer by using ChatGPT to fill in the gaps a writer would play on specific tasks.
% Point: the GenAI does not need to have the holistic ability of the writer, it just needs to be good at the specific aspects that are missing.

% The other interesting dynamic here is the extent to which the working relationships between humans and AI are collaborative vs delegative.
% Sometimes practitioners engage deeply with the GenAI, e.g. to brainstorm ideas, or the acting out personas example, while other times they simply outsource tasks to the GenAI.
% Of course, this reflects relationships in human collaboration to some extent;
% (maybe cite the data science paper here about how data scientists see data workers as subhuman.)



% --where is human agency here?
% --what is collaborative vs delegative?
% --replacing human relationships with machine
% --roles absorbing other roles


\subsection{Technology Crafting: A New Form of Job Crafting}

Finally, we observed several instances of job crafting with technology that do not fit the traditional job crafting framework. This form of crafting, which we label \textit{technology crafting}, is characterized by actions that are exclusively aimed at reconfiguring a specific \textit{technology} to improve one's work experience. In particular, this occurred through actions taken to manage or reconfigure GenAI tools that practitioners would not have otherwise engaged in except to integrate said tools into their practices. In these practices, GenAI became the focal point for modification and adoption, just like any other task or relationship.

Unlike prior research on AI \cite{Perez2022} in which users had little control over the structure and function of AI tools (e.g., algorithmic management through a tool), leaving employees to craft only reactively, GenAI was easily configurable and personalizable, supporting proactive crafting. At an extreme, this entailed fine-tuning a model with custom data to make it better at a very specific task or setting up self-hosting to control sensitive data sharing. This kind of labor includes behaviors like curating prompts, tuning model parameters, or managing when to have a tool be active or inactive to better fit one's working style.

These forms of technology crafting differed from previously-studied forms in two key ways. First, it required continuous interaction with GenAI, imposing temporal demands. Second, the black-box nature of the technology introduced variability in control, making it challenging for professionals to accurately predict the outcomes.

% unlike patchwork, our practitioners tend to have agency in the adoption of genai tools, so the invisible labor that they do happens throughout the cycle of their work, not just after the integration of the tool. This is important to recognize


% we argue that the Job crafting framework in its current form is limited in its ability to explain emerging AI managerial labor and reconfigurations that are exclusively aimed to customize the technology itself with an intent to improve practitioners’ work.






% Finally, we observed a new form of job crafting that does not quite fit under the standard definitions of crafting in these works.
% In what we refer to as tech crafting, practitioners expended effort to modify the extent or nature of how GenAI was integrated into their workflows.

% This could take the form of performing the kind of managerial labor the viewed necessary to practically use GenAI. This kind of labor, especially prompt design, is not something that they would otherwise have engaged in for their job--they specifically took on this kind of work in order to use GenAI.

% More interestingly, tech crafting could take the form of configuring or reconfiguring technology to fit the desires, requirements, or values of a practitioner.
% For example, choosing to host an open-source model locally for privacy reasons, and going through the effort and reasoning required to make that possible.
% Similarly, fine-tuning a generative model to fit a specific use case or even simply managing when to have the tool on or off, are examples of crafting the use of technology to fit a desired work experience.
\section{Limitations and Conclusion}
This work has a few limitations. To start, we focused our search on GenAI-enabled work practices performed in the HCI community. For this purspose, we limited ourselves to the ACM digital library. As more work emerges around how GenAI is being used, looking at broader research communities will help to tell a more comprehensive story. Further, the papers that we found relevant to our research objective were mostly qualitative. While this was appropriate to the nature of our question, quantitative survey studies can complement our narratives that we identified.

Finally, although GenAI tools are becoming accessible in fields beyond technology, the reviewed studies predominantly focused on technology-related occupations, highlighting a critical need for HCI studies to examine GenAI's impact across a broader range of professions.

In summary, this paper analyzed 23 papers to understand how GenAI is being used by practitioners to craft their jobs. We found that practitioners used GenAI to transform targeted aspects of the tasks they were performing, as well as to shape their roles and relationships. Based on our findings, we discussed how bottom-up usage of these tools was changing roles in unconventional ways, shifting task demand from high-level abstract thinking to more routine tasks, and facilitating the decomposition of roles into piecework. 
%We also suggest a need to expand the job crafting framework to consider ways in which practitioners craft the technology they use to transform their work experiences.


%
% ---- Bibliography ----
%
% BibTeX users should specify bibliography style 'splncs04'.
% References will then be sorted and formatted in the correct style.
%
\bibliographystyle{splncs04}
\bibliography{mainbib}
%

\end{document}
