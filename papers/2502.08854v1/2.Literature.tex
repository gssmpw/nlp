\section{Related Work}


\subsection{AI's Role in Work Transformations}

While the relationship between automation and work has long been studied,
\cite{frank2019toward} argue that the impact of AI-driven trends on labor may be harder to predict.
Autor~\cite{autor2015there} notes that automation tends to increase the value of complementary tasks where humans have comparative advantages.
% Autor~\cite{autor2015there} has argued that automation tends to complement tasks that it cannot substitute for, potentially leading to a bifurcation whereby automation hollows out middle tasks and creates demand for high-level abstract tasks and manual labor. However, Autor notes that machine-learning-driven AI, using inductive methods, diverges from traditional automation techniques in ways that might threaten tasks that utilize tacit knowledge, which traditionally have resisted automation. This raises the question of how and to what extent AI technologies might transform human work. 
%Through a quantitative lens, Park and Kim analyzed automation at a task level from 2008 to 2020, finding that machine control and hazardous tasks have become more automated, while systems analysis and creative and critical thinking have become less so \cite{park2022data}. However, a fair amount of work has also tried to answer this question through the experiences of workers in domains that have begun to integrate AI.
Early work in HCI suggests that, indeed, AI's limitations favor complementary transformations of work rather than simply displacing human labor. Fox et al. describe patchwork as a form of new invisible labor that emerges around the gaps between AI expectations and reality~\cite{fox2023patchwork}. This labor takes different forms, including \textit{compensating patchwork}, where humans fill in for AI when it breaks down, \textit{peripheral patchwork}, where humans watch over supposedly autonomous machines, and \textit{collaborative patchwork}, where a human works in concert with the AI to help adapt to contextual changes like the weather. However, how these changes will affect the dynamics of work across different professions remains to be seen.

%Workers also create patches--manual reconfigurations of AI systems to make them more reliable.
%In a similar vein, Levy observes that, while future models of automation in trucking may delegate more tasks completely to AI, the current reality is one of integrating AI into truckers' work through \textit{compelled hybridization} like smart wearable or camera systems to monitor fatigue while driving~\cite{levy2023}.
% Outside the realm of manual work, Wang et al. spoke to data scientists about the advent of AutoAI, which is geared towards automating data science tasks. Overall, they found that the data scientists viewed continued automation in their field as inevitable, but also believed that human expertise would never be fully displaced~\cite{wang2019human}.

% Overall, the evidence seems to suggest a near-term a divide between the vision of AI-driven automation in the workplace and the messy reality of how integrating AI into human work actually changes that work. With that said, the nature and availability of AI technologies in the workplace have drastically changed over the past two years with the rise of generative AI models. 

% \subsection{GenAI's Early Signs of Impact on Working Communities}

Rapid innovation in GenAI and their multi-modal task capabilities positions them to support professional work in new ways.
%such as hospitality \cite{Hsu_Tan_Stantic_2024},education \cite{Chan_Hu_2023}, and manufacturing \cite{Bendoly2023}. 
%\cite{23Jiang,Small2024}. 
Recent research predicts that almost 80\% of the U.S. workforce could have at least 10\% of their work tasks altered by GenAI technologies~\cite{Eloundou_2023}, exposing even high-skill roles to GenAI~\cite{Felten2023}.    

Two research narratives have emerged. The first has explored the potential of GenAI to bolster workers' tasks, skills, and outcomes. Experimental studies have indicated positive worker performance \cite{Al2024,Brynjolfsson2023}. 
%For instance, an initial study with customer support agents using Gen AI reflected a 14\% increase in the number of issues that they resolved per hour\cite{Brynjolfsson2023}. 
% A more recent study exploring the impact of GenAI tools on professional work in ten different job categories demonstrated at least 26 \% savings in their task duration\cite{Dell2023}. Here, professional consultants in a controlled experiment either delegated portions of work or continually interacted with GenAI, indicating nuances in their GenAI usage. GenAI tools might also augment employees' creative processes by provoking idea generation through serendipity~\cite{Epstein_2022}. 
Another stream of research has investigated shortcomings of GenAI technologies that negatively impact aspects of work. Recent studies have shown that professionals using GenAI may introduce biases and errors into their work processes, affecting the artifacts they produce %\cite{Miyazaki2024,
\cite{Kidd_Birhane_2023}. More broadly, studies have argued the negative impacts of these micro-changes on the development of professional capacities, with a strong potential to dilute human skills such as creativity and critical thinking \cite{Walczak_2023}. 

\subsection{HCI, Work, and Gen AI}
Several projects have already examined how specific communities or fields are engaging with GenAI. One question of interest is how practitioners in different fields are adopting such tools.
For example, Takaffoli et al., studying UX designers, found, among other things, that they used GenAI more for research tasks than design tasks, and more at an individual level than at a team level~\cite{takaffoli2024generative}.


Some projects focused on how GenAI becomes adopted in a given context. Boulus et al.~\cite{boulus2024genai} identified five stages in developing a functional augmentative relationship with GenAI.
%(1) playing around, (2) infatuation, (3) commitment, characterized by more serious considerations of how to use the tool, including financial and ethical considerations, (4) frustration with the tool as continued use exposes limitations or friction due to changes in the human or the AI's behavior, and (5) enlightenment, characterized by realization and adjustment to the complementary imperfections of themselves and the tool.
\cite{22}, meanwhile, found initial resistance among game design interns who were encouraged to use GenAI by program directors. Interestingly, they found that interns were more likely to embrace its use in programming tasks than artistic ones, considering ethics and quality.
%Their reasoning focused on ethical questions like ownership and agency, as well as variance between the quality of work produced by GenAI across the two domains. 
% The authors also found differences in perception of GenAI as a tool, rather than a threat to replace them~\cite{22}.

Others examined practitioner perceptions of how GenAI will transform their industries.
Li et al. found that experienced UX designers were confident that their uniquely human abilities would relegate GenAI tools to assistive roles, but worried that increased usage of these tools could negatively impact junior designers~\cite{li2024user}. In particular, they worried that pressure to keep up with the speed of GenAI  could lead to ``creativity exhaustion'' among human designers. 
%Kalving et al., through surveys, focus-groups, and interviews with designers, found a generally optimistic view of AI as a collaborator for human designers, couched in ethical concerns over issues like authenticity, and marked by an acceptance that designers would need to adapt, perhaps moving away from traditional UI design and focusing on uniquely human capacities like empathy for users~\cite{kalving2024ai}.

We were primarily interested in how practitioners are using GenAI on their own to transform the nature of their work and experiences. In particular, we were interested in the following research question (RQ): \textbf{what are the practitioner-led work transformations in response to the introduction of GenAI in their jobs?}


\subsection{Job Crafting: A Theoretical Lens} \label{RW-jobcrafting}
To address this question, we applied the theoretical framework of \textit{job crafting}, which originated as a theory to understand the design of people's work responsibilities, activities, and relationships. 
%What sets job crafting apart from other work design frameworks is that it attributes agency to workers, rather than managers or the organization, in designing their work. The framework focuses on the bottom-up physical and cognitive changes that workers initiate to modify their role or relationships boundaries in at least a semi-permanent way \cite{Tims_2012}. 
The idea was introduced to describe how professionals invest significant energy in configuring their jobs \cite{Wrzesniewski_2001}. A set of studies identified three main aspects of job crafting. First are the \textit{physical task} changes to the job that affect the `form' or the `number of activities.' Second are \textit{cognitive} changes that affect how individuals see the job. Third are \textit{relational} changes that influence interactions with colleagues. 

% Another strand of job crafting research views jobs as sets of demands (e.g., a tight deadline imposed on an journalist) and resources (e.g., a journalist using a fact-checking tool to make their research easier) and describes how in job crafting, workers attempt to reduce their demands and increase their resources.
\cite{zhang_2019} synthesized job crafting research by suggesting two higher-order constructs: (1) \textit{approach} crafting, which involves actively creating opportunities that align with one's professional preferences, strengths, and goals, and (2) \textit{avoidance} crafting, which describes practices employed to reduce negative work outcomes. 

Workers' crafting behaviors can occur in response to either proactive (e.g., an author wanting to publish a new experimental work) or reactive motivators (e.g., a writer coping with a new change in the organization) \cite{Lazazzara_2020}. 
% Interestingly, workers who engage in crafting behaviors proactively have been shown to benefit with respect to increased job performance, satisfaction, and wellbeing \cite{Petrou2012,Gordon2018}. 
% Additionally, not all workers in a job setting engage in job crafting at the same level. This is quite diverse based on the rank \cite{Berg2010}, degree of autonomy \cite{Berg2010}, and the resources the workers \cite{Petrou2012} have available to them. These combined insights indicate that job crafting is a complex process that is unique to every worker depending on their job \cite{Petrou2012}.
%their role, and their work environment, which can become a part of their daily work lives \cite{Petrou2012}, especially to improve meaning and increase moments of hopefulness \cite{Blustein2023}.
Job crafting theory has been studied extensively with full-time employees in traditional organizational contexts across various domains \cite{Berg2010,Fuller2017}. 
% Within these settings, studies have focused on both proactive and reactive crafting practices in specific work situations, such as  shaping identity \cite{Kenny2020}, goal-setting \cite{Bruning2019}, planning and scheduling \cite{Kossek2016}, work-transitions \cite{Gascoigne2018}, work-life balance \cite{Gravador2018}, and developing professional relationships \cite{Bizzi_2017}.
With recent transformations in job structures, job crafting has also been studied in contexts of non-traditional forms of work, such as gig work \cite{Wong_Fieseler_2021}. 
% For example, \cite{Panteli2022} explored how female IT contractors craft their jobs to thrive in a male-dominated sector. Similarly, \cite{Wong_Fieseler_2021} found that gig workers who engaged in individual and collaborative job crafting behaviors indicated higher resilience. 

Despite the burgeoning interest in job crafting, there is limited understanding of how AI technologies are shaping workers' crafting perspectives. Studying changing worker-led practices through the lens of job crafting in the advent of GenAI technologies is beneficial for multiple reasons. The job crafting lens allows us to position agency in the hands of the workers, the actual users of the GenAI technology. This framework also allows us to balance some of the techno-deterministic narrative surrounding the GenAI push with a social constructivist understanding of how individuals integrate GenAI into their work. Lastly, it provides a systematic approach to examine the nuanced sociotechnical practices at a micro-level.  

