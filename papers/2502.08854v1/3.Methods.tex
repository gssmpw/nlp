\section{Methods}
We performed a systematic literature review using the thematic synthesis approach to answer our broad research objectives \cite{Thomas2008}. Rather than beginning with a presupposed research question or hypothesis, thematic synthesis enables researchers to start with a high-level objective and use a systematic, inductive analysis to provide comprehensive answers while iteratively refining the objective. This method is particularly suited for reviews aimed not only at summarizing but also at \textit{extending} the existing literature through synthesis, producing higher-order structures between broad concepts \cite{Xiao_Watson_2019}.
% In this sense, the study draws on the long-standing tradition of meta-synthesis, such as meta-ethnography that focuses on producing inferences that expand the current understanding of the literature \cite{Thorne2004}.

% \begin{figure}
%     \centering
%     \includegraphics[width=0.5\linewidth]{images/cscw_analysis.png}
% \caption{A flowchart showing the overall steps taken to screen and shortlist the key literature for analysis}
%     \label{fig:analysis-process}
% \end{figure}

Thematic synthesis is conducted in three steps \cite{Thomas2008}. First, researchers identify, collect, and filter relevant literature to build a study corpus. Second, the study corpus is analyzed to extract core descriptive themes through the coding process, also known as second-order constructs. Third, the themes are clustered and synthesized into analytical themes, or third-order constructs.

\subsection{Searching \& Scoping the Literature}

\subsubsection{Identification}
We began the research process with a broad objective: understanding the relationship between recent AI developments and worker practices. Before initiating the main search, we conducted a preliminary exploration of the research landscape. We looked for papers relevant to AI and work practices in four prominent digital repositories containing HCI work: ACM, Elsevier, Springer, and IEEE. In parallel, we reached out to 15 expert researchers in the fields of technology, organizational studies, and labor studies, asking them to recommend relevant papers and key research gaps. We identified these experts from specific academic groups, mailing lists, prior workshops, and online communities. We shared a brief excerpt of our research objective and our proposed approach before asking them to suggest two to four literature items, excluding gray literature items such as workshops and work-in-progress.

Our preliminary exploration and expert recommendations revealed that papers describing how workers engaged with AI tools and adopted the technology in a bottom-up manner seemed the most relevant to our objective. We also observed a lack of consensus among the qualitative studies regarding how AI use was shaping practices beyond particular working groups. In light of these insights, we refined our final search to reduce the scope exclusively to how workers employed GenAI tools and the bottom-up, proactive efforts through which they used these to transform their work identity, practices, and relationships, for which the ACM Digital Library yielded the most relevant studies.


\subsubsection{Collection}
Based on our refined objective, we started our focused search in the ACM Digital Library through two main areas of keyword search: ones that described and captured \textit{GenAI} technologies and ones that captured practitioner's bottom-up \textit{work practices}. We experimented with the keywords until we achieved a good balance of finding diverse papers while minimizing noise, motivated by a purposive sampling strategy ~\cite{Xiao_Watson_2019}. Purposive sampling focuses on studies that can assist in conceptual synthesis through rich \textit{interpretation} rather than exhaustive search. Table \ref{tab:search_query} presents the final keywords used to find the initial corpus. We constructed our query as the conjunction of the disjunction of terms in each of these areas, over both the title and abstract fields. %Thus, an entry containing at least one of the terms in each area, in either the title or the abstract, was contained in our result set. 

\begin{table}
    \centering
    \begin{tabular}{p{1cm} p{1 cm} p{7.5cm} p{1.3cm}}
        \hline
       \textbf{KC} & \textbf{Op}  & \textbf{Search Terms}   & \textbf{Scope}\\
        \hline
        GenAI &
        &  
       ``generative artificial intelligence'' OR ``generative ai'' OR ``gen AI'' OR ``genAI'' OR ``large language models'' OR ``chatgpt'' OR ``chat-gpt'' OR ``stable diffusion'' OR ``dall-e'' OR ``midjourney'' OR ``AI-generated content'' OR ``text generation AI'' OR ``image generation AI'' OR ``AI creativity tools'' OR ``AI art generation'' OR ``AI-enhanced tools''&  
      
        Title,\ \ \ \ \ \ \ \ \ \ \ \ Abstract\\
       &  (AND) & & \\
        Work Practices &
        &
         ``labor'' OR ``jobs'' OR ``worker'' OR ``practitioner'' OR ``professional'' OR ``staff'' OR ``workforce'' OR ``work practices'' OR ``employment'' OR ``occupational'' OR career OR ``work habits'' OR ``job performance'' OR ``work design'' OR employee OR ``work patterns'' OR ``work routines'' OR ``work strategies'' OR ``employment practices'' OR ``task management'' OR ``job adaptation'' OR ``work changes'' &
         Title,\ \ \ \ \ \ \ \ \ \ \ \ Abstract\\     
        \hline
    \end{tabular}
    \caption{Final search query used to find relevant papers in the ACM Digital Library. The key concepts (KC) were joined by an AND operation (Op) and search terms were separated by OR operation. These keywords were applied to titles and abstracts.}
    \label{tab:search_query}
\end{table}

\subsubsection{Screening}
Our search query included results from 2022 to 2024 (until July), yielding 665 results. From an initial dataset of 665 papers, we performed a pass to identify the most relevant papers in two stages: 1) title screening and abstract screening, and 2) full article screening. To screen the papers effectively, we followed clear inclusion and exclusion criteria. We focused on only those experiences that described and presented evidence of change in work practices, avoiding speculative work \cite{he2024ai} or initial experiences from the system pilots \cite{petridis2023anglekindling}. Papers were included if they talked about a specific working community, their work practices, and the use of GenAI in these practices. We present a more detailed list of inclusion and exclusion criteria in Table \ref{tab:exclusion_inclusion}.

While we mainly focused on qualitative studies over papers focused on lived experiences, we included some survey and/or mixed-method papers that also designed interventions around what they learned.
% mixed-methods papers that used surveys or described intervention designs around experiences. 
Thematic analysis is particularly well-suited to analyzing and synthesizing such rich qualitative research, enabling both consensus-building and the development of extended arguments \cite{Barnett2009}. 
  
The title and abstract screening process was divided between two reviewers who rated papers independently. To reduce preconceived biases, we did not include authors' names or affiliations. To kick-start the process, both authors reviewed a smaller sample of papers together to build consensus on how to apply the filtering criteria. Once the authors started independent review, they used prolonged deliberation \cite{creswell} as a method to discuss, mark, and resolve uncertain cases. During the title and abstract filtering stage, we reduced the set of 665 papers down to 27. During our subsequent full paper review stage, we eliminated four more papers using the same criteria, resulting in a final set of 23 papers.


\begin{table}
    \centering
    \begin{tabular}{p{5cm} p{7cm}}
        \hline
        Inclusion Criteria   & Exclusion Criteria \\
        \hline
        \begin{enumerate}
            \item Studies that captured workers' change of practices as one of their key themes.
            \item Studies that focused on GenAI.
         
            
        \end{enumerate} &
        \begin{enumerate}
            \item Studies that focused on speculative experiences.
            \item Papers that focused on quantitative and computational methodologies. 
            \item Papers that covered experimental studies (e.g., comparative studies between practitioners' use and non-use of GenAI).
            \item Papers that focused on GenAI-based system design and its evaluation through practitioner feedback.
            \item Papers that conducted systematic review.
            \item Low-quality studies.    
            \
        \end{enumerate}
        \\
        \hline
    \end{tabular}
    \caption{Table depicting the inclusion and exclusion criteria for finding relevant papers.}
    \label{tab:exclusion_inclusion}
\end{table}

\subsection{Analysis}
We started our analysis by familiarizing ourselves with the kinds of reported data in the papers and their style of reporting. 
%All the papers in our final list followed a consistent structure of reporting, with most of their empirical data being available in the findings, in the form of key concepts and quotations. This aligned with the recommendations suggested by \cite{Thomas2008}'s best practices. 
Afterward, the authors went through each paper's finding section line-by-line, engaging with their concepts and coding the key insights in order to develop second-order constructs. Once we developed codes for a few papers, we used peer debriefing \cite{creswell} to discuss the codes and work on the disagreements. This process was repeated until all the papers were coded, producing a total of 31 codes. 

In the second phase of our analysis, we focused on developing third-order constructs that went beyond the data presented in the papers. We used an abductive approach, utilizing the theoretical framing of job crafting \cite{Wrzesniewski_2001} to produce interpretations that mapped to our research objective. For instance, the authors employed the notion of task and cognitive crafting to analyze how practitioners reconfigured their tasks and repositioned their beliefs around job identity. This process was repeated until comprehensive analytical themes emerged. To help with the process, authors developed multiple conceptual maps that helped them draw connections across the descriptive themes \cite{Miles2020}. This analysis ended with \textit{thirteen} analytical themes, including \textit{`AI managerial labor,'} \textit{`Expanding role capacities,'} and \textit{`Displacing human dependencies with GenAI.'} 
%A full accounting of the key themes and relevant codes are presented in the attached supplementary materials. 
 


