
\subsection{Transforming Role Attributes} \label{sec:role}
The second transformation involved practitioners using GenAI to enhance their roles, not only shifting their perspectives of these roles (cognitive crafting) but also expanding their role capacities (task crafting). Some of these transformations were temporary, wherein they augmented their abilities through GenAI for the purpose of completing a particular task. Other practices were more permanent, which allowed practitioners to focus on professional development by learning skills relevant to their work. 

\subsubsection{Augmenting Abilities Through GenAI}
In many cases, practitioners used GenAI to expand the range of tasks they could perform without improving their own skills or acquiring additional knowledge in the long term. 
%We refer to this practice as augmenting abilities through GenAI and noticed a high prevalence in our sample studies.  

One area of work in which participants augmented their abilities with GenAI was in the realm of communication. An industrial designer in~\cite{18}, for example, described how CAD-generated images filled in visual details that allowed him to tell a story around the packaging he was designing. In another domain, fact-checkers described using GenAI to connect instantly with audience members on a broader scale, or to convert fact checks into an audio format to share through different modalities like TikTok~\cite{1}. Despite not knowing how to do these tasks before, practitioners could accomplish them successfully using GenAI. 

Other practitioners used GenAI to fill perceived gaps in their critical thinking abilities, such as by using GenAI to make external assessments (e.g., evaluating ideas for a startup)~\cite{21} or to improve their sense-making abilities. A designer in~\cite{18} used GenAI to reveal complex connections between their current work and previous work, connections that they were initially not aware of. A few practitioners used GenAI for more routine but core tasks where they experienced a disadvantage, such as proofreading manuscripts in a non-native language \cite{13}. GenAI also allowed practitioners to perform tasks beyond their skill set. 
%For example, fact-checkers used GenAI to generate search queries in languages they didn't speak or identify the sources of images~\cite{1}. 
This could take the form of GenAI filling in for missing expertise in constrained situations, e.g., ``\textit{In our startup, we don't have a dedicated UX writing role. Our designers often use ChatGPT to assess the appropriateness of the UX content in our design}''~\cite{2}.


\subsubsection{Learning Through GenAI}
Using GenAI in multiple steps enabled practitioners to develop specific abilities and knowledge as a by-product of their dyadic interactions. Multiple developers argued that GenAI provided quicker and more effective access to information for knowledge acquisition than traditional search engines or reading the documentation directly~\cite{8,3}. 
%For example, a software developer claimed that,``\textit{what has been really helping me with this tool is precisely the process of trying to understand something that I’m not currently grasping about what a certain code is supposed to be doing}''~\cite{16}.  

Some practitioners exhibited explicit intent to use GenAI for learning purposes. Examples included developers using GenAI to learn how to do static type checking with type hints in Python~\cite{23}, exploring solutions to problems they'd never needed to solve before~\cite{4}, and understanding machine learning libraries~\cite{23}.

We also observed several counterintuitive examples, where, in attempts to improve productivity through GenAI, some practitioners skipped the learning process altogether. These practitioners experienced significant pressure to use the GenAI tools to decrease their project's delivery time but felt this compromised knowledge acquisition and skill development. For instance, a designer in~\cite{18} shared, ``\textit{there's no longer a day or a week of reading up for the project; instead I collect a bunch of materials and pretty much dump it into GPT-4.}''

Along these lines, the ability to use GenAI in this way reduced practitioners' sense of control and ownership around their work. This could lead practitioners to perceive their roles as evolving into assistants to GenAI. A graphic artist working with Midjourney found his role shifting to handing off ideas to the AI to illustrate, then fixing any errors, a change that he described as ``heartbreaking''~\cite{9}. Nonetheless, other practitioners found the role of GenAI to become more specialized over time, discovered ``progressive ownership'' in adjusting its outputs, or identified the value of human qualities in their work processes.To cope, practitioners engaged in the crafting practice of \textit{role re-framing}, shifting their sense of worth from their technical skills to their original ideas. As their experience grew, they began viewing themselves as managers of the GenAI tools. 

