\section{Limitations and Conclusion}
This work has a few limitations. To start, we focused our search on GenAI-enabled work practices performed in the HCI community. For this purspose, we limited ourselves to the ACM digital library. As more work emerges around how GenAI is being used, looking at broader research communities will help to tell a more comprehensive story. Further, the papers that we found relevant to our research objective were mostly qualitative. While this was appropriate to the nature of our question, quantitative survey studies can complement our narratives that we identified.

Finally, although GenAI tools are becoming accessible in fields beyond technology, the reviewed studies predominantly focused on technology-related occupations, highlighting a critical need for HCI studies to examine GenAI's impact across a broader range of professions.

In summary, this paper analyzed 23 papers to understand how GenAI is being used by practitioners to craft their jobs. We found that practitioners used GenAI to transform targeted aspects of the tasks they were performing, as well as to shape their roles and relationships. Based on our findings, we discussed how bottom-up usage of these tools was changing roles in unconventional ways, shifting task demand from high-level abstract thinking to more routine tasks, and facilitating the decomposition of roles into piecework. 
%We also suggest a need to expand the job crafting framework to consider ways in which practitioners craft the technology they use to transform their work experiences.

