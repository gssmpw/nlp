
\section{Introduction}
Like any large-scale change, technological innovations, such as the emergence of personal computers, smart devices, and the advent of the ``first wave''~\cite{Autor2023} of AI-integration, have contributed to destabilization in workers' lives \cite{Anteby_2016}. Destabilization is not necessarily negative, and technology-enabled destabilization can sometimes create positive outcomes for workers, such as increased productivity \cite{Brynjolfsson2023}. But for the majority of workers, these changes create a period of vulnerability and challenges that compel modification of their practices to keep themselves relevant. Studying these \textit{worker-led transformations} in response to the evolving techno-centric labor landscape can illuminate how practitioners might not only regain but maintain stability despite occupational destabilization. 

While blue-collar workers have been the most significant victims of the aforementioned occupational destabilization~\cite{Ford_2015},  GenAI (e.g., ChatGPT) is destabilizing white-collar occupations~\cite{Felten2023} as well.
%\cite{Kalleberg_2009} as well. 
%In fact, GenAI is poised to shape and impact as many as 80\% of occupations in the U.S. alone \cite{Eloundou_2023}. This scale of impact is unprecedented. 
Emerging studies on the effect of GenAI tools in this space tend to take a top-down approach, limiting workers' experiences to narrow metrics within controlled settings \cite{Brynjolfsson2023}. In contrast, recent HCI studies have taken a bottom-up approach to capture more nuanced worker experiences. While these studies offer valuable insights, they are typically limited to specific occupations, highlighting the need for a broader understanding across various fields. Addressing this gap, our research objective was to \textbf{understand how practitioners across diverse occupations have transformed their own work in response to GenAI integration}. 

We conducted a systematic literature review of 23 papers in the ACM Digital Library from the past two years (2022–2024). This period marked the widespread adoption of off-the-shelf GenAI tools. Through this review, we aimed to develop generalizable insights into the lived experiences of practitioners.
%, which could help workers safeguard themselves from potential harms of GenAI while enhancing their overall job outcomes.
Our findings show that, with the integration of GenAI into their work, practitioners across 18 professions transformed their \textit{tasks} in various ways. Many delegated peripheral tasks to GenAI, freeing up time for higher-priority core activities, while some actively engaged with GenAI to enhance the quality of core tasks. In certain instances, practitioners even assigned core tasks to GenAI. However, this flexibility brought with it a new burden: the need for ``AI managerial labor'': additional work to oversee and manage GenAI effectively.

Practitioners also transformed their \textit{collaborations} by increasing their reliance on GenAI where they might have otherwise relied on other stakeholders. In extreme cases, practitioners used GenAI to bypass peers and subordinates to achieve positive outcomes or circumvent negative ones. 
%These kinds of GenAI-driven reconfigurations could foster grappling with changing roles and identities. For example, practitioners could view themselves as assistants to GenAI when offloading core tasks or managers when they delegating peripheral tasks. 

Our findings highlight the emergence of several tensions. First, while practitioners adopted GenAI to reduce their workload, resulting demands to manage GenAI and its outputs shifted their roles away from their professional identities. Second, applying GenAI to specific task components fragmented traditional workflows into piecework, eroding established boundaries and safeguards, with certain tasks being absorbed into other roles. Our study makes the following contributions:

\begin{enumerate}
    \item We present comprehensive, worker-led transformations in identity, tasks, and relationships across 18 professions in response to the introduction of GenAI.
    \item We compile different forms of emerging labor being taken up by practitioners engaging in white-collar jobs exposed to GenAI under the umbrella of \textit{AI managerial labor}. 
    \item We lay out several tensions that emerged as a direct result of practitioners integrating GenAI into their workflows. 
\end{enumerate}

