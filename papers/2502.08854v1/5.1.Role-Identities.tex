\subsection{The Dynamics of Task Transformation with GenAI}
Our insights show that practitioners primarily turned to GenAI to increase their overall work productivity and efficiency. However, in the process, they created a need for new forms of labor to manage the AI (see Section \ref{sec:aimanagerial}). In some ways, these findings reflect prior work around the long-term transformation of technology-driven automation on work \cite{autor2003skill}. \cite{autor2015there} argued that automation can increase the value of complementary tasks. 
%For example, the rise of AI-driven automation for data collection tasks has created demand for practitioners to make sense of this new data.
While this has traditionally lead to a polarization of labor, where routine tasks are automated and create demand for tasks which require non-routine human abilities (abstract reasoning or manual interaction with messy environments), our findings suggest a shift with GenAI.



% At the same time, our findings diverge from the typical pattern described by \cite{autor2015there} in how this displacement occurs. \cite{autor2015there} suggested that the demand for routine well-structured tasks (also referred to as ``middle-skilled'' activities) tends to give way to tasks that complement automation and require unique human abilities, such as high-level abstract thinking or manual interaction with the messy, unpredictable world. 
% Our GenAI-specific findings demonstrate an opposite trend. 
In addition to routine tasks, some practitioners voluntarily gave GenAI abstract and creative tasks, while regularly taking on tasks to manage GenAI like fixing mistakes in its outputs. This departure reflects the unique capabilities of GenAI tools, which are more capable than previous forms of AI in completing tasks that require complex human abilities. Forms of AI managerial labor like verifying outputs still require human intuition. It remains to be seen whether these will also be susceptible to automation and how the market will value them.

% The pockets of tasks that were still exclusive to practitioners were those that involved human ability to infer contextual preferences and how those map to the outputs. This human quality characterizes much of the AI managerial labor that we observed.

\subsection{Arising Tensions in Shifting Role Identities}
Aforementioned transformations in tasks were also accompanied by practitioners grappling with their shifting role identities.  Workers can perceive threats to their professional identity when they experience changes to their role~\cite{petriglieri2011under}. Workers often respond to such threats by either perceiving a sense of identity crises or perceiving an opportunity for identity growth~\cite{zikic2016happens}. In our findings, GenAI assumed this dual role wherein it was perceived as a threat by the practitioners when they offloaded tasks that they perceived were central to their professional identity (see Section \ref{sec:role}). This contributed to these practitioners perceiving GenAI to devalue their role and contributed to identity crises.

In contrast, others re-framed their roles in response to GenAI, perceived it as a positive force, and focused on identity growth. These findings echo \cite{zikic2016happens}'s work around threat perceptions of immigrant workers and their resultant experiences of identity crisis or growth when adapting to a new workplace. Future studies need to focus on designing support structures that help practitioners handle these threats in ways that lead to growth rather than identity crisis.

