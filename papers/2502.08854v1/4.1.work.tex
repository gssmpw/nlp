
\subsection{Transforming Work Processes}
\label{sec:workprocesses}
Practitioners primarily transformed their work by leveraging GenAI to craft and reshape various stages of their regular tasks. This concept is known as task crafting, and our analysis revealed it was utilized across seven different stages of work: discovery, analysis, artifact creation, quality assurance, delivery, logistics, and overall management of work.
%(see Figure \fixme{\ref{fig:task-processes}}).

% The second category focused on developing the practitioner's own professional role, both in short- and long-term. The third category focused on the crafting labor that went into the technology to fine-tune and customize it to improve its efficacy the crafting process. 

\subsubsection{Discovery}
A significant proportion of the transformed practices observed in the papers we reviewed involved tasks in the discovery phase of the participants' work. The discovery phase consists of early-stage processes where practitioners learn more about the problem at hand. One way practitioners used GenAI in the discovery phase was by delegating relevant efforts to it, reducing their scope (avoidance crafting). A common theme in this category was offloading time-consuming information foraging tasks to GenAI. For example, a fact checker in~\cite{1} described how they delegated their laborious work of finding trending topics to ChatGPT:

\begin{quote}
    \textit{``We take the top 200 headlines from the last 24 hours from those sites [...] and run them through ChatGPT, asking it to summarize the main narratives [...] and extract the names of people, places, entities [...] and then send that to me by email. So every six hours, [...] we get an email.''}
\end{quote}

Software developers applied similar strategies to more specific instances of information foraging, such as aggregating syntax examples for guidance~\cite{4}. 
%All of these examples reduced the effort needed from practitioners to complete their tasks.

In contrast to delegating effort, several practitioners actively engaged with GenAI tools to improve aspects of their tasks in the discovery stage (approach crafting). This included practitioners seeking inspiration from GenAI as they embarked on projects. For example, developers used GenAI to seek out new problems to solve~\cite{4}, while visual artists engaged with GenAI playfully to find serendipitous inspiration~\cite{9}. It should be noted that seeking inspiration through GenAI was not limited to initial exploration. A participant in~\cite{6} used GenAI to help them flesh out ideas that they had already partially formed, while designers in~\cite{6,18} used GenAI to help develop mood boards. Developers, on the other hand, used it to explore new ways to solve problems, including learning complex logic creation and identifying ways to reuse solutions \cite{4}.

Another common activity observed in this category was using GenAI for brainstorming, with participants finding ways to speed up idea generation~\cite{10}. This involved coaxing responses or even including GenAI in their existing brainstorming techniques like reverse-thinking~\cite{21}. For example, a user interface designer \cite{6} explained using used GenAI to expand the scope of their ideation:   

\begin{quote}
    ``\textit{I would ask him to tell me alternatives. Push it to think about it in a different way. I always ask – any other ideas? And it would always come back with something}.''
\end{quote}

Practitioners also used GenAI to kick-start their creative processes, such as as generating a first draft of a written project ~\cite{10} or providing guidance on what kinds of software libraries might be useful for a problem~\cite{12}. 

%As one UX designer ~\cite{2} put it: 

% \begin{quote}
%     ``\textit{GenAI tools can help nowadays in generating basic ideas to help us populate the blank canvas, thereby aiding in overcoming the fear of the `empty canvas.'}''.
% \end{quote}

Additionally, GenAI helped kick-start work in areas where practitioners' confidence in their expertise wavered, such as an amateur programmer learning programming concepts to help them write code for a new project~\cite{6}, or software developers turning to GenAI to develop deeper understandings of their craft and to strategize solving challenging problems~\cite{16}.

\subsubsection{Analysis} Certain practitioners leveraged GenAI to change how they performed analysis-focused tasks. These tasks typically involved taking raw information and finding specific patterns. Similar to the discovery phase, practitioners delegated analysis tasks to GenAI that they felt were particularly time-consuming. For instance, a UX researcher~\cite{10} used a GenAI tool embedded in Miro\footnote{www.miro.com},
a digital whiteboarding tool, to summarize brainstorming data.

Surprisingly, practitioners also \textit{actively} engaged with GenAI to make their analysis process more efficient. For instance, \cite{10} showed how end-user-facing practitioners repeatedly consulted with GenAI to synthesize insights from different forms of raw data. A product designer shared how they did the analysis:

\begin{quote}
    \textit{``It was like 75 open ended survey responses and I [...] strip them of [the participant number] and dump them into ChatGPT, and asked [ChatGPT] to generate 5 insights based on the 75.'' }
\end{quote}

Practitioners engaged with GenAI for other analysis activities as well. Some used it for text analysis, some for clustering data for analysis (e.g., thematic analysis), others for categorizing user testing observations, and still others for identifying pain points in customer emails~\cite{21}.

\subsubsection{Artifact Creation} 
In the intermediate to late stages of their processes, users relied on GenAI to help create work-related artifacts. The types of artifacts varied, including personas generated from user transcripts~\cite{21}, product descriptions~\cite{21}, merchandise, and game assets~\cite{20}. Surprisingly, the practices tended to lean toward avoidance crafting rather than active collaboration with GenAI. The level of delegation ranged greatly. On the conservative end of the spectrum, some practitioners outsourced only a small portion of artifact creation, e.g.: 

\begin{quote}
``\textit{I could generate any of this in Photoshop or Illustrator [...] the fact that it was able to render these things on an aesthetic level that was exceeding my bar or at my bar, and doing it in an instant, was mind-boggling. What it did is it gave me time to dabble in other areas.}''~\cite{6}
\end{quote}

On the far end of the spectrum, practitioners used GenAI to perform the entire artifact creation, e.g., software developers asking GenAI to generate entire new features or classes from scratch~\cite{19,11}. Amplified output capacity was one of the most positive consequences of task crafting. A set of professionals in diverse roles at a game design firm, for example, ``were unanimous in considering the creation of more content in less time a strength of [GenAI] systems ''~\cite{9}. 
%~\cite{3} particularly associate this effect with content creation and creativity in ways that could influence practitioner's self-perceptions: ``by leveraging ChatGPT's capabilities to expand their creative or informational output, participants experienced a sense of productivity and accomplishment.''

\subsubsection{Quality Assurance \& Delivery} 
Beyond creating task-related artifacts, multiple papers provided evidence of participants using GenAI for quality assurance and delivery tasks. These practices were primarily aimed at offloading participants' responsibilities (avoidance crafting). For quality assurance tasks, this often meant relying on GenAI for code maintenance in software development, such as refactoring code snippets~\cite{4,19}. Those in user-facing roles, such as UX practitioners, focused on designing evaluation methods to maintain or even improve product quality. For example, some used GenAI to generate user test case scenarios, directly integrating the scenarios' output into their projects~\cite{21}.

Reliance on GenAI increased during the delivery stage, with practitioners delegating the creation of resources meant to communicate key ideas to stakeholders. These resources included pitch documents to present content strategy, images to visualize concepts, and presentations to aid communication during handover~\cite{9,10,12,21}. UX designers in~\cite{10,21} described using Midjourney to generate visual representations and mock-up descriptions, all with the goal of reducing communication challenges and enhancing clarity. Developers, in turn, leveraged GenAI extensively for creating various forms of code documentation~\cite{4,19} and reports~\cite{8}.
% This was particularly evident in~\cite{8}, where software developers used GenAI to minimize the effort required for writing reports.
% One developer shared:

% \begin{quote}
% \textit{``The tool’s ability to generate coherent and relevant content and provide valuable insights and suggestions significantly supported participants’ writing activities, enabling them to produce these types of artifacts with greater ease.''}    
% \end{quote}
% matt: this is not a participant quote; it's a quote from a paper (it is paper R8, referring to generation of writing artifacts like reports).

\subsubsection{Task Management \& Logistics} Within this stage, we aggregated all the secondary tasks that practitioners performed to ensure the smooth functioning of their roles. These tasks included planning, coordinating, and managing various resources and processes. Practitioners engaged with GenAI within this stage in two distinct ways. First, they used GenAI to streamline their overall workflow. This included using GenAI to help formulate clear project goals and develop concrete support mechanisms to achieve those goals~\cite{6}. %For example, a sci-fi writer from the same study shared how they leveraged GenAI to create a ``hero's journey'' document -- used by writers to track progress and plot elements. The writer then used ChatGPT to further break down the document into manageable steps, addressing various criteria such as deadlines and project budget.
Speech language pathologists used GenAI to optimize the documentation processes required at different stages of their workflow, such as recording intake forms and writing evaluation reports~\cite{7}. One speech language pathologist posted in an online community:

\begin{quote}
    \textit{``[...] I’m in the process of creating a Google Form for speech/language intake. After collecting responses (ensuring privacy by removing personal details), you can efficiently transfer this data into ChatGPT alongside a preset prompt template and your evaluation notes. [\dots] ''}
    %This approach can greatly expedite the report writing process, using AI to structure and incorporate information seamlessly into a Word document!''}
\end{quote}

Practitioners also used GenAI to delegate specific logistical activities within tasks to enhance work efficiency. One knowledge worker expressed in \cite{3}, ``\textit{It feels good to outsource this [kind of] work to ChatGPT because I don't enjoy it much}''~\cite{3}. Common delegated activities included summarizing material or simplifying dense content. For example, practitioners frequently used GenAI to condense their own written content for presentations or meeting notes while they focused on conveying critical business decisions. We found this strategy to be particularly prevalent in UX practitioners \cite{10} and knowledge workers \cite{3}. 
%In contrast, fact-checkers relied on GenAI tools to break down and restructure ``dense content'' into a format that would be simpler and more digestible for their end-users \cite{1}.

Another common logistical activity was delegating short but highly repetitive tasks to GenAI, such as generating repetitive content. Participants frequently offloaded such tasks to capable programs. For example, software developers used GenAI to generate boiler plate code~\cite{16}, while fact-checkers performed transcription and translation tasks using GenAI~\cite{1}. Similarly, many practitioners relied on GenAI to fix repetitive but predictable issues, such as a software developer using ChatGPT to write a bash script to change the capitalization convention of file names~\cite{11}. GenAI helped practitioners save significant time in these cases by reducing the need for context switching between different applications~\cite{8}.
%Streamlining or delegating work to GenAI allowed practitioners to focus on tasks they found more important or engaging, e.g., preparing for discussions or concentrating on more advanced aspects of their work~\cite{3}. 


