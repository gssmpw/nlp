\subsection{The Versatility of GenAI \& Fractured Roles}
In order to redefine their tasks and roles, practitioners frequently restructured their work in innovative ways to engage effectively with GenAI. This happened across various stages of their workflows (see Section \ref{sec:workprocesses}) and through two primary approaches. First, practitioners segmented their own tasks into distinct parts, identifying elements that GenAI could independently accomplish, and then collaborated on these. Second, practitioners selectively undertook tasks typically managed by external stakeholders, completing them with GenAI’s support. This approach of compartmentalizing tasks with GenAI’s assistance began to resemble a form of piecework\cite{hagan1973piece}. Piecework is a process of decomposing complex work into smaller ``pieces'' that could be performed by individuals with varied skill sets~\cite{alkhatib2017examining}, with recent HCI applying this concept to modern crowd-work \cite{irani2015cultural}. 

It's possible that using GenAI as a pieceworker could alleviate some of the issues that typically arise around piecework. For example, GenAI's versatile features enabled practitioners to assign piecework to it while overseeing the tasks. Unlike traditional piecework dynamics, one might argue that situations where a practitioner chooses to parcel out elements of their work to GenAI gives them unprecedented agency in what aspects of their roles become piecework. In an ideal scenario, this could allow the practitioner to focus on more fulfilling aspects of their roles without exploiting other workers.

However, from practitioners' perspectives, transforming their tasks into piecework brought multiple uncertainties to their roles. One common uncertainty came from the fact that the affordances of the GenAI tool, rather than the preferences of the practitioner, tend to determine what aspects of work get off-loaded. If, for example, tweaking GenAI-generated images is significantly faster than a human artist producing those images, continuing to do this work manually can carry significant risk of falling behind. In addition, the potential for some roles to absorb pieces of other roles can introduce power asymmetries that undermine relationships between neighboring roles. Practitioners can also run the risk of transforming visible aspects of their work into invisible (e.g., prompt designing). This can lead to acrimonious relationships between workers and management~\cite{alkhatib2017examining}.

 
 
 
%  the opportunity cost of not realizing these productivity gains could make doing those aspects of the work untenable, something we saw play into practitioners' choices to use GenAI tools.
% In this case, the practitioner does not really have agency over what aspects of their role become piecework. In cases where most of a role gets transformed into piecework, effected practitioners may end up caught between losing work to GenAI and dealing with the negative dynamics that accompany piecework.

% practitioners, it is questionable the extent to which practitioners can actually control what gets converted to piecework or not.

% Beyond shifting identities, we noticed that crafting with GenAI could lead to traditional roles being broken down into task-sized pieces. We attribute this to the versatility of GenAI tools which allow practitioners to target specific tasks as-needed. 

% Across the papers that we surveyed, it was common for practitioners to break off elements of their work that would not normally be treated separately, in order to delegate them to GenAI. For example, illustrating an idea during the ideation process or seeking inspiration to get unstuck are not typically the kinds of activities that would be outsourced, individually, to another human. However, performing these kinds of microtasks were quite typical uses of GenAI.

% We saw this decomposition happening in two primary ways: (1) practitioners breaking off and absorbing pieces of external roles by asking GenAI to fulfill tasks outside their skillset, and (2) practitioners splitting off tasks that they would normally perform themselves and offloading them to GenAI. An example of GenAI crafting fracturing an external role is a UX design team choosing to use GenAI as-needed instead of hiring a UX writer to evaluate their writing, splitting this kind of work off from UX writers. Alternatively, we saw examples of crafting like a software developer asking GenAI to write class definitions, or a UX designer asking GenAI to craft a persona from user data. In these cases, the crafting practitioner explicitly carves pieces of their own work out of their roles. 

% The practice of decomposing larger tasks to be performed and compensated on-demand relates to the historical phenomenon of piecework~\cite{alkhatib2017examining, hagan1973piece}. Piecework was a movement that emerged in the 19th century around decomposing complex tasks into small pieces that could be performed by individuals with either limited or extremely specialized expertise. Recent scholarship in HCI has sought to apply this concept to modern crowd-work on platforms like Amazon Mechanical Turk~\cite{irani2015cultural}. 

% We argue that practitioners breaking off parts of their roles into tasks that GenAI can perform as needed resembles piecework, with some important distinctions. Most importantly, the piecework in this case is being performed by GenAI, rather than a human worker. GenAI's versatility and the flexibility of the modalities through which we interact with it, such as language and images, position it as an ideal pieceworker, with limited expertise to tackle large problems comprehensively, but perfectly capable of handling constrained tasks with enough oversight. Just like riveters, who do not undergo extended metalwork apprenticeships but become experts at one focused task, GenAI models can be finetuned with custom data for custom tasks, like we saw one fact-checker doing to generate search queries.

% From a worker perspective, transforming more comprehensive roles into piecework can have negative consequences.
% While advocates of piecework have historically pointed to increased worker autonomy and the ability to use it to solve complex problems at scale, detractors have highlighted instances where it enables worker exploitation through asymmetries of power or information, tends to overlook invisible labor performed by workers, and can lead to acrimonious relationships between workers and management~\cite{alkhatib2017examining}.

% It's possible that using GenAI as a pieceworker could alleviate some of the issues that typically arise around piecework.
% Indeed, unlike traditional piecework dynamics, one might argue that situations where a practitioner chooses to parcel out elements of their work to a GenAI gives them unprecedented agency in what aspects of their roles become piecework. In an ideal scenario, this could allow the practitioner to focus on more fulfilling aspects of their roles, parting out less desirable pieces, without exploiting other workers.

% However, based on our earlier discussion of role shifts, it also seems possible that the affordances of the GenAI tool, rather than the preferences of the practitioner, ultimately determine what aspects of work get off-loaded to the GenAI. If, for example, tweaking GenAI-generated images is significantly faster than producing those images as a human artist, the opportunity cost of not realizing these productivity gains could make doing those aspects of the work untenable, something we saw play into practitioners' choices to use GenAI tools.
% In this case, the practitioner does not really have agency over what aspects of their role become piecework. In cases where most of a role gets transformed into piecework, effected practitioners may end up caught between losing work to GenAI and dealing with the negative dynamics that accompany piecework.

% Overall, GenAI's versatility positions it as a sort of ideal pieceworker, enabling the breaking down of traditional roles, either externally or internally. While this may have positive implications for some practitioners, it is questionable the extent to which practitioners can actually control what gets converted to piecework or not. Insofar as piecework can negatively affect work experiences, it is important to determine the extent to which markets can be designed to preserve practitioners' agency in how this plays out. This is complicated by the fact that our findings indicate that many of these shifts are happening organically due to choices that practitioners make on their own. At the very least, conversations need to happen around how we value and incentivize piecework, as it is performed either by a human or an AI.









%  On one hand, unlike traditional instances of piece work, the worker themselves exert agency in choosing to break off aspects of their work. In an ideal scenario, this could lead to situations like those practitioners who used GenAI to save time and allow them to focus on more fulfilling aspects of their roles. However, there is also a possibility that the affordances of the GenAI tool, rather than the preferences of the practitioner, determine how work is off-loaded to the AI.
% For example, if tweaking passable outputs from a GenAI tool is significantly faster than relying on human-generated work, the productivity gains could create market pressures that make doing the work yourself untenable. This could explain the examples we saw of graphic designers feeling forced to use GenAI tools to keep up with the market\fixme{the example in the intro}, or struggling with a sense of lost control and feeling the need to re-imagine themselves as AI managers.


% The flexibility of the modalities through which we interact with GenAI, e.g. language and visual images, position it to act as a sort of chorus of workers, with limited expertise to tackle large wicked problems on its own, but perfectly capable of handling constrained chunks of tasks at a level that is acceptable with enough oversight. The ability to fine-tune models with custom data, like we saw one fact-checker doing, even presents the possibility to create the kinds of focused expertise that ~\citet{alkhatib2017examining} say characterize late-stage, advanced piecework, e.g., workers who don't undergo extended apprenticeships as metalworkers but, by focusing on one task, become expert riveters. The striking compatibility of GenAI to perform this kind of piecework, and the demonstrated willingness of practitioners to engage it thusly, raises the question of what the implications will be for those whose work aligns with this trend. Even though GenAI is performing the piecework, using it in this way is likely to affect the structure and dynamics of work for those who normally perform the tasks being parted out.














% Traditionally, piecework describes this...
% good and bad...



% What we see here has a couple of clear departures from traditional piecework, however.

% a) GenAI is the piece worker
% b) workers can sometimes have agency in the kinds of work they parcel off





% Looking closer at the sense practitioners had of their roles shifting, we also observed increasing complexities in how the tasks responsibilities encompassed by specific roles could fracture in unintended ways due to the choices that they made about how to use GenAI. 

% This is because the versatility of GenAI tools affords targeting individual pieces of work as the need arises. Across the papers that we surveyed, it was common for practitioners to break off elements of their work that would not normally be treated separately, in order to delegate them to GenAI. For example, illustrating an idea during the ideation process, refactoring how filenames are capitalized, or seeking inspiration to get unstuck are not typically the kinds of activities that would be outsourced, individually, to another human. However, performing these kinds of microtasks were quite typical uses of GenAI.

% These dynamics of decomposing larger tasks to be performed on-demand by a GenAI resemble the historical phenomenon of piecework \cite{}. Piecework was a movement that emerged in the 19th century around decomposing complex tasks into small pieces that could be performed by individuals with either limited or extremely specialized expertise. Under this system, workers were paid per piece of work, instead of for a set amount of time.

% % Examples of this spanned from coal miners  being paid per unit of coal that they produced~\cite{hagan1973piece} to astronomer George Airy mailing out pieces of complex computations to distribute them among workers with rudimentary math backgrounds~\cite{alkhatib2017examining}. 
% Recent scholarship in HCI has sought to apply this concept to modern crowd-work on platforms like Amazon Mechanical Turk, what Irani describes as ``cognitive piecework''~\cite{irani2015cultural}. While advocates of piecework have historically pointed to increased worker autonomy and the ability to use it to solve complex problems at scale, detractors have highlighted instances where it enables worker exploitation through asymmetries of power or information, tends to overlook invisible labor performed by workers, and can lead to acrimonious relationships between workers and management~\cite{alkhatib2017examining}.

% The major difference in our study is that the piece work is being performed, not by humans, but by GenAI tools. The flexibility of the modalities through which we interact with GenAI, e.g. language and visual images, position it to act as a sort of chorus of workers, with limited expertise to tackle large wicked problems on its own, but perfectly capable of handling constrained chunks of tasks at a level that is acceptable with enough oversight. The ability to fine-tune models with custom data, like we saw one fact-checker doing, even presents the possibility to create the kinds of focused expertise that ~\citet{alkhatib2017examining} say characterize late-stage, advanced piecework, e.g., workers who don't undergo extended apprenticeships as metalworkers but, by focusing on one task, become expert riveters. The striking compatibility of GenAI to perform this kind of piecework, and the demonstrated willingness of practitioners to engage it thusly, raises the question of what the implications will be for those whose work aligns with this trend. Even though GenAI is performing the piecework, using it in this way is likely to affect the structure and dynamics of work for those who normally perform the tasks being parted out.

% We broadly observed two ways in which the roles were decomposed into tasks that could be delegated to GenAI, with different consequences. In one, the practitioner extends or augments their own abilities using the GenAI to accomplish tasks that are traditionally bound up in other roles. For instance: a manager using a text-to-image generator to illustration presentation materials to sidestep working with an artist, or a UX design team avoiding hiring a UX writer by making use of an LLM when needed. This kind of behavior has consequences for those in the roles that would traditionally be relied on to fulfill these kinds of tasks. At a structural level, those pieces of work that practitioners are turning to GenAI to fulfill are broken out from the traditional role that contained them into work that is requested and fulfilled by the piece. In a practical sense, other roles absorb these pieces, fragmenting the traditional roles into the pieces that are not substitutable by GenAI.

% A second, and more interesting way that we see role fracturing happening is when practitioners reconfigure their own work to give some of it to the GenAI. This does not describe the cases where we saw practitioners inviting GenAI in to expand some aspect of an activity they were doing (e.g. as an extra brainstorming perspective). Instead, these include behaviors like asking the GenAI to make actual design artifacts or analyze user data for them. While this may be done out of exigency, these kinds of choices have the potential to segment and remove those pieces of work from one's own role.

% This a complex dynamic to interpret. On one hand, unlike traditional instances of piece work, the worker themselves exert agency in choosing to break off aspects of their work. In an ideal scenario, this could lead to situations like those practitioners who used GenAI to save time and allow them to focus on more fulfilling aspects of their roles. However, there is also a possibility that the affordances of the GenAI tool, rather than the preferences of the practitioner, determine how work is off-loaded to the AI.
% For example, if tweaking passable outputs from a GenAI tool is significantly faster than relying on human-generated work, the productivity gains could create market pressures that make doing the work yourself untenable. This could explain the examples we saw of graphic designers feeling forced to use GenAI tools to keep up with the market\fixme{the example in the intro}, or struggling with a sense of lost control and feeling the need to re-imagine themselves as AI managers.

% What are the implications of this? At the very least, conversations need to happen around how we value and incentivize both piece work and AI managerial work. It is one thing if a push towards role fragmentation and piece work originated top-down, but our survey suggests that it is emerging naturally in the ecosystem as a result of the decisions practitioners are making to improve their own work experiences.

% Secondly, are there ways to preserve agency over what kinds of work to parcel out, independent of the affordances of GenAI? Are there precedents for successfully structuring markets around drivers other than productivity? Or are there ways that we can shape the development of models and the infrastructure around them to preserve human agency over how to divide work and collaborate with GenAI in a way that doesn't sacrifice its benefits? Are there ways to organize practitioners to resist market pressure to adopt GenAI tools where they do not wish to? (precedent from piece work match girls or slow working)


% ``The literature on the history of piecework does not frame the
% question as whether piecework is inherently ethical or unethical, instead asking what conditions render it exploitative''~\cite{alkhati}
% Argument:
% very common to isolate elements of work to delegate to the GenAI in ways that we normally would not think of doing before
% This can lead to fragmentation of work that would otherwise be part of a human's job into pieces that are fulfilled as-needed.
% The implications of this are twofold:
% (1) GenAI enables the fragmentation of traditional roles
% --> this is bad--piecework is bad
% (2) Piecework is done by tech instead of humans
% --> this may not be bad--piecework is bad


% piecework:
% where it happens, it is dehumanizing
% but humans were always still needed, and we needed infrastructure to transform this work into piecework/manage pieceworkers, etc.
% now humans may not be needed, and the opportunities to piecefy work are frightening


% Closely tied to shifting roles was how the versatility of GenAI corresponded to the fragmentation of traditional roles into pieces that could be absorbed by other roles. 

% The term piecework has traditionally been used to describe compensation that is determined by individual pieces of work, going back to coal miners in the 19th century being paid per unit of coal that they produced~\cite{hagan1973piece}. More recently, the same idea has been applied to gig-based microwork on platforms like Amazon Mechanical Turk, what Irani describes as ``cognitive piecework''~\cite{irani2015cultural}.
% HCI researchers have criticized the consequences of piecework for worker welfare; arguing that it makes work invisible~\cite{irani2013turkopticon} while handcuffing workers' control over their time under the myth of self-reliance~\cite{dubal2020time}.

% There are a few interesting wrinkles in the relationship between GenAI and piecework.
% To start, as Irani and others point out, piecework, particularly with respect to data work, is generaly a key prerequisite to producing data-driven AI models like those used in GenAI.
% Secondly, at one point, crowdsourced piecework was envisioned as a way to make up for AI's shortcomings at the kinds of complex tasks that we saw practitioners employing it towards.
% For example, Bernstein et al. prototyped Soylent, a word processor that integrated crowd workers to perform tasks like shortening or improving text~\cite{bernstein2010soylent}.
% The creators of Soylent saw its key contribution as the idea that integrating distributed human labor into an interface could (1) reduce reliance on individual humans, e.g. colleagues who might help with proofreading, and (2) seamlessly compensate for the inability of computers to do editing tasks requiring complex cognition. While this form of hybrid interface never took off, it is worth pointing out that it sought to achieve its goals by fragmenting the work of one human into something that could be accomplished by many pieceworkers.

% In our survey, however, we saw practitioners using GenAI tools precisely for the kind of small but complex cognition tasks that Bernstein et al. were trying to substitute piecework for--e.g. text summarization; but this time without the human pieceworker.
% And yet we argue that the implications of using GenAI in this way are similar--replacing the need for an individual human who would normally perform this kind of work.
% The flexibility of GenAI, including its ability to handle complex tasks previously thought to require human cognition, has the same fragmentary effect that technologies like the Soylent Word Processor would have for the labor consumer--allowing them to chunk the things they need done into pieces to be completed on-demand.



% As we think about role shifts, one affordance of GenAI that stood out in the diversity of ways that practitioners were crafting their work with it was its versatility to target specific tasks. While we did find examples of practitioners using GenAI to support their processes more generally, we found many cases where it was used for extremely specialized tasks at different stages. One of GenAI’s most unique affordances is its ability to play a variety of specialized roles–some practitioners attributed their productivity gains to being able to consolidate a suite of tools into a single one.

% A consequence of this extreme versatility is the ability to use GenAI to fulfill task-specific needs that may otherwise require a specialist with relevant expertise. This could lead to certain work that is traditionally part of a larger role instead being broken down into smaller, application-specific pieces. When such pieces can be fulfilled using GenAI, practitioners that would otherwise need to collaborate with a professional can simply absorb those capabilities into their own roles.

% Throughout the papers that we surveyed, we saw numerous cases of practitioners thinking about pieces of tasks in isolation that they normally would not have, without GenAI. 
% For example, illustrating an idea during the ideation process, refactoring how filenames are capitalized, or seeking inspiration to get unstuck are not typically the kinds of activities that would be outsourced, individually, to another human.
% However, these kinds of microtasks were quite typical uses of GenAI.




% Interestingly, this kind of crowdsourced cognitive gig-work, what Amazon once referred to as ``artificial artificial intelligence'', was envisioned to substitute for AI's shortcomings.
% Prototypes like Bernstein et al.'s Soylent word processor~\cite{bernstein2010soylent} stood in for shortcomings in text-based AI by using human crowd-workers to do things like shorten or improve sentences, at the click of a button.
% The motivation for creating such a crowd-powered intelligent system was twofold: (1) to reduce reliance on individual humans, e.g. colleagues who might help with proofreading, and (2) to compensate for the inability of computers to do editing tasks requiring complex cognition.
% While this form of hybrid interface never took off, it is worth pointing out that it sought to achieve its goals by fragmenting the work of one human into something that could be accomplished by many pieceworkers.

% With the advent of GenAI, however, we observe practitioners using GenAI tools precisely for the kind of small but complex cognition tasks that Bernstein et al. were trying to substitute piece-work for--e.g. text summarization; but this time without the human piece-worker.
% And yet we argue that the implications of using GenAI in this way are similar--replacing the need for an individual human who would normally perform this kind of work.
% The flexibility of GenAI, including its ability to handle complex tasks previously thought to require human cognition, has the same fragmentary effect that technologies like the Soylent Word Processor would have for the labor consumer--allowing them to chunk the things they need done into pieces to be completed on-demand.

% From the perspective of a practioner employing GenAI to craft their work, this can play in two ways. On one hand, they might use GenAI to address those pieces of work that they would ordinarily rely on human relationships for. In the Soylent example, this would be analogous to foregoing asking a friend or a copy-editor to review something you wrote by using GenAI to proofread each paragraph instead.
% In such cases, the practitioner is simply absorbing tasks that would otherwise be delegated to other roles.
% This of course has implications for the practitioners who would normally be filling those roles, if not the practitioner using GenAI.

% On the other hand, as discussed above, there were cases when humans began to split off tasks they would ordinarily perform to the GenAI.
% This kind of delegation runs the risk of removing those pieces from one's own role, particularly if the GenAI is good enough that the productivity gains outweigh the immediate costs.
% In some cases, such role reduction may be desirable, but in other cases they may ultimately be self-defeating.
% Unfortunately, even if one chooses not to engage in this kind of self-defeating crafting, pressures from others doing so and their short-term productivity gains may force one's hand.

% Even if not all of a role can be absorbed via GenAI, the fracturing of roles can have negative effects.
% Especially through studies of piece-work in the gig economy, HCI researchers have found that it tends to...
% tradeoff--sense of control vs piecework and productivity
% --what if we no longer are able to do our jobs as well as GenAI?



% \subsection{From Human Relationships to Collaboration with Machines}
% Concurrent to this, we observe some shift away from dependence on human relationships enabled by GenAI.
% Motivations for this could span from a desire not to waste other people's time to filling a role that would otherwise not be feasible, e.g. if a company doesn't have access to a UX writer, to capitalizing on opportunities to sidestep engagement with a human that could cost time or money.
% In some cases, GenAI fills roles that humans simply could not fill--acting out user personas or even speaking for inanimate objects in the design process.
% In other cases, it is directly filling roles that would otherwise be filled by a human relationship.


% Interestingly, this is happening at the task level, due to the ability of GenAI tools to target specific tasks.
% This ability to target specific tasks, e.g., illustrating a pitch deck or transcription, allows human roles to expand in ways that absorb other roles.
% For example, a UX designer role can absorb the role typically filled by a UX writer by using ChatGPT to fill in the gaps a writer would play on specific tasks.
% Point: the GenAI does not need to have the holistic ability of the writer, it just needs to be good at the specific aspects that are missing.

% The other interesting dynamic here is the extent to which the working relationships between humans and AI are collaborative vs delegative.
% Sometimes practitioners engage deeply with the GenAI, e.g. to brainstorm ideas, or the acting out personas example, while other times they simply outsource tasks to the GenAI.
% Of course, this reflects relationships in human collaboration to some extent;
% (maybe cite the data science paper here about how data scientists see data workers as subhuman.)



% --where is human agency here?
% --what is collaborative vs delegative?
% --replacing human relationships with machine
% --roles absorbing other roles
