
\begin{abstract}
    % Deep learning has achieved remarkable success in medical image analysis. However, its clinical application is often hindered by data heterogeneity caused by variations in scanner vendors, imaging protocols, and operators. 
    % Approaches such as invariant risk minimization (IRM) aim to address this challenge of out-of-distribution generalization. For instance, VIRM improves upon IRM by tackling the issue of insufficient feature support overlap, demonstrating promising potential. 
    % Nonetheless, these methods face limitations in medical imaging due to the scarcity of annotated data and the inefficiency of augmentation strategies.
    % To address these issues, we propose a novel domain-oriented direction selector to replace the random augmentation strategy used in VIRM. 
    % Our method leverages inter-domain covariance as an indicator for augmentation direction, guiding data augmentation towards the target domain. This approach effectively reduces domain discrepancies and enhances generalization performance.
    % Experiments on a multi-center diabetic retinopathy dataset demonstrate that our method outperforms state-of-the-art approaches, particularly under limited data conditions and significant domain heterogeneity. 
    % The code for our method has been made publicly available.
    Deep learning has achieved remarkable success in medical image classification.
However, its clinical application is often hindered by data heterogeneity caused by variations in scanner vendors, imaging protocols, and operators. 
Approaches such as invariant risk minimization (IRM) aim to address this challenge of out-of-distribution generalization. 
For instance, VIRM improves upon IRM by tackling the issue of insufficient feature support overlap, demonstrating promising potential. 
Nonetheless, these methods face limitations in medical imaging due to the scarcity of annotated data and the inefficiency of augmentation strategies. 
To address these issues, we propose a novel domain-oriented direction selector to replace the random augmentation strategy used in VIRM. 
Our method leverages inter-domain covariance as a guider for augmentation direction, guiding data augmentation towards the target domain. 
This approach effectively reduces domain discrepancies and enhances generalization performance. 
Experiments on a multi-center diabetic retinopathy dataset demonstrate that our method outperforms state-of-the-art approaches, particularly under limited data conditions and significant domain heterogeneity. 
The code for our method has been made publicly available.
\end{abstract}
    
\begin{IEEEkeywords}
    Domain Generalization, Invariant Risk Minimization, Semantic Data Augmentation, Medical Image Classification
\end{IEEEkeywords}
