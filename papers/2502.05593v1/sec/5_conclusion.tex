\section{Conclusion}
\label{sec:conclusion}

% In this work, we proposed a novel medical vicinal invariant risk minimization (MedVIRM) framework to improve the generalization performance of invariant risk minimization (IRM) by expanding the feature overlap between domains. 
% Specifically, we introduced an augmentation direction to enhance feature diversity while maintaining label consistency.
% Our experiments on challenging datasets demonstrated significant improvements in both accuracy and robustness compared to baseline methods, highlighting the potential of MedVIRM for real-world domain generalization tasks.
% Despite these promising results, our study has some limitations. The method was primarily evaluated on synthetic and small-scale datasets, and its computational efficiency in large-scale scenarios needs further investigation. 
% Future work could explore more efficient implementations and evaluate the framework on diverse, real-world medical datasets.
% In conclusion, MedVIRM represents a step forward in domain generalization research, providing a practical and effective solution to challenges in medical applications.

In this work, we proposed a novel domain-oriented framework to enhance the generalization performance of invariant risk minimization (IRM) by expanding the feature overlap between domains. 
Specifically, we introduced an augmentation direction selector that enhances feature diversity while maintaining label consistency. 
Our method was not specifically designed for the diabetic retinopathy grading task, but rather as a general framework for domain generalization in medical image analysis. 
This flexibility suggests that our approach has the potential for broad applicability across different medical imaging domains. 
Our experiments on challenging datasets demonstrated significant improvements in accuracy and robustness compared to baseline methods, highlighting the potential of our approach for real-world domain generalization tasks. 
Despite promising results, the method was primarily evaluated on synthetic and small-scale datasets, and its scalability in large-scale scenarios needs further investigation. 
Future work should explore more efficient implementations and evaluate the framework on diverse, real-world medical datasets.
