%%%%%%%%%%%%%%%%%%%%%%%%%%%%%%%%%%%%%%%%%%%%%%%%%%%
% Discussion \& Future Work
%%%%%%%%%%%%%%%%%%%%%%%%%%%%%%%%%%%%%%%%%%%%%%%%%%%
\section{Discussion \& Future Work}
\label{sec:future-work}

%In this section, we discuss various aspects of \Method and potential directions for future work.

\textbf{Generalizability of \Method}
In the quantitative study, we mainly focus on \Dataset~\citesensorqa as it is the only available sensor-based QA dataset that aligns with our problem statement. However, \Method is not limited to this dataset. As demonstrated in the user study (Sec.\ref{sec:deployment}), \Method can generalize to broader real-life scenarios and answer high-level qualitative questions.
Moreover, the design of \Method can be easily expanded.
Specifically, \Method can incorporate more sensors via adding additional modality encoders.
It can also adapt to more diverse questions by expanding the solution templates or using more powerful LLMs such as OpenAI's o3-mini~\cite{openaio3} or DeepSeek~\cite{deepseek} during question decomposition.
Last but not least, \Method can serve a larger user base by collecting more data and equipping with better generalization designs such as~\cite{xu2023practically}. \Method is a general and flexible framework designed for fusing long-duration multimodal sensors knowledge with language and answering versatile questions.


%Generalizability of SensorChat: extension to larger number of users and more sensors. If we have more data available, SensorChat’s performance can be further improved
%Generalizability of SensorChat to more complex questions

\textbf{Personalization of \Method} This work emphasizes \Method's ability to generalize across a broad range of questions and users. However, providing personalized responses is equally crucial, particularly for health-related analyses. We foresee the potential to tailor \Method for individual users by personalizing the LLMs~\cite{kim2024health}, a direction we intend to explore in future research.

\textbf{Evaluation Metrics} \Method uses traditional NLP metrics and accuracy to evaluate the quality and precision of generated answers. While these metrics excel in objectiveness, they do not capture subjective user satisfaction, which is crucial for assessing a system's readiness for a broad market. 
We leave a comprehensive evaluation of \Method with subjective metrics as our future work.



\textbf{Efficiency of \Method} In this work, we demonstrate that \Method can be deployed on Jetson-level edge devices using common quantization techniques. However, further optimization is needed for real-time LLM service on edge devices. The long-term goal for systems like \Method is to run on mobile devices. Recent studies~\cite{liu2024mobilellm, zhuang2024litemoe} have explored enabling real-time LLM serving on mobile devices, and these techniques could be integrated into \Method in future work. Additionally, we recognize the potential synergy between \Method and vector databases~\cite{zhou2024llm} or retrieval-augmented generation~\cite{zhao2024retrieval}. The techniques in these domains can be integrated into \Method to improve query performance and efficiency.
%We discuss the potential to further improve \Method with vector databases~\cite{zhou2024llm} or retrival augmented generation~\cite{zhao2024retrieval} in Sec.~\ref{sec:future-work}.

