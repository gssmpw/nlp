%%
%% This is file `sample-sigconf.tex',
%% generated with the docstrip utility.
%%
%% The original source files were:
%%
%% provide a.dtx  (with options: `sigconf')
%% 
%% IMPORTANT NOTICE:
%% 
%% For the copyright see the source file.
%% 
%% Any modified versions of this file must be renamed
%% with new filenames distinct from sample-sigconf.tex.
%% 
%% For distribution of the original source see the terms
%% for copying and modification in the file samples.dtx.
%% 
%% This generated file may be distributed as long as the
%% original source files, as listed above, are part of the
%% same distribution. (The sources need not necessarily be
%% in the same archive or directory.)
%%
%%
%% Commands for TeXCount
%TC:macro \cite [option:text,text]
%TC:macro \citep [option:text,text]
%TC:macro \citet [option:text,text]
%TC:envir table 0 1
%TC:envir table* 0 1
%TC:envir tabular [ignore] word
%TC:envir displaymath 0 word
%TC:envir math 0 word
%TC:envir comment 0 0
%%
%%
%% The first command in your LaTeX source must be the \documentclass
%% command.
%%
%% For submission and review of your manuscript please change the
%% command to \documentclass[manuscript, screen, review]{acmart}.
%%
%% When submitting camera ready or to TAPS, please change the command
%% to \documentclass[sigconf]{acmart} or whichever template is required
%% for your publication.
%%
%%
\documentclass[acmlarge]{acmart}

\usepackage{amsmath,amsfonts} % amssymb
%\usepackage{algorithmic}
\usepackage[ruled,vlined]{algorithm2e}
\usepackage{graphicx}
\usepackage{textcomp}
\usepackage{xcolor}
\usepackage{tabularx}
\usepackage{caption}
\usepackage{subcaption}
\usepackage[inline]{enumitem}
\newcommand{\Ch}{\checkmark}
\newcommand{\X}{$\times$}
\usepackage{bm}
\usepackage{bbm}
\usepackage{url}
\usepackage{amsthm}
\usepackage{wrapfig}
\usepackage{multicol}
\usepackage{multirow}
\newcommand\mycommfont[1]{\small\ttfamily\textcolor{blue}{#1}}
\SetCommentSty{mycommfont}

\usepackage{xspace}
\newcommand{\Method}{SensorChat\xspace}
\newcommand{\MethodC}{SensorChat\textsubscript{C}\xspace}
\newcommand{\MethodE}{SensorChat\textsubscript{E}\xspace}
\newcommand{\Dataset}{SensorQA\xspace}
\newcommand{\citesensorqa}{\cite{sensorqa}\xspace}
\newcommand*{\benjamin}{\textcolor{blue}}
\newcommand*{\lanxiang}{\textcolor{orange}}
% Define colors
\definecolor{myred}{RGB}{170,0,0} % Dark red
\definecolor{myblue}{RGB}{0,0,170} % Dark blue
\definecolor{mygreen}{RGB}{0,100,0} % Dark green


%%
%% \BibTeX command to typeset BibTeX logo in the docs
\AtBeginDocument{%
  \providecommand\BibTeX{{%
    Bib\TeX}}}

%% Rights management information.  This information is sent to you
%% when you complete the rights form.  These commands have SAMPLE
%% values in them; it is your responsibility as an author to replace
%% the commands and values with those provided to you when you
%% complete the rights form.
\setcopyright{acmlicensed}
\copyrightyear{2025}
\acmYear{2025}
\acmDOI{XXXXXXX.XXXXXXX}
%% These commands are for a PROCEEDINGS abstract or paper.
%\acmJournal{}
%\acmVolume{0}
%\acmNumber{0}
%\acmArticle{0}
%\acmMonth{0}


%%
%% Submission ID.
%% Use this when submitting an article to a sponsored event. You'll
%% receive a unique submission ID from the organizers
%% of the event, and this ID should be used as the parameter to this command.
%%\acmSubmissionID{123-A56-BU3}

%%
%% For managing citations, it is recommended to use bibliography
%% files in BibTeX format.
%%
%% You can then either use BibTeX with the ACM-Reference-Format style,
%% or BibLaTeX with the acmnumeric or acmauthoryear sytles, that include
%% support for advanced citation of software artefact from the
%% biblatex-software package, also separately available on CTAN.
%%
%% Look at the sample-*-biblatex.tex files for templates showcasing
%% the biblatex styles.
%%

%%
%% The majority of ACM publications use numbered citations and
%% references.  The command \citestyle{authoryear} switches to the
%% "author year" style.
%%
%% If you are preparing content for an event
%% sponsored by ACM SIGGRAPH, you must use the "author year" style of
%% citations and references.
%% Uncommenting
%% the next command will enable that style.
%%\citestyle{acmauthoryear}


%%
%% end of the preamble, start of the body of the document source.
\begin{document}

%%
%% The "title" command has an optional parameter,
%% allowing the author to define a "short title" to be used in page headers.
\title{\Method: Answering Qualitative and Quantitative Questions during Long-Term Multimodal Sensor Interactions}

%\Method: Time Series Sensor-based Question Answering System for Daily-Life Monitoring

%\Method: Multimodal Sensor-based Question Answering System for Daily-Life Monitoring

% \Method: A Real-Time Question Answering System for Daily-Life Monitoring using Time-Series Sensors

%%
%% The "author" command and its associated commands are used to define
%% the authors and their affiliations.
%% Of note is the shared affiliation of the first two authors, and the
%% "authornote" and "authornotemark" commands
%% used to denote shared contribution to the research.
\author{Xiaofan Yu}
\email{x1yu@ucsd.edu}
\orcid{0000-0002-9638-6184}
\affiliation{%
  \institution{University of California San Diego}
  \city{La Jolla}
  \state{California}
  \country{USA}
}

\author{Lanxiang Hu}
\email{lah003@ucsd.edu}
\orcid{0000-0003-0641-3677}
\affiliation{%
  \institution{University of California San Diego}
  \city{La Jolla}
  \state{California}
  \country{USA}
}

\author{Benjamin Reichman}
\email{bzr@gatech.edu}
\orcid{0009-0004-3854-7930}
\affiliation{%
  \institution{Georgia Institute of Technology}
  \city{Atlanta}
  \state{Georgia}
  \country{USA}
}

\author{Dylan Chu}
\email{dchu@ucsd.edu}
\orcid{0009-0001-5511-3286}
\affiliation{%
  \institution{University of California San Diego}
  \city{La Jolla}
  \state{California}
  \country{USA}
}

\author{Rushil Chandrupatla}
\email{ruchandrupatla@ucsd.edu}
\orcid{0009-0006-5447-8693}
\affiliation{%
  \institution{University of California San Diego}
  \city{La Jolla}
  \state{California}
  \country{USA}
}


\author{Xiyuan Zhang}
\email{xiyuanzh@ucsd.edu}
\orcid{0000-0002-8908-1307}
\affiliation{%
  \institution{University of California San Diego}
  \city{La Jolla}
  \state{California}
  \country{USA}
}

\author{Larry Heck}
\email{larryheck@gatech.edu}
\orcid{0000-0003-3358-6362}
\affiliation{%
  \institution{Georgia Institute of Technology}
  \city{Atlanta}
  \state{Georgia}
  \country{USA}
}

\author{Tajana \v{S}imuni\'{c} Rosing}
\email{tajana@ucsd.edu}
\orcid{0000-0002-6954-997X}
\affiliation{%
  \institution{University of California San Diego}
  \city{La Jolla}
  \state{California}
  \country{USA}
}


%%
%% By default, the full list of authors will be used in the page
%% headers. Often, this list is too long, and will overlap
%% other information printed in the page headers. This command allows
%% the author to define a more concise list
%% of authors' names for this purpose.
\renewcommand{\shortauthors}{}
\newcommand{\xiyuan}[1]{\textcolor{violet}{\textbf{Xiyuan:} #1}}
\newcommand{\lx}[1]{\textcolor{blue}{\textbf{Lanxiang:} #1}}


%%
%% The abstract is a short summary of the work to be presented in the
%% article.
\begin{abstract}
Natural language interaction with sensing systems is crucial for enabling all users to comprehend sensor data and its impact on their everyday lives.
However, existing systems, which typically operate in a Question Answering (QA) manner, are significantly limited in terms of the \textit{duration} and \textit{complexity} of sensor data they can handle.

In this work, we introduce \Method, the first end-to-end QA system designed for long-term sensor monitoring with multimodal and high-dimensional data including time series. \Method effectively answers both qualitative (requiring high-level reasoning) and quantitative (requiring accurate responses derived from sensor data) questions in real-world scenarios. To achieve this, \Method uses an innovative three-stage pipeline that includes question decomposition, sensor data query, and answer assembly.
The first and third stages leverage Large Language Models (LLMs) for intuitive human interactions and to guide the sensor data query process.
Unlike existing multimodal LLMs, \Method incorporates an explicit query stage to precisely extract factual information from long-duration sensor data.
We implement \Method and demonstrate its capability for real-time interactions on a cloud server while also being able to run entirely on edge platforms after quantization. Comprehensive QA evaluations show that \Method achieves up to 26\% higher answer accuracy than state-of-the-art systems on quantitative questions. Additionally, a user study with eight volunteers highlights \Method's effectiveness in handling qualitative and open-ended questions.
\end{abstract}

%, enhanced with in-context learning and few-shot learning.
%Among the three stages, the sensor data query stage makes the key contribution by aligning the sensor embedding proposes a novel contrastive sensor-text pretraining loss and 
%The intermediate sensor data query stage ensures accurate sensor information extraction. 
%%
%% The code below is generated by the tool at http://dl.acm.org/ccs.cfm.
%% Please copy and paste the code instead of the example below.
%%
\begin{CCSXML}
<ccs2012>
   <concept>
       <concept_id>10010147.10010257</concept_id>
       <concept_desc>Computing methodologies~Machine learning</concept_desc>
       <concept_significance>500</concept_significance>
       </concept>
   <concept>
       <concept_id>10010147.10010178.10010179</concept_id>
       <concept_desc>Computing methodologies~Natural language processing</concept_desc>
       <concept_significance>500</concept_significance>
       </concept>
   <concept>
       <concept_id>10010520.10010553.10010562</concept_id>
       <concept_desc>Computer systems organization~Embedded systems</concept_desc>
       <concept_significance>500</concept_significance>
       </concept>
 </ccs2012>
\end{CCSXML}

\ccsdesc[500]{Computer systems organization~Embedded systems}
\ccsdesc[500]{Computing methodologies~Machine learning}
\ccsdesc[300]{Computing methodologies~Natural language processing}

%%
%% Keywords. The author(s) should pick words that accurately describe
%% the work being presented. Separate the keywords with commas.
\keywords{Question Answering, Multimodal Sensors, Large Language Models}
%% A "teaser" image appears between the author and affiliation
%% information and the body of the document, and typically spans the
%% page.

\settopmatter{printacmref=false}

%\received{20 February 2007}
%\received[revised]{12 March 2009}
%\received[accepted]{5 June 2009}

%%
%% This command processes the author and affiliation and title
%% information and builds the first part of the formatted document.
\maketitle


\IEEEPARstart{H}{yperspectral} image (HSI) can record the spectral characteristics of ground objects~\cite{6555921}. As a key technology of HSI processing, HSI classification is aimed at assigning a unique category label to each pixel based on the spectral and spatial characteristics of this pixel~\cite{FPGA,10078841,10696913,10167502}, which is widely used in agriculture~\cite{WHU-Hi}, forest~\cite{ITreeDet}, city~\cite{WANG2022113058}, ocean~\cite{WHU-Hi} studies and so on.

The existing HSI classifiers~\cite{FPGA,10325566,10047983,9573256} typically assume the closed-set setting, where all HSI pixels are presumed to belong to one of the \textit{known} classes. However, due to the practical limitations of field investigations across wide geographical areas and the high annotation costs associated with the limited availability of domain experts, it is inevitable to have outliers in the vast study area~\cite{MDL4OW,Fang_OpenSet,Kang_OpenSet}. These outliers do not belong to any known classes and will be referred as \textit{unknown} classes hereafter. A classifier based on closed-set assumption will misclassify the unknown class as one of the known classes. For example, in the University of Pavia HSI dataset (Fig.~\ref{fig:open_set_example}), objects such as vehicles, buildings with red roofs, carports, and swimming pools are ignored from the original annotations~\cite{MDL4OW}. These objects are misclassified as one of predefined known classes.

\begin{figure}[!t]
    \centering
    \includegraphics[width=0.98\columnwidth]{example_paviau.png}
    \caption{Comparison of classification results between closed-set based classifier and open-set based classifier for the University of Pavia dataset. The dataset originally contains nine \textit{known} land cover classes, however, significant misclassifications occur in the \textit{unknown} classes in closed-set based results. For instance, these unknown buildings with red roofs are misclassified as Bare S., Meadows, and other known materials by closed-set based classifier~\cite{FPGA}. Note that there is a significant overlap in the distribution of spectral curves between known and unknown classes in HSI datasets, which poses a major problem to open-set HSI classification.}
    \label{fig:open_set_example}
\end{figure}

Open-set classification (Fig.~\ref{fig:open_set_example}), as a critical task for safely deploying models in real-world scenarios, addresses the above problem by accurately classifying known class samples and rejecting unknown class outliers~\cite{OpenMax,MDL4OW,Fang_OpenSet}. Moreover, the recent advanced researches have explored training with an auxiliary unknown classes dataset to regularize the classifiers to produce lower confidence~\cite{Entropy,WOODS} or higher energies~\cite{Energy} on these unknown classes samples.

Despite its promise, there are some limitations when open-set classification meets HSI. First, the limited number of training samples, combined with significant spectral overlap between known and unknown classes (see Fig.~\ref{fig:open_set_example}), causes the classifier to overfit on the training samples. Second, the distribution of the auxiliary unknown classes dataset may not align well with the distribution of real-world unknown classes, potentially leading to the misclassification of the test-time data. Finally, it is labor-intensive to ensure the collected extra unknown classes dataset does not overlap with the known classes.

To mitigate these limitations, this paper leverages unlabeled ``in-the-wild'' hyperspectral data (referred to as ``wild data''), which can be collected \textit{freely} during deploying HSI classifiers in the open real-world environments, and has been largely neglected for open-set HSI classification purposes. Such data is abundant, has a better match to the test-time distribution than the collected auxiliary unknown classes dataset, and does not require any annotation workloads. Moreover, the information about unknown classes stored in the wild data can be leveraged to promote the rejection of unknown classes in the case of spectral overlap. While leveraging wild data naturally suits open-set HSI classification, it also poses a unique challenge: wild data is not pure and consists of both known and unknown classes. This challenge originates from the marginal distribution of wild data, which can be modeled by the Huber contamination model~\cite{Huber}:
\begin{equation}
    \mathbb{P}_{wild}=\pi\mathbb{P}_{k}+(1-\pi)\mathbb{P}_{u},
    \label{eq:huber_contamination_model}
\end{equation}
where $\mathbb{P}_{k}$ and $\mathbb{P}_{u}$ represent the distributions of known and unknown classes, respectively. Here, $\pi=\pi_{1}+\dots+\pi_{C}$, and $\pi_{c}$ refers to the probability (or class prior~\cite{DistPU}) of the known class $c \in [1,C]$ in $\mathbb{P}_{wild}$.
The known component of wild data acts as noise, potentially disrupting the training process (further analysis can be found in Section~\ref{sec:Methodology}). 

\begin{center}
    \fbox{\begin{minipage}{23em}
        This paper aims to propose a novel framework---\textit{HOpenCls}---to effectively leverage wild data for open-set HSI classification. Wild data is easily available as it's naturally generated during classifier deployment in real-world environments. This framework can be regarded as training open-set HSI classifiers in their \textit{living environments}.
    \end{minipage}}
\end{center}

To handle the lack of ``clean'' unknown classes datasets, the insight of this paper is to formulate a positive-unlabeled (PU) learning problem~\cite{DistPU,T-HOneCls} in the rejection of unknown classes: learning a binary classifier to classify positive (known) and negative (unknown) classes only from positive and unlabeled (wild) data. What's more, the high intra-class variance of positive class and the high class prior of positive class are potential factors that limit the ability of PU learning methods to address unknown class rejection task. To overcome these limitations, the multi-label strategy is introduced to the \textit{HOpenCls} to decouple the original unknown classes rejection task into multiple sub-PU learning tasks, where the $c$-th sub-PU learning task is responsible for classifying the known class $c$ against all other classes. Compared to the original unknown classes rejection task, each sub-PU learning task exhibits reduced intra-class variance and class prior in the positive class.

Beyond the mathematical reformulation, a key contribution of this paper is a novel PU learning method inspired by the abnormal gradient weights found in wild data. First of all, when the auxiliary unknown classes dataset is replaced by the wild data, this paper demonstrates that the adverse effects impeding the rejection of unknown classes originate from the larger gradient weights associated with the component of known classes in the wild data. Therefore, a gradient contraction (Grad-C) module is designed to reduce the gradient weights associated with all training wild data, and then, the gradient weights of wild unknown samples are recovered by the gradient expansion (Grad-E) module to enhance the fitting capability of the classifier. Compared to other PU learning methods~\cite{nnPU,DistPU,PUET,HOneCls}, the combination of Grad-C and Grad-E modules provides the capability to reject unknown classes in a class prior-free manner. Given the spectral overlap characteristics in HSI, estimating class priors for each known class is highly challenging~\cite{T-HOneCls}, and the class prior-free PU learning method is more suitable for open-set HSI classification.

Extensive experiments have been conducted to evaluate the proposed \textit{HOpenCls}. For thorough comparison, two groups of methods are compared: (1) trained with only $\mathbb{P}_{k}$ data, and (2) trained with both $\mathbb{P}_{k}$ data and an additional dataset. The experimental results demonstrate that the proposed framework substantially enhances the classifier's ability to reject unknown classes, leading to a marked improvement in open-set HSI classification performance. Taking the challenging WHU-Hi-HongHu dataset as an example, \textit{HOpenCls} boosts the overall accuracy in open-set classification (Open OA) by 8.20\% compared to the strongest baseline, with significantly improving the metric of unknown classes rejection (F1\textsuperscript{U}) by 38.91\%. The key contributions of this paper can be summarized as follows:
\begin{itemize}
    \item[1)] This paper proposes a novel framework, \textit{HOpenCls}, for open-set HSI classification, designed to effectively leverage wild data. To the best of our knowledge, this paper pioneers the exploration of PU learning for open-set HSI classification.
    \item[2)] The multi-PU head is designed to incorporate the multi-label strategy into \textit{HOpenCls}, decoupling the original unknown classes rejection task into multiple sub-PU learning tasks. As demonstrated in the experimental section, the multi-PU strategy is crucial for bridging PU learning with open-set HSI classification.
    \item[3)] The Grad-C and Grad-E modules, derived from the theoretical analysis of abnormal gradient weights, are proposed for the rejection of unknown classes. The combination of these modules forms a novel class prior-free PU learning method.
    \item[4)] Extensive comparisons and ablations are conducted across: (1) a diverse range of datasets, and (2) varying assumptions about the relationship between the auxiliary dataset distribution and the test-time distribution. The proposed \textit{HOpenCls} achieves state-of-the-art performance, demonstrating significant improvements over existing methods.
\end{itemize}
\section{Related Work}\label{sec:relatedwork}

\par
Different GPU simulators have been developed to explore and propose architectural changes to these architectures. Some of the most popular ones are single-thread simulators, such as Multi2Sim \cite{multi2sim} or GPGPU-Sim \cite{gpgpusimOriginal}. The former models the AMD Evergreen \cite{amdevergreen} architecture, while the latter models the NVIDIA Tesla \cite{teslaHotchips}. Recently, GPGPU-Sim was updated and renamed as the Accel-sim framework \cite{accelsim} to include some major features introduced in the NVIDIA Volta \cite{voltaPaper} architecture.

\par
Some previous works have developed parallel GPU simulators. The first one is Barra \cite{barra}, a GPU functional simulator focused on the NVIDIA Tesla architecture, which achieves a speed-up of 3.53x with 4 threads. However, this simulator models an old architecture and does not provide a timing model. Another work that models the NVIDIA Tesla architecture is GpuTejas \cite{gputejas}, which includes a timing model and achieves a mean speed-up of 17.33x with 64 threads. Unfortunately, executing GpuTejas in parallel has an indeterministic behavior, leading to accuracy simulation errors of up to 7.7\% compared to the single-threaded execution. One of the most successful parallel simulators is MGPUSim \cite{mgpusim}, an event-driven simulator that includes functional and timing simulation targeting the AMD GCN3 \cite{amdgcn3}. MGPUSim follows a conservative parallel simulation approach for parallelizing the different concurrent events during the simulation, preventing any deviation error from executing the simulator in parallel. It achieves a mean speed-up of 2.5x when executed with 4 threads.

\par
Several works have parallelized the GPGPU-Sim simulator. MAFIA \cite{mafia} can run different kernels concurrently in multiple threads but cannot parallelize single-kernel simulations. Lee et al. \cite{parallelGPUSim1} \cite{parallelGPUSim2} have proposed a simulator framework built on top of GPGPU-Sim. Their proposal needs at least three threads in order to run. Two threads are always dedicated to executing the Interconnect-Memory Subsystem and the Work Distribution and Control components. The rest of the threads are devoted to parallelizing the execution of the multiple SMs of the GPU. Lee et al. approach has an average 3\% simulation error compared to the original sequential simulation, achieving an average speed-up of 5x and up to 8.9x in some benchmarks.

\par
Some simulators, such as NVAS \cite{nvas}, address the highly time-consuming problem of simulations by reducing the detail of some components. For example, modeling the GPU on-chip interconnects in low detail in NVAS reports a 2.13x speed-up and less than 1\% benefit in mean absolute error compared to a high-fidelity model. Avalos et al. \cite{pcaKernelGpuSampling} rely on sampling techniques to simulate huge workloads.

\par
In contrast to previous works, we follow a simple approach to parallelize the Accel-sim framework simulator, the most modern academic GPU simulator used for research and capable of executing modern NVIDIA GPU architectures and workloads. Our proposal employs OpenMP \cite{openmp} to implement a scalable implementation that allows parallelizing the simulator with a user-defined number of threads. Moreover, our approach does not compromise the simulation accuracy and determinism when the simulator runs in parallel and provides the same results as the sequential version. Thus, it eases developing and debugging tasks. This makes our work more robust than the implementations of Lee et al. \cite{parallelGPUSim1} \cite{parallelGPUSim2}, and GpuTejas \cite{gputejas}, where the parallel version results differ from the single-threaded one. Moreover, our work is orthogonal to approaches such as the ones followed by NVAS \cite{nvas} and Avalos et al. \cite{pcaKernelGpuSampling}, which reduce the detail of some components and use sampling to speed up simulations even more.

% \section{The Rationale of DPTS}
\section{Method Rationale}

\label{sec:motivation}

% \ycj{Rationale?}
% \ycj{Pre-analysis/Motivation and Problem Definition/Motivation and Challenge Analysis/Pilot Study?}
 
% \ycj{Add a short overview statement here.}


{
In this section, we present empirical findings that highlight the key challenges of tree search in LLM and provide the rationale behind our proposed DPTS. 
% First, the inherent sequential nature of tree search complicates parallel execution, leading to irregular node expansions and varying path lengths. Second, excessive exploitation of low-confidence paths wastes computational resources, suggesting that a confidence-based pruning strategy could mitigate this inefficiency. Finally, tree search methods that prioritize breadth often suffer from frequent switching between paths, hindering deep exploitation and resulting in token and expansion redundancy. Addressing these challenges can significantly enhance the efficiency and effectiveness of tree search methods.
}
First, the frequent switch between paths complicates parallel execution and causes shallow thinking, disrupting the model’s ability to engage in efficient deep reasoning (Sec.~\ref{sec:3.1}). 
% sequential nature of tree search complicates parallel execution, leading to irregular node expansions and varying path lengths. Existing tree search algorithms frequently switch between paths, causing shallow thinking and disrupting the model’s ability to engage in deep reasoning, ultimately degrading generation quality. 
Second, excessive exploitation of low-confidence paths results in redundant rollouts and wastes effort on fewer possible candidates (Sec.~\ref{sec:3.2}). 
% consuming unnecessary computational resources. Without an effective pruning strategy,..., reducing overall efficiency and slowing down search convergence.

% \subsection{Unpredictable Growth Behavior}
\subsection{Frequent Switching}
\label{sec:3.1}

\begin{figure}[t]
    \centering
    \includegraphics[width=0.93\linewidth]
    {figs/draw_switch_times.pdf}
    \vspace{-0.1in}
    \caption{Statistics for switch from the best path to the suboptimal (blue), and total switch (green).} 
    \vspace{-0.1in}
    \label{fig:motivation_switch_path}
\end{figure}
% \ycj{Analysis of why vanilla solution doesn't work well?}

% Tree search has become one of the key paradigms for enabling deep reasoning in large language models. However, the inherent sequential nature of tree structures presents significant challenges for GPU parallelism. Regardless of the search algorithm employed, the hierarchical and distributed nature of tree-structured data introduces intrinsic difficulties in memory management—whether using a contiguous memory pool or fragmented memory blocks, the irregularity of tree search remains unavoidable.  % 这一段说的 tree-structured data 难以 parallel

Tree search inherently exhibits retrospective and recursive behaviors, making efficient parallel execution difficult. Even if each node is constrained to generate the same number of tokens, the focus switching between different reasoning trajectories and the diverse path lengths makes it incompatible with the end-to-end parallelism on GPUs. The detailed illustrations for this phenomenon can be found in Appendix~\ref{sec:app:trees}. 

% This behavior demonstrates the fundamental characteristics of tree search that make the efficient parallelization a challenging problem: \textit{diverse path lengths, varying child depths, irregular node jumps, and recursive retrospective exploitation}. 

The focus switching between paths also makes the tree search fail in focused reasoning trajectory~\cite{wang2025thoughts}, which prevents deep thinking and leads to a tendency of shallow exploitation. We quantify the switch times of the reasoning focus on each sample in the Math500 dataset. Figure~\ref{fig:motivation_switch_path} counts the total switch, which is about 35 on average. As well as the switch from the best path to a suboptimal or incorrect one, which is up to 3 times for a single sample. It demonstrates the instability of the tree search algorithm in maintaining a focused reasoning trajectory. 


% 下面这段我放 appendix 了
\jwt{
% Tree search is key for enabling deep reasoning in large language models, but its sequential nature presents challenges for efficient GPU parallelism. Tree-structured data introduces difficulties in memory management, with irregular patterns in node expansions and varying path lengths. This leads to difficulties in maintaining parallelism, as tree search is inherently recursive and retrospective, causing inefficient execution when different paths vary in depth and termination points.

% Figure~\ref{fig:motivation_dfs_trees} visualizes depth-first search (DFS) trees, highlighting the irregular expansion process. Darker nodes represent high-confidence paths, while lighter nodes indicate lower-confidence ones. The figure demonstrates that tree search does not follow a predictable spatial or hierarchical pattern, resulting in diverse path lengths, varying child depths, and irregular node jumps. These behaviors complicate parallel execution and efficient resource utilization.

% To address these challenges, it is crucial to focus on improving the handling of irregular growth in tree structures. By introducing strategies that can better manage the dynamic nature of tree search, we can optimize memory usage and reduce the inherent complexity, enabling more effective parallelization.

}


% These findings highlight a critical shortcoming of BFS: frequent switching between different reasoning paths prevents deep exploitation, leading to significant computational redundancy. This behavior results in excessive token generation, unnecessary expansions on suboptimal paths, and instability in maintaining a coherent line of reasoning—factors that ultimately hinder search efficiency and inference speed.

% 下面这段我放 appendix 了
% \jwt{
%  Tree search focusing on width expansion explores a wide range of paths but suffers from frequent switching between them, preventing deep reasoning and leading to shallow exploitation. This behavior causes two inefficiencies: incomplete reasoning and excessive expansions. These algorithms often generates more tokens and expansions than necessary, exploring many suboptimal paths before finding the best one, which results in significant computational redundancy.

%  Figure~\ref{fig:motivation_bfs_trees} shows how BFS expands in a flat, top-down manner, leading to shallow exploitation. Figure~\ref{fig:motivation_waste_tokens}(left) compares the total tokens generated (blue line) to those required for the best path (yellow line), revealing excessive token redundancy. Figure~\ref{fig:motivation_waste_tokens}(right) highlights unnecessary node expansions, where many explored nodes do not contribute to the final solution. Finally, Figure~\ref{fig:motivation_switch_path} analyzes node-switching frequency, showing that BFS frequently shifts between optimal and suboptimal paths, leading to instability in reasoning.

%  The high frequency of path-switching can be mitigated by introducing mechanisms that help the model maintain focus on the most promising paths.
% }



\subsection{Redundant Exploration}
\label{sec:3.2}


The lack of early termination in existing tree search algorithms leads to excessive exploitation and redundant searching. Observations in Figure~\ref{fig:motivation_low_confidence} show that low-confidence nodes rarely contribute to the best solutions, either terminated with suboptimal results (yellow) or failing to be the first to reach the best path (orange). The average probability of the suboptimal results brought by low confidence is 91.3\%, while the probability of those nodes being the earliest best path is only 6.2\%. It suggests that low-confidence nodes have little potential to reach the best solution, it is even hard to be the first one. It means that most low-confidence nodes have less contribution to the final results but waste computational resources. 
% We reorder the nodes based on the former probability for clearer illustration (the plot with original order is showcased in Appendix~\ref{app:sec:low_reward_original}). 

% One of the fundamental inefficiencies in traditional tree search lies in its inability to terminate early when exploring suboptimal paths. In most cases, a search path is only abandoned when it reaches the termination condition, regardless of whether it is already apparent that the path is unlikely to yield an optimal solution. This behavior can lead to substantial computational redundancy, as a large number of unnecessary expansions and token generations are performed on low-confidence nodes. 


% To quantify the extent of this redundancy, we conducted an observational experiment. In Figure~\ref{fig:motivation_low_confidence}, we analyze the probability that continuing the rollout from a low-confidence node leads to less contribution. Here, we define a node as low confidence if its score is lower than the average confidence of previously visited nodes (refer to Eq. (\ref{eq:theta}))—a deliberately aggressive threshold since such occurrences are relatively frequent (highlighted in yellow). And we also reorder the nodes based this probability for clearer illustration (the plot with original order is showcased in Appendix~\ref{app:sec:low_reward_original}). However, as our observations indicate, the probability that these low-confidence nodes ultimately contribute to the optimal path is extremely low (highlighted in orange). Additionally, in most cases where the optimal path is eventually reached, it is not the first time an optimal solution is discovered (highlighted in blue), meaning that a higher-confidence node had already identified the correct path earlier. 

% Between the yellow and orange regions lies an additional scenario: the probability that continuing a rollout from a low-confidence node either fails to generate an answer at termination or produces an incorrect answer. This category accounts for the majority of cases, further reinforcing the inefficiency of expanding low-confidence paths.

% Based on these observations, it suggest that a node’s prior confidence may serve as a reasonable predictor of whether the search path will ultimately lead to a valid solution. While this does not guarantee a perfect pruning strategy, it indicates that integrating confidence-based heuristics could significantly reduce unnecessary rollouts, improving the overall efficiency of tree search methods.

% 合并到上面的黑色文本里了
\jwt{
% A major inefficiency in traditional tree search is the lack of early termination when exploring low-confidence paths. This results in wasted computational resources, as the algorithm continues expanding these paths even when their confidence scores remain low. Observations show that low-confidence nodes rarely contribute to optimal solutions, suggesting that confidence-based pruning can reduce unnecessary exploitation and improve efficiency.

% Figure~\ref{fig:motivation_dfs_trees} (Tree 2) illustrates the issue, showing that low-confidence nodes (rightmost branches) are expanded despite their low likelihood of contributing to the final answer. Figure~\ref{fig:motivation_low_confidence} further quantifies this inefficiency: yellow regions indicate frequent occurrences of low-confidence nodes, while orange regions show that they rarely lead to optimal solutions. The blue regions reveal that even when the best path is found, a higher-confidence node had typically identified it earlier, confirming the redundancy of expanding low-confidence nodes.

% These findings emphasize the importance of selectively pruning low-confidence paths early in the search process. By incorporating mechanisms that assess the likelihood of a node contributing to the optimal solution, unnecessary expansions can be avoided, leading to a significant reduction in computational overhead.
}

\begin{figure}[t]
    % \centering
    \includegraphics[width=0.85\linewidth]{figs/low_reward_original.pdf}
    \vspace{-0.1in}
    \caption{Probabilities with reordered samples of those have prior confidence below $\theta_{es}(\lambda=1)$ in Eq.~\ref{eq:theta} and do not terminate with the highest reward score (yellow), and paths that are not the earliest best path (orange), which means there is already at least one path that has terminated with the same reward score. }
    \vspace{-0.2in}
    \label{fig:motivation_low_confidence}
\end{figure}

These findings emphasize the importance of maintaining the focus on deep reasoning and pruning low-confidence paths for efficient inference. 


%%%%%%%%%%%%%%%%%%%%%%%%%%%%%%%%%%%%%%%%%%%%%%%%%%%
% Model
%%%%%%%%%%%%%%%%%%%%%%%%%%%%%%%%%%%%%%%%%%%%%%%%%%%
\section{\Method Design}
\label{sec:model}


%The collected QA dataset enables a concrete benchmark for the QA problem in daily-life activities monitoring. 
%Analyzing the dataset reveals unique challenges, such as handling heterogeneous question types and variable time spans. 
%In this section, we provide the details of \Method. 
%\Method is the first end-to-end QA system that enables natural QA interactions with users. 
%We envision \Method as a next-generation mobile system that assists in person's everyday tasks. \Method uses the most intuitive communication of natural language to ``chat'' with users, thus seamlessly embedding into users' lives.
%

%\Method advances SOTA systems from the following perspectives. \textbf{First,} \Method handles a wide range of natural language questions from existence to time queries, and provides clear, understandable answers through LLMs. This addresses the challenges in prior systems~\cite{xing2021deepsqa,nie2022conversational,englhardt2024classification} which restricted users to predefined question or answer sets.
%\textbf{Second,}  \Method can answer questions based on long durations of fine-grained timeseries sensor data, such as counting the total exercising time in a week, by integrating a sensor query stage. This differs from existing systems that use low-dimensional sensor inputs (e.g., daily step counts) or fixed windows~\cite{englhardt2024classification,kim2024health,yang2024drhouse,moon2023anymal,han2024onellm,moon-etal-2023-imu2clip}. %To do so, \Method designs an intermediate stage for accurate and efficient encoding and query of sensor data.
%\textbf{Third}, \Method generates accurate answers to quantitative questions where SOTA systems struggle with~\cite{xing2021deepsqa,zhang2023llama,moon-etal-2023-imu2clip,moon2023anymal}. This is achieved via a carefully designed pipeline that integrates LLMs and the sensor data query stage.
%\textbf{Last but not least,} \Method is optimized for edge systems, delivering real-time responses on the edge desktop. 
%\Method is the first real system that enables natural and real-time interactions with multimodal sensors on edge devices.






%%%%% Overall steps, why do you design in this way
%\vspace{-2mm}
\subsection{Overview of \Method} 

%This is critical for \Method to handle quantitative questions.
%, which process natural language, with a sensor data query stage, which processes raw sensor signals. 
%The integration must leverage the strengths of each module while mitigating their weaknesses. Specifically, LLMs excel at understanding diverse questions but can generate hallucinations that distort quantitative results. On the other hand, the sensor query stage provides accurate quantitative data but may struggle with unexpected queries. Additionally, the complexity of the query search must be carefully managed to ensure real-time responses.
%The challenge of creating a pipeline where these modules operate accurately and robustly in real-world scenarios remains unresolved.
%
To address the challenge of accurate QA over long-duration sensor data, we design a novel three-stage pipeline in \Method, consisting of \textit{question decomposition, sensor data query}, and \textit{answer assembly}.
Specifically, we use LLMs in question decomposition and answer assembly, recognizing their essential roles in correctly interpreting user queries and generating natural language answers.
The intermediate sensor query stage is the \textit{key component} of \Method.
First, the query stage uses a pretrained sensor encoder to effectively encode high-dimensional and long-duration multimodal sensor timeseries into meaningful embeddings. Second, the query stage performs a similarity search in the embedding space to retrieve all sensor information relevant to the original question. This ensures precise extraction of sensor context, making a significant contribution to the accuracy of the final answer.
To the best of the authors' knowledge, \Method is the first system to incorporate an explicit sensor data query stage for accurately handling sensor-based language tasks in long-term monitoring.
%We discuss the potential to further improve \Method with vector databases~\cite{zhou2024llm} or retrival augmented generation~\cite{zhao2024retrieval} in Sec.~\ref{sec:future-work}.
%We recognize the potential to further improve \Method with vector databases~\cite{zhou2024llm} or retrival augmented generation~\cite{zhao2024retrieval}, which we leave for future investigation and discuss 

%Each stage and its interfaces are carefully designed to leverage the strengths of each module while mitigating their weaknesses. 
%The \textbf{key technical challenge} for \Method lies in creating a pipeline where these modules operate \textit{accurately, robustly and efficiently} in real-world scenarios, a problem that remains unsolved before \Method.
%, such as the hallucinations in LLMs.
%The key innovation is to use the LLM as both the front- and back-end for processing questions and answers, with the sensor data query stage in the middle to ensure accurate sensor information extraction. E
%The key technical challenge for \Method lies in \textit{effectively integrating LLM with accurate sensory queries}, so that the whole pipeline works accurately and robustly in real-world scenarios.


The complete pipeline of \Method is illustrated in Fig.~\ref{fig:overview}. For example, given a question like, “\textit{How long did I exercise last week in the morning?}” (\textcircled{1} in Fig.~\ref{fig:overview}), \Method first decomposes the question and generates specific sensor data queries using LLMs. These queries include details such as the context of interest, date and time ranges, and the summarization function (\textcircled{3}). In this case, the query might specify a context of “\textit{do exercise},” a time span of “\textit{last week},” a time of day of “\textit{morning},” and a summarization function of \texttt{CalculateDuration}. To enhance the precision of question decomposition, \Method uses solution templates (\textcircled{2}) with carefully designed prompts. Next, the sensor query stage encodes the activity text “\textit{exercise}” into the embedding space (\textcircled{4}) and retrieves sensor embeddings that are sufficiently similar to the text embedding (\textcircled{5}). These sensor embeddings are encoded offline from the full-history raw sensor data. Both the sensor and text encoders are pretrained offline to align their outputs in the same embedding space, ensuring accurate sensor data retrieval.
Therefore, only text encoding and similarity-based query searches are performed online.
Additional properties, such as the date (“\textit{last week}”) and time of day (“\textit{morning}”), are used to constrain the sensor query range. The final step of the sensor data query stage involves summarizing the relevant sensor embeddings into a textual context with a summarization function (\textcircled{6}). For instance, with the \texttt{CalculateDuration} function, the summarization step calculates the total duration of the retrieved sensor embeddings. This results in a sensor context such as: “\textit{Among all days last week, you exercised for 35 minutes on Monday morning and 55 minutes on Thursday morning.}” The summarization function is identified during question decomposition. Finally, the original question and the sensor context are fed into the answer assembly stage, where a fine-tuned LLM generates the final answer (\textcircled{7}). With the precise sensor context, the model produces an accurate response, such as: “\textit{You exercised a total of 1 hour and 30 minutes last week.}” (\textcircled{8}).

\begin{figure*}[tb]
  \centering
  \includegraphics[width=0.95\textwidth]{figs/overview3.png}
  \vspace{-3mm}
  \caption{The system diagram of \Method including three stages.}
  \vspace{-4mm}
  \label{fig:overview}
\end{figure*}
%\textbf{First,} with the goal of handling heterogeneous questions and answers in the real world, \Method integrates LLMs in the first stage of question decomposition and the third stage of answer assembly. This design leverages the extensive knowledge of pretrained/finetuned LLMs to ensure accurate and natural interactions with users.
%\textbf{Second,} to achieve an exhaustive search across the entire sensor lifespan, \Method develops a sensor data query mechanism to improve answer accuracy.
%The sensor information (in text) and natural language questions are fused in the third stage.
%Without introducing a new modality, \Method avoids the intensive training required by multimodal LLMs, making it a practical solution for real-world applications.

%Inspired by the limitations of prior designs as discussed in Sec.~\ref{sec:analysis}, \Method focuses on handling the diverse interactions in the real world, especially the diverse scenarios and long sensor time spans.
%In \Method, we design a novel framework that fuses LLMs with sensor database query to leverage the strengths of both.
%\Method addresses the limited question and answer types in prior QA systems~\cite{xing2021deepsqa,nie2022ai} as well as the fixed sensor window in existing multimodal LLM works~\cite{moon2023anymal,han2024onellm}.


%leverages the exceptional reasoning capability of pretrained LLMs to handle arbitrary questions, while integrating with a carefully designed a sensor query database to effectively extract relevant multimodal sensor data in the full history.

%In this section, we first give an overview of \Method and then provide more details about its key designs.


%In \Method, we propose a novel framework to address two key limitations in previous designs: (i) heterogeneous question and answer types and (ii) variable time scales of the queries.
%These capabilities are crucial for real-world sensor applications but have not been sufficiently addressed in existing studies~\cite{xing2021deepsqa,han2024onellm,moon2023anymal}.


%The framework also needs to integrate with accurate sensor data queries to ensure correct answer generation, rather than relying on the LLM's potentially inaccurate predictions.
%\Method is the first design to effectively fuse sensor data with natural language. Although recent works in multimodal LLM~\cite{han2024onellm,moon2023anymal} provide a path for integrating data from other modalities into LLMs, they require vast amounts of multimodal data (e.g., over 1M samples) for finetuning. 
%\Dataset is insufficient to obtain satisfactory finetuning results, as shown in the previous section.

 
%Unfortunately, \Dataset is the only dataset currently available for our natural QA task with sensors, making multimodal LLM infeasible in our scenario.


%To address the limitations of (1) heterogeneous question and answer types and (2) fixed sensor scales, our approach leverages the extensive knowledge of pretrained LLMs to handle diverse questions while designing a sensor data query mechanism to adaptively search the entire database. 



%We use \Dataset to finetune the LLM in the last stage thus the model adapts to our sensor-specific task and is able to generate answers in the desired format.

%In \Method, all three stages work together to produce accurate, high-quality answers. The sensor query stage in the middle stage effectively integrates sensor data with natural language: its query search is guided by the first stage's decomposition results, while its extracted information (in text) is fused with the original question in the last stage, both powered by pretrained LLMs. This approach avoids the data-intensive training required by multimodal LLMs, making \Method a practical solution for real-world applications. 


In the following lines, we describe the key design elements of \Method: the offline contrastive sensor-text encoder pretraining (Sec.~\ref{sec:pretraining}) and the online three-stage pipeline (Sec.~\ref{sec:three-stage}).

%that enable accurate answers from long-duration raw sensor timeseries.
%We first detail the offline contrastive sensor-text pretraining in \Method (Sec.~\ref{sec:pretraining}), which is critical for training well-aligned sensor and text encoders and ensuring precise sensor data queries. To achieve this, we introduce a novel contrastive pretraining loss tailored for partial contexts. 
%Next, we detail the online three-stage pipeline of \Method in Sec.~\ref{sec:three-stage}, including how LLMs are utilized for question decomposition and answer assembly, and how the sensor data query stage enables effective fusion of sensor and text information through similarity-based searches.




\subsection{Contrastive Sensor-Text Pretraining for Partial Contexts}
\label{sec:pretraining}

The sensor and text encoders are crucial for effective sensor-text fusion, serving as a "bridge" between high-dimensional sensor timeseries and semantic text. A high-quality sensor encoder ensures that \Method can accurately and comprehensively retrieve relevant sensor embeddings during the sensor data query stage. Similarly, a robust text encoder is essential for handling the arbitrary contextual text generated during the question decomposition stage.

However, achieving this level of alignment poses significant challenges due to the lack of suitable techniques. Existing pretraining methods, such as CLIP and its variants~\cite{radford2021learning,moon-etal-2023-imu2clip}, are effective for pretraining encoders with paired inputs, i.e., one piece of sensor data and one sentence. Unfortunately, encoders trained in this manner struggle to accurately identify similar sensor embeddings when provided with partial or arbitrary context instead of complete sentences.
For instance, using CLIP, a piece of sensor data might align well with the full sentence "The person is sitting and working on computers at school." However, when a user is specifically interested in a partial context like "working on computers," the encoded text embedding may not closely match the original sensor embeddings, leading to reduced accuracy in sensor data query as we show in Sec.~\ref{sec:ablation}. To address this challenge, we introduce a novel contrastive sensor-text pretraining loss for partial contexts.



%tailored for multi-label contexts. 
We pretrain our model on a large-scale multimodal sensor dataset with annotations, where each sample $\{\mathbf{x}_t, w_t\}$ consists of raw time series sensor data $\mathbf{x}_t$ collected at time $t$ and an associated set of partial context labels $w_t$. $w_t$ can be extracted from label annotations. For instance, in a single-label human activity classification dataset,  $w_t$ may contain a single phrase (e.g., $\{\textrm{"standing"}\}$), whereas in a multi-label dataset, it may include multiple phrases (e.g., $\{ \textrm{"at school", "working on computers"}\}$).
We employ separate encoders for sensor and text inputs. 
Formally, let $\theta$ denote the complete sensor encoder model. The encoded sensor embedding is given by $\mathbf{z}^s_t = \theta(\mathbf{x}_t)$.
The text encoder, denoted by $\phi$, maps arbitrary phrases to a text embedding $\mathbf{z}^w_t = \phi(w_t)$.
Key notations used throughout our work are summarized in Table~\ref{tbl:notation}.
%
The details of the sensor encoder are shown in Fig.~\ref{fig:sensor_encoder}. We use distinct sensor encoder for each sensor modality (e.g., IMUs, audio and phone status) to accommodate the varying complexity of different sensor data types. For instance, a Transformer-based encoder is used for high-dimensional time series data while a simple linear layer is designed for encoding phone status. The outputs from these modality-specific encoders are concatenated and passed through a fusion layer to generate the final sensor embedding. Our framework allows for missing modalities by padding with mean values and is flexible for future expansion to additional modalities.

\begin{figure}[t]
\begin{center}
\includegraphics[width=0.75\textwidth]{figs/sensor_encoder.png} 
\vspace{-4mm}
\caption{Visualization of the contrastive sensor-text pretraining in \Method.}
\label{fig:sensor_encoder}
\end{center}
\vspace{-6mm}
\end{figure}


%which effectively aligns sensor embeddings with the semantic information in labels~\cite{zhang2023navigating}. 
%Important notations are listed in Table~\ref{tbl:notation}.
%We integrate the timeseries collected from $d$ multimodal sensors into a single sample $\mathbf{x}_t \in \mathbb{R}^{d \times \tau}$, where $t$ denotes the timestamp and $\tau$ denotes the time window length. 
%In the preprocessing stage, we normalize data based on complete readings from the same sensor.
%Missing modalities or readings are padded with zeros.
%The sensor encoder $\theta$ encodes $\mathbf{x}_t$, while the label encoder $\phi$ encodes text labels $w_c$, such as "at school". We then compute the logits by taking the dot product between the sensor and label embeddings:
%$g(\mathbf{x}_t; \theta, \phi) = \sigma \left ( \left [ f(\mathbf{x}_t;\theta) \circ f(w_c;\phi) \right ]_{w_c \in \mathcal{W}} \right )$,
%where $\sigma$ is a sigmoid function to obtain the predicted probabilities.
%
We introduce a new pretraining loss to align sensor embeddings $\mathbf{z}^s_t$ and partial text embeddings $\mathbf{z}^w_t$. Different from CLIP~\cite{radford2021learning}, our loss function enables alignment between sensor data and all partial phrases, defined as follows:
\begin{equation}
 \mathcal{L} = \sum_t {\frac{-1}{|w_t|} \sum_{w \in w_t} \log \frac{\exp(\mathbf{z}^s_t \cdot \mathbf{z}^w_t / \tau)}{\sum_{a \in A(w)} \exp (\mathbf{z}^s_t \cdot \mathbf{z}^a / \tau)}}, \label{eq:loss} 
\end{equation}
$|w_t|$ is the cardinality of set $w_t$, $\tau$ is a scalar temperature parameter. The term $A(w) \equiv A \setminus \{w\}$ represents the collection of negative contexts, defined as all possible phrases except the positive phrase $w$.
The intuition behind our loss function is illustrated in Fig.~\ref{fig:sensor_encoder}. Consider a sensor embedding $\mathbf{x}_t$ and a set of partial text labels $w_t = \{ \textrm{sitting, working on computers}\}$.
Our loss function encourages high similarity between the sensor embedding $\mathbf{z}^s_t$ and all positive text embeddings corresponding to $w_t$, namely "sitting" and "working on computers", while distinguishing them from other negative contexts such as "walking" or "at school". We draw inspiration from the supervised contrast learning loss~\cite{khosla2020supervised}. However, our loss function differs in that it explicitly models the similarity between sensor embeddings and multiple text phrases rather than between samples of the same modality.
%During encoder training, we minimize binary cross-entropy loss using logits $g(\mathbf{x}_t; \theta, \phi)$ and ground truth multi-class labels $y_t$. 
%In the final deployment of \Method, the pretrained encoders are fixed, and new sensor data are encoded into logits.
%The logits is a vector with the same size as $y_t$, indicating the probability of each activity.
After pretraining, we store all sensor embeddings in a database for online queries.
%We further optimize the efficiency in Sec.~\ref{sec:edge}.

\iffalse
\begin{wraptable}{r}{0.44\textwidth}
\footnotesize
\vspace{-5mm}
\caption{List of important notations.}
\label{tbl:notation}
\vspace{-5mm}
\begin{center}
\hspace{-2mm}\begin{tabular}{p{3em} p{22em}} 
 \toprule
 \hspace{-2mm}Symbol & Meaning \\
 \midrule
 %$(q^i, a^i)$ & The $i$th question-answer pair \\
 %$\mathbf{x}^i$ & The full-history sensor time series data w.r.t. $i$th QA pair \\
 $d$ & Number of different sensors \\
 %$T^i$ & Total duration of sensor data $\mathbf{x}^i$ \\
 $\tau$ & Window length of each sensor sample \\
 $\{\mathbf{x}_t, y_t\}$ & The sensor time series data and context labels at time $t$ \\
 %$c^i$ & The extracted sensor context w.r.t. $i$th QA pair \\
 %$T$ & Time window length of the raw timeseries sensor data \\
 $\theta, \phi$ & Pretrained sensor and label encoders \\
 $h$ & Threshold in query search \\
 %$\mathbf{t}$ & Global timestamps of sensor data \\
 %$N$ & Total number of QA pairs in our \Method dataset \\
 %$l, m$ & Maximum length of tokenized questions and answers \\
 %$\theta$ & Autoregressive Large Language Model to finetuning \\
 %$q$ & Number of provided templates during question decomposition \\
 %$k$ & Number of generated variants during dataset augmentation \\
 %$lr$ & Learning rate of during finetuning \\
 %$\tau$ & Temperature for LLM generation \\
 %$\phi$ & Pretrained transformer encoder for encoding raw sensor data \\
 %$b$ & The dimension of sensor embeddings \\
 \bottomrule
\end{tabular}
\end{center}
\vspace{-4mm}
\end{wraptable}
\fi

\begin{table}[t]
\small
\caption{List of important notations.}
\label{tbl:notation}
\vspace{-4mm}
\begin{center}
\begin{tabular}{p{3em} p{21em} | p{3em} p{21em} } 
\toprule
Symbol & Meaning & Symbol & Meaning \\
\midrule
 $\mathbf{x}_t$ & Multimodal sensor data sampled at $t$ &
 $w_t$ & Partial context labels at $t$ \\
 $d$ & Number of different sensors &
 $T$ & Total duration of sensor data \\
 $\theta$ & Sensor encoder &
 $\phi$ & Text encoder \\
 $\mathbf{z}^s_t$ & Sensor embedding of $\mathbf{x}_t$ &
 $\mathbf{z}^w_t$ & Text embedding of $w_t$ \\
 $A(w)$ & Set of negative contexts of $w$ &
 $\tau$ & Temperature scalar \\
 $f$ & Trained similarity function between embeddings & 
 $h$ & Threshold in query search \\
\bottomrule
\end{tabular}
\end{center}
\vspace{-4mm}
\end{table}


\subsection{Three Stages in \Method}
\label{sec:three-stage}

In this section, we explain the detailed designs in each stage of \Method, including question decomposition, sensor data query and answer assembly.
All stages need to work accurately, robustly and collaboratively to achieve natural and precise answer generation in the end.

%\Method uses LLMs in the question decomposition and answer assembly stages to interact freely with users using natural language.
%Our major goal is to utilize the extensive knowledge in LLMs for comprehension and summarizing with minimal hallucinations.
%The first and third stages of \Method use LLMs to to interact freely with users using natural language.
%correctly understand diverse types of questions and construct accurate answers. These stages allow \Method 
%More specifically, as shown in Fig.~\ref{fig:overview}, the question decomposition stage uses GPT~\cite{gpt-4} or LLaMA~\cite{touvron2023llama}, treating the decomposition as a zero-shot reasoning task due to the lack of finetuning data.
%The answer assembly stage uses a custom LLaMA~\cite{touvron2023llama} from fine-tuning on \Dataset. The finetuning process helps \Method produce answers in a desired format. 
%We introduce more details in the following.

%Due to the lack of data to this question decomposition, we formulate the task as zero-shot generation, enhanced by in-context learning and chain-of-thought reasoning to improve solution quality.
%As detailed in the left portion of Fig.~\ref{fig:overview}, we utilize both closed-source LLMs, such as GPT~\cite{gpt-4}, and open-source LLMs, like Llama~\cite{touvron2023llama}. 
\iffalse
\begin{figure}[t]
\begin{center}
\begin{tabular}{c}
    \includegraphics[width=0.95\textwidth]{figs/solution_template.png} \\
    {\small (a) Example solution template for ICL. \vspace{1mm}} \\
    
    \includegraphics[width=0.95\textwidth]{figs/question_decompose.png} \\
    {\small (b) The prompt design for the question decomposition stage includes ICL solution templates (in blue) and bolded text for CoT. \vspace{1mm}}  \\

    \includegraphics[width=0.5\textwidth]{figs/finetuning.png} \\
    {\small (c) Prompt design in the answer assembly stage for LLM finetuning.} \\
\end{tabular}
\vspace{-4mm}
\caption{Prompts design in \Method to better leverage the power of LLMs.}
\label{fig:prompts}
\end{center}
\vspace{-6mm}
\end{figure}
\fi


\subsubsection{Question Decomposition}
\label{sec:decomposition}

%Explicit question decomposition has not been utilized in existing sensor-based QA systems, which either used a limited question set~\cite{xing2021deepsqa,nie2022conversational,englhardt2024classification} or relied on a single LLM for end-to-end problem solving~\cite{englhardt2024classification,kim2024health,yang2024drhouse,zhang2023llama,moon-etal-2023-imu2clip,moon2023anymal}.
%In contrast, \Method integrates explicit question decomposition to guide accurate sensor data queries, thus empowering \Method to handle quantitative questions.

%Question decomposition stage is crucial for accurately interpreting various question types and determining the corresponding sensor query requests. 
The question decomposition stage processes the user question and identifies specific query triggers for the sensor data query stage, such as the context of interest, date and time ranges, and the summarization function, as illustrated in the leftmost box in Fig.~\ref{fig:overview}.
The primary challenge in designing this stage lies in generating \textit{accurate} decompositions for \textit{arbitrary} user inputs, including various questions types as discussed in Sec.~\ref{sec:motivation}.
We choose to rely on LLMs for this stage due to their outstanding capability in handling natural language inputs.
However, LLMs are not without limitations as they can produce erroneous outputs or hallucinations~\cite{huang2023survey}, especially in our case where a dedicated dataset for similar decomposition tasks is unavailable.


%%%Difference from SOTA, what makes this challenging


%However, a significant challenge is the lack of a dedicated dataset for this decomposition. Without such a dataset and finetuning, it is difficult to ensure that LLMs generate desired decompositions with minimal errors due to hallucinations.
%
%
%The \textit{primary challenge} in question decomposition is handling arbitrary user questions without a dedicated dataset. Existing datasets like DeepSQA~\cite{xing2021deepsqa} and SensorQA~\cite{} focus on end-to-end answers, not decomposition. While GPT models excel in general reasoning~\cite{gpt-4}, guiding them to generate precise query arguments in our specific format without pre-existing data is difficult. 
To address this challenge, we propose a \textit{few-shot learning} approach to prompt pre-trained LLMs, enhanced with in-context learning~\cite{alayrac2022flamingo,shao2023prompting} and chain of thought techniques~\cite{chuCoTReasoningSurvey2024,lu2022learn}.
An example prompt is illustrated in Fig.~\ref{fig:question-decompose}, where LLM is instructed to mark different arguments with distinct symbols, such as "<<" and ">>" for function names. This helps in accurately extracting various arguments from the LLM output. 
In \Method, we accommodate a variety of real-life scenarios by extracting contexts (such as activities), dates, times of day, and summarization functions during the query process, as shown in Fig.~\ref{fig:question-decompose}.
\Method supports multiple extracted phrases to handle complex questions such as "How long did I work at school on Monday and Tuesday?"
\Method is designed to be flexible, allowing for future expansion with additional decomposition terms.

We next explain more details on the in-context learning and chain-of-thought techniques used in \Method, both for ensuring high-quality question decompositions.



%We explain more details about each technique next.
%In in-context learning, we adaptively select solution templates and feed them to GPTs along with the question to decompose.
%The CoT design requires LLMs to perform detailed reasoning 
%We also introduce in-context learning and chain-of-thought prompting to improve zero-shot reasoning and avoid too ``creative'' generations.
%In this stage, we introduce in-context learning and chain-of-thought prompting to enhance the effectiveness of zero-shot reasoning, as detailed below.
%Due to the lack of data for this specific task, we opt for prompting GPTs in a zero-shot manner. 




\textbf{In-Context Learning (ICL)} integrates a few examples directly into the prompt during inference, enabling LLMs to adapt effectively to specific tasks without requiring fine-tuning~\cite{alayrac2022flamingo,shao2023prompting}. 
Building on this insight, we design a library of general solution templates covering diverse QA scenarios and incorporate them as in-context examples to improve decomposition accuracy (Fig.~\ref{fig:question-decompose}).
We observe that decomposition solutions for the same question type often share similarities. For instance, "how long" questions typically map to the \texttt{CalculateDuration} function. To utilize this question-specific property, in \Method, we first classify the question type using a BERT model~\cite{devlin2018bert}. Depending on the question type, we dynamically select templates that best match the question category, a strategy shown to enhance ICL performance~\cite{fu2023gpt4aigchip}.
Fig.~\ref{fig:solution-template} illustrates an example of a solution template for a time query. Including an explanation in the solution template is particularly helpful for complex reasoning tasks. %, as detailed with chain of thought.
%For example, a solution to ``\textit{How long did I work yesterday?}'' could be ``\textit{Query the database with the function <<CalculateDuration>> using the activity ((in the main workplace)), and the date [[yesterday]]}''.
%During decomposition, we adaptively select the two templates that solve the most similar questions, which has been shown to improve ICL performance~\cite{fu2023gpt4aigchip}. Using just two templates helps balance prompt length and performance.
%To ensure accurate decomposition across various question types, we create distinct templates tailored to each category.
%For example, the template in Fig.\ref{fig} (b) is used for time queries, while another template addresses action queries like ``What did I do on Tuesday?''
%This adaptive template selection offers specialized guidance based on the question category, thereby enhancing decomposition accuracy. 
%In \Method, we provide 2-3 templates for each question category listed in Table~\ref{tab:question_profile}.



%\textbf{In-context learning}
%includes task examples directly into the prompt during inference, which has shown superb performances in adapting LLM outputs to specific tasks without finetuning~\cite{alayrac2022flamingo,shao2023prompting}. Based on this insight, we design solution templates as in-context examples, which are embedded into prompts as illustrated in Fig.~\ref{fig:prompts} (a).
%An example of such a solution template is shown in Fig.\ref{fig:prompts} (b).
%For each question, we provide a corresponding solution and a detailed explanation. To ensure accurate decomposition across various question types, we create distinct templates tailored to different question categories.
%For instance, the template example in Fig.~\ref{fig:prompts} (b) is used for time queries, while a separate template with ``What did I do on Tuesday?'' is used for action queries. 
%This adaptive template selection offers specialized guidance based on the question category, thereby enhancing decomposition accuracy. In \Method, we provide 2-3 templates for each question category outlined in Table~\ref{tab:question_profile}.

%These templates serve two main purposes: first, they guide LLMs to highlight different properties with various symbols, enabling accurate extraction of query arguments from the generated text. Secondly, 
%We create multiple templates for each question category and use different templates based on the question to be decomposed.
%A concrete template example for the category of time query questions is shown in Fig.~\ref{fig:prompts} (b). We select a typical question within this category and provide a solution as well as an explanation. 

\textbf{Chain-of-Thought (CoT) Prompting} explicitly requests LLMs to generate their reasoning process, which improves accuracy in zero- or few-shot logical reasoning~\cite{chuCoTReasoningSurvey2024,lu2022learn}. In \Method, we include ``please generate step-by-step explanations'' in prompts to encourage CoT, as shown in Fig.~\ref{fig:question-decompose}.
CoT and detailed explanations in solution templates improve the quality of logical reasoning, especially for questions requiring multi-step thoughts.
For example, applying CoT to the question ``What did I do right after waking up on Wednesday?'' helps produce a step-by-step thinking process, such as: ``To find the activity right after waking up on Wednesday, we can query the start and end times of all activities on Wednesday, therefore we can use the function <<DetectFirstTime>> and <<DetectLastTime>> with ((all activities)) on [[Wednesday]]''.

\begin{figure}[t]
    \centering    
    \begin{subfigure}[b]{0.95\textwidth}
        \centering
        \includegraphics[width=\textwidth]{figs/question_decompose.png}
        \vspace{-6mm}
        \caption{The prompt design for the question decomposition stage includes ICL and CoT.}
        \label{fig:question-decompose}
    \end{subfigure}

    \begin{subfigure}[b]{0.95\textwidth}
        \centering
        \includegraphics[width=\textwidth]{figs/solution_template.png}
        \vspace{-6mm}
        \caption{Example solution template for the time query question category.}
        \label{fig:solution-template}
    \end{subfigure} %\hspace{0.05\textwidth} % Add horizontal space between subfigures if needed

    \begin{subfigure}[b]{0.6\textwidth}
        \centering
        \includegraphics[width=0.8\textwidth]{figs/finetuning.png}
        \vspace{-2mm}
        \caption{Prompt design in the answer assembly stage for LLM finetuning.}
        \label{fig:finetuning}
    \end{subfigure}
    \vspace{-4mm}
    \caption{Prompts design in \Method to better leverage the power of LLMs.}
    \label{fig:prompts}
    \vspace{-4mm}
\end{figure}


\subsubsection{Sensor Data Query}
\label{sec:sensor-data-query}

The sensor query stage is the core module of \Method, responsible for searching the sensor database and extracting relevant information, which directly impacts the accuracy of \Method's answers.
Existing sensor-based QA systems did not employ an explicit search module, restricting to low-dimensional sensor features~\cite{englhardt2024classification,kim2024health,yang2024drhouse} or fixed windows of sensor signals~\cite{xing2021deepsqa,moon-etal-2023-imu2clip,moon2023anymal,chen2024sensor2text,arakawa2024prism}.
In contrast, \Method introduces this explicit query stage to handle long-duration, fine-grained timeseries sensor signals and provide accurate information to queries, which is not possible with prior QA designs.

%In this stage, \Method invokes functions with arguments provided by the question decomposition stage and returns the extracted information as text
%For example, a query calling \texttt{CountingDays} using activity ``\textit{at home}'' and dates ``\textit{last week}'' may return one line of text ``\textit{You spent 4 days at home last week}'', which is then passed to the answer assembly stage.


%%%Difference from SOTA - what is missing from SOTA? what makes this challenging

%Sensor data query is the core component of \Method, which extracts relevant sensor data from raw timeseries in a long time span and composes the factual sensor information in text, , as shown in the middle part of Fig.~\ref{fig:overview}.

The sensor and label encoders are pretrained offline as explained in Sec.~\ref{sec:pretraining}. 
With the pretrained sensor encoder, the full-history raw sensor signals are converted into an embedding database beforehand.
During online user interactions, as shown in the middle part of Fig.~\ref{fig:overview}, the sensor query stage only needs to encode the context (e.g., activity) generated in the question decomposition stage using the text encoder (\textcircled{4} in Fig.~\ref{fig:overview}) , performs a similarity search among the sensor database given the text (\textcircled{5}), and summarizes the retrieved sensor embeddings into textual information for the final answer assembly (\textcircled{6}).
The design of the sensor query stage faces two primary challenges: (1) ensuring accurate and efficient searches within the sensor embedding database, and (2) constructing adequate output context based on users' queries. 
\Method addresses these challenges through an efficient similarity-based embedding search mechanism and a set of carefully designed summarization functions, as detailed below.
%(1) handling diverse queries from practical user questions, and (2) performing accurate and efficient searches on timeseries data.
%The sensor data encoder converts raw timeseries data to lightweight embeddings, enabling accurate and efficient searches within the embedding database.
%We next explain the detailed designs.



%We include the list of important notations in Table~\ref{tbl:notation}.



%While designing many functions can enable finer-grained sensor queries, it may also confuse the LLM during the decomposition stage. To balance these factors, we designed six functions in \Method, covering common question and answer types listed in Table~\ref{tab:sensorqa_profile}.
%\textcolor{red}{Note: we don't need to cover exhaustive functions, as stage 1 and stage 3 will project the scenarios to these questions.}
%%% What are the main challenge of this stage?
%The \textit{key challenge} lies in accurately extracting sensor data from an extensive duration. 
%This involves two major components: first, training a sensor encoder that accurately encodes raw data into embeddings for database storage, ensuring query \textit{precision}. Second, developing a simple yet comprehensive set of query functions to extract the most \textit{relevant} information. These components jointly determine \Method's ability to provide correct answers in the final stage.
%first, training a sensor encoder that accurately encodes raw sensor data into embeddings for database storage. A well-designed encoder ensures the \textit{precision} of sensor queries. Secondly, it is crucial to develop a simple yet comprehensive set of query functions to effectively extract the most \textit{relevant} sensor information from the long duration. Together, these two components determine whether \Method can provide correct answers in the final stage.

%Both sensor data encoder and query functions are crucial in ensuring the relevance and precision of the extracted sensor information, which 



\textbf{Similarity-based Embedding Search} 
%We design efficient query search in \Method based on the lightweight logits.
Our goal is to accurately identify all sensor samples relevant to the query context.
For example, if the query context is ``exercise'', \Method aims to retrieve all sensor samples associated with activities similar to exercise.
If multiple text are generated during question decomposition, \Method performs an embedding search for each query text individually.
The pretraining in Sec.~\ref{sec:pretraining} ensures that the sensor embedding space is well aligned with the text embedding space even using partial context. 
Therefore, the similarity comparison given any pair of sensor and text embeddings can be achieved by training a similarity function $f$ in the embedding space: 
\begin{equation}
f(\mathbf{z}^s_t, \mathbf{z}^w) = f \big(\theta(\mathbf{x}_t), \phi(w) \big) \in [0, 1]
\end{equation}
The output of $f$ is a scalar value between 0 and 1, representing the similarity between the sensor and text samples.
With $f$, the similarity search in the embedding space is significantly more efficient than searching directly through raw, high-dimensional time-series data. The efficiency can be further improved by narrowing the search scope based on the date and time-of-day arguments identified during question decomposition.

\textbf{Summarization Function Design}
We design a set of summarization functions to "summarize" the queried sensor samples and generate contextual information for the answer assembly stage. The specific summarization function to be used is determined by \Method during question decomposition.
For example, a question of "\textit{how long}" should be directed to the \texttt{CalculateDuration} function, while a question of "\textit{what did I do}" should be handled by the \texttt{DetectActivity} function.
Each summarization function uses a unique template to return text information.
The numerical value in the returned text is determined by the embedding search results.
For instance, when querying \texttt{CalculateDuration} with the activity ``\textit{cooking}'', date ``\textit{Sunday}'', and time ``\textit{morning}'', the text output can be: "\textit{You spent {$\gamma$} minutes cooking on Sunday morning}", where $\gamma$ is calculated as follows:
\begin{equation}
\gamma = \sum_{t \in T_{SundayMorning}} \Bigg[ f \big( \theta(\mathbf{x}_t), \phi( \textrm{"cooking"}) \big) > h \Bigg].
\end{equation}
%\begin{equation}
%    \gamma = \sum_{t \in T_{SundayMorning}} [ g(\mathbf{x}_t;\theta, \phi) > h ].
%\end{equation}
Here $T_{SundayMorning}$ represents all timestamps within Sunday morning. 
The notation $[Cond]$ gives $1$, when the inner condition {\it Cond} is met; otherwise $0$.
$h$ is a predetermined threshold.

In \Method, we carefully design a set of summarization functions to account for diverse scenarios in real life, including time quries, activity quries, counting, etc. The details of those functions are explained in Appendix~\ref{sec:query-function}.

\subsubsection{Answer Assembly}
\label{sec:answer-assembly}
As shown in the right box of Fig.~\ref{fig:overview}, the final stage of answer assembly integrates question and sensor information to generate the final answer.
State-of-the-art methods~\cite{xing2021deepsqa,zhang2023llama,moon-etal-2023-imu2clip} rely on ``black-box'' fusion of natural language and sensory data, often leading to ineffective fusion and inaccurate answers (see Sec.~\ref{sec:motivation}).
In contrast, \Method summarizes query results to text and directly fuses them with the question in the prompt, as illustrated in Fig.~\ref{fig:finetuning}.
Our intuition is that, in contrast to processing and fusing with other modalities, LLMs are the most professional in dealing with text.
\Method is capable of answering both qualitative and quantitative questions by combining the original question and the extracted fine-grained activity information from long-duration, high-dimensional sensors.
At this answer assembly stage, we finetune a LLM such as LLaMA~\cite{zhang2023llama} to adapt the model to the desired answer style. Fine-tuning is chosen over few-shot learning as it delivers better performance with the presence of high-quality datasets like \Dataset~\citesensorqa (see Sec.\ref{sec:ablation}). We use Low Rank Adaptation (LoRA)~\cite{hu2021lora} due to its parameter efficiency and comparable performance to full fine-tuning.




%the \texttt{CalculateDuration} function is widely used in time compare, time query and existence questions.
%\texttt{CalculateFrequency} and \texttt{ActivityDetection} are designed for the types of counting and activity query questions, respectively. The \texttt{CalculateDays} function is added to address day counting as a subset of the counting category.
%The \texttt{DetectFirstTime} and \texttt{DetectLastTime} are for counter the needs of concrete timestamp queries.
%Finally, the arguments \texttt{Activity, Date} and \texttt{Time\_of\_Day} are given as additional constraints to limit the search range within the sensor's duration.
%The above four functions are able to cover the majority of question and answer types.
%The sensor database comprises a full history of sensor data with timestamps and a pretrained label classifier, which we explain further in Sec.~\ref{sec:reality}.

%, since days counting differs from regular activity frequency counting.
%is the most commonly used, calculating the duration of a specified activity on a specified date for a certain period. \texttt{CalculateDuration} is widely used in handling time compare, time query and existence questions. Functions 

%We recognize a few potential directions to further improve the sensor query such as using vector databases~\cite{zhou2024llm} or retrival augmented generation~\cite{zhao2024retrieval}, which we leave for future investigation.
%We recognize the potential of enhancing \Method with vector database~\cite{zhou2024llm} or retrieval augmented generation techniques~\cite{zhao2024retrieval}. However, we emphasize that   We discuss the possible extensions to \Method in Sec.~\ref{sec:discussion}.
% 

%\subsection{Optimizing \Method for the Edge}
%\label{sec:edge}
%Deploying \Method on edge devices is critical in preserving user's privacy.
%As shown from Fig.~\ref{fig:overview}, the computational complexity of \Method mainly comes from two parts: (1) the LLMs in question decomposition and answer assembly, and (2) the sensor data encoder and and logits search.
%We use two strategies to optimize them.

%\textbf{Optimizing LLM deployments}
%We optimize the LLMs in \Method using quantization, which has achieved real-time token generation on desktop-level systems in recent works~\cite{lin2023awq,kim2023squeezellm}.
%Thanks to \Method's design of only using text inputs for both question decomposition and answer assembly, the LLMs in \Method can be integrated with any state-of-the-art quantization techniques, such as AWQ~\cite{lin2023awq}.
%In contrast, multimodal LLMs cannot be fully quantized by these techniques due to the additional adapter modules.
%The only thing needed is a calibration dataset, which can be created using our solution templates and finetuning data.
%We construct a calibration dataset from our data 

%\textbf{Optimizing the Encoding of Sensor Data}
%Although inferences on the sensor and label encoders can be inefficient on edge devices in real time, these encoding operations can be performed offline when the QA part is idle.
%We further optimize the encoding stage by performing the encoding inference offline, when the QA part is idle.
%During user interactions, \Method only needs to search the precomputed logits and ensure real-time responses without delay.
%Moreover, the sensor encoder ``compresses'' raw time series data into logits, thus significantly reducing storage requirements as shown in Sec.~\ref{sec:memory}.
\section{System Implementation}
\label{sec:system-implementation}
%We use GPT 3.5 Turbo~\cite{gpt-3.5} and LLaMA3-8B~\cite{touvron2023llama} in the question decomposition stage, while finetuning the open-source LLaMA2-7B and LLaMA3-8B~\cite{touvron2023llama} in the answer assembly stage.
%We focus on the small models because of the resource limitation of edge devices.


\iffalse
\begin{figure}[t]
   \centering
    \setlength{\tabcolsep}{0.2pt}
    \begin{tabular}{ccc}
        \vspace{-2mm}
        \includegraphics[width=0.35\textwidth, height=3.6cm]{figs/system.png} &
        \includegraphics[width=0.32\textwidth, height=3.4cm]{figs/sensorchat_cloud.png} &
        \includegraphics[width=0.27\textwidth, height=3.4cm]{figs/sensorchat_edge.png} \\
\end{tabular}
\vspace{-3mm}
    \caption{Sensitivity of key hyperparameters.}
    \label{fig:system}
    \vspace{-5mm}
\end{figure}
\fi



We implement \Method on real-world systems. We envision \Method as a personal assistant that provides accurate and timely answers to user questions, as outlined in our problem statement (Sec.~\ref{sec:problem-statement}).
Fig.~\ref{fig:system} visualizes the general system pipeline of \Method.
We employ smartphones and smartwatches to collect multimodal sensor data from users in daily lives (\textcircled{1} in Fig.~\ref{fig:system_diagram}).
Implemented based on the ExtraSensory App~\cite{vaizman2018extrasensory}, the mobile devices automatically gathers data for 20 seconds every minute, including 40Hz IMU signals, 13 MFCC audio features from a 22kHz microphone, and other phone state information including compass, GPS location, Wi-Fi status, light intensity, battery level, etc. The details can be found in the ExtraSensory App manual~\cite{vaizman2018extrasensory}.
These sensor data are then transmitted to a system running \Method (\textcircled{2} in Fig.~\ref{fig:system_diagram}). We offer two variants of \Method, designed for a cloud server and an edge environment respectively. Their detailed implementations and trade-offs are discussed below. Finally, users can interact with \Method directly through a chatting interface using natural language, shown as \textcircled{3} in Fig.~\ref{fig:system_diagram}.


\textbf{\MethodC and \MethodE} We offer two system variants of \Method, as shown in Fig.~\ref{fig:sensorchat_cloud} and \ref{fig:sensorchat_edge}.
\begin{itemize}
    \item \textbf{\MethodC}, designed for a cloud environment, uses GPT-3.5-Turbo~\cite{gpt-3.5} for question decomposition and a full-size finetuned LLaMA2-7B model~\cite{touvron2023llama} for answer assembly. We deploy and test \Method on a cloud server equipped with an NVIDIA A100 GPU~\cite{a100}.
    \item \textbf{\MethodE}, designed for an edge environment, uses quantized LLaMA model~\cite{touvron2023llama} for both question decomposition and answer assembly. The question decomposition model is quantized from the official LLaMA3-8B, while the answer assembly model is quantized from our fine-tuned version of LLaMA2-7B. 
    We use Activation-aware Weight Quantization (AWQ), a state-of-the-art quantization method for LLMs, known for its hardware efficiency. We deploy and test \MethodE on a NVIDIA Jetson Orin NX module~\cite{jetsonorin} with 16GB RAM.
\end{itemize}
\MethodC and \MethodE accommodate two typical use scenarios. \MethodC is expected to deliver superior QA performance with the full-precision LLMs in the cloud, at the cost of intensive resource consumption. Additionally, \MethodC requires the users to transmit the full sensor history to the cloud server.
On the other hand, \MethodE runs entirely on a local edge platform belonging to the user, eliminating the need to transmit user data to the cloud and thus preserving user privacy. However, its QA and latency performance degrade compared to \MethodC.  

%a Linux desktop and an NVIDIA Jetson Orin~\cite{}.
%RPi 5 enjoys a 2.4GHz quad-core Cortex-A76 CPU and 8GB RAM. 
%The desktop is equipped with an Intel Core i7-8700 CPU . The Jetson platform features a dual-core NVIDIA Denver 2 CPU, a quad-core ARM Cortex-A57 MPCore, an NVIDIA Pascal GPU with 256 CUDA cores, and 8GB of RAM. %We measure response generation speed on both platforms.

\begin{figure}[t]
  \begin{subfigure}[b]{0.38\textwidth}
        \centering
        \includegraphics[width=\textwidth]{figs/system.png}
        \vspace{-5mm}
        \caption{Real system implementation.}
        \label{fig:system_diagram}
    \end{subfigure} %\hspace{0.05\textwidth} % Add horizontal space between subfigures if needed
    \begin{subfigure}[b]{0.3\textwidth}
        \centering
        \includegraphics[width=\textwidth]{figs/sensorchat_cloud.png}
        \vspace{-5mm}
        \caption{\MethodC diagram.}
        \label{fig:sensorchat_cloud}
    \end{subfigure}
    \begin{subfigure}[b]{0.28\textwidth}
        \centering
        \includegraphics[width=\textwidth]{figs/sensorchat_edge.png}
        \vspace{-5mm}
        \caption{\MethodE diagram.}
        \label{fig:sensorchat_edge}
    \end{subfigure}
    \vspace{-4mm}
    \caption{System implementation details of \Method.}
    \label{fig:system}
    \vspace{-4mm}
\end{figure}


\textbf{Implementation Details}
The algorithm part of \Method is implemented with Python and PyTorch~\cite{paszke2019pytorch}.
For the offline sensor encoder, \Method uses a Transformer architecture~\cite{vaswani2017attention} with 6 encoder layers, 8 attention heads, and a feedforward network size of 2048 for time series data. For the low-dimensional phone status data, \Method employs a fully connected layer as the encoder. The fusion layer is also implemented as a fully connected layer. The label encoder is initialized from the pretrained CLIP ViT-B/32 label encoder~\cite{radford2021learning}. Both the sensor and label embedding spaces share a dimension size of 512. Offline pretraining is performed with the partial-context loss proposed in Sec.~\ref{sec:pretraining} using a temperature scalar of $\tau=0.1$. We use the Adam optimizer with a learning rate of $1e-5$ over 100 epochs.
%We test three candidate GPTs for question decomposition: GPT-3.5-Turbo~\cite{gpt-3.5}, GPT-4~\cite{gpt-4}, and quantized LLaMA3-8B~\cite{lin2023awq}, as shown in Fig.~\ref{fig:overview}. 
%In the answer assembly stage, \Method uses the open source LLaMA2-7B and LLaMA3-8B models~\cite{touvron2023llama}. 
%We focus on smaller models so that they can be deployed on edge devices after quantization.

For question decomposition, we design two solution templates for each question category. Limiting the number of templates to two helps balance prompt length and performance.
For online data queries, we initialize the classifier $f$ as a multilayer perceptron with one hidden layer and a ReLU activation function. The hidden layer size is set to 512 and the query threshold is configured to $h = 0.5$.
For finetuning the LLM in answer assembly, we apply low rank adaptation (LoRA) finetuning~\cite{hu2021lora} on an A100 GPU, using a batch size of 8, a learning rate of 0.0002 and 10 epochs.
The LoRA rank $r$ is 16, the LoRA scaling factor \textit{alpha} is 16, and the LoRA dropout is 0.1.
During answer generation, \Method uses a max sequence length of 1024 and a geberating temperature of $0.2$.
%

%%%%%%%%%%%%%%%%%%%%%%%%%%%%%%%%%%%%%%%%%%%%%%%%%%%
% Evaluation
%%%%%%%%%%%%%%%%%%%%%%%%%%%%%%%%%%%%%%%%%%%%%%%%%%%
\vspace{-2mm}
\section{Evaluation on State-of-the-Art Dataset}
\label{sec:evaluation}

In this section, we thoroughly evaluate \Method on the state-of-the-art dataset focusing on quantitative questions. We further validate \Method in a real world study with open-ended, qualitative questions in Sec.~\ref{sec:deployment}.


\subsection{Dataset and Metrics}

In this evaluation section, we focus mainly on the \Dataset dataset~\citesensorqa and quantitative questions. A detailed introduction to \Dataset~\citesensorqa is provided in Sec.~\ref{sec:motivation}. 
To the best of our knowledge, \Dataset is the first and only available benchmarking dataset for QA interactions that use long-term timeseries sensor data and reflect practical user interests.
While we focus on \Dataset~\citesensorqa in this section, we emphasize that the motivation and design of \Method are broadly applicable and can be extended to other practical sensing applications.
To ensure the best alignment between the QA pairs and sensor information, we conduct offline encoders pretraining on the ExtraSensory multimodal sensor dataset~\cite{vaizman2017recognizing}, which servers as the sensor data source for \Dataset.
During pretraining, all sensor samples are aligned by a time window of 20 seconds.
%The experiments can be extended to other sensor application that has data readily available in the future.


%For all experiments, we randomly select 80\% of the QA pairs in \Dataset for training or finetuning and use the rest 20\% for testing. 
%\textcolor{red}{In Sec.XX, we further evaluate the gene}
%For all methods, we use the first 48 users' data for training and the remaining 12 for testing. 
%The training process for \Method includes training the sensor and label encoder with sensor data, followed by finetuning the LLaMA model using both the QA data and the stored sensor database.


\textbf{Dataset Variants and Metrics}
We evaluate three versions of \Dataset~\cite{sensorqa} using various metrics to assess both the quality and accuracy of the generated answers.
\begin{itemize}[topsep=0pt, itemsep=0pt]
    \item \textbf{Full answers} refer to the original full responses  in \Dataset. We evaluate the model's performance on the full answers dataset using Rouge-1, Rouge-2, and Rouge-L scores~\cite{eyal-etal-2019-question}. Rouge scores measure the overlap of n-grams between the machine-generated content and the ground-truth answers, expressed as F-1 scores. Higher Rouge scores indicate greater similarity between the generated and true answers.
    \item \textbf{Short answers} are the 1-2 key words extracted from the full answers by GPT-3.5-Turbo~\cite{gpt-3.5}, offered with the original \Dataset dataset~\citesensorqa. We use the exact match accuracy on the short answers to evaluate the precision of generated answers, as detailed in Sec.~\ref{sec:motivation}.
    \item \textbf{Multiple choices} are generated by prompting GPT-3.5-Turbo~\cite{gpt-3.5} to create three additional choices similar to the correct short answer. An example QA can be "Which day did I spend the most time with coworkers? A. Friday, B. Monday, C. Thursday, D. Wednesday", with the correct answer being "D" or "D. Wednesday." The models are expected to accurately select the correct answer from the four candidates. We evaluate the performance based on exact answer selection accuracy.
\end{itemize}
We create the multiple-choice version in addition to the full answers and short answers provided in the original \Dataset dataset~\citesensorqa, to further assess the model's ability in distinguishing similar facts based on sensor data. 
We use different dataset variants to assess various aspects of the models. The full answers dataset evaluates overall language quality, while the short answers and multiple-choice evaluations focus on the model's ability to learn underlying facts rather than relying solely on patterns in language token generation. Further discussion of the evaluation metrics can be found in Sec.~\ref{sec:future-work}.
%the original questions and answers collected from workers, a short-answer version, and a multiple-choice version. For the short-answer version, we use GPT-3.5 to extract 1-2 key words from the original answers. For the multiple-choice version, we use GPT-3.5 to generate three additional choices that are similar to the correct short answer, converting it into a multiple-choice question. The full-answer version is suitable for evaluating the language quality of the responses, while the short-answer and multiple-choice versions facilitate exact match accuracy evaluation, highlighting the precision of the answers.


\vspace{-2mm}
\subsection{State-of-the-Art Baselines}



We compare \Method against several state-of-the-art baselines that leverage different modalities combinations, including \textcolor{mygreen}{text}, \textcolor{myred}{vision+text}, and \textcolor{myblue}{sensor+text} data. These comparisons highlight the effectiveness of integrating multiple modalities to address the sensor-based QA tasks. 
We consider both closed-source and open-source baselines for a comprehensive analysis.

%For visual inputs, we use the activity graphs created when generating \Dataset.
 % to address natural QA interactions with sensors.

We first evaluate the state-of-the-art closed-source generative models using various modalities:
\begin{itemize}[topsep=0pt, itemsep=0pt]
    \item \textbf{GPT-3.5-Turbo~\cite{gpt-3.5} and GPT-4~\cite{gpt-4}} are \textcolor{mygreen}{text-only} baselines taking only the questions as inputs.
    \item \textbf{GPT-4-Turbo~\cite{gpt-4} and GPT-4o~\cite{gpt-4}} are \textcolor{myred}{vision+text} baselines taking the activity graphs and the questions as inputs. For all \textcolor{myred}{vision+text} baselines, we feed the activity graphs in \Dataset~\citesensorqa, similar to Fig.~\ref{fig:example_qas}, along with the questions into the model.
    \item \textbf{IMU2CLIP+GPT-4\footnote{\url{https://github.com/facebookresearch/imu2clip}}~\cite{moon-etal-2023-imu2clip}} is the state-of-the-art \textcolor{myblue}{sensor+text} GPT baseline as explained in Sec.~\ref{sec:motivation}.
\end{itemize}
For these closed-source generative models, we use few-shot learning (FSL). Specifically, we incorporate a set of QA examples from \Dataset~\citesensorqa into the prompt for each question input. We adopt $10$ samples per question based on a grid search of $\{2, 5, 10, 15\}$.





%The various combinations of modalities include \textcolor{mygreen}{text-only}, \textcolor{myred}{vision+text} and \textcolor{myblue}{sensor+text}.
%Given that the innovative aspect of our \Method QA model lies in its fine-tuning approach, we select baselines that either train a neural network or finetune a language model using the \Method training dataset and subsequently evaluate their performance on the \Method test dataset.

We further conduct comprehensive evaluation with the state-of-the-art open-source models using various modalities:
\begin{itemize}[topsep=0pt, itemsep=0pt]
    \item \textbf{T5~\cite{2020t5}} and \textbf{LLaMA\footnote{\url{https://www.llama.com/}}~\cite{touvron2023llama}} are popular \textcolor{mygreen}{text-only} language models.
    
    \item \textbf{LLaMA-Adapter\footnote{\url{https://github.com/OpenGVLab/LLaMA-Adapter}}~\cite{zhang2023llama}} is a recent \textcolor{myred}{vision+text} framework offering a lightweight method for fine-tuning instruction-following and multimodal LLaMA models. It integrates vision inputs (i.e., activity graphs from \Dataset~\citesensorqa) with LLMs using a transformer adapter. We utilize the latest LLaMA-Adapter V2 model.
    
    \item \textbf{LLaVA-1.5\footnote{\url{https://llava-vl.github.io/}}~\cite{liu2024improved}} represents state-of-the-art \textcolor{myred}{vision+text} model. LLaVA connects pre-trained CLIP ViT-L/14 visual encoder~\cite{radford2021learning} and large language model Vicuna~\cite{vicuna2023}, using a projection matrix. LLaVA-1.5~\cite{liu2024improved} achieves state-of-the-art performance on 11 benchmarks through simple modifications to the original LLaVA and the use of more extensive public datasets for finetuning.
    
    \item \textbf{DeepSQA\footnote{\url{https://github.com/nesl/DeepSQA}}~\cite{xing2021deepsqa}} trains a CNN-LSTM model with compositional attention to fuse \textcolor{myblue}{sensor+text} modalities and predict from a fixed and limited set of candidate answers given questions and IMU signals. We adapt their implementation to use the full-history timeseries data as input, to align with \Method's setup.
    
    \item \textbf{OneLLM\footnote{\url{https://github.com/csuhan/OneLLM}}~\cite{han2023onellm}} is a state-of-the-art multimodal LLM framework that processes \textcolor{myblue}{sensor+text} modalities using a universal pretrained CLIP encoder and a mixture of projection experts for modality alignment. We adapt their implementation and feed the full-history timeseries data from the IMU tokenizer.
\end{itemize}
All baselines based on LLaMA (that is, LLaMA, LLaMA-Adapter, and OneLLM) use LLaMA2-7B~\cite{touvron2023llama}.
For T5 and DeepSQA, we train the models directly on the \Dataset dataset~\citesensorqa.
For the LLM baselines, we apply LoRA fine-tuning (FT)~\cite{hu2021lora} using the samples from \Dataset~\citesensorqa. 
For all methods that require training, we randomly select 80\% of the QA pairs in \Dataset~\citesensorqa as training samples and reserve the remaining 20\% for testing. We explore alternative splitting schemes in Sec.~\ref{sec:generalizability} to demonstrate \Method's generalizability to unseen users.
All baselines adopt the same hyperparameters as those specified in their official codebases.
%
We did not compare with the latest works of Sensor2Text~\cite{chen2024sensor2text}, PrISM-Q\&A~\cite{arakawa2024prism}, and DrHouse~\cite{yang2024drhouse} due to limited access to their open-source code and models.



\begin{figure*}[tb]
  \centering
  \begin{subfigure}[b]{0.82\textwidth}
        \centering
        \includegraphics[width=\textwidth]{figs/qual_example_1.png}
        \vspace{-6mm}
        \caption{Example of a time query question on a single-day duration.}
        \label{fig:qual-daily}
    \end{subfigure}
    
    \begin{subfigure}[b]{0.82\textwidth}
        \centering
        \includegraphics[width=\textwidth]{figs/qual_example_2.png}
        \vspace{-6mm}
        \caption{Example of a counting question on a multi-day duration.}
        \label{fig:qual-week}
    \end{subfigure}
  \vspace{-4mm}
  \caption{Qualitative results of \Method in comparison to state-of-the-art methods.}
  \vspace{-6mm}
  \label{fig:qual_results}
\end{figure*}




\subsection{Qualitative Performance}
\label{sec:qualitative}

We illustrate a qualitative comparison of \Method with the top-performing baselines in Fig.~\ref{fig:qual_results}. 
Specifically, Fig.~\ref{fig:qual_results} (a) focuses on time-related queries and (b) focuses on counting questions, which are the most challenging for SOTA methods given the long-duration time series sensor inputs (see Sec.~\ref{sec:motivation}).
Key phrases in the answers are highlighted in the green if they closely match the true answer, and in the red if they do not. The two presented examples are very challenging for state-of-the-art baselines. %, with GPT-4o in Fig.~\ref{fig:qual_results} (b) being the only correct instance.
\Method, on the other hand, consistently produces more accurate answers which can be attributed to its novel three-stage pipeline.

Answering the question in Fig.~\ref{fig:qual_results} (a) requires two major steps: (i) calculating the total time spent in school and in the main workplace, and (ii) computing the difference between them.
\Method accurately answers it by first decomposing and then querying the durations for ``at school'' and ``in the main workplace'' respectively, resulting in ``\textit{You spent 11 hours and 27 minutes at school Tuesday. You spent 9 hours and 3 minutes at main workplace}''. Finally, \Method integrates the above text and determines the time difference during the answering stage.
Although \Method's answer is 15 minutes off from the true value, possibly due to inaccuracies in the sensor encoder or LLM reasoning, \Method still approximates the ground truth with an accuracy unmatched by other baselines.


In Fig.~\ref{fig:qual_results} (b), counting the total days spent at home requires long-term reasoning where  LLaMA-Adapter~\cite{zhang2023llama} and  DeepSQA~\cite{xing2021deepsqa} usually fall short.
\Method decomposes this question and queries the duration of "at home" on "each day", then leaving the counting task to the answering stage.
In a nutshell, \Method relies on question decomposition and sensor data queries to extract relevant key sensor information, while the answer assembly stage handles reasoning and produces the final answer.
This collaboration across the three stages allows \Method to effectively manage a wide range of tasks, particularly those requiring multi-step reasoning and quantitative analyzes, which highlights \Method's advancements over existing works.
%The three-stage design in \Method and the intelligent work division between each stage enables \Method to properly handle such challenging tasks requiring multi-step reasoning.


\begin{table*}[t]
{
\footnotesize
\centering
\begin{tabular}{c|c|c|ccc|c|c}
\toprule
 \textbf{Modalities} & \textbf{Method} & \textbf{FSL/FT$^1$} & \multicolumn{3}{c|}{\textbf{Full Answers}} & \textbf{Short Answers} & \textbf{Multiple Choices} \\
 & & & Rouge-1 ($\uparrow$) & Rouge-2 ($\uparrow$) & Rouge-L ($\uparrow$) & Accuracy ($\uparrow$) & Accuracy ($\uparrow$) \\
\midrule
\textcolor{mygreen}{Text} & GPT-3.5-Turbo~\cite{gpt-3.5} & FSL & 0.35 & 0.23 & 0.32 & 0.03 & 0.33 \\
\textcolor{mygreen}{Text} & GPT-4~\cite{gpt-4} & FSL & 0.66 & 0.51 & 0.64 & 0.16 & 0.34 \\
\textcolor{mygreen}{Text} & T5-Base~\cite{2020t5} & FT & 0.71 & 0.55 & 0.69 & 0.25 & 0.52 \\
%\hline
\textcolor{mygreen}{Text} &  LLaMA2-7B~\cite{llama2} & FT & \underline{0.72} & \underline{0.62} & \underline{0.72} & 0.26 & \underline{0.56} \\
\hline
\textcolor{myred}{Vision+Text} & GPT-4-Turbo~\cite{gpt-4} & FSL & 0.38 & 0.28 & 0.36 & 0.14 & 0.24 \\
\textcolor{myred}{Vision+Text} & GPT-4o~\cite{gpt-4} & FSL & 0.39 & 0.28 & 0.37 & 0.20 & 0.07 \\
%\hline
\textcolor{myred}{Vision+Text} & LLaMA-Adapter~\cite{zhang2023llama} & FT & 0.73 & $0.57$ & $0.71$ & \underline{0.28} & 0.54 \\
%\hline
\textcolor{myred}{Vision+Text} & LLaVA-1.5~\cite{liu2024improved} & FT & $0.62$ &  $0.46$ & $0.60$ & 0.21 & 0.47 \\
\hline
\textcolor{myblue}{Sensor+Text} & IMU2CLIP-GPT4~\cite{moon-etal-2023-imu2clip} & FSL & 0.44 & 0.28 & 0.40 & 0.13 & 0.18 \\ 
\textcolor{myblue}{Sensor+Text} & DeepSQA~\cite{xing2021deepsqa} & FT & 0.34 & 0.05 & 0.34 & 0.27 & - \\
\textcolor{myblue}{Sensor+Text} & OneLLM~\cite{han2024onellm} & FT & 0.12 & 0.04 & 0.12 & 0.05 & 0.30 \\
\midrule
\textcolor{myblue}{Sensor+Text} & \textbf{\MethodC} & FT & \textbf{0.77}& \textbf{0.62} & \textbf{0.75} & \textbf{0.54} & \textbf{0.70} \\
\textcolor{myblue}{Sensor+Text} & \textbf{\MethodE} & FT & 0.76 & 0.60 & 0.74 & 0.49 & 0.67 \\
\bottomrule
\multicolumn{8}{l}{$^{1}$\small{FS: Few-Shot Learning. FT: Finetuning.}} \\
\end{tabular}
}
\vspace{-1mm}
\caption{Quantitative results of \Method compared against state-of-the-art methods. Bold and underlined values show the best results (all achieved by \Method) and the best among baselines.} % *DeepSQA only considers classification problem thus is only evaluated on the exact-match version of the dataset.}
\vspace{-7mm}
\label{tab:quant_results}
\end{table*}




\subsection{Quantitative Performance}

Table~\ref{tab:quant_results} presents the quantitative results of all methods on the three variants of \Dataset~\citesensorqa: full answers, short answers, and multiple-choice. Note that DeepSQA~\cite{xing2021deepsqa} was not evaluated on the multiple-choice version due to its inability to handle dynamic answer choices.
It is important to highlight that exact match accuracy for short and multiple-choice answers is a strict metric, as it requires the model to generate answers in the \textit{exact} same form as the correct ones. For instance, "4 hours" and "3 hours 50 min" would be considered different. Answering multiple-choice questions can also be particularly challenging, as candidate answers differ only slightly, such as "A. 10 min" vs. "B. 20 min," making them difficult to distinguish. In practical applications, however, a QA system does not necessarily need to achieve perfect exact match accuracy to be useful. We leave the exploration of more advanced metrics that better align with user satisfaction for future work, as discussed in Sec.~\ref{sec:future-work}.

As shown in Table~\ref{tab:quant_results}, our \Method outperforms the best state-of-the-art methods with the \textbf{highest Rouge scores on full answers}, \textbf{26\% higher accuracy on short answers}, and \textbf{14\% higher accuracy on multiple choices}. 
The top Rouge scores highlight \Method's ability to generate high-quality natural language that are the most similar with the ground-truth answers in \Dataset~\citesensorqa.
Even under the strict exact match evaluation, \Method achieves 54\% accuracy on short answers and 70\% on multiple-choice questions. These accuracy improvements demonstrate \Method's effectiveness in learning the underlying activity facts from the long-duration, multimodal time series sensor data. As explained in Sec.~\ref{sec:qualitative}, all three stages are crucial for achieving high accuracy, including correct question decomposition to invoke the right functions, precise sensor queries to accurately extract activity information from sensor data, and effective answer assembly for generating natural language responses.
\MethodC performs slightly better than \MethodE mainly due to the stronger model capability of GPTs compared to quantized LLaMA in question decomposition. However, \MethodE, as a pure edge solution, preserves better user privacy as discussed in Sec.~\ref{sec:system-implementation}.
%In future work, we plan to explore new metrics that allow small time drifts thus better fit practical use cases.

%The full answer dataset uses Rouge scores to measure the ``similarity'' between the model's generated answers and the true answers. Hence the full answer dataset focuses on the general text quality rather than precision. For example, an answer with similar wording but different key information, e.g., ``1 hour'' vs ``10 hours'', may still receive high Rouge scores.
%The short-answer and multiple-choice variants assess precision by comparing whether the exact key information is present in the answer.
%Together, these three datasets provide a comprehensive evaluation of the user-sensor QA task.


In contrast, all baselines struggle with quantitative accuracy, with an highest accuracy of merely 28\% on the short answers.
Despite GPT-4's strong reasoning capabilities and its multimodal variants (GPT-4-Turbo and GPT-4o), the models do not perform optimally for processing sensor data and daily life activities, as they are not specifically trained for these tasks.
The Rouge scores are lower because GPTs tend to generate longer text, resulting in less similarity with the ground-truth answers.
Interestingly, text-only baselines with finetuning, such as LLaMA2-7B~\cite{touvron2023llama} and T5~\cite{2020t5}, achieve some of the highest accuracy among the baselines, with 27\% accuracy on short answers and 56\% accuracy on multiple-choice datasets. These results suggest a sensor bias in the dataset, where some questions can be ``guessed'' correctly without sensor data.
Accuracy scores lower than these baselines indicate ineffective fusion of text and sensor data, as seen with models like Llava-1.5~\cite{liu2024improved} and OneLLM~\cite{han2024onellm}, whose poor performance stems from mismatches in pretraining and finetuning data formats. For example, OneLLM was designed for aligning raw IMU signals with fixed window sizes, making it challenging to adapt to long-duration sensor data in our task. Finally, DeepSQA~\cite{xing2021deepsqa} and LLaMA-Adapter~\cite{zhang2023llama} perform best among the baselines but struggle with accurate quantitative analysis on long-duration time series, as discussed in Sec.\ref{sec:motivation} and Sec.\ref{sec:qualitative}.


%through LLM-powered question decomposition and answer assembly stages. The accuracy gains demonstrate the effectiveness of sensor query and the fusion with text. 
%The GPT series are hindered by limited prompt length. If the narrative text does not fit in the length, then the prompt will be truncated and important information may be lost, leading to poor quality answers such as IMU2CLIP+GPT-4 in Fig~\ref{fig:qual_results} (b).
%The LLaMA-based models can adjust the answer according to the question and sensory input (if any), but struggles with accurate quantitative outputs. 

\iffalse
\subsection{Memory Requirements on Edge Devices}
\label{sec:memory}
\textcolor{red}{To be updated} Fig.~\ref{fig:memory} compares the model size requirements of all finetuning methods. The left plot shows unquantized LLM models while the right plot shows quantized LLMs by AWQ~\cite{lin2023awq} and small models such as DeepSQA~\cite{xing2021deepsqa} and T5~\cite{2020t5}.
Without quantization, all LLM-based methods require at least 13.5GB memory to accommodate model weights, making them unsuitable for edge deployment.
With AWQ quantization, \Method reduces its memory footprint to 3.8GB and fits into the 8GB RAM of a Jetson TX2. While non-LLM models, such as DeepSQA and T5, use less than 1GB of memory, they provide less natural and accurate answers as shown in the previous section. Multimodal LLMs require more memory to support adapter layers, where AWQ cannot be applied directly.
%
Notably, \Method consumes nearly the same memory as the text-only LLaMA2-7B. 
Such a negligible memory overhead can be attributed to the compression of encoding raw timeseries into logits, which reduces the total dataset size from 24G to 484K, achieving a compression ratio of approximately 50K times.
The sensor encoder and logits database enable \Method to run lightweight query search on edge devices.
\fi
%Compression ratio from raw data to logits.

\begin{figure}
\begin{center}
\vspace{-4mm}
\includegraphics[width=0.8\textwidth]{figs/efficiency2.png} 
\vspace{-4mm}
\caption{The answer accuracy and latency trade-offs of all methods measured on the cloud and edge platforms. 
We evaluate \MethodC on A100~\cite{a100} and \MethodE on Jetson Orin NX~\cite{jetsonorin}.
All LLM-based models are quantized to 4-bit weights with AWQ~\cite{lin2023awq} for Jetson Orin deployments.}
\label{fig:latency}
\end{center}
\vspace{-4mm}
\end{figure}


\subsection{End-to-End Answer Generation Latency}
To evaluate efficiency, we measure the end-to-end answer generation latency of \MethodC and \MethodE on their respective platforms - cloud server and NVIDIA Jetson Orin. For a fair comparison, we quantize all LLM-based baselines to 4-bit weights using AWQ~\cite{lin2023awq} for the Jetson Orin experiments, matching \MethodE’s setup. \textbf{\Method takes an average generation latency of 2.3s on the cloud and 10.0s on the Jetson Orin}.  
Specifically, \MethodC takes an average of 1.2s for question decomposition, 0.5s for sensor data query, and 0.6s for answer assembly. \MethodE takes an average of 2.5s, 4.9s, and 2.5s for each stage, respectively.


The accuracy-latency trade-off of \Method and locally running baselines is shown in Fig.~\ref{fig:latency}. Despite higher latency due to its dual-LLM design, \Method outperforms other baselines in accuracy. \Method still achieves real-time responses on the cloud and can be deployed on resource-constrained devices with reasonable generation latency.
Notably, quantization causes negligible accuracy loss in \MethodE compared to the full-precision \MethodC. This is due to \Method's design, which converts sensor data into text, making the final answer assembly a purely language-based task. Techniques like AWQ~\cite{lin2023awq} minimize accuracy degradation in language tasks after quantization.
We plan to further optimize the latency of \Method on edge devices via algorithm-system co-design in future work.
%All quantized LLM-based methods achieve negligible accuracy degradation compared to the full-precision models as in Table~\ref{tab:baseline_results_unfiltered}.
%Methods closer to the top left corner are preferred for their higher accuracy and lower latency.
%This indicates that \Method achieves real-time interactions on desktop-level edge devices and acts a feasible solution on more resource-constrained devices.
%On Jetson TX2, DeepSQA~\cite{xing2021deepsqa} encounters OOM issues with the large multi-day timeseries input. T5-Base without TensorRT~\cite{tensorrt} optimization cannot run neither due to its large model size. While T5-Small~\cite{2020t5} is very lightweight, it lacks sensor data, leading to low accuracy.
%Remarkably, \Method's latency is nearly identical to the text-only LLaMA2-7B, indicating minimal overhead from query searches. It also reduces latency by 38\% compared to the complex OneLLM model.


%On the desktop, all methods are very efficient.

\iffalse
\begin{figure*}[tb]
  \centering
  \begin{subfigure}[b]{0.44\textwidth}
        \centering
        \includegraphics[width=\textwidth]{figs/memory.png} 
        \vspace{-4mm}
        \caption{Memory requirements of quantized and unquantized models from various methods. The red horizontal line shows the RAM size limit of 8GB on Jetson TX2~\cite{jetsontx2}.}
        \label{fig:memory}
    \end{subfigure} \hspace{0.02\textwidth} % Add horizontal space 
    \begin{subfigure}[b]{0.44\textwidth}
        \centering
        \includegraphics[width=\textwidth]{figs/efficiency.png} 
        \vspace{-4mm}
        \caption{Efficiency of fine-tuning methods on desktop and Jetson TX2~\cite{jetsontx2}. The footnote $Q$ indicates quantized models.}
        \label{fig:efficiency}
    \end{subfigure}
  \vspace{-4mm}
  \caption{\textcolor{red}{To be updated.}}
  \vspace{-6mm}
  \label{fig:qual_results}
\end{figure*}
\fi





%\begin{itemize}
%    \item original AWQ (4bit)
%    \item personalized AWQ
%    \item original squeeze LLM (3bit)
%    \item personalized squeeze LLM
%\end{itemize}



\subsection{Ablation Studies}
\label{sec:ablation}

In this section, we comprehensively examine the impact of key design choices in \Method. Without loss of generality, we use \MethodC as the base model.
%major design variants in \Method by isolating one stage while keeping the other two fixed.
%This is because all three stages are essential for \Method and removing any one of them would disable the system entirely. 
%For example, without question decomposition, \Method would not know how to query the database. 
%We skip the answer assembly from discussion as it has few variants that are impactful to end-to-end performances.


\begin{table}[!t]
\small
\centering
%\vspace{-2mm}
\caption{Impact of three major stage in \MethodC. Bold values highlight the best results.}
\vspace{-4mm}
\label{tbl:ablation}
\begin{tabular}{c|ccc|c} 
\toprule
\small
\textbf{Setup} & \multicolumn{3}{c|}{\textbf{Full Answers}} & \textbf{Short Answers} \\ 
& Rouge-1 ($\uparrow$) & Rouge-2 ($\uparrow$) & Rouge-L ($\uparrow$) & Accuracy ($\uparrow$) \\
\midrule
w/o Question Decomposition & 0.73 & 0.57 & 0.71 & 0.35 \\
w/o Sensor Data Query & 0.72 & 0.62 & 0.72 & 0.26 \\ 
w/o Answer Assembly & 0.26 & 0.08 & 0.24 & 0.0 \\
\midrule
Full \Method & \textbf{0.77}& \textbf{0.62} & \textbf{0.75} & \textbf{0.54} \\
\bottomrule
\end{tabular}
\vspace{-2mm}
\end{table}

\textbf{Impact of Each Stage in \Method} 
We first evaluate the individual contribution of each stage in \Method by removing one of them from the pipeline. 
By removing question decomposition, we use a fixed and general decomposition templates for all questions. Removing sensor data query reverts the model to a language-only approach. By removing answer assembly, we directly output the queried sensor context from the second stage.
Table~\ref{tbl:ablation} summarizes the results on full and short answers including both quality and quantitative accuracy.
As observed, removing any stage leads to a significant performance drop. 
In \Method, all three stages must work collaboratively to deliver high-quality, accurate answers across diverse question types in \Dataset.
Among the three stages, the answer assembly stage has the most significant impact, as it directly influences the final output. Removing it results in severely degraded performance, with near-zero accuracy on short answers. However, the question decomposition and sensor data query stages are equally crucial.


\begin{table}[!t]
\small
\centering
%\vspace{-2mm}
\caption{Impact of various designs for sensor-text pretraining in \MethodC. Bold values highlight the best results.}
\vspace{-4mm}
\label{tbl:ablation-sensor-feature}
\begin{tabular}{c|c|c|c} 
\toprule
\small
\textbf{Sensor Data} & \textbf{Training Loss} & \textbf{Online Querying} & \textbf{Multiple Choices} \\ 
& & Accuracy ($\uparrow$) & Answer Accuracy ($\uparrow$) \\
\midrule
Statistical features & Partial-Context Loss & 0.91 & 0.62 \\
Time series & IMU2CLIP~\cite{moon-etal-2023-imu2clip} & 0.90 & 0.61  \\ \midrule
Time series & Partial-Context Loss & \textbf{0.98} & \textbf{0.70} \\
\bottomrule
\end{tabular}
\vspace{-2mm}
\end{table}

% If use F1, statistical features: 0.58, CLIP: 0.56, timeseries & our loss: 0.83

\textbf{Impact of Sensor Features and Pretraining Loss Functions}
We next evaluate the impact of sensor features and loss functions during offline encoder pretraining.
Specifically, we compare using statistical features (e.g., mean acceleration) versus raw time series inputs, and IMU2CLIP loss~\cite{moon-etal-2023-imu2clip} versus our proposed contrastive sensor-text pretraining loss for partial context (see Sec.\ref{sec:pretraining}). 
For IMU2CLIP, text samples are generated by combining all valid labels into one sentence. We report the online querying accuracy and multiple-choice answer accuracy to assess the influence to sensor information extraction.
As shown in Table~\ref{tbl:ablation-sensor-feature}, statistical features result in low querying and answer accuracies. This validates our motivation to design \Method that high-dimensional time series sensor data are critical for fine-grained activity information. IMU2CLIP training, even with fine-grained data, yields poorer querying and answering accuracies, highlighting its limited ability associating sensor embeddings from partial text query. Our proposed loss function, which aligns sensor and text encoders for partial context queries, proves more effective. These findings emphasize the importance of selecting appropriate sensor features and loss functions during pretraining in order to achieve high-performance QA.


\textbf{Impact of LLM Design Choices}
%Question decomposition, as the first stage, is critical for \Method diving into the correct direction.
We finally evaluate the impact of various design choices for LLMs. 
In question decomposition, we assess the contribution of in-context learning (ICL), chain-of-thought (CoT) techniques, and different backbone LLMs.
The results are summarized in Table~\ref{tbl:ablation-detailed-design}. 
Both ICL and CoT are crucial for high-quality and accurate answers. This is because an effective question decomposition improves sensor data queries. The solution templates in ICL are more essential to \Method as removing ICL reduces answer accuracy by 11\%. CoT enhances reasoning and slightly boosts accuracy by 1-5\%. Interestingly, using a more advanced backbone (GPT-4 vs. GPT-3.5-Turbo) results in minimal improvement in answer quality, as GPT-4, while generating richer text, does not follow instructions as well as GPT-3.5-Turbo according to our observation.

For answer assembly, we evaluate the effectiveness of finetuning compared to zero-shot or few-shot learning, as well as different LLaMA backbones. As shown in Table~\ref{tbl:ablation-detailed-design}, zero-shot learning results in poor performance, while few-shot learning improves answer quality but still lags behind finetuning. This highlights that finetuning is the most effective approach when a dataset like \Dataset is available. Using a more advanced LLaMA backbone, such as LLaMA3-8B, has minimal impact. Finetuning and the dataset prove to be more important than the model architecture during answer assembly.

%shows the performance of various question decomposition designs, including different LLMs and the use of in-context learning (ICL) and chain-of-thought (CoT) techniques.
%The closed-source GPT models generally yield more accurate decompositions due to their larger parameter sets.
%GPT-4 queries take 1x longer, while the latencies for queries without ICL and without CoT appear to be uncorrelated with prompt length, possibly due to OpenAI's internal setup.
%Interestingly, the latencies of GPT decompositions may not be directly related to prompt length. 
%Surprisingly, GPT-3.5-Turbo without ICL demonstrates the longest average latency, despite having shorter prompts without solution templates. This may be due to OpenAI's internal setup. GPT-4 queries take approximately 1x longer time to return. 
%Notably, using GPTs here only requires sending the questions to cloud but not the raw sensor data, thus still protecting sensitive information.
%The quantized LLaMA3-8B~\cite{lin2023awq} offers a trade-off between privacy and performance. Using quantized LLaMA3-8B preserves perfect privacy by avoiding transmitting both questions and sensor data to cloud, but leads to 4-11\% lower final answer accuracy and a longer latency per query of 8.1 seconds on a desktop.


%\textbf{Impact of Finetuning Designs}
%Table~\ref{tbl:ablation-finetuning}



\begin{table}[!t]
\small
\centering
\caption{Impact of various design choices for LLMs in \MethodC. Bold values highlight the best results.}
\vspace{-4mm}
\label{tbl:ablation-detailed-design}
\begin{tabular}{c|c|c|ccc|c} 
\toprule
\small
\textbf{Stage} & \textbf{Setup} & \textbf{Model in Stage} & \multicolumn{3}{c|}{\textbf{Full Answers}} & \textbf{Short Answer} \\ 
& & & Rouge-1 ($\uparrow$) & Rouge-2 ($\uparrow$) & Rouge-L ($\uparrow$) & Accuracy ($\uparrow$) \\
\midrule
& w/o ICL & GPT-3.5-Turbo & 0.75 & 0.59 & 0.72 & 0.43 \\
Question & w/o CoT & GPT-3.5-Turbo & 0.76 & 0.61 & 0.74 & 0.50 \\ 
Decomposition & Full & GPT-4 & \textbf{0.77} & \textbf{0.62} & \textbf{0.75} & 0.49 \\
& Full & GPT-3.5-Turbo & \textbf{0.77} & \textbf{0.62} & \textbf{0.75} & \textbf{0.54} \\ \hline
& Zero-shot learning & LLaMA2-7B & 0.16 & 0.07 & 0.14 & 0.0 \\
Answer & Few-shot learning & LLaMA2-7B & 0.43 & 0.29 & 0.41 & 0.24 \\
Assembly & Finetuning & LLaMA3-8B & 0.76 & 0.61 & 0.74 & 0.53 \\
& Finetuning & LLaMA2-7B & \textbf{0.77} & \textbf{0.62} & \textbf{0.75} & \textbf{0.54} \\
\bottomrule
\end{tabular}
\vspace{-2mm}
\end{table}








\iffalse
\begin{table}[!t]
\footnotesize
\centering
\vspace{-2mm}
\caption{Impact of designs in finetuning.}
\vspace{-2mm}
\label{tbl:ablation-finetuning}
\begin{tabular}{c|c|c|c} 
\toprule
\small
\textbf{Model} & \textbf{Method} & \textbf{Short Answer} & \textbf{Multiple Choices} \\ 
& & \textbf{Accuracy} & \textbf{Accuracy} \\ \hline
\multirow{2}{*}{LLaMA2-7B} & LoRA Finetuning & 0.54 & 0.69 \\
 & Full Finetuning & & \\
\multirow{2}{*}{LLaMA3-8B} & LoRA Finetuning & 0.53 & 0.70 \\
& Full Finetuning & \\
\bottomrule
\end{tabular}
%\vspace{-2mm}
\end{table}
\fi


\begin{figure}[t]
   \centering
    \setlength{\tabcolsep}{0.2pt}
\begin{tabular}{cccc}
        \vspace{-2mm}
        \includegraphics[width=0.22\textwidth, height=2.2cm]{figs/lr-gpt-shortened.png} &
        \includegraphics[width=0.22\textwidth, height=2.15cm]{figs/rank-gpt-shortened.png} &
        \includegraphics[width=0.22\textwidth, height=2.2cm]{figs/temperature-gpt-shortened.png} &
        \includegraphics[width=0.22\textwidth, height=2.15cm]{figs/query-threshold.png} \\ 
        %{\footnotesize (a) Working memory size} &
        %{\footnotesize (b) Novelty threshold} &
        %{\footnotesize (c) Merging sensitivity} \\
\end{tabular}
\vspace{-3mm}
    \caption{Sensitivity of key hyperparameters.}
    \label{fig:sensitivity}
    \vspace{-5mm}
\end{figure}



\subsection{Sensitivity Analyses}
%q, k, $\tau$, learning rate, lora rank
Fig.~\ref{fig:sensitivity} shows the sensitivity of key parameters in \Method.
while the less sensitive ones are omitted due to space limitation.
%We focus on short answer accuracy, as other metrics show similar trends.
The default parameter setting is the same as described in Sec.~\ref{sec:system-implementation}.
We mainly focus on evaluating the short answers to assess the parameters' impact on factual information extraction.

\textbf{Learning Rate in Answer Assembly} 
Fig.~\ref{fig:sensitivity} (leftmost) shows the short answers accuracy for learning rates of $\{1e-5, 5e-5, 1e-4, 2e-4, 5e-4\}$ during finetuning.
Larger learning rates result in bigger gradient steps during LoRA finetuning, with $1e-4$ providing the best performance for our task.

\textbf{LoRA Rank in Answer Assembly}
Fig.~\ref{fig:sensitivity} (middle left) shows the short answers accuracy for LoRA ranks of $\{8, 16, 32, 64\}$ during finetuning.
The rank affects the size of the LoRA adapter weights. Higher ranks mean more parameters and a larger weight space to optimize. 
For our task, varying ranks have little impact on final accuracy, with rank 32 achieving the best performance for short answers.

\textbf{Generating Temperature in Answer Assembly}
Fig.~\ref{fig:sensitivity} (middle right) shows the short answers accuracy for generating temperatures of $\{0.01, 0.1, 0.2, 0.5\}$ during the final answer generation.
Higher temperatures instruct the LLM to use more ``creativity''. For our task, varying temperatures have negligible impact on the short answers accuracy, indicating minimal impact to the sensor information extraction.

\textbf{Query Threshold in Sensor Data Query} Fig.~\ref{fig:sensitivity} (rightmost) shows the F1 scores of online querying at different query thresholds $h=\{0.2, 0.4, 0.5, 0.6, 0.8\}$. %As discussed in Sec.~\ref{sec:ablation}, the higher the BA, the more accurate the activity classification, and the better answer accuracy we get from \Method.
%While increasing the threshold reduces the BA, the reduction is minimal thus \Method is generally robust to various $h$.
Raising the threshold $h$ excludes less confident positive predictions, which can improve F1 scores. However, this may also overlook some detailed events, potentially reducing answer accuracy. Ideally, $h$ should be calibrated individually for each user to achieve the best results.

\iffalse
\textbf{Temperature during answer generation}
Fig.~\ref{fig:sensitivity} (right) shows the final accuracy for temperatures of $\{0.01, 0.1, 0.2, 0.5\}$.
Temperature balances LLaMA's predictability and creativity. A higher temperature encourages exploration and can be helpful in answering creative questions. However, on the short answer dataset, temperature has little impact on final performance.
\fi


\begin{figure}[t]
  \begin{subfigure}[b]{0.55\textwidth}
        \centering
        \includegraphics[width=\textwidth]{figs/legend_split.png}
        \vspace{-5mm}
    \end{subfigure}

    \begin{subfigure}[b]{0.35\textwidth}
        \centering
        \includegraphics[width=\textwidth]{figs/exact_split.png}
        \vspace{-6mm}
        \caption{Average short-answer accuracy on different users.}
        \label{fig:exact_split}
    \end{subfigure} \hspace{0.02\textwidth} % Add horizontal space between subfigures if needed
    \begin{subfigure}[b]{0.35\textwidth}
        \centering
        \includegraphics[width=\textwidth]{figs/acc_split.png}
        \vspace{-6mm}
        \caption{Average online querying accuracy on different users during sensor data query.}
        \label{fig:acc_split}
    \end{subfigure} 
    \vspace{-4mm}
    \caption{Generalization of key learning components in \Method.}
    \label{fig:generalization}
    \vspace{-4mm}
\end{figure}

\subsection{Generalization and Robustness}
\label{sec:generalizability}
We evaluate \Method's generalization and robustness across different users' sensor data and QA inputs. In addition to the standard 80/20 random split, we compare results with a split where training is performed on the first 48 out of 60 users and testing includes all users. To ensure a fair comparison, we equalize the training set size in both splits by duplicating samples in the smaller set.
Our evaluation focuses on two key learning processes in \Method: LLM fine-tuning in answer assembly and sensor-text encoder pretraining.

Fig.~\ref{fig:exact_split} presents the short-answer accuracy when fine-tuning on all users' QA data versus only the first 48 users. The results show similar accuracy for both seen and unseen users, demonstrating \Method's strong generalizability in answer assembly. This is likely due to \Method's design of treating answer assembly as a pure language task. Since all users' language tokens follow similar distributions in a sensor-based QA task, generalization remains robust across user variations.

Fig.\ref{fig:acc_split} presents the online querying accuracy when pretraining with all users' sensor data versus only the first 48 users. Unlike language fine-tuning, limiting sensor data to the first 48 users leads to accuracy degradation on unseen users due to variations in data distributions. Therefore, improving \Method's generalization to new users primarily depends on developing a robust sensor and text encoder, which is a well-studied problem in existing literature~\cite{xu2023practically}. We leave the investigation for combining with these techniques in future work.
%The performance of \Method mainly depends on two learning components: the sensor and text encoder obtained in pretraining, and the finetuned LLM in answer assembly. Therefore we assess the generalization of \Method 

%%%%%%%%%%%%%%%%%%%%%%%%%%%%%%%%%%%%%%%%%%%%%%%%%%%
% Evaluation on real deployment
%%%%%%%%%%%%%%%%%%%%%%%%%%%%%%%%%%%%%%%%%%%%%%%%%%%
\section{Real User Study}
\label{sec:deployment}


We conduct a user study to evaluate \Method's real-world applicability. 
In addition to dataset-based evaluations in Sec.~\ref{sec:evaluation} which focus on quantitative questions, this study places \Method into the ``wilderness''. Participants will interact with \Method immediately after data collection and may pose arbitrary questions, especially the qualitative and open-ended ones.
Our study is approved by the Institutional Review Board Committee.

\textbf{User Study Setup} We recruited eight volunteers (five males and three females) who were instructed to follow their normal daily routines while carrying a smartphone with the ExtraSensoryApp~\cite{vaizman2018extrasensory}. The smartphone models include Huawei Mate 10 Pro, LG G7 ThinQ and Google Pixel 2. The app automatically collected multimodal sensor data and transmitted it to \Method whenever a network connection was available. Reporting activity labels was optional for the participants. The data collection phase lasted one to three days, with valid samples ranging from 52 to 1366 minutes. Sensor data availability varied due to factors such as phone model and usage patterns. After the data collection phase, volunteers were invited to interact with \MethodC in-person through a chat-based graphical interface. They were encouraged to ask any questions about their lives during the data collection phase and observe \MethodC's response generation in real time. Finally, we gathered feedback from the participants including ratings on answer content, latency, and practical value of \Method on a scale from 1 to 5. %, with 1 being the least favorable and 5 being the most favorable. 
%
We specifically select \MethodC for the user study to evaluate \Method's full functionality in real-world conditions. We leave user evaluation and optimization of \MethodE in a purely edge scenario for future work.


\textbf{User Feedback Results}
Fig.~\ref{fig:user_study} displays the quantitative feedback ratings gathered from the eight participants. \textbf{\Method received an average score of 3.12 for answer content, 4.50 for latency, and 4.00 for practical utility}, supporting \Method's applicability in real life scenarios.
Participant comments praised \Method's natural responses, e.g., ``\textit{The answers were formatted in an easy to understand way}''.
However, concerns were also raised about quantitative accuracy, such as ``\textit{some numbers were a little off}'' and ``\textit{it mentioned activities I never did}''.
We attribute these issues to the limited performance of the sensor encoder when generalizing from a dataset to real-world users, as we observed noisy outputs from the sensor query stage, occasionally detecting activities that never occurred.
Fortunately, generalizing models trained on one dataset to different settings is a well-studied problem~\cite{xu2023practically}. Integrating these techniques into \Method could enhance its practical performance, which we leave for future investigation.
%If the sensor encoder does not transfer well to a new user, the resulting queried data and answers can be inaccurate. 

%More data and few-shot transfer learning could help, which we plan to explore in future work. 
In terms of latency, all participants give positive feedbacks regarding the end-to-end answer generation latency, with comments such as "\textit{I do not feel as if I had to wait a long time for the answers}" and "\textit{I think it is faster than I thought previously}".
Most participants are positive in terms of the practical utility of \Method in their daily lives, e.g., ``\textit{I believe it can help with my wellness management significantly.}'' The less positive comments mentioned the challenge of adapting \Method to individual users for generating more personalized and useful responses, an issue that we leave for future exploration.

%i.e., adapting both sensor encoder and the LLM components to better fit specific users' lifestyle and needs, 
%can be attributed to two potential reasons on the model's performance: (1) the sensor encoder may not predict accurately, and (2) the LLM components may not perform optimally for specific participants' requests.

%Based on the detailed comments from participants, \Method gives more convincing answers in qualitative questions such as ``''The quality of \Method's answers can be potentially improved from...

\textbf{Generalization to Qualitative and Open-Ended Questions}
Participants have asked creative and open-ended questions to \Method that do not exist in \Dataset, providing a more comprehensive assessment of \Method's practical utility.
%For example, one participant asked, ``Was I in a crowded area yesterday?''. While it is not instantly obvious which labels or sensors are relevant, \Method is able to reason by decomposing the question into an activity query, and infers that the participant likely visited crowded places based on labels such as "in school," "talking," and "with friends."
For example, one participant asked, ``\textit{How would you rate my lifestyle and what improvements do you suggest?}'' \Method responded with, ``\textit{I would rate this lifestyle as 2 out of 5 stars and would suggest you improve on your exercise and talking,}'' based on the duration of each activity where exercise and talking were lacking.
The three-stage pipeline, especially the explicit sensor data query stage, enables \Method to analyze users' overall lifestyle and fitness, in addition to answering quantitative questions in \Dataset~\citesensorqa. 
Specifically, \Method achieves this through comprehensively querying a list of activity schedule in the sensor data query stage and then analyzing them with LLM during the answer assembly stage.
Participants appreciated \Method's ability to effectively handle both quantitative and qualitative questions, e.g., "\textit{I think it could be very useful to be able to answer these qualitative and quantitative questions about my lifestyle,}" highlighting its broad applicability to diverse real-life scenarios.

%With its LLM-powered backbone, \Method shows notable expertise in answering these kinds of creative questions. %, which clearly advances the capabilities of existing systems.

\begin{figure*}[t]
  \centering
  \includegraphics[width=0.98\textwidth]{figs/user_study.png}
  \vspace{-4mm}
  \caption{Satisfactory ratings from eight participants on three questions.}
  \vspace{-6mm}
  \label{fig:user_study}
\end{figure*}

%%%%%%%%%%%%%%%%%%%%%%%%%%%%%%%%%%%%%%%%%%%%%%%%%%%
% Discussion \& Future Work
%%%%%%%%%%%%%%%%%%%%%%%%%%%%%%%%%%%%%%%%%%%%%%%%%%%
\section{Discussion \& Future Work}
\label{sec:future-work}

%In this section, we discuss various aspects of \Method and potential directions for future work.

\textbf{Generalizability of \Method}
In the quantitative study, we mainly focus on \Dataset~\citesensorqa as it is the only available sensor-based QA dataset that aligns with our problem statement. However, \Method is not limited to this dataset. As demonstrated in the user study (Sec.\ref{sec:deployment}), \Method can generalize to broader real-life scenarios and answer high-level qualitative questions.
Moreover, the design of \Method can be easily expanded.
Specifically, \Method can incorporate more sensors via adding additional modality encoders.
It can also adapt to more diverse questions by expanding the solution templates or using more powerful LLMs such as OpenAI's o3-mini~\cite{openaio3} or DeepSeek~\cite{deepseek} during question decomposition.
Last but not least, \Method can serve a larger user base by collecting more data and equipping with better generalization designs such as~\cite{xu2023practically}. \Method is a general and flexible framework designed for fusing long-duration multimodal sensors knowledge with language and answering versatile questions.


%Generalizability of SensorChat: extension to larger number of users and more sensors. If we have more data available, SensorChat’s performance can be further improved
%Generalizability of SensorChat to more complex questions

\textbf{Personalization of \Method} This work emphasizes \Method's ability to generalize across a broad range of questions and users. However, providing personalized responses is equally crucial, particularly for health-related analyses. We foresee the potential to tailor \Method for individual users by personalizing the LLMs~\cite{kim2024health}, a direction we intend to explore in future research.

\textbf{Evaluation Metrics} \Method uses traditional NLP metrics and accuracy to evaluate the quality and precision of generated answers. While these metrics excel in objectiveness, they do not capture subjective user satisfaction, which is crucial for assessing a system's readiness for a broad market. 
We leave a comprehensive evaluation of \Method with subjective metrics as our future work.



\textbf{Efficiency of \Method} In this work, we demonstrate that \Method can be deployed on Jetson-level edge devices using common quantization techniques. However, further optimization is needed for real-time LLM service on edge devices. The long-term goal for systems like \Method is to run on mobile devices. Recent studies~\cite{liu2024mobilellm, zhuang2024litemoe} have explored enabling real-time LLM serving on mobile devices, and these techniques could be integrated into \Method in future work. Additionally, we recognize the potential synergy between \Method and vector databases~\cite{zhou2024llm} or retrieval-augmented generation~\cite{zhao2024retrieval}. The techniques in these domains can be integrated into \Method to improve query performance and efficiency.
%We discuss the potential to further improve \Method with vector databases~\cite{zhou2024llm} or retrival augmented generation~\cite{zhao2024retrieval} in Sec.~\ref{sec:future-work}.


%%%%%%%%%%%%%%%%%%%%%%%%%%%%%%%%%%%%%%%%%%%%%%%%%%%
% Conclusion
%%%%%%%%%%%%%%%%%%%%%%%%%%%%%%%%%%%%%%%%%%%%%%%%%%%
\section{Conclusion}
\label{sec:conclusion}

Natural interactions between users and multimodal sensors can unlock the full potential of sensor data in real-world applications. However, existing systems struggle to process long-duration, high-dimensional sensor data, often resulting in unsatisfactory answers to long-term questions.
In this paper, we present \Method, the first edge system designed to handle long-duration, time-series sensor data for both qualitative and quantitative question answering. \Method introduces a novel three-stage pipeline including question decomposition, sensor data query and answer assembly.
All stages and interfaces are carefully designed to effectively fuse the natural language and the sensor modality.
Evaluation results and user studies demonstrate \Method's effectiveness in answering a wide range of qualitative and quantitative questions. The cloud version of \Method system supports real-time interactions, while the edge version is optimized to run on Jetson-level devices.
%show that \Method enhances the answer accuracy by up to 26\% compared to state-of-the-art QA systems while achieving an average generating latency of 0.63s on the desktop.
%We further validate the practical value and challenges with \Method in a real user study.

%%
%% The acknowledgments section is defined using the "acks" environment
%% (and NOT an unnumbered section). This ensures the proper
%% identification of the section in the article metadata, and the
%% consistent spelling of the heading.
\begin{acks}
This work was supported in part by National Science Foundation under Grants \#2003279, \#1826967, \#2100237, \#2112167, \#1911095, \#2112665, \#2120019, \#2211386 and in part by PRISM and CoCoSys, centers in JUMP 2.0, an SRC program sponsored by DARPA.
\end{acks}
%%
%% The next two lines define the bibliography style to be used, and
%% the bibliography file.
\bibliographystyle{ACM-Reference-Format}
\bibliography{sample-base}


%%
%% If your work has an appendix, this is the place to put it.
\appendix
\section{Summarization Function Design}
\label{sec:query-function}
%The design of functions must balance the ability to handle various Q\&A types with the complexity of obtaining accurate question decomposition. 

We design the following summarization functions to handle various real-life QA scenarios.
\begin{itemize}
    \item \texttt{CalculateDuration} is used for time comparisons, time queries, and existence questions.
    \item \texttt{DetectActivity} handles activity queries. \texttt{CountingFrequency} and \texttt{CountingDays} handle single- and multi-day counting. 
    \item \texttt{DetectFirstTime} and \texttt{DetectLastTime} are used for specific timestamp queries.
\end{itemize}
%To handle diverse queries, one may need a large number of query functions and arguments.
We emphasize that the function set can be expanded in the future to accommodate a broader range of queries.
However, since the summarization function is selected by the LLM during question decomposition, it is crucial to limit the number of functions to avoid overwhelming or confusing the model.
Fortunately, one strength of \Method is its ability to transform questions into existing functions during question decomposition, eliminating the need to create new ones unnecessarily.
For instance, a question like "What did I do right after waking up?" can be decomposed into calls to \texttt{DetectFirstTime} and \texttt{DetectLastTime}, rather than requiring the creation of a new function such as \texttt{DetectNextActivity}.
To enable this transformation, we include a list of available functions in the prompt, as shown in Fig.~\ref{fig:question-decompose}.
This approach ensures that the current set of six summarization functions is sufficient to effectively handle the vast majority of user queries.






\end{document}
\endinput
%%
%% End of file `sample-sigconf.tex'.
