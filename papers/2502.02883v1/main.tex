%%
%% This is file `sample-sigconf.tex',
%% generated with the docstrip utility.
%%
%% The original source files were:
%%
%% provide a.dtx  (with options: `sigconf')
%% 
%% IMPORTANT NOTICE:
%% 
%% For the copyright see the source file.
%% 
%% Any modified versions of this file must be renamed
%% with new filenames distinct from sample-sigconf.tex.
%% 
%% For distribution of the original source see the terms
%% for copying and modification in the file samples.dtx.
%% 
%% This generated file may be distributed as long as the
%% original source files, as listed above, are part of the
%% same distribution. (The sources need not necessarily be
%% in the same archive or directory.)
%%
%%
%% Commands for TeXCount
%TC:macro \cite [option:text,text]
%TC:macro \citep [option:text,text]
%TC:macro \citet [option:text,text]
%TC:envir table 0 1
%TC:envir table* 0 1
%TC:envir tabular [ignore] word
%TC:envir displaymath 0 word
%TC:envir math 0 word
%TC:envir comment 0 0
%%
%%
%% The first command in your LaTeX source must be the \documentclass
%% command.
%%
%% For submission and review of your manuscript please change the
%% command to \documentclass[manuscript, screen, review]{acmart}.
%%
%% When submitting camera ready or to TAPS, please change the command
%% to \documentclass[sigconf]{acmart} or whichever template is required
%% for your publication.
%%
%%
\documentclass[acmlarge]{acmart}

\usepackage{amsmath,amsfonts} % amssymb
%\usepackage{algorithmic}
\usepackage[ruled,vlined]{algorithm2e}
\usepackage{graphicx}
\usepackage{textcomp}
\usepackage{xcolor}
\usepackage{tabularx}
\usepackage{caption}
\usepackage{subcaption}
\usepackage[inline]{enumitem}
\newcommand{\Ch}{\checkmark}
\newcommand{\X}{$\times$}
\usepackage{bm}
\usepackage{bbm}
\usepackage{url}
\usepackage{amsthm}
\usepackage{wrapfig}
\usepackage{multicol}
\usepackage{multirow}
\newcommand\mycommfont[1]{\small\ttfamily\textcolor{blue}{#1}}
\SetCommentSty{mycommfont}

\usepackage{xspace}
\newcommand{\Method}{SensorChat\xspace}
\newcommand{\MethodC}{SensorChat\textsubscript{C}\xspace}
\newcommand{\MethodE}{SensorChat\textsubscript{E}\xspace}
\newcommand{\Dataset}{SensorQA\xspace}
\newcommand{\citesensorqa}{\cite{sensorqa}\xspace}
\newcommand*{\benjamin}{\textcolor{blue}}
\newcommand*{\lanxiang}{\textcolor{orange}}
% Define colors
\definecolor{myred}{RGB}{170,0,0} % Dark red
\definecolor{myblue}{RGB}{0,0,170} % Dark blue
\definecolor{mygreen}{RGB}{0,100,0} % Dark green


%%
%% \BibTeX command to typeset BibTeX logo in the docs
\AtBeginDocument{%
  \providecommand\BibTeX{{%
    Bib\TeX}}}

%% Rights management information.  This information is sent to you
%% when you complete the rights form.  These commands have SAMPLE
%% values in them; it is your responsibility as an author to replace
%% the commands and values with those provided to you when you
%% complete the rights form.
\setcopyright{acmlicensed}
\copyrightyear{2025}
\acmYear{2025}
\acmDOI{XXXXXXX.XXXXXXX}
%% These commands are for a PROCEEDINGS abstract or paper.
%\acmJournal{}
%\acmVolume{0}
%\acmNumber{0}
%\acmArticle{0}
%\acmMonth{0}


%%
%% Submission ID.
%% Use this when submitting an article to a sponsored event. You'll
%% receive a unique submission ID from the organizers
%% of the event, and this ID should be used as the parameter to this command.
%%\acmSubmissionID{123-A56-BU3}

%%
%% For managing citations, it is recommended to use bibliography
%% files in BibTeX format.
%%
%% You can then either use BibTeX with the ACM-Reference-Format style,
%% or BibLaTeX with the acmnumeric or acmauthoryear sytles, that include
%% support for advanced citation of software artefact from the
%% biblatex-software package, also separately available on CTAN.
%%
%% Look at the sample-*-biblatex.tex files for templates showcasing
%% the biblatex styles.
%%

%%
%% The majority of ACM publications use numbered citations and
%% references.  The command \citestyle{authoryear} switches to the
%% "author year" style.
%%
%% If you are preparing content for an event
%% sponsored by ACM SIGGRAPH, you must use the "author year" style of
%% citations and references.
%% Uncommenting
%% the next command will enable that style.
%%\citestyle{acmauthoryear}


%%
%% end of the preamble, start of the body of the document source.
\begin{document}

%%
%% The "title" command has an optional parameter,
%% allowing the author to define a "short title" to be used in page headers.
\title{\Method: Answering Qualitative and Quantitative Questions during Long-Term Multimodal Sensor Interactions}

%\Method: Time Series Sensor-based Question Answering System for Daily-Life Monitoring

%\Method: Multimodal Sensor-based Question Answering System for Daily-Life Monitoring

% \Method: A Real-Time Question Answering System for Daily-Life Monitoring using Time-Series Sensors

%%
%% The "author" command and its associated commands are used to define
%% the authors and their affiliations.
%% Of note is the shared affiliation of the first two authors, and the
%% "authornote" and "authornotemark" commands
%% used to denote shared contribution to the research.
\author{Xiaofan Yu}
\email{x1yu@ucsd.edu}
\orcid{0000-0002-9638-6184}
\affiliation{%
  \institution{University of California San Diego}
  \city{La Jolla}
  \state{California}
  \country{USA}
}

\author{Lanxiang Hu}
\email{lah003@ucsd.edu}
\orcid{0000-0003-0641-3677}
\affiliation{%
  \institution{University of California San Diego}
  \city{La Jolla}
  \state{California}
  \country{USA}
}

\author{Benjamin Reichman}
\email{bzr@gatech.edu}
\orcid{0009-0004-3854-7930}
\affiliation{%
  \institution{Georgia Institute of Technology}
  \city{Atlanta}
  \state{Georgia}
  \country{USA}
}

\author{Dylan Chu}
\email{dchu@ucsd.edu}
\orcid{0009-0001-5511-3286}
\affiliation{%
  \institution{University of California San Diego}
  \city{La Jolla}
  \state{California}
  \country{USA}
}

\author{Rushil Chandrupatla}
\email{ruchandrupatla@ucsd.edu}
\orcid{0009-0006-5447-8693}
\affiliation{%
  \institution{University of California San Diego}
  \city{La Jolla}
  \state{California}
  \country{USA}
}


\author{Xiyuan Zhang}
\email{xiyuanzh@ucsd.edu}
\orcid{0000-0002-8908-1307}
\affiliation{%
  \institution{University of California San Diego}
  \city{La Jolla}
  \state{California}
  \country{USA}
}

\author{Larry Heck}
\email{larryheck@gatech.edu}
\orcid{0000-0003-3358-6362}
\affiliation{%
  \institution{Georgia Institute of Technology}
  \city{Atlanta}
  \state{Georgia}
  \country{USA}
}

\author{Tajana \v{S}imuni\'{c} Rosing}
\email{tajana@ucsd.edu}
\orcid{0000-0002-6954-997X}
\affiliation{%
  \institution{University of California San Diego}
  \city{La Jolla}
  \state{California}
  \country{USA}
}


%%
%% By default, the full list of authors will be used in the page
%% headers. Often, this list is too long, and will overlap
%% other information printed in the page headers. This command allows
%% the author to define a more concise list
%% of authors' names for this purpose.
\renewcommand{\shortauthors}{}
\newcommand{\xiyuan}[1]{\textcolor{violet}{\textbf{Xiyuan:} #1}}
\newcommand{\lx}[1]{\textcolor{blue}{\textbf{Lanxiang:} #1}}


%%
%% The abstract is a short summary of the work to be presented in the
%% article.
\begin{abstract}
Natural language interaction with sensing systems is crucial for enabling all users to comprehend sensor data and its impact on their everyday lives.
However, existing systems, which typically operate in a Question Answering (QA) manner, are significantly limited in terms of the \textit{duration} and \textit{complexity} of sensor data they can handle.

In this work, we introduce \Method, the first end-to-end QA system designed for long-term sensor monitoring with multimodal and high-dimensional data including time series. \Method effectively answers both qualitative (requiring high-level reasoning) and quantitative (requiring accurate responses derived from sensor data) questions in real-world scenarios. To achieve this, \Method uses an innovative three-stage pipeline that includes question decomposition, sensor data query, and answer assembly.
The first and third stages leverage Large Language Models (LLMs) for intuitive human interactions and to guide the sensor data query process.
Unlike existing multimodal LLMs, \Method incorporates an explicit query stage to precisely extract factual information from long-duration sensor data.
We implement \Method and demonstrate its capability for real-time interactions on a cloud server while also being able to run entirely on edge platforms after quantization. Comprehensive QA evaluations show that \Method achieves up to 26\% higher answer accuracy than state-of-the-art systems on quantitative questions. Additionally, a user study with eight volunteers highlights \Method's effectiveness in handling qualitative and open-ended questions.
\end{abstract}

%, enhanced with in-context learning and few-shot learning.
%Among the three stages, the sensor data query stage makes the key contribution by aligning the sensor embedding proposes a novel contrastive sensor-text pretraining loss and 
%The intermediate sensor data query stage ensures accurate sensor information extraction. 
%%
%% The code below is generated by the tool at http://dl.acm.org/ccs.cfm.
%% Please copy and paste the code instead of the example below.
%%
\begin{CCSXML}
<ccs2012>
   <concept>
       <concept_id>10010147.10010257</concept_id>
       <concept_desc>Computing methodologies~Machine learning</concept_desc>
       <concept_significance>500</concept_significance>
       </concept>
   <concept>
       <concept_id>10010147.10010178.10010179</concept_id>
       <concept_desc>Computing methodologies~Natural language processing</concept_desc>
       <concept_significance>500</concept_significance>
       </concept>
   <concept>
       <concept_id>10010520.10010553.10010562</concept_id>
       <concept_desc>Computer systems organization~Embedded systems</concept_desc>
       <concept_significance>500</concept_significance>
       </concept>
 </ccs2012>
\end{CCSXML}

\ccsdesc[500]{Computer systems organization~Embedded systems}
\ccsdesc[500]{Computing methodologies~Machine learning}
\ccsdesc[300]{Computing methodologies~Natural language processing}

%%
%% Keywords. The author(s) should pick words that accurately describe
%% the work being presented. Separate the keywords with commas.
\keywords{Question Answering, Multimodal Sensors, Large Language Models}
%% A "teaser" image appears between the author and affiliation
%% information and the body of the document, and typically spans the
%% page.

\settopmatter{printacmref=false}

%\received{20 February 2007}
%\received[revised]{12 March 2009}
%\received[accepted]{5 June 2009}

%%
%% This command processes the author and affiliation and title
%% information and builds the first part of the formatted document.
\maketitle


\section{Introduction}\label{sec:intro}

Augmented and Virtual Reality (AR/VR) has made significant advancements in recent years in terms of quality and affordability \cite{meta_motiv1, meta_motiv2}, through the use of machine learning algorithms.
One crucial algorithm is depth estimation from stereo sensors, which plays a vital role in spatial computing, hand tracking \cite{depth_gesture_recog}, and %spatial and
passthrough rendering.
Conventional DNN based stereo depth algorithms use expensive hierarchical processing \cite{stereonet, mobilestereonet, hitnet}, 
% resulting in high storage and processing energy
\je{which are challenging to accelerate on power constrained platforms}.
Increased resolutions and frame rates in newer systems further increase these costs \cite{near_sensor_distributed}.
Consequently, accelerating these networks on AR/VR devices while meeting real-time latency requirements and operating within the energy budget of limited battery devices presents a challenge.

In this work, we propose \textit{\projname{}}, an AR/VR stereo depth system comprising a flexible architecture for processing dynamic Regions of Interest (ROIs), and a comprehensive mapping methodology to optimize ROI processing for energy efficiency while maintaining real-time performance. Our contributions are as follows:

\begin{itemize}[leftmargin=*]
    % \item Analysis of stereo depth compute across variable ROI sizes;
    \item \je{The \textit{SteROI-D Algorithm}, which leverages Region-of-Interest (ROI) Sparsity to reduce per-frame depth extraction cost, and interleaved object detection and tracking to reduce ROI detection cost;}
    % \item A flexible ROI-based stereo depth processing system and a comprehensive mapping methodology to achieve near-ideal average inference energy by balancing resource availability for the largest ROIs while maintaining high efficiency for average ROIs;
    % \item Special Compute Unit (SCU) and multipacket routing to enable real time, high framerate operation; and
    \item \je{Special Compute Units (SCUs) and NoC Multipackets, to address compute and communication challenges in accelerating stereo depth networks;}
    % \item An evaluation of this end-to-end system design across a range of algorithm components and candidate datasets.
    \item \je{\textit{Binned Mapping}, a method for split online-offline algorithm mapping to enable efficient processing for a continuous range of ROI sizes; and}
    \item \je{A design space exploration framework for jointly optimizing an accelerator's SRAM allocation with it's Binned Mapping.}
\end{itemize}

To our knowledge, this is the first study to achieve ROI-based stereo depth processing.
This is also the first work to address variable ROI processing through a mapping-system co-design approach via an efficient design space exploration.
While prior work has exploited ROIs for eye tracking on AR devices, it was limited to a static architecture \cite{eyecod}.
Furthermore, although prior work has also proposed lightweight stereo depth systems for AR devices, they have not leveraged ROI sparsity \cite{tiefenrausch}.
%%%%%%%%%%%%%%%%%%%%%%%%%%%%%%%%%%%%%%%%%%%%%%%%%%%
% Related Work
%%%%%%%%%%%%%%%%%%%%%%%%%%%%%%%%%%%%%%%%%%%%%%%%%%%
\section{Related Work}
\label{sec:related-work}




%\subsection{Question Answering using Sensor Data}
\textbf{Question Answering using Sensor Data}
The QA problem has been extensively studied across various domains, including text~\cite{rogers2023qa}, visual~\cite{schwenk2022okvqa}, medical~\cite{pal2022medmcqa}, and remote sensing~\cite{hu2023rsgpt}. In the sensor domain, early works \textit{AI Therapist}~\cite{nie2022conversational} and its successor CaiTI~\cite{nie2024llm} utilized smart home devices, such as Amazon Echo, to engage in conversations with users and assess mental well-being. 
DeepSQA~\cite{xing2021deepsqa} was the first to benchmark time-series sensor-based QA for human activity recognition. It introduced SQA-GEN, an automated QA generation tool that gathers 1-minute sensor readings and generates valid Q\&A pairs by exhaustively searching six pre-defined question templates. They mainly focused on quantitative questions including time query, counting and action compare.
The authors also evaluated traditional neural network models, including CNNs and LSTMs, finding that the ConvLSTM network with Compositional Attention achieved the highest QA accuracy.
%However, DeepSQA is limited by a restricted variety of questions and answers, formulating the problem as a classification task. In contrast, \Method supports natural question types and varied time lengths.

Recent contributions have pioneered the integration of LLMs for generating more intuitive answers and better interpreting sensor data. Englhardt \textit{et al.}~\cite{englhardt2024classification}, Health-LLM~\cite{kim2024health}, and DrHouse~\cite{yang2024drhouse} converted physiological data from wearable devices, such as heart rates and daily step counts, into text prompts for LLMs, enabling more sophisticated medical and healthcare diagnoses.
The latest Sensor2Text~\cite{chen2024sensor2text} and PrISM-Q\&A~\cite{arakawa2024prism} explored natural language interactions between users and wearable devices to understand and support daily activities using sensor data, such as advising on "What should I do next with this?" Both approaches utilized LLMs as their backbone and fed sensor embeddings into the models.

While promising, existing systems exhibit significant limitations in handling long duration and complex sensor data required for accurate answers. Table~\ref{tbl:related_works} highlights the key differences between \Method and existing QA systems. In summary, \Method greatly enhances capabilities of existing systems by (1) answering questions based on long-duration sensor data spanning weeks or months, compared to the short windows of seconds or minutes in previous systems~\cite{xing2021deepsqa,chen2024sensor2text,arakawa2024prism}, and (2) encoding high-dimensional time-series data to extract fine-grained activity details for reasoning, unlike prior systems that are limited to low-dimensional sensor data~\cite{englhardt2024classification,kim2024health,yang2024drhouse}.
%However, these works use coarse-grained sensor data which cannot be easily adapted to timeseries and detailed quantitative questions.
%\Method instead focuses on analyzing rich timeseries data.
%However, these works have two drawbacks: (1) they merely input text-formatted sensor data into LLM which may not work well for timeseries; (2) their usage of closed-source LLMs such as GPT-4 and external knowledge databases are challenging for edge devices. \Method addresses both drawbacks. 
%However, all above works focused on health-related aspects and used lower dimensional of data compared to raw sensor time series. Their approaches cannot be directly transferred to daily-life activity monitoring with raw time series sensors.

%fused low-dimensional sensors data such as from smart home sensors or wristband with LLMs to infer health status.
%infer mental health and assess the user's daily functioning using GPT-3. %Their approach integrated a GPT-3-based natural language processing core to interact with users as well as detect abnormal mental status, based on 37 dimensions suggested by the therapists.

%considering only the top 27 answers and 

%A major constraint lies in their language models, which predict outputs as a classification problem - such as a 0-2 mental health core in \textit{AI therapist} and only the top 27 answers considered as the ground-truth labels in DeepSQA. This design restricts their models to a highly constrained QA scenario that may not be applicable in real life.
%In contrast, the distinctiveness of \Method lies in its capacity to accommodate "arbitrary" and "unpredictable" questions, namely open-ended question answering.

%\subsection{LLMs for Multimodal Reasoning}
%\vspace{1mm}
\textbf{LLMs for Multimodal Reasoning}
%The surge of LLM has triggered a wave of innovations in text understanding and reasoning applications.
Recent works investigated multimodal LLMs that transform other data modalities into a sequence of tokens for LLM integration~\cite{zhang2023llama,lin2024vila}.
IMU2CLIP~\cite{moon-etal-2023-imu2clip} and TENT~\cite{zhou2023tent} employed contrastive pretraining to align text with various timeseries sensor signals.
%To enable this transformation, IMU2CLIP~\cite{moon-etal-2023-imu2clip} utilized contrastive pretraining to align IMU signals with text narration~\cite{grauman2022ego4d}. Similarly, TENT~\cite{zhou2023tent} employed contrastive pretraining to align text with a broader set of IoT sensor signals, including camera video, LiDAR, and mmWave data. Both approaches have limited sensor-specific reasoning capabilities as they used a frozen LLM without further tuning or did not utilize an LLM at all. 
Recent designs like AnyMAL~\cite{moon2023anymal} and OneLLM~\cite{han2024onellm} proposed fine-tuning multimodal LLMs to process up to eight different modalities, including IMU time series, for reasoning tasks.
However, direct integration of time series sensor data to LLMs is constrained by LLM's inherent weakness in handling long-context inputs~\cite{li2024long,gu2023mamba}, making them unsuitable for processing long-duration sensor signals. In contrast, \Method overcomes this limitation by incorporating a dedicated sensor data query stage, enabling it to handle long-term queries effectively.
%requires a large instruction dataset with well-aligned multimodal data, which is lacking in multimodal sensor-based QA. 
%that accept IMU signal inputs using carefully prepared instruction datasets.

%\begin{wraptable}{r}{0.55\textwidth}
%\vspace{-4mm}
\begin{table}
\small
\caption{Comparing {\Method} and existing QA systems.}
\label{tbl:related_works}
\vspace{-4mm}
\begin{center}
\begin{tabular}{ccc} % note: no vertical bars at all
\toprule
\textbf{Existing QA systems for sensor data} & \textbf{Long-duration} & \textbf{High-dimensional} \\
& \textbf{sensor data} & \textbf{time series sensors} \\
\midrule
DeepSQA~\cite{xing2021deepsqa}, Sensor2Text~\cite{chen2024sensor2text}, PrISM-Q\&A~\cite{arakawa2024prism} & \X & \Ch \\
Englhardt \textit{et al.}~\cite{englhardt2024classification}, Health-LLM~\cite{kim2024health}, DrHouse~\cite{yang2024drhouse}& \Ch & \X \\ \midrule
 \textbf{\Method (this work)} & \textbf{\Ch} & \textbf{\Ch} \\
\bottomrule
\end{tabular}
\end{center}
\vspace{-4mm}
\end{table}
%\end{wraptable}


\iffalse
\begin{figure*}[htbp]
    \centering
    \begin{subfigure}[b]{0.48\textwidth}
        \centering
        \includegraphics[width=\textwidth]{figs/frequency_query.png}
        \vspace{-6mm}
        \caption{Example question of frequency query.}
        \label{fig:daily-freq}
    \end{subfigure}
    \hfill
    \begin{subfigure}[b]{0.48\textwidth}
        \centering
        \includegraphics[width=\textwidth]{figs/time_query.png}
        \vspace{-6mm}
        \caption{Example question of time query.}
        \label{fig:daily-time}
    \end{subfigure}
    
    \begin{subfigure}[b]{0.48\textwidth}
        \centering
        \includegraphics[width=\textwidth]{figs/day_query.png}
        \vspace{-6mm}
        \caption{Example question of day query.}
        \label{fig:weekly-day}
    \end{subfigure}
    \hfill
    \begin{subfigure}[b]{0.48\textwidth}
        \centering
        \includegraphics[width=\textwidth]{figs/activity_query.png}
        \vspace{-6mm}
        \caption{Example question of activity query.}
        \label{fig:weekly-activity}
    \end{subfigure}
    \vspace{-4mm}
    \caption{Example QA pairs that we collect from AMT. (a) and (b) are generated from daily graph, while (c) and (d) are generated from weekly graph.}
    \label{fig:example_qas}
    \vspace{-4mm}
\end{figure*}
\fi





\iffalse
Limited prior research has studied finetuning LLMs for specific reasoning tasks. Based on IMU2CLIP, AnyMAL~\cite{moon2023anymal} further finetuned the LLM by training projection layers or applying Low-Rank Adaptation (LoRA).
Nevertheless, the general fine-tuning approach of AnyMAL struggles to accurately address sensor-specific queries, such as those pertaining to specific times and locations. These challenges are effectively tackled in \Method through the introduction of two novel fine-tuning techniques, setting \Method apart from existing methodologies.
\fi

%\subsection{Human Activity Monitoring}
%\vspace{1mm}
\textbf{Mobile Systems for Daily Life Monitoring}
Researchers have explored a variety of sensing technologies for monitoring human lives, including built-in sensors on smart phones~\cite{zhang2020pdlens}, cameras~\cite{radu2019vision2sensor}, Wi-Fi signals~\cite{yang2024mm} and mmWave radar~\cite{weng2024large}.
Although these efforts have resulted in numerous open-source datasets~\cite{misc_human_activity_recognition_using_smartphones_240,misc_mhealth_dataset_319,misc_opportunity_activity_recognition_226,vaizman2017recognizing} and powerful machine learning models~\cite{ma2019attnsense,xu2021limu,deldari2022cocoa,ouyang2022cosmo,ouyang2023harmony,xu2023practically}, they fail to handle queries in natural language, which are more creative and open-ended.
%the vast majority of the existing work has focused on human activity recognition. %, a \textit{passive} way of processing information where the user has no control. 
%How to further interpret the data beyond activity classification has remained less explored.
In this paper, we design \Method to facilitate long-term QA interactions based on time series sensor data.
The methodology of \Method can be further integrated with other sensing modalities and applications.
% \section{The Rationale of DPTS}
\section{Method Rationale}

\label{sec:motivation}

% \ycj{Rationale?}
% \ycj{Pre-analysis/Motivation and Problem Definition/Motivation and Challenge Analysis/Pilot Study?}
 
% \ycj{Add a short overview statement here.}


{
In this section, we present empirical findings that highlight the key challenges of tree search in LLM and provide the rationale behind our proposed DPTS. 
% First, the inherent sequential nature of tree search complicates parallel execution, leading to irregular node expansions and varying path lengths. Second, excessive exploitation of low-confidence paths wastes computational resources, suggesting that a confidence-based pruning strategy could mitigate this inefficiency. Finally, tree search methods that prioritize breadth often suffer from frequent switching between paths, hindering deep exploitation and resulting in token and expansion redundancy. Addressing these challenges can significantly enhance the efficiency and effectiveness of tree search methods.
}
First, the frequent switch between paths complicates parallel execution and causes shallow thinking, disrupting the model’s ability to engage in efficient deep reasoning (Sec.~\ref{sec:3.1}). 
% sequential nature of tree search complicates parallel execution, leading to irregular node expansions and varying path lengths. Existing tree search algorithms frequently switch between paths, causing shallow thinking and disrupting the model’s ability to engage in deep reasoning, ultimately degrading generation quality. 
Second, excessive exploitation of low-confidence paths results in redundant rollouts and wastes effort on fewer possible candidates (Sec.~\ref{sec:3.2}). 
% consuming unnecessary computational resources. Without an effective pruning strategy,..., reducing overall efficiency and slowing down search convergence.

% \subsection{Unpredictable Growth Behavior}
\subsection{Frequent Switching}
\label{sec:3.1}

\begin{figure}[t]
    \centering
    \includegraphics[width=0.93\linewidth]
    {figs/draw_switch_times.pdf}
    \vspace{-0.1in}
    \caption{Statistics for switch from the best path to the suboptimal (blue), and total switch (green).} 
    \vspace{-0.1in}
    \label{fig:motivation_switch_path}
\end{figure}
% \ycj{Analysis of why vanilla solution doesn't work well?}

% Tree search has become one of the key paradigms for enabling deep reasoning in large language models. However, the inherent sequential nature of tree structures presents significant challenges for GPU parallelism. Regardless of the search algorithm employed, the hierarchical and distributed nature of tree-structured data introduces intrinsic difficulties in memory management—whether using a contiguous memory pool or fragmented memory blocks, the irregularity of tree search remains unavoidable.  % 这一段说的 tree-structured data 难以 parallel

Tree search inherently exhibits retrospective and recursive behaviors, making efficient parallel execution difficult. Even if each node is constrained to generate the same number of tokens, the focus switching between different reasoning trajectories and the diverse path lengths makes it incompatible with the end-to-end parallelism on GPUs. The detailed illustrations for this phenomenon can be found in Appendix~\ref{sec:app:trees}. 

% This behavior demonstrates the fundamental characteristics of tree search that make the efficient parallelization a challenging problem: \textit{diverse path lengths, varying child depths, irregular node jumps, and recursive retrospective exploitation}. 

The focus switching between paths also makes the tree search fail in focused reasoning trajectory~\cite{wang2025thoughts}, which prevents deep thinking and leads to a tendency of shallow exploitation. We quantify the switch times of the reasoning focus on each sample in the Math500 dataset. Figure~\ref{fig:motivation_switch_path} counts the total switch, which is about 35 on average. As well as the switch from the best path to a suboptimal or incorrect one, which is up to 3 times for a single sample. It demonstrates the instability of the tree search algorithm in maintaining a focused reasoning trajectory. 


% 下面这段我放 appendix 了
\jwt{
% Tree search is key for enabling deep reasoning in large language models, but its sequential nature presents challenges for efficient GPU parallelism. Tree-structured data introduces difficulties in memory management, with irregular patterns in node expansions and varying path lengths. This leads to difficulties in maintaining parallelism, as tree search is inherently recursive and retrospective, causing inefficient execution when different paths vary in depth and termination points.

% Figure~\ref{fig:motivation_dfs_trees} visualizes depth-first search (DFS) trees, highlighting the irregular expansion process. Darker nodes represent high-confidence paths, while lighter nodes indicate lower-confidence ones. The figure demonstrates that tree search does not follow a predictable spatial or hierarchical pattern, resulting in diverse path lengths, varying child depths, and irregular node jumps. These behaviors complicate parallel execution and efficient resource utilization.

% To address these challenges, it is crucial to focus on improving the handling of irregular growth in tree structures. By introducing strategies that can better manage the dynamic nature of tree search, we can optimize memory usage and reduce the inherent complexity, enabling more effective parallelization.

}


% These findings highlight a critical shortcoming of BFS: frequent switching between different reasoning paths prevents deep exploitation, leading to significant computational redundancy. This behavior results in excessive token generation, unnecessary expansions on suboptimal paths, and instability in maintaining a coherent line of reasoning—factors that ultimately hinder search efficiency and inference speed.

% 下面这段我放 appendix 了
% \jwt{
%  Tree search focusing on width expansion explores a wide range of paths but suffers from frequent switching between them, preventing deep reasoning and leading to shallow exploitation. This behavior causes two inefficiencies: incomplete reasoning and excessive expansions. These algorithms often generates more tokens and expansions than necessary, exploring many suboptimal paths before finding the best one, which results in significant computational redundancy.

%  Figure~\ref{fig:motivation_bfs_trees} shows how BFS expands in a flat, top-down manner, leading to shallow exploitation. Figure~\ref{fig:motivation_waste_tokens}(left) compares the total tokens generated (blue line) to those required for the best path (yellow line), revealing excessive token redundancy. Figure~\ref{fig:motivation_waste_tokens}(right) highlights unnecessary node expansions, where many explored nodes do not contribute to the final solution. Finally, Figure~\ref{fig:motivation_switch_path} analyzes node-switching frequency, showing that BFS frequently shifts between optimal and suboptimal paths, leading to instability in reasoning.

%  The high frequency of path-switching can be mitigated by introducing mechanisms that help the model maintain focus on the most promising paths.
% }



\subsection{Redundant Exploration}
\label{sec:3.2}


The lack of early termination in existing tree search algorithms leads to excessive exploitation and redundant searching. Observations in Figure~\ref{fig:motivation_low_confidence} show that low-confidence nodes rarely contribute to the best solutions, either terminated with suboptimal results (yellow) or failing to be the first to reach the best path (orange). The average probability of the suboptimal results brought by low confidence is 91.3\%, while the probability of those nodes being the earliest best path is only 6.2\%. It suggests that low-confidence nodes have little potential to reach the best solution, it is even hard to be the first one. It means that most low-confidence nodes have less contribution to the final results but waste computational resources. 
% We reorder the nodes based on the former probability for clearer illustration (the plot with original order is showcased in Appendix~\ref{app:sec:low_reward_original}). 

% One of the fundamental inefficiencies in traditional tree search lies in its inability to terminate early when exploring suboptimal paths. In most cases, a search path is only abandoned when it reaches the termination condition, regardless of whether it is already apparent that the path is unlikely to yield an optimal solution. This behavior can lead to substantial computational redundancy, as a large number of unnecessary expansions and token generations are performed on low-confidence nodes. 


% To quantify the extent of this redundancy, we conducted an observational experiment. In Figure~\ref{fig:motivation_low_confidence}, we analyze the probability that continuing the rollout from a low-confidence node leads to less contribution. Here, we define a node as low confidence if its score is lower than the average confidence of previously visited nodes (refer to Eq. (\ref{eq:theta}))—a deliberately aggressive threshold since such occurrences are relatively frequent (highlighted in yellow). And we also reorder the nodes based this probability for clearer illustration (the plot with original order is showcased in Appendix~\ref{app:sec:low_reward_original}). However, as our observations indicate, the probability that these low-confidence nodes ultimately contribute to the optimal path is extremely low (highlighted in orange). Additionally, in most cases where the optimal path is eventually reached, it is not the first time an optimal solution is discovered (highlighted in blue), meaning that a higher-confidence node had already identified the correct path earlier. 

% Between the yellow and orange regions lies an additional scenario: the probability that continuing a rollout from a low-confidence node either fails to generate an answer at termination or produces an incorrect answer. This category accounts for the majority of cases, further reinforcing the inefficiency of expanding low-confidence paths.

% Based on these observations, it suggest that a node’s prior confidence may serve as a reasonable predictor of whether the search path will ultimately lead to a valid solution. While this does not guarantee a perfect pruning strategy, it indicates that integrating confidence-based heuristics could significantly reduce unnecessary rollouts, improving the overall efficiency of tree search methods.

% 合并到上面的黑色文本里了
\jwt{
% A major inefficiency in traditional tree search is the lack of early termination when exploring low-confidence paths. This results in wasted computational resources, as the algorithm continues expanding these paths even when their confidence scores remain low. Observations show that low-confidence nodes rarely contribute to optimal solutions, suggesting that confidence-based pruning can reduce unnecessary exploitation and improve efficiency.

% Figure~\ref{fig:motivation_dfs_trees} (Tree 2) illustrates the issue, showing that low-confidence nodes (rightmost branches) are expanded despite their low likelihood of contributing to the final answer. Figure~\ref{fig:motivation_low_confidence} further quantifies this inefficiency: yellow regions indicate frequent occurrences of low-confidence nodes, while orange regions show that they rarely lead to optimal solutions. The blue regions reveal that even when the best path is found, a higher-confidence node had typically identified it earlier, confirming the redundancy of expanding low-confidence nodes.

% These findings emphasize the importance of selectively pruning low-confidence paths early in the search process. By incorporating mechanisms that assess the likelihood of a node contributing to the optimal solution, unnecessary expansions can be avoided, leading to a significant reduction in computational overhead.
}

\begin{figure}[t]
    % \centering
    \includegraphics[width=0.85\linewidth]{figs/low_reward_original.pdf}
    \vspace{-0.1in}
    \caption{Probabilities with reordered samples of those have prior confidence below $\theta_{es}(\lambda=1)$ in Eq.~\ref{eq:theta} and do not terminate with the highest reward score (yellow), and paths that are not the earliest best path (orange), which means there is already at least one path that has terminated with the same reward score. }
    \vspace{-0.2in}
    \label{fig:motivation_low_confidence}
\end{figure}

These findings emphasize the importance of maintaining the focus on deep reasoning and pruning low-confidence paths for efficient inference. 


\section{Proposed Method}
\label{sec:method}
To address the aforementioned challenges, we propose an innovative framework that allows for efficient reasoning, termed Dynamic Parallel Tree Search (DPTS). 
% The proposed parallel framework serves as the foundation of DPTS. Furthermore, the search and the transition mechanism enables efficient and adaptive exploitation. 
In the generation phase, the Parallelism Streamline in Sec.~\ref{sec:method_parallel} supports fine-grained and flexible paralleled expansion for arbitrary paths. 
In the selection phase, the Search and Transition Mechanism in Sec.~\ref{sec:search_and_transition} enables less redundant exploration by identifying the highly potential solutions to focus reasoning. 

% \begin{algorithm}[ht]
% \caption{Algorithmic process DPTS}
% \label{alg:algorithm_parallel}
% \begin{algorithmic}[1]
% \REQUIRE LLM generation function $llm(x)$, PRM reward function $prm(x)$, Query $q$, Candidate Node Pool $N = \varnothing$, Parallel Queue $P = \varnothing$, Exploit Node Proportion $p$, Tree Width $w$. \\
% \ENSURE End node with best path reward $n^*$.

% {\color{ForestGreen}{// Step 1: Initialize the root node}} \\
% \STATE $r \gets \text{generate\_node}(q, \mathrm{None})$
% \STATE $N \gets N \cup \{r\}$

% \WHILE{\text{within computational budget}} 

%     % {\color{ForestGreen}{// Step 2: Adjust Parallel Queue}} \\
%     \STATE $ P_\text{size} \gets \text{Eq. (\ref{eq:queue_size})}$  \\
    
%     {\color{ForestGreen}{// Step 3: Perform searching}} \\
%     \STATE $P \gets \text{Search}(P, P_\text{size}, N)$ (Algorithm~\ref{alg:searching})
%     \STATE $\theta_{\mathrm{es}}, \theta_{\mathrm{ds}} \gets \text{Eq. (\ref{eq:theta})}$ \\
    
%     {\color{ForestGreen}{// Step 4: Parallelism by Eq. (\ref{eq:kv}) and (\ref{eq:seq})}} \\
%     \STATE $\mathbf{n} \gets \text{generate\_node}(
%     \mathrm{Seq}^{all}, \mathrm{KV}^{all})$ \\ 
%     {\color{ForestGreen}{// Step 5: Update new nodes by Eq.~(\ref{eq:n_new})}} \\
%     % \STATE $\mathrm{KV}^{\text{all}}, \mathrm{Seq}^{\text{all}} \gets \text{Eq. (\ref{eq:kv}), Eq. (\ref{eq:seq})}$
%     % \STATE $\mathbf{n} \gets \mathtt{generate\_node}(\mathrm{Seq}^{\text{all}}, \mathrm{KV}^{\text{all}})$
%     {\color{ForestGreen}{// Step 6: Terminate and reward}} \\
%     \STATE $N\gets \text{Reward}(P, N)$ (\text{Algorithm~\ref{app:algo:reward}}) \\
%     {\color{ForestGreen}{// Step 7: Perform transition}} \\
    
%     \STATE $P \gets \text{Transition}(P, \theta_{\mathrm{es}}, \theta_{\mathrm{ds}})$ (\text{Algorithm~\ref{alg:transition}})
% \ENDWHILE

% \RETURN $\max_{\text{reward}}(\forall n \in N)$
% \end{algorithmic}
% \end{algorithm}

\begin{figure*}
    \centering
    \includegraphics[width=\linewidth]{figs/overview.pdf}
    \vspace{-0.35in}
    \caption{Overview of the proposed DPTS framework. The right part demonstrates the Parallelism Streamline, while the left and middle illustrate the proposed Search and Transition Mechanism. }
    \label{fig:overview}
    \vspace{-0.1in}
\end{figure*}

\subsection{Parallelism Streamline}
\label{sec:method_parallel}

% As the foundation of our framework, we fully parallelize the generation process to maximize GPU utilization. Thus, we first introduce the parallel execution architecture, as illustrated in the right side of Figure~\ref{fig:overview}. It consists of three phases: Adaptive Parallelism Queue, which has dynamically adjustable length determined by the available GPU memory; Data Collection and Preparation for parallel inference; Generation and Updating for efficient storage. Additional details can be found in Appendix~\ref{app:sec:parallel_reasoning} due to the limited length.  
% The core component of this architecture are the parallel queue $P$, which holds the currently selected optimal nodes; candidate node pool $N$, which collects all the unexpanded nodes; the data structure of each node. 

% 

% The selection process will be detailed in later sections on searching and transition mechanisms.


% \begin{algorithm}[ht]
% \caption{Process of EETS}
% \label{alg:algorithm_parallel}
% % \resizebox{\linewidth}{!}{
% \begin{algorithmic}[1]
% \REQUIRE Query $q$, candidate node pool $N=\emptyset$, parallel queue $P=\emptyset$, exploit node proportion $p$, tree width $\mathtt{tw}$. \\ 
% \ENSURE  Node with best path $n^*$. \\ 
% % \FOR {$s \gets S$}
%     \STATE $t_0 = \mathtt{time()}$
    
%     {\color{ForestGreen}{// Initializing root node $r$}} 
%     \STATE $r \gets \mathtt{generate\_node}(q, \mathrm{None})$
%     \STATE $N \gets N \cup \{r\}$
    
%     \WHILE{$\mathtt{time()}-t_0<t_\mathrm{{max}}$} 
%     \STATE $\mathtt{queue\_size} \gets \frac{1-O_{init}}{O_{peak} - O_{init}}$  
%     % {\color{ForestGreen}{// Adaptive parallelism number}} 
    
%     % \IF{$|P|<\mathtt{queue\_size}$}
%     % \STATE $N'\gets$ Descending $N$ based on conf. \\
%     % % , $\forall n \in N$
%     % {\color{ForestGreen}{// EE Searching}} \\
%     % \STATE $P \gets P \cup N'[:\mathtt{queue\_size}-|P|]$
%     % \STATE $P'\gets$ Descending $P$ based on conf.
%     % % , $\forall n \in P$
%     % \STATE $n$.mode $\gets \mathtt{EXPLORE}$, $\forall n \in P'[:p|P|]$
%     % \STATE $n$.mode $\gets \mathtt{EXPLOIT}$, $\forall n \in P'[p|P|:]$
%     % \ENDIF
%     {\color{ForestGreen}{// E\&E Searching}} \\
%     \STATE $P\gets$ Algorithm\ref{alg:searching}$(P$, $\mathtt{queue\_size}, N)$ \\
%     % 更新 theta
%     \STATE $\theta_{early\_stop}, \theta_{deep\_seek} \gets $ Eq. (\ref{eq:theta})
    
%     {\color{ForestGreen}{// E\&E Parallelism}} \\
%     % data preparation & generation
%     \STATE $\mathrm{KV}^{all}, \mathrm{Seq}^{all} \gets$ Eq. (\ref{eq:kv}), Eq. (\ref{eq:seq})
%     \STATE $\mathbf{n} \gets \mathtt{generate\_node}(\mathrm{Seq}^{all}, \mathrm{KV}^{all})$
    
%     % updating new node
%     \FORALL{$n \in P$}
%         \STATE $n$.children $\gets$ Eq. (\ref{eq:n_new}) 
%         \FORALL{$m\in n$.children}
%             \IF{$\mathtt{is\_terminate}(m)$}
%                 \STATE $m$.reward$\gets\mathtt{reward}(m)$
%             \ELSE
%                 \STATE $N\gets N\cup \{m\}$ \\ 
%             \ENDIF
%         \ENDFOR
%         % \STATE $a^i \gets P'[i]$
%         % \FOR{$j \gets 1$ to $\mathtt{tw}$}
%         % \STATE $\mathrm{KV}^{n^{ij}}\gets \mathbf{n}.\mathtt{past\_kv}_{[ij, \dots, |\mathrm{KV}^{all}|:]}$
%         % \STATE $\mathrm{Seq}^{1\sim n^{ij}}\gets \mathbf{n}.\mathtt{output}_{[ij]}$
%         % \STATE $n^{ij}\gets[id, a, \mathrm{KV}^{n^{ij}}, \mathrm{Seq}^{1\sim n^{ij}}]$
%         % \ENDFOR
%         % \STATE $a^i$.children$\gets[n^{i1}, \dots, n^{i\mathtt{tw}}]$
%     \ENDFOR
%     % updating N and transition
    
%     {\color{ForestGreen}{// E\&E Transition}} \\
%     \STATE $P \gets$Algorithm\ref{alg:transition}$(P, \theta_{early\_stop}, \theta_{deep\_seek})$
%     % \FORALL{$n \in P$}
%     % \STATE $n^*=\max_\mathrm{conf.}$($n$.children)
%     % \IF{$n$.mode $=$ $\mathtt{EXPLORE}$ \AND $n^*$.conf. $<\theta_{early\_stop}$ \OR $n$.mode $=$ $\mathtt{EXPLOIT}$ \AND $n^*$.conf. $<\theta_{deep\_seek}$}
%     %     % \IF{}
%     %     \STATE $P \gets P \setminus n$
%     %     \ELSE
%     %     \STATE $P \gets P\cup \{n^*\}$
%     %     \STATE $n^*$.mode $\gets \mathtt{EXPLORE}$ 
%     % \ENDIF
%     % \ENDFOR
%     % \FORALL{$n \in P$}
%     % \IF{$\mathtt{is\_terminate}(n)$}
%     % \STATE $P\gets P \setminus n$
%     % \STATE $n$.reward$\gets\mathtt{reward}(n)$
%     % \ENDIF
%     % \ENDFOR
% \ENDWHILE
% \RETURN $\max_\mathrm{reward}(\forall n \in N)$
% \end{algorithmic}
% \end{algorithm}

% \jwt{
As illustrated in Figure~\ref{fig:overview}, We fully parallelize the tree search process in our framework with three main components: \textbf{Tree Structure Building}, \textbf{KV Cache Handling}, and \textbf{Adaptive Parallel Generation}. Each component is designed to optimize memory usage and parallel execution during the reasoning process. 



\subsubsection{Tree Structure Building}
The tree search framework relies on a tree structure where each node represents a reasoning state.  Specifically, the node data structure includes the following elements:
\begin{itemize}[itemsep=0pt,parsep=0pt,left=0pt,topsep=0pt]
    \item \textbf{Node ID}: A unique identifier for each node.
    \item \textbf{Parent Node}: A reference to the parent node, establishing the hierarchical structure of the tree.
    \item \textbf{Prior Confidence}: The confidence of the node, based on prior knowledge and model predictions.
    % \item \textbf{Posterior Reward}:
    \item \textbf{Key-Value Cache} (\(\text{KV}^n\)): The key-value cache specific to this node, storing intermediate results during the reasoning process.
    \item \textbf{Token Sequence} (\(\text{Seq}^{1 \sim n}\)): The complete token sequence from the root node to the current node, representing the reasoning path taken so far.
\end{itemize}
The key challenge lies in managing the KV cache~\cite{floridi2020gpt}. Instead of storing the entire sequences, each node only retains its own KV cache. This significantly reduces memory usage, particularly when dealing with a large number of nodes in the tree. By keeping each node's cache isolated, we avoid redundant memory usage while ensuring that each node has necessary information to continue reasoning process.

\subsubsection{KV Cache Handling}

The KV cache for each node is stored separately, and during parallel execution, these caches need to be collected and concatenated for efficient parallelism. The key challenge is that tree search paths have varying lengths, which means that both the KV caches and the input sequences for different nodes will vary in size and be hard to parallel. 

To address this, we use a simple but straightforward padding technique to ensure that all sequences have consistent lengths before being processed. Specifically, for nodes with shorter KV caches, we apply left padding with zeros. Similarly, input sequences are padded with a predefined padding token to match the longest sequence in the batch. 
This padding ensures that all nodes are processed in parallel with consistent sequence lengths and corresponding KV cache, allowing for efficient batch processing across the tree search. \dyf{The details of padding and concatenating are given in Appendix Eq. (\ref{eq:kv}) and (\ref{eq:seq}).}

Besides data collecting and preparation, we also clean up the useless KV cache either the leaf node is terminated, or all the children's branches are exploited and finished. In this way, we release the memory, making room for new reasoning paths. 

\subsubsection{Adaptive Parallel Generation}
To further utilize the computational resources, we introduce an adaptive parallelism queue, which dynamically adjusts the number of parallel paths based on the available GPU memory. The parallelism queue size, denoted \( |P| \), is used to restrict the number of exploitation and exploration paths in Sec.~\ref{sec:method_searching}. It is calculated by the available and the peak memory usage during previous generations: 
% $|P| = \frac{O_{\text{max}} - O_{\text{init}}}{O_{\text{peak}} - O_{\text{init}}},$ \label{eq:queue_size}
\begin{equation}
\label{eq:queue_size}
|P| = \frac{O_{\text{max}} - O_{\text{init}}}{O_{\text{peak}} - O_{\text{init}}},
\end{equation}
where $O_{\text{max}}$ is the total memory budget, \( O_{\text{peak}} \) represents the peak memory usage from the previous generation, and \( O_{\text{init}} \) is the memory consumption during model initialization. 
\dyf{As the tree grows, the memory occupation of intermediate results continues to increase even with KV cache cleaning. Since memory overflow is one of the termination conditions, it is important to adaptively adjust the parallel number, preventing excessive memory allocation and early termination.} 
% mechanism ensures that the number of parallel paths is adjusted by the available memory resources, preventing excessive memory allocation and early termination of searching process}. 
 

After the generation phase, the newly generated sequences and KV caches are stored based on the tree width. The sequences for each node are completely stored, while the KV caches are stored partially with only the new tokens generated at this step (details are provided in Appendix~\ref{eq:n_new}). 
The new nodes are then added to the candidate node pool \( N \), where they will be available for subsequent selection processes in the tree search.

% \hspace{0pt}
\vspace{0.05in}

In summary, our Parallelism Streamline is a well-structured streamline to optimize both memory usage and parallel execution. 
% By carefully engineering the tree structure, KV cache, and parallel generation processes, we ensure that the reasoning capabilities of LLM are significantly enhanced while maintaining computational efficiency. 
The overall process is showcased in Algorithm~\ref{alg:algorithm_parallel}. More details can be found in Appendix~\ref{app:sec:parallel_reasoning} due to the limited length. 


\begin{algorithm}[ht]
\caption{Algorithmic process DPTS}
\label{alg:algorithm_parallel}
\begin{algorithmic}[1]
\REQUIRE LLM generation function $llm(x)$, PRM reward function $prm(x)$, Query $q$, Candidate Node Pool $N = \varnothing$, Parallel Queue $P = \varnothing$, Exploit Node Proportion $p$, Tree Width $w$. \\
\ENSURE End node with best path reward $n^*$.

{\color{ForestGreen}{// Step 1: Initialize the root node}} \\
\STATE $r \gets \text{generate\_node}(q, \mathrm{None})$
\STATE $N \gets N \cup \{r\}$

\WHILE{\text{within computational budget}} 

    % {\color{ForestGreen}{// Step 2: Adjust Parallel Queue}} \\
    \STATE $ P_\text{size} \gets \text{Eq. (\ref{eq:queue_size})}$  \\
    
    {\color{ForestGreen}{// Step 2: Perform searching}} \\
    \STATE $P \gets \text{Search}(P, P_\text{size}, N)$ (Algorithm~\ref{alg:searching})
    \STATE $\theta_{\mathrm{es}}, \theta_{\mathrm{ds}} \gets \text{Eq. (\ref{eq:theta})}$ \\
    
    {\color{ForestGreen}{// Step 3: Parallelism by Eq. (\ref{eq:kv}) and (\ref{eq:seq})}} \\
    \STATE $\mathbf{n} \gets \text{generate\_node}(
    \mathrm{Seq}^{all}, \mathrm{KV}^{all})$ \\ 
    {\color{ForestGreen}{// Step 4: Update new nodes by Eq.~(\ref{eq:n_new})}} \\
    % \STATE $\mathrm{KV}^{\text{all}}, \mathrm{Seq}^{\text{all}} \gets \text{Eq. (\ref{eq:kv}), Eq. (\ref{eq:seq})}$
    % \STATE $\mathbf{n} \gets \mathtt{generate\_node}(\mathrm{Seq}^{\text{all}}, \mathrm{KV}^{\text{all}})$
    {\color{ForestGreen}{// Step 5: Terminate and reward}} \\
    \STATE $N\gets \text{Reward}(P, N)$ (\text{Algorithm~\ref{app:algo:reward}}) \\
    {\color{ForestGreen}{// Step 6: Perform transition}} \\
    
    \STATE $P \gets \text{Transition}(P, \theta_{\mathrm{es}}, \theta_{\mathrm{ds}})$ (\text{Algorithm~\ref{alg:transition}})
\ENDWHILE

\RETURN $\max_{\text{reward}}(\forall n \in N)$
\end{algorithmic}
\end{algorithm}

% The dynamic adjustment of parallelism, coupled with effective padding strategies and efficient node management, allows for scalable and high-performance inference in complex tree search scenarios.  



% \textbf{Data Structure.} For preliminary, we first organize every thought steps as nodes of ToT. Each node in the parallel reasoning process is represented by a data structure that maintains essential information for efficient tree search. This structure includes the node's unique identifier, a reference to its parent node, prior confidence scores, the complete sequence of tokens from the root to the current node, and the node-specific key-value (KV) cache. To optimize memory usage, we avoid redundant storage of the entire sequence of past KV caches, retaining only the KV cache specific to each node. This ensures that the memory consumption is minimized without sacrificing the ability to track relevant state information.

% \The core of our parallel reasoning approach lies in the dynamic management of the parallelism queue, which plays a central role in balancing memory usage and computational efficiency. The size of this queue is adaptively adjusted based on the available GPU memory, ensuring that the framework can handle varying memory demands throughout the inference process. Specifically, 
% % the queue size is determined by the peak memory usage observed during prior generations, allowing us to dynamically allocate computational resources while avoiding memory overload.

% For data preparation phrase, to handle the variable path lengths in the tree search, input data is prepared by padding both the KV caches and input sequences to achieve uniform dimensions. Since the path lengths can differ significantly between nodes, we perform padding on both the KV caches and input sequences to ensure consistent input dimensions. The padding process involves left-padding the KV caches for shorter paths and right-padding the input sequences, resulting in a unified matrix that can be processed in parallel by the large language model (LLM). This approach allows the model to handle varying sequence lengths efficiently without introducing unnecessary complexity.

% Once the data is prepared, the framework proceeds to the generation and updating stage. In this phase, new output sequences and their corresponding KV caches are generated for each node in parallel. The results are then partitioned based on the tree width, with each partition containing only the newly generated information. This segmentation minimizes the memory overhead by retaining only the relevant KV cache segments associated with the new tokens. The output sequences, on the other hand, are stored in their entirety to avoid fragmentation, which would otherwise introduce unnecessary latency. Newly generated nodes are subsequently updated into the candidate node pool, where they become available for further exploitation in subsequent cycles.

% Together, these components of the parallel reasoning process allow the framework to efficiently leverage computational resources, maintain flexibility in handling diverse paths, and ensure high-speed inference while keeping memory consumption in check.
% }

\subsection{Search and Transition Mechanism}
\label{sec:search_and_transition}
In this section, we introduce the \textbf{Search} and \textbf{Transition} Mechanism in DPTS, which is a hybrid search algorithm that balances exploitation and exploration through separate management and dynamic conversion. 

\subsubsection{Search}
\label{sec:method_searching}

% To achieve a balance between exploitation and exploration, the parallel queue  $P$  is dynamically partitioned: the top  $p|P|$  highest-scoring nodes are designated as exploit nodes, while the remaining  $(1 - p)|P|$  nodes are explore nodes. These nodes are selected from the candidate node pool $N$ based on their confidence, as illustrated in Figure~\ref{fig:overview} (left). 

% \textbf{Exploit nodes updating} follows a dynamic node inheritance mechanism. if the new node confidence above the threshold, the best child of the exploit node can proceed to the next generation. However, if the confidence of new node falls below the threshold, or the number of exploit node less than $p|P|$, additional nodes are selected from the highest-scoring candidates in the pool  $N$  to ensure the total number of exploitation paths. This dynamic adjustment ensures that the exploitation process remains continuous and adaptive to the evolving search space. 

% \textbf{Explore nodes selection} focuses on extensive searching high-confidence solutions and refining reasoning direction. Unlike exploit nodes, explore nodes are not inherited from the previous generation cycle. Instead, before each expansion step, the highest-scoring nodes from pool $N$, excepting the exploit nodes that are already selected, are collected as explore nodes. This mechanism ensures that explore nodes are always updated based on the latest search progress, allowing for efficient and extensive exploration while avoid unnecessary detours into suboptimal regions. 

% \jwt{
The Search Mechanism aims to balance exploration and exploitation by dynamically partitioning the nodes in parallel queue \( P \) into two categories: \textit{explore nodes} and \textit{exploit nodes}. 
% By allocating resources efficiently to both types of nodes, the search process is able to explore new areas of the search space while exploiting high-confidence paths for refinement. 

As illustrated in Figure~\ref{fig:overview} (left), these nodes are selected from the candidate node pool \( N \).
% 设置了一个超参数p,这个超参数根据什么设置的 % 目前是 half-half
The partition ratio $p$ can be manually adjusted according to the task and memory budget. 
%And we simply set to $p=0.5$ for balance. 这句应该放实验里面 % okk
At initialization, the top \( p|P| \) highest-scoring nodes are assigned as \textit{exploitation nodes}, while the remaining \( (1 - p)|P| \) nodes are assigned as \textit{exploration nodes}. While during searching progress, the proportion of the two types of nodes dynamically fluctuates based on the transition mechanism in Sec.~\ref{sec:method_transition}. 
The primary distinction between these two nodes lies in their origins and roles during the search process.

\paragraph{Exploitation Nodes} 
 The exploitation nodes are primarily inherited from parent exploitation nodes, focusing on refining the most promising paths in the search space. When a child node’s confidence exceeds a predefined threshold, it inherits the status of its parent exploitation node and continues that path. This inheritance ensures that the most promising paths are deepened and further refined. Additionally, when the number of exploitation nodes falls below a predefined threshold, new high-confidence candidate nodes from the pool \( N \) are selected to fill the gap, ensuring that the number of exploitation nodes remains adequate for the search process. This strategy enables the exploitation of high-potential paths while maintaining the focus on areas with high confidence.
 % These nodes are selected before each expansion step from the candidate node pool \( N \), excluding those already designated as exploitation nodes. Exploitation nodes are not inherited from the previous generation; instead, they are always chosen based on the latest progress in the search space. This approach ensures that exploitation targets the most confident solutions while avoiding redundant exploration of paths that are unlikely to lead to better results.

\paragraph{Exploration Nodes} 
In contrast to exploitation nodes, the exploration nodes are not inherited from previous nodes but are dynamically selected from the candidate nodes pool. These nodes are responsible for discovering new paths that may have high potential but low current confidence in the search space. At each reasoning step, the exploration nodes are reselected from the candidate pool \( N \), choosing the highest-confidence nodes that are not already assigned as exploitation nodes. The dynamic re-selection of exploration nodes allows the search process to adapt to changing circumstances and uncover new regions of the search space that may lead to better solutions.

% These nodes are dynamically updated through an inheritance mechanism, where a new child node is inherited by an exploration node only if its confidence exceeds a certain threshold. If the confidence of the new child node falls below the threshold, or if the number of explore nodes drops below \( p|P| \), additional nodes are selected from the pool \( N \) to maintain the desired number of explore nodes. This dynamic selection ensures that exploitation  remains adaptive to the evolving search space, allowing the algorithm to discover new, potentially fruitful paths as it progresses.


% }


% \begin{algorithm}[ht]
%     \caption{Searching}
% \label{alg:searching}
% \begin{algorithmic}[1]
% \REQUIRE Parallel queue $P$, parallel number $\mathtt{queue\_size}$, candidate node pool $N$. 
% \ENSURE Updated $P$. 
% \IF{$|P|<\mathtt{queue\_size}$}
%     \STATE $N'\gets$ Descending $N$ based on conf. \\
%     % , $\forall n \in N$
%     \STATE $P \gets P \cup N'[:\mathtt{queue\_size}-|P|]$
%     \STATE $P'\gets$ Descending $P$ based on conf.
%     % , $\forall n \in P$
%     \STATE $n$.mode $\gets \mathtt{EXPLOIT}$, $\forall n \in P'[:p|P|]$
%     \STATE $n$.mode $\gets \mathtt{EXPLORE}$, $\forall n \in P'[p|P|:]$
% \ENDIF
% \RETURN $P$
% \end{algorithmic}
% \end{algorithm}

\subsubsection{Transition}
\label{sec:method_transition}

% While partitioning parallel nodes into exploit and explore helps mitigate redundant exploitation and shallow expansion issues, it does not entirely eliminate inefficiencies. One key limitation is that the initially selected exploit nodes are not guaranteed to be on the optimal path. Even when an exploit node is later found to be suboptimal, it continues expanding until reach the termination condition (timeout or out of memory), leading to unnecessary resource consumption. Additionally, before an exploit node terminates, newly identified high-confidence nodes cannot be deeply exploit since they are still classified as explore nodes. These weaken the flexibility and limit the efficiency of parallel reasoning. 

% To alleviate this issue, we introduce a bidirectional transition mechanism in the DPTS framework. As illustrated in Figure~\ref{fig:overview} (middle), it enables dynamic transition between exploit and explore nodes, enhancing flexibility and adaptability. It is a two-way transition: \textit{Early Stop (Exploit → Explore)} and \textit{Deep Seek (Explore → Exploit)}. 
% % These two strategies allow the search process to adaptively reallocate computational resources, ensuring efficient reasoning while maintaining a balance between exploitation and exploration.

% In the \textbf{Early Stop transition}, a threshold $\theta_{{early\_stop}}$ is introduced to dynamically evict low-potential exploit nodes. Specifically, we evaluate the best child node after each expansion. If the confidence is lower than $\theta_{{early\_stop}}$, none of the child would be added in the queue $P$ for the next cycle, preventing further redundant exploitation. Conversely, if the best child’s confidence exceeds $\theta_{{es}}$, it inherits the parent’s status and continues to exploit in the next cycle. As discussed in Sec.~\ref{sec:motivation}, the prior confidence is a reasonable indicator for forecasting the potential of the path. The threshold $\theta_{{es}}$ is empirically set as
% \begin{equation}
% \label{eq:theta}
% \theta_{{es}}  = \begin{cases}
% \lambda_1\frac{\sum^{N_\mathrm{Exp.}}_{n}n.\mathrm{conf.}}{|N_\mathrm{Exp.}|}, \text{if $t\le t^*$} \\ 
% \max(\forall_{n\in N_\mathrm{Exp.}} n.\mathrm{conf.}), \text{otherwise}
% \end{cases}
% \end{equation}
% where $N_\mathrm{Exp.}$ means all previously selected and expanded nodes, $\lambda_1$ is a coefficient, $t$ is the number of current terminated paths and $t^*$ is a predefined threshold to change the $\theta_{{early\_stop}}$. 
% We provide experimental evidence in Appendix~\ref{app:best_path_index} to show that early-terminated paths are more likely to be the optimal solution. 

% In contrast, the \textbf{Deep Seek transition} allows high-confidence explore nodes to convert into exploit nodes, enabling promising nodes to dig deeper. Specifically, explore node with a confidence exceeding the threshold $\theta_{{deep\_seek}}$ is promoted to an exploit node, where $\theta_{{deep\_seek}}$ has the similar formula as $\theta_{early\_stop}$ with $\lambda_{2}$. 
% % which is empirically set as $\theta_{{deep\_seek}}=\theta_{{early\_stop}}$ and we consistently obtain good results. 
% Using this mechanism, the number of exploit nodes may temporarily exceed $p|P|$. Fortunately, the increasing number of high-confidence nodes naturally raises $\theta_{{early\_stop}}$, which in turn a larger ratio of exploit nodes meet the Early Stop condition. This creates an adaptive system where the two types of nodes are dynamically balanced throughout the searching process. 

% \jwt{
While the Search Mechanism ensures a balance between exploration and exploitation, the redundant issue is not entirely mitigated. 
% there are cases where inefficiencies arise. 
One example is that the initial exploitation nodes are not guaranteed to be the optimal solution. However, they only stop exploiting till reach the termination condition. 
% may fail to lead to optimal paths.
Another issue occurs when high-confidence nodes are assigned as exploration nodes, but they will wait for the computation resources and do not roll out till the previous paths terminate. 
% which may waste resources because of over-exploring through already existing promising paths. 

To address these issues, we introduce the Transition Mechanism, which consists of two main strategies: \textit{Early Stop} (Exploitation → Exploration) and \textit{Deep Seek} (Exploration → Exploitation). As illustrated in Figure~\ref{fig:overview} (middle), these strategies allow an evolving search space with node transits between the two statuses. 
It helps the tree maintain focused reasoning, ensuring the efficient allocation and utilization of limited computational resources throughout the whole search process. 
% This flexibility enhances the search process by reallocating computational resources as needed to maintain an optimal balance between exploration and exploitation.

\paragraph{Early Stop (Exploitation → Exploration)} 
The Early Stop~\cite{yao2007early} strategy allows relatively low-confidence exploitation nodes to transition into explore nodes, eliminating redundant exploitation on suboptimal paths. 
During the expansion process, if the best child node of an explore node has a confidence lower than a certain threshold \( \theta_{\mathrm{es}} \), the child node will be excluded from the queue \( P \) in the next cycle. This prevents further exploration of paths that are unlikely to lead to optimal solutions, saving computational resources. Conversely, if the child node’s confidence exceeds \( \theta_{\mathrm{es}} \), it inherits the parent’s status and continues to explore in the next cycle. This mechanism ensures that only the most promising explore nodes continue to expand, optimizing both exploration and resource usage.
The threshold \( \theta_{\mathrm{es}} \) is defined as follows:
\begin{equation}
\label{eq:theta}
\mathsf{\theta}_{\mathrm{es}} = \begin{cases}
\lambda_\mathrm{es} \, \frac{1}{|\mathcal{N}|} \sum\limits_{i \in \mathcal{N}} c_i, & \text{if } t \leq t^* \\
\underset{i \in \mathcal{N}}{\max} \, c_i, & \text{otherwise}
\end{cases}
\end{equation}
where \( \mathcal{N} \) is the set of previously expanded nodes, \( c_i \) represents the confidence of node \( i \), \( \lambda_\mathrm{es} \) is a coefficient that adjusts the threshold, \( t \) is the number of currently terminated paths, and \( t^* \) is a predefined threshold after which \( \theta_{\mathrm{es}} \) is adjusted.


\paragraph{Deep Seek (Exploration → Exploitation)} 
The Deep Seek strategy addresses the issue of inefficient over-exploration and shallow thinking, ensuring promising exploration nodes can be dug deeper. Specifically, exploration nodes with confidence exceeding a threshold \( \theta_{\mathrm{ds}} \) with $\lambda_\mathrm{ds}$ are promoted to exploitation nodes. 
% This threshold is typically set to \( \theta_{\mathrm{ds}} = \theta_{\mathrm{es}} \), according to empirical results. % 这个实验里也打算加一下不等的,因为我想了想感觉等号不一定理论上合理。。
As a result, the number of exploitation nodes may temporarily exceed \( p|P| \). But as more high-confidence nodes are promoted, \( \theta_{\mathrm{es}} \) increases, and thus more exploration nodes are stopped under the Early Stop strategy. This creates a dynamic balance between exploration and exploitation throughout the search process. 

% \hspace{0pt}
\vspace{0.07in}

In a word, the proposed Search and Transition Mechanism in DPTS effectively manages the trade-off between exploitation and exploration with dynamic and bidirectional transition. 
% By dynamically partitioning the parallel queue and  the transition strategies, our approach adapts to the evolving search space and optimizes the utilization of computational resources. It enables the model to focus on the most promising branches while avoiding redundant or suboptimal expansions. 
Detailed algorithms in this part can be found in Appendix~\ref{app:sec:parallel_reasoning}.  

% \begin{algorithm}[ht]
%     \caption{Transition}
% \label{alg:transition}
% \begin{algorithmic}[1]
% \REQUIRE Parallel queue $P$, transition thresholds $\theta_{early\_stop}$ and $\theta_{deep\_seek}$. 
% \ENSURE Updated $P$. 
% \FORALL{$n \in P$}
%     \STATE $n^*=\max_\mathrm{conf.}$($n$.children)
%     \IF{$n$.mode $=$ $\mathtt{EXPLOIT}$ \AND $n^*$.conf. $<\theta_{early\_stop}$ \OR $n$.mode $=$ $\mathtt{EXPLORE}$ \AND $n^*$.conf. $<\theta_{deep\_seek}$}
%         \STATE $P \gets P \setminus n$
%     \ELSE
%         \STATE $P \gets P\cup \{n^*\}$
%         \STATE $n^*$.mode $\gets \mathtt{EXPLOIT}$ 
%     \ENDIF
% \ENDFOR
% \RETURN $P$
% \end{algorithmic}
% \end{algorithm}


\section{System Implementation}
\label{sec:system-implementation}
%We use GPT 3.5 Turbo~\cite{gpt-3.5} and LLaMA3-8B~\cite{touvron2023llama} in the question decomposition stage, while finetuning the open-source LLaMA2-7B and LLaMA3-8B~\cite{touvron2023llama} in the answer assembly stage.
%We focus on the small models because of the resource limitation of edge devices.


\iffalse
\begin{figure}[t]
   \centering
    \setlength{\tabcolsep}{0.2pt}
    \begin{tabular}{ccc}
        \vspace{-2mm}
        \includegraphics[width=0.35\textwidth, height=3.6cm]{figs/system.png} &
        \includegraphics[width=0.32\textwidth, height=3.4cm]{figs/sensorchat_cloud.png} &
        \includegraphics[width=0.27\textwidth, height=3.4cm]{figs/sensorchat_edge.png} \\
\end{tabular}
\vspace{-3mm}
    \caption{Sensitivity of key hyperparameters.}
    \label{fig:system}
    \vspace{-5mm}
\end{figure}
\fi



We implement \Method on real-world systems. We envision \Method as a personal assistant that provides accurate and timely answers to user questions, as outlined in our problem statement (Sec.~\ref{sec:problem-statement}).
Fig.~\ref{fig:system} visualizes the general system pipeline of \Method.
We employ smartphones and smartwatches to collect multimodal sensor data from users in daily lives (\textcircled{1} in Fig.~\ref{fig:system_diagram}).
Implemented based on the ExtraSensory App~\cite{vaizman2018extrasensory}, the mobile devices automatically gathers data for 20 seconds every minute, including 40Hz IMU signals, 13 MFCC audio features from a 22kHz microphone, and other phone state information including compass, GPS location, Wi-Fi status, light intensity, battery level, etc. The details can be found in the ExtraSensory App manual~\cite{vaizman2018extrasensory}.
These sensor data are then transmitted to a system running \Method (\textcircled{2} in Fig.~\ref{fig:system_diagram}). We offer two variants of \Method, designed for a cloud server and an edge environment respectively. Their detailed implementations and trade-offs are discussed below. Finally, users can interact with \Method directly through a chatting interface using natural language, shown as \textcircled{3} in Fig.~\ref{fig:system_diagram}.


\textbf{\MethodC and \MethodE} We offer two system variants of \Method, as shown in Fig.~\ref{fig:sensorchat_cloud} and \ref{fig:sensorchat_edge}.
\begin{itemize}
    \item \textbf{\MethodC}, designed for a cloud environment, uses GPT-3.5-Turbo~\cite{gpt-3.5} for question decomposition and a full-size finetuned LLaMA2-7B model~\cite{touvron2023llama} for answer assembly. We deploy and test \Method on a cloud server equipped with an NVIDIA A100 GPU~\cite{a100}.
    \item \textbf{\MethodE}, designed for an edge environment, uses quantized LLaMA model~\cite{touvron2023llama} for both question decomposition and answer assembly. The question decomposition model is quantized from the official LLaMA3-8B, while the answer assembly model is quantized from our fine-tuned version of LLaMA2-7B. 
    We use Activation-aware Weight Quantization (AWQ), a state-of-the-art quantization method for LLMs, known for its hardware efficiency. We deploy and test \MethodE on a NVIDIA Jetson Orin NX module~\cite{jetsonorin} with 16GB RAM.
\end{itemize}
\MethodC and \MethodE accommodate two typical use scenarios. \MethodC is expected to deliver superior QA performance with the full-precision LLMs in the cloud, at the cost of intensive resource consumption. Additionally, \MethodC requires the users to transmit the full sensor history to the cloud server.
On the other hand, \MethodE runs entirely on a local edge platform belonging to the user, eliminating the need to transmit user data to the cloud and thus preserving user privacy. However, its QA and latency performance degrade compared to \MethodC.  

%a Linux desktop and an NVIDIA Jetson Orin~\cite{}.
%RPi 5 enjoys a 2.4GHz quad-core Cortex-A76 CPU and 8GB RAM. 
%The desktop is equipped with an Intel Core i7-8700 CPU . The Jetson platform features a dual-core NVIDIA Denver 2 CPU, a quad-core ARM Cortex-A57 MPCore, an NVIDIA Pascal GPU with 256 CUDA cores, and 8GB of RAM. %We measure response generation speed on both platforms.

\begin{figure}[t]
  \begin{subfigure}[b]{0.38\textwidth}
        \centering
        \includegraphics[width=\textwidth]{figs/system.png}
        \vspace{-5mm}
        \caption{Real system implementation.}
        \label{fig:system_diagram}
    \end{subfigure} %\hspace{0.05\textwidth} % Add horizontal space between subfigures if needed
    \begin{subfigure}[b]{0.3\textwidth}
        \centering
        \includegraphics[width=\textwidth]{figs/sensorchat_cloud.png}
        \vspace{-5mm}
        \caption{\MethodC diagram.}
        \label{fig:sensorchat_cloud}
    \end{subfigure}
    \begin{subfigure}[b]{0.28\textwidth}
        \centering
        \includegraphics[width=\textwidth]{figs/sensorchat_edge.png}
        \vspace{-5mm}
        \caption{\MethodE diagram.}
        \label{fig:sensorchat_edge}
    \end{subfigure}
    \vspace{-4mm}
    \caption{System implementation details of \Method.}
    \label{fig:system}
    \vspace{-4mm}
\end{figure}


\textbf{Implementation Details}
The algorithm part of \Method is implemented with Python and PyTorch~\cite{paszke2019pytorch}.
For the offline sensor encoder, \Method uses a Transformer architecture~\cite{vaswani2017attention} with 6 encoder layers, 8 attention heads, and a feedforward network size of 2048 for time series data. For the low-dimensional phone status data, \Method employs a fully connected layer as the encoder. The fusion layer is also implemented as a fully connected layer. The label encoder is initialized from the pretrained CLIP ViT-B/32 label encoder~\cite{radford2021learning}. Both the sensor and label embedding spaces share a dimension size of 512. Offline pretraining is performed with the partial-context loss proposed in Sec.~\ref{sec:pretraining} using a temperature scalar of $\tau=0.1$. We use the Adam optimizer with a learning rate of $1e-5$ over 100 epochs.
%We test three candidate GPTs for question decomposition: GPT-3.5-Turbo~\cite{gpt-3.5}, GPT-4~\cite{gpt-4}, and quantized LLaMA3-8B~\cite{lin2023awq}, as shown in Fig.~\ref{fig:overview}. 
%In the answer assembly stage, \Method uses the open source LLaMA2-7B and LLaMA3-8B models~\cite{touvron2023llama}. 
%We focus on smaller models so that they can be deployed on edge devices after quantization.

For question decomposition, we design two solution templates for each question category. Limiting the number of templates to two helps balance prompt length and performance.
For online data queries, we initialize the classifier $f$ as a multilayer perceptron with one hidden layer and a ReLU activation function. The hidden layer size is set to 512 and the query threshold is configured to $h = 0.5$.
For finetuning the LLM in answer assembly, we apply low rank adaptation (LoRA) finetuning~\cite{hu2021lora} on an A100 GPU, using a batch size of 8, a learning rate of 0.0002 and 10 epochs.
The LoRA rank $r$ is 16, the LoRA scaling factor \textit{alpha} is 16, and the LoRA dropout is 0.1.
During answer generation, \Method uses a max sequence length of 1024 and a geberating temperature of $0.2$.
%

%%%%%%%%%%%%%%%%%%%%%%%%%%%%%%%%%%%%%%%%%%%%%%%%%%%
% Evaluation
%%%%%%%%%%%%%%%%%%%%%%%%%%%%%%%%%%%%%%%%%%%%%%%%%%%
\vspace{-2mm}
\section{Evaluation on State-of-the-Art Dataset}
\label{sec:evaluation}

In this section, we thoroughly evaluate \Method on the state-of-the-art dataset focusing on quantitative questions. We further validate \Method in a real world study with open-ended, qualitative questions in Sec.~\ref{sec:deployment}.


\subsection{Dataset and Metrics}

In this evaluation section, we focus mainly on the \Dataset dataset~\citesensorqa and quantitative questions. A detailed introduction to \Dataset~\citesensorqa is provided in Sec.~\ref{sec:motivation}. 
To the best of our knowledge, \Dataset is the first and only available benchmarking dataset for QA interactions that use long-term timeseries sensor data and reflect practical user interests.
While we focus on \Dataset~\citesensorqa in this section, we emphasize that the motivation and design of \Method are broadly applicable and can be extended to other practical sensing applications.
To ensure the best alignment between the QA pairs and sensor information, we conduct offline encoders pretraining on the ExtraSensory multimodal sensor dataset~\cite{vaizman2017recognizing}, which servers as the sensor data source for \Dataset.
During pretraining, all sensor samples are aligned by a time window of 20 seconds.
%The experiments can be extended to other sensor application that has data readily available in the future.


%For all experiments, we randomly select 80\% of the QA pairs in \Dataset for training or finetuning and use the rest 20\% for testing. 
%\textcolor{red}{In Sec.XX, we further evaluate the gene}
%For all methods, we use the first 48 users' data for training and the remaining 12 for testing. 
%The training process for \Method includes training the sensor and label encoder with sensor data, followed by finetuning the LLaMA model using both the QA data and the stored sensor database.


\textbf{Dataset Variants and Metrics}
We evaluate three versions of \Dataset~\cite{sensorqa} using various metrics to assess both the quality and accuracy of the generated answers.
\begin{itemize}[topsep=0pt, itemsep=0pt]
    \item \textbf{Full answers} refer to the original full responses  in \Dataset. We evaluate the model's performance on the full answers dataset using Rouge-1, Rouge-2, and Rouge-L scores~\cite{eyal-etal-2019-question}. Rouge scores measure the overlap of n-grams between the machine-generated content and the ground-truth answers, expressed as F-1 scores. Higher Rouge scores indicate greater similarity between the generated and true answers.
    \item \textbf{Short answers} are the 1-2 key words extracted from the full answers by GPT-3.5-Turbo~\cite{gpt-3.5}, offered with the original \Dataset dataset~\citesensorqa. We use the exact match accuracy on the short answers to evaluate the precision of generated answers, as detailed in Sec.~\ref{sec:motivation}.
    \item \textbf{Multiple choices} are generated by prompting GPT-3.5-Turbo~\cite{gpt-3.5} to create three additional choices similar to the correct short answer. An example QA can be "Which day did I spend the most time with coworkers? A. Friday, B. Monday, C. Thursday, D. Wednesday", with the correct answer being "D" or "D. Wednesday." The models are expected to accurately select the correct answer from the four candidates. We evaluate the performance based on exact answer selection accuracy.
\end{itemize}
We create the multiple-choice version in addition to the full answers and short answers provided in the original \Dataset dataset~\citesensorqa, to further assess the model's ability in distinguishing similar facts based on sensor data. 
We use different dataset variants to assess various aspects of the models. The full answers dataset evaluates overall language quality, while the short answers and multiple-choice evaluations focus on the model's ability to learn underlying facts rather than relying solely on patterns in language token generation. Further discussion of the evaluation metrics can be found in Sec.~\ref{sec:future-work}.
%the original questions and answers collected from workers, a short-answer version, and a multiple-choice version. For the short-answer version, we use GPT-3.5 to extract 1-2 key words from the original answers. For the multiple-choice version, we use GPT-3.5 to generate three additional choices that are similar to the correct short answer, converting it into a multiple-choice question. The full-answer version is suitable for evaluating the language quality of the responses, while the short-answer and multiple-choice versions facilitate exact match accuracy evaluation, highlighting the precision of the answers.


\vspace{-2mm}
\subsection{State-of-the-Art Baselines}



We compare \Method against several state-of-the-art baselines that leverage different modalities combinations, including \textcolor{mygreen}{text}, \textcolor{myred}{vision+text}, and \textcolor{myblue}{sensor+text} data. These comparisons highlight the effectiveness of integrating multiple modalities to address the sensor-based QA tasks. 
We consider both closed-source and open-source baselines for a comprehensive analysis.

%For visual inputs, we use the activity graphs created when generating \Dataset.
 % to address natural QA interactions with sensors.

We first evaluate the state-of-the-art closed-source generative models using various modalities:
\begin{itemize}[topsep=0pt, itemsep=0pt]
    \item \textbf{GPT-3.5-Turbo~\cite{gpt-3.5} and GPT-4~\cite{gpt-4}} are \textcolor{mygreen}{text-only} baselines taking only the questions as inputs.
    \item \textbf{GPT-4-Turbo~\cite{gpt-4} and GPT-4o~\cite{gpt-4}} are \textcolor{myred}{vision+text} baselines taking the activity graphs and the questions as inputs. For all \textcolor{myred}{vision+text} baselines, we feed the activity graphs in \Dataset~\citesensorqa, similar to Fig.~\ref{fig:example_qas}, along with the questions into the model.
    \item \textbf{IMU2CLIP+GPT-4\footnote{\url{https://github.com/facebookresearch/imu2clip}}~\cite{moon-etal-2023-imu2clip}} is the state-of-the-art \textcolor{myblue}{sensor+text} GPT baseline as explained in Sec.~\ref{sec:motivation}.
\end{itemize}
For these closed-source generative models, we use few-shot learning (FSL). Specifically, we incorporate a set of QA examples from \Dataset~\citesensorqa into the prompt for each question input. We adopt $10$ samples per question based on a grid search of $\{2, 5, 10, 15\}$.





%The various combinations of modalities include \textcolor{mygreen}{text-only}, \textcolor{myred}{vision+text} and \textcolor{myblue}{sensor+text}.
%Given that the innovative aspect of our \Method QA model lies in its fine-tuning approach, we select baselines that either train a neural network or finetune a language model using the \Method training dataset and subsequently evaluate their performance on the \Method test dataset.

We further conduct comprehensive evaluation with the state-of-the-art open-source models using various modalities:
\begin{itemize}[topsep=0pt, itemsep=0pt]
    \item \textbf{T5~\cite{2020t5}} and \textbf{LLaMA\footnote{\url{https://www.llama.com/}}~\cite{touvron2023llama}} are popular \textcolor{mygreen}{text-only} language models.
    
    \item \textbf{LLaMA-Adapter\footnote{\url{https://github.com/OpenGVLab/LLaMA-Adapter}}~\cite{zhang2023llama}} is a recent \textcolor{myred}{vision+text} framework offering a lightweight method for fine-tuning instruction-following and multimodal LLaMA models. It integrates vision inputs (i.e., activity graphs from \Dataset~\citesensorqa) with LLMs using a transformer adapter. We utilize the latest LLaMA-Adapter V2 model.
    
    \item \textbf{LLaVA-1.5\footnote{\url{https://llava-vl.github.io/}}~\cite{liu2024improved}} represents state-of-the-art \textcolor{myred}{vision+text} model. LLaVA connects pre-trained CLIP ViT-L/14 visual encoder~\cite{radford2021learning} and large language model Vicuna~\cite{vicuna2023}, using a projection matrix. LLaVA-1.5~\cite{liu2024improved} achieves state-of-the-art performance on 11 benchmarks through simple modifications to the original LLaVA and the use of more extensive public datasets for finetuning.
    
    \item \textbf{DeepSQA\footnote{\url{https://github.com/nesl/DeepSQA}}~\cite{xing2021deepsqa}} trains a CNN-LSTM model with compositional attention to fuse \textcolor{myblue}{sensor+text} modalities and predict from a fixed and limited set of candidate answers given questions and IMU signals. We adapt their implementation to use the full-history timeseries data as input, to align with \Method's setup.
    
    \item \textbf{OneLLM\footnote{\url{https://github.com/csuhan/OneLLM}}~\cite{han2023onellm}} is a state-of-the-art multimodal LLM framework that processes \textcolor{myblue}{sensor+text} modalities using a universal pretrained CLIP encoder and a mixture of projection experts for modality alignment. We adapt their implementation and feed the full-history timeseries data from the IMU tokenizer.
\end{itemize}
All baselines based on LLaMA (that is, LLaMA, LLaMA-Adapter, and OneLLM) use LLaMA2-7B~\cite{touvron2023llama}.
For T5 and DeepSQA, we train the models directly on the \Dataset dataset~\citesensorqa.
For the LLM baselines, we apply LoRA fine-tuning (FT)~\cite{hu2021lora} using the samples from \Dataset~\citesensorqa. 
For all methods that require training, we randomly select 80\% of the QA pairs in \Dataset~\citesensorqa as training samples and reserve the remaining 20\% for testing. We explore alternative splitting schemes in Sec.~\ref{sec:generalizability} to demonstrate \Method's generalizability to unseen users.
All baselines adopt the same hyperparameters as those specified in their official codebases.
%
We did not compare with the latest works of Sensor2Text~\cite{chen2024sensor2text}, PrISM-Q\&A~\cite{arakawa2024prism}, and DrHouse~\cite{yang2024drhouse} due to limited access to their open-source code and models.



\begin{figure*}[tb]
  \centering
  \begin{subfigure}[b]{0.82\textwidth}
        \centering
        \includegraphics[width=\textwidth]{figs/qual_example_1.png}
        \vspace{-6mm}
        \caption{Example of a time query question on a single-day duration.}
        \label{fig:qual-daily}
    \end{subfigure}
    
    \begin{subfigure}[b]{0.82\textwidth}
        \centering
        \includegraphics[width=\textwidth]{figs/qual_example_2.png}
        \vspace{-6mm}
        \caption{Example of a counting question on a multi-day duration.}
        \label{fig:qual-week}
    \end{subfigure}
  \vspace{-4mm}
  \caption{Qualitative results of \Method in comparison to state-of-the-art methods.}
  \vspace{-6mm}
  \label{fig:qual_results}
\end{figure*}




\subsection{Qualitative Performance}
\label{sec:qualitative}

We illustrate a qualitative comparison of \Method with the top-performing baselines in Fig.~\ref{fig:qual_results}. 
Specifically, Fig.~\ref{fig:qual_results} (a) focuses on time-related queries and (b) focuses on counting questions, which are the most challenging for SOTA methods given the long-duration time series sensor inputs (see Sec.~\ref{sec:motivation}).
Key phrases in the answers are highlighted in the green if they closely match the true answer, and in the red if they do not. The two presented examples are very challenging for state-of-the-art baselines. %, with GPT-4o in Fig.~\ref{fig:qual_results} (b) being the only correct instance.
\Method, on the other hand, consistently produces more accurate answers which can be attributed to its novel three-stage pipeline.

Answering the question in Fig.~\ref{fig:qual_results} (a) requires two major steps: (i) calculating the total time spent in school and in the main workplace, and (ii) computing the difference between them.
\Method accurately answers it by first decomposing and then querying the durations for ``at school'' and ``in the main workplace'' respectively, resulting in ``\textit{You spent 11 hours and 27 minutes at school Tuesday. You spent 9 hours and 3 minutes at main workplace}''. Finally, \Method integrates the above text and determines the time difference during the answering stage.
Although \Method's answer is 15 minutes off from the true value, possibly due to inaccuracies in the sensor encoder or LLM reasoning, \Method still approximates the ground truth with an accuracy unmatched by other baselines.


In Fig.~\ref{fig:qual_results} (b), counting the total days spent at home requires long-term reasoning where  LLaMA-Adapter~\cite{zhang2023llama} and  DeepSQA~\cite{xing2021deepsqa} usually fall short.
\Method decomposes this question and queries the duration of "at home" on "each day", then leaving the counting task to the answering stage.
In a nutshell, \Method relies on question decomposition and sensor data queries to extract relevant key sensor information, while the answer assembly stage handles reasoning and produces the final answer.
This collaboration across the three stages allows \Method to effectively manage a wide range of tasks, particularly those requiring multi-step reasoning and quantitative analyzes, which highlights \Method's advancements over existing works.
%The three-stage design in \Method and the intelligent work division between each stage enables \Method to properly handle such challenging tasks requiring multi-step reasoning.


\begin{table*}[t]
{
\footnotesize
\centering
\begin{tabular}{c|c|c|ccc|c|c}
\toprule
 \textbf{Modalities} & \textbf{Method} & \textbf{FSL/FT$^1$} & \multicolumn{3}{c|}{\textbf{Full Answers}} & \textbf{Short Answers} & \textbf{Multiple Choices} \\
 & & & Rouge-1 ($\uparrow$) & Rouge-2 ($\uparrow$) & Rouge-L ($\uparrow$) & Accuracy ($\uparrow$) & Accuracy ($\uparrow$) \\
\midrule
\textcolor{mygreen}{Text} & GPT-3.5-Turbo~\cite{gpt-3.5} & FSL & 0.35 & 0.23 & 0.32 & 0.03 & 0.33 \\
\textcolor{mygreen}{Text} & GPT-4~\cite{gpt-4} & FSL & 0.66 & 0.51 & 0.64 & 0.16 & 0.34 \\
\textcolor{mygreen}{Text} & T5-Base~\cite{2020t5} & FT & 0.71 & 0.55 & 0.69 & 0.25 & 0.52 \\
%\hline
\textcolor{mygreen}{Text} &  LLaMA2-7B~\cite{llama2} & FT & \underline{0.72} & \underline{0.62} & \underline{0.72} & 0.26 & \underline{0.56} \\
\hline
\textcolor{myred}{Vision+Text} & GPT-4-Turbo~\cite{gpt-4} & FSL & 0.38 & 0.28 & 0.36 & 0.14 & 0.24 \\
\textcolor{myred}{Vision+Text} & GPT-4o~\cite{gpt-4} & FSL & 0.39 & 0.28 & 0.37 & 0.20 & 0.07 \\
%\hline
\textcolor{myred}{Vision+Text} & LLaMA-Adapter~\cite{zhang2023llama} & FT & 0.73 & $0.57$ & $0.71$ & \underline{0.28} & 0.54 \\
%\hline
\textcolor{myred}{Vision+Text} & LLaVA-1.5~\cite{liu2024improved} & FT & $0.62$ &  $0.46$ & $0.60$ & 0.21 & 0.47 \\
\hline
\textcolor{myblue}{Sensor+Text} & IMU2CLIP-GPT4~\cite{moon-etal-2023-imu2clip} & FSL & 0.44 & 0.28 & 0.40 & 0.13 & 0.18 \\ 
\textcolor{myblue}{Sensor+Text} & DeepSQA~\cite{xing2021deepsqa} & FT & 0.34 & 0.05 & 0.34 & 0.27 & - \\
\textcolor{myblue}{Sensor+Text} & OneLLM~\cite{han2024onellm} & FT & 0.12 & 0.04 & 0.12 & 0.05 & 0.30 \\
\midrule
\textcolor{myblue}{Sensor+Text} & \textbf{\MethodC} & FT & \textbf{0.77}& \textbf{0.62} & \textbf{0.75} & \textbf{0.54} & \textbf{0.70} \\
\textcolor{myblue}{Sensor+Text} & \textbf{\MethodE} & FT & 0.76 & 0.60 & 0.74 & 0.49 & 0.67 \\
\bottomrule
\multicolumn{8}{l}{$^{1}$\small{FS: Few-Shot Learning. FT: Finetuning.}} \\
\end{tabular}
}
\vspace{-1mm}
\caption{Quantitative results of \Method compared against state-of-the-art methods. Bold and underlined values show the best results (all achieved by \Method) and the best among baselines.} % *DeepSQA only considers classification problem thus is only evaluated on the exact-match version of the dataset.}
\vspace{-7mm}
\label{tab:quant_results}
\end{table*}




\subsection{Quantitative Performance}

Table~\ref{tab:quant_results} presents the quantitative results of all methods on the three variants of \Dataset~\citesensorqa: full answers, short answers, and multiple-choice. Note that DeepSQA~\cite{xing2021deepsqa} was not evaluated on the multiple-choice version due to its inability to handle dynamic answer choices.
It is important to highlight that exact match accuracy for short and multiple-choice answers is a strict metric, as it requires the model to generate answers in the \textit{exact} same form as the correct ones. For instance, "4 hours" and "3 hours 50 min" would be considered different. Answering multiple-choice questions can also be particularly challenging, as candidate answers differ only slightly, such as "A. 10 min" vs. "B. 20 min," making them difficult to distinguish. In practical applications, however, a QA system does not necessarily need to achieve perfect exact match accuracy to be useful. We leave the exploration of more advanced metrics that better align with user satisfaction for future work, as discussed in Sec.~\ref{sec:future-work}.

As shown in Table~\ref{tab:quant_results}, our \Method outperforms the best state-of-the-art methods with the \textbf{highest Rouge scores on full answers}, \textbf{26\% higher accuracy on short answers}, and \textbf{14\% higher accuracy on multiple choices}. 
The top Rouge scores highlight \Method's ability to generate high-quality natural language that are the most similar with the ground-truth answers in \Dataset~\citesensorqa.
Even under the strict exact match evaluation, \Method achieves 54\% accuracy on short answers and 70\% on multiple-choice questions. These accuracy improvements demonstrate \Method's effectiveness in learning the underlying activity facts from the long-duration, multimodal time series sensor data. As explained in Sec.~\ref{sec:qualitative}, all three stages are crucial for achieving high accuracy, including correct question decomposition to invoke the right functions, precise sensor queries to accurately extract activity information from sensor data, and effective answer assembly for generating natural language responses.
\MethodC performs slightly better than \MethodE mainly due to the stronger model capability of GPTs compared to quantized LLaMA in question decomposition. However, \MethodE, as a pure edge solution, preserves better user privacy as discussed in Sec.~\ref{sec:system-implementation}.
%In future work, we plan to explore new metrics that allow small time drifts thus better fit practical use cases.

%The full answer dataset uses Rouge scores to measure the ``similarity'' between the model's generated answers and the true answers. Hence the full answer dataset focuses on the general text quality rather than precision. For example, an answer with similar wording but different key information, e.g., ``1 hour'' vs ``10 hours'', may still receive high Rouge scores.
%The short-answer and multiple-choice variants assess precision by comparing whether the exact key information is present in the answer.
%Together, these three datasets provide a comprehensive evaluation of the user-sensor QA task.


In contrast, all baselines struggle with quantitative accuracy, with an highest accuracy of merely 28\% on the short answers.
Despite GPT-4's strong reasoning capabilities and its multimodal variants (GPT-4-Turbo and GPT-4o), the models do not perform optimally for processing sensor data and daily life activities, as they are not specifically trained for these tasks.
The Rouge scores are lower because GPTs tend to generate longer text, resulting in less similarity with the ground-truth answers.
Interestingly, text-only baselines with finetuning, such as LLaMA2-7B~\cite{touvron2023llama} and T5~\cite{2020t5}, achieve some of the highest accuracy among the baselines, with 27\% accuracy on short answers and 56\% accuracy on multiple-choice datasets. These results suggest a sensor bias in the dataset, where some questions can be ``guessed'' correctly without sensor data.
Accuracy scores lower than these baselines indicate ineffective fusion of text and sensor data, as seen with models like Llava-1.5~\cite{liu2024improved} and OneLLM~\cite{han2024onellm}, whose poor performance stems from mismatches in pretraining and finetuning data formats. For example, OneLLM was designed for aligning raw IMU signals with fixed window sizes, making it challenging to adapt to long-duration sensor data in our task. Finally, DeepSQA~\cite{xing2021deepsqa} and LLaMA-Adapter~\cite{zhang2023llama} perform best among the baselines but struggle with accurate quantitative analysis on long-duration time series, as discussed in Sec.\ref{sec:motivation} and Sec.\ref{sec:qualitative}.


%through LLM-powered question decomposition and answer assembly stages. The accuracy gains demonstrate the effectiveness of sensor query and the fusion with text. 
%The GPT series are hindered by limited prompt length. If the narrative text does not fit in the length, then the prompt will be truncated and important information may be lost, leading to poor quality answers such as IMU2CLIP+GPT-4 in Fig~\ref{fig:qual_results} (b).
%The LLaMA-based models can adjust the answer according to the question and sensory input (if any), but struggles with accurate quantitative outputs. 

\iffalse
\subsection{Memory Requirements on Edge Devices}
\label{sec:memory}
\textcolor{red}{To be updated} Fig.~\ref{fig:memory} compares the model size requirements of all finetuning methods. The left plot shows unquantized LLM models while the right plot shows quantized LLMs by AWQ~\cite{lin2023awq} and small models such as DeepSQA~\cite{xing2021deepsqa} and T5~\cite{2020t5}.
Without quantization, all LLM-based methods require at least 13.5GB memory to accommodate model weights, making them unsuitable for edge deployment.
With AWQ quantization, \Method reduces its memory footprint to 3.8GB and fits into the 8GB RAM of a Jetson TX2. While non-LLM models, such as DeepSQA and T5, use less than 1GB of memory, they provide less natural and accurate answers as shown in the previous section. Multimodal LLMs require more memory to support adapter layers, where AWQ cannot be applied directly.
%
Notably, \Method consumes nearly the same memory as the text-only LLaMA2-7B. 
Such a negligible memory overhead can be attributed to the compression of encoding raw timeseries into logits, which reduces the total dataset size from 24G to 484K, achieving a compression ratio of approximately 50K times.
The sensor encoder and logits database enable \Method to run lightweight query search on edge devices.
\fi
%Compression ratio from raw data to logits.

\begin{figure}
\begin{center}
\vspace{-4mm}
\includegraphics[width=0.8\textwidth]{figs/efficiency2.png} 
\vspace{-4mm}
\caption{The answer accuracy and latency trade-offs of all methods measured on the cloud and edge platforms. 
We evaluate \MethodC on A100~\cite{a100} and \MethodE on Jetson Orin NX~\cite{jetsonorin}.
All LLM-based models are quantized to 4-bit weights with AWQ~\cite{lin2023awq} for Jetson Orin deployments.}
\label{fig:latency}
\end{center}
\vspace{-4mm}
\end{figure}


\subsection{End-to-End Answer Generation Latency}
To evaluate efficiency, we measure the end-to-end answer generation latency of \MethodC and \MethodE on their respective platforms - cloud server and NVIDIA Jetson Orin. For a fair comparison, we quantize all LLM-based baselines to 4-bit weights using AWQ~\cite{lin2023awq} for the Jetson Orin experiments, matching \MethodE’s setup. \textbf{\Method takes an average generation latency of 2.3s on the cloud and 10.0s on the Jetson Orin}.  
Specifically, \MethodC takes an average of 1.2s for question decomposition, 0.5s for sensor data query, and 0.6s for answer assembly. \MethodE takes an average of 2.5s, 4.9s, and 2.5s for each stage, respectively.


The accuracy-latency trade-off of \Method and locally running baselines is shown in Fig.~\ref{fig:latency}. Despite higher latency due to its dual-LLM design, \Method outperforms other baselines in accuracy. \Method still achieves real-time responses on the cloud and can be deployed on resource-constrained devices with reasonable generation latency.
Notably, quantization causes negligible accuracy loss in \MethodE compared to the full-precision \MethodC. This is due to \Method's design, which converts sensor data into text, making the final answer assembly a purely language-based task. Techniques like AWQ~\cite{lin2023awq} minimize accuracy degradation in language tasks after quantization.
We plan to further optimize the latency of \Method on edge devices via algorithm-system co-design in future work.
%All quantized LLM-based methods achieve negligible accuracy degradation compared to the full-precision models as in Table~\ref{tab:baseline_results_unfiltered}.
%Methods closer to the top left corner are preferred for their higher accuracy and lower latency.
%This indicates that \Method achieves real-time interactions on desktop-level edge devices and acts a feasible solution on more resource-constrained devices.
%On Jetson TX2, DeepSQA~\cite{xing2021deepsqa} encounters OOM issues with the large multi-day timeseries input. T5-Base without TensorRT~\cite{tensorrt} optimization cannot run neither due to its large model size. While T5-Small~\cite{2020t5} is very lightweight, it lacks sensor data, leading to low accuracy.
%Remarkably, \Method's latency is nearly identical to the text-only LLaMA2-7B, indicating minimal overhead from query searches. It also reduces latency by 38\% compared to the complex OneLLM model.


%On the desktop, all methods are very efficient.

\iffalse
\begin{figure*}[tb]
  \centering
  \begin{subfigure}[b]{0.44\textwidth}
        \centering
        \includegraphics[width=\textwidth]{figs/memory.png} 
        \vspace{-4mm}
        \caption{Memory requirements of quantized and unquantized models from various methods. The red horizontal line shows the RAM size limit of 8GB on Jetson TX2~\cite{jetsontx2}.}
        \label{fig:memory}
    \end{subfigure} \hspace{0.02\textwidth} % Add horizontal space 
    \begin{subfigure}[b]{0.44\textwidth}
        \centering
        \includegraphics[width=\textwidth]{figs/efficiency.png} 
        \vspace{-4mm}
        \caption{Efficiency of fine-tuning methods on desktop and Jetson TX2~\cite{jetsontx2}. The footnote $Q$ indicates quantized models.}
        \label{fig:efficiency}
    \end{subfigure}
  \vspace{-4mm}
  \caption{\textcolor{red}{To be updated.}}
  \vspace{-6mm}
  \label{fig:qual_results}
\end{figure*}
\fi





%\begin{itemize}
%    \item original AWQ (4bit)
%    \item personalized AWQ
%    \item original squeeze LLM (3bit)
%    \item personalized squeeze LLM
%\end{itemize}



\subsection{Ablation Studies}
\label{sec:ablation}

In this section, we comprehensively examine the impact of key design choices in \Method. Without loss of generality, we use \MethodC as the base model.
%major design variants in \Method by isolating one stage while keeping the other two fixed.
%This is because all three stages are essential for \Method and removing any one of them would disable the system entirely. 
%For example, without question decomposition, \Method would not know how to query the database. 
%We skip the answer assembly from discussion as it has few variants that are impactful to end-to-end performances.


\begin{table}[!t]
\small
\centering
%\vspace{-2mm}
\caption{Impact of three major stage in \MethodC. Bold values highlight the best results.}
\vspace{-4mm}
\label{tbl:ablation}
\begin{tabular}{c|ccc|c} 
\toprule
\small
\textbf{Setup} & \multicolumn{3}{c|}{\textbf{Full Answers}} & \textbf{Short Answers} \\ 
& Rouge-1 ($\uparrow$) & Rouge-2 ($\uparrow$) & Rouge-L ($\uparrow$) & Accuracy ($\uparrow$) \\
\midrule
w/o Question Decomposition & 0.73 & 0.57 & 0.71 & 0.35 \\
w/o Sensor Data Query & 0.72 & 0.62 & 0.72 & 0.26 \\ 
w/o Answer Assembly & 0.26 & 0.08 & 0.24 & 0.0 \\
\midrule
Full \Method & \textbf{0.77}& \textbf{0.62} & \textbf{0.75} & \textbf{0.54} \\
\bottomrule
\end{tabular}
\vspace{-2mm}
\end{table}

\textbf{Impact of Each Stage in \Method} 
We first evaluate the individual contribution of each stage in \Method by removing one of them from the pipeline. 
By removing question decomposition, we use a fixed and general decomposition templates for all questions. Removing sensor data query reverts the model to a language-only approach. By removing answer assembly, we directly output the queried sensor context from the second stage.
Table~\ref{tbl:ablation} summarizes the results on full and short answers including both quality and quantitative accuracy.
As observed, removing any stage leads to a significant performance drop. 
In \Method, all three stages must work collaboratively to deliver high-quality, accurate answers across diverse question types in \Dataset.
Among the three stages, the answer assembly stage has the most significant impact, as it directly influences the final output. Removing it results in severely degraded performance, with near-zero accuracy on short answers. However, the question decomposition and sensor data query stages are equally crucial.


\begin{table}[!t]
\small
\centering
%\vspace{-2mm}
\caption{Impact of various designs for sensor-text pretraining in \MethodC. Bold values highlight the best results.}
\vspace{-4mm}
\label{tbl:ablation-sensor-feature}
\begin{tabular}{c|c|c|c} 
\toprule
\small
\textbf{Sensor Data} & \textbf{Training Loss} & \textbf{Online Querying} & \textbf{Multiple Choices} \\ 
& & Accuracy ($\uparrow$) & Answer Accuracy ($\uparrow$) \\
\midrule
Statistical features & Partial-Context Loss & 0.91 & 0.62 \\
Time series & IMU2CLIP~\cite{moon-etal-2023-imu2clip} & 0.90 & 0.61  \\ \midrule
Time series & Partial-Context Loss & \textbf{0.98} & \textbf{0.70} \\
\bottomrule
\end{tabular}
\vspace{-2mm}
\end{table}

% If use F1, statistical features: 0.58, CLIP: 0.56, timeseries & our loss: 0.83

\textbf{Impact of Sensor Features and Pretraining Loss Functions}
We next evaluate the impact of sensor features and loss functions during offline encoder pretraining.
Specifically, we compare using statistical features (e.g., mean acceleration) versus raw time series inputs, and IMU2CLIP loss~\cite{moon-etal-2023-imu2clip} versus our proposed contrastive sensor-text pretraining loss for partial context (see Sec.\ref{sec:pretraining}). 
For IMU2CLIP, text samples are generated by combining all valid labels into one sentence. We report the online querying accuracy and multiple-choice answer accuracy to assess the influence to sensor information extraction.
As shown in Table~\ref{tbl:ablation-sensor-feature}, statistical features result in low querying and answer accuracies. This validates our motivation to design \Method that high-dimensional time series sensor data are critical for fine-grained activity information. IMU2CLIP training, even with fine-grained data, yields poorer querying and answering accuracies, highlighting its limited ability associating sensor embeddings from partial text query. Our proposed loss function, which aligns sensor and text encoders for partial context queries, proves more effective. These findings emphasize the importance of selecting appropriate sensor features and loss functions during pretraining in order to achieve high-performance QA.


\textbf{Impact of LLM Design Choices}
%Question decomposition, as the first stage, is critical for \Method diving into the correct direction.
We finally evaluate the impact of various design choices for LLMs. 
In question decomposition, we assess the contribution of in-context learning (ICL), chain-of-thought (CoT) techniques, and different backbone LLMs.
The results are summarized in Table~\ref{tbl:ablation-detailed-design}. 
Both ICL and CoT are crucial for high-quality and accurate answers. This is because an effective question decomposition improves sensor data queries. The solution templates in ICL are more essential to \Method as removing ICL reduces answer accuracy by 11\%. CoT enhances reasoning and slightly boosts accuracy by 1-5\%. Interestingly, using a more advanced backbone (GPT-4 vs. GPT-3.5-Turbo) results in minimal improvement in answer quality, as GPT-4, while generating richer text, does not follow instructions as well as GPT-3.5-Turbo according to our observation.

For answer assembly, we evaluate the effectiveness of finetuning compared to zero-shot or few-shot learning, as well as different LLaMA backbones. As shown in Table~\ref{tbl:ablation-detailed-design}, zero-shot learning results in poor performance, while few-shot learning improves answer quality but still lags behind finetuning. This highlights that finetuning is the most effective approach when a dataset like \Dataset is available. Using a more advanced LLaMA backbone, such as LLaMA3-8B, has minimal impact. Finetuning and the dataset prove to be more important than the model architecture during answer assembly.

%shows the performance of various question decomposition designs, including different LLMs and the use of in-context learning (ICL) and chain-of-thought (CoT) techniques.
%The closed-source GPT models generally yield more accurate decompositions due to their larger parameter sets.
%GPT-4 queries take 1x longer, while the latencies for queries without ICL and without CoT appear to be uncorrelated with prompt length, possibly due to OpenAI's internal setup.
%Interestingly, the latencies of GPT decompositions may not be directly related to prompt length. 
%Surprisingly, GPT-3.5-Turbo without ICL demonstrates the longest average latency, despite having shorter prompts without solution templates. This may be due to OpenAI's internal setup. GPT-4 queries take approximately 1x longer time to return. 
%Notably, using GPTs here only requires sending the questions to cloud but not the raw sensor data, thus still protecting sensitive information.
%The quantized LLaMA3-8B~\cite{lin2023awq} offers a trade-off between privacy and performance. Using quantized LLaMA3-8B preserves perfect privacy by avoiding transmitting both questions and sensor data to cloud, but leads to 4-11\% lower final answer accuracy and a longer latency per query of 8.1 seconds on a desktop.


%\textbf{Impact of Finetuning Designs}
%Table~\ref{tbl:ablation-finetuning}



\begin{table}[!t]
\small
\centering
\caption{Impact of various design choices for LLMs in \MethodC. Bold values highlight the best results.}
\vspace{-4mm}
\label{tbl:ablation-detailed-design}
\begin{tabular}{c|c|c|ccc|c} 
\toprule
\small
\textbf{Stage} & \textbf{Setup} & \textbf{Model in Stage} & \multicolumn{3}{c|}{\textbf{Full Answers}} & \textbf{Short Answer} \\ 
& & & Rouge-1 ($\uparrow$) & Rouge-2 ($\uparrow$) & Rouge-L ($\uparrow$) & Accuracy ($\uparrow$) \\
\midrule
& w/o ICL & GPT-3.5-Turbo & 0.75 & 0.59 & 0.72 & 0.43 \\
Question & w/o CoT & GPT-3.5-Turbo & 0.76 & 0.61 & 0.74 & 0.50 \\ 
Decomposition & Full & GPT-4 & \textbf{0.77} & \textbf{0.62} & \textbf{0.75} & 0.49 \\
& Full & GPT-3.5-Turbo & \textbf{0.77} & \textbf{0.62} & \textbf{0.75} & \textbf{0.54} \\ \hline
& Zero-shot learning & LLaMA2-7B & 0.16 & 0.07 & 0.14 & 0.0 \\
Answer & Few-shot learning & LLaMA2-7B & 0.43 & 0.29 & 0.41 & 0.24 \\
Assembly & Finetuning & LLaMA3-8B & 0.76 & 0.61 & 0.74 & 0.53 \\
& Finetuning & LLaMA2-7B & \textbf{0.77} & \textbf{0.62} & \textbf{0.75} & \textbf{0.54} \\
\bottomrule
\end{tabular}
\vspace{-2mm}
\end{table}








\iffalse
\begin{table}[!t]
\footnotesize
\centering
\vspace{-2mm}
\caption{Impact of designs in finetuning.}
\vspace{-2mm}
\label{tbl:ablation-finetuning}
\begin{tabular}{c|c|c|c} 
\toprule
\small
\textbf{Model} & \textbf{Method} & \textbf{Short Answer} & \textbf{Multiple Choices} \\ 
& & \textbf{Accuracy} & \textbf{Accuracy} \\ \hline
\multirow{2}{*}{LLaMA2-7B} & LoRA Finetuning & 0.54 & 0.69 \\
 & Full Finetuning & & \\
\multirow{2}{*}{LLaMA3-8B} & LoRA Finetuning & 0.53 & 0.70 \\
& Full Finetuning & \\
\bottomrule
\end{tabular}
%\vspace{-2mm}
\end{table}
\fi


\begin{figure}[t]
   \centering
    \setlength{\tabcolsep}{0.2pt}
\begin{tabular}{cccc}
        \vspace{-2mm}
        \includegraphics[width=0.22\textwidth, height=2.2cm]{figs/lr-gpt-shortened.png} &
        \includegraphics[width=0.22\textwidth, height=2.15cm]{figs/rank-gpt-shortened.png} &
        \includegraphics[width=0.22\textwidth, height=2.2cm]{figs/temperature-gpt-shortened.png} &
        \includegraphics[width=0.22\textwidth, height=2.15cm]{figs/query-threshold.png} \\ 
        %{\footnotesize (a) Working memory size} &
        %{\footnotesize (b) Novelty threshold} &
        %{\footnotesize (c) Merging sensitivity} \\
\end{tabular}
\vspace{-3mm}
    \caption{Sensitivity of key hyperparameters.}
    \label{fig:sensitivity}
    \vspace{-5mm}
\end{figure}



\subsection{Sensitivity Analyses}
%q, k, $\tau$, learning rate, lora rank
Fig.~\ref{fig:sensitivity} shows the sensitivity of key parameters in \Method.
while the less sensitive ones are omitted due to space limitation.
%We focus on short answer accuracy, as other metrics show similar trends.
The default parameter setting is the same as described in Sec.~\ref{sec:system-implementation}.
We mainly focus on evaluating the short answers to assess the parameters' impact on factual information extraction.

\textbf{Learning Rate in Answer Assembly} 
Fig.~\ref{fig:sensitivity} (leftmost) shows the short answers accuracy for learning rates of $\{1e-5, 5e-5, 1e-4, 2e-4, 5e-4\}$ during finetuning.
Larger learning rates result in bigger gradient steps during LoRA finetuning, with $1e-4$ providing the best performance for our task.

\textbf{LoRA Rank in Answer Assembly}
Fig.~\ref{fig:sensitivity} (middle left) shows the short answers accuracy for LoRA ranks of $\{8, 16, 32, 64\}$ during finetuning.
The rank affects the size of the LoRA adapter weights. Higher ranks mean more parameters and a larger weight space to optimize. 
For our task, varying ranks have little impact on final accuracy, with rank 32 achieving the best performance for short answers.

\textbf{Generating Temperature in Answer Assembly}
Fig.~\ref{fig:sensitivity} (middle right) shows the short answers accuracy for generating temperatures of $\{0.01, 0.1, 0.2, 0.5\}$ during the final answer generation.
Higher temperatures instruct the LLM to use more ``creativity''. For our task, varying temperatures have negligible impact on the short answers accuracy, indicating minimal impact to the sensor information extraction.

\textbf{Query Threshold in Sensor Data Query} Fig.~\ref{fig:sensitivity} (rightmost) shows the F1 scores of online querying at different query thresholds $h=\{0.2, 0.4, 0.5, 0.6, 0.8\}$. %As discussed in Sec.~\ref{sec:ablation}, the higher the BA, the more accurate the activity classification, and the better answer accuracy we get from \Method.
%While increasing the threshold reduces the BA, the reduction is minimal thus \Method is generally robust to various $h$.
Raising the threshold $h$ excludes less confident positive predictions, which can improve F1 scores. However, this may also overlook some detailed events, potentially reducing answer accuracy. Ideally, $h$ should be calibrated individually for each user to achieve the best results.

\iffalse
\textbf{Temperature during answer generation}
Fig.~\ref{fig:sensitivity} (right) shows the final accuracy for temperatures of $\{0.01, 0.1, 0.2, 0.5\}$.
Temperature balances LLaMA's predictability and creativity. A higher temperature encourages exploration and can be helpful in answering creative questions. However, on the short answer dataset, temperature has little impact on final performance.
\fi


\begin{figure}[t]
  \begin{subfigure}[b]{0.55\textwidth}
        \centering
        \includegraphics[width=\textwidth]{figs/legend_split.png}
        \vspace{-5mm}
    \end{subfigure}

    \begin{subfigure}[b]{0.35\textwidth}
        \centering
        \includegraphics[width=\textwidth]{figs/exact_split.png}
        \vspace{-6mm}
        \caption{Average short-answer accuracy on different users.}
        \label{fig:exact_split}
    \end{subfigure} \hspace{0.02\textwidth} % Add horizontal space between subfigures if needed
    \begin{subfigure}[b]{0.35\textwidth}
        \centering
        \includegraphics[width=\textwidth]{figs/acc_split.png}
        \vspace{-6mm}
        \caption{Average online querying accuracy on different users during sensor data query.}
        \label{fig:acc_split}
    \end{subfigure} 
    \vspace{-4mm}
    \caption{Generalization of key learning components in \Method.}
    \label{fig:generalization}
    \vspace{-4mm}
\end{figure}

\subsection{Generalization and Robustness}
\label{sec:generalizability}
We evaluate \Method's generalization and robustness across different users' sensor data and QA inputs. In addition to the standard 80/20 random split, we compare results with a split where training is performed on the first 48 out of 60 users and testing includes all users. To ensure a fair comparison, we equalize the training set size in both splits by duplicating samples in the smaller set.
Our evaluation focuses on two key learning processes in \Method: LLM fine-tuning in answer assembly and sensor-text encoder pretraining.

Fig.~\ref{fig:exact_split} presents the short-answer accuracy when fine-tuning on all users' QA data versus only the first 48 users. The results show similar accuracy for both seen and unseen users, demonstrating \Method's strong generalizability in answer assembly. This is likely due to \Method's design of treating answer assembly as a pure language task. Since all users' language tokens follow similar distributions in a sensor-based QA task, generalization remains robust across user variations.

Fig.\ref{fig:acc_split} presents the online querying accuracy when pretraining with all users' sensor data versus only the first 48 users. Unlike language fine-tuning, limiting sensor data to the first 48 users leads to accuracy degradation on unseen users due to variations in data distributions. Therefore, improving \Method's generalization to new users primarily depends on developing a robust sensor and text encoder, which is a well-studied problem in existing literature~\cite{xu2023practically}. We leave the investigation for combining with these techniques in future work.
%The performance of \Method mainly depends on two learning components: the sensor and text encoder obtained in pretraining, and the finetuned LLM in answer assembly. Therefore we assess the generalization of \Method 

%%%%%%%%%%%%%%%%%%%%%%%%%%%%%%%%%%%%%%%%%%%%%%%%%%%
% Evaluation on real deployment
%%%%%%%%%%%%%%%%%%%%%%%%%%%%%%%%%%%%%%%%%%%%%%%%%%%
\section{Real User Study}
\label{sec:deployment}


We conduct a user study to evaluate \Method's real-world applicability. 
In addition to dataset-based evaluations in Sec.~\ref{sec:evaluation} which focus on quantitative questions, this study places \Method into the ``wilderness''. Participants will interact with \Method immediately after data collection and may pose arbitrary questions, especially the qualitative and open-ended ones.
Our study is approved by the Institutional Review Board Committee.

\textbf{User Study Setup} We recruited eight volunteers (five males and three females) who were instructed to follow their normal daily routines while carrying a smartphone with the ExtraSensoryApp~\cite{vaizman2018extrasensory}. The smartphone models include Huawei Mate 10 Pro, LG G7 ThinQ and Google Pixel 2. The app automatically collected multimodal sensor data and transmitted it to \Method whenever a network connection was available. Reporting activity labels was optional for the participants. The data collection phase lasted one to three days, with valid samples ranging from 52 to 1366 minutes. Sensor data availability varied due to factors such as phone model and usage patterns. After the data collection phase, volunteers were invited to interact with \MethodC in-person through a chat-based graphical interface. They were encouraged to ask any questions about their lives during the data collection phase and observe \MethodC's response generation in real time. Finally, we gathered feedback from the participants including ratings on answer content, latency, and practical value of \Method on a scale from 1 to 5. %, with 1 being the least favorable and 5 being the most favorable. 
%
We specifically select \MethodC for the user study to evaluate \Method's full functionality in real-world conditions. We leave user evaluation and optimization of \MethodE in a purely edge scenario for future work.


\textbf{User Feedback Results}
Fig.~\ref{fig:user_study} displays the quantitative feedback ratings gathered from the eight participants. \textbf{\Method received an average score of 3.12 for answer content, 4.50 for latency, and 4.00 for practical utility}, supporting \Method's applicability in real life scenarios.
Participant comments praised \Method's natural responses, e.g., ``\textit{The answers were formatted in an easy to understand way}''.
However, concerns were also raised about quantitative accuracy, such as ``\textit{some numbers were a little off}'' and ``\textit{it mentioned activities I never did}''.
We attribute these issues to the limited performance of the sensor encoder when generalizing from a dataset to real-world users, as we observed noisy outputs from the sensor query stage, occasionally detecting activities that never occurred.
Fortunately, generalizing models trained on one dataset to different settings is a well-studied problem~\cite{xu2023practically}. Integrating these techniques into \Method could enhance its practical performance, which we leave for future investigation.
%If the sensor encoder does not transfer well to a new user, the resulting queried data and answers can be inaccurate. 

%More data and few-shot transfer learning could help, which we plan to explore in future work. 
In terms of latency, all participants give positive feedbacks regarding the end-to-end answer generation latency, with comments such as "\textit{I do not feel as if I had to wait a long time for the answers}" and "\textit{I think it is faster than I thought previously}".
Most participants are positive in terms of the practical utility of \Method in their daily lives, e.g., ``\textit{I believe it can help with my wellness management significantly.}'' The less positive comments mentioned the challenge of adapting \Method to individual users for generating more personalized and useful responses, an issue that we leave for future exploration.

%i.e., adapting both sensor encoder and the LLM components to better fit specific users' lifestyle and needs, 
%can be attributed to two potential reasons on the model's performance: (1) the sensor encoder may not predict accurately, and (2) the LLM components may not perform optimally for specific participants' requests.

%Based on the detailed comments from participants, \Method gives more convincing answers in qualitative questions such as ``''The quality of \Method's answers can be potentially improved from...

\textbf{Generalization to Qualitative and Open-Ended Questions}
Participants have asked creative and open-ended questions to \Method that do not exist in \Dataset, providing a more comprehensive assessment of \Method's practical utility.
%For example, one participant asked, ``Was I in a crowded area yesterday?''. While it is not instantly obvious which labels or sensors are relevant, \Method is able to reason by decomposing the question into an activity query, and infers that the participant likely visited crowded places based on labels such as "in school," "talking," and "with friends."
For example, one participant asked, ``\textit{How would you rate my lifestyle and what improvements do you suggest?}'' \Method responded with, ``\textit{I would rate this lifestyle as 2 out of 5 stars and would suggest you improve on your exercise and talking,}'' based on the duration of each activity where exercise and talking were lacking.
The three-stage pipeline, especially the explicit sensor data query stage, enables \Method to analyze users' overall lifestyle and fitness, in addition to answering quantitative questions in \Dataset~\citesensorqa. 
Specifically, \Method achieves this through comprehensively querying a list of activity schedule in the sensor data query stage and then analyzing them with LLM during the answer assembly stage.
Participants appreciated \Method's ability to effectively handle both quantitative and qualitative questions, e.g., "\textit{I think it could be very useful to be able to answer these qualitative and quantitative questions about my lifestyle,}" highlighting its broad applicability to diverse real-life scenarios.

%With its LLM-powered backbone, \Method shows notable expertise in answering these kinds of creative questions. %, which clearly advances the capabilities of existing systems.

\begin{figure*}[t]
  \centering
  \includegraphics[width=0.98\textwidth]{figs/user_study.png}
  \vspace{-4mm}
  \caption{Satisfactory ratings from eight participants on three questions.}
  \vspace{-6mm}
  \label{fig:user_study}
\end{figure*}

%%%%%%%%%%%%%%%%%%%%%%%%%%%%%%%%%%%%%%%%%%%%%%%%%%%
% Discussion \& Future Work
%%%%%%%%%%%%%%%%%%%%%%%%%%%%%%%%%%%%%%%%%%%%%%%%%%%
\section{Discussion \& Future Work}
\label{sec:future-work}

%In this section, we discuss various aspects of \Method and potential directions for future work.

\textbf{Generalizability of \Method}
In the quantitative study, we mainly focus on \Dataset~\citesensorqa as it is the only available sensor-based QA dataset that aligns with our problem statement. However, \Method is not limited to this dataset. As demonstrated in the user study (Sec.\ref{sec:deployment}), \Method can generalize to broader real-life scenarios and answer high-level qualitative questions.
Moreover, the design of \Method can be easily expanded.
Specifically, \Method can incorporate more sensors via adding additional modality encoders.
It can also adapt to more diverse questions by expanding the solution templates or using more powerful LLMs such as OpenAI's o3-mini~\cite{openaio3} or DeepSeek~\cite{deepseek} during question decomposition.
Last but not least, \Method can serve a larger user base by collecting more data and equipping with better generalization designs such as~\cite{xu2023practically}. \Method is a general and flexible framework designed for fusing long-duration multimodal sensors knowledge with language and answering versatile questions.


%Generalizability of SensorChat: extension to larger number of users and more sensors. If we have more data available, SensorChat’s performance can be further improved
%Generalizability of SensorChat to more complex questions

\textbf{Personalization of \Method} This work emphasizes \Method's ability to generalize across a broad range of questions and users. However, providing personalized responses is equally crucial, particularly for health-related analyses. We foresee the potential to tailor \Method for individual users by personalizing the LLMs~\cite{kim2024health}, a direction we intend to explore in future research.

\textbf{Evaluation Metrics} \Method uses traditional NLP metrics and accuracy to evaluate the quality and precision of generated answers. While these metrics excel in objectiveness, they do not capture subjective user satisfaction, which is crucial for assessing a system's readiness for a broad market. 
We leave a comprehensive evaluation of \Method with subjective metrics as our future work.



\textbf{Efficiency of \Method} In this work, we demonstrate that \Method can be deployed on Jetson-level edge devices using common quantization techniques. However, further optimization is needed for real-time LLM service on edge devices. The long-term goal for systems like \Method is to run on mobile devices. Recent studies~\cite{liu2024mobilellm, zhuang2024litemoe} have explored enabling real-time LLM serving on mobile devices, and these techniques could be integrated into \Method in future work. Additionally, we recognize the potential synergy between \Method and vector databases~\cite{zhou2024llm} or retrieval-augmented generation~\cite{zhao2024retrieval}. The techniques in these domains can be integrated into \Method to improve query performance and efficiency.
%We discuss the potential to further improve \Method with vector databases~\cite{zhou2024llm} or retrival augmented generation~\cite{zhao2024retrieval} in Sec.~\ref{sec:future-work}.


%%%%%%%%%%%%%%%%%%%%%%%%%%%%%%%%%%%%%%%%%%%%%%%%%%%
% Conclusion
%%%%%%%%%%%%%%%%%%%%%%%%%%%%%%%%%%%%%%%%%%%%%%%%%%%
\section{Conclusion}
\label{sec:conclusion}

Natural interactions between users and multimodal sensors can unlock the full potential of sensor data in real-world applications. However, existing systems struggle to process long-duration, high-dimensional sensor data, often resulting in unsatisfactory answers to long-term questions.
In this paper, we present \Method, the first edge system designed to handle long-duration, time-series sensor data for both qualitative and quantitative question answering. \Method introduces a novel three-stage pipeline including question decomposition, sensor data query and answer assembly.
All stages and interfaces are carefully designed to effectively fuse the natural language and the sensor modality.
Evaluation results and user studies demonstrate \Method's effectiveness in answering a wide range of qualitative and quantitative questions. The cloud version of \Method system supports real-time interactions, while the edge version is optimized to run on Jetson-level devices.
%show that \Method enhances the answer accuracy by up to 26\% compared to state-of-the-art QA systems while achieving an average generating latency of 0.63s on the desktop.
%We further validate the practical value and challenges with \Method in a real user study.

%%
%% The acknowledgments section is defined using the "acks" environment
%% (and NOT an unnumbered section). This ensures the proper
%% identification of the section in the article metadata, and the
%% consistent spelling of the heading.
\begin{acks}
This work was supported in part by National Science Foundation under Grants \#2003279, \#1826967, \#2100237, \#2112167, \#1911095, \#2112665, \#2120019, \#2211386 and in part by PRISM and CoCoSys, centers in JUMP 2.0, an SRC program sponsored by DARPA.
\end{acks}
%%
%% The next two lines define the bibliography style to be used, and
%% the bibliography file.
\bibliographystyle{ACM-Reference-Format}
\bibliography{sample-base}


%%
%% If your work has an appendix, this is the place to put it.
\appendix
\section{Summarization Function Design}
\label{sec:query-function}
%The design of functions must balance the ability to handle various Q\&A types with the complexity of obtaining accurate question decomposition. 

We design the following summarization functions to handle various real-life QA scenarios.
\begin{itemize}
    \item \texttt{CalculateDuration} is used for time comparisons, time queries, and existence questions.
    \item \texttt{DetectActivity} handles activity queries. \texttt{CountingFrequency} and \texttt{CountingDays} handle single- and multi-day counting. 
    \item \texttt{DetectFirstTime} and \texttt{DetectLastTime} are used for specific timestamp queries.
\end{itemize}
%To handle diverse queries, one may need a large number of query functions and arguments.
We emphasize that the function set can be expanded in the future to accommodate a broader range of queries.
However, since the summarization function is selected by the LLM during question decomposition, it is crucial to limit the number of functions to avoid overwhelming or confusing the model.
Fortunately, one strength of \Method is its ability to transform questions into existing functions during question decomposition, eliminating the need to create new ones unnecessarily.
For instance, a question like "What did I do right after waking up?" can be decomposed into calls to \texttt{DetectFirstTime} and \texttt{DetectLastTime}, rather than requiring the creation of a new function such as \texttt{DetectNextActivity}.
To enable this transformation, we include a list of available functions in the prompt, as shown in Fig.~\ref{fig:question-decompose}.
This approach ensures that the current set of six summarization functions is sufficient to effectively handle the vast majority of user queries.






\end{document}
\endinput
%%
%% End of file `sample-sigconf.tex'.
