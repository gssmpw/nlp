\section{Summarization Function Design}
\label{sec:query-function}
%The design of functions must balance the ability to handle various Q\&A types with the complexity of obtaining accurate question decomposition. 

We design the following summarization functions to handle various real-life QA scenarios.
\begin{itemize}
    \item \texttt{CalculateDuration} is used for time comparisons, time queries, and existence questions.
    \item \texttt{DetectActivity} handles activity queries. \texttt{CountingFrequency} and \texttt{CountingDays} handle single- and multi-day counting. 
    \item \texttt{DetectFirstTime} and \texttt{DetectLastTime} are used for specific timestamp queries.
\end{itemize}
%To handle diverse queries, one may need a large number of query functions and arguments.
We emphasize that the function set can be expanded in the future to accommodate a broader range of queries.
However, since the summarization function is selected by the LLM during question decomposition, it is crucial to limit the number of functions to avoid overwhelming or confusing the model.
Fortunately, one strength of \Method is its ability to transform questions into existing functions during question decomposition, eliminating the need to create new ones unnecessarily.
For instance, a question like "What did I do right after waking up?" can be decomposed into calls to \texttt{DetectFirstTime} and \texttt{DetectLastTime}, rather than requiring the creation of a new function such as \texttt{DetectNextActivity}.
To enable this transformation, we include a list of available functions in the prompt, as shown in Fig.~\ref{fig:question-decompose}.
This approach ensures that the current set of six summarization functions is sufficient to effectively handle the vast majority of user queries.


