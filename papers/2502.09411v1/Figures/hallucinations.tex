\begin{figure}[htpb]
    \centering
    \setlength{\tabcolsep}{0.2pt}
\setlength{\abovecaptionskip}{5pt}
    \renewcommand{\arraystretch}{0.5}

\begin{adjustbox}{max width=\columnwidth}
    \begin{tabular}{c@{\hskip 0.5em} c@{\hskip 0.2em} c@{\hskip 0.2em} c}

        & Model output & +Reference & ImageRAG \\

        \raisebox{0.44in}{\rotatebox[origin=t]{90}{\fontsize{8.8pt}{8.8pt}\selectfont Cradle (SDXL)}} &
        \includegraphics[clip,width=25mm]{Assets/hallucinations/prompt_516_no_imageRAG.png} &
        \includegraphics[clip,width=25mm]{Assets/hallucinations/image_325036.jpg} &
        \includegraphics[clip,width=25mm]{Assets/hallucinations/prompt_516.png}
        \\
        \raisebox{0.41in}{\rotatebox[origin=t]{90}{\fontsize{8.8pt}{8.8pt}\selectfont Chime (OmniGen)}} &
        \includegraphics[clip,width=25mm]{Assets/hallucinations/prompt_494_no_imageRAG.png} &
        \includegraphics[clip,width=25mm]{Assets/hallucinations/image_746802.jpg} &
        \includegraphics[clip,width=25mm]{Assets/hallucinations/prompt_494.png}

    \end{tabular}
\end{adjustbox}
    \caption{
    \textbf{Hallucinations.} When models do not know the meaning of a prompt, they may ``hallucinate'' and generate unrelated images (left).
    By applying our method to retrieve and utilize relevant references (mid), the base models can generate appropriate images (right).
    }
    \label{fig:hallucinations}\vspace{-8pt}
\end{figure}