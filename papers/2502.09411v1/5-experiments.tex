\vspace{-4pt}
\section{Experiments}
\label{sec:experiments}

In this section, we describe quantitative and qualitative experiments performed to evaluate the effectiveness of our method. To assess its adaptability to different model types, we experiment with applying \emph{ImageRAG} to two models, namely OmniGen~\cite{xiao2024omnigen} and SDXL~\cite{podellsdxl}, each representing a different model type.

\vspace{-1pt}
\subsection{Quantitative comparisons}
\label{sec:quant}
\begin{table*}
\caption{Comparisons on fine-grained image generation with text-to-image models. We use the ImageNet~\cite{deng2009imagenet}, iNaturalist~\cite{van2018inaturalist}, CUB~\cite{wah2011caltech}, and Aircraft~\cite{majifine} datasets. 
For each set, we report mean ($\pm$ standard error) CLIP, SigLIP text-to-image similarities, and DINO similarity between real and generated images. 
Middle rows feature OmniGen-based models, while the bottom features SDXL-based models.
In each part, best results are \textbf{bolded}. 
}
  \label{tab:long_tail}
  \adjustbox{max width=1.005\linewidth}{
  \centering
  \begin{tabular}{@{}ccccccccccccc}
    \toprule
     & \multicolumn{3}{c}{ImageNet} & \multicolumn{3}{c}{iNaturalist} & \multicolumn{3}{c}{CUB} &
     \multicolumn{3}{c}{Aircraft} \\
    \cmidrule(lr){2-4} \cmidrule(lr){5-7} \cmidrule(lr){8-10} \cmidrule(lr){11-13}
     & CLIP $\uparrow$ & SigLIP $\uparrow$ & DINO $\uparrow$ & CLIP $\uparrow$ & SigLIP $\uparrow$ & DINO $\uparrow$ & CLIP $\uparrow$ & SigLIP $\uparrow$ & DINO $\uparrow$ & CLIP $\uparrow$ & SigLIP $\uparrow$ & DINO $\uparrow$ \\ 
    \midrule 
    FLUX & 
    $0.262 \pm 0.001$ & $0.132 \pm 0.001$ & $0.711 \pm 0.003$ 
    & $0.201 \pm 0.002$ & $0.048 \pm 0.002$ & $0.644 \pm 0.002$ 
    & $0.254 \pm 0.004$ & $0.126 \pm 0.003$ & $0.759 \pm 0.004$
    & $0.243 \pm 0.007$ & 
    $0.116 \pm 0.006$ & $0.725 \pm 0.011$ \\
    Pixart-$\Sigma$ &
    $0.262 \pm 0.001$ & $0.121 \pm 0.001$ & $0.691 \pm 0.003$
    & $0.162 \pm 0.002$ & $0.027 \pm 0.002$ & $0.611 \pm 0.002$
    & $0.232 \pm 0.004$ & $0.101 \pm 0.003$ & $0.736 \pm 0.004$
    & $0.160 \pm 0.008$ & 
    $0.054 \pm 0.006$ & $0.634 \pm 0.011$ \\
    \midrule
    OmniGen & $0.247 \pm 0.002$ & $0.122 \pm 0.001$ & $0.692 \pm 0.003$ & 
    $0.155 \pm 0.002$ & $0.014 \pm 0.001$ & $0.595 \pm 0.002$ & 
    $0.231 \pm 0.005$ & $0.109 \pm 0.003$ & $0.747 \pm 0.005$ & 
    $0.181 \pm 0.010$ & $0.073 \pm 0.007$ & $0.656 \pm 0.013$ \\ 
    GraPE-O & $0.251 \pm 0.002$ & $0.123 \pm 0.001$ & $0.692 \pm 0.003$ & 
    $0.157 \pm 0.002$ & $0.016 \pm 0.002$ & $0.604 \pm 0.001$ & $0.240 \pm 0.005$ & $0.115 \pm 0.003$ & $0.747 \pm 0.005$ & 
    $0.191 \pm 0.009$ & $0.073 \pm 0.007$ & $0.647 \pm 0.013$ \\
    ImageRAG-O & \textbf{0.264} $\pm$ \textbf{0.001} & \textbf{0.134} $\pm$ \textbf{0.001} & \textbf{0.708} $\pm$ \textbf{0.002} & 
    \textbf{0.197} $\pm$ \textbf{0.002} & \textbf{0.095} $\pm$ \textbf{0.002} & \textbf{0.701} $\pm$ \textbf{0.002} & \textbf{0.253} $\pm$ \textbf{0.003} & \textbf{0.125} $\pm$ \textbf{0.002} & \textbf{0.760} $\pm$ \textbf{0.003} & 
    \textbf{0.228} $\pm$ \textbf{0.006} & \textbf{0.103} $\pm$ \textbf{0.005} & \textbf{0.747} $\pm$ \textbf{0.010} \\ 
    \midrule
    SDXL & $0.267 \pm 0.002$ & $0.136 \pm 0.001$ & $0.700 \pm 0.003$ &
    \textbf{0.259} $\pm$ \textbf{0.002} & $0.096 \pm 0.002$ & $0.698 \pm 0.003$ &
    \textbf{0.315} $\pm$ \textbf{0.001} & $0.172 \pm 0.003$ & $0.782 \pm 0.002$ & 
    $0.264 \pm 0.006$ & \textbf{0.145} $\pm$ \textbf{0.005} & \textbf{0.771} $\pm$ \textbf{0.010} \\
    ImageRAG-SD & \textbf{0.274} $\pm$ \textbf{0.001} & \textbf{0.141} $\pm$ \textbf{0.001} & \textbf{0.709} $\pm$ \textbf{0.002} & $0.243 \pm 0.002$ &
    \textbf{0.118} $\pm$ \textbf{0.001} & \textbf{0.724} $\pm$ \textbf{0.002}
    & $0.314 \pm 0.001$ & \textbf{0.174} $\pm$ \textbf{0.002} & \textbf{0.784} $\pm$ \textbf{0.001} & 
    \textbf{0.272} $\pm$ \textbf{0.005} & $0.141 \pm 0.005$ & $0.756 \pm 0.011$ \\
    \bottomrule
  \end{tabular}
  }
\end{table*}
We evaluate the ability of our method to improve T2I generation of rare and fine-grained concepts by comparing the results of OmniGen and SDXL with their results when applying \emph{ImageRAG} to them.
As additional baselines, we compare with FLUX~\cite{flux2023}, Pixart-$\Sigma$~\cite{chen2025pixart}, and GraPE~\cite{goswami2024grape}. The last is an iterative LLM-based image generation method which employs editing tools to insert missing objects. We use their OmniGen-based version.
We use each method to generate images of each class in the following datasets: ImageNet~\cite{deng2009imagenet}, iNaturalist~\cite{van2018inaturalist}, CUB~\cite{wah2011caltech}, and Aircraft~\cite{majifine}.
For iNaturalist, we use the first 1000 classes.
\cref{tab:long_tail} shows evaluation results of all methods using CLIP~\cite{radford2021learning}, SigLIP~\cite{zhai2023sigmoid} and DINO~\cite{zhangdino} similarities. For fairness, we use open-CLIP for evaluation, while using OpenAI CLIP for retrieval.
As demonstrated in \cref{tab:long_tail}, both OmniGen and SDXL results improve when using our method for the generation of rare concepts and fine-grained categories.

\vspace{-3pt}
\subsection{Proprietary Data Generation}
\begin{table}[!thb]
\caption{Proprietary data usage experiment. Results for using each dataset as the retrieval-dataset (``Proprietary-\textless{}model\textgreater{}'') vs.\ using our subset from LAION as the retrieval-dataset (``LAION-\textless{}model\textgreater{}'').
Here, ``O'' indicates OmniGen based models, ``SD'' indicates SDXL based models. 
Best results for each model are \textbf{bolded}.
}
  \label{tab:prop_ds}
  \adjustbox{max width=\columnwidth}{
  \centering
  \begin{tabular}{@{}ccccccc}
    \toprule
     & \multicolumn{3}{c}{ImageNet}  
     & \multicolumn{3}{c}{Aircraft} \\
    \cmidrule(lr){2-4} \cmidrule(lr){5-7} 
     & CLIP $\uparrow$ & SigLIP $\uparrow$ & DINO $\uparrow$ & 
     CLIP $\uparrow$ & SigLIP $\uparrow$ & DINO $\uparrow$ \\ 
    \midrule
    LAION-O &
    $0.264 \pm 0.001$ & $0.134 \pm 0.001$ & $0.708 \pm 0.002$ &
    $0.228 \pm 0.006$ & $0.103 \pm 0.005$ & $0.747 \pm 0.010$ 
     \\
    Proprietary-O &
    \textbf{0.266} $\pm$ \textbf{0.001} & \textbf{0.136} $\pm$ \textbf{0.001} & \textbf{0.710} $\pm$ \textbf{0.002} &
    \textbf{0.244} $\pm$ \textbf{0.007} & \textbf{0.109} $\pm$ \textbf{0.005} & 
    \textbf{0.786} $\pm$ \textbf{0.010} 
     \\ 
    \midrule
    LAION-SD &
    $0.274 \pm 0.001$ & $0.141 \pm 0.001$ & $0.709 \pm 0.002$ &
     $0.272 \pm 0.005$ & $0.141 \pm 0.005$ & $0.756 \pm 0.011$ \\
    Proprietary-SD & 
    \textbf{0.288} $\pm$ \textbf{0.001} & \textbf{0.142} $\pm$ \textbf{0.001} & \textbf{0.736} $\pm$ \textbf{0.003} &
     \textbf{0.280} $\pm$ \textbf{0.005} & \textbf{0.152} $\pm$ \textbf{0.003} & \textbf{0.785} $\pm$ \textbf{0.009} \\
    \bottomrule
  \end{tabular}
  }
\end{table}
A common use for RAG in NLP is generation based on proprietary data \cite{lewis2020retrieval},
where the retrieval dataset is a proprietary one.
A similar application in image generation would be generating images based on a proprietary gallery of images. 
It could be for personalization, where the gallery is of a personal concept, e.g. images of a person's dog, or it could be a company brand or a private collection of images that could broaden the knowledge of a model.
Our LAION-based experiments explored the scenario where a user has access to a general, large-scale set. Here, we further evaluate the performance of \emph{ImageRAG} when we have access to a potentially smaller, specialized dataset. Hence, we repeat the experiments with the datasets used in \cref{tab:long_tail}, but this time retrieve samples from within each dataset rather than from the LAION~\cite{schuhmann2022laion} subset.
 Results are reported in \cref{tab:prop_ds,tab:prop_ds_app}.
 We observe that although applying our method with the generic dataset of a relatively small subset from LAION already improves the results, they improve even further when using the proprietary datasets for retrieval.

\vspace{-3pt}
\subsection{Ablation Studies}
\label{subsec:ablations}

\paragraph{Synthetic multiple} 
Thus far, we have exclusively considered that the number of generated synthetic records equals the number of records in the real data, \ie, $N = \synthetic{N}$. We now consider the case when more synthetic data is made available to a data-based adversary ($\synthetic{\mathcal{A}}$). Specifically, we denote the \emph{synthetic multiple} $m = \nicefrac{\synthetic{N}}{N}$ and evaluate how different MIAs perform for varying values of $m$.
%
Figure~\ref{fig:synthetic_multiple} shows how the ROC AUC score varies as $m$ increases. As expected, the ROC AUC score for the attack that uses membership signals computed using a 2-gram model trained on synthetic data increases when more synthetic data is available. In contrast, attacks based on similarity metrics do not seem to benefit significantly from this additional synthetic data.

\begin{figure}[htb]
  \centering
  \begin{subfigure}{0.4\textwidth}
    \centering
    \includegraphics[width=\textwidth]{figures/synthetic_multiple_sst2.pdf}
  \end{subfigure}
  \hspace{0.05\textwidth}
  \begin{subfigure}{0.4\textwidth}
    \includegraphics[width=\textwidth]{figures/synthetic_multiple_agnews.pdf}
  \end{subfigure}
  \caption{ROC AUC score for increasing value of the synthetic multiple $m$ across data-based attack methods for SST-2 (left) and AG News (right). Canaries are synthetically generated with target perplexity of $\mathcal{P}_{\textrm{target}}=250$,  with a natural label, with no in-distribution prefix ($F=0$), and inserted $n_\textrm{rep}=12$ times.} 
  \label{fig:synthetic_multiple}
\end{figure} 


\paragraph{Hyperparameters in data-based attacks}
The data-based attacks that we presented in Sec.~\ref{sec:membership_method} rely on certain hyperparameters.
%
The attack that uses $n$-gram models to compute membership signals is parameterized by the order $n$. Using a too small value for $n$ might not suffice to capture the information leaked from canaries into the synthetic data used to train the $n$-gram model. When using a too large order $n$, on the other hand, we would expect less overlap between $n$-grams present in the synthetic data and the canaries, lowering the membership signal.

Further, the similarity-based methods rely on the computation of the mean similarity of the closest $k$ synthetic records to the a canary. When $k$ is very small, \eg $k=1$, the method takes into account a single synthetic record, potentially missing on leakage of membership information from other close synthetic data records. When $k$ becomes too large, larger regions of the synthetic data are taken into account, which might dilute the membership signal among the noise.

Table~\ref{tab:ablations_synthetic} reports the ROC AUC scores of data-based attacks for different values of the hyperparameters $n$ and $k$ when using standard canaries (Sec.~\ref{sec:baseline_results}). We find that for both datasets, training a $2$-gram model on the synthetic data to compute the membership signal yields the best performance. For the data-based MIAs relying on the similarity between the canary and the synthetic records, both when considering Jaccard distance and cosine distance in the embedding space, we find that considering the $k=25$ closest synthetic records yields the best performance. 

\begin{table}[ht]
    \centering
    \begin{tabular}{ccc@{\hskip 20pt}cc@{\hskip 20pt}cc}
    \toprule
         & \multicolumn{2}{c}{$n$-gram} 
         & \multicolumn{2}{c}{$\textsc{SIM}_\textrm{Jac}$} 
         & \multicolumn{2}{c}{$\textsc{SIM}_\textrm{emb}$} \\
        \cmidrule(lr){2-3} \cmidrule(lr){4-5} \cmidrule(lr){6-7}
        Dataset & $n$ & AUC & $k$ & AUC & $k$ & AUC\\
        \midrule
        \multirow{4}{*}{\parbox{1.8cm}{\centering SST-2}} 
        & $1$ & $0.415$ & $1$ & $0.520$ & $1$ & $0.516$ \\ 
        & $2$ & \bm{$0.616$} & $5$ & $0.535$ & $5$ & $0.516$ \\ 
        & $3$ & $0.581$ & $10$ & $0.538$ & $10$ & $0.519$ \\ 
        & $4$ & $0.530$ & $25$ & \bm{$0.547$} & $25$ & \bm{$0.530$} \\   
        \midrule
        \multirow{4}{*}{\parbox{1.8cm}{\centering AG News}} 
        & $1$ & $0.603$ & $1$ & $0.522$ & $1$ & $0.503$ \\ 
        & $2$ & \bm{$0.644$} & $5$ & $0.525$ & $5$ & $0.498$ \\ 
        & $3$ & $0.567$ & $10$ & $0.537$ & $10$ & $0.503$ \\ 
        & $4$ & $0.527$ & $25$ & \bm{$0.552$} & $25$ & \bm{$0.506$} \\        
        \bottomrule
    \end{tabular}
    \caption{Ablation over hyperparameters of data-based MIAs. We report ROC AUC scores across different values of the hyperparameters $n$ and $k$ (see Sec.~\ref{sec:membership_method}). Canaries are synthetically generated with target perplexity $\mathcal{P}_\textrm{target}=250$, with a natural label, with no in-distribution prefix ($F=0$), and inserted $n_\textrm{rep}=12$ times.}
    \label{tab:ablations_synthetic}
\end{table}
To evaluate the contribution of each part of our method, we conduct an ablation study testing different components and report our results in \cref{tab:ablations}.
First, we want to ensure the performance gap is not based on simply interpreting rare words using an LLM. Hence, we evaluate OmniGen and SDXL over rephrased text prompts, without providing reference images. 
To do so, we asked GPT to rephrase the prompts, to make them easier for a T2I generative model, by explicitly asking it to change rare words to their description if necessary. The full prompt can be found in \cref{app:prompts}.
As the results show, rephrasing was not enough for a meaningful improvement in the results (``Rephrased prompt'' in \cref{tab:ablations}).
Next, we investigate the importance of using detailed image captions for retrieval, rather than just listing the missing concepts or using the original prompt. We do so by evaluating our method when retrieving the concepts directly without generating compatible image captions for each missing concept (``Retrieve concepts'' in \cref{tab:ablations}), and when retrieving the prompt directly (``Retrieve prompt'' in \cref{tab:ablations}). While retrieval with each of them introduced some improvement over the initial results, retrieving detailed captions improved the results even further.

\begin{figure}[htp]
  \centering
   \includegraphics[width=1.0\columnwidth]{Assets/num_samples_to_clip.pdf}
   \caption{\textbf{Retrieval dataset size vs. CLIP score on ImageNet (left) and Aircraft (right).} 
   Dashed lines represent the scores of the base models.
   Even relatively small, unspecialized retrieval sets can already improve results. More data leads to further increased scores. However, small sets may not contain relevant retrieval examples, and their use may harm results, particularly for stronger models.
   }
   \label{fig:samples_to_score}
\end{figure}
Next, we investigate the effect of the retrieval-dataset size. We tested our method over ImageNet~\cite{deng2009imagenet} and Aircraft~\cite{majifine} when using 
% 0 (the original models), 
\num{1000}, \num{10000}, \num{100000}, and \num{350000} examples from LAION~\cite{schuhmann2022laion}. \cref{fig:samples_to_score} shows that increasing the dataset size typically leads to better results. However, even using a relatively small dataset can already lead to improvements. For OmniGen, \num{1000} examples were enough to see an improvement over the baseline model. SDXL has a stronger baseline, hence more examples were needed for improvement.
\begin{table}
\caption{Similarity metric ablation study (OmniGen). Results of our method using different similarity metrics for image retrieval.
Best results are \textbf{bolded}. 
}
  \label{tab:sim_metrics}
  \adjustbox{max width=\columnwidth}{
  \centering
  \begin{tabular}{@{}ccccccc}
    \toprule
     & \multicolumn{3}{c}{ImageNet} &  
     \multicolumn{3}{c}{CUB} \\
    \cmidrule(lr){2-4} \cmidrule(lr){5-7} 
     & CLIP $\uparrow$ & SigLIP $\uparrow$ & DINO $\uparrow$ & CLIP $\uparrow$ & SigLIP $\uparrow$ & DINO $\uparrow$ \\ 
    \midrule
    GPT Re-rank & \textbf{0.265} $\pm$ \textbf{0.001} & \textbf{0.135} $\pm$ \textbf{0.001} & $0.707 \pm 0.002$ & 
    \textbf{0.255} $\pm$ \textbf{0.004} & \textbf{0.125} $\pm$ \textbf{0.003} & $0.762 \pm 0.004$
     \\
     BM25 Re-rank & $0.264 \pm 0.001$ & $0.134 \pm 0.001$ & $0.707 \pm 0.002$ & 
     $0.253 \pm 0.003$ & $0.123 \pm 0.003$ & \textbf{0.763} $\pm$ \textbf{0.004}
     \\
    SigLIP & $0.259 \pm 0.006$ & $0.133 \pm 0.001$ & $0.704 \pm 0.002$ & 
    $0.243 \pm 0.004$ & $0.116 \pm 0.003$ & $0.761 \pm 0.004$
     \\
    CLIP & $0.264 \pm 0.001$ & $0.134 \pm 0.001$ & \textbf{0.708} $\pm$ \textbf{0.002} & 
    $0.253 \pm 0.003$ & \textbf{0.125} $\pm$ \textbf{0.002} & $0.760 \pm 0.003$ 
     \\ 
    % \midrule
    \bottomrule
  \end{tabular}
  }
\end{table}

Finally, we investigate the effect of different similarity metrics for retrieval. We used CLIP~\cite{radford2021learning}, SigLIP~\cite{zhai2023sigmoid}, and re-ranking with GPT~\cite{hurst2024gpt} and BM25~\cite{robertson2009probabilistic} over image captions generated by GPT for the retrieved candidates. Re-ranking was performed after retrieving $3$ candidates from each of CLIP and SigLIP. Results are reported in \cref{tab:sim_metrics}.
Although re-ranking with GPT produced slightly better results, they were not significant enough to justify the cost of applying this complex strategy vs. a more straightforward CLIP metric. Hence, our other experiments use CLIP. Nevertheless, all the different metrics improved the generation abilities of the base model by providing helpful references.

\subsection{Qualitative comparisons}
\label{sec:qual}
\section{User evaluation with frequent users of mobile ASR: Lab study and online survey }
To evaluate the usability of our approach, we decided to conduct an in-person lab evaluation of the SpeechCompass phone case and the speech-to-text application (described in Section~\ref{subsection:app}), with frequent users of mobile transcription technology. We first conducted a large-scale online pilot study to inform the design of the in-person lab evaluation, which we conducted with eight deaf or hard-of-hearing participants, set up to mimic a realistic conversation scenario. 

\begin{figure*}
  \centering
  \includegraphics[width=0.75\linewidth]{images/second_study.pdf}
  \caption{Participants' preferences for different visualization techniques in the online survey. A) Results indicating how valuable the specific indicator would be for the user. B) Preferences for the specific indicators for speech direction.} 
  \label{fig:user_preferences_online} 
\end{figure*}


\subsection{Large-scale, online survey (n=494)} In this survey, we use screenshots of our interactive UI prototypes to solicit initial user
feedback on the potential for our proposed approach, to guide the design of a more realistic in-person lab study.

The study was conducted using the same Google Surveys deployment and screening methodology as for the foundational study, detailed in Section 3. The participants were shown different UI renderings and were asked to rate them. The large-scale online survey could only show static images of the interfaces, due to limitations of the survey tool. Out of 985 respondents we focus our analysis on the 494 participants who use captioning technology multiple times per week or more frequently. 

As shown in Figure~\ref{fig:user_preferences_online}A, the colored text was found to be valuable by 60\% of participants. Glyph indicators for speech direction, which included arrow and circle+line indicators, were found valuable by 70\%. The Edge indicator and the mini map had a less positive reception. 

To better understand which glyph indicators were favored, we also asked targeted questions about them, as shown in Figure~\ref{fig:user_preferences_online}B. \emph{Circle + line} was preferred by 13.1\% more respondents than the \emph{highlight box} (45.1\% vs 32.0\%), and the \emph{arrow} was preferred by 21.9\% more respondents than the \emph{circle + line} (51.2\% vs 29.3\%).


\subsection{Lab study (n=8)}
\alex{explain and emphasize intention}
We recruited 8 participants from our institution who were frequent users of captioning technology. Five were female, three were male, and all were deaf or hard of hearing. One participant was 25--34 years old, four were 34--44, one was 45--53, and two were 65+ (we are only allowed to collect age ranges at our institution). 


% setup: https://docs.google.com/document/d/1akr5HVMgJb8Kd9KaEZJcdXn2S0IbHhd8JdBPTE0TiA0/edit?usp=sharing
The study took place in a quiet lab over approximately 60 minutes and used the phone-case prototype (Figure~\ref{fig:pcb_design}) with our mobile ASR application (Figure~\ref{fig:phone_interfaces}). First, the participant was introduced to the technology, prototype, and the purpose of the study. Then, the participant was asked to fill out a background survey, which included demographic questions and their current use and experienced challenges with transcription technology. Afterward, the participant was introduced to different visualization scenarios with the SpeechCompass application. The participant used the SpeechCompass transcription while sitting between the two experimenters, as they all sat around a small table with the SpeechCompass phone case in the center. In each of the seven conditions, which ran for 5 minutes, the experimenters sat across from each other and had short conversations about different topics. The participants were instructed to turn off hearing aid devices if they used any, and were asked to use the SoundCompass UI and transcript to follow the conversation. The experimenters' casual conversations included topics like weekend plans, hobbies, and the weather. The seven conditions, which used the ASR, diarization, and localization functionality for different visualization techniques, are shown in Figure~\ref{fig:ui_options} and presented with more UI context in Figure~\ref{fig:phone_interfaces}. The conditions were:
\begin{enumerate}
    \item \textbf{Transcription only}. The transcribed text is shown in white on a black background. 
    
    \item \textbf{Edge indicator}. A circle (``dot'') that moves around the edge of the screen to point to the currently active speaker. The color of the dot changes based on the direction. 
    
    \item \textbf{Arrow indicator}. A glyph using a colored arrow next to a white text block. The glyph points in the direction of the associated speech. 
    
    \item \textbf{Circle + line indicator}. A glyph using a circle with a directional line next to a white text block. The glyph points in the direction of the speech associated with the text. 
    
    \item \textbf{Mini map}. A colored circle with a smaller circle (``dot'') moves around its edge to point to the currently active speaker. The color of the dot changes based on the direction. 
    
    \item \textbf{Colored text}. The text is colored based on the direction that the associated speech was coming from. 
    
    \item \textbf{Everything on}. All indicators are turned on (except the Circle + line, as it couldn't be used simultaneously with the arrow). 
\end{enumerate}

%five isolated visualization techniques, baseline with just text transcription (no speaker information), and with all visualization turned one. Minimap was shown with an arrow, since we envisioned it would be combined with other techniques. 
After participants had completed all conditions, they filled out a form that asked them to rate how desirable each of the five visual indicator styles (\textit{Edge indicator}, \textit{Arrow}, \textit{Circle  + line}, \textit{Colored map}, and \textit{Colored text}) were on a 7-point Likert scale, from \emph{-3: Strongly dislike} to \emph{+3: Strongly like}. Finally, they were asked to rate the overall value of directional feedback to the transcription experience, how strongly they would recommend these features to users of mobile captioning, and whether they had any general free-form feedback about SpeechCompass. 

\begin{figure*}
  \centering
  \includegraphics[width=0.65\linewidth]{images/study_setup.png}
  \caption{Examples of seven visualization scenarios that participants experienced in the in-person study.} 
  \label{fig:ui_options} 
\end{figure*}

%After running the scenarios, participants filled out the second part of the survey, which asked them to rate each scenario and overall impression on a scale from -3:strongly dislike to +3:strongly like. Finally, the participants filled out free form feedback about the study. 

\begin{figure*}
  \centering
  \includegraphics[width=0.65\linewidth]{images/box_plot_in_person_study_results.png}
  \caption{Boxplots of results of the in-person study. A) Participants' preferences for different visualization techniques. B) Overall opinions about augmented mobile ASR application.\alex{love these plots -- maybe to B you could also add the question about multi-people conversations as the leftmost, since it is also on same scale?} } 
  \label{fig:user_preferences} 
\end{figure*}

\subsection{Results}
Mobile transcription apps (e.g., Android Live Transcribe) were the most used communication technology for the participants. Specifically, three used them multiple times per day, one used them daily, three used them multiple times per week, and one used them rarely. 

75\% of participants frequently experienced the scenario where multiple people would get mixed up in the transcript (two multiple times per day, two daily, two multiple times per week). All participants agreed that it was challenging to participate in conversations when speech was combined from multiple people. 
%Similarly to the online survey, we asked participants to select the biggest challenges they experienced in their use of transcription technology (same options as in Figure~\ref{fig: survey-challenges}). where the majority (6/8) selected \textit{"Have to look away from the person I am talking to"}.  
\\

A Kruskal-Wallis (KW) test found a significant effect
on participant preferences for visualization techniques (P=.014).
The post-hoc pair-wise analyses using the Wilcoxon test with Bonferroni correction did, however, not show statistical significance between any pairs.
Of the five visual indicator styles that participants experienced, \emph{Colored text} was the most well-received (mean ($\bar{x})=2.625$), as it was rated positively by all the participants. %, with six strong like (+3), one like (+2), and one slight like (+1). 
The \emph{Arrow} indicator was also well-received ($\bar{x}=1.125$), with six positive, one negative, and one neutral participant.
%(one strong like (+3), three like (+2) and one slight like (+1)) and one dislike (-1) and one neutral (0)). 
Several participants noted that \emph{Arrow} and \emph{Colored text} worked well together: \emph{"Arrows + color seem to be most easier way to indicate the direction." (P2)} and \emph{"The combination of the colored text with the arrow was the most effective for me." (P7)}.

The other indicator styles received more mixed feedback. The feedback for both \emph{Edge indicator} ($\bar{x}=0.25$) and \emph{Circle + line} ($\bar{x}=-0.125$) was split between four negative and four positive participants. 
Some participants were concerned that \emph{Edge indicator} was distracting and not sufficiently discreet: \textit{"I do prefer the tool be as discrete as possible and would perhaps choose to avoid bright colored things moving around since this would be eye-catching and this kind of attention is often undesired" (P3)} and 
\textit{"Indicator moving around the edge was distracting and causing a bit of eye strain" (P2)}.
On the other hand, another participant found this style particularly useful: \textit{"the color dot moving to the speaker direction worked REALLY well" (P1)}. 
For \emph{Circle + line}, some participants struggled with its legibility: \textit{"If the analog direction indicators were larger (and translucent, or set behind)" (P8)} and \textit{"The lines in a circle were a bit slower and not as accurate (buggy)" (P5)}.
The \emph{Mini map} was rated positively by five participants and negatively by three. The most favorable participant stated: \emph{"this is also great for environmental awareness for those with single-sided hearing or no hearing at all." (P3)} and a participant who disliked the \emph{Edge indicator} commented: \emph{"steady map in the corner worked a bit better (P5)"}.

Overall, all participants agreed with the value of directional feedback ($\bar{x}=2.88$, seven Strongly agree:+3 and one Agree:+2) and would recommend these features to other users of captioning technology ($\bar{x}=2.63$, five Strongly agree:+3 and three Agree:+2): \textit{"I really liked that almost immediately I could tell that there was a speaker change, so that as soon as the text started to show up, I could better contextualize that text as attributed to a new speaker." (P1)}, \textit{"I'm very happy to see this tool being developed, it's a great addition to other speech recognition tools!" (P3)}, and \textit{"This prototype is definitely a life changer and I strongly believe that it will improve the quality of access to communication with speakers for many users" (P6)}.

\subsection{Discussion}
Consistent with the large-scale survey, the value of the diarization and localization features was immediate to all users. The participants were asked if directional guidance would be valuable in their mobile transcription experience. All eight users agreed. Also, all eight users would recommend this feature to mobile captioning users. 

While the large-scale survey helped inform our testing and exclude conditions (e.g., \emph{Highlight box}), the lab study allowed us to more rigorously evaluate the techniques in a realistic scenario. This difference became significant for the \emph{Edge indicator} and \emph{Mini map}, where issues, such as discreetness and distracting aspects, became evident during live usage. 

The results suggest that the combination of \textit{Colored text} and \textit{Arrow} would meet the preferences of most users, thanks to the balance of directional encoding and clarity. The arrow has redundant benefits too, since colored text might not always be reliably visible depending on lighting and screen conditions (e.g., strong sunlight, or dim display) and might also not be usable for colorblind users. The mixed feedback for other techniques indicates that the interface may also benefit from mechanisms that would allow users to customize the visualization style. Such customization could also apply to rendering properties, such as color, transparency, and line thickness, as some participants found \textit{Circle + line} particularly difficult to interpret. In both the large-scale survey and the in-person lab study, the \textit{Arrow} was preferred over \textit{Circle + line}. Through more customization options and extended usage in their daily lives, participants will be able to provide more nuanced feedback about these techniques. 


% Edge indicator and mini map had a less positive reception. However, they were rated more positively than those in the in-person study. Since participants didn't experience the working prototype, the discreet and distracting aspects that were observed in the in-person study were not captured. 

% In both online and in-person study, the arrow directional glyph was preferred to circle+line.



% This dichotomy demonstrates that users should be given a way to customize their experience. For example, the edge indicator received strong likes and dislikes from different participants. 


% This indicates that the interface designers should make the directional glyphs as easy to read as possible.


% The results of the online survey followed what was observed in the in-person study. Edge indicator and mini map had a less positive reception. However, they were rated more positively than those in the in-person study. Since participants didn't experience the working prototype, the discreet and distracting aspects that were observed in the in-person study were not captured. In both online and in-person study, the arrow directional glyph was preferred to circle+line.

% As indicated in the survey, the value of the diarization and localization features was immediate to all users. The participants were asked if directional feedback is valuable in their mobile transcription experience. All eight users agreed. Also, all eight users would recommend this feature to mobile captioning users. 


% \textit{"I really liked that almost immediately I could tell that there was a speaker change, so that as soon as the text started to show up, I could better contextualize that text as attributed to a new speaker." (P1)}

% P3
% Arrows + color seem to be most easier way to indicate the direction.
% \emph{"Arrows + color seem to be most easier way to indicate the direction." (P2)}
% P4
% \textit{"I'm very happy to see this tool being developed, it's a great addition to other speech recognition tools!" (P3)
% }
% \textit{"it was great to see so many options being offered" (P3)
% }
% P6 
% \textit{"This prototype is definitely a life changer and I strongly beleve that it will improve the quality of access to communication with speakers for many users" (P6)}

% P8
% The combination of the colored text with the arrow was the most effective for me.

% \emph{"The combination of the colored text with the arrow was the most effective for me." (P7)}
\begin{figure*}[htpb]
    \centering
    \setlength{\tabcolsep}{0.7pt}

    \begin{tabular}{c c | c c c c}
        Prompt & Reference & OmniGen & SDXL & \makecell{ImageRAG \\ (OmniGen)} & \makecell{ImageRAG \\ (SDXL+IP)} \\
        \raisebox{0.5in}{Chow} &
        \includegraphics[clip,width=25mm,height=25mm]{Assets/comp/chow_ref.jpg} &
        \includegraphics[clip,width=25mm]{Assets/comp/chow_o.jpg} &
        \includegraphics[clip,width=25mm]{Assets/comp/chow_sd.jpg} &
        \includegraphics[clip,width=25mm]{Assets/comp/chow_imagerag_o.jpg} &
        \includegraphics[clip,width=25mm]{Assets/comp/chow_imagerag_sd.jpg} \\
        \raisebox{0.5in}{Boston bull} &
        \includegraphics[clip,width=25mm,height=25mm]{Assets/comp/boston_bull_ref.jpg} &
        \includegraphics[clip,width=25mm]{Assets/comp/boston_bull_o.jpg} &
        \includegraphics[clip,width=25mm]{Assets/comp/boston_bull_sd.jpg} &
        \includegraphics[clip,width=25mm]{Assets/comp/boston_bull_imagerag_o.jpg} &
        \includegraphics[clip,width=25mm]{Assets/comp/boston_bull_imagerag_sd.jpg} \\
        \raisebox{0.5in}{Cab} &
        \includegraphics[clip,width=25mm,height=25mm]{Assets/comp/cab_ref.jpg} &
        \includegraphics[clip,width=25mm]{Assets/comp/cab_o.jpg} &
        \includegraphics[clip,width=25mm]{Assets/comp/cab_sd.jpg} &
        \includegraphics[clip,width=25mm]{Assets/comp/cab_imagerag_o.jpg} &
        \includegraphics[clip,width=25mm]{Assets/comp/cab_imagerag_sd.jpg} \\
        \raisebox{0.5in}{Academic gown} &
        \includegraphics[clip,width=25mm,height=25mm]{Assets/comp/academic_gown_ref.jpg} &
        \includegraphics[clip,width=25mm]{Assets/comp/academic_gown_o.jpg} &
        \includegraphics[clip,width=25mm]{Assets/comp/academic_gown_sd.jpg} &
        \includegraphics[clip,width=25mm]{Assets/comp/academic_gown_imagerag_o.jpg} &
        \includegraphics[clip,width=25mm]{Assets/comp/academic_gown_imagerag_sd.jpg} \\
        \raisebox{0.5in}{Unicycle} &
        \includegraphics[clip,width=25mm,height=25mm]{Assets/comp/unicycle_ref.jpg} &
        \includegraphics[clip,width=25mm]{Assets/comp/unicycle_o.jpg} &
        \includegraphics[clip,width=25mm]{Assets/comp/unicycle_sd.jpg} &
        \includegraphics[clip,width=25mm]{Assets/comp/unicycle_imagerag_o.jpg} &
        \includegraphics[clip,width=25mm]{Assets/comp/unicycle_imagerag_sd.jpg} \\
        \raisebox{0.5in}{Geococcyx} &
        \includegraphics[clip,width=25mm,height=25mm]{Assets/comp/geo_ref.jpg} &
        \includegraphics[clip,width=25mm]{Assets/comp/geo_o.jpg} &
        \includegraphics[clip,width=25mm]{Assets/comp/geo_sd.jpg} &
        \includegraphics[clip,width=25mm]{Assets/comp/geo_imagerag_o.jpg} &
        \includegraphics[clip,width=25mm]{Assets/comp/geo_imagerag_sd.jpg} \\
        \raisebox{0.5in}{Cyanocitta cristata} &
        \includegraphics[clip,width=25mm,height=25mm]{Assets/comp/iNat_20_ref.jpg} &
        \includegraphics[clip,width=25mm]{Assets/comp/iNat_20_o.jpg} &
        \includegraphics[clip,width=25mm]{Assets/comp/iNat_20_sd.jpg} &
        \includegraphics[clip,width=25mm]{Assets/comp/iNat_20_imagerag_o.jpg} &
        \includegraphics[clip,width=25mm]{Assets/comp/iNat_20_imagerag_sd.jpg} \\
        
    \end{tabular}
    
    \caption{\textbf{Qualitative comparisons: rare concept generation.} Examples from ImageNet~\cite{deng2009imagenet}, CUB~\cite{wah2011caltech} and iNaturalist~\cite{van2018inaturalist}. The left-most image column is the retrieved reference using \emph{ImageRAG} for each prompt. OmniGen and SDXL both struggle with the uncommon concepts, sometimes generating similar concepts such as a bull or a cow instead of the dog breed ``Boston bull'', while in other times, they generate completely unrelated images, as in the case of ``Chow'', or ``Geococcyx''. When using \emph{ImageRAG} both models generate the correct concept.
    }
    \label{fig:qual_comp}
\end{figure*}

\cref{fig:qual_comp} shows qualitative examples from the ImageNet~\cite{deng2009imagenet}, CUB~\cite{wah2011caltech} and iNaturalist~\cite{van2018inaturalist} datasets, comparing the results of OmniGen and SDXL with and without our method.

To further assess the quality of our results, we conduct a user study with 46 participants and a total of 767 comparisons.
We perform two types of studies --- one that evaluates SDXL and OmniGen with and without our method, and another that compares our results with other retrieval-based generation models. Specifically, we compare to models explicitly trained for the task of image generation using retrieved images: RDM~\cite{blattmann2022retrieval}, knn-diffusion~\cite{sheyninknn}, and ReImagen~\cite{chenre}. Since these are largely proprietary models with no API, we compare to images and prompts published in their papers.

In both cases, we ask participants to compare two images at a time: one created with our approach, and one using a baseline. We ask users to choose the one they prefer in terms of adherence to the text prompt,
visual quality, and overall preference. Since some prompts contain uncommon concepts, we supply users with a real image of the least familiar concept in each prompt (not taken from our dataset).
When running \emph{ImageRAG} for the user study, we disable the decision step where GPT is asked if the initial image matches the prompt. This is done to avoid showing a user the same image twice in cases where GPT deems the initial image to be a good fit.
As demonstrated in \cref{fig:user_study} participants favored \emph{ImageRAG} over all other methods in all three criteria of text alignment, visual quality, and overall preference.
\cref{app:user_study} supplies more information about questions asked in the study, visual comparison examples for each retrieval-based generation model (\cref{fig:retrieval_comp}), and more comparisons to SDXL (\cref{fig:rare_sd}) and OmniGen (\cref{fig:rare_o}) with and without \emph{ImageRAG}. 
\cref{fig:creative} shows additional visual results of our method with more complex and creative prompts.
