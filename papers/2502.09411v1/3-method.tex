\vspace{-5pt}
\section{Method}
\label{sec:method}
\begin{figure*}[t]
\begin{center}
\includegraphics[width=.85\linewidth]{fig_overview_v3.pdf}
\end{center}
\caption{
FastAtlas Overview: In each frame, we compute charts spanning fully or partially visible triangles (a), determine texture space bounding boxes for the visible portions of the view-space projections of each chart, and tightly pack these boxes into atlases (b, here $2K \times 2K$). We simultaneously bijectively parameterize and shade the charts into their atlas boxes, obtaining high quality texture space shading (c), and use this shading to render the shaded frames (d).}
\label{fig:overview}
\label{fig:alg_overview}
\end{figure*}

\section{Overview}
\label{sec:overview}
Our work has two core contributions: a real-time, GPU-based algorithm for tight packing of general parameterized charts into compact atlases; and a real-time TSS method that
utilizes this packing.  

\paragraph*{FastAtlas Packing.}
FastAtlas runs entirely on the GPU as a series of compute shaders. It takes the bounding boxes of parameterized charts as input, and packs them into an atlas (Fig~\ref{fig:overview}b, Sec.~\ref{sec:pack}). As such, the only input it requires are the dimensions of the bounding boxes.
Its outputs are deterministic; identical input charts are packed into identical atlases. This is critical for TSS and similar applications, as it ensures that consecutive frames taken from the same camera view have the same shading. Even minute shading differences across such frames can cause sampling jitter, leading to undesirable flicker \cite{baker2012rock}. 
While prior methods such as \cite{mueller2018shading,hladky2019tessellated,hladky2021snakebinning,Neff2022MSA} cap the dimensions of the charts that can be packed as-is for a given atlas size, and scale down all charts that exceed these dimensions, we scale all charts by the same factor, and do so only when strictly necessary to achieve packing success (Figs~\ref{fig:atlas},~\ref{fig:sas_issues}). 

\paragraph*{TSS using FastAtlas.}
Our end-to-end TSS atlas generation method combines the packing method above with a novel approach for computing seamless per-frame charts. 
We define our charts as the connected components of the visible surfaces in each frame (Fig.~\ref{fig:overview}a), and efficiently compute them using a parallel union-find algorithm (Sec.~\ref{sec:visible}). Since the boundaries of these charts coincide with the contours of the rendered surface, they are {\em invisible} to the viewer. This approach 
eliminates the artifacts caused by shading discontinuities along visible seams (Fig.~\ref{fig:seams}). 

\begin{parWithWrapFigure}
\begin{wrapfigure}{l}{.27\columnwidth}%
\includegraphics[width=\linewidth]{fig_inset_view_plane.pdf}%
\end{wrapfigure}
We bijectively parametrize the {\em visible portions} of our charts by projecting them to view space (inset). This maps a constant number of texels to each pixel in the final rendered output, evenly distributing residual undersampling error across all image pixels. While conceptually straightforward, efficiently parameterizing charts containing partially visible triangles using viewspace projection is non-trivial, as the visible portions may no longer be triangular (e.g. green triangle in the inset); applying naive projection to triangles with vertices behind the camera may produce ill-posed results. Clipping triangles before projection is both computationally expensive and significantly complicates downstream operations. We avoid explicit clipping by observing that all that is required for atlas packing is the dimensions of, potentially conservative, bounding boxes of these projected visible portions. We compute such bounding boxes without explicit chart clipping by adapting a conservative screen coverage estimator \shortcite{Blinn:CalculatingScreenCoverage} (Sec.~\ref{sec:box}). We then pack the computed boxes using FastAtlas. 
\end{parWithWrapFigure}

Finally, we shade the visible portion of each chart into its corresponding atlas bounding box (Fig~\ref{fig:overview}c). 
The resulting texture is then used during rasterization as a standard texture map (Fig. ~\ref{fig:overview}d). 
Our framework is compatible with all existing approaches for texture space shading, including forward shading, raytraced illumination, or deferred shading in texture space \cite{baker:2016}. In the examples shown, we use the standard forward shading based rendering pipeline included in the G3D Innovation Engine \cite{G3D17}, a commercial grade renderer.


Our goal is to increase the robustness of T2I models, particularly with rare or unseen concepts, which they struggle to generate. To do so, we investigate a retrieval-augmented generation approach, through which we dynamically select images that can provide the model with missing visual cues. Importantly, we focus on models that were not trained for RAG, and show that existing image conditioning tools can be leveraged to support RAG post-hoc.
As depicted in \cref{fig:overview}, given a text prompt and a T2I generative model, we start by generating an image with the given prompt. Then, we query a VLM with the image, and ask it to decide if the image matches the prompt. If it does not, we aim to retrieve images representing the concepts that are missing from the image, and provide them as additional context to the model to guide it toward better alignment with the prompt.
In the following sections, we describe our method by answering key questions:
(1) How do we know which images to retrieve? 
(2) How can we retrieve the required images? 
and (3) How can we use the retrieved images for unknown concept generation?
By answering these questions, we achieve our goal of generating new concepts that the model struggles to generate on its own.

\vspace{-3pt}
\subsection{Which images to retrieve?}
The amount of images we can pass to a model is limited, hence we need to decide which images to pass as references to guide the generation of a base model. As T2I models are already capable of generating many concepts successfully, an efficient strategy would be passing only concepts they struggle to generate as references, and not all the concepts in a prompt.
To find the challenging concepts,
we utilize a VLM and apply a step-by-step method, as depicted in the bottom part of \cref{fig:overview}. First, we generate an initial image with a T2I model. Then, we provide the VLM with the initial prompt and image, and ask it if they match. If not, we ask the VLM to identify missing concepts and
focus on content and style, since these are easy to convey through visual cues.
As demonstrated in \cref{tab:ablations}, empirical experiments show that image retrieval from detailed image captions yields better results than retrieval from brief, generic concept descriptions.
Therefore, after identifying the missing concepts, we ask the VLM to suggest detailed image captions for images that describe each of the concepts. 

\vspace{-4pt}
\subsubsection{Error Handling}
\label{subsec:err_hand}

The VLM may sometimes fail to identify the missing concepts in an image, and will respond that it is ``unable to respond''. In these rare cases, we allow up to 3 query repetitions, while increasing the query temperature in each repetition. Increasing the temperature allows for more diverse responses by encouraging the model to sample less probable words.
In most cases, using our suggested step-by-step method yields better results than retrieving images directly from the given prompt (see 
\cref{subsec:ablations}).
However, if the VLM still fails to identify the missing concepts after multiple attempts, we fall back to retrieving images directly from the prompt, as it usually means the VLM does not know what is the meaning of the prompt.

The used prompts can be found in \cref{app:prompts}.
Next, we turn to retrieve images based on the acquired image captions.

\vspace{-3pt}
\subsection{How to retrieve the required images?}

Given $n$ image captions, our goal is to retrieve the images that are most similar to these captions from a dataset. 
To retrieve images matching a given image caption, we compare the caption to all the images in the dataset using a text-image similarity metric and retrieve the top $k$ most similar images.
Text-to-image retrieval is an active research field~\cite{radford2021learning, zhai2023sigmoid, ray2024cola, vendrowinquire}, where no single method is perfect.
Retrieval is especially hard when the dataset does not contain an exact match to the query \cite{biswas2024efficient} or when the task is fine-grained retrieval, that depends on subtle details~\cite{wei2022fine}.
Hence, a common retrieval workflow is to first retrieve image candidates using pre-computed embeddings, and then re-rank the retrieved candidates using a different, often more expensive but accurate, method \cite{vendrowinquire}.
Following this workflow, we experimented with cosine similarity over different embeddings, and with multiple re-ranking methods of reference candidates.
Although re-ranking sometimes yields better results compared to simply using cosine similarity between CLIP~\cite{radford2021learning} embeddings, the difference was not significant in most of our experiments. Therefore, for simplicity, we use cosine similarity between CLIP embeddings as our similarity metric (see \cref{tab:sim_metrics}, \cref{subsec:ablations} for more details about our experiments with different similarity metrics).

\vspace{-3pt}
\subsection{How to use the retrieved images?}
Putting it all together, after retrieving relevant images, all that is left to do is to use them as context so they are beneficial for the model.
We experimented with two types of models; models that are trained to receive images as input in addition to text and have ICL capabilities (e.g., OmniGen~\cite{xiao2024omnigen}), and T2I models augmented with an image encoder in post-training (e.g., SDXL~\cite{podellsdxl} with IP-adapter~\cite{ye2023ip}).
As the first model type has ICL capabilities, we can supply the retrieved images as examples that it can learn from, by adjusting the original prompt.
Although the second model type lacks true ICL capabilities, it offers image-based control functionalities, which we can leverage for applying RAG over it with our method.
Hence, for both model types, we augment the input prompt to contain a reference of the retrieved images as examples.
Formally, given a prompt $p$, $n$ concepts, and $k$ compatible images for each concept, we use the following template to create a new prompt:
``According to these examples of 
$\mathord{<}c_1\mathord{>:<}img_{1,1}\mathord{>}, ... , \mathord{<}img_{1,k}\mathord{>}, ... , \mathord{<}c_n\mathord{>:<}img_{n,1}\mathord{>}, ... , $
$\mathord{<}img_{n,k}\mathord{>}$,
generate $\mathord{<}p\mathord{>}$'', 
where $c_i$ for $i\in{[1,n]}$ is a compatible image caption of the image $\mathord{<}img_{i,j}\mathord{>},  j\in{[1,k]}$. 

This prompt allows models to learn missing concepts from the images, guiding them to generate the required result. 

\textbf{Personalized Generation}: 
For models that support multiple input images, we can apply our method for personalized generation as well, to generate rare concept combinations with personal concepts. In this case, we use one image for personal content, and 1+ other reference images for missing concepts. For example, given an image of a specific cat, we can generate diverse images of it, ranging from a mug featuring the cat to a lego of it or atypical situations like the cat writing code or teaching a classroom of dogs (\cref{fig:personalization}).
\vspace{-2pt}
\begin{figure}[htp]
  \centering
   \includegraphics[width=\linewidth]{Assets/personalization.pdf}
   \caption{\textbf{Personalized generation example.}
   \emph{ImageRAG} can work in parallel with personalization methods and enhance their capabilities. For example, although OmniGen can generate images of a subject based on an image, it struggles to generate some concepts. Using references retrieved by our method, it can generate the required result.
}
   \label{fig:personalization}\vspace{-10pt}
\end{figure}