\documentclass[accepted]{uai2025} % for initial submission
%\documentclass[accepted]{uai2025} % after acceptance, for a revised version; 
% also before submission to see how the non-anonymous paper would look like 
                        
%% There is a class option to choose the math font
% \documentclass[mathfont=ptmx]{uai2025} % ptmx math instead of Computer
                                         % Modern (has noticeable issues)
% \documentclass[mathfont=newtx]{uai2025} % newtx fonts (improves upon
                                          % ptmx; less tested, no support)
% NOTE: Only keep *one* line above as appropriate, as it will be replaced
%       automatically for papers to be published. Do not make any other
%       change above this note for an accepted version.

%% Choose your variant of English; be consistent
\usepackage[american]{babel}
% \usepackage[british]{babel}

%% Some suggested packages, as needed:
\usepackage{natbib} % has a nice set of citation styles and commands
    %\bibliographystyle{plainnat}
% \renewcommand{\bibsection}{\subsubsection*{References}}
\usepackage{mathtools} % amsmath with fixes and additions
% \usepackage{siunitx} % for proper typesetting of numbers and units
\usepackage{booktabs} % commands to create good-looking tables
\usepackage{tikz} % nice language for creating drawings and diagrams
\usepackage{xspace}

\usepackage{titlesec}
\usepackage{titletoc}

\usepackage[utf8]{inputenc} % allow utf-8 input
\usepackage[T1]{fontenc}    % use 8-bit T1 fonts
\usepackage{hyperref}       % hyperlinks
\usepackage{url}            % simple URL typesetting
\usepackage{booktabs}       % professional-quality tables
\usepackage{amsfonts}       % blackboard math symbols
\usepackage{nicefrac}       % compact symbols for 1/2, etc.
\usepackage{microtype}      % microtypography
\usepackage{xcolor}         % colors
\usepackage{amsmath}
\usepackage{graphicx}
\usepackage{cleveref}
\usepackage{algorithm}
%\usepackage{algorithmic}
\usepackage{amsthm}
\newtheorem{theorem}{Theorem}[section]  
\newtheorem{assumption}{Assumption}
\newtheorem{lemma}{Lemma}
\newtheorem{corollary}{Corollary}
\bibliographystyle{plainnat}
\usepackage{thmtools}
\usepackage{thm-restate}
\usepackage[T1]{fontenc}

\usepackage{compact}
\newtheorem{definition}{Definition}[section] % 
\newtheorem{proposition}{Proposition}[section] 
\usepackage{amsthm}



\usepackage{amsmath,amssymb}
\usepackage{algorithm}
\usepackage{algpseudocode}  % or {algorithmic} if you prefer


\newcommand{\lk}[1]{\textcolor{red}{[\textbf{Lingkai}: #1]}}
\newcommand{\yuqi}[1]{\textcolor{blue}{[\textbf{Yuqi }: #1]}}
\newcommand{\haichuan}[1]{\textcolor{green}{[\textbf{Haichuan }: #1]}}
\newcommand{\ak}[1]{\textcolor{teal}{[\textbf{Adam }: #1]}}


\newcommand{\argmax}{\mathop{\mathrm{arg\,max}}}
\newcommand{\argmin}{\mathop{\mathrm{arg\,min}}}
\newcommand{\ours}{\textsc{DiffOracle}\xspace}
%% Provided macros
% \smaller: Because the class footnote size is essentially LaTeX's \small,
%           redefining \footnotesize, we provide the original \footnotesize
%           using this macro.
%           (Use only sparingly, e.g., in drawings, as it is quite small.)

%% Self-defined macros
\newcommand{\swap}[3][-]{#3#1#2} % just an example

\title{Robust Optimization with Diffusion Models for Green Security}

% The standard author block has changed for UAI 2025 to provide
% more space for long author lists and allow for complex affiliations
%
% All author information is authomatically removed by the class for the
% anonymous submission version of your paper, so you can already add your
% information below.
%
% Add authors
\author[1]{\href{lingkaikong@g.harvard.edu}{Lingkai Kong}{}}
\author[1]{Haichuan Wang}
\author[1]{Yuqi Pan}
\author[1]{Cheol Woo Kim}
\author[1]{Mingxiao Song}
\author[1]{Alayna Nguyen}
\author[1]{\\Tonghan Wang}
\author[2]{Haifeng Xu}
\author[1]{Milind Tambe}
% Add affiliations after the authors
\affil[1]{%
    Harvard University
}
\affil[2]{%
    University of Chicago

}

  
  \begin{document}
\maketitle

\begin{abstract}
% \textit{We could cast this more generally as green security games, which includes: protecting against illegal logging, illegal fishing, poaching, illegal environmental pollution/dumping and other topics, depending on whether this goes in a very applied direction.}


% We aim to use diffusion models to predict poacher behavior, with the resulting predictions informing patrol planning. Given the potential noisiness of the poacher dataset, we propose addressing this challenge through a distributionally robust optimization approach. Specifically, we will employ a game-theoretic framework to solve the resulting minimax optimization problem. Our key innovation is to develop a guided diffusion technique capable of directly sampling from the worst-case distribution.

% In green security, defenders patrol by forecasting the adversarial behavior of poachers, illegal loggers, and illegal fishers. However, this behavior is often highly uncertain and multimodal. To address this challenge, we propose leveraging a conditional diffusion model for adversarial behavior prediction and focus on robust patrol optimization. However, the learned diffusion model can be imperfect due to noise and limited size in data. we formulate the problem as a two-player game between the defender and nature, where nature player selects the adversarial behavior distribution within a constrained space. This formulation enhances the robustness of the patrol strategy to imperfections in the learned diffusion model. To accurately estimate the expected utility function, we combine twisted Sequential Monte Carlo sampling and diffusion model, ensuring asymptotic exactness. We evaluate our approach on both synthetic and real-world poaching datasets, demonstrating its effectiveness.

% In green security, defenders must forecast the behavior of adversaries such as poachers, illegal loggers, and illegal fishers to plan effective patrols. However, these behaviors are often highly uncertain and complicated. To address this challenge, we propose using a conditional diffusion model for adversarial behavior prediction. Because the learned diffusion model can be imperfect due to noise and limited data, we formulate the patrol optimization problem as a two-player game between the defender and nature, where nature selects an adversarial behavior distribution from a constrained neighborhood around the model’s predicted distribution. This game-theoretic framework enhances the robustness of patrol strategies against model imperfections. We solve the resulting game via a double oracle algorithm. To accurately estimate expected utilities, we integrate twisted Sequential Monte Carlo sampling with the diffusion model, ensuring asymptotic exactness. Theoretically, our algorithm can converge to an epsilon-equilibrium with arbitrarily high probability using finite iterations and finite samples from the diffusion model. We evaluate our method on both synthetic and real-world poaching datasets, demonstrating its effectiveness.


% In green security, defenders must predict the behavior of adversaries such as poachers, illegal loggers, and illegal fishers to plan effective patrols. However, these behaviors are often highly uncertain and complex. Previous approaches typically rely on Gaussian processes or linear models, which have limited expressiveness.
% % To address this challenge, we propose using a conditional diffusion model for adversarial behavior prediction. Since the learned diffusion model may be imperfect due to noise and limited data, we formulate the patrol optimization problem as a two-player game between the defender and nature. In this game, nature selects an adversarial behavior distribution from a constrained neighborhood around the model’s predicted distribution. 
% The use of a diffusion model introduces new challenges in applying game theory to solve the following robust optimization problem, such as a constrained mixed strategy space and the need to sample from an unnormalized distribution to estimate the utility. To tackle these issues, we introduce mixed strategy of mixed strategies and leverage a twisted sequential Monte Carlo sampler for efficient sampling from the diffusion model. Theoretically, our algorithm is guaranteed to converge to an $\epsilon$-equilibrium with high probability using a finite number of iterations and finite number of samples from the diffusion model. We evaluate our method on both synthetic and real-world poaching datasets, demonstrating its effectiveness.

In  green security, defenders must forecast adversarial behavior—such as poaching, illegal logging, and illegal fishing—to plan effective patrols. These behaviors are often highly uncertain and complex. Prior work has leveraged game theory to design robust patrol strategies to handle uncertainty, but existing adversarial behavior models primarily rely on Gaussian processes or linear models, which lack the expressiveness needed to capture intricate behavioral patterns.  To address this limitation, we propose a conditional diffusion model for adversary behavior modeling, leveraging its strong distribution-fitting capabilities. To the best of our knowledge, this is the first application of diffusion models in the green security domain.  
Integrating diffusion models into game-theoretic optimization, however, presents new challenges, including a constrained mixed strategy space and the need to sample from an unnormalized distribution to estimate utilities. To tackle these challenges, we introduce a mixed strategy of mixed strategies and employ a twisted Sequential Monte Carlo (SMC) sampler for accurate sampling.  Theoretically, our algorithm is guaranteed to converge to an \(\epsilon\)-equilibrium with high probability using a finite number of iterations and samples. Empirically, we evaluate our approach on both synthetic and real-world poaching datasets, demonstrating its effectiveness.  



\end{abstract}


\section{Introduction}


\begin{figure}[t]
\centering
\includegraphics[width=0.6\columnwidth]{figures/evaluation_desiderata_V5.pdf}
\vspace{-0.5cm}
\caption{\systemName is a platform for conducting realistic evaluations of code LLMs, collecting human preferences of coding models with real users, real tasks, and in realistic environments, aimed at addressing the limitations of existing evaluations.
}
\label{fig:motivation}
\end{figure}

\begin{figure*}[t]
\centering
\includegraphics[width=\textwidth]{figures/system_design_v2.png}
\caption{We introduce \systemName, a VSCode extension to collect human preferences of code directly in a developer's IDE. \systemName enables developers to use code completions from various models. The system comprises a) the interface in the user's IDE which presents paired completions to users (left), b) a sampling strategy that picks model pairs to reduce latency (right, top), and c) a prompting scheme that allows diverse LLMs to perform code completions with high fidelity.
Users can select between the top completion (green box) using \texttt{tab} or the bottom completion (blue box) using \texttt{shift+tab}.}
\label{fig:overview}
\end{figure*}

As model capabilities improve, large language models (LLMs) are increasingly integrated into user environments and workflows.
For example, software developers code with AI in integrated developer environments (IDEs)~\citep{peng2023impact}, doctors rely on notes generated through ambient listening~\citep{oberst2024science}, and lawyers consider case evidence identified by electronic discovery systems~\citep{yang2024beyond}.
Increasing deployment of models in productivity tools demands evaluation that more closely reflects real-world circumstances~\citep{hutchinson2022evaluation, saxon2024benchmarks, kapoor2024ai}.
While newer benchmarks and live platforms incorporate human feedback to capture real-world usage, they almost exclusively focus on evaluating LLMs in chat conversations~\citep{zheng2023judging,dubois2023alpacafarm,chiang2024chatbot, kirk2024the}.
Model evaluation must move beyond chat-based interactions and into specialized user environments.



 

In this work, we focus on evaluating LLM-based coding assistants. 
Despite the popularity of these tools---millions of developers use Github Copilot~\citep{Copilot}---existing
evaluations of the coding capabilities of new models exhibit multiple limitations (Figure~\ref{fig:motivation}, bottom).
Traditional ML benchmarks evaluate LLM capabilities by measuring how well a model can complete static, interview-style coding tasks~\citep{chen2021evaluating,austin2021program,jain2024livecodebench, white2024livebench} and lack \emph{real users}. 
User studies recruit real users to evaluate the effectiveness of LLMs as coding assistants, but are often limited to simple programming tasks as opposed to \emph{real tasks}~\citep{vaithilingam2022expectation,ross2023programmer, mozannar2024realhumaneval}.
Recent efforts to collect human feedback such as Chatbot Arena~\citep{chiang2024chatbot} are still removed from a \emph{realistic environment}, resulting in users and data that deviate from typical software development processes.
We introduce \systemName to address these limitations (Figure~\ref{fig:motivation}, top), and we describe our three main contributions below.


\textbf{We deploy \systemName in-the-wild to collect human preferences on code.} 
\systemName is a Visual Studio Code extension, collecting preferences directly in a developer's IDE within their actual workflow (Figure~\ref{fig:overview}).
\systemName provides developers with code completions, akin to the type of support provided by Github Copilot~\citep{Copilot}. 
Over the past 3 months, \systemName has served over~\completions suggestions from 10 state-of-the-art LLMs, 
gathering \sampleCount~votes from \userCount~users.
To collect user preferences,
\systemName presents a novel interface that shows users paired code completions from two different LLMs, which are determined based on a sampling strategy that aims to 
mitigate latency while preserving coverage across model comparisons.
Additionally, we devise a prompting scheme that allows a diverse set of models to perform code completions with high fidelity.
See Section~\ref{sec:system} and Section~\ref{sec:deployment} for details about system design and deployment respectively.



\textbf{We construct a leaderboard of user preferences and find notable differences from existing static benchmarks and human preference leaderboards.}
In general, we observe that smaller models seem to overperform in static benchmarks compared to our leaderboard, while performance among larger models is mixed (Section~\ref{sec:leaderboard_calculation}).
We attribute these differences to the fact that \systemName is exposed to users and tasks that differ drastically from code evaluations in the past. 
Our data spans 103 programming languages and 24 natural languages as well as a variety of real-world applications and code structures, while static benchmarks tend to focus on a specific programming and natural language and task (e.g. coding competition problems).
Additionally, while all of \systemName interactions contain code contexts and the majority involve infilling tasks, a much smaller fraction of Chatbot Arena's coding tasks contain code context, with infilling tasks appearing even more rarely. 
We analyze our data in depth in Section~\ref{subsec:comparison}.



\textbf{We derive new insights into user preferences of code by analyzing \systemName's diverse and distinct data distribution.}
We compare user preferences across different stratifications of input data (e.g., common versus rare languages) and observe which affect observed preferences most (Section~\ref{sec:analysis}).
For example, while user preferences stay relatively consistent across various programming languages, they differ drastically between different task categories (e.g. frontend/backend versus algorithm design).
We also observe variations in user preference due to different features related to code structure 
(e.g., context length and completion patterns).
We open-source \systemName and release a curated subset of code contexts.
Altogether, our results highlight the necessity of model evaluation in realistic and domain-specific settings.





\section{Related Works}
\label{sec:related_works}


\noindent\textbf{Diffusion-based Video Generation. }
The advancement of diffusion models \cite{rombach2022high, ramesh2022hierarchical, zheng2022entropy} has led to significant progress in video generation. Due to the scarcity of high-quality video-text datasets \cite{Blattmann2023, Blattmann2023a}, researchers have adapted existing text-to-image (T2I) models to facilitate text-to-video (T2V) generation. Notable examples include AnimateDiff \cite{Guo2023}, Align your Latents \cite{Blattmann2023a}, PYoCo \cite{ge2023preserve}, and Emu Video \cite{girdhar2023emu}. Further advancements, such as LVDM \cite{he2022latent}, VideoCrafter \cite{chen2023videocrafter1, chen2024videocrafter2}, ModelScope \cite{wang2023modelscope}, LAVIE \cite{wang2023lavie}, and VideoFactory \cite{wang2024videofactory}, have refined these approaches by fine-tuning both spatial and temporal blocks, leveraging T2I models for initialization to improve video quality.
Recently, Sora \cite{brooks2024video} and CogVideoX \cite{yang2024cogvideox} enhance video generation by introducing Transformer-based diffusion backbones \cite{Peebles2023, Ma2024, Yu2024} and utilizing 3D-VAE, unlocking the potential for realistic world simulators. Additionally, SVD \cite{Blattmann2023}, SEINE \cite{chen2023seine}, PixelDance \cite{zeng2024make} and PIA \cite{zhang2024pia} have made significant strides in image-to-video generation, achieving notable improvements in quality and flexibility.
Further, I2VGen-XL \cite{zhang2023i2vgen}, DynamicCrafter \cite{Xing2023}, and Moonshot \cite{zhang2024moonshot} incorporate additional cross-attention layers to strengthen conditional signals during generation.



\noindent\textbf{Controllable Generation.}
Controllable generation has become a central focus in both image \citep{Zhang2023,jiang2024survey, Mou2024, Zheng2023, peng2024controlnext, ye2023ip, wu2024spherediffusion, song2024moma, wu2024ifadapter} and video \citep{gong2024atomovideo, zhang2024moonshot, guo2025sparsectrl, jiang2024videobooth} generation, enabling users to direct the output through various types of control. A wide range of controllable inputs has been explored, including text descriptions, pose \citep{ma2024follow,wang2023disco,hu2024animate,xu2024magicanimate}, audio \citep{tang2023anytoany,tian2024emo,he2024co}, identity representations \citep{chefer2024still,wang2024customvideo,wu2024customcrafter}, trajectory \citep{yin2023dragnuwa,chen2024motion,li2024generative,wu2024motionbooth, namekata2024sg}.


\noindent\textbf{Text-based Camera Control.}
Text-based camera control methods use natural language descriptions to guide camera motion in video generation. AnimateDiff \cite{Guo2023} and SVD \cite{Blattmann2023} fine-tune LoRAs \cite{hu2021lora} for specific camera movements based on text input. 
Image conductor\cite{li2024image} proposed to separate different camera and object motions through camera LoRA weight and object LoRA weight to achieve more precise motion control.
In contrast, MotionMaster \cite{hu2024motionmaster} and Peekaboo \cite{jain2024peekaboo} offer training-free approaches for generating coarse-grained camera motions, though with limited precision. VideoComposer \cite{wang2024videocomposer} adjusts pixel-level motion vectors to provide finer control, but challenges remain in achieving precise camera control.

\noindent\textbf{Trajectory-based Camera Control.}
MotionCtrl \cite{Wang2024Motionctrl}, CameraCtrl \cite{He2024Cameractrl}, and Direct-a-Video \cite{yang2024direct} use camera pose as input to enhance control, while CVD \cite{kuang2024collaborative} extends CameraCtrl for multi-view generation, though still limited by motion complexity. To improve geometric consistency, Pose-guided diffusion \cite{tseng2023consistent}, CamCo \cite{Xu2024}, and CamI2V \cite{zheng2024cami2v} apply epipolar constraints for consistent viewpoints. VD3D \cite{bahmani2024vd3d} introduces a ControlNet\cite{Zhang2023}-like conditioning mechanism with spatiotemporal camera embeddings, enabling more precise control.
CamTrol \cite{hou2024training} offers a training-free approach that renders static point clouds into multi-view frames for video generation. Cavia \cite{xu2024cavia} introduces view-integrated attention mechanisms to improve viewpoint and temporal consistency, while I2VControl-Camera \cite{feng2024i2vcontrol} refines camera movement by employing point trajectories in the camera coordinate system. Despite these advancements, challenges in maintaining camera control and scene-scale consistency remain, which our method seeks to address. It is noted that 4Dim~\cite{watson2024controlling} introduces absolute scale but in  4D novel view synthesis (NVS) of scenes.



\section{Background}\label{sec:backgrnd}

\subsection{Cold Start Latency and Mitigation Techniques}

Traditional FaaS platforms mitigate cold starts through snapshotting, lightweight virtualization, and warm-state management. Snapshot-based methods like \textbf{REAP} and \textbf{Catalyzer} reduce initialization time by preloading or restoring container states but require significant memory and I/O resources, limiting scalability~\cite{dong_catalyzer_2020, ustiugov_benchmarking_2021}. Lightweight virtualization solutions, such as \textbf{Firecracker} microVMs, achieve fast startup times with strong isolation but depend on robust infrastructure, making them less adaptable to fluctuating workloads~\cite{agache_firecracker_2020}. Warm-state management techniques like \textbf{Faa\$T}~\cite{romero_faa_2021} and \textbf{Kraken}~\cite{vivek_kraken_2021} keep frequently invoked containers ready, balancing readiness and cost efficiency under predictable workloads but incurring overhead when demand is erratic~\cite{romero_faa_2021, vivek_kraken_2021}. While these methods perform well in resource-rich cloud environments, their resource intensity challenges applicability in edge settings.

\subsubsection{Edge FaaS Perspective}

In edge environments, cold start mitigation emphasizes lightweight designs, resource sharing, and hybrid task distribution. Lightweight execution environments like unikernels~\cite{edward_sock_2018} and \textbf{Firecracker}~\cite{agache_firecracker_2020}, as used by \textbf{TinyFaaS}~\cite{pfandzelter_tinyfaas_2020}, minimize resource usage and initialization delays but require careful orchestration to avoid resource contention. Function co-location, demonstrated by \textbf{Photons}~\cite{v_dukic_photons_2020}, reduces redundant initializations by sharing runtime resources among related functions, though this complicates isolation in multi-tenant setups~\cite{v_dukic_photons_2020}. Hybrid offloading frameworks like \textbf{GeoFaaS}~\cite{malekabbasi_geofaas_2024} balance edge-cloud workloads by offloading latency-tolerant tasks to the cloud and reserving edge resources for real-time operations, requiring reliable connectivity and efficient task management. These edge-specific strategies address cold starts effectively but introduce challenges in scalability and orchestration.

\subsection{Predictive Scaling and Caching Techniques}

Efficient resource allocation is vital for maintaining low latency and high availability in serverless platforms. Predictive scaling and caching techniques dynamically provision resources and reduce cold start latency by leveraging workload prediction and state retention.
Traditional FaaS platforms use predictive scaling and caching to optimize resources, employing techniques (OFC, FaasCache) to reduce cold starts. However, these methods rely on centralized orchestration and workload predictability, limiting their effectiveness in dynamic, resource-constrained edge environments.



\subsubsection{Edge FaaS Perspective}

Edge FaaS platforms adapt predictive scaling and caching techniques to constrain resources and heterogeneous environments. \textbf{EDGE-Cache}~\cite{kim_delay-aware_2022} uses traffic profiling to selectively retain high-priority functions, reducing memory overhead while maintaining readiness for frequent requests. Hybrid frameworks like \textbf{GeoFaaS}~\cite{malekabbasi_geofaas_2024} implement distributed caching to balance resources between edge and cloud nodes, enabling low-latency processing for critical tasks while offloading less critical workloads. Machine learning methods, such as clustering-based workload predictors~\cite{gao_machine_2020} and GRU-based models~\cite{guo_applying_2018}, enhance resource provisioning in edge systems by efficiently forecasting workload spikes. These innovations effectively address cold start challenges in edge environments, though their dependency on accurate predictions and robust orchestration poses scalability challenges.

\subsection{Decentralized Orchestration, Function Placement, and Scheduling}

Efficient orchestration in serverless platforms involves workload distribution, resource optimization, and performance assurance. While traditional FaaS platforms rely on centralized control, edge environments require decentralized and adaptive strategies to address unique challenges such as resource constraints and heterogeneous hardware.



\subsubsection{Edge FaaS Perspective}

Edge FaaS platforms adopt decentralized and adaptive orchestration frameworks to meet the demands of resource-constrained environments. Systems like \textbf{Wukong} distribute scheduling across edge nodes, enhancing data locality and scalability while reducing network latency. Lightweight frameworks such as \textbf{OpenWhisk Lite}~\cite{kravchenko_kpavelopenwhisk-light_2024} optimize resource allocation by decentralizing scheduling policies, minimizing cold starts and latency in edge setups~\cite{benjamin_wukong_2020}. Hybrid solutions like \textbf{OpenFaaS}~\cite{noauthor_openfaasfaas_2024} and \textbf{EdgeMatrix}~\cite{shen_edgematrix_2023} combine edge-cloud orchestration to balance resource utilization, retaining latency-sensitive functions at the edge while offloading non-critical workloads to the cloud. While these approaches improve flexibility, they face challenges in maintaining coordination and ensuring consistent performance across distributed nodes.


\section{Method}\label{sec:method}
\begin{figure}
    \centering
    \includegraphics[width=0.85\textwidth]{imgs/heatmap_acc.pdf}
    \caption{\textbf{Visualization of the proposed periodic Bayesian flow with mean parameter $\mu$ and accumulated accuracy parameter $c$ which corresponds to the entropy/uncertainty}. For $x = 0.3, \beta(1) = 1000$ and $\alpha_i$ defined in \cref{appd:bfn_cir}, this figure plots three colored stochastic parameter trajectories for receiver mean parameter $m$ and accumulated accuracy parameter $c$, superimposed on a log-scale heatmap of the Bayesian flow distribution $p_F(m|x,\senderacc)$ and $p_F(c|x,\senderacc)$. Note the \emph{non-monotonicity} and \emph{non-additive} property of $c$ which could inform the network the entropy of the mean parameter $m$ as a condition and the \emph{periodicity} of $m$. %\jj{Shrink the figures to save space}\hanlin{Do we need to make this figure one-column?}
    }
    \label{fig:vmbf_vis}
    \vskip -0.1in
\end{figure}
% \begin{wrapfigure}{r}{0.5\textwidth}
%     \centering
%     \includegraphics[width=0.49\textwidth]{imgs/heatmap_acc.pdf}
%     \caption{\textbf{Visualization of hyper-torus Bayesian flow based on von Mises Distribution}. For $x = 0.3, \beta(1) = 1000$ and $\alpha_i$ defined in \cref{appd:bfn_cir}, this figure plots three colored stochastic parameter trajectories for receiver mean parameter $m$ and accumulated accuracy parameter $c$, superimposed on a log-scale heatmap of the Bayesian flow distribution $p_F(m|x,\senderacc)$ and $p_F(c|x,\senderacc)$. Note the \emph{non-monotonicity} and \emph{non-additive} property of $c$. \jj{Shrink the figures to save space}}
%     \label{fig:vmbf_vis}
%     \vspace{-30pt}
% \end{wrapfigure}


In this section, we explain the detailed design of CrysBFN tackling theoretical and practical challenges. First, we describe how to derive our new formulation of Bayesian Flow Networks over hyper-torus $\mathbb{T}^{D}$ from scratch. Next, we illustrate the two key differences between \modelname and the original form of BFN: $1)$ a meticulously designed novel base distribution with different Bayesian update rules; and $2)$ different properties over the accuracy scheduling resulted from the periodicity and the new Bayesian update rules. Then, we present in detail the overall framework of \modelname over each manifold of the crystal space (\textit{i.e.} fractional coordinates, lattice vectors, atom types) respecting \textit{periodic E(3) invariance}. 

% In this section, we first demonstrate how to build Bayesian flow on hyper-torus $\mathbb{T}^{D}$ by overcoming theoretical and practical problems to provide a low-noise parameter-space approach to fractional atom coordinate generation. Next, we present how \modelname models each manifold of crystal space respecting \textit{periodic E(3) invariance}. 

\subsection{Periodic Bayesian Flow on Hyper-torus \texorpdfstring{$\mathbb{T}^{D}$}{}} 
For generative modeling of fractional coordinates in crystal, we first construct a periodic Bayesian flow on \texorpdfstring{$\mathbb{T}^{D}$}{} by designing every component of the totally new Bayesian update process which we demonstrate to be distinct from the original Bayesian flow (please see \cref{fig:non_add}). 
 %:) 
 
 The fractional atom coordinate system \citep{jiao2023crystal} inherently distributes over a hyper-torus support $\mathbb{T}^{3\times N}$. Hence, the normal distribution support on $\R$ used in the original \citep{bfn} is not suitable for this scenario. 
% The key problem of generative modeling for crystal is the periodicity of Cartesian atom coordinates $\vX$ requiring:
% \begin{equation}\label{eq:periodcity}
% p(\vA,\vL,\vX)=p(\vA,\vL,\vX+\vec{LK}),\text{where}~\vec{K}=\vec{k}\vec{1}_{1\times N},\forall\vec{k}\in\mathbb{Z}^{3\times1}
% \end{equation}
% However, there does not exist such a distribution supporting on $\R$ to model such property because the integration of such distribution over $\R$ will not be finite and equal to 1. Therefore, the normal distribution used in \citet{bfn} can not meet this condition.

To tackle this problem, the circular distribution~\citep{mardia2009directional} over the finite interval $[-\pi,\pi)$ is a natural choice as the base distribution for deriving the BFN on $\mathbb{T}^D$. 
% one natural choice is to 
% we would like to consider the circular distribution over the finite interval as the base 
% we find that circular distributions \citep{mardia2009directional} defined on a finite interval with lengths of $2\pi$ can be used as the instantiation of input distribution for the BFN on $\mathbb{T}^D$.
Specifically, circular distributions enjoy desirable periodic properties: $1)$ the integration over any interval length of $2\pi$ equals 1; $2)$ the probability distribution function is periodic with period $2\pi$.  Sharing the same intrinsic with fractional coordinates, such periodic property of circular distribution makes it suitable for the instantiation of BFN's input distribution, in parameterizing the belief towards ground truth $\x$ on $\mathbb{T}^D$. 
% \yuxuan{this is very complicated from my perspective.} \hanlin{But this property is exactly beautiful and perfectly fit into the BFN.}

\textbf{von Mises Distribution and its Bayesian Update} We choose von Mises distribution \citep{mardia2009directional} from various circular distributions as the form of input distribution, based on the appealing conjugacy property required in the derivation of the BFN framework.
% to leverage the Bayesian conjugacy property of von Mises distribution which is required by the BFN framework. 
That is, the posterior of a von Mises distribution parameterized likelihood is still in the family of von Mises distributions. The probability density function of von Mises distribution with mean direction parameter $m$ and concentration parameter $c$ (describing the entropy/uncertainty of $m$) is defined as: 
\begin{equation}
f(x|m,c)=vM(x|m,c)=\frac{\exp(c\cos(x-m))}{2\pi I_0(c)}
\end{equation}
where $I_0(c)$ is zeroth order modified Bessel function of the first kind as the normalizing constant. Given the last univariate belief parameterized by von Mises distribution with parameter $\theta_{i-1}=\{m_{i-1},\ c_{i-1}\}$ and the sample $y$ from sender distribution with unknown data sample $x$ and known accuracy $\alpha$ describing the entropy/uncertainty of $y$,  Bayesian update for the receiver is deducted as:
\begin{equation}
 h(\{m_{i-1},c_{i-1}\},y,\alpha)=\{m_i,c_i \}, \text{where}
\end{equation}
\begin{equation}\label{eq:h_m}
m_i=\text{atan2}(\alpha\sin y+c_{i-1}\sin m_{i-1}, {\alpha\cos y+c_{i-1}\cos m_{i-1}})
\end{equation}
\begin{equation}\label{eq:h_c}
c_i =\sqrt{\alpha^2+c_{i-1}^2+2\alpha c_{i-1}\cos(y-m_{i-1})}
\end{equation}
The proof of the above equations can be found in \cref{apdx:bayesian_update_function}. The atan2 function refers to  2-argument arctangent. Independently conducting  Bayesian update for each dimension, we can obtain the Bayesian update distribution by marginalizing $\y$:
\begin{equation}
p_U(\vtheta'|\vtheta,\bold{x};\alpha)=\mathbb{E}_{p_S(\bold{y}|\bold{x};\alpha)}\delta(\vtheta'-h(\vtheta,\bold{y},\alpha))=\mathbb{E}_{vM(\bold{y}|\bold{x},\alpha)}\delta(\vtheta'-h(\vtheta,\bold{y},\alpha))
\end{equation} 
\begin{figure}
    \centering
    \vskip -0.15in
    \includegraphics[width=0.95\linewidth]{imgs/non_add.pdf}
    \caption{An intuitive illustration of non-additive accuracy Bayesian update on the torus. The lengths of arrows represent the uncertainty/entropy of the belief (\emph{e.g.}~$1/\sigma^2$ for Gaussian and $c$ for von Mises). The directions of the arrows represent the believed location (\emph{e.g.}~ $\mu$ for Gaussian and $m$ for von Mises).}
    \label{fig:non_add}
    \vskip -0.15in
\end{figure}
\textbf{Non-additive Accuracy} 
The additive accuracy is a nice property held with the Gaussian-formed sender distribution of the original BFN expressed as:
\begin{align}
\label{eq:standard_id}
    \update(\parsn{}'' \mid \parsn{}, \x; \alpha_a+\alpha_b) = \E_{\update(\parsn{}' \mid \parsn{}, \x; \alpha_a)} \update(\parsn{}'' \mid \parsn{}', \x; \alpha_b)
\end{align}
Such property is mainly derived based on the standard identity of Gaussian variable:
\begin{equation}
X \sim \mathcal{N}\left(\mu_X, \sigma_X^2\right), Y \sim \mathcal{N}\left(\mu_Y, \sigma_Y^2\right) \Longrightarrow X+Y \sim \mathcal{N}\left(\mu_X+\mu_Y, \sigma_X^2+\sigma_Y^2\right)
\end{equation}
The additive accuracy property makes it feasible to derive the Bayesian flow distribution $
p_F(\boldsymbol{\theta} \mid \mathbf{x} ; i)=p_U\left(\boldsymbol{\theta} \mid \boldsymbol{\theta}_0, \mathbf{x}, \sum_{k=1}^{i} \alpha_i \right)
$ for the simulation-free training of \cref{eq:loss_n}.
It should be noted that the standard identity in \cref{eq:standard_id} does not hold in the von Mises distribution. Hence there exists an important difference between the original Bayesian flow defined on Euclidean space and the Bayesian flow of circular data on $\mathbb{T}^D$ based on von Mises distribution. With prior $\btheta = \{\bold{0},\bold{0}\}$, we could formally represent the non-additive accuracy issue as:
% The additive accuracy property implies the fact that the "confidence" for the data sample after observing a series of the noisy samples with accuracy ${\alpha_1, \cdots, \alpha_i}$ could be  as the accuracy sum  which could be  
% Here we 
% Here we emphasize the specific property of BFN based on von Mises distribution.
% Note that 
% \begin{equation}
% \update(\parsn'' \mid \parsn, \x; \alpha_a+\alpha_b) \ne \E_{\update(\parsn' \mid \parsn, \x; \alpha_a)} \update(\parsn'' \mid \parsn', \x; \alpha_b)
% \end{equation}
% \oyyw{please check whether the below equation is better}
% \yuxuan{I fill somehow confusing on what is the update distribution with $\alpha$. }
% \begin{equation}
% \update(\parsn{}'' \mid \parsn{}, \x; \alpha_a+\alpha_b) \ne \E_{\update(\parsn{}' \mid \parsn{}, \x; \alpha_a)} \update(\parsn{}'' \mid \parsn{}', \x; \alpha_b)
% \end{equation}
% We give an intuitive visualization of such difference in \cref{fig:non_add}. The untenability of this property can materialize by considering the following case: with prior $\btheta = \{\bold{0},\bold{0}\}$, check the two-step Bayesian update distribution with $\alpha_a,\alpha_b$ and one-step Bayesian update with $\alpha=\alpha_a+\alpha_b$:
\begin{align}
\label{eq:nonadd}
     &\update(c'' \mid \parsn, \x; \alpha_a+\alpha_b)  = \delta(c-\alpha_a-\alpha_b)
     \ne  \mathbb{E}_{p_U(\parsn' \mid \parsn, \x; \alpha_a)}\update(c'' \mid \parsn', \x; \alpha_b) \nonumber \\&= \mathbb{E}_{vM(\bold{y}_b|\bold{x},\alpha_a)}\mathbb{E}_{vM(\bold{y}_a|\bold{x},\alpha_b)}\delta(c-||[\alpha_a \cos\y_a+\alpha_b\cos \y_b,\alpha_a \sin\y_a+\alpha_b\sin \y_b]^T||_2)
\end{align}
A more intuitive visualization could be found in \cref{fig:non_add}. This fundamental difference between periodic Bayesian flow and that of \citet{bfn} presents both theoretical and practical challenges, which we will explain and address in the following contents.

% This makes constructing Bayesian flow based on von Mises distribution intrinsically different from previous Bayesian flows (\citet{bfn}).

% Thus, we must reformulate the framework of Bayesian flow networks  accordingly. % and do necessary reformulations of BFN. 

% \yuxuan{overall I feel this part is complicated by using the language of update distribution. I would like to suggest simply use bayesian update, to provide intuitive explantion.}\hanlin{See the illustration in \cref{fig:non_add}}

% That introduces a cascade of problems, and we investigate the following issues: $(1)$ Accuracies between sender and receiver are not synchronized and need to be differentiated. $(2)$ There is no tractable Bayesian flow distribution for a one-step sample conditioned on a given time step $i$, and naively simulating the Bayesian flow results in computational overhead. $(3)$ It is difficult to control the entropy of the Bayesian flow. $(4)$ Accuracy is no longer a function of $t$ and becomes a distribution conditioned on $t$, which can be different across dimensions.
%\jj{Edited till here}

\textbf{Entropy Conditioning} As a common practice in generative models~\citep{ddpm,flowmatching,bfn}, timestep $t$ is widely used to distinguish among generation states by feeding the timestep information into the networks. However, this paper shows that for periodic Bayesian flow, the accumulated accuracy $\vc_i$ is more effective than time-based conditioning by informing the network about the entropy and certainty of the states $\parsnt{i}$. This stems from the intrinsic non-additive accuracy which makes the receiver's accumulated accuracy $c$ not bijective function of $t$, but a distribution conditioned on accumulated accuracies $\vc_i$ instead. Therefore, the entropy parameter $\vc$ is taken logarithm and fed into the network to describe the entropy of the input corrupted structure. We verify this consideration in \cref{sec:exp_ablation}. 
% \yuxuan{implement variant. traditionally, the timestep is widely used to distinguish the different states by putting the timestep embedding into the networks. citation of FM, diffusion, BFN. However, we find that conditioned on time in periodic flow could not provide extra benefits. To further boost the performance, we introduce a simple yet effective modification term entropy conditional. This is based on that the accumulated accuracy which represents the current uncertainty or entropy could be a better indicator to distinguish different states. + Describe how you do this. }



\textbf{Reformulations of BFN}. Recall the original update function with Gaussian sender distribution, after receiving noisy samples $\y_1,\y_2,\dots,\y_i$ with accuracies $\senderacc$, the accumulated accuracies of the receiver side could be analytically obtained by the additive property and it is consistent with the sender side.
% Since observing sample $\y$ with $\alpha_i$ can not result in exact accuracy increment $\alpha_i$ for receiver, the accuracies between sender and receiver are not synchronized which need to be differentiated. 
However, as previously mentioned, this does not apply to periodic Bayesian flow, and some of the notations in original BFN~\citep{bfn} need to be adjusted accordingly. We maintain the notations of sender side's one-step accuracy $\alpha$ and added accuracy $\beta$, and alter the notation of receiver's accuracy parameter as $c$, which is needed to be simulated by cascade of Bayesian updates. We emphasize that the receiver's accumulated accuracy $c$ is no longer a function of $t$ (differently from the Gaussian case), and it becomes a distribution conditioned on received accuracies $\senderacc$ from the sender. Therefore, we represent the Bayesian flow distribution of von Mises distribution as $p_F(\btheta|\x;\alpha_1,\alpha_2,\dots,\alpha_i)$. And the original simulation-free training with Bayesian flow distribution is no longer applicable in this scenario.
% Different from previous BFNs where the accumulated accuracy $\rho$ is not explicitly modeled, the accumulated accuracy parameter $c$ (visualized in \cref{fig:vmbf_vis}) needs to be explicitly modeled by feeding it to the network to avoid information loss.
% the randomaccuracy parameter $c$ (visualized in \cref{fig:vmbf_vis}) implies that there exists information in $c$ from the sender just like $m$, meaning that $c$ also should be fed into the network to avoid information loss. 
% We ablate this consideration in  \cref{sec:exp_ablation}. 

\textbf{Fast Sampling from Equivalent Bayesian Flow Distribution} Based on the above reformulations, the Bayesian flow distribution of von Mises distribution is reframed as: 
\begin{equation}\label{eq:flow_frac}
p_F(\btheta_i|\x;\alpha_1,\alpha_2,\dots,\alpha_i)=\E_{\update(\parsnt{1} \mid \parsnt{0}, \x ; \alphat{1})}\dots\E_{\update(\parsn_{i-1} \mid \parsnt{i-2}, \x; \alphat{i-1})} \update(\parsnt{i} | \parsnt{i-1},\x;\alphat{i} )
\end{equation}
Naively sampling from \cref{eq:flow_frac} requires slow auto-regressive iterated simulation, making training unaffordable. Noticing the mathematical properties of \cref{eq:h_m,eq:h_c}, we  transform \cref{eq:flow_frac} to the equivalent form:
\begin{equation}\label{eq:cirflow_equiv}
p_F(\vec{m}_i|\x;\alpha_1,\alpha_2,\dots,\alpha_i)=\E_{vM(\y_1|\x,\alpha_1)\dots vM(\y_i|\x,\alpha_i)} \delta(\vec{m}_i-\text{atan2}(\sum_{j=1}^i \alpha_j \cos \y_j,\sum_{j=1}^i \alpha_j \sin \y_j))
\end{equation}
\begin{equation}\label{eq:cirflow_equiv2}
p_F(\vec{c}_i|\x;\alpha_1,\alpha_2,\dots,\alpha_i)=\E_{vM(\y_1|\x,\alpha_1)\dots vM(\y_i|\x,\alpha_i)}  \delta(\vec{c}_i-||[\sum_{j=1}^i \alpha_j \cos \y_j,\sum_{j=1}^i \alpha_j \sin \y_j]^T||_2)
\end{equation}
which bypasses the computation of intermediate variables and allows pure tensor operations, with negligible computational overhead.
\begin{restatable}{proposition}{cirflowequiv}
The probability density function of Bayesian flow distribution defined by \cref{eq:cirflow_equiv,eq:cirflow_equiv2} is equivalent to the original definition in \cref{eq:flow_frac}. 
\end{restatable}
\textbf{Numerical Determination of Linear Entropy Sender Accuracy Schedule} ~Original BFN designs the accuracy schedule $\beta(t)$ to make the entropy of input distribution linearly decrease. As for crystal generation task, to ensure information coherence between modalities, we choose a sender accuracy schedule $\senderacc$ that makes the receiver's belief entropy $H(t_i)=H(p_I(\cdot|\vtheta_i))=H(p_I(\cdot|\vc_i))$ linearly decrease \emph{w.r.t.} time $t_i$, given the initial and final accuracy parameter $c(0)$ and $c(1)$. Due to the intractability of \cref{eq:vm_entropy}, we first use numerical binary search in $[0,c(1)]$ to determine the receiver's $c(t_i)$ for $i=1,\dots, n$ by solving the equation $H(c(t_i))=(1-t_i)H(c(0))+tH(c(1))$. Next, with $c(t_i)$, we conduct numerical binary search for each $\alpha_i$ in $[0,c(1)]$ by solving the equations $\E_{y\sim vM(x,\alpha_i)}[\sqrt{\alpha_i^2+c_{i-1}^2+2\alpha_i c_{i-1}\cos(y-m_{i-1})}]=c(t_i)$ from $i=1$ to $i=n$ for arbitrarily selected $x\in[-\pi,\pi)$.

After tackling all those issues, we have now arrived at a new BFN architecture for effectively modeling crystals. Such BFN can also be adapted to other type of data located in hyper-torus $\mathbb{T}^{D}$.

\subsection{Equivariant Bayesian Flow for Crystal}
With the above Bayesian flow designed for generative modeling of fractional coordinate $\vF$, we are able to build equivariant Bayesian flow for each modality of crystal. In this section, we first give an overview of the general training and sampling algorithm of \modelname (visualized in \cref{fig:framework}). Then, we describe the details of the Bayesian flow of every modality. The training and sampling algorithm can be found in \cref{alg:train} and \cref{alg:sampling}.

\textbf{Overview} Operating in the parameter space $\bthetaM=\{\bthetaA,\bthetaL,\bthetaF\}$, \modelname generates high-fidelity crystals through a joint BFN sampling process on the parameter of  atom type $\bthetaA$, lattice parameter $\vec{\theta}^L=\{\bmuL,\brhoL\}$, and the parameter of fractional coordinate matrix $\bthetaF=\{\bmF,\bcF\}$. We index the $n$-steps of the generation process in a discrete manner $i$, and denote the corresponding continuous notation $t_i=i/n$ from prior parameter $\thetaM_0$ to a considerably low variance parameter $\thetaM_n$ (\emph{i.e.} large $\vrho^L,\bmF$, and centered $\bthetaA$).

At training time, \modelname samples time $i\sim U\{1,n\}$ and $\bthetaM_{i-1}$ from the Bayesian flow distribution of each modality, serving as the input to the network. The network $\net$ outputs $\net(\parsnt{i-1}^\mathcal{M},t_{i-1})=\net(\parsnt{i-1}^A,\parsnt{i-1}^F,\parsnt{i-1}^L,t_{i-1})$ and conducts gradient descents on loss function \cref{eq:loss_n} for each modality. After proper training, the sender distribution $p_S$ can be approximated by the receiver distribution $p_R$. 

At inference time, from predefined $\thetaM_0$, we conduct transitions from $\thetaM_{i-1}$ to $\thetaM_{i}$ by: $(1)$ sampling $\y_i\sim p_R(\bold{y}|\thetaM_{i-1};t_i,\alpha_i)$ according to network prediction $\predM{i-1}$; and $(2)$ performing Bayesian update $h(\thetaM_{i-1},\y^\calM_{i-1},\alpha_i)$ for each dimension. 

% Alternatively, we complete this transition using the flow-back technique by sampling 
% $\thetaM_{i}$ from Bayesian flow distribution $\flow(\btheta^M_{i}|\predM{i-1};t_{i-1})$. 

% The training objective of $\net$ is to minimize the KL divergence between sender distribution and receiver distribution for every modality as defined in \cref{eq:loss_n} which is equivalent to optimizing the negative variational lower bound $\calL^{VLB}$ as discussed in \cref{sec:preliminaries}. 

%In the following part, we will present the Bayesian flow of each modality in detail.

\textbf{Bayesian Flow of Fractional Coordinate $\vF$}~The distribution of the prior parameter $\bthetaF_0$ is defined as:
\begin{equation}\label{eq:prior_frac}
    p(\bthetaF_0) \defeq \{vM(\vm_0^F|\vec{0}_{3\times N},\vec{0}_{3\times N}),\delta(\vc_0^F-\vec{0}_{3\times N})\} = \{U(\vec{0},\vec{1}),\delta(\vc_0^F-\vec{0}_{3\times N})\}
\end{equation}
Note that this prior distribution of $\vm_0^F$ is uniform over $[\vec{0},\vec{1})$, ensuring the periodic translation invariance property in \cref{De:pi}. The training objective is minimizing the KL divergence between sender and receiver distribution (deduction can be found in \cref{appd:cir_loss}): 
%\oyyw{replace $\vF$ with $\x$?} \hanlin{notations follow Preliminary?}
\begin{align}\label{loss_frac}
\calL_F = n \E_{i \sim \ui{n}, \flow(\parsn{}^F \mid \vF ; \senderacc)} \alpha_i\frac{I_1(\alpha_i)}{I_0(\alpha_i)}(1-\cos(\vF-\predF{i-1}))
\end{align}
where $I_0(x)$ and $I_1(x)$ are the zeroth and the first order of modified Bessel functions. The transition from $\bthetaF_{i-1}$ to $\bthetaF_{i}$ is the Bayesian update distribution based on network prediction:
\begin{equation}\label{eq:transi_frac}
    p(\btheta^F_{i}|\parsnt{i-1}^\calM)=\mathbb{E}_{vM(\bold{y}|\predF{i-1},\alpha_i)}\delta(\btheta^F_{i}-h(\btheta^F_{i-1},\bold{y},\alpha_i))
\end{equation}
\begin{restatable}{proposition}{fracinv}
With $\net_{F}$ as a periodic translation equivariant function namely $\net_F(\parsnt{}^A,w(\parsnt{}^F+\vt),\parsnt{}^L,t)=w(\net_F(\parsnt{}^A,\parsnt{}^F,\parsnt{}^L,t)+\vt), \forall\vt\in\R^3$, the marginal distribution of $p(\vF_n)$ defined by \cref{eq:prior_frac,eq:transi_frac} is periodic translation invariant. 
\end{restatable}
\textbf{Bayesian Flow of Lattice Parameter \texorpdfstring{$\boldsymbol{L}$}{}}   
Noting the lattice parameter $\bm{L}$ located in Euclidean space, we set prior as the parameter of a isotropic multivariate normal distribution $\btheta^L_0\defeq\{\vmu_0^L,\vrho_0^L\}=\{\bm{0}_{3\times3},\bm{1}_{3\times3}\}$
% \begin{equation}\label{eq:lattice_prior}
% \btheta^L_0\defeq\{\vmu_0^L,\vrho_0^L\}=\{\bm{0}_{3\times3},\bm{1}_{3\times3}\}
% \end{equation}
such that the prior distribution of the Markov process on $\vmu^L$ is the Dirac distribution $\delta(\vec{\mu_0}-\vec{0})$ and $\delta(\vec{\rho_0}-\vec{1})$, 
% \begin{equation}
%     p_I^L(\boldsymbol{L}|\btheta_0^L)=\mathcal{N}(\bm{L}|\bm{0},\bm{I})
% \end{equation}
which ensures O(3)-invariance of prior distribution of $\vL$. By Eq. 77 from \citet{bfn}, the Bayesian flow distribution of the lattice parameter $\bm{L}$ is: 
\begin{align}% =p_U(\bmuL|\btheta_0^L,\bm{L},\beta(t))
p_F^L(\bmuL|\bm{L};t) &=\mathcal{N}(\bmuL|\gamma(t)\bm{L},\gamma(t)(1-\gamma(t))\bm{I}) 
\end{align}
where $\gamma(t) = 1 - \sigma_1^{2t}$ and $\sigma_1$ is the predefined hyper-parameter controlling the variance of input distribution at $t=1$ under linear entropy accuracy schedule. The variance parameter $\vrho$ does not need to be modeled and fed to the network, since it is deterministic given the accuracy schedule. After sampling $\bmuL_i$ from $p_F^L$, the training objective is defined as minimizing KL divergence between sender and receiver distribution (based on Eq. 96 in \citet{bfn}):
\begin{align}
\mathcal{L}_{L} = \frac{n}{2}\left(1-\sigma_1^{2/n}\right)\E_{i \sim \ui{n}}\E_{\flow(\bmuL_{i-1} |\vL ; t_{i-1})}  \frac{\left\|\vL -\predL{i-1}\right\|^2}{\sigma_1^{2i/n}},\label{eq:lattice_loss}
\end{align}
where the prediction term $\predL{i-1}$ is the lattice parameter part of network output. After training, the generation process is defined as the Bayesian update distribution given network prediction:
\begin{equation}\label{eq:lattice_sampling}
    p(\bmuL_{i}|\parsnt{i-1}^\calM)=\update^L(\bmuL_{i}|\predL{i-1},\bmuL_{i-1};t_{i-1})
\end{equation}
    

% The final prediction of the lattice parameter is given by $\bmuL_n = \predL{n-1}$.
% \begin{equation}\label{eq:final_lattice}
%     \bmuL_n = \predL{n-1}
% \end{equation}

\begin{restatable}{proposition}{latticeinv}\label{prop:latticeinv}
With $\net_{L}$ as  O(3)-equivariant function namely $\net_L(\parsnt{}^A,\parsnt{}^F,\vQ\parsnt{}^L,t)=\vQ\net_L(\parsnt{}^A,\parsnt{}^F,\parsnt{}^L,t),\forall\vQ^T\vQ=\vI$, the marginal distribution of $p(\bmuL_n)$ defined by \cref{eq:lattice_sampling} is O(3)-invariant. 
\end{restatable}


\textbf{Bayesian Flow of Atom Types \texorpdfstring{$\boldsymbol{A}$}{}} 
Given that atom types are discrete random variables located in a simplex $\calS^K$, the prior parameter of $\boldsymbol{A}$ is the discrete uniform distribution over the vocabulary $\parsnt{0}^A \defeq \frac{1}{K}\vec{1}_{1\times N}$. 
% \begin{align}\label{eq:disc_input_prior}
% \parsnt{0}^A \defeq \frac{1}{K}\vec{1}_{1\times N}
% \end{align}
% \begin{align}
%     (\oh{j}{K})_k \defeq \delta_{j k}, \text{where }\oh{j}{K}\in \R^{K},\oh{\vA}{KD} \defeq \left(\oh{a_1}{K},\dots,\oh{a_N}{K}\right) \in \R^{K\times N}
% \end{align}
With the notation of the projection from the class index $j$ to the length $K$ one-hot vector $ (\oh{j}{K})_k \defeq \delta_{j k}, \text{where }\oh{j}{K}\in \R^{K},\oh{\vA}{KD} \defeq \left(\oh{a_1}{K},\dots,\oh{a_N}{K}\right) \in \R^{K\times N}$, the Bayesian flow distribution of atom types $\vA$ is derived in \citet{bfn}:
\begin{align}
\flow^{A}(\parsn^A \mid \vA; t) &= \E_{\N{\y \mid \beta^A(t)\left(K \oh{\vA}{K\times N} - \vec{1}_{K\times N}\right)}{\beta^A(t) K \vec{I}_{K\times N \times N}}} \delta\left(\parsn^A - \frac{e^{\y}\parsnt{0}^A}{\sum_{k=1}^K e^{\y_k}(\parsnt{0})_{k}^A}\right).
\end{align}
where $\beta^A(t)$ is the predefined accuracy schedule for atom types. Sampling $\btheta_i^A$ from $p_F^A$ as the training signal, the training objective is the $n$-step discrete-time loss for discrete variable \citep{bfn}: 
% \oyyw{can we simplify the next equation? Such as remove $K \times N, K \times N \times N$}
% \begin{align}
% &\calL_A = n\E_{i \sim U\{1,n\},\flow^A(\parsn^A \mid \vA ; t_{i-1}),\N{\y \mid \alphat{i}\left(K \oh{\vA}{KD} - \vec{1}_{K\times N}\right)}{\alphat{i} K \vec{I}_{K\times N \times N}}} \ln \N{\y \mid \alphat{i}\left(K \oh{\vA}{K\times N} - \vec{1}_{K\times N}\right)}{\alphat{i} K \vec{I}_{K\times N \times N}}\nonumber\\
% &\qquad\qquad\qquad-\sum_{d=1}^N \ln \left(\sum_{k=1}^K \out^{(d)}(k \mid \parsn^A; t_{i-1}) \N{\ydd{d} \mid \alphat{i}\left(K\oh{k}{K}- \vec{1}_{K\times N}\right)}{\alphat{i} K \vec{I}_{K\times N \times N}}\right)\label{discdisc_t_loss_exp}
% \end{align}
\begin{align}
&\calL_A = n\E_{i \sim U\{1,n\},\flow^A(\parsn^A \mid \vA ; t_{i-1}),\N{\y \mid \alphat{i}\left(K \oh{\vA}{KD} - \vec{1}\right)}{\alphat{i} K \vec{I}}} \ln \N{\y \mid \alphat{i}\left(K \oh{\vA}{K\times N} - \vec{1}\right)}{\alphat{i} K \vec{I}}\nonumber\\
&\qquad\qquad\qquad-\sum_{d=1}^N \ln \left(\sum_{k=1}^K \out^{(d)}(k \mid \parsn^A; t_{i-1}) \N{\ydd{d} \mid \alphat{i}\left(K\oh{k}{K}- \vec{1}\right)}{\alphat{i} K \vec{I}}\right)\label{discdisc_t_loss_exp}
\end{align}
where $\vec{I}\in \R^{K\times N \times N}$ and $\vec{1}\in\R^{K\times D}$. When sampling, the transition from $\bthetaA_{i-1}$ to $\bthetaA_{i}$ is derived as:
\begin{equation}
    p(\btheta^A_{i}|\parsnt{i-1}^\calM)=\update^A(\btheta^A_{i}|\btheta^A_{i-1},\predA{i-1};t_{i-1})
\end{equation}

The detailed training and sampling algorithm could be found in \cref{alg:train} and \cref{alg:sampling}.




\section{Experiments}
\label{sec:experiments}
The experiments are designed to address two key research questions.
First, \textbf{RQ1} evaluates whether the average $L_2$-norm of the counterfactual perturbation vectors ($\overline{||\perturb||}$) decreases as the model overfits the data, thereby providing further empirical validation for our hypothesis.
Second, \textbf{RQ2} evaluates the ability of the proposed counterfactual regularized loss, as defined in (\ref{eq:regularized_loss2}), to mitigate overfitting when compared to existing regularization techniques.

% The experiments are designed to address three key research questions. First, \textbf{RQ1} investigates whether the mean perturbation vector norm decreases as the model overfits the data, aiming to further validate our intuition. Second, \textbf{RQ2} explores whether the mean perturbation vector norm can be effectively leveraged as a regularization term during training, offering insights into its potential role in mitigating overfitting. Finally, \textbf{RQ3} examines whether our counterfactual regularizer enables the model to achieve superior performance compared to existing regularization methods, thus highlighting its practical advantage.

\subsection{Experimental Setup}
\textbf{\textit{Datasets, Models, and Tasks.}}
The experiments are conducted on three datasets: \textit{Water Potability}~\cite{kadiwal2020waterpotability}, \textit{Phomene}~\cite{phomene}, and \textit{CIFAR-10}~\cite{krizhevsky2009learning}. For \textit{Water Potability} and \textit{Phomene}, we randomly select $80\%$ of the samples for the training set, and the remaining $20\%$ for the test set, \textit{CIFAR-10} comes already split. Furthermore, we consider the following models: Logistic Regression, Multi-Layer Perceptron (MLP) with 100 and 30 neurons on each hidden layer, and PreactResNet-18~\cite{he2016cvecvv} as a Convolutional Neural Network (CNN) architecture.
We focus on binary classification tasks and leave the extension to multiclass scenarios for future work. However, for datasets that are inherently multiclass, we transform the problem into a binary classification task by selecting two classes, aligning with our assumption.

\smallskip
\noindent\textbf{\textit{Evaluation Measures.}} To characterize the degree of overfitting, we use the test loss, as it serves as a reliable indicator of the model's generalization capability to unseen data. Additionally, we evaluate the predictive performance of each model using the test accuracy.

\smallskip
\noindent\textbf{\textit{Baselines.}} We compare CF-Reg with the following regularization techniques: L1 (``Lasso''), L2 (``Ridge''), and Dropout.

\smallskip
\noindent\textbf{\textit{Configurations.}}
For each model, we adopt specific configurations as follows.
\begin{itemize}
\item \textit{Logistic Regression:} To induce overfitting in the model, we artificially increase the dimensionality of the data beyond the number of training samples by applying a polynomial feature expansion. This approach ensures that the model has enough capacity to overfit the training data, allowing us to analyze the impact of our counterfactual regularizer. The degree of the polynomial is chosen as the smallest degree that makes the number of features greater than the number of data.
\item \textit{Neural Networks (MLP and CNN):} To take advantage of the closed-form solution for computing the optimal perturbation vector as defined in (\ref{eq:opt-delta}), we use a local linear approximation of the neural network models. Hence, given an instance $\inst_i$, we consider the (optimal) counterfactual not with respect to $\model$ but with respect to:
\begin{equation}
\label{eq:taylor}
    \model^{lin}(\inst) = \model(\inst_i) + \nabla_{\inst}\model(\inst_i)(\inst - \inst_i),
\end{equation}
where $\model^{lin}$ represents the first-order Taylor approximation of $\model$ at $\inst_i$.
Note that this step is unnecessary for Logistic Regression, as it is inherently a linear model.
\end{itemize}

\smallskip
\noindent \textbf{\textit{Implementation Details.}} We run all experiments on a machine equipped with an AMD Ryzen 9 7900 12-Core Processor and an NVIDIA GeForce RTX 4090 GPU. Our implementation is based on the PyTorch Lightning framework. We use stochastic gradient descent as the optimizer with a learning rate of $\eta = 0.001$ and no weight decay. We use a batch size of $128$. The training and test steps are conducted for $6000$ epochs on the \textit{Water Potability} and \textit{Phoneme} datasets, while for the \textit{CIFAR-10} dataset, they are performed for $200$ epochs.
Finally, the contribution $w_i^{\varepsilon}$ of each training point $\inst_i$ is uniformly set as $w_i^{\varepsilon} = 1~\forall i\in \{1,\ldots,m\}$.

The source code implementation for our experiments is available at the following GitHub repository: \url{https://anonymous.4open.science/r/COCE-80B4/README.md} 

\subsection{RQ1: Counterfactual Perturbation vs. Overfitting}
To address \textbf{RQ1}, we analyze the relationship between the test loss and the average $L_2$-norm of the counterfactual perturbation vectors ($\overline{||\perturb||}$) over training epochs.

In particular, Figure~\ref{fig:delta_loss_epochs} depicts the evolution of $\overline{||\perturb||}$ alongside the test loss for an MLP trained \textit{without} regularization on the \textit{Water Potability} dataset. 
\begin{figure}[ht]
    \centering
    \includegraphics[width=0.85\linewidth]{img/delta_loss_epochs.png}
    \caption{The average counterfactual perturbation vector $\overline{||\perturb||}$ (left $y$-axis) and the cross-entropy test loss (right $y$-axis) over training epochs ($x$-axis) for an MLP trained on the \textit{Water Potability} dataset \textit{without} regularization.}
    \label{fig:delta_loss_epochs}
\end{figure}

The plot shows a clear trend as the model starts to overfit the data (evidenced by an increase in test loss). 
Notably, $\overline{||\perturb||}$ begins to decrease, which aligns with the hypothesis that the average distance to the optimal counterfactual example gets smaller as the model's decision boundary becomes increasingly adherent to the training data.

It is worth noting that this trend is heavily influenced by the choice of the counterfactual generator model. In particular, the relationship between $\overline{||\perturb||}$ and the degree of overfitting may become even more pronounced when leveraging more accurate counterfactual generators. However, these models often come at the cost of higher computational complexity, and their exploration is left to future work.

Nonetheless, we expect that $\overline{||\perturb||}$ will eventually stabilize at a plateau, as the average $L_2$-norm of the optimal counterfactual perturbations cannot vanish to zero.

% Additionally, the choice of employing the score-based counterfactual explanation framework to generate counterfactuals was driven to promote computational efficiency.

% Future enhancements to the framework may involve adopting models capable of generating more precise counterfactuals. While such approaches may yield to performance improvements, they are likely to come at the cost of increased computational complexity.


\subsection{RQ2: Counterfactual Regularization Performance}
To answer \textbf{RQ2}, we evaluate the effectiveness of the proposed counterfactual regularization (CF-Reg) by comparing its performance against existing baselines: unregularized training loss (No-Reg), L1 regularization (L1-Reg), L2 regularization (L2-Reg), and Dropout.
Specifically, for each model and dataset combination, Table~\ref{tab:regularization_comparison} presents the mean value and standard deviation of test accuracy achieved by each method across 5 random initialization. 

The table illustrates that our regularization technique consistently delivers better results than existing methods across all evaluated scenarios, except for one case -- i.e., Logistic Regression on the \textit{Phomene} dataset. 
However, this setting exhibits an unusual pattern, as the highest model accuracy is achieved without any regularization. Even in this case, CF-Reg still surpasses other regularization baselines.

From the results above, we derive the following key insights. First, CF-Reg proves to be effective across various model types, ranging from simple linear models (Logistic Regression) to deep architectures like MLPs and CNNs, and across diverse datasets, including both tabular and image data. 
Second, CF-Reg's strong performance on the \textit{Water} dataset with Logistic Regression suggests that its benefits may be more pronounced when applied to simpler models. However, the unexpected outcome on the \textit{Phoneme} dataset calls for further investigation into this phenomenon.


\begin{table*}[h!]
    \centering
    \caption{Mean value and standard deviation of test accuracy across 5 random initializations for different model, dataset, and regularization method. The best results are highlighted in \textbf{bold}.}
    \label{tab:regularization_comparison}
    \begin{tabular}{|c|c|c|c|c|c|c|}
        \hline
        \textbf{Model} & \textbf{Dataset} & \textbf{No-Reg} & \textbf{L1-Reg} & \textbf{L2-Reg} & \textbf{Dropout} & \textbf{CF-Reg (ours)} \\ \hline
        Logistic Regression   & \textit{Water}   & $0.6595 \pm 0.0038$   & $0.6729 \pm 0.0056$   & $0.6756 \pm 0.0046$  & N/A    & $\mathbf{0.6918 \pm 0.0036}$                     \\ \hline
        MLP   & \textit{Water}   & $0.6756 \pm 0.0042$   & $0.6790 \pm 0.0058$   & $0.6790 \pm 0.0023$  & $0.6750 \pm 0.0036$    & $\mathbf{0.6802 \pm 0.0046}$                    \\ \hline
%        MLP   & \textit{Adult}   & $0.8404 \pm 0.0010$   & $\mathbf{0.8495 \pm 0.0007}$   & $0.8489 \pm 0.0014$  & $\mathbf{0.8495 \pm 0.0016}$     & $0.8449 \pm 0.0019$                    \\ \hline
        Logistic Regression   & \textit{Phomene}   & $\mathbf{0.8148 \pm 0.0020}$   & $0.8041 \pm 0.0028$   & $0.7835 \pm 0.0176$  & N/A    & $0.8098 \pm 0.0055$                     \\ \hline
        MLP   & \textit{Phomene}   & $0.8677 \pm 0.0033$   & $0.8374 \pm 0.0080$   & $0.8673 \pm 0.0045$  & $0.8672 \pm 0.0042$     & $\mathbf{0.8718 \pm 0.0040}$                    \\ \hline
        CNN   & \textit{CIFAR-10} & $0.6670 \pm 0.0233$   & $0.6229 \pm 0.0850$   & $0.7348 \pm 0.0365$   & N/A    & $\mathbf{0.7427 \pm 0.0571}$                     \\ \hline
    \end{tabular}
\end{table*}

\begin{table*}[htb!]
    \centering
    \caption{Hyperparameter configurations utilized for the generation of Table \ref{tab:regularization_comparison}. For our regularization the hyperparameters are reported as $\mathbf{\alpha/\beta}$.}
    \label{tab:performance_parameters}
    \begin{tabular}{|c|c|c|c|c|c|c|}
        \hline
        \textbf{Model} & \textbf{Dataset} & \textbf{No-Reg} & \textbf{L1-Reg} & \textbf{L2-Reg} & \textbf{Dropout} & \textbf{CF-Reg (ours)} \\ \hline
        Logistic Regression   & \textit{Water}   & N/A   & $0.0093$   & $0.6927$  & N/A    & $0.3791/1.0355$                     \\ \hline
        MLP   & \textit{Water}   & N/A   & $0.0007$   & $0.0022$  & $0.0002$    & $0.2567/1.9775$                    \\ \hline
        Logistic Regression   &
        \textit{Phomene}   & N/A   & $0.0097$   & $0.7979$  & N/A    & $0.0571/1.8516$                     \\ \hline
        MLP   & \textit{Phomene}   & N/A   & $0.0007$   & $4.24\cdot10^{-5}$  & $0.0015$    & $0.0516/2.2700$                    \\ \hline
       % MLP   & \textit{Adult}   & N/A   & $0.0018$   & $0.0018$  & $0.0601$     & $0.0764/2.2068$                    \\ \hline
        CNN   & \textit{CIFAR-10} & N/A   & $0.0050$   & $0.0864$ & N/A    & $0.3018/
        2.1502$                     \\ \hline
    \end{tabular}
\end{table*}

\begin{table*}[htb!]
    \centering
    \caption{Mean value and standard deviation of training time across 5 different runs. The reported time (in seconds) corresponds to the generation of each entry in Table \ref{tab:regularization_comparison}. Times are }
    \label{tab:times}
    \begin{tabular}{|c|c|c|c|c|c|c|}
        \hline
        \textbf{Model} & \textbf{Dataset} & \textbf{No-Reg} & \textbf{L1-Reg} & \textbf{L2-Reg} & \textbf{Dropout} & \textbf{CF-Reg (ours)} \\ \hline
        Logistic Regression   & \textit{Water}   & $222.98 \pm 1.07$   & $239.94 \pm 2.59$   & $241.60 \pm 1.88$  & N/A    & $251.50 \pm 1.93$                     \\ \hline
        MLP   & \textit{Water}   & $225.71 \pm 3.85$   & $250.13 \pm 4.44$   & $255.78 \pm 2.38$  & $237.83 \pm 3.45$    & $266.48 \pm 3.46$                    \\ \hline
        Logistic Regression   & \textit{Phomene}   & $266.39 \pm 0.82$ & $367.52 \pm 6.85$   & $361.69 \pm 4.04$  & N/A   & $310.48 \pm 0.76$                    \\ \hline
        MLP   &
        \textit{Phomene} & $335.62 \pm 1.77$   & $390.86 \pm 2.11$   & $393.96 \pm 1.95$ & $363.51 \pm 5.07$    & $403.14 \pm 1.92$                     \\ \hline
       % MLP   & \textit{Adult}   & N/A   & $0.0018$   & $0.0018$  & $0.0601$     & $0.0764/2.2068$                    \\ \hline
        CNN   & \textit{CIFAR-10} & $370.09 \pm 0.18$   & $395.71 \pm 0.55$   & $401.38 \pm 0.16$ & N/A    & $1287.8 \pm 0.26$                     \\ \hline
    \end{tabular}
\end{table*}

\subsection{Feasibility of our Method}
A crucial requirement for any regularization technique is that it should impose minimal impact on the overall training process.
In this respect, CF-Reg introduces an overhead that depends on the time required to find the optimal counterfactual example for each training instance. 
As such, the more sophisticated the counterfactual generator model probed during training the higher would be the time required. However, a more advanced counterfactual generator might provide a more effective regularization. We discuss this trade-off in more details in Section~\ref{sec:discussion}.

Table~\ref{tab:times} presents the average training time ($\pm$ standard deviation) for each model and dataset combination listed in Table~\ref{tab:regularization_comparison}.
We can observe that the higher accuracy achieved by CF-Reg using the score-based counterfactual generator comes with only minimal overhead. However, when applied to deep neural networks with many hidden layers, such as \textit{PreactResNet-18}, the forward derivative computation required for the linearization of the network introduces a more noticeable computational cost, explaining the longer training times in the table.

\subsection{Hyperparameter Sensitivity Analysis}
The proposed counterfactual regularization technique relies on two key hyperparameters: $\alpha$ and $\beta$. The former is intrinsic to the loss formulation defined in (\ref{eq:cf-train}), while the latter is closely tied to the choice of the score-based counterfactual explanation method used.

Figure~\ref{fig:test_alpha_beta} illustrates how the test accuracy of an MLP trained on the \textit{Water Potability} dataset changes for different combinations of $\alpha$ and $\beta$.

\begin{figure}[ht]
    \centering
    \includegraphics[width=0.85\linewidth]{img/test_acc_alpha_beta.png}
    \caption{The test accuracy of an MLP trained on the \textit{Water Potability} dataset, evaluated while varying the weight of our counterfactual regularizer ($\alpha$) for different values of $\beta$.}
    \label{fig:test_alpha_beta}
\end{figure}

We observe that, for a fixed $\beta$, increasing the weight of our counterfactual regularizer ($\alpha$) can slightly improve test accuracy until a sudden drop is noticed for $\alpha > 0.1$.
This behavior was expected, as the impact of our penalty, like any regularization term, can be disruptive if not properly controlled.

Moreover, this finding further demonstrates that our regularization method, CF-Reg, is inherently data-driven. Therefore, it requires specific fine-tuning based on the combination of the model and dataset at hand.
\section{Conclusion}
In this work, we propose a simple yet effective approach, called SMILE, for graph few-shot learning with fewer tasks. Specifically, we introduce a novel dual-level mixup strategy, including within-task and across-task mixup, for enriching the diversity of nodes within each task and the diversity of tasks. Also, we incorporate the degree-based prior information to learn expressive node embeddings. Theoretically, we prove that SMILE effectively enhances the model's generalization performance. Empirically, we conduct extensive experiments on multiple benchmarks and the results suggest that SMILE significantly outperforms other baselines, including both in-domain and cross-domain few-shot settings.

%\bibliographystyle{abbrv}
\bibliography{references}




%%%%%%%%%%%%%%%%%%%%%%%%%%%%%%%%%%%%%%%%%%%%%%%%%%%%%%%%%%%%


\newpage
\appendix
\onecolumn


\begin{center}
	{\Large \textbf{Appendix for \ours}}
\end{center}

\startcontents[sections]
\printcontents[sections]{l}{1}{\setcounter{tocdepth}{2}}

\section{Algorithm~\ref{alg:twisted-smc}}
The full procedure of the twised Sequential Monte-Carlo sampler is given in Algorithm~\ref{alg:twisted-smc}.


\begin{algorithm}[!h]
\footnotesize
\caption{Twisted SMC for Diffusion Model}
\label{alg:twisted-smc}
\begin{algorithmic}[1]
\Require Pretrained diffusion model, number of particles $N$, time horizon $T$, $\Phi(\mathbf{z})$ (Eq.~\ref{eq:twisting})
\State Initialize $\mathbf{z}_n^T \sim p_{\theta}(\mathbf{z}^T)$,\; $w_n \gets \Phi(\mathbf{z}_n^T)$
\For{$t = T, \dots, 1$}
  \State \textbf{Resample:} \\
    $\quad\quad\{\mathbf{z}_n^t\}_{n=1}^N \sim \mathrm{Multinomial}\bigl(\{\mathbf{z}_n^t\}_{n=1}^N;\,\{w_n^t\}_{n=1}^N\bigr)$
  \For{$k = 1 \dots K$}
    \State $\displaystyle 
           \hat{s}_k \gets s_\theta(\mathbf{z}_k^t,\mathbf{c},t) \;-\; 
             \gamma \,\nabla_{\mathbf{z}_k^t}\Bigl[U(\pi^*_{i-1},\mathbf{z})\Bigr]$
    \State $\mathbf{z}_k^{t-1} \sim 
            \mathcal{N}\bigl(\mathbf{z}_k^t + \sigma^2 \hat{s}_k,\;\hat{\beta}^2\bigr)$
    \State $\displaystyle
           w_k^{t-1} \gets 
           \frac{p_{\theta}\bigl(\mathbf{z}_k^{t-1} \mid \mathbf{z}_k^t,\mathbf{c}\bigr)\,\Phi(\mathbf{z}_k^{t-1})}{
                 \hat{p}_{\theta}\bigl(\mathbf{z}_k^{t-1} \mid \mathbf{z}_k^t,\mathbf{c}\bigr)\,\Phi(\mathbf{z}_k^{t})}$
  \EndFor
\EndFor
\State \textbf{Output:} Weighted particles $\{\mathbf{z}_k^0,\, w_k^0\}_{k=1}^K$
\end{algorithmic}
\end{algorithm}

\section{Examples of mixed strategy over mixed strategies}\label{appdx:example}
Let us consider a national park with $3$ target regions to protect, and poachers' pure strategies specify how many snares to put in each target region. Two examples of poacher pure strategies could be $\mathbf{z}_1 = (3, 4, 3)$ and $\mathbf{z}_2 = (0, 0, 10)$. Each entry in the pure strategy determines the number of snares a poacher will place in the corresponding target region. Let us denote poachers' pure strategy space as $\mathcal{Z} = \{\mathbf{z}_1, \mathbf{z}_2\}$. 

A mixed strategy $\tau$ is a distribution on the pure strategy space, i.e., $\tau \in \Delta(\mathcal{Z})$. Denote the subset of mixed strategies which satisfy the constraint 
$D_{\rm KL}(\tau(\mathbf{z}) || p_{\theta}(\mathbf{z} | \mathbf{c})) \leq \rho$ as $\mathcal{T}$. One such example $\tau_1$ could be $P(\mathbf{z}_1) = 0.1$ and $P(\mathbf{z}_2) = 0.9$. Another degenerate example of mixed strategy $\tau_2$ could be $P(\mathbf{z}_1) = 0$ and $P(\mathbf{z}_2) = 1$.

A mixed strategy over mixed strategies $\sigma$ is a distribution on the constrained mixed strategy space, i.e., $\sigma \in \Delta(\mathcal{T})$. One example of mixed strategy over mixed strategies $\sigma_1$ could be $P(\tau_1) = 0.1$ and $P(\tau_2) = 0.9$. Another degenerate example $\sigma_2$ could be $P(\tau_1) = 0$ and $P(\tau_2) = 1$.

A mixed strategy over mixed strategies is still a distribution on the original pure strategy space, i.e., $\sigma \in \Delta(\mathcal{Z})$. For example, an alternative way to view $\sigma_1$ could be $$P(\mathbf{z}_1) = P(\sigma_1(\tau_1)) \cdot P(\tau_1(\mathbf{z}_1)) + P(\sigma_1(\tau_2)) \cdot P(\tau_2(\mathbf{z}_1)) = 0.01$$ 
and 
$$P(\mathbf{z}_2) = P(\sigma_1(\tau_1)) \cdot P(\tau_1(\mathbf{z}_2)) + P(\sigma_1(\tau_2)) \cdot P(\tau_2(\mathbf{z}_2)) = 0.99$$
However, it is proven in Proposition $\ref{thm:mixed_over_mixed}$ that all mixed strategy over mixed strategies $\sigma$ satisfy $D_{\rm KL}(\sigma || p_{\theta}(\mathbf{z} | \mathbf{c})) \leq \rho$, which is not generally true for elements in $\Delta(\mathcal{Z})$.

From Section $\ref{sec:double_oracle}$ onward, readers can interpret $\mathcal{T}$ as the pure strategy space and $\sigma$ as a standard mixed strategy. Despite each pure strategy $\tau \in \mathcal{T}$ being a distribution, 
all standard terminologies of game theory remain applicable.



% The pure strategy $z = (3, 4, 3)$ means the poacher will put $3$, $4$, and $5$ snares in each target region, respectively. 



\section{Proof of Proposition \ref{thm:mixed_over_mixed}}\label{appdx:mixed_over_mixed}
% $$D_{\mathrm{KL}}(\tau(\mathbf{z}) \,\|\, p_{\theta}(\mathbf{z}\mid \mathbf{c})) \leq \rho \}$$

We now show that for any $\pi(\mathbf{x})\in \Delta(\mathcal{X})$, 

\begin{align*}
\min_{\tau(\mathbf{z})} \left\{ \mathbb{E}_{\pi(\mathbf{x})}\mathbb{E}_{\tau(\mathbf{z})} \left[ u(\mathbf{x}, \mathbf{z}) \right] : D_{\mathrm{KL}}(\tau(\mathbf{z}) \,\|\ p_{\theta}(\mathbf{z} | \mathbf{c})) \leq \rho \right\}=\\
\min_{\sigma(\tau)} \left\{ \mathbb{E}_{\pi(\mathbf{x})}\mathbb{E}_{\sigma(\tau)} \left(\mathbb{E}_{\tau(\mathbf{z})}\left[ u(\mathbf{x}, \mathbf{z}) \right] \right): D_{\mathrm{KL}}(\tau(\mathbf{z}) \,\|\,  p_{\theta}(\mathbf{z} | \mathbf{c})) \leq \rho \right\}.
\end{align*}
From this, the original theorem follows.

Consider any solution $\tau'(\mathbf{z})$ that attains the minimum on the left-hand side. Define a degenerate distribution over strategies $\sigma'(\tau) = \delta[\tau = \tau']$, i.e., it places all its mass on $\tau'$. Note that $\tau'$ satisfies the divergence constraint on the left, so $\sigma'(\tau)$ will also satisfy the corresponding constraint on the right-hand side. Since the expected value under $\sigma'(\tau)$ matches the value attained by $\tau'$, we have the left side is not smaller than the right side.

Now take any solution $\sigma'(\tau)$ that attains the minimum on the right side. Define
$\tau'(\mathbf{z}) = \mathbb{E}_{\sigma'(\tau)}[\tau(\mathbf{z})]$. 
Because a mixture over mixed strategies is itself a valid mixed strategy in $\Delta(\mathcal{Z})$, $\tau'(\mathbf{z})$ is admissible on the left side.

By the convexity of the divergence measure $D$, we have:
\[
D_{\mathrm{KL}}(\tau'(\mathbf{z})\,\|\ p_\theta(\mathbf{z}| \mathbf{c})) 
= D_{\mathrm{KL}}\bigl(\mathbb{E}_{\sigma'(\tau)}\tau(\mathbf{z}) \,\|\  p_\theta(\mathbf{z}| \mathbf{c})\bigr)
\leq \mathbb{E}_{\sigma'(\tau)}[D_{\mathrm{KL}}(\tau(\mathbf{z}) \,\|\ p_\theta(\mathbf{z}|\mathbf{c})]
\leq \rho.
\]
Here, the first inequality follows from the convexity of $D$, and the second inequality is by the construction of $\sigma'(\tau)$, which satisfies the original constraint on the right side.

Thus, $\tau'(\mathbf{z})$ satisfies the left side constraint and attains the same expected value as $\sigma'(\tau)$. We then obtain that the left side is not larger than the right side.

Combining both parts, we conclude the proof. 

% \section{Appendix: Double Oracle}

% \paragraph{Conventional Notations}
% Let $w$ denote the realization of an infinite sequence of payoff matrix outcome. We denote $(p_i^*(w), q_i^*(w))$ as the equilibrium to the subgame $(X_i, Y_i, \hat{u}_i)$, where $\hat{u}_i$ is the sample approximation of the utility matrix at $i$-th iteration of Algorithm \ref{alg:double-oracle}. Let $\Delta_i$ denote the largest difference among the cell of the true payoff table and sample payoff table at the $i$-th iteration of the algorithm. We assume the utility is bounded between $[0, M]$.

% Lemma \ref{lem:utility_bound} shows that for any strategy pair, the sample estimation error is bounded by the largest sample estimation error in the payoff matrix.
% \begin{lemma}\label{lem:utility_bound}
%     For any $p_i \in \Delta X_i$ and $q_i \in \Delta Y_i$, we have $|\hat{U_i}(p_i, q_i) - U(p_i, q_i)| \leq \Delta_i$.
% \end{lemma}
% \begin{proof}
%     We write 
%     \begin{equation}\label{eq:utility_bound_pf}
%         |\hat{U_i}(p_i, q_i) - U(p_i, q_i)| = \sum_{x}\sum_{y} p_i(x) \cdot q_i (y) \cdot |\hat{U_i}(x, y) - U(x, y)|
%     \end{equation}
%    $|\hat{U_i}(x, y) - U(x, y)|$ denotes the sample estimation error for pure strategy pair $(x,y)$ in the payoff matrix. The maximum on the right-hand side of \ref{eq:utility_bound_pf} is obtained when putting all the probability mass on the strategy pair with the largest sample estimation error, which is $\Delta_i$.
% \end{proof}

% Lemma \ref{lem:order_stats_bound} proposes a sampling scheme that bounds the largest sample estimation error in the payoff matrix at every iteration of the algorithm. Because of lemma \ref{lem:utility_bound}, the proposed sampling scheme bounds the sample estimation error for the utility of any strategy pair as well. 
% \begin{lemma}\label{lem:order_stats_bound}
%     Given any $(\delta, \mathsf{p})$, if we sample
%     $$N_i = \left\lceil \frac{M(i+1)^2 \cdot \mathsf{p}}{\delta^2} \right\rceil$$
%     particles for each cell at the $i$-th iteration of the algorithm, then $\|\Delta\|_\infty$ with probability at least $1-\mathsf{prob}$. (We omit the time subscript of $\|\Delta\|_\infty$ because this bound holds for every iteration). 
% \end{lemma}

% \begin{proof}
%     For any cell $(j,k)$ in the matrix, we apply the Chebyshev bound of twisted sampling:
%     $$P(|\Delta_{j,k}| > \delta) \leq \frac{M}{N \cdot \delta^2}$$
%     Since at $i$-th iteration, there are $(i+1)^2$ cells in the payoff matrix, we apply the union bound and obtain:
%      $$P(\|\Delta\|_\infty > \delta) \leq \frac{M (i+1)^2}{N\cdot \delta^2}$$
%     By setting $N_i = \frac{M(i+1)^2 \cdot \mathsf{p}}{\delta^2}$, we have 
%     $$P(\|\Delta\|_\infty > \delta) \leq \mathsf{p}$$
% \end{proof}

% % \begin{lemma}
% %     Given a pure strategy $x$, if for every $i$,
% %     $$U(p_i^*(w)) + 2\delta \geq U(x, q_i^*(w)) with probability 1-p$$
% %     and $U(p_i^*(w), q_i^*(w)) \to U(p^*, q^*)$ and $U(x, q_i^*(w)) \to U(x, q^*)$, then 
% %     $$U(p^*, q^*) + 2\delta \geq U(x, q^*) with probability 1-p$$
% % \end{lemma}
% Lemma \ref{lem:prob-preserve} establishes that the probabilistic guarantee on utility bounds is preserved in the limit as the strategy pair converges.
% \begin{lemma}\label{lem:prob-preserve}
%     Given a pure strategy \( x \), if for every \( i \),
%     \[
%     U(p_i^*, q_i^*) + 2\delta \geq U(x, q_i^*) \quad \text{with probability at least } 1 - p,
%     \]
%     and for any $w$
%     \[
%     U(p_i^*(w), q_i^*(w)) \to U(p^*(w), q^*(w)) \quad \text{and} \quad U(x, q_i^*(w)) \to U(x, q^*(w)),
%     \]
%     then
%     \[
%     U(p^*, q^*) + 2\delta \geq U(x, q^*) \quad \text{with probability at least } 1 - p.
%     \]
% \end{lemma}
% \begin{proof}
% Let's consider the event $E_i = \left\{ 
% w \mid U(p_i^*(w), q_i^*(w)) + 2\delta < U(x, q_i^*(w))
% \right\}$, which means the collection of elementary outcomes that makes the $i$-th item in the weak convergent subsequence satisfies the inequality and $E = \left\{ 
% w \mid U(p^*(w), q^*(w)) + 2\delta < U(x, q^*(w))
% \right\}$. Let $\liminf\limits_{i \to \infty} E_i = 
%     \bigcup\limits_{n=1}^\infty 
% \bigcap\limits_{i=n}^\infty E_i$, where $\liminf\limits_{i \to \infty} E_i$ is the event that $w$ belongs to $E_i$ for all sufficiently large $i$. 

% Fix a $w$. Since  \[
%     U(p_i^*(w), q_i^*(w)) \to U(p^*(w), q^*(w)) \quad \text{and} \quad U(x, q_i^*(w)) \to U(x, q^*(w)),
%     \]
% if $U(p^*(w), q^*(w)) + 2\delta < U(x, q^*(w))$, then there exists $i_0$ such that when $i > i_0$,
% \[
%     U(p_i^*(w), q_i^*(w)) + 2\delta < U(x, q_i^*(w)) 
% \]
% Hence, $E \subseteq \liminf\limits_{i \to \infty} E_i$.
% By monotonicity of probability measure,
% \begin{align}\label{eq:monotonicity-of-measure}
%     P(E) \leq P(\liminf\limits_{i \to \infty} E_i)
% \end{align}
% % By applying Fatou's lemma to the indicator function $\mathbf{1}_{E_i}(w)$, we have
% % \begin{align*}
% %     P(\liminf\limits_{i \to \infty} E_i) \leq \liminf\limits_{i \to \infty}P(E_i)
% % \end{align*}
% Also, we have 
% \begin{align}\label{eq:fatou-lemma}
%     P(\liminf\limits_{i \to \infty} E_i) \leq \liminf\limits_{i \to \infty}P(E_i) \leq \mathsf{p}
% \end{align}
% where the first inequality follows from applying Fatou's lemma to the indicator function $\mathbf{1}_{E_i}(w)$, and the second inequality follows because $\forall i, P(E_i) \leq p $. Combining \ref{eq:monotonicity-of-measure} and \ref{eq:fatou-lemma}, we obtain 
% \[
%     U(p^*, q^*) + 2\delta \geq U(x, q^*) \quad \text{with probability at least } 1 - \mathsf{p}.
% \]
% \end{proof}

% \begin{theorem}
%     Let $G = (X, Y, u)$ be a continuous game. If $G$ is an infinite game and $\epsilon > 0$, given any $(\delta, \mathsf{p})$, by sampling
%      $$N_i =\left\lceil  \frac{M(i+1)^2 \cdot \mathsf{p}}{\delta^2} \right\rceil$$
%      particles for each cell in the payoff matrix at the $i$-th iteration of the algorithm, the algorithm will stop in a finite number of iterations, and when it stops, the algorithm will converge to a  $10\delta + \epsilon$ equilibrium with probability at least $1-\mathsf{p}$.
% \end{theorem}

% \begin{proof}
%     % (Technically, every $\hat{U}$ should have a time subscript, but for notation simplicity let's omit it for now. It won't impact the proof because every time we use convergence or weak convergence, we apply the convergence result to the original utility function).

%     % Let $w$ denote the realization of the infinite sequence of payoff matrix outcome. We denote $(p_i^*(w), q_i^*(w))$ as the equilibrium to the subgame $(X_i, Y_i, \hat{u}_i)$, where $\hat{u}_i$ is the sample approximation of the utility matrix at $i$-th iteration. 
    
%     We list here several results that are already proven in \cite{adam2021double}.
%     \begin{enumerate}
%         \item There exists a weakly convergent subsequence, which for simplicity, will be denoted by the same indices. Therefore, $p_i^*(w) \Rightarrow p^*(w)$ for some $p^*(w)$ and $q_i^*(w) \Rightarrow q^*(w)$ for some $q^*(w)$, where $\Rightarrow$ denotes weak convergence.
%         \item If $p_i \Rightarrow p$ in $\Delta_X$ and $q_i \Rightarrow q$ in $\Delta_Y$, then $U(p_i, q_i) \to U(p,q)$. If $p_i \Rightarrow p$ in $\Delta_X$ and $y_i \to y$ in $Y$, then $U(p_i, y_i) \to U(p,y)$.
%         \item For any $p \in \Delta_X$ we have
%         $$\min_{y \in Y} U(p,y) = \min_{q \in \Delta_Y} U(p,q)$$
%     \end{enumerate}
%     % From \cite{adam2021double}, we know there exists a weakly convergent subsequence, which for simplicity, will be denoted by the same indices. Therefore, $p_i^*(w) \Rightarrow p^*$ for some $p^*$ and $q_i^*(w) \Rightarrow q^*$ for some $q^*$, where $\Rightarrow$ denotes weak convergence.
%     By lemma \ref{lem:utility_bound} and \ref{lem:order_stats_bound}, our sampling scheme ensures that for any strategy pair $(p,q)$ and iteration $i$, we have $|U(p,q) - \hat{U}_i(p,q)| \leq \delta$ with probability at least $1-\mathsf{p}$.

%     Consider any $x$ such that $x \in X_{i_0}$ for some $i_0$. Take an arbitrary $i \geq i_0$, which implies $x \in X_i$. Since $(p_i^*, q_i^*)$ is an equilibrium of the subgame $(X_i, Y_i, \hat{u}_i)$, we  get
%     $$\hat{U}_i(p_i^*, q_i^*) \geq \hat{U}_i(x, q_i^*)$$
%     Since $U(p_i^*, q_i^*)$ and $\hat{U}_i(p_i^*, q_i^*)$ differ by at most $\delta$ with probability at least $1-\mathsf{p}$, we have
%     \begin{align*}
%         U(p_i^*, q_i^*) + 2\delta \geq U(x, q_i^*) \to U(x, q^*) \text{ with probability at least } 1 - \mathsf{p}.
%     \end{align*}
%     % $$U(p_i^*, q_i^*) + 2\delta \geq U(x, q_i^*) \to U(x, q^*)$$
%     Since $U(p_i^*, q_i^*) \to U(p^*, q^*)$, by lemma $\ref{lem:prob-preserve}$ we have
%     \begin{align}\label{eq:for-closed-x}
%         U(p^*, q^*) + 2\delta \geq U(x, q^*) \text{ with probability at least } 1 - \mathsf{p}
%     \end{align}
%     for all $x \in \cup X_i$. Since $U$ is continuous, the previous inequality holds for all $x \in cl(\cup X_i)$.
    
%  Fix now an arbitrary $x \in X$. Note $x_{i+1}$ is best response to $\hat{U}_i$ (since ranger oracle uses finite sample estimation of payoff matrix), and we have
%     $$\hat{U}_i(x_{i+1}, q_i^*) \geq \hat{U}_i(x, q_i^*)$$

%      Because $U$ and $\hat{U}_i$ differ by at most $\delta$ with probability at least $1-\mathsf{p}$, we have
%     \begin{align}\label{eq:original-eq-6}
%         U(x_{i+1}, q_i^*) + 2\delta \geq U(x, q_i^*) \to U(x, q^*) \text{ with probability at least } 1 - \mathsf{p}
%     \end{align}
%      Since $x_{i+1} \in X_{i+1}$ and by compactness of $X$, we can select a convergence subsequence $x_i \to \tilde{x}$, where $\tilde{x} \in cl(\cup X_i)$. This allows us to use \ref{eq:for-closed-x} to obtain 
%     \begin{align}\label{eq:original-eq-7}
%         U(x_{i+1}, q_i^*) \to U(\tilde{x}, q^*) \leq U(p^*, q^*) + 2 \delta \text{ with probability at least } 1 - \mathsf{p}
%     \end{align}

%     Now we first show that our algorithm will hit the terminating condition with finite iteration. Combining \ref{eq:original-eq-6} and \ref{eq:original-eq-7}, there exists a sufficiently large $L_0$ such that if $i > L_0$, we have \begin{align}\label{eq:ranger_util_range}
%         U(x_{i+1}, q_i^*) \in (U(p^*, q^*)-2\delta-\frac{\epsilon}{2},U(p^*, q^*) + \frac{\epsilon}{2}) \text{ with probability at least } 1 - \mathsf{p}
%     \end{align}
%     Repeat the analogous argument in the other variable: there exists a sufficiently large $L_1$ such that for $i > L_1$, we have \begin{align}\label{eq:poacher_util_range}
%    U(p_i^*, y_{i+1}) \in (U(p^*, q^*)-2\delta-\frac{\epsilon}{2},U(p^*, q^*) + \frac{\epsilon}{2}) \text{ with probability at least } 1 - \mathsf{p}
%     \end{align}
%      The two sides are not symmetrical because the best response for the poacher doesn't use the finite sample approximation of payoff matrix, thus having a smaller error.
%      Take $L = max\{L_0, L_1\}$. For every $i > L$, combine \ref{eq:ranger_util_range} and \ref{eq:poacher_util_range}, we have
%      \begin{align*}
%          U(x_{i+1}, q_i^*)- U(p_i^*, y_{i+1}) \in (- 2\delta - \epsilon, 4\delta + \epsilon) \text{ with probability at least } 1 - \mathsf{p}
%      \end{align*}
%     This implies
%     \begin{align*}
%         \hat{U}_i(x_{i+1}, q_i^*)- \hat{U}_i(p_i^*, y_{i+1}) \in (- 4\delta - \epsilon, 6\delta + \epsilon) \text{ with probability at least } 1 - \mathsf{p}
%     \end{align*}
%     The chance that the algorithm has not stopped after $L+t$ iterations is bounded above by $\mathsf{p}^t$. Since $0 <\mathsf{p} < 1$, this implies the algorithm will terminate in finite iterations. 

%      When the algorithm terminates, we have the following:
%      \begin{equation}\label{eq:direction1}
%     \begin{aligned}
%         U(p_i^*, q_i^*) & \leq U(x_{i+1}, q_i^*) + 2\delta \\
%         & \leq \hat{U}_i(x_{i+1},q_i^*) + \delta + 2\delta \\
%         & \leq \hat{U}_i(p_i^*, y_{i+1}) + 4\delta + 3\delta + \epsilon \\
%         & \leq \hat{U}_i(p_i^*, y') + 7\delta + \epsilon \text{ where } y'=\arg\min_{y \in \Delta_Y} U(p_i^*, y)\\
%         & \leq \min_{y' \in Y}U(p_i^*, y') + \delta + 7\delta + \epsilon \\
%         & = \min_{q \in \Delta Y}U(p_i^*, q) + 8\delta + \epsilon
%     \end{aligned}
%     \end{equation}
%     with probability at least $1-\mathsf{p}$. The first relation follows from the definition of best response (with a relaxation $2\delta$ introduced as the best response is to $\hat{U}_i$), the second and fifth relations come from the error bound on utility deviation, the third relation comes from the terminating condition, and the fourth relation comes from best response. Similarly, we have 
%     \begin{equation}\label{eq:direction2}
%         \begin{aligned}
%         U(p_i^*, q_i^*) & \geq U(p_i^*, y_{i+1}) - 2\delta \\
%         & \geq \hat{U}_i(p_i^*,y_{i+1}) - \delta - 2\delta \\
%         & \geq \hat{U}_i(x_{i+1},q_i^*) - 6\delta - 3\delta - \epsilon \\
%         & \geq \hat{U}_i(x', q_i^{*}) - 9\delta - \epsilon \text{ where } x'=\arg\max_{x \in \Delta_X} U(x, q_i^*)\\
%         & \geq \max_{x' \in X}U(x', q_i^*) - \delta - 9\delta - \epsilon \\
%         & = \max_{p \in \Delta X}U(p, q_i^*) - 10\delta - \epsilon
%         \end{aligned}
%     \end{equation}
%     % \begin{align}\label{eq:direction2}
%     %     U(p_i^*, q_i^*) & \geq U(p_i^*, y_{i+1}) - 2\delta \\
%     %     & \geq \hat{U}_i(p_i^*,y_{i+1}) - \delta - 2\delta \\
%     %     & \geq \hat{U}_i(x_{i+1},q_i^*) - 6\delta - 3\delta - \epsilon \\
%     %     & \geq \hat{U}_i(x', q_i^{*}) - 9\delta + \epsilon \text{ where } x'=\arg\min_{x \in \Delta_X} U(x, q_i^*)\\
%     %     & \geq \min_{x' \in X}U(x', q_i^*) - \delta - 9\delta + \epsilon \\
%     %     & = \min_{q \in \Delta Y}U(p_i^*, q) - 10\delta - \epsilon
%     % \end{align}
    
%      with probability at least $1-\mathsf{p}$. Combining \ref{eq:direction1} and \ref{eq:direction2}, we show that $(p_i^*, q_i^*)$ is a $10\delta+\epsilon$ equilibrium with probability at least $1-\mathsf{p}$.
% \end{proof}

    




     
    % % $$U(x_{i+1}, q_i^*)- U(p_i^*, y_{i+1}) \in (- 2\delta - \epsilon, 4\delta + \epsilon)$$
    % with probability at least $1-\mathsf{prob}$.
   
    
    %  Combine \ref{eq:original-eq-6} and \ref{eq:original-eq-7} we have 
    % $$U(p^*, q^*) + 4 \delta \geq U(x, q^*) \text{ with probability at least } 1 - \mathsf{p}$$
    %  Repeating the analogous argument in the other variable yields 
    % $$U(p^*, q^*) - 2 \delta \leq U(p^*, y) \text{ with probability at least } 1 - \mathsf{p}$$
    % Note the two sides are not symmetrical because the best response for the poacher does not use finite sample estimation, which reduces the error. 

    


    
% \newpage
    
%     We know that $p_i^* \Rightarrow p^*$ for some $p^*$ and $q_i^* \Rightarrow q^*$ for some $q^*$, where $\Rightarrow$ denotes weak convergence.  

%     Consider any $x$ such that $x \in X_{i_0}$ for some $i_0$. Take an arbitrary $i \geq i_0$, which implies $x \in X_i$. Since $(p_i^*, q_i^*)$ is an equilibrium of the subgame $(X_i, Y_i, \hat{u})$, we  get
%     $$\hat{U}_i(p_i^*, q_i^*) \geq \hat{U}_i(x, q_i^*)$$
%     Since $U(p_i^*, q_i^*)$ and $\hat{U}_i(p_i^*, q_i^*)$ differ by at most $\delta$ with prob $1-\mathsf{prob}$, we have 
%     $$U(p_i^*, q_i^*) + 2\delta \geq U(x, q_i^*) \to U(x, q^*)$$
%     Since $U(p_i^*, q_i^*) \to U(p^*, q^*)$, this implies
%     \begin{align}\label{eq-for-closedx}
%         U(p^*, q^*) + 2\delta \geq U(x, q^*)
%     \end{align}
%     for all $x \in \cup X_i$. Since $U$ is continuous, the previous inequality holds for all $x \in cl(\cup X_i)$.

%     Fix now an arbitrary $x \in X$. Note $x_{i+1}$ is best response to $\hat{U}_i$ (ranger oracle uses finite sample estimation), so we have
%     $$\hat{U}_i(x_{i+1}, q_i^*) \geq \hat{U}_i(x, q_i^*)$$
%     Because $U$ and $\hat{U}_i$ differ by at most $\delta$ with probability at least $1-\mathsf{prob}$, we have
%     \begin{align}\label{eq6}
%         U(x_{i+1}, q_i^*) + 2\delta \geq U(x, q_i^*) \to U(x, q^*)
%     \end{align}
%     Since $x_{i+1} \in X_{i+1}$ and by compactness of $X$, we can select a convergence subsequence $x_i \to \tilde{x}$, where $\tilde{x} \in cl(\cup X_i)$. This allows us to use \ref{eq-for-closedx} to obtain 
%     \begin{align}\label{eq7}
%         U(x_{i+1}, q_i^*) \to U(\tilde{x}, q^*) \leq U(p^*, q^*) + 2 \delta
%     \end{align}
%     Combine \ref{eq6} and \ref{eq7} we have 
%     $$U(p^*, q^*) + 4 \delta \geq U(x, q^*)$$
%     Repeating the analogous argument in the other variable yields 
%     $$U(p^*, q^*) - 2 \delta \leq U(p^*, y)$$
%     (The two sides are not symmetrical because the best response for the poacher is perfect as it does not use finite sample estimation). 
%     % All the derivation above holds with $$\|\Delta\|_\infty < \delta$$, which happens with probability at least $1-\mathsf{prob}$.  
%     % Hence, $(p^*, q^*)$ is a $4\delta - equilibrium$ with probability at least $1-\mathsf{prob}$. 

   

%     Note by \ref{eq7} we have 
%     $$U(p^*, q^*) +2 \delta \geq U(\tilde{x}, q^*) \leftarrow U(x_{i+1}, q_i^*)$$
%     By best response,
%     $$U(x_{i+1}, q_i^*)  + 2\delta \geq U(p_i^*, q_i^*) \rightarrow U(p^*, q^*)$$
%     Hence, there exists a sufficiently large $L_0$ such that if $i > L_0$, we have 
%     $$U(x_{i+1}, q_i^*) \in (U(p^*, q^*)-2\delta-\frac{\epsilon}{2},U(p^*, q^*)+2\delta + \frac{\epsilon}{2})$$
%     with probability at least $1-\mathsf{prob}$.
%     Similarly, there exists a sufficiently large $L_1$ such that for $i > L_1$, we have 
%      $$U(p_i^*, y_{i+1}) \in (U(p^*, q^*)-2\delta-\frac{\epsilon}{2},U(p^*, q^*) + \frac{\epsilon}{2})$$
%     with probability at least $1-\mathsf{prob}$. Again, the two sides are not symmetrical because diffusion (poacher) oracle does not involve finite sampling error. Take $L = max\{L_0, L_1\}$. For every $i > L$, we have
%     $$U(x_{i+1}, q_i^*)- U(p_i^*, y_{i+1}) \in (- 2\delta - \epsilon, 4\delta + \epsilon)$$
%     with probability at least $1-\mathsf{prob}$.
%     This implies
%      $$\hat{U}_i(x_{i+1}, q_i^*)- \hat{U}_i(p_i^*, y_{i+1}) \in (- 6\delta - \epsilon, 6\delta + \epsilon)$$
%     with probability at least $1-\mathsf{prob}$.
%     Let $i = L+1+k$, the chance that the algorithm has not stopped is less than $\mathsf{prob}^k$. Since $0 < \mathsf{prob} < 1$, this shows the terminating condition will happen in finite iterations with probability $1$.

%     When the algorithm stops, we have the following:
%     \begin{align*}
%         U(p_i^*, q_i^*) & \leq U(x_{i+1}, q_i^*) + 2\delta \\
%         & \leq U(p_i^*, y_{i+1}) + 8\delta + \epsilon \\
%         &  = min_{y' \in Y}U(p_i^*, y') + 2\delta + 8\delta + \epsilon \\
%         & = min_{q \in \Delta Y}U(p_i^*, q) + 10\delta + \epsilon
%     \end{align*}
%     with probability at least $1-\mathsf{prob}$. The first and the third relation above follow from the definition of best response (with a relaxation $2\delta$ introduced as the best response is to $\hat{U}_i$, and second from terminating condition. We repeat the same procedure for the other variables, and we obtain $(p_i^*, q_i^*)$ is a $10\delta+\epsilon$ equilibrium with probability at least $1-\mathsf{prob}$.


    
% \end{proof}






% \section{Appendix: Approximation Error of the Ranger Oracle}

% \subsection{Problem Setup}

% Suppose the goal is to solve a stochastic optimization problem:
% \[
% x^* = \arg\min_{x \in \mathcal{X}} \mathbb{E}_q[F(x, \xi)],
% \]
% where \( \xi \sim q \) is a random variable, and \( F(x, \xi) \) is the objective function.

% In the SMC context, we approximate the expectation \( \mathbb{E}_q[F(x, \xi)] \) using particle estimates:
% \[
% \hat{\mathbb{E}}[F(x, \xi)] = \frac{1}{N} \sum_{n=1}^N F(x, X_t^n),
% \]
% where \( \{X_t^n\}_{n=1}^N \) are the particles sampled according to the SMC algorithm.

% The \textbf{particle-based stochastic optimization problem} becomes:
% \[
% \hat{x}_N = \arg\min_{x \in \mathcal{X}} \hat{\mathbb{E}}[F(x, \xi)].
% \]

% We are interested in bounding the \textbf{objective gap}:
% \[
% \mathbb{E}_q[F(\hat{x}_N, \xi)] - \mathbb{E}_q[F(x^*, \xi)].
% \]


% \subsection{Derivation Steps}

% \subsubsection{Step 1: Decompose the Objective Gap}

% Decompose the gap into two components:
% \[
% \mathbb{E}_q[F(\hat{x}_N, \xi)] - \mathbb{E}_q[F(x^*, \xi)] = \underbrace{\mathbb{E}_q[F(\hat{x}_N, \xi)] - \hat{\mathbb{E}}[F(\hat{x}_N, \xi)]}_{\text{(I)}} + \underbrace{\hat{\mathbb{E}}[F(\hat{x}_N, \xi)] - \hat{\mathbb{E}}[F(x^*, \xi)]}_{\text{(II)}} + \underbrace{\hat{\mathbb{E}}[F(x^*, \xi)] - \mathbb{E}_q[F(x^*, \xi)]}_{\text{(III)}}.
% \]

% - (I) and (III): Error from approximating the true expectation with the particle estimate.

% - (II): Suboptimality of the particle-based solution \( \hat{x}_N \) in the approximate problem.


% \subsubsection{Step 2: Bound the Approximation Errors (I) and (III)}

% From the \textbf{MSE bounds in Proposition 11.3} in \cite{chopin2020introduction}, the particle estimate satisfies:
% \[
% \mathbb{E}\left[\left(\hat{\mathbb{E}}[F(x, \xi)] - \mathbb{E}_q[F(x, \xi)]\right)^2\right] \leq \frac{c_t \|F(x, \cdot)\|_\infty^2}{N},
% \]
% where $c_t$ is a constant.

% Using \textbf{Chebyshev's inequality}, the probability that the particle estimate deviates significantly from the true expectation is bounded as:
% \[
% \mathbb{P}\left(\left|\hat{\mathbb{E}}[F(x, \xi)] - \mathbb{E}_q[F(x, \xi)]\right| > \epsilon\right) \leq \frac{c_t \|F(x, \cdot)\|_\infty^2}{N \epsilon^2}.
% \]

% Thus, for any \( x \in \mathcal{X} \), the particle estimate converges in probability to the true expectation.

% Applying this bound to both \( \hat{x}_N \) and \( x^* \), we have:
% \[
% \mathbb{P}\left(|\mathbb{E}_q[F(\hat{x}_N, \xi)] - \hat{\mathbb{E}}[F(\hat{x}_N, \xi)]| > \epsilon\right) \leq \frac{c_t \|F(x, \cdot)\|_\infty^2}{N \epsilon^2},
% \]
% and
% \[
% \mathbb{P}\left(|\hat{\mathbb{E}}[F(x^*, \xi)] - \mathbb{E}_q[F(x^*, \xi)]| > \epsilon\right) \leq \frac{c_t \|F(x, \cdot)\|_\infty^2}{N \epsilon^2}.
% \]

% \subsubsection{Step 3: Bound the Suboptimality Term (II)}

% By definition of \( \hat{x}_N \), it minimizes the particle-based objective:
% \[
% \hat{x}_N = \arg\min_{x \in \mathcal{X}} \hat{\mathbb{E}}[F(x, \xi)].
% \]

% Thus:
% \[
% \hat{\mathbb{E}}[F(\hat{x}_N, \xi)] \leq \hat{\mathbb{E}}[F(x^*, \xi)].
% \]

% Rearranging, we find that:
% \[
% \hat{\mathbb{E}}[F(\hat{x}_N, \xi)] - \hat{\mathbb{E}}[F(x^*, \xi)] \leq 0.
% \]

% This means the suboptimality term is non-positive.

% \subsubsection{Step 4: Combine the Bounds}

% Combining the results:
% \[
% \mathbb{E}_q[F(\hat{x}_N, \xi)] - \mathbb{E}_q[F(x^*, \xi)] \leq 2 \sup_{x \in \mathcal{X}} \left|\hat{\mathbb{E}}[F(x, \xi)] - \mathbb{E}_q[F(x, \xi)]\right|.
% \]

% Using the uniform bound from Step 2:
% \[
% \sup_{x \in \mathcal{X}} \left|\hat{\mathbb{E}}[F(x, \xi)] - \mathbb{E}_q[F(x, \xi)]\right| = \mathcal{O}_p\left(\sqrt{\frac{1}{N}}\right).
% \]

% Thus, the \textbf{objective gap} satisfies:
% \[
% \mathbb{E}_q[F(\hat{x}_N, \xi)] - \mathbb{E}_q[F(x^*, \xi)] = \mathcal{O}_p\left(\sqrt{\frac{1}{N}}\right).
% \]


% \subsubsection{Final Probabilistic Bound}

% Using the probabilistic result from Chebyshev's inequality, for any \( \epsilon > 0 \), we can state that:
% \[
% \mathbb{P}\left(\mathbb{E}_q[F(\hat{x}_N, \xi)] - \mathbb{E}_q[F(x^*, \xi)] > \epsilon\right) \leq \frac{c_t \|F(x, \cdot)\|_\infty^2}{N \epsilon^2}.
% \]

% This shows that the \textbf{objective gap decreases as \( \mathcal{O}(1/N) \) in probability}, and for sufficiently large \( N \), the particle-based stochastic optimization converges to the true optimal solution with high probability.

% \subsubsection{Concavity of the utility function in ranger effort}
% \begin{assumption}[Concavity of Utility Function]
%     We assume that the utility function is twice differentiable with respect to $a$, and we assume that $\frac{\partial^2 V(a,q(z))}{\partial a^2} \leq 0 \ \forall a$, i.e., the utility function is concave in ranger effort.
% \end{assumption}

% The concavity assumption implies that $\frac{\partial V(a,q(z))}{\partial a}$ is decreasing in $a$, which is equivalent to saying there is diminishing marginal return to ranger effort. The intuition is within the first unit of effort, the most obvious snares in the field are first removed. Hence, the same effort will lead to less snares found later. %This phenomenon is also documented in [ideally we can find some conservation literature here]. 

% \begin{assumption}
%     The distribution learned by diffusion model, i.e., $\pi(\mathbf{z})$ has full support.
% \end{assumption}
\subsection{Error Gap Between Aligned and Misaligned Data}\label{subsec:proof-align-misalign}







\thmalignment*

\begin{proof}

For the aligned case, we can derive the mean squared error (MSE) as follows:
\begin{equation}\label{eq:mse_aligned}
    \mathrm{MSE}_\mathrm{aligned} = \inf_{\boldsymbol{\alpha} \in R^{m^P}, \boldsymbol{\beta} \in R^{m^S}} \|\mathbf{y} - \mathbf{X}^P \boldsymbol{\alpha} - \mathbf{X}^S \boldsymbol{\beta}\|
\end{equation}
The ordinary least squares (OLS) estimator of $\boldsymbol{\alpha}$ is given by:
\begin{equation}
    \hat{\boldsymbol{\alpha}} := (\mathbf{X}^{P \top} \mathbf{X}^P)^{-1} \mathbf{X}^P (\mathbf{y} - \mathbb{E}[\mathbf{R}] \mathbf{X}^S \boldsymbol{\beta}) 
\end{equation}
For a permutation matrix $\mathbf{R}$ under uniform distribution, we have $\mathbb{E}[\mathbf{R}] = \frac{1}{n}\mathds{1}^\top \mathds{1}$. Therefore:
\begin{equation}\label{eq:alpha_hat}
    \hat{\boldsymbol{\alpha}} = (\mathbf{X}^{P \top} \mathbf{X}^P)^{-1} \mathbf{X}^P (\mathbf{y} - \frac{1}{n} \mathds{1}^\top \mathds{1} \mathbf{X}^S \boldsymbol{\beta}) 
\end{equation}
The MSE for the misaligned case can be expressed as:
\begin{align}
    \mathrm{MSE}_{\mathrm{misaligned}} 
    & = \inf_{\boldsymbol{\beta}} \inf_{\boldsymbol{\alpha}} \mathbb{E}_\mathbf{R} \|\mathbf{y} - \mathbf{X}^P \boldsymbol{\alpha} - \mathbf{R} \mathbf{X}^S \boldsymbol{\beta}\|_2^2 \\
    & = \inf_{\boldsymbol{\beta}} \mathbb{E}_\mathbf{R} \|\mathbf{y} - \mathbf{X}^P \hat{\boldsymbol{\alpha}} - \mathbf{R} \mathbf{X}^S \boldsymbol{\beta}\|_2^2 \\
\end{align}
Substituting $\hat{\boldsymbol{\alpha}}$ from equation~\ref{eq:alpha_hat}, we obtain:
\begin{align}
    \mathrm{MSE}_{\mathrm{misaligned}} 
    & = \inf_{\boldsymbol{\beta}} \mathbb{E}_\mathbf{R} \left\|\mathbf{y} - \mathbf{X}^P (\mathbf{X}^{P \top} \mathbf{X}^P)^{-1} (\mathbf{X}^P \mathbf{y} - \mathbf{X}^P \frac{1}{n} 1^\top 1 \mathbf{X}^S \boldsymbol{\beta}) - \mathbf{R} \mathbf{X}^S \boldsymbol{\beta}\right\|_2^2 \\
    & = \inf_{\boldsymbol{\beta}} \mathbb{E}_\mathbf{R} \left\| (\mathbf{I} - \mathbf{X}^P (\mathbf{X}^{P \top} \mathbf{X}^P)^{-1} \mathbf{X}^P)\mathbf{y} + (\mathbf{X}^P (\mathbf{X}^{P \top} \mathbf{X}^P)^{-1} \mathbf{X}^P \frac{1}{n} \mathds{1}^\top \mathds{1} \mathbf{X}^S \boldsymbol{\beta}) - \mathbf{R} \mathbf{X}^S \boldsymbol{\beta}\right\|_2^2 
\end{align}
Since $\mathbf{X}^P (\mathbf{X}^{P \top} \mathbf{X}^P)^{-1} \mathbf{X}^P$ is a projection matrix that projects any vector onto the column space of $\mathbf{X}^P$, and $\mathbf{X}^S \boldsymbol{\beta}$ is orthogonal to the column space of $\mathbf{X}^P$, the term $\mathbf{X}^P (\mathbf{X}^{P \top} \mathbf{X}^P)^{-1} \mathbf{X}^P \frac{1}{n} \mathds{1}^\top \mathds{1} \mathbf{X}^S \boldsymbol{\beta} = 0$. Thus:
\begin{align}
    \mathrm{MSE}_{\mathrm{misaligned}}
    & = \inf_{\boldsymbol{\beta}} \mathbb{E}_\mathbf{R} \left\| (\mathbf{I} - \mathbf{X}^P (\mathbf{X}^{P \top} \mathbf{X}^P)^{-1} \mathbf{X}^P)\mathbf{y} - \mathbf{R} \mathbf{X}^S \boldsymbol{\beta}\right\|_2^2 \\
    & = \inf_{\boldsymbol{\beta}} \mathbb{E}_\mathbf{R} \left[\left\|\mathbf{R} \mathbf{X}^S \boldsymbol{\beta}\right\|_2^2 - 2\left[(\mathbf{I} - \mathbf{X}^P (\mathbf{X}^{P \top} \mathbf{X}^P)^{-1} \mathbf{X}^P)\mathbf{y}\right]^\top \mathbf{R} \mathbf{X}^S \boldsymbol{\beta} + \left\|(\mathbf{I} - \mathbf{X}^P (\mathbf{X}^{P \top} \mathbf{X}^P)^{-1} \mathbf{X}^P)\mathbf{y}\right\|_2^2\right]
\end{align}
By properties of permutation matrices:
\begin{equation}
    \mathbb{E}_\mathbf{R}\| \mathbf{R} \mathbf{X}^S \boldsymbol{\beta}\|_2^2 = \|\mathbf{X}^S \boldsymbol{\beta}\|_2^2; \; \mathbb{E}_\mathbf{R} [\mathbf{R}]= \frac{1}{n}\mathds{1}^\top \mathds{1}
\end{equation}
Therefore:
\begin{align}
    \mathrm{MSE}_{\mathrm{misaligned}}
    & = \inf_{\boldsymbol{\beta}} \left[\left\|\mathbf{X}^S \boldsymbol{\beta}\right\|_2^2 - 2\left[(\mathbf{I} - \mathbf{X}^P (\mathbf{X}^{P \top} \mathbf{X}^P)^{-1} \mathbf{X}^P)\mathbf{y}\right]^\top \frac{1}{n}\mathds{1}^\top \mathds{1} \mathbf{X}^S \boldsymbol{\beta} + \left\|(\mathbf{I} - \mathbf{X}^P (\mathbf{X}^{P \top} \mathbf{X}^P)^{-1} \mathbf{X}^P)\mathbf{y}\right\|_2^2\right]
\end{align}
Since $\mathbf{I} - \mathbf{X}^P (\mathbf{X}^{P \top} \mathbf{X}^P)^{-1} \mathbf{X}^P$ projects any vector onto the orthogonal complement of the column space of $\mathbf{X}^P$, the term $\left[(\mathbf{I} - \mathbf{X}^P (\mathbf{X}^{P \top} \mathbf{X}^P)^{-1} \mathbf{X}^P)\mathbf{y}\right]^\top \frac{1}{n}\mathds{1}^\top \mathds{1} \mathbf{X}^S \boldsymbol{\beta} = 0$. Hence:
\begin{align}
    \mathrm{MSE}_{\mathrm{misaligned}}
    & = \inf_{\boldsymbol{\beta}} \left[\left\|\mathbf{X}^S \boldsymbol{\beta}\right\|_2^2 + \left\|(\mathbf{I} - \mathbf{X}^P (\mathbf{X}^{P \top} \mathbf{X}^P)^{-1} \mathbf{X}^P)\mathbf{y}\right\|_2^2\right] \\
    & = \inf_{\boldsymbol{\beta}} \left\|\mathbf{X}^S \boldsymbol{\beta}\right\|_2^2 + \left\|(\mathbf{I} - \mathbf{X}^P (\mathbf{X}^{P \top} \mathbf{X}^P)^{-1} \mathbf{X}^P)\mathbf{y}\right\|_2^2 \\
\end{align}
The minimum is attained at $\boldsymbol{\beta} = \mathbf{0}$, yielding:
\begin{align}
    \mathrm{MSE}_{\mathrm{misaligned}}
    & = \left\|(\mathbf{I} - \mathbf{X}^P (\mathbf{X}^{P \top} \mathbf{X}^P)^{-1} \mathbf{X}^P)\mathbf{y}\right\|_2^2 \\
    & = \inf_{\boldsymbol{\alpha} \in \mathbb{R}^{m^P}, \boldsymbol{\beta} = \mathbf{0}} \left\|\mathbf{y} - \mathbf{X}^P \boldsymbol{\alpha} - \mathbf{X}^S \boldsymbol{\beta}\right\|_2^2 \\
\end{align}
Comparing with Equation~\ref{eq:mse_aligned}, we conclude:
\begin{equation}
    \mathrm{MSE}_{\mathrm{misaligned}} \geq \inf_{\boldsymbol{\alpha} \in \mathbb{R}^{m^P}, \boldsymbol{\beta} \in \mathbb{R}^{m^S}} \left\|\mathbf{y} - \mathbf{X}^P \boldsymbol{\alpha} - \mathbf{X}^S \boldsymbol{\beta}\right\|_2^2 = \mathrm{MSE}_{\mathrm{aligned}}
\end{equation}
\end{proof}





















\subsection{Approximation Capacity of Cluster Sampler}\label{subsec:proof-cluster-sampler}

\begin{definition}[Definition of optimal cluster sampler]
    Assume the inputs are uniformly bounded by some constant $B$. 
    The optimal cluster sampler is defined by the uniform equi-continuous cluster sampler function which achieves the minimal optimization loss for the prediction task in \cref{fig:leal-framework}.
    \begin{equation}
        \textrm{Optimal cluster sampler} := \arginf_{\textrm{Uniform equi-continuous cluster sampler}} \textrm{Loss}(\textrm{cluster sampler})
    \end{equation}
    The cluster sampler is defined over bounded inputs ($|X^P|_{\infty} \leq B, |X^S|_{\infty} \leq B$) from $\mathbb{R}^{m^P} \times \mathbb{R}^{n^S \times m^S}$ and output in $\mathbb{R}^{n^S}$.
\end{definition}

\begin{remark}
    The existence of such optimal cluster sampler is guaranteed by the boundedness and uniform equi-continuity of the set of cluster sampler functions. 
\end{remark}


\thmclustersampler*

\begin{proof}
    We just need to prove the statement for small $\epsilon \leq 6$.

    The input of cluster sampler is $1 \times m^P$ and output is $n^S \times m^S$, the final prediction is to generate a sample probabilities:
    \begin{equation}
        (n^S * m^S, 1 * m^P) \to (n^S * d, 1 * C) \to (n^S * C, 1 * C) \to n^S * 1. 
    \end{equation}

    Also, since there is no weight depends on dimension $n_2$, we can reduce the approximation statement to that there exists trainable weight such that the continuous function $h$ can be approximated:
    \begin{equation}
        (1 * m^S, 1 * m^P) \to (n^S * d, 1 * C) \to (n^S * C, 1 * C) \to 1 * 1. 
    \end{equation}

    Notice that the layer operation of secondary embedding and trainable centroids weights $(C \times d)$ is continuous and the pretrained encoder as a neural network (which is a universal approximator) can approximates any continuous function $f$ composited with inverse embedding. 
    For simplicity, we will consider $m^P = m^S = 1$. 
    For any continuous function $h(p, s) \in [0, 1]$,
    we just need to show there exists trainable weight $\theta_1$, $\theta_2$ such that 
    \begin{equation}
        f(p; \theta_1) \odot g(s; \theta_2) = \sum_{i=1}^C f_i(p; \theta_1) \odot g_i(s; \theta_2). 
    \end{equation}
    Here $f(p; \theta_1) \in \mathbb{R}^C$ is a function of $p$ parameterized by $\theta_1$ and $g(s; \theta_1) \in \mathbb{R}^C$ is a function of $s$ parameterized by $\theta_2$.  
    As any continuous function $f(p, s)$ has a corresponding Taylor series expansion, it means for any $\epsilon > 0$, there exists $C$ which depends on error $\epsilon$ such that
    \begin{equation}
        \sup_p \sup_s |h(p, s) -\sum_{i=1}^C pol_{1,i}(p) pol_{2,i}(s)| \leq \frac{\epsilon}{2}. 
    \end{equation}
    Furthermore, as polynomial functions are continuous function, therefore $f_i$ can be used to approximate the polynomial function $pol_{1, i}$ and $g$ can be used to approximate the polynomial function $pol_{2, i}$.
    \begin{align}
        \sup_p |pol_{1,i}(p) - f_i(p; \theta_1)| & \leq \frac{\epsilon}{6B} \\ 
        \sup_s |pol_{2,i}(s) - g_i(s; \theta_2)| & \leq \frac{\epsilon}{6B}. 
    \end{align}
    Here $B := \max(1, \sup_p \max_{i} |pol_{1, i}(p)|, \sup_s \max_{i} |pol_{2, i}(s)|).$ 
    We show that the cluster sampler is capable to approximate any desirable continuous cluster sampler. 
    \begin{equation}
        \sup_p \sup_s |h(p, s) -\sum_{i=1}^C f_i(p; \theta_1) g_i(s; \theta_2)| \leq \frac{\epsilon}{2} + \frac{\epsilon}{6B} * B + \frac{\epsilon}{6B} (B + \frac{\epsilon}{6B}) = \frac{5}{6} \epsilon + \frac{\epsilon^2}{36B^2} < \epsilon. 
    \end{equation}
    The last inequality comes from $\epsilon < 6$. 
    The universal approximation capacity of the cluster sampler is proved. 
\end{proof}

\begin{remark}
    Since we are working with a cluster sampler with specific manually designed structure, it mainly comes from the fact the student's t-kernel introduce a suitable implicit bias to more efficiently learn the cluster sample probability $(n_2 \times 1)$. 
\end{remark}

\section{Experimental Details}
\label{sec:exp-details}

\subsection{Datasets}
\label{appendix:data}
\paragraph{Synthetic Dataset} To better reflect real-world conditions, regions are connected based on a predefined topology. We randomly generate 5,100 graphs, each with 30 nodes and 20 edges. The first 4,800 graphs are used for training, the next 200 for validation, and the remaining 100 for testing. Each node is assigned a randomly generated 10-dimensional feature vector. Next, we establish a stochastic mapping from a node’s features to its poaching count, capturing the complex relationships observed in real-world scenarios. Poaching counts are sampled from a Gamma distribution parameterized by shape and scale values.  We randomly initialize two Graph Convolutional Networks (GCNs). For each node, one of the two GCNs is selected with equal probability to map the node’s features to a continuous value, which is then scaled by a factor of 20. This value serves as the shape parameter of the Gamma distribution. The poaching count is then drawn from the Gamma distribution, where the scale parameter is set to 1 if the first GCN is chosen and 0.9 if the second is chosen.  To incorporate adversarial noise, we apply perturbations inversely proportional to the poaching count—nodes with lower poaching counts receive higher noise levels. Finally, the poaching count for each node is capped within the range \([0,40]\) and scaled by 0.2 to align the overall distribution with real-world data.


\paragraph{Real-world Dataset}
We use poaching data from Murchison Falls National Park (MFNP) in Uganda, collected between 2010 and 2021. The protected area is discretized into 1 × 1 km grid cells. For each cell, we measure ranger patrol effort (in kilometers patrolled) as the conditional variable for the diffusion model, while the monthly number of detected illegal activity instances of each cell serves as the adversarial behavior. Following~\citep{basak2016abstraction}, we represent the park as a graph to capture geospatial connectivity among these cells. To focus on high-risk regions, we subsample 20 subgraphs from the entire graph. Specifically, at each time step we identify the 20 cells with the highest poaching counts. Each of these cells is treated as a central node, and we iteratively add the neighboring cell with the highest poaching count until the subgraph reaches 20 nodes. This procedure yields 532 training samples, 62 validation samples, and 31 test samples.


\subsection{Implementation details}
\label{appendix:details}
We use a three-layer Graph Convolutional Network (GCN)~\citep{kipf2022semi} with a hidden dimension of 128 as the backbone of the diffusion model. The diffusion process follows the DDPM framework~\citep{ho2020denoising} with \( T = 1000 \) time steps and a variance schedule from \( 10^{-4} \) to \( 0.02 \). Optimization is performed using Adam~\citep{kingma2014adam} with a learning rate of \( 10^{-3} \), and the model is trained for 5000 epochs. To estimate the expected utility, we draw 500 samples from the diffusion model. All comparison methods run for 30 iterations. The mirror ascent oracle uses a step size of 0.1 and runs for 100 iterations. The step size in the mirror ascent step for the baselines is also 0.1.





The actions of the poacher and ranger in grid $j$, represented by $z_j$ and $x_j$ respectively, influence the wildlife population in the area. We model the wildlife population in grid $j$ as follows: 
$$\max(N_0(j)e^r-\alpha e^{\psi \mathbf{z}_j - \theta \mathbf{x}_j}, 0),$$
where $N_0(j)$ is the initial wildlife population in the area and $r$ denotes the natural growth rate of the wildlife. The parameter $\alpha$ captures the impact of both the ranger’s and poacher’s actions on the wildlife population, $\psi$ reflects the strength of poaching, and $\theta$ measures the effectiveness of patrol effort. The utility for the ranger is then represented as the sum of wildlife population across all grids:
$$U(\mathbf{x}, \mathbf{z}) = \sum_{j=1}^K \max(N_0(j)e^r-\alpha e^{\psi \mathbf{z}_j - \theta \mathbf{x}_j}, 0)$$.


\paragraph{Forecasting Experiments.} 
We use the poaching dataset described in Appendix~\ref{appendix:data}. 
Following~\citet{pmlr-v161-xu21a}, linear regression and Gaussian processes predict the poaching count for each $1\times 1$ km cell individually, using two features: the previous month's patrol effort in the current cell and the aggregated patrol effort from neighboring cells.
For linear regression, we employ the scikit-learn implementation, while for Gaussian processes, we use the GPy library with both the RBF and Matérn kernels. The training procedure for the diffusion model follows Appendix~\ref{appendix:details}, with its support constrained to $[0,3]$. For each test instance, we generate 500 samples and use the mean prediction.  We also attempted to impose constraints on the baseline's output but found that this only degraded its performance.








\end{document}