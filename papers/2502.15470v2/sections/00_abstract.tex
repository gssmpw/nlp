\begin{abstract}
Large language models (LLMs) are widely used for natural language understanding and text generation. An LLM model relies on a time-consuming step called LLM decoding to generate output tokens. Several prior works focus on improving the performance of LLM decoding using parallelism techniques, such as batching and speculative decoding. 
State-of-the-art LLM decoding has both compute-bound and memory-bound kernels. Some prior works \emph{statically identify} and map these different kernels to a heterogeneous architecture consisting of both processing-in-memory (PIM) units and computation-centric accelerators (e.g., GPUs). We observe that characteristics of LLM decoding kernels (e.g., whether or not a kernel is memory-bound) can change \emph{dynamically} due to parameter changes to meet user and/or system demands, making (1) \emph{static} kernel mapping to PIM units and computation-centric accelerators suboptimal, and (2) one-size-fits-all approach of designing PIM units inefficient due to a large degree of heterogeneity even in memory-bound kernels.

In this paper, we aim to accelerate LLM decoding while considering the dynamically changing characteristics of the kernels involved. We propose PAPI (\textbf{PA}rallel Decoding with \textbf{PI}M), a PIM-enabled heterogeneous architecture that exploits dynamic scheduling of compute-bound or memory-bound kernels to suitable hardware units. PAPI has two key mechanisms: 
(1) \emph{online kernel characterization} to dynamically schedule kernels to the most suitable hardware units at runtime and (2) a PIM-enabled heterogeneous computing system that harmoniously orchestrates both computation-centric processing units (GPU) and hybrid PIM units with different computing capabilities. Our experimental results on three broadly-used LLMs (i.e., LLaMA-65B, GPT-3 66B, and GPT-3 175B) show that PAPI achieves 1.8$\times$ and 11.1$\times$ speedups over a state-of-the-art heterogeneous LLM accelerator (i.e., GPU and PIM) and a state-of-the-art PIM-only LLM accelerator, respectively.
\end{abstract}
