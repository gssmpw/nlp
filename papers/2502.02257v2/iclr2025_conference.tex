
\documentclass{article} % For LaTeX2e
\usepackage[dvipsnames, table, xcdraw]{xcolor}
\usepackage{iclr2025_conference,times}

% Optional math commands from https://github.com/goodfeli/dlbook_notation.
% %%%%% NEW MATH DEFINITIONS %%%%%

% \usepackage{amsmath,amsfonts,bm}
\usepackage{amsmath,amsfonts}

\usepackage{pifont}


\newcommand{\R}{\mathbb{R}}


\def\va{{\mathbf{a}}}
\def\vg{{\mathbf{g}}}

% Sets
\def\sR{\mathbb{R}}
\def\sC{\mathbb{C}}
\def\sZ{\mathbb{Z}}
\def\sN{\mathbb{N}}
\def\sQ{\mathbb{Q}}

\def\sS{\mathcal{S}}



% Vectors
\def\vzero{{\mathbf{0}}}
\def\vone{{\mathbf{1}}}
\def\vmu{{\mathbf{\mu}}}
\def\vtheta{{\mathbf{\theta}}}
\def\va{{\mathbf{a}}}
\def\vb{{\mathbf{b}}}
\def\vc{{\mathbf{c}}}
\def\vd{{\mathbf{d}}}
\def\ve{{\mathbf{e}}}
\def\vf{{\mathbf{f}}}
\def\vg{{\mathbf{g}}}
\def\vh{{\mathbf{h}}}
\def\vi{{\mathbf{i}}}
\def\vj{{\mathbf{j}}}
\def\vk{{\mathbf{k}}}
\def\vl{{\mathbf{l}}}
\def\vm{{\mathbf{m}}}
\def\vn{{\mathbf{n}}}
\def\vo{{\mathbf{o}}}
\def\vp{{\mathbf{p}}}
\def\vq{{\mathbf{q}}}
\def\vr{{\mathbf{r}}}
\def\vs{{\mathbf{s}}}
\def\vt{{\mathbf{t}}}
\def\vu{{\mathbf{u}}}
\def\vv{{\mathbf{v}}}
\def\vw{{\mathbf{w}}}
\def\vx{{\mathbf{x}}}
\def\vy{{\mathbf{y}}}
\def\vz{{\mathbf{z}}}
\def\vzeta{{\mathbf{\zeta}}}

% Matrix
\def\mA{{\mathbf{A}}}
\def\mB{{\mathbf{B}}}
\def\mC{{\mathbf{C}}}
\def\mD{{\mathbf{D}}}
\def\mE{{\mathbf{E}}}
\def\mF{{\mathbf{F}}}
\def\mG{{\mathbf{G}}}
\def\mH{{\mathbf{H}}}
\def\mI{{\mathbf{I}}}
\def\mJ{{\mathbf{J}}}
\def\mK{{\mathbf{K}}}
\def\mL{{\mathbf{L}}}
\def\mM{{\mathbf{M}}}
\def\mN{{\mathbf{N}}}
\def\mO{{\mathbf{O}}}
\def\mP{{\mathbf{P}}}
\def\mQ{{\mathbf{Q}}}
\def\mR{{\mathbf{R}}}
\def\mS{{\mathbf{S}}}
\def\mT{{\mathbf{T}}}
\def\mU{{\mathbf{U}}}
\def\mV{{\mathbf{V}}}
\def\mW{{\mathbf{W}}}
\def\mX{{\mathbf{X}}}
\def\mY{{\mathbf{Y}}}
\def\mZ{{\mathbf{Z}}}
\def\mBeta{{\mathbf{\beta}}}
\def\mPhi{{\mathbf{\Phi}}}
\def\mLambda{{\mathbf{\Lambda}}}
\def\mSigma{{\mathbf{\Sigma}}}


% Expectation
% \def\eE{\mathop{\mathbb{E}}\limits}
\def\eE{\mathbb{E}}

% Probability
\def\pP{\mathbb{P}}

% Tilde
\def\tf{\tilde{f}}
\def\tS{\tilde{S}}
\def\wtF{\widetilde{\mathcal{F}}}
\def\whR{\widehat{R}}
\def\tvx{\tilde{\mathbf{x}}}
\def\ty{\tilde{y}}


\def\defeq{\overset{\textup{def}}{=}}
% \def\defeq{\overset{.}{=}}
\def\defone{\overset{\text{\ding{172}}}{=}}
\def\deftwo{\overset{\text{\ding{173}}}{=}}
\def\leqone{\overset{\text{\ding{172}}}{\leq}}
\def\leqtwo{\overset{\text{\ding{173}}}{\leq}}
\def\leqthree{\overset{\text{\ding{174}}}{\leq}}
\def\leqfour{\overset{\text{\ding{175}}}{\leq}}
\def\eqone{\overset{\text{\ding{172}}}{=}}
\def\eqtwo{\overset{\text{\ding{173}}}{=}}
\def\eqthree{\overset{\text{\ding{174}}}{=}}
\def\eqfour{\overset{\text{\ding{175}}}{=}}
\def\geqfive{\overset{\text{\ding{176}}}{\geq}}

% \usepackage{hyperref}
\usepackage[hidelinks, breaklinks, colorlinks, linkcolor=Red, citecolor=RoyalBlue]{hyperref}
\usepackage{url}

\usepackage{multirow}
\usepackage{makecell}
\usepackage{booktabs}
\usepackage{graphicx}
\usepackage{algpseudocode}
\usepackage{eqparbox}
\usepackage{algorithmicx}
\usepackage{algorithm}
\usepackage{array}
\usepackage{pifont}
\usepackage{wrapfig}
\usepackage{amsmath}
\usepackage{amsfonts}
\usepackage{colortbl}
\usepackage{fontenc}
\usepackage{cleveref}
\usepackage{tikz}
\usepackage{xspace}
%\usepackage{caption}

\newcommand{\figref}[1]{Fig.~\ref{#1}}
\newcommand{\tabref}[1]{Tab.~\ref{#1}}
\newcommand{\Figref}[1]{Figure~\ref{#1}}
\newcommand{\Tabref}[1]{Table~\ref{#1}}
\newcommand{\eqnref}[1]{Eq.~(\ref{#1})}
\newcommand{\secref}[1]{Sec.~\ref{#1}}
\newcommand{\Secref}[1]{Section~\ref{#1}}
\newcommand{\Appref}[1]{Appendix~\ref{#1}}

\definecolor{baselinecolor}{rgb}{0.9, 0.9, 1.}
\definecolor{graycolor}{gray}{0.9}
\definecolor{Green}{rgb}{0.0, 0.5, 0.0}
\definecolor{Green}{rgb}{0.0, 0.5, 0.0}
\definecolor{rebuttal}{rgb}{0.6, 0.6, 1.}
\newcommand{\baseline}[1]{\cellcolor{baselinecolor}{#1}}
\newcommand{\graybase}[1]{\cellcolor{graycolor}{#1}}


\newcommand{\plus}[1]{\scriptsize\bf\textcolor{Green}{#1}}
\newcommand{\tinyplus}[1]{\tiny\bf\textcolor{Green}{#1}}

\newcommand{\cmark}{\ding{51}}%
\newcommand{\xmark}{\ding{55}}%

\makeatletter
\DeclareRobustCommand\onedot{\futurelet\@let@token\@onedot}
\def\@onedot{\ifx\@let@token.\else.\null\fi\xspace}
\def\eg{\emph{e.g}\onedot} \def\Eg{\emph{E.g}\onedot}
\def\ie{\emph{i.e}\onedot} \def\Ie{\emph{I.e}\onedot}
\def\cf{\emph{c.f}\onedot} \def\Cf{\emph{C.f}\onedot}
\def\etc{\emph{etc}\onedot} \def\vs{\emph{vs}\onedot}
\def\wrt{w.r.t\onedot} \def\dof{d.o.f\onedot}
\def\etal{\emph{et al}\onedot}
\makeatother

% \title{Formatting Instructions for ICLR 2025 \\ Conference Submissions}
\title{UNIP: Rethinking Pre-trained Attention Patterns for Infrared Semantic Segmentation}
% Authors must not appear in the submitted version. They should be hidden
% as long as the \iclrfinalcopy macro remains commented out below.
% Non-anonymous submissions will be rejected without review.

% \author{Antiquus S.~Hippocampus, Natalia Cerebro \& Amelie P. Amygdale \thanks{ Use footnote for providing further information
% about author (webpage, alternative address)---\emph{not} for acknowledging
% funding agencies.  Funding acknowledgements go at the end of the paper.} \\
% Department of Computer Science\\
% Cranberry-Lemon University\\
% Pittsburgh, PA 15213, USA \\
% \texttt{\{hippo,brain,jen\}@cs.cranberry-lemon.edu} \\
% \And
% Ji Q. Ren \& Yevgeny LeNet \\
% Department of Computational Neuroscience \\
% University of the Witwatersrand \\
% Joburg, South Africa \\
% \texttt{\{robot,net\}@wits.ac.za} \\
% \AND
% Coauthor \\
% Affiliation \\
% Address \\
% \texttt{email}
% }

\author{Tao Zhang\textsuperscript{1,2} \ \
Jinyong Wen\textsuperscript{1,2} \ \
Zhen Chen\textsuperscript{3} \ \
Kun Ding\textsuperscript{1}\thanks{Corresponding author} \ \
Shiming Xiang\textsuperscript{1,2} \ \
Chunhong Pan\textsuperscript{1} \\
\textsuperscript{1}MAIS, Institute of Automation, Chinese Academy of Sciences, China \\
\textsuperscript{2}School of Artificial Intelligence, University of Chinese Academy of Sciences, China \\
\textsuperscript{3}School of Automation Science and Electrical Engineering, Beihang University, China \\ 
\texttt{\{zhangtao2021,wenjinyong2019,kun.ding\}@ia.ac.cn;} \\ \texttt{\{czhen\}@buaa.edu.cn;\{smxiang, chpan\}@nlpr.ia.ac.cn}\\
}

% The \author macro works with any number of authors. There are two commands
% used to separate the names and addresses of multiple authors: \And and \AND.
%
% Using \And between authors leaves it to \LaTeX{} to determine where to break
% the lines. Using \AND forces a linebreak at that point. So, if \LaTeX{}
% puts 3 of 4 authors names on the first line, and the last on the second
% line, try using \AND instead of \And before the third author name.

\newcommand{\fix}{\marginpar{FIX}}
\newcommand{\new}{\marginpar{NEW}}


\iclrfinalcopy % Uncomment for camera-ready version, but NOT for submission.
\begin{document}


\maketitle


\begin{abstract}
Pre-training techniques significantly enhance the performance of semantic segmentation tasks with limited training data. However, the efficacy under a large domain gap between pre-training (\eg RGB) and fine-tuning (\eg infrared) remains underexplored. In this study, we first benchmark the infrared semantic segmentation performance of various pre-training methods and reveal several phenomena distinct from the RGB domain. Next, our layerwise analysis of pre-trained attention maps uncovers that: (1) There are three typical attention patterns (local, hybrid, and global); (2) Pre-training tasks notably influence the pattern distribution across layers; (3) The hybrid pattern is crucial for semantic segmentation as it attends to both nearby and foreground elements; (4) The texture bias impedes model generalization in infrared tasks. Building on these insights, we propose \textbf{UNIP}, a \textbf{UN}ified \textbf{I}nfrared \textbf{P}re-training framework, to enhance the pre-trained model performance. This framework uses the hybrid-attention distillation NMI-HAD as the pre-training target, a large-scale mixed dataset InfMix for pre-training, and a last-layer feature pyramid network LL-FPN for fine-tuning. Experimental results show that UNIP outperforms various pre-training methods by up to \textbf{13.5\%} in average mIoU on three infrared segmentation tasks, evaluated using fine-tuning and linear probing metrics. UNIP-S\footnote{We use the term \textit{method-size} to denote the vision transformer (ViT) of a specific \textit{size} pre-trained by a specific \textit{method}. T, S, B, and L refer to the ViT-Tiny, ViT-Small, ViT-Base, and ViT-Large, respectively.} achieves performance on par with MAE-L while requiring only \textbf{1/10} of the computational cost. Furthermore, UNIP significantly surpasses state-of-the-art (SOTA) infrared or RGB segmentation methods and demonstrates broad potential for application in other modalities, such as RGB and depth. Our code is available at \href{https://github.com/casiatao/UNIP}{https://github.com/casiatao/UNIP}.

\end{abstract}

\documentclass[../main.tex]{subfiles}
\graphicspath{{../images/}}
\makeatletter
\def\input@path{{../images/}}
\makeatother
\begin{document}
\section{Introduction}
\begin{figure}
\centering
\begin{tikzpicture}
\node[inner sep=0pt] (ws) at (0, 0) {
\includegraphics[height=.4\textwidth, trim={10cm 0 10cm 0},clip]{world_space.png}};
\node[inner sep=0pt] (cs) at (6,0) {\includegraphics[height=.4\textwidth, trim={10cm 1cm 10cm 4cm},clip]{conf_space.png}};
\end{tikzpicture}
\vspace{-5pt}
\label{fig:pbrm_intro}
\caption{\textbf{Left}: Shows world space obstacles as grey spheres. Robots start and goal configuration is colored red and green, respectively. Configurations along the computed path are colored transparent blue. \textbf{Right:} Mapped world space scenario to configuration space. Obstacle region is the grey mesh. Red spheres are collision-free regions computed by the neural SCDF. The optimized shortest path in the convex corridor is the blue curve.}
\vspace{-25pt}
\end{figure}
Motion planning is the problem of finding a collision-free trajectory that connects a given start and goal configuration. The planning takes place in the configuration space of the robot. For single body robots, like mobile robots or drones, the configuration space and the world space are usually the same. This simplifies the planning, since explicit obstacle representations are available which enables geometrical tools like separating hyperplanes, smallest distance to obstacles etc., to be used when designing motion planning algorithms. For multi-body robots like manipulators, the situation is completely different. The world space obstacles are usually mapped to non-convex regions, and to make the problem even harder, the mapping is usually not known. Forming explicit representations of the obstacle region in the configuration space is usually too expensive or intractable. Despite all of this, sampling based planners are used with great success, which mainly is due to their use of implicit representations of the obstacle region. The basic idea is to construct a graph in the configuration space that covers and connects the collision-free region. From this graph, a path can be extracted that connects a given start and goal configuration. The approach is computationally expensive, since the graph is constructed with the smallest geometrical building block available, points, which represents a collision-check. Furthermore, the extracted paths from the graph are non-smooth and jagged due to the stochastic nature of the approach. This adds an additional post-processing step to the process, where the paths are shortcutted and smoothened, before the path can be used for tracking. Clearly a lot of time is invested to form this graph and produce smooth paths. Thus, if the obstacles start to move, then all of this work is done in no use, since all points that make up this graph need to be re-verified, which is simply too time consuming to be done in real time.
\\\\
In this work, we want to address the existing drawbacks of the sampling based planners. Our main contribution is an improved motion planner where each vertex in the graph covers a collision-free region in the form of a sphere instead of a point and where the edges are formed with neighboring intersecting spheres. This representation has the advantage of instead of returning piecewise linear paths, returning a sequence of overlapping spheres, i.e. a convex corridor, that connects a given start and goal configuration, illustrated in Figure \ref{fig:pbrm_intro}. This convex corridor allows us to use convex optimization to produce smooth trajectories, instead of computationally expensive post-processing methods. The representation further allows us to estimate the coverage of the collision-free space, which gives us awareness and feedback in the offline roadmap construction phase. Finally, our representation is simple to adapt to moving obstacles, simply requery for the new radii and recheck for intersections. 
\\\\
The spherical collision-free regions are formed using a signed distance function (SDF), which is a function that returns the smallest distance from an arbitrary point to the boundary of an obstacle. As the name implies, the distance is signed, thus if the point is inside the obstacle it is negative otherwise positive. If the distance is positive, a sphere with radius equal to the distance is guaranteed to cover a collision-free region. Using an SDF in motion planning is not new, but what is novel about our approach is that we express the distance in the configuration space instead of the world space and by doing so allows us to form these convex collision-free regions. We refer to the resulting SDF as a signed configuration distance function (SCDF). Computing an SCDF analytically is non-trivial, our approach is therefore to parameterize the SCDF with a deep neural network and learn the mapping by supervised learning. Our resulting neural SCDF can compute distances for different parameter values of obstacle shapes and we also show how multiple distances can be combined, thus making our approach flexible.
\section{Related work}
Motion planning algorithms can roughly be divided into three families, grid-based, sampling based and optimization based methods. Grid-based methods (GBM) discretize the planning space from which a graph is then compiled. A standard search method is A$^\star$ \citep{a_star}, which is classified as an \textit{informed} search method, since it employs a heuristic function to speed up the search. A$^\star$ guarantees to return an optimal path at the level of discretization used. GBMs usually discretize the planning space by a regular lattice and this limits the GBMs to problems with low dimensionality due to the curse of dimensionality. Thus, GBMs are usually limited to single-body robots where the degrees of freedom (DOF) are low. To overcome the inherent scaling problem with the GBMs, stochastic methods are usually used for multi-body robots. These methods are termed as sampling-based methods (SBM) and core members within this family are the rapidly-exploring random trees (RRT) \citep{rrt} and the probabilistic roadmap (PRM) \citep{prm}. RRT grows a tree from the start configuration and explores the collision-free region in a rapid way until it is able to connect to the goal region. RRT is usually improved by bi-directional planning \citep{rrt_connect}, i.e. an additional tree is grown from the goal configuration and the trees are tested for connection after any tree has been expanded. RRT is a single-query method, thus it searches for a path from scratch each time it is queried. Contrary to this, PRM is a multi-query method, which solves for multiple queries without starting from scratch. PRM does this by creating a roadmap (graph) that covers the collision-free space as an offline step. The graph is then used to solve for multiple queries. PRMs are used in cases where the environment does not change since the extra offline step is too computationally costly and needs to be re-done if the environment is changed. In our work, we address this inherent issue by using a different roadmap representation. Our vertices in the graph cover a collision-free region in the form of spheres and we form the edges by checking for intersecting spheres. If something in the environment changes, we recompute the spheres radii and recheck the intersections, without relying on collision detection. We use a trained neural network to compute the sphere radius, therefore querying for the radius can be done fast, hence our representation enables the PRM for dynamic environments.
\\\\
In the recent decades, optimization based methods (OBM) \citep{chomp, schulman, itomp, stomp} have been introduced as an alternative to SBM for multi-body robots. Like the SBM, the OBMs scale well to higher dimensional problems and produce smoother motion. It is common to use a SDF in the optimization since it is a smooth function, thus enabling gradient-based methods. However, the standard way of expressing the SDF is in world space. The distance therefore needs to be mapped to the configuration space by the forward kinematics. This mapping makes the optimization problem a non-linear program (NLP), which is computationally expensive to solve. Recently, a different approach has been proposed. In \cite{mp_gcs} motion planning is formulated as a convex optimization problem by using the graph of convex sets framework \citep{gcs}. The underlying idea is to decompose the collision-free space into intersecting convex sets from which a convex optimization problem is formulated. In cases where an explicit representation of the obstacles in the configuration space exists, like for single-body robots, creating collision-free convex regions can be done fast \citep{iris}. For multi-body robots, this is non-trivial. Existing work does this successfully \citep{iris_nlp, iris_c} by an optimization based approach, but the methods are still too time consuming to be used in the presence of moving obstacles. Our approach is instead to use deep learning to learn an SDF expressed in the configuration space. With this, we can query for shortest distances to the collision boundary, which allows us to expand spherical regions which are collision-free. Our approach is fast and therefore enables our suggested roadmap planner to be used in dynamic environments.
\\\\
Recent research has focused on learning collision detection \citep{fk_kernel_distance, diffco, graphdistnet} by predicting the signed distance between the robot links and the surrounding obstacles in the world space. The learned SDF is used in trajectory optimization but since the distance is expressed in the world space, the problem becomes an NLP and therefore takes a long time to solve. We take a novel approach and suggest to instead express the signed distance in the configuration space. This allows us to improve the PRM at the same time as it enables convex optimization for trajectory optimization, which runs faster and is more reliable than NLP solvers. In \cite{cspf} a learned signed distance function in the configuration space is proposed similar to our approach. However, their approach is restricted to point cloud representations, while we propose to represent the obstacles as parameterized geometric shapes, e.g. spheres. Furthermore, we also show how to use our learned SCDF to improve an existing roadmap planner.
\section{Problem formulation}
A robot is located in the world space, $\W \subset \R^3 $. The unique location of the robot is given by its configuration $\q \in \C$, where $\C$ is the configuration space. The set of points covered by the robots bodies at a certain configuration is expressed as $\B(\q) \subset \W$. The robot is surrounded by $\NrObst$ obstacles $\O = \bigcup_{i=1}^{\NrObst} \O_i$, where  $\O_i \subset \W$. The representation of the obstacle in the configuration space is the set $\C\O_i = \{\q \in \C \: |\: \B(\q) \cap \O_i \neq \emptyset \}$. The obstacle space is formed as $\Co = \bigcup_{i=1}^{\NrObst} \C \O_i$. The complement is referred to as the free space, $\Cf = \C \setminus \Co$. The path planning problem is a tuple, ($\Cf$, $\qStart$, $\qGoal$), where we want to connect a query pair, consisting of a start, $\qStart$, and goal configuration, $\qGoal$, with a geometric path, $\q(s): [0, 1] \mapsto \Cf$, such that $\q(0)=\qStart$ and $\q(1)=\qGoal$, or report correctly when such a path does not exist.
\end{document}

\begin{figure}[t]
    \centering
    \begin{tikzpicture}
       \node at (0,0) {
        \includegraphics[width=\linewidth]{figures/benchmark_v6.pdf}
      };
      % Draw the table
      \node[anchor=south west, xshift=-5.4cm, yshift=-1.85cm] at (current page.south west) {
        \scriptsize
        \setlength{\tabcolsep}{0.8mm}{
        \scalebox{0.9}{
        \begin{tabular}{lcc}
          \hline
          Model & Param & FLOPs \\
          \hline
          ViT-T & 11.0M & 36.7G \\
          ViT-S & 41.9M & 146.2G \\
          ViT-B & 163.7M & 584.0G \\
          ViT-L & 441.3M & 1192.6G \\
          \hline
        \end{tabular}}}
      };
    \end{tikzpicture}
    \vspace{-20pt}
    \caption{The performance of pre-trained models across various methods and sizes. \textit{Left}: The average fine-tuning (FT) performance on three infrared semantic segmentation datasets, along with the associated computational cost. \textit{Middle}: The average linear probing (LP) performance on three infrared datasets. \textit{Right}: The fine-tuning performance on ImageNet \citep{imagenet}. The \textcolor{gray}{gray} dotted lines and corresponding values highlight the performance gains of UNIP over other methods. Detailed results for each dataset are presented in \tabref{tab:benchmark}.}
    \label{fig:benchmark}
    \vspace{-4pt}
\end{figure}

\section{How do pre-training methods perform on infrared tasks?}
\label{sec:benchmark}
In this section, we benchmark six pre-training methods on three infrared semantic segmentation datasets and discuss several key phenomena.

\subsection{Infrared Segmentation Benchmark of RGB Pre-trained Models}
\label{sec:benchmark_setup}
\textbf{Pre-trained Backbone.} Pre-training of the Vision Transformer (ViT) \citep{vit} has gained widespread attention and demonstrated powerful performance in various fields. Many recent pre-training methods \citep{deit,mae,iBOT,dinov2} use ViT for experiments, making pre-trained ViT models readily available. Therefore, ViT models of various sizes are set as the evaluation backbone.

\textbf{Pre-training Methods.} Both supervised and self-supervised methods are investigated. For supervised approaches, we use \textbf{DeiT} \citep{deit} and \textbf{DeiT III} \citep{deit3}, which perform image classification on ImageNet for pre-training. In self-supervised methods, we study contrastive learning (CL) and masked image modeling (MIM). CL methods like \textbf{DINO} \citep{dino} encourage features from different views of the same image to be close, while keeping features from different images distinct. MIM methods like \textbf{MAE} \citep{mae} and \textbf{CrossMAE} \citep{crossmae} focus on reconstructing masked image patches by learning context relations. Although \textbf{iBOT} \citep{iBOT} combines CL with masked feature prediction, we classify it as a CL method due to its similar characteristics to DINO. \textbf{The above methods are selected because they all pre-train vanilla ViT models on ImageNet}, without additional pre-trained tokenizers like BeiT \citep{beit} or MILAN \citep{milan}, or larger datasets like EVA \citep{eva} and DINOv2 \citep{dinov2}. This allows us to focus on the impact of the pre-training tasks alone.

\textbf{Evaluation Datasets.} The evaluation is conducted on three infrared semantic segmentation datasets: SODA \citep{soda}, MFNet-T \citep{mfnet}, and SCUT-Seg \citep{mcnet}. Notably, MFNet is an RGB-Thermal paired dataset. The thermal part MFNet-T is used for benchmarks while the RGB part MFNet-RGB is employed in further investigations. Additionally, RGB datasets like ImageNet-1K \citep{imagenet} and ADE20K \citep{ade20k} are also used for comparison. Details about these datasets can be found in \Appref{app:evaluation_datasets}.

\textbf{Evaluation Metrics.} We employ two metrics: \textit{fine-tuning} (FT) and \textit{linear probing} (LP). FT (\figref{fig:ft_lp}\textcolor{red}{a}) is the primary metric, where both the pre-trained model and the decoder are tuned with the labeled datasets. In LP (\figref{fig:ft_lp}\textcolor{red}{b}), only a linear head is updated while all other parameters remain frozen. \textbf{Average (Avg) FT or LP performance in subsequent sections denotes the mean mIoU across three infrared semantic segmentation datasets.} More details are available in \Appref{app:appendix_evaluation}.

\textbf{Benchmark Results.} In the benchmark, all models are trained for 100 epochs for both evaluation metrics. Typical results are illustrated in \figref{fig:benchmark}, with ImageNet \citep{imagenet} fine-tuning performance included for comparison. The complete results of each dataset are detailed in \tabref{tab:benchmark}. 



\subsection{What insights can we gain from this benchmark?}
\label{sec:benchmark_multi_layer}

\begin{wraptable}{r}{0.36\textwidth}
    \vspace{-7.5mm}
    \caption{Pearson coefficients between average FT and other metrics.}
    \label{tab:correlation_coeff}
    \vspace{1mm}
    \centering
    \scriptsize
    \setlength{\tabcolsep}{0.8mm}{
    \scalebox{1.0}{
    \begin{tabular}{l c c c c}
        \toprule
        Metric & Small & Base & Large & Mean \\
        \midrule
        Avg FT \& Avg LP & \textbf{0.89} & \textbf{0.93} & \textbf{0.81} & \textbf{0.88}  \\
        Avg FT \& IN1K FT & 0.78 & 0.12 & -0.66 & 0.08 \\
        \bottomrule
    \end{tabular}}}
    \vspace{-4mm}
\end{wraptable}

\textbf{The infrared FT performance is strongly positively correlated with LP, but has no clear relationship with ImageNet FT.} \tabref{tab:correlation_coeff} presents the Pearson \citep{pearson} correlation coefficients between different metrics. For each metric pair, the coefficients are calculated across six pre-training methods in \figref{fig:benchmark}. Notably, the coefficients between average infrared LP and FT are close to 1, indicating that models with better LP performance generally exhibit better FT performance. Conversely, the ImageNet FT performance does not consistently correlate with infrared FT results across various model sizes, likely due to domain and task differences. Therefore, using ImageNet accuracy to predict transfer performance on infrared segmentation datasets is not reliable, underscoring the importance of benchmarking on infrared segmentation datasets.

\textbf{Supervised and CL methods outperform MIM methods, especially for small models.} As depicted in \figref{fig:benchmark} and \tabref{tab:benchmark}, the performance of supervised and CL methods like DeiT, DeiT III, DINO, and iBOT is similar across both metrics, except for the LP of DeiT-S. For the LP metric, MIM methods of various sizes consistently lag behind supervised and CL methods by a significant margin, matching observations in the RGB domain \citep{mae} that MIM representations are less linearly separable. In terms of FT, smaller MIM models (ViT-T, S, and B) still underperform supervised and CL methods, while larger models (ViT-L) are more competitive. For instance, MAE-S is far behind iBOT-S (55.39\% vs 62.09\%), but MAE-L performs comparably to iBOT-L (64.35\% vs 64.97\%). As we will discuss in \secref{sec:investigation}, the discrepancy in the attention pattern distribution and texture bias between different models accounts for their distinct infrared segmentation performance.

\textbf{Larger models perform better, but their computational cost increases sharply.}
As illustrated in \figref{fig:benchmark}, larger MIM models bring considerable performance gains over smaller models. However, for supervised and CL methods, small models are already well-trained, and the performance improvement from larger models is marginal compared to the significant increase in computational cost. For example, iBOT-L surpasses iBOT-S by only 2.85\% (64.97\% vs 62.09\%), while the parameter count and FLOPs increase by 10.5$\times$ (441M vs 42M) and 8.2$\times$ (1193G vs 146G), respectively. Given that infrared images are often processed on edge devices with limited computing budgets, using large models to pursue better performance is not cost-effective. Therefore, we believe improving small models is a more effective approach. In \secref{sec:distill}, we propose several strategies to elevate the performance of small models to be on par with larger models.

\section{What matters for infrared semantic segmentation?}
\label{sec:investigation}
To determine which characteristics of the pre-trained models are critical for infrared semantic segmentation, we analyze different models from multiple perspectives.

\subsection{The Pre-training tasks influence attention patterns}
\label{sec:attention_pattern}
The self-attention mechanism is a key component of ViT. In semantic segmentation tasks, the spatial interactions between tokens are crucial. Thus, we visualize the attention maps of pre-trained models. 

\textbf{Attention maps of supervised/CL and MIM methods differ significantly.} As shown in \figref{fig:query_attn}\textcolor{red}{a}, DINO-S exhibits three distinct attention patterns: (1) \textit{Local}: In shallow layers (Layer 5), different query tokens focus only on their spatially nearby key tokens; (2) \textit{Hybrid}: In middle layers (Layer 9), query tokens attend to both nearby tokens and foreground tokens; (3) \textit{Global}: In deep layers (Layer 12), different query tokens all focus on foreground tokens with nearly identical attention maps, a phenomenon known as \textit{attention collapse} \citep{ssl_vit}. This attention pattern distribution is consistent across different sizes in CL and supervised methods, as shown in \figref{fig:query_attn_cl}. However, in MIM methods like MAE, the distribution varies. In MAE-S (\figref{fig:query_attn}\textcolor{red}{b}), attention maps are mainly \textit{local}, with slight \textit{hybrid} patterns emerging in deep layers. Conversely, in MAE-L (\figref{fig:query_attn}\textcolor{red}{c}), shallow and deep layers exhibit \textit{local} patterns, while middle layers show \textit{hybrid} patterns. CKA \citep{CKA} analysis in \Appref{app:CKA} reveals similar phenomena regarding feature representation.

\begin{figure}[t]
\centering
\includegraphics[width=\linewidth]{figures/query_attn_v9.pdf}
\vspace{-20pt}
\caption{Attention maps for different query tokens in three representative layers. Each query token's attention map corresponds to a row in the attention matrix, averaged over different heads.}
\label{fig:query_attn}
\vspace{-8pt}
\end{figure}


\textbf{Differences in attention patterns stem from the pre-training tasks.} CL methods, similar to supervised approaches, treat views from the same image as belonging to the same class. This setup encourages models to focus on foreground tokens, as images in the same class often share similar foreground objects but may differ in background. Consequently, attention maps in later layers present \textit{global} patterns. The \textit{local} and \textit{hybrid} patterns can be regarded as the intermediate states in forming the \textit{global} pattern. This high-level pre-training task causes models of different sizes and methods to have similar pattern distributions across layers.
In contrast, pre-training tasks of MIM methods focus on reconstructing features or raw pixels of masked tokens, which is a relatively low-level task relying heavily on spatially nearby tokens. Consequently, models are not compelled to capture global image information, leading small models to primarily exhibit \textit{local} patterns. In larger models like MAE-L, the increased representation capacity allows \textit{hybrid} patterns to spontaneously emerge in the middle layers to capture broader context. In deep layers near the decoder, \textit{local} patterns reappear to support the pre-training task of reconstructing nearby masked tokens.

As a supplement, iBOT exhibits similar patterns with DINO in shallow and middle layers but shows less \textit{attention collapse} in deep layers (see \figref{fig:query_attn_cl}). This can be attributed to that iBOT combines DINO with masked feature prediction \citep{iBOT}, which encourages the later layers to leverage spatial information to predict features of masked tokens.


\subsection{How to quantitively identify different attention patterns?}
\label{sec:nmi}


Three distinct attention patterns are qualitatively summarized in \secref{sec:attention_pattern}, prompting the question of whether a metric can quantitatively measure them. Attention distance measures the average distance 
\begin{wrapfigure}{r}{0.32\textwidth}
\vspace{-1pt}
\centering
\includegraphics[width=1.0\linewidth]{figures/nmi_v9.pdf}
\vspace{-23pt}
\caption{NMI on ImageNet.}
\label{fig:nmi}
\vspace{-11pt}
\end{wrapfigure}
between the query and key tokens, while attention entropy implies the concentration of the attention distribution. However, both metrics depict the relationship between one query and multiple key tokens and are unable to reflect differences in the attention maps of various queries. \textbf{We find that the normalized mutual information (NMI) between query and key tokens is an effective indicator.} The calculation process is elaborated in \Appref{app:NMI}. Let $A\in \mathbb{R}^{N\times N}$ denote the attention matrix, where $N$ is the number of tokens. The NMI is a function of $A$, ranging from 0 to 1.  We highlight two special cases to clarify NMI: (1) When query tokens focus solely on their spatially corresponding key tokens (an extreme \textit{local} pattern), $A$ becomes an identity matrix. Thus the joint probability of query and key tokens is equivalent to their marginal probability and the NMI reaches its maximum value of 1; (2) When all query tokens attend to the same key tokens (an extreme \textit{global} pattern), each row of $A$ is identical. Consequently, the probability distribution of query and key tokens are independent, leading to the minimum NMI of 0.


\textbf{Therefore, \textit{local} patterns have larger NMI values, while \textit{global} patterns exhibit lower ones.} As illustrated in \figref{fig:nmi}, the NMI of DINO-S and iBOT-S decreases with depth and consistently stays below that of MAE models. Especially, the NMI of DINO-S approaches 0 in later layers, revealing its \textit{attention collapse}. In contrast, the NMI of MAE-L first decreases and then increases, due to the \textit{hybrid} patterns in middle layers and \textit{local} patterns in later layers. 


\subsection{The Hybrid pattern matters for semantic segmentation}
\label{sec:lg_matter}
Semantic segmentation is a dense prediction task where all pixels in an image are classified into different semantic classes. Local information is crucial as nearby pixels usually belong to the same class, while global cues are also essential since instances of the same class may appear in different positions within the image. Therefore, we hypothesize that \textbf{\textit{hybrid} patterns, which capture both local and global information, are more important for semantic segmentation than purely \textit{local} or \textit{global} patterns}. To demonstrate this, we conduct the \textit{layerwise linear probing} (LLP) experiments, where frozen features of only one layer are passed to the linear head, as shown in \figref{fig:ft_lp}\textcolor{red}{d}.

\textbf{The LLP performance peaks where \textit{hybrid} attention patterns emerge.} As shown in \figref{fig:layerwise_lp}, supervised and CL methods peak at about three-quarters of the model's depth. Large MIM models (ViT-L) perform better in the middle layers. These peaks commonly occur near the \textit{hybrid} patterns. In contrast, the performance of small MIM models (ViT-S and ViT-B) gradually increases with depth, peaking in the last two layers. This is because, although all layers exhibit \textit{local} patterns, deep layers focus more on foreground tokens (\figref{fig:query_attn}\textcolor{red}{b}) and have smaller NMI values (\figref{fig:nmi}), leading to better LLP performance. Additionally, the performance degradation in iBOT's deep layers is less pronounced than that in DINO and supervised methods. This aligns with observations that iBOT's deep-layer attention maps contain more local information (\secref{sec:attention_pattern}) and have larger NMI values than those of DINO (\figref{fig:nmi}), underscoring the importance of the \textit{hybrid} attention pattern.



\textbf{This hypothesis can explain the phenomena in \secref{sec:benchmark}.} Small MIM models struggle to learn the \textit{hybrid} pattern, resulting in a notable performance gap compared to supervised and CL methods. Conversely, large MIM models successfully develop the \textit{hybrid} pattern, making their fine-tuning performance comparable to other methods. iBOT performs best across different model sizes and evaluation metrics because the \textit{hybrid} pattern occurs more frequently than in other methods.



\subsection{The Texture bias hinders the model's generalization on infrared images}
\label{sec:texture}


\begin{wraptable}{r}{0.51\textwidth}
    \vspace{-7.5mm}
    \caption{The FT performance on RGB and infrared semantic segmentation datasets.}
    \label{tab:rgb_infrared}
    \centering
    \scriptsize
    \setlength{\tabcolsep}{1.0mm}{
    \scalebox{1.0}{
    \begin{tabular}{l c c c c c}
        \toprule
        \multirow{2}{*}{Methods} & \multicolumn{2}{c}{RGB} & \multicolumn{3}{c}{Infrared} \\
        \cmidrule(lr){2-3} \cmidrule(lr){4-6}
        & ADE20K & MFNet-RGB & MFNet-T & SODA & SCUT-Seg \\
        \midrule
        DeiT-B & 47.4 & 57.07 & \textbf{48.59} & 69.73 & 69.35 \\
        DINO-B & 46.8 & 55.20 & 48.54 & \textbf{69.79} & \textbf{69.82} \\
        MAE-B & \textbf{48.1} & \textbf{57.29} & 46.78 & 68.18 & 67.86 \\
        \bottomrule
    \end{tabular}}}
    \vspace{-4mm}
\end{wraptable}

When transferring RGB pre-trained models to infrared tasks, the distribution shift between these modalities significantly impacts performance. A major difference between RGB and infrared images is that RGB images can capture fine-grained textures, which are scarce in infrared images. Therefore, we assume that \textbf{the model's bias towards texture would exacerbate the distribution shift, thereby impairing the transfer performance on infrared tasks.}

\begin{figure}[t]
\centering
\includegraphics[width=\linewidth]{figures/layer_lp_v9.pdf}
\vspace{-20pt}
\caption{The layerwise linear probing performance of different methods on SODA \citep{soda}.}
\label{fig:layerwise_lp}
\vspace{-8pt}
\end{figure}

According to \citet{ssl_vit}, MIM methods are texture-biased while CL and supervised methods are shape-biased. This bias is evident in attention maps in \figref{fig:query_attn}, where MAE models focus on textures while DINO emphasizes edges. We conduct experiments to investigate the bias's impact on infrared segmentation. As shown in \tabref{tab:rgb_infrared}, MAE-B outperforms DeiT-B and DINO-B on RGB datasets like ADE20K \citep{ade20k} and MFNet-RGB \citep{mfnet}, but consistently underperforms on infrared datasets. Notably, the paired MFNet-RGB and MFNet-T share the same scenario and image counts, differing only in modality. This indicates that MAE models pre-trained on ImageNet rely on low-level texture information to reconstruct masked patches, leading to poor generalization on texture-less infrared images. Therefore, \textbf{reducing the texture bias is a promising way to enhance the transfer performance of RGB-pre-trained models on infrared tasks.}

\section{How to improve the performance on infrared segmentation?}
\label{sec:distill}
As discussed in \secref{sec:benchmark_multi_layer}, scaling up model sizes for better performance is impractical for resource-constrained scenarios. Therefore, we focus on enhancing small pre-trained models by introducing a comprehensive framework, UNIP, and validating its effectiveness through extensive experiments.

\subsection{UNIP: a unified infrared pre-training framework}
\label{sec:unip}
UNIP improves small pre-trained models by optimizing the pre-training task, constructing an appropriate pre-training dataset, and refining the fine-tuning architecture, as depicted in \figref{fig:architecture}.

\textbf{NMI-Guided Hybrid Attention Pattern Distillation (NMI-HAD).} Compressing knowledge from large models into smaller ones is an effective strategy to enhance performance without increasing parameter count. Previous works use various distillation targets like logits \citep{dino} and features \citep{efficient_sam}. However, they often overlook the relationship between distillation targets and attention patterns. As revealed in \secref{sec:lg_matter}, the \textit{hybrid} attention pattern is crucial for semantic segmentation, with NMI values linked to attention patterns. Therefore, we propose using \textit{hybird} patterns as the distillation target and introduce the NMI-guided \textit{hybrid} attention pattern distillation. First, the NMI value $\text{NMI}(A_l)$ of each teacher model's layer is calculated on ImageNet-1K:
\begin{equation}
    \text{NMI}(A_l)=\frac{1}{M}\sum_{m=1}^M\text{NMI}(A_l^m),  \quad A_l^m=\text{softmax}\left(\frac{Q_l^m (K_l^m)^T}{\sqrt{d}}\right), \quad l=\frac{L}{2}+1,...,L,
    \label{eq:nmi_cal}
\end{equation}
where $A_l^m$ denotes the $m$-th head attention matrix in the $l$-th layer. $L$ and $d$ are the number of layers and the dimension of the teacher model. The method for calculating NMI is detailed in \Appref{app:NMI}. Note that we only consider layers in the latter half of the model, as shallow layers do not capture sufficient knowledge. Next, NMI values are used to identify the location of the \textit{hybrid} attention pattern. The attention map of the layer whose NMI is closest to an empirical value $s$ is utilized as the distillation target $A_T$. Finally, the attention map $A_S$ in the last layer of the student model is forced to imitate $A_T$ by employing the Kullback-Leibler (KL) divergence constraints:
\begin{equation}
    A_T = \mathop{\arg\max}_{A_l}\Delta \text{NMI}(A_l), \quad \Delta \text{NMI}(A_l) =-\left\vert \text{NMI}(A_l)-s \right\vert, \quad
    \mathcal{L}=\frac{1}{M}\sum_{m=1}^M \text{KL}(A_{T}^m||A_{S}^m), 
    \label{eq:distill}
\end{equation}
Empirically, the NMI values of \textit{hybrid} patterns range between 0.06 and 0.12. We find that setting $s$ within this range yields good results. In all our experiments, we set it to 0.09 by default.


\begin{wraptable}{r}{0.47\textwidth}
    \vspace{-3.5mm}
    \centering
    \caption{Comparisons of infrared pre-training datasets. \#Subset denotes the number of datasets from which the images are collected.}
    \label{tab:dataset_comparison}
    \vspace{1mm}
    \scriptsize
    \setlength{\tabcolsep}{1.0mm}{
    \scalebox{1.0}{
    \begin{tabular}{l c c c c}
        \toprule
        Dataset & \#Image & \#Subset & Width & Height \\
        \midrule
        MSIP \citep{pad} & 178,756 & 8 & 844 & 596 \\
        Inf30 \citep{infmae} & 305,241 & - & 700 & 562 \\
        InfPre (ours) & \textbf{541,088} & \textbf{23} & \textbf{1,075} & \textbf{686} \\
        % InfMix (ours) & 859,375 & 25 & & \\
        \bottomrule
    \end{tabular}}}
    \vspace{-3.5mm}
\end{wraptable}

\textbf{InfMix Dataset.} To alleviate the distribution shift and reduce texture bias when distilling RGB pre-trained models for infrared tasks, we develop InfMix, a mixed dataset for distillation. InfMix comprises \textbf{859,375} images from both RGB and infrared modalities, constructed through four steps. (1) Infrared images play a key role in mitigating the distribution shift. However, existing datasets often lack diversity and sufficient images, so we collect a large and unlabelled infrared pre-training dataset called \textbf{InfPre}. It consists of \textbf{541,088} images from \textbf{23} infrared-related datasets. Compared to the other two datasets in \tabref{tab:dataset_comparison}, InfPre offers a larger number of higher-resolution images sourced from more diverse datasets. Importantly, three segmentation datasets used in the benchmark are excluded from InfPre for fair comparison. Details on data collection and deduplication can be found in \Appref{app:infpre}. (2) A subset of ImageNet-1K \citep{imagenet} is used, comprising \textbf{200,000} images evenly sampled from 1,000 classes. Since these images are part of the teacher model's pre-training data, they can anchor the student representation space close to the teacher's, thereby aiding in transferring the teacher's general feature extraction capabilities to the student. (3) The training set of COCO \citep{coco}, with \textbf{118,287} images, is also included to further enrich the pre-training dataset. Unlike single-object-centric images in ImageNet, COCO images typically depict larger scenes with multiple objects, making them more similar to infrared images, as indicated in \tabref{tab:domain_gap} in the appendix. (4) Images from ImageNet and COCO are converted to grayscale (three identical channels) to resemble infrared images more closely, as noted in \tabref{tab:domain_gap}.

\textbf{Last-Layer Feature Pyramid Network (LL-FPN).} To adapt the non-hierarchical ViT to multi-scale decoders in dense prediction tasks, previous works \citep{mae,iBOT} typically generate multi-scale feature maps from different layers of ViT, as shown in \figref{fig:ft_lp}\textcolor{red}{a}. \textbf{However, we find this multi-layer design unnecessary for our distilled models.} In these models, the \textit{hybrid} patterns in later layers equip the final features with both local and global information, making them suitable for multi-scale feature map generation. Inspired by ViTDet \citep{vitdet}, we propose using the last-layer feature pyramid network during fine-tuning. It constructs all feature maps of different scales upon the last layer's features, as illustrated in \figref{fig:architecture} and \figref{fig:ft_lp}\textcolor{red}{c}. As a bonus, this approach enhances the representation capacity of each scale branch compared to the configuration in \figref{fig:ft_lp}\textcolor{red}{a}, since they go through the entire backbone, leading to improved fine-tuning performance.

\begin{table}[t]
    \centering
    \caption{The infrared semantic segmentation performance of different models. Across various pre-trained methods, UNIP models significantly surpass pre-trained models of the same size. Remarkably, they even outperform their teacher models, despite the latter having more parameters.}
    \label{tab:main_distill}
    % \vspace{-2mm}
    \centering
    \scriptsize
    \setlength{\tabcolsep}{1.2mm}{
    \scalebox{1.0}{
    \begin{tabular}{l c c c c c c c c c}
        \toprule
        \multirow{2}{*}{Methods} & \multirow{2}{*}{Params(M)} & \multicolumn{4}{c}{Fine-tuning (FT)} & \multicolumn{4}{c}{Linear Probing (LP)} \\ 
        \cmidrule(lr){3-6} \cmidrule(lr){7-10}
        & & SODA & MFNet-T & SCUT-Seg & Average FT & SODA & MFNet-T & SCUT-Seg & Average LP \\
        \midrule
        \textcolor{gray}{MAE-L (Teacher)} & \textcolor{gray}{441.3} & \textcolor{gray}{71.04} & \textcolor{gray}{51.17} & \textcolor{gray}{70.83} & \textcolor{gray}{64.35} & \textcolor{gray}{52.20} & \textcolor{gray}{31.21} & \textcolor{gray}{43.71} & \textcolor{gray}{42.37} \\
        MAE-T & 11.0 & 52.85 & 35.93 & 51.31 & 46.70 & 23.75 & 15.79 & 27.18 & 22.24 \\
        \rowcolor{cyan!15} UNIP-T & 11.0 & \textbf{64.83} & \textbf{48.77} & \textbf{67.22} & \textbf{60.27} (\plus{+13.57}) & \textbf{44.12} & \textbf{28.26} & \textbf{35.09} & \textbf{35.82} (\plus{+13.58}) \\
        MAE-S & 41.9 & 63.36 & 42.44 & 60.38 & 55.39 & 38.17 & 21.14 & 34.15 & 31.15 \\
        \rowcolor{cyan!15} UNIP-S & 41.9 & \textbf{70.99} & \textbf{51.32} & \textbf{70.79} & \textbf{64.37} (\plus{+8.98}) & \textbf{55.25} & \textbf{33.49} & \textbf{43.37} & \textbf{44.04} (\plus{+12.89}) \\
        MAE-B & 163.7 & 68.18 & 46.78 & 67.86 & 60.94 & 43.01 & 23.42 & 37.48 & 34.64 \\
        \rowcolor{cyan!15} UNIP-B & 163.7 & \textbf{71.47} & \textbf{52.55} & \textbf{71.82} & \textbf{65.28} (\plus{+4.34}) & \textbf{58.82} & \textbf{34.75} & \textbf{48.74} & \textbf{47.43} (\plus{+12.79}) \\
        \midrule
        \textcolor{gray}{DINO-B (Teacher)} & \textcolor{gray}{163.7} & \textcolor{gray}{69.79} & \textcolor{gray}{48.54} & \textcolor{gray}{69.82} & \textcolor{gray}{62.72} & \textcolor{gray}{59.33} & \textcolor{gray}{34.86} & \textcolor{gray}{47.23} & \textcolor{gray}{47.14} \\
        DINO-S & 41.9 & 68.56 & 47.98 & 68.74 & 61.76 & 56.02 & 32.94 & 45.94 & 44.97 \\
        \rowcolor{cyan!15} UNIP-S & 41.9 & \textbf{69.35} & \textbf{49.95} & \textbf{69.70} & \textbf{63.00} (\plus{+1.24}) & \textbf{57.76} & \textbf{34.15} & \textbf{46.37} & \textbf{46.09} (\plus{+1.12}) \\
        \midrule
        \textcolor{gray}{iBOT-L (Teacher)} & \textcolor{gray}{441.3} & \textcolor{gray}{71.75} & \textcolor{gray}{51.66} & \textcolor{gray}{71.49} & \textcolor{gray}{64.97} & \textcolor{gray}{61.73} & \textcolor{gray}{36.68} & \textcolor{gray}{50.12} & \textcolor{gray}{49.51} \\
        iBOT-S & 41.9 & 69.33 & 47.15 & 69.80 & 62.09 & 57.10 & 33.87 & 45.82 & 45.60 \\
        \rowcolor{cyan!15} UNIP-S & 41.9 & \textbf{70.75} & \textbf{51.81} & \textbf{71.55} & \textbf{64.70} (\plus{+2.61}) & \textbf{60.28} & \textbf{37.16} & \textbf{47.68} & \textbf{48.37} (\plus{+2.77}) \\
        iBOT-B & 163.7 & 71.15 & 48.98 & 71.26 & 63.80 & 60.05 & 34.34 & 49.12 & 47.84 \\
        \rowcolor{cyan!15} UNIP-B & 163.7 & \textbf{71.75} & \textbf{51.46} & \textbf{72.00} & \textbf{65.07} (\plus{+1.27}) & \textbf{63.14} & \textbf{39.08} & \textbf{52.53} & \textbf{51.58} (\plus{+3.74}) \\    
        \bottomrule
    \end{tabular}}}
    \vspace{-4mm}
\end{table} 


\subsection{Experiments}
\label{sec:unip_experiments}

The MAE-L, DINO-B, and iBOT-L are utilized as teacher models for distillation, and the 18th, 9th, and 21st layers are used as the target layer, according to \eqnref{eq:nmi_cal} and \eqnref{eq:distill}. Unless otherwise specified, the distillation, fine-tuning, and linear probing processes are each conducted for 100 epochs. For ablation studies, we mainly focus on the fine-tuning metric as it reflects the model's highest achievable performance. More details about experimental settings can be found in \Appref{app:pre-training}.

\begin{table}[t]
  \begin{minipage}{0.49\linewidth}
  \caption{Comparisons with other segmentation methods (FT). All the compared results except PAD are borrowed from TINN. Training epochs of SODA and MFNet-T are 200 and 300.}
    \label{tab:sota_comparison}
    \vspace{1mm}
    \centering
    \scriptsize
    \setlength{\tabcolsep}{0.8mm}{
    \scalebox{1.0}{
    \begin{tabular}{l c c c c c c}
        \toprule
        Methods & Params(M) & SODA & MFNet-T \\
        \midrule
        DeepLab V3+ \citep{deeplabv3+} & 62.7  & 68.73 & 49.80  \\
        PSPNet \citep{pspnet} & 68.1 & 68.68 & 45.24 \\
        UPerNet \citep{upernet} & 72.3 & 67.45 & 48.56 \\
        SegFormer \citep{segformer} & 84.7 & 67.86 & 50.68 \\
        ViT-Adapter \citep{vitadapter} & 99.8 & 68.12 & 50.62 \\
        Mask2Former \citep{mask2former} & 216.0 & 67.58 & 51.30 \\
        MaskDINO \citep{maskdino} & 223.0 & 66.32 & 51.03 \\
        \midrule
        EC-CNN \citep{soda} & 54.5 & 65.87 & 47.56 \\
        MCNet \citep{mcnet} & 35.7 & 63.89 & 43.15 \\
        PAD (MAE-B) \citep{pad} & 164.9 & 69.71 & 50.14 \\
        TINN \citep{tinn} & 85.3 & 69.45 & 51.93  \\
        \midrule
        \rowcolor{cyan!15} UNIP-T (MAE-L) & 11.0 & 67.29 & 50.39 \\
        \rowcolor{cyan!15} UNIP-S (MAE-L) & 41.9 & \underline{71.35} & \underline{53.76} \\
        \rowcolor{cyan!15} UNIP-B (MAE-L) & 163.7 & \textbf{72.19} & \textbf{54.35} \\
        \bottomrule
    \end{tabular}}}
    \vspace{-4mm}
  \end{minipage}
  \hfill
  \begin{minipage}{0.5\linewidth}
    \centering
    \caption{Impact of distillation targets (UNIP-S).}
    \label{tab:target_ablation}
    \scriptsize
    \setlength{\tabcolsep}{1.0mm}{
    \scalebox{0.97}{
    \begin{tabular}{c c c c c c}
        \toprule
        Target (Teacher) & Layer & SODA & MFNet-T & SCUT-Seg & Avg FT \\
        \midrule
        \multirow{2}{*}{\makecell[c]{Feature \\ (MAE-L)}} & 18 & 65.66 & 48.44 & 66.55 & 60.22  \\
        & 24 & 66.86 & 49.00 & 66.61 & 60.82  \\
        \midrule
        \multirow{2}{*}{\makecell[c]{Attention \\ (MAE-L)}} & \cellcolor{cyan!15}18 & \cellcolor{cyan!15}\textbf{70.99} & \cellcolor{cyan!15}\textbf{51.32} & \cellcolor{cyan!15}\textbf{70.79} & \cellcolor{cyan!15}\textbf{64.37}  \\
        & 24 & 67.74 & 50.39 & 69.00 & 62.38  \\
        \bottomrule
    \end{tabular}}}
    \vspace{-2mm}
    \caption{Impact for the LL-FPN (UNIP-S). \xmark\, and \cmark\, denote the ones in \figref{fig:ft_lp}\textcolor{red}{a} and \figref{fig:ft_lp}\textcolor{red}{c}.}
    \label{tab:fpn_ablation}
    \centering
    \scriptsize
    \setlength{\tabcolsep}{0.8mm}{
    \scalebox{0.95}{
    \begin{tabular}{l c c c c c}
        \toprule
        Teacher (Layer) & LL-FPN & SODA & MFNet-T & SCUT-Seg & Avg FT \\
        \midrule
        \multirow{2}{*}{\makecell[c]{\textcolor{gray}{\textit{Hybrid} Pattern} \\ MAE-L (Layer 18)}} & \xmark & 69.54 & 50.18 & 70.63 & 63.45 \\
        & \cellcolor{cyan!15}\cmark & \cellcolor{cyan!15}\textbf{70.99} & \cellcolor{cyan!15}\textbf{51.32} & \cellcolor{cyan!15}\textbf{70.79} & \cellcolor{cyan!15}\textbf{64.37} \\
        \midrule
        \multirow{2}{*}{\makecell[c]{\textcolor{gray}{\textit{Local} Pattern} \\ MAE-L (Layer 24)}} & \xmark & 67.64 & 49.93 & 68.68 & 62.08 \\
        & \cmark & \textbf{67.74} & \textbf{50.39} & \textbf{69.00} & \textbf{62.38} \\
        \midrule
        \multirow{2}{*}{\makecell[c]{\textcolor{gray}{\textit{Hybrid} Pattern} \\ DINO-B (Layer 9)}} & \xmark & 68.56 & 49.44 & 68.49 & 62.16 \\
        & \cellcolor{cyan!15}\cmark & \cellcolor{cyan!15}\textbf{69.35} & \cellcolor{cyan!15}\textbf{49.95} & \cellcolor{cyan!15}\textbf{69.70} & \cellcolor{cyan!15}\textbf{63.00} \\
        \midrule
        \multirow{2}{*}{\makecell[c]{\textcolor{gray}{\textit{Global} Pattern} \\ DINO-B (Layer 12)}} & \xmark & \textbf{68.62} & 47.36 & \textbf{69.71} & 61.90 \\
        & \cmark & 68.50 & \textbf{48.40} & 69.67 & \textbf{62.19} \\
        \bottomrule
    \end{tabular}}}
    \vspace{-4mm}
  \end{minipage}
\end{table}




\textbf{Improvements of UNIP.} As shown in \tabref{tab:main_distill}, UNIP significantly enhances the performance of small models across both metrics, often exhibiting comparable or even better performance than teacher models. With MAE-L as the teacher, UNIP-T, UNIP-S, and UNIP-B achieve average mIoU gains of \textbf{13.57\%}, \textbf{8.98\%}, and \textbf{4.34\%} in fine-tuning, and \textbf{13.58\%}, \textbf{12.98\%}, and \textbf{12.79\%} in linear probing. Notably, UNIP-S performs comparably to MAE-L with only \textbf{1/10} of the computational cost. UNIP-B even outperforms MAE-L by \textbf{0.93\%} in FT and \textbf{5.06\%} in LP. Using iBOT-L as the teacher, UNIP-S transcends iBOT-S by \textbf{2.61\%} in FT and \textbf{2.77\%} in LP. Meanwhile, UNIP-B shows gains of \textbf{1.27\%} in FT and \textbf{3.74\%} in LP, exceeding its teacher iBOT-L. Even with a smaller teacher like DINO-B, UNIP-S still enhances performance by as least \textbf{1.12\%}. \tabref{tab:sota_comparison} compares the fine-tuning performance of UNIP with other RGB or infrared segmentation methods. With fewer than half the parameters, UNIP-S, distilled from MAE-L, surpasses the universal segmentation method Mask2Former \citep{mask2former} by \textbf{3.77\%} on SODA and \textbf{2.46\%} on MFNet-T. It also outperforms TINN \citep{tinn}, specially designed for infrared semantic segmentation, by \textbf{1.9\%} on SODA and \textbf{1.83\%} on MFNet-T. A larger model UNIP-B further widens this performance gap, indicating that UNIP can greatly unleash the potential of the vanilla ViT for infrared semantic segmentation.


\begin{wrapfigure}{r}{0.51\textwidth}
\vspace{-10pt}
\centering
\includegraphics[width=1.0\linewidth]{figures/ft_nmi_v3.pdf}
\vspace{-22pt}
\caption{The average FT and NMI of each target layer. Each model is distilled for 20 epochs.}
\label{fig:ft_nmi}
\vspace{-11pt}
\end{wrapfigure}

\textbf{Impact of Distillation Target Layers.} \figref{fig:ft_nmi} displays the average fine-tuning performance using different layers of MAE-L and DINO-B as the distillation target layer. Notably, both models exhibit a strong positive correlation between average FT performance and $\Delta$NMI in \eqnref{eq:distill}, as indicated by a large Pearson coefficient. Furthermore, the peaks of FT and $\Delta$NMI occur in the same layer, highlighting the effectiveness of the NMI-HAD.


\begin{wraptable}{r}{0.51\textwidth}
    \vspace{-7.5mm}
    \caption{Ablations for components of the InfMix dataset. The teacher and student models are MAE-L and UNIP-S. All datasets are distilled for the same number of iterations for fair comparison.}
    \label{tab:dataset_composition_ablation}
    \centering
    \scriptsize
    \setlength{\tabcolsep}{0.8mm}{
    \scalebox{1.0}{
    \begin{tabular}{l c c c c c}
        \toprule
        Dataset & \#Images & SODA & MFNet-T & SCUT-Seg & Avg FT \\
        \midrule
        \rowcolor{cyan!15} InfMix & 859,375 & \textbf{70.99} & \textbf{51.32} & 70.79 & \textbf{64.37} \\
        -- w/o IN1K & 659,375 & 69.41 & 51.13 & 70.21 & 63.58 \\
        -- w/o COCO & 741,088 & 69.62 & 51.29 & 69.58 & 63.50 \\
        -- w/o Grayscale & 859,375 & 69.73 & 50.71 & \textbf{71.09} & 63.84 \\
        \midrule
        ImageNet-1K & 1,281,167 & 69.39	& 49.11 & 69.63 & 62.71 \\ 
        InfPre & 541,088 & 68.45 & 51.27 & 67.87 & 62.53 \\
        \bottomrule
    \end{tabular}}}
    \vspace{-4mm}
\end{wraptable}
\textbf{Impact of Pre-training Datasets.} \tabref{tab:dataset_composition_ablation} illustrates the performance of different datasets. As anticipated, all components of the InfMix dataset are necessary, including the infrared dataset InfPre, the ImageNet subset \citep{imagenet}, the COCO training set  \citep{coco}, and the grayscale operation. Remarkably, InfMix significantly outperforms single-modality datasets like ImageNet and InfPre. This improvement can be attributed to the complementary strengths of both modalities: infrared images help mitigate the distribution shift issue, while RGB images enhance general feature extraction capabilities. The mixed dataset effectively balances these two aspects. Moreover, \tabref{tab:data_ratio} in the appendix displays the scaling characteristics of pre-training data, demonstrating the necessity of constructing the larger InfMix dataset.


\begin{wrapfigure}{r}{0.25\textwidth}
\vspace{-11pt}
\centering
\includegraphics[width=1.0\linewidth]{figures/layer_select_s_v2.pdf}
\vspace{-22pt}
\caption{The average FT when employing different $s$ in \eqnref{eq:distill}.}
\label{fig:para_s}
\vspace{-15pt}
\end{wrapfigure}
\textbf{Impact of the Hyperparameter $s$.} The parameter $s$ in \eqnref{eq:distill} determines the layer chosen for distillation. As presented in \figref{fig:para_s}, when $s$ ranges from 0.06 to 0.12, the selected layer remains nearly constant: the 18th layer for MAE-L and the 9th layer for DINO-B. Therefore, the performance of UNIP is relatively stable with respect to $s$.


\textbf{Comparison with the Feature Distillation.} As compared in \tabref{tab:target_ablation}, the performance of feature distillation consistently lags behind attention distillation across different layers, implying the latter's superiority. We believe this is because attention distillation only restricts the relationship between tokens, whereas feature distillation imposes direct constraints on each token's features. Excessive constraints on features may intensify the distribution shift and hinder the generalization of distilled models. Additionally, the performance of feature distillation across different layers is similar, likely due to the skip connections in ViT, which enhance feature similarities between layers. In contrast, the attention maps of different layers differ significantly, as revealed in \secref{sec:attention_pattern}.


\begin{wraptable}{r}{0.48\textwidth}
    \vspace{-7.5mm}
    \centering
    \caption{Comparisons of pre-training methods.}
    \label{tab:other_methods}
    \scriptsize
    \setlength{\tabcolsep}{1.0mm}{
    \scalebox{1.0}{
    \begin{tabular}{l c c}
        \toprule
        Method & Avg FT & Training Time (h) \\
        \midrule
        Continual Pre-trained (MAE-S) & 58.53 & 75.0 (1x RTX3090) \\
        UNIP-S (MAE-L distilled) & \textbf{64.37} & \textbf{72.5} (1x RTX3090)  \\
        \bottomrule
    \end{tabular}}}
    \vspace{-4mm}
\end{wraptable}
\textbf{Comparison with Contiunal Pre-training on Target Domain.}
We initialize MAE-S with RGB pre-trained weights and further pre-train it on InfMix for 100 epochs. As shown in \tabref{tab:other_methods}, this continually pre-trained MAE-S (58.53\%) exceeds the RGB pre-trained one (55.39\% in \tabref{tab:main_distill}). However, it still underperforms UNIP-S by 5.84\% and requires more training time, highlighting the efficiency of UNIP over continual pre-training.

\textbf{Impact of the LL-FPN.} \tabref{tab:fpn_ablation} shows the performance of models distilled from various teachers. While LL-FPN enhances performance for all models, the improvements are much greater when using \textit{hybrid} patterns as distillation targets than \textit{local} or \textit{global} patterns. This demonstrates LL-FPN's superiority and good compatibility with the \textit{hybrid} pattern, supporting the analysis in \secref{sec:unip}.


\begin{wraptable}{r}{0.6\textwidth}
    \vspace{-8mm}
    \caption{The LLP performance on RGB and depth datasets. Training epochs are 30 for ADE20K and 100 for others.}
    \label{tab:rgb_llp}
    \vspace{1mm}
    \centering
    \scriptsize
    \setlength{\tabcolsep}{0.8mm}{
    \scalebox{0.93}{
    \begin{tabular}{l c c c c c}
        \toprule
        \multirow{3}{*}{\textcolor{gray}{(Modality)} Dataset} & \multicolumn{2}{c}{DINO-S} & \multicolumn{2}{c}{DeiT-S} \\
        \cmidrule(lr){2-3} \cmidrule(lr){4-5}
        & Layer 9 & Layer 12  & Layer 9 & Layer 12 \\
        & \textcolor{gray}{(\textit{Hybrid})} & \textcolor{gray}{(\textit{Global})} & \textcolor{gray}{(\textit{Hybrid})} & \textcolor{gray}{(\textit{Global})} \\
        \midrule
        \textcolor{gray}{(RGB)} ADE20K \citep{ade20k} & \textbf{26.11} & 23.15 & \textbf{24.35} & 22.68  \\
        \textcolor{gray}{(RGB)} MFNet-RGB \citep{mfnet} & \textbf{38.94} & 37.53  & \textbf{30.43} & 29.44 \\
        \midrule
        \textcolor{gray}{(Depth)} NYUDepthv2 \citep{nyuv2} & \textbf{17.25} & 15.29 & \textbf{5.55} & 5.15 \\
        \textcolor{gray}{(Depth)} SUN-RGBD \citep{sunrgbd} & \textbf{13.17} & 11.41 & \textbf{5.61} & 4.94 \\
        \bottomrule
    \end{tabular}}}
    \vspace{-4mm}
\end{wraptable}

\textbf{Applicability to Other Modalities.} We extend the LLP experiments in \secref{sec:lg_matter} to the RGB and depth modalities. As shown in \tabref{tab:rgb_llp}, for both DINO-S and DeiT-S, the LLP performance of middle layers (the \textit{local} pattern) surpasses that of deep layers (the \textit{global} pattern) across all RGB and depth semantic segmentation datasets. This mirrors the phenomenon in the infrared domain discussed in \secref{sec:lg_matter}, underscoring the importance of \textit{hybrid} patterns for semantic segmentation tasks, regardless of dataset size or modality. Therefore, we believe that UNIP can be effectively extended to other modalities.

\textbf{Visualizations.} We present visualizations of distilled models in the appendix. As shown in \figref{fig:distill_query_attn}\textcolor{red}{c}, the deep layers of UNIP-S exhibit \textit{hybrid} patterns, indicating that UNIP effectively transfers these patterns from the teacher to the student. The CKA alignment between DION-S and UNIP-S in shallow and middle layers, shown in \figref{fig:cka}, further demonstrates this from a feature representation perspective. Additionally, compared to MAE-L, attention maps in UNIP-S focus more on shape information than textures, as evident in \figref{fig:texture_query_attn}. The emergence of \textit{hybrid} patterns in deep layers and the reduced bias towards texture both contribute to the excellent performance of UNIP.




\section{Conclusion and Discussion}
In this work, we comprehensively benchmark the infrared segmentation performance of different pre-training methods and uncover several valuable insights. We further analyze the pre-trained attention maps and identify the importance of \textit{hybrid} patterns for semantic segmentation. Finally, we propose the UNIP framework to improve the performance of small ViT models. Extensive experimental results demonstrate the effectiveness of our dataset and method. UNIP presents a viable approach for selective knowledge distillation in domain transfer settings. We hope our analysis can provide meaningful insights into the characteristics and differences among pre-training methods, ultimately contributing to the advancements of visual pre-training and downstream transfer learning.

\textbf{Limitations and Future Work.} Due to limited computing resources, we validate UNIP's effectiveness only in the infrared domain for semantic segmentation. However, we believe UNIP can be effectively extended to other modalities, such as RGB and depth images, as the superiority of \textit{hybrid} patterns in these modalities is demonstrated in \tabref{tab:rgb_llp}. Exploring its potential in other dense prediction tasks, like object detection and depth estimation, is also worthwhile. Moreover, combining \textit{hybrid} patterns from different pre-trained methods could 
be a promising avenue.




% \subsubsection*{Acknowledgments}
% Use unnumbered third level headings for the acknowledgments. All
% acknowledgments, including those to funding agencies, go at the end of the paper.

\section*{Reproducibility Statement}
Reproducibility is a priority in our research. In this statement, we outline the measures taken to ensure our work can be reproduced.

\textbf{Source Code.} The source code of our work is available at this \href{https://github.com/casiatao/UNIP}{link}. Researchers can access and utilize our code to reproduce the experimental results in this paper. The source code and pre-trained model weights will be made publicly available.

\textbf{Experimental Setup and Details.} In the main paper, the basic experimental configurations are presented in \secref{sec:benchmark_setup} (benchmark) and in \secref{sec:unip_experiments} (UNIP). In \Appref{app:experimental_details}, we provide the detailed settings, including the implementation details of the benchmark (\Appref{app:benchmark_details}) and UNIP (\Appref{app:pre-training}), the comparisons of different evaluation metrics and their hyperparameter settings (\Appref{app:appendix_evaluation}), and the evaluation datasets usage (\Appref{app:evaluation_datasets}).

\textbf{Datasets.} We outline the construction steps of our InfMix dataset in \secref{sec:unip}. In \Appref{app:pretraining_dataset}, we further present more details about the dataset collection and preprocessing.

By highlighting these references, we intend to improve the reproducibility of our work, helping other researchers verify and build on our findings. We're open to any questions or requests for more information about our methods, as we aspire to ensure our research is transparent and reliable.



\bibliography{iclr2025_conference}
\bibliographystyle{iclr2025_conference}

\newpage
\appendix
\section*{Appendix}
In \secref{app:related_work}, we discuss the related works. In \secref{app:relationship}, we review the relationship between our motivations and proposed methods. In \secref{app:experimental_details}, we provide detailed descriptions of the experimental settings, including the complete benchmark results, evaluation metrics and datasets, and the experimental specifics of UNIP. Further analysis is conducted in \secref{app:appendix_analysis}, covering (1) the relationship between NMI and attention patterns, and (2) the CKA analysis of feature representation. Additional experimental results are presented in \secref{app:more_experiments}. Finally, we provide more details of the pre-training dataset in \secref{app:pretraining_dataset} and offer additional visualization results in \secref{app:more_visualization}.

\section{Related Work}
% \subsection{Vision Language Model}
% 시각장애인에서 상황을 설명할 DB가 없으니 만들었다. 그리고 이를 VLM에 튜닝했다.
\subsection{Technical approaches for assisting the visually-impaired}


\subsection{Datasets for visual instruction tuning}

This section demonstrates the relationships between SM\_ACTV, MEM\_UTIL,
and GPU\_UTIL. It also explains the correlations among counters and imbalance
metrics. Heatmaps provide insights into how different
factors interact. This section investigates the following question:
%
\begin{RQcallout}
    {\bf RQ5 }{\it \RQfive{}}
\end{RQcallout}

\subsection{Relationship between SM\_ACTV, MEM\_UTIL and GPU\_UTIL}

We analyze the relationship between SM\_ACTV, MEM\_UTIL and GPU\_UTIL
to understand how memory transfer (MEM\_UTIL) and
computation (SM\_ACTV) components contribute to GPU\_UTIL.
Figure~\ref{fig:memcopy_sm_spatial_gputil} visualizes
GPU\_UTIL as a function of mean (left), spatial imbalance (middle),
and temporal imbalance (right) metric values of SM\_ACTV and MEM\_UTIL.
Colors represent the corresponding metric for GPU\_UTIL. Cells with no
jobs shown in orange.

In these heatmaps, each cell in the plots represents a binned value range for
SM\_ACTV and MEM\_UTIL. For instance, in the left plot, we divide the mean
counter values into 10 bins. Jobs with mean counter values falling within a
specific bin are grouped into the
corresponding cell. The color of each cell represents the mean GPU\_UTIL of all
jobs within that bin.

% To assess consistency, we compute mean GPU\_UTIL for each cell and
% evaluate its stability using the Standard Error of the Mean (SEM). SEM
% is defined as: $\text{SEM} = \frac{\sigma}{\sqrt{n}}$, where $\sigma$ is the
% standard deviation and $n$ is the sample size.
% Since SEM values are consistently low, we do not report them in this paper.

The left plot in Figure~\ref{fig:memcopy_sm_spatial_gputil} shows
that high mean of GPU\_UTIL (dark blue) occurs when both
mean SM\_ACTV and MEM\_UTIL are high, which indicates efficient use
of both compute and memory transfer resources. The cells with
mean GPU\_UTIL value of greater than 70\% are annotated.

The middle plot in Figure~\ref{fig:memcopy_sm_spatial_gputil} presents spatial
imbalance using the same axes. Annotated cells represent spatial imbalance of
GPU\_UTIL values below 0.3. Unlike the middle plot, the brighter cells are mostly
clustered around the low spatial imbalance of MEM\_UTIL values (\textless0.2) with
a few exceptions on top left. As spatial imbalance of MEM\_UTIL increases
(moving up the y-axis), spatial imbalance in GPU\_UTIL decreases,
even when spatial imbalance in SM\_ACTV remains low. However, when the
spatial imbalance of MEM\_UTIL is low (\textless0.2), the spatial
imbalance of GPU\_UTIL remains relatively more stable as the spatial
imbalance of SM\_ACTV increases. This suggests that spatial imbalance of
MEM\_UTIL has a stronger effect on spatial imbalance of GPU\_UTIL.

The right plot in Figure~\ref{fig:memcopy_sm_spatial_gputil} visualizes
temporal imbalances of SM\_ACTV and MEM\_UTIL, colored by temporal imbalance
of GPU\_UTIL. Cells with mean temporal imbalance of GPU\_UTIL values
below 0.3 are annotated. As SM\_ACTV increases (moving right along the x-axis),
the temporal imbalance of GPU\_UTIL also increases (darker colors), regardless of
MEM\_UTIL. However, increasing temporal imbalance of MEM\_UTIL (moving up) has
a weaker effect on GPU\_UTIL. This suggests that temporal imbalance
of SM\_ACTV has a stronger influence on GPU\_UTIL.

\subsection{Correlation among Hardware Counters}

Figure~\ref{fig:correlations} presents three correlation
heatmaps examining relationship between mean counters values (left),
spatial imbalances (middle), and temporal imbalances (right). We calculate
correlations using jobs with at least one active precision counter.
Values are ranged from -1 (perfect negative correlation) to +1 (perfect
positive correlation), and 0 indicates no monotonic relationship.

The left heatmap in Figure~\ref{fig:correlations} shows correlations
between mean counter values per job. Mean of SM\_ACTV
has the strongest correlations with GPU\_UTIL (0.72), while MEM\_UTIL
exhibits a relatively lower correlation (0.59). This indicates that
computation activity plays a more direct role in mean GPU\_UTIL compared
to data transfer activity (from GPU to CPU or vice versa). HBM usage (0.46)
and GPU cores show weaker correlations, which suggests that
they are less critical in determining mean of GPU\_UTIL.

The middle heatmap in Figure~\ref{fig:correlations} demonstrates spatial imbalance
correlations. Spatial imbalance of GPU\_UTIL is strongly correlated with MEM\_UTIL
(0.86) and slightly less with SM\_ACTV (0.77), consistent with
Figure~\ref{fig:memcopy_sm_spatial_gputil}. Spatial imbalance of SM\_ACTV
has high correlations with spatial imbalance of tensor (93\%), FP16 (91\%),
FP32 (97\%), and FP64 core usage (97\%). Spatial imbalance of MEM\_UTIL is more
strongly correlated with FP32 (0.92) than other precisions. Spatial imbalance of
FP16\_ACTV shows stronger correlations with other counters than its temporal
imbalance.

The right heatmap in Figure~\ref{fig:correlations} examines temporal imbalance
correlations. Temporal imbalance of GPU\_UTIL is strongly correlated
with temporal imbalance of SM\_ACTV (0.89) and MEM\_UTIL (0.83). This
indicates that fluctuations in GPU\_UTIL closely related to variations in compute
and memory transfer activity. Temporal imbalance of SM\_ACTV also shows
high correlations with FP64\_ACTV (0.93) and FP32 (0.85).
Correlation between temporal and spatial imbalance of FP16\_ACTV and FP64\_ACTV
cannot be calculated (black cells) due to insufficient data, as only two jobs use both.

\section{Experimental Details}
\label{app:experimental_details}
\subsection{Benchmark details}
\label{app:benchmark_details}
Frontier language models demonstrate a remarkable mismatch between their problem-solving capabilities and poor out-of-box verification capabilities.
These limitations have largely been attributed to the inability of current language models to self-diagnose hallucinations or enforce rigour \citep{zhang_how_2023,orgad_llms_2024,snyder_early_2024,kamoi_evaluating_2024, tyen_llms_2024, DBLP:conf/iclr/0009CMZYSZ24}.
However, our findings that models can be directed to accurately perform verifications at scale suggest that these out-of-box limitations can be addressed with standard methods like instruction tuning.
We compiled a set of challenging reasoning problems and candidate solutions to provide a benchmark for these deficits.

Each entry in this benchmark consists of a question, a correct candidate response, and an incorrect candidate response, and is manually curated from the residuals of our sampling-based search experiments (Section~\ref{section:pipeline}).
An example entry from this benchmark can be found below (see Appendix~\ref{app:examplebenchmark} for more).

\vspace{0.4cm}
\begin{tcolorbox}[title=Question from LiveBench Reasoning (Web-of-Lies Puzzle), breakable]
In this question, assume each person either always tells the truth or always lies. The person at the campground thinks their friend is lying. Mateo is at the aquarium. The person at the restaurant says the person at the hotel lies. Farid is at the movie theater. The person at the movie theater says the person at the campground lies. Ryan is at the shopping mall. The person at the cafe says the person at the campground lies. The person at the observatory says the person at the museum lies. The person at the museum says the person at the restaurant tells the truth. The person at the campground says the person at the airport tells the truth. The person at the campground tells the truth. Luna is at the restaurant. The person at the movie theater saw a firetruck. Isabella is at the observatory. The person at the amusement park says the person at the museum tells the truth. The person at the restaurant says the person at the shopping mall tells the truth. Liam is at the beach. Ayaan is at the airport. The person at the restaurant says the person at the beach tells the truth. Maya is at the amusement park. The person at the restaurant says the person at the aquarium lies. The person at the restaurant says the person at the movie theater lies. The person at the library says the person at the campground tells the truth. Elowen is at the hotel. Priya is at the library. The person at the gym says the person at the restaurant tells the truth. Kehinde is at the cafe. Anika is at the museum. Quan is at the campground. Soren is at the gym. Does the person at the museum tell the truth? Does the person at the restaurant tell the truth? Does the person at the movie theater tell the truth? Think step by step, and then put your answer in **bold** as a list of three words, yes or no (for example, **yes, no, yes**). If you don't know, guess.
\end{tcolorbox}
\definecolor{mydarkgreen}{rgb}{0.0, 0.5, 0.0} %


\begin{paracol}{2}

\switchcolumn[0]
\begin{tcolorbox}[breakable, colframe=mydarkgreen!70, colback=mydarkgreen!10, title=Correct response]
Here's how we can break down this logic puzzle:

1. **Quan is at the campground and tells the truth.** This is a given fact.

...

4. **The person at the restaurant (Luna) says the person at the hotel (Elowen) lies.**  We don't know yet if Luna is lying or telling the truth.

...

10. **Since Luna tells the truth about Farid, Luna *always* tells the truth.**  This unlocks several other pieces of information.

...

13. **The person at the cafe (Kehinde) says the person at the campground lies.** Since Quan tells the truth, Kehinde is lying.

Therefore:
* **Anika (museum):** Tells the truth.
* **Luna (restaurant):** Tells the truth.
* **Farid (movie theater):** Lies.

**yes, yes, no**
\end{tcolorbox}

\switchcolumn
\begin{tcolorbox}[breakable, colframe=red!70, colback=red!10, title=Incorrect Response Due to Hallucination]
Here's how we can solve this logic puzzle:

1. **Quan is at the campground and tells the truth.** This is given.

2. **The person at the airport (Ayaan) tells the truth.** Quan (campground) says the person at the airport tells the truth, and Quan tells the truth.

...

4. **Luna (restaurant) lies.**  The person at the restaurant says the person at the hotel (Elowen) lies. \textbf{\color{red} Later, the person at the restaurant says the person at the campground (Quan) lies.} Since Quan tells the truth, Luna (restaurant) must be lying.

...

Therefore:
* **Anika (museum): lies**
* **Luna (restaurant): lies**
* **Farid (movie theater): lies**

**Answer: no, no, no** 
\end{tcolorbox}
\end{paracol}

\noindent\begin{minipage}{\textwidth}
\captionof{figure}{Example of an entry in our verification benchmark. The question is sourced from the LiveBench Reasoning benchmark, and the two responses are generated by Gemini v1.5 Pro. The green response has the correct final answer; the red response has the wrong final answer due to hallucinating a non-existent clause.}
\label{fig:example}
\vspace{0.4cm}
\end{minipage}

\noindent
On each entry, our benchmark studies verification accuracy on two tasks:
\begin{enumerate}
    \item \textbf{Scoring task.} When given only the question and one of the responses, is the model able to discern the correctness of the response?
    \item \textbf{Comparison task.} When provided the whole tuple with the correctness labels of the responses masked and a guarantee that at least one response is correct, is the model able to discern which response is correct and which is incorrect?
\end{enumerate}

\noindent
The scoring task is also evaluated over a separate set of (question, response) pairs where the response reaches the correct final answer by coincidence but contains fatal errors and should be labeled by a reasonable verifier as being incorrect; an example can be found in Appendix~\ref{app:examplebenchmark}.
In the scoring task, models are provided only with the task description; in the comparison task, models are provided only with the task description and a suggestion to identify disagreements between responses in its reasoning.

Table~\ref{tab:benchmark} lists the baseline performances of current commercial model offerings on this benchmark.
Gemini v1.5 Pro is omitted from the benchmark as the entries in the benchmark are curated from the residuals of Gemini v1.5 Pro.
The prompts used in Table~\ref{tab:benchmark} are provided in Appendix~\ref{app:benchmarkprompts}.

As we previously observed, and has been noted in prior works \citep{tyen_llms_2024, kamoi_evaluating_2024}, verification errors are typically due to low recall.
Even the easier comparison task, models perform only marginally better---and often worse---than random chance.
In many cases, Consistency@5 performs worse than one-shot inference because Consistency simply averages out noise from an output distribution, meaning that a model biased towards producing an incorrect answer will do so with higher probability under Consistency.
Addressing these deficits in verification capabilities---which we see as low-hanging fruit for post-training---would enable not only better sampling-based search, but also other downstream applications of verification including reinforcement learning \citep[e.g.][]{o1-preview,deepseekai2025deepseekr1incentivizingreasoningcapability}, data flywheeling \citep[e.g.,][]{welleck_generating_2022}, and end-user experience (see Section~\ref{sec:related} for further discussion).


\begin{table}[htbp]
\centering
\begin{tabular}{llcccccc}
\toprule
\textbf{Model} & \textbf{Metric} & \multicolumn{3}{c}{\textbf{Scoring Accuracy}} & \multicolumn{1}{c}{\textbf{Comparison Accuracy}} \\
\cmidrule(lr){3-5} \cmidrule(lr){6-6}
 &  & \textbf{Correct} & \textbf{Wrong} & \textbf{Flawed} &  \\
\midrule
\multirow{2}{*}{GPT-4o} & Pass@1    & 76.5\%  & 31.0\% & 22.2\% & 43.2\%\\
 & Consistency@5 & 77.4\% & 30.0\% & 11.1\% & 35.4\% \\
\midrule
\multirow{2}{*}{Claude 3.5 Sonnet} & Pass@1 & 89.6\% & 22.5\% & 33.3\% & 56.1\% \\
 & Consistency@5 & 90.3\% & 17.5\% & 33.3\% & 61.2\% \\
\midrule
\multirow{2}{*}{o1-preview} & Pass@1 & 100\% & 68.8\% & 80.0\% & 84.5\% \\
 & Consistency@5 & 100\% & 79.4\% & 88.8\% & 92\% \\
\midrule
\multirow{2}{*}{Gemini 2.0 Flash} & Pass@1 & 73.5\% & 44.5\% & 60\% & 58\%  \\
 & Consistency@5 & 77.4\% & 42.5\% & 66.6\% & 58.7\% \\
\midrule
\multirow{2}{*}{Gemini 2.0 Thinking Flash} & Pass@1 & 75.4\% & 56.5\% & 53.3\%  & 80\%  \\
 & Consistency@5 & 77.4\%  & 55\% & 55.5\%  & 89.1\% \\
\midrule
\multicolumn{2}{c}{Random guessing}  & 80\% & 20\% & 20\% & 50\% \\
\bottomrule
\end{tabular}
\caption{Accuracy rates of commercial language models on our verification benchmark. For the task of response scoring (Scoring Accuracy), accuracy rates are broken down for entries that require identifying a correct response as being correct (Correct), entries that require identifying a wrong response as being wrong (Wrong), and entries that require identifying a wrong response that coincidentally reaches the correct answer as being wrong (Flawed).
GPT-4o and Claude 3.5 Sonnet only perform marginally better than random guessing across all tasks. o1-Preview performs better, but still fails to identify 20-30\% of wrong responses.
}
\label{tab:benchmark}
\end{table}


\textbf{Reproduction of small MAE models.} The MAE-T and MAE-S are reproduced following the settings in \citet{mae}. We make several adjustments to the decoder to make it suitable for small encoders. For both MAE-S and MAE-T, the decoder includes 8 transformer blocks, each with 8 attention heads. The decoder dimensions in MAE-S and MAE-T are 256 and 192, respectively.

\textbf{Implementation details.} The weights of all pre-trained models are downloaded from corresponding official repositories. The models are trained for 100 epochs using MMSegmentation \citep{mmseg}. For different methods and model sizes, we keep the learning rate constant and sweep the layerwise decay rate across \{0.5, 0.65, 0.75, 0.85, 1.0\}. To adapt models pre-trained on three-channel RGB images for single-channel infrared images, we duplicate the infrared images three times to create pseudo-three-channel images.

\subsection{Evaluation metrics}
\label{app:appendix_evaluation}

\textbf{Fine-tuning.} \textit{Fine-tuning} is the default evaluation metric in this work, which utilizes the pre-trained model as the backbone of existing semantic segmentation models. Following previous works \citep{mae, iBOT}, we employ UperNet \citep{upernet} as the semantic segmentation model. As illustrated in \figref{fig:ft_lp}\textcolor{red}{a}, to build the feature pyramid based on the non-hierarchical ViT model, features from different layers are passed through the MaxPooling layers or DeConv layers, to obtain features of different resolutions. These multi-scale features are then input into the decoder for segmentation results. Following \citet{mae} and \citet{iBOT}, we use features of the \{4, 6, 8, 12\} layers in ViT-T, ViT-S, and ViT-B, and the features of the \{8, 12, 16, 24\} layers in ViT-L, to build the feature pyramid. Remarkably, in fine-tuning, all parameters including the pre-trained model, the feature pyramid, and the decoder, are tuned with the labeled downstream datasets. Hyperparameters are listed in \tabref{tab:setting_seg}.

\textbf{Linear Probing.} As mentioned above, \textit{fine-tuning} introduces additional learnable parameters and alters the pre-trained feature representation. Its performance may not fully reflect the characteristics of the pre-trained features. Therefore, \textit{linear probing} is also employed as an evaluation metric. As shown in \figref{fig:ft_lp}\textcolor{red}{b}, features from different layers are resized to $1/4$ of the input resolution and then concatenated together. Finally, a linear head ($1\times1$ conv) utilizes these concatenated features to predict segmentation results. Notably, only the linear head is trainable, while all other parameters are frozen. The layer settings of output features are the same as \textit{fine-tuning}.

\textbf{Fine-tuning (LL-FPN).} This metric is discussed in \secref{sec:distill}, which aims to enhance the fine-tuning performance of UNIP models by using the last layer to obtain features of different resolutions, as depicted in \figref{fig:ft_lp}\textcolor{red}{c}. Specifically, we employ the features of the \{12, 12, 12, 12\} layers in ViT-T, ViT-S, and ViT-B, and the features of the \{24, 24, 24, 24\} layers in ViT-L, to build the feature pyramid. Other settings remain the same as \textit{fine-tuning}.

\begin{figure}[t]
\centering
\includegraphics[width=\linewidth]{figures/ft_lp_v7.pdf}
\vspace{-20pt}
\caption{Illustrations of different transfer architectures for semantic segmentation tasks.}
\label{fig:ft_lp}
\vspace{-8pt}
\end{figure}

\textbf{Layerwise Linear Probing.} This metric is a layerwise version of the \textit{linear probing} metric. It is designed to assess the pre-trained feature representation at each layer. As shown in \figref{fig:ft_lp}\textcolor{red}{d}, only the features of a single layer are forwarded to the linear head following the resize operation. Other settings are the same as \textit{linear probing}.

\begin{table}[t]
    \caption{Settings of semantic segmentation.}
    \label{tab:setting_seg}
    \centering
    \footnotesize
    \setlength{\tabcolsep}{1.5mm}{
    \scalebox{1.0}{
    \begin{tabular}{l  c c c}
         \toprule
         Hyperparameters & SODA & MFNet-T & SCUT-seg \\
         \midrule
         Input resolution & 512 $\times$ 512 & 512 $\times$ 512 & 512 $\times$ 512 \\
         Training epochs & 100 / 200 & 100 / 300 & 100 \\
         Training iterations  &  14400 / 28800  &  9800 / 29400  &  16800 \\
         Peak learning rate  & 1e-4 & 1e-4 & 1e-4 \\
         Batch size  & 8 & 8 & 8  \\
         Optimizer  & AdamW & AdamW & AdamW \\
         Weight decay  & 0.05 & 0.05 & 0.05  \\
         Optimizer momentum  & $\beta_1, \beta_2=$ 0.9, 0.999 & $\beta_1, \beta_2=$ 0.9, 0.999 & $\beta_1, \beta_2=$ 0.9, 0.999 \\
         Learning rate schedule  & Linear decay & Linear decay & Linear decay \\
         Minimal learning rate & 0 & 0 & 0 \\
         Warmup steps  & 1500 / 3000 & 1000 / 3000 & 1700 \\
         \bottomrule
    \end{tabular}}}
    \vspace{-2mm}
\end{table}


\subsection{Evaluation datasets}
\label{app:evaluation_datasets}

\textbf{SODA} \citep{soda}. This dataset features a variety of indoor and outdoor scenes. It comprises 1,168 training images and 1,000 test images, spanning 20 distinct semantic categories, including road, building, car, chair, lamp, table, monitor, and others.

\textbf{MFNet} \citep{mfnet}. This dataset focuses on RGBT semantic segmentation for automotive driving scenarios and includes 1,569 image pairs of infrared and RGB images. It is divided into 784 training images, 392 validation images, and 393 test images, covering 8 semantic categories such as car, person, bike, curve, and others. When benchmarking the performance of different pre-training methods, we combine the validation set with the test set, resulting in a larger test set of 785 images. When comparing UNIP models with other SOTA semantic segmentation models, we follow their settings, \ie, using the original 393 test images for evaluation.

\textbf{SCUT-Seg} \citep{mcnet}. This dataset includes 1345 training images and 665 test images in nighttime driving scenes. It has 10 classes including road, person, fence, pole, and others.

\textbf{ADE20K} \citep{ade20k}. ADE20K is a large-scale RGB semantic segmentation dataset, covering a variety of scenes from indoor to outdoor and nature to urban. It consists of 20,210 training images and 2,000 test images, with 150 different semantic categories.

\textbf{ImageNet-1K} \citep{imagenet}. ImageNet-1K is a subset of the ImageNet database, consisting of 1,000 categories with roughly 1.2 million training images, 50,000 validation images, and 100,000 test images. It is widely used in computer vision research like image classification and pre-training.

\subsection{UNIP.}
\label{app:pre-training}

\begin{table}[t]
  \begin{minipage}{0.47\linewidth}
  \centering
    \caption{Configurations of ViT for semantic segmentation tasks.}
    \label{tab:vit_config}
    \centering
    \footnotesize
    \setlength{\tabcolsep}{1.2mm}{
    \scalebox{1.0}{
    \begin{tabular}{l c c c}
        \toprule
        Model & Dimension & Head Num & Depth \\
        \midrule
        ViT-T & 192 & 3 & 12 \\
        ViT-S & 384 & 6 & 12 \\
        ViT-B & 768 & 12 & 12 \\
        ViT-L & 1024 & 16 & 24 \\
        \bottomrule
    \end{tabular}}}
    \vspace{-4mm}
  \end{minipage}
  \hfill
  \begin{minipage}{0.52\linewidth}
  \caption{Settings of pre-training.}
    \label{tab:distill_setting}
    \centering
    \footnotesize
    \setlength{\tabcolsep}{1.5mm}{
    \scalebox{1.0}{
    \begin{tabular}{l c}
        \toprule
        Hyperparameters  &  Value \\
        \midrule
        Input resolution  &  224 $\times$  224 \\
        Training epochs &  100 \\
        Warmup epochs  &  5  \\
        Optimizer  &  AdamW  \\
        Base learning rate  & 1e-4  \\
        Weight decay  & 0.05  \\
        Optimizer momentum  &  $\beta_1, \beta_2=$ 0.9, 0.95 \\
        Batch size  &  4096 \\
        Learning rate schedule  &  Cosine decay  \\
        Augmentation & Random resized cropping \& \\
        & Random horizontal flipping \\
        \bottomrule
    \end{tabular}}}
    \vspace{-4mm}
  \end{minipage}
\end{table}





\textbf{Head Misalignment.} To solve the head misalignment between teacher and student models during distillation, we experiment with two methods. (1) The first method is the adaptive block proposed in  \citet{tinymim}. Specifically, during distillation, the number of attention heads in the student model's last layer is adjusted to be the same as that of the teacher model by changing the head dimension while keeping the overall dimension constant. When performing fine-tuning or linear probing on downstream tasks, the number of attention heads is reverted to the standard setting in \tabref{tab:vit_config}. (2) The second method involves adding a self-attention layer at the end of the student model during distillation. The number of attention heads in the extra attention layer is equivalent to the teacher model's. This layer is removed when transferring to downstream tasks. These two methods achieve similar performance, but the latter consumes slightly more training time. Therefore, we use the first method in practice.

\textbf{Feature Distillation.} For the feature distillation in \tabref{tab:target_ablation}, we employ a linear projection layer to match the dimension of the student model to that of the teacher model. The distillation and fine-tuning settings are the same as UNIP. The loss function is the cosine similarity loss between the $L_2$ normalized student feature $l_2(F_T)$ and teacher feature $l_2(F_S)$: 
\begin{align}
    L=1-\cos(l_2(F_T)\cdot l_2(F_S)).
\end{align}

\textbf{Implementation Details.} All experiments are conducted using the PyTorch toolkit \citep{pytorch} on 8 NVIDIA RTX 3090 GPUs. The default settings are shown in \tabref{tab:distill_setting}. We use the linear \textit{learning rate} scaling rule: $lr=base\_lr \times$ batchsize / 256, following \citet{mae}. The semantic segmentation settings of UNIP models are the same as those in \Appref{app:appendix_evaluation}.






\section{Additional analysis}
\label{app:appendix_analysis}

\subsection{Normalized Mutual Information}
\label{app:NMI}
The Normalized Mutual Information (NMI) is employed in \secref{sec:nmi} to measure the attention patterns. Let $p(q_i)$ denote the marginal probability of the $i$-th query token and $p(k_j)$ denote the marginal probability of the $j$-th key token. Since query tokens are evenly distributed across every spatial coordinate, $p(q_i)$ can be formulated as:
\begin{equation}
    p(q_i)=\frac{1}{N}, \quad i=1,2,...,N.
    \label{eq:marginal_q}
\end{equation}
Assume $A^m\in \mathbb{R}^{N\times N}$ represents the $m$-th head of the attention matrix after the softmax operation without the \textit{class} token, where $N$ is the number of spatial tokens. The attention scores from each query token to all key tokens sum to 1, \ie, $\sum_{j=1}^N A_{i,j}^m=1, i=1,2,...,N. $
Thus, each row of $A$ can be viewed as the conditional probability distribution of key tokens given the query token:
\begin{equation}
    p(k_j\vert q_i)=A_{i,j}^m.
\end{equation}
Then the joint probability of $q_i$ and $k_j$ can be calculated as:
\begin{equation}
    p(q_i,k_j)=p(k_j\vert q_i)p(q_i)=\frac{1}{N}A_{i,j}^m.
    \label{eq:joint_qk}
\end{equation}
The marginal probability of $k_j$ is:
\begin{equation}
    p(k_j)=\sum_{i=1}^Np(q_i,k_j)=\frac{1}{N}\sum_{i=1}^NA_{i,j}^m.
    \label{eq:marginal_k}
\end{equation}
The mutual information of query and key tokens can be formulated as:
\begin{equation}
\begin{aligned}
    I^m(Q;K)&=\sum_{i=1}^N\sum_{j=1}^Np(q_i,k_j)\log{\frac{p(q_i,k_j)}{p(q_i)p(k_j)}} \\
    &=\sum_{i=1}^N\sum_{j=1}^N\frac{1}{N}A_{i,j}^m\log{\frac{NA_{i,j}^m}{\sum_{i=1}^NA_{i,j}^m}}.
\end{aligned}
\end{equation}
The entropy of query and key tokens can be calculated as:
\begin{gather}
    H^m(Q)=-\sum_{i=1}^N p(q_i)\log{p(q_i)}=-\sum_{i=1}^N\frac{1}{N}\log{\frac{1}{N}}, \\
    H^m(K)=-\sum_{i=1}^N p(k_j)\log{p(k_j)}=-\sum_{j=1}^N\left(\frac{1}{N}\sum_{i=1}^NA_{i,j}^m\log{\frac{1}{N}\sum_{i=1}^NA_{i,j}^m}\right).
\end{gather}
Therefore, the NMI of the $m$ head is:
\begin{equation}
    \text{NMI}^m(Q;K)=\frac{I^m(Q;K)}{\sqrt{H^m(Q)H^m(K)}}.
\end{equation}
The final NMI is calculated by averaging on all heads:
\begin{equation}
    \text{NMI}(Q;K)=\frac{1}{M}\sum_{m=1}^M \text{NMI}^m(Q;K).
\end{equation}

The value of NMI ranges from 0 to 1. It reaches the maximum value of 1 when the joint probability of the query and key tokens is the same as their marginal probability:
\begin{equation}
    p(q_i,k_i)=p(q_i)=p(k_i), \quad i=1,2,...,N.
    \label{eq:nmi_1}
\end{equation}
According to \eqnref{eq:marginal_q}, \eqnref{eq:joint_qk}, \eqnref{eq:marginal_k}, and \eqnref{eq:nmi_1}, it can be derived that
\begin{equation}
A^m_{i,j}=
\begin{cases} 
1, & \mbox{if }i=j, \\
0, & \mbox{if }i\neq j,
\end{cases}
\end{equation}
which implies that the attention matrix of each head is an identity matrix. This indicates that each query token focuses only on the key token at the same spatial position, which is a particular case of the \textit{local} attention pattern.

On the other hand, the NMI has a value of 0 when the query and key tokens are independent:
\begin{equation}
    p(q_i,k_j)=p(q_i)p(k_j), \quad i=1,2,...,N, j=1,2,...,N.
    \label{eq:nmi_0}
\end{equation}
According to \eqnref{eq:marginal_q}, \eqnref{eq:joint_qk}, \eqnref{eq:marginal_k}, and \eqnref{eq:nmi_0}, we can derive that
\begin{equation}
    A^m_{i,j}=A^m_{k,j}, \quad i=1,2,...,N, k=1,2,...,N, j=1,2,...N,
\end{equation}
which indicates that every row of the attention matrix is the same. This means that each query token has the same attention maps for all key tokens, which is a particular case of the \textit{global} attention pattern.
\textbf{Therefore, a higher NMI value indicates a stronger relationship between the query and key tokens and a more local attention pattern. Conversely, a lower NMI value means that different query tokens have more similar and global attention patterns for key tokens.}




\subsection{Centered Kernel Alignment}
\label{app:CKA}
In this section, we extend the analysis in \secref{sec:attention_pattern} from the attention pattern to the feature representation. Let $x^l$ denote the input features of the $l$-th block of the ViT model. The features of the next block $x^{l+1}$ can be formulated as:
\begin{align}
    &x_{tmp} = x^l + \text{Attention}(\text{LN}(x^l), \\
    &x^{l+1} = x_{tmp} + \text{FFN}(\text{LN}(x_{tmp}),
\end{align}
where $\text{Attention},\text{FFN},\text{LN}$ refer to the self-attention module, the feedforward module, and the LayerNorm layer, respectively. Obviously, the self-attention module plays a crucial role in transforming the feature representation. The \textit{global} attention pattern will bring the features of different tokens closer since different query tokens interact similarly with all key tokens. In contrast, the \textit{local} attention pattern will make the features of different tokens further apart.

\begin{figure}[t]
\centering
\includegraphics[width=\linewidth]{figures/cka_v3.pdf}
\vspace{-22pt}
\caption{CKA representation analysis of different models. UNIP-S aligns well with DINO-S in the shallow and middle layers, indicating that the \textit{hybrid} patterns are effectively distilled from MAE-L.}
\label{fig:cka}
\vspace{-15pt}
\end{figure}

To investigate the relationships between features of different layers and models, we use the centered kernel alignment (CKA), a metric that measures the similarity between two feature maps. The details of CKA can refer to \citet{CKA}. As shown in \figref{fig:cka}, features in the later layers of MAE-S, \eg, the 10th and 11th layers, are similar to features in the shallow layers of DINO-S, \eg, the 4th, 5th, and 6th layer, implying that the features of MAE-S are relatively lower level compared to DINO-S. \textbf{This is consistent with observations in \secref{sec:attention_pattern} that the \textit{local} attention patterns are distributed in the shallow layers of DINO-S, but are present in all layers of MAE.}

For MAE-L, the features in the middle layers (13th to 20th) exhibit high similarity with the middle-layer features of DINO-S (9th and 10th), due to the \textit{hybrid} patterns in these layers. On the contrary, the features in the later layers (the 22nd and 23rd layers) gradually resemble the shallow layers of DINO-S, which can be attributed to the \textit{local} patterns in the later layers of MAE-L.

It is noteworthy that the UNIP model effectively imitates the features in the middle layers of MAE-L. Its features align more closely with DINO-S than those of MAE-S, especially in the shallow and middle layers, demonstrating that attention distillation can implicitly change the features of distilled models like what feature distillation explicitly does.

\newpage
\section{Omitted Details in \pref{sec: experiment}}\label{app: experiment}
For completeness, we include the pseudo code for the benchmark used in our experiment, that is, \texttt{LinUCB} using only the observed feedback;
see \pref{alg:linUCB}.

\begin{algorithm}[h]
\caption{LinUCB with Delayed Feedback}\label{alg:linUCB}
Input: action set $\calA$, a parameter $\lambda>0$.

Initialize: $\wh{\theta}_1$ arbitrarily, $\beta_t = \sqrt{\lambda} + \sqrt{2\log T+n\log(1+\frac{t}{n\lambda})}$ for all $t\in[T]$, $H_1 = \lambda I$.

\For{$t=1,2,\dots,T$}{
     Pick 
     \begin{align*}
         a_t=
         \begin{cases}
            \argmin_{a\in\calA} \inner{a,\wh{\theta}_t} - \beta \|a\|_{H_t}^{-1}, &\mbox{in the loss case,} \\
            \argmax_{a\in\calA} \inner{a,\wh{\theta}_t} + \beta \|a\|_{H_t}^{-1}, &\mbox{in the reward case.}
        \end{cases}
     \end{align*}
     
     Observe the payoff $u_\tau$ for all $\tau$ such that $\tau+d_{\tau}\in (t-1,t]$.

     Update $H_{t+1} = H_t + \sum_{\tau:\tau+d_{\tau}\in(t-1,t]}a_{\tau}a_{\tau}^\top$ and $\wh{\theta}_{t+1}=H_{t+1}^{-1}\sum_{\tau:\tau+d_{\tau}\leq t}a_{\tau}u_{\tau}$.
     
}
\end{algorithm}

\section{Pre-training Dataset.}
\label{app:pretraining_dataset}
\subsection{The InfPre dataset.}
\label{app:infpre}

The InfPre dataset is constructed by collecting images from 23 infrared-related visual datasets. The details of the extracted datasets are presented in \tabref{tab:infpre}. To reduce the redundancy in images with similar backgrounds, we employ two sampling methods: fixed-interval sampling and similarity-based sampling. For datasets containing diverse image sequences with different backgrounds, frames are sampled at fixed intervals (\eg 2, 5, and 10) within each sequence. For datasets captured in the same location, we only sample frames that are less similar to each other. The cosine similarity of image embeddings extracted by DINO-B is used as the similarity metric. Images with high similarity to those already sampled images will be discarded. The Faiss \citep{faiss} library is utilized to accelerate the sampling process.

\begin{table}[t]
    \centering
    \caption{Details of the InfPre dataset. \#Image and \#Extraced image represent the number of original and extracted images from the dataset. Interval and Similarity denote the fixed-interval and similarity-based sampling methods, respectively. The value after the slash indicates the fixed interval or similarity threshold.}
    \label{tab:infpre}
    \scriptsize
    \setlength{\tabcolsep}{1.0mm}{
    \scalebox{0.97}{
    \begin{tabular}{l c c c c c c c}
        \toprule
        Dataset & Task & Scenario & \#Image & \makecell[c]{\#Extracted \\ Image} & \makecell[c]{Average \\ Width} & \makecell[c]{Average \\ Height} & Sampling \\
        \midrule
        RGBT-CC \citep{rgbt_cc} & Crowd Counting & Urban & 2,030 & 2,030 & 636 & 484 & - \\
        KAIST \citep{kaist} & Object Detection & Driving & 95,328 & 9,546 & 640 & 512 & Interval / 10 \\
        Infrared City \citep{infrared_city} & Video Translation & Driving & 200,000+ & 20,187 & 256 & 256 & Interval / 10\\
        CVC-09 \citep{cvc9} & Object Detection & Driving & 13,184 & 13,184 & 640 & 480 & - \\
        CVC-14 \citep{cvc14} & Object Detection & Driving & 8518 & 8518 & 640 & 471 & - \\
        VAP \citep{vap} & Semantic Segmentation & Indoors & 23,080 & 2,309 & 640 & 480 & Interval / 10 \\
        RGBT-234 \citep{rgbt-234} & Object Tracking & Surveillance & 117,612 & 11,762 & 628 & 459 & Interval / 10 \\
        LTD \citep{ltd} & Concept Drift & Surveillance & 26,820,000 & 15,749 & 384 & 288 & Similarity / 0.95 \\
        Rain \citep{rainsnow} & Semantic Segmentation & Surveillance & 130,800 & 25,920 & 640 & 480 & Interval / 5 \\
        Infrared Security \citep{security} & Object Detection & Surveillance & 8,999 & 8,999 & 495 & 386 & - \\
        LLVIP \citep{llvip} & Object Detection & Surveillance & 15,488 & 15,488 & 1,280 & 1,024 & - \\
        LSOTB-TIR \citep{lsotb} & Object Tracking & Diverse & 600,000+ & 61,154 & 925 & 623 & Interval / 10 \\
        Dual-Sensor \citep{dual_sensor} & - & Driving & 73,638 & 14,728 & 384 & 288 & Interval / 5 \\
        LasHeR \citep{lasher} & Object Tracking & Diverse & 740,000+ & 74,035 & 879 & 554 & Interval / 10 \\
        VT5000 \citep{vt5000} & Salient Object Detection & Diverse & 5,000 & 5,000 & 640 & 480 & - \\
        Infrared Vehicle \citep{infrared_vehicle} & Object Detection & Driving & 13166 & 13,166 & 815 & 613 & - \\
        Infrared Ship \citep{ship} & Object Detection & Marine & 9,402 & 9,402 & 772 & 591 & - \\
        DroneVehicle \citep{dronevehicle} & Object Detection & Aerial & 28,439 & 28,439 & 640 & 512 & - \\
        Infrared Aerial \citep{aerial} & Object Detection & Aerial & 11,045 & 11,045 & 627 & 502 & - \\
        VTUAV \citep{vtuav} & Object Tracking & Aerial & 1,700,000 & 166,986 & 1,920 & 1,080 & Interval / 10 \\
        M3FD \citep{m3fd} & Object Detection & Driving & 4,200 & 4,200 & 1,00,1 & 744 & - \\
        OTCBVS IRIS Face \citep{face} & - & Human Face & 4,199 & 4,199 & 320 & 240 & - \\
        Multispectral \citep{multispectral_det} & Object Detection & Driving & 15,042 & 15,042 & 480 & 368 & - \\
        \midrule
        InfPre & Pre-training & Diverse & - & 541,088 & 1,075 & 686 & - \\ 
        \bottomrule
    \end{tabular}}}
    \vspace{-3mm}
\end{table}

\subsection{The InfMix dataset.}


\begin{table}[t]
    \centering
    \caption{The cosine similarity between pre-training and infrared segmentation datasets. The embeddings of images are extracted by DINO-B. The similarity is averaged over all pairwise images from different datasets.}
    \label{tab:domain_gap}
    \vspace{1mm}
    \scriptsize
    \setlength{\tabcolsep}{1.5mm}{
    \scalebox{1.1}{
    \begin{tabular}{l c c c c}
        \toprule
        \multirow{2}{*}{Pre-training dataset} & \multicolumn{4}{c}{Downstream dataset} \\
        \cmidrule(lr){2-5} 
        & SODA & MFNet-T & SCUT-Seg & Mean \\
        \midrule
        ImageNet-1K \citep{imagenet} & 0.083 & 0.074 & 0.081 & 0.079 \\
        COCO \citep{coco} & 0.111 & 0.101 & 0.106 & 0.106 \\
        InfMix & 0.200 & 0.227 & 0.236 & 0.221 \\
        InfMix (gray) & \textbf{0.216} & \textbf{0.246} & \textbf{0.254} & \textbf{0.239} \\
        \bottomrule
    \end{tabular}}}
    \vspace{-3mm}
\end{table}

The InfMix dataset combines the InfPre, the subset of ImageNet-1k \citep{imagenet}, and the training set of COCO \citep{coco}, totaling 859,375 images. \tabref{tab:domain_gap} compares the similarity between various pre-training datasets and three infrared segmentation datasets used in our benchmark. Notably, compared to RGB datasets like ImageNet-1k and COCO, the mixed dataset exhibits higher similarity with infrared downstream tasks, thereby mitigating the representation shift between pre-training and downstream data. Moreover, converting RGB images to grayscale further enhances this similarity, resulting in better fine-tuning performance, as shown in \tabref{tab:dataset_composition_ablation}.


\section{More visualizations.}
\label{app:more_visualization}

We provide additional visualization results in this section. \figref{fig:query_attn_cl} shows the attention maps of different supervised and CL methods of various sizes. The comparison of attention maps between MAE and UNIP is displayed in \figref{fig:distill_query_attn} and \figref{fig:texture_query_attn}. The attention maps of RGB image inputs are visualized in \figref{fig:query_attn_rgb}, exhibiting nearly identical attention pattern distribution with infrared images in \figref{fig:query_attn}.


\begin{figure}[ht]
\centering
\includegraphics[width=\linewidth]{figures/add_query_attn_v4.pdf}
\vspace{-22pt}
\caption{Visualizations of attention maps in supervised and CL models. The attention maps are averaged over different heads. All CL and supervised methods share similar attention pattern distribution across layers.}
\label{fig:query_attn_cl}
\end{figure}

\begin{figure}[ht]
\centering
\includegraphics[width=\linewidth]{figures/distill_query_attn_v2.pdf}
\vspace{-22pt}
\caption{Visualizations of layerwise attention maps in MAE and UNIP-S distilled from MAE-L. The \textit{hybrid} patterns emerge in the later layers of UNIP-S but in the middle layers of MAE-L.}
\label{fig:distill_query_attn}
\end{figure}


\begin{figure}[ht]
\centering
\includegraphics[width=\linewidth]{figures/texture_query_attn_v1.pdf}
\vspace{-22pt}
\caption{Visualizations of attention maps in MAE and UNIP distilled from MAE-L. Attention maps from the 12th layer of MAE-S, the 18th layer of MAE-L, and the 12th layer of UNIP-S are displayed, respectively. Compared to MAE-S and MAE-L, UNIP-S exhibits reduced texture bias, emphasizing shape information over textures.}
\label{fig:texture_query_attn}
\vspace{-15pt}
\end{figure}


\begin{figure}[t]
\centering
\includegraphics[width=\linewidth]{figures/query_attn_rgb.pdf}
\vspace{-24pt}
\caption{Attention maps of RGB image inputs for different query tokens in three representative layers. Each query token's attention map corresponds to a row in the attention matrix, averaged over different heads. Images are from ImageNet \citep{imagenet}.}
\label{fig:query_attn_rgb}
\vspace{-14pt}
\end{figure}


\end{document}
